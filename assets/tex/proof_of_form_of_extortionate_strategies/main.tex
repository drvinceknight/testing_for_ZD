\documentclass[a4]{article}

\usepackage{amsmath}
\usepackage{amssymb}
\usepackage[margin=1.5cm,
            includefoot,
            footskip=30pt]{geometry}

\title{Proof of algebraic condition for extortionate strategies}
\author{}
\date{}

\begin{document}

\maketitle

The defining equations for an extortionate strategy are:


\begin{align}
    \tilde p_1 & = \alpha (R - P) + \beta (R - P)\label{eqn:definition_for_p_1}\\
    \tilde p_2 & = \alpha (S - P) + \beta (T - P)\label{eqn:definition_for_p_2}\\
    \tilde p_3 & = \alpha (T - P) + \beta (S - P)\label{eqn:definition_for_p_3}\\
    \tilde p_4 & = 0
\end{align}

Using equation (\ref{eqn:definition_for_p_2}, \(\alpha\) is isolated

\begin{equation}\label{eqn:initial_expression_for_alpha}
    \alpha = \frac{-\beta (P - T) - \tilde p_2}
                  {P - S}
\end{equation}

Substituting this value in to equation (\ref{eqn:definition_for_p_3}), \(\beta\)
is isolated:

\begin{equation}\label{eqn:expression_for_beta}
    \beta = -\frac{P\tilde p_1 - P \tilde p_2 + S \tilde p_2 - T \tilde p_1}
                  {(S - T)(2 P - S - T)}
\end{equation}

Substituting this back in to (\ref{eqn:initial_expression_for_alpha}) gives:

\begin{equation}\label{eqn:expression_for_alpha}
    \alpha = \frac{-\tilde p_2 + (P - T)(P \tilde p_1 - P\tilde p_2 + S\tilde p_2 - T\tilde p_1)}
                  {(S - T)(2P - S - T)(P - S)}
\end{equation}

Substituting equations
(\ref{eqn:expression_for_beta}-\ref{eqn:expression_for_alpha}) in to
equation (\ref{eqn:definition_for_p_1}) gives the required expression for
\(p_1\).

Taking the ratio of equations
(\ref{eqn:expression_for_beta}-\ref{eqn:expression_for_alpha}) gives the
required expression for \(\chi\).

Finally, the condition \(\chi > 1\) corresponds to:


\begin{equation}
\tilde p_2 (P - T) + \tilde p_3 (S - P) >
                                      \tilde p_2 (P - S) + \tilde p_3 (T - P)
\end{equation}

which can be simplified to:

\begin{equation}
    \tilde p_2 > - \tilde p_3
\end{equation}

recalling that \(\tilde p_2 = p_2 - 1\) and \(\tilde p_3 = p_3\) gives the
required result.

\end{document}
