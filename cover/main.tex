\documentclass[11pt]{letter}

\setlength{\oddsidemargin}{-.10in}
\setlength{\textwidth}{6.7in}
\setlength{\textheight}{11in}
\setlength{\topmargin}{-.40in}

\begin{document}

\signature{The authors}

\begin{letter}{}

\textbf{Cover letter: Recognising and evaluating the effectiveness
       of extortion in the Iterated Prisoner's Dilemma}

To whom it may concern,

We present strong and novel results indicating that extortionate behaviours are
not as robust as adaptable behaviours, extending and improving on fundamental
recent results in evolutionary game theory including highly-cited manuscripts
appearing in Plos One. All research software designed for the work
and all data is made available for all to use according to the very best open
scientific principles.

In 2012, Press and Dyson published a paper in PNAS entitled: ``Iterated
Prisoner’s Dilemma contains strategies that dominate any evolutionary
opponent''. This work has obtained a lot of interest as it seemed to indicate an
evolutionary advantage to extortionate behaviour which puts in doubt a
large amount of work showing how and why cooperative behaviour emerges in
complex systems.

The work we present here extends this observations and looks in to identifying
when and were extortion takes place.
We analyze more than 200 strategies/behaviours from the literature and
many original contributors, obtained through open scientific processes and
available to all to use. A linear algebraic approach is used to determine if a
given strategy is behaving in an extortionate way against a given opponent. Some
of these strategies are classic strategies from the literature and others have
been recently developed using machine learning and reinforcement learning
techniques. This allows us to obtain experimental evidence detailing that whilst
extortionate behaviour can be advantageous, it needs to be combined with
adaptability to be evolutionarily beneficial.

\closing{Sincerely,}

\end{letter}

\end{document}
