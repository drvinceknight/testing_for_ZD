\documentclass[11pt]{letter}

\setlength{\oddsidemargin}{-.10in}
\setlength{\textwidth}{6.7in}
\setlength{\textheight}{11in}
\setlength{\topmargin}{-.40in}

\begin{document}

\signature{The authors}

\begin{letter}{}

\textbf{Cover letter: Recognising and evaluating the effectiveness
       of extortion in the Iterated Prisoner's Dilemma}

To whom it may concern,

In 2012, Press and Dyson published a paper in PNAS entitled: ``Iterated
Prisoner’s Dilemma contains strategies that dominate any evolutionary
opponent''. This work has obtained a lot of interest as it seemed to indicate an
evolutionary advantage to extortionate behaviour. Essentially putting in doubt a
large amount of work showing how and why cooperative behaviour emerges in
complex systems.

There have been numerous follow ups to this work showing that perhaps the
findings where not as robust as originally claimed. One such example is the work
of Adami and Hintze: ``Evolutionary instability of zero-determinant strategies
demonstrates that winning is not everything'' which in 2013 was published in
Nature Communications and showed that whilst extortionate behaviours will better
any given agent this advantage does not extend in an evolutionary way.

The work we present here, extends these observations by reversing them. We
consider a large set of more than 200 behaviours obtained through open
scientific processes and available to all to use and using a linear algebraic
approach to determine if a strategy is behaving in an extortionate way. Some of
these strategies are classic strategies from the literature whilst others are
more up to date and are the result of machine learning and reinforcement
learning techniques. This allows us to obtain experimental evidence detailing
that whilst extortionate behaviour can be advantageous, it needs to be combined
with adaptability to be evolutionarily beneficial.

We present this work for publication in Nature Communications as it not only
extends a number of papers published in this journal, presents strong novel
findings relating to the mechanisms for which behaviours can emerge in complex
systems but also makes available a number of tools and data according to best
open scientific principles.

\closing{Sincerely,}

\end{letter}

\end{document}
