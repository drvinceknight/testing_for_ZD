\documentclass[11pt]{letter}

\setlength{\oddsidemargin}{-.10in}
\setlength{\textwidth}{6.7in}
\setlength{\textheight}{11in}
\setlength{\topmargin}{-.40in}

\begin{document}

\signature{The authors}

\begin{letter}{}

\textbf{Cover letter: Recognising and evaluating the effectiveness
       of extortion in the Iterated Prisoner's Dilemma}

To whom it may concern,

We present strong and novel results indicating that extortionate behaviours are
not as robust as adaptable behaviours, extending and improving on fundamental
recent results in evolutionary game theory including highly-cited manuscripts
appearing in Nature Communications. All research software designed for the work
and all data is made available for all to use according to the very best open
scientific principles.

In 2012, Press and Dyson published a paper in PNAS entitled: ``Iterated
Prisoner’s Dilemma contains strategies that dominate any evolutionary
opponent''. This work has obtained a lot of interest as it seemed to indicate an
evolutionary advantage to extortionate behaviour which puts in doubt a
large amount of work showing how and why cooperative behaviour emerges in
complex systems.

This area of research is within the scope of Nature Communications as
demonstrated by a number of publications in the field. One such example is the work
of Adami and Hintze: ``Evolutionary instability of zero-determinant strategies
demonstrates that winning is not everything'' which in 2013 was published in
Nature Communications and showed that even though extortionate behaviours will never
not lose against any agent in expectation, this advantage does not extend in an
evolutionary way. More recently, in 2019 Becks and Milinski, also published in
Nature Communications a paper entitled: ``Extortion strategies resist
disciplining when higher competitiveness is rewarded with extra gain'' where
they investigated the effect of reward in social
experiments.

The work we present here extends these observations by reversing them. We
analyze more than 200 strategies/behaviours from the literature and many
original contributors, obtained through open scientific processes and available
to all to use. A linear algebraic approach is used to determine if a given
strategy is behaving in an extortionate way against a given opponent. Some of
these strategies are
classic strategies from the literature and others have been recently developed
using machine learning and reinforcement learning techniques. This allows us to
obtain experimental evidence detailing that whilst extortionate behaviour can be
advantageous, it needs to be combined with adaptability to be evolutionarily
beneficial.

\closing{Sincerely,}

\end{letter}

\end{document}
