\documentclass[a4paper]{article}

\usepackage{amsmath}
\usepackage{amssymb}
\usepackage[margin=1.5cm,
            includefoot,
            footskip=30pt]{geometry}
\usepackage{layout}
\usepackage{graphicx}
\usepackage{subcaption}

\usepackage{biblatex}
\usepackage{pdfpages}
\usepackage{booktabs}

\usepackage{lscape}

\usepackage{authblk}

\bibliography{main.bib}
\newcommand{\SSe}{\text{SSE}}

\title{Recognising and evaluating the effectiveness
       of extortion in the Iterated Prisoner's Dilemma}
\author[1]{Vincent A. Knight*}
\author[2]{Marc Harper}
\author[1]{Nikoleta E. Glynatsi}
\author[1]{Jonathan Gillard}
\affil[1]{Cardiff University, School of Mathematics, Cardiff, United Kingdom}
\affil[2]{Google Inc., Mountain View, CA, United States of America}
\date{\today}



\begin{document}

\maketitle

\begin{abstract}

Establishing and maintaining mutual cooperation in agent to agent interactions
can be considered as a question of direct reciprocity. This mechanism can be
readily applied to the iterated prisoners dilemma. Agents cooperate, at a
small cost to themselves, in the hope of obtaining a future benefit. In 2012
a new type of strategy was introduced: zero-determinant strategies which
made way for strategies that were provable extortionate.

In the established literature, most of the studies of the effectiveness or lack
thereof of zero-determinant strategies is done by placing some
zero-determinant strategy in a specific scenario (collection of agents) and
evaluating their performance either numerically or theoretically.

While an interesting class of strategies, the definitions of extortionate
strategies are algebraically rigid, apply only to memory-one strategies, and
require complete knowledge of a strategy (memory-one cooperation
probabilities). The contribution of this work is a method to detect
extortionate behaviour from the history of play of a strategy. This inverts
the paradigm of most studies: instead of observing the effectiveness of some
theoretically extortionate strategies, the largest known collection of
strategies will be observed and their performance compared to how
theoretically extortionate they are.

When applied to a corpus of
\documentclass[a4paper]{article}

\usepackage{amsmath}
\usepackage{amssymb}
\usepackage[margin=1.5cm,
            includefoot,
            footskip=30pt]{geometry}
\usepackage{layout}
\usepackage{graphicx}
\usepackage{subcaption}

\usepackage{biblatex}
\usepackage{pdfpages}

\bibliography{main.bib}

\title{Suspicion: Recognising and evaluating the effectiveness
       of extortion in the Iterated Prisoner's Dilemma}
\author{Vincent A. Knight \and Nikoleta E. Glynatsi}
\date{\today}



\begin{document}

\maketitle

\begin{abstract}
    The Iterated Prisoner's Dilemma is a model for rational and evolutionary
    interactive behaviour. It has applications both in the study of human social
    behaviour as well as in biology.
    It is used to understand when and how a rational individual might
    accept an immediate cost to their own utility for the direct benefit of
    another.

    Much attention has been given to a class of strategies called
    Zero Determinant strategies. It has been theoretically shown that these
    strategies can ``extort'' any player.

    In this work, an approach to identify if observed strategies are playing in
    an extortionate way is described. Furthermore, experimental analysis of
    a large tournament with \documentclass[a4paper]{article}

\usepackage{amsmath}
\usepackage{amssymb}
\usepackage[margin=1.5cm,
            includefoot,
            footskip=30pt]{geometry}
\usepackage{layout}
\usepackage{graphicx}
\usepackage{subcaption}

\usepackage{biblatex}
\usepackage{pdfpages}

\bibliography{main.bib}

\title{Suspicion: Recognising and evaluating the effectiveness
       of extortion in the Iterated Prisoner's Dilemma}
\author{Vincent A. Knight \and Nikoleta E. Glynatsi}
\date{\today}



\begin{document}

\maketitle

\begin{abstract}
    The Iterated Prisoner's Dilemma is a model for rational and evolutionary
    interactive behaviour. It has applications both in the study of human social
    behaviour as well as in biology.
    It is used to understand when and how a rational individual might
    accept an immediate cost to their own utility for the direct benefit of
    another.

    Much attention has been given to a class of strategies called
    Zero Determinant strategies. It has been theoretically shown that these
    strategies can ``extort'' any player.

    In this work, an approach to identify if observed strategies are playing in
    an extortionate way is described. Furthermore, experimental analysis of
    a large tournament with \documentclass[a4paper]{article}

\usepackage{amsmath}
\usepackage{amssymb}
\usepackage[margin=1.5cm,
            includefoot,
            footskip=30pt]{geometry}
\usepackage{layout}
\usepackage{graphicx}
\usepackage{subcaption}

\usepackage{biblatex}
\usepackage{pdfpages}

\bibliography{main.bib}

\title{Suspicion: Recognising and evaluating the effectiveness
       of extortion in the Iterated Prisoner's Dilemma}
\author{Vincent A. Knight \and Nikoleta E. Glynatsi}
\date{\today}



\begin{document}

\maketitle

\begin{abstract}
    The Iterated Prisoner's Dilemma is a model for rational and evolutionary
    interactive behaviour. It has applications both in the study of human social
    behaviour as well as in biology.
    It is used to understand when and how a rational individual might
    accept an immediate cost to their own utility for the direct benefit of
    another.

    Much attention has been given to a class of strategies called
    Zero Determinant strategies. It has been theoretically shown that these
    strategies can ``extort'' any player.

    In this work, an approach to identify if observed strategies are playing in
    an extortionate way is described. Furthermore, experimental analysis of
    a large tournament with \input{assets/tex/number_of_full_strategies/main.tex}
    strategies is considered. In this setting
    the most highly performing strategies do not play in an extortionate way
    against each other but do against lower performing strategies.
    This suggests that whilst the theory of Zero Determinant strategies
    indicates that memory is not of fundamental importance to the evolution of
    cooperative behaviour, this is incomplete.
\end{abstract}

\section{Introduction}\label{sec:introduction}

Agent based game theoretic models have become a stalwart of the underpinning
mathematics of interactive behaviours. One of the major pieces of work
in this area is the pair of original computer tournaments run by Robert
Axelrod~\cite{Axelrod1980, Axelrod1980a}. These tournaments pitted submitted
computer strategies against each other in plays of the Iterated Prisoner's
Dilemma. A common game where agents can choose to pay a slight cost to their
immediate utility in the hope of building a reputation. This has been used in
economic and evolutionary game theory to understand the evolution of cooperative
behaviour.

Recently, a class of strategies was described in~\cite{Press2012} that can
provably extort any given opponent. In~\cite{Hilbe2013, Moran1707} some
questions have already been asked about the true effectiveness of these
strategies in an evolutionary setting. Here another question is asked: is it
possible to recognise this extortionate behaviour? A mathematical procedure for
suspicion is presented: in the same way that the continued actions of an
extortionate individual might raise suspicion.

This work makes use of the Axelrod Python library~\cite{Knight2018, Knight2016}
with a large number of Prisoner Dilemma strategies available to give an
extensive numerical example of the ideas presented.  The approach is presented
in Section~\ref{sec:delta-zd-strategies}.  All of the code and data discussed
in Section~\ref{sec:numerical-experiments} is open sourced, archived and
written according to best scientific principles~\cite{Wilson2014}. The data
archive can be found at~\cite{vincent_knight_2018_1297075}.

\section{Recognising Extortion}\label{sec:delta-zd-strategies}

In~\cite{Press2012}, given a match between 2 memory-one strategies, the concept
of Zero Determinant (ZD) strategies is introduced. The main result of that paper
shows that given two memory one players \(p, q\in\mathbb{R}^4\) a linear
relationship between the players' scores could be forced by one of the players.

Using the notation of~\cite{Press2012}, assuming the utilities for player \(p\)
are given by \(S_x=(R, S, T, P)\) and for player \(q\) by \(S_y=(R, T, S, P)\)
and that the stationary scores of each player is given by \(S_X\) and \(S_Y\)
respectively. The main result of~\cite{Press2012} is that if

\begin{equation}\label{eqn:linear_relationship_for_p}
    \tilde p=\alpha S_x + \beta S_y + \gamma
\end{equation}

or

\begin{equation}\label{eqn:linear_relationship_for_q}
    \tilde q=\alpha S_x + \beta S_y + \gamma
\end{equation}

where \(\tilde p = (1 - p_1, 1 - p_2, p_3, p_4)\) and
\(\tilde q = (1 - q_1, 1 - q_2, q_3, q_4)\) then:

\begin{equation}
    \alpha S_X + \beta S_Y + \gamma = 0
\end{equation}

In~\cite{Press2012} a particular type of ZD strategy is defined: extortionate
strategies. If:

\begin{equation}\label{eqn:constraint_for_extortion}
    \gamma = - P(\alpha + \beta)
\end{equation}

then the player can ensure they get a score \(\chi\) times
larger than the opponent. This extortion coefficient is given by:

\begin{equation}\label{eqn:definition_of_chi}
    \chi=\frac{-\beta}{\alpha}
\end{equation}

Thus, if (\ref{eqn:constraint_for_extortion}) holds and \(\chi >1\) a player is
said to extort their opponent.
Here, the reverse problem is considered: given a
\(p\in\mathbb{R}^4\) how does one identify \(\alpha, \beta\) if they
exist and is the strategy in fact acting in an extortionate way?

These conditions correspond to:

\begin{align}
    \tilde p_1 & = \alpha R + \beta R - P (\alpha + \beta)
            \label{eqn:condition_for_tilde_p1}\\
    \tilde p_2 & = \alpha S + \beta T - P (\alpha + \beta)
            \label{eqn:condition_for_tilde_p2}\\
    \tilde p_3 & = \alpha T + \beta S - P (\alpha + \beta)
            \label{eqn:condition_for_tilde_p3}\\
    \tilde p_4 & = \alpha P + \beta P - P (\alpha + \beta)
            \label{eqn:condition_for_tilde_p4}
\end{align}

Equation (\ref{eqn:condition_for_tilde_p4}) ensures that \(p_4=\tilde p_4=0\).
Equations (\ref{eqn:condition_for_tilde_p1}-\ref{eqn:condition_for_tilde_p3})
can be used to eliminate \(\alpha, \beta\), giving:

\begin{equation}\label{eqn:planar_definition_of_extortion}
    \tilde p_1 = \frac{(R - P)(\tilde p_2 + \tilde p_3)}{S + T - 2P}
\end{equation}

with:

\begin{equation}\label{eqn:definition_of_chi}
    \chi = \frac{\tilde p_2 (P - T) + \tilde p_3 (S - P)}
                {\tilde p_2 (P - S) + \tilde p_3 (T - P)}
\end{equation}

Given a strategy \(p\in\mathbb{R}^{4\times 1}\) equations
(\ref{eqn:condition_for_tilde_p4}), (\ref{eqn:planar_definition_of_extortion}-\ref{eqn:definition_of_chi}) can be used to check if
a strategy is extortionate. The conditions correspond to:

\begin{align}
    p_1 & = \frac{(R-P)(p_2 + p_3) - R + T + S - P}{S + T - 2P}
     \label{eqn:condition_for_p1}\\
    p_4 & = 0 \label{eqn:condition_for_p4}\\
    1 & > p_2 + p_3\label{eqn:condition_for_chi}
\end{align}

The algebraic steps necessary to prove these results are available in the
supporting materials.

All extortionate strategies reside on a triangular (\ref{eqn:condition_for_chi})
plane (\ref{eqn:condition_for_p1}) in 3 dimensions (\ref{eqn:condition_for_p4}).
Using this formulation it can be seen that a necessary (but not sufficient)
condition for an extortionate strategy is that it cooperates on average less
than 50\% of the time when in a state of disagreement with the opponent.

As an example, consider the known extortionate strategy \(p=(8 / 9, 1 / 2, 1 /
3, 0)\) from~\cite{Stewart2012} which is referred to as \texttt{Extort-2}. In
this case, for the standard values of \((R, T, S, P)\) constraint
(\ref{eqn:condition_for_p1}) corresponds to:

\begin{equation}
    p_1 = \frac{2(p_2 + p_3) + 1}{3}
\end{equation}

It is clear that in this case all constraints hold.

This approach could in fact be used to confirm that a given strategy is acting
in an extortionate manner even if it is not a memory one strategy. However, in
practice, if a closed form for \(p\) is not known, then due to measurement
and/or numerical error this would not work.

This problem can be written in the following linear algebraic form where
\(x=(\alpha, \beta)\)
and \(p^*=(\tilde p_1 - 1, tilde_2 - 1, p_3)\):

\begin{equation}\label{eqn:linear_algebraic_equation_for_p}
    Cx= p^*
\end{equation}

\(C\) corresponds to equations
(\ref{eqn:condition_for_tilde_p1}-\ref{eqn:condition_for_tilde_p3}) and is
given by:

\begin{equation}\label{eqn:definition_of_C}
    C =
    \begin{bmatrix}
        R - P & R- P \\
        S - P & T- P \\
        T - P & S- P \\
    \end{bmatrix}
\end{equation}

Note that in general, equation (\ref{eqn:linear_algebraic_equation_for_p}) will
not necessarily have a solution. From the Rouch\'{e}-Capelli theorem if there is
a solution it is unique as \(\text{rank}(C)=2\) which is the dimension of the
variable \(x\). The best fitting \(x\) is found by minimizing:

\begin{equation}\label{eqn:r_squared}
    \text{SSError} = \|C x- p^*\|_2^2 = \sum_{i=1}^{3}\left((C\bar x)_i-p_i^*\right)^2
\end{equation}

Note that \(\text{SSError}\), which is the square of the Frobenius
norm~\cite{Golub2013}, becomes a measure of how close a strategy is to being an
extortionate strategy. Suspicion
of extortion then corresponds to a threshold on \(\text{SSError}\).

By observing interactions (human or otherwise), their memory one representation
can be inferred and this approach can be used to recognise extortionate
behaviour. The notion of comparing theoretic and actual plays of the IPD is not
novel, see for example~\cite{Rand2013}. Immediately it is noted that if the
environment is noisy~\cite{Wu1995} then no strategy can be considered to be
extortionate as \(p_4>0\).

In the next section, this idea will be illustrated by observing the interactions
that take place in a computer based tournament of the IPD\@.

\section{Numerical experiments}\label{sec:numerical-experiments}

In~\cite{Stewart2012} results from a tournament with
\input{./assets/tex/number_of_stewart_plotkin_strategies/main.tex} strategies,
was presented with specific consideration given to ZD strategies. This
tournament is reproduced here using the Axelrod-Python
project~\cite{Knight2016}. To obtain a good measure of the corresponding
transition rates for each strategy all matches have been run for
\input{assets/tex/number_of_turns/main.tex} turns and every match has been
repeated \input{assets/tex/number_of_repetitions/main.tex} times. All of this
interaction data is available at~\cite{vincent_knight_2018_1297075}. A good
match between the inferred Markov chain and the state distribution of the actual
interactions has been verified. Data for this is presented in the supplementary
materials.

Figure~\ref{fig:SSError_overall_in_stewart_plotkin} shows the \(\text{SSError}\)
values for all the strategies in the tournament, as reported
in~\cite{Stewart2012} the extortionate strategy (which has an expected
\(\text{SSError}\) approximately 0) gains a large number of wins.

\begin{figure}[!htbp]
    \centering
    \includegraphics[width=.8\textwidth]{./assets/img/SSError_overall_in_stewart_plotkin/main.pdf}
    \caption{\(\text{SSError}\) and state probabilities for the strategies
        of~\cite{Stewart2012}, ordered both by number of wins and overall score.
        Note that \(P(DC)\) is not shown as it corresponds to the transpose of
        \(P(CD)\). Cooperator and Defector are omitted as they do not visit all
        the states.}
    \label{fig:SSError_overall_in_stewart_plotkin}
\end{figure}

Here, the work of~\cite{Stewart2012} is extended by investigating a tournament
with \input{assets/tex/number_of_full_strategies/main.tex}
strategies.

The results of this analysis are shown in
Figure~\ref{fig:SSError_and_probabilities_in_full}. The top ranking strategies
by number of wins seem to be extortionate (but not against all strategies) and
it can be seen that a small sub group of strategies achieve mutual defection.
All the top ranking strategies according to score achieve mutual cooperation and
do not extort each other, however they
\textbf{do} exhibit extortionate behaviour towards a number of the lower ranking
strategies.

\begin{figure}[!htbp]
    \centering
    \includegraphics[width=.8\textwidth]{./assets/img/SSError_and_probabilities_in_full/main.pdf}
    \caption{\(\text{SSError}\) for the strategies for the full tournament. Only
    strategy interactions for which \(p_4=0\) and \(\chi>1\) are displayed.}
    \label{fig:SSError_and_probabilities_in_full}
\end{figure}

\section{Conclusion}\label{sec:conclusion}

This work defines an approach to measure whether or not a player is playing a
strategy that corresponds to an extortionate strategy as defined
in~\cite{Press2012}: a mathematical model for suspicion. Indeed, all
extortionate strategies have been
 classified as lying on a triangular plane.
This rigorous classification fails to be robust to small measurement error, thus
a statistical approach is proposed.
This is done through a linear algebraic approach for approximating the solution
of a linear system. Using this, a large number of pairwise interactions is
simulated and in fact very few strategies are found to act extortionately.

The work of~\cite{Press2012}, whilst showing that a clever approach to taking
advantage of another memory one strategy exists: this is incomplete. Whilst the
elegance of this result is very attractive, just as the simplicity of the
victory of Tit For Tat in Axelrod's original tournaments was, it is incomplete.
Extortionate strategies achieve a high number of wins but they do not
achieve a high score which corresponds to the fitness landscape in an
evolutionary sense. From the large number of interactions a payoff matrix \(S\)
can be measured where \(S_{ij}\) denotes the score (using standard values of
\((R, S, T, P) = (3, 0, 5, 1)\)) of the \(i\)th strategy
against the \(j\)th strategy. Using this, the replicator equation
describes the evolution of the system based on a population density fitness
function:

\begin{equation}\label{eqn:replicator_dynamics}
    \frac{dx}{dt} = x(S-x^TS x)
\end{equation}

Equation (\ref{eqn:replicator_dynamics}) is solved numerically through an
integration technique described in~\cite{Petzold1983} and
Figure~\ref{fig:replicator_dynamics} shows the evolution of the distribution of
the system: the various strategies are ranked by scores. It is clear to see that
only the high ranking strategies survive the evolutionary process (in fact,
only \input{./assets/img/replicator_dynamics/main.tex}
have a final distribution greater than \(10 ^ {-2}\)). This confirms the
findings of~\cite{Moran1707} in which sophisticated strategies resist
evolutionary invasion of shorter memory strategies. Recalling
Figure~\ref{fig:SSError_and_probabilities_in_full} this demonstrates that:

\begin{itemize}
    \item Cooperation emerges through the evolutionary process: the high scoring
        strategies do not exhibit extortionate behaviour towards each other.
    \item Extortionate strategies do not survive the evolutionary process.
\end{itemize}

\begin{figure}[!htbp]
    \centering
    \includegraphics[width=.8\textwidth]{./assets/img/replicator_dynamics/main.pdf}
    \caption{Numerical simulation of the replicator equation
    (\ref{eqn:replicator_dynamics}): strategies are ordered by score, only the strategies with a high score survive the evolutionary process.}
    \label{fig:replicator_dynamics}
\end{figure}

This work can be used to classify plays of the IPD\@: data can be collected from
actual interactions (in lab or in the field). Furthermore, this allows for a
classification method similar to the notion of fingerprinting presented
in~\cite{Ashlock2008}. Trained strategies can potentially be classified as
extortionate or not or it could be possible to even constrain the reinforcement
learning approaches that are becoming prevalent in the literature.
Alternatively, this mathematical approach for recognising extortion could be
used in sophisticated strategies to defend against invasion. Arguably, some of
the strategies considered here exhibit this behaviour, indeed as described
in~\cite{Harper2017}, the top ranking strategies in the full tournament are
obtained using evolutionary reinforcement learning techniques, thus, suspicion
of extortionate behaviour could in fact be an evolutionary trait.

\section*{Acknowledgements}

The following open source software libraries were used in this research:

\begin{itemize}
    \item The Axelrod ~\cite{Knight2016, Knight2018} library (IPD strategies and
        tournaments).
    \item The sympy library~\cite{Meurer2017} (verification of all symbolic
        calculations).
    \item The matplotlib~\cite{Droettboom2018} library (visualisation).
    \item The pandas~\cite{Structures2010}, dask~\cite{Dask2016} and
        NumPy~\cite{Oliphant2015} libraries (data manipulation).
    \item The SciPy~\cite{Jones2001} library (numerical integration of the
        replicator equation).
\end{itemize}

This work was performed using the computational facilities of the Advanced
Research Computing @ Cardiff (ARCCA) Division, Cardiff University.

\printbibliography

\newpage
\section*{Supplementary materials}

\includepdf{assets/pdf/proof_of_form_of_extortionate_strategies/main.pdf}

\newpage

Using the pair wise interactions the transition rates \(p,
q\) can be measured and the steady state probabilities inferred and compared to
the actual probabilities of each state.
This is done numerically by computing the singular eigenvector of the
matrix \(A\) \cite{Stewart2009}:

\[
    A =
    \begin{bmatrix}
        p_1 q_1 & p_1 (1 - q_1) & (1 - p_1) q_1 & (1 -p_1) (1 - q_1) \\
        p_2 q_2 & p_2 (1 - q_2) & (1 - p_2) q_2 & (1 -p_2) (1 - q_2) \\
        p_3 q_3 & p_3 (1 - q_3) & (1 - p_3) q_3 & (1 -p_3) (1 - q_3) \\
        p_4 q_4 & p_4 (1 - q_4) & (1 - p_4) q_4 & (1 -p_4) (1 - q_4) \\
    \end{bmatrix}
\]

Figure~\ref{fig:computed_probabilities_vs_theoretic_probabilities} shows a
regression line fitted to every pairwise interaction with a reported
\(\text{SSError}\) value (pairwise interactions with missing states were
omitted). This serves to validate the approach: a part from some edge cases the
relationship is consistent.

\begin{figure}[!htbp]
    \centering
    \includegraphics[width=.8\textwidth]{./assets/img/computed_probabilities_vs_theoretic_probabilities/main.pdf}
    \caption{The
        relationship between the steady state probabilities inferred from the
        measured transitions and the actual steady state probabilities. A linear
        regression line is included validating the approach.}
    \label{fig:computed_probabilities_vs_theoretic_probabilities}
\end{figure}


\end{document}

    strategies is considered. In this setting
    the most highly performing strategies do not play in an extortionate way
    against each other but do against lower performing strategies.
    This suggests that whilst the theory of Zero Determinant strategies
    indicates that memory is not of fundamental importance to the evolution of
    cooperative behaviour, this is incomplete.
\end{abstract}

\section{Introduction}\label{sec:introduction}

Agent based game theoretic models have become a stalwart of the underpinning
mathematics of interactive behaviours. One of the major pieces of work
in this area is the pair of original computer tournaments run by Robert
Axelrod~\cite{Axelrod1980, Axelrod1980a}. These tournaments pitted submitted
computer strategies against each other in plays of the Iterated Prisoner's
Dilemma. A common game where agents can choose to pay a slight cost to their
immediate utility in the hope of building a reputation. This has been used in
economic and evolutionary game theory to understand the evolution of cooperative
behaviour.

Recently, a class of strategies was described in~\cite{Press2012} that can
provably extort any given opponent. In~\cite{Hilbe2013, Moran1707} some
questions have already been asked about the true effectiveness of these
strategies in an evolutionary setting. Here another question is asked: is it
possible to recognise this extortionate behaviour? A mathematical procedure for
suspicion is presented: in the same way that the continued actions of an
extortionate individual might raise suspicion.

This work makes use of the Axelrod Python library~\cite{Knight2018, Knight2016}
with a large number of Prisoner Dilemma strategies available to give an
extensive numerical example of the ideas presented.  The approach is presented
in Section~\ref{sec:delta-zd-strategies}.  All of the code and data discussed
in Section~\ref{sec:numerical-experiments} is open sourced, archived and
written according to best scientific principles~\cite{Wilson2014}. The data
archive can be found at~\cite{vincent_knight_2018_1297075}.

\section{Recognising Extortion}\label{sec:delta-zd-strategies}

In~\cite{Press2012}, given a match between 2 memory-one strategies, the concept
of Zero Determinant (ZD) strategies is introduced. The main result of that paper
shows that given two memory one players \(p, q\in\mathbb{R}^4\) a linear
relationship between the players' scores could be forced by one of the players.

Using the notation of~\cite{Press2012}, assuming the utilities for player \(p\)
are given by \(S_x=(R, S, T, P)\) and for player \(q\) by \(S_y=(R, T, S, P)\)
and that the stationary scores of each player is given by \(S_X\) and \(S_Y\)
respectively. The main result of~\cite{Press2012} is that if

\begin{equation}\label{eqn:linear_relationship_for_p}
    \tilde p=\alpha S_x + \beta S_y + \gamma
\end{equation}

or

\begin{equation}\label{eqn:linear_relationship_for_q}
    \tilde q=\alpha S_x + \beta S_y + \gamma
\end{equation}

where \(\tilde p = (1 - p_1, 1 - p_2, p_3, p_4)\) and
\(\tilde q = (1 - q_1, 1 - q_2, q_3, q_4)\) then:

\begin{equation}
    \alpha S_X + \beta S_Y + \gamma = 0
\end{equation}

In~\cite{Press2012} a particular type of ZD strategy is defined: extortionate
strategies. If:

\begin{equation}\label{eqn:constraint_for_extortion}
    \gamma = - P(\alpha + \beta)
\end{equation}

then the player can ensure they get a score \(\chi\) times
larger than the opponent. This extortion coefficient is given by:

\begin{equation}\label{eqn:definition_of_chi}
    \chi=\frac{-\beta}{\alpha}
\end{equation}

Thus, if (\ref{eqn:constraint_for_extortion}) holds and \(\chi >1\) a player is
said to extort their opponent.
Here, the reverse problem is considered: given a
\(p\in\mathbb{R}^4\) how does one identify \(\alpha, \beta\) if they
exist and is the strategy in fact acting in an extortionate way?

These conditions correspond to:

\begin{align}
    \tilde p_1 & = \alpha R + \beta R - P (\alpha + \beta)
            \label{eqn:condition_for_tilde_p1}\\
    \tilde p_2 & = \alpha S + \beta T - P (\alpha + \beta)
            \label{eqn:condition_for_tilde_p2}\\
    \tilde p_3 & = \alpha T + \beta S - P (\alpha + \beta)
            \label{eqn:condition_for_tilde_p3}\\
    \tilde p_4 & = \alpha P + \beta P - P (\alpha + \beta)
            \label{eqn:condition_for_tilde_p4}
\end{align}

Equation (\ref{eqn:condition_for_tilde_p4}) ensures that \(p_4=\tilde p_4=0\).
Equations (\ref{eqn:condition_for_tilde_p1}-\ref{eqn:condition_for_tilde_p3})
can be used to eliminate \(\alpha, \beta\), giving:

\begin{equation}\label{eqn:planar_definition_of_extortion}
    \tilde p_1 = \frac{(R - P)(\tilde p_2 + \tilde p_3)}{S + T - 2P}
\end{equation}

with:

\begin{equation}\label{eqn:definition_of_chi}
    \chi = \frac{\tilde p_2 (P - T) + \tilde p_3 (S - P)}
                {\tilde p_2 (P - S) + \tilde p_3 (T - P)}
\end{equation}

Given a strategy \(p\in\mathbb{R}^{4\times 1}\) equations
(\ref{eqn:condition_for_tilde_p4}), (\ref{eqn:planar_definition_of_extortion}-\ref{eqn:definition_of_chi}) can be used to check if
a strategy is extortionate. The conditions correspond to:

\begin{align}
    p_1 & = \frac{(R-P)(p_2 + p_3) - R + T + S - P}{S + T - 2P}
     \label{eqn:condition_for_p1}\\
    p_4 & = 0 \label{eqn:condition_for_p4}\\
    1 & > p_2 + p_3\label{eqn:condition_for_chi}
\end{align}

The algebraic steps necessary to prove these results are available in the
supporting materials.

All extortionate strategies reside on a triangular (\ref{eqn:condition_for_chi})
plane (\ref{eqn:condition_for_p1}) in 3 dimensions (\ref{eqn:condition_for_p4}).
Using this formulation it can be seen that a necessary (but not sufficient)
condition for an extortionate strategy is that it cooperates on average less
than 50\% of the time when in a state of disagreement with the opponent.

As an example, consider the known extortionate strategy \(p=(8 / 9, 1 / 2, 1 /
3, 0)\) from~\cite{Stewart2012} which is referred to as \texttt{Extort-2}. In
this case, for the standard values of \((R, T, S, P)\) constraint
(\ref{eqn:condition_for_p1}) corresponds to:

\begin{equation}
    p_1 = \frac{2(p_2 + p_3) + 1}{3}
\end{equation}

It is clear that in this case all constraints hold.

This approach could in fact be used to confirm that a given strategy is acting
in an extortionate manner even if it is not a memory one strategy. However, in
practice, if a closed form for \(p\) is not known, then due to measurement
and/or numerical error this would not work.

This problem can be written in the following linear algebraic form where
\(x=(\alpha, \beta)\)
and \(p^*=(\tilde p_1 - 1, tilde_2 - 1, p_3)\):

\begin{equation}\label{eqn:linear_algebraic_equation_for_p}
    Cx= p^*
\end{equation}

\(C\) corresponds to equations
(\ref{eqn:condition_for_tilde_p1}-\ref{eqn:condition_for_tilde_p3}) and is
given by:

\begin{equation}\label{eqn:definition_of_C}
    C =
    \begin{bmatrix}
        R - P & R- P \\
        S - P & T- P \\
        T - P & S- P \\
    \end{bmatrix}
\end{equation}

Note that in general, equation (\ref{eqn:linear_algebraic_equation_for_p}) will
not necessarily have a solution. From the Rouch\'{e}-Capelli theorem if there is
a solution it is unique as \(\text{rank}(C)=2\) which is the dimension of the
variable \(x\). The best fitting \(x\) is found by minimizing:

\begin{equation}\label{eqn:r_squared}
    \text{SSError} = \|C x- p^*\|_2^2 = \sum_{i=1}^{3}\left((C\bar x)_i-p_i^*\right)^2
\end{equation}

Note that \(\text{SSError}\), which is the square of the Frobenius
norm~\cite{Golub2013}, becomes a measure of how close a strategy is to being an
extortionate strategy. Suspicion
of extortion then corresponds to a threshold on \(\text{SSError}\).

By observing interactions (human or otherwise), their memory one representation
can be inferred and this approach can be used to recognise extortionate
behaviour. The notion of comparing theoretic and actual plays of the IPD is not
novel, see for example~\cite{Rand2013}. Immediately it is noted that if the
environment is noisy~\cite{Wu1995} then no strategy can be considered to be
extortionate as \(p_4>0\).

In the next section, this idea will be illustrated by observing the interactions
that take place in a computer based tournament of the IPD\@.

\section{Numerical experiments}\label{sec:numerical-experiments}

In~\cite{Stewart2012} results from a tournament with
\documentclass[a4paper]{article}

\usepackage{amsmath}
\usepackage{amssymb}
\usepackage[margin=1.5cm,
            includefoot,
            footskip=30pt]{geometry}
\usepackage{layout}
\usepackage{graphicx}
\usepackage{subcaption}

\usepackage{biblatex}
\usepackage{pdfpages}

\bibliography{main.bib}

\title{Suspicion: Recognising and evaluating the effectiveness
       of extortion in the Iterated Prisoner's Dilemma}
\author{Vincent A. Knight \and Nikoleta E. Glynatsi}
\date{\today}



\begin{document}

\maketitle

\begin{abstract}
    The Iterated Prisoner's Dilemma is a model for rational and evolutionary
    interactive behaviour. It has applications both in the study of human social
    behaviour as well as in biology.
    It is used to understand when and how a rational individual might
    accept an immediate cost to their own utility for the direct benefit of
    another.

    Much attention has been given to a class of strategies called
    Zero Determinant strategies. It has been theoretically shown that these
    strategies can ``extort'' any player.

    In this work, an approach to identify if observed strategies are playing in
    an extortionate way is described. Furthermore, experimental analysis of
    a large tournament with \input{assets/tex/number_of_full_strategies/main.tex}
    strategies is considered. In this setting
    the most highly performing strategies do not play in an extortionate way
    against each other but do against lower performing strategies.
    This suggests that whilst the theory of Zero Determinant strategies
    indicates that memory is not of fundamental importance to the evolution of
    cooperative behaviour, this is incomplete.
\end{abstract}

\section{Introduction}\label{sec:introduction}

Agent based game theoretic models have become a stalwart of the underpinning
mathematics of interactive behaviours. One of the major pieces of work
in this area is the pair of original computer tournaments run by Robert
Axelrod~\cite{Axelrod1980, Axelrod1980a}. These tournaments pitted submitted
computer strategies against each other in plays of the Iterated Prisoner's
Dilemma. A common game where agents can choose to pay a slight cost to their
immediate utility in the hope of building a reputation. This has been used in
economic and evolutionary game theory to understand the evolution of cooperative
behaviour.

Recently, a class of strategies was described in~\cite{Press2012} that can
provably extort any given opponent. In~\cite{Hilbe2013, Moran1707} some
questions have already been asked about the true effectiveness of these
strategies in an evolutionary setting. Here another question is asked: is it
possible to recognise this extortionate behaviour? A mathematical procedure for
suspicion is presented: in the same way that the continued actions of an
extortionate individual might raise suspicion.

This work makes use of the Axelrod Python library~\cite{Knight2018, Knight2016}
with a large number of Prisoner Dilemma strategies available to give an
extensive numerical example of the ideas presented.  The approach is presented
in Section~\ref{sec:delta-zd-strategies}.  All of the code and data discussed
in Section~\ref{sec:numerical-experiments} is open sourced, archived and
written according to best scientific principles~\cite{Wilson2014}. The data
archive can be found at~\cite{vincent_knight_2018_1297075}.

\section{Recognising Extortion}\label{sec:delta-zd-strategies}

In~\cite{Press2012}, given a match between 2 memory-one strategies, the concept
of Zero Determinant (ZD) strategies is introduced. The main result of that paper
shows that given two memory one players \(p, q\in\mathbb{R}^4\) a linear
relationship between the players' scores could be forced by one of the players.

Using the notation of~\cite{Press2012}, assuming the utilities for player \(p\)
are given by \(S_x=(R, S, T, P)\) and for player \(q\) by \(S_y=(R, T, S, P)\)
and that the stationary scores of each player is given by \(S_X\) and \(S_Y\)
respectively. The main result of~\cite{Press2012} is that if

\begin{equation}\label{eqn:linear_relationship_for_p}
    \tilde p=\alpha S_x + \beta S_y + \gamma
\end{equation}

or

\begin{equation}\label{eqn:linear_relationship_for_q}
    \tilde q=\alpha S_x + \beta S_y + \gamma
\end{equation}

where \(\tilde p = (1 - p_1, 1 - p_2, p_3, p_4)\) and
\(\tilde q = (1 - q_1, 1 - q_2, q_3, q_4)\) then:

\begin{equation}
    \alpha S_X + \beta S_Y + \gamma = 0
\end{equation}

In~\cite{Press2012} a particular type of ZD strategy is defined: extortionate
strategies. If:

\begin{equation}\label{eqn:constraint_for_extortion}
    \gamma = - P(\alpha + \beta)
\end{equation}

then the player can ensure they get a score \(\chi\) times
larger than the opponent. This extortion coefficient is given by:

\begin{equation}\label{eqn:definition_of_chi}
    \chi=\frac{-\beta}{\alpha}
\end{equation}

Thus, if (\ref{eqn:constraint_for_extortion}) holds and \(\chi >1\) a player is
said to extort their opponent.
Here, the reverse problem is considered: given a
\(p\in\mathbb{R}^4\) how does one identify \(\alpha, \beta\) if they
exist and is the strategy in fact acting in an extortionate way?

These conditions correspond to:

\begin{align}
    \tilde p_1 & = \alpha R + \beta R - P (\alpha + \beta)
            \label{eqn:condition_for_tilde_p1}\\
    \tilde p_2 & = \alpha S + \beta T - P (\alpha + \beta)
            \label{eqn:condition_for_tilde_p2}\\
    \tilde p_3 & = \alpha T + \beta S - P (\alpha + \beta)
            \label{eqn:condition_for_tilde_p3}\\
    \tilde p_4 & = \alpha P + \beta P - P (\alpha + \beta)
            \label{eqn:condition_for_tilde_p4}
\end{align}

Equation (\ref{eqn:condition_for_tilde_p4}) ensures that \(p_4=\tilde p_4=0\).
Equations (\ref{eqn:condition_for_tilde_p1}-\ref{eqn:condition_for_tilde_p3})
can be used to eliminate \(\alpha, \beta\), giving:

\begin{equation}\label{eqn:planar_definition_of_extortion}
    \tilde p_1 = \frac{(R - P)(\tilde p_2 + \tilde p_3)}{S + T - 2P}
\end{equation}

with:

\begin{equation}\label{eqn:definition_of_chi}
    \chi = \frac{\tilde p_2 (P - T) + \tilde p_3 (S - P)}
                {\tilde p_2 (P - S) + \tilde p_3 (T - P)}
\end{equation}

Given a strategy \(p\in\mathbb{R}^{4\times 1}\) equations
(\ref{eqn:condition_for_tilde_p4}), (\ref{eqn:planar_definition_of_extortion}-\ref{eqn:definition_of_chi}) can be used to check if
a strategy is extortionate. The conditions correspond to:

\begin{align}
    p_1 & = \frac{(R-P)(p_2 + p_3) - R + T + S - P}{S + T - 2P}
     \label{eqn:condition_for_p1}\\
    p_4 & = 0 \label{eqn:condition_for_p4}\\
    1 & > p_2 + p_3\label{eqn:condition_for_chi}
\end{align}

The algebraic steps necessary to prove these results are available in the
supporting materials.

All extortionate strategies reside on a triangular (\ref{eqn:condition_for_chi})
plane (\ref{eqn:condition_for_p1}) in 3 dimensions (\ref{eqn:condition_for_p4}).
Using this formulation it can be seen that a necessary (but not sufficient)
condition for an extortionate strategy is that it cooperates on average less
than 50\% of the time when in a state of disagreement with the opponent.

As an example, consider the known extortionate strategy \(p=(8 / 9, 1 / 2, 1 /
3, 0)\) from~\cite{Stewart2012} which is referred to as \texttt{Extort-2}. In
this case, for the standard values of \((R, T, S, P)\) constraint
(\ref{eqn:condition_for_p1}) corresponds to:

\begin{equation}
    p_1 = \frac{2(p_2 + p_3) + 1}{3}
\end{equation}

It is clear that in this case all constraints hold.

This approach could in fact be used to confirm that a given strategy is acting
in an extortionate manner even if it is not a memory one strategy. However, in
practice, if a closed form for \(p\) is not known, then due to measurement
and/or numerical error this would not work.

This problem can be written in the following linear algebraic form where
\(x=(\alpha, \beta)\)
and \(p^*=(\tilde p_1 - 1, tilde_2 - 1, p_3)\):

\begin{equation}\label{eqn:linear_algebraic_equation_for_p}
    Cx= p^*
\end{equation}

\(C\) corresponds to equations
(\ref{eqn:condition_for_tilde_p1}-\ref{eqn:condition_for_tilde_p3}) and is
given by:

\begin{equation}\label{eqn:definition_of_C}
    C =
    \begin{bmatrix}
        R - P & R- P \\
        S - P & T- P \\
        T - P & S- P \\
    \end{bmatrix}
\end{equation}

Note that in general, equation (\ref{eqn:linear_algebraic_equation_for_p}) will
not necessarily have a solution. From the Rouch\'{e}-Capelli theorem if there is
a solution it is unique as \(\text{rank}(C)=2\) which is the dimension of the
variable \(x\). The best fitting \(x\) is found by minimizing:

\begin{equation}\label{eqn:r_squared}
    \text{SSError} = \|C x- p^*\|_2^2 = \sum_{i=1}^{3}\left((C\bar x)_i-p_i^*\right)^2
\end{equation}

Note that \(\text{SSError}\), which is the square of the Frobenius
norm~\cite{Golub2013}, becomes a measure of how close a strategy is to being an
extortionate strategy. Suspicion
of extortion then corresponds to a threshold on \(\text{SSError}\).

By observing interactions (human or otherwise), their memory one representation
can be inferred and this approach can be used to recognise extortionate
behaviour. The notion of comparing theoretic and actual plays of the IPD is not
novel, see for example~\cite{Rand2013}. Immediately it is noted that if the
environment is noisy~\cite{Wu1995} then no strategy can be considered to be
extortionate as \(p_4>0\).

In the next section, this idea will be illustrated by observing the interactions
that take place in a computer based tournament of the IPD\@.

\section{Numerical experiments}\label{sec:numerical-experiments}

In~\cite{Stewart2012} results from a tournament with
\input{./assets/tex/number_of_stewart_plotkin_strategies/main.tex} strategies,
was presented with specific consideration given to ZD strategies. This
tournament is reproduced here using the Axelrod-Python
project~\cite{Knight2016}. To obtain a good measure of the corresponding
transition rates for each strategy all matches have been run for
\input{assets/tex/number_of_turns/main.tex} turns and every match has been
repeated \input{assets/tex/number_of_repetitions/main.tex} times. All of this
interaction data is available at~\cite{vincent_knight_2018_1297075}. A good
match between the inferred Markov chain and the state distribution of the actual
interactions has been verified. Data for this is presented in the supplementary
materials.

Figure~\ref{fig:SSError_overall_in_stewart_plotkin} shows the \(\text{SSError}\)
values for all the strategies in the tournament, as reported
in~\cite{Stewart2012} the extortionate strategy (which has an expected
\(\text{SSError}\) approximately 0) gains a large number of wins.

\begin{figure}[!htbp]
    \centering
    \includegraphics[width=.8\textwidth]{./assets/img/SSError_overall_in_stewart_plotkin/main.pdf}
    \caption{\(\text{SSError}\) and state probabilities for the strategies
        of~\cite{Stewart2012}, ordered both by number of wins and overall score.
        Note that \(P(DC)\) is not shown as it corresponds to the transpose of
        \(P(CD)\). Cooperator and Defector are omitted as they do not visit all
        the states.}
    \label{fig:SSError_overall_in_stewart_plotkin}
\end{figure}

Here, the work of~\cite{Stewart2012} is extended by investigating a tournament
with \input{assets/tex/number_of_full_strategies/main.tex}
strategies.

The results of this analysis are shown in
Figure~\ref{fig:SSError_and_probabilities_in_full}. The top ranking strategies
by number of wins seem to be extortionate (but not against all strategies) and
it can be seen that a small sub group of strategies achieve mutual defection.
All the top ranking strategies according to score achieve mutual cooperation and
do not extort each other, however they
\textbf{do} exhibit extortionate behaviour towards a number of the lower ranking
strategies.

\begin{figure}[!htbp]
    \centering
    \includegraphics[width=.8\textwidth]{./assets/img/SSError_and_probabilities_in_full/main.pdf}
    \caption{\(\text{SSError}\) for the strategies for the full tournament. Only
    strategy interactions for which \(p_4=0\) and \(\chi>1\) are displayed.}
    \label{fig:SSError_and_probabilities_in_full}
\end{figure}

\section{Conclusion}\label{sec:conclusion}

This work defines an approach to measure whether or not a player is playing a
strategy that corresponds to an extortionate strategy as defined
in~\cite{Press2012}: a mathematical model for suspicion. Indeed, all
extortionate strategies have been
 classified as lying on a triangular plane.
This rigorous classification fails to be robust to small measurement error, thus
a statistical approach is proposed.
This is done through a linear algebraic approach for approximating the solution
of a linear system. Using this, a large number of pairwise interactions is
simulated and in fact very few strategies are found to act extortionately.

The work of~\cite{Press2012}, whilst showing that a clever approach to taking
advantage of another memory one strategy exists: this is incomplete. Whilst the
elegance of this result is very attractive, just as the simplicity of the
victory of Tit For Tat in Axelrod's original tournaments was, it is incomplete.
Extortionate strategies achieve a high number of wins but they do not
achieve a high score which corresponds to the fitness landscape in an
evolutionary sense. From the large number of interactions a payoff matrix \(S\)
can be measured where \(S_{ij}\) denotes the score (using standard values of
\((R, S, T, P) = (3, 0, 5, 1)\)) of the \(i\)th strategy
against the \(j\)th strategy. Using this, the replicator equation
describes the evolution of the system based on a population density fitness
function:

\begin{equation}\label{eqn:replicator_dynamics}
    \frac{dx}{dt} = x(S-x^TS x)
\end{equation}

Equation (\ref{eqn:replicator_dynamics}) is solved numerically through an
integration technique described in~\cite{Petzold1983} and
Figure~\ref{fig:replicator_dynamics} shows the evolution of the distribution of
the system: the various strategies are ranked by scores. It is clear to see that
only the high ranking strategies survive the evolutionary process (in fact,
only \input{./assets/img/replicator_dynamics/main.tex}
have a final distribution greater than \(10 ^ {-2}\)). This confirms the
findings of~\cite{Moran1707} in which sophisticated strategies resist
evolutionary invasion of shorter memory strategies. Recalling
Figure~\ref{fig:SSError_and_probabilities_in_full} this demonstrates that:

\begin{itemize}
    \item Cooperation emerges through the evolutionary process: the high scoring
        strategies do not exhibit extortionate behaviour towards each other.
    \item Extortionate strategies do not survive the evolutionary process.
\end{itemize}

\begin{figure}[!htbp]
    \centering
    \includegraphics[width=.8\textwidth]{./assets/img/replicator_dynamics/main.pdf}
    \caption{Numerical simulation of the replicator equation
    (\ref{eqn:replicator_dynamics}): strategies are ordered by score, only the strategies with a high score survive the evolutionary process.}
    \label{fig:replicator_dynamics}
\end{figure}

This work can be used to classify plays of the IPD\@: data can be collected from
actual interactions (in lab or in the field). Furthermore, this allows for a
classification method similar to the notion of fingerprinting presented
in~\cite{Ashlock2008}. Trained strategies can potentially be classified as
extortionate or not or it could be possible to even constrain the reinforcement
learning approaches that are becoming prevalent in the literature.
Alternatively, this mathematical approach for recognising extortion could be
used in sophisticated strategies to defend against invasion. Arguably, some of
the strategies considered here exhibit this behaviour, indeed as described
in~\cite{Harper2017}, the top ranking strategies in the full tournament are
obtained using evolutionary reinforcement learning techniques, thus, suspicion
of extortionate behaviour could in fact be an evolutionary trait.

\section*{Acknowledgements}

The following open source software libraries were used in this research:

\begin{itemize}
    \item The Axelrod ~\cite{Knight2016, Knight2018} library (IPD strategies and
        tournaments).
    \item The sympy library~\cite{Meurer2017} (verification of all symbolic
        calculations).
    \item The matplotlib~\cite{Droettboom2018} library (visualisation).
    \item The pandas~\cite{Structures2010}, dask~\cite{Dask2016} and
        NumPy~\cite{Oliphant2015} libraries (data manipulation).
    \item The SciPy~\cite{Jones2001} library (numerical integration of the
        replicator equation).
\end{itemize}

This work was performed using the computational facilities of the Advanced
Research Computing @ Cardiff (ARCCA) Division, Cardiff University.

\printbibliography

\newpage
\section*{Supplementary materials}

\includepdf{assets/pdf/proof_of_form_of_extortionate_strategies/main.pdf}

\newpage

Using the pair wise interactions the transition rates \(p,
q\) can be measured and the steady state probabilities inferred and compared to
the actual probabilities of each state.
This is done numerically by computing the singular eigenvector of the
matrix \(A\) \cite{Stewart2009}:

\[
    A =
    \begin{bmatrix}
        p_1 q_1 & p_1 (1 - q_1) & (1 - p_1) q_1 & (1 -p_1) (1 - q_1) \\
        p_2 q_2 & p_2 (1 - q_2) & (1 - p_2) q_2 & (1 -p_2) (1 - q_2) \\
        p_3 q_3 & p_3 (1 - q_3) & (1 - p_3) q_3 & (1 -p_3) (1 - q_3) \\
        p_4 q_4 & p_4 (1 - q_4) & (1 - p_4) q_4 & (1 -p_4) (1 - q_4) \\
    \end{bmatrix}
\]

Figure~\ref{fig:computed_probabilities_vs_theoretic_probabilities} shows a
regression line fitted to every pairwise interaction with a reported
\(\text{SSError}\) value (pairwise interactions with missing states were
omitted). This serves to validate the approach: a part from some edge cases the
relationship is consistent.

\begin{figure}[!htbp]
    \centering
    \includegraphics[width=.8\textwidth]{./assets/img/computed_probabilities_vs_theoretic_probabilities/main.pdf}
    \caption{The
        relationship between the steady state probabilities inferred from the
        measured transitions and the actual steady state probabilities. A linear
        regression line is included validating the approach.}
    \label{fig:computed_probabilities_vs_theoretic_probabilities}
\end{figure}


\end{document}
 strategies,
was presented with specific consideration given to ZD strategies. This
tournament is reproduced here using the Axelrod-Python
project~\cite{Knight2016}. To obtain a good measure of the corresponding
transition rates for each strategy all matches have been run for
\documentclass[a4paper]{article}

\usepackage{amsmath}
\usepackage{amssymb}
\usepackage[margin=1.5cm,
            includefoot,
            footskip=30pt]{geometry}
\usepackage{layout}
\usepackage{graphicx}
\usepackage{subcaption}

\usepackage{biblatex}
\usepackage{pdfpages}

\bibliography{main.bib}

\title{Suspicion: Recognising and evaluating the effectiveness
       of extortion in the Iterated Prisoner's Dilemma}
\author{Vincent A. Knight \and Nikoleta E. Glynatsi}
\date{\today}



\begin{document}

\maketitle

\begin{abstract}
    The Iterated Prisoner's Dilemma is a model for rational and evolutionary
    interactive behaviour. It has applications both in the study of human social
    behaviour as well as in biology.
    It is used to understand when and how a rational individual might
    accept an immediate cost to their own utility for the direct benefit of
    another.

    Much attention has been given to a class of strategies called
    Zero Determinant strategies. It has been theoretically shown that these
    strategies can ``extort'' any player.

    In this work, an approach to identify if observed strategies are playing in
    an extortionate way is described. Furthermore, experimental analysis of
    a large tournament with \input{assets/tex/number_of_full_strategies/main.tex}
    strategies is considered. In this setting
    the most highly performing strategies do not play in an extortionate way
    against each other but do against lower performing strategies.
    This suggests that whilst the theory of Zero Determinant strategies
    indicates that memory is not of fundamental importance to the evolution of
    cooperative behaviour, this is incomplete.
\end{abstract}

\section{Introduction}\label{sec:introduction}

Agent based game theoretic models have become a stalwart of the underpinning
mathematics of interactive behaviours. One of the major pieces of work
in this area is the pair of original computer tournaments run by Robert
Axelrod~\cite{Axelrod1980, Axelrod1980a}. These tournaments pitted submitted
computer strategies against each other in plays of the Iterated Prisoner's
Dilemma. A common game where agents can choose to pay a slight cost to their
immediate utility in the hope of building a reputation. This has been used in
economic and evolutionary game theory to understand the evolution of cooperative
behaviour.

Recently, a class of strategies was described in~\cite{Press2012} that can
provably extort any given opponent. In~\cite{Hilbe2013, Moran1707} some
questions have already been asked about the true effectiveness of these
strategies in an evolutionary setting. Here another question is asked: is it
possible to recognise this extortionate behaviour? A mathematical procedure for
suspicion is presented: in the same way that the continued actions of an
extortionate individual might raise suspicion.

This work makes use of the Axelrod Python library~\cite{Knight2018, Knight2016}
with a large number of Prisoner Dilemma strategies available to give an
extensive numerical example of the ideas presented.  The approach is presented
in Section~\ref{sec:delta-zd-strategies}.  All of the code and data discussed
in Section~\ref{sec:numerical-experiments} is open sourced, archived and
written according to best scientific principles~\cite{Wilson2014}. The data
archive can be found at~\cite{vincent_knight_2018_1297075}.

\section{Recognising Extortion}\label{sec:delta-zd-strategies}

In~\cite{Press2012}, given a match between 2 memory-one strategies, the concept
of Zero Determinant (ZD) strategies is introduced. The main result of that paper
shows that given two memory one players \(p, q\in\mathbb{R}^4\) a linear
relationship between the players' scores could be forced by one of the players.

Using the notation of~\cite{Press2012}, assuming the utilities for player \(p\)
are given by \(S_x=(R, S, T, P)\) and for player \(q\) by \(S_y=(R, T, S, P)\)
and that the stationary scores of each player is given by \(S_X\) and \(S_Y\)
respectively. The main result of~\cite{Press2012} is that if

\begin{equation}\label{eqn:linear_relationship_for_p}
    \tilde p=\alpha S_x + \beta S_y + \gamma
\end{equation}

or

\begin{equation}\label{eqn:linear_relationship_for_q}
    \tilde q=\alpha S_x + \beta S_y + \gamma
\end{equation}

where \(\tilde p = (1 - p_1, 1 - p_2, p_3, p_4)\) and
\(\tilde q = (1 - q_1, 1 - q_2, q_3, q_4)\) then:

\begin{equation}
    \alpha S_X + \beta S_Y + \gamma = 0
\end{equation}

In~\cite{Press2012} a particular type of ZD strategy is defined: extortionate
strategies. If:

\begin{equation}\label{eqn:constraint_for_extortion}
    \gamma = - P(\alpha + \beta)
\end{equation}

then the player can ensure they get a score \(\chi\) times
larger than the opponent. This extortion coefficient is given by:

\begin{equation}\label{eqn:definition_of_chi}
    \chi=\frac{-\beta}{\alpha}
\end{equation}

Thus, if (\ref{eqn:constraint_for_extortion}) holds and \(\chi >1\) a player is
said to extort their opponent.
Here, the reverse problem is considered: given a
\(p\in\mathbb{R}^4\) how does one identify \(\alpha, \beta\) if they
exist and is the strategy in fact acting in an extortionate way?

These conditions correspond to:

\begin{align}
    \tilde p_1 & = \alpha R + \beta R - P (\alpha + \beta)
            \label{eqn:condition_for_tilde_p1}\\
    \tilde p_2 & = \alpha S + \beta T - P (\alpha + \beta)
            \label{eqn:condition_for_tilde_p2}\\
    \tilde p_3 & = \alpha T + \beta S - P (\alpha + \beta)
            \label{eqn:condition_for_tilde_p3}\\
    \tilde p_4 & = \alpha P + \beta P - P (\alpha + \beta)
            \label{eqn:condition_for_tilde_p4}
\end{align}

Equation (\ref{eqn:condition_for_tilde_p4}) ensures that \(p_4=\tilde p_4=0\).
Equations (\ref{eqn:condition_for_tilde_p1}-\ref{eqn:condition_for_tilde_p3})
can be used to eliminate \(\alpha, \beta\), giving:

\begin{equation}\label{eqn:planar_definition_of_extortion}
    \tilde p_1 = \frac{(R - P)(\tilde p_2 + \tilde p_3)}{S + T - 2P}
\end{equation}

with:

\begin{equation}\label{eqn:definition_of_chi}
    \chi = \frac{\tilde p_2 (P - T) + \tilde p_3 (S - P)}
                {\tilde p_2 (P - S) + \tilde p_3 (T - P)}
\end{equation}

Given a strategy \(p\in\mathbb{R}^{4\times 1}\) equations
(\ref{eqn:condition_for_tilde_p4}), (\ref{eqn:planar_definition_of_extortion}-\ref{eqn:definition_of_chi}) can be used to check if
a strategy is extortionate. The conditions correspond to:

\begin{align}
    p_1 & = \frac{(R-P)(p_2 + p_3) - R + T + S - P}{S + T - 2P}
     \label{eqn:condition_for_p1}\\
    p_4 & = 0 \label{eqn:condition_for_p4}\\
    1 & > p_2 + p_3\label{eqn:condition_for_chi}
\end{align}

The algebraic steps necessary to prove these results are available in the
supporting materials.

All extortionate strategies reside on a triangular (\ref{eqn:condition_for_chi})
plane (\ref{eqn:condition_for_p1}) in 3 dimensions (\ref{eqn:condition_for_p4}).
Using this formulation it can be seen that a necessary (but not sufficient)
condition for an extortionate strategy is that it cooperates on average less
than 50\% of the time when in a state of disagreement with the opponent.

As an example, consider the known extortionate strategy \(p=(8 / 9, 1 / 2, 1 /
3, 0)\) from~\cite{Stewart2012} which is referred to as \texttt{Extort-2}. In
this case, for the standard values of \((R, T, S, P)\) constraint
(\ref{eqn:condition_for_p1}) corresponds to:

\begin{equation}
    p_1 = \frac{2(p_2 + p_3) + 1}{3}
\end{equation}

It is clear that in this case all constraints hold.

This approach could in fact be used to confirm that a given strategy is acting
in an extortionate manner even if it is not a memory one strategy. However, in
practice, if a closed form for \(p\) is not known, then due to measurement
and/or numerical error this would not work.

This problem can be written in the following linear algebraic form where
\(x=(\alpha, \beta)\)
and \(p^*=(\tilde p_1 - 1, tilde_2 - 1, p_3)\):

\begin{equation}\label{eqn:linear_algebraic_equation_for_p}
    Cx= p^*
\end{equation}

\(C\) corresponds to equations
(\ref{eqn:condition_for_tilde_p1}-\ref{eqn:condition_for_tilde_p3}) and is
given by:

\begin{equation}\label{eqn:definition_of_C}
    C =
    \begin{bmatrix}
        R - P & R- P \\
        S - P & T- P \\
        T - P & S- P \\
    \end{bmatrix}
\end{equation}

Note that in general, equation (\ref{eqn:linear_algebraic_equation_for_p}) will
not necessarily have a solution. From the Rouch\'{e}-Capelli theorem if there is
a solution it is unique as \(\text{rank}(C)=2\) which is the dimension of the
variable \(x\). The best fitting \(x\) is found by minimizing:

\begin{equation}\label{eqn:r_squared}
    \text{SSError} = \|C x- p^*\|_2^2 = \sum_{i=1}^{3}\left((C\bar x)_i-p_i^*\right)^2
\end{equation}

Note that \(\text{SSError}\), which is the square of the Frobenius
norm~\cite{Golub2013}, becomes a measure of how close a strategy is to being an
extortionate strategy. Suspicion
of extortion then corresponds to a threshold on \(\text{SSError}\).

By observing interactions (human or otherwise), their memory one representation
can be inferred and this approach can be used to recognise extortionate
behaviour. The notion of comparing theoretic and actual plays of the IPD is not
novel, see for example~\cite{Rand2013}. Immediately it is noted that if the
environment is noisy~\cite{Wu1995} then no strategy can be considered to be
extortionate as \(p_4>0\).

In the next section, this idea will be illustrated by observing the interactions
that take place in a computer based tournament of the IPD\@.

\section{Numerical experiments}\label{sec:numerical-experiments}

In~\cite{Stewart2012} results from a tournament with
\input{./assets/tex/number_of_stewart_plotkin_strategies/main.tex} strategies,
was presented with specific consideration given to ZD strategies. This
tournament is reproduced here using the Axelrod-Python
project~\cite{Knight2016}. To obtain a good measure of the corresponding
transition rates for each strategy all matches have been run for
\input{assets/tex/number_of_turns/main.tex} turns and every match has been
repeated \input{assets/tex/number_of_repetitions/main.tex} times. All of this
interaction data is available at~\cite{vincent_knight_2018_1297075}. A good
match between the inferred Markov chain and the state distribution of the actual
interactions has been verified. Data for this is presented in the supplementary
materials.

Figure~\ref{fig:SSError_overall_in_stewart_plotkin} shows the \(\text{SSError}\)
values for all the strategies in the tournament, as reported
in~\cite{Stewart2012} the extortionate strategy (which has an expected
\(\text{SSError}\) approximately 0) gains a large number of wins.

\begin{figure}[!htbp]
    \centering
    \includegraphics[width=.8\textwidth]{./assets/img/SSError_overall_in_stewart_plotkin/main.pdf}
    \caption{\(\text{SSError}\) and state probabilities for the strategies
        of~\cite{Stewart2012}, ordered both by number of wins and overall score.
        Note that \(P(DC)\) is not shown as it corresponds to the transpose of
        \(P(CD)\). Cooperator and Defector are omitted as they do not visit all
        the states.}
    \label{fig:SSError_overall_in_stewart_plotkin}
\end{figure}

Here, the work of~\cite{Stewart2012} is extended by investigating a tournament
with \input{assets/tex/number_of_full_strategies/main.tex}
strategies.

The results of this analysis are shown in
Figure~\ref{fig:SSError_and_probabilities_in_full}. The top ranking strategies
by number of wins seem to be extortionate (but not against all strategies) and
it can be seen that a small sub group of strategies achieve mutual defection.
All the top ranking strategies according to score achieve mutual cooperation and
do not extort each other, however they
\textbf{do} exhibit extortionate behaviour towards a number of the lower ranking
strategies.

\begin{figure}[!htbp]
    \centering
    \includegraphics[width=.8\textwidth]{./assets/img/SSError_and_probabilities_in_full/main.pdf}
    \caption{\(\text{SSError}\) for the strategies for the full tournament. Only
    strategy interactions for which \(p_4=0\) and \(\chi>1\) are displayed.}
    \label{fig:SSError_and_probabilities_in_full}
\end{figure}

\section{Conclusion}\label{sec:conclusion}

This work defines an approach to measure whether or not a player is playing a
strategy that corresponds to an extortionate strategy as defined
in~\cite{Press2012}: a mathematical model for suspicion. Indeed, all
extortionate strategies have been
 classified as lying on a triangular plane.
This rigorous classification fails to be robust to small measurement error, thus
a statistical approach is proposed.
This is done through a linear algebraic approach for approximating the solution
of a linear system. Using this, a large number of pairwise interactions is
simulated and in fact very few strategies are found to act extortionately.

The work of~\cite{Press2012}, whilst showing that a clever approach to taking
advantage of another memory one strategy exists: this is incomplete. Whilst the
elegance of this result is very attractive, just as the simplicity of the
victory of Tit For Tat in Axelrod's original tournaments was, it is incomplete.
Extortionate strategies achieve a high number of wins but they do not
achieve a high score which corresponds to the fitness landscape in an
evolutionary sense. From the large number of interactions a payoff matrix \(S\)
can be measured where \(S_{ij}\) denotes the score (using standard values of
\((R, S, T, P) = (3, 0, 5, 1)\)) of the \(i\)th strategy
against the \(j\)th strategy. Using this, the replicator equation
describes the evolution of the system based on a population density fitness
function:

\begin{equation}\label{eqn:replicator_dynamics}
    \frac{dx}{dt} = x(S-x^TS x)
\end{equation}

Equation (\ref{eqn:replicator_dynamics}) is solved numerically through an
integration technique described in~\cite{Petzold1983} and
Figure~\ref{fig:replicator_dynamics} shows the evolution of the distribution of
the system: the various strategies are ranked by scores. It is clear to see that
only the high ranking strategies survive the evolutionary process (in fact,
only \input{./assets/img/replicator_dynamics/main.tex}
have a final distribution greater than \(10 ^ {-2}\)). This confirms the
findings of~\cite{Moran1707} in which sophisticated strategies resist
evolutionary invasion of shorter memory strategies. Recalling
Figure~\ref{fig:SSError_and_probabilities_in_full} this demonstrates that:

\begin{itemize}
    \item Cooperation emerges through the evolutionary process: the high scoring
        strategies do not exhibit extortionate behaviour towards each other.
    \item Extortionate strategies do not survive the evolutionary process.
\end{itemize}

\begin{figure}[!htbp]
    \centering
    \includegraphics[width=.8\textwidth]{./assets/img/replicator_dynamics/main.pdf}
    \caption{Numerical simulation of the replicator equation
    (\ref{eqn:replicator_dynamics}): strategies are ordered by score, only the strategies with a high score survive the evolutionary process.}
    \label{fig:replicator_dynamics}
\end{figure}

This work can be used to classify plays of the IPD\@: data can be collected from
actual interactions (in lab or in the field). Furthermore, this allows for a
classification method similar to the notion of fingerprinting presented
in~\cite{Ashlock2008}. Trained strategies can potentially be classified as
extortionate or not or it could be possible to even constrain the reinforcement
learning approaches that are becoming prevalent in the literature.
Alternatively, this mathematical approach for recognising extortion could be
used in sophisticated strategies to defend against invasion. Arguably, some of
the strategies considered here exhibit this behaviour, indeed as described
in~\cite{Harper2017}, the top ranking strategies in the full tournament are
obtained using evolutionary reinforcement learning techniques, thus, suspicion
of extortionate behaviour could in fact be an evolutionary trait.

\section*{Acknowledgements}

The following open source software libraries were used in this research:

\begin{itemize}
    \item The Axelrod ~\cite{Knight2016, Knight2018} library (IPD strategies and
        tournaments).
    \item The sympy library~\cite{Meurer2017} (verification of all symbolic
        calculations).
    \item The matplotlib~\cite{Droettboom2018} library (visualisation).
    \item The pandas~\cite{Structures2010}, dask~\cite{Dask2016} and
        NumPy~\cite{Oliphant2015} libraries (data manipulation).
    \item The SciPy~\cite{Jones2001} library (numerical integration of the
        replicator equation).
\end{itemize}

This work was performed using the computational facilities of the Advanced
Research Computing @ Cardiff (ARCCA) Division, Cardiff University.

\printbibliography

\newpage
\section*{Supplementary materials}

\includepdf{assets/pdf/proof_of_form_of_extortionate_strategies/main.pdf}

\newpage

Using the pair wise interactions the transition rates \(p,
q\) can be measured and the steady state probabilities inferred and compared to
the actual probabilities of each state.
This is done numerically by computing the singular eigenvector of the
matrix \(A\) \cite{Stewart2009}:

\[
    A =
    \begin{bmatrix}
        p_1 q_1 & p_1 (1 - q_1) & (1 - p_1) q_1 & (1 -p_1) (1 - q_1) \\
        p_2 q_2 & p_2 (1 - q_2) & (1 - p_2) q_2 & (1 -p_2) (1 - q_2) \\
        p_3 q_3 & p_3 (1 - q_3) & (1 - p_3) q_3 & (1 -p_3) (1 - q_3) \\
        p_4 q_4 & p_4 (1 - q_4) & (1 - p_4) q_4 & (1 -p_4) (1 - q_4) \\
    \end{bmatrix}
\]

Figure~\ref{fig:computed_probabilities_vs_theoretic_probabilities} shows a
regression line fitted to every pairwise interaction with a reported
\(\text{SSError}\) value (pairwise interactions with missing states were
omitted). This serves to validate the approach: a part from some edge cases the
relationship is consistent.

\begin{figure}[!htbp]
    \centering
    \includegraphics[width=.8\textwidth]{./assets/img/computed_probabilities_vs_theoretic_probabilities/main.pdf}
    \caption{The
        relationship between the steady state probabilities inferred from the
        measured transitions and the actual steady state probabilities. A linear
        regression line is included validating the approach.}
    \label{fig:computed_probabilities_vs_theoretic_probabilities}
\end{figure}


\end{document}
 turns and every match has been
repeated \documentclass[a4paper]{article}

\usepackage{amsmath}
\usepackage{amssymb}
\usepackage[margin=1.5cm,
            includefoot,
            footskip=30pt]{geometry}
\usepackage{layout}
\usepackage{graphicx}
\usepackage{subcaption}

\usepackage{biblatex}
\usepackage{pdfpages}

\bibliography{main.bib}

\title{Suspicion: Recognising and evaluating the effectiveness
       of extortion in the Iterated Prisoner's Dilemma}
\author{Vincent A. Knight \and Nikoleta E. Glynatsi}
\date{\today}



\begin{document}

\maketitle

\begin{abstract}
    The Iterated Prisoner's Dilemma is a model for rational and evolutionary
    interactive behaviour. It has applications both in the study of human social
    behaviour as well as in biology.
    It is used to understand when and how a rational individual might
    accept an immediate cost to their own utility for the direct benefit of
    another.

    Much attention has been given to a class of strategies called
    Zero Determinant strategies. It has been theoretically shown that these
    strategies can ``extort'' any player.

    In this work, an approach to identify if observed strategies are playing in
    an extortionate way is described. Furthermore, experimental analysis of
    a large tournament with \input{assets/tex/number_of_full_strategies/main.tex}
    strategies is considered. In this setting
    the most highly performing strategies do not play in an extortionate way
    against each other but do against lower performing strategies.
    This suggests that whilst the theory of Zero Determinant strategies
    indicates that memory is not of fundamental importance to the evolution of
    cooperative behaviour, this is incomplete.
\end{abstract}

\section{Introduction}\label{sec:introduction}

Agent based game theoretic models have become a stalwart of the underpinning
mathematics of interactive behaviours. One of the major pieces of work
in this area is the pair of original computer tournaments run by Robert
Axelrod~\cite{Axelrod1980, Axelrod1980a}. These tournaments pitted submitted
computer strategies against each other in plays of the Iterated Prisoner's
Dilemma. A common game where agents can choose to pay a slight cost to their
immediate utility in the hope of building a reputation. This has been used in
economic and evolutionary game theory to understand the evolution of cooperative
behaviour.

Recently, a class of strategies was described in~\cite{Press2012} that can
provably extort any given opponent. In~\cite{Hilbe2013, Moran1707} some
questions have already been asked about the true effectiveness of these
strategies in an evolutionary setting. Here another question is asked: is it
possible to recognise this extortionate behaviour? A mathematical procedure for
suspicion is presented: in the same way that the continued actions of an
extortionate individual might raise suspicion.

This work makes use of the Axelrod Python library~\cite{Knight2018, Knight2016}
with a large number of Prisoner Dilemma strategies available to give an
extensive numerical example of the ideas presented.  The approach is presented
in Section~\ref{sec:delta-zd-strategies}.  All of the code and data discussed
in Section~\ref{sec:numerical-experiments} is open sourced, archived and
written according to best scientific principles~\cite{Wilson2014}. The data
archive can be found at~\cite{vincent_knight_2018_1297075}.

\section{Recognising Extortion}\label{sec:delta-zd-strategies}

In~\cite{Press2012}, given a match between 2 memory-one strategies, the concept
of Zero Determinant (ZD) strategies is introduced. The main result of that paper
shows that given two memory one players \(p, q\in\mathbb{R}^4\) a linear
relationship between the players' scores could be forced by one of the players.

Using the notation of~\cite{Press2012}, assuming the utilities for player \(p\)
are given by \(S_x=(R, S, T, P)\) and for player \(q\) by \(S_y=(R, T, S, P)\)
and that the stationary scores of each player is given by \(S_X\) and \(S_Y\)
respectively. The main result of~\cite{Press2012} is that if

\begin{equation}\label{eqn:linear_relationship_for_p}
    \tilde p=\alpha S_x + \beta S_y + \gamma
\end{equation}

or

\begin{equation}\label{eqn:linear_relationship_for_q}
    \tilde q=\alpha S_x + \beta S_y + \gamma
\end{equation}

where \(\tilde p = (1 - p_1, 1 - p_2, p_3, p_4)\) and
\(\tilde q = (1 - q_1, 1 - q_2, q_3, q_4)\) then:

\begin{equation}
    \alpha S_X + \beta S_Y + \gamma = 0
\end{equation}

In~\cite{Press2012} a particular type of ZD strategy is defined: extortionate
strategies. If:

\begin{equation}\label{eqn:constraint_for_extortion}
    \gamma = - P(\alpha + \beta)
\end{equation}

then the player can ensure they get a score \(\chi\) times
larger than the opponent. This extortion coefficient is given by:

\begin{equation}\label{eqn:definition_of_chi}
    \chi=\frac{-\beta}{\alpha}
\end{equation}

Thus, if (\ref{eqn:constraint_for_extortion}) holds and \(\chi >1\) a player is
said to extort their opponent.
Here, the reverse problem is considered: given a
\(p\in\mathbb{R}^4\) how does one identify \(\alpha, \beta\) if they
exist and is the strategy in fact acting in an extortionate way?

These conditions correspond to:

\begin{align}
    \tilde p_1 & = \alpha R + \beta R - P (\alpha + \beta)
            \label{eqn:condition_for_tilde_p1}\\
    \tilde p_2 & = \alpha S + \beta T - P (\alpha + \beta)
            \label{eqn:condition_for_tilde_p2}\\
    \tilde p_3 & = \alpha T + \beta S - P (\alpha + \beta)
            \label{eqn:condition_for_tilde_p3}\\
    \tilde p_4 & = \alpha P + \beta P - P (\alpha + \beta)
            \label{eqn:condition_for_tilde_p4}
\end{align}

Equation (\ref{eqn:condition_for_tilde_p4}) ensures that \(p_4=\tilde p_4=0\).
Equations (\ref{eqn:condition_for_tilde_p1}-\ref{eqn:condition_for_tilde_p3})
can be used to eliminate \(\alpha, \beta\), giving:

\begin{equation}\label{eqn:planar_definition_of_extortion}
    \tilde p_1 = \frac{(R - P)(\tilde p_2 + \tilde p_3)}{S + T - 2P}
\end{equation}

with:

\begin{equation}\label{eqn:definition_of_chi}
    \chi = \frac{\tilde p_2 (P - T) + \tilde p_3 (S - P)}
                {\tilde p_2 (P - S) + \tilde p_3 (T - P)}
\end{equation}

Given a strategy \(p\in\mathbb{R}^{4\times 1}\) equations
(\ref{eqn:condition_for_tilde_p4}), (\ref{eqn:planar_definition_of_extortion}-\ref{eqn:definition_of_chi}) can be used to check if
a strategy is extortionate. The conditions correspond to:

\begin{align}
    p_1 & = \frac{(R-P)(p_2 + p_3) - R + T + S - P}{S + T - 2P}
     \label{eqn:condition_for_p1}\\
    p_4 & = 0 \label{eqn:condition_for_p4}\\
    1 & > p_2 + p_3\label{eqn:condition_for_chi}
\end{align}

The algebraic steps necessary to prove these results are available in the
supporting materials.

All extortionate strategies reside on a triangular (\ref{eqn:condition_for_chi})
plane (\ref{eqn:condition_for_p1}) in 3 dimensions (\ref{eqn:condition_for_p4}).
Using this formulation it can be seen that a necessary (but not sufficient)
condition for an extortionate strategy is that it cooperates on average less
than 50\% of the time when in a state of disagreement with the opponent.

As an example, consider the known extortionate strategy \(p=(8 / 9, 1 / 2, 1 /
3, 0)\) from~\cite{Stewart2012} which is referred to as \texttt{Extort-2}. In
this case, for the standard values of \((R, T, S, P)\) constraint
(\ref{eqn:condition_for_p1}) corresponds to:

\begin{equation}
    p_1 = \frac{2(p_2 + p_3) + 1}{3}
\end{equation}

It is clear that in this case all constraints hold.

This approach could in fact be used to confirm that a given strategy is acting
in an extortionate manner even if it is not a memory one strategy. However, in
practice, if a closed form for \(p\) is not known, then due to measurement
and/or numerical error this would not work.

This problem can be written in the following linear algebraic form where
\(x=(\alpha, \beta)\)
and \(p^*=(\tilde p_1 - 1, tilde_2 - 1, p_3)\):

\begin{equation}\label{eqn:linear_algebraic_equation_for_p}
    Cx= p^*
\end{equation}

\(C\) corresponds to equations
(\ref{eqn:condition_for_tilde_p1}-\ref{eqn:condition_for_tilde_p3}) and is
given by:

\begin{equation}\label{eqn:definition_of_C}
    C =
    \begin{bmatrix}
        R - P & R- P \\
        S - P & T- P \\
        T - P & S- P \\
    \end{bmatrix}
\end{equation}

Note that in general, equation (\ref{eqn:linear_algebraic_equation_for_p}) will
not necessarily have a solution. From the Rouch\'{e}-Capelli theorem if there is
a solution it is unique as \(\text{rank}(C)=2\) which is the dimension of the
variable \(x\). The best fitting \(x\) is found by minimizing:

\begin{equation}\label{eqn:r_squared}
    \text{SSError} = \|C x- p^*\|_2^2 = \sum_{i=1}^{3}\left((C\bar x)_i-p_i^*\right)^2
\end{equation}

Note that \(\text{SSError}\), which is the square of the Frobenius
norm~\cite{Golub2013}, becomes a measure of how close a strategy is to being an
extortionate strategy. Suspicion
of extortion then corresponds to a threshold on \(\text{SSError}\).

By observing interactions (human or otherwise), their memory one representation
can be inferred and this approach can be used to recognise extortionate
behaviour. The notion of comparing theoretic and actual plays of the IPD is not
novel, see for example~\cite{Rand2013}. Immediately it is noted that if the
environment is noisy~\cite{Wu1995} then no strategy can be considered to be
extortionate as \(p_4>0\).

In the next section, this idea will be illustrated by observing the interactions
that take place in a computer based tournament of the IPD\@.

\section{Numerical experiments}\label{sec:numerical-experiments}

In~\cite{Stewart2012} results from a tournament with
\input{./assets/tex/number_of_stewart_plotkin_strategies/main.tex} strategies,
was presented with specific consideration given to ZD strategies. This
tournament is reproduced here using the Axelrod-Python
project~\cite{Knight2016}. To obtain a good measure of the corresponding
transition rates for each strategy all matches have been run for
\input{assets/tex/number_of_turns/main.tex} turns and every match has been
repeated \input{assets/tex/number_of_repetitions/main.tex} times. All of this
interaction data is available at~\cite{vincent_knight_2018_1297075}. A good
match between the inferred Markov chain and the state distribution of the actual
interactions has been verified. Data for this is presented in the supplementary
materials.

Figure~\ref{fig:SSError_overall_in_stewart_plotkin} shows the \(\text{SSError}\)
values for all the strategies in the tournament, as reported
in~\cite{Stewart2012} the extortionate strategy (which has an expected
\(\text{SSError}\) approximately 0) gains a large number of wins.

\begin{figure}[!htbp]
    \centering
    \includegraphics[width=.8\textwidth]{./assets/img/SSError_overall_in_stewart_plotkin/main.pdf}
    \caption{\(\text{SSError}\) and state probabilities for the strategies
        of~\cite{Stewart2012}, ordered both by number of wins and overall score.
        Note that \(P(DC)\) is not shown as it corresponds to the transpose of
        \(P(CD)\). Cooperator and Defector are omitted as they do not visit all
        the states.}
    \label{fig:SSError_overall_in_stewart_plotkin}
\end{figure}

Here, the work of~\cite{Stewart2012} is extended by investigating a tournament
with \input{assets/tex/number_of_full_strategies/main.tex}
strategies.

The results of this analysis are shown in
Figure~\ref{fig:SSError_and_probabilities_in_full}. The top ranking strategies
by number of wins seem to be extortionate (but not against all strategies) and
it can be seen that a small sub group of strategies achieve mutual defection.
All the top ranking strategies according to score achieve mutual cooperation and
do not extort each other, however they
\textbf{do} exhibit extortionate behaviour towards a number of the lower ranking
strategies.

\begin{figure}[!htbp]
    \centering
    \includegraphics[width=.8\textwidth]{./assets/img/SSError_and_probabilities_in_full/main.pdf}
    \caption{\(\text{SSError}\) for the strategies for the full tournament. Only
    strategy interactions for which \(p_4=0\) and \(\chi>1\) are displayed.}
    \label{fig:SSError_and_probabilities_in_full}
\end{figure}

\section{Conclusion}\label{sec:conclusion}

This work defines an approach to measure whether or not a player is playing a
strategy that corresponds to an extortionate strategy as defined
in~\cite{Press2012}: a mathematical model for suspicion. Indeed, all
extortionate strategies have been
 classified as lying on a triangular plane.
This rigorous classification fails to be robust to small measurement error, thus
a statistical approach is proposed.
This is done through a linear algebraic approach for approximating the solution
of a linear system. Using this, a large number of pairwise interactions is
simulated and in fact very few strategies are found to act extortionately.

The work of~\cite{Press2012}, whilst showing that a clever approach to taking
advantage of another memory one strategy exists: this is incomplete. Whilst the
elegance of this result is very attractive, just as the simplicity of the
victory of Tit For Tat in Axelrod's original tournaments was, it is incomplete.
Extortionate strategies achieve a high number of wins but they do not
achieve a high score which corresponds to the fitness landscape in an
evolutionary sense. From the large number of interactions a payoff matrix \(S\)
can be measured where \(S_{ij}\) denotes the score (using standard values of
\((R, S, T, P) = (3, 0, 5, 1)\)) of the \(i\)th strategy
against the \(j\)th strategy. Using this, the replicator equation
describes the evolution of the system based on a population density fitness
function:

\begin{equation}\label{eqn:replicator_dynamics}
    \frac{dx}{dt} = x(S-x^TS x)
\end{equation}

Equation (\ref{eqn:replicator_dynamics}) is solved numerically through an
integration technique described in~\cite{Petzold1983} and
Figure~\ref{fig:replicator_dynamics} shows the evolution of the distribution of
the system: the various strategies are ranked by scores. It is clear to see that
only the high ranking strategies survive the evolutionary process (in fact,
only \input{./assets/img/replicator_dynamics/main.tex}
have a final distribution greater than \(10 ^ {-2}\)). This confirms the
findings of~\cite{Moran1707} in which sophisticated strategies resist
evolutionary invasion of shorter memory strategies. Recalling
Figure~\ref{fig:SSError_and_probabilities_in_full} this demonstrates that:

\begin{itemize}
    \item Cooperation emerges through the evolutionary process: the high scoring
        strategies do not exhibit extortionate behaviour towards each other.
    \item Extortionate strategies do not survive the evolutionary process.
\end{itemize}

\begin{figure}[!htbp]
    \centering
    \includegraphics[width=.8\textwidth]{./assets/img/replicator_dynamics/main.pdf}
    \caption{Numerical simulation of the replicator equation
    (\ref{eqn:replicator_dynamics}): strategies are ordered by score, only the strategies with a high score survive the evolutionary process.}
    \label{fig:replicator_dynamics}
\end{figure}

This work can be used to classify plays of the IPD\@: data can be collected from
actual interactions (in lab or in the field). Furthermore, this allows for a
classification method similar to the notion of fingerprinting presented
in~\cite{Ashlock2008}. Trained strategies can potentially be classified as
extortionate or not or it could be possible to even constrain the reinforcement
learning approaches that are becoming prevalent in the literature.
Alternatively, this mathematical approach for recognising extortion could be
used in sophisticated strategies to defend against invasion. Arguably, some of
the strategies considered here exhibit this behaviour, indeed as described
in~\cite{Harper2017}, the top ranking strategies in the full tournament are
obtained using evolutionary reinforcement learning techniques, thus, suspicion
of extortionate behaviour could in fact be an evolutionary trait.

\section*{Acknowledgements}

The following open source software libraries were used in this research:

\begin{itemize}
    \item The Axelrod ~\cite{Knight2016, Knight2018} library (IPD strategies and
        tournaments).
    \item The sympy library~\cite{Meurer2017} (verification of all symbolic
        calculations).
    \item The matplotlib~\cite{Droettboom2018} library (visualisation).
    \item The pandas~\cite{Structures2010}, dask~\cite{Dask2016} and
        NumPy~\cite{Oliphant2015} libraries (data manipulation).
    \item The SciPy~\cite{Jones2001} library (numerical integration of the
        replicator equation).
\end{itemize}

This work was performed using the computational facilities of the Advanced
Research Computing @ Cardiff (ARCCA) Division, Cardiff University.

\printbibliography

\newpage
\section*{Supplementary materials}

\includepdf{assets/pdf/proof_of_form_of_extortionate_strategies/main.pdf}

\newpage

Using the pair wise interactions the transition rates \(p,
q\) can be measured and the steady state probabilities inferred and compared to
the actual probabilities of each state.
This is done numerically by computing the singular eigenvector of the
matrix \(A\) \cite{Stewart2009}:

\[
    A =
    \begin{bmatrix}
        p_1 q_1 & p_1 (1 - q_1) & (1 - p_1) q_1 & (1 -p_1) (1 - q_1) \\
        p_2 q_2 & p_2 (1 - q_2) & (1 - p_2) q_2 & (1 -p_2) (1 - q_2) \\
        p_3 q_3 & p_3 (1 - q_3) & (1 - p_3) q_3 & (1 -p_3) (1 - q_3) \\
        p_4 q_4 & p_4 (1 - q_4) & (1 - p_4) q_4 & (1 -p_4) (1 - q_4) \\
    \end{bmatrix}
\]

Figure~\ref{fig:computed_probabilities_vs_theoretic_probabilities} shows a
regression line fitted to every pairwise interaction with a reported
\(\text{SSError}\) value (pairwise interactions with missing states were
omitted). This serves to validate the approach: a part from some edge cases the
relationship is consistent.

\begin{figure}[!htbp]
    \centering
    \includegraphics[width=.8\textwidth]{./assets/img/computed_probabilities_vs_theoretic_probabilities/main.pdf}
    \caption{The
        relationship between the steady state probabilities inferred from the
        measured transitions and the actual steady state probabilities. A linear
        regression line is included validating the approach.}
    \label{fig:computed_probabilities_vs_theoretic_probabilities}
\end{figure}


\end{document}
 times. All of this
interaction data is available at~\cite{vincent_knight_2018_1297075}. A good
match between the inferred Markov chain and the state distribution of the actual
interactions has been verified. Data for this is presented in the supplementary
materials.

Figure~\ref{fig:SSError_overall_in_stewart_plotkin} shows the \(\text{SSError}\)
values for all the strategies in the tournament, as reported
in~\cite{Stewart2012} the extortionate strategy (which has an expected
\(\text{SSError}\) approximately 0) gains a large number of wins.

\begin{figure}[!htbp]
    \centering
    \includegraphics[width=.8\textwidth]{./assets/img/SSError_overall_in_stewart_plotkin/main.pdf}
    \caption{\(\text{SSError}\) and state probabilities for the strategies
        of~\cite{Stewart2012}, ordered both by number of wins and overall score.
        Note that \(P(DC)\) is not shown as it corresponds to the transpose of
        \(P(CD)\). Cooperator and Defector are omitted as they do not visit all
        the states.}
    \label{fig:SSError_overall_in_stewart_plotkin}
\end{figure}

Here, the work of~\cite{Stewart2012} is extended by investigating a tournament
with \documentclass[a4paper]{article}

\usepackage{amsmath}
\usepackage{amssymb}
\usepackage[margin=1.5cm,
            includefoot,
            footskip=30pt]{geometry}
\usepackage{layout}
\usepackage{graphicx}
\usepackage{subcaption}

\usepackage{biblatex}
\usepackage{pdfpages}

\bibliography{main.bib}

\title{Suspicion: Recognising and evaluating the effectiveness
       of extortion in the Iterated Prisoner's Dilemma}
\author{Vincent A. Knight \and Nikoleta E. Glynatsi}
\date{\today}



\begin{document}

\maketitle

\begin{abstract}
    The Iterated Prisoner's Dilemma is a model for rational and evolutionary
    interactive behaviour. It has applications both in the study of human social
    behaviour as well as in biology.
    It is used to understand when and how a rational individual might
    accept an immediate cost to their own utility for the direct benefit of
    another.

    Much attention has been given to a class of strategies called
    Zero Determinant strategies. It has been theoretically shown that these
    strategies can ``extort'' any player.

    In this work, an approach to identify if observed strategies are playing in
    an extortionate way is described. Furthermore, experimental analysis of
    a large tournament with \input{assets/tex/number_of_full_strategies/main.tex}
    strategies is considered. In this setting
    the most highly performing strategies do not play in an extortionate way
    against each other but do against lower performing strategies.
    This suggests that whilst the theory of Zero Determinant strategies
    indicates that memory is not of fundamental importance to the evolution of
    cooperative behaviour, this is incomplete.
\end{abstract}

\section{Introduction}\label{sec:introduction}

Agent based game theoretic models have become a stalwart of the underpinning
mathematics of interactive behaviours. One of the major pieces of work
in this area is the pair of original computer tournaments run by Robert
Axelrod~\cite{Axelrod1980, Axelrod1980a}. These tournaments pitted submitted
computer strategies against each other in plays of the Iterated Prisoner's
Dilemma. A common game where agents can choose to pay a slight cost to their
immediate utility in the hope of building a reputation. This has been used in
economic and evolutionary game theory to understand the evolution of cooperative
behaviour.

Recently, a class of strategies was described in~\cite{Press2012} that can
provably extort any given opponent. In~\cite{Hilbe2013, Moran1707} some
questions have already been asked about the true effectiveness of these
strategies in an evolutionary setting. Here another question is asked: is it
possible to recognise this extortionate behaviour? A mathematical procedure for
suspicion is presented: in the same way that the continued actions of an
extortionate individual might raise suspicion.

This work makes use of the Axelrod Python library~\cite{Knight2018, Knight2016}
with a large number of Prisoner Dilemma strategies available to give an
extensive numerical example of the ideas presented.  The approach is presented
in Section~\ref{sec:delta-zd-strategies}.  All of the code and data discussed
in Section~\ref{sec:numerical-experiments} is open sourced, archived and
written according to best scientific principles~\cite{Wilson2014}. The data
archive can be found at~\cite{vincent_knight_2018_1297075}.

\section{Recognising Extortion}\label{sec:delta-zd-strategies}

In~\cite{Press2012}, given a match between 2 memory-one strategies, the concept
of Zero Determinant (ZD) strategies is introduced. The main result of that paper
shows that given two memory one players \(p, q\in\mathbb{R}^4\) a linear
relationship between the players' scores could be forced by one of the players.

Using the notation of~\cite{Press2012}, assuming the utilities for player \(p\)
are given by \(S_x=(R, S, T, P)\) and for player \(q\) by \(S_y=(R, T, S, P)\)
and that the stationary scores of each player is given by \(S_X\) and \(S_Y\)
respectively. The main result of~\cite{Press2012} is that if

\begin{equation}\label{eqn:linear_relationship_for_p}
    \tilde p=\alpha S_x + \beta S_y + \gamma
\end{equation}

or

\begin{equation}\label{eqn:linear_relationship_for_q}
    \tilde q=\alpha S_x + \beta S_y + \gamma
\end{equation}

where \(\tilde p = (1 - p_1, 1 - p_2, p_3, p_4)\) and
\(\tilde q = (1 - q_1, 1 - q_2, q_3, q_4)\) then:

\begin{equation}
    \alpha S_X + \beta S_Y + \gamma = 0
\end{equation}

In~\cite{Press2012} a particular type of ZD strategy is defined: extortionate
strategies. If:

\begin{equation}\label{eqn:constraint_for_extortion}
    \gamma = - P(\alpha + \beta)
\end{equation}

then the player can ensure they get a score \(\chi\) times
larger than the opponent. This extortion coefficient is given by:

\begin{equation}\label{eqn:definition_of_chi}
    \chi=\frac{-\beta}{\alpha}
\end{equation}

Thus, if (\ref{eqn:constraint_for_extortion}) holds and \(\chi >1\) a player is
said to extort their opponent.
Here, the reverse problem is considered: given a
\(p\in\mathbb{R}^4\) how does one identify \(\alpha, \beta\) if they
exist and is the strategy in fact acting in an extortionate way?

These conditions correspond to:

\begin{align}
    \tilde p_1 & = \alpha R + \beta R - P (\alpha + \beta)
            \label{eqn:condition_for_tilde_p1}\\
    \tilde p_2 & = \alpha S + \beta T - P (\alpha + \beta)
            \label{eqn:condition_for_tilde_p2}\\
    \tilde p_3 & = \alpha T + \beta S - P (\alpha + \beta)
            \label{eqn:condition_for_tilde_p3}\\
    \tilde p_4 & = \alpha P + \beta P - P (\alpha + \beta)
            \label{eqn:condition_for_tilde_p4}
\end{align}

Equation (\ref{eqn:condition_for_tilde_p4}) ensures that \(p_4=\tilde p_4=0\).
Equations (\ref{eqn:condition_for_tilde_p1}-\ref{eqn:condition_for_tilde_p3})
can be used to eliminate \(\alpha, \beta\), giving:

\begin{equation}\label{eqn:planar_definition_of_extortion}
    \tilde p_1 = \frac{(R - P)(\tilde p_2 + \tilde p_3)}{S + T - 2P}
\end{equation}

with:

\begin{equation}\label{eqn:definition_of_chi}
    \chi = \frac{\tilde p_2 (P - T) + \tilde p_3 (S - P)}
                {\tilde p_2 (P - S) + \tilde p_3 (T - P)}
\end{equation}

Given a strategy \(p\in\mathbb{R}^{4\times 1}\) equations
(\ref{eqn:condition_for_tilde_p4}), (\ref{eqn:planar_definition_of_extortion}-\ref{eqn:definition_of_chi}) can be used to check if
a strategy is extortionate. The conditions correspond to:

\begin{align}
    p_1 & = \frac{(R-P)(p_2 + p_3) - R + T + S - P}{S + T - 2P}
     \label{eqn:condition_for_p1}\\
    p_4 & = 0 \label{eqn:condition_for_p4}\\
    1 & > p_2 + p_3\label{eqn:condition_for_chi}
\end{align}

The algebraic steps necessary to prove these results are available in the
supporting materials.

All extortionate strategies reside on a triangular (\ref{eqn:condition_for_chi})
plane (\ref{eqn:condition_for_p1}) in 3 dimensions (\ref{eqn:condition_for_p4}).
Using this formulation it can be seen that a necessary (but not sufficient)
condition for an extortionate strategy is that it cooperates on average less
than 50\% of the time when in a state of disagreement with the opponent.

As an example, consider the known extortionate strategy \(p=(8 / 9, 1 / 2, 1 /
3, 0)\) from~\cite{Stewart2012} which is referred to as \texttt{Extort-2}. In
this case, for the standard values of \((R, T, S, P)\) constraint
(\ref{eqn:condition_for_p1}) corresponds to:

\begin{equation}
    p_1 = \frac{2(p_2 + p_3) + 1}{3}
\end{equation}

It is clear that in this case all constraints hold.

This approach could in fact be used to confirm that a given strategy is acting
in an extortionate manner even if it is not a memory one strategy. However, in
practice, if a closed form for \(p\) is not known, then due to measurement
and/or numerical error this would not work.

This problem can be written in the following linear algebraic form where
\(x=(\alpha, \beta)\)
and \(p^*=(\tilde p_1 - 1, tilde_2 - 1, p_3)\):

\begin{equation}\label{eqn:linear_algebraic_equation_for_p}
    Cx= p^*
\end{equation}

\(C\) corresponds to equations
(\ref{eqn:condition_for_tilde_p1}-\ref{eqn:condition_for_tilde_p3}) and is
given by:

\begin{equation}\label{eqn:definition_of_C}
    C =
    \begin{bmatrix}
        R - P & R- P \\
        S - P & T- P \\
        T - P & S- P \\
    \end{bmatrix}
\end{equation}

Note that in general, equation (\ref{eqn:linear_algebraic_equation_for_p}) will
not necessarily have a solution. From the Rouch\'{e}-Capelli theorem if there is
a solution it is unique as \(\text{rank}(C)=2\) which is the dimension of the
variable \(x\). The best fitting \(x\) is found by minimizing:

\begin{equation}\label{eqn:r_squared}
    \text{SSError} = \|C x- p^*\|_2^2 = \sum_{i=1}^{3}\left((C\bar x)_i-p_i^*\right)^2
\end{equation}

Note that \(\text{SSError}\), which is the square of the Frobenius
norm~\cite{Golub2013}, becomes a measure of how close a strategy is to being an
extortionate strategy. Suspicion
of extortion then corresponds to a threshold on \(\text{SSError}\).

By observing interactions (human or otherwise), their memory one representation
can be inferred and this approach can be used to recognise extortionate
behaviour. The notion of comparing theoretic and actual plays of the IPD is not
novel, see for example~\cite{Rand2013}. Immediately it is noted that if the
environment is noisy~\cite{Wu1995} then no strategy can be considered to be
extortionate as \(p_4>0\).

In the next section, this idea will be illustrated by observing the interactions
that take place in a computer based tournament of the IPD\@.

\section{Numerical experiments}\label{sec:numerical-experiments}

In~\cite{Stewart2012} results from a tournament with
\input{./assets/tex/number_of_stewart_plotkin_strategies/main.tex} strategies,
was presented with specific consideration given to ZD strategies. This
tournament is reproduced here using the Axelrod-Python
project~\cite{Knight2016}. To obtain a good measure of the corresponding
transition rates for each strategy all matches have been run for
\input{assets/tex/number_of_turns/main.tex} turns and every match has been
repeated \input{assets/tex/number_of_repetitions/main.tex} times. All of this
interaction data is available at~\cite{vincent_knight_2018_1297075}. A good
match between the inferred Markov chain and the state distribution of the actual
interactions has been verified. Data for this is presented in the supplementary
materials.

Figure~\ref{fig:SSError_overall_in_stewart_plotkin} shows the \(\text{SSError}\)
values for all the strategies in the tournament, as reported
in~\cite{Stewart2012} the extortionate strategy (which has an expected
\(\text{SSError}\) approximately 0) gains a large number of wins.

\begin{figure}[!htbp]
    \centering
    \includegraphics[width=.8\textwidth]{./assets/img/SSError_overall_in_stewart_plotkin/main.pdf}
    \caption{\(\text{SSError}\) and state probabilities for the strategies
        of~\cite{Stewart2012}, ordered both by number of wins and overall score.
        Note that \(P(DC)\) is not shown as it corresponds to the transpose of
        \(P(CD)\). Cooperator and Defector are omitted as they do not visit all
        the states.}
    \label{fig:SSError_overall_in_stewart_plotkin}
\end{figure}

Here, the work of~\cite{Stewart2012} is extended by investigating a tournament
with \input{assets/tex/number_of_full_strategies/main.tex}
strategies.

The results of this analysis are shown in
Figure~\ref{fig:SSError_and_probabilities_in_full}. The top ranking strategies
by number of wins seem to be extortionate (but not against all strategies) and
it can be seen that a small sub group of strategies achieve mutual defection.
All the top ranking strategies according to score achieve mutual cooperation and
do not extort each other, however they
\textbf{do} exhibit extortionate behaviour towards a number of the lower ranking
strategies.

\begin{figure}[!htbp]
    \centering
    \includegraphics[width=.8\textwidth]{./assets/img/SSError_and_probabilities_in_full/main.pdf}
    \caption{\(\text{SSError}\) for the strategies for the full tournament. Only
    strategy interactions for which \(p_4=0\) and \(\chi>1\) are displayed.}
    \label{fig:SSError_and_probabilities_in_full}
\end{figure}

\section{Conclusion}\label{sec:conclusion}

This work defines an approach to measure whether or not a player is playing a
strategy that corresponds to an extortionate strategy as defined
in~\cite{Press2012}: a mathematical model for suspicion. Indeed, all
extortionate strategies have been
 classified as lying on a triangular plane.
This rigorous classification fails to be robust to small measurement error, thus
a statistical approach is proposed.
This is done through a linear algebraic approach for approximating the solution
of a linear system. Using this, a large number of pairwise interactions is
simulated and in fact very few strategies are found to act extortionately.

The work of~\cite{Press2012}, whilst showing that a clever approach to taking
advantage of another memory one strategy exists: this is incomplete. Whilst the
elegance of this result is very attractive, just as the simplicity of the
victory of Tit For Tat in Axelrod's original tournaments was, it is incomplete.
Extortionate strategies achieve a high number of wins but they do not
achieve a high score which corresponds to the fitness landscape in an
evolutionary sense. From the large number of interactions a payoff matrix \(S\)
can be measured where \(S_{ij}\) denotes the score (using standard values of
\((R, S, T, P) = (3, 0, 5, 1)\)) of the \(i\)th strategy
against the \(j\)th strategy. Using this, the replicator equation
describes the evolution of the system based on a population density fitness
function:

\begin{equation}\label{eqn:replicator_dynamics}
    \frac{dx}{dt} = x(S-x^TS x)
\end{equation}

Equation (\ref{eqn:replicator_dynamics}) is solved numerically through an
integration technique described in~\cite{Petzold1983} and
Figure~\ref{fig:replicator_dynamics} shows the evolution of the distribution of
the system: the various strategies are ranked by scores. It is clear to see that
only the high ranking strategies survive the evolutionary process (in fact,
only \input{./assets/img/replicator_dynamics/main.tex}
have a final distribution greater than \(10 ^ {-2}\)). This confirms the
findings of~\cite{Moran1707} in which sophisticated strategies resist
evolutionary invasion of shorter memory strategies. Recalling
Figure~\ref{fig:SSError_and_probabilities_in_full} this demonstrates that:

\begin{itemize}
    \item Cooperation emerges through the evolutionary process: the high scoring
        strategies do not exhibit extortionate behaviour towards each other.
    \item Extortionate strategies do not survive the evolutionary process.
\end{itemize}

\begin{figure}[!htbp]
    \centering
    \includegraphics[width=.8\textwidth]{./assets/img/replicator_dynamics/main.pdf}
    \caption{Numerical simulation of the replicator equation
    (\ref{eqn:replicator_dynamics}): strategies are ordered by score, only the strategies with a high score survive the evolutionary process.}
    \label{fig:replicator_dynamics}
\end{figure}

This work can be used to classify plays of the IPD\@: data can be collected from
actual interactions (in lab or in the field). Furthermore, this allows for a
classification method similar to the notion of fingerprinting presented
in~\cite{Ashlock2008}. Trained strategies can potentially be classified as
extortionate or not or it could be possible to even constrain the reinforcement
learning approaches that are becoming prevalent in the literature.
Alternatively, this mathematical approach for recognising extortion could be
used in sophisticated strategies to defend against invasion. Arguably, some of
the strategies considered here exhibit this behaviour, indeed as described
in~\cite{Harper2017}, the top ranking strategies in the full tournament are
obtained using evolutionary reinforcement learning techniques, thus, suspicion
of extortionate behaviour could in fact be an evolutionary trait.

\section*{Acknowledgements}

The following open source software libraries were used in this research:

\begin{itemize}
    \item The Axelrod ~\cite{Knight2016, Knight2018} library (IPD strategies and
        tournaments).
    \item The sympy library~\cite{Meurer2017} (verification of all symbolic
        calculations).
    \item The matplotlib~\cite{Droettboom2018} library (visualisation).
    \item The pandas~\cite{Structures2010}, dask~\cite{Dask2016} and
        NumPy~\cite{Oliphant2015} libraries (data manipulation).
    \item The SciPy~\cite{Jones2001} library (numerical integration of the
        replicator equation).
\end{itemize}

This work was performed using the computational facilities of the Advanced
Research Computing @ Cardiff (ARCCA) Division, Cardiff University.

\printbibliography

\newpage
\section*{Supplementary materials}

\includepdf{assets/pdf/proof_of_form_of_extortionate_strategies/main.pdf}

\newpage

Using the pair wise interactions the transition rates \(p,
q\) can be measured and the steady state probabilities inferred and compared to
the actual probabilities of each state.
This is done numerically by computing the singular eigenvector of the
matrix \(A\) \cite{Stewart2009}:

\[
    A =
    \begin{bmatrix}
        p_1 q_1 & p_1 (1 - q_1) & (1 - p_1) q_1 & (1 -p_1) (1 - q_1) \\
        p_2 q_2 & p_2 (1 - q_2) & (1 - p_2) q_2 & (1 -p_2) (1 - q_2) \\
        p_3 q_3 & p_3 (1 - q_3) & (1 - p_3) q_3 & (1 -p_3) (1 - q_3) \\
        p_4 q_4 & p_4 (1 - q_4) & (1 - p_4) q_4 & (1 -p_4) (1 - q_4) \\
    \end{bmatrix}
\]

Figure~\ref{fig:computed_probabilities_vs_theoretic_probabilities} shows a
regression line fitted to every pairwise interaction with a reported
\(\text{SSError}\) value (pairwise interactions with missing states were
omitted). This serves to validate the approach: a part from some edge cases the
relationship is consistent.

\begin{figure}[!htbp]
    \centering
    \includegraphics[width=.8\textwidth]{./assets/img/computed_probabilities_vs_theoretic_probabilities/main.pdf}
    \caption{The
        relationship between the steady state probabilities inferred from the
        measured transitions and the actual steady state probabilities. A linear
        regression line is included validating the approach.}
    \label{fig:computed_probabilities_vs_theoretic_probabilities}
\end{figure}


\end{document}

strategies.

The results of this analysis are shown in
Figure~\ref{fig:SSError_and_probabilities_in_full}. The top ranking strategies
by number of wins seem to be extortionate (but not against all strategies) and
it can be seen that a small sub group of strategies achieve mutual defection.
All the top ranking strategies according to score achieve mutual cooperation and
do not extort each other, however they
\textbf{do} exhibit extortionate behaviour towards a number of the lower ranking
strategies.

\begin{figure}[!htbp]
    \centering
    \includegraphics[width=.8\textwidth]{./assets/img/SSError_and_probabilities_in_full/main.pdf}
    \caption{\(\text{SSError}\) for the strategies for the full tournament. Only
    strategy interactions for which \(p_4=0\) and \(\chi>1\) are displayed.}
    \label{fig:SSError_and_probabilities_in_full}
\end{figure}

\section{Conclusion}\label{sec:conclusion}

This work defines an approach to measure whether or not a player is playing a
strategy that corresponds to an extortionate strategy as defined
in~\cite{Press2012}: a mathematical model for suspicion. Indeed, all
extortionate strategies have been
 classified as lying on a triangular plane.
This rigorous classification fails to be robust to small measurement error, thus
a statistical approach is proposed.
This is done through a linear algebraic approach for approximating the solution
of a linear system. Using this, a large number of pairwise interactions is
simulated and in fact very few strategies are found to act extortionately.

The work of~\cite{Press2012}, whilst showing that a clever approach to taking
advantage of another memory one strategy exists: this is incomplete. Whilst the
elegance of this result is very attractive, just as the simplicity of the
victory of Tit For Tat in Axelrod's original tournaments was, it is incomplete.
Extortionate strategies achieve a high number of wins but they do not
achieve a high score which corresponds to the fitness landscape in an
evolutionary sense. From the large number of interactions a payoff matrix \(S\)
can be measured where \(S_{ij}\) denotes the score (using standard values of
\((R, S, T, P) = (3, 0, 5, 1)\)) of the \(i\)th strategy
against the \(j\)th strategy. Using this, the replicator equation
describes the evolution of the system based on a population density fitness
function:

\begin{equation}\label{eqn:replicator_dynamics}
    \frac{dx}{dt} = x(S-x^TS x)
\end{equation}

Equation (\ref{eqn:replicator_dynamics}) is solved numerically through an
integration technique described in~\cite{Petzold1983} and
Figure~\ref{fig:replicator_dynamics} shows the evolution of the distribution of
the system: the various strategies are ranked by scores. It is clear to see that
only the high ranking strategies survive the evolutionary process (in fact,
only \documentclass[a4paper]{article}

\usepackage{amsmath}
\usepackage{amssymb}
\usepackage[margin=1.5cm,
            includefoot,
            footskip=30pt]{geometry}
\usepackage{layout}
\usepackage{graphicx}
\usepackage{subcaption}

\usepackage{biblatex}
\usepackage{pdfpages}

\bibliography{main.bib}

\title{Suspicion: Recognising and evaluating the effectiveness
       of extortion in the Iterated Prisoner's Dilemma}
\author{Vincent A. Knight \and Nikoleta E. Glynatsi}
\date{\today}



\begin{document}

\maketitle

\begin{abstract}
    The Iterated Prisoner's Dilemma is a model for rational and evolutionary
    interactive behaviour. It has applications both in the study of human social
    behaviour as well as in biology.
    It is used to understand when and how a rational individual might
    accept an immediate cost to their own utility for the direct benefit of
    another.

    Much attention has been given to a class of strategies called
    Zero Determinant strategies. It has been theoretically shown that these
    strategies can ``extort'' any player.

    In this work, an approach to identify if observed strategies are playing in
    an extortionate way is described. Furthermore, experimental analysis of
    a large tournament with \input{assets/tex/number_of_full_strategies/main.tex}
    strategies is considered. In this setting
    the most highly performing strategies do not play in an extortionate way
    against each other but do against lower performing strategies.
    This suggests that whilst the theory of Zero Determinant strategies
    indicates that memory is not of fundamental importance to the evolution of
    cooperative behaviour, this is incomplete.
\end{abstract}

\section{Introduction}\label{sec:introduction}

Agent based game theoretic models have become a stalwart of the underpinning
mathematics of interactive behaviours. One of the major pieces of work
in this area is the pair of original computer tournaments run by Robert
Axelrod~\cite{Axelrod1980, Axelrod1980a}. These tournaments pitted submitted
computer strategies against each other in plays of the Iterated Prisoner's
Dilemma. A common game where agents can choose to pay a slight cost to their
immediate utility in the hope of building a reputation. This has been used in
economic and evolutionary game theory to understand the evolution of cooperative
behaviour.

Recently, a class of strategies was described in~\cite{Press2012} that can
provably extort any given opponent. In~\cite{Hilbe2013, Moran1707} some
questions have already been asked about the true effectiveness of these
strategies in an evolutionary setting. Here another question is asked: is it
possible to recognise this extortionate behaviour? A mathematical procedure for
suspicion is presented: in the same way that the continued actions of an
extortionate individual might raise suspicion.

This work makes use of the Axelrod Python library~\cite{Knight2018, Knight2016}
with a large number of Prisoner Dilemma strategies available to give an
extensive numerical example of the ideas presented.  The approach is presented
in Section~\ref{sec:delta-zd-strategies}.  All of the code and data discussed
in Section~\ref{sec:numerical-experiments} is open sourced, archived and
written according to best scientific principles~\cite{Wilson2014}. The data
archive can be found at~\cite{vincent_knight_2018_1297075}.

\section{Recognising Extortion}\label{sec:delta-zd-strategies}

In~\cite{Press2012}, given a match between 2 memory-one strategies, the concept
of Zero Determinant (ZD) strategies is introduced. The main result of that paper
shows that given two memory one players \(p, q\in\mathbb{R}^4\) a linear
relationship between the players' scores could be forced by one of the players.

Using the notation of~\cite{Press2012}, assuming the utilities for player \(p\)
are given by \(S_x=(R, S, T, P)\) and for player \(q\) by \(S_y=(R, T, S, P)\)
and that the stationary scores of each player is given by \(S_X\) and \(S_Y\)
respectively. The main result of~\cite{Press2012} is that if

\begin{equation}\label{eqn:linear_relationship_for_p}
    \tilde p=\alpha S_x + \beta S_y + \gamma
\end{equation}

or

\begin{equation}\label{eqn:linear_relationship_for_q}
    \tilde q=\alpha S_x + \beta S_y + \gamma
\end{equation}

where \(\tilde p = (1 - p_1, 1 - p_2, p_3, p_4)\) and
\(\tilde q = (1 - q_1, 1 - q_2, q_3, q_4)\) then:

\begin{equation}
    \alpha S_X + \beta S_Y + \gamma = 0
\end{equation}

In~\cite{Press2012} a particular type of ZD strategy is defined: extortionate
strategies. If:

\begin{equation}\label{eqn:constraint_for_extortion}
    \gamma = - P(\alpha + \beta)
\end{equation}

then the player can ensure they get a score \(\chi\) times
larger than the opponent. This extortion coefficient is given by:

\begin{equation}\label{eqn:definition_of_chi}
    \chi=\frac{-\beta}{\alpha}
\end{equation}

Thus, if (\ref{eqn:constraint_for_extortion}) holds and \(\chi >1\) a player is
said to extort their opponent.
Here, the reverse problem is considered: given a
\(p\in\mathbb{R}^4\) how does one identify \(\alpha, \beta\) if they
exist and is the strategy in fact acting in an extortionate way?

These conditions correspond to:

\begin{align}
    \tilde p_1 & = \alpha R + \beta R - P (\alpha + \beta)
            \label{eqn:condition_for_tilde_p1}\\
    \tilde p_2 & = \alpha S + \beta T - P (\alpha + \beta)
            \label{eqn:condition_for_tilde_p2}\\
    \tilde p_3 & = \alpha T + \beta S - P (\alpha + \beta)
            \label{eqn:condition_for_tilde_p3}\\
    \tilde p_4 & = \alpha P + \beta P - P (\alpha + \beta)
            \label{eqn:condition_for_tilde_p4}
\end{align}

Equation (\ref{eqn:condition_for_tilde_p4}) ensures that \(p_4=\tilde p_4=0\).
Equations (\ref{eqn:condition_for_tilde_p1}-\ref{eqn:condition_for_tilde_p3})
can be used to eliminate \(\alpha, \beta\), giving:

\begin{equation}\label{eqn:planar_definition_of_extortion}
    \tilde p_1 = \frac{(R - P)(\tilde p_2 + \tilde p_3)}{S + T - 2P}
\end{equation}

with:

\begin{equation}\label{eqn:definition_of_chi}
    \chi = \frac{\tilde p_2 (P - T) + \tilde p_3 (S - P)}
                {\tilde p_2 (P - S) + \tilde p_3 (T - P)}
\end{equation}

Given a strategy \(p\in\mathbb{R}^{4\times 1}\) equations
(\ref{eqn:condition_for_tilde_p4}), (\ref{eqn:planar_definition_of_extortion}-\ref{eqn:definition_of_chi}) can be used to check if
a strategy is extortionate. The conditions correspond to:

\begin{align}
    p_1 & = \frac{(R-P)(p_2 + p_3) - R + T + S - P}{S + T - 2P}
     \label{eqn:condition_for_p1}\\
    p_4 & = 0 \label{eqn:condition_for_p4}\\
    1 & > p_2 + p_3\label{eqn:condition_for_chi}
\end{align}

The algebraic steps necessary to prove these results are available in the
supporting materials.

All extortionate strategies reside on a triangular (\ref{eqn:condition_for_chi})
plane (\ref{eqn:condition_for_p1}) in 3 dimensions (\ref{eqn:condition_for_p4}).
Using this formulation it can be seen that a necessary (but not sufficient)
condition for an extortionate strategy is that it cooperates on average less
than 50\% of the time when in a state of disagreement with the opponent.

As an example, consider the known extortionate strategy \(p=(8 / 9, 1 / 2, 1 /
3, 0)\) from~\cite{Stewart2012} which is referred to as \texttt{Extort-2}. In
this case, for the standard values of \((R, T, S, P)\) constraint
(\ref{eqn:condition_for_p1}) corresponds to:

\begin{equation}
    p_1 = \frac{2(p_2 + p_3) + 1}{3}
\end{equation}

It is clear that in this case all constraints hold.

This approach could in fact be used to confirm that a given strategy is acting
in an extortionate manner even if it is not a memory one strategy. However, in
practice, if a closed form for \(p\) is not known, then due to measurement
and/or numerical error this would not work.

This problem can be written in the following linear algebraic form where
\(x=(\alpha, \beta)\)
and \(p^*=(\tilde p_1 - 1, tilde_2 - 1, p_3)\):

\begin{equation}\label{eqn:linear_algebraic_equation_for_p}
    Cx= p^*
\end{equation}

\(C\) corresponds to equations
(\ref{eqn:condition_for_tilde_p1}-\ref{eqn:condition_for_tilde_p3}) and is
given by:

\begin{equation}\label{eqn:definition_of_C}
    C =
    \begin{bmatrix}
        R - P & R- P \\
        S - P & T- P \\
        T - P & S- P \\
    \end{bmatrix}
\end{equation}

Note that in general, equation (\ref{eqn:linear_algebraic_equation_for_p}) will
not necessarily have a solution. From the Rouch\'{e}-Capelli theorem if there is
a solution it is unique as \(\text{rank}(C)=2\) which is the dimension of the
variable \(x\). The best fitting \(x\) is found by minimizing:

\begin{equation}\label{eqn:r_squared}
    \text{SSError} = \|C x- p^*\|_2^2 = \sum_{i=1}^{3}\left((C\bar x)_i-p_i^*\right)^2
\end{equation}

Note that \(\text{SSError}\), which is the square of the Frobenius
norm~\cite{Golub2013}, becomes a measure of how close a strategy is to being an
extortionate strategy. Suspicion
of extortion then corresponds to a threshold on \(\text{SSError}\).

By observing interactions (human or otherwise), their memory one representation
can be inferred and this approach can be used to recognise extortionate
behaviour. The notion of comparing theoretic and actual plays of the IPD is not
novel, see for example~\cite{Rand2013}. Immediately it is noted that if the
environment is noisy~\cite{Wu1995} then no strategy can be considered to be
extortionate as \(p_4>0\).

In the next section, this idea will be illustrated by observing the interactions
that take place in a computer based tournament of the IPD\@.

\section{Numerical experiments}\label{sec:numerical-experiments}

In~\cite{Stewart2012} results from a tournament with
\input{./assets/tex/number_of_stewart_plotkin_strategies/main.tex} strategies,
was presented with specific consideration given to ZD strategies. This
tournament is reproduced here using the Axelrod-Python
project~\cite{Knight2016}. To obtain a good measure of the corresponding
transition rates for each strategy all matches have been run for
\input{assets/tex/number_of_turns/main.tex} turns and every match has been
repeated \input{assets/tex/number_of_repetitions/main.tex} times. All of this
interaction data is available at~\cite{vincent_knight_2018_1297075}. A good
match between the inferred Markov chain and the state distribution of the actual
interactions has been verified. Data for this is presented in the supplementary
materials.

Figure~\ref{fig:SSError_overall_in_stewart_plotkin} shows the \(\text{SSError}\)
values for all the strategies in the tournament, as reported
in~\cite{Stewart2012} the extortionate strategy (which has an expected
\(\text{SSError}\) approximately 0) gains a large number of wins.

\begin{figure}[!htbp]
    \centering
    \includegraphics[width=.8\textwidth]{./assets/img/SSError_overall_in_stewart_plotkin/main.pdf}
    \caption{\(\text{SSError}\) and state probabilities for the strategies
        of~\cite{Stewart2012}, ordered both by number of wins and overall score.
        Note that \(P(DC)\) is not shown as it corresponds to the transpose of
        \(P(CD)\). Cooperator and Defector are omitted as they do not visit all
        the states.}
    \label{fig:SSError_overall_in_stewart_plotkin}
\end{figure}

Here, the work of~\cite{Stewart2012} is extended by investigating a tournament
with \input{assets/tex/number_of_full_strategies/main.tex}
strategies.

The results of this analysis are shown in
Figure~\ref{fig:SSError_and_probabilities_in_full}. The top ranking strategies
by number of wins seem to be extortionate (but not against all strategies) and
it can be seen that a small sub group of strategies achieve mutual defection.
All the top ranking strategies according to score achieve mutual cooperation and
do not extort each other, however they
\textbf{do} exhibit extortionate behaviour towards a number of the lower ranking
strategies.

\begin{figure}[!htbp]
    \centering
    \includegraphics[width=.8\textwidth]{./assets/img/SSError_and_probabilities_in_full/main.pdf}
    \caption{\(\text{SSError}\) for the strategies for the full tournament. Only
    strategy interactions for which \(p_4=0\) and \(\chi>1\) are displayed.}
    \label{fig:SSError_and_probabilities_in_full}
\end{figure}

\section{Conclusion}\label{sec:conclusion}

This work defines an approach to measure whether or not a player is playing a
strategy that corresponds to an extortionate strategy as defined
in~\cite{Press2012}: a mathematical model for suspicion. Indeed, all
extortionate strategies have been
 classified as lying on a triangular plane.
This rigorous classification fails to be robust to small measurement error, thus
a statistical approach is proposed.
This is done through a linear algebraic approach for approximating the solution
of a linear system. Using this, a large number of pairwise interactions is
simulated and in fact very few strategies are found to act extortionately.

The work of~\cite{Press2012}, whilst showing that a clever approach to taking
advantage of another memory one strategy exists: this is incomplete. Whilst the
elegance of this result is very attractive, just as the simplicity of the
victory of Tit For Tat in Axelrod's original tournaments was, it is incomplete.
Extortionate strategies achieve a high number of wins but they do not
achieve a high score which corresponds to the fitness landscape in an
evolutionary sense. From the large number of interactions a payoff matrix \(S\)
can be measured where \(S_{ij}\) denotes the score (using standard values of
\((R, S, T, P) = (3, 0, 5, 1)\)) of the \(i\)th strategy
against the \(j\)th strategy. Using this, the replicator equation
describes the evolution of the system based on a population density fitness
function:

\begin{equation}\label{eqn:replicator_dynamics}
    \frac{dx}{dt} = x(S-x^TS x)
\end{equation}

Equation (\ref{eqn:replicator_dynamics}) is solved numerically through an
integration technique described in~\cite{Petzold1983} and
Figure~\ref{fig:replicator_dynamics} shows the evolution of the distribution of
the system: the various strategies are ranked by scores. It is clear to see that
only the high ranking strategies survive the evolutionary process (in fact,
only \input{./assets/img/replicator_dynamics/main.tex}
have a final distribution greater than \(10 ^ {-2}\)). This confirms the
findings of~\cite{Moran1707} in which sophisticated strategies resist
evolutionary invasion of shorter memory strategies. Recalling
Figure~\ref{fig:SSError_and_probabilities_in_full} this demonstrates that:

\begin{itemize}
    \item Cooperation emerges through the evolutionary process: the high scoring
        strategies do not exhibit extortionate behaviour towards each other.
    \item Extortionate strategies do not survive the evolutionary process.
\end{itemize}

\begin{figure}[!htbp]
    \centering
    \includegraphics[width=.8\textwidth]{./assets/img/replicator_dynamics/main.pdf}
    \caption{Numerical simulation of the replicator equation
    (\ref{eqn:replicator_dynamics}): strategies are ordered by score, only the strategies with a high score survive the evolutionary process.}
    \label{fig:replicator_dynamics}
\end{figure}

This work can be used to classify plays of the IPD\@: data can be collected from
actual interactions (in lab or in the field). Furthermore, this allows for a
classification method similar to the notion of fingerprinting presented
in~\cite{Ashlock2008}. Trained strategies can potentially be classified as
extortionate or not or it could be possible to even constrain the reinforcement
learning approaches that are becoming prevalent in the literature.
Alternatively, this mathematical approach for recognising extortion could be
used in sophisticated strategies to defend against invasion. Arguably, some of
the strategies considered here exhibit this behaviour, indeed as described
in~\cite{Harper2017}, the top ranking strategies in the full tournament are
obtained using evolutionary reinforcement learning techniques, thus, suspicion
of extortionate behaviour could in fact be an evolutionary trait.

\section*{Acknowledgements}

The following open source software libraries were used in this research:

\begin{itemize}
    \item The Axelrod ~\cite{Knight2016, Knight2018} library (IPD strategies and
        tournaments).
    \item The sympy library~\cite{Meurer2017} (verification of all symbolic
        calculations).
    \item The matplotlib~\cite{Droettboom2018} library (visualisation).
    \item The pandas~\cite{Structures2010}, dask~\cite{Dask2016} and
        NumPy~\cite{Oliphant2015} libraries (data manipulation).
    \item The SciPy~\cite{Jones2001} library (numerical integration of the
        replicator equation).
\end{itemize}

This work was performed using the computational facilities of the Advanced
Research Computing @ Cardiff (ARCCA) Division, Cardiff University.

\printbibliography

\newpage
\section*{Supplementary materials}

\includepdf{assets/pdf/proof_of_form_of_extortionate_strategies/main.pdf}

\newpage

Using the pair wise interactions the transition rates \(p,
q\) can be measured and the steady state probabilities inferred and compared to
the actual probabilities of each state.
This is done numerically by computing the singular eigenvector of the
matrix \(A\) \cite{Stewart2009}:

\[
    A =
    \begin{bmatrix}
        p_1 q_1 & p_1 (1 - q_1) & (1 - p_1) q_1 & (1 -p_1) (1 - q_1) \\
        p_2 q_2 & p_2 (1 - q_2) & (1 - p_2) q_2 & (1 -p_2) (1 - q_2) \\
        p_3 q_3 & p_3 (1 - q_3) & (1 - p_3) q_3 & (1 -p_3) (1 - q_3) \\
        p_4 q_4 & p_4 (1 - q_4) & (1 - p_4) q_4 & (1 -p_4) (1 - q_4) \\
    \end{bmatrix}
\]

Figure~\ref{fig:computed_probabilities_vs_theoretic_probabilities} shows a
regression line fitted to every pairwise interaction with a reported
\(\text{SSError}\) value (pairwise interactions with missing states were
omitted). This serves to validate the approach: a part from some edge cases the
relationship is consistent.

\begin{figure}[!htbp]
    \centering
    \includegraphics[width=.8\textwidth]{./assets/img/computed_probabilities_vs_theoretic_probabilities/main.pdf}
    \caption{The
        relationship between the steady state probabilities inferred from the
        measured transitions and the actual steady state probabilities. A linear
        regression line is included validating the approach.}
    \label{fig:computed_probabilities_vs_theoretic_probabilities}
\end{figure}


\end{document}

have a final distribution greater than \(10 ^ {-2}\)). This confirms the
findings of~\cite{Moran1707} in which sophisticated strategies resist
evolutionary invasion of shorter memory strategies. Recalling
Figure~\ref{fig:SSError_and_probabilities_in_full} this demonstrates that:

\begin{itemize}
    \item Cooperation emerges through the evolutionary process: the high scoring
        strategies do not exhibit extortionate behaviour towards each other.
    \item Extortionate strategies do not survive the evolutionary process.
\end{itemize}

\begin{figure}[!htbp]
    \centering
    \includegraphics[width=.8\textwidth]{./assets/img/replicator_dynamics/main.pdf}
    \caption{Numerical simulation of the replicator equation
    (\ref{eqn:replicator_dynamics}): strategies are ordered by score, only the strategies with a high score survive the evolutionary process.}
    \label{fig:replicator_dynamics}
\end{figure}

This work can be used to classify plays of the IPD\@: data can be collected from
actual interactions (in lab or in the field). Furthermore, this allows for a
classification method similar to the notion of fingerprinting presented
in~\cite{Ashlock2008}. Trained strategies can potentially be classified as
extortionate or not or it could be possible to even constrain the reinforcement
learning approaches that are becoming prevalent in the literature.
Alternatively, this mathematical approach for recognising extortion could be
used in sophisticated strategies to defend against invasion. Arguably, some of
the strategies considered here exhibit this behaviour, indeed as described
in~\cite{Harper2017}, the top ranking strategies in the full tournament are
obtained using evolutionary reinforcement learning techniques, thus, suspicion
of extortionate behaviour could in fact be an evolutionary trait.

\section*{Acknowledgements}

The following open source software libraries were used in this research:

\begin{itemize}
    \item The Axelrod ~\cite{Knight2016, Knight2018} library (IPD strategies and
        tournaments).
    \item The sympy library~\cite{Meurer2017} (verification of all symbolic
        calculations).
    \item The matplotlib~\cite{Droettboom2018} library (visualisation).
    \item The pandas~\cite{Structures2010}, dask~\cite{Dask2016} and
        NumPy~\cite{Oliphant2015} libraries (data manipulation).
    \item The SciPy~\cite{Jones2001} library (numerical integration of the
        replicator equation).
\end{itemize}

This work was performed using the computational facilities of the Advanced
Research Computing @ Cardiff (ARCCA) Division, Cardiff University.

\printbibliography

\newpage
\section*{Supplementary materials}

\includepdf{assets/pdf/proof_of_form_of_extortionate_strategies/main.pdf}

\newpage

Using the pair wise interactions the transition rates \(p,
q\) can be measured and the steady state probabilities inferred and compared to
the actual probabilities of each state.
This is done numerically by computing the singular eigenvector of the
matrix \(A\) \cite{Stewart2009}:

\[
    A =
    \begin{bmatrix}
        p_1 q_1 & p_1 (1 - q_1) & (1 - p_1) q_1 & (1 -p_1) (1 - q_1) \\
        p_2 q_2 & p_2 (1 - q_2) & (1 - p_2) q_2 & (1 -p_2) (1 - q_2) \\
        p_3 q_3 & p_3 (1 - q_3) & (1 - p_3) q_3 & (1 -p_3) (1 - q_3) \\
        p_4 q_4 & p_4 (1 - q_4) & (1 - p_4) q_4 & (1 -p_4) (1 - q_4) \\
    \end{bmatrix}
\]

Figure~\ref{fig:computed_probabilities_vs_theoretic_probabilities} shows a
regression line fitted to every pairwise interaction with a reported
\(\text{SSError}\) value (pairwise interactions with missing states were
omitted). This serves to validate the approach: a part from some edge cases the
relationship is consistent.

\begin{figure}[!htbp]
    \centering
    \includegraphics[width=.8\textwidth]{./assets/img/computed_probabilities_vs_theoretic_probabilities/main.pdf}
    \caption{The
        relationship between the steady state probabilities inferred from the
        measured transitions and the actual steady state probabilities. A linear
        regression line is included validating the approach.}
    \label{fig:computed_probabilities_vs_theoretic_probabilities}
\end{figure}


\end{document}

    strategies is considered. In this setting
    the most highly performing strategies do not play in an extortionate way
    against each other but do against lower performing strategies.
    This suggests that whilst the theory of Zero Determinant strategies
    indicates that memory is not of fundamental importance to the evolution of
    cooperative behaviour, this is incomplete.
\end{abstract}

\section{Introduction}\label{sec:introduction}

Agent based game theoretic models have become a stalwart of the underpinning
mathematics of interactive behaviours. One of the major pieces of work
in this area is the pair of original computer tournaments run by Robert
Axelrod~\cite{Axelrod1980, Axelrod1980a}. These tournaments pitted submitted
computer strategies against each other in plays of the Iterated Prisoner's
Dilemma. A common game where agents can choose to pay a slight cost to their
immediate utility in the hope of building a reputation. This has been used in
economic and evolutionary game theory to understand the evolution of cooperative
behaviour.

Recently, a class of strategies was described in~\cite{Press2012} that can
provably extort any given opponent. In~\cite{Hilbe2013, Moran1707} some
questions have already been asked about the true effectiveness of these
strategies in an evolutionary setting. Here another question is asked: is it
possible to recognise this extortionate behaviour? A mathematical procedure for
suspicion is presented: in the same way that the continued actions of an
extortionate individual might raise suspicion.

This work makes use of the Axelrod Python library~\cite{Knight2018, Knight2016}
with a large number of Prisoner Dilemma strategies available to give an
extensive numerical example of the ideas presented.  The approach is presented
in Section~\ref{sec:delta-zd-strategies}.  All of the code and data discussed
in Section~\ref{sec:numerical-experiments} is open sourced, archived and
written according to best scientific principles~\cite{Wilson2014}. The data
archive can be found at~\cite{vincent_knight_2018_1297075}.

\section{Recognising Extortion}\label{sec:delta-zd-strategies}

In~\cite{Press2012}, given a match between 2 memory-one strategies, the concept
of Zero Determinant (ZD) strategies is introduced. The main result of that paper
shows that given two memory one players \(p, q\in\mathbb{R}^4\) a linear
relationship between the players' scores could be forced by one of the players.

Using the notation of~\cite{Press2012}, assuming the utilities for player \(p\)
are given by \(S_x=(R, S, T, P)\) and for player \(q\) by \(S_y=(R, T, S, P)\)
and that the stationary scores of each player is given by \(S_X\) and \(S_Y\)
respectively. The main result of~\cite{Press2012} is that if

\begin{equation}\label{eqn:linear_relationship_for_p}
    \tilde p=\alpha S_x + \beta S_y + \gamma
\end{equation}

or

\begin{equation}\label{eqn:linear_relationship_for_q}
    \tilde q=\alpha S_x + \beta S_y + \gamma
\end{equation}

where \(\tilde p = (1 - p_1, 1 - p_2, p_3, p_4)\) and
\(\tilde q = (1 - q_1, 1 - q_2, q_3, q_4)\) then:

\begin{equation}
    \alpha S_X + \beta S_Y + \gamma = 0
\end{equation}

In~\cite{Press2012} a particular type of ZD strategy is defined: extortionate
strategies. If:

\begin{equation}\label{eqn:constraint_for_extortion}
    \gamma = - P(\alpha + \beta)
\end{equation}

then the player can ensure they get a score \(\chi\) times
larger than the opponent. This extortion coefficient is given by:

\begin{equation}\label{eqn:definition_of_chi}
    \chi=\frac{-\beta}{\alpha}
\end{equation}

Thus, if (\ref{eqn:constraint_for_extortion}) holds and \(\chi >1\) a player is
said to extort their opponent.
Here, the reverse problem is considered: given a
\(p\in\mathbb{R}^4\) how does one identify \(\alpha, \beta\) if they
exist and is the strategy in fact acting in an extortionate way?

These conditions correspond to:

\begin{align}
    \tilde p_1 & = \alpha R + \beta R - P (\alpha + \beta)
            \label{eqn:condition_for_tilde_p1}\\
    \tilde p_2 & = \alpha S + \beta T - P (\alpha + \beta)
            \label{eqn:condition_for_tilde_p2}\\
    \tilde p_3 & = \alpha T + \beta S - P (\alpha + \beta)
            \label{eqn:condition_for_tilde_p3}\\
    \tilde p_4 & = \alpha P + \beta P - P (\alpha + \beta)
            \label{eqn:condition_for_tilde_p4}
\end{align}

Equation (\ref{eqn:condition_for_tilde_p4}) ensures that \(p_4=\tilde p_4=0\).
Equations (\ref{eqn:condition_for_tilde_p1}-\ref{eqn:condition_for_tilde_p3})
can be used to eliminate \(\alpha, \beta\), giving:

\begin{equation}\label{eqn:planar_definition_of_extortion}
    \tilde p_1 = \frac{(R - P)(\tilde p_2 + \tilde p_3)}{S + T - 2P}
\end{equation}

with:

\begin{equation}\label{eqn:definition_of_chi}
    \chi = \frac{\tilde p_2 (P - T) + \tilde p_3 (S - P)}
                {\tilde p_2 (P - S) + \tilde p_3 (T - P)}
\end{equation}

Given a strategy \(p\in\mathbb{R}^{4\times 1}\) equations
(\ref{eqn:condition_for_tilde_p4}), (\ref{eqn:planar_definition_of_extortion}-\ref{eqn:definition_of_chi}) can be used to check if
a strategy is extortionate. The conditions correspond to:

\begin{align}
    p_1 & = \frac{(R-P)(p_2 + p_3) - R + T + S - P}{S + T - 2P}
     \label{eqn:condition_for_p1}\\
    p_4 & = 0 \label{eqn:condition_for_p4}\\
    1 & > p_2 + p_3\label{eqn:condition_for_chi}
\end{align}

The algebraic steps necessary to prove these results are available in the
supporting materials.

All extortionate strategies reside on a triangular (\ref{eqn:condition_for_chi})
plane (\ref{eqn:condition_for_p1}) in 3 dimensions (\ref{eqn:condition_for_p4}).
Using this formulation it can be seen that a necessary (but not sufficient)
condition for an extortionate strategy is that it cooperates on average less
than 50\% of the time when in a state of disagreement with the opponent.

As an example, consider the known extortionate strategy \(p=(8 / 9, 1 / 2, 1 /
3, 0)\) from~\cite{Stewart2012} which is referred to as \texttt{Extort-2}. In
this case, for the standard values of \((R, T, S, P)\) constraint
(\ref{eqn:condition_for_p1}) corresponds to:

\begin{equation}
    p_1 = \frac{2(p_2 + p_3) + 1}{3}
\end{equation}

It is clear that in this case all constraints hold.

This approach could in fact be used to confirm that a given strategy is acting
in an extortionate manner even if it is not a memory one strategy. However, in
practice, if a closed form for \(p\) is not known, then due to measurement
and/or numerical error this would not work.

This problem can be written in the following linear algebraic form where
\(x=(\alpha, \beta)\)
and \(p^*=(\tilde p_1 - 1, tilde_2 - 1, p_3)\):

\begin{equation}\label{eqn:linear_algebraic_equation_for_p}
    Cx= p^*
\end{equation}

\(C\) corresponds to equations
(\ref{eqn:condition_for_tilde_p1}-\ref{eqn:condition_for_tilde_p3}) and is
given by:

\begin{equation}\label{eqn:definition_of_C}
    C =
    \begin{bmatrix}
        R - P & R- P \\
        S - P & T- P \\
        T - P & S- P \\
    \end{bmatrix}
\end{equation}

Note that in general, equation (\ref{eqn:linear_algebraic_equation_for_p}) will
not necessarily have a solution. From the Rouch\'{e}-Capelli theorem if there is
a solution it is unique as \(\text{rank}(C)=2\) which is the dimension of the
variable \(x\). The best fitting \(x\) is found by minimizing:

\begin{equation}\label{eqn:r_squared}
    \text{SSError} = \|C x- p^*\|_2^2 = \sum_{i=1}^{3}\left((C\bar x)_i-p_i^*\right)^2
\end{equation}

Note that \(\text{SSError}\), which is the square of the Frobenius
norm~\cite{Golub2013}, becomes a measure of how close a strategy is to being an
extortionate strategy. Suspicion
of extortion then corresponds to a threshold on \(\text{SSError}\).

By observing interactions (human or otherwise), their memory one representation
can be inferred and this approach can be used to recognise extortionate
behaviour. The notion of comparing theoretic and actual plays of the IPD is not
novel, see for example~\cite{Rand2013}. Immediately it is noted that if the
environment is noisy~\cite{Wu1995} then no strategy can be considered to be
extortionate as \(p_4>0\).

In the next section, this idea will be illustrated by observing the interactions
that take place in a computer based tournament of the IPD\@.

\section{Numerical experiments}\label{sec:numerical-experiments}

In~\cite{Stewart2012} results from a tournament with
\documentclass[a4paper]{article}

\usepackage{amsmath}
\usepackage{amssymb}
\usepackage[margin=1.5cm,
            includefoot,
            footskip=30pt]{geometry}
\usepackage{layout}
\usepackage{graphicx}
\usepackage{subcaption}

\usepackage{biblatex}
\usepackage{pdfpages}

\bibliography{main.bib}

\title{Suspicion: Recognising and evaluating the effectiveness
       of extortion in the Iterated Prisoner's Dilemma}
\author{Vincent A. Knight \and Nikoleta E. Glynatsi}
\date{\today}



\begin{document}

\maketitle

\begin{abstract}
    The Iterated Prisoner's Dilemma is a model for rational and evolutionary
    interactive behaviour. It has applications both in the study of human social
    behaviour as well as in biology.
    It is used to understand when and how a rational individual might
    accept an immediate cost to their own utility for the direct benefit of
    another.

    Much attention has been given to a class of strategies called
    Zero Determinant strategies. It has been theoretically shown that these
    strategies can ``extort'' any player.

    In this work, an approach to identify if observed strategies are playing in
    an extortionate way is described. Furthermore, experimental analysis of
    a large tournament with \documentclass[a4paper]{article}

\usepackage{amsmath}
\usepackage{amssymb}
\usepackage[margin=1.5cm,
            includefoot,
            footskip=30pt]{geometry}
\usepackage{layout}
\usepackage{graphicx}
\usepackage{subcaption}

\usepackage{biblatex}
\usepackage{pdfpages}

\bibliography{main.bib}

\title{Suspicion: Recognising and evaluating the effectiveness
       of extortion in the Iterated Prisoner's Dilemma}
\author{Vincent A. Knight \and Nikoleta E. Glynatsi}
\date{\today}



\begin{document}

\maketitle

\begin{abstract}
    The Iterated Prisoner's Dilemma is a model for rational and evolutionary
    interactive behaviour. It has applications both in the study of human social
    behaviour as well as in biology.
    It is used to understand when and how a rational individual might
    accept an immediate cost to their own utility for the direct benefit of
    another.

    Much attention has been given to a class of strategies called
    Zero Determinant strategies. It has been theoretically shown that these
    strategies can ``extort'' any player.

    In this work, an approach to identify if observed strategies are playing in
    an extortionate way is described. Furthermore, experimental analysis of
    a large tournament with \input{assets/tex/number_of_full_strategies/main.tex}
    strategies is considered. In this setting
    the most highly performing strategies do not play in an extortionate way
    against each other but do against lower performing strategies.
    This suggests that whilst the theory of Zero Determinant strategies
    indicates that memory is not of fundamental importance to the evolution of
    cooperative behaviour, this is incomplete.
\end{abstract}

\section{Introduction}\label{sec:introduction}

Agent based game theoretic models have become a stalwart of the underpinning
mathematics of interactive behaviours. One of the major pieces of work
in this area is the pair of original computer tournaments run by Robert
Axelrod~\cite{Axelrod1980, Axelrod1980a}. These tournaments pitted submitted
computer strategies against each other in plays of the Iterated Prisoner's
Dilemma. A common game where agents can choose to pay a slight cost to their
immediate utility in the hope of building a reputation. This has been used in
economic and evolutionary game theory to understand the evolution of cooperative
behaviour.

Recently, a class of strategies was described in~\cite{Press2012} that can
provably extort any given opponent. In~\cite{Hilbe2013, Moran1707} some
questions have already been asked about the true effectiveness of these
strategies in an evolutionary setting. Here another question is asked: is it
possible to recognise this extortionate behaviour? A mathematical procedure for
suspicion is presented: in the same way that the continued actions of an
extortionate individual might raise suspicion.

This work makes use of the Axelrod Python library~\cite{Knight2018, Knight2016}
with a large number of Prisoner Dilemma strategies available to give an
extensive numerical example of the ideas presented.  The approach is presented
in Section~\ref{sec:delta-zd-strategies}.  All of the code and data discussed
in Section~\ref{sec:numerical-experiments} is open sourced, archived and
written according to best scientific principles~\cite{Wilson2014}. The data
archive can be found at~\cite{vincent_knight_2018_1297075}.

\section{Recognising Extortion}\label{sec:delta-zd-strategies}

In~\cite{Press2012}, given a match between 2 memory-one strategies, the concept
of Zero Determinant (ZD) strategies is introduced. The main result of that paper
shows that given two memory one players \(p, q\in\mathbb{R}^4\) a linear
relationship between the players' scores could be forced by one of the players.

Using the notation of~\cite{Press2012}, assuming the utilities for player \(p\)
are given by \(S_x=(R, S, T, P)\) and for player \(q\) by \(S_y=(R, T, S, P)\)
and that the stationary scores of each player is given by \(S_X\) and \(S_Y\)
respectively. The main result of~\cite{Press2012} is that if

\begin{equation}\label{eqn:linear_relationship_for_p}
    \tilde p=\alpha S_x + \beta S_y + \gamma
\end{equation}

or

\begin{equation}\label{eqn:linear_relationship_for_q}
    \tilde q=\alpha S_x + \beta S_y + \gamma
\end{equation}

where \(\tilde p = (1 - p_1, 1 - p_2, p_3, p_4)\) and
\(\tilde q = (1 - q_1, 1 - q_2, q_3, q_4)\) then:

\begin{equation}
    \alpha S_X + \beta S_Y + \gamma = 0
\end{equation}

In~\cite{Press2012} a particular type of ZD strategy is defined: extortionate
strategies. If:

\begin{equation}\label{eqn:constraint_for_extortion}
    \gamma = - P(\alpha + \beta)
\end{equation}

then the player can ensure they get a score \(\chi\) times
larger than the opponent. This extortion coefficient is given by:

\begin{equation}\label{eqn:definition_of_chi}
    \chi=\frac{-\beta}{\alpha}
\end{equation}

Thus, if (\ref{eqn:constraint_for_extortion}) holds and \(\chi >1\) a player is
said to extort their opponent.
Here, the reverse problem is considered: given a
\(p\in\mathbb{R}^4\) how does one identify \(\alpha, \beta\) if they
exist and is the strategy in fact acting in an extortionate way?

These conditions correspond to:

\begin{align}
    \tilde p_1 & = \alpha R + \beta R - P (\alpha + \beta)
            \label{eqn:condition_for_tilde_p1}\\
    \tilde p_2 & = \alpha S + \beta T - P (\alpha + \beta)
            \label{eqn:condition_for_tilde_p2}\\
    \tilde p_3 & = \alpha T + \beta S - P (\alpha + \beta)
            \label{eqn:condition_for_tilde_p3}\\
    \tilde p_4 & = \alpha P + \beta P - P (\alpha + \beta)
            \label{eqn:condition_for_tilde_p4}
\end{align}

Equation (\ref{eqn:condition_for_tilde_p4}) ensures that \(p_4=\tilde p_4=0\).
Equations (\ref{eqn:condition_for_tilde_p1}-\ref{eqn:condition_for_tilde_p3})
can be used to eliminate \(\alpha, \beta\), giving:

\begin{equation}\label{eqn:planar_definition_of_extortion}
    \tilde p_1 = \frac{(R - P)(\tilde p_2 + \tilde p_3)}{S + T - 2P}
\end{equation}

with:

\begin{equation}\label{eqn:definition_of_chi}
    \chi = \frac{\tilde p_2 (P - T) + \tilde p_3 (S - P)}
                {\tilde p_2 (P - S) + \tilde p_3 (T - P)}
\end{equation}

Given a strategy \(p\in\mathbb{R}^{4\times 1}\) equations
(\ref{eqn:condition_for_tilde_p4}), (\ref{eqn:planar_definition_of_extortion}-\ref{eqn:definition_of_chi}) can be used to check if
a strategy is extortionate. The conditions correspond to:

\begin{align}
    p_1 & = \frac{(R-P)(p_2 + p_3) - R + T + S - P}{S + T - 2P}
     \label{eqn:condition_for_p1}\\
    p_4 & = 0 \label{eqn:condition_for_p4}\\
    1 & > p_2 + p_3\label{eqn:condition_for_chi}
\end{align}

The algebraic steps necessary to prove these results are available in the
supporting materials.

All extortionate strategies reside on a triangular (\ref{eqn:condition_for_chi})
plane (\ref{eqn:condition_for_p1}) in 3 dimensions (\ref{eqn:condition_for_p4}).
Using this formulation it can be seen that a necessary (but not sufficient)
condition for an extortionate strategy is that it cooperates on average less
than 50\% of the time when in a state of disagreement with the opponent.

As an example, consider the known extortionate strategy \(p=(8 / 9, 1 / 2, 1 /
3, 0)\) from~\cite{Stewart2012} which is referred to as \texttt{Extort-2}. In
this case, for the standard values of \((R, T, S, P)\) constraint
(\ref{eqn:condition_for_p1}) corresponds to:

\begin{equation}
    p_1 = \frac{2(p_2 + p_3) + 1}{3}
\end{equation}

It is clear that in this case all constraints hold.

This approach could in fact be used to confirm that a given strategy is acting
in an extortionate manner even if it is not a memory one strategy. However, in
practice, if a closed form for \(p\) is not known, then due to measurement
and/or numerical error this would not work.

This problem can be written in the following linear algebraic form where
\(x=(\alpha, \beta)\)
and \(p^*=(\tilde p_1 - 1, tilde_2 - 1, p_3)\):

\begin{equation}\label{eqn:linear_algebraic_equation_for_p}
    Cx= p^*
\end{equation}

\(C\) corresponds to equations
(\ref{eqn:condition_for_tilde_p1}-\ref{eqn:condition_for_tilde_p3}) and is
given by:

\begin{equation}\label{eqn:definition_of_C}
    C =
    \begin{bmatrix}
        R - P & R- P \\
        S - P & T- P \\
        T - P & S- P \\
    \end{bmatrix}
\end{equation}

Note that in general, equation (\ref{eqn:linear_algebraic_equation_for_p}) will
not necessarily have a solution. From the Rouch\'{e}-Capelli theorem if there is
a solution it is unique as \(\text{rank}(C)=2\) which is the dimension of the
variable \(x\). The best fitting \(x\) is found by minimizing:

\begin{equation}\label{eqn:r_squared}
    \text{SSError} = \|C x- p^*\|_2^2 = \sum_{i=1}^{3}\left((C\bar x)_i-p_i^*\right)^2
\end{equation}

Note that \(\text{SSError}\), which is the square of the Frobenius
norm~\cite{Golub2013}, becomes a measure of how close a strategy is to being an
extortionate strategy. Suspicion
of extortion then corresponds to a threshold on \(\text{SSError}\).

By observing interactions (human or otherwise), their memory one representation
can be inferred and this approach can be used to recognise extortionate
behaviour. The notion of comparing theoretic and actual plays of the IPD is not
novel, see for example~\cite{Rand2013}. Immediately it is noted that if the
environment is noisy~\cite{Wu1995} then no strategy can be considered to be
extortionate as \(p_4>0\).

In the next section, this idea will be illustrated by observing the interactions
that take place in a computer based tournament of the IPD\@.

\section{Numerical experiments}\label{sec:numerical-experiments}

In~\cite{Stewart2012} results from a tournament with
\input{./assets/tex/number_of_stewart_plotkin_strategies/main.tex} strategies,
was presented with specific consideration given to ZD strategies. This
tournament is reproduced here using the Axelrod-Python
project~\cite{Knight2016}. To obtain a good measure of the corresponding
transition rates for each strategy all matches have been run for
\input{assets/tex/number_of_turns/main.tex} turns and every match has been
repeated \input{assets/tex/number_of_repetitions/main.tex} times. All of this
interaction data is available at~\cite{vincent_knight_2018_1297075}. A good
match between the inferred Markov chain and the state distribution of the actual
interactions has been verified. Data for this is presented in the supplementary
materials.

Figure~\ref{fig:SSError_overall_in_stewart_plotkin} shows the \(\text{SSError}\)
values for all the strategies in the tournament, as reported
in~\cite{Stewart2012} the extortionate strategy (which has an expected
\(\text{SSError}\) approximately 0) gains a large number of wins.

\begin{figure}[!htbp]
    \centering
    \includegraphics[width=.8\textwidth]{./assets/img/SSError_overall_in_stewart_plotkin/main.pdf}
    \caption{\(\text{SSError}\) and state probabilities for the strategies
        of~\cite{Stewart2012}, ordered both by number of wins and overall score.
        Note that \(P(DC)\) is not shown as it corresponds to the transpose of
        \(P(CD)\). Cooperator and Defector are omitted as they do not visit all
        the states.}
    \label{fig:SSError_overall_in_stewart_plotkin}
\end{figure}

Here, the work of~\cite{Stewart2012} is extended by investigating a tournament
with \input{assets/tex/number_of_full_strategies/main.tex}
strategies.

The results of this analysis are shown in
Figure~\ref{fig:SSError_and_probabilities_in_full}. The top ranking strategies
by number of wins seem to be extortionate (but not against all strategies) and
it can be seen that a small sub group of strategies achieve mutual defection.
All the top ranking strategies according to score achieve mutual cooperation and
do not extort each other, however they
\textbf{do} exhibit extortionate behaviour towards a number of the lower ranking
strategies.

\begin{figure}[!htbp]
    \centering
    \includegraphics[width=.8\textwidth]{./assets/img/SSError_and_probabilities_in_full/main.pdf}
    \caption{\(\text{SSError}\) for the strategies for the full tournament. Only
    strategy interactions for which \(p_4=0\) and \(\chi>1\) are displayed.}
    \label{fig:SSError_and_probabilities_in_full}
\end{figure}

\section{Conclusion}\label{sec:conclusion}

This work defines an approach to measure whether or not a player is playing a
strategy that corresponds to an extortionate strategy as defined
in~\cite{Press2012}: a mathematical model for suspicion. Indeed, all
extortionate strategies have been
 classified as lying on a triangular plane.
This rigorous classification fails to be robust to small measurement error, thus
a statistical approach is proposed.
This is done through a linear algebraic approach for approximating the solution
of a linear system. Using this, a large number of pairwise interactions is
simulated and in fact very few strategies are found to act extortionately.

The work of~\cite{Press2012}, whilst showing that a clever approach to taking
advantage of another memory one strategy exists: this is incomplete. Whilst the
elegance of this result is very attractive, just as the simplicity of the
victory of Tit For Tat in Axelrod's original tournaments was, it is incomplete.
Extortionate strategies achieve a high number of wins but they do not
achieve a high score which corresponds to the fitness landscape in an
evolutionary sense. From the large number of interactions a payoff matrix \(S\)
can be measured where \(S_{ij}\) denotes the score (using standard values of
\((R, S, T, P) = (3, 0, 5, 1)\)) of the \(i\)th strategy
against the \(j\)th strategy. Using this, the replicator equation
describes the evolution of the system based on a population density fitness
function:

\begin{equation}\label{eqn:replicator_dynamics}
    \frac{dx}{dt} = x(S-x^TS x)
\end{equation}

Equation (\ref{eqn:replicator_dynamics}) is solved numerically through an
integration technique described in~\cite{Petzold1983} and
Figure~\ref{fig:replicator_dynamics} shows the evolution of the distribution of
the system: the various strategies are ranked by scores. It is clear to see that
only the high ranking strategies survive the evolutionary process (in fact,
only \input{./assets/img/replicator_dynamics/main.tex}
have a final distribution greater than \(10 ^ {-2}\)). This confirms the
findings of~\cite{Moran1707} in which sophisticated strategies resist
evolutionary invasion of shorter memory strategies. Recalling
Figure~\ref{fig:SSError_and_probabilities_in_full} this demonstrates that:

\begin{itemize}
    \item Cooperation emerges through the evolutionary process: the high scoring
        strategies do not exhibit extortionate behaviour towards each other.
    \item Extortionate strategies do not survive the evolutionary process.
\end{itemize}

\begin{figure}[!htbp]
    \centering
    \includegraphics[width=.8\textwidth]{./assets/img/replicator_dynamics/main.pdf}
    \caption{Numerical simulation of the replicator equation
    (\ref{eqn:replicator_dynamics}): strategies are ordered by score, only the strategies with a high score survive the evolutionary process.}
    \label{fig:replicator_dynamics}
\end{figure}

This work can be used to classify plays of the IPD\@: data can be collected from
actual interactions (in lab or in the field). Furthermore, this allows for a
classification method similar to the notion of fingerprinting presented
in~\cite{Ashlock2008}. Trained strategies can potentially be classified as
extortionate or not or it could be possible to even constrain the reinforcement
learning approaches that are becoming prevalent in the literature.
Alternatively, this mathematical approach for recognising extortion could be
used in sophisticated strategies to defend against invasion. Arguably, some of
the strategies considered here exhibit this behaviour, indeed as described
in~\cite{Harper2017}, the top ranking strategies in the full tournament are
obtained using evolutionary reinforcement learning techniques, thus, suspicion
of extortionate behaviour could in fact be an evolutionary trait.

\section*{Acknowledgements}

The following open source software libraries were used in this research:

\begin{itemize}
    \item The Axelrod ~\cite{Knight2016, Knight2018} library (IPD strategies and
        tournaments).
    \item The sympy library~\cite{Meurer2017} (verification of all symbolic
        calculations).
    \item The matplotlib~\cite{Droettboom2018} library (visualisation).
    \item The pandas~\cite{Structures2010}, dask~\cite{Dask2016} and
        NumPy~\cite{Oliphant2015} libraries (data manipulation).
    \item The SciPy~\cite{Jones2001} library (numerical integration of the
        replicator equation).
\end{itemize}

This work was performed using the computational facilities of the Advanced
Research Computing @ Cardiff (ARCCA) Division, Cardiff University.

\printbibliography

\newpage
\section*{Supplementary materials}

\includepdf{assets/pdf/proof_of_form_of_extortionate_strategies/main.pdf}

\newpage

Using the pair wise interactions the transition rates \(p,
q\) can be measured and the steady state probabilities inferred and compared to
the actual probabilities of each state.
This is done numerically by computing the singular eigenvector of the
matrix \(A\) \cite{Stewart2009}:

\[
    A =
    \begin{bmatrix}
        p_1 q_1 & p_1 (1 - q_1) & (1 - p_1) q_1 & (1 -p_1) (1 - q_1) \\
        p_2 q_2 & p_2 (1 - q_2) & (1 - p_2) q_2 & (1 -p_2) (1 - q_2) \\
        p_3 q_3 & p_3 (1 - q_3) & (1 - p_3) q_3 & (1 -p_3) (1 - q_3) \\
        p_4 q_4 & p_4 (1 - q_4) & (1 - p_4) q_4 & (1 -p_4) (1 - q_4) \\
    \end{bmatrix}
\]

Figure~\ref{fig:computed_probabilities_vs_theoretic_probabilities} shows a
regression line fitted to every pairwise interaction with a reported
\(\text{SSError}\) value (pairwise interactions with missing states were
omitted). This serves to validate the approach: a part from some edge cases the
relationship is consistent.

\begin{figure}[!htbp]
    \centering
    \includegraphics[width=.8\textwidth]{./assets/img/computed_probabilities_vs_theoretic_probabilities/main.pdf}
    \caption{The
        relationship between the steady state probabilities inferred from the
        measured transitions and the actual steady state probabilities. A linear
        regression line is included validating the approach.}
    \label{fig:computed_probabilities_vs_theoretic_probabilities}
\end{figure}


\end{document}

    strategies is considered. In this setting
    the most highly performing strategies do not play in an extortionate way
    against each other but do against lower performing strategies.
    This suggests that whilst the theory of Zero Determinant strategies
    indicates that memory is not of fundamental importance to the evolution of
    cooperative behaviour, this is incomplete.
\end{abstract}

\section{Introduction}\label{sec:introduction}

Agent based game theoretic models have become a stalwart of the underpinning
mathematics of interactive behaviours. One of the major pieces of work
in this area is the pair of original computer tournaments run by Robert
Axelrod~\cite{Axelrod1980, Axelrod1980a}. These tournaments pitted submitted
computer strategies against each other in plays of the Iterated Prisoner's
Dilemma. A common game where agents can choose to pay a slight cost to their
immediate utility in the hope of building a reputation. This has been used in
economic and evolutionary game theory to understand the evolution of cooperative
behaviour.

Recently, a class of strategies was described in~\cite{Press2012} that can
provably extort any given opponent. In~\cite{Hilbe2013, Moran1707} some
questions have already been asked about the true effectiveness of these
strategies in an evolutionary setting. Here another question is asked: is it
possible to recognise this extortionate behaviour? A mathematical procedure for
suspicion is presented: in the same way that the continued actions of an
extortionate individual might raise suspicion.

This work makes use of the Axelrod Python library~\cite{Knight2018, Knight2016}
with a large number of Prisoner Dilemma strategies available to give an
extensive numerical example of the ideas presented.  The approach is presented
in Section~\ref{sec:delta-zd-strategies}.  All of the code and data discussed
in Section~\ref{sec:numerical-experiments} is open sourced, archived and
written according to best scientific principles~\cite{Wilson2014}. The data
archive can be found at~\cite{vincent_knight_2018_1297075}.

\section{Recognising Extortion}\label{sec:delta-zd-strategies}

In~\cite{Press2012}, given a match between 2 memory-one strategies, the concept
of Zero Determinant (ZD) strategies is introduced. The main result of that paper
shows that given two memory one players \(p, q\in\mathbb{R}^4\) a linear
relationship between the players' scores could be forced by one of the players.

Using the notation of~\cite{Press2012}, assuming the utilities for player \(p\)
are given by \(S_x=(R, S, T, P)\) and for player \(q\) by \(S_y=(R, T, S, P)\)
and that the stationary scores of each player is given by \(S_X\) and \(S_Y\)
respectively. The main result of~\cite{Press2012} is that if

\begin{equation}\label{eqn:linear_relationship_for_p}
    \tilde p=\alpha S_x + \beta S_y + \gamma
\end{equation}

or

\begin{equation}\label{eqn:linear_relationship_for_q}
    \tilde q=\alpha S_x + \beta S_y + \gamma
\end{equation}

where \(\tilde p = (1 - p_1, 1 - p_2, p_3, p_4)\) and
\(\tilde q = (1 - q_1, 1 - q_2, q_3, q_4)\) then:

\begin{equation}
    \alpha S_X + \beta S_Y + \gamma = 0
\end{equation}

In~\cite{Press2012} a particular type of ZD strategy is defined: extortionate
strategies. If:

\begin{equation}\label{eqn:constraint_for_extortion}
    \gamma = - P(\alpha + \beta)
\end{equation}

then the player can ensure they get a score \(\chi\) times
larger than the opponent. This extortion coefficient is given by:

\begin{equation}\label{eqn:definition_of_chi}
    \chi=\frac{-\beta}{\alpha}
\end{equation}

Thus, if (\ref{eqn:constraint_for_extortion}) holds and \(\chi >1\) a player is
said to extort their opponent.
Here, the reverse problem is considered: given a
\(p\in\mathbb{R}^4\) how does one identify \(\alpha, \beta\) if they
exist and is the strategy in fact acting in an extortionate way?

These conditions correspond to:

\begin{align}
    \tilde p_1 & = \alpha R + \beta R - P (\alpha + \beta)
            \label{eqn:condition_for_tilde_p1}\\
    \tilde p_2 & = \alpha S + \beta T - P (\alpha + \beta)
            \label{eqn:condition_for_tilde_p2}\\
    \tilde p_3 & = \alpha T + \beta S - P (\alpha + \beta)
            \label{eqn:condition_for_tilde_p3}\\
    \tilde p_4 & = \alpha P + \beta P - P (\alpha + \beta)
            \label{eqn:condition_for_tilde_p4}
\end{align}

Equation (\ref{eqn:condition_for_tilde_p4}) ensures that \(p_4=\tilde p_4=0\).
Equations (\ref{eqn:condition_for_tilde_p1}-\ref{eqn:condition_for_tilde_p3})
can be used to eliminate \(\alpha, \beta\), giving:

\begin{equation}\label{eqn:planar_definition_of_extortion}
    \tilde p_1 = \frac{(R - P)(\tilde p_2 + \tilde p_3)}{S + T - 2P}
\end{equation}

with:

\begin{equation}\label{eqn:definition_of_chi}
    \chi = \frac{\tilde p_2 (P - T) + \tilde p_3 (S - P)}
                {\tilde p_2 (P - S) + \tilde p_3 (T - P)}
\end{equation}

Given a strategy \(p\in\mathbb{R}^{4\times 1}\) equations
(\ref{eqn:condition_for_tilde_p4}), (\ref{eqn:planar_definition_of_extortion}-\ref{eqn:definition_of_chi}) can be used to check if
a strategy is extortionate. The conditions correspond to:

\begin{align}
    p_1 & = \frac{(R-P)(p_2 + p_3) - R + T + S - P}{S + T - 2P}
     \label{eqn:condition_for_p1}\\
    p_4 & = 0 \label{eqn:condition_for_p4}\\
    1 & > p_2 + p_3\label{eqn:condition_for_chi}
\end{align}

The algebraic steps necessary to prove these results are available in the
supporting materials.

All extortionate strategies reside on a triangular (\ref{eqn:condition_for_chi})
plane (\ref{eqn:condition_for_p1}) in 3 dimensions (\ref{eqn:condition_for_p4}).
Using this formulation it can be seen that a necessary (but not sufficient)
condition for an extortionate strategy is that it cooperates on average less
than 50\% of the time when in a state of disagreement with the opponent.

As an example, consider the known extortionate strategy \(p=(8 / 9, 1 / 2, 1 /
3, 0)\) from~\cite{Stewart2012} which is referred to as \texttt{Extort-2}. In
this case, for the standard values of \((R, T, S, P)\) constraint
(\ref{eqn:condition_for_p1}) corresponds to:

\begin{equation}
    p_1 = \frac{2(p_2 + p_3) + 1}{3}
\end{equation}

It is clear that in this case all constraints hold.

This approach could in fact be used to confirm that a given strategy is acting
in an extortionate manner even if it is not a memory one strategy. However, in
practice, if a closed form for \(p\) is not known, then due to measurement
and/or numerical error this would not work.

This problem can be written in the following linear algebraic form where
\(x=(\alpha, \beta)\)
and \(p^*=(\tilde p_1 - 1, tilde_2 - 1, p_3)\):

\begin{equation}\label{eqn:linear_algebraic_equation_for_p}
    Cx= p^*
\end{equation}

\(C\) corresponds to equations
(\ref{eqn:condition_for_tilde_p1}-\ref{eqn:condition_for_tilde_p3}) and is
given by:

\begin{equation}\label{eqn:definition_of_C}
    C =
    \begin{bmatrix}
        R - P & R- P \\
        S - P & T- P \\
        T - P & S- P \\
    \end{bmatrix}
\end{equation}

Note that in general, equation (\ref{eqn:linear_algebraic_equation_for_p}) will
not necessarily have a solution. From the Rouch\'{e}-Capelli theorem if there is
a solution it is unique as \(\text{rank}(C)=2\) which is the dimension of the
variable \(x\). The best fitting \(x\) is found by minimizing:

\begin{equation}\label{eqn:r_squared}
    \text{SSError} = \|C x- p^*\|_2^2 = \sum_{i=1}^{3}\left((C\bar x)_i-p_i^*\right)^2
\end{equation}

Note that \(\text{SSError}\), which is the square of the Frobenius
norm~\cite{Golub2013}, becomes a measure of how close a strategy is to being an
extortionate strategy. Suspicion
of extortion then corresponds to a threshold on \(\text{SSError}\).

By observing interactions (human or otherwise), their memory one representation
can be inferred and this approach can be used to recognise extortionate
behaviour. The notion of comparing theoretic and actual plays of the IPD is not
novel, see for example~\cite{Rand2013}. Immediately it is noted that if the
environment is noisy~\cite{Wu1995} then no strategy can be considered to be
extortionate as \(p_4>0\).

In the next section, this idea will be illustrated by observing the interactions
that take place in a computer based tournament of the IPD\@.

\section{Numerical experiments}\label{sec:numerical-experiments}

In~\cite{Stewart2012} results from a tournament with
\documentclass[a4paper]{article}

\usepackage{amsmath}
\usepackage{amssymb}
\usepackage[margin=1.5cm,
            includefoot,
            footskip=30pt]{geometry}
\usepackage{layout}
\usepackage{graphicx}
\usepackage{subcaption}

\usepackage{biblatex}
\usepackage{pdfpages}

\bibliography{main.bib}

\title{Suspicion: Recognising and evaluating the effectiveness
       of extortion in the Iterated Prisoner's Dilemma}
\author{Vincent A. Knight \and Nikoleta E. Glynatsi}
\date{\today}



\begin{document}

\maketitle

\begin{abstract}
    The Iterated Prisoner's Dilemma is a model for rational and evolutionary
    interactive behaviour. It has applications both in the study of human social
    behaviour as well as in biology.
    It is used to understand when and how a rational individual might
    accept an immediate cost to their own utility for the direct benefit of
    another.

    Much attention has been given to a class of strategies called
    Zero Determinant strategies. It has been theoretically shown that these
    strategies can ``extort'' any player.

    In this work, an approach to identify if observed strategies are playing in
    an extortionate way is described. Furthermore, experimental analysis of
    a large tournament with \input{assets/tex/number_of_full_strategies/main.tex}
    strategies is considered. In this setting
    the most highly performing strategies do not play in an extortionate way
    against each other but do against lower performing strategies.
    This suggests that whilst the theory of Zero Determinant strategies
    indicates that memory is not of fundamental importance to the evolution of
    cooperative behaviour, this is incomplete.
\end{abstract}

\section{Introduction}\label{sec:introduction}

Agent based game theoretic models have become a stalwart of the underpinning
mathematics of interactive behaviours. One of the major pieces of work
in this area is the pair of original computer tournaments run by Robert
Axelrod~\cite{Axelrod1980, Axelrod1980a}. These tournaments pitted submitted
computer strategies against each other in plays of the Iterated Prisoner's
Dilemma. A common game where agents can choose to pay a slight cost to their
immediate utility in the hope of building a reputation. This has been used in
economic and evolutionary game theory to understand the evolution of cooperative
behaviour.

Recently, a class of strategies was described in~\cite{Press2012} that can
provably extort any given opponent. In~\cite{Hilbe2013, Moran1707} some
questions have already been asked about the true effectiveness of these
strategies in an evolutionary setting. Here another question is asked: is it
possible to recognise this extortionate behaviour? A mathematical procedure for
suspicion is presented: in the same way that the continued actions of an
extortionate individual might raise suspicion.

This work makes use of the Axelrod Python library~\cite{Knight2018, Knight2016}
with a large number of Prisoner Dilemma strategies available to give an
extensive numerical example of the ideas presented.  The approach is presented
in Section~\ref{sec:delta-zd-strategies}.  All of the code and data discussed
in Section~\ref{sec:numerical-experiments} is open sourced, archived and
written according to best scientific principles~\cite{Wilson2014}. The data
archive can be found at~\cite{vincent_knight_2018_1297075}.

\section{Recognising Extortion}\label{sec:delta-zd-strategies}

In~\cite{Press2012}, given a match between 2 memory-one strategies, the concept
of Zero Determinant (ZD) strategies is introduced. The main result of that paper
shows that given two memory one players \(p, q\in\mathbb{R}^4\) a linear
relationship between the players' scores could be forced by one of the players.

Using the notation of~\cite{Press2012}, assuming the utilities for player \(p\)
are given by \(S_x=(R, S, T, P)\) and for player \(q\) by \(S_y=(R, T, S, P)\)
and that the stationary scores of each player is given by \(S_X\) and \(S_Y\)
respectively. The main result of~\cite{Press2012} is that if

\begin{equation}\label{eqn:linear_relationship_for_p}
    \tilde p=\alpha S_x + \beta S_y + \gamma
\end{equation}

or

\begin{equation}\label{eqn:linear_relationship_for_q}
    \tilde q=\alpha S_x + \beta S_y + \gamma
\end{equation}

where \(\tilde p = (1 - p_1, 1 - p_2, p_3, p_4)\) and
\(\tilde q = (1 - q_1, 1 - q_2, q_3, q_4)\) then:

\begin{equation}
    \alpha S_X + \beta S_Y + \gamma = 0
\end{equation}

In~\cite{Press2012} a particular type of ZD strategy is defined: extortionate
strategies. If:

\begin{equation}\label{eqn:constraint_for_extortion}
    \gamma = - P(\alpha + \beta)
\end{equation}

then the player can ensure they get a score \(\chi\) times
larger than the opponent. This extortion coefficient is given by:

\begin{equation}\label{eqn:definition_of_chi}
    \chi=\frac{-\beta}{\alpha}
\end{equation}

Thus, if (\ref{eqn:constraint_for_extortion}) holds and \(\chi >1\) a player is
said to extort their opponent.
Here, the reverse problem is considered: given a
\(p\in\mathbb{R}^4\) how does one identify \(\alpha, \beta\) if they
exist and is the strategy in fact acting in an extortionate way?

These conditions correspond to:

\begin{align}
    \tilde p_1 & = \alpha R + \beta R - P (\alpha + \beta)
            \label{eqn:condition_for_tilde_p1}\\
    \tilde p_2 & = \alpha S + \beta T - P (\alpha + \beta)
            \label{eqn:condition_for_tilde_p2}\\
    \tilde p_3 & = \alpha T + \beta S - P (\alpha + \beta)
            \label{eqn:condition_for_tilde_p3}\\
    \tilde p_4 & = \alpha P + \beta P - P (\alpha + \beta)
            \label{eqn:condition_for_tilde_p4}
\end{align}

Equation (\ref{eqn:condition_for_tilde_p4}) ensures that \(p_4=\tilde p_4=0\).
Equations (\ref{eqn:condition_for_tilde_p1}-\ref{eqn:condition_for_tilde_p3})
can be used to eliminate \(\alpha, \beta\), giving:

\begin{equation}\label{eqn:planar_definition_of_extortion}
    \tilde p_1 = \frac{(R - P)(\tilde p_2 + \tilde p_3)}{S + T - 2P}
\end{equation}

with:

\begin{equation}\label{eqn:definition_of_chi}
    \chi = \frac{\tilde p_2 (P - T) + \tilde p_3 (S - P)}
                {\tilde p_2 (P - S) + \tilde p_3 (T - P)}
\end{equation}

Given a strategy \(p\in\mathbb{R}^{4\times 1}\) equations
(\ref{eqn:condition_for_tilde_p4}), (\ref{eqn:planar_definition_of_extortion}-\ref{eqn:definition_of_chi}) can be used to check if
a strategy is extortionate. The conditions correspond to:

\begin{align}
    p_1 & = \frac{(R-P)(p_2 + p_3) - R + T + S - P}{S + T - 2P}
     \label{eqn:condition_for_p1}\\
    p_4 & = 0 \label{eqn:condition_for_p4}\\
    1 & > p_2 + p_3\label{eqn:condition_for_chi}
\end{align}

The algebraic steps necessary to prove these results are available in the
supporting materials.

All extortionate strategies reside on a triangular (\ref{eqn:condition_for_chi})
plane (\ref{eqn:condition_for_p1}) in 3 dimensions (\ref{eqn:condition_for_p4}).
Using this formulation it can be seen that a necessary (but not sufficient)
condition for an extortionate strategy is that it cooperates on average less
than 50\% of the time when in a state of disagreement with the opponent.

As an example, consider the known extortionate strategy \(p=(8 / 9, 1 / 2, 1 /
3, 0)\) from~\cite{Stewart2012} which is referred to as \texttt{Extort-2}. In
this case, for the standard values of \((R, T, S, P)\) constraint
(\ref{eqn:condition_for_p1}) corresponds to:

\begin{equation}
    p_1 = \frac{2(p_2 + p_3) + 1}{3}
\end{equation}

It is clear that in this case all constraints hold.

This approach could in fact be used to confirm that a given strategy is acting
in an extortionate manner even if it is not a memory one strategy. However, in
practice, if a closed form for \(p\) is not known, then due to measurement
and/or numerical error this would not work.

This problem can be written in the following linear algebraic form where
\(x=(\alpha, \beta)\)
and \(p^*=(\tilde p_1 - 1, tilde_2 - 1, p_3)\):

\begin{equation}\label{eqn:linear_algebraic_equation_for_p}
    Cx= p^*
\end{equation}

\(C\) corresponds to equations
(\ref{eqn:condition_for_tilde_p1}-\ref{eqn:condition_for_tilde_p3}) and is
given by:

\begin{equation}\label{eqn:definition_of_C}
    C =
    \begin{bmatrix}
        R - P & R- P \\
        S - P & T- P \\
        T - P & S- P \\
    \end{bmatrix}
\end{equation}

Note that in general, equation (\ref{eqn:linear_algebraic_equation_for_p}) will
not necessarily have a solution. From the Rouch\'{e}-Capelli theorem if there is
a solution it is unique as \(\text{rank}(C)=2\) which is the dimension of the
variable \(x\). The best fitting \(x\) is found by minimizing:

\begin{equation}\label{eqn:r_squared}
    \text{SSError} = \|C x- p^*\|_2^2 = \sum_{i=1}^{3}\left((C\bar x)_i-p_i^*\right)^2
\end{equation}

Note that \(\text{SSError}\), which is the square of the Frobenius
norm~\cite{Golub2013}, becomes a measure of how close a strategy is to being an
extortionate strategy. Suspicion
of extortion then corresponds to a threshold on \(\text{SSError}\).

By observing interactions (human or otherwise), their memory one representation
can be inferred and this approach can be used to recognise extortionate
behaviour. The notion of comparing theoretic and actual plays of the IPD is not
novel, see for example~\cite{Rand2013}. Immediately it is noted that if the
environment is noisy~\cite{Wu1995} then no strategy can be considered to be
extortionate as \(p_4>0\).

In the next section, this idea will be illustrated by observing the interactions
that take place in a computer based tournament of the IPD\@.

\section{Numerical experiments}\label{sec:numerical-experiments}

In~\cite{Stewart2012} results from a tournament with
\input{./assets/tex/number_of_stewart_plotkin_strategies/main.tex} strategies,
was presented with specific consideration given to ZD strategies. This
tournament is reproduced here using the Axelrod-Python
project~\cite{Knight2016}. To obtain a good measure of the corresponding
transition rates for each strategy all matches have been run for
\input{assets/tex/number_of_turns/main.tex} turns and every match has been
repeated \input{assets/tex/number_of_repetitions/main.tex} times. All of this
interaction data is available at~\cite{vincent_knight_2018_1297075}. A good
match between the inferred Markov chain and the state distribution of the actual
interactions has been verified. Data for this is presented in the supplementary
materials.

Figure~\ref{fig:SSError_overall_in_stewart_plotkin} shows the \(\text{SSError}\)
values for all the strategies in the tournament, as reported
in~\cite{Stewart2012} the extortionate strategy (which has an expected
\(\text{SSError}\) approximately 0) gains a large number of wins.

\begin{figure}[!htbp]
    \centering
    \includegraphics[width=.8\textwidth]{./assets/img/SSError_overall_in_stewart_plotkin/main.pdf}
    \caption{\(\text{SSError}\) and state probabilities for the strategies
        of~\cite{Stewart2012}, ordered both by number of wins and overall score.
        Note that \(P(DC)\) is not shown as it corresponds to the transpose of
        \(P(CD)\). Cooperator and Defector are omitted as they do not visit all
        the states.}
    \label{fig:SSError_overall_in_stewart_plotkin}
\end{figure}

Here, the work of~\cite{Stewart2012} is extended by investigating a tournament
with \input{assets/tex/number_of_full_strategies/main.tex}
strategies.

The results of this analysis are shown in
Figure~\ref{fig:SSError_and_probabilities_in_full}. The top ranking strategies
by number of wins seem to be extortionate (but not against all strategies) and
it can be seen that a small sub group of strategies achieve mutual defection.
All the top ranking strategies according to score achieve mutual cooperation and
do not extort each other, however they
\textbf{do} exhibit extortionate behaviour towards a number of the lower ranking
strategies.

\begin{figure}[!htbp]
    \centering
    \includegraphics[width=.8\textwidth]{./assets/img/SSError_and_probabilities_in_full/main.pdf}
    \caption{\(\text{SSError}\) for the strategies for the full tournament. Only
    strategy interactions for which \(p_4=0\) and \(\chi>1\) are displayed.}
    \label{fig:SSError_and_probabilities_in_full}
\end{figure}

\section{Conclusion}\label{sec:conclusion}

This work defines an approach to measure whether or not a player is playing a
strategy that corresponds to an extortionate strategy as defined
in~\cite{Press2012}: a mathematical model for suspicion. Indeed, all
extortionate strategies have been
 classified as lying on a triangular plane.
This rigorous classification fails to be robust to small measurement error, thus
a statistical approach is proposed.
This is done through a linear algebraic approach for approximating the solution
of a linear system. Using this, a large number of pairwise interactions is
simulated and in fact very few strategies are found to act extortionately.

The work of~\cite{Press2012}, whilst showing that a clever approach to taking
advantage of another memory one strategy exists: this is incomplete. Whilst the
elegance of this result is very attractive, just as the simplicity of the
victory of Tit For Tat in Axelrod's original tournaments was, it is incomplete.
Extortionate strategies achieve a high number of wins but they do not
achieve a high score which corresponds to the fitness landscape in an
evolutionary sense. From the large number of interactions a payoff matrix \(S\)
can be measured where \(S_{ij}\) denotes the score (using standard values of
\((R, S, T, P) = (3, 0, 5, 1)\)) of the \(i\)th strategy
against the \(j\)th strategy. Using this, the replicator equation
describes the evolution of the system based on a population density fitness
function:

\begin{equation}\label{eqn:replicator_dynamics}
    \frac{dx}{dt} = x(S-x^TS x)
\end{equation}

Equation (\ref{eqn:replicator_dynamics}) is solved numerically through an
integration technique described in~\cite{Petzold1983} and
Figure~\ref{fig:replicator_dynamics} shows the evolution of the distribution of
the system: the various strategies are ranked by scores. It is clear to see that
only the high ranking strategies survive the evolutionary process (in fact,
only \input{./assets/img/replicator_dynamics/main.tex}
have a final distribution greater than \(10 ^ {-2}\)). This confirms the
findings of~\cite{Moran1707} in which sophisticated strategies resist
evolutionary invasion of shorter memory strategies. Recalling
Figure~\ref{fig:SSError_and_probabilities_in_full} this demonstrates that:

\begin{itemize}
    \item Cooperation emerges through the evolutionary process: the high scoring
        strategies do not exhibit extortionate behaviour towards each other.
    \item Extortionate strategies do not survive the evolutionary process.
\end{itemize}

\begin{figure}[!htbp]
    \centering
    \includegraphics[width=.8\textwidth]{./assets/img/replicator_dynamics/main.pdf}
    \caption{Numerical simulation of the replicator equation
    (\ref{eqn:replicator_dynamics}): strategies are ordered by score, only the strategies with a high score survive the evolutionary process.}
    \label{fig:replicator_dynamics}
\end{figure}

This work can be used to classify plays of the IPD\@: data can be collected from
actual interactions (in lab or in the field). Furthermore, this allows for a
classification method similar to the notion of fingerprinting presented
in~\cite{Ashlock2008}. Trained strategies can potentially be classified as
extortionate or not or it could be possible to even constrain the reinforcement
learning approaches that are becoming prevalent in the literature.
Alternatively, this mathematical approach for recognising extortion could be
used in sophisticated strategies to defend against invasion. Arguably, some of
the strategies considered here exhibit this behaviour, indeed as described
in~\cite{Harper2017}, the top ranking strategies in the full tournament are
obtained using evolutionary reinforcement learning techniques, thus, suspicion
of extortionate behaviour could in fact be an evolutionary trait.

\section*{Acknowledgements}

The following open source software libraries were used in this research:

\begin{itemize}
    \item The Axelrod ~\cite{Knight2016, Knight2018} library (IPD strategies and
        tournaments).
    \item The sympy library~\cite{Meurer2017} (verification of all symbolic
        calculations).
    \item The matplotlib~\cite{Droettboom2018} library (visualisation).
    \item The pandas~\cite{Structures2010}, dask~\cite{Dask2016} and
        NumPy~\cite{Oliphant2015} libraries (data manipulation).
    \item The SciPy~\cite{Jones2001} library (numerical integration of the
        replicator equation).
\end{itemize}

This work was performed using the computational facilities of the Advanced
Research Computing @ Cardiff (ARCCA) Division, Cardiff University.

\printbibliography

\newpage
\section*{Supplementary materials}

\includepdf{assets/pdf/proof_of_form_of_extortionate_strategies/main.pdf}

\newpage

Using the pair wise interactions the transition rates \(p,
q\) can be measured and the steady state probabilities inferred and compared to
the actual probabilities of each state.
This is done numerically by computing the singular eigenvector of the
matrix \(A\) \cite{Stewart2009}:

\[
    A =
    \begin{bmatrix}
        p_1 q_1 & p_1 (1 - q_1) & (1 - p_1) q_1 & (1 -p_1) (1 - q_1) \\
        p_2 q_2 & p_2 (1 - q_2) & (1 - p_2) q_2 & (1 -p_2) (1 - q_2) \\
        p_3 q_3 & p_3 (1 - q_3) & (1 - p_3) q_3 & (1 -p_3) (1 - q_3) \\
        p_4 q_4 & p_4 (1 - q_4) & (1 - p_4) q_4 & (1 -p_4) (1 - q_4) \\
    \end{bmatrix}
\]

Figure~\ref{fig:computed_probabilities_vs_theoretic_probabilities} shows a
regression line fitted to every pairwise interaction with a reported
\(\text{SSError}\) value (pairwise interactions with missing states were
omitted). This serves to validate the approach: a part from some edge cases the
relationship is consistent.

\begin{figure}[!htbp]
    \centering
    \includegraphics[width=.8\textwidth]{./assets/img/computed_probabilities_vs_theoretic_probabilities/main.pdf}
    \caption{The
        relationship between the steady state probabilities inferred from the
        measured transitions and the actual steady state probabilities. A linear
        regression line is included validating the approach.}
    \label{fig:computed_probabilities_vs_theoretic_probabilities}
\end{figure}


\end{document}
 strategies,
was presented with specific consideration given to ZD strategies. This
tournament is reproduced here using the Axelrod-Python
project~\cite{Knight2016}. To obtain a good measure of the corresponding
transition rates for each strategy all matches have been run for
\documentclass[a4paper]{article}

\usepackage{amsmath}
\usepackage{amssymb}
\usepackage[margin=1.5cm,
            includefoot,
            footskip=30pt]{geometry}
\usepackage{layout}
\usepackage{graphicx}
\usepackage{subcaption}

\usepackage{biblatex}
\usepackage{pdfpages}

\bibliography{main.bib}

\title{Suspicion: Recognising and evaluating the effectiveness
       of extortion in the Iterated Prisoner's Dilemma}
\author{Vincent A. Knight \and Nikoleta E. Glynatsi}
\date{\today}



\begin{document}

\maketitle

\begin{abstract}
    The Iterated Prisoner's Dilemma is a model for rational and evolutionary
    interactive behaviour. It has applications both in the study of human social
    behaviour as well as in biology.
    It is used to understand when and how a rational individual might
    accept an immediate cost to their own utility for the direct benefit of
    another.

    Much attention has been given to a class of strategies called
    Zero Determinant strategies. It has been theoretically shown that these
    strategies can ``extort'' any player.

    In this work, an approach to identify if observed strategies are playing in
    an extortionate way is described. Furthermore, experimental analysis of
    a large tournament with \input{assets/tex/number_of_full_strategies/main.tex}
    strategies is considered. In this setting
    the most highly performing strategies do not play in an extortionate way
    against each other but do against lower performing strategies.
    This suggests that whilst the theory of Zero Determinant strategies
    indicates that memory is not of fundamental importance to the evolution of
    cooperative behaviour, this is incomplete.
\end{abstract}

\section{Introduction}\label{sec:introduction}

Agent based game theoretic models have become a stalwart of the underpinning
mathematics of interactive behaviours. One of the major pieces of work
in this area is the pair of original computer tournaments run by Robert
Axelrod~\cite{Axelrod1980, Axelrod1980a}. These tournaments pitted submitted
computer strategies against each other in plays of the Iterated Prisoner's
Dilemma. A common game where agents can choose to pay a slight cost to their
immediate utility in the hope of building a reputation. This has been used in
economic and evolutionary game theory to understand the evolution of cooperative
behaviour.

Recently, a class of strategies was described in~\cite{Press2012} that can
provably extort any given opponent. In~\cite{Hilbe2013, Moran1707} some
questions have already been asked about the true effectiveness of these
strategies in an evolutionary setting. Here another question is asked: is it
possible to recognise this extortionate behaviour? A mathematical procedure for
suspicion is presented: in the same way that the continued actions of an
extortionate individual might raise suspicion.

This work makes use of the Axelrod Python library~\cite{Knight2018, Knight2016}
with a large number of Prisoner Dilemma strategies available to give an
extensive numerical example of the ideas presented.  The approach is presented
in Section~\ref{sec:delta-zd-strategies}.  All of the code and data discussed
in Section~\ref{sec:numerical-experiments} is open sourced, archived and
written according to best scientific principles~\cite{Wilson2014}. The data
archive can be found at~\cite{vincent_knight_2018_1297075}.

\section{Recognising Extortion}\label{sec:delta-zd-strategies}

In~\cite{Press2012}, given a match between 2 memory-one strategies, the concept
of Zero Determinant (ZD) strategies is introduced. The main result of that paper
shows that given two memory one players \(p, q\in\mathbb{R}^4\) a linear
relationship between the players' scores could be forced by one of the players.

Using the notation of~\cite{Press2012}, assuming the utilities for player \(p\)
are given by \(S_x=(R, S, T, P)\) and for player \(q\) by \(S_y=(R, T, S, P)\)
and that the stationary scores of each player is given by \(S_X\) and \(S_Y\)
respectively. The main result of~\cite{Press2012} is that if

\begin{equation}\label{eqn:linear_relationship_for_p}
    \tilde p=\alpha S_x + \beta S_y + \gamma
\end{equation}

or

\begin{equation}\label{eqn:linear_relationship_for_q}
    \tilde q=\alpha S_x + \beta S_y + \gamma
\end{equation}

where \(\tilde p = (1 - p_1, 1 - p_2, p_3, p_4)\) and
\(\tilde q = (1 - q_1, 1 - q_2, q_3, q_4)\) then:

\begin{equation}
    \alpha S_X + \beta S_Y + \gamma = 0
\end{equation}

In~\cite{Press2012} a particular type of ZD strategy is defined: extortionate
strategies. If:

\begin{equation}\label{eqn:constraint_for_extortion}
    \gamma = - P(\alpha + \beta)
\end{equation}

then the player can ensure they get a score \(\chi\) times
larger than the opponent. This extortion coefficient is given by:

\begin{equation}\label{eqn:definition_of_chi}
    \chi=\frac{-\beta}{\alpha}
\end{equation}

Thus, if (\ref{eqn:constraint_for_extortion}) holds and \(\chi >1\) a player is
said to extort their opponent.
Here, the reverse problem is considered: given a
\(p\in\mathbb{R}^4\) how does one identify \(\alpha, \beta\) if they
exist and is the strategy in fact acting in an extortionate way?

These conditions correspond to:

\begin{align}
    \tilde p_1 & = \alpha R + \beta R - P (\alpha + \beta)
            \label{eqn:condition_for_tilde_p1}\\
    \tilde p_2 & = \alpha S + \beta T - P (\alpha + \beta)
            \label{eqn:condition_for_tilde_p2}\\
    \tilde p_3 & = \alpha T + \beta S - P (\alpha + \beta)
            \label{eqn:condition_for_tilde_p3}\\
    \tilde p_4 & = \alpha P + \beta P - P (\alpha + \beta)
            \label{eqn:condition_for_tilde_p4}
\end{align}

Equation (\ref{eqn:condition_for_tilde_p4}) ensures that \(p_4=\tilde p_4=0\).
Equations (\ref{eqn:condition_for_tilde_p1}-\ref{eqn:condition_for_tilde_p3})
can be used to eliminate \(\alpha, \beta\), giving:

\begin{equation}\label{eqn:planar_definition_of_extortion}
    \tilde p_1 = \frac{(R - P)(\tilde p_2 + \tilde p_3)}{S + T - 2P}
\end{equation}

with:

\begin{equation}\label{eqn:definition_of_chi}
    \chi = \frac{\tilde p_2 (P - T) + \tilde p_3 (S - P)}
                {\tilde p_2 (P - S) + \tilde p_3 (T - P)}
\end{equation}

Given a strategy \(p\in\mathbb{R}^{4\times 1}\) equations
(\ref{eqn:condition_for_tilde_p4}), (\ref{eqn:planar_definition_of_extortion}-\ref{eqn:definition_of_chi}) can be used to check if
a strategy is extortionate. The conditions correspond to:

\begin{align}
    p_1 & = \frac{(R-P)(p_2 + p_3) - R + T + S - P}{S + T - 2P}
     \label{eqn:condition_for_p1}\\
    p_4 & = 0 \label{eqn:condition_for_p4}\\
    1 & > p_2 + p_3\label{eqn:condition_for_chi}
\end{align}

The algebraic steps necessary to prove these results are available in the
supporting materials.

All extortionate strategies reside on a triangular (\ref{eqn:condition_for_chi})
plane (\ref{eqn:condition_for_p1}) in 3 dimensions (\ref{eqn:condition_for_p4}).
Using this formulation it can be seen that a necessary (but not sufficient)
condition for an extortionate strategy is that it cooperates on average less
than 50\% of the time when in a state of disagreement with the opponent.

As an example, consider the known extortionate strategy \(p=(8 / 9, 1 / 2, 1 /
3, 0)\) from~\cite{Stewart2012} which is referred to as \texttt{Extort-2}. In
this case, for the standard values of \((R, T, S, P)\) constraint
(\ref{eqn:condition_for_p1}) corresponds to:

\begin{equation}
    p_1 = \frac{2(p_2 + p_3) + 1}{3}
\end{equation}

It is clear that in this case all constraints hold.

This approach could in fact be used to confirm that a given strategy is acting
in an extortionate manner even if it is not a memory one strategy. However, in
practice, if a closed form for \(p\) is not known, then due to measurement
and/or numerical error this would not work.

This problem can be written in the following linear algebraic form where
\(x=(\alpha, \beta)\)
and \(p^*=(\tilde p_1 - 1, tilde_2 - 1, p_3)\):

\begin{equation}\label{eqn:linear_algebraic_equation_for_p}
    Cx= p^*
\end{equation}

\(C\) corresponds to equations
(\ref{eqn:condition_for_tilde_p1}-\ref{eqn:condition_for_tilde_p3}) and is
given by:

\begin{equation}\label{eqn:definition_of_C}
    C =
    \begin{bmatrix}
        R - P & R- P \\
        S - P & T- P \\
        T - P & S- P \\
    \end{bmatrix}
\end{equation}

Note that in general, equation (\ref{eqn:linear_algebraic_equation_for_p}) will
not necessarily have a solution. From the Rouch\'{e}-Capelli theorem if there is
a solution it is unique as \(\text{rank}(C)=2\) which is the dimension of the
variable \(x\). The best fitting \(x\) is found by minimizing:

\begin{equation}\label{eqn:r_squared}
    \text{SSError} = \|C x- p^*\|_2^2 = \sum_{i=1}^{3}\left((C\bar x)_i-p_i^*\right)^2
\end{equation}

Note that \(\text{SSError}\), which is the square of the Frobenius
norm~\cite{Golub2013}, becomes a measure of how close a strategy is to being an
extortionate strategy. Suspicion
of extortion then corresponds to a threshold on \(\text{SSError}\).

By observing interactions (human or otherwise), their memory one representation
can be inferred and this approach can be used to recognise extortionate
behaviour. The notion of comparing theoretic and actual plays of the IPD is not
novel, see for example~\cite{Rand2013}. Immediately it is noted that if the
environment is noisy~\cite{Wu1995} then no strategy can be considered to be
extortionate as \(p_4>0\).

In the next section, this idea will be illustrated by observing the interactions
that take place in a computer based tournament of the IPD\@.

\section{Numerical experiments}\label{sec:numerical-experiments}

In~\cite{Stewart2012} results from a tournament with
\input{./assets/tex/number_of_stewart_plotkin_strategies/main.tex} strategies,
was presented with specific consideration given to ZD strategies. This
tournament is reproduced here using the Axelrod-Python
project~\cite{Knight2016}. To obtain a good measure of the corresponding
transition rates for each strategy all matches have been run for
\input{assets/tex/number_of_turns/main.tex} turns and every match has been
repeated \input{assets/tex/number_of_repetitions/main.tex} times. All of this
interaction data is available at~\cite{vincent_knight_2018_1297075}. A good
match between the inferred Markov chain and the state distribution of the actual
interactions has been verified. Data for this is presented in the supplementary
materials.

Figure~\ref{fig:SSError_overall_in_stewart_plotkin} shows the \(\text{SSError}\)
values for all the strategies in the tournament, as reported
in~\cite{Stewart2012} the extortionate strategy (which has an expected
\(\text{SSError}\) approximately 0) gains a large number of wins.

\begin{figure}[!htbp]
    \centering
    \includegraphics[width=.8\textwidth]{./assets/img/SSError_overall_in_stewart_plotkin/main.pdf}
    \caption{\(\text{SSError}\) and state probabilities for the strategies
        of~\cite{Stewart2012}, ordered both by number of wins and overall score.
        Note that \(P(DC)\) is not shown as it corresponds to the transpose of
        \(P(CD)\). Cooperator and Defector are omitted as they do not visit all
        the states.}
    \label{fig:SSError_overall_in_stewart_plotkin}
\end{figure}

Here, the work of~\cite{Stewart2012} is extended by investigating a tournament
with \input{assets/tex/number_of_full_strategies/main.tex}
strategies.

The results of this analysis are shown in
Figure~\ref{fig:SSError_and_probabilities_in_full}. The top ranking strategies
by number of wins seem to be extortionate (but not against all strategies) and
it can be seen that a small sub group of strategies achieve mutual defection.
All the top ranking strategies according to score achieve mutual cooperation and
do not extort each other, however they
\textbf{do} exhibit extortionate behaviour towards a number of the lower ranking
strategies.

\begin{figure}[!htbp]
    \centering
    \includegraphics[width=.8\textwidth]{./assets/img/SSError_and_probabilities_in_full/main.pdf}
    \caption{\(\text{SSError}\) for the strategies for the full tournament. Only
    strategy interactions for which \(p_4=0\) and \(\chi>1\) are displayed.}
    \label{fig:SSError_and_probabilities_in_full}
\end{figure}

\section{Conclusion}\label{sec:conclusion}

This work defines an approach to measure whether or not a player is playing a
strategy that corresponds to an extortionate strategy as defined
in~\cite{Press2012}: a mathematical model for suspicion. Indeed, all
extortionate strategies have been
 classified as lying on a triangular plane.
This rigorous classification fails to be robust to small measurement error, thus
a statistical approach is proposed.
This is done through a linear algebraic approach for approximating the solution
of a linear system. Using this, a large number of pairwise interactions is
simulated and in fact very few strategies are found to act extortionately.

The work of~\cite{Press2012}, whilst showing that a clever approach to taking
advantage of another memory one strategy exists: this is incomplete. Whilst the
elegance of this result is very attractive, just as the simplicity of the
victory of Tit For Tat in Axelrod's original tournaments was, it is incomplete.
Extortionate strategies achieve a high number of wins but they do not
achieve a high score which corresponds to the fitness landscape in an
evolutionary sense. From the large number of interactions a payoff matrix \(S\)
can be measured where \(S_{ij}\) denotes the score (using standard values of
\((R, S, T, P) = (3, 0, 5, 1)\)) of the \(i\)th strategy
against the \(j\)th strategy. Using this, the replicator equation
describes the evolution of the system based on a population density fitness
function:

\begin{equation}\label{eqn:replicator_dynamics}
    \frac{dx}{dt} = x(S-x^TS x)
\end{equation}

Equation (\ref{eqn:replicator_dynamics}) is solved numerically through an
integration technique described in~\cite{Petzold1983} and
Figure~\ref{fig:replicator_dynamics} shows the evolution of the distribution of
the system: the various strategies are ranked by scores. It is clear to see that
only the high ranking strategies survive the evolutionary process (in fact,
only \input{./assets/img/replicator_dynamics/main.tex}
have a final distribution greater than \(10 ^ {-2}\)). This confirms the
findings of~\cite{Moran1707} in which sophisticated strategies resist
evolutionary invasion of shorter memory strategies. Recalling
Figure~\ref{fig:SSError_and_probabilities_in_full} this demonstrates that:

\begin{itemize}
    \item Cooperation emerges through the evolutionary process: the high scoring
        strategies do not exhibit extortionate behaviour towards each other.
    \item Extortionate strategies do not survive the evolutionary process.
\end{itemize}

\begin{figure}[!htbp]
    \centering
    \includegraphics[width=.8\textwidth]{./assets/img/replicator_dynamics/main.pdf}
    \caption{Numerical simulation of the replicator equation
    (\ref{eqn:replicator_dynamics}): strategies are ordered by score, only the strategies with a high score survive the evolutionary process.}
    \label{fig:replicator_dynamics}
\end{figure}

This work can be used to classify plays of the IPD\@: data can be collected from
actual interactions (in lab or in the field). Furthermore, this allows for a
classification method similar to the notion of fingerprinting presented
in~\cite{Ashlock2008}. Trained strategies can potentially be classified as
extortionate or not or it could be possible to even constrain the reinforcement
learning approaches that are becoming prevalent in the literature.
Alternatively, this mathematical approach for recognising extortion could be
used in sophisticated strategies to defend against invasion. Arguably, some of
the strategies considered here exhibit this behaviour, indeed as described
in~\cite{Harper2017}, the top ranking strategies in the full tournament are
obtained using evolutionary reinforcement learning techniques, thus, suspicion
of extortionate behaviour could in fact be an evolutionary trait.

\section*{Acknowledgements}

The following open source software libraries were used in this research:

\begin{itemize}
    \item The Axelrod ~\cite{Knight2016, Knight2018} library (IPD strategies and
        tournaments).
    \item The sympy library~\cite{Meurer2017} (verification of all symbolic
        calculations).
    \item The matplotlib~\cite{Droettboom2018} library (visualisation).
    \item The pandas~\cite{Structures2010}, dask~\cite{Dask2016} and
        NumPy~\cite{Oliphant2015} libraries (data manipulation).
    \item The SciPy~\cite{Jones2001} library (numerical integration of the
        replicator equation).
\end{itemize}

This work was performed using the computational facilities of the Advanced
Research Computing @ Cardiff (ARCCA) Division, Cardiff University.

\printbibliography

\newpage
\section*{Supplementary materials}

\includepdf{assets/pdf/proof_of_form_of_extortionate_strategies/main.pdf}

\newpage

Using the pair wise interactions the transition rates \(p,
q\) can be measured and the steady state probabilities inferred and compared to
the actual probabilities of each state.
This is done numerically by computing the singular eigenvector of the
matrix \(A\) \cite{Stewart2009}:

\[
    A =
    \begin{bmatrix}
        p_1 q_1 & p_1 (1 - q_1) & (1 - p_1) q_1 & (1 -p_1) (1 - q_1) \\
        p_2 q_2 & p_2 (1 - q_2) & (1 - p_2) q_2 & (1 -p_2) (1 - q_2) \\
        p_3 q_3 & p_3 (1 - q_3) & (1 - p_3) q_3 & (1 -p_3) (1 - q_3) \\
        p_4 q_4 & p_4 (1 - q_4) & (1 - p_4) q_4 & (1 -p_4) (1 - q_4) \\
    \end{bmatrix}
\]

Figure~\ref{fig:computed_probabilities_vs_theoretic_probabilities} shows a
regression line fitted to every pairwise interaction with a reported
\(\text{SSError}\) value (pairwise interactions with missing states were
omitted). This serves to validate the approach: a part from some edge cases the
relationship is consistent.

\begin{figure}[!htbp]
    \centering
    \includegraphics[width=.8\textwidth]{./assets/img/computed_probabilities_vs_theoretic_probabilities/main.pdf}
    \caption{The
        relationship between the steady state probabilities inferred from the
        measured transitions and the actual steady state probabilities. A linear
        regression line is included validating the approach.}
    \label{fig:computed_probabilities_vs_theoretic_probabilities}
\end{figure}


\end{document}
 turns and every match has been
repeated \documentclass[a4paper]{article}

\usepackage{amsmath}
\usepackage{amssymb}
\usepackage[margin=1.5cm,
            includefoot,
            footskip=30pt]{geometry}
\usepackage{layout}
\usepackage{graphicx}
\usepackage{subcaption}

\usepackage{biblatex}
\usepackage{pdfpages}

\bibliography{main.bib}

\title{Suspicion: Recognising and evaluating the effectiveness
       of extortion in the Iterated Prisoner's Dilemma}
\author{Vincent A. Knight \and Nikoleta E. Glynatsi}
\date{\today}



\begin{document}

\maketitle

\begin{abstract}
    The Iterated Prisoner's Dilemma is a model for rational and evolutionary
    interactive behaviour. It has applications both in the study of human social
    behaviour as well as in biology.
    It is used to understand when and how a rational individual might
    accept an immediate cost to their own utility for the direct benefit of
    another.

    Much attention has been given to a class of strategies called
    Zero Determinant strategies. It has been theoretically shown that these
    strategies can ``extort'' any player.

    In this work, an approach to identify if observed strategies are playing in
    an extortionate way is described. Furthermore, experimental analysis of
    a large tournament with \input{assets/tex/number_of_full_strategies/main.tex}
    strategies is considered. In this setting
    the most highly performing strategies do not play in an extortionate way
    against each other but do against lower performing strategies.
    This suggests that whilst the theory of Zero Determinant strategies
    indicates that memory is not of fundamental importance to the evolution of
    cooperative behaviour, this is incomplete.
\end{abstract}

\section{Introduction}\label{sec:introduction}

Agent based game theoretic models have become a stalwart of the underpinning
mathematics of interactive behaviours. One of the major pieces of work
in this area is the pair of original computer tournaments run by Robert
Axelrod~\cite{Axelrod1980, Axelrod1980a}. These tournaments pitted submitted
computer strategies against each other in plays of the Iterated Prisoner's
Dilemma. A common game where agents can choose to pay a slight cost to their
immediate utility in the hope of building a reputation. This has been used in
economic and evolutionary game theory to understand the evolution of cooperative
behaviour.

Recently, a class of strategies was described in~\cite{Press2012} that can
provably extort any given opponent. In~\cite{Hilbe2013, Moran1707} some
questions have already been asked about the true effectiveness of these
strategies in an evolutionary setting. Here another question is asked: is it
possible to recognise this extortionate behaviour? A mathematical procedure for
suspicion is presented: in the same way that the continued actions of an
extortionate individual might raise suspicion.

This work makes use of the Axelrod Python library~\cite{Knight2018, Knight2016}
with a large number of Prisoner Dilemma strategies available to give an
extensive numerical example of the ideas presented.  The approach is presented
in Section~\ref{sec:delta-zd-strategies}.  All of the code and data discussed
in Section~\ref{sec:numerical-experiments} is open sourced, archived and
written according to best scientific principles~\cite{Wilson2014}. The data
archive can be found at~\cite{vincent_knight_2018_1297075}.

\section{Recognising Extortion}\label{sec:delta-zd-strategies}

In~\cite{Press2012}, given a match between 2 memory-one strategies, the concept
of Zero Determinant (ZD) strategies is introduced. The main result of that paper
shows that given two memory one players \(p, q\in\mathbb{R}^4\) a linear
relationship between the players' scores could be forced by one of the players.

Using the notation of~\cite{Press2012}, assuming the utilities for player \(p\)
are given by \(S_x=(R, S, T, P)\) and for player \(q\) by \(S_y=(R, T, S, P)\)
and that the stationary scores of each player is given by \(S_X\) and \(S_Y\)
respectively. The main result of~\cite{Press2012} is that if

\begin{equation}\label{eqn:linear_relationship_for_p}
    \tilde p=\alpha S_x + \beta S_y + \gamma
\end{equation}

or

\begin{equation}\label{eqn:linear_relationship_for_q}
    \tilde q=\alpha S_x + \beta S_y + \gamma
\end{equation}

where \(\tilde p = (1 - p_1, 1 - p_2, p_3, p_4)\) and
\(\tilde q = (1 - q_1, 1 - q_2, q_3, q_4)\) then:

\begin{equation}
    \alpha S_X + \beta S_Y + \gamma = 0
\end{equation}

In~\cite{Press2012} a particular type of ZD strategy is defined: extortionate
strategies. If:

\begin{equation}\label{eqn:constraint_for_extortion}
    \gamma = - P(\alpha + \beta)
\end{equation}

then the player can ensure they get a score \(\chi\) times
larger than the opponent. This extortion coefficient is given by:

\begin{equation}\label{eqn:definition_of_chi}
    \chi=\frac{-\beta}{\alpha}
\end{equation}

Thus, if (\ref{eqn:constraint_for_extortion}) holds and \(\chi >1\) a player is
said to extort their opponent.
Here, the reverse problem is considered: given a
\(p\in\mathbb{R}^4\) how does one identify \(\alpha, \beta\) if they
exist and is the strategy in fact acting in an extortionate way?

These conditions correspond to:

\begin{align}
    \tilde p_1 & = \alpha R + \beta R - P (\alpha + \beta)
            \label{eqn:condition_for_tilde_p1}\\
    \tilde p_2 & = \alpha S + \beta T - P (\alpha + \beta)
            \label{eqn:condition_for_tilde_p2}\\
    \tilde p_3 & = \alpha T + \beta S - P (\alpha + \beta)
            \label{eqn:condition_for_tilde_p3}\\
    \tilde p_4 & = \alpha P + \beta P - P (\alpha + \beta)
            \label{eqn:condition_for_tilde_p4}
\end{align}

Equation (\ref{eqn:condition_for_tilde_p4}) ensures that \(p_4=\tilde p_4=0\).
Equations (\ref{eqn:condition_for_tilde_p1}-\ref{eqn:condition_for_tilde_p3})
can be used to eliminate \(\alpha, \beta\), giving:

\begin{equation}\label{eqn:planar_definition_of_extortion}
    \tilde p_1 = \frac{(R - P)(\tilde p_2 + \tilde p_3)}{S + T - 2P}
\end{equation}

with:

\begin{equation}\label{eqn:definition_of_chi}
    \chi = \frac{\tilde p_2 (P - T) + \tilde p_3 (S - P)}
                {\tilde p_2 (P - S) + \tilde p_3 (T - P)}
\end{equation}

Given a strategy \(p\in\mathbb{R}^{4\times 1}\) equations
(\ref{eqn:condition_for_tilde_p4}), (\ref{eqn:planar_definition_of_extortion}-\ref{eqn:definition_of_chi}) can be used to check if
a strategy is extortionate. The conditions correspond to:

\begin{align}
    p_1 & = \frac{(R-P)(p_2 + p_3) - R + T + S - P}{S + T - 2P}
     \label{eqn:condition_for_p1}\\
    p_4 & = 0 \label{eqn:condition_for_p4}\\
    1 & > p_2 + p_3\label{eqn:condition_for_chi}
\end{align}

The algebraic steps necessary to prove these results are available in the
supporting materials.

All extortionate strategies reside on a triangular (\ref{eqn:condition_for_chi})
plane (\ref{eqn:condition_for_p1}) in 3 dimensions (\ref{eqn:condition_for_p4}).
Using this formulation it can be seen that a necessary (but not sufficient)
condition for an extortionate strategy is that it cooperates on average less
than 50\% of the time when in a state of disagreement with the opponent.

As an example, consider the known extortionate strategy \(p=(8 / 9, 1 / 2, 1 /
3, 0)\) from~\cite{Stewart2012} which is referred to as \texttt{Extort-2}. In
this case, for the standard values of \((R, T, S, P)\) constraint
(\ref{eqn:condition_for_p1}) corresponds to:

\begin{equation}
    p_1 = \frac{2(p_2 + p_3) + 1}{3}
\end{equation}

It is clear that in this case all constraints hold.

This approach could in fact be used to confirm that a given strategy is acting
in an extortionate manner even if it is not a memory one strategy. However, in
practice, if a closed form for \(p\) is not known, then due to measurement
and/or numerical error this would not work.

This problem can be written in the following linear algebraic form where
\(x=(\alpha, \beta)\)
and \(p^*=(\tilde p_1 - 1, tilde_2 - 1, p_3)\):

\begin{equation}\label{eqn:linear_algebraic_equation_for_p}
    Cx= p^*
\end{equation}

\(C\) corresponds to equations
(\ref{eqn:condition_for_tilde_p1}-\ref{eqn:condition_for_tilde_p3}) and is
given by:

\begin{equation}\label{eqn:definition_of_C}
    C =
    \begin{bmatrix}
        R - P & R- P \\
        S - P & T- P \\
        T - P & S- P \\
    \end{bmatrix}
\end{equation}

Note that in general, equation (\ref{eqn:linear_algebraic_equation_for_p}) will
not necessarily have a solution. From the Rouch\'{e}-Capelli theorem if there is
a solution it is unique as \(\text{rank}(C)=2\) which is the dimension of the
variable \(x\). The best fitting \(x\) is found by minimizing:

\begin{equation}\label{eqn:r_squared}
    \text{SSError} = \|C x- p^*\|_2^2 = \sum_{i=1}^{3}\left((C\bar x)_i-p_i^*\right)^2
\end{equation}

Note that \(\text{SSError}\), which is the square of the Frobenius
norm~\cite{Golub2013}, becomes a measure of how close a strategy is to being an
extortionate strategy. Suspicion
of extortion then corresponds to a threshold on \(\text{SSError}\).

By observing interactions (human or otherwise), their memory one representation
can be inferred and this approach can be used to recognise extortionate
behaviour. The notion of comparing theoretic and actual plays of the IPD is not
novel, see for example~\cite{Rand2013}. Immediately it is noted that if the
environment is noisy~\cite{Wu1995} then no strategy can be considered to be
extortionate as \(p_4>0\).

In the next section, this idea will be illustrated by observing the interactions
that take place in a computer based tournament of the IPD\@.

\section{Numerical experiments}\label{sec:numerical-experiments}

In~\cite{Stewart2012} results from a tournament with
\input{./assets/tex/number_of_stewart_plotkin_strategies/main.tex} strategies,
was presented with specific consideration given to ZD strategies. This
tournament is reproduced here using the Axelrod-Python
project~\cite{Knight2016}. To obtain a good measure of the corresponding
transition rates for each strategy all matches have been run for
\input{assets/tex/number_of_turns/main.tex} turns and every match has been
repeated \input{assets/tex/number_of_repetitions/main.tex} times. All of this
interaction data is available at~\cite{vincent_knight_2018_1297075}. A good
match between the inferred Markov chain and the state distribution of the actual
interactions has been verified. Data for this is presented in the supplementary
materials.

Figure~\ref{fig:SSError_overall_in_stewart_plotkin} shows the \(\text{SSError}\)
values for all the strategies in the tournament, as reported
in~\cite{Stewart2012} the extortionate strategy (which has an expected
\(\text{SSError}\) approximately 0) gains a large number of wins.

\begin{figure}[!htbp]
    \centering
    \includegraphics[width=.8\textwidth]{./assets/img/SSError_overall_in_stewart_plotkin/main.pdf}
    \caption{\(\text{SSError}\) and state probabilities for the strategies
        of~\cite{Stewart2012}, ordered both by number of wins and overall score.
        Note that \(P(DC)\) is not shown as it corresponds to the transpose of
        \(P(CD)\). Cooperator and Defector are omitted as they do not visit all
        the states.}
    \label{fig:SSError_overall_in_stewart_plotkin}
\end{figure}

Here, the work of~\cite{Stewart2012} is extended by investigating a tournament
with \input{assets/tex/number_of_full_strategies/main.tex}
strategies.

The results of this analysis are shown in
Figure~\ref{fig:SSError_and_probabilities_in_full}. The top ranking strategies
by number of wins seem to be extortionate (but not against all strategies) and
it can be seen that a small sub group of strategies achieve mutual defection.
All the top ranking strategies according to score achieve mutual cooperation and
do not extort each other, however they
\textbf{do} exhibit extortionate behaviour towards a number of the lower ranking
strategies.

\begin{figure}[!htbp]
    \centering
    \includegraphics[width=.8\textwidth]{./assets/img/SSError_and_probabilities_in_full/main.pdf}
    \caption{\(\text{SSError}\) for the strategies for the full tournament. Only
    strategy interactions for which \(p_4=0\) and \(\chi>1\) are displayed.}
    \label{fig:SSError_and_probabilities_in_full}
\end{figure}

\section{Conclusion}\label{sec:conclusion}

This work defines an approach to measure whether or not a player is playing a
strategy that corresponds to an extortionate strategy as defined
in~\cite{Press2012}: a mathematical model for suspicion. Indeed, all
extortionate strategies have been
 classified as lying on a triangular plane.
This rigorous classification fails to be robust to small measurement error, thus
a statistical approach is proposed.
This is done through a linear algebraic approach for approximating the solution
of a linear system. Using this, a large number of pairwise interactions is
simulated and in fact very few strategies are found to act extortionately.

The work of~\cite{Press2012}, whilst showing that a clever approach to taking
advantage of another memory one strategy exists: this is incomplete. Whilst the
elegance of this result is very attractive, just as the simplicity of the
victory of Tit For Tat in Axelrod's original tournaments was, it is incomplete.
Extortionate strategies achieve a high number of wins but they do not
achieve a high score which corresponds to the fitness landscape in an
evolutionary sense. From the large number of interactions a payoff matrix \(S\)
can be measured where \(S_{ij}\) denotes the score (using standard values of
\((R, S, T, P) = (3, 0, 5, 1)\)) of the \(i\)th strategy
against the \(j\)th strategy. Using this, the replicator equation
describes the evolution of the system based on a population density fitness
function:

\begin{equation}\label{eqn:replicator_dynamics}
    \frac{dx}{dt} = x(S-x^TS x)
\end{equation}

Equation (\ref{eqn:replicator_dynamics}) is solved numerically through an
integration technique described in~\cite{Petzold1983} and
Figure~\ref{fig:replicator_dynamics} shows the evolution of the distribution of
the system: the various strategies are ranked by scores. It is clear to see that
only the high ranking strategies survive the evolutionary process (in fact,
only \input{./assets/img/replicator_dynamics/main.tex}
have a final distribution greater than \(10 ^ {-2}\)). This confirms the
findings of~\cite{Moran1707} in which sophisticated strategies resist
evolutionary invasion of shorter memory strategies. Recalling
Figure~\ref{fig:SSError_and_probabilities_in_full} this demonstrates that:

\begin{itemize}
    \item Cooperation emerges through the evolutionary process: the high scoring
        strategies do not exhibit extortionate behaviour towards each other.
    \item Extortionate strategies do not survive the evolutionary process.
\end{itemize}

\begin{figure}[!htbp]
    \centering
    \includegraphics[width=.8\textwidth]{./assets/img/replicator_dynamics/main.pdf}
    \caption{Numerical simulation of the replicator equation
    (\ref{eqn:replicator_dynamics}): strategies are ordered by score, only the strategies with a high score survive the evolutionary process.}
    \label{fig:replicator_dynamics}
\end{figure}

This work can be used to classify plays of the IPD\@: data can be collected from
actual interactions (in lab or in the field). Furthermore, this allows for a
classification method similar to the notion of fingerprinting presented
in~\cite{Ashlock2008}. Trained strategies can potentially be classified as
extortionate or not or it could be possible to even constrain the reinforcement
learning approaches that are becoming prevalent in the literature.
Alternatively, this mathematical approach for recognising extortion could be
used in sophisticated strategies to defend against invasion. Arguably, some of
the strategies considered here exhibit this behaviour, indeed as described
in~\cite{Harper2017}, the top ranking strategies in the full tournament are
obtained using evolutionary reinforcement learning techniques, thus, suspicion
of extortionate behaviour could in fact be an evolutionary trait.

\section*{Acknowledgements}

The following open source software libraries were used in this research:

\begin{itemize}
    \item The Axelrod ~\cite{Knight2016, Knight2018} library (IPD strategies and
        tournaments).
    \item The sympy library~\cite{Meurer2017} (verification of all symbolic
        calculations).
    \item The matplotlib~\cite{Droettboom2018} library (visualisation).
    \item The pandas~\cite{Structures2010}, dask~\cite{Dask2016} and
        NumPy~\cite{Oliphant2015} libraries (data manipulation).
    \item The SciPy~\cite{Jones2001} library (numerical integration of the
        replicator equation).
\end{itemize}

This work was performed using the computational facilities of the Advanced
Research Computing @ Cardiff (ARCCA) Division, Cardiff University.

\printbibliography

\newpage
\section*{Supplementary materials}

\includepdf{assets/pdf/proof_of_form_of_extortionate_strategies/main.pdf}

\newpage

Using the pair wise interactions the transition rates \(p,
q\) can be measured and the steady state probabilities inferred and compared to
the actual probabilities of each state.
This is done numerically by computing the singular eigenvector of the
matrix \(A\) \cite{Stewart2009}:

\[
    A =
    \begin{bmatrix}
        p_1 q_1 & p_1 (1 - q_1) & (1 - p_1) q_1 & (1 -p_1) (1 - q_1) \\
        p_2 q_2 & p_2 (1 - q_2) & (1 - p_2) q_2 & (1 -p_2) (1 - q_2) \\
        p_3 q_3 & p_3 (1 - q_3) & (1 - p_3) q_3 & (1 -p_3) (1 - q_3) \\
        p_4 q_4 & p_4 (1 - q_4) & (1 - p_4) q_4 & (1 -p_4) (1 - q_4) \\
    \end{bmatrix}
\]

Figure~\ref{fig:computed_probabilities_vs_theoretic_probabilities} shows a
regression line fitted to every pairwise interaction with a reported
\(\text{SSError}\) value (pairwise interactions with missing states were
omitted). This serves to validate the approach: a part from some edge cases the
relationship is consistent.

\begin{figure}[!htbp]
    \centering
    \includegraphics[width=.8\textwidth]{./assets/img/computed_probabilities_vs_theoretic_probabilities/main.pdf}
    \caption{The
        relationship between the steady state probabilities inferred from the
        measured transitions and the actual steady state probabilities. A linear
        regression line is included validating the approach.}
    \label{fig:computed_probabilities_vs_theoretic_probabilities}
\end{figure}


\end{document}
 times. All of this
interaction data is available at~\cite{vincent_knight_2018_1297075}. A good
match between the inferred Markov chain and the state distribution of the actual
interactions has been verified. Data for this is presented in the supplementary
materials.

Figure~\ref{fig:SSError_overall_in_stewart_plotkin} shows the \(\text{SSError}\)
values for all the strategies in the tournament, as reported
in~\cite{Stewart2012} the extortionate strategy (which has an expected
\(\text{SSError}\) approximately 0) gains a large number of wins.

\begin{figure}[!htbp]
    \centering
    \includegraphics[width=.8\textwidth]{./assets/img/SSError_overall_in_stewart_plotkin/main.pdf}
    \caption{\(\text{SSError}\) and state probabilities for the strategies
        of~\cite{Stewart2012}, ordered both by number of wins and overall score.
        Note that \(P(DC)\) is not shown as it corresponds to the transpose of
        \(P(CD)\). Cooperator and Defector are omitted as they do not visit all
        the states.}
    \label{fig:SSError_overall_in_stewart_plotkin}
\end{figure}

Here, the work of~\cite{Stewart2012} is extended by investigating a tournament
with \documentclass[a4paper]{article}

\usepackage{amsmath}
\usepackage{amssymb}
\usepackage[margin=1.5cm,
            includefoot,
            footskip=30pt]{geometry}
\usepackage{layout}
\usepackage{graphicx}
\usepackage{subcaption}

\usepackage{biblatex}
\usepackage{pdfpages}

\bibliography{main.bib}

\title{Suspicion: Recognising and evaluating the effectiveness
       of extortion in the Iterated Prisoner's Dilemma}
\author{Vincent A. Knight \and Nikoleta E. Glynatsi}
\date{\today}



\begin{document}

\maketitle

\begin{abstract}
    The Iterated Prisoner's Dilemma is a model for rational and evolutionary
    interactive behaviour. It has applications both in the study of human social
    behaviour as well as in biology.
    It is used to understand when and how a rational individual might
    accept an immediate cost to their own utility for the direct benefit of
    another.

    Much attention has been given to a class of strategies called
    Zero Determinant strategies. It has been theoretically shown that these
    strategies can ``extort'' any player.

    In this work, an approach to identify if observed strategies are playing in
    an extortionate way is described. Furthermore, experimental analysis of
    a large tournament with \input{assets/tex/number_of_full_strategies/main.tex}
    strategies is considered. In this setting
    the most highly performing strategies do not play in an extortionate way
    against each other but do against lower performing strategies.
    This suggests that whilst the theory of Zero Determinant strategies
    indicates that memory is not of fundamental importance to the evolution of
    cooperative behaviour, this is incomplete.
\end{abstract}

\section{Introduction}\label{sec:introduction}

Agent based game theoretic models have become a stalwart of the underpinning
mathematics of interactive behaviours. One of the major pieces of work
in this area is the pair of original computer tournaments run by Robert
Axelrod~\cite{Axelrod1980, Axelrod1980a}. These tournaments pitted submitted
computer strategies against each other in plays of the Iterated Prisoner's
Dilemma. A common game where agents can choose to pay a slight cost to their
immediate utility in the hope of building a reputation. This has been used in
economic and evolutionary game theory to understand the evolution of cooperative
behaviour.

Recently, a class of strategies was described in~\cite{Press2012} that can
provably extort any given opponent. In~\cite{Hilbe2013, Moran1707} some
questions have already been asked about the true effectiveness of these
strategies in an evolutionary setting. Here another question is asked: is it
possible to recognise this extortionate behaviour? A mathematical procedure for
suspicion is presented: in the same way that the continued actions of an
extortionate individual might raise suspicion.

This work makes use of the Axelrod Python library~\cite{Knight2018, Knight2016}
with a large number of Prisoner Dilemma strategies available to give an
extensive numerical example of the ideas presented.  The approach is presented
in Section~\ref{sec:delta-zd-strategies}.  All of the code and data discussed
in Section~\ref{sec:numerical-experiments} is open sourced, archived and
written according to best scientific principles~\cite{Wilson2014}. The data
archive can be found at~\cite{vincent_knight_2018_1297075}.

\section{Recognising Extortion}\label{sec:delta-zd-strategies}

In~\cite{Press2012}, given a match between 2 memory-one strategies, the concept
of Zero Determinant (ZD) strategies is introduced. The main result of that paper
shows that given two memory one players \(p, q\in\mathbb{R}^4\) a linear
relationship between the players' scores could be forced by one of the players.

Using the notation of~\cite{Press2012}, assuming the utilities for player \(p\)
are given by \(S_x=(R, S, T, P)\) and for player \(q\) by \(S_y=(R, T, S, P)\)
and that the stationary scores of each player is given by \(S_X\) and \(S_Y\)
respectively. The main result of~\cite{Press2012} is that if

\begin{equation}\label{eqn:linear_relationship_for_p}
    \tilde p=\alpha S_x + \beta S_y + \gamma
\end{equation}

or

\begin{equation}\label{eqn:linear_relationship_for_q}
    \tilde q=\alpha S_x + \beta S_y + \gamma
\end{equation}

where \(\tilde p = (1 - p_1, 1 - p_2, p_3, p_4)\) and
\(\tilde q = (1 - q_1, 1 - q_2, q_3, q_4)\) then:

\begin{equation}
    \alpha S_X + \beta S_Y + \gamma = 0
\end{equation}

In~\cite{Press2012} a particular type of ZD strategy is defined: extortionate
strategies. If:

\begin{equation}\label{eqn:constraint_for_extortion}
    \gamma = - P(\alpha + \beta)
\end{equation}

then the player can ensure they get a score \(\chi\) times
larger than the opponent. This extortion coefficient is given by:

\begin{equation}\label{eqn:definition_of_chi}
    \chi=\frac{-\beta}{\alpha}
\end{equation}

Thus, if (\ref{eqn:constraint_for_extortion}) holds and \(\chi >1\) a player is
said to extort their opponent.
Here, the reverse problem is considered: given a
\(p\in\mathbb{R}^4\) how does one identify \(\alpha, \beta\) if they
exist and is the strategy in fact acting in an extortionate way?

These conditions correspond to:

\begin{align}
    \tilde p_1 & = \alpha R + \beta R - P (\alpha + \beta)
            \label{eqn:condition_for_tilde_p1}\\
    \tilde p_2 & = \alpha S + \beta T - P (\alpha + \beta)
            \label{eqn:condition_for_tilde_p2}\\
    \tilde p_3 & = \alpha T + \beta S - P (\alpha + \beta)
            \label{eqn:condition_for_tilde_p3}\\
    \tilde p_4 & = \alpha P + \beta P - P (\alpha + \beta)
            \label{eqn:condition_for_tilde_p4}
\end{align}

Equation (\ref{eqn:condition_for_tilde_p4}) ensures that \(p_4=\tilde p_4=0\).
Equations (\ref{eqn:condition_for_tilde_p1}-\ref{eqn:condition_for_tilde_p3})
can be used to eliminate \(\alpha, \beta\), giving:

\begin{equation}\label{eqn:planar_definition_of_extortion}
    \tilde p_1 = \frac{(R - P)(\tilde p_2 + \tilde p_3)}{S + T - 2P}
\end{equation}

with:

\begin{equation}\label{eqn:definition_of_chi}
    \chi = \frac{\tilde p_2 (P - T) + \tilde p_3 (S - P)}
                {\tilde p_2 (P - S) + \tilde p_3 (T - P)}
\end{equation}

Given a strategy \(p\in\mathbb{R}^{4\times 1}\) equations
(\ref{eqn:condition_for_tilde_p4}), (\ref{eqn:planar_definition_of_extortion}-\ref{eqn:definition_of_chi}) can be used to check if
a strategy is extortionate. The conditions correspond to:

\begin{align}
    p_1 & = \frac{(R-P)(p_2 + p_3) - R + T + S - P}{S + T - 2P}
     \label{eqn:condition_for_p1}\\
    p_4 & = 0 \label{eqn:condition_for_p4}\\
    1 & > p_2 + p_3\label{eqn:condition_for_chi}
\end{align}

The algebraic steps necessary to prove these results are available in the
supporting materials.

All extortionate strategies reside on a triangular (\ref{eqn:condition_for_chi})
plane (\ref{eqn:condition_for_p1}) in 3 dimensions (\ref{eqn:condition_for_p4}).
Using this formulation it can be seen that a necessary (but not sufficient)
condition for an extortionate strategy is that it cooperates on average less
than 50\% of the time when in a state of disagreement with the opponent.

As an example, consider the known extortionate strategy \(p=(8 / 9, 1 / 2, 1 /
3, 0)\) from~\cite{Stewart2012} which is referred to as \texttt{Extort-2}. In
this case, for the standard values of \((R, T, S, P)\) constraint
(\ref{eqn:condition_for_p1}) corresponds to:

\begin{equation}
    p_1 = \frac{2(p_2 + p_3) + 1}{3}
\end{equation}

It is clear that in this case all constraints hold.

This approach could in fact be used to confirm that a given strategy is acting
in an extortionate manner even if it is not a memory one strategy. However, in
practice, if a closed form for \(p\) is not known, then due to measurement
and/or numerical error this would not work.

This problem can be written in the following linear algebraic form where
\(x=(\alpha, \beta)\)
and \(p^*=(\tilde p_1 - 1, tilde_2 - 1, p_3)\):

\begin{equation}\label{eqn:linear_algebraic_equation_for_p}
    Cx= p^*
\end{equation}

\(C\) corresponds to equations
(\ref{eqn:condition_for_tilde_p1}-\ref{eqn:condition_for_tilde_p3}) and is
given by:

\begin{equation}\label{eqn:definition_of_C}
    C =
    \begin{bmatrix}
        R - P & R- P \\
        S - P & T- P \\
        T - P & S- P \\
    \end{bmatrix}
\end{equation}

Note that in general, equation (\ref{eqn:linear_algebraic_equation_for_p}) will
not necessarily have a solution. From the Rouch\'{e}-Capelli theorem if there is
a solution it is unique as \(\text{rank}(C)=2\) which is the dimension of the
variable \(x\). The best fitting \(x\) is found by minimizing:

\begin{equation}\label{eqn:r_squared}
    \text{SSError} = \|C x- p^*\|_2^2 = \sum_{i=1}^{3}\left((C\bar x)_i-p_i^*\right)^2
\end{equation}

Note that \(\text{SSError}\), which is the square of the Frobenius
norm~\cite{Golub2013}, becomes a measure of how close a strategy is to being an
extortionate strategy. Suspicion
of extortion then corresponds to a threshold on \(\text{SSError}\).

By observing interactions (human or otherwise), their memory one representation
can be inferred and this approach can be used to recognise extortionate
behaviour. The notion of comparing theoretic and actual plays of the IPD is not
novel, see for example~\cite{Rand2013}. Immediately it is noted that if the
environment is noisy~\cite{Wu1995} then no strategy can be considered to be
extortionate as \(p_4>0\).

In the next section, this idea will be illustrated by observing the interactions
that take place in a computer based tournament of the IPD\@.

\section{Numerical experiments}\label{sec:numerical-experiments}

In~\cite{Stewart2012} results from a tournament with
\input{./assets/tex/number_of_stewart_plotkin_strategies/main.tex} strategies,
was presented with specific consideration given to ZD strategies. This
tournament is reproduced here using the Axelrod-Python
project~\cite{Knight2016}. To obtain a good measure of the corresponding
transition rates for each strategy all matches have been run for
\input{assets/tex/number_of_turns/main.tex} turns and every match has been
repeated \input{assets/tex/number_of_repetitions/main.tex} times. All of this
interaction data is available at~\cite{vincent_knight_2018_1297075}. A good
match between the inferred Markov chain and the state distribution of the actual
interactions has been verified. Data for this is presented in the supplementary
materials.

Figure~\ref{fig:SSError_overall_in_stewart_plotkin} shows the \(\text{SSError}\)
values for all the strategies in the tournament, as reported
in~\cite{Stewart2012} the extortionate strategy (which has an expected
\(\text{SSError}\) approximately 0) gains a large number of wins.

\begin{figure}[!htbp]
    \centering
    \includegraphics[width=.8\textwidth]{./assets/img/SSError_overall_in_stewart_plotkin/main.pdf}
    \caption{\(\text{SSError}\) and state probabilities for the strategies
        of~\cite{Stewart2012}, ordered both by number of wins and overall score.
        Note that \(P(DC)\) is not shown as it corresponds to the transpose of
        \(P(CD)\). Cooperator and Defector are omitted as they do not visit all
        the states.}
    \label{fig:SSError_overall_in_stewart_plotkin}
\end{figure}

Here, the work of~\cite{Stewart2012} is extended by investigating a tournament
with \input{assets/tex/number_of_full_strategies/main.tex}
strategies.

The results of this analysis are shown in
Figure~\ref{fig:SSError_and_probabilities_in_full}. The top ranking strategies
by number of wins seem to be extortionate (but not against all strategies) and
it can be seen that a small sub group of strategies achieve mutual defection.
All the top ranking strategies according to score achieve mutual cooperation and
do not extort each other, however they
\textbf{do} exhibit extortionate behaviour towards a number of the lower ranking
strategies.

\begin{figure}[!htbp]
    \centering
    \includegraphics[width=.8\textwidth]{./assets/img/SSError_and_probabilities_in_full/main.pdf}
    \caption{\(\text{SSError}\) for the strategies for the full tournament. Only
    strategy interactions for which \(p_4=0\) and \(\chi>1\) are displayed.}
    \label{fig:SSError_and_probabilities_in_full}
\end{figure}

\section{Conclusion}\label{sec:conclusion}

This work defines an approach to measure whether or not a player is playing a
strategy that corresponds to an extortionate strategy as defined
in~\cite{Press2012}: a mathematical model for suspicion. Indeed, all
extortionate strategies have been
 classified as lying on a triangular plane.
This rigorous classification fails to be robust to small measurement error, thus
a statistical approach is proposed.
This is done through a linear algebraic approach for approximating the solution
of a linear system. Using this, a large number of pairwise interactions is
simulated and in fact very few strategies are found to act extortionately.

The work of~\cite{Press2012}, whilst showing that a clever approach to taking
advantage of another memory one strategy exists: this is incomplete. Whilst the
elegance of this result is very attractive, just as the simplicity of the
victory of Tit For Tat in Axelrod's original tournaments was, it is incomplete.
Extortionate strategies achieve a high number of wins but they do not
achieve a high score which corresponds to the fitness landscape in an
evolutionary sense. From the large number of interactions a payoff matrix \(S\)
can be measured where \(S_{ij}\) denotes the score (using standard values of
\((R, S, T, P) = (3, 0, 5, 1)\)) of the \(i\)th strategy
against the \(j\)th strategy. Using this, the replicator equation
describes the evolution of the system based on a population density fitness
function:

\begin{equation}\label{eqn:replicator_dynamics}
    \frac{dx}{dt} = x(S-x^TS x)
\end{equation}

Equation (\ref{eqn:replicator_dynamics}) is solved numerically through an
integration technique described in~\cite{Petzold1983} and
Figure~\ref{fig:replicator_dynamics} shows the evolution of the distribution of
the system: the various strategies are ranked by scores. It is clear to see that
only the high ranking strategies survive the evolutionary process (in fact,
only \input{./assets/img/replicator_dynamics/main.tex}
have a final distribution greater than \(10 ^ {-2}\)). This confirms the
findings of~\cite{Moran1707} in which sophisticated strategies resist
evolutionary invasion of shorter memory strategies. Recalling
Figure~\ref{fig:SSError_and_probabilities_in_full} this demonstrates that:

\begin{itemize}
    \item Cooperation emerges through the evolutionary process: the high scoring
        strategies do not exhibit extortionate behaviour towards each other.
    \item Extortionate strategies do not survive the evolutionary process.
\end{itemize}

\begin{figure}[!htbp]
    \centering
    \includegraphics[width=.8\textwidth]{./assets/img/replicator_dynamics/main.pdf}
    \caption{Numerical simulation of the replicator equation
    (\ref{eqn:replicator_dynamics}): strategies are ordered by score, only the strategies with a high score survive the evolutionary process.}
    \label{fig:replicator_dynamics}
\end{figure}

This work can be used to classify plays of the IPD\@: data can be collected from
actual interactions (in lab or in the field). Furthermore, this allows for a
classification method similar to the notion of fingerprinting presented
in~\cite{Ashlock2008}. Trained strategies can potentially be classified as
extortionate or not or it could be possible to even constrain the reinforcement
learning approaches that are becoming prevalent in the literature.
Alternatively, this mathematical approach for recognising extortion could be
used in sophisticated strategies to defend against invasion. Arguably, some of
the strategies considered here exhibit this behaviour, indeed as described
in~\cite{Harper2017}, the top ranking strategies in the full tournament are
obtained using evolutionary reinforcement learning techniques, thus, suspicion
of extortionate behaviour could in fact be an evolutionary trait.

\section*{Acknowledgements}

The following open source software libraries were used in this research:

\begin{itemize}
    \item The Axelrod ~\cite{Knight2016, Knight2018} library (IPD strategies and
        tournaments).
    \item The sympy library~\cite{Meurer2017} (verification of all symbolic
        calculations).
    \item The matplotlib~\cite{Droettboom2018} library (visualisation).
    \item The pandas~\cite{Structures2010}, dask~\cite{Dask2016} and
        NumPy~\cite{Oliphant2015} libraries (data manipulation).
    \item The SciPy~\cite{Jones2001} library (numerical integration of the
        replicator equation).
\end{itemize}

This work was performed using the computational facilities of the Advanced
Research Computing @ Cardiff (ARCCA) Division, Cardiff University.

\printbibliography

\newpage
\section*{Supplementary materials}

\includepdf{assets/pdf/proof_of_form_of_extortionate_strategies/main.pdf}

\newpage

Using the pair wise interactions the transition rates \(p,
q\) can be measured and the steady state probabilities inferred and compared to
the actual probabilities of each state.
This is done numerically by computing the singular eigenvector of the
matrix \(A\) \cite{Stewart2009}:

\[
    A =
    \begin{bmatrix}
        p_1 q_1 & p_1 (1 - q_1) & (1 - p_1) q_1 & (1 -p_1) (1 - q_1) \\
        p_2 q_2 & p_2 (1 - q_2) & (1 - p_2) q_2 & (1 -p_2) (1 - q_2) \\
        p_3 q_3 & p_3 (1 - q_3) & (1 - p_3) q_3 & (1 -p_3) (1 - q_3) \\
        p_4 q_4 & p_4 (1 - q_4) & (1 - p_4) q_4 & (1 -p_4) (1 - q_4) \\
    \end{bmatrix}
\]

Figure~\ref{fig:computed_probabilities_vs_theoretic_probabilities} shows a
regression line fitted to every pairwise interaction with a reported
\(\text{SSError}\) value (pairwise interactions with missing states were
omitted). This serves to validate the approach: a part from some edge cases the
relationship is consistent.

\begin{figure}[!htbp]
    \centering
    \includegraphics[width=.8\textwidth]{./assets/img/computed_probabilities_vs_theoretic_probabilities/main.pdf}
    \caption{The
        relationship between the steady state probabilities inferred from the
        measured transitions and the actual steady state probabilities. A linear
        regression line is included validating the approach.}
    \label{fig:computed_probabilities_vs_theoretic_probabilities}
\end{figure}


\end{document}

strategies.

The results of this analysis are shown in
Figure~\ref{fig:SSError_and_probabilities_in_full}. The top ranking strategies
by number of wins seem to be extortionate (but not against all strategies) and
it can be seen that a small sub group of strategies achieve mutual defection.
All the top ranking strategies according to score achieve mutual cooperation and
do not extort each other, however they
\textbf{do} exhibit extortionate behaviour towards a number of the lower ranking
strategies.

\begin{figure}[!htbp]
    \centering
    \includegraphics[width=.8\textwidth]{./assets/img/SSError_and_probabilities_in_full/main.pdf}
    \caption{\(\text{SSError}\) for the strategies for the full tournament. Only
    strategy interactions for which \(p_4=0\) and \(\chi>1\) are displayed.}
    \label{fig:SSError_and_probabilities_in_full}
\end{figure}

\section{Conclusion}\label{sec:conclusion}

This work defines an approach to measure whether or not a player is playing a
strategy that corresponds to an extortionate strategy as defined
in~\cite{Press2012}: a mathematical model for suspicion. Indeed, all
extortionate strategies have been
 classified as lying on a triangular plane.
This rigorous classification fails to be robust to small measurement error, thus
a statistical approach is proposed.
This is done through a linear algebraic approach for approximating the solution
of a linear system. Using this, a large number of pairwise interactions is
simulated and in fact very few strategies are found to act extortionately.

The work of~\cite{Press2012}, whilst showing that a clever approach to taking
advantage of another memory one strategy exists: this is incomplete. Whilst the
elegance of this result is very attractive, just as the simplicity of the
victory of Tit For Tat in Axelrod's original tournaments was, it is incomplete.
Extortionate strategies achieve a high number of wins but they do not
achieve a high score which corresponds to the fitness landscape in an
evolutionary sense. From the large number of interactions a payoff matrix \(S\)
can be measured where \(S_{ij}\) denotes the score (using standard values of
\((R, S, T, P) = (3, 0, 5, 1)\)) of the \(i\)th strategy
against the \(j\)th strategy. Using this, the replicator equation
describes the evolution of the system based on a population density fitness
function:

\begin{equation}\label{eqn:replicator_dynamics}
    \frac{dx}{dt} = x(S-x^TS x)
\end{equation}

Equation (\ref{eqn:replicator_dynamics}) is solved numerically through an
integration technique described in~\cite{Petzold1983} and
Figure~\ref{fig:replicator_dynamics} shows the evolution of the distribution of
the system: the various strategies are ranked by scores. It is clear to see that
only the high ranking strategies survive the evolutionary process (in fact,
only \documentclass[a4paper]{article}

\usepackage{amsmath}
\usepackage{amssymb}
\usepackage[margin=1.5cm,
            includefoot,
            footskip=30pt]{geometry}
\usepackage{layout}
\usepackage{graphicx}
\usepackage{subcaption}

\usepackage{biblatex}
\usepackage{pdfpages}

\bibliography{main.bib}

\title{Suspicion: Recognising and evaluating the effectiveness
       of extortion in the Iterated Prisoner's Dilemma}
\author{Vincent A. Knight \and Nikoleta E. Glynatsi}
\date{\today}



\begin{document}

\maketitle

\begin{abstract}
    The Iterated Prisoner's Dilemma is a model for rational and evolutionary
    interactive behaviour. It has applications both in the study of human social
    behaviour as well as in biology.
    It is used to understand when and how a rational individual might
    accept an immediate cost to their own utility for the direct benefit of
    another.

    Much attention has been given to a class of strategies called
    Zero Determinant strategies. It has been theoretically shown that these
    strategies can ``extort'' any player.

    In this work, an approach to identify if observed strategies are playing in
    an extortionate way is described. Furthermore, experimental analysis of
    a large tournament with \input{assets/tex/number_of_full_strategies/main.tex}
    strategies is considered. In this setting
    the most highly performing strategies do not play in an extortionate way
    against each other but do against lower performing strategies.
    This suggests that whilst the theory of Zero Determinant strategies
    indicates that memory is not of fundamental importance to the evolution of
    cooperative behaviour, this is incomplete.
\end{abstract}

\section{Introduction}\label{sec:introduction}

Agent based game theoretic models have become a stalwart of the underpinning
mathematics of interactive behaviours. One of the major pieces of work
in this area is the pair of original computer tournaments run by Robert
Axelrod~\cite{Axelrod1980, Axelrod1980a}. These tournaments pitted submitted
computer strategies against each other in plays of the Iterated Prisoner's
Dilemma. A common game where agents can choose to pay a slight cost to their
immediate utility in the hope of building a reputation. This has been used in
economic and evolutionary game theory to understand the evolution of cooperative
behaviour.

Recently, a class of strategies was described in~\cite{Press2012} that can
provably extort any given opponent. In~\cite{Hilbe2013, Moran1707} some
questions have already been asked about the true effectiveness of these
strategies in an evolutionary setting. Here another question is asked: is it
possible to recognise this extortionate behaviour? A mathematical procedure for
suspicion is presented: in the same way that the continued actions of an
extortionate individual might raise suspicion.

This work makes use of the Axelrod Python library~\cite{Knight2018, Knight2016}
with a large number of Prisoner Dilemma strategies available to give an
extensive numerical example of the ideas presented.  The approach is presented
in Section~\ref{sec:delta-zd-strategies}.  All of the code and data discussed
in Section~\ref{sec:numerical-experiments} is open sourced, archived and
written according to best scientific principles~\cite{Wilson2014}. The data
archive can be found at~\cite{vincent_knight_2018_1297075}.

\section{Recognising Extortion}\label{sec:delta-zd-strategies}

In~\cite{Press2012}, given a match between 2 memory-one strategies, the concept
of Zero Determinant (ZD) strategies is introduced. The main result of that paper
shows that given two memory one players \(p, q\in\mathbb{R}^4\) a linear
relationship between the players' scores could be forced by one of the players.

Using the notation of~\cite{Press2012}, assuming the utilities for player \(p\)
are given by \(S_x=(R, S, T, P)\) and for player \(q\) by \(S_y=(R, T, S, P)\)
and that the stationary scores of each player is given by \(S_X\) and \(S_Y\)
respectively. The main result of~\cite{Press2012} is that if

\begin{equation}\label{eqn:linear_relationship_for_p}
    \tilde p=\alpha S_x + \beta S_y + \gamma
\end{equation}

or

\begin{equation}\label{eqn:linear_relationship_for_q}
    \tilde q=\alpha S_x + \beta S_y + \gamma
\end{equation}

where \(\tilde p = (1 - p_1, 1 - p_2, p_3, p_4)\) and
\(\tilde q = (1 - q_1, 1 - q_2, q_3, q_4)\) then:

\begin{equation}
    \alpha S_X + \beta S_Y + \gamma = 0
\end{equation}

In~\cite{Press2012} a particular type of ZD strategy is defined: extortionate
strategies. If:

\begin{equation}\label{eqn:constraint_for_extortion}
    \gamma = - P(\alpha + \beta)
\end{equation}

then the player can ensure they get a score \(\chi\) times
larger than the opponent. This extortion coefficient is given by:

\begin{equation}\label{eqn:definition_of_chi}
    \chi=\frac{-\beta}{\alpha}
\end{equation}

Thus, if (\ref{eqn:constraint_for_extortion}) holds and \(\chi >1\) a player is
said to extort their opponent.
Here, the reverse problem is considered: given a
\(p\in\mathbb{R}^4\) how does one identify \(\alpha, \beta\) if they
exist and is the strategy in fact acting in an extortionate way?

These conditions correspond to:

\begin{align}
    \tilde p_1 & = \alpha R + \beta R - P (\alpha + \beta)
            \label{eqn:condition_for_tilde_p1}\\
    \tilde p_2 & = \alpha S + \beta T - P (\alpha + \beta)
            \label{eqn:condition_for_tilde_p2}\\
    \tilde p_3 & = \alpha T + \beta S - P (\alpha + \beta)
            \label{eqn:condition_for_tilde_p3}\\
    \tilde p_4 & = \alpha P + \beta P - P (\alpha + \beta)
            \label{eqn:condition_for_tilde_p4}
\end{align}

Equation (\ref{eqn:condition_for_tilde_p4}) ensures that \(p_4=\tilde p_4=0\).
Equations (\ref{eqn:condition_for_tilde_p1}-\ref{eqn:condition_for_tilde_p3})
can be used to eliminate \(\alpha, \beta\), giving:

\begin{equation}\label{eqn:planar_definition_of_extortion}
    \tilde p_1 = \frac{(R - P)(\tilde p_2 + \tilde p_3)}{S + T - 2P}
\end{equation}

with:

\begin{equation}\label{eqn:definition_of_chi}
    \chi = \frac{\tilde p_2 (P - T) + \tilde p_3 (S - P)}
                {\tilde p_2 (P - S) + \tilde p_3 (T - P)}
\end{equation}

Given a strategy \(p\in\mathbb{R}^{4\times 1}\) equations
(\ref{eqn:condition_for_tilde_p4}), (\ref{eqn:planar_definition_of_extortion}-\ref{eqn:definition_of_chi}) can be used to check if
a strategy is extortionate. The conditions correspond to:

\begin{align}
    p_1 & = \frac{(R-P)(p_2 + p_3) - R + T + S - P}{S + T - 2P}
     \label{eqn:condition_for_p1}\\
    p_4 & = 0 \label{eqn:condition_for_p4}\\
    1 & > p_2 + p_3\label{eqn:condition_for_chi}
\end{align}

The algebraic steps necessary to prove these results are available in the
supporting materials.

All extortionate strategies reside on a triangular (\ref{eqn:condition_for_chi})
plane (\ref{eqn:condition_for_p1}) in 3 dimensions (\ref{eqn:condition_for_p4}).
Using this formulation it can be seen that a necessary (but not sufficient)
condition for an extortionate strategy is that it cooperates on average less
than 50\% of the time when in a state of disagreement with the opponent.

As an example, consider the known extortionate strategy \(p=(8 / 9, 1 / 2, 1 /
3, 0)\) from~\cite{Stewart2012} which is referred to as \texttt{Extort-2}. In
this case, for the standard values of \((R, T, S, P)\) constraint
(\ref{eqn:condition_for_p1}) corresponds to:

\begin{equation}
    p_1 = \frac{2(p_2 + p_3) + 1}{3}
\end{equation}

It is clear that in this case all constraints hold.

This approach could in fact be used to confirm that a given strategy is acting
in an extortionate manner even if it is not a memory one strategy. However, in
practice, if a closed form for \(p\) is not known, then due to measurement
and/or numerical error this would not work.

This problem can be written in the following linear algebraic form where
\(x=(\alpha, \beta)\)
and \(p^*=(\tilde p_1 - 1, tilde_2 - 1, p_3)\):

\begin{equation}\label{eqn:linear_algebraic_equation_for_p}
    Cx= p^*
\end{equation}

\(C\) corresponds to equations
(\ref{eqn:condition_for_tilde_p1}-\ref{eqn:condition_for_tilde_p3}) and is
given by:

\begin{equation}\label{eqn:definition_of_C}
    C =
    \begin{bmatrix}
        R - P & R- P \\
        S - P & T- P \\
        T - P & S- P \\
    \end{bmatrix}
\end{equation}

Note that in general, equation (\ref{eqn:linear_algebraic_equation_for_p}) will
not necessarily have a solution. From the Rouch\'{e}-Capelli theorem if there is
a solution it is unique as \(\text{rank}(C)=2\) which is the dimension of the
variable \(x\). The best fitting \(x\) is found by minimizing:

\begin{equation}\label{eqn:r_squared}
    \text{SSError} = \|C x- p^*\|_2^2 = \sum_{i=1}^{3}\left((C\bar x)_i-p_i^*\right)^2
\end{equation}

Note that \(\text{SSError}\), which is the square of the Frobenius
norm~\cite{Golub2013}, becomes a measure of how close a strategy is to being an
extortionate strategy. Suspicion
of extortion then corresponds to a threshold on \(\text{SSError}\).

By observing interactions (human or otherwise), their memory one representation
can be inferred and this approach can be used to recognise extortionate
behaviour. The notion of comparing theoretic and actual plays of the IPD is not
novel, see for example~\cite{Rand2013}. Immediately it is noted that if the
environment is noisy~\cite{Wu1995} then no strategy can be considered to be
extortionate as \(p_4>0\).

In the next section, this idea will be illustrated by observing the interactions
that take place in a computer based tournament of the IPD\@.

\section{Numerical experiments}\label{sec:numerical-experiments}

In~\cite{Stewart2012} results from a tournament with
\input{./assets/tex/number_of_stewart_plotkin_strategies/main.tex} strategies,
was presented with specific consideration given to ZD strategies. This
tournament is reproduced here using the Axelrod-Python
project~\cite{Knight2016}. To obtain a good measure of the corresponding
transition rates for each strategy all matches have been run for
\input{assets/tex/number_of_turns/main.tex} turns and every match has been
repeated \input{assets/tex/number_of_repetitions/main.tex} times. All of this
interaction data is available at~\cite{vincent_knight_2018_1297075}. A good
match between the inferred Markov chain and the state distribution of the actual
interactions has been verified. Data for this is presented in the supplementary
materials.

Figure~\ref{fig:SSError_overall_in_stewart_plotkin} shows the \(\text{SSError}\)
values for all the strategies in the tournament, as reported
in~\cite{Stewart2012} the extortionate strategy (which has an expected
\(\text{SSError}\) approximately 0) gains a large number of wins.

\begin{figure}[!htbp]
    \centering
    \includegraphics[width=.8\textwidth]{./assets/img/SSError_overall_in_stewart_plotkin/main.pdf}
    \caption{\(\text{SSError}\) and state probabilities for the strategies
        of~\cite{Stewart2012}, ordered both by number of wins and overall score.
        Note that \(P(DC)\) is not shown as it corresponds to the transpose of
        \(P(CD)\). Cooperator and Defector are omitted as they do not visit all
        the states.}
    \label{fig:SSError_overall_in_stewart_plotkin}
\end{figure}

Here, the work of~\cite{Stewart2012} is extended by investigating a tournament
with \input{assets/tex/number_of_full_strategies/main.tex}
strategies.

The results of this analysis are shown in
Figure~\ref{fig:SSError_and_probabilities_in_full}. The top ranking strategies
by number of wins seem to be extortionate (but not against all strategies) and
it can be seen that a small sub group of strategies achieve mutual defection.
All the top ranking strategies according to score achieve mutual cooperation and
do not extort each other, however they
\textbf{do} exhibit extortionate behaviour towards a number of the lower ranking
strategies.

\begin{figure}[!htbp]
    \centering
    \includegraphics[width=.8\textwidth]{./assets/img/SSError_and_probabilities_in_full/main.pdf}
    \caption{\(\text{SSError}\) for the strategies for the full tournament. Only
    strategy interactions for which \(p_4=0\) and \(\chi>1\) are displayed.}
    \label{fig:SSError_and_probabilities_in_full}
\end{figure}

\section{Conclusion}\label{sec:conclusion}

This work defines an approach to measure whether or not a player is playing a
strategy that corresponds to an extortionate strategy as defined
in~\cite{Press2012}: a mathematical model for suspicion. Indeed, all
extortionate strategies have been
 classified as lying on a triangular plane.
This rigorous classification fails to be robust to small measurement error, thus
a statistical approach is proposed.
This is done through a linear algebraic approach for approximating the solution
of a linear system. Using this, a large number of pairwise interactions is
simulated and in fact very few strategies are found to act extortionately.

The work of~\cite{Press2012}, whilst showing that a clever approach to taking
advantage of another memory one strategy exists: this is incomplete. Whilst the
elegance of this result is very attractive, just as the simplicity of the
victory of Tit For Tat in Axelrod's original tournaments was, it is incomplete.
Extortionate strategies achieve a high number of wins but they do not
achieve a high score which corresponds to the fitness landscape in an
evolutionary sense. From the large number of interactions a payoff matrix \(S\)
can be measured where \(S_{ij}\) denotes the score (using standard values of
\((R, S, T, P) = (3, 0, 5, 1)\)) of the \(i\)th strategy
against the \(j\)th strategy. Using this, the replicator equation
describes the evolution of the system based on a population density fitness
function:

\begin{equation}\label{eqn:replicator_dynamics}
    \frac{dx}{dt} = x(S-x^TS x)
\end{equation}

Equation (\ref{eqn:replicator_dynamics}) is solved numerically through an
integration technique described in~\cite{Petzold1983} and
Figure~\ref{fig:replicator_dynamics} shows the evolution of the distribution of
the system: the various strategies are ranked by scores. It is clear to see that
only the high ranking strategies survive the evolutionary process (in fact,
only \input{./assets/img/replicator_dynamics/main.tex}
have a final distribution greater than \(10 ^ {-2}\)). This confirms the
findings of~\cite{Moran1707} in which sophisticated strategies resist
evolutionary invasion of shorter memory strategies. Recalling
Figure~\ref{fig:SSError_and_probabilities_in_full} this demonstrates that:

\begin{itemize}
    \item Cooperation emerges through the evolutionary process: the high scoring
        strategies do not exhibit extortionate behaviour towards each other.
    \item Extortionate strategies do not survive the evolutionary process.
\end{itemize}

\begin{figure}[!htbp]
    \centering
    \includegraphics[width=.8\textwidth]{./assets/img/replicator_dynamics/main.pdf}
    \caption{Numerical simulation of the replicator equation
    (\ref{eqn:replicator_dynamics}): strategies are ordered by score, only the strategies with a high score survive the evolutionary process.}
    \label{fig:replicator_dynamics}
\end{figure}

This work can be used to classify plays of the IPD\@: data can be collected from
actual interactions (in lab or in the field). Furthermore, this allows for a
classification method similar to the notion of fingerprinting presented
in~\cite{Ashlock2008}. Trained strategies can potentially be classified as
extortionate or not or it could be possible to even constrain the reinforcement
learning approaches that are becoming prevalent in the literature.
Alternatively, this mathematical approach for recognising extortion could be
used in sophisticated strategies to defend against invasion. Arguably, some of
the strategies considered here exhibit this behaviour, indeed as described
in~\cite{Harper2017}, the top ranking strategies in the full tournament are
obtained using evolutionary reinforcement learning techniques, thus, suspicion
of extortionate behaviour could in fact be an evolutionary trait.

\section*{Acknowledgements}

The following open source software libraries were used in this research:

\begin{itemize}
    \item The Axelrod ~\cite{Knight2016, Knight2018} library (IPD strategies and
        tournaments).
    \item The sympy library~\cite{Meurer2017} (verification of all symbolic
        calculations).
    \item The matplotlib~\cite{Droettboom2018} library (visualisation).
    \item The pandas~\cite{Structures2010}, dask~\cite{Dask2016} and
        NumPy~\cite{Oliphant2015} libraries (data manipulation).
    \item The SciPy~\cite{Jones2001} library (numerical integration of the
        replicator equation).
\end{itemize}

This work was performed using the computational facilities of the Advanced
Research Computing @ Cardiff (ARCCA) Division, Cardiff University.

\printbibliography

\newpage
\section*{Supplementary materials}

\includepdf{assets/pdf/proof_of_form_of_extortionate_strategies/main.pdf}

\newpage

Using the pair wise interactions the transition rates \(p,
q\) can be measured and the steady state probabilities inferred and compared to
the actual probabilities of each state.
This is done numerically by computing the singular eigenvector of the
matrix \(A\) \cite{Stewart2009}:

\[
    A =
    \begin{bmatrix}
        p_1 q_1 & p_1 (1 - q_1) & (1 - p_1) q_1 & (1 -p_1) (1 - q_1) \\
        p_2 q_2 & p_2 (1 - q_2) & (1 - p_2) q_2 & (1 -p_2) (1 - q_2) \\
        p_3 q_3 & p_3 (1 - q_3) & (1 - p_3) q_3 & (1 -p_3) (1 - q_3) \\
        p_4 q_4 & p_4 (1 - q_4) & (1 - p_4) q_4 & (1 -p_4) (1 - q_4) \\
    \end{bmatrix}
\]

Figure~\ref{fig:computed_probabilities_vs_theoretic_probabilities} shows a
regression line fitted to every pairwise interaction with a reported
\(\text{SSError}\) value (pairwise interactions with missing states were
omitted). This serves to validate the approach: a part from some edge cases the
relationship is consistent.

\begin{figure}[!htbp]
    \centering
    \includegraphics[width=.8\textwidth]{./assets/img/computed_probabilities_vs_theoretic_probabilities/main.pdf}
    \caption{The
        relationship between the steady state probabilities inferred from the
        measured transitions and the actual steady state probabilities. A linear
        regression line is included validating the approach.}
    \label{fig:computed_probabilities_vs_theoretic_probabilities}
\end{figure}


\end{document}

have a final distribution greater than \(10 ^ {-2}\)). This confirms the
findings of~\cite{Moran1707} in which sophisticated strategies resist
evolutionary invasion of shorter memory strategies. Recalling
Figure~\ref{fig:SSError_and_probabilities_in_full} this demonstrates that:

\begin{itemize}
    \item Cooperation emerges through the evolutionary process: the high scoring
        strategies do not exhibit extortionate behaviour towards each other.
    \item Extortionate strategies do not survive the evolutionary process.
\end{itemize}

\begin{figure}[!htbp]
    \centering
    \includegraphics[width=.8\textwidth]{./assets/img/replicator_dynamics/main.pdf}
    \caption{Numerical simulation of the replicator equation
    (\ref{eqn:replicator_dynamics}): strategies are ordered by score, only the strategies with a high score survive the evolutionary process.}
    \label{fig:replicator_dynamics}
\end{figure}

This work can be used to classify plays of the IPD\@: data can be collected from
actual interactions (in lab or in the field). Furthermore, this allows for a
classification method similar to the notion of fingerprinting presented
in~\cite{Ashlock2008}. Trained strategies can potentially be classified as
extortionate or not or it could be possible to even constrain the reinforcement
learning approaches that are becoming prevalent in the literature.
Alternatively, this mathematical approach for recognising extortion could be
used in sophisticated strategies to defend against invasion. Arguably, some of
the strategies considered here exhibit this behaviour, indeed as described
in~\cite{Harper2017}, the top ranking strategies in the full tournament are
obtained using evolutionary reinforcement learning techniques, thus, suspicion
of extortionate behaviour could in fact be an evolutionary trait.

\section*{Acknowledgements}

The following open source software libraries were used in this research:

\begin{itemize}
    \item The Axelrod ~\cite{Knight2016, Knight2018} library (IPD strategies and
        tournaments).
    \item The sympy library~\cite{Meurer2017} (verification of all symbolic
        calculations).
    \item The matplotlib~\cite{Droettboom2018} library (visualisation).
    \item The pandas~\cite{Structures2010}, dask~\cite{Dask2016} and
        NumPy~\cite{Oliphant2015} libraries (data manipulation).
    \item The SciPy~\cite{Jones2001} library (numerical integration of the
        replicator equation).
\end{itemize}

This work was performed using the computational facilities of the Advanced
Research Computing @ Cardiff (ARCCA) Division, Cardiff University.

\printbibliography

\newpage
\section*{Supplementary materials}

\includepdf{assets/pdf/proof_of_form_of_extortionate_strategies/main.pdf}

\newpage

Using the pair wise interactions the transition rates \(p,
q\) can be measured and the steady state probabilities inferred and compared to
the actual probabilities of each state.
This is done numerically by computing the singular eigenvector of the
matrix \(A\) \cite{Stewart2009}:

\[
    A =
    \begin{bmatrix}
        p_1 q_1 & p_1 (1 - q_1) & (1 - p_1) q_1 & (1 -p_1) (1 - q_1) \\
        p_2 q_2 & p_2 (1 - q_2) & (1 - p_2) q_2 & (1 -p_2) (1 - q_2) \\
        p_3 q_3 & p_3 (1 - q_3) & (1 - p_3) q_3 & (1 -p_3) (1 - q_3) \\
        p_4 q_4 & p_4 (1 - q_4) & (1 - p_4) q_4 & (1 -p_4) (1 - q_4) \\
    \end{bmatrix}
\]

Figure~\ref{fig:computed_probabilities_vs_theoretic_probabilities} shows a
regression line fitted to every pairwise interaction with a reported
\(\text{SSError}\) value (pairwise interactions with missing states were
omitted). This serves to validate the approach: a part from some edge cases the
relationship is consistent.

\begin{figure}[!htbp]
    \centering
    \includegraphics[width=.8\textwidth]{./assets/img/computed_probabilities_vs_theoretic_probabilities/main.pdf}
    \caption{The
        relationship between the steady state probabilities inferred from the
        measured transitions and the actual steady state probabilities. A linear
        regression line is included validating the approach.}
    \label{fig:computed_probabilities_vs_theoretic_probabilities}
\end{figure}


\end{document}
 strategies,
was presented with specific consideration given to ZD strategies. This
tournament is reproduced here using the Axelrod-Python
project~\cite{Knight2016}. To obtain a good measure of the corresponding
transition rates for each strategy all matches have been run for
\documentclass[a4paper]{article}

\usepackage{amsmath}
\usepackage{amssymb}
\usepackage[margin=1.5cm,
            includefoot,
            footskip=30pt]{geometry}
\usepackage{layout}
\usepackage{graphicx}
\usepackage{subcaption}

\usepackage{biblatex}
\usepackage{pdfpages}

\bibliography{main.bib}

\title{Suspicion: Recognising and evaluating the effectiveness
       of extortion in the Iterated Prisoner's Dilemma}
\author{Vincent A. Knight \and Nikoleta E. Glynatsi}
\date{\today}



\begin{document}

\maketitle

\begin{abstract}
    The Iterated Prisoner's Dilemma is a model for rational and evolutionary
    interactive behaviour. It has applications both in the study of human social
    behaviour as well as in biology.
    It is used to understand when and how a rational individual might
    accept an immediate cost to their own utility for the direct benefit of
    another.

    Much attention has been given to a class of strategies called
    Zero Determinant strategies. It has been theoretically shown that these
    strategies can ``extort'' any player.

    In this work, an approach to identify if observed strategies are playing in
    an extortionate way is described. Furthermore, experimental analysis of
    a large tournament with \documentclass[a4paper]{article}

\usepackage{amsmath}
\usepackage{amssymb}
\usepackage[margin=1.5cm,
            includefoot,
            footskip=30pt]{geometry}
\usepackage{layout}
\usepackage{graphicx}
\usepackage{subcaption}

\usepackage{biblatex}
\usepackage{pdfpages}

\bibliography{main.bib}

\title{Suspicion: Recognising and evaluating the effectiveness
       of extortion in the Iterated Prisoner's Dilemma}
\author{Vincent A. Knight \and Nikoleta E. Glynatsi}
\date{\today}



\begin{document}

\maketitle

\begin{abstract}
    The Iterated Prisoner's Dilemma is a model for rational and evolutionary
    interactive behaviour. It has applications both in the study of human social
    behaviour as well as in biology.
    It is used to understand when and how a rational individual might
    accept an immediate cost to their own utility for the direct benefit of
    another.

    Much attention has been given to a class of strategies called
    Zero Determinant strategies. It has been theoretically shown that these
    strategies can ``extort'' any player.

    In this work, an approach to identify if observed strategies are playing in
    an extortionate way is described. Furthermore, experimental analysis of
    a large tournament with \input{assets/tex/number_of_full_strategies/main.tex}
    strategies is considered. In this setting
    the most highly performing strategies do not play in an extortionate way
    against each other but do against lower performing strategies.
    This suggests that whilst the theory of Zero Determinant strategies
    indicates that memory is not of fundamental importance to the evolution of
    cooperative behaviour, this is incomplete.
\end{abstract}

\section{Introduction}\label{sec:introduction}

Agent based game theoretic models have become a stalwart of the underpinning
mathematics of interactive behaviours. One of the major pieces of work
in this area is the pair of original computer tournaments run by Robert
Axelrod~\cite{Axelrod1980, Axelrod1980a}. These tournaments pitted submitted
computer strategies against each other in plays of the Iterated Prisoner's
Dilemma. A common game where agents can choose to pay a slight cost to their
immediate utility in the hope of building a reputation. This has been used in
economic and evolutionary game theory to understand the evolution of cooperative
behaviour.

Recently, a class of strategies was described in~\cite{Press2012} that can
provably extort any given opponent. In~\cite{Hilbe2013, Moran1707} some
questions have already been asked about the true effectiveness of these
strategies in an evolutionary setting. Here another question is asked: is it
possible to recognise this extortionate behaviour? A mathematical procedure for
suspicion is presented: in the same way that the continued actions of an
extortionate individual might raise suspicion.

This work makes use of the Axelrod Python library~\cite{Knight2018, Knight2016}
with a large number of Prisoner Dilemma strategies available to give an
extensive numerical example of the ideas presented.  The approach is presented
in Section~\ref{sec:delta-zd-strategies}.  All of the code and data discussed
in Section~\ref{sec:numerical-experiments} is open sourced, archived and
written according to best scientific principles~\cite{Wilson2014}. The data
archive can be found at~\cite{vincent_knight_2018_1297075}.

\section{Recognising Extortion}\label{sec:delta-zd-strategies}

In~\cite{Press2012}, given a match between 2 memory-one strategies, the concept
of Zero Determinant (ZD) strategies is introduced. The main result of that paper
shows that given two memory one players \(p, q\in\mathbb{R}^4\) a linear
relationship between the players' scores could be forced by one of the players.

Using the notation of~\cite{Press2012}, assuming the utilities for player \(p\)
are given by \(S_x=(R, S, T, P)\) and for player \(q\) by \(S_y=(R, T, S, P)\)
and that the stationary scores of each player is given by \(S_X\) and \(S_Y\)
respectively. The main result of~\cite{Press2012} is that if

\begin{equation}\label{eqn:linear_relationship_for_p}
    \tilde p=\alpha S_x + \beta S_y + \gamma
\end{equation}

or

\begin{equation}\label{eqn:linear_relationship_for_q}
    \tilde q=\alpha S_x + \beta S_y + \gamma
\end{equation}

where \(\tilde p = (1 - p_1, 1 - p_2, p_3, p_4)\) and
\(\tilde q = (1 - q_1, 1 - q_2, q_3, q_4)\) then:

\begin{equation}
    \alpha S_X + \beta S_Y + \gamma = 0
\end{equation}

In~\cite{Press2012} a particular type of ZD strategy is defined: extortionate
strategies. If:

\begin{equation}\label{eqn:constraint_for_extortion}
    \gamma = - P(\alpha + \beta)
\end{equation}

then the player can ensure they get a score \(\chi\) times
larger than the opponent. This extortion coefficient is given by:

\begin{equation}\label{eqn:definition_of_chi}
    \chi=\frac{-\beta}{\alpha}
\end{equation}

Thus, if (\ref{eqn:constraint_for_extortion}) holds and \(\chi >1\) a player is
said to extort their opponent.
Here, the reverse problem is considered: given a
\(p\in\mathbb{R}^4\) how does one identify \(\alpha, \beta\) if they
exist and is the strategy in fact acting in an extortionate way?

These conditions correspond to:

\begin{align}
    \tilde p_1 & = \alpha R + \beta R - P (\alpha + \beta)
            \label{eqn:condition_for_tilde_p1}\\
    \tilde p_2 & = \alpha S + \beta T - P (\alpha + \beta)
            \label{eqn:condition_for_tilde_p2}\\
    \tilde p_3 & = \alpha T + \beta S - P (\alpha + \beta)
            \label{eqn:condition_for_tilde_p3}\\
    \tilde p_4 & = \alpha P + \beta P - P (\alpha + \beta)
            \label{eqn:condition_for_tilde_p4}
\end{align}

Equation (\ref{eqn:condition_for_tilde_p4}) ensures that \(p_4=\tilde p_4=0\).
Equations (\ref{eqn:condition_for_tilde_p1}-\ref{eqn:condition_for_tilde_p3})
can be used to eliminate \(\alpha, \beta\), giving:

\begin{equation}\label{eqn:planar_definition_of_extortion}
    \tilde p_1 = \frac{(R - P)(\tilde p_2 + \tilde p_3)}{S + T - 2P}
\end{equation}

with:

\begin{equation}\label{eqn:definition_of_chi}
    \chi = \frac{\tilde p_2 (P - T) + \tilde p_3 (S - P)}
                {\tilde p_2 (P - S) + \tilde p_3 (T - P)}
\end{equation}

Given a strategy \(p\in\mathbb{R}^{4\times 1}\) equations
(\ref{eqn:condition_for_tilde_p4}), (\ref{eqn:planar_definition_of_extortion}-\ref{eqn:definition_of_chi}) can be used to check if
a strategy is extortionate. The conditions correspond to:

\begin{align}
    p_1 & = \frac{(R-P)(p_2 + p_3) - R + T + S - P}{S + T - 2P}
     \label{eqn:condition_for_p1}\\
    p_4 & = 0 \label{eqn:condition_for_p4}\\
    1 & > p_2 + p_3\label{eqn:condition_for_chi}
\end{align}

The algebraic steps necessary to prove these results are available in the
supporting materials.

All extortionate strategies reside on a triangular (\ref{eqn:condition_for_chi})
plane (\ref{eqn:condition_for_p1}) in 3 dimensions (\ref{eqn:condition_for_p4}).
Using this formulation it can be seen that a necessary (but not sufficient)
condition for an extortionate strategy is that it cooperates on average less
than 50\% of the time when in a state of disagreement with the opponent.

As an example, consider the known extortionate strategy \(p=(8 / 9, 1 / 2, 1 /
3, 0)\) from~\cite{Stewart2012} which is referred to as \texttt{Extort-2}. In
this case, for the standard values of \((R, T, S, P)\) constraint
(\ref{eqn:condition_for_p1}) corresponds to:

\begin{equation}
    p_1 = \frac{2(p_2 + p_3) + 1}{3}
\end{equation}

It is clear that in this case all constraints hold.

This approach could in fact be used to confirm that a given strategy is acting
in an extortionate manner even if it is not a memory one strategy. However, in
practice, if a closed form for \(p\) is not known, then due to measurement
and/or numerical error this would not work.

This problem can be written in the following linear algebraic form where
\(x=(\alpha, \beta)\)
and \(p^*=(\tilde p_1 - 1, tilde_2 - 1, p_3)\):

\begin{equation}\label{eqn:linear_algebraic_equation_for_p}
    Cx= p^*
\end{equation}

\(C\) corresponds to equations
(\ref{eqn:condition_for_tilde_p1}-\ref{eqn:condition_for_tilde_p3}) and is
given by:

\begin{equation}\label{eqn:definition_of_C}
    C =
    \begin{bmatrix}
        R - P & R- P \\
        S - P & T- P \\
        T - P & S- P \\
    \end{bmatrix}
\end{equation}

Note that in general, equation (\ref{eqn:linear_algebraic_equation_for_p}) will
not necessarily have a solution. From the Rouch\'{e}-Capelli theorem if there is
a solution it is unique as \(\text{rank}(C)=2\) which is the dimension of the
variable \(x\). The best fitting \(x\) is found by minimizing:

\begin{equation}\label{eqn:r_squared}
    \text{SSError} = \|C x- p^*\|_2^2 = \sum_{i=1}^{3}\left((C\bar x)_i-p_i^*\right)^2
\end{equation}

Note that \(\text{SSError}\), which is the square of the Frobenius
norm~\cite{Golub2013}, becomes a measure of how close a strategy is to being an
extortionate strategy. Suspicion
of extortion then corresponds to a threshold on \(\text{SSError}\).

By observing interactions (human or otherwise), their memory one representation
can be inferred and this approach can be used to recognise extortionate
behaviour. The notion of comparing theoretic and actual plays of the IPD is not
novel, see for example~\cite{Rand2013}. Immediately it is noted that if the
environment is noisy~\cite{Wu1995} then no strategy can be considered to be
extortionate as \(p_4>0\).

In the next section, this idea will be illustrated by observing the interactions
that take place in a computer based tournament of the IPD\@.

\section{Numerical experiments}\label{sec:numerical-experiments}

In~\cite{Stewart2012} results from a tournament with
\input{./assets/tex/number_of_stewart_plotkin_strategies/main.tex} strategies,
was presented with specific consideration given to ZD strategies. This
tournament is reproduced here using the Axelrod-Python
project~\cite{Knight2016}. To obtain a good measure of the corresponding
transition rates for each strategy all matches have been run for
\input{assets/tex/number_of_turns/main.tex} turns and every match has been
repeated \input{assets/tex/number_of_repetitions/main.tex} times. All of this
interaction data is available at~\cite{vincent_knight_2018_1297075}. A good
match between the inferred Markov chain and the state distribution of the actual
interactions has been verified. Data for this is presented in the supplementary
materials.

Figure~\ref{fig:SSError_overall_in_stewart_plotkin} shows the \(\text{SSError}\)
values for all the strategies in the tournament, as reported
in~\cite{Stewart2012} the extortionate strategy (which has an expected
\(\text{SSError}\) approximately 0) gains a large number of wins.

\begin{figure}[!htbp]
    \centering
    \includegraphics[width=.8\textwidth]{./assets/img/SSError_overall_in_stewart_plotkin/main.pdf}
    \caption{\(\text{SSError}\) and state probabilities for the strategies
        of~\cite{Stewart2012}, ordered both by number of wins and overall score.
        Note that \(P(DC)\) is not shown as it corresponds to the transpose of
        \(P(CD)\). Cooperator and Defector are omitted as they do not visit all
        the states.}
    \label{fig:SSError_overall_in_stewart_plotkin}
\end{figure}

Here, the work of~\cite{Stewart2012} is extended by investigating a tournament
with \input{assets/tex/number_of_full_strategies/main.tex}
strategies.

The results of this analysis are shown in
Figure~\ref{fig:SSError_and_probabilities_in_full}. The top ranking strategies
by number of wins seem to be extortionate (but not against all strategies) and
it can be seen that a small sub group of strategies achieve mutual defection.
All the top ranking strategies according to score achieve mutual cooperation and
do not extort each other, however they
\textbf{do} exhibit extortionate behaviour towards a number of the lower ranking
strategies.

\begin{figure}[!htbp]
    \centering
    \includegraphics[width=.8\textwidth]{./assets/img/SSError_and_probabilities_in_full/main.pdf}
    \caption{\(\text{SSError}\) for the strategies for the full tournament. Only
    strategy interactions for which \(p_4=0\) and \(\chi>1\) are displayed.}
    \label{fig:SSError_and_probabilities_in_full}
\end{figure}

\section{Conclusion}\label{sec:conclusion}

This work defines an approach to measure whether or not a player is playing a
strategy that corresponds to an extortionate strategy as defined
in~\cite{Press2012}: a mathematical model for suspicion. Indeed, all
extortionate strategies have been
 classified as lying on a triangular plane.
This rigorous classification fails to be robust to small measurement error, thus
a statistical approach is proposed.
This is done through a linear algebraic approach for approximating the solution
of a linear system. Using this, a large number of pairwise interactions is
simulated and in fact very few strategies are found to act extortionately.

The work of~\cite{Press2012}, whilst showing that a clever approach to taking
advantage of another memory one strategy exists: this is incomplete. Whilst the
elegance of this result is very attractive, just as the simplicity of the
victory of Tit For Tat in Axelrod's original tournaments was, it is incomplete.
Extortionate strategies achieve a high number of wins but they do not
achieve a high score which corresponds to the fitness landscape in an
evolutionary sense. From the large number of interactions a payoff matrix \(S\)
can be measured where \(S_{ij}\) denotes the score (using standard values of
\((R, S, T, P) = (3, 0, 5, 1)\)) of the \(i\)th strategy
against the \(j\)th strategy. Using this, the replicator equation
describes the evolution of the system based on a population density fitness
function:

\begin{equation}\label{eqn:replicator_dynamics}
    \frac{dx}{dt} = x(S-x^TS x)
\end{equation}

Equation (\ref{eqn:replicator_dynamics}) is solved numerically through an
integration technique described in~\cite{Petzold1983} and
Figure~\ref{fig:replicator_dynamics} shows the evolution of the distribution of
the system: the various strategies are ranked by scores. It is clear to see that
only the high ranking strategies survive the evolutionary process (in fact,
only \input{./assets/img/replicator_dynamics/main.tex}
have a final distribution greater than \(10 ^ {-2}\)). This confirms the
findings of~\cite{Moran1707} in which sophisticated strategies resist
evolutionary invasion of shorter memory strategies. Recalling
Figure~\ref{fig:SSError_and_probabilities_in_full} this demonstrates that:

\begin{itemize}
    \item Cooperation emerges through the evolutionary process: the high scoring
        strategies do not exhibit extortionate behaviour towards each other.
    \item Extortionate strategies do not survive the evolutionary process.
\end{itemize}

\begin{figure}[!htbp]
    \centering
    \includegraphics[width=.8\textwidth]{./assets/img/replicator_dynamics/main.pdf}
    \caption{Numerical simulation of the replicator equation
    (\ref{eqn:replicator_dynamics}): strategies are ordered by score, only the strategies with a high score survive the evolutionary process.}
    \label{fig:replicator_dynamics}
\end{figure}

This work can be used to classify plays of the IPD\@: data can be collected from
actual interactions (in lab or in the field). Furthermore, this allows for a
classification method similar to the notion of fingerprinting presented
in~\cite{Ashlock2008}. Trained strategies can potentially be classified as
extortionate or not or it could be possible to even constrain the reinforcement
learning approaches that are becoming prevalent in the literature.
Alternatively, this mathematical approach for recognising extortion could be
used in sophisticated strategies to defend against invasion. Arguably, some of
the strategies considered here exhibit this behaviour, indeed as described
in~\cite{Harper2017}, the top ranking strategies in the full tournament are
obtained using evolutionary reinforcement learning techniques, thus, suspicion
of extortionate behaviour could in fact be an evolutionary trait.

\section*{Acknowledgements}

The following open source software libraries were used in this research:

\begin{itemize}
    \item The Axelrod ~\cite{Knight2016, Knight2018} library (IPD strategies and
        tournaments).
    \item The sympy library~\cite{Meurer2017} (verification of all symbolic
        calculations).
    \item The matplotlib~\cite{Droettboom2018} library (visualisation).
    \item The pandas~\cite{Structures2010}, dask~\cite{Dask2016} and
        NumPy~\cite{Oliphant2015} libraries (data manipulation).
    \item The SciPy~\cite{Jones2001} library (numerical integration of the
        replicator equation).
\end{itemize}

This work was performed using the computational facilities of the Advanced
Research Computing @ Cardiff (ARCCA) Division, Cardiff University.

\printbibliography

\newpage
\section*{Supplementary materials}

\includepdf{assets/pdf/proof_of_form_of_extortionate_strategies/main.pdf}

\newpage

Using the pair wise interactions the transition rates \(p,
q\) can be measured and the steady state probabilities inferred and compared to
the actual probabilities of each state.
This is done numerically by computing the singular eigenvector of the
matrix \(A\) \cite{Stewart2009}:

\[
    A =
    \begin{bmatrix}
        p_1 q_1 & p_1 (1 - q_1) & (1 - p_1) q_1 & (1 -p_1) (1 - q_1) \\
        p_2 q_2 & p_2 (1 - q_2) & (1 - p_2) q_2 & (1 -p_2) (1 - q_2) \\
        p_3 q_3 & p_3 (1 - q_3) & (1 - p_3) q_3 & (1 -p_3) (1 - q_3) \\
        p_4 q_4 & p_4 (1 - q_4) & (1 - p_4) q_4 & (1 -p_4) (1 - q_4) \\
    \end{bmatrix}
\]

Figure~\ref{fig:computed_probabilities_vs_theoretic_probabilities} shows a
regression line fitted to every pairwise interaction with a reported
\(\text{SSError}\) value (pairwise interactions with missing states were
omitted). This serves to validate the approach: a part from some edge cases the
relationship is consistent.

\begin{figure}[!htbp]
    \centering
    \includegraphics[width=.8\textwidth]{./assets/img/computed_probabilities_vs_theoretic_probabilities/main.pdf}
    \caption{The
        relationship between the steady state probabilities inferred from the
        measured transitions and the actual steady state probabilities. A linear
        regression line is included validating the approach.}
    \label{fig:computed_probabilities_vs_theoretic_probabilities}
\end{figure}


\end{document}

    strategies is considered. In this setting
    the most highly performing strategies do not play in an extortionate way
    against each other but do against lower performing strategies.
    This suggests that whilst the theory of Zero Determinant strategies
    indicates that memory is not of fundamental importance to the evolution of
    cooperative behaviour, this is incomplete.
\end{abstract}

\section{Introduction}\label{sec:introduction}

Agent based game theoretic models have become a stalwart of the underpinning
mathematics of interactive behaviours. One of the major pieces of work
in this area is the pair of original computer tournaments run by Robert
Axelrod~\cite{Axelrod1980, Axelrod1980a}. These tournaments pitted submitted
computer strategies against each other in plays of the Iterated Prisoner's
Dilemma. A common game where agents can choose to pay a slight cost to their
immediate utility in the hope of building a reputation. This has been used in
economic and evolutionary game theory to understand the evolution of cooperative
behaviour.

Recently, a class of strategies was described in~\cite{Press2012} that can
provably extort any given opponent. In~\cite{Hilbe2013, Moran1707} some
questions have already been asked about the true effectiveness of these
strategies in an evolutionary setting. Here another question is asked: is it
possible to recognise this extortionate behaviour? A mathematical procedure for
suspicion is presented: in the same way that the continued actions of an
extortionate individual might raise suspicion.

This work makes use of the Axelrod Python library~\cite{Knight2018, Knight2016}
with a large number of Prisoner Dilemma strategies available to give an
extensive numerical example of the ideas presented.  The approach is presented
in Section~\ref{sec:delta-zd-strategies}.  All of the code and data discussed
in Section~\ref{sec:numerical-experiments} is open sourced, archived and
written according to best scientific principles~\cite{Wilson2014}. The data
archive can be found at~\cite{vincent_knight_2018_1297075}.

\section{Recognising Extortion}\label{sec:delta-zd-strategies}

In~\cite{Press2012}, given a match between 2 memory-one strategies, the concept
of Zero Determinant (ZD) strategies is introduced. The main result of that paper
shows that given two memory one players \(p, q\in\mathbb{R}^4\) a linear
relationship between the players' scores could be forced by one of the players.

Using the notation of~\cite{Press2012}, assuming the utilities for player \(p\)
are given by \(S_x=(R, S, T, P)\) and for player \(q\) by \(S_y=(R, T, S, P)\)
and that the stationary scores of each player is given by \(S_X\) and \(S_Y\)
respectively. The main result of~\cite{Press2012} is that if

\begin{equation}\label{eqn:linear_relationship_for_p}
    \tilde p=\alpha S_x + \beta S_y + \gamma
\end{equation}

or

\begin{equation}\label{eqn:linear_relationship_for_q}
    \tilde q=\alpha S_x + \beta S_y + \gamma
\end{equation}

where \(\tilde p = (1 - p_1, 1 - p_2, p_3, p_4)\) and
\(\tilde q = (1 - q_1, 1 - q_2, q_3, q_4)\) then:

\begin{equation}
    \alpha S_X + \beta S_Y + \gamma = 0
\end{equation}

In~\cite{Press2012} a particular type of ZD strategy is defined: extortionate
strategies. If:

\begin{equation}\label{eqn:constraint_for_extortion}
    \gamma = - P(\alpha + \beta)
\end{equation}

then the player can ensure they get a score \(\chi\) times
larger than the opponent. This extortion coefficient is given by:

\begin{equation}\label{eqn:definition_of_chi}
    \chi=\frac{-\beta}{\alpha}
\end{equation}

Thus, if (\ref{eqn:constraint_for_extortion}) holds and \(\chi >1\) a player is
said to extort their opponent.
Here, the reverse problem is considered: given a
\(p\in\mathbb{R}^4\) how does one identify \(\alpha, \beta\) if they
exist and is the strategy in fact acting in an extortionate way?

These conditions correspond to:

\begin{align}
    \tilde p_1 & = \alpha R + \beta R - P (\alpha + \beta)
            \label{eqn:condition_for_tilde_p1}\\
    \tilde p_2 & = \alpha S + \beta T - P (\alpha + \beta)
            \label{eqn:condition_for_tilde_p2}\\
    \tilde p_3 & = \alpha T + \beta S - P (\alpha + \beta)
            \label{eqn:condition_for_tilde_p3}\\
    \tilde p_4 & = \alpha P + \beta P - P (\alpha + \beta)
            \label{eqn:condition_for_tilde_p4}
\end{align}

Equation (\ref{eqn:condition_for_tilde_p4}) ensures that \(p_4=\tilde p_4=0\).
Equations (\ref{eqn:condition_for_tilde_p1}-\ref{eqn:condition_for_tilde_p3})
can be used to eliminate \(\alpha, \beta\), giving:

\begin{equation}\label{eqn:planar_definition_of_extortion}
    \tilde p_1 = \frac{(R - P)(\tilde p_2 + \tilde p_3)}{S + T - 2P}
\end{equation}

with:

\begin{equation}\label{eqn:definition_of_chi}
    \chi = \frac{\tilde p_2 (P - T) + \tilde p_3 (S - P)}
                {\tilde p_2 (P - S) + \tilde p_3 (T - P)}
\end{equation}

Given a strategy \(p\in\mathbb{R}^{4\times 1}\) equations
(\ref{eqn:condition_for_tilde_p4}), (\ref{eqn:planar_definition_of_extortion}-\ref{eqn:definition_of_chi}) can be used to check if
a strategy is extortionate. The conditions correspond to:

\begin{align}
    p_1 & = \frac{(R-P)(p_2 + p_3) - R + T + S - P}{S + T - 2P}
     \label{eqn:condition_for_p1}\\
    p_4 & = 0 \label{eqn:condition_for_p4}\\
    1 & > p_2 + p_3\label{eqn:condition_for_chi}
\end{align}

The algebraic steps necessary to prove these results are available in the
supporting materials.

All extortionate strategies reside on a triangular (\ref{eqn:condition_for_chi})
plane (\ref{eqn:condition_for_p1}) in 3 dimensions (\ref{eqn:condition_for_p4}).
Using this formulation it can be seen that a necessary (but not sufficient)
condition for an extortionate strategy is that it cooperates on average less
than 50\% of the time when in a state of disagreement with the opponent.

As an example, consider the known extortionate strategy \(p=(8 / 9, 1 / 2, 1 /
3, 0)\) from~\cite{Stewart2012} which is referred to as \texttt{Extort-2}. In
this case, for the standard values of \((R, T, S, P)\) constraint
(\ref{eqn:condition_for_p1}) corresponds to:

\begin{equation}
    p_1 = \frac{2(p_2 + p_3) + 1}{3}
\end{equation}

It is clear that in this case all constraints hold.

This approach could in fact be used to confirm that a given strategy is acting
in an extortionate manner even if it is not a memory one strategy. However, in
practice, if a closed form for \(p\) is not known, then due to measurement
and/or numerical error this would not work.

This problem can be written in the following linear algebraic form where
\(x=(\alpha, \beta)\)
and \(p^*=(\tilde p_1 - 1, tilde_2 - 1, p_3)\):

\begin{equation}\label{eqn:linear_algebraic_equation_for_p}
    Cx= p^*
\end{equation}

\(C\) corresponds to equations
(\ref{eqn:condition_for_tilde_p1}-\ref{eqn:condition_for_tilde_p3}) and is
given by:

\begin{equation}\label{eqn:definition_of_C}
    C =
    \begin{bmatrix}
        R - P & R- P \\
        S - P & T- P \\
        T - P & S- P \\
    \end{bmatrix}
\end{equation}

Note that in general, equation (\ref{eqn:linear_algebraic_equation_for_p}) will
not necessarily have a solution. From the Rouch\'{e}-Capelli theorem if there is
a solution it is unique as \(\text{rank}(C)=2\) which is the dimension of the
variable \(x\). The best fitting \(x\) is found by minimizing:

\begin{equation}\label{eqn:r_squared}
    \text{SSError} = \|C x- p^*\|_2^2 = \sum_{i=1}^{3}\left((C\bar x)_i-p_i^*\right)^2
\end{equation}

Note that \(\text{SSError}\), which is the square of the Frobenius
norm~\cite{Golub2013}, becomes a measure of how close a strategy is to being an
extortionate strategy. Suspicion
of extortion then corresponds to a threshold on \(\text{SSError}\).

By observing interactions (human or otherwise), their memory one representation
can be inferred and this approach can be used to recognise extortionate
behaviour. The notion of comparing theoretic and actual plays of the IPD is not
novel, see for example~\cite{Rand2013}. Immediately it is noted that if the
environment is noisy~\cite{Wu1995} then no strategy can be considered to be
extortionate as \(p_4>0\).

In the next section, this idea will be illustrated by observing the interactions
that take place in a computer based tournament of the IPD\@.

\section{Numerical experiments}\label{sec:numerical-experiments}

In~\cite{Stewart2012} results from a tournament with
\documentclass[a4paper]{article}

\usepackage{amsmath}
\usepackage{amssymb}
\usepackage[margin=1.5cm,
            includefoot,
            footskip=30pt]{geometry}
\usepackage{layout}
\usepackage{graphicx}
\usepackage{subcaption}

\usepackage{biblatex}
\usepackage{pdfpages}

\bibliography{main.bib}

\title{Suspicion: Recognising and evaluating the effectiveness
       of extortion in the Iterated Prisoner's Dilemma}
\author{Vincent A. Knight \and Nikoleta E. Glynatsi}
\date{\today}



\begin{document}

\maketitle

\begin{abstract}
    The Iterated Prisoner's Dilemma is a model for rational and evolutionary
    interactive behaviour. It has applications both in the study of human social
    behaviour as well as in biology.
    It is used to understand when and how a rational individual might
    accept an immediate cost to their own utility for the direct benefit of
    another.

    Much attention has been given to a class of strategies called
    Zero Determinant strategies. It has been theoretically shown that these
    strategies can ``extort'' any player.

    In this work, an approach to identify if observed strategies are playing in
    an extortionate way is described. Furthermore, experimental analysis of
    a large tournament with \input{assets/tex/number_of_full_strategies/main.tex}
    strategies is considered. In this setting
    the most highly performing strategies do not play in an extortionate way
    against each other but do against lower performing strategies.
    This suggests that whilst the theory of Zero Determinant strategies
    indicates that memory is not of fundamental importance to the evolution of
    cooperative behaviour, this is incomplete.
\end{abstract}

\section{Introduction}\label{sec:introduction}

Agent based game theoretic models have become a stalwart of the underpinning
mathematics of interactive behaviours. One of the major pieces of work
in this area is the pair of original computer tournaments run by Robert
Axelrod~\cite{Axelrod1980, Axelrod1980a}. These tournaments pitted submitted
computer strategies against each other in plays of the Iterated Prisoner's
Dilemma. A common game where agents can choose to pay a slight cost to their
immediate utility in the hope of building a reputation. This has been used in
economic and evolutionary game theory to understand the evolution of cooperative
behaviour.

Recently, a class of strategies was described in~\cite{Press2012} that can
provably extort any given opponent. In~\cite{Hilbe2013, Moran1707} some
questions have already been asked about the true effectiveness of these
strategies in an evolutionary setting. Here another question is asked: is it
possible to recognise this extortionate behaviour? A mathematical procedure for
suspicion is presented: in the same way that the continued actions of an
extortionate individual might raise suspicion.

This work makes use of the Axelrod Python library~\cite{Knight2018, Knight2016}
with a large number of Prisoner Dilemma strategies available to give an
extensive numerical example of the ideas presented.  The approach is presented
in Section~\ref{sec:delta-zd-strategies}.  All of the code and data discussed
in Section~\ref{sec:numerical-experiments} is open sourced, archived and
written according to best scientific principles~\cite{Wilson2014}. The data
archive can be found at~\cite{vincent_knight_2018_1297075}.

\section{Recognising Extortion}\label{sec:delta-zd-strategies}

In~\cite{Press2012}, given a match between 2 memory-one strategies, the concept
of Zero Determinant (ZD) strategies is introduced. The main result of that paper
shows that given two memory one players \(p, q\in\mathbb{R}^4\) a linear
relationship between the players' scores could be forced by one of the players.

Using the notation of~\cite{Press2012}, assuming the utilities for player \(p\)
are given by \(S_x=(R, S, T, P)\) and for player \(q\) by \(S_y=(R, T, S, P)\)
and that the stationary scores of each player is given by \(S_X\) and \(S_Y\)
respectively. The main result of~\cite{Press2012} is that if

\begin{equation}\label{eqn:linear_relationship_for_p}
    \tilde p=\alpha S_x + \beta S_y + \gamma
\end{equation}

or

\begin{equation}\label{eqn:linear_relationship_for_q}
    \tilde q=\alpha S_x + \beta S_y + \gamma
\end{equation}

where \(\tilde p = (1 - p_1, 1 - p_2, p_3, p_4)\) and
\(\tilde q = (1 - q_1, 1 - q_2, q_3, q_4)\) then:

\begin{equation}
    \alpha S_X + \beta S_Y + \gamma = 0
\end{equation}

In~\cite{Press2012} a particular type of ZD strategy is defined: extortionate
strategies. If:

\begin{equation}\label{eqn:constraint_for_extortion}
    \gamma = - P(\alpha + \beta)
\end{equation}

then the player can ensure they get a score \(\chi\) times
larger than the opponent. This extortion coefficient is given by:

\begin{equation}\label{eqn:definition_of_chi}
    \chi=\frac{-\beta}{\alpha}
\end{equation}

Thus, if (\ref{eqn:constraint_for_extortion}) holds and \(\chi >1\) a player is
said to extort their opponent.
Here, the reverse problem is considered: given a
\(p\in\mathbb{R}^4\) how does one identify \(\alpha, \beta\) if they
exist and is the strategy in fact acting in an extortionate way?

These conditions correspond to:

\begin{align}
    \tilde p_1 & = \alpha R + \beta R - P (\alpha + \beta)
            \label{eqn:condition_for_tilde_p1}\\
    \tilde p_2 & = \alpha S + \beta T - P (\alpha + \beta)
            \label{eqn:condition_for_tilde_p2}\\
    \tilde p_3 & = \alpha T + \beta S - P (\alpha + \beta)
            \label{eqn:condition_for_tilde_p3}\\
    \tilde p_4 & = \alpha P + \beta P - P (\alpha + \beta)
            \label{eqn:condition_for_tilde_p4}
\end{align}

Equation (\ref{eqn:condition_for_tilde_p4}) ensures that \(p_4=\tilde p_4=0\).
Equations (\ref{eqn:condition_for_tilde_p1}-\ref{eqn:condition_for_tilde_p3})
can be used to eliminate \(\alpha, \beta\), giving:

\begin{equation}\label{eqn:planar_definition_of_extortion}
    \tilde p_1 = \frac{(R - P)(\tilde p_2 + \tilde p_3)}{S + T - 2P}
\end{equation}

with:

\begin{equation}\label{eqn:definition_of_chi}
    \chi = \frac{\tilde p_2 (P - T) + \tilde p_3 (S - P)}
                {\tilde p_2 (P - S) + \tilde p_3 (T - P)}
\end{equation}

Given a strategy \(p\in\mathbb{R}^{4\times 1}\) equations
(\ref{eqn:condition_for_tilde_p4}), (\ref{eqn:planar_definition_of_extortion}-\ref{eqn:definition_of_chi}) can be used to check if
a strategy is extortionate. The conditions correspond to:

\begin{align}
    p_1 & = \frac{(R-P)(p_2 + p_3) - R + T + S - P}{S + T - 2P}
     \label{eqn:condition_for_p1}\\
    p_4 & = 0 \label{eqn:condition_for_p4}\\
    1 & > p_2 + p_3\label{eqn:condition_for_chi}
\end{align}

The algebraic steps necessary to prove these results are available in the
supporting materials.

All extortionate strategies reside on a triangular (\ref{eqn:condition_for_chi})
plane (\ref{eqn:condition_for_p1}) in 3 dimensions (\ref{eqn:condition_for_p4}).
Using this formulation it can be seen that a necessary (but not sufficient)
condition for an extortionate strategy is that it cooperates on average less
than 50\% of the time when in a state of disagreement with the opponent.

As an example, consider the known extortionate strategy \(p=(8 / 9, 1 / 2, 1 /
3, 0)\) from~\cite{Stewart2012} which is referred to as \texttt{Extort-2}. In
this case, for the standard values of \((R, T, S, P)\) constraint
(\ref{eqn:condition_for_p1}) corresponds to:

\begin{equation}
    p_1 = \frac{2(p_2 + p_3) + 1}{3}
\end{equation}

It is clear that in this case all constraints hold.

This approach could in fact be used to confirm that a given strategy is acting
in an extortionate manner even if it is not a memory one strategy. However, in
practice, if a closed form for \(p\) is not known, then due to measurement
and/or numerical error this would not work.

This problem can be written in the following linear algebraic form where
\(x=(\alpha, \beta)\)
and \(p^*=(\tilde p_1 - 1, tilde_2 - 1, p_3)\):

\begin{equation}\label{eqn:linear_algebraic_equation_for_p}
    Cx= p^*
\end{equation}

\(C\) corresponds to equations
(\ref{eqn:condition_for_tilde_p1}-\ref{eqn:condition_for_tilde_p3}) and is
given by:

\begin{equation}\label{eqn:definition_of_C}
    C =
    \begin{bmatrix}
        R - P & R- P \\
        S - P & T- P \\
        T - P & S- P \\
    \end{bmatrix}
\end{equation}

Note that in general, equation (\ref{eqn:linear_algebraic_equation_for_p}) will
not necessarily have a solution. From the Rouch\'{e}-Capelli theorem if there is
a solution it is unique as \(\text{rank}(C)=2\) which is the dimension of the
variable \(x\). The best fitting \(x\) is found by minimizing:

\begin{equation}\label{eqn:r_squared}
    \text{SSError} = \|C x- p^*\|_2^2 = \sum_{i=1}^{3}\left((C\bar x)_i-p_i^*\right)^2
\end{equation}

Note that \(\text{SSError}\), which is the square of the Frobenius
norm~\cite{Golub2013}, becomes a measure of how close a strategy is to being an
extortionate strategy. Suspicion
of extortion then corresponds to a threshold on \(\text{SSError}\).

By observing interactions (human or otherwise), their memory one representation
can be inferred and this approach can be used to recognise extortionate
behaviour. The notion of comparing theoretic and actual plays of the IPD is not
novel, see for example~\cite{Rand2013}. Immediately it is noted that if the
environment is noisy~\cite{Wu1995} then no strategy can be considered to be
extortionate as \(p_4>0\).

In the next section, this idea will be illustrated by observing the interactions
that take place in a computer based tournament of the IPD\@.

\section{Numerical experiments}\label{sec:numerical-experiments}

In~\cite{Stewart2012} results from a tournament with
\input{./assets/tex/number_of_stewart_plotkin_strategies/main.tex} strategies,
was presented with specific consideration given to ZD strategies. This
tournament is reproduced here using the Axelrod-Python
project~\cite{Knight2016}. To obtain a good measure of the corresponding
transition rates for each strategy all matches have been run for
\input{assets/tex/number_of_turns/main.tex} turns and every match has been
repeated \input{assets/tex/number_of_repetitions/main.tex} times. All of this
interaction data is available at~\cite{vincent_knight_2018_1297075}. A good
match between the inferred Markov chain and the state distribution of the actual
interactions has been verified. Data for this is presented in the supplementary
materials.

Figure~\ref{fig:SSError_overall_in_stewart_plotkin} shows the \(\text{SSError}\)
values for all the strategies in the tournament, as reported
in~\cite{Stewart2012} the extortionate strategy (which has an expected
\(\text{SSError}\) approximately 0) gains a large number of wins.

\begin{figure}[!htbp]
    \centering
    \includegraphics[width=.8\textwidth]{./assets/img/SSError_overall_in_stewart_plotkin/main.pdf}
    \caption{\(\text{SSError}\) and state probabilities for the strategies
        of~\cite{Stewart2012}, ordered both by number of wins and overall score.
        Note that \(P(DC)\) is not shown as it corresponds to the transpose of
        \(P(CD)\). Cooperator and Defector are omitted as they do not visit all
        the states.}
    \label{fig:SSError_overall_in_stewart_plotkin}
\end{figure}

Here, the work of~\cite{Stewart2012} is extended by investigating a tournament
with \input{assets/tex/number_of_full_strategies/main.tex}
strategies.

The results of this analysis are shown in
Figure~\ref{fig:SSError_and_probabilities_in_full}. The top ranking strategies
by number of wins seem to be extortionate (but not against all strategies) and
it can be seen that a small sub group of strategies achieve mutual defection.
All the top ranking strategies according to score achieve mutual cooperation and
do not extort each other, however they
\textbf{do} exhibit extortionate behaviour towards a number of the lower ranking
strategies.

\begin{figure}[!htbp]
    \centering
    \includegraphics[width=.8\textwidth]{./assets/img/SSError_and_probabilities_in_full/main.pdf}
    \caption{\(\text{SSError}\) for the strategies for the full tournament. Only
    strategy interactions for which \(p_4=0\) and \(\chi>1\) are displayed.}
    \label{fig:SSError_and_probabilities_in_full}
\end{figure}

\section{Conclusion}\label{sec:conclusion}

This work defines an approach to measure whether or not a player is playing a
strategy that corresponds to an extortionate strategy as defined
in~\cite{Press2012}: a mathematical model for suspicion. Indeed, all
extortionate strategies have been
 classified as lying on a triangular plane.
This rigorous classification fails to be robust to small measurement error, thus
a statistical approach is proposed.
This is done through a linear algebraic approach for approximating the solution
of a linear system. Using this, a large number of pairwise interactions is
simulated and in fact very few strategies are found to act extortionately.

The work of~\cite{Press2012}, whilst showing that a clever approach to taking
advantage of another memory one strategy exists: this is incomplete. Whilst the
elegance of this result is very attractive, just as the simplicity of the
victory of Tit For Tat in Axelrod's original tournaments was, it is incomplete.
Extortionate strategies achieve a high number of wins but they do not
achieve a high score which corresponds to the fitness landscape in an
evolutionary sense. From the large number of interactions a payoff matrix \(S\)
can be measured where \(S_{ij}\) denotes the score (using standard values of
\((R, S, T, P) = (3, 0, 5, 1)\)) of the \(i\)th strategy
against the \(j\)th strategy. Using this, the replicator equation
describes the evolution of the system based on a population density fitness
function:

\begin{equation}\label{eqn:replicator_dynamics}
    \frac{dx}{dt} = x(S-x^TS x)
\end{equation}

Equation (\ref{eqn:replicator_dynamics}) is solved numerically through an
integration technique described in~\cite{Petzold1983} and
Figure~\ref{fig:replicator_dynamics} shows the evolution of the distribution of
the system: the various strategies are ranked by scores. It is clear to see that
only the high ranking strategies survive the evolutionary process (in fact,
only \input{./assets/img/replicator_dynamics/main.tex}
have a final distribution greater than \(10 ^ {-2}\)). This confirms the
findings of~\cite{Moran1707} in which sophisticated strategies resist
evolutionary invasion of shorter memory strategies. Recalling
Figure~\ref{fig:SSError_and_probabilities_in_full} this demonstrates that:

\begin{itemize}
    \item Cooperation emerges through the evolutionary process: the high scoring
        strategies do not exhibit extortionate behaviour towards each other.
    \item Extortionate strategies do not survive the evolutionary process.
\end{itemize}

\begin{figure}[!htbp]
    \centering
    \includegraphics[width=.8\textwidth]{./assets/img/replicator_dynamics/main.pdf}
    \caption{Numerical simulation of the replicator equation
    (\ref{eqn:replicator_dynamics}): strategies are ordered by score, only the strategies with a high score survive the evolutionary process.}
    \label{fig:replicator_dynamics}
\end{figure}

This work can be used to classify plays of the IPD\@: data can be collected from
actual interactions (in lab or in the field). Furthermore, this allows for a
classification method similar to the notion of fingerprinting presented
in~\cite{Ashlock2008}. Trained strategies can potentially be classified as
extortionate or not or it could be possible to even constrain the reinforcement
learning approaches that are becoming prevalent in the literature.
Alternatively, this mathematical approach for recognising extortion could be
used in sophisticated strategies to defend against invasion. Arguably, some of
the strategies considered here exhibit this behaviour, indeed as described
in~\cite{Harper2017}, the top ranking strategies in the full tournament are
obtained using evolutionary reinforcement learning techniques, thus, suspicion
of extortionate behaviour could in fact be an evolutionary trait.

\section*{Acknowledgements}

The following open source software libraries were used in this research:

\begin{itemize}
    \item The Axelrod ~\cite{Knight2016, Knight2018} library (IPD strategies and
        tournaments).
    \item The sympy library~\cite{Meurer2017} (verification of all symbolic
        calculations).
    \item The matplotlib~\cite{Droettboom2018} library (visualisation).
    \item The pandas~\cite{Structures2010}, dask~\cite{Dask2016} and
        NumPy~\cite{Oliphant2015} libraries (data manipulation).
    \item The SciPy~\cite{Jones2001} library (numerical integration of the
        replicator equation).
\end{itemize}

This work was performed using the computational facilities of the Advanced
Research Computing @ Cardiff (ARCCA) Division, Cardiff University.

\printbibliography

\newpage
\section*{Supplementary materials}

\includepdf{assets/pdf/proof_of_form_of_extortionate_strategies/main.pdf}

\newpage

Using the pair wise interactions the transition rates \(p,
q\) can be measured and the steady state probabilities inferred and compared to
the actual probabilities of each state.
This is done numerically by computing the singular eigenvector of the
matrix \(A\) \cite{Stewart2009}:

\[
    A =
    \begin{bmatrix}
        p_1 q_1 & p_1 (1 - q_1) & (1 - p_1) q_1 & (1 -p_1) (1 - q_1) \\
        p_2 q_2 & p_2 (1 - q_2) & (1 - p_2) q_2 & (1 -p_2) (1 - q_2) \\
        p_3 q_3 & p_3 (1 - q_3) & (1 - p_3) q_3 & (1 -p_3) (1 - q_3) \\
        p_4 q_4 & p_4 (1 - q_4) & (1 - p_4) q_4 & (1 -p_4) (1 - q_4) \\
    \end{bmatrix}
\]

Figure~\ref{fig:computed_probabilities_vs_theoretic_probabilities} shows a
regression line fitted to every pairwise interaction with a reported
\(\text{SSError}\) value (pairwise interactions with missing states were
omitted). This serves to validate the approach: a part from some edge cases the
relationship is consistent.

\begin{figure}[!htbp]
    \centering
    \includegraphics[width=.8\textwidth]{./assets/img/computed_probabilities_vs_theoretic_probabilities/main.pdf}
    \caption{The
        relationship between the steady state probabilities inferred from the
        measured transitions and the actual steady state probabilities. A linear
        regression line is included validating the approach.}
    \label{fig:computed_probabilities_vs_theoretic_probabilities}
\end{figure}


\end{document}
 strategies,
was presented with specific consideration given to ZD strategies. This
tournament is reproduced here using the Axelrod-Python
project~\cite{Knight2016}. To obtain a good measure of the corresponding
transition rates for each strategy all matches have been run for
\documentclass[a4paper]{article}

\usepackage{amsmath}
\usepackage{amssymb}
\usepackage[margin=1.5cm,
            includefoot,
            footskip=30pt]{geometry}
\usepackage{layout}
\usepackage{graphicx}
\usepackage{subcaption}

\usepackage{biblatex}
\usepackage{pdfpages}

\bibliography{main.bib}

\title{Suspicion: Recognising and evaluating the effectiveness
       of extortion in the Iterated Prisoner's Dilemma}
\author{Vincent A. Knight \and Nikoleta E. Glynatsi}
\date{\today}



\begin{document}

\maketitle

\begin{abstract}
    The Iterated Prisoner's Dilemma is a model for rational and evolutionary
    interactive behaviour. It has applications both in the study of human social
    behaviour as well as in biology.
    It is used to understand when and how a rational individual might
    accept an immediate cost to their own utility for the direct benefit of
    another.

    Much attention has been given to a class of strategies called
    Zero Determinant strategies. It has been theoretically shown that these
    strategies can ``extort'' any player.

    In this work, an approach to identify if observed strategies are playing in
    an extortionate way is described. Furthermore, experimental analysis of
    a large tournament with \input{assets/tex/number_of_full_strategies/main.tex}
    strategies is considered. In this setting
    the most highly performing strategies do not play in an extortionate way
    against each other but do against lower performing strategies.
    This suggests that whilst the theory of Zero Determinant strategies
    indicates that memory is not of fundamental importance to the evolution of
    cooperative behaviour, this is incomplete.
\end{abstract}

\section{Introduction}\label{sec:introduction}

Agent based game theoretic models have become a stalwart of the underpinning
mathematics of interactive behaviours. One of the major pieces of work
in this area is the pair of original computer tournaments run by Robert
Axelrod~\cite{Axelrod1980, Axelrod1980a}. These tournaments pitted submitted
computer strategies against each other in plays of the Iterated Prisoner's
Dilemma. A common game where agents can choose to pay a slight cost to their
immediate utility in the hope of building a reputation. This has been used in
economic and evolutionary game theory to understand the evolution of cooperative
behaviour.

Recently, a class of strategies was described in~\cite{Press2012} that can
provably extort any given opponent. In~\cite{Hilbe2013, Moran1707} some
questions have already been asked about the true effectiveness of these
strategies in an evolutionary setting. Here another question is asked: is it
possible to recognise this extortionate behaviour? A mathematical procedure for
suspicion is presented: in the same way that the continued actions of an
extortionate individual might raise suspicion.

This work makes use of the Axelrod Python library~\cite{Knight2018, Knight2016}
with a large number of Prisoner Dilemma strategies available to give an
extensive numerical example of the ideas presented.  The approach is presented
in Section~\ref{sec:delta-zd-strategies}.  All of the code and data discussed
in Section~\ref{sec:numerical-experiments} is open sourced, archived and
written according to best scientific principles~\cite{Wilson2014}. The data
archive can be found at~\cite{vincent_knight_2018_1297075}.

\section{Recognising Extortion}\label{sec:delta-zd-strategies}

In~\cite{Press2012}, given a match between 2 memory-one strategies, the concept
of Zero Determinant (ZD) strategies is introduced. The main result of that paper
shows that given two memory one players \(p, q\in\mathbb{R}^4\) a linear
relationship between the players' scores could be forced by one of the players.

Using the notation of~\cite{Press2012}, assuming the utilities for player \(p\)
are given by \(S_x=(R, S, T, P)\) and for player \(q\) by \(S_y=(R, T, S, P)\)
and that the stationary scores of each player is given by \(S_X\) and \(S_Y\)
respectively. The main result of~\cite{Press2012} is that if

\begin{equation}\label{eqn:linear_relationship_for_p}
    \tilde p=\alpha S_x + \beta S_y + \gamma
\end{equation}

or

\begin{equation}\label{eqn:linear_relationship_for_q}
    \tilde q=\alpha S_x + \beta S_y + \gamma
\end{equation}

where \(\tilde p = (1 - p_1, 1 - p_2, p_3, p_4)\) and
\(\tilde q = (1 - q_1, 1 - q_2, q_3, q_4)\) then:

\begin{equation}
    \alpha S_X + \beta S_Y + \gamma = 0
\end{equation}

In~\cite{Press2012} a particular type of ZD strategy is defined: extortionate
strategies. If:

\begin{equation}\label{eqn:constraint_for_extortion}
    \gamma = - P(\alpha + \beta)
\end{equation}

then the player can ensure they get a score \(\chi\) times
larger than the opponent. This extortion coefficient is given by:

\begin{equation}\label{eqn:definition_of_chi}
    \chi=\frac{-\beta}{\alpha}
\end{equation}

Thus, if (\ref{eqn:constraint_for_extortion}) holds and \(\chi >1\) a player is
said to extort their opponent.
Here, the reverse problem is considered: given a
\(p\in\mathbb{R}^4\) how does one identify \(\alpha, \beta\) if they
exist and is the strategy in fact acting in an extortionate way?

These conditions correspond to:

\begin{align}
    \tilde p_1 & = \alpha R + \beta R - P (\alpha + \beta)
            \label{eqn:condition_for_tilde_p1}\\
    \tilde p_2 & = \alpha S + \beta T - P (\alpha + \beta)
            \label{eqn:condition_for_tilde_p2}\\
    \tilde p_3 & = \alpha T + \beta S - P (\alpha + \beta)
            \label{eqn:condition_for_tilde_p3}\\
    \tilde p_4 & = \alpha P + \beta P - P (\alpha + \beta)
            \label{eqn:condition_for_tilde_p4}
\end{align}

Equation (\ref{eqn:condition_for_tilde_p4}) ensures that \(p_4=\tilde p_4=0\).
Equations (\ref{eqn:condition_for_tilde_p1}-\ref{eqn:condition_for_tilde_p3})
can be used to eliminate \(\alpha, \beta\), giving:

\begin{equation}\label{eqn:planar_definition_of_extortion}
    \tilde p_1 = \frac{(R - P)(\tilde p_2 + \tilde p_3)}{S + T - 2P}
\end{equation}

with:

\begin{equation}\label{eqn:definition_of_chi}
    \chi = \frac{\tilde p_2 (P - T) + \tilde p_3 (S - P)}
                {\tilde p_2 (P - S) + \tilde p_3 (T - P)}
\end{equation}

Given a strategy \(p\in\mathbb{R}^{4\times 1}\) equations
(\ref{eqn:condition_for_tilde_p4}), (\ref{eqn:planar_definition_of_extortion}-\ref{eqn:definition_of_chi}) can be used to check if
a strategy is extortionate. The conditions correspond to:

\begin{align}
    p_1 & = \frac{(R-P)(p_2 + p_3) - R + T + S - P}{S + T - 2P}
     \label{eqn:condition_for_p1}\\
    p_4 & = 0 \label{eqn:condition_for_p4}\\
    1 & > p_2 + p_3\label{eqn:condition_for_chi}
\end{align}

The algebraic steps necessary to prove these results are available in the
supporting materials.

All extortionate strategies reside on a triangular (\ref{eqn:condition_for_chi})
plane (\ref{eqn:condition_for_p1}) in 3 dimensions (\ref{eqn:condition_for_p4}).
Using this formulation it can be seen that a necessary (but not sufficient)
condition for an extortionate strategy is that it cooperates on average less
than 50\% of the time when in a state of disagreement with the opponent.

As an example, consider the known extortionate strategy \(p=(8 / 9, 1 / 2, 1 /
3, 0)\) from~\cite{Stewart2012} which is referred to as \texttt{Extort-2}. In
this case, for the standard values of \((R, T, S, P)\) constraint
(\ref{eqn:condition_for_p1}) corresponds to:

\begin{equation}
    p_1 = \frac{2(p_2 + p_3) + 1}{3}
\end{equation}

It is clear that in this case all constraints hold.

This approach could in fact be used to confirm that a given strategy is acting
in an extortionate manner even if it is not a memory one strategy. However, in
practice, if a closed form for \(p\) is not known, then due to measurement
and/or numerical error this would not work.

This problem can be written in the following linear algebraic form where
\(x=(\alpha, \beta)\)
and \(p^*=(\tilde p_1 - 1, tilde_2 - 1, p_3)\):

\begin{equation}\label{eqn:linear_algebraic_equation_for_p}
    Cx= p^*
\end{equation}

\(C\) corresponds to equations
(\ref{eqn:condition_for_tilde_p1}-\ref{eqn:condition_for_tilde_p3}) and is
given by:

\begin{equation}\label{eqn:definition_of_C}
    C =
    \begin{bmatrix}
        R - P & R- P \\
        S - P & T- P \\
        T - P & S- P \\
    \end{bmatrix}
\end{equation}

Note that in general, equation (\ref{eqn:linear_algebraic_equation_for_p}) will
not necessarily have a solution. From the Rouch\'{e}-Capelli theorem if there is
a solution it is unique as \(\text{rank}(C)=2\) which is the dimension of the
variable \(x\). The best fitting \(x\) is found by minimizing:

\begin{equation}\label{eqn:r_squared}
    \text{SSError} = \|C x- p^*\|_2^2 = \sum_{i=1}^{3}\left((C\bar x)_i-p_i^*\right)^2
\end{equation}

Note that \(\text{SSError}\), which is the square of the Frobenius
norm~\cite{Golub2013}, becomes a measure of how close a strategy is to being an
extortionate strategy. Suspicion
of extortion then corresponds to a threshold on \(\text{SSError}\).

By observing interactions (human or otherwise), their memory one representation
can be inferred and this approach can be used to recognise extortionate
behaviour. The notion of comparing theoretic and actual plays of the IPD is not
novel, see for example~\cite{Rand2013}. Immediately it is noted that if the
environment is noisy~\cite{Wu1995} then no strategy can be considered to be
extortionate as \(p_4>0\).

In the next section, this idea will be illustrated by observing the interactions
that take place in a computer based tournament of the IPD\@.

\section{Numerical experiments}\label{sec:numerical-experiments}

In~\cite{Stewart2012} results from a tournament with
\input{./assets/tex/number_of_stewart_plotkin_strategies/main.tex} strategies,
was presented with specific consideration given to ZD strategies. This
tournament is reproduced here using the Axelrod-Python
project~\cite{Knight2016}. To obtain a good measure of the corresponding
transition rates for each strategy all matches have been run for
\input{assets/tex/number_of_turns/main.tex} turns and every match has been
repeated \input{assets/tex/number_of_repetitions/main.tex} times. All of this
interaction data is available at~\cite{vincent_knight_2018_1297075}. A good
match between the inferred Markov chain and the state distribution of the actual
interactions has been verified. Data for this is presented in the supplementary
materials.

Figure~\ref{fig:SSError_overall_in_stewart_plotkin} shows the \(\text{SSError}\)
values for all the strategies in the tournament, as reported
in~\cite{Stewart2012} the extortionate strategy (which has an expected
\(\text{SSError}\) approximately 0) gains a large number of wins.

\begin{figure}[!htbp]
    \centering
    \includegraphics[width=.8\textwidth]{./assets/img/SSError_overall_in_stewart_plotkin/main.pdf}
    \caption{\(\text{SSError}\) and state probabilities for the strategies
        of~\cite{Stewart2012}, ordered both by number of wins and overall score.
        Note that \(P(DC)\) is not shown as it corresponds to the transpose of
        \(P(CD)\). Cooperator and Defector are omitted as they do not visit all
        the states.}
    \label{fig:SSError_overall_in_stewart_plotkin}
\end{figure}

Here, the work of~\cite{Stewart2012} is extended by investigating a tournament
with \input{assets/tex/number_of_full_strategies/main.tex}
strategies.

The results of this analysis are shown in
Figure~\ref{fig:SSError_and_probabilities_in_full}. The top ranking strategies
by number of wins seem to be extortionate (but not against all strategies) and
it can be seen that a small sub group of strategies achieve mutual defection.
All the top ranking strategies according to score achieve mutual cooperation and
do not extort each other, however they
\textbf{do} exhibit extortionate behaviour towards a number of the lower ranking
strategies.

\begin{figure}[!htbp]
    \centering
    \includegraphics[width=.8\textwidth]{./assets/img/SSError_and_probabilities_in_full/main.pdf}
    \caption{\(\text{SSError}\) for the strategies for the full tournament. Only
    strategy interactions for which \(p_4=0\) and \(\chi>1\) are displayed.}
    \label{fig:SSError_and_probabilities_in_full}
\end{figure}

\section{Conclusion}\label{sec:conclusion}

This work defines an approach to measure whether or not a player is playing a
strategy that corresponds to an extortionate strategy as defined
in~\cite{Press2012}: a mathematical model for suspicion. Indeed, all
extortionate strategies have been
 classified as lying on a triangular plane.
This rigorous classification fails to be robust to small measurement error, thus
a statistical approach is proposed.
This is done through a linear algebraic approach for approximating the solution
of a linear system. Using this, a large number of pairwise interactions is
simulated and in fact very few strategies are found to act extortionately.

The work of~\cite{Press2012}, whilst showing that a clever approach to taking
advantage of another memory one strategy exists: this is incomplete. Whilst the
elegance of this result is very attractive, just as the simplicity of the
victory of Tit For Tat in Axelrod's original tournaments was, it is incomplete.
Extortionate strategies achieve a high number of wins but they do not
achieve a high score which corresponds to the fitness landscape in an
evolutionary sense. From the large number of interactions a payoff matrix \(S\)
can be measured where \(S_{ij}\) denotes the score (using standard values of
\((R, S, T, P) = (3, 0, 5, 1)\)) of the \(i\)th strategy
against the \(j\)th strategy. Using this, the replicator equation
describes the evolution of the system based on a population density fitness
function:

\begin{equation}\label{eqn:replicator_dynamics}
    \frac{dx}{dt} = x(S-x^TS x)
\end{equation}

Equation (\ref{eqn:replicator_dynamics}) is solved numerically through an
integration technique described in~\cite{Petzold1983} and
Figure~\ref{fig:replicator_dynamics} shows the evolution of the distribution of
the system: the various strategies are ranked by scores. It is clear to see that
only the high ranking strategies survive the evolutionary process (in fact,
only \input{./assets/img/replicator_dynamics/main.tex}
have a final distribution greater than \(10 ^ {-2}\)). This confirms the
findings of~\cite{Moran1707} in which sophisticated strategies resist
evolutionary invasion of shorter memory strategies. Recalling
Figure~\ref{fig:SSError_and_probabilities_in_full} this demonstrates that:

\begin{itemize}
    \item Cooperation emerges through the evolutionary process: the high scoring
        strategies do not exhibit extortionate behaviour towards each other.
    \item Extortionate strategies do not survive the evolutionary process.
\end{itemize}

\begin{figure}[!htbp]
    \centering
    \includegraphics[width=.8\textwidth]{./assets/img/replicator_dynamics/main.pdf}
    \caption{Numerical simulation of the replicator equation
    (\ref{eqn:replicator_dynamics}): strategies are ordered by score, only the strategies with a high score survive the evolutionary process.}
    \label{fig:replicator_dynamics}
\end{figure}

This work can be used to classify plays of the IPD\@: data can be collected from
actual interactions (in lab or in the field). Furthermore, this allows for a
classification method similar to the notion of fingerprinting presented
in~\cite{Ashlock2008}. Trained strategies can potentially be classified as
extortionate or not or it could be possible to even constrain the reinforcement
learning approaches that are becoming prevalent in the literature.
Alternatively, this mathematical approach for recognising extortion could be
used in sophisticated strategies to defend against invasion. Arguably, some of
the strategies considered here exhibit this behaviour, indeed as described
in~\cite{Harper2017}, the top ranking strategies in the full tournament are
obtained using evolutionary reinforcement learning techniques, thus, suspicion
of extortionate behaviour could in fact be an evolutionary trait.

\section*{Acknowledgements}

The following open source software libraries were used in this research:

\begin{itemize}
    \item The Axelrod ~\cite{Knight2016, Knight2018} library (IPD strategies and
        tournaments).
    \item The sympy library~\cite{Meurer2017} (verification of all symbolic
        calculations).
    \item The matplotlib~\cite{Droettboom2018} library (visualisation).
    \item The pandas~\cite{Structures2010}, dask~\cite{Dask2016} and
        NumPy~\cite{Oliphant2015} libraries (data manipulation).
    \item The SciPy~\cite{Jones2001} library (numerical integration of the
        replicator equation).
\end{itemize}

This work was performed using the computational facilities of the Advanced
Research Computing @ Cardiff (ARCCA) Division, Cardiff University.

\printbibliography

\newpage
\section*{Supplementary materials}

\includepdf{assets/pdf/proof_of_form_of_extortionate_strategies/main.pdf}

\newpage

Using the pair wise interactions the transition rates \(p,
q\) can be measured and the steady state probabilities inferred and compared to
the actual probabilities of each state.
This is done numerically by computing the singular eigenvector of the
matrix \(A\) \cite{Stewart2009}:

\[
    A =
    \begin{bmatrix}
        p_1 q_1 & p_1 (1 - q_1) & (1 - p_1) q_1 & (1 -p_1) (1 - q_1) \\
        p_2 q_2 & p_2 (1 - q_2) & (1 - p_2) q_2 & (1 -p_2) (1 - q_2) \\
        p_3 q_3 & p_3 (1 - q_3) & (1 - p_3) q_3 & (1 -p_3) (1 - q_3) \\
        p_4 q_4 & p_4 (1 - q_4) & (1 - p_4) q_4 & (1 -p_4) (1 - q_4) \\
    \end{bmatrix}
\]

Figure~\ref{fig:computed_probabilities_vs_theoretic_probabilities} shows a
regression line fitted to every pairwise interaction with a reported
\(\text{SSError}\) value (pairwise interactions with missing states were
omitted). This serves to validate the approach: a part from some edge cases the
relationship is consistent.

\begin{figure}[!htbp]
    \centering
    \includegraphics[width=.8\textwidth]{./assets/img/computed_probabilities_vs_theoretic_probabilities/main.pdf}
    \caption{The
        relationship between the steady state probabilities inferred from the
        measured transitions and the actual steady state probabilities. A linear
        regression line is included validating the approach.}
    \label{fig:computed_probabilities_vs_theoretic_probabilities}
\end{figure}


\end{document}
 turns and every match has been
repeated \documentclass[a4paper]{article}

\usepackage{amsmath}
\usepackage{amssymb}
\usepackage[margin=1.5cm,
            includefoot,
            footskip=30pt]{geometry}
\usepackage{layout}
\usepackage{graphicx}
\usepackage{subcaption}

\usepackage{biblatex}
\usepackage{pdfpages}

\bibliography{main.bib}

\title{Suspicion: Recognising and evaluating the effectiveness
       of extortion in the Iterated Prisoner's Dilemma}
\author{Vincent A. Knight \and Nikoleta E. Glynatsi}
\date{\today}



\begin{document}

\maketitle

\begin{abstract}
    The Iterated Prisoner's Dilemma is a model for rational and evolutionary
    interactive behaviour. It has applications both in the study of human social
    behaviour as well as in biology.
    It is used to understand when and how a rational individual might
    accept an immediate cost to their own utility for the direct benefit of
    another.

    Much attention has been given to a class of strategies called
    Zero Determinant strategies. It has been theoretically shown that these
    strategies can ``extort'' any player.

    In this work, an approach to identify if observed strategies are playing in
    an extortionate way is described. Furthermore, experimental analysis of
    a large tournament with \input{assets/tex/number_of_full_strategies/main.tex}
    strategies is considered. In this setting
    the most highly performing strategies do not play in an extortionate way
    against each other but do against lower performing strategies.
    This suggests that whilst the theory of Zero Determinant strategies
    indicates that memory is not of fundamental importance to the evolution of
    cooperative behaviour, this is incomplete.
\end{abstract}

\section{Introduction}\label{sec:introduction}

Agent based game theoretic models have become a stalwart of the underpinning
mathematics of interactive behaviours. One of the major pieces of work
in this area is the pair of original computer tournaments run by Robert
Axelrod~\cite{Axelrod1980, Axelrod1980a}. These tournaments pitted submitted
computer strategies against each other in plays of the Iterated Prisoner's
Dilemma. A common game where agents can choose to pay a slight cost to their
immediate utility in the hope of building a reputation. This has been used in
economic and evolutionary game theory to understand the evolution of cooperative
behaviour.

Recently, a class of strategies was described in~\cite{Press2012} that can
provably extort any given opponent. In~\cite{Hilbe2013, Moran1707} some
questions have already been asked about the true effectiveness of these
strategies in an evolutionary setting. Here another question is asked: is it
possible to recognise this extortionate behaviour? A mathematical procedure for
suspicion is presented: in the same way that the continued actions of an
extortionate individual might raise suspicion.

This work makes use of the Axelrod Python library~\cite{Knight2018, Knight2016}
with a large number of Prisoner Dilemma strategies available to give an
extensive numerical example of the ideas presented.  The approach is presented
in Section~\ref{sec:delta-zd-strategies}.  All of the code and data discussed
in Section~\ref{sec:numerical-experiments} is open sourced, archived and
written according to best scientific principles~\cite{Wilson2014}. The data
archive can be found at~\cite{vincent_knight_2018_1297075}.

\section{Recognising Extortion}\label{sec:delta-zd-strategies}

In~\cite{Press2012}, given a match between 2 memory-one strategies, the concept
of Zero Determinant (ZD) strategies is introduced. The main result of that paper
shows that given two memory one players \(p, q\in\mathbb{R}^4\) a linear
relationship between the players' scores could be forced by one of the players.

Using the notation of~\cite{Press2012}, assuming the utilities for player \(p\)
are given by \(S_x=(R, S, T, P)\) and for player \(q\) by \(S_y=(R, T, S, P)\)
and that the stationary scores of each player is given by \(S_X\) and \(S_Y\)
respectively. The main result of~\cite{Press2012} is that if

\begin{equation}\label{eqn:linear_relationship_for_p}
    \tilde p=\alpha S_x + \beta S_y + \gamma
\end{equation}

or

\begin{equation}\label{eqn:linear_relationship_for_q}
    \tilde q=\alpha S_x + \beta S_y + \gamma
\end{equation}

where \(\tilde p = (1 - p_1, 1 - p_2, p_3, p_4)\) and
\(\tilde q = (1 - q_1, 1 - q_2, q_3, q_4)\) then:

\begin{equation}
    \alpha S_X + \beta S_Y + \gamma = 0
\end{equation}

In~\cite{Press2012} a particular type of ZD strategy is defined: extortionate
strategies. If:

\begin{equation}\label{eqn:constraint_for_extortion}
    \gamma = - P(\alpha + \beta)
\end{equation}

then the player can ensure they get a score \(\chi\) times
larger than the opponent. This extortion coefficient is given by:

\begin{equation}\label{eqn:definition_of_chi}
    \chi=\frac{-\beta}{\alpha}
\end{equation}

Thus, if (\ref{eqn:constraint_for_extortion}) holds and \(\chi >1\) a player is
said to extort their opponent.
Here, the reverse problem is considered: given a
\(p\in\mathbb{R}^4\) how does one identify \(\alpha, \beta\) if they
exist and is the strategy in fact acting in an extortionate way?

These conditions correspond to:

\begin{align}
    \tilde p_1 & = \alpha R + \beta R - P (\alpha + \beta)
            \label{eqn:condition_for_tilde_p1}\\
    \tilde p_2 & = \alpha S + \beta T - P (\alpha + \beta)
            \label{eqn:condition_for_tilde_p2}\\
    \tilde p_3 & = \alpha T + \beta S - P (\alpha + \beta)
            \label{eqn:condition_for_tilde_p3}\\
    \tilde p_4 & = \alpha P + \beta P - P (\alpha + \beta)
            \label{eqn:condition_for_tilde_p4}
\end{align}

Equation (\ref{eqn:condition_for_tilde_p4}) ensures that \(p_4=\tilde p_4=0\).
Equations (\ref{eqn:condition_for_tilde_p1}-\ref{eqn:condition_for_tilde_p3})
can be used to eliminate \(\alpha, \beta\), giving:

\begin{equation}\label{eqn:planar_definition_of_extortion}
    \tilde p_1 = \frac{(R - P)(\tilde p_2 + \tilde p_3)}{S + T - 2P}
\end{equation}

with:

\begin{equation}\label{eqn:definition_of_chi}
    \chi = \frac{\tilde p_2 (P - T) + \tilde p_3 (S - P)}
                {\tilde p_2 (P - S) + \tilde p_3 (T - P)}
\end{equation}

Given a strategy \(p\in\mathbb{R}^{4\times 1}\) equations
(\ref{eqn:condition_for_tilde_p4}), (\ref{eqn:planar_definition_of_extortion}-\ref{eqn:definition_of_chi}) can be used to check if
a strategy is extortionate. The conditions correspond to:

\begin{align}
    p_1 & = \frac{(R-P)(p_2 + p_3) - R + T + S - P}{S + T - 2P}
     \label{eqn:condition_for_p1}\\
    p_4 & = 0 \label{eqn:condition_for_p4}\\
    1 & > p_2 + p_3\label{eqn:condition_for_chi}
\end{align}

The algebraic steps necessary to prove these results are available in the
supporting materials.

All extortionate strategies reside on a triangular (\ref{eqn:condition_for_chi})
plane (\ref{eqn:condition_for_p1}) in 3 dimensions (\ref{eqn:condition_for_p4}).
Using this formulation it can be seen that a necessary (but not sufficient)
condition for an extortionate strategy is that it cooperates on average less
than 50\% of the time when in a state of disagreement with the opponent.

As an example, consider the known extortionate strategy \(p=(8 / 9, 1 / 2, 1 /
3, 0)\) from~\cite{Stewart2012} which is referred to as \texttt{Extort-2}. In
this case, for the standard values of \((R, T, S, P)\) constraint
(\ref{eqn:condition_for_p1}) corresponds to:

\begin{equation}
    p_1 = \frac{2(p_2 + p_3) + 1}{3}
\end{equation}

It is clear that in this case all constraints hold.

This approach could in fact be used to confirm that a given strategy is acting
in an extortionate manner even if it is not a memory one strategy. However, in
practice, if a closed form for \(p\) is not known, then due to measurement
and/or numerical error this would not work.

This problem can be written in the following linear algebraic form where
\(x=(\alpha, \beta)\)
and \(p^*=(\tilde p_1 - 1, tilde_2 - 1, p_3)\):

\begin{equation}\label{eqn:linear_algebraic_equation_for_p}
    Cx= p^*
\end{equation}

\(C\) corresponds to equations
(\ref{eqn:condition_for_tilde_p1}-\ref{eqn:condition_for_tilde_p3}) and is
given by:

\begin{equation}\label{eqn:definition_of_C}
    C =
    \begin{bmatrix}
        R - P & R- P \\
        S - P & T- P \\
        T - P & S- P \\
    \end{bmatrix}
\end{equation}

Note that in general, equation (\ref{eqn:linear_algebraic_equation_for_p}) will
not necessarily have a solution. From the Rouch\'{e}-Capelli theorem if there is
a solution it is unique as \(\text{rank}(C)=2\) which is the dimension of the
variable \(x\). The best fitting \(x\) is found by minimizing:

\begin{equation}\label{eqn:r_squared}
    \text{SSError} = \|C x- p^*\|_2^2 = \sum_{i=1}^{3}\left((C\bar x)_i-p_i^*\right)^2
\end{equation}

Note that \(\text{SSError}\), which is the square of the Frobenius
norm~\cite{Golub2013}, becomes a measure of how close a strategy is to being an
extortionate strategy. Suspicion
of extortion then corresponds to a threshold on \(\text{SSError}\).

By observing interactions (human or otherwise), their memory one representation
can be inferred and this approach can be used to recognise extortionate
behaviour. The notion of comparing theoretic and actual plays of the IPD is not
novel, see for example~\cite{Rand2013}. Immediately it is noted that if the
environment is noisy~\cite{Wu1995} then no strategy can be considered to be
extortionate as \(p_4>0\).

In the next section, this idea will be illustrated by observing the interactions
that take place in a computer based tournament of the IPD\@.

\section{Numerical experiments}\label{sec:numerical-experiments}

In~\cite{Stewart2012} results from a tournament with
\input{./assets/tex/number_of_stewart_plotkin_strategies/main.tex} strategies,
was presented with specific consideration given to ZD strategies. This
tournament is reproduced here using the Axelrod-Python
project~\cite{Knight2016}. To obtain a good measure of the corresponding
transition rates for each strategy all matches have been run for
\input{assets/tex/number_of_turns/main.tex} turns and every match has been
repeated \input{assets/tex/number_of_repetitions/main.tex} times. All of this
interaction data is available at~\cite{vincent_knight_2018_1297075}. A good
match between the inferred Markov chain and the state distribution of the actual
interactions has been verified. Data for this is presented in the supplementary
materials.

Figure~\ref{fig:SSError_overall_in_stewart_plotkin} shows the \(\text{SSError}\)
values for all the strategies in the tournament, as reported
in~\cite{Stewart2012} the extortionate strategy (which has an expected
\(\text{SSError}\) approximately 0) gains a large number of wins.

\begin{figure}[!htbp]
    \centering
    \includegraphics[width=.8\textwidth]{./assets/img/SSError_overall_in_stewart_plotkin/main.pdf}
    \caption{\(\text{SSError}\) and state probabilities for the strategies
        of~\cite{Stewart2012}, ordered both by number of wins and overall score.
        Note that \(P(DC)\) is not shown as it corresponds to the transpose of
        \(P(CD)\). Cooperator and Defector are omitted as they do not visit all
        the states.}
    \label{fig:SSError_overall_in_stewart_plotkin}
\end{figure}

Here, the work of~\cite{Stewart2012} is extended by investigating a tournament
with \input{assets/tex/number_of_full_strategies/main.tex}
strategies.

The results of this analysis are shown in
Figure~\ref{fig:SSError_and_probabilities_in_full}. The top ranking strategies
by number of wins seem to be extortionate (but not against all strategies) and
it can be seen that a small sub group of strategies achieve mutual defection.
All the top ranking strategies according to score achieve mutual cooperation and
do not extort each other, however they
\textbf{do} exhibit extortionate behaviour towards a number of the lower ranking
strategies.

\begin{figure}[!htbp]
    \centering
    \includegraphics[width=.8\textwidth]{./assets/img/SSError_and_probabilities_in_full/main.pdf}
    \caption{\(\text{SSError}\) for the strategies for the full tournament. Only
    strategy interactions for which \(p_4=0\) and \(\chi>1\) are displayed.}
    \label{fig:SSError_and_probabilities_in_full}
\end{figure}

\section{Conclusion}\label{sec:conclusion}

This work defines an approach to measure whether or not a player is playing a
strategy that corresponds to an extortionate strategy as defined
in~\cite{Press2012}: a mathematical model for suspicion. Indeed, all
extortionate strategies have been
 classified as lying on a triangular plane.
This rigorous classification fails to be robust to small measurement error, thus
a statistical approach is proposed.
This is done through a linear algebraic approach for approximating the solution
of a linear system. Using this, a large number of pairwise interactions is
simulated and in fact very few strategies are found to act extortionately.

The work of~\cite{Press2012}, whilst showing that a clever approach to taking
advantage of another memory one strategy exists: this is incomplete. Whilst the
elegance of this result is very attractive, just as the simplicity of the
victory of Tit For Tat in Axelrod's original tournaments was, it is incomplete.
Extortionate strategies achieve a high number of wins but they do not
achieve a high score which corresponds to the fitness landscape in an
evolutionary sense. From the large number of interactions a payoff matrix \(S\)
can be measured where \(S_{ij}\) denotes the score (using standard values of
\((R, S, T, P) = (3, 0, 5, 1)\)) of the \(i\)th strategy
against the \(j\)th strategy. Using this, the replicator equation
describes the evolution of the system based on a population density fitness
function:

\begin{equation}\label{eqn:replicator_dynamics}
    \frac{dx}{dt} = x(S-x^TS x)
\end{equation}

Equation (\ref{eqn:replicator_dynamics}) is solved numerically through an
integration technique described in~\cite{Petzold1983} and
Figure~\ref{fig:replicator_dynamics} shows the evolution of the distribution of
the system: the various strategies are ranked by scores. It is clear to see that
only the high ranking strategies survive the evolutionary process (in fact,
only \input{./assets/img/replicator_dynamics/main.tex}
have a final distribution greater than \(10 ^ {-2}\)). This confirms the
findings of~\cite{Moran1707} in which sophisticated strategies resist
evolutionary invasion of shorter memory strategies. Recalling
Figure~\ref{fig:SSError_and_probabilities_in_full} this demonstrates that:

\begin{itemize}
    \item Cooperation emerges through the evolutionary process: the high scoring
        strategies do not exhibit extortionate behaviour towards each other.
    \item Extortionate strategies do not survive the evolutionary process.
\end{itemize}

\begin{figure}[!htbp]
    \centering
    \includegraphics[width=.8\textwidth]{./assets/img/replicator_dynamics/main.pdf}
    \caption{Numerical simulation of the replicator equation
    (\ref{eqn:replicator_dynamics}): strategies are ordered by score, only the strategies with a high score survive the evolutionary process.}
    \label{fig:replicator_dynamics}
\end{figure}

This work can be used to classify plays of the IPD\@: data can be collected from
actual interactions (in lab or in the field). Furthermore, this allows for a
classification method similar to the notion of fingerprinting presented
in~\cite{Ashlock2008}. Trained strategies can potentially be classified as
extortionate or not or it could be possible to even constrain the reinforcement
learning approaches that are becoming prevalent in the literature.
Alternatively, this mathematical approach for recognising extortion could be
used in sophisticated strategies to defend against invasion. Arguably, some of
the strategies considered here exhibit this behaviour, indeed as described
in~\cite{Harper2017}, the top ranking strategies in the full tournament are
obtained using evolutionary reinforcement learning techniques, thus, suspicion
of extortionate behaviour could in fact be an evolutionary trait.

\section*{Acknowledgements}

The following open source software libraries were used in this research:

\begin{itemize}
    \item The Axelrod ~\cite{Knight2016, Knight2018} library (IPD strategies and
        tournaments).
    \item The sympy library~\cite{Meurer2017} (verification of all symbolic
        calculations).
    \item The matplotlib~\cite{Droettboom2018} library (visualisation).
    \item The pandas~\cite{Structures2010}, dask~\cite{Dask2016} and
        NumPy~\cite{Oliphant2015} libraries (data manipulation).
    \item The SciPy~\cite{Jones2001} library (numerical integration of the
        replicator equation).
\end{itemize}

This work was performed using the computational facilities of the Advanced
Research Computing @ Cardiff (ARCCA) Division, Cardiff University.

\printbibliography

\newpage
\section*{Supplementary materials}

\includepdf{assets/pdf/proof_of_form_of_extortionate_strategies/main.pdf}

\newpage

Using the pair wise interactions the transition rates \(p,
q\) can be measured and the steady state probabilities inferred and compared to
the actual probabilities of each state.
This is done numerically by computing the singular eigenvector of the
matrix \(A\) \cite{Stewart2009}:

\[
    A =
    \begin{bmatrix}
        p_1 q_1 & p_1 (1 - q_1) & (1 - p_1) q_1 & (1 -p_1) (1 - q_1) \\
        p_2 q_2 & p_2 (1 - q_2) & (1 - p_2) q_2 & (1 -p_2) (1 - q_2) \\
        p_3 q_3 & p_3 (1 - q_3) & (1 - p_3) q_3 & (1 -p_3) (1 - q_3) \\
        p_4 q_4 & p_4 (1 - q_4) & (1 - p_4) q_4 & (1 -p_4) (1 - q_4) \\
    \end{bmatrix}
\]

Figure~\ref{fig:computed_probabilities_vs_theoretic_probabilities} shows a
regression line fitted to every pairwise interaction with a reported
\(\text{SSError}\) value (pairwise interactions with missing states were
omitted). This serves to validate the approach: a part from some edge cases the
relationship is consistent.

\begin{figure}[!htbp]
    \centering
    \includegraphics[width=.8\textwidth]{./assets/img/computed_probabilities_vs_theoretic_probabilities/main.pdf}
    \caption{The
        relationship between the steady state probabilities inferred from the
        measured transitions and the actual steady state probabilities. A linear
        regression line is included validating the approach.}
    \label{fig:computed_probabilities_vs_theoretic_probabilities}
\end{figure}


\end{document}
 times. All of this
interaction data is available at~\cite{vincent_knight_2018_1297075}. A good
match between the inferred Markov chain and the state distribution of the actual
interactions has been verified. Data for this is presented in the supplementary
materials.

Figure~\ref{fig:SSError_overall_in_stewart_plotkin} shows the \(\text{SSError}\)
values for all the strategies in the tournament, as reported
in~\cite{Stewart2012} the extortionate strategy (which has an expected
\(\text{SSError}\) approximately 0) gains a large number of wins.

\begin{figure}[!htbp]
    \centering
    \includegraphics[width=.8\textwidth]{./assets/img/SSError_overall_in_stewart_plotkin/main.pdf}
    \caption{\(\text{SSError}\) and state probabilities for the strategies
        of~\cite{Stewart2012}, ordered both by number of wins and overall score.
        Note that \(P(DC)\) is not shown as it corresponds to the transpose of
        \(P(CD)\). Cooperator and Defector are omitted as they do not visit all
        the states.}
    \label{fig:SSError_overall_in_stewart_plotkin}
\end{figure}

Here, the work of~\cite{Stewart2012} is extended by investigating a tournament
with \documentclass[a4paper]{article}

\usepackage{amsmath}
\usepackage{amssymb}
\usepackage[margin=1.5cm,
            includefoot,
            footskip=30pt]{geometry}
\usepackage{layout}
\usepackage{graphicx}
\usepackage{subcaption}

\usepackage{biblatex}
\usepackage{pdfpages}

\bibliography{main.bib}

\title{Suspicion: Recognising and evaluating the effectiveness
       of extortion in the Iterated Prisoner's Dilemma}
\author{Vincent A. Knight \and Nikoleta E. Glynatsi}
\date{\today}



\begin{document}

\maketitle

\begin{abstract}
    The Iterated Prisoner's Dilemma is a model for rational and evolutionary
    interactive behaviour. It has applications both in the study of human social
    behaviour as well as in biology.
    It is used to understand when and how a rational individual might
    accept an immediate cost to their own utility for the direct benefit of
    another.

    Much attention has been given to a class of strategies called
    Zero Determinant strategies. It has been theoretically shown that these
    strategies can ``extort'' any player.

    In this work, an approach to identify if observed strategies are playing in
    an extortionate way is described. Furthermore, experimental analysis of
    a large tournament with \input{assets/tex/number_of_full_strategies/main.tex}
    strategies is considered. In this setting
    the most highly performing strategies do not play in an extortionate way
    against each other but do against lower performing strategies.
    This suggests that whilst the theory of Zero Determinant strategies
    indicates that memory is not of fundamental importance to the evolution of
    cooperative behaviour, this is incomplete.
\end{abstract}

\section{Introduction}\label{sec:introduction}

Agent based game theoretic models have become a stalwart of the underpinning
mathematics of interactive behaviours. One of the major pieces of work
in this area is the pair of original computer tournaments run by Robert
Axelrod~\cite{Axelrod1980, Axelrod1980a}. These tournaments pitted submitted
computer strategies against each other in plays of the Iterated Prisoner's
Dilemma. A common game where agents can choose to pay a slight cost to their
immediate utility in the hope of building a reputation. This has been used in
economic and evolutionary game theory to understand the evolution of cooperative
behaviour.

Recently, a class of strategies was described in~\cite{Press2012} that can
provably extort any given opponent. In~\cite{Hilbe2013, Moran1707} some
questions have already been asked about the true effectiveness of these
strategies in an evolutionary setting. Here another question is asked: is it
possible to recognise this extortionate behaviour? A mathematical procedure for
suspicion is presented: in the same way that the continued actions of an
extortionate individual might raise suspicion.

This work makes use of the Axelrod Python library~\cite{Knight2018, Knight2016}
with a large number of Prisoner Dilemma strategies available to give an
extensive numerical example of the ideas presented.  The approach is presented
in Section~\ref{sec:delta-zd-strategies}.  All of the code and data discussed
in Section~\ref{sec:numerical-experiments} is open sourced, archived and
written according to best scientific principles~\cite{Wilson2014}. The data
archive can be found at~\cite{vincent_knight_2018_1297075}.

\section{Recognising Extortion}\label{sec:delta-zd-strategies}

In~\cite{Press2012}, given a match between 2 memory-one strategies, the concept
of Zero Determinant (ZD) strategies is introduced. The main result of that paper
shows that given two memory one players \(p, q\in\mathbb{R}^4\) a linear
relationship between the players' scores could be forced by one of the players.

Using the notation of~\cite{Press2012}, assuming the utilities for player \(p\)
are given by \(S_x=(R, S, T, P)\) and for player \(q\) by \(S_y=(R, T, S, P)\)
and that the stationary scores of each player is given by \(S_X\) and \(S_Y\)
respectively. The main result of~\cite{Press2012} is that if

\begin{equation}\label{eqn:linear_relationship_for_p}
    \tilde p=\alpha S_x + \beta S_y + \gamma
\end{equation}

or

\begin{equation}\label{eqn:linear_relationship_for_q}
    \tilde q=\alpha S_x + \beta S_y + \gamma
\end{equation}

where \(\tilde p = (1 - p_1, 1 - p_2, p_3, p_4)\) and
\(\tilde q = (1 - q_1, 1 - q_2, q_3, q_4)\) then:

\begin{equation}
    \alpha S_X + \beta S_Y + \gamma = 0
\end{equation}

In~\cite{Press2012} a particular type of ZD strategy is defined: extortionate
strategies. If:

\begin{equation}\label{eqn:constraint_for_extortion}
    \gamma = - P(\alpha + \beta)
\end{equation}

then the player can ensure they get a score \(\chi\) times
larger than the opponent. This extortion coefficient is given by:

\begin{equation}\label{eqn:definition_of_chi}
    \chi=\frac{-\beta}{\alpha}
\end{equation}

Thus, if (\ref{eqn:constraint_for_extortion}) holds and \(\chi >1\) a player is
said to extort their opponent.
Here, the reverse problem is considered: given a
\(p\in\mathbb{R}^4\) how does one identify \(\alpha, \beta\) if they
exist and is the strategy in fact acting in an extortionate way?

These conditions correspond to:

\begin{align}
    \tilde p_1 & = \alpha R + \beta R - P (\alpha + \beta)
            \label{eqn:condition_for_tilde_p1}\\
    \tilde p_2 & = \alpha S + \beta T - P (\alpha + \beta)
            \label{eqn:condition_for_tilde_p2}\\
    \tilde p_3 & = \alpha T + \beta S - P (\alpha + \beta)
            \label{eqn:condition_for_tilde_p3}\\
    \tilde p_4 & = \alpha P + \beta P - P (\alpha + \beta)
            \label{eqn:condition_for_tilde_p4}
\end{align}

Equation (\ref{eqn:condition_for_tilde_p4}) ensures that \(p_4=\tilde p_4=0\).
Equations (\ref{eqn:condition_for_tilde_p1}-\ref{eqn:condition_for_tilde_p3})
can be used to eliminate \(\alpha, \beta\), giving:

\begin{equation}\label{eqn:planar_definition_of_extortion}
    \tilde p_1 = \frac{(R - P)(\tilde p_2 + \tilde p_3)}{S + T - 2P}
\end{equation}

with:

\begin{equation}\label{eqn:definition_of_chi}
    \chi = \frac{\tilde p_2 (P - T) + \tilde p_3 (S - P)}
                {\tilde p_2 (P - S) + \tilde p_3 (T - P)}
\end{equation}

Given a strategy \(p\in\mathbb{R}^{4\times 1}\) equations
(\ref{eqn:condition_for_tilde_p4}), (\ref{eqn:planar_definition_of_extortion}-\ref{eqn:definition_of_chi}) can be used to check if
a strategy is extortionate. The conditions correspond to:

\begin{align}
    p_1 & = \frac{(R-P)(p_2 + p_3) - R + T + S - P}{S + T - 2P}
     \label{eqn:condition_for_p1}\\
    p_4 & = 0 \label{eqn:condition_for_p4}\\
    1 & > p_2 + p_3\label{eqn:condition_for_chi}
\end{align}

The algebraic steps necessary to prove these results are available in the
supporting materials.

All extortionate strategies reside on a triangular (\ref{eqn:condition_for_chi})
plane (\ref{eqn:condition_for_p1}) in 3 dimensions (\ref{eqn:condition_for_p4}).
Using this formulation it can be seen that a necessary (but not sufficient)
condition for an extortionate strategy is that it cooperates on average less
than 50\% of the time when in a state of disagreement with the opponent.

As an example, consider the known extortionate strategy \(p=(8 / 9, 1 / 2, 1 /
3, 0)\) from~\cite{Stewart2012} which is referred to as \texttt{Extort-2}. In
this case, for the standard values of \((R, T, S, P)\) constraint
(\ref{eqn:condition_for_p1}) corresponds to:

\begin{equation}
    p_1 = \frac{2(p_2 + p_3) + 1}{3}
\end{equation}

It is clear that in this case all constraints hold.

This approach could in fact be used to confirm that a given strategy is acting
in an extortionate manner even if it is not a memory one strategy. However, in
practice, if a closed form for \(p\) is not known, then due to measurement
and/or numerical error this would not work.

This problem can be written in the following linear algebraic form where
\(x=(\alpha, \beta)\)
and \(p^*=(\tilde p_1 - 1, tilde_2 - 1, p_3)\):

\begin{equation}\label{eqn:linear_algebraic_equation_for_p}
    Cx= p^*
\end{equation}

\(C\) corresponds to equations
(\ref{eqn:condition_for_tilde_p1}-\ref{eqn:condition_for_tilde_p3}) and is
given by:

\begin{equation}\label{eqn:definition_of_C}
    C =
    \begin{bmatrix}
        R - P & R- P \\
        S - P & T- P \\
        T - P & S- P \\
    \end{bmatrix}
\end{equation}

Note that in general, equation (\ref{eqn:linear_algebraic_equation_for_p}) will
not necessarily have a solution. From the Rouch\'{e}-Capelli theorem if there is
a solution it is unique as \(\text{rank}(C)=2\) which is the dimension of the
variable \(x\). The best fitting \(x\) is found by minimizing:

\begin{equation}\label{eqn:r_squared}
    \text{SSError} = \|C x- p^*\|_2^2 = \sum_{i=1}^{3}\left((C\bar x)_i-p_i^*\right)^2
\end{equation}

Note that \(\text{SSError}\), which is the square of the Frobenius
norm~\cite{Golub2013}, becomes a measure of how close a strategy is to being an
extortionate strategy. Suspicion
of extortion then corresponds to a threshold on \(\text{SSError}\).

By observing interactions (human or otherwise), their memory one representation
can be inferred and this approach can be used to recognise extortionate
behaviour. The notion of comparing theoretic and actual plays of the IPD is not
novel, see for example~\cite{Rand2013}. Immediately it is noted that if the
environment is noisy~\cite{Wu1995} then no strategy can be considered to be
extortionate as \(p_4>0\).

In the next section, this idea will be illustrated by observing the interactions
that take place in a computer based tournament of the IPD\@.

\section{Numerical experiments}\label{sec:numerical-experiments}

In~\cite{Stewart2012} results from a tournament with
\input{./assets/tex/number_of_stewart_plotkin_strategies/main.tex} strategies,
was presented with specific consideration given to ZD strategies. This
tournament is reproduced here using the Axelrod-Python
project~\cite{Knight2016}. To obtain a good measure of the corresponding
transition rates for each strategy all matches have been run for
\input{assets/tex/number_of_turns/main.tex} turns and every match has been
repeated \input{assets/tex/number_of_repetitions/main.tex} times. All of this
interaction data is available at~\cite{vincent_knight_2018_1297075}. A good
match between the inferred Markov chain and the state distribution of the actual
interactions has been verified. Data for this is presented in the supplementary
materials.

Figure~\ref{fig:SSError_overall_in_stewart_plotkin} shows the \(\text{SSError}\)
values for all the strategies in the tournament, as reported
in~\cite{Stewart2012} the extortionate strategy (which has an expected
\(\text{SSError}\) approximately 0) gains a large number of wins.

\begin{figure}[!htbp]
    \centering
    \includegraphics[width=.8\textwidth]{./assets/img/SSError_overall_in_stewart_plotkin/main.pdf}
    \caption{\(\text{SSError}\) and state probabilities for the strategies
        of~\cite{Stewart2012}, ordered both by number of wins and overall score.
        Note that \(P(DC)\) is not shown as it corresponds to the transpose of
        \(P(CD)\). Cooperator and Defector are omitted as they do not visit all
        the states.}
    \label{fig:SSError_overall_in_stewart_plotkin}
\end{figure}

Here, the work of~\cite{Stewart2012} is extended by investigating a tournament
with \input{assets/tex/number_of_full_strategies/main.tex}
strategies.

The results of this analysis are shown in
Figure~\ref{fig:SSError_and_probabilities_in_full}. The top ranking strategies
by number of wins seem to be extortionate (but not against all strategies) and
it can be seen that a small sub group of strategies achieve mutual defection.
All the top ranking strategies according to score achieve mutual cooperation and
do not extort each other, however they
\textbf{do} exhibit extortionate behaviour towards a number of the lower ranking
strategies.

\begin{figure}[!htbp]
    \centering
    \includegraphics[width=.8\textwidth]{./assets/img/SSError_and_probabilities_in_full/main.pdf}
    \caption{\(\text{SSError}\) for the strategies for the full tournament. Only
    strategy interactions for which \(p_4=0\) and \(\chi>1\) are displayed.}
    \label{fig:SSError_and_probabilities_in_full}
\end{figure}

\section{Conclusion}\label{sec:conclusion}

This work defines an approach to measure whether or not a player is playing a
strategy that corresponds to an extortionate strategy as defined
in~\cite{Press2012}: a mathematical model for suspicion. Indeed, all
extortionate strategies have been
 classified as lying on a triangular plane.
This rigorous classification fails to be robust to small measurement error, thus
a statistical approach is proposed.
This is done through a linear algebraic approach for approximating the solution
of a linear system. Using this, a large number of pairwise interactions is
simulated and in fact very few strategies are found to act extortionately.

The work of~\cite{Press2012}, whilst showing that a clever approach to taking
advantage of another memory one strategy exists: this is incomplete. Whilst the
elegance of this result is very attractive, just as the simplicity of the
victory of Tit For Tat in Axelrod's original tournaments was, it is incomplete.
Extortionate strategies achieve a high number of wins but they do not
achieve a high score which corresponds to the fitness landscape in an
evolutionary sense. From the large number of interactions a payoff matrix \(S\)
can be measured where \(S_{ij}\) denotes the score (using standard values of
\((R, S, T, P) = (3, 0, 5, 1)\)) of the \(i\)th strategy
against the \(j\)th strategy. Using this, the replicator equation
describes the evolution of the system based on a population density fitness
function:

\begin{equation}\label{eqn:replicator_dynamics}
    \frac{dx}{dt} = x(S-x^TS x)
\end{equation}

Equation (\ref{eqn:replicator_dynamics}) is solved numerically through an
integration technique described in~\cite{Petzold1983} and
Figure~\ref{fig:replicator_dynamics} shows the evolution of the distribution of
the system: the various strategies are ranked by scores. It is clear to see that
only the high ranking strategies survive the evolutionary process (in fact,
only \input{./assets/img/replicator_dynamics/main.tex}
have a final distribution greater than \(10 ^ {-2}\)). This confirms the
findings of~\cite{Moran1707} in which sophisticated strategies resist
evolutionary invasion of shorter memory strategies. Recalling
Figure~\ref{fig:SSError_and_probabilities_in_full} this demonstrates that:

\begin{itemize}
    \item Cooperation emerges through the evolutionary process: the high scoring
        strategies do not exhibit extortionate behaviour towards each other.
    \item Extortionate strategies do not survive the evolutionary process.
\end{itemize}

\begin{figure}[!htbp]
    \centering
    \includegraphics[width=.8\textwidth]{./assets/img/replicator_dynamics/main.pdf}
    \caption{Numerical simulation of the replicator equation
    (\ref{eqn:replicator_dynamics}): strategies are ordered by score, only the strategies with a high score survive the evolutionary process.}
    \label{fig:replicator_dynamics}
\end{figure}

This work can be used to classify plays of the IPD\@: data can be collected from
actual interactions (in lab or in the field). Furthermore, this allows for a
classification method similar to the notion of fingerprinting presented
in~\cite{Ashlock2008}. Trained strategies can potentially be classified as
extortionate or not or it could be possible to even constrain the reinforcement
learning approaches that are becoming prevalent in the literature.
Alternatively, this mathematical approach for recognising extortion could be
used in sophisticated strategies to defend against invasion. Arguably, some of
the strategies considered here exhibit this behaviour, indeed as described
in~\cite{Harper2017}, the top ranking strategies in the full tournament are
obtained using evolutionary reinforcement learning techniques, thus, suspicion
of extortionate behaviour could in fact be an evolutionary trait.

\section*{Acknowledgements}

The following open source software libraries were used in this research:

\begin{itemize}
    \item The Axelrod ~\cite{Knight2016, Knight2018} library (IPD strategies and
        tournaments).
    \item The sympy library~\cite{Meurer2017} (verification of all symbolic
        calculations).
    \item The matplotlib~\cite{Droettboom2018} library (visualisation).
    \item The pandas~\cite{Structures2010}, dask~\cite{Dask2016} and
        NumPy~\cite{Oliphant2015} libraries (data manipulation).
    \item The SciPy~\cite{Jones2001} library (numerical integration of the
        replicator equation).
\end{itemize}

This work was performed using the computational facilities of the Advanced
Research Computing @ Cardiff (ARCCA) Division, Cardiff University.

\printbibliography

\newpage
\section*{Supplementary materials}

\includepdf{assets/pdf/proof_of_form_of_extortionate_strategies/main.pdf}

\newpage

Using the pair wise interactions the transition rates \(p,
q\) can be measured and the steady state probabilities inferred and compared to
the actual probabilities of each state.
This is done numerically by computing the singular eigenvector of the
matrix \(A\) \cite{Stewart2009}:

\[
    A =
    \begin{bmatrix}
        p_1 q_1 & p_1 (1 - q_1) & (1 - p_1) q_1 & (1 -p_1) (1 - q_1) \\
        p_2 q_2 & p_2 (1 - q_2) & (1 - p_2) q_2 & (1 -p_2) (1 - q_2) \\
        p_3 q_3 & p_3 (1 - q_3) & (1 - p_3) q_3 & (1 -p_3) (1 - q_3) \\
        p_4 q_4 & p_4 (1 - q_4) & (1 - p_4) q_4 & (1 -p_4) (1 - q_4) \\
    \end{bmatrix}
\]

Figure~\ref{fig:computed_probabilities_vs_theoretic_probabilities} shows a
regression line fitted to every pairwise interaction with a reported
\(\text{SSError}\) value (pairwise interactions with missing states were
omitted). This serves to validate the approach: a part from some edge cases the
relationship is consistent.

\begin{figure}[!htbp]
    \centering
    \includegraphics[width=.8\textwidth]{./assets/img/computed_probabilities_vs_theoretic_probabilities/main.pdf}
    \caption{The
        relationship between the steady state probabilities inferred from the
        measured transitions and the actual steady state probabilities. A linear
        regression line is included validating the approach.}
    \label{fig:computed_probabilities_vs_theoretic_probabilities}
\end{figure}


\end{document}

strategies.

The results of this analysis are shown in
Figure~\ref{fig:SSError_and_probabilities_in_full}. The top ranking strategies
by number of wins seem to be extortionate (but not against all strategies) and
it can be seen that a small sub group of strategies achieve mutual defection.
All the top ranking strategies according to score achieve mutual cooperation and
do not extort each other, however they
\textbf{do} exhibit extortionate behaviour towards a number of the lower ranking
strategies.

\begin{figure}[!htbp]
    \centering
    \includegraphics[width=.8\textwidth]{./assets/img/SSError_and_probabilities_in_full/main.pdf}
    \caption{\(\text{SSError}\) for the strategies for the full tournament. Only
    strategy interactions for which \(p_4=0\) and \(\chi>1\) are displayed.}
    \label{fig:SSError_and_probabilities_in_full}
\end{figure}

\section{Conclusion}\label{sec:conclusion}

This work defines an approach to measure whether or not a player is playing a
strategy that corresponds to an extortionate strategy as defined
in~\cite{Press2012}: a mathematical model for suspicion. Indeed, all
extortionate strategies have been
 classified as lying on a triangular plane.
This rigorous classification fails to be robust to small measurement error, thus
a statistical approach is proposed.
This is done through a linear algebraic approach for approximating the solution
of a linear system. Using this, a large number of pairwise interactions is
simulated and in fact very few strategies are found to act extortionately.

The work of~\cite{Press2012}, whilst showing that a clever approach to taking
advantage of another memory one strategy exists: this is incomplete. Whilst the
elegance of this result is very attractive, just as the simplicity of the
victory of Tit For Tat in Axelrod's original tournaments was, it is incomplete.
Extortionate strategies achieve a high number of wins but they do not
achieve a high score which corresponds to the fitness landscape in an
evolutionary sense. From the large number of interactions a payoff matrix \(S\)
can be measured where \(S_{ij}\) denotes the score (using standard values of
\((R, S, T, P) = (3, 0, 5, 1)\)) of the \(i\)th strategy
against the \(j\)th strategy. Using this, the replicator equation
describes the evolution of the system based on a population density fitness
function:

\begin{equation}\label{eqn:replicator_dynamics}
    \frac{dx}{dt} = x(S-x^TS x)
\end{equation}

Equation (\ref{eqn:replicator_dynamics}) is solved numerically through an
integration technique described in~\cite{Petzold1983} and
Figure~\ref{fig:replicator_dynamics} shows the evolution of the distribution of
the system: the various strategies are ranked by scores. It is clear to see that
only the high ranking strategies survive the evolutionary process (in fact,
only \documentclass[a4paper]{article}

\usepackage{amsmath}
\usepackage{amssymb}
\usepackage[margin=1.5cm,
            includefoot,
            footskip=30pt]{geometry}
\usepackage{layout}
\usepackage{graphicx}
\usepackage{subcaption}

\usepackage{biblatex}
\usepackage{pdfpages}

\bibliography{main.bib}

\title{Suspicion: Recognising and evaluating the effectiveness
       of extortion in the Iterated Prisoner's Dilemma}
\author{Vincent A. Knight \and Nikoleta E. Glynatsi}
\date{\today}



\begin{document}

\maketitle

\begin{abstract}
    The Iterated Prisoner's Dilemma is a model for rational and evolutionary
    interactive behaviour. It has applications both in the study of human social
    behaviour as well as in biology.
    It is used to understand when and how a rational individual might
    accept an immediate cost to their own utility for the direct benefit of
    another.

    Much attention has been given to a class of strategies called
    Zero Determinant strategies. It has been theoretically shown that these
    strategies can ``extort'' any player.

    In this work, an approach to identify if observed strategies are playing in
    an extortionate way is described. Furthermore, experimental analysis of
    a large tournament with \input{assets/tex/number_of_full_strategies/main.tex}
    strategies is considered. In this setting
    the most highly performing strategies do not play in an extortionate way
    against each other but do against lower performing strategies.
    This suggests that whilst the theory of Zero Determinant strategies
    indicates that memory is not of fundamental importance to the evolution of
    cooperative behaviour, this is incomplete.
\end{abstract}

\section{Introduction}\label{sec:introduction}

Agent based game theoretic models have become a stalwart of the underpinning
mathematics of interactive behaviours. One of the major pieces of work
in this area is the pair of original computer tournaments run by Robert
Axelrod~\cite{Axelrod1980, Axelrod1980a}. These tournaments pitted submitted
computer strategies against each other in plays of the Iterated Prisoner's
Dilemma. A common game where agents can choose to pay a slight cost to their
immediate utility in the hope of building a reputation. This has been used in
economic and evolutionary game theory to understand the evolution of cooperative
behaviour.

Recently, a class of strategies was described in~\cite{Press2012} that can
provably extort any given opponent. In~\cite{Hilbe2013, Moran1707} some
questions have already been asked about the true effectiveness of these
strategies in an evolutionary setting. Here another question is asked: is it
possible to recognise this extortionate behaviour? A mathematical procedure for
suspicion is presented: in the same way that the continued actions of an
extortionate individual might raise suspicion.

This work makes use of the Axelrod Python library~\cite{Knight2018, Knight2016}
with a large number of Prisoner Dilemma strategies available to give an
extensive numerical example of the ideas presented.  The approach is presented
in Section~\ref{sec:delta-zd-strategies}.  All of the code and data discussed
in Section~\ref{sec:numerical-experiments} is open sourced, archived and
written according to best scientific principles~\cite{Wilson2014}. The data
archive can be found at~\cite{vincent_knight_2018_1297075}.

\section{Recognising Extortion}\label{sec:delta-zd-strategies}

In~\cite{Press2012}, given a match between 2 memory-one strategies, the concept
of Zero Determinant (ZD) strategies is introduced. The main result of that paper
shows that given two memory one players \(p, q\in\mathbb{R}^4\) a linear
relationship between the players' scores could be forced by one of the players.

Using the notation of~\cite{Press2012}, assuming the utilities for player \(p\)
are given by \(S_x=(R, S, T, P)\) and for player \(q\) by \(S_y=(R, T, S, P)\)
and that the stationary scores of each player is given by \(S_X\) and \(S_Y\)
respectively. The main result of~\cite{Press2012} is that if

\begin{equation}\label{eqn:linear_relationship_for_p}
    \tilde p=\alpha S_x + \beta S_y + \gamma
\end{equation}

or

\begin{equation}\label{eqn:linear_relationship_for_q}
    \tilde q=\alpha S_x + \beta S_y + \gamma
\end{equation}

where \(\tilde p = (1 - p_1, 1 - p_2, p_3, p_4)\) and
\(\tilde q = (1 - q_1, 1 - q_2, q_3, q_4)\) then:

\begin{equation}
    \alpha S_X + \beta S_Y + \gamma = 0
\end{equation}

In~\cite{Press2012} a particular type of ZD strategy is defined: extortionate
strategies. If:

\begin{equation}\label{eqn:constraint_for_extortion}
    \gamma = - P(\alpha + \beta)
\end{equation}

then the player can ensure they get a score \(\chi\) times
larger than the opponent. This extortion coefficient is given by:

\begin{equation}\label{eqn:definition_of_chi}
    \chi=\frac{-\beta}{\alpha}
\end{equation}

Thus, if (\ref{eqn:constraint_for_extortion}) holds and \(\chi >1\) a player is
said to extort their opponent.
Here, the reverse problem is considered: given a
\(p\in\mathbb{R}^4\) how does one identify \(\alpha, \beta\) if they
exist and is the strategy in fact acting in an extortionate way?

These conditions correspond to:

\begin{align}
    \tilde p_1 & = \alpha R + \beta R - P (\alpha + \beta)
            \label{eqn:condition_for_tilde_p1}\\
    \tilde p_2 & = \alpha S + \beta T - P (\alpha + \beta)
            \label{eqn:condition_for_tilde_p2}\\
    \tilde p_3 & = \alpha T + \beta S - P (\alpha + \beta)
            \label{eqn:condition_for_tilde_p3}\\
    \tilde p_4 & = \alpha P + \beta P - P (\alpha + \beta)
            \label{eqn:condition_for_tilde_p4}
\end{align}

Equation (\ref{eqn:condition_for_tilde_p4}) ensures that \(p_4=\tilde p_4=0\).
Equations (\ref{eqn:condition_for_tilde_p1}-\ref{eqn:condition_for_tilde_p3})
can be used to eliminate \(\alpha, \beta\), giving:

\begin{equation}\label{eqn:planar_definition_of_extortion}
    \tilde p_1 = \frac{(R - P)(\tilde p_2 + \tilde p_3)}{S + T - 2P}
\end{equation}

with:

\begin{equation}\label{eqn:definition_of_chi}
    \chi = \frac{\tilde p_2 (P - T) + \tilde p_3 (S - P)}
                {\tilde p_2 (P - S) + \tilde p_3 (T - P)}
\end{equation}

Given a strategy \(p\in\mathbb{R}^{4\times 1}\) equations
(\ref{eqn:condition_for_tilde_p4}), (\ref{eqn:planar_definition_of_extortion}-\ref{eqn:definition_of_chi}) can be used to check if
a strategy is extortionate. The conditions correspond to:

\begin{align}
    p_1 & = \frac{(R-P)(p_2 + p_3) - R + T + S - P}{S + T - 2P}
     \label{eqn:condition_for_p1}\\
    p_4 & = 0 \label{eqn:condition_for_p4}\\
    1 & > p_2 + p_3\label{eqn:condition_for_chi}
\end{align}

The algebraic steps necessary to prove these results are available in the
supporting materials.

All extortionate strategies reside on a triangular (\ref{eqn:condition_for_chi})
plane (\ref{eqn:condition_for_p1}) in 3 dimensions (\ref{eqn:condition_for_p4}).
Using this formulation it can be seen that a necessary (but not sufficient)
condition for an extortionate strategy is that it cooperates on average less
than 50\% of the time when in a state of disagreement with the opponent.

As an example, consider the known extortionate strategy \(p=(8 / 9, 1 / 2, 1 /
3, 0)\) from~\cite{Stewart2012} which is referred to as \texttt{Extort-2}. In
this case, for the standard values of \((R, T, S, P)\) constraint
(\ref{eqn:condition_for_p1}) corresponds to:

\begin{equation}
    p_1 = \frac{2(p_2 + p_3) + 1}{3}
\end{equation}

It is clear that in this case all constraints hold.

This approach could in fact be used to confirm that a given strategy is acting
in an extortionate manner even if it is not a memory one strategy. However, in
practice, if a closed form for \(p\) is not known, then due to measurement
and/or numerical error this would not work.

This problem can be written in the following linear algebraic form where
\(x=(\alpha, \beta)\)
and \(p^*=(\tilde p_1 - 1, tilde_2 - 1, p_3)\):

\begin{equation}\label{eqn:linear_algebraic_equation_for_p}
    Cx= p^*
\end{equation}

\(C\) corresponds to equations
(\ref{eqn:condition_for_tilde_p1}-\ref{eqn:condition_for_tilde_p3}) and is
given by:

\begin{equation}\label{eqn:definition_of_C}
    C =
    \begin{bmatrix}
        R - P & R- P \\
        S - P & T- P \\
        T - P & S- P \\
    \end{bmatrix}
\end{equation}

Note that in general, equation (\ref{eqn:linear_algebraic_equation_for_p}) will
not necessarily have a solution. From the Rouch\'{e}-Capelli theorem if there is
a solution it is unique as \(\text{rank}(C)=2\) which is the dimension of the
variable \(x\). The best fitting \(x\) is found by minimizing:

\begin{equation}\label{eqn:r_squared}
    \text{SSError} = \|C x- p^*\|_2^2 = \sum_{i=1}^{3}\left((C\bar x)_i-p_i^*\right)^2
\end{equation}

Note that \(\text{SSError}\), which is the square of the Frobenius
norm~\cite{Golub2013}, becomes a measure of how close a strategy is to being an
extortionate strategy. Suspicion
of extortion then corresponds to a threshold on \(\text{SSError}\).

By observing interactions (human or otherwise), their memory one representation
can be inferred and this approach can be used to recognise extortionate
behaviour. The notion of comparing theoretic and actual plays of the IPD is not
novel, see for example~\cite{Rand2013}. Immediately it is noted that if the
environment is noisy~\cite{Wu1995} then no strategy can be considered to be
extortionate as \(p_4>0\).

In the next section, this idea will be illustrated by observing the interactions
that take place in a computer based tournament of the IPD\@.

\section{Numerical experiments}\label{sec:numerical-experiments}

In~\cite{Stewart2012} results from a tournament with
\input{./assets/tex/number_of_stewart_plotkin_strategies/main.tex} strategies,
was presented with specific consideration given to ZD strategies. This
tournament is reproduced here using the Axelrod-Python
project~\cite{Knight2016}. To obtain a good measure of the corresponding
transition rates for each strategy all matches have been run for
\input{assets/tex/number_of_turns/main.tex} turns and every match has been
repeated \input{assets/tex/number_of_repetitions/main.tex} times. All of this
interaction data is available at~\cite{vincent_knight_2018_1297075}. A good
match between the inferred Markov chain and the state distribution of the actual
interactions has been verified. Data for this is presented in the supplementary
materials.

Figure~\ref{fig:SSError_overall_in_stewart_plotkin} shows the \(\text{SSError}\)
values for all the strategies in the tournament, as reported
in~\cite{Stewart2012} the extortionate strategy (which has an expected
\(\text{SSError}\) approximately 0) gains a large number of wins.

\begin{figure}[!htbp]
    \centering
    \includegraphics[width=.8\textwidth]{./assets/img/SSError_overall_in_stewart_plotkin/main.pdf}
    \caption{\(\text{SSError}\) and state probabilities for the strategies
        of~\cite{Stewart2012}, ordered both by number of wins and overall score.
        Note that \(P(DC)\) is not shown as it corresponds to the transpose of
        \(P(CD)\). Cooperator and Defector are omitted as they do not visit all
        the states.}
    \label{fig:SSError_overall_in_stewart_plotkin}
\end{figure}

Here, the work of~\cite{Stewart2012} is extended by investigating a tournament
with \input{assets/tex/number_of_full_strategies/main.tex}
strategies.

The results of this analysis are shown in
Figure~\ref{fig:SSError_and_probabilities_in_full}. The top ranking strategies
by number of wins seem to be extortionate (but not against all strategies) and
it can be seen that a small sub group of strategies achieve mutual defection.
All the top ranking strategies according to score achieve mutual cooperation and
do not extort each other, however they
\textbf{do} exhibit extortionate behaviour towards a number of the lower ranking
strategies.

\begin{figure}[!htbp]
    \centering
    \includegraphics[width=.8\textwidth]{./assets/img/SSError_and_probabilities_in_full/main.pdf}
    \caption{\(\text{SSError}\) for the strategies for the full tournament. Only
    strategy interactions for which \(p_4=0\) and \(\chi>1\) are displayed.}
    \label{fig:SSError_and_probabilities_in_full}
\end{figure}

\section{Conclusion}\label{sec:conclusion}

This work defines an approach to measure whether or not a player is playing a
strategy that corresponds to an extortionate strategy as defined
in~\cite{Press2012}: a mathematical model for suspicion. Indeed, all
extortionate strategies have been
 classified as lying on a triangular plane.
This rigorous classification fails to be robust to small measurement error, thus
a statistical approach is proposed.
This is done through a linear algebraic approach for approximating the solution
of a linear system. Using this, a large number of pairwise interactions is
simulated and in fact very few strategies are found to act extortionately.

The work of~\cite{Press2012}, whilst showing that a clever approach to taking
advantage of another memory one strategy exists: this is incomplete. Whilst the
elegance of this result is very attractive, just as the simplicity of the
victory of Tit For Tat in Axelrod's original tournaments was, it is incomplete.
Extortionate strategies achieve a high number of wins but they do not
achieve a high score which corresponds to the fitness landscape in an
evolutionary sense. From the large number of interactions a payoff matrix \(S\)
can be measured where \(S_{ij}\) denotes the score (using standard values of
\((R, S, T, P) = (3, 0, 5, 1)\)) of the \(i\)th strategy
against the \(j\)th strategy. Using this, the replicator equation
describes the evolution of the system based on a population density fitness
function:

\begin{equation}\label{eqn:replicator_dynamics}
    \frac{dx}{dt} = x(S-x^TS x)
\end{equation}

Equation (\ref{eqn:replicator_dynamics}) is solved numerically through an
integration technique described in~\cite{Petzold1983} and
Figure~\ref{fig:replicator_dynamics} shows the evolution of the distribution of
the system: the various strategies are ranked by scores. It is clear to see that
only the high ranking strategies survive the evolutionary process (in fact,
only \input{./assets/img/replicator_dynamics/main.tex}
have a final distribution greater than \(10 ^ {-2}\)). This confirms the
findings of~\cite{Moran1707} in which sophisticated strategies resist
evolutionary invasion of shorter memory strategies. Recalling
Figure~\ref{fig:SSError_and_probabilities_in_full} this demonstrates that:

\begin{itemize}
    \item Cooperation emerges through the evolutionary process: the high scoring
        strategies do not exhibit extortionate behaviour towards each other.
    \item Extortionate strategies do not survive the evolutionary process.
\end{itemize}

\begin{figure}[!htbp]
    \centering
    \includegraphics[width=.8\textwidth]{./assets/img/replicator_dynamics/main.pdf}
    \caption{Numerical simulation of the replicator equation
    (\ref{eqn:replicator_dynamics}): strategies are ordered by score, only the strategies with a high score survive the evolutionary process.}
    \label{fig:replicator_dynamics}
\end{figure}

This work can be used to classify plays of the IPD\@: data can be collected from
actual interactions (in lab or in the field). Furthermore, this allows for a
classification method similar to the notion of fingerprinting presented
in~\cite{Ashlock2008}. Trained strategies can potentially be classified as
extortionate or not or it could be possible to even constrain the reinforcement
learning approaches that are becoming prevalent in the literature.
Alternatively, this mathematical approach for recognising extortion could be
used in sophisticated strategies to defend against invasion. Arguably, some of
the strategies considered here exhibit this behaviour, indeed as described
in~\cite{Harper2017}, the top ranking strategies in the full tournament are
obtained using evolutionary reinforcement learning techniques, thus, suspicion
of extortionate behaviour could in fact be an evolutionary trait.

\section*{Acknowledgements}

The following open source software libraries were used in this research:

\begin{itemize}
    \item The Axelrod ~\cite{Knight2016, Knight2018} library (IPD strategies and
        tournaments).
    \item The sympy library~\cite{Meurer2017} (verification of all symbolic
        calculations).
    \item The matplotlib~\cite{Droettboom2018} library (visualisation).
    \item The pandas~\cite{Structures2010}, dask~\cite{Dask2016} and
        NumPy~\cite{Oliphant2015} libraries (data manipulation).
    \item The SciPy~\cite{Jones2001} library (numerical integration of the
        replicator equation).
\end{itemize}

This work was performed using the computational facilities of the Advanced
Research Computing @ Cardiff (ARCCA) Division, Cardiff University.

\printbibliography

\newpage
\section*{Supplementary materials}

\includepdf{assets/pdf/proof_of_form_of_extortionate_strategies/main.pdf}

\newpage

Using the pair wise interactions the transition rates \(p,
q\) can be measured and the steady state probabilities inferred and compared to
the actual probabilities of each state.
This is done numerically by computing the singular eigenvector of the
matrix \(A\) \cite{Stewart2009}:

\[
    A =
    \begin{bmatrix}
        p_1 q_1 & p_1 (1 - q_1) & (1 - p_1) q_1 & (1 -p_1) (1 - q_1) \\
        p_2 q_2 & p_2 (1 - q_2) & (1 - p_2) q_2 & (1 -p_2) (1 - q_2) \\
        p_3 q_3 & p_3 (1 - q_3) & (1 - p_3) q_3 & (1 -p_3) (1 - q_3) \\
        p_4 q_4 & p_4 (1 - q_4) & (1 - p_4) q_4 & (1 -p_4) (1 - q_4) \\
    \end{bmatrix}
\]

Figure~\ref{fig:computed_probabilities_vs_theoretic_probabilities} shows a
regression line fitted to every pairwise interaction with a reported
\(\text{SSError}\) value (pairwise interactions with missing states were
omitted). This serves to validate the approach: a part from some edge cases the
relationship is consistent.

\begin{figure}[!htbp]
    \centering
    \includegraphics[width=.8\textwidth]{./assets/img/computed_probabilities_vs_theoretic_probabilities/main.pdf}
    \caption{The
        relationship between the steady state probabilities inferred from the
        measured transitions and the actual steady state probabilities. A linear
        regression line is included validating the approach.}
    \label{fig:computed_probabilities_vs_theoretic_probabilities}
\end{figure}


\end{document}

have a final distribution greater than \(10 ^ {-2}\)). This confirms the
findings of~\cite{Moran1707} in which sophisticated strategies resist
evolutionary invasion of shorter memory strategies. Recalling
Figure~\ref{fig:SSError_and_probabilities_in_full} this demonstrates that:

\begin{itemize}
    \item Cooperation emerges through the evolutionary process: the high scoring
        strategies do not exhibit extortionate behaviour towards each other.
    \item Extortionate strategies do not survive the evolutionary process.
\end{itemize}

\begin{figure}[!htbp]
    \centering
    \includegraphics[width=.8\textwidth]{./assets/img/replicator_dynamics/main.pdf}
    \caption{Numerical simulation of the replicator equation
    (\ref{eqn:replicator_dynamics}): strategies are ordered by score, only the strategies with a high score survive the evolutionary process.}
    \label{fig:replicator_dynamics}
\end{figure}

This work can be used to classify plays of the IPD\@: data can be collected from
actual interactions (in lab or in the field). Furthermore, this allows for a
classification method similar to the notion of fingerprinting presented
in~\cite{Ashlock2008}. Trained strategies can potentially be classified as
extortionate or not or it could be possible to even constrain the reinforcement
learning approaches that are becoming prevalent in the literature.
Alternatively, this mathematical approach for recognising extortion could be
used in sophisticated strategies to defend against invasion. Arguably, some of
the strategies considered here exhibit this behaviour, indeed as described
in~\cite{Harper2017}, the top ranking strategies in the full tournament are
obtained using evolutionary reinforcement learning techniques, thus, suspicion
of extortionate behaviour could in fact be an evolutionary trait.

\section*{Acknowledgements}

The following open source software libraries were used in this research:

\begin{itemize}
    \item The Axelrod ~\cite{Knight2016, Knight2018} library (IPD strategies and
        tournaments).
    \item The sympy library~\cite{Meurer2017} (verification of all symbolic
        calculations).
    \item The matplotlib~\cite{Droettboom2018} library (visualisation).
    \item The pandas~\cite{Structures2010}, dask~\cite{Dask2016} and
        NumPy~\cite{Oliphant2015} libraries (data manipulation).
    \item The SciPy~\cite{Jones2001} library (numerical integration of the
        replicator equation).
\end{itemize}

This work was performed using the computational facilities of the Advanced
Research Computing @ Cardiff (ARCCA) Division, Cardiff University.

\printbibliography

\newpage
\section*{Supplementary materials}

\includepdf{assets/pdf/proof_of_form_of_extortionate_strategies/main.pdf}

\newpage

Using the pair wise interactions the transition rates \(p,
q\) can be measured and the steady state probabilities inferred and compared to
the actual probabilities of each state.
This is done numerically by computing the singular eigenvector of the
matrix \(A\) \cite{Stewart2009}:

\[
    A =
    \begin{bmatrix}
        p_1 q_1 & p_1 (1 - q_1) & (1 - p_1) q_1 & (1 -p_1) (1 - q_1) \\
        p_2 q_2 & p_2 (1 - q_2) & (1 - p_2) q_2 & (1 -p_2) (1 - q_2) \\
        p_3 q_3 & p_3 (1 - q_3) & (1 - p_3) q_3 & (1 -p_3) (1 - q_3) \\
        p_4 q_4 & p_4 (1 - q_4) & (1 - p_4) q_4 & (1 -p_4) (1 - q_4) \\
    \end{bmatrix}
\]

Figure~\ref{fig:computed_probabilities_vs_theoretic_probabilities} shows a
regression line fitted to every pairwise interaction with a reported
\(\text{SSError}\) value (pairwise interactions with missing states were
omitted). This serves to validate the approach: a part from some edge cases the
relationship is consistent.

\begin{figure}[!htbp]
    \centering
    \includegraphics[width=.8\textwidth]{./assets/img/computed_probabilities_vs_theoretic_probabilities/main.pdf}
    \caption{The
        relationship between the steady state probabilities inferred from the
        measured transitions and the actual steady state probabilities. A linear
        regression line is included validating the approach.}
    \label{fig:computed_probabilities_vs_theoretic_probabilities}
\end{figure}


\end{document}
 turns and every match has been
repeated \documentclass[a4paper]{article}

\usepackage{amsmath}
\usepackage{amssymb}
\usepackage[margin=1.5cm,
            includefoot,
            footskip=30pt]{geometry}
\usepackage{layout}
\usepackage{graphicx}
\usepackage{subcaption}

\usepackage{biblatex}
\usepackage{pdfpages}

\bibliography{main.bib}

\title{Suspicion: Recognising and evaluating the effectiveness
       of extortion in the Iterated Prisoner's Dilemma}
\author{Vincent A. Knight \and Nikoleta E. Glynatsi}
\date{\today}



\begin{document}

\maketitle

\begin{abstract}
    The Iterated Prisoner's Dilemma is a model for rational and evolutionary
    interactive behaviour. It has applications both in the study of human social
    behaviour as well as in biology.
    It is used to understand when and how a rational individual might
    accept an immediate cost to their own utility for the direct benefit of
    another.

    Much attention has been given to a class of strategies called
    Zero Determinant strategies. It has been theoretically shown that these
    strategies can ``extort'' any player.

    In this work, an approach to identify if observed strategies are playing in
    an extortionate way is described. Furthermore, experimental analysis of
    a large tournament with \documentclass[a4paper]{article}

\usepackage{amsmath}
\usepackage{amssymb}
\usepackage[margin=1.5cm,
            includefoot,
            footskip=30pt]{geometry}
\usepackage{layout}
\usepackage{graphicx}
\usepackage{subcaption}

\usepackage{biblatex}
\usepackage{pdfpages}

\bibliography{main.bib}

\title{Suspicion: Recognising and evaluating the effectiveness
       of extortion in the Iterated Prisoner's Dilemma}
\author{Vincent A. Knight \and Nikoleta E. Glynatsi}
\date{\today}



\begin{document}

\maketitle

\begin{abstract}
    The Iterated Prisoner's Dilemma is a model for rational and evolutionary
    interactive behaviour. It has applications both in the study of human social
    behaviour as well as in biology.
    It is used to understand when and how a rational individual might
    accept an immediate cost to their own utility for the direct benefit of
    another.

    Much attention has been given to a class of strategies called
    Zero Determinant strategies. It has been theoretically shown that these
    strategies can ``extort'' any player.

    In this work, an approach to identify if observed strategies are playing in
    an extortionate way is described. Furthermore, experimental analysis of
    a large tournament with \input{assets/tex/number_of_full_strategies/main.tex}
    strategies is considered. In this setting
    the most highly performing strategies do not play in an extortionate way
    against each other but do against lower performing strategies.
    This suggests that whilst the theory of Zero Determinant strategies
    indicates that memory is not of fundamental importance to the evolution of
    cooperative behaviour, this is incomplete.
\end{abstract}

\section{Introduction}\label{sec:introduction}

Agent based game theoretic models have become a stalwart of the underpinning
mathematics of interactive behaviours. One of the major pieces of work
in this area is the pair of original computer tournaments run by Robert
Axelrod~\cite{Axelrod1980, Axelrod1980a}. These tournaments pitted submitted
computer strategies against each other in plays of the Iterated Prisoner's
Dilemma. A common game where agents can choose to pay a slight cost to their
immediate utility in the hope of building a reputation. This has been used in
economic and evolutionary game theory to understand the evolution of cooperative
behaviour.

Recently, a class of strategies was described in~\cite{Press2012} that can
provably extort any given opponent. In~\cite{Hilbe2013, Moran1707} some
questions have already been asked about the true effectiveness of these
strategies in an evolutionary setting. Here another question is asked: is it
possible to recognise this extortionate behaviour? A mathematical procedure for
suspicion is presented: in the same way that the continued actions of an
extortionate individual might raise suspicion.

This work makes use of the Axelrod Python library~\cite{Knight2018, Knight2016}
with a large number of Prisoner Dilemma strategies available to give an
extensive numerical example of the ideas presented.  The approach is presented
in Section~\ref{sec:delta-zd-strategies}.  All of the code and data discussed
in Section~\ref{sec:numerical-experiments} is open sourced, archived and
written according to best scientific principles~\cite{Wilson2014}. The data
archive can be found at~\cite{vincent_knight_2018_1297075}.

\section{Recognising Extortion}\label{sec:delta-zd-strategies}

In~\cite{Press2012}, given a match between 2 memory-one strategies, the concept
of Zero Determinant (ZD) strategies is introduced. The main result of that paper
shows that given two memory one players \(p, q\in\mathbb{R}^4\) a linear
relationship between the players' scores could be forced by one of the players.

Using the notation of~\cite{Press2012}, assuming the utilities for player \(p\)
are given by \(S_x=(R, S, T, P)\) and for player \(q\) by \(S_y=(R, T, S, P)\)
and that the stationary scores of each player is given by \(S_X\) and \(S_Y\)
respectively. The main result of~\cite{Press2012} is that if

\begin{equation}\label{eqn:linear_relationship_for_p}
    \tilde p=\alpha S_x + \beta S_y + \gamma
\end{equation}

or

\begin{equation}\label{eqn:linear_relationship_for_q}
    \tilde q=\alpha S_x + \beta S_y + \gamma
\end{equation}

where \(\tilde p = (1 - p_1, 1 - p_2, p_3, p_4)\) and
\(\tilde q = (1 - q_1, 1 - q_2, q_3, q_4)\) then:

\begin{equation}
    \alpha S_X + \beta S_Y + \gamma = 0
\end{equation}

In~\cite{Press2012} a particular type of ZD strategy is defined: extortionate
strategies. If:

\begin{equation}\label{eqn:constraint_for_extortion}
    \gamma = - P(\alpha + \beta)
\end{equation}

then the player can ensure they get a score \(\chi\) times
larger than the opponent. This extortion coefficient is given by:

\begin{equation}\label{eqn:definition_of_chi}
    \chi=\frac{-\beta}{\alpha}
\end{equation}

Thus, if (\ref{eqn:constraint_for_extortion}) holds and \(\chi >1\) a player is
said to extort their opponent.
Here, the reverse problem is considered: given a
\(p\in\mathbb{R}^4\) how does one identify \(\alpha, \beta\) if they
exist and is the strategy in fact acting in an extortionate way?

These conditions correspond to:

\begin{align}
    \tilde p_1 & = \alpha R + \beta R - P (\alpha + \beta)
            \label{eqn:condition_for_tilde_p1}\\
    \tilde p_2 & = \alpha S + \beta T - P (\alpha + \beta)
            \label{eqn:condition_for_tilde_p2}\\
    \tilde p_3 & = \alpha T + \beta S - P (\alpha + \beta)
            \label{eqn:condition_for_tilde_p3}\\
    \tilde p_4 & = \alpha P + \beta P - P (\alpha + \beta)
            \label{eqn:condition_for_tilde_p4}
\end{align}

Equation (\ref{eqn:condition_for_tilde_p4}) ensures that \(p_4=\tilde p_4=0\).
Equations (\ref{eqn:condition_for_tilde_p1}-\ref{eqn:condition_for_tilde_p3})
can be used to eliminate \(\alpha, \beta\), giving:

\begin{equation}\label{eqn:planar_definition_of_extortion}
    \tilde p_1 = \frac{(R - P)(\tilde p_2 + \tilde p_3)}{S + T - 2P}
\end{equation}

with:

\begin{equation}\label{eqn:definition_of_chi}
    \chi = \frac{\tilde p_2 (P - T) + \tilde p_3 (S - P)}
                {\tilde p_2 (P - S) + \tilde p_3 (T - P)}
\end{equation}

Given a strategy \(p\in\mathbb{R}^{4\times 1}\) equations
(\ref{eqn:condition_for_tilde_p4}), (\ref{eqn:planar_definition_of_extortion}-\ref{eqn:definition_of_chi}) can be used to check if
a strategy is extortionate. The conditions correspond to:

\begin{align}
    p_1 & = \frac{(R-P)(p_2 + p_3) - R + T + S - P}{S + T - 2P}
     \label{eqn:condition_for_p1}\\
    p_4 & = 0 \label{eqn:condition_for_p4}\\
    1 & > p_2 + p_3\label{eqn:condition_for_chi}
\end{align}

The algebraic steps necessary to prove these results are available in the
supporting materials.

All extortionate strategies reside on a triangular (\ref{eqn:condition_for_chi})
plane (\ref{eqn:condition_for_p1}) in 3 dimensions (\ref{eqn:condition_for_p4}).
Using this formulation it can be seen that a necessary (but not sufficient)
condition for an extortionate strategy is that it cooperates on average less
than 50\% of the time when in a state of disagreement with the opponent.

As an example, consider the known extortionate strategy \(p=(8 / 9, 1 / 2, 1 /
3, 0)\) from~\cite{Stewart2012} which is referred to as \texttt{Extort-2}. In
this case, for the standard values of \((R, T, S, P)\) constraint
(\ref{eqn:condition_for_p1}) corresponds to:

\begin{equation}
    p_1 = \frac{2(p_2 + p_3) + 1}{3}
\end{equation}

It is clear that in this case all constraints hold.

This approach could in fact be used to confirm that a given strategy is acting
in an extortionate manner even if it is not a memory one strategy. However, in
practice, if a closed form for \(p\) is not known, then due to measurement
and/or numerical error this would not work.

This problem can be written in the following linear algebraic form where
\(x=(\alpha, \beta)\)
and \(p^*=(\tilde p_1 - 1, tilde_2 - 1, p_3)\):

\begin{equation}\label{eqn:linear_algebraic_equation_for_p}
    Cx= p^*
\end{equation}

\(C\) corresponds to equations
(\ref{eqn:condition_for_tilde_p1}-\ref{eqn:condition_for_tilde_p3}) and is
given by:

\begin{equation}\label{eqn:definition_of_C}
    C =
    \begin{bmatrix}
        R - P & R- P \\
        S - P & T- P \\
        T - P & S- P \\
    \end{bmatrix}
\end{equation}

Note that in general, equation (\ref{eqn:linear_algebraic_equation_for_p}) will
not necessarily have a solution. From the Rouch\'{e}-Capelli theorem if there is
a solution it is unique as \(\text{rank}(C)=2\) which is the dimension of the
variable \(x\). The best fitting \(x\) is found by minimizing:

\begin{equation}\label{eqn:r_squared}
    \text{SSError} = \|C x- p^*\|_2^2 = \sum_{i=1}^{3}\left((C\bar x)_i-p_i^*\right)^2
\end{equation}

Note that \(\text{SSError}\), which is the square of the Frobenius
norm~\cite{Golub2013}, becomes a measure of how close a strategy is to being an
extortionate strategy. Suspicion
of extortion then corresponds to a threshold on \(\text{SSError}\).

By observing interactions (human or otherwise), their memory one representation
can be inferred and this approach can be used to recognise extortionate
behaviour. The notion of comparing theoretic and actual plays of the IPD is not
novel, see for example~\cite{Rand2013}. Immediately it is noted that if the
environment is noisy~\cite{Wu1995} then no strategy can be considered to be
extortionate as \(p_4>0\).

In the next section, this idea will be illustrated by observing the interactions
that take place in a computer based tournament of the IPD\@.

\section{Numerical experiments}\label{sec:numerical-experiments}

In~\cite{Stewart2012} results from a tournament with
\input{./assets/tex/number_of_stewart_plotkin_strategies/main.tex} strategies,
was presented with specific consideration given to ZD strategies. This
tournament is reproduced here using the Axelrod-Python
project~\cite{Knight2016}. To obtain a good measure of the corresponding
transition rates for each strategy all matches have been run for
\input{assets/tex/number_of_turns/main.tex} turns and every match has been
repeated \input{assets/tex/number_of_repetitions/main.tex} times. All of this
interaction data is available at~\cite{vincent_knight_2018_1297075}. A good
match between the inferred Markov chain and the state distribution of the actual
interactions has been verified. Data for this is presented in the supplementary
materials.

Figure~\ref{fig:SSError_overall_in_stewart_plotkin} shows the \(\text{SSError}\)
values for all the strategies in the tournament, as reported
in~\cite{Stewart2012} the extortionate strategy (which has an expected
\(\text{SSError}\) approximately 0) gains a large number of wins.

\begin{figure}[!htbp]
    \centering
    \includegraphics[width=.8\textwidth]{./assets/img/SSError_overall_in_stewart_plotkin/main.pdf}
    \caption{\(\text{SSError}\) and state probabilities for the strategies
        of~\cite{Stewart2012}, ordered both by number of wins and overall score.
        Note that \(P(DC)\) is not shown as it corresponds to the transpose of
        \(P(CD)\). Cooperator and Defector are omitted as they do not visit all
        the states.}
    \label{fig:SSError_overall_in_stewart_plotkin}
\end{figure}

Here, the work of~\cite{Stewart2012} is extended by investigating a tournament
with \input{assets/tex/number_of_full_strategies/main.tex}
strategies.

The results of this analysis are shown in
Figure~\ref{fig:SSError_and_probabilities_in_full}. The top ranking strategies
by number of wins seem to be extortionate (but not against all strategies) and
it can be seen that a small sub group of strategies achieve mutual defection.
All the top ranking strategies according to score achieve mutual cooperation and
do not extort each other, however they
\textbf{do} exhibit extortionate behaviour towards a number of the lower ranking
strategies.

\begin{figure}[!htbp]
    \centering
    \includegraphics[width=.8\textwidth]{./assets/img/SSError_and_probabilities_in_full/main.pdf}
    \caption{\(\text{SSError}\) for the strategies for the full tournament. Only
    strategy interactions for which \(p_4=0\) and \(\chi>1\) are displayed.}
    \label{fig:SSError_and_probabilities_in_full}
\end{figure}

\section{Conclusion}\label{sec:conclusion}

This work defines an approach to measure whether or not a player is playing a
strategy that corresponds to an extortionate strategy as defined
in~\cite{Press2012}: a mathematical model for suspicion. Indeed, all
extortionate strategies have been
 classified as lying on a triangular plane.
This rigorous classification fails to be robust to small measurement error, thus
a statistical approach is proposed.
This is done through a linear algebraic approach for approximating the solution
of a linear system. Using this, a large number of pairwise interactions is
simulated and in fact very few strategies are found to act extortionately.

The work of~\cite{Press2012}, whilst showing that a clever approach to taking
advantage of another memory one strategy exists: this is incomplete. Whilst the
elegance of this result is very attractive, just as the simplicity of the
victory of Tit For Tat in Axelrod's original tournaments was, it is incomplete.
Extortionate strategies achieve a high number of wins but they do not
achieve a high score which corresponds to the fitness landscape in an
evolutionary sense. From the large number of interactions a payoff matrix \(S\)
can be measured where \(S_{ij}\) denotes the score (using standard values of
\((R, S, T, P) = (3, 0, 5, 1)\)) of the \(i\)th strategy
against the \(j\)th strategy. Using this, the replicator equation
describes the evolution of the system based on a population density fitness
function:

\begin{equation}\label{eqn:replicator_dynamics}
    \frac{dx}{dt} = x(S-x^TS x)
\end{equation}

Equation (\ref{eqn:replicator_dynamics}) is solved numerically through an
integration technique described in~\cite{Petzold1983} and
Figure~\ref{fig:replicator_dynamics} shows the evolution of the distribution of
the system: the various strategies are ranked by scores. It is clear to see that
only the high ranking strategies survive the evolutionary process (in fact,
only \input{./assets/img/replicator_dynamics/main.tex}
have a final distribution greater than \(10 ^ {-2}\)). This confirms the
findings of~\cite{Moran1707} in which sophisticated strategies resist
evolutionary invasion of shorter memory strategies. Recalling
Figure~\ref{fig:SSError_and_probabilities_in_full} this demonstrates that:

\begin{itemize}
    \item Cooperation emerges through the evolutionary process: the high scoring
        strategies do not exhibit extortionate behaviour towards each other.
    \item Extortionate strategies do not survive the evolutionary process.
\end{itemize}

\begin{figure}[!htbp]
    \centering
    \includegraphics[width=.8\textwidth]{./assets/img/replicator_dynamics/main.pdf}
    \caption{Numerical simulation of the replicator equation
    (\ref{eqn:replicator_dynamics}): strategies are ordered by score, only the strategies with a high score survive the evolutionary process.}
    \label{fig:replicator_dynamics}
\end{figure}

This work can be used to classify plays of the IPD\@: data can be collected from
actual interactions (in lab or in the field). Furthermore, this allows for a
classification method similar to the notion of fingerprinting presented
in~\cite{Ashlock2008}. Trained strategies can potentially be classified as
extortionate or not or it could be possible to even constrain the reinforcement
learning approaches that are becoming prevalent in the literature.
Alternatively, this mathematical approach for recognising extortion could be
used in sophisticated strategies to defend against invasion. Arguably, some of
the strategies considered here exhibit this behaviour, indeed as described
in~\cite{Harper2017}, the top ranking strategies in the full tournament are
obtained using evolutionary reinforcement learning techniques, thus, suspicion
of extortionate behaviour could in fact be an evolutionary trait.

\section*{Acknowledgements}

The following open source software libraries were used in this research:

\begin{itemize}
    \item The Axelrod ~\cite{Knight2016, Knight2018} library (IPD strategies and
        tournaments).
    \item The sympy library~\cite{Meurer2017} (verification of all symbolic
        calculations).
    \item The matplotlib~\cite{Droettboom2018} library (visualisation).
    \item The pandas~\cite{Structures2010}, dask~\cite{Dask2016} and
        NumPy~\cite{Oliphant2015} libraries (data manipulation).
    \item The SciPy~\cite{Jones2001} library (numerical integration of the
        replicator equation).
\end{itemize}

This work was performed using the computational facilities of the Advanced
Research Computing @ Cardiff (ARCCA) Division, Cardiff University.

\printbibliography

\newpage
\section*{Supplementary materials}

\includepdf{assets/pdf/proof_of_form_of_extortionate_strategies/main.pdf}

\newpage

Using the pair wise interactions the transition rates \(p,
q\) can be measured and the steady state probabilities inferred and compared to
the actual probabilities of each state.
This is done numerically by computing the singular eigenvector of the
matrix \(A\) \cite{Stewart2009}:

\[
    A =
    \begin{bmatrix}
        p_1 q_1 & p_1 (1 - q_1) & (1 - p_1) q_1 & (1 -p_1) (1 - q_1) \\
        p_2 q_2 & p_2 (1 - q_2) & (1 - p_2) q_2 & (1 -p_2) (1 - q_2) \\
        p_3 q_3 & p_3 (1 - q_3) & (1 - p_3) q_3 & (1 -p_3) (1 - q_3) \\
        p_4 q_4 & p_4 (1 - q_4) & (1 - p_4) q_4 & (1 -p_4) (1 - q_4) \\
    \end{bmatrix}
\]

Figure~\ref{fig:computed_probabilities_vs_theoretic_probabilities} shows a
regression line fitted to every pairwise interaction with a reported
\(\text{SSError}\) value (pairwise interactions with missing states were
omitted). This serves to validate the approach: a part from some edge cases the
relationship is consistent.

\begin{figure}[!htbp]
    \centering
    \includegraphics[width=.8\textwidth]{./assets/img/computed_probabilities_vs_theoretic_probabilities/main.pdf}
    \caption{The
        relationship between the steady state probabilities inferred from the
        measured transitions and the actual steady state probabilities. A linear
        regression line is included validating the approach.}
    \label{fig:computed_probabilities_vs_theoretic_probabilities}
\end{figure}


\end{document}

    strategies is considered. In this setting
    the most highly performing strategies do not play in an extortionate way
    against each other but do against lower performing strategies.
    This suggests that whilst the theory of Zero Determinant strategies
    indicates that memory is not of fundamental importance to the evolution of
    cooperative behaviour, this is incomplete.
\end{abstract}

\section{Introduction}\label{sec:introduction}

Agent based game theoretic models have become a stalwart of the underpinning
mathematics of interactive behaviours. One of the major pieces of work
in this area is the pair of original computer tournaments run by Robert
Axelrod~\cite{Axelrod1980, Axelrod1980a}. These tournaments pitted submitted
computer strategies against each other in plays of the Iterated Prisoner's
Dilemma. A common game where agents can choose to pay a slight cost to their
immediate utility in the hope of building a reputation. This has been used in
economic and evolutionary game theory to understand the evolution of cooperative
behaviour.

Recently, a class of strategies was described in~\cite{Press2012} that can
provably extort any given opponent. In~\cite{Hilbe2013, Moran1707} some
questions have already been asked about the true effectiveness of these
strategies in an evolutionary setting. Here another question is asked: is it
possible to recognise this extortionate behaviour? A mathematical procedure for
suspicion is presented: in the same way that the continued actions of an
extortionate individual might raise suspicion.

This work makes use of the Axelrod Python library~\cite{Knight2018, Knight2016}
with a large number of Prisoner Dilemma strategies available to give an
extensive numerical example of the ideas presented.  The approach is presented
in Section~\ref{sec:delta-zd-strategies}.  All of the code and data discussed
in Section~\ref{sec:numerical-experiments} is open sourced, archived and
written according to best scientific principles~\cite{Wilson2014}. The data
archive can be found at~\cite{vincent_knight_2018_1297075}.

\section{Recognising Extortion}\label{sec:delta-zd-strategies}

In~\cite{Press2012}, given a match between 2 memory-one strategies, the concept
of Zero Determinant (ZD) strategies is introduced. The main result of that paper
shows that given two memory one players \(p, q\in\mathbb{R}^4\) a linear
relationship between the players' scores could be forced by one of the players.

Using the notation of~\cite{Press2012}, assuming the utilities for player \(p\)
are given by \(S_x=(R, S, T, P)\) and for player \(q\) by \(S_y=(R, T, S, P)\)
and that the stationary scores of each player is given by \(S_X\) and \(S_Y\)
respectively. The main result of~\cite{Press2012} is that if

\begin{equation}\label{eqn:linear_relationship_for_p}
    \tilde p=\alpha S_x + \beta S_y + \gamma
\end{equation}

or

\begin{equation}\label{eqn:linear_relationship_for_q}
    \tilde q=\alpha S_x + \beta S_y + \gamma
\end{equation}

where \(\tilde p = (1 - p_1, 1 - p_2, p_3, p_4)\) and
\(\tilde q = (1 - q_1, 1 - q_2, q_3, q_4)\) then:

\begin{equation}
    \alpha S_X + \beta S_Y + \gamma = 0
\end{equation}

In~\cite{Press2012} a particular type of ZD strategy is defined: extortionate
strategies. If:

\begin{equation}\label{eqn:constraint_for_extortion}
    \gamma = - P(\alpha + \beta)
\end{equation}

then the player can ensure they get a score \(\chi\) times
larger than the opponent. This extortion coefficient is given by:

\begin{equation}\label{eqn:definition_of_chi}
    \chi=\frac{-\beta}{\alpha}
\end{equation}

Thus, if (\ref{eqn:constraint_for_extortion}) holds and \(\chi >1\) a player is
said to extort their opponent.
Here, the reverse problem is considered: given a
\(p\in\mathbb{R}^4\) how does one identify \(\alpha, \beta\) if they
exist and is the strategy in fact acting in an extortionate way?

These conditions correspond to:

\begin{align}
    \tilde p_1 & = \alpha R + \beta R - P (\alpha + \beta)
            \label{eqn:condition_for_tilde_p1}\\
    \tilde p_2 & = \alpha S + \beta T - P (\alpha + \beta)
            \label{eqn:condition_for_tilde_p2}\\
    \tilde p_3 & = \alpha T + \beta S - P (\alpha + \beta)
            \label{eqn:condition_for_tilde_p3}\\
    \tilde p_4 & = \alpha P + \beta P - P (\alpha + \beta)
            \label{eqn:condition_for_tilde_p4}
\end{align}

Equation (\ref{eqn:condition_for_tilde_p4}) ensures that \(p_4=\tilde p_4=0\).
Equations (\ref{eqn:condition_for_tilde_p1}-\ref{eqn:condition_for_tilde_p3})
can be used to eliminate \(\alpha, \beta\), giving:

\begin{equation}\label{eqn:planar_definition_of_extortion}
    \tilde p_1 = \frac{(R - P)(\tilde p_2 + \tilde p_3)}{S + T - 2P}
\end{equation}

with:

\begin{equation}\label{eqn:definition_of_chi}
    \chi = \frac{\tilde p_2 (P - T) + \tilde p_3 (S - P)}
                {\tilde p_2 (P - S) + \tilde p_3 (T - P)}
\end{equation}

Given a strategy \(p\in\mathbb{R}^{4\times 1}\) equations
(\ref{eqn:condition_for_tilde_p4}), (\ref{eqn:planar_definition_of_extortion}-\ref{eqn:definition_of_chi}) can be used to check if
a strategy is extortionate. The conditions correspond to:

\begin{align}
    p_1 & = \frac{(R-P)(p_2 + p_3) - R + T + S - P}{S + T - 2P}
     \label{eqn:condition_for_p1}\\
    p_4 & = 0 \label{eqn:condition_for_p4}\\
    1 & > p_2 + p_3\label{eqn:condition_for_chi}
\end{align}

The algebraic steps necessary to prove these results are available in the
supporting materials.

All extortionate strategies reside on a triangular (\ref{eqn:condition_for_chi})
plane (\ref{eqn:condition_for_p1}) in 3 dimensions (\ref{eqn:condition_for_p4}).
Using this formulation it can be seen that a necessary (but not sufficient)
condition for an extortionate strategy is that it cooperates on average less
than 50\% of the time when in a state of disagreement with the opponent.

As an example, consider the known extortionate strategy \(p=(8 / 9, 1 / 2, 1 /
3, 0)\) from~\cite{Stewart2012} which is referred to as \texttt{Extort-2}. In
this case, for the standard values of \((R, T, S, P)\) constraint
(\ref{eqn:condition_for_p1}) corresponds to:

\begin{equation}
    p_1 = \frac{2(p_2 + p_3) + 1}{3}
\end{equation}

It is clear that in this case all constraints hold.

This approach could in fact be used to confirm that a given strategy is acting
in an extortionate manner even if it is not a memory one strategy. However, in
practice, if a closed form for \(p\) is not known, then due to measurement
and/or numerical error this would not work.

This problem can be written in the following linear algebraic form where
\(x=(\alpha, \beta)\)
and \(p^*=(\tilde p_1 - 1, tilde_2 - 1, p_3)\):

\begin{equation}\label{eqn:linear_algebraic_equation_for_p}
    Cx= p^*
\end{equation}

\(C\) corresponds to equations
(\ref{eqn:condition_for_tilde_p1}-\ref{eqn:condition_for_tilde_p3}) and is
given by:

\begin{equation}\label{eqn:definition_of_C}
    C =
    \begin{bmatrix}
        R - P & R- P \\
        S - P & T- P \\
        T - P & S- P \\
    \end{bmatrix}
\end{equation}

Note that in general, equation (\ref{eqn:linear_algebraic_equation_for_p}) will
not necessarily have a solution. From the Rouch\'{e}-Capelli theorem if there is
a solution it is unique as \(\text{rank}(C)=2\) which is the dimension of the
variable \(x\). The best fitting \(x\) is found by minimizing:

\begin{equation}\label{eqn:r_squared}
    \text{SSError} = \|C x- p^*\|_2^2 = \sum_{i=1}^{3}\left((C\bar x)_i-p_i^*\right)^2
\end{equation}

Note that \(\text{SSError}\), which is the square of the Frobenius
norm~\cite{Golub2013}, becomes a measure of how close a strategy is to being an
extortionate strategy. Suspicion
of extortion then corresponds to a threshold on \(\text{SSError}\).

By observing interactions (human or otherwise), their memory one representation
can be inferred and this approach can be used to recognise extortionate
behaviour. The notion of comparing theoretic and actual plays of the IPD is not
novel, see for example~\cite{Rand2013}. Immediately it is noted that if the
environment is noisy~\cite{Wu1995} then no strategy can be considered to be
extortionate as \(p_4>0\).

In the next section, this idea will be illustrated by observing the interactions
that take place in a computer based tournament of the IPD\@.

\section{Numerical experiments}\label{sec:numerical-experiments}

In~\cite{Stewart2012} results from a tournament with
\documentclass[a4paper]{article}

\usepackage{amsmath}
\usepackage{amssymb}
\usepackage[margin=1.5cm,
            includefoot,
            footskip=30pt]{geometry}
\usepackage{layout}
\usepackage{graphicx}
\usepackage{subcaption}

\usepackage{biblatex}
\usepackage{pdfpages}

\bibliography{main.bib}

\title{Suspicion: Recognising and evaluating the effectiveness
       of extortion in the Iterated Prisoner's Dilemma}
\author{Vincent A. Knight \and Nikoleta E. Glynatsi}
\date{\today}



\begin{document}

\maketitle

\begin{abstract}
    The Iterated Prisoner's Dilemma is a model for rational and evolutionary
    interactive behaviour. It has applications both in the study of human social
    behaviour as well as in biology.
    It is used to understand when and how a rational individual might
    accept an immediate cost to their own utility for the direct benefit of
    another.

    Much attention has been given to a class of strategies called
    Zero Determinant strategies. It has been theoretically shown that these
    strategies can ``extort'' any player.

    In this work, an approach to identify if observed strategies are playing in
    an extortionate way is described. Furthermore, experimental analysis of
    a large tournament with \input{assets/tex/number_of_full_strategies/main.tex}
    strategies is considered. In this setting
    the most highly performing strategies do not play in an extortionate way
    against each other but do against lower performing strategies.
    This suggests that whilst the theory of Zero Determinant strategies
    indicates that memory is not of fundamental importance to the evolution of
    cooperative behaviour, this is incomplete.
\end{abstract}

\section{Introduction}\label{sec:introduction}

Agent based game theoretic models have become a stalwart of the underpinning
mathematics of interactive behaviours. One of the major pieces of work
in this area is the pair of original computer tournaments run by Robert
Axelrod~\cite{Axelrod1980, Axelrod1980a}. These tournaments pitted submitted
computer strategies against each other in plays of the Iterated Prisoner's
Dilemma. A common game where agents can choose to pay a slight cost to their
immediate utility in the hope of building a reputation. This has been used in
economic and evolutionary game theory to understand the evolution of cooperative
behaviour.

Recently, a class of strategies was described in~\cite{Press2012} that can
provably extort any given opponent. In~\cite{Hilbe2013, Moran1707} some
questions have already been asked about the true effectiveness of these
strategies in an evolutionary setting. Here another question is asked: is it
possible to recognise this extortionate behaviour? A mathematical procedure for
suspicion is presented: in the same way that the continued actions of an
extortionate individual might raise suspicion.

This work makes use of the Axelrod Python library~\cite{Knight2018, Knight2016}
with a large number of Prisoner Dilemma strategies available to give an
extensive numerical example of the ideas presented.  The approach is presented
in Section~\ref{sec:delta-zd-strategies}.  All of the code and data discussed
in Section~\ref{sec:numerical-experiments} is open sourced, archived and
written according to best scientific principles~\cite{Wilson2014}. The data
archive can be found at~\cite{vincent_knight_2018_1297075}.

\section{Recognising Extortion}\label{sec:delta-zd-strategies}

In~\cite{Press2012}, given a match between 2 memory-one strategies, the concept
of Zero Determinant (ZD) strategies is introduced. The main result of that paper
shows that given two memory one players \(p, q\in\mathbb{R}^4\) a linear
relationship between the players' scores could be forced by one of the players.

Using the notation of~\cite{Press2012}, assuming the utilities for player \(p\)
are given by \(S_x=(R, S, T, P)\) and for player \(q\) by \(S_y=(R, T, S, P)\)
and that the stationary scores of each player is given by \(S_X\) and \(S_Y\)
respectively. The main result of~\cite{Press2012} is that if

\begin{equation}\label{eqn:linear_relationship_for_p}
    \tilde p=\alpha S_x + \beta S_y + \gamma
\end{equation}

or

\begin{equation}\label{eqn:linear_relationship_for_q}
    \tilde q=\alpha S_x + \beta S_y + \gamma
\end{equation}

where \(\tilde p = (1 - p_1, 1 - p_2, p_3, p_4)\) and
\(\tilde q = (1 - q_1, 1 - q_2, q_3, q_4)\) then:

\begin{equation}
    \alpha S_X + \beta S_Y + \gamma = 0
\end{equation}

In~\cite{Press2012} a particular type of ZD strategy is defined: extortionate
strategies. If:

\begin{equation}\label{eqn:constraint_for_extortion}
    \gamma = - P(\alpha + \beta)
\end{equation}

then the player can ensure they get a score \(\chi\) times
larger than the opponent. This extortion coefficient is given by:

\begin{equation}\label{eqn:definition_of_chi}
    \chi=\frac{-\beta}{\alpha}
\end{equation}

Thus, if (\ref{eqn:constraint_for_extortion}) holds and \(\chi >1\) a player is
said to extort their opponent.
Here, the reverse problem is considered: given a
\(p\in\mathbb{R}^4\) how does one identify \(\alpha, \beta\) if they
exist and is the strategy in fact acting in an extortionate way?

These conditions correspond to:

\begin{align}
    \tilde p_1 & = \alpha R + \beta R - P (\alpha + \beta)
            \label{eqn:condition_for_tilde_p1}\\
    \tilde p_2 & = \alpha S + \beta T - P (\alpha + \beta)
            \label{eqn:condition_for_tilde_p2}\\
    \tilde p_3 & = \alpha T + \beta S - P (\alpha + \beta)
            \label{eqn:condition_for_tilde_p3}\\
    \tilde p_4 & = \alpha P + \beta P - P (\alpha + \beta)
            \label{eqn:condition_for_tilde_p4}
\end{align}

Equation (\ref{eqn:condition_for_tilde_p4}) ensures that \(p_4=\tilde p_4=0\).
Equations (\ref{eqn:condition_for_tilde_p1}-\ref{eqn:condition_for_tilde_p3})
can be used to eliminate \(\alpha, \beta\), giving:

\begin{equation}\label{eqn:planar_definition_of_extortion}
    \tilde p_1 = \frac{(R - P)(\tilde p_2 + \tilde p_3)}{S + T - 2P}
\end{equation}

with:

\begin{equation}\label{eqn:definition_of_chi}
    \chi = \frac{\tilde p_2 (P - T) + \tilde p_3 (S - P)}
                {\tilde p_2 (P - S) + \tilde p_3 (T - P)}
\end{equation}

Given a strategy \(p\in\mathbb{R}^{4\times 1}\) equations
(\ref{eqn:condition_for_tilde_p4}), (\ref{eqn:planar_definition_of_extortion}-\ref{eqn:definition_of_chi}) can be used to check if
a strategy is extortionate. The conditions correspond to:

\begin{align}
    p_1 & = \frac{(R-P)(p_2 + p_3) - R + T + S - P}{S + T - 2P}
     \label{eqn:condition_for_p1}\\
    p_4 & = 0 \label{eqn:condition_for_p4}\\
    1 & > p_2 + p_3\label{eqn:condition_for_chi}
\end{align}

The algebraic steps necessary to prove these results are available in the
supporting materials.

All extortionate strategies reside on a triangular (\ref{eqn:condition_for_chi})
plane (\ref{eqn:condition_for_p1}) in 3 dimensions (\ref{eqn:condition_for_p4}).
Using this formulation it can be seen that a necessary (but not sufficient)
condition for an extortionate strategy is that it cooperates on average less
than 50\% of the time when in a state of disagreement with the opponent.

As an example, consider the known extortionate strategy \(p=(8 / 9, 1 / 2, 1 /
3, 0)\) from~\cite{Stewart2012} which is referred to as \texttt{Extort-2}. In
this case, for the standard values of \((R, T, S, P)\) constraint
(\ref{eqn:condition_for_p1}) corresponds to:

\begin{equation}
    p_1 = \frac{2(p_2 + p_3) + 1}{3}
\end{equation}

It is clear that in this case all constraints hold.

This approach could in fact be used to confirm that a given strategy is acting
in an extortionate manner even if it is not a memory one strategy. However, in
practice, if a closed form for \(p\) is not known, then due to measurement
and/or numerical error this would not work.

This problem can be written in the following linear algebraic form where
\(x=(\alpha, \beta)\)
and \(p^*=(\tilde p_1 - 1, tilde_2 - 1, p_3)\):

\begin{equation}\label{eqn:linear_algebraic_equation_for_p}
    Cx= p^*
\end{equation}

\(C\) corresponds to equations
(\ref{eqn:condition_for_tilde_p1}-\ref{eqn:condition_for_tilde_p3}) and is
given by:

\begin{equation}\label{eqn:definition_of_C}
    C =
    \begin{bmatrix}
        R - P & R- P \\
        S - P & T- P \\
        T - P & S- P \\
    \end{bmatrix}
\end{equation}

Note that in general, equation (\ref{eqn:linear_algebraic_equation_for_p}) will
not necessarily have a solution. From the Rouch\'{e}-Capelli theorem if there is
a solution it is unique as \(\text{rank}(C)=2\) which is the dimension of the
variable \(x\). The best fitting \(x\) is found by minimizing:

\begin{equation}\label{eqn:r_squared}
    \text{SSError} = \|C x- p^*\|_2^2 = \sum_{i=1}^{3}\left((C\bar x)_i-p_i^*\right)^2
\end{equation}

Note that \(\text{SSError}\), which is the square of the Frobenius
norm~\cite{Golub2013}, becomes a measure of how close a strategy is to being an
extortionate strategy. Suspicion
of extortion then corresponds to a threshold on \(\text{SSError}\).

By observing interactions (human or otherwise), their memory one representation
can be inferred and this approach can be used to recognise extortionate
behaviour. The notion of comparing theoretic and actual plays of the IPD is not
novel, see for example~\cite{Rand2013}. Immediately it is noted that if the
environment is noisy~\cite{Wu1995} then no strategy can be considered to be
extortionate as \(p_4>0\).

In the next section, this idea will be illustrated by observing the interactions
that take place in a computer based tournament of the IPD\@.

\section{Numerical experiments}\label{sec:numerical-experiments}

In~\cite{Stewart2012} results from a tournament with
\input{./assets/tex/number_of_stewart_plotkin_strategies/main.tex} strategies,
was presented with specific consideration given to ZD strategies. This
tournament is reproduced here using the Axelrod-Python
project~\cite{Knight2016}. To obtain a good measure of the corresponding
transition rates for each strategy all matches have been run for
\input{assets/tex/number_of_turns/main.tex} turns and every match has been
repeated \input{assets/tex/number_of_repetitions/main.tex} times. All of this
interaction data is available at~\cite{vincent_knight_2018_1297075}. A good
match between the inferred Markov chain and the state distribution of the actual
interactions has been verified. Data for this is presented in the supplementary
materials.

Figure~\ref{fig:SSError_overall_in_stewart_plotkin} shows the \(\text{SSError}\)
values for all the strategies in the tournament, as reported
in~\cite{Stewart2012} the extortionate strategy (which has an expected
\(\text{SSError}\) approximately 0) gains a large number of wins.

\begin{figure}[!htbp]
    \centering
    \includegraphics[width=.8\textwidth]{./assets/img/SSError_overall_in_stewart_plotkin/main.pdf}
    \caption{\(\text{SSError}\) and state probabilities for the strategies
        of~\cite{Stewart2012}, ordered both by number of wins and overall score.
        Note that \(P(DC)\) is not shown as it corresponds to the transpose of
        \(P(CD)\). Cooperator and Defector are omitted as they do not visit all
        the states.}
    \label{fig:SSError_overall_in_stewart_plotkin}
\end{figure}

Here, the work of~\cite{Stewart2012} is extended by investigating a tournament
with \input{assets/tex/number_of_full_strategies/main.tex}
strategies.

The results of this analysis are shown in
Figure~\ref{fig:SSError_and_probabilities_in_full}. The top ranking strategies
by number of wins seem to be extortionate (but not against all strategies) and
it can be seen that a small sub group of strategies achieve mutual defection.
All the top ranking strategies according to score achieve mutual cooperation and
do not extort each other, however they
\textbf{do} exhibit extortionate behaviour towards a number of the lower ranking
strategies.

\begin{figure}[!htbp]
    \centering
    \includegraphics[width=.8\textwidth]{./assets/img/SSError_and_probabilities_in_full/main.pdf}
    \caption{\(\text{SSError}\) for the strategies for the full tournament. Only
    strategy interactions for which \(p_4=0\) and \(\chi>1\) are displayed.}
    \label{fig:SSError_and_probabilities_in_full}
\end{figure}

\section{Conclusion}\label{sec:conclusion}

This work defines an approach to measure whether or not a player is playing a
strategy that corresponds to an extortionate strategy as defined
in~\cite{Press2012}: a mathematical model for suspicion. Indeed, all
extortionate strategies have been
 classified as lying on a triangular plane.
This rigorous classification fails to be robust to small measurement error, thus
a statistical approach is proposed.
This is done through a linear algebraic approach for approximating the solution
of a linear system. Using this, a large number of pairwise interactions is
simulated and in fact very few strategies are found to act extortionately.

The work of~\cite{Press2012}, whilst showing that a clever approach to taking
advantage of another memory one strategy exists: this is incomplete. Whilst the
elegance of this result is very attractive, just as the simplicity of the
victory of Tit For Tat in Axelrod's original tournaments was, it is incomplete.
Extortionate strategies achieve a high number of wins but they do not
achieve a high score which corresponds to the fitness landscape in an
evolutionary sense. From the large number of interactions a payoff matrix \(S\)
can be measured where \(S_{ij}\) denotes the score (using standard values of
\((R, S, T, P) = (3, 0, 5, 1)\)) of the \(i\)th strategy
against the \(j\)th strategy. Using this, the replicator equation
describes the evolution of the system based on a population density fitness
function:

\begin{equation}\label{eqn:replicator_dynamics}
    \frac{dx}{dt} = x(S-x^TS x)
\end{equation}

Equation (\ref{eqn:replicator_dynamics}) is solved numerically through an
integration technique described in~\cite{Petzold1983} and
Figure~\ref{fig:replicator_dynamics} shows the evolution of the distribution of
the system: the various strategies are ranked by scores. It is clear to see that
only the high ranking strategies survive the evolutionary process (in fact,
only \input{./assets/img/replicator_dynamics/main.tex}
have a final distribution greater than \(10 ^ {-2}\)). This confirms the
findings of~\cite{Moran1707} in which sophisticated strategies resist
evolutionary invasion of shorter memory strategies. Recalling
Figure~\ref{fig:SSError_and_probabilities_in_full} this demonstrates that:

\begin{itemize}
    \item Cooperation emerges through the evolutionary process: the high scoring
        strategies do not exhibit extortionate behaviour towards each other.
    \item Extortionate strategies do not survive the evolutionary process.
\end{itemize}

\begin{figure}[!htbp]
    \centering
    \includegraphics[width=.8\textwidth]{./assets/img/replicator_dynamics/main.pdf}
    \caption{Numerical simulation of the replicator equation
    (\ref{eqn:replicator_dynamics}): strategies are ordered by score, only the strategies with a high score survive the evolutionary process.}
    \label{fig:replicator_dynamics}
\end{figure}

This work can be used to classify plays of the IPD\@: data can be collected from
actual interactions (in lab or in the field). Furthermore, this allows for a
classification method similar to the notion of fingerprinting presented
in~\cite{Ashlock2008}. Trained strategies can potentially be classified as
extortionate or not or it could be possible to even constrain the reinforcement
learning approaches that are becoming prevalent in the literature.
Alternatively, this mathematical approach for recognising extortion could be
used in sophisticated strategies to defend against invasion. Arguably, some of
the strategies considered here exhibit this behaviour, indeed as described
in~\cite{Harper2017}, the top ranking strategies in the full tournament are
obtained using evolutionary reinforcement learning techniques, thus, suspicion
of extortionate behaviour could in fact be an evolutionary trait.

\section*{Acknowledgements}

The following open source software libraries were used in this research:

\begin{itemize}
    \item The Axelrod ~\cite{Knight2016, Knight2018} library (IPD strategies and
        tournaments).
    \item The sympy library~\cite{Meurer2017} (verification of all symbolic
        calculations).
    \item The matplotlib~\cite{Droettboom2018} library (visualisation).
    \item The pandas~\cite{Structures2010}, dask~\cite{Dask2016} and
        NumPy~\cite{Oliphant2015} libraries (data manipulation).
    \item The SciPy~\cite{Jones2001} library (numerical integration of the
        replicator equation).
\end{itemize}

This work was performed using the computational facilities of the Advanced
Research Computing @ Cardiff (ARCCA) Division, Cardiff University.

\printbibliography

\newpage
\section*{Supplementary materials}

\includepdf{assets/pdf/proof_of_form_of_extortionate_strategies/main.pdf}

\newpage

Using the pair wise interactions the transition rates \(p,
q\) can be measured and the steady state probabilities inferred and compared to
the actual probabilities of each state.
This is done numerically by computing the singular eigenvector of the
matrix \(A\) \cite{Stewart2009}:

\[
    A =
    \begin{bmatrix}
        p_1 q_1 & p_1 (1 - q_1) & (1 - p_1) q_1 & (1 -p_1) (1 - q_1) \\
        p_2 q_2 & p_2 (1 - q_2) & (1 - p_2) q_2 & (1 -p_2) (1 - q_2) \\
        p_3 q_3 & p_3 (1 - q_3) & (1 - p_3) q_3 & (1 -p_3) (1 - q_3) \\
        p_4 q_4 & p_4 (1 - q_4) & (1 - p_4) q_4 & (1 -p_4) (1 - q_4) \\
    \end{bmatrix}
\]

Figure~\ref{fig:computed_probabilities_vs_theoretic_probabilities} shows a
regression line fitted to every pairwise interaction with a reported
\(\text{SSError}\) value (pairwise interactions with missing states were
omitted). This serves to validate the approach: a part from some edge cases the
relationship is consistent.

\begin{figure}[!htbp]
    \centering
    \includegraphics[width=.8\textwidth]{./assets/img/computed_probabilities_vs_theoretic_probabilities/main.pdf}
    \caption{The
        relationship between the steady state probabilities inferred from the
        measured transitions and the actual steady state probabilities. A linear
        regression line is included validating the approach.}
    \label{fig:computed_probabilities_vs_theoretic_probabilities}
\end{figure}


\end{document}
 strategies,
was presented with specific consideration given to ZD strategies. This
tournament is reproduced here using the Axelrod-Python
project~\cite{Knight2016}. To obtain a good measure of the corresponding
transition rates for each strategy all matches have been run for
\documentclass[a4paper]{article}

\usepackage{amsmath}
\usepackage{amssymb}
\usepackage[margin=1.5cm,
            includefoot,
            footskip=30pt]{geometry}
\usepackage{layout}
\usepackage{graphicx}
\usepackage{subcaption}

\usepackage{biblatex}
\usepackage{pdfpages}

\bibliography{main.bib}

\title{Suspicion: Recognising and evaluating the effectiveness
       of extortion in the Iterated Prisoner's Dilemma}
\author{Vincent A. Knight \and Nikoleta E. Glynatsi}
\date{\today}



\begin{document}

\maketitle

\begin{abstract}
    The Iterated Prisoner's Dilemma is a model for rational and evolutionary
    interactive behaviour. It has applications both in the study of human social
    behaviour as well as in biology.
    It is used to understand when and how a rational individual might
    accept an immediate cost to their own utility for the direct benefit of
    another.

    Much attention has been given to a class of strategies called
    Zero Determinant strategies. It has been theoretically shown that these
    strategies can ``extort'' any player.

    In this work, an approach to identify if observed strategies are playing in
    an extortionate way is described. Furthermore, experimental analysis of
    a large tournament with \input{assets/tex/number_of_full_strategies/main.tex}
    strategies is considered. In this setting
    the most highly performing strategies do not play in an extortionate way
    against each other but do against lower performing strategies.
    This suggests that whilst the theory of Zero Determinant strategies
    indicates that memory is not of fundamental importance to the evolution of
    cooperative behaviour, this is incomplete.
\end{abstract}

\section{Introduction}\label{sec:introduction}

Agent based game theoretic models have become a stalwart of the underpinning
mathematics of interactive behaviours. One of the major pieces of work
in this area is the pair of original computer tournaments run by Robert
Axelrod~\cite{Axelrod1980, Axelrod1980a}. These tournaments pitted submitted
computer strategies against each other in plays of the Iterated Prisoner's
Dilemma. A common game where agents can choose to pay a slight cost to their
immediate utility in the hope of building a reputation. This has been used in
economic and evolutionary game theory to understand the evolution of cooperative
behaviour.

Recently, a class of strategies was described in~\cite{Press2012} that can
provably extort any given opponent. In~\cite{Hilbe2013, Moran1707} some
questions have already been asked about the true effectiveness of these
strategies in an evolutionary setting. Here another question is asked: is it
possible to recognise this extortionate behaviour? A mathematical procedure for
suspicion is presented: in the same way that the continued actions of an
extortionate individual might raise suspicion.

This work makes use of the Axelrod Python library~\cite{Knight2018, Knight2016}
with a large number of Prisoner Dilemma strategies available to give an
extensive numerical example of the ideas presented.  The approach is presented
in Section~\ref{sec:delta-zd-strategies}.  All of the code and data discussed
in Section~\ref{sec:numerical-experiments} is open sourced, archived and
written according to best scientific principles~\cite{Wilson2014}. The data
archive can be found at~\cite{vincent_knight_2018_1297075}.

\section{Recognising Extortion}\label{sec:delta-zd-strategies}

In~\cite{Press2012}, given a match between 2 memory-one strategies, the concept
of Zero Determinant (ZD) strategies is introduced. The main result of that paper
shows that given two memory one players \(p, q\in\mathbb{R}^4\) a linear
relationship between the players' scores could be forced by one of the players.

Using the notation of~\cite{Press2012}, assuming the utilities for player \(p\)
are given by \(S_x=(R, S, T, P)\) and for player \(q\) by \(S_y=(R, T, S, P)\)
and that the stationary scores of each player is given by \(S_X\) and \(S_Y\)
respectively. The main result of~\cite{Press2012} is that if

\begin{equation}\label{eqn:linear_relationship_for_p}
    \tilde p=\alpha S_x + \beta S_y + \gamma
\end{equation}

or

\begin{equation}\label{eqn:linear_relationship_for_q}
    \tilde q=\alpha S_x + \beta S_y + \gamma
\end{equation}

where \(\tilde p = (1 - p_1, 1 - p_2, p_3, p_4)\) and
\(\tilde q = (1 - q_1, 1 - q_2, q_3, q_4)\) then:

\begin{equation}
    \alpha S_X + \beta S_Y + \gamma = 0
\end{equation}

In~\cite{Press2012} a particular type of ZD strategy is defined: extortionate
strategies. If:

\begin{equation}\label{eqn:constraint_for_extortion}
    \gamma = - P(\alpha + \beta)
\end{equation}

then the player can ensure they get a score \(\chi\) times
larger than the opponent. This extortion coefficient is given by:

\begin{equation}\label{eqn:definition_of_chi}
    \chi=\frac{-\beta}{\alpha}
\end{equation}

Thus, if (\ref{eqn:constraint_for_extortion}) holds and \(\chi >1\) a player is
said to extort their opponent.
Here, the reverse problem is considered: given a
\(p\in\mathbb{R}^4\) how does one identify \(\alpha, \beta\) if they
exist and is the strategy in fact acting in an extortionate way?

These conditions correspond to:

\begin{align}
    \tilde p_1 & = \alpha R + \beta R - P (\alpha + \beta)
            \label{eqn:condition_for_tilde_p1}\\
    \tilde p_2 & = \alpha S + \beta T - P (\alpha + \beta)
            \label{eqn:condition_for_tilde_p2}\\
    \tilde p_3 & = \alpha T + \beta S - P (\alpha + \beta)
            \label{eqn:condition_for_tilde_p3}\\
    \tilde p_4 & = \alpha P + \beta P - P (\alpha + \beta)
            \label{eqn:condition_for_tilde_p4}
\end{align}

Equation (\ref{eqn:condition_for_tilde_p4}) ensures that \(p_4=\tilde p_4=0\).
Equations (\ref{eqn:condition_for_tilde_p1}-\ref{eqn:condition_for_tilde_p3})
can be used to eliminate \(\alpha, \beta\), giving:

\begin{equation}\label{eqn:planar_definition_of_extortion}
    \tilde p_1 = \frac{(R - P)(\tilde p_2 + \tilde p_3)}{S + T - 2P}
\end{equation}

with:

\begin{equation}\label{eqn:definition_of_chi}
    \chi = \frac{\tilde p_2 (P - T) + \tilde p_3 (S - P)}
                {\tilde p_2 (P - S) + \tilde p_3 (T - P)}
\end{equation}

Given a strategy \(p\in\mathbb{R}^{4\times 1}\) equations
(\ref{eqn:condition_for_tilde_p4}), (\ref{eqn:planar_definition_of_extortion}-\ref{eqn:definition_of_chi}) can be used to check if
a strategy is extortionate. The conditions correspond to:

\begin{align}
    p_1 & = \frac{(R-P)(p_2 + p_3) - R + T + S - P}{S + T - 2P}
     \label{eqn:condition_for_p1}\\
    p_4 & = 0 \label{eqn:condition_for_p4}\\
    1 & > p_2 + p_3\label{eqn:condition_for_chi}
\end{align}

The algebraic steps necessary to prove these results are available in the
supporting materials.

All extortionate strategies reside on a triangular (\ref{eqn:condition_for_chi})
plane (\ref{eqn:condition_for_p1}) in 3 dimensions (\ref{eqn:condition_for_p4}).
Using this formulation it can be seen that a necessary (but not sufficient)
condition for an extortionate strategy is that it cooperates on average less
than 50\% of the time when in a state of disagreement with the opponent.

As an example, consider the known extortionate strategy \(p=(8 / 9, 1 / 2, 1 /
3, 0)\) from~\cite{Stewart2012} which is referred to as \texttt{Extort-2}. In
this case, for the standard values of \((R, T, S, P)\) constraint
(\ref{eqn:condition_for_p1}) corresponds to:

\begin{equation}
    p_1 = \frac{2(p_2 + p_3) + 1}{3}
\end{equation}

It is clear that in this case all constraints hold.

This approach could in fact be used to confirm that a given strategy is acting
in an extortionate manner even if it is not a memory one strategy. However, in
practice, if a closed form for \(p\) is not known, then due to measurement
and/or numerical error this would not work.

This problem can be written in the following linear algebraic form where
\(x=(\alpha, \beta)\)
and \(p^*=(\tilde p_1 - 1, tilde_2 - 1, p_3)\):

\begin{equation}\label{eqn:linear_algebraic_equation_for_p}
    Cx= p^*
\end{equation}

\(C\) corresponds to equations
(\ref{eqn:condition_for_tilde_p1}-\ref{eqn:condition_for_tilde_p3}) and is
given by:

\begin{equation}\label{eqn:definition_of_C}
    C =
    \begin{bmatrix}
        R - P & R- P \\
        S - P & T- P \\
        T - P & S- P \\
    \end{bmatrix}
\end{equation}

Note that in general, equation (\ref{eqn:linear_algebraic_equation_for_p}) will
not necessarily have a solution. From the Rouch\'{e}-Capelli theorem if there is
a solution it is unique as \(\text{rank}(C)=2\) which is the dimension of the
variable \(x\). The best fitting \(x\) is found by minimizing:

\begin{equation}\label{eqn:r_squared}
    \text{SSError} = \|C x- p^*\|_2^2 = \sum_{i=1}^{3}\left((C\bar x)_i-p_i^*\right)^2
\end{equation}

Note that \(\text{SSError}\), which is the square of the Frobenius
norm~\cite{Golub2013}, becomes a measure of how close a strategy is to being an
extortionate strategy. Suspicion
of extortion then corresponds to a threshold on \(\text{SSError}\).

By observing interactions (human or otherwise), their memory one representation
can be inferred and this approach can be used to recognise extortionate
behaviour. The notion of comparing theoretic and actual plays of the IPD is not
novel, see for example~\cite{Rand2013}. Immediately it is noted that if the
environment is noisy~\cite{Wu1995} then no strategy can be considered to be
extortionate as \(p_4>0\).

In the next section, this idea will be illustrated by observing the interactions
that take place in a computer based tournament of the IPD\@.

\section{Numerical experiments}\label{sec:numerical-experiments}

In~\cite{Stewart2012} results from a tournament with
\input{./assets/tex/number_of_stewart_plotkin_strategies/main.tex} strategies,
was presented with specific consideration given to ZD strategies. This
tournament is reproduced here using the Axelrod-Python
project~\cite{Knight2016}. To obtain a good measure of the corresponding
transition rates for each strategy all matches have been run for
\input{assets/tex/number_of_turns/main.tex} turns and every match has been
repeated \input{assets/tex/number_of_repetitions/main.tex} times. All of this
interaction data is available at~\cite{vincent_knight_2018_1297075}. A good
match between the inferred Markov chain and the state distribution of the actual
interactions has been verified. Data for this is presented in the supplementary
materials.

Figure~\ref{fig:SSError_overall_in_stewart_plotkin} shows the \(\text{SSError}\)
values for all the strategies in the tournament, as reported
in~\cite{Stewart2012} the extortionate strategy (which has an expected
\(\text{SSError}\) approximately 0) gains a large number of wins.

\begin{figure}[!htbp]
    \centering
    \includegraphics[width=.8\textwidth]{./assets/img/SSError_overall_in_stewart_plotkin/main.pdf}
    \caption{\(\text{SSError}\) and state probabilities for the strategies
        of~\cite{Stewart2012}, ordered both by number of wins and overall score.
        Note that \(P(DC)\) is not shown as it corresponds to the transpose of
        \(P(CD)\). Cooperator and Defector are omitted as they do not visit all
        the states.}
    \label{fig:SSError_overall_in_stewart_plotkin}
\end{figure}

Here, the work of~\cite{Stewart2012} is extended by investigating a tournament
with \input{assets/tex/number_of_full_strategies/main.tex}
strategies.

The results of this analysis are shown in
Figure~\ref{fig:SSError_and_probabilities_in_full}. The top ranking strategies
by number of wins seem to be extortionate (but not against all strategies) and
it can be seen that a small sub group of strategies achieve mutual defection.
All the top ranking strategies according to score achieve mutual cooperation and
do not extort each other, however they
\textbf{do} exhibit extortionate behaviour towards a number of the lower ranking
strategies.

\begin{figure}[!htbp]
    \centering
    \includegraphics[width=.8\textwidth]{./assets/img/SSError_and_probabilities_in_full/main.pdf}
    \caption{\(\text{SSError}\) for the strategies for the full tournament. Only
    strategy interactions for which \(p_4=0\) and \(\chi>1\) are displayed.}
    \label{fig:SSError_and_probabilities_in_full}
\end{figure}

\section{Conclusion}\label{sec:conclusion}

This work defines an approach to measure whether or not a player is playing a
strategy that corresponds to an extortionate strategy as defined
in~\cite{Press2012}: a mathematical model for suspicion. Indeed, all
extortionate strategies have been
 classified as lying on a triangular plane.
This rigorous classification fails to be robust to small measurement error, thus
a statistical approach is proposed.
This is done through a linear algebraic approach for approximating the solution
of a linear system. Using this, a large number of pairwise interactions is
simulated and in fact very few strategies are found to act extortionately.

The work of~\cite{Press2012}, whilst showing that a clever approach to taking
advantage of another memory one strategy exists: this is incomplete. Whilst the
elegance of this result is very attractive, just as the simplicity of the
victory of Tit For Tat in Axelrod's original tournaments was, it is incomplete.
Extortionate strategies achieve a high number of wins but they do not
achieve a high score which corresponds to the fitness landscape in an
evolutionary sense. From the large number of interactions a payoff matrix \(S\)
can be measured where \(S_{ij}\) denotes the score (using standard values of
\((R, S, T, P) = (3, 0, 5, 1)\)) of the \(i\)th strategy
against the \(j\)th strategy. Using this, the replicator equation
describes the evolution of the system based on a population density fitness
function:

\begin{equation}\label{eqn:replicator_dynamics}
    \frac{dx}{dt} = x(S-x^TS x)
\end{equation}

Equation (\ref{eqn:replicator_dynamics}) is solved numerically through an
integration technique described in~\cite{Petzold1983} and
Figure~\ref{fig:replicator_dynamics} shows the evolution of the distribution of
the system: the various strategies are ranked by scores. It is clear to see that
only the high ranking strategies survive the evolutionary process (in fact,
only \input{./assets/img/replicator_dynamics/main.tex}
have a final distribution greater than \(10 ^ {-2}\)). This confirms the
findings of~\cite{Moran1707} in which sophisticated strategies resist
evolutionary invasion of shorter memory strategies. Recalling
Figure~\ref{fig:SSError_and_probabilities_in_full} this demonstrates that:

\begin{itemize}
    \item Cooperation emerges through the evolutionary process: the high scoring
        strategies do not exhibit extortionate behaviour towards each other.
    \item Extortionate strategies do not survive the evolutionary process.
\end{itemize}

\begin{figure}[!htbp]
    \centering
    \includegraphics[width=.8\textwidth]{./assets/img/replicator_dynamics/main.pdf}
    \caption{Numerical simulation of the replicator equation
    (\ref{eqn:replicator_dynamics}): strategies are ordered by score, only the strategies with a high score survive the evolutionary process.}
    \label{fig:replicator_dynamics}
\end{figure}

This work can be used to classify plays of the IPD\@: data can be collected from
actual interactions (in lab or in the field). Furthermore, this allows for a
classification method similar to the notion of fingerprinting presented
in~\cite{Ashlock2008}. Trained strategies can potentially be classified as
extortionate or not or it could be possible to even constrain the reinforcement
learning approaches that are becoming prevalent in the literature.
Alternatively, this mathematical approach for recognising extortion could be
used in sophisticated strategies to defend against invasion. Arguably, some of
the strategies considered here exhibit this behaviour, indeed as described
in~\cite{Harper2017}, the top ranking strategies in the full tournament are
obtained using evolutionary reinforcement learning techniques, thus, suspicion
of extortionate behaviour could in fact be an evolutionary trait.

\section*{Acknowledgements}

The following open source software libraries were used in this research:

\begin{itemize}
    \item The Axelrod ~\cite{Knight2016, Knight2018} library (IPD strategies and
        tournaments).
    \item The sympy library~\cite{Meurer2017} (verification of all symbolic
        calculations).
    \item The matplotlib~\cite{Droettboom2018} library (visualisation).
    \item The pandas~\cite{Structures2010}, dask~\cite{Dask2016} and
        NumPy~\cite{Oliphant2015} libraries (data manipulation).
    \item The SciPy~\cite{Jones2001} library (numerical integration of the
        replicator equation).
\end{itemize}

This work was performed using the computational facilities of the Advanced
Research Computing @ Cardiff (ARCCA) Division, Cardiff University.

\printbibliography

\newpage
\section*{Supplementary materials}

\includepdf{assets/pdf/proof_of_form_of_extortionate_strategies/main.pdf}

\newpage

Using the pair wise interactions the transition rates \(p,
q\) can be measured and the steady state probabilities inferred and compared to
the actual probabilities of each state.
This is done numerically by computing the singular eigenvector of the
matrix \(A\) \cite{Stewart2009}:

\[
    A =
    \begin{bmatrix}
        p_1 q_1 & p_1 (1 - q_1) & (1 - p_1) q_1 & (1 -p_1) (1 - q_1) \\
        p_2 q_2 & p_2 (1 - q_2) & (1 - p_2) q_2 & (1 -p_2) (1 - q_2) \\
        p_3 q_3 & p_3 (1 - q_3) & (1 - p_3) q_3 & (1 -p_3) (1 - q_3) \\
        p_4 q_4 & p_4 (1 - q_4) & (1 - p_4) q_4 & (1 -p_4) (1 - q_4) \\
    \end{bmatrix}
\]

Figure~\ref{fig:computed_probabilities_vs_theoretic_probabilities} shows a
regression line fitted to every pairwise interaction with a reported
\(\text{SSError}\) value (pairwise interactions with missing states were
omitted). This serves to validate the approach: a part from some edge cases the
relationship is consistent.

\begin{figure}[!htbp]
    \centering
    \includegraphics[width=.8\textwidth]{./assets/img/computed_probabilities_vs_theoretic_probabilities/main.pdf}
    \caption{The
        relationship between the steady state probabilities inferred from the
        measured transitions and the actual steady state probabilities. A linear
        regression line is included validating the approach.}
    \label{fig:computed_probabilities_vs_theoretic_probabilities}
\end{figure}


\end{document}
 turns and every match has been
repeated \documentclass[a4paper]{article}

\usepackage{amsmath}
\usepackage{amssymb}
\usepackage[margin=1.5cm,
            includefoot,
            footskip=30pt]{geometry}
\usepackage{layout}
\usepackage{graphicx}
\usepackage{subcaption}

\usepackage{biblatex}
\usepackage{pdfpages}

\bibliography{main.bib}

\title{Suspicion: Recognising and evaluating the effectiveness
       of extortion in the Iterated Prisoner's Dilemma}
\author{Vincent A. Knight \and Nikoleta E. Glynatsi}
\date{\today}



\begin{document}

\maketitle

\begin{abstract}
    The Iterated Prisoner's Dilemma is a model for rational and evolutionary
    interactive behaviour. It has applications both in the study of human social
    behaviour as well as in biology.
    It is used to understand when and how a rational individual might
    accept an immediate cost to their own utility for the direct benefit of
    another.

    Much attention has been given to a class of strategies called
    Zero Determinant strategies. It has been theoretically shown that these
    strategies can ``extort'' any player.

    In this work, an approach to identify if observed strategies are playing in
    an extortionate way is described. Furthermore, experimental analysis of
    a large tournament with \input{assets/tex/number_of_full_strategies/main.tex}
    strategies is considered. In this setting
    the most highly performing strategies do not play in an extortionate way
    against each other but do against lower performing strategies.
    This suggests that whilst the theory of Zero Determinant strategies
    indicates that memory is not of fundamental importance to the evolution of
    cooperative behaviour, this is incomplete.
\end{abstract}

\section{Introduction}\label{sec:introduction}

Agent based game theoretic models have become a stalwart of the underpinning
mathematics of interactive behaviours. One of the major pieces of work
in this area is the pair of original computer tournaments run by Robert
Axelrod~\cite{Axelrod1980, Axelrod1980a}. These tournaments pitted submitted
computer strategies against each other in plays of the Iterated Prisoner's
Dilemma. A common game where agents can choose to pay a slight cost to their
immediate utility in the hope of building a reputation. This has been used in
economic and evolutionary game theory to understand the evolution of cooperative
behaviour.

Recently, a class of strategies was described in~\cite{Press2012} that can
provably extort any given opponent. In~\cite{Hilbe2013, Moran1707} some
questions have already been asked about the true effectiveness of these
strategies in an evolutionary setting. Here another question is asked: is it
possible to recognise this extortionate behaviour? A mathematical procedure for
suspicion is presented: in the same way that the continued actions of an
extortionate individual might raise suspicion.

This work makes use of the Axelrod Python library~\cite{Knight2018, Knight2016}
with a large number of Prisoner Dilemma strategies available to give an
extensive numerical example of the ideas presented.  The approach is presented
in Section~\ref{sec:delta-zd-strategies}.  All of the code and data discussed
in Section~\ref{sec:numerical-experiments} is open sourced, archived and
written according to best scientific principles~\cite{Wilson2014}. The data
archive can be found at~\cite{vincent_knight_2018_1297075}.

\section{Recognising Extortion}\label{sec:delta-zd-strategies}

In~\cite{Press2012}, given a match between 2 memory-one strategies, the concept
of Zero Determinant (ZD) strategies is introduced. The main result of that paper
shows that given two memory one players \(p, q\in\mathbb{R}^4\) a linear
relationship between the players' scores could be forced by one of the players.

Using the notation of~\cite{Press2012}, assuming the utilities for player \(p\)
are given by \(S_x=(R, S, T, P)\) and for player \(q\) by \(S_y=(R, T, S, P)\)
and that the stationary scores of each player is given by \(S_X\) and \(S_Y\)
respectively. The main result of~\cite{Press2012} is that if

\begin{equation}\label{eqn:linear_relationship_for_p}
    \tilde p=\alpha S_x + \beta S_y + \gamma
\end{equation}

or

\begin{equation}\label{eqn:linear_relationship_for_q}
    \tilde q=\alpha S_x + \beta S_y + \gamma
\end{equation}

where \(\tilde p = (1 - p_1, 1 - p_2, p_3, p_4)\) and
\(\tilde q = (1 - q_1, 1 - q_2, q_3, q_4)\) then:

\begin{equation}
    \alpha S_X + \beta S_Y + \gamma = 0
\end{equation}

In~\cite{Press2012} a particular type of ZD strategy is defined: extortionate
strategies. If:

\begin{equation}\label{eqn:constraint_for_extortion}
    \gamma = - P(\alpha + \beta)
\end{equation}

then the player can ensure they get a score \(\chi\) times
larger than the opponent. This extortion coefficient is given by:

\begin{equation}\label{eqn:definition_of_chi}
    \chi=\frac{-\beta}{\alpha}
\end{equation}

Thus, if (\ref{eqn:constraint_for_extortion}) holds and \(\chi >1\) a player is
said to extort their opponent.
Here, the reverse problem is considered: given a
\(p\in\mathbb{R}^4\) how does one identify \(\alpha, \beta\) if they
exist and is the strategy in fact acting in an extortionate way?

These conditions correspond to:

\begin{align}
    \tilde p_1 & = \alpha R + \beta R - P (\alpha + \beta)
            \label{eqn:condition_for_tilde_p1}\\
    \tilde p_2 & = \alpha S + \beta T - P (\alpha + \beta)
            \label{eqn:condition_for_tilde_p2}\\
    \tilde p_3 & = \alpha T + \beta S - P (\alpha + \beta)
            \label{eqn:condition_for_tilde_p3}\\
    \tilde p_4 & = \alpha P + \beta P - P (\alpha + \beta)
            \label{eqn:condition_for_tilde_p4}
\end{align}

Equation (\ref{eqn:condition_for_tilde_p4}) ensures that \(p_4=\tilde p_4=0\).
Equations (\ref{eqn:condition_for_tilde_p1}-\ref{eqn:condition_for_tilde_p3})
can be used to eliminate \(\alpha, \beta\), giving:

\begin{equation}\label{eqn:planar_definition_of_extortion}
    \tilde p_1 = \frac{(R - P)(\tilde p_2 + \tilde p_3)}{S + T - 2P}
\end{equation}

with:

\begin{equation}\label{eqn:definition_of_chi}
    \chi = \frac{\tilde p_2 (P - T) + \tilde p_3 (S - P)}
                {\tilde p_2 (P - S) + \tilde p_3 (T - P)}
\end{equation}

Given a strategy \(p\in\mathbb{R}^{4\times 1}\) equations
(\ref{eqn:condition_for_tilde_p4}), (\ref{eqn:planar_definition_of_extortion}-\ref{eqn:definition_of_chi}) can be used to check if
a strategy is extortionate. The conditions correspond to:

\begin{align}
    p_1 & = \frac{(R-P)(p_2 + p_3) - R + T + S - P}{S + T - 2P}
     \label{eqn:condition_for_p1}\\
    p_4 & = 0 \label{eqn:condition_for_p4}\\
    1 & > p_2 + p_3\label{eqn:condition_for_chi}
\end{align}

The algebraic steps necessary to prove these results are available in the
supporting materials.

All extortionate strategies reside on a triangular (\ref{eqn:condition_for_chi})
plane (\ref{eqn:condition_for_p1}) in 3 dimensions (\ref{eqn:condition_for_p4}).
Using this formulation it can be seen that a necessary (but not sufficient)
condition for an extortionate strategy is that it cooperates on average less
than 50\% of the time when in a state of disagreement with the opponent.

As an example, consider the known extortionate strategy \(p=(8 / 9, 1 / 2, 1 /
3, 0)\) from~\cite{Stewart2012} which is referred to as \texttt{Extort-2}. In
this case, for the standard values of \((R, T, S, P)\) constraint
(\ref{eqn:condition_for_p1}) corresponds to:

\begin{equation}
    p_1 = \frac{2(p_2 + p_3) + 1}{3}
\end{equation}

It is clear that in this case all constraints hold.

This approach could in fact be used to confirm that a given strategy is acting
in an extortionate manner even if it is not a memory one strategy. However, in
practice, if a closed form for \(p\) is not known, then due to measurement
and/or numerical error this would not work.

This problem can be written in the following linear algebraic form where
\(x=(\alpha, \beta)\)
and \(p^*=(\tilde p_1 - 1, tilde_2 - 1, p_3)\):

\begin{equation}\label{eqn:linear_algebraic_equation_for_p}
    Cx= p^*
\end{equation}

\(C\) corresponds to equations
(\ref{eqn:condition_for_tilde_p1}-\ref{eqn:condition_for_tilde_p3}) and is
given by:

\begin{equation}\label{eqn:definition_of_C}
    C =
    \begin{bmatrix}
        R - P & R- P \\
        S - P & T- P \\
        T - P & S- P \\
    \end{bmatrix}
\end{equation}

Note that in general, equation (\ref{eqn:linear_algebraic_equation_for_p}) will
not necessarily have a solution. From the Rouch\'{e}-Capelli theorem if there is
a solution it is unique as \(\text{rank}(C)=2\) which is the dimension of the
variable \(x\). The best fitting \(x\) is found by minimizing:

\begin{equation}\label{eqn:r_squared}
    \text{SSError} = \|C x- p^*\|_2^2 = \sum_{i=1}^{3}\left((C\bar x)_i-p_i^*\right)^2
\end{equation}

Note that \(\text{SSError}\), which is the square of the Frobenius
norm~\cite{Golub2013}, becomes a measure of how close a strategy is to being an
extortionate strategy. Suspicion
of extortion then corresponds to a threshold on \(\text{SSError}\).

By observing interactions (human or otherwise), their memory one representation
can be inferred and this approach can be used to recognise extortionate
behaviour. The notion of comparing theoretic and actual plays of the IPD is not
novel, see for example~\cite{Rand2013}. Immediately it is noted that if the
environment is noisy~\cite{Wu1995} then no strategy can be considered to be
extortionate as \(p_4>0\).

In the next section, this idea will be illustrated by observing the interactions
that take place in a computer based tournament of the IPD\@.

\section{Numerical experiments}\label{sec:numerical-experiments}

In~\cite{Stewart2012} results from a tournament with
\input{./assets/tex/number_of_stewart_plotkin_strategies/main.tex} strategies,
was presented with specific consideration given to ZD strategies. This
tournament is reproduced here using the Axelrod-Python
project~\cite{Knight2016}. To obtain a good measure of the corresponding
transition rates for each strategy all matches have been run for
\input{assets/tex/number_of_turns/main.tex} turns and every match has been
repeated \input{assets/tex/number_of_repetitions/main.tex} times. All of this
interaction data is available at~\cite{vincent_knight_2018_1297075}. A good
match between the inferred Markov chain and the state distribution of the actual
interactions has been verified. Data for this is presented in the supplementary
materials.

Figure~\ref{fig:SSError_overall_in_stewart_plotkin} shows the \(\text{SSError}\)
values for all the strategies in the tournament, as reported
in~\cite{Stewart2012} the extortionate strategy (which has an expected
\(\text{SSError}\) approximately 0) gains a large number of wins.

\begin{figure}[!htbp]
    \centering
    \includegraphics[width=.8\textwidth]{./assets/img/SSError_overall_in_stewart_plotkin/main.pdf}
    \caption{\(\text{SSError}\) and state probabilities for the strategies
        of~\cite{Stewart2012}, ordered both by number of wins and overall score.
        Note that \(P(DC)\) is not shown as it corresponds to the transpose of
        \(P(CD)\). Cooperator and Defector are omitted as they do not visit all
        the states.}
    \label{fig:SSError_overall_in_stewart_plotkin}
\end{figure}

Here, the work of~\cite{Stewart2012} is extended by investigating a tournament
with \input{assets/tex/number_of_full_strategies/main.tex}
strategies.

The results of this analysis are shown in
Figure~\ref{fig:SSError_and_probabilities_in_full}. The top ranking strategies
by number of wins seem to be extortionate (but not against all strategies) and
it can be seen that a small sub group of strategies achieve mutual defection.
All the top ranking strategies according to score achieve mutual cooperation and
do not extort each other, however they
\textbf{do} exhibit extortionate behaviour towards a number of the lower ranking
strategies.

\begin{figure}[!htbp]
    \centering
    \includegraphics[width=.8\textwidth]{./assets/img/SSError_and_probabilities_in_full/main.pdf}
    \caption{\(\text{SSError}\) for the strategies for the full tournament. Only
    strategy interactions for which \(p_4=0\) and \(\chi>1\) are displayed.}
    \label{fig:SSError_and_probabilities_in_full}
\end{figure}

\section{Conclusion}\label{sec:conclusion}

This work defines an approach to measure whether or not a player is playing a
strategy that corresponds to an extortionate strategy as defined
in~\cite{Press2012}: a mathematical model for suspicion. Indeed, all
extortionate strategies have been
 classified as lying on a triangular plane.
This rigorous classification fails to be robust to small measurement error, thus
a statistical approach is proposed.
This is done through a linear algebraic approach for approximating the solution
of a linear system. Using this, a large number of pairwise interactions is
simulated and in fact very few strategies are found to act extortionately.

The work of~\cite{Press2012}, whilst showing that a clever approach to taking
advantage of another memory one strategy exists: this is incomplete. Whilst the
elegance of this result is very attractive, just as the simplicity of the
victory of Tit For Tat in Axelrod's original tournaments was, it is incomplete.
Extortionate strategies achieve a high number of wins but they do not
achieve a high score which corresponds to the fitness landscape in an
evolutionary sense. From the large number of interactions a payoff matrix \(S\)
can be measured where \(S_{ij}\) denotes the score (using standard values of
\((R, S, T, P) = (3, 0, 5, 1)\)) of the \(i\)th strategy
against the \(j\)th strategy. Using this, the replicator equation
describes the evolution of the system based on a population density fitness
function:

\begin{equation}\label{eqn:replicator_dynamics}
    \frac{dx}{dt} = x(S-x^TS x)
\end{equation}

Equation (\ref{eqn:replicator_dynamics}) is solved numerically through an
integration technique described in~\cite{Petzold1983} and
Figure~\ref{fig:replicator_dynamics} shows the evolution of the distribution of
the system: the various strategies are ranked by scores. It is clear to see that
only the high ranking strategies survive the evolutionary process (in fact,
only \input{./assets/img/replicator_dynamics/main.tex}
have a final distribution greater than \(10 ^ {-2}\)). This confirms the
findings of~\cite{Moran1707} in which sophisticated strategies resist
evolutionary invasion of shorter memory strategies. Recalling
Figure~\ref{fig:SSError_and_probabilities_in_full} this demonstrates that:

\begin{itemize}
    \item Cooperation emerges through the evolutionary process: the high scoring
        strategies do not exhibit extortionate behaviour towards each other.
    \item Extortionate strategies do not survive the evolutionary process.
\end{itemize}

\begin{figure}[!htbp]
    \centering
    \includegraphics[width=.8\textwidth]{./assets/img/replicator_dynamics/main.pdf}
    \caption{Numerical simulation of the replicator equation
    (\ref{eqn:replicator_dynamics}): strategies are ordered by score, only the strategies with a high score survive the evolutionary process.}
    \label{fig:replicator_dynamics}
\end{figure}

This work can be used to classify plays of the IPD\@: data can be collected from
actual interactions (in lab or in the field). Furthermore, this allows for a
classification method similar to the notion of fingerprinting presented
in~\cite{Ashlock2008}. Trained strategies can potentially be classified as
extortionate or not or it could be possible to even constrain the reinforcement
learning approaches that are becoming prevalent in the literature.
Alternatively, this mathematical approach for recognising extortion could be
used in sophisticated strategies to defend against invasion. Arguably, some of
the strategies considered here exhibit this behaviour, indeed as described
in~\cite{Harper2017}, the top ranking strategies in the full tournament are
obtained using evolutionary reinforcement learning techniques, thus, suspicion
of extortionate behaviour could in fact be an evolutionary trait.

\section*{Acknowledgements}

The following open source software libraries were used in this research:

\begin{itemize}
    \item The Axelrod ~\cite{Knight2016, Knight2018} library (IPD strategies and
        tournaments).
    \item The sympy library~\cite{Meurer2017} (verification of all symbolic
        calculations).
    \item The matplotlib~\cite{Droettboom2018} library (visualisation).
    \item The pandas~\cite{Structures2010}, dask~\cite{Dask2016} and
        NumPy~\cite{Oliphant2015} libraries (data manipulation).
    \item The SciPy~\cite{Jones2001} library (numerical integration of the
        replicator equation).
\end{itemize}

This work was performed using the computational facilities of the Advanced
Research Computing @ Cardiff (ARCCA) Division, Cardiff University.

\printbibliography

\newpage
\section*{Supplementary materials}

\includepdf{assets/pdf/proof_of_form_of_extortionate_strategies/main.pdf}

\newpage

Using the pair wise interactions the transition rates \(p,
q\) can be measured and the steady state probabilities inferred and compared to
the actual probabilities of each state.
This is done numerically by computing the singular eigenvector of the
matrix \(A\) \cite{Stewart2009}:

\[
    A =
    \begin{bmatrix}
        p_1 q_1 & p_1 (1 - q_1) & (1 - p_1) q_1 & (1 -p_1) (1 - q_1) \\
        p_2 q_2 & p_2 (1 - q_2) & (1 - p_2) q_2 & (1 -p_2) (1 - q_2) \\
        p_3 q_3 & p_3 (1 - q_3) & (1 - p_3) q_3 & (1 -p_3) (1 - q_3) \\
        p_4 q_4 & p_4 (1 - q_4) & (1 - p_4) q_4 & (1 -p_4) (1 - q_4) \\
    \end{bmatrix}
\]

Figure~\ref{fig:computed_probabilities_vs_theoretic_probabilities} shows a
regression line fitted to every pairwise interaction with a reported
\(\text{SSError}\) value (pairwise interactions with missing states were
omitted). This serves to validate the approach: a part from some edge cases the
relationship is consistent.

\begin{figure}[!htbp]
    \centering
    \includegraphics[width=.8\textwidth]{./assets/img/computed_probabilities_vs_theoretic_probabilities/main.pdf}
    \caption{The
        relationship between the steady state probabilities inferred from the
        measured transitions and the actual steady state probabilities. A linear
        regression line is included validating the approach.}
    \label{fig:computed_probabilities_vs_theoretic_probabilities}
\end{figure}


\end{document}
 times. All of this
interaction data is available at~\cite{vincent_knight_2018_1297075}. A good
match between the inferred Markov chain and the state distribution of the actual
interactions has been verified. Data for this is presented in the supplementary
materials.

Figure~\ref{fig:SSError_overall_in_stewart_plotkin} shows the \(\text{SSError}\)
values for all the strategies in the tournament, as reported
in~\cite{Stewart2012} the extortionate strategy (which has an expected
\(\text{SSError}\) approximately 0) gains a large number of wins.

\begin{figure}[!htbp]
    \centering
    \includegraphics[width=.8\textwidth]{./assets/img/SSError_overall_in_stewart_plotkin/main.pdf}
    \caption{\(\text{SSError}\) and state probabilities for the strategies
        of~\cite{Stewart2012}, ordered both by number of wins and overall score.
        Note that \(P(DC)\) is not shown as it corresponds to the transpose of
        \(P(CD)\). Cooperator and Defector are omitted as they do not visit all
        the states.}
    \label{fig:SSError_overall_in_stewart_plotkin}
\end{figure}

Here, the work of~\cite{Stewart2012} is extended by investigating a tournament
with \documentclass[a4paper]{article}

\usepackage{amsmath}
\usepackage{amssymb}
\usepackage[margin=1.5cm,
            includefoot,
            footskip=30pt]{geometry}
\usepackage{layout}
\usepackage{graphicx}
\usepackage{subcaption}

\usepackage{biblatex}
\usepackage{pdfpages}

\bibliography{main.bib}

\title{Suspicion: Recognising and evaluating the effectiveness
       of extortion in the Iterated Prisoner's Dilemma}
\author{Vincent A. Knight \and Nikoleta E. Glynatsi}
\date{\today}



\begin{document}

\maketitle

\begin{abstract}
    The Iterated Prisoner's Dilemma is a model for rational and evolutionary
    interactive behaviour. It has applications both in the study of human social
    behaviour as well as in biology.
    It is used to understand when and how a rational individual might
    accept an immediate cost to their own utility for the direct benefit of
    another.

    Much attention has been given to a class of strategies called
    Zero Determinant strategies. It has been theoretically shown that these
    strategies can ``extort'' any player.

    In this work, an approach to identify if observed strategies are playing in
    an extortionate way is described. Furthermore, experimental analysis of
    a large tournament with \input{assets/tex/number_of_full_strategies/main.tex}
    strategies is considered. In this setting
    the most highly performing strategies do not play in an extortionate way
    against each other but do against lower performing strategies.
    This suggests that whilst the theory of Zero Determinant strategies
    indicates that memory is not of fundamental importance to the evolution of
    cooperative behaviour, this is incomplete.
\end{abstract}

\section{Introduction}\label{sec:introduction}

Agent based game theoretic models have become a stalwart of the underpinning
mathematics of interactive behaviours. One of the major pieces of work
in this area is the pair of original computer tournaments run by Robert
Axelrod~\cite{Axelrod1980, Axelrod1980a}. These tournaments pitted submitted
computer strategies against each other in plays of the Iterated Prisoner's
Dilemma. A common game where agents can choose to pay a slight cost to their
immediate utility in the hope of building a reputation. This has been used in
economic and evolutionary game theory to understand the evolution of cooperative
behaviour.

Recently, a class of strategies was described in~\cite{Press2012} that can
provably extort any given opponent. In~\cite{Hilbe2013, Moran1707} some
questions have already been asked about the true effectiveness of these
strategies in an evolutionary setting. Here another question is asked: is it
possible to recognise this extortionate behaviour? A mathematical procedure for
suspicion is presented: in the same way that the continued actions of an
extortionate individual might raise suspicion.

This work makes use of the Axelrod Python library~\cite{Knight2018, Knight2016}
with a large number of Prisoner Dilemma strategies available to give an
extensive numerical example of the ideas presented.  The approach is presented
in Section~\ref{sec:delta-zd-strategies}.  All of the code and data discussed
in Section~\ref{sec:numerical-experiments} is open sourced, archived and
written according to best scientific principles~\cite{Wilson2014}. The data
archive can be found at~\cite{vincent_knight_2018_1297075}.

\section{Recognising Extortion}\label{sec:delta-zd-strategies}

In~\cite{Press2012}, given a match between 2 memory-one strategies, the concept
of Zero Determinant (ZD) strategies is introduced. The main result of that paper
shows that given two memory one players \(p, q\in\mathbb{R}^4\) a linear
relationship between the players' scores could be forced by one of the players.

Using the notation of~\cite{Press2012}, assuming the utilities for player \(p\)
are given by \(S_x=(R, S, T, P)\) and for player \(q\) by \(S_y=(R, T, S, P)\)
and that the stationary scores of each player is given by \(S_X\) and \(S_Y\)
respectively. The main result of~\cite{Press2012} is that if

\begin{equation}\label{eqn:linear_relationship_for_p}
    \tilde p=\alpha S_x + \beta S_y + \gamma
\end{equation}

or

\begin{equation}\label{eqn:linear_relationship_for_q}
    \tilde q=\alpha S_x + \beta S_y + \gamma
\end{equation}

where \(\tilde p = (1 - p_1, 1 - p_2, p_3, p_4)\) and
\(\tilde q = (1 - q_1, 1 - q_2, q_3, q_4)\) then:

\begin{equation}
    \alpha S_X + \beta S_Y + \gamma = 0
\end{equation}

In~\cite{Press2012} a particular type of ZD strategy is defined: extortionate
strategies. If:

\begin{equation}\label{eqn:constraint_for_extortion}
    \gamma = - P(\alpha + \beta)
\end{equation}

then the player can ensure they get a score \(\chi\) times
larger than the opponent. This extortion coefficient is given by:

\begin{equation}\label{eqn:definition_of_chi}
    \chi=\frac{-\beta}{\alpha}
\end{equation}

Thus, if (\ref{eqn:constraint_for_extortion}) holds and \(\chi >1\) a player is
said to extort their opponent.
Here, the reverse problem is considered: given a
\(p\in\mathbb{R}^4\) how does one identify \(\alpha, \beta\) if they
exist and is the strategy in fact acting in an extortionate way?

These conditions correspond to:

\begin{align}
    \tilde p_1 & = \alpha R + \beta R - P (\alpha + \beta)
            \label{eqn:condition_for_tilde_p1}\\
    \tilde p_2 & = \alpha S + \beta T - P (\alpha + \beta)
            \label{eqn:condition_for_tilde_p2}\\
    \tilde p_3 & = \alpha T + \beta S - P (\alpha + \beta)
            \label{eqn:condition_for_tilde_p3}\\
    \tilde p_4 & = \alpha P + \beta P - P (\alpha + \beta)
            \label{eqn:condition_for_tilde_p4}
\end{align}

Equation (\ref{eqn:condition_for_tilde_p4}) ensures that \(p_4=\tilde p_4=0\).
Equations (\ref{eqn:condition_for_tilde_p1}-\ref{eqn:condition_for_tilde_p3})
can be used to eliminate \(\alpha, \beta\), giving:

\begin{equation}\label{eqn:planar_definition_of_extortion}
    \tilde p_1 = \frac{(R - P)(\tilde p_2 + \tilde p_3)}{S + T - 2P}
\end{equation}

with:

\begin{equation}\label{eqn:definition_of_chi}
    \chi = \frac{\tilde p_2 (P - T) + \tilde p_3 (S - P)}
                {\tilde p_2 (P - S) + \tilde p_3 (T - P)}
\end{equation}

Given a strategy \(p\in\mathbb{R}^{4\times 1}\) equations
(\ref{eqn:condition_for_tilde_p4}), (\ref{eqn:planar_definition_of_extortion}-\ref{eqn:definition_of_chi}) can be used to check if
a strategy is extortionate. The conditions correspond to:

\begin{align}
    p_1 & = \frac{(R-P)(p_2 + p_3) - R + T + S - P}{S + T - 2P}
     \label{eqn:condition_for_p1}\\
    p_4 & = 0 \label{eqn:condition_for_p4}\\
    1 & > p_2 + p_3\label{eqn:condition_for_chi}
\end{align}

The algebraic steps necessary to prove these results are available in the
supporting materials.

All extortionate strategies reside on a triangular (\ref{eqn:condition_for_chi})
plane (\ref{eqn:condition_for_p1}) in 3 dimensions (\ref{eqn:condition_for_p4}).
Using this formulation it can be seen that a necessary (but not sufficient)
condition for an extortionate strategy is that it cooperates on average less
than 50\% of the time when in a state of disagreement with the opponent.

As an example, consider the known extortionate strategy \(p=(8 / 9, 1 / 2, 1 /
3, 0)\) from~\cite{Stewart2012} which is referred to as \texttt{Extort-2}. In
this case, for the standard values of \((R, T, S, P)\) constraint
(\ref{eqn:condition_for_p1}) corresponds to:

\begin{equation}
    p_1 = \frac{2(p_2 + p_3) + 1}{3}
\end{equation}

It is clear that in this case all constraints hold.

This approach could in fact be used to confirm that a given strategy is acting
in an extortionate manner even if it is not a memory one strategy. However, in
practice, if a closed form for \(p\) is not known, then due to measurement
and/or numerical error this would not work.

This problem can be written in the following linear algebraic form where
\(x=(\alpha, \beta)\)
and \(p^*=(\tilde p_1 - 1, tilde_2 - 1, p_3)\):

\begin{equation}\label{eqn:linear_algebraic_equation_for_p}
    Cx= p^*
\end{equation}

\(C\) corresponds to equations
(\ref{eqn:condition_for_tilde_p1}-\ref{eqn:condition_for_tilde_p3}) and is
given by:

\begin{equation}\label{eqn:definition_of_C}
    C =
    \begin{bmatrix}
        R - P & R- P \\
        S - P & T- P \\
        T - P & S- P \\
    \end{bmatrix}
\end{equation}

Note that in general, equation (\ref{eqn:linear_algebraic_equation_for_p}) will
not necessarily have a solution. From the Rouch\'{e}-Capelli theorem if there is
a solution it is unique as \(\text{rank}(C)=2\) which is the dimension of the
variable \(x\). The best fitting \(x\) is found by minimizing:

\begin{equation}\label{eqn:r_squared}
    \text{SSError} = \|C x- p^*\|_2^2 = \sum_{i=1}^{3}\left((C\bar x)_i-p_i^*\right)^2
\end{equation}

Note that \(\text{SSError}\), which is the square of the Frobenius
norm~\cite{Golub2013}, becomes a measure of how close a strategy is to being an
extortionate strategy. Suspicion
of extortion then corresponds to a threshold on \(\text{SSError}\).

By observing interactions (human or otherwise), their memory one representation
can be inferred and this approach can be used to recognise extortionate
behaviour. The notion of comparing theoretic and actual plays of the IPD is not
novel, see for example~\cite{Rand2013}. Immediately it is noted that if the
environment is noisy~\cite{Wu1995} then no strategy can be considered to be
extortionate as \(p_4>0\).

In the next section, this idea will be illustrated by observing the interactions
that take place in a computer based tournament of the IPD\@.

\section{Numerical experiments}\label{sec:numerical-experiments}

In~\cite{Stewart2012} results from a tournament with
\input{./assets/tex/number_of_stewart_plotkin_strategies/main.tex} strategies,
was presented with specific consideration given to ZD strategies. This
tournament is reproduced here using the Axelrod-Python
project~\cite{Knight2016}. To obtain a good measure of the corresponding
transition rates for each strategy all matches have been run for
\input{assets/tex/number_of_turns/main.tex} turns and every match has been
repeated \input{assets/tex/number_of_repetitions/main.tex} times. All of this
interaction data is available at~\cite{vincent_knight_2018_1297075}. A good
match between the inferred Markov chain and the state distribution of the actual
interactions has been verified. Data for this is presented in the supplementary
materials.

Figure~\ref{fig:SSError_overall_in_stewart_plotkin} shows the \(\text{SSError}\)
values for all the strategies in the tournament, as reported
in~\cite{Stewart2012} the extortionate strategy (which has an expected
\(\text{SSError}\) approximately 0) gains a large number of wins.

\begin{figure}[!htbp]
    \centering
    \includegraphics[width=.8\textwidth]{./assets/img/SSError_overall_in_stewart_plotkin/main.pdf}
    \caption{\(\text{SSError}\) and state probabilities for the strategies
        of~\cite{Stewart2012}, ordered both by number of wins and overall score.
        Note that \(P(DC)\) is not shown as it corresponds to the transpose of
        \(P(CD)\). Cooperator and Defector are omitted as they do not visit all
        the states.}
    \label{fig:SSError_overall_in_stewart_plotkin}
\end{figure}

Here, the work of~\cite{Stewart2012} is extended by investigating a tournament
with \input{assets/tex/number_of_full_strategies/main.tex}
strategies.

The results of this analysis are shown in
Figure~\ref{fig:SSError_and_probabilities_in_full}. The top ranking strategies
by number of wins seem to be extortionate (but not against all strategies) and
it can be seen that a small sub group of strategies achieve mutual defection.
All the top ranking strategies according to score achieve mutual cooperation and
do not extort each other, however they
\textbf{do} exhibit extortionate behaviour towards a number of the lower ranking
strategies.

\begin{figure}[!htbp]
    \centering
    \includegraphics[width=.8\textwidth]{./assets/img/SSError_and_probabilities_in_full/main.pdf}
    \caption{\(\text{SSError}\) for the strategies for the full tournament. Only
    strategy interactions for which \(p_4=0\) and \(\chi>1\) are displayed.}
    \label{fig:SSError_and_probabilities_in_full}
\end{figure}

\section{Conclusion}\label{sec:conclusion}

This work defines an approach to measure whether or not a player is playing a
strategy that corresponds to an extortionate strategy as defined
in~\cite{Press2012}: a mathematical model for suspicion. Indeed, all
extortionate strategies have been
 classified as lying on a triangular plane.
This rigorous classification fails to be robust to small measurement error, thus
a statistical approach is proposed.
This is done through a linear algebraic approach for approximating the solution
of a linear system. Using this, a large number of pairwise interactions is
simulated and in fact very few strategies are found to act extortionately.

The work of~\cite{Press2012}, whilst showing that a clever approach to taking
advantage of another memory one strategy exists: this is incomplete. Whilst the
elegance of this result is very attractive, just as the simplicity of the
victory of Tit For Tat in Axelrod's original tournaments was, it is incomplete.
Extortionate strategies achieve a high number of wins but they do not
achieve a high score which corresponds to the fitness landscape in an
evolutionary sense. From the large number of interactions a payoff matrix \(S\)
can be measured where \(S_{ij}\) denotes the score (using standard values of
\((R, S, T, P) = (3, 0, 5, 1)\)) of the \(i\)th strategy
against the \(j\)th strategy. Using this, the replicator equation
describes the evolution of the system based on a population density fitness
function:

\begin{equation}\label{eqn:replicator_dynamics}
    \frac{dx}{dt} = x(S-x^TS x)
\end{equation}

Equation (\ref{eqn:replicator_dynamics}) is solved numerically through an
integration technique described in~\cite{Petzold1983} and
Figure~\ref{fig:replicator_dynamics} shows the evolution of the distribution of
the system: the various strategies are ranked by scores. It is clear to see that
only the high ranking strategies survive the evolutionary process (in fact,
only \input{./assets/img/replicator_dynamics/main.tex}
have a final distribution greater than \(10 ^ {-2}\)). This confirms the
findings of~\cite{Moran1707} in which sophisticated strategies resist
evolutionary invasion of shorter memory strategies. Recalling
Figure~\ref{fig:SSError_and_probabilities_in_full} this demonstrates that:

\begin{itemize}
    \item Cooperation emerges through the evolutionary process: the high scoring
        strategies do not exhibit extortionate behaviour towards each other.
    \item Extortionate strategies do not survive the evolutionary process.
\end{itemize}

\begin{figure}[!htbp]
    \centering
    \includegraphics[width=.8\textwidth]{./assets/img/replicator_dynamics/main.pdf}
    \caption{Numerical simulation of the replicator equation
    (\ref{eqn:replicator_dynamics}): strategies are ordered by score, only the strategies with a high score survive the evolutionary process.}
    \label{fig:replicator_dynamics}
\end{figure}

This work can be used to classify plays of the IPD\@: data can be collected from
actual interactions (in lab or in the field). Furthermore, this allows for a
classification method similar to the notion of fingerprinting presented
in~\cite{Ashlock2008}. Trained strategies can potentially be classified as
extortionate or not or it could be possible to even constrain the reinforcement
learning approaches that are becoming prevalent in the literature.
Alternatively, this mathematical approach for recognising extortion could be
used in sophisticated strategies to defend against invasion. Arguably, some of
the strategies considered here exhibit this behaviour, indeed as described
in~\cite{Harper2017}, the top ranking strategies in the full tournament are
obtained using evolutionary reinforcement learning techniques, thus, suspicion
of extortionate behaviour could in fact be an evolutionary trait.

\section*{Acknowledgements}

The following open source software libraries were used in this research:

\begin{itemize}
    \item The Axelrod ~\cite{Knight2016, Knight2018} library (IPD strategies and
        tournaments).
    \item The sympy library~\cite{Meurer2017} (verification of all symbolic
        calculations).
    \item The matplotlib~\cite{Droettboom2018} library (visualisation).
    \item The pandas~\cite{Structures2010}, dask~\cite{Dask2016} and
        NumPy~\cite{Oliphant2015} libraries (data manipulation).
    \item The SciPy~\cite{Jones2001} library (numerical integration of the
        replicator equation).
\end{itemize}

This work was performed using the computational facilities of the Advanced
Research Computing @ Cardiff (ARCCA) Division, Cardiff University.

\printbibliography

\newpage
\section*{Supplementary materials}

\includepdf{assets/pdf/proof_of_form_of_extortionate_strategies/main.pdf}

\newpage

Using the pair wise interactions the transition rates \(p,
q\) can be measured and the steady state probabilities inferred and compared to
the actual probabilities of each state.
This is done numerically by computing the singular eigenvector of the
matrix \(A\) \cite{Stewart2009}:

\[
    A =
    \begin{bmatrix}
        p_1 q_1 & p_1 (1 - q_1) & (1 - p_1) q_1 & (1 -p_1) (1 - q_1) \\
        p_2 q_2 & p_2 (1 - q_2) & (1 - p_2) q_2 & (1 -p_2) (1 - q_2) \\
        p_3 q_3 & p_3 (1 - q_3) & (1 - p_3) q_3 & (1 -p_3) (1 - q_3) \\
        p_4 q_4 & p_4 (1 - q_4) & (1 - p_4) q_4 & (1 -p_4) (1 - q_4) \\
    \end{bmatrix}
\]

Figure~\ref{fig:computed_probabilities_vs_theoretic_probabilities} shows a
regression line fitted to every pairwise interaction with a reported
\(\text{SSError}\) value (pairwise interactions with missing states were
omitted). This serves to validate the approach: a part from some edge cases the
relationship is consistent.

\begin{figure}[!htbp]
    \centering
    \includegraphics[width=.8\textwidth]{./assets/img/computed_probabilities_vs_theoretic_probabilities/main.pdf}
    \caption{The
        relationship between the steady state probabilities inferred from the
        measured transitions and the actual steady state probabilities. A linear
        regression line is included validating the approach.}
    \label{fig:computed_probabilities_vs_theoretic_probabilities}
\end{figure}


\end{document}

strategies.

The results of this analysis are shown in
Figure~\ref{fig:SSError_and_probabilities_in_full}. The top ranking strategies
by number of wins seem to be extortionate (but not against all strategies) and
it can be seen that a small sub group of strategies achieve mutual defection.
All the top ranking strategies according to score achieve mutual cooperation and
do not extort each other, however they
\textbf{do} exhibit extortionate behaviour towards a number of the lower ranking
strategies.

\begin{figure}[!htbp]
    \centering
    \includegraphics[width=.8\textwidth]{./assets/img/SSError_and_probabilities_in_full/main.pdf}
    \caption{\(\text{SSError}\) for the strategies for the full tournament. Only
    strategy interactions for which \(p_4=0\) and \(\chi>1\) are displayed.}
    \label{fig:SSError_and_probabilities_in_full}
\end{figure}

\section{Conclusion}\label{sec:conclusion}

This work defines an approach to measure whether or not a player is playing a
strategy that corresponds to an extortionate strategy as defined
in~\cite{Press2012}: a mathematical model for suspicion. Indeed, all
extortionate strategies have been
 classified as lying on a triangular plane.
This rigorous classification fails to be robust to small measurement error, thus
a statistical approach is proposed.
This is done through a linear algebraic approach for approximating the solution
of a linear system. Using this, a large number of pairwise interactions is
simulated and in fact very few strategies are found to act extortionately.

The work of~\cite{Press2012}, whilst showing that a clever approach to taking
advantage of another memory one strategy exists: this is incomplete. Whilst the
elegance of this result is very attractive, just as the simplicity of the
victory of Tit For Tat in Axelrod's original tournaments was, it is incomplete.
Extortionate strategies achieve a high number of wins but they do not
achieve a high score which corresponds to the fitness landscape in an
evolutionary sense. From the large number of interactions a payoff matrix \(S\)
can be measured where \(S_{ij}\) denotes the score (using standard values of
\((R, S, T, P) = (3, 0, 5, 1)\)) of the \(i\)th strategy
against the \(j\)th strategy. Using this, the replicator equation
describes the evolution of the system based on a population density fitness
function:

\begin{equation}\label{eqn:replicator_dynamics}
    \frac{dx}{dt} = x(S-x^TS x)
\end{equation}

Equation (\ref{eqn:replicator_dynamics}) is solved numerically through an
integration technique described in~\cite{Petzold1983} and
Figure~\ref{fig:replicator_dynamics} shows the evolution of the distribution of
the system: the various strategies are ranked by scores. It is clear to see that
only the high ranking strategies survive the evolutionary process (in fact,
only \documentclass[a4paper]{article}

\usepackage{amsmath}
\usepackage{amssymb}
\usepackage[margin=1.5cm,
            includefoot,
            footskip=30pt]{geometry}
\usepackage{layout}
\usepackage{graphicx}
\usepackage{subcaption}

\usepackage{biblatex}
\usepackage{pdfpages}

\bibliography{main.bib}

\title{Suspicion: Recognising and evaluating the effectiveness
       of extortion in the Iterated Prisoner's Dilemma}
\author{Vincent A. Knight \and Nikoleta E. Glynatsi}
\date{\today}



\begin{document}

\maketitle

\begin{abstract}
    The Iterated Prisoner's Dilemma is a model for rational and evolutionary
    interactive behaviour. It has applications both in the study of human social
    behaviour as well as in biology.
    It is used to understand when and how a rational individual might
    accept an immediate cost to their own utility for the direct benefit of
    another.

    Much attention has been given to a class of strategies called
    Zero Determinant strategies. It has been theoretically shown that these
    strategies can ``extort'' any player.

    In this work, an approach to identify if observed strategies are playing in
    an extortionate way is described. Furthermore, experimental analysis of
    a large tournament with \input{assets/tex/number_of_full_strategies/main.tex}
    strategies is considered. In this setting
    the most highly performing strategies do not play in an extortionate way
    against each other but do against lower performing strategies.
    This suggests that whilst the theory of Zero Determinant strategies
    indicates that memory is not of fundamental importance to the evolution of
    cooperative behaviour, this is incomplete.
\end{abstract}

\section{Introduction}\label{sec:introduction}

Agent based game theoretic models have become a stalwart of the underpinning
mathematics of interactive behaviours. One of the major pieces of work
in this area is the pair of original computer tournaments run by Robert
Axelrod~\cite{Axelrod1980, Axelrod1980a}. These tournaments pitted submitted
computer strategies against each other in plays of the Iterated Prisoner's
Dilemma. A common game where agents can choose to pay a slight cost to their
immediate utility in the hope of building a reputation. This has been used in
economic and evolutionary game theory to understand the evolution of cooperative
behaviour.

Recently, a class of strategies was described in~\cite{Press2012} that can
provably extort any given opponent. In~\cite{Hilbe2013, Moran1707} some
questions have already been asked about the true effectiveness of these
strategies in an evolutionary setting. Here another question is asked: is it
possible to recognise this extortionate behaviour? A mathematical procedure for
suspicion is presented: in the same way that the continued actions of an
extortionate individual might raise suspicion.

This work makes use of the Axelrod Python library~\cite{Knight2018, Knight2016}
with a large number of Prisoner Dilemma strategies available to give an
extensive numerical example of the ideas presented.  The approach is presented
in Section~\ref{sec:delta-zd-strategies}.  All of the code and data discussed
in Section~\ref{sec:numerical-experiments} is open sourced, archived and
written according to best scientific principles~\cite{Wilson2014}. The data
archive can be found at~\cite{vincent_knight_2018_1297075}.

\section{Recognising Extortion}\label{sec:delta-zd-strategies}

In~\cite{Press2012}, given a match between 2 memory-one strategies, the concept
of Zero Determinant (ZD) strategies is introduced. The main result of that paper
shows that given two memory one players \(p, q\in\mathbb{R}^4\) a linear
relationship between the players' scores could be forced by one of the players.

Using the notation of~\cite{Press2012}, assuming the utilities for player \(p\)
are given by \(S_x=(R, S, T, P)\) and for player \(q\) by \(S_y=(R, T, S, P)\)
and that the stationary scores of each player is given by \(S_X\) and \(S_Y\)
respectively. The main result of~\cite{Press2012} is that if

\begin{equation}\label{eqn:linear_relationship_for_p}
    \tilde p=\alpha S_x + \beta S_y + \gamma
\end{equation}

or

\begin{equation}\label{eqn:linear_relationship_for_q}
    \tilde q=\alpha S_x + \beta S_y + \gamma
\end{equation}

where \(\tilde p = (1 - p_1, 1 - p_2, p_3, p_4)\) and
\(\tilde q = (1 - q_1, 1 - q_2, q_3, q_4)\) then:

\begin{equation}
    \alpha S_X + \beta S_Y + \gamma = 0
\end{equation}

In~\cite{Press2012} a particular type of ZD strategy is defined: extortionate
strategies. If:

\begin{equation}\label{eqn:constraint_for_extortion}
    \gamma = - P(\alpha + \beta)
\end{equation}

then the player can ensure they get a score \(\chi\) times
larger than the opponent. This extortion coefficient is given by:

\begin{equation}\label{eqn:definition_of_chi}
    \chi=\frac{-\beta}{\alpha}
\end{equation}

Thus, if (\ref{eqn:constraint_for_extortion}) holds and \(\chi >1\) a player is
said to extort their opponent.
Here, the reverse problem is considered: given a
\(p\in\mathbb{R}^4\) how does one identify \(\alpha, \beta\) if they
exist and is the strategy in fact acting in an extortionate way?

These conditions correspond to:

\begin{align}
    \tilde p_1 & = \alpha R + \beta R - P (\alpha + \beta)
            \label{eqn:condition_for_tilde_p1}\\
    \tilde p_2 & = \alpha S + \beta T - P (\alpha + \beta)
            \label{eqn:condition_for_tilde_p2}\\
    \tilde p_3 & = \alpha T + \beta S - P (\alpha + \beta)
            \label{eqn:condition_for_tilde_p3}\\
    \tilde p_4 & = \alpha P + \beta P - P (\alpha + \beta)
            \label{eqn:condition_for_tilde_p4}
\end{align}

Equation (\ref{eqn:condition_for_tilde_p4}) ensures that \(p_4=\tilde p_4=0\).
Equations (\ref{eqn:condition_for_tilde_p1}-\ref{eqn:condition_for_tilde_p3})
can be used to eliminate \(\alpha, \beta\), giving:

\begin{equation}\label{eqn:planar_definition_of_extortion}
    \tilde p_1 = \frac{(R - P)(\tilde p_2 + \tilde p_3)}{S + T - 2P}
\end{equation}

with:

\begin{equation}\label{eqn:definition_of_chi}
    \chi = \frac{\tilde p_2 (P - T) + \tilde p_3 (S - P)}
                {\tilde p_2 (P - S) + \tilde p_3 (T - P)}
\end{equation}

Given a strategy \(p\in\mathbb{R}^{4\times 1}\) equations
(\ref{eqn:condition_for_tilde_p4}), (\ref{eqn:planar_definition_of_extortion}-\ref{eqn:definition_of_chi}) can be used to check if
a strategy is extortionate. The conditions correspond to:

\begin{align}
    p_1 & = \frac{(R-P)(p_2 + p_3) - R + T + S - P}{S + T - 2P}
     \label{eqn:condition_for_p1}\\
    p_4 & = 0 \label{eqn:condition_for_p4}\\
    1 & > p_2 + p_3\label{eqn:condition_for_chi}
\end{align}

The algebraic steps necessary to prove these results are available in the
supporting materials.

All extortionate strategies reside on a triangular (\ref{eqn:condition_for_chi})
plane (\ref{eqn:condition_for_p1}) in 3 dimensions (\ref{eqn:condition_for_p4}).
Using this formulation it can be seen that a necessary (but not sufficient)
condition for an extortionate strategy is that it cooperates on average less
than 50\% of the time when in a state of disagreement with the opponent.

As an example, consider the known extortionate strategy \(p=(8 / 9, 1 / 2, 1 /
3, 0)\) from~\cite{Stewart2012} which is referred to as \texttt{Extort-2}. In
this case, for the standard values of \((R, T, S, P)\) constraint
(\ref{eqn:condition_for_p1}) corresponds to:

\begin{equation}
    p_1 = \frac{2(p_2 + p_3) + 1}{3}
\end{equation}

It is clear that in this case all constraints hold.

This approach could in fact be used to confirm that a given strategy is acting
in an extortionate manner even if it is not a memory one strategy. However, in
practice, if a closed form for \(p\) is not known, then due to measurement
and/or numerical error this would not work.

This problem can be written in the following linear algebraic form where
\(x=(\alpha, \beta)\)
and \(p^*=(\tilde p_1 - 1, tilde_2 - 1, p_3)\):

\begin{equation}\label{eqn:linear_algebraic_equation_for_p}
    Cx= p^*
\end{equation}

\(C\) corresponds to equations
(\ref{eqn:condition_for_tilde_p1}-\ref{eqn:condition_for_tilde_p3}) and is
given by:

\begin{equation}\label{eqn:definition_of_C}
    C =
    \begin{bmatrix}
        R - P & R- P \\
        S - P & T- P \\
        T - P & S- P \\
    \end{bmatrix}
\end{equation}

Note that in general, equation (\ref{eqn:linear_algebraic_equation_for_p}) will
not necessarily have a solution. From the Rouch\'{e}-Capelli theorem if there is
a solution it is unique as \(\text{rank}(C)=2\) which is the dimension of the
variable \(x\). The best fitting \(x\) is found by minimizing:

\begin{equation}\label{eqn:r_squared}
    \text{SSError} = \|C x- p^*\|_2^2 = \sum_{i=1}^{3}\left((C\bar x)_i-p_i^*\right)^2
\end{equation}

Note that \(\text{SSError}\), which is the square of the Frobenius
norm~\cite{Golub2013}, becomes a measure of how close a strategy is to being an
extortionate strategy. Suspicion
of extortion then corresponds to a threshold on \(\text{SSError}\).

By observing interactions (human or otherwise), their memory one representation
can be inferred and this approach can be used to recognise extortionate
behaviour. The notion of comparing theoretic and actual plays of the IPD is not
novel, see for example~\cite{Rand2013}. Immediately it is noted that if the
environment is noisy~\cite{Wu1995} then no strategy can be considered to be
extortionate as \(p_4>0\).

In the next section, this idea will be illustrated by observing the interactions
that take place in a computer based tournament of the IPD\@.

\section{Numerical experiments}\label{sec:numerical-experiments}

In~\cite{Stewart2012} results from a tournament with
\input{./assets/tex/number_of_stewart_plotkin_strategies/main.tex} strategies,
was presented with specific consideration given to ZD strategies. This
tournament is reproduced here using the Axelrod-Python
project~\cite{Knight2016}. To obtain a good measure of the corresponding
transition rates for each strategy all matches have been run for
\input{assets/tex/number_of_turns/main.tex} turns and every match has been
repeated \input{assets/tex/number_of_repetitions/main.tex} times. All of this
interaction data is available at~\cite{vincent_knight_2018_1297075}. A good
match between the inferred Markov chain and the state distribution of the actual
interactions has been verified. Data for this is presented in the supplementary
materials.

Figure~\ref{fig:SSError_overall_in_stewart_plotkin} shows the \(\text{SSError}\)
values for all the strategies in the tournament, as reported
in~\cite{Stewart2012} the extortionate strategy (which has an expected
\(\text{SSError}\) approximately 0) gains a large number of wins.

\begin{figure}[!htbp]
    \centering
    \includegraphics[width=.8\textwidth]{./assets/img/SSError_overall_in_stewart_plotkin/main.pdf}
    \caption{\(\text{SSError}\) and state probabilities for the strategies
        of~\cite{Stewart2012}, ordered both by number of wins and overall score.
        Note that \(P(DC)\) is not shown as it corresponds to the transpose of
        \(P(CD)\). Cooperator and Defector are omitted as they do not visit all
        the states.}
    \label{fig:SSError_overall_in_stewart_plotkin}
\end{figure}

Here, the work of~\cite{Stewart2012} is extended by investigating a tournament
with \input{assets/tex/number_of_full_strategies/main.tex}
strategies.

The results of this analysis are shown in
Figure~\ref{fig:SSError_and_probabilities_in_full}. The top ranking strategies
by number of wins seem to be extortionate (but not against all strategies) and
it can be seen that a small sub group of strategies achieve mutual defection.
All the top ranking strategies according to score achieve mutual cooperation and
do not extort each other, however they
\textbf{do} exhibit extortionate behaviour towards a number of the lower ranking
strategies.

\begin{figure}[!htbp]
    \centering
    \includegraphics[width=.8\textwidth]{./assets/img/SSError_and_probabilities_in_full/main.pdf}
    \caption{\(\text{SSError}\) for the strategies for the full tournament. Only
    strategy interactions for which \(p_4=0\) and \(\chi>1\) are displayed.}
    \label{fig:SSError_and_probabilities_in_full}
\end{figure}

\section{Conclusion}\label{sec:conclusion}

This work defines an approach to measure whether or not a player is playing a
strategy that corresponds to an extortionate strategy as defined
in~\cite{Press2012}: a mathematical model for suspicion. Indeed, all
extortionate strategies have been
 classified as lying on a triangular plane.
This rigorous classification fails to be robust to small measurement error, thus
a statistical approach is proposed.
This is done through a linear algebraic approach for approximating the solution
of a linear system. Using this, a large number of pairwise interactions is
simulated and in fact very few strategies are found to act extortionately.

The work of~\cite{Press2012}, whilst showing that a clever approach to taking
advantage of another memory one strategy exists: this is incomplete. Whilst the
elegance of this result is very attractive, just as the simplicity of the
victory of Tit For Tat in Axelrod's original tournaments was, it is incomplete.
Extortionate strategies achieve a high number of wins but they do not
achieve a high score which corresponds to the fitness landscape in an
evolutionary sense. From the large number of interactions a payoff matrix \(S\)
can be measured where \(S_{ij}\) denotes the score (using standard values of
\((R, S, T, P) = (3, 0, 5, 1)\)) of the \(i\)th strategy
against the \(j\)th strategy. Using this, the replicator equation
describes the evolution of the system based on a population density fitness
function:

\begin{equation}\label{eqn:replicator_dynamics}
    \frac{dx}{dt} = x(S-x^TS x)
\end{equation}

Equation (\ref{eqn:replicator_dynamics}) is solved numerically through an
integration technique described in~\cite{Petzold1983} and
Figure~\ref{fig:replicator_dynamics} shows the evolution of the distribution of
the system: the various strategies are ranked by scores. It is clear to see that
only the high ranking strategies survive the evolutionary process (in fact,
only \input{./assets/img/replicator_dynamics/main.tex}
have a final distribution greater than \(10 ^ {-2}\)). This confirms the
findings of~\cite{Moran1707} in which sophisticated strategies resist
evolutionary invasion of shorter memory strategies. Recalling
Figure~\ref{fig:SSError_and_probabilities_in_full} this demonstrates that:

\begin{itemize}
    \item Cooperation emerges through the evolutionary process: the high scoring
        strategies do not exhibit extortionate behaviour towards each other.
    \item Extortionate strategies do not survive the evolutionary process.
\end{itemize}

\begin{figure}[!htbp]
    \centering
    \includegraphics[width=.8\textwidth]{./assets/img/replicator_dynamics/main.pdf}
    \caption{Numerical simulation of the replicator equation
    (\ref{eqn:replicator_dynamics}): strategies are ordered by score, only the strategies with a high score survive the evolutionary process.}
    \label{fig:replicator_dynamics}
\end{figure}

This work can be used to classify plays of the IPD\@: data can be collected from
actual interactions (in lab or in the field). Furthermore, this allows for a
classification method similar to the notion of fingerprinting presented
in~\cite{Ashlock2008}. Trained strategies can potentially be classified as
extortionate or not or it could be possible to even constrain the reinforcement
learning approaches that are becoming prevalent in the literature.
Alternatively, this mathematical approach for recognising extortion could be
used in sophisticated strategies to defend against invasion. Arguably, some of
the strategies considered here exhibit this behaviour, indeed as described
in~\cite{Harper2017}, the top ranking strategies in the full tournament are
obtained using evolutionary reinforcement learning techniques, thus, suspicion
of extortionate behaviour could in fact be an evolutionary trait.

\section*{Acknowledgements}

The following open source software libraries were used in this research:

\begin{itemize}
    \item The Axelrod ~\cite{Knight2016, Knight2018} library (IPD strategies and
        tournaments).
    \item The sympy library~\cite{Meurer2017} (verification of all symbolic
        calculations).
    \item The matplotlib~\cite{Droettboom2018} library (visualisation).
    \item The pandas~\cite{Structures2010}, dask~\cite{Dask2016} and
        NumPy~\cite{Oliphant2015} libraries (data manipulation).
    \item The SciPy~\cite{Jones2001} library (numerical integration of the
        replicator equation).
\end{itemize}

This work was performed using the computational facilities of the Advanced
Research Computing @ Cardiff (ARCCA) Division, Cardiff University.

\printbibliography

\newpage
\section*{Supplementary materials}

\includepdf{assets/pdf/proof_of_form_of_extortionate_strategies/main.pdf}

\newpage

Using the pair wise interactions the transition rates \(p,
q\) can be measured and the steady state probabilities inferred and compared to
the actual probabilities of each state.
This is done numerically by computing the singular eigenvector of the
matrix \(A\) \cite{Stewart2009}:

\[
    A =
    \begin{bmatrix}
        p_1 q_1 & p_1 (1 - q_1) & (1 - p_1) q_1 & (1 -p_1) (1 - q_1) \\
        p_2 q_2 & p_2 (1 - q_2) & (1 - p_2) q_2 & (1 -p_2) (1 - q_2) \\
        p_3 q_3 & p_3 (1 - q_3) & (1 - p_3) q_3 & (1 -p_3) (1 - q_3) \\
        p_4 q_4 & p_4 (1 - q_4) & (1 - p_4) q_4 & (1 -p_4) (1 - q_4) \\
    \end{bmatrix}
\]

Figure~\ref{fig:computed_probabilities_vs_theoretic_probabilities} shows a
regression line fitted to every pairwise interaction with a reported
\(\text{SSError}\) value (pairwise interactions with missing states were
omitted). This serves to validate the approach: a part from some edge cases the
relationship is consistent.

\begin{figure}[!htbp]
    \centering
    \includegraphics[width=.8\textwidth]{./assets/img/computed_probabilities_vs_theoretic_probabilities/main.pdf}
    \caption{The
        relationship between the steady state probabilities inferred from the
        measured transitions and the actual steady state probabilities. A linear
        regression line is included validating the approach.}
    \label{fig:computed_probabilities_vs_theoretic_probabilities}
\end{figure}


\end{document}

have a final distribution greater than \(10 ^ {-2}\)). This confirms the
findings of~\cite{Moran1707} in which sophisticated strategies resist
evolutionary invasion of shorter memory strategies. Recalling
Figure~\ref{fig:SSError_and_probabilities_in_full} this demonstrates that:

\begin{itemize}
    \item Cooperation emerges through the evolutionary process: the high scoring
        strategies do not exhibit extortionate behaviour towards each other.
    \item Extortionate strategies do not survive the evolutionary process.
\end{itemize}

\begin{figure}[!htbp]
    \centering
    \includegraphics[width=.8\textwidth]{./assets/img/replicator_dynamics/main.pdf}
    \caption{Numerical simulation of the replicator equation
    (\ref{eqn:replicator_dynamics}): strategies are ordered by score, only the strategies with a high score survive the evolutionary process.}
    \label{fig:replicator_dynamics}
\end{figure}

This work can be used to classify plays of the IPD\@: data can be collected from
actual interactions (in lab or in the field). Furthermore, this allows for a
classification method similar to the notion of fingerprinting presented
in~\cite{Ashlock2008}. Trained strategies can potentially be classified as
extortionate or not or it could be possible to even constrain the reinforcement
learning approaches that are becoming prevalent in the literature.
Alternatively, this mathematical approach for recognising extortion could be
used in sophisticated strategies to defend against invasion. Arguably, some of
the strategies considered here exhibit this behaviour, indeed as described
in~\cite{Harper2017}, the top ranking strategies in the full tournament are
obtained using evolutionary reinforcement learning techniques, thus, suspicion
of extortionate behaviour could in fact be an evolutionary trait.

\section*{Acknowledgements}

The following open source software libraries were used in this research:

\begin{itemize}
    \item The Axelrod ~\cite{Knight2016, Knight2018} library (IPD strategies and
        tournaments).
    \item The sympy library~\cite{Meurer2017} (verification of all symbolic
        calculations).
    \item The matplotlib~\cite{Droettboom2018} library (visualisation).
    \item The pandas~\cite{Structures2010}, dask~\cite{Dask2016} and
        NumPy~\cite{Oliphant2015} libraries (data manipulation).
    \item The SciPy~\cite{Jones2001} library (numerical integration of the
        replicator equation).
\end{itemize}

This work was performed using the computational facilities of the Advanced
Research Computing @ Cardiff (ARCCA) Division, Cardiff University.

\printbibliography

\newpage
\section*{Supplementary materials}

\includepdf{assets/pdf/proof_of_form_of_extortionate_strategies/main.pdf}

\newpage

Using the pair wise interactions the transition rates \(p,
q\) can be measured and the steady state probabilities inferred and compared to
the actual probabilities of each state.
This is done numerically by computing the singular eigenvector of the
matrix \(A\) \cite{Stewart2009}:

\[
    A =
    \begin{bmatrix}
        p_1 q_1 & p_1 (1 - q_1) & (1 - p_1) q_1 & (1 -p_1) (1 - q_1) \\
        p_2 q_2 & p_2 (1 - q_2) & (1 - p_2) q_2 & (1 -p_2) (1 - q_2) \\
        p_3 q_3 & p_3 (1 - q_3) & (1 - p_3) q_3 & (1 -p_3) (1 - q_3) \\
        p_4 q_4 & p_4 (1 - q_4) & (1 - p_4) q_4 & (1 -p_4) (1 - q_4) \\
    \end{bmatrix}
\]

Figure~\ref{fig:computed_probabilities_vs_theoretic_probabilities} shows a
regression line fitted to every pairwise interaction with a reported
\(\text{SSError}\) value (pairwise interactions with missing states were
omitted). This serves to validate the approach: a part from some edge cases the
relationship is consistent.

\begin{figure}[!htbp]
    \centering
    \includegraphics[width=.8\textwidth]{./assets/img/computed_probabilities_vs_theoretic_probabilities/main.pdf}
    \caption{The
        relationship between the steady state probabilities inferred from the
        measured transitions and the actual steady state probabilities. A linear
        regression line is included validating the approach.}
    \label{fig:computed_probabilities_vs_theoretic_probabilities}
\end{figure}


\end{document}
 times. All of this
interaction data is available at~\cite{vincent_knight_2018_1297075}. A good
match between the inferred Markov chain and the state distribution of the actual
interactions has been verified. Data for this is presented in the supplementary
materials.

Figure~\ref{fig:SSError_overall_in_stewart_plotkin} shows the \(\text{SSError}\)
values for all the strategies in the tournament, as reported
in~\cite{Stewart2012} the extortionate strategy (which has an expected
\(\text{SSError}\) approximately 0) gains a large number of wins.

\begin{figure}[!htbp]
    \centering
    \includegraphics[width=.8\textwidth]{./assets/img/SSError_overall_in_stewart_plotkin/main.pdf}
    \caption{\(\text{SSError}\) and state probabilities for the strategies
        of~\cite{Stewart2012}, ordered both by number of wins and overall score.
        Note that \(P(DC)\) is not shown as it corresponds to the transpose of
        \(P(CD)\). Cooperator and Defector are omitted as they do not visit all
        the states.}
    \label{fig:SSError_overall_in_stewart_plotkin}
\end{figure}

Here, the work of~\cite{Stewart2012} is extended by investigating a tournament
with \documentclass[a4paper]{article}

\usepackage{amsmath}
\usepackage{amssymb}
\usepackage[margin=1.5cm,
            includefoot,
            footskip=30pt]{geometry}
\usepackage{layout}
\usepackage{graphicx}
\usepackage{subcaption}

\usepackage{biblatex}
\usepackage{pdfpages}

\bibliography{main.bib}

\title{Suspicion: Recognising and evaluating the effectiveness
       of extortion in the Iterated Prisoner's Dilemma}
\author{Vincent A. Knight \and Nikoleta E. Glynatsi}
\date{\today}



\begin{document}

\maketitle

\begin{abstract}
    The Iterated Prisoner's Dilemma is a model for rational and evolutionary
    interactive behaviour. It has applications both in the study of human social
    behaviour as well as in biology.
    It is used to understand when and how a rational individual might
    accept an immediate cost to their own utility for the direct benefit of
    another.

    Much attention has been given to a class of strategies called
    Zero Determinant strategies. It has been theoretically shown that these
    strategies can ``extort'' any player.

    In this work, an approach to identify if observed strategies are playing in
    an extortionate way is described. Furthermore, experimental analysis of
    a large tournament with \documentclass[a4paper]{article}

\usepackage{amsmath}
\usepackage{amssymb}
\usepackage[margin=1.5cm,
            includefoot,
            footskip=30pt]{geometry}
\usepackage{layout}
\usepackage{graphicx}
\usepackage{subcaption}

\usepackage{biblatex}
\usepackage{pdfpages}

\bibliography{main.bib}

\title{Suspicion: Recognising and evaluating the effectiveness
       of extortion in the Iterated Prisoner's Dilemma}
\author{Vincent A. Knight \and Nikoleta E. Glynatsi}
\date{\today}



\begin{document}

\maketitle

\begin{abstract}
    The Iterated Prisoner's Dilemma is a model for rational and evolutionary
    interactive behaviour. It has applications both in the study of human social
    behaviour as well as in biology.
    It is used to understand when and how a rational individual might
    accept an immediate cost to their own utility for the direct benefit of
    another.

    Much attention has been given to a class of strategies called
    Zero Determinant strategies. It has been theoretically shown that these
    strategies can ``extort'' any player.

    In this work, an approach to identify if observed strategies are playing in
    an extortionate way is described. Furthermore, experimental analysis of
    a large tournament with \input{assets/tex/number_of_full_strategies/main.tex}
    strategies is considered. In this setting
    the most highly performing strategies do not play in an extortionate way
    against each other but do against lower performing strategies.
    This suggests that whilst the theory of Zero Determinant strategies
    indicates that memory is not of fundamental importance to the evolution of
    cooperative behaviour, this is incomplete.
\end{abstract}

\section{Introduction}\label{sec:introduction}

Agent based game theoretic models have become a stalwart of the underpinning
mathematics of interactive behaviours. One of the major pieces of work
in this area is the pair of original computer tournaments run by Robert
Axelrod~\cite{Axelrod1980, Axelrod1980a}. These tournaments pitted submitted
computer strategies against each other in plays of the Iterated Prisoner's
Dilemma. A common game where agents can choose to pay a slight cost to their
immediate utility in the hope of building a reputation. This has been used in
economic and evolutionary game theory to understand the evolution of cooperative
behaviour.

Recently, a class of strategies was described in~\cite{Press2012} that can
provably extort any given opponent. In~\cite{Hilbe2013, Moran1707} some
questions have already been asked about the true effectiveness of these
strategies in an evolutionary setting. Here another question is asked: is it
possible to recognise this extortionate behaviour? A mathematical procedure for
suspicion is presented: in the same way that the continued actions of an
extortionate individual might raise suspicion.

This work makes use of the Axelrod Python library~\cite{Knight2018, Knight2016}
with a large number of Prisoner Dilemma strategies available to give an
extensive numerical example of the ideas presented.  The approach is presented
in Section~\ref{sec:delta-zd-strategies}.  All of the code and data discussed
in Section~\ref{sec:numerical-experiments} is open sourced, archived and
written according to best scientific principles~\cite{Wilson2014}. The data
archive can be found at~\cite{vincent_knight_2018_1297075}.

\section{Recognising Extortion}\label{sec:delta-zd-strategies}

In~\cite{Press2012}, given a match between 2 memory-one strategies, the concept
of Zero Determinant (ZD) strategies is introduced. The main result of that paper
shows that given two memory one players \(p, q\in\mathbb{R}^4\) a linear
relationship between the players' scores could be forced by one of the players.

Using the notation of~\cite{Press2012}, assuming the utilities for player \(p\)
are given by \(S_x=(R, S, T, P)\) and for player \(q\) by \(S_y=(R, T, S, P)\)
and that the stationary scores of each player is given by \(S_X\) and \(S_Y\)
respectively. The main result of~\cite{Press2012} is that if

\begin{equation}\label{eqn:linear_relationship_for_p}
    \tilde p=\alpha S_x + \beta S_y + \gamma
\end{equation}

or

\begin{equation}\label{eqn:linear_relationship_for_q}
    \tilde q=\alpha S_x + \beta S_y + \gamma
\end{equation}

where \(\tilde p = (1 - p_1, 1 - p_2, p_3, p_4)\) and
\(\tilde q = (1 - q_1, 1 - q_2, q_3, q_4)\) then:

\begin{equation}
    \alpha S_X + \beta S_Y + \gamma = 0
\end{equation}

In~\cite{Press2012} a particular type of ZD strategy is defined: extortionate
strategies. If:

\begin{equation}\label{eqn:constraint_for_extortion}
    \gamma = - P(\alpha + \beta)
\end{equation}

then the player can ensure they get a score \(\chi\) times
larger than the opponent. This extortion coefficient is given by:

\begin{equation}\label{eqn:definition_of_chi}
    \chi=\frac{-\beta}{\alpha}
\end{equation}

Thus, if (\ref{eqn:constraint_for_extortion}) holds and \(\chi >1\) a player is
said to extort their opponent.
Here, the reverse problem is considered: given a
\(p\in\mathbb{R}^4\) how does one identify \(\alpha, \beta\) if they
exist and is the strategy in fact acting in an extortionate way?

These conditions correspond to:

\begin{align}
    \tilde p_1 & = \alpha R + \beta R - P (\alpha + \beta)
            \label{eqn:condition_for_tilde_p1}\\
    \tilde p_2 & = \alpha S + \beta T - P (\alpha + \beta)
            \label{eqn:condition_for_tilde_p2}\\
    \tilde p_3 & = \alpha T + \beta S - P (\alpha + \beta)
            \label{eqn:condition_for_tilde_p3}\\
    \tilde p_4 & = \alpha P + \beta P - P (\alpha + \beta)
            \label{eqn:condition_for_tilde_p4}
\end{align}

Equation (\ref{eqn:condition_for_tilde_p4}) ensures that \(p_4=\tilde p_4=0\).
Equations (\ref{eqn:condition_for_tilde_p1}-\ref{eqn:condition_for_tilde_p3})
can be used to eliminate \(\alpha, \beta\), giving:

\begin{equation}\label{eqn:planar_definition_of_extortion}
    \tilde p_1 = \frac{(R - P)(\tilde p_2 + \tilde p_3)}{S + T - 2P}
\end{equation}

with:

\begin{equation}\label{eqn:definition_of_chi}
    \chi = \frac{\tilde p_2 (P - T) + \tilde p_3 (S - P)}
                {\tilde p_2 (P - S) + \tilde p_3 (T - P)}
\end{equation}

Given a strategy \(p\in\mathbb{R}^{4\times 1}\) equations
(\ref{eqn:condition_for_tilde_p4}), (\ref{eqn:planar_definition_of_extortion}-\ref{eqn:definition_of_chi}) can be used to check if
a strategy is extortionate. The conditions correspond to:

\begin{align}
    p_1 & = \frac{(R-P)(p_2 + p_3) - R + T + S - P}{S + T - 2P}
     \label{eqn:condition_for_p1}\\
    p_4 & = 0 \label{eqn:condition_for_p4}\\
    1 & > p_2 + p_3\label{eqn:condition_for_chi}
\end{align}

The algebraic steps necessary to prove these results are available in the
supporting materials.

All extortionate strategies reside on a triangular (\ref{eqn:condition_for_chi})
plane (\ref{eqn:condition_for_p1}) in 3 dimensions (\ref{eqn:condition_for_p4}).
Using this formulation it can be seen that a necessary (but not sufficient)
condition for an extortionate strategy is that it cooperates on average less
than 50\% of the time when in a state of disagreement with the opponent.

As an example, consider the known extortionate strategy \(p=(8 / 9, 1 / 2, 1 /
3, 0)\) from~\cite{Stewart2012} which is referred to as \texttt{Extort-2}. In
this case, for the standard values of \((R, T, S, P)\) constraint
(\ref{eqn:condition_for_p1}) corresponds to:

\begin{equation}
    p_1 = \frac{2(p_2 + p_3) + 1}{3}
\end{equation}

It is clear that in this case all constraints hold.

This approach could in fact be used to confirm that a given strategy is acting
in an extortionate manner even if it is not a memory one strategy. However, in
practice, if a closed form for \(p\) is not known, then due to measurement
and/or numerical error this would not work.

This problem can be written in the following linear algebraic form where
\(x=(\alpha, \beta)\)
and \(p^*=(\tilde p_1 - 1, tilde_2 - 1, p_3)\):

\begin{equation}\label{eqn:linear_algebraic_equation_for_p}
    Cx= p^*
\end{equation}

\(C\) corresponds to equations
(\ref{eqn:condition_for_tilde_p1}-\ref{eqn:condition_for_tilde_p3}) and is
given by:

\begin{equation}\label{eqn:definition_of_C}
    C =
    \begin{bmatrix}
        R - P & R- P \\
        S - P & T- P \\
        T - P & S- P \\
    \end{bmatrix}
\end{equation}

Note that in general, equation (\ref{eqn:linear_algebraic_equation_for_p}) will
not necessarily have a solution. From the Rouch\'{e}-Capelli theorem if there is
a solution it is unique as \(\text{rank}(C)=2\) which is the dimension of the
variable \(x\). The best fitting \(x\) is found by minimizing:

\begin{equation}\label{eqn:r_squared}
    \text{SSError} = \|C x- p^*\|_2^2 = \sum_{i=1}^{3}\left((C\bar x)_i-p_i^*\right)^2
\end{equation}

Note that \(\text{SSError}\), which is the square of the Frobenius
norm~\cite{Golub2013}, becomes a measure of how close a strategy is to being an
extortionate strategy. Suspicion
of extortion then corresponds to a threshold on \(\text{SSError}\).

By observing interactions (human or otherwise), their memory one representation
can be inferred and this approach can be used to recognise extortionate
behaviour. The notion of comparing theoretic and actual plays of the IPD is not
novel, see for example~\cite{Rand2013}. Immediately it is noted that if the
environment is noisy~\cite{Wu1995} then no strategy can be considered to be
extortionate as \(p_4>0\).

In the next section, this idea will be illustrated by observing the interactions
that take place in a computer based tournament of the IPD\@.

\section{Numerical experiments}\label{sec:numerical-experiments}

In~\cite{Stewart2012} results from a tournament with
\input{./assets/tex/number_of_stewart_plotkin_strategies/main.tex} strategies,
was presented with specific consideration given to ZD strategies. This
tournament is reproduced here using the Axelrod-Python
project~\cite{Knight2016}. To obtain a good measure of the corresponding
transition rates for each strategy all matches have been run for
\input{assets/tex/number_of_turns/main.tex} turns and every match has been
repeated \input{assets/tex/number_of_repetitions/main.tex} times. All of this
interaction data is available at~\cite{vincent_knight_2018_1297075}. A good
match between the inferred Markov chain and the state distribution of the actual
interactions has been verified. Data for this is presented in the supplementary
materials.

Figure~\ref{fig:SSError_overall_in_stewart_plotkin} shows the \(\text{SSError}\)
values for all the strategies in the tournament, as reported
in~\cite{Stewart2012} the extortionate strategy (which has an expected
\(\text{SSError}\) approximately 0) gains a large number of wins.

\begin{figure}[!htbp]
    \centering
    \includegraphics[width=.8\textwidth]{./assets/img/SSError_overall_in_stewart_plotkin/main.pdf}
    \caption{\(\text{SSError}\) and state probabilities for the strategies
        of~\cite{Stewart2012}, ordered both by number of wins and overall score.
        Note that \(P(DC)\) is not shown as it corresponds to the transpose of
        \(P(CD)\). Cooperator and Defector are omitted as they do not visit all
        the states.}
    \label{fig:SSError_overall_in_stewart_plotkin}
\end{figure}

Here, the work of~\cite{Stewart2012} is extended by investigating a tournament
with \input{assets/tex/number_of_full_strategies/main.tex}
strategies.

The results of this analysis are shown in
Figure~\ref{fig:SSError_and_probabilities_in_full}. The top ranking strategies
by number of wins seem to be extortionate (but not against all strategies) and
it can be seen that a small sub group of strategies achieve mutual defection.
All the top ranking strategies according to score achieve mutual cooperation and
do not extort each other, however they
\textbf{do} exhibit extortionate behaviour towards a number of the lower ranking
strategies.

\begin{figure}[!htbp]
    \centering
    \includegraphics[width=.8\textwidth]{./assets/img/SSError_and_probabilities_in_full/main.pdf}
    \caption{\(\text{SSError}\) for the strategies for the full tournament. Only
    strategy interactions for which \(p_4=0\) and \(\chi>1\) are displayed.}
    \label{fig:SSError_and_probabilities_in_full}
\end{figure}

\section{Conclusion}\label{sec:conclusion}

This work defines an approach to measure whether or not a player is playing a
strategy that corresponds to an extortionate strategy as defined
in~\cite{Press2012}: a mathematical model for suspicion. Indeed, all
extortionate strategies have been
 classified as lying on a triangular plane.
This rigorous classification fails to be robust to small measurement error, thus
a statistical approach is proposed.
This is done through a linear algebraic approach for approximating the solution
of a linear system. Using this, a large number of pairwise interactions is
simulated and in fact very few strategies are found to act extortionately.

The work of~\cite{Press2012}, whilst showing that a clever approach to taking
advantage of another memory one strategy exists: this is incomplete. Whilst the
elegance of this result is very attractive, just as the simplicity of the
victory of Tit For Tat in Axelrod's original tournaments was, it is incomplete.
Extortionate strategies achieve a high number of wins but they do not
achieve a high score which corresponds to the fitness landscape in an
evolutionary sense. From the large number of interactions a payoff matrix \(S\)
can be measured where \(S_{ij}\) denotes the score (using standard values of
\((R, S, T, P) = (3, 0, 5, 1)\)) of the \(i\)th strategy
against the \(j\)th strategy. Using this, the replicator equation
describes the evolution of the system based on a population density fitness
function:

\begin{equation}\label{eqn:replicator_dynamics}
    \frac{dx}{dt} = x(S-x^TS x)
\end{equation}

Equation (\ref{eqn:replicator_dynamics}) is solved numerically through an
integration technique described in~\cite{Petzold1983} and
Figure~\ref{fig:replicator_dynamics} shows the evolution of the distribution of
the system: the various strategies are ranked by scores. It is clear to see that
only the high ranking strategies survive the evolutionary process (in fact,
only \input{./assets/img/replicator_dynamics/main.tex}
have a final distribution greater than \(10 ^ {-2}\)). This confirms the
findings of~\cite{Moran1707} in which sophisticated strategies resist
evolutionary invasion of shorter memory strategies. Recalling
Figure~\ref{fig:SSError_and_probabilities_in_full} this demonstrates that:

\begin{itemize}
    \item Cooperation emerges through the evolutionary process: the high scoring
        strategies do not exhibit extortionate behaviour towards each other.
    \item Extortionate strategies do not survive the evolutionary process.
\end{itemize}

\begin{figure}[!htbp]
    \centering
    \includegraphics[width=.8\textwidth]{./assets/img/replicator_dynamics/main.pdf}
    \caption{Numerical simulation of the replicator equation
    (\ref{eqn:replicator_dynamics}): strategies are ordered by score, only the strategies with a high score survive the evolutionary process.}
    \label{fig:replicator_dynamics}
\end{figure}

This work can be used to classify plays of the IPD\@: data can be collected from
actual interactions (in lab or in the field). Furthermore, this allows for a
classification method similar to the notion of fingerprinting presented
in~\cite{Ashlock2008}. Trained strategies can potentially be classified as
extortionate or not or it could be possible to even constrain the reinforcement
learning approaches that are becoming prevalent in the literature.
Alternatively, this mathematical approach for recognising extortion could be
used in sophisticated strategies to defend against invasion. Arguably, some of
the strategies considered here exhibit this behaviour, indeed as described
in~\cite{Harper2017}, the top ranking strategies in the full tournament are
obtained using evolutionary reinforcement learning techniques, thus, suspicion
of extortionate behaviour could in fact be an evolutionary trait.

\section*{Acknowledgements}

The following open source software libraries were used in this research:

\begin{itemize}
    \item The Axelrod ~\cite{Knight2016, Knight2018} library (IPD strategies and
        tournaments).
    \item The sympy library~\cite{Meurer2017} (verification of all symbolic
        calculations).
    \item The matplotlib~\cite{Droettboom2018} library (visualisation).
    \item The pandas~\cite{Structures2010}, dask~\cite{Dask2016} and
        NumPy~\cite{Oliphant2015} libraries (data manipulation).
    \item The SciPy~\cite{Jones2001} library (numerical integration of the
        replicator equation).
\end{itemize}

This work was performed using the computational facilities of the Advanced
Research Computing @ Cardiff (ARCCA) Division, Cardiff University.

\printbibliography

\newpage
\section*{Supplementary materials}

\includepdf{assets/pdf/proof_of_form_of_extortionate_strategies/main.pdf}

\newpage

Using the pair wise interactions the transition rates \(p,
q\) can be measured and the steady state probabilities inferred and compared to
the actual probabilities of each state.
This is done numerically by computing the singular eigenvector of the
matrix \(A\) \cite{Stewart2009}:

\[
    A =
    \begin{bmatrix}
        p_1 q_1 & p_1 (1 - q_1) & (1 - p_1) q_1 & (1 -p_1) (1 - q_1) \\
        p_2 q_2 & p_2 (1 - q_2) & (1 - p_2) q_2 & (1 -p_2) (1 - q_2) \\
        p_3 q_3 & p_3 (1 - q_3) & (1 - p_3) q_3 & (1 -p_3) (1 - q_3) \\
        p_4 q_4 & p_4 (1 - q_4) & (1 - p_4) q_4 & (1 -p_4) (1 - q_4) \\
    \end{bmatrix}
\]

Figure~\ref{fig:computed_probabilities_vs_theoretic_probabilities} shows a
regression line fitted to every pairwise interaction with a reported
\(\text{SSError}\) value (pairwise interactions with missing states were
omitted). This serves to validate the approach: a part from some edge cases the
relationship is consistent.

\begin{figure}[!htbp]
    \centering
    \includegraphics[width=.8\textwidth]{./assets/img/computed_probabilities_vs_theoretic_probabilities/main.pdf}
    \caption{The
        relationship between the steady state probabilities inferred from the
        measured transitions and the actual steady state probabilities. A linear
        regression line is included validating the approach.}
    \label{fig:computed_probabilities_vs_theoretic_probabilities}
\end{figure}


\end{document}

    strategies is considered. In this setting
    the most highly performing strategies do not play in an extortionate way
    against each other but do against lower performing strategies.
    This suggests that whilst the theory of Zero Determinant strategies
    indicates that memory is not of fundamental importance to the evolution of
    cooperative behaviour, this is incomplete.
\end{abstract}

\section{Introduction}\label{sec:introduction}

Agent based game theoretic models have become a stalwart of the underpinning
mathematics of interactive behaviours. One of the major pieces of work
in this area is the pair of original computer tournaments run by Robert
Axelrod~\cite{Axelrod1980, Axelrod1980a}. These tournaments pitted submitted
computer strategies against each other in plays of the Iterated Prisoner's
Dilemma. A common game where agents can choose to pay a slight cost to their
immediate utility in the hope of building a reputation. This has been used in
economic and evolutionary game theory to understand the evolution of cooperative
behaviour.

Recently, a class of strategies was described in~\cite{Press2012} that can
provably extort any given opponent. In~\cite{Hilbe2013, Moran1707} some
questions have already been asked about the true effectiveness of these
strategies in an evolutionary setting. Here another question is asked: is it
possible to recognise this extortionate behaviour? A mathematical procedure for
suspicion is presented: in the same way that the continued actions of an
extortionate individual might raise suspicion.

This work makes use of the Axelrod Python library~\cite{Knight2018, Knight2016}
with a large number of Prisoner Dilemma strategies available to give an
extensive numerical example of the ideas presented.  The approach is presented
in Section~\ref{sec:delta-zd-strategies}.  All of the code and data discussed
in Section~\ref{sec:numerical-experiments} is open sourced, archived and
written according to best scientific principles~\cite{Wilson2014}. The data
archive can be found at~\cite{vincent_knight_2018_1297075}.

\section{Recognising Extortion}\label{sec:delta-zd-strategies}

In~\cite{Press2012}, given a match between 2 memory-one strategies, the concept
of Zero Determinant (ZD) strategies is introduced. The main result of that paper
shows that given two memory one players \(p, q\in\mathbb{R}^4\) a linear
relationship between the players' scores could be forced by one of the players.

Using the notation of~\cite{Press2012}, assuming the utilities for player \(p\)
are given by \(S_x=(R, S, T, P)\) and for player \(q\) by \(S_y=(R, T, S, P)\)
and that the stationary scores of each player is given by \(S_X\) and \(S_Y\)
respectively. The main result of~\cite{Press2012} is that if

\begin{equation}\label{eqn:linear_relationship_for_p}
    \tilde p=\alpha S_x + \beta S_y + \gamma
\end{equation}

or

\begin{equation}\label{eqn:linear_relationship_for_q}
    \tilde q=\alpha S_x + \beta S_y + \gamma
\end{equation}

where \(\tilde p = (1 - p_1, 1 - p_2, p_3, p_4)\) and
\(\tilde q = (1 - q_1, 1 - q_2, q_3, q_4)\) then:

\begin{equation}
    \alpha S_X + \beta S_Y + \gamma = 0
\end{equation}

In~\cite{Press2012} a particular type of ZD strategy is defined: extortionate
strategies. If:

\begin{equation}\label{eqn:constraint_for_extortion}
    \gamma = - P(\alpha + \beta)
\end{equation}

then the player can ensure they get a score \(\chi\) times
larger than the opponent. This extortion coefficient is given by:

\begin{equation}\label{eqn:definition_of_chi}
    \chi=\frac{-\beta}{\alpha}
\end{equation}

Thus, if (\ref{eqn:constraint_for_extortion}) holds and \(\chi >1\) a player is
said to extort their opponent.
Here, the reverse problem is considered: given a
\(p\in\mathbb{R}^4\) how does one identify \(\alpha, \beta\) if they
exist and is the strategy in fact acting in an extortionate way?

These conditions correspond to:

\begin{align}
    \tilde p_1 & = \alpha R + \beta R - P (\alpha + \beta)
            \label{eqn:condition_for_tilde_p1}\\
    \tilde p_2 & = \alpha S + \beta T - P (\alpha + \beta)
            \label{eqn:condition_for_tilde_p2}\\
    \tilde p_3 & = \alpha T + \beta S - P (\alpha + \beta)
            \label{eqn:condition_for_tilde_p3}\\
    \tilde p_4 & = \alpha P + \beta P - P (\alpha + \beta)
            \label{eqn:condition_for_tilde_p4}
\end{align}

Equation (\ref{eqn:condition_for_tilde_p4}) ensures that \(p_4=\tilde p_4=0\).
Equations (\ref{eqn:condition_for_tilde_p1}-\ref{eqn:condition_for_tilde_p3})
can be used to eliminate \(\alpha, \beta\), giving:

\begin{equation}\label{eqn:planar_definition_of_extortion}
    \tilde p_1 = \frac{(R - P)(\tilde p_2 + \tilde p_3)}{S + T - 2P}
\end{equation}

with:

\begin{equation}\label{eqn:definition_of_chi}
    \chi = \frac{\tilde p_2 (P - T) + \tilde p_3 (S - P)}
                {\tilde p_2 (P - S) + \tilde p_3 (T - P)}
\end{equation}

Given a strategy \(p\in\mathbb{R}^{4\times 1}\) equations
(\ref{eqn:condition_for_tilde_p4}), (\ref{eqn:planar_definition_of_extortion}-\ref{eqn:definition_of_chi}) can be used to check if
a strategy is extortionate. The conditions correspond to:

\begin{align}
    p_1 & = \frac{(R-P)(p_2 + p_3) - R + T + S - P}{S + T - 2P}
     \label{eqn:condition_for_p1}\\
    p_4 & = 0 \label{eqn:condition_for_p4}\\
    1 & > p_2 + p_3\label{eqn:condition_for_chi}
\end{align}

The algebraic steps necessary to prove these results are available in the
supporting materials.

All extortionate strategies reside on a triangular (\ref{eqn:condition_for_chi})
plane (\ref{eqn:condition_for_p1}) in 3 dimensions (\ref{eqn:condition_for_p4}).
Using this formulation it can be seen that a necessary (but not sufficient)
condition for an extortionate strategy is that it cooperates on average less
than 50\% of the time when in a state of disagreement with the opponent.

As an example, consider the known extortionate strategy \(p=(8 / 9, 1 / 2, 1 /
3, 0)\) from~\cite{Stewart2012} which is referred to as \texttt{Extort-2}. In
this case, for the standard values of \((R, T, S, P)\) constraint
(\ref{eqn:condition_for_p1}) corresponds to:

\begin{equation}
    p_1 = \frac{2(p_2 + p_3) + 1}{3}
\end{equation}

It is clear that in this case all constraints hold.

This approach could in fact be used to confirm that a given strategy is acting
in an extortionate manner even if it is not a memory one strategy. However, in
practice, if a closed form for \(p\) is not known, then due to measurement
and/or numerical error this would not work.

This problem can be written in the following linear algebraic form where
\(x=(\alpha, \beta)\)
and \(p^*=(\tilde p_1 - 1, tilde_2 - 1, p_3)\):

\begin{equation}\label{eqn:linear_algebraic_equation_for_p}
    Cx= p^*
\end{equation}

\(C\) corresponds to equations
(\ref{eqn:condition_for_tilde_p1}-\ref{eqn:condition_for_tilde_p3}) and is
given by:

\begin{equation}\label{eqn:definition_of_C}
    C =
    \begin{bmatrix}
        R - P & R- P \\
        S - P & T- P \\
        T - P & S- P \\
    \end{bmatrix}
\end{equation}

Note that in general, equation (\ref{eqn:linear_algebraic_equation_for_p}) will
not necessarily have a solution. From the Rouch\'{e}-Capelli theorem if there is
a solution it is unique as \(\text{rank}(C)=2\) which is the dimension of the
variable \(x\). The best fitting \(x\) is found by minimizing:

\begin{equation}\label{eqn:r_squared}
    \text{SSError} = \|C x- p^*\|_2^2 = \sum_{i=1}^{3}\left((C\bar x)_i-p_i^*\right)^2
\end{equation}

Note that \(\text{SSError}\), which is the square of the Frobenius
norm~\cite{Golub2013}, becomes a measure of how close a strategy is to being an
extortionate strategy. Suspicion
of extortion then corresponds to a threshold on \(\text{SSError}\).

By observing interactions (human or otherwise), their memory one representation
can be inferred and this approach can be used to recognise extortionate
behaviour. The notion of comparing theoretic and actual plays of the IPD is not
novel, see for example~\cite{Rand2013}. Immediately it is noted that if the
environment is noisy~\cite{Wu1995} then no strategy can be considered to be
extortionate as \(p_4>0\).

In the next section, this idea will be illustrated by observing the interactions
that take place in a computer based tournament of the IPD\@.

\section{Numerical experiments}\label{sec:numerical-experiments}

In~\cite{Stewart2012} results from a tournament with
\documentclass[a4paper]{article}

\usepackage{amsmath}
\usepackage{amssymb}
\usepackage[margin=1.5cm,
            includefoot,
            footskip=30pt]{geometry}
\usepackage{layout}
\usepackage{graphicx}
\usepackage{subcaption}

\usepackage{biblatex}
\usepackage{pdfpages}

\bibliography{main.bib}

\title{Suspicion: Recognising and evaluating the effectiveness
       of extortion in the Iterated Prisoner's Dilemma}
\author{Vincent A. Knight \and Nikoleta E. Glynatsi}
\date{\today}



\begin{document}

\maketitle

\begin{abstract}
    The Iterated Prisoner's Dilemma is a model for rational and evolutionary
    interactive behaviour. It has applications both in the study of human social
    behaviour as well as in biology.
    It is used to understand when and how a rational individual might
    accept an immediate cost to their own utility for the direct benefit of
    another.

    Much attention has been given to a class of strategies called
    Zero Determinant strategies. It has been theoretically shown that these
    strategies can ``extort'' any player.

    In this work, an approach to identify if observed strategies are playing in
    an extortionate way is described. Furthermore, experimental analysis of
    a large tournament with \input{assets/tex/number_of_full_strategies/main.tex}
    strategies is considered. In this setting
    the most highly performing strategies do not play in an extortionate way
    against each other but do against lower performing strategies.
    This suggests that whilst the theory of Zero Determinant strategies
    indicates that memory is not of fundamental importance to the evolution of
    cooperative behaviour, this is incomplete.
\end{abstract}

\section{Introduction}\label{sec:introduction}

Agent based game theoretic models have become a stalwart of the underpinning
mathematics of interactive behaviours. One of the major pieces of work
in this area is the pair of original computer tournaments run by Robert
Axelrod~\cite{Axelrod1980, Axelrod1980a}. These tournaments pitted submitted
computer strategies against each other in plays of the Iterated Prisoner's
Dilemma. A common game where agents can choose to pay a slight cost to their
immediate utility in the hope of building a reputation. This has been used in
economic and evolutionary game theory to understand the evolution of cooperative
behaviour.

Recently, a class of strategies was described in~\cite{Press2012} that can
provably extort any given opponent. In~\cite{Hilbe2013, Moran1707} some
questions have already been asked about the true effectiveness of these
strategies in an evolutionary setting. Here another question is asked: is it
possible to recognise this extortionate behaviour? A mathematical procedure for
suspicion is presented: in the same way that the continued actions of an
extortionate individual might raise suspicion.

This work makes use of the Axelrod Python library~\cite{Knight2018, Knight2016}
with a large number of Prisoner Dilemma strategies available to give an
extensive numerical example of the ideas presented.  The approach is presented
in Section~\ref{sec:delta-zd-strategies}.  All of the code and data discussed
in Section~\ref{sec:numerical-experiments} is open sourced, archived and
written according to best scientific principles~\cite{Wilson2014}. The data
archive can be found at~\cite{vincent_knight_2018_1297075}.

\section{Recognising Extortion}\label{sec:delta-zd-strategies}

In~\cite{Press2012}, given a match between 2 memory-one strategies, the concept
of Zero Determinant (ZD) strategies is introduced. The main result of that paper
shows that given two memory one players \(p, q\in\mathbb{R}^4\) a linear
relationship between the players' scores could be forced by one of the players.

Using the notation of~\cite{Press2012}, assuming the utilities for player \(p\)
are given by \(S_x=(R, S, T, P)\) and for player \(q\) by \(S_y=(R, T, S, P)\)
and that the stationary scores of each player is given by \(S_X\) and \(S_Y\)
respectively. The main result of~\cite{Press2012} is that if

\begin{equation}\label{eqn:linear_relationship_for_p}
    \tilde p=\alpha S_x + \beta S_y + \gamma
\end{equation}

or

\begin{equation}\label{eqn:linear_relationship_for_q}
    \tilde q=\alpha S_x + \beta S_y + \gamma
\end{equation}

where \(\tilde p = (1 - p_1, 1 - p_2, p_3, p_4)\) and
\(\tilde q = (1 - q_1, 1 - q_2, q_3, q_4)\) then:

\begin{equation}
    \alpha S_X + \beta S_Y + \gamma = 0
\end{equation}

In~\cite{Press2012} a particular type of ZD strategy is defined: extortionate
strategies. If:

\begin{equation}\label{eqn:constraint_for_extortion}
    \gamma = - P(\alpha + \beta)
\end{equation}

then the player can ensure they get a score \(\chi\) times
larger than the opponent. This extortion coefficient is given by:

\begin{equation}\label{eqn:definition_of_chi}
    \chi=\frac{-\beta}{\alpha}
\end{equation}

Thus, if (\ref{eqn:constraint_for_extortion}) holds and \(\chi >1\) a player is
said to extort their opponent.
Here, the reverse problem is considered: given a
\(p\in\mathbb{R}^4\) how does one identify \(\alpha, \beta\) if they
exist and is the strategy in fact acting in an extortionate way?

These conditions correspond to:

\begin{align}
    \tilde p_1 & = \alpha R + \beta R - P (\alpha + \beta)
            \label{eqn:condition_for_tilde_p1}\\
    \tilde p_2 & = \alpha S + \beta T - P (\alpha + \beta)
            \label{eqn:condition_for_tilde_p2}\\
    \tilde p_3 & = \alpha T + \beta S - P (\alpha + \beta)
            \label{eqn:condition_for_tilde_p3}\\
    \tilde p_4 & = \alpha P + \beta P - P (\alpha + \beta)
            \label{eqn:condition_for_tilde_p4}
\end{align}

Equation (\ref{eqn:condition_for_tilde_p4}) ensures that \(p_4=\tilde p_4=0\).
Equations (\ref{eqn:condition_for_tilde_p1}-\ref{eqn:condition_for_tilde_p3})
can be used to eliminate \(\alpha, \beta\), giving:

\begin{equation}\label{eqn:planar_definition_of_extortion}
    \tilde p_1 = \frac{(R - P)(\tilde p_2 + \tilde p_3)}{S + T - 2P}
\end{equation}

with:

\begin{equation}\label{eqn:definition_of_chi}
    \chi = \frac{\tilde p_2 (P - T) + \tilde p_3 (S - P)}
                {\tilde p_2 (P - S) + \tilde p_3 (T - P)}
\end{equation}

Given a strategy \(p\in\mathbb{R}^{4\times 1}\) equations
(\ref{eqn:condition_for_tilde_p4}), (\ref{eqn:planar_definition_of_extortion}-\ref{eqn:definition_of_chi}) can be used to check if
a strategy is extortionate. The conditions correspond to:

\begin{align}
    p_1 & = \frac{(R-P)(p_2 + p_3) - R + T + S - P}{S + T - 2P}
     \label{eqn:condition_for_p1}\\
    p_4 & = 0 \label{eqn:condition_for_p4}\\
    1 & > p_2 + p_3\label{eqn:condition_for_chi}
\end{align}

The algebraic steps necessary to prove these results are available in the
supporting materials.

All extortionate strategies reside on a triangular (\ref{eqn:condition_for_chi})
plane (\ref{eqn:condition_for_p1}) in 3 dimensions (\ref{eqn:condition_for_p4}).
Using this formulation it can be seen that a necessary (but not sufficient)
condition for an extortionate strategy is that it cooperates on average less
than 50\% of the time when in a state of disagreement with the opponent.

As an example, consider the known extortionate strategy \(p=(8 / 9, 1 / 2, 1 /
3, 0)\) from~\cite{Stewart2012} which is referred to as \texttt{Extort-2}. In
this case, for the standard values of \((R, T, S, P)\) constraint
(\ref{eqn:condition_for_p1}) corresponds to:

\begin{equation}
    p_1 = \frac{2(p_2 + p_3) + 1}{3}
\end{equation}

It is clear that in this case all constraints hold.

This approach could in fact be used to confirm that a given strategy is acting
in an extortionate manner even if it is not a memory one strategy. However, in
practice, if a closed form for \(p\) is not known, then due to measurement
and/or numerical error this would not work.

This problem can be written in the following linear algebraic form where
\(x=(\alpha, \beta)\)
and \(p^*=(\tilde p_1 - 1, tilde_2 - 1, p_3)\):

\begin{equation}\label{eqn:linear_algebraic_equation_for_p}
    Cx= p^*
\end{equation}

\(C\) corresponds to equations
(\ref{eqn:condition_for_tilde_p1}-\ref{eqn:condition_for_tilde_p3}) and is
given by:

\begin{equation}\label{eqn:definition_of_C}
    C =
    \begin{bmatrix}
        R - P & R- P \\
        S - P & T- P \\
        T - P & S- P \\
    \end{bmatrix}
\end{equation}

Note that in general, equation (\ref{eqn:linear_algebraic_equation_for_p}) will
not necessarily have a solution. From the Rouch\'{e}-Capelli theorem if there is
a solution it is unique as \(\text{rank}(C)=2\) which is the dimension of the
variable \(x\). The best fitting \(x\) is found by minimizing:

\begin{equation}\label{eqn:r_squared}
    \text{SSError} = \|C x- p^*\|_2^2 = \sum_{i=1}^{3}\left((C\bar x)_i-p_i^*\right)^2
\end{equation}

Note that \(\text{SSError}\), which is the square of the Frobenius
norm~\cite{Golub2013}, becomes a measure of how close a strategy is to being an
extortionate strategy. Suspicion
of extortion then corresponds to a threshold on \(\text{SSError}\).

By observing interactions (human or otherwise), their memory one representation
can be inferred and this approach can be used to recognise extortionate
behaviour. The notion of comparing theoretic and actual plays of the IPD is not
novel, see for example~\cite{Rand2013}. Immediately it is noted that if the
environment is noisy~\cite{Wu1995} then no strategy can be considered to be
extortionate as \(p_4>0\).

In the next section, this idea will be illustrated by observing the interactions
that take place in a computer based tournament of the IPD\@.

\section{Numerical experiments}\label{sec:numerical-experiments}

In~\cite{Stewart2012} results from a tournament with
\input{./assets/tex/number_of_stewart_plotkin_strategies/main.tex} strategies,
was presented with specific consideration given to ZD strategies. This
tournament is reproduced here using the Axelrod-Python
project~\cite{Knight2016}. To obtain a good measure of the corresponding
transition rates for each strategy all matches have been run for
\input{assets/tex/number_of_turns/main.tex} turns and every match has been
repeated \input{assets/tex/number_of_repetitions/main.tex} times. All of this
interaction data is available at~\cite{vincent_knight_2018_1297075}. A good
match between the inferred Markov chain and the state distribution of the actual
interactions has been verified. Data for this is presented in the supplementary
materials.

Figure~\ref{fig:SSError_overall_in_stewart_plotkin} shows the \(\text{SSError}\)
values for all the strategies in the tournament, as reported
in~\cite{Stewart2012} the extortionate strategy (which has an expected
\(\text{SSError}\) approximately 0) gains a large number of wins.

\begin{figure}[!htbp]
    \centering
    \includegraphics[width=.8\textwidth]{./assets/img/SSError_overall_in_stewart_plotkin/main.pdf}
    \caption{\(\text{SSError}\) and state probabilities for the strategies
        of~\cite{Stewart2012}, ordered both by number of wins and overall score.
        Note that \(P(DC)\) is not shown as it corresponds to the transpose of
        \(P(CD)\). Cooperator and Defector are omitted as they do not visit all
        the states.}
    \label{fig:SSError_overall_in_stewart_plotkin}
\end{figure}

Here, the work of~\cite{Stewart2012} is extended by investigating a tournament
with \input{assets/tex/number_of_full_strategies/main.tex}
strategies.

The results of this analysis are shown in
Figure~\ref{fig:SSError_and_probabilities_in_full}. The top ranking strategies
by number of wins seem to be extortionate (but not against all strategies) and
it can be seen that a small sub group of strategies achieve mutual defection.
All the top ranking strategies according to score achieve mutual cooperation and
do not extort each other, however they
\textbf{do} exhibit extortionate behaviour towards a number of the lower ranking
strategies.

\begin{figure}[!htbp]
    \centering
    \includegraphics[width=.8\textwidth]{./assets/img/SSError_and_probabilities_in_full/main.pdf}
    \caption{\(\text{SSError}\) for the strategies for the full tournament. Only
    strategy interactions for which \(p_4=0\) and \(\chi>1\) are displayed.}
    \label{fig:SSError_and_probabilities_in_full}
\end{figure}

\section{Conclusion}\label{sec:conclusion}

This work defines an approach to measure whether or not a player is playing a
strategy that corresponds to an extortionate strategy as defined
in~\cite{Press2012}: a mathematical model for suspicion. Indeed, all
extortionate strategies have been
 classified as lying on a triangular plane.
This rigorous classification fails to be robust to small measurement error, thus
a statistical approach is proposed.
This is done through a linear algebraic approach for approximating the solution
of a linear system. Using this, a large number of pairwise interactions is
simulated and in fact very few strategies are found to act extortionately.

The work of~\cite{Press2012}, whilst showing that a clever approach to taking
advantage of another memory one strategy exists: this is incomplete. Whilst the
elegance of this result is very attractive, just as the simplicity of the
victory of Tit For Tat in Axelrod's original tournaments was, it is incomplete.
Extortionate strategies achieve a high number of wins but they do not
achieve a high score which corresponds to the fitness landscape in an
evolutionary sense. From the large number of interactions a payoff matrix \(S\)
can be measured where \(S_{ij}\) denotes the score (using standard values of
\((R, S, T, P) = (3, 0, 5, 1)\)) of the \(i\)th strategy
against the \(j\)th strategy. Using this, the replicator equation
describes the evolution of the system based on a population density fitness
function:

\begin{equation}\label{eqn:replicator_dynamics}
    \frac{dx}{dt} = x(S-x^TS x)
\end{equation}

Equation (\ref{eqn:replicator_dynamics}) is solved numerically through an
integration technique described in~\cite{Petzold1983} and
Figure~\ref{fig:replicator_dynamics} shows the evolution of the distribution of
the system: the various strategies are ranked by scores. It is clear to see that
only the high ranking strategies survive the evolutionary process (in fact,
only \input{./assets/img/replicator_dynamics/main.tex}
have a final distribution greater than \(10 ^ {-2}\)). This confirms the
findings of~\cite{Moran1707} in which sophisticated strategies resist
evolutionary invasion of shorter memory strategies. Recalling
Figure~\ref{fig:SSError_and_probabilities_in_full} this demonstrates that:

\begin{itemize}
    \item Cooperation emerges through the evolutionary process: the high scoring
        strategies do not exhibit extortionate behaviour towards each other.
    \item Extortionate strategies do not survive the evolutionary process.
\end{itemize}

\begin{figure}[!htbp]
    \centering
    \includegraphics[width=.8\textwidth]{./assets/img/replicator_dynamics/main.pdf}
    \caption{Numerical simulation of the replicator equation
    (\ref{eqn:replicator_dynamics}): strategies are ordered by score, only the strategies with a high score survive the evolutionary process.}
    \label{fig:replicator_dynamics}
\end{figure}

This work can be used to classify plays of the IPD\@: data can be collected from
actual interactions (in lab or in the field). Furthermore, this allows for a
classification method similar to the notion of fingerprinting presented
in~\cite{Ashlock2008}. Trained strategies can potentially be classified as
extortionate or not or it could be possible to even constrain the reinforcement
learning approaches that are becoming prevalent in the literature.
Alternatively, this mathematical approach for recognising extortion could be
used in sophisticated strategies to defend against invasion. Arguably, some of
the strategies considered here exhibit this behaviour, indeed as described
in~\cite{Harper2017}, the top ranking strategies in the full tournament are
obtained using evolutionary reinforcement learning techniques, thus, suspicion
of extortionate behaviour could in fact be an evolutionary trait.

\section*{Acknowledgements}

The following open source software libraries were used in this research:

\begin{itemize}
    \item The Axelrod ~\cite{Knight2016, Knight2018} library (IPD strategies and
        tournaments).
    \item The sympy library~\cite{Meurer2017} (verification of all symbolic
        calculations).
    \item The matplotlib~\cite{Droettboom2018} library (visualisation).
    \item The pandas~\cite{Structures2010}, dask~\cite{Dask2016} and
        NumPy~\cite{Oliphant2015} libraries (data manipulation).
    \item The SciPy~\cite{Jones2001} library (numerical integration of the
        replicator equation).
\end{itemize}

This work was performed using the computational facilities of the Advanced
Research Computing @ Cardiff (ARCCA) Division, Cardiff University.

\printbibliography

\newpage
\section*{Supplementary materials}

\includepdf{assets/pdf/proof_of_form_of_extortionate_strategies/main.pdf}

\newpage

Using the pair wise interactions the transition rates \(p,
q\) can be measured and the steady state probabilities inferred and compared to
the actual probabilities of each state.
This is done numerically by computing the singular eigenvector of the
matrix \(A\) \cite{Stewart2009}:

\[
    A =
    \begin{bmatrix}
        p_1 q_1 & p_1 (1 - q_1) & (1 - p_1) q_1 & (1 -p_1) (1 - q_1) \\
        p_2 q_2 & p_2 (1 - q_2) & (1 - p_2) q_2 & (1 -p_2) (1 - q_2) \\
        p_3 q_3 & p_3 (1 - q_3) & (1 - p_3) q_3 & (1 -p_3) (1 - q_3) \\
        p_4 q_4 & p_4 (1 - q_4) & (1 - p_4) q_4 & (1 -p_4) (1 - q_4) \\
    \end{bmatrix}
\]

Figure~\ref{fig:computed_probabilities_vs_theoretic_probabilities} shows a
regression line fitted to every pairwise interaction with a reported
\(\text{SSError}\) value (pairwise interactions with missing states were
omitted). This serves to validate the approach: a part from some edge cases the
relationship is consistent.

\begin{figure}[!htbp]
    \centering
    \includegraphics[width=.8\textwidth]{./assets/img/computed_probabilities_vs_theoretic_probabilities/main.pdf}
    \caption{The
        relationship between the steady state probabilities inferred from the
        measured transitions and the actual steady state probabilities. A linear
        regression line is included validating the approach.}
    \label{fig:computed_probabilities_vs_theoretic_probabilities}
\end{figure}


\end{document}
 strategies,
was presented with specific consideration given to ZD strategies. This
tournament is reproduced here using the Axelrod-Python
project~\cite{Knight2016}. To obtain a good measure of the corresponding
transition rates for each strategy all matches have been run for
\documentclass[a4paper]{article}

\usepackage{amsmath}
\usepackage{amssymb}
\usepackage[margin=1.5cm,
            includefoot,
            footskip=30pt]{geometry}
\usepackage{layout}
\usepackage{graphicx}
\usepackage{subcaption}

\usepackage{biblatex}
\usepackage{pdfpages}

\bibliography{main.bib}

\title{Suspicion: Recognising and evaluating the effectiveness
       of extortion in the Iterated Prisoner's Dilemma}
\author{Vincent A. Knight \and Nikoleta E. Glynatsi}
\date{\today}



\begin{document}

\maketitle

\begin{abstract}
    The Iterated Prisoner's Dilemma is a model for rational and evolutionary
    interactive behaviour. It has applications both in the study of human social
    behaviour as well as in biology.
    It is used to understand when and how a rational individual might
    accept an immediate cost to their own utility for the direct benefit of
    another.

    Much attention has been given to a class of strategies called
    Zero Determinant strategies. It has been theoretically shown that these
    strategies can ``extort'' any player.

    In this work, an approach to identify if observed strategies are playing in
    an extortionate way is described. Furthermore, experimental analysis of
    a large tournament with \input{assets/tex/number_of_full_strategies/main.tex}
    strategies is considered. In this setting
    the most highly performing strategies do not play in an extortionate way
    against each other but do against lower performing strategies.
    This suggests that whilst the theory of Zero Determinant strategies
    indicates that memory is not of fundamental importance to the evolution of
    cooperative behaviour, this is incomplete.
\end{abstract}

\section{Introduction}\label{sec:introduction}

Agent based game theoretic models have become a stalwart of the underpinning
mathematics of interactive behaviours. One of the major pieces of work
in this area is the pair of original computer tournaments run by Robert
Axelrod~\cite{Axelrod1980, Axelrod1980a}. These tournaments pitted submitted
computer strategies against each other in plays of the Iterated Prisoner's
Dilemma. A common game where agents can choose to pay a slight cost to their
immediate utility in the hope of building a reputation. This has been used in
economic and evolutionary game theory to understand the evolution of cooperative
behaviour.

Recently, a class of strategies was described in~\cite{Press2012} that can
provably extort any given opponent. In~\cite{Hilbe2013, Moran1707} some
questions have already been asked about the true effectiveness of these
strategies in an evolutionary setting. Here another question is asked: is it
possible to recognise this extortionate behaviour? A mathematical procedure for
suspicion is presented: in the same way that the continued actions of an
extortionate individual might raise suspicion.

This work makes use of the Axelrod Python library~\cite{Knight2018, Knight2016}
with a large number of Prisoner Dilemma strategies available to give an
extensive numerical example of the ideas presented.  The approach is presented
in Section~\ref{sec:delta-zd-strategies}.  All of the code and data discussed
in Section~\ref{sec:numerical-experiments} is open sourced, archived and
written according to best scientific principles~\cite{Wilson2014}. The data
archive can be found at~\cite{vincent_knight_2018_1297075}.

\section{Recognising Extortion}\label{sec:delta-zd-strategies}

In~\cite{Press2012}, given a match between 2 memory-one strategies, the concept
of Zero Determinant (ZD) strategies is introduced. The main result of that paper
shows that given two memory one players \(p, q\in\mathbb{R}^4\) a linear
relationship between the players' scores could be forced by one of the players.

Using the notation of~\cite{Press2012}, assuming the utilities for player \(p\)
are given by \(S_x=(R, S, T, P)\) and for player \(q\) by \(S_y=(R, T, S, P)\)
and that the stationary scores of each player is given by \(S_X\) and \(S_Y\)
respectively. The main result of~\cite{Press2012} is that if

\begin{equation}\label{eqn:linear_relationship_for_p}
    \tilde p=\alpha S_x + \beta S_y + \gamma
\end{equation}

or

\begin{equation}\label{eqn:linear_relationship_for_q}
    \tilde q=\alpha S_x + \beta S_y + \gamma
\end{equation}

where \(\tilde p = (1 - p_1, 1 - p_2, p_3, p_4)\) and
\(\tilde q = (1 - q_1, 1 - q_2, q_3, q_4)\) then:

\begin{equation}
    \alpha S_X + \beta S_Y + \gamma = 0
\end{equation}

In~\cite{Press2012} a particular type of ZD strategy is defined: extortionate
strategies. If:

\begin{equation}\label{eqn:constraint_for_extortion}
    \gamma = - P(\alpha + \beta)
\end{equation}

then the player can ensure they get a score \(\chi\) times
larger than the opponent. This extortion coefficient is given by:

\begin{equation}\label{eqn:definition_of_chi}
    \chi=\frac{-\beta}{\alpha}
\end{equation}

Thus, if (\ref{eqn:constraint_for_extortion}) holds and \(\chi >1\) a player is
said to extort their opponent.
Here, the reverse problem is considered: given a
\(p\in\mathbb{R}^4\) how does one identify \(\alpha, \beta\) if they
exist and is the strategy in fact acting in an extortionate way?

These conditions correspond to:

\begin{align}
    \tilde p_1 & = \alpha R + \beta R - P (\alpha + \beta)
            \label{eqn:condition_for_tilde_p1}\\
    \tilde p_2 & = \alpha S + \beta T - P (\alpha + \beta)
            \label{eqn:condition_for_tilde_p2}\\
    \tilde p_3 & = \alpha T + \beta S - P (\alpha + \beta)
            \label{eqn:condition_for_tilde_p3}\\
    \tilde p_4 & = \alpha P + \beta P - P (\alpha + \beta)
            \label{eqn:condition_for_tilde_p4}
\end{align}

Equation (\ref{eqn:condition_for_tilde_p4}) ensures that \(p_4=\tilde p_4=0\).
Equations (\ref{eqn:condition_for_tilde_p1}-\ref{eqn:condition_for_tilde_p3})
can be used to eliminate \(\alpha, \beta\), giving:

\begin{equation}\label{eqn:planar_definition_of_extortion}
    \tilde p_1 = \frac{(R - P)(\tilde p_2 + \tilde p_3)}{S + T - 2P}
\end{equation}

with:

\begin{equation}\label{eqn:definition_of_chi}
    \chi = \frac{\tilde p_2 (P - T) + \tilde p_3 (S - P)}
                {\tilde p_2 (P - S) + \tilde p_3 (T - P)}
\end{equation}

Given a strategy \(p\in\mathbb{R}^{4\times 1}\) equations
(\ref{eqn:condition_for_tilde_p4}), (\ref{eqn:planar_definition_of_extortion}-\ref{eqn:definition_of_chi}) can be used to check if
a strategy is extortionate. The conditions correspond to:

\begin{align}
    p_1 & = \frac{(R-P)(p_2 + p_3) - R + T + S - P}{S + T - 2P}
     \label{eqn:condition_for_p1}\\
    p_4 & = 0 \label{eqn:condition_for_p4}\\
    1 & > p_2 + p_3\label{eqn:condition_for_chi}
\end{align}

The algebraic steps necessary to prove these results are available in the
supporting materials.

All extortionate strategies reside on a triangular (\ref{eqn:condition_for_chi})
plane (\ref{eqn:condition_for_p1}) in 3 dimensions (\ref{eqn:condition_for_p4}).
Using this formulation it can be seen that a necessary (but not sufficient)
condition for an extortionate strategy is that it cooperates on average less
than 50\% of the time when in a state of disagreement with the opponent.

As an example, consider the known extortionate strategy \(p=(8 / 9, 1 / 2, 1 /
3, 0)\) from~\cite{Stewart2012} which is referred to as \texttt{Extort-2}. In
this case, for the standard values of \((R, T, S, P)\) constraint
(\ref{eqn:condition_for_p1}) corresponds to:

\begin{equation}
    p_1 = \frac{2(p_2 + p_3) + 1}{3}
\end{equation}

It is clear that in this case all constraints hold.

This approach could in fact be used to confirm that a given strategy is acting
in an extortionate manner even if it is not a memory one strategy. However, in
practice, if a closed form for \(p\) is not known, then due to measurement
and/or numerical error this would not work.

This problem can be written in the following linear algebraic form where
\(x=(\alpha, \beta)\)
and \(p^*=(\tilde p_1 - 1, tilde_2 - 1, p_3)\):

\begin{equation}\label{eqn:linear_algebraic_equation_for_p}
    Cx= p^*
\end{equation}

\(C\) corresponds to equations
(\ref{eqn:condition_for_tilde_p1}-\ref{eqn:condition_for_tilde_p3}) and is
given by:

\begin{equation}\label{eqn:definition_of_C}
    C =
    \begin{bmatrix}
        R - P & R- P \\
        S - P & T- P \\
        T - P & S- P \\
    \end{bmatrix}
\end{equation}

Note that in general, equation (\ref{eqn:linear_algebraic_equation_for_p}) will
not necessarily have a solution. From the Rouch\'{e}-Capelli theorem if there is
a solution it is unique as \(\text{rank}(C)=2\) which is the dimension of the
variable \(x\). The best fitting \(x\) is found by minimizing:

\begin{equation}\label{eqn:r_squared}
    \text{SSError} = \|C x- p^*\|_2^2 = \sum_{i=1}^{3}\left((C\bar x)_i-p_i^*\right)^2
\end{equation}

Note that \(\text{SSError}\), which is the square of the Frobenius
norm~\cite{Golub2013}, becomes a measure of how close a strategy is to being an
extortionate strategy. Suspicion
of extortion then corresponds to a threshold on \(\text{SSError}\).

By observing interactions (human or otherwise), their memory one representation
can be inferred and this approach can be used to recognise extortionate
behaviour. The notion of comparing theoretic and actual plays of the IPD is not
novel, see for example~\cite{Rand2013}. Immediately it is noted that if the
environment is noisy~\cite{Wu1995} then no strategy can be considered to be
extortionate as \(p_4>0\).

In the next section, this idea will be illustrated by observing the interactions
that take place in a computer based tournament of the IPD\@.

\section{Numerical experiments}\label{sec:numerical-experiments}

In~\cite{Stewart2012} results from a tournament with
\input{./assets/tex/number_of_stewart_plotkin_strategies/main.tex} strategies,
was presented with specific consideration given to ZD strategies. This
tournament is reproduced here using the Axelrod-Python
project~\cite{Knight2016}. To obtain a good measure of the corresponding
transition rates for each strategy all matches have been run for
\input{assets/tex/number_of_turns/main.tex} turns and every match has been
repeated \input{assets/tex/number_of_repetitions/main.tex} times. All of this
interaction data is available at~\cite{vincent_knight_2018_1297075}. A good
match between the inferred Markov chain and the state distribution of the actual
interactions has been verified. Data for this is presented in the supplementary
materials.

Figure~\ref{fig:SSError_overall_in_stewart_plotkin} shows the \(\text{SSError}\)
values for all the strategies in the tournament, as reported
in~\cite{Stewart2012} the extortionate strategy (which has an expected
\(\text{SSError}\) approximately 0) gains a large number of wins.

\begin{figure}[!htbp]
    \centering
    \includegraphics[width=.8\textwidth]{./assets/img/SSError_overall_in_stewart_plotkin/main.pdf}
    \caption{\(\text{SSError}\) and state probabilities for the strategies
        of~\cite{Stewart2012}, ordered both by number of wins and overall score.
        Note that \(P(DC)\) is not shown as it corresponds to the transpose of
        \(P(CD)\). Cooperator and Defector are omitted as they do not visit all
        the states.}
    \label{fig:SSError_overall_in_stewart_plotkin}
\end{figure}

Here, the work of~\cite{Stewart2012} is extended by investigating a tournament
with \input{assets/tex/number_of_full_strategies/main.tex}
strategies.

The results of this analysis are shown in
Figure~\ref{fig:SSError_and_probabilities_in_full}. The top ranking strategies
by number of wins seem to be extortionate (but not against all strategies) and
it can be seen that a small sub group of strategies achieve mutual defection.
All the top ranking strategies according to score achieve mutual cooperation and
do not extort each other, however they
\textbf{do} exhibit extortionate behaviour towards a number of the lower ranking
strategies.

\begin{figure}[!htbp]
    \centering
    \includegraphics[width=.8\textwidth]{./assets/img/SSError_and_probabilities_in_full/main.pdf}
    \caption{\(\text{SSError}\) for the strategies for the full tournament. Only
    strategy interactions for which \(p_4=0\) and \(\chi>1\) are displayed.}
    \label{fig:SSError_and_probabilities_in_full}
\end{figure}

\section{Conclusion}\label{sec:conclusion}

This work defines an approach to measure whether or not a player is playing a
strategy that corresponds to an extortionate strategy as defined
in~\cite{Press2012}: a mathematical model for suspicion. Indeed, all
extortionate strategies have been
 classified as lying on a triangular plane.
This rigorous classification fails to be robust to small measurement error, thus
a statistical approach is proposed.
This is done through a linear algebraic approach for approximating the solution
of a linear system. Using this, a large number of pairwise interactions is
simulated and in fact very few strategies are found to act extortionately.

The work of~\cite{Press2012}, whilst showing that a clever approach to taking
advantage of another memory one strategy exists: this is incomplete. Whilst the
elegance of this result is very attractive, just as the simplicity of the
victory of Tit For Tat in Axelrod's original tournaments was, it is incomplete.
Extortionate strategies achieve a high number of wins but they do not
achieve a high score which corresponds to the fitness landscape in an
evolutionary sense. From the large number of interactions a payoff matrix \(S\)
can be measured where \(S_{ij}\) denotes the score (using standard values of
\((R, S, T, P) = (3, 0, 5, 1)\)) of the \(i\)th strategy
against the \(j\)th strategy. Using this, the replicator equation
describes the evolution of the system based on a population density fitness
function:

\begin{equation}\label{eqn:replicator_dynamics}
    \frac{dx}{dt} = x(S-x^TS x)
\end{equation}

Equation (\ref{eqn:replicator_dynamics}) is solved numerically through an
integration technique described in~\cite{Petzold1983} and
Figure~\ref{fig:replicator_dynamics} shows the evolution of the distribution of
the system: the various strategies are ranked by scores. It is clear to see that
only the high ranking strategies survive the evolutionary process (in fact,
only \input{./assets/img/replicator_dynamics/main.tex}
have a final distribution greater than \(10 ^ {-2}\)). This confirms the
findings of~\cite{Moran1707} in which sophisticated strategies resist
evolutionary invasion of shorter memory strategies. Recalling
Figure~\ref{fig:SSError_and_probabilities_in_full} this demonstrates that:

\begin{itemize}
    \item Cooperation emerges through the evolutionary process: the high scoring
        strategies do not exhibit extortionate behaviour towards each other.
    \item Extortionate strategies do not survive the evolutionary process.
\end{itemize}

\begin{figure}[!htbp]
    \centering
    \includegraphics[width=.8\textwidth]{./assets/img/replicator_dynamics/main.pdf}
    \caption{Numerical simulation of the replicator equation
    (\ref{eqn:replicator_dynamics}): strategies are ordered by score, only the strategies with a high score survive the evolutionary process.}
    \label{fig:replicator_dynamics}
\end{figure}

This work can be used to classify plays of the IPD\@: data can be collected from
actual interactions (in lab or in the field). Furthermore, this allows for a
classification method similar to the notion of fingerprinting presented
in~\cite{Ashlock2008}. Trained strategies can potentially be classified as
extortionate or not or it could be possible to even constrain the reinforcement
learning approaches that are becoming prevalent in the literature.
Alternatively, this mathematical approach for recognising extortion could be
used in sophisticated strategies to defend against invasion. Arguably, some of
the strategies considered here exhibit this behaviour, indeed as described
in~\cite{Harper2017}, the top ranking strategies in the full tournament are
obtained using evolutionary reinforcement learning techniques, thus, suspicion
of extortionate behaviour could in fact be an evolutionary trait.

\section*{Acknowledgements}

The following open source software libraries were used in this research:

\begin{itemize}
    \item The Axelrod ~\cite{Knight2016, Knight2018} library (IPD strategies and
        tournaments).
    \item The sympy library~\cite{Meurer2017} (verification of all symbolic
        calculations).
    \item The matplotlib~\cite{Droettboom2018} library (visualisation).
    \item The pandas~\cite{Structures2010}, dask~\cite{Dask2016} and
        NumPy~\cite{Oliphant2015} libraries (data manipulation).
    \item The SciPy~\cite{Jones2001} library (numerical integration of the
        replicator equation).
\end{itemize}

This work was performed using the computational facilities of the Advanced
Research Computing @ Cardiff (ARCCA) Division, Cardiff University.

\printbibliography

\newpage
\section*{Supplementary materials}

\includepdf{assets/pdf/proof_of_form_of_extortionate_strategies/main.pdf}

\newpage

Using the pair wise interactions the transition rates \(p,
q\) can be measured and the steady state probabilities inferred and compared to
the actual probabilities of each state.
This is done numerically by computing the singular eigenvector of the
matrix \(A\) \cite{Stewart2009}:

\[
    A =
    \begin{bmatrix}
        p_1 q_1 & p_1 (1 - q_1) & (1 - p_1) q_1 & (1 -p_1) (1 - q_1) \\
        p_2 q_2 & p_2 (1 - q_2) & (1 - p_2) q_2 & (1 -p_2) (1 - q_2) \\
        p_3 q_3 & p_3 (1 - q_3) & (1 - p_3) q_3 & (1 -p_3) (1 - q_3) \\
        p_4 q_4 & p_4 (1 - q_4) & (1 - p_4) q_4 & (1 -p_4) (1 - q_4) \\
    \end{bmatrix}
\]

Figure~\ref{fig:computed_probabilities_vs_theoretic_probabilities} shows a
regression line fitted to every pairwise interaction with a reported
\(\text{SSError}\) value (pairwise interactions with missing states were
omitted). This serves to validate the approach: a part from some edge cases the
relationship is consistent.

\begin{figure}[!htbp]
    \centering
    \includegraphics[width=.8\textwidth]{./assets/img/computed_probabilities_vs_theoretic_probabilities/main.pdf}
    \caption{The
        relationship between the steady state probabilities inferred from the
        measured transitions and the actual steady state probabilities. A linear
        regression line is included validating the approach.}
    \label{fig:computed_probabilities_vs_theoretic_probabilities}
\end{figure}


\end{document}
 turns and every match has been
repeated \documentclass[a4paper]{article}

\usepackage{amsmath}
\usepackage{amssymb}
\usepackage[margin=1.5cm,
            includefoot,
            footskip=30pt]{geometry}
\usepackage{layout}
\usepackage{graphicx}
\usepackage{subcaption}

\usepackage{biblatex}
\usepackage{pdfpages}

\bibliography{main.bib}

\title{Suspicion: Recognising and evaluating the effectiveness
       of extortion in the Iterated Prisoner's Dilemma}
\author{Vincent A. Knight \and Nikoleta E. Glynatsi}
\date{\today}



\begin{document}

\maketitle

\begin{abstract}
    The Iterated Prisoner's Dilemma is a model for rational and evolutionary
    interactive behaviour. It has applications both in the study of human social
    behaviour as well as in biology.
    It is used to understand when and how a rational individual might
    accept an immediate cost to their own utility for the direct benefit of
    another.

    Much attention has been given to a class of strategies called
    Zero Determinant strategies. It has been theoretically shown that these
    strategies can ``extort'' any player.

    In this work, an approach to identify if observed strategies are playing in
    an extortionate way is described. Furthermore, experimental analysis of
    a large tournament with \input{assets/tex/number_of_full_strategies/main.tex}
    strategies is considered. In this setting
    the most highly performing strategies do not play in an extortionate way
    against each other but do against lower performing strategies.
    This suggests that whilst the theory of Zero Determinant strategies
    indicates that memory is not of fundamental importance to the evolution of
    cooperative behaviour, this is incomplete.
\end{abstract}

\section{Introduction}\label{sec:introduction}

Agent based game theoretic models have become a stalwart of the underpinning
mathematics of interactive behaviours. One of the major pieces of work
in this area is the pair of original computer tournaments run by Robert
Axelrod~\cite{Axelrod1980, Axelrod1980a}. These tournaments pitted submitted
computer strategies against each other in plays of the Iterated Prisoner's
Dilemma. A common game where agents can choose to pay a slight cost to their
immediate utility in the hope of building a reputation. This has been used in
economic and evolutionary game theory to understand the evolution of cooperative
behaviour.

Recently, a class of strategies was described in~\cite{Press2012} that can
provably extort any given opponent. In~\cite{Hilbe2013, Moran1707} some
questions have already been asked about the true effectiveness of these
strategies in an evolutionary setting. Here another question is asked: is it
possible to recognise this extortionate behaviour? A mathematical procedure for
suspicion is presented: in the same way that the continued actions of an
extortionate individual might raise suspicion.

This work makes use of the Axelrod Python library~\cite{Knight2018, Knight2016}
with a large number of Prisoner Dilemma strategies available to give an
extensive numerical example of the ideas presented.  The approach is presented
in Section~\ref{sec:delta-zd-strategies}.  All of the code and data discussed
in Section~\ref{sec:numerical-experiments} is open sourced, archived and
written according to best scientific principles~\cite{Wilson2014}. The data
archive can be found at~\cite{vincent_knight_2018_1297075}.

\section{Recognising Extortion}\label{sec:delta-zd-strategies}

In~\cite{Press2012}, given a match between 2 memory-one strategies, the concept
of Zero Determinant (ZD) strategies is introduced. The main result of that paper
shows that given two memory one players \(p, q\in\mathbb{R}^4\) a linear
relationship between the players' scores could be forced by one of the players.

Using the notation of~\cite{Press2012}, assuming the utilities for player \(p\)
are given by \(S_x=(R, S, T, P)\) and for player \(q\) by \(S_y=(R, T, S, P)\)
and that the stationary scores of each player is given by \(S_X\) and \(S_Y\)
respectively. The main result of~\cite{Press2012} is that if

\begin{equation}\label{eqn:linear_relationship_for_p}
    \tilde p=\alpha S_x + \beta S_y + \gamma
\end{equation}

or

\begin{equation}\label{eqn:linear_relationship_for_q}
    \tilde q=\alpha S_x + \beta S_y + \gamma
\end{equation}

where \(\tilde p = (1 - p_1, 1 - p_2, p_3, p_4)\) and
\(\tilde q = (1 - q_1, 1 - q_2, q_3, q_4)\) then:

\begin{equation}
    \alpha S_X + \beta S_Y + \gamma = 0
\end{equation}

In~\cite{Press2012} a particular type of ZD strategy is defined: extortionate
strategies. If:

\begin{equation}\label{eqn:constraint_for_extortion}
    \gamma = - P(\alpha + \beta)
\end{equation}

then the player can ensure they get a score \(\chi\) times
larger than the opponent. This extortion coefficient is given by:

\begin{equation}\label{eqn:definition_of_chi}
    \chi=\frac{-\beta}{\alpha}
\end{equation}

Thus, if (\ref{eqn:constraint_for_extortion}) holds and \(\chi >1\) a player is
said to extort their opponent.
Here, the reverse problem is considered: given a
\(p\in\mathbb{R}^4\) how does one identify \(\alpha, \beta\) if they
exist and is the strategy in fact acting in an extortionate way?

These conditions correspond to:

\begin{align}
    \tilde p_1 & = \alpha R + \beta R - P (\alpha + \beta)
            \label{eqn:condition_for_tilde_p1}\\
    \tilde p_2 & = \alpha S + \beta T - P (\alpha + \beta)
            \label{eqn:condition_for_tilde_p2}\\
    \tilde p_3 & = \alpha T + \beta S - P (\alpha + \beta)
            \label{eqn:condition_for_tilde_p3}\\
    \tilde p_4 & = \alpha P + \beta P - P (\alpha + \beta)
            \label{eqn:condition_for_tilde_p4}
\end{align}

Equation (\ref{eqn:condition_for_tilde_p4}) ensures that \(p_4=\tilde p_4=0\).
Equations (\ref{eqn:condition_for_tilde_p1}-\ref{eqn:condition_for_tilde_p3})
can be used to eliminate \(\alpha, \beta\), giving:

\begin{equation}\label{eqn:planar_definition_of_extortion}
    \tilde p_1 = \frac{(R - P)(\tilde p_2 + \tilde p_3)}{S + T - 2P}
\end{equation}

with:

\begin{equation}\label{eqn:definition_of_chi}
    \chi = \frac{\tilde p_2 (P - T) + \tilde p_3 (S - P)}
                {\tilde p_2 (P - S) + \tilde p_3 (T - P)}
\end{equation}

Given a strategy \(p\in\mathbb{R}^{4\times 1}\) equations
(\ref{eqn:condition_for_tilde_p4}), (\ref{eqn:planar_definition_of_extortion}-\ref{eqn:definition_of_chi}) can be used to check if
a strategy is extortionate. The conditions correspond to:

\begin{align}
    p_1 & = \frac{(R-P)(p_2 + p_3) - R + T + S - P}{S + T - 2P}
     \label{eqn:condition_for_p1}\\
    p_4 & = 0 \label{eqn:condition_for_p4}\\
    1 & > p_2 + p_3\label{eqn:condition_for_chi}
\end{align}

The algebraic steps necessary to prove these results are available in the
supporting materials.

All extortionate strategies reside on a triangular (\ref{eqn:condition_for_chi})
plane (\ref{eqn:condition_for_p1}) in 3 dimensions (\ref{eqn:condition_for_p4}).
Using this formulation it can be seen that a necessary (but not sufficient)
condition for an extortionate strategy is that it cooperates on average less
than 50\% of the time when in a state of disagreement with the opponent.

As an example, consider the known extortionate strategy \(p=(8 / 9, 1 / 2, 1 /
3, 0)\) from~\cite{Stewart2012} which is referred to as \texttt{Extort-2}. In
this case, for the standard values of \((R, T, S, P)\) constraint
(\ref{eqn:condition_for_p1}) corresponds to:

\begin{equation}
    p_1 = \frac{2(p_2 + p_3) + 1}{3}
\end{equation}

It is clear that in this case all constraints hold.

This approach could in fact be used to confirm that a given strategy is acting
in an extortionate manner even if it is not a memory one strategy. However, in
practice, if a closed form for \(p\) is not known, then due to measurement
and/or numerical error this would not work.

This problem can be written in the following linear algebraic form where
\(x=(\alpha, \beta)\)
and \(p^*=(\tilde p_1 - 1, tilde_2 - 1, p_3)\):

\begin{equation}\label{eqn:linear_algebraic_equation_for_p}
    Cx= p^*
\end{equation}

\(C\) corresponds to equations
(\ref{eqn:condition_for_tilde_p1}-\ref{eqn:condition_for_tilde_p3}) and is
given by:

\begin{equation}\label{eqn:definition_of_C}
    C =
    \begin{bmatrix}
        R - P & R- P \\
        S - P & T- P \\
        T - P & S- P \\
    \end{bmatrix}
\end{equation}

Note that in general, equation (\ref{eqn:linear_algebraic_equation_for_p}) will
not necessarily have a solution. From the Rouch\'{e}-Capelli theorem if there is
a solution it is unique as \(\text{rank}(C)=2\) which is the dimension of the
variable \(x\). The best fitting \(x\) is found by minimizing:

\begin{equation}\label{eqn:r_squared}
    \text{SSError} = \|C x- p^*\|_2^2 = \sum_{i=1}^{3}\left((C\bar x)_i-p_i^*\right)^2
\end{equation}

Note that \(\text{SSError}\), which is the square of the Frobenius
norm~\cite{Golub2013}, becomes a measure of how close a strategy is to being an
extortionate strategy. Suspicion
of extortion then corresponds to a threshold on \(\text{SSError}\).

By observing interactions (human or otherwise), their memory one representation
can be inferred and this approach can be used to recognise extortionate
behaviour. The notion of comparing theoretic and actual plays of the IPD is not
novel, see for example~\cite{Rand2013}. Immediately it is noted that if the
environment is noisy~\cite{Wu1995} then no strategy can be considered to be
extortionate as \(p_4>0\).

In the next section, this idea will be illustrated by observing the interactions
that take place in a computer based tournament of the IPD\@.

\section{Numerical experiments}\label{sec:numerical-experiments}

In~\cite{Stewart2012} results from a tournament with
\input{./assets/tex/number_of_stewart_plotkin_strategies/main.tex} strategies,
was presented with specific consideration given to ZD strategies. This
tournament is reproduced here using the Axelrod-Python
project~\cite{Knight2016}. To obtain a good measure of the corresponding
transition rates for each strategy all matches have been run for
\input{assets/tex/number_of_turns/main.tex} turns and every match has been
repeated \input{assets/tex/number_of_repetitions/main.tex} times. All of this
interaction data is available at~\cite{vincent_knight_2018_1297075}. A good
match between the inferred Markov chain and the state distribution of the actual
interactions has been verified. Data for this is presented in the supplementary
materials.

Figure~\ref{fig:SSError_overall_in_stewart_plotkin} shows the \(\text{SSError}\)
values for all the strategies in the tournament, as reported
in~\cite{Stewart2012} the extortionate strategy (which has an expected
\(\text{SSError}\) approximately 0) gains a large number of wins.

\begin{figure}[!htbp]
    \centering
    \includegraphics[width=.8\textwidth]{./assets/img/SSError_overall_in_stewart_plotkin/main.pdf}
    \caption{\(\text{SSError}\) and state probabilities for the strategies
        of~\cite{Stewart2012}, ordered both by number of wins and overall score.
        Note that \(P(DC)\) is not shown as it corresponds to the transpose of
        \(P(CD)\). Cooperator and Defector are omitted as they do not visit all
        the states.}
    \label{fig:SSError_overall_in_stewart_plotkin}
\end{figure}

Here, the work of~\cite{Stewart2012} is extended by investigating a tournament
with \input{assets/tex/number_of_full_strategies/main.tex}
strategies.

The results of this analysis are shown in
Figure~\ref{fig:SSError_and_probabilities_in_full}. The top ranking strategies
by number of wins seem to be extortionate (but not against all strategies) and
it can be seen that a small sub group of strategies achieve mutual defection.
All the top ranking strategies according to score achieve mutual cooperation and
do not extort each other, however they
\textbf{do} exhibit extortionate behaviour towards a number of the lower ranking
strategies.

\begin{figure}[!htbp]
    \centering
    \includegraphics[width=.8\textwidth]{./assets/img/SSError_and_probabilities_in_full/main.pdf}
    \caption{\(\text{SSError}\) for the strategies for the full tournament. Only
    strategy interactions for which \(p_4=0\) and \(\chi>1\) are displayed.}
    \label{fig:SSError_and_probabilities_in_full}
\end{figure}

\section{Conclusion}\label{sec:conclusion}

This work defines an approach to measure whether or not a player is playing a
strategy that corresponds to an extortionate strategy as defined
in~\cite{Press2012}: a mathematical model for suspicion. Indeed, all
extortionate strategies have been
 classified as lying on a triangular plane.
This rigorous classification fails to be robust to small measurement error, thus
a statistical approach is proposed.
This is done through a linear algebraic approach for approximating the solution
of a linear system. Using this, a large number of pairwise interactions is
simulated and in fact very few strategies are found to act extortionately.

The work of~\cite{Press2012}, whilst showing that a clever approach to taking
advantage of another memory one strategy exists: this is incomplete. Whilst the
elegance of this result is very attractive, just as the simplicity of the
victory of Tit For Tat in Axelrod's original tournaments was, it is incomplete.
Extortionate strategies achieve a high number of wins but they do not
achieve a high score which corresponds to the fitness landscape in an
evolutionary sense. From the large number of interactions a payoff matrix \(S\)
can be measured where \(S_{ij}\) denotes the score (using standard values of
\((R, S, T, P) = (3, 0, 5, 1)\)) of the \(i\)th strategy
against the \(j\)th strategy. Using this, the replicator equation
describes the evolution of the system based on a population density fitness
function:

\begin{equation}\label{eqn:replicator_dynamics}
    \frac{dx}{dt} = x(S-x^TS x)
\end{equation}

Equation (\ref{eqn:replicator_dynamics}) is solved numerically through an
integration technique described in~\cite{Petzold1983} and
Figure~\ref{fig:replicator_dynamics} shows the evolution of the distribution of
the system: the various strategies are ranked by scores. It is clear to see that
only the high ranking strategies survive the evolutionary process (in fact,
only \input{./assets/img/replicator_dynamics/main.tex}
have a final distribution greater than \(10 ^ {-2}\)). This confirms the
findings of~\cite{Moran1707} in which sophisticated strategies resist
evolutionary invasion of shorter memory strategies. Recalling
Figure~\ref{fig:SSError_and_probabilities_in_full} this demonstrates that:

\begin{itemize}
    \item Cooperation emerges through the evolutionary process: the high scoring
        strategies do not exhibit extortionate behaviour towards each other.
    \item Extortionate strategies do not survive the evolutionary process.
\end{itemize}

\begin{figure}[!htbp]
    \centering
    \includegraphics[width=.8\textwidth]{./assets/img/replicator_dynamics/main.pdf}
    \caption{Numerical simulation of the replicator equation
    (\ref{eqn:replicator_dynamics}): strategies are ordered by score, only the strategies with a high score survive the evolutionary process.}
    \label{fig:replicator_dynamics}
\end{figure}

This work can be used to classify plays of the IPD\@: data can be collected from
actual interactions (in lab or in the field). Furthermore, this allows for a
classification method similar to the notion of fingerprinting presented
in~\cite{Ashlock2008}. Trained strategies can potentially be classified as
extortionate or not or it could be possible to even constrain the reinforcement
learning approaches that are becoming prevalent in the literature.
Alternatively, this mathematical approach for recognising extortion could be
used in sophisticated strategies to defend against invasion. Arguably, some of
the strategies considered here exhibit this behaviour, indeed as described
in~\cite{Harper2017}, the top ranking strategies in the full tournament are
obtained using evolutionary reinforcement learning techniques, thus, suspicion
of extortionate behaviour could in fact be an evolutionary trait.

\section*{Acknowledgements}

The following open source software libraries were used in this research:

\begin{itemize}
    \item The Axelrod ~\cite{Knight2016, Knight2018} library (IPD strategies and
        tournaments).
    \item The sympy library~\cite{Meurer2017} (verification of all symbolic
        calculations).
    \item The matplotlib~\cite{Droettboom2018} library (visualisation).
    \item The pandas~\cite{Structures2010}, dask~\cite{Dask2016} and
        NumPy~\cite{Oliphant2015} libraries (data manipulation).
    \item The SciPy~\cite{Jones2001} library (numerical integration of the
        replicator equation).
\end{itemize}

This work was performed using the computational facilities of the Advanced
Research Computing @ Cardiff (ARCCA) Division, Cardiff University.

\printbibliography

\newpage
\section*{Supplementary materials}

\includepdf{assets/pdf/proof_of_form_of_extortionate_strategies/main.pdf}

\newpage

Using the pair wise interactions the transition rates \(p,
q\) can be measured and the steady state probabilities inferred and compared to
the actual probabilities of each state.
This is done numerically by computing the singular eigenvector of the
matrix \(A\) \cite{Stewart2009}:

\[
    A =
    \begin{bmatrix}
        p_1 q_1 & p_1 (1 - q_1) & (1 - p_1) q_1 & (1 -p_1) (1 - q_1) \\
        p_2 q_2 & p_2 (1 - q_2) & (1 - p_2) q_2 & (1 -p_2) (1 - q_2) \\
        p_3 q_3 & p_3 (1 - q_3) & (1 - p_3) q_3 & (1 -p_3) (1 - q_3) \\
        p_4 q_4 & p_4 (1 - q_4) & (1 - p_4) q_4 & (1 -p_4) (1 - q_4) \\
    \end{bmatrix}
\]

Figure~\ref{fig:computed_probabilities_vs_theoretic_probabilities} shows a
regression line fitted to every pairwise interaction with a reported
\(\text{SSError}\) value (pairwise interactions with missing states were
omitted). This serves to validate the approach: a part from some edge cases the
relationship is consistent.

\begin{figure}[!htbp]
    \centering
    \includegraphics[width=.8\textwidth]{./assets/img/computed_probabilities_vs_theoretic_probabilities/main.pdf}
    \caption{The
        relationship between the steady state probabilities inferred from the
        measured transitions and the actual steady state probabilities. A linear
        regression line is included validating the approach.}
    \label{fig:computed_probabilities_vs_theoretic_probabilities}
\end{figure}


\end{document}
 times. All of this
interaction data is available at~\cite{vincent_knight_2018_1297075}. A good
match between the inferred Markov chain and the state distribution of the actual
interactions has been verified. Data for this is presented in the supplementary
materials.

Figure~\ref{fig:SSError_overall_in_stewart_plotkin} shows the \(\text{SSError}\)
values for all the strategies in the tournament, as reported
in~\cite{Stewart2012} the extortionate strategy (which has an expected
\(\text{SSError}\) approximately 0) gains a large number of wins.

\begin{figure}[!htbp]
    \centering
    \includegraphics[width=.8\textwidth]{./assets/img/SSError_overall_in_stewart_plotkin/main.pdf}
    \caption{\(\text{SSError}\) and state probabilities for the strategies
        of~\cite{Stewart2012}, ordered both by number of wins and overall score.
        Note that \(P(DC)\) is not shown as it corresponds to the transpose of
        \(P(CD)\). Cooperator and Defector are omitted as they do not visit all
        the states.}
    \label{fig:SSError_overall_in_stewart_plotkin}
\end{figure}

Here, the work of~\cite{Stewart2012} is extended by investigating a tournament
with \documentclass[a4paper]{article}

\usepackage{amsmath}
\usepackage{amssymb}
\usepackage[margin=1.5cm,
            includefoot,
            footskip=30pt]{geometry}
\usepackage{layout}
\usepackage{graphicx}
\usepackage{subcaption}

\usepackage{biblatex}
\usepackage{pdfpages}

\bibliography{main.bib}

\title{Suspicion: Recognising and evaluating the effectiveness
       of extortion in the Iterated Prisoner's Dilemma}
\author{Vincent A. Knight \and Nikoleta E. Glynatsi}
\date{\today}



\begin{document}

\maketitle

\begin{abstract}
    The Iterated Prisoner's Dilemma is a model for rational and evolutionary
    interactive behaviour. It has applications both in the study of human social
    behaviour as well as in biology.
    It is used to understand when and how a rational individual might
    accept an immediate cost to their own utility for the direct benefit of
    another.

    Much attention has been given to a class of strategies called
    Zero Determinant strategies. It has been theoretically shown that these
    strategies can ``extort'' any player.

    In this work, an approach to identify if observed strategies are playing in
    an extortionate way is described. Furthermore, experimental analysis of
    a large tournament with \input{assets/tex/number_of_full_strategies/main.tex}
    strategies is considered. In this setting
    the most highly performing strategies do not play in an extortionate way
    against each other but do against lower performing strategies.
    This suggests that whilst the theory of Zero Determinant strategies
    indicates that memory is not of fundamental importance to the evolution of
    cooperative behaviour, this is incomplete.
\end{abstract}

\section{Introduction}\label{sec:introduction}

Agent based game theoretic models have become a stalwart of the underpinning
mathematics of interactive behaviours. One of the major pieces of work
in this area is the pair of original computer tournaments run by Robert
Axelrod~\cite{Axelrod1980, Axelrod1980a}. These tournaments pitted submitted
computer strategies against each other in plays of the Iterated Prisoner's
Dilemma. A common game where agents can choose to pay a slight cost to their
immediate utility in the hope of building a reputation. This has been used in
economic and evolutionary game theory to understand the evolution of cooperative
behaviour.

Recently, a class of strategies was described in~\cite{Press2012} that can
provably extort any given opponent. In~\cite{Hilbe2013, Moran1707} some
questions have already been asked about the true effectiveness of these
strategies in an evolutionary setting. Here another question is asked: is it
possible to recognise this extortionate behaviour? A mathematical procedure for
suspicion is presented: in the same way that the continued actions of an
extortionate individual might raise suspicion.

This work makes use of the Axelrod Python library~\cite{Knight2018, Knight2016}
with a large number of Prisoner Dilemma strategies available to give an
extensive numerical example of the ideas presented.  The approach is presented
in Section~\ref{sec:delta-zd-strategies}.  All of the code and data discussed
in Section~\ref{sec:numerical-experiments} is open sourced, archived and
written according to best scientific principles~\cite{Wilson2014}. The data
archive can be found at~\cite{vincent_knight_2018_1297075}.

\section{Recognising Extortion}\label{sec:delta-zd-strategies}

In~\cite{Press2012}, given a match between 2 memory-one strategies, the concept
of Zero Determinant (ZD) strategies is introduced. The main result of that paper
shows that given two memory one players \(p, q\in\mathbb{R}^4\) a linear
relationship between the players' scores could be forced by one of the players.

Using the notation of~\cite{Press2012}, assuming the utilities for player \(p\)
are given by \(S_x=(R, S, T, P)\) and for player \(q\) by \(S_y=(R, T, S, P)\)
and that the stationary scores of each player is given by \(S_X\) and \(S_Y\)
respectively. The main result of~\cite{Press2012} is that if

\begin{equation}\label{eqn:linear_relationship_for_p}
    \tilde p=\alpha S_x + \beta S_y + \gamma
\end{equation}

or

\begin{equation}\label{eqn:linear_relationship_for_q}
    \tilde q=\alpha S_x + \beta S_y + \gamma
\end{equation}

where \(\tilde p = (1 - p_1, 1 - p_2, p_3, p_4)\) and
\(\tilde q = (1 - q_1, 1 - q_2, q_3, q_4)\) then:

\begin{equation}
    \alpha S_X + \beta S_Y + \gamma = 0
\end{equation}

In~\cite{Press2012} a particular type of ZD strategy is defined: extortionate
strategies. If:

\begin{equation}\label{eqn:constraint_for_extortion}
    \gamma = - P(\alpha + \beta)
\end{equation}

then the player can ensure they get a score \(\chi\) times
larger than the opponent. This extortion coefficient is given by:

\begin{equation}\label{eqn:definition_of_chi}
    \chi=\frac{-\beta}{\alpha}
\end{equation}

Thus, if (\ref{eqn:constraint_for_extortion}) holds and \(\chi >1\) a player is
said to extort their opponent.
Here, the reverse problem is considered: given a
\(p\in\mathbb{R}^4\) how does one identify \(\alpha, \beta\) if they
exist and is the strategy in fact acting in an extortionate way?

These conditions correspond to:

\begin{align}
    \tilde p_1 & = \alpha R + \beta R - P (\alpha + \beta)
            \label{eqn:condition_for_tilde_p1}\\
    \tilde p_2 & = \alpha S + \beta T - P (\alpha + \beta)
            \label{eqn:condition_for_tilde_p2}\\
    \tilde p_3 & = \alpha T + \beta S - P (\alpha + \beta)
            \label{eqn:condition_for_tilde_p3}\\
    \tilde p_4 & = \alpha P + \beta P - P (\alpha + \beta)
            \label{eqn:condition_for_tilde_p4}
\end{align}

Equation (\ref{eqn:condition_for_tilde_p4}) ensures that \(p_4=\tilde p_4=0\).
Equations (\ref{eqn:condition_for_tilde_p1}-\ref{eqn:condition_for_tilde_p3})
can be used to eliminate \(\alpha, \beta\), giving:

\begin{equation}\label{eqn:planar_definition_of_extortion}
    \tilde p_1 = \frac{(R - P)(\tilde p_2 + \tilde p_3)}{S + T - 2P}
\end{equation}

with:

\begin{equation}\label{eqn:definition_of_chi}
    \chi = \frac{\tilde p_2 (P - T) + \tilde p_3 (S - P)}
                {\tilde p_2 (P - S) + \tilde p_3 (T - P)}
\end{equation}

Given a strategy \(p\in\mathbb{R}^{4\times 1}\) equations
(\ref{eqn:condition_for_tilde_p4}), (\ref{eqn:planar_definition_of_extortion}-\ref{eqn:definition_of_chi}) can be used to check if
a strategy is extortionate. The conditions correspond to:

\begin{align}
    p_1 & = \frac{(R-P)(p_2 + p_3) - R + T + S - P}{S + T - 2P}
     \label{eqn:condition_for_p1}\\
    p_4 & = 0 \label{eqn:condition_for_p4}\\
    1 & > p_2 + p_3\label{eqn:condition_for_chi}
\end{align}

The algebraic steps necessary to prove these results are available in the
supporting materials.

All extortionate strategies reside on a triangular (\ref{eqn:condition_for_chi})
plane (\ref{eqn:condition_for_p1}) in 3 dimensions (\ref{eqn:condition_for_p4}).
Using this formulation it can be seen that a necessary (but not sufficient)
condition for an extortionate strategy is that it cooperates on average less
than 50\% of the time when in a state of disagreement with the opponent.

As an example, consider the known extortionate strategy \(p=(8 / 9, 1 / 2, 1 /
3, 0)\) from~\cite{Stewart2012} which is referred to as \texttt{Extort-2}. In
this case, for the standard values of \((R, T, S, P)\) constraint
(\ref{eqn:condition_for_p1}) corresponds to:

\begin{equation}
    p_1 = \frac{2(p_2 + p_3) + 1}{3}
\end{equation}

It is clear that in this case all constraints hold.

This approach could in fact be used to confirm that a given strategy is acting
in an extortionate manner even if it is not a memory one strategy. However, in
practice, if a closed form for \(p\) is not known, then due to measurement
and/or numerical error this would not work.

This problem can be written in the following linear algebraic form where
\(x=(\alpha, \beta)\)
and \(p^*=(\tilde p_1 - 1, tilde_2 - 1, p_3)\):

\begin{equation}\label{eqn:linear_algebraic_equation_for_p}
    Cx= p^*
\end{equation}

\(C\) corresponds to equations
(\ref{eqn:condition_for_tilde_p1}-\ref{eqn:condition_for_tilde_p3}) and is
given by:

\begin{equation}\label{eqn:definition_of_C}
    C =
    \begin{bmatrix}
        R - P & R- P \\
        S - P & T- P \\
        T - P & S- P \\
    \end{bmatrix}
\end{equation}

Note that in general, equation (\ref{eqn:linear_algebraic_equation_for_p}) will
not necessarily have a solution. From the Rouch\'{e}-Capelli theorem if there is
a solution it is unique as \(\text{rank}(C)=2\) which is the dimension of the
variable \(x\). The best fitting \(x\) is found by minimizing:

\begin{equation}\label{eqn:r_squared}
    \text{SSError} = \|C x- p^*\|_2^2 = \sum_{i=1}^{3}\left((C\bar x)_i-p_i^*\right)^2
\end{equation}

Note that \(\text{SSError}\), which is the square of the Frobenius
norm~\cite{Golub2013}, becomes a measure of how close a strategy is to being an
extortionate strategy. Suspicion
of extortion then corresponds to a threshold on \(\text{SSError}\).

By observing interactions (human or otherwise), their memory one representation
can be inferred and this approach can be used to recognise extortionate
behaviour. The notion of comparing theoretic and actual plays of the IPD is not
novel, see for example~\cite{Rand2013}. Immediately it is noted that if the
environment is noisy~\cite{Wu1995} then no strategy can be considered to be
extortionate as \(p_4>0\).

In the next section, this idea will be illustrated by observing the interactions
that take place in a computer based tournament of the IPD\@.

\section{Numerical experiments}\label{sec:numerical-experiments}

In~\cite{Stewart2012} results from a tournament with
\input{./assets/tex/number_of_stewart_plotkin_strategies/main.tex} strategies,
was presented with specific consideration given to ZD strategies. This
tournament is reproduced here using the Axelrod-Python
project~\cite{Knight2016}. To obtain a good measure of the corresponding
transition rates for each strategy all matches have been run for
\input{assets/tex/number_of_turns/main.tex} turns and every match has been
repeated \input{assets/tex/number_of_repetitions/main.tex} times. All of this
interaction data is available at~\cite{vincent_knight_2018_1297075}. A good
match between the inferred Markov chain and the state distribution of the actual
interactions has been verified. Data for this is presented in the supplementary
materials.

Figure~\ref{fig:SSError_overall_in_stewart_plotkin} shows the \(\text{SSError}\)
values for all the strategies in the tournament, as reported
in~\cite{Stewart2012} the extortionate strategy (which has an expected
\(\text{SSError}\) approximately 0) gains a large number of wins.

\begin{figure}[!htbp]
    \centering
    \includegraphics[width=.8\textwidth]{./assets/img/SSError_overall_in_stewart_plotkin/main.pdf}
    \caption{\(\text{SSError}\) and state probabilities for the strategies
        of~\cite{Stewart2012}, ordered both by number of wins and overall score.
        Note that \(P(DC)\) is not shown as it corresponds to the transpose of
        \(P(CD)\). Cooperator and Defector are omitted as they do not visit all
        the states.}
    \label{fig:SSError_overall_in_stewart_plotkin}
\end{figure}

Here, the work of~\cite{Stewart2012} is extended by investigating a tournament
with \input{assets/tex/number_of_full_strategies/main.tex}
strategies.

The results of this analysis are shown in
Figure~\ref{fig:SSError_and_probabilities_in_full}. The top ranking strategies
by number of wins seem to be extortionate (but not against all strategies) and
it can be seen that a small sub group of strategies achieve mutual defection.
All the top ranking strategies according to score achieve mutual cooperation and
do not extort each other, however they
\textbf{do} exhibit extortionate behaviour towards a number of the lower ranking
strategies.

\begin{figure}[!htbp]
    \centering
    \includegraphics[width=.8\textwidth]{./assets/img/SSError_and_probabilities_in_full/main.pdf}
    \caption{\(\text{SSError}\) for the strategies for the full tournament. Only
    strategy interactions for which \(p_4=0\) and \(\chi>1\) are displayed.}
    \label{fig:SSError_and_probabilities_in_full}
\end{figure}

\section{Conclusion}\label{sec:conclusion}

This work defines an approach to measure whether or not a player is playing a
strategy that corresponds to an extortionate strategy as defined
in~\cite{Press2012}: a mathematical model for suspicion. Indeed, all
extortionate strategies have been
 classified as lying on a triangular plane.
This rigorous classification fails to be robust to small measurement error, thus
a statistical approach is proposed.
This is done through a linear algebraic approach for approximating the solution
of a linear system. Using this, a large number of pairwise interactions is
simulated and in fact very few strategies are found to act extortionately.

The work of~\cite{Press2012}, whilst showing that a clever approach to taking
advantage of another memory one strategy exists: this is incomplete. Whilst the
elegance of this result is very attractive, just as the simplicity of the
victory of Tit For Tat in Axelrod's original tournaments was, it is incomplete.
Extortionate strategies achieve a high number of wins but they do not
achieve a high score which corresponds to the fitness landscape in an
evolutionary sense. From the large number of interactions a payoff matrix \(S\)
can be measured where \(S_{ij}\) denotes the score (using standard values of
\((R, S, T, P) = (3, 0, 5, 1)\)) of the \(i\)th strategy
against the \(j\)th strategy. Using this, the replicator equation
describes the evolution of the system based on a population density fitness
function:

\begin{equation}\label{eqn:replicator_dynamics}
    \frac{dx}{dt} = x(S-x^TS x)
\end{equation}

Equation (\ref{eqn:replicator_dynamics}) is solved numerically through an
integration technique described in~\cite{Petzold1983} and
Figure~\ref{fig:replicator_dynamics} shows the evolution of the distribution of
the system: the various strategies are ranked by scores. It is clear to see that
only the high ranking strategies survive the evolutionary process (in fact,
only \input{./assets/img/replicator_dynamics/main.tex}
have a final distribution greater than \(10 ^ {-2}\)). This confirms the
findings of~\cite{Moran1707} in which sophisticated strategies resist
evolutionary invasion of shorter memory strategies. Recalling
Figure~\ref{fig:SSError_and_probabilities_in_full} this demonstrates that:

\begin{itemize}
    \item Cooperation emerges through the evolutionary process: the high scoring
        strategies do not exhibit extortionate behaviour towards each other.
    \item Extortionate strategies do not survive the evolutionary process.
\end{itemize}

\begin{figure}[!htbp]
    \centering
    \includegraphics[width=.8\textwidth]{./assets/img/replicator_dynamics/main.pdf}
    \caption{Numerical simulation of the replicator equation
    (\ref{eqn:replicator_dynamics}): strategies are ordered by score, only the strategies with a high score survive the evolutionary process.}
    \label{fig:replicator_dynamics}
\end{figure}

This work can be used to classify plays of the IPD\@: data can be collected from
actual interactions (in lab or in the field). Furthermore, this allows for a
classification method similar to the notion of fingerprinting presented
in~\cite{Ashlock2008}. Trained strategies can potentially be classified as
extortionate or not or it could be possible to even constrain the reinforcement
learning approaches that are becoming prevalent in the literature.
Alternatively, this mathematical approach for recognising extortion could be
used in sophisticated strategies to defend against invasion. Arguably, some of
the strategies considered here exhibit this behaviour, indeed as described
in~\cite{Harper2017}, the top ranking strategies in the full tournament are
obtained using evolutionary reinforcement learning techniques, thus, suspicion
of extortionate behaviour could in fact be an evolutionary trait.

\section*{Acknowledgements}

The following open source software libraries were used in this research:

\begin{itemize}
    \item The Axelrod ~\cite{Knight2016, Knight2018} library (IPD strategies and
        tournaments).
    \item The sympy library~\cite{Meurer2017} (verification of all symbolic
        calculations).
    \item The matplotlib~\cite{Droettboom2018} library (visualisation).
    \item The pandas~\cite{Structures2010}, dask~\cite{Dask2016} and
        NumPy~\cite{Oliphant2015} libraries (data manipulation).
    \item The SciPy~\cite{Jones2001} library (numerical integration of the
        replicator equation).
\end{itemize}

This work was performed using the computational facilities of the Advanced
Research Computing @ Cardiff (ARCCA) Division, Cardiff University.

\printbibliography

\newpage
\section*{Supplementary materials}

\includepdf{assets/pdf/proof_of_form_of_extortionate_strategies/main.pdf}

\newpage

Using the pair wise interactions the transition rates \(p,
q\) can be measured and the steady state probabilities inferred and compared to
the actual probabilities of each state.
This is done numerically by computing the singular eigenvector of the
matrix \(A\) \cite{Stewart2009}:

\[
    A =
    \begin{bmatrix}
        p_1 q_1 & p_1 (1 - q_1) & (1 - p_1) q_1 & (1 -p_1) (1 - q_1) \\
        p_2 q_2 & p_2 (1 - q_2) & (1 - p_2) q_2 & (1 -p_2) (1 - q_2) \\
        p_3 q_3 & p_3 (1 - q_3) & (1 - p_3) q_3 & (1 -p_3) (1 - q_3) \\
        p_4 q_4 & p_4 (1 - q_4) & (1 - p_4) q_4 & (1 -p_4) (1 - q_4) \\
    \end{bmatrix}
\]

Figure~\ref{fig:computed_probabilities_vs_theoretic_probabilities} shows a
regression line fitted to every pairwise interaction with a reported
\(\text{SSError}\) value (pairwise interactions with missing states were
omitted). This serves to validate the approach: a part from some edge cases the
relationship is consistent.

\begin{figure}[!htbp]
    \centering
    \includegraphics[width=.8\textwidth]{./assets/img/computed_probabilities_vs_theoretic_probabilities/main.pdf}
    \caption{The
        relationship between the steady state probabilities inferred from the
        measured transitions and the actual steady state probabilities. A linear
        regression line is included validating the approach.}
    \label{fig:computed_probabilities_vs_theoretic_probabilities}
\end{figure}


\end{document}

strategies.

The results of this analysis are shown in
Figure~\ref{fig:SSError_and_probabilities_in_full}. The top ranking strategies
by number of wins seem to be extortionate (but not against all strategies) and
it can be seen that a small sub group of strategies achieve mutual defection.
All the top ranking strategies according to score achieve mutual cooperation and
do not extort each other, however they
\textbf{do} exhibit extortionate behaviour towards a number of the lower ranking
strategies.

\begin{figure}[!htbp]
    \centering
    \includegraphics[width=.8\textwidth]{./assets/img/SSError_and_probabilities_in_full/main.pdf}
    \caption{\(\text{SSError}\) for the strategies for the full tournament. Only
    strategy interactions for which \(p_4=0\) and \(\chi>1\) are displayed.}
    \label{fig:SSError_and_probabilities_in_full}
\end{figure}

\section{Conclusion}\label{sec:conclusion}

This work defines an approach to measure whether or not a player is playing a
strategy that corresponds to an extortionate strategy as defined
in~\cite{Press2012}: a mathematical model for suspicion. Indeed, all
extortionate strategies have been
 classified as lying on a triangular plane.
This rigorous classification fails to be robust to small measurement error, thus
a statistical approach is proposed.
This is done through a linear algebraic approach for approximating the solution
of a linear system. Using this, a large number of pairwise interactions is
simulated and in fact very few strategies are found to act extortionately.

The work of~\cite{Press2012}, whilst showing that a clever approach to taking
advantage of another memory one strategy exists: this is incomplete. Whilst the
elegance of this result is very attractive, just as the simplicity of the
victory of Tit For Tat in Axelrod's original tournaments was, it is incomplete.
Extortionate strategies achieve a high number of wins but they do not
achieve a high score which corresponds to the fitness landscape in an
evolutionary sense. From the large number of interactions a payoff matrix \(S\)
can be measured where \(S_{ij}\) denotes the score (using standard values of
\((R, S, T, P) = (3, 0, 5, 1)\)) of the \(i\)th strategy
against the \(j\)th strategy. Using this, the replicator equation
describes the evolution of the system based on a population density fitness
function:

\begin{equation}\label{eqn:replicator_dynamics}
    \frac{dx}{dt} = x(S-x^TS x)
\end{equation}

Equation (\ref{eqn:replicator_dynamics}) is solved numerically through an
integration technique described in~\cite{Petzold1983} and
Figure~\ref{fig:replicator_dynamics} shows the evolution of the distribution of
the system: the various strategies are ranked by scores. It is clear to see that
only the high ranking strategies survive the evolutionary process (in fact,
only \documentclass[a4paper]{article}

\usepackage{amsmath}
\usepackage{amssymb}
\usepackage[margin=1.5cm,
            includefoot,
            footskip=30pt]{geometry}
\usepackage{layout}
\usepackage{graphicx}
\usepackage{subcaption}

\usepackage{biblatex}
\usepackage{pdfpages}

\bibliography{main.bib}

\title{Suspicion: Recognising and evaluating the effectiveness
       of extortion in the Iterated Prisoner's Dilemma}
\author{Vincent A. Knight \and Nikoleta E. Glynatsi}
\date{\today}



\begin{document}

\maketitle

\begin{abstract}
    The Iterated Prisoner's Dilemma is a model for rational and evolutionary
    interactive behaviour. It has applications both in the study of human social
    behaviour as well as in biology.
    It is used to understand when and how a rational individual might
    accept an immediate cost to their own utility for the direct benefit of
    another.

    Much attention has been given to a class of strategies called
    Zero Determinant strategies. It has been theoretically shown that these
    strategies can ``extort'' any player.

    In this work, an approach to identify if observed strategies are playing in
    an extortionate way is described. Furthermore, experimental analysis of
    a large tournament with \input{assets/tex/number_of_full_strategies/main.tex}
    strategies is considered. In this setting
    the most highly performing strategies do not play in an extortionate way
    against each other but do against lower performing strategies.
    This suggests that whilst the theory of Zero Determinant strategies
    indicates that memory is not of fundamental importance to the evolution of
    cooperative behaviour, this is incomplete.
\end{abstract}

\section{Introduction}\label{sec:introduction}

Agent based game theoretic models have become a stalwart of the underpinning
mathematics of interactive behaviours. One of the major pieces of work
in this area is the pair of original computer tournaments run by Robert
Axelrod~\cite{Axelrod1980, Axelrod1980a}. These tournaments pitted submitted
computer strategies against each other in plays of the Iterated Prisoner's
Dilemma. A common game where agents can choose to pay a slight cost to their
immediate utility in the hope of building a reputation. This has been used in
economic and evolutionary game theory to understand the evolution of cooperative
behaviour.

Recently, a class of strategies was described in~\cite{Press2012} that can
provably extort any given opponent. In~\cite{Hilbe2013, Moran1707} some
questions have already been asked about the true effectiveness of these
strategies in an evolutionary setting. Here another question is asked: is it
possible to recognise this extortionate behaviour? A mathematical procedure for
suspicion is presented: in the same way that the continued actions of an
extortionate individual might raise suspicion.

This work makes use of the Axelrod Python library~\cite{Knight2018, Knight2016}
with a large number of Prisoner Dilemma strategies available to give an
extensive numerical example of the ideas presented.  The approach is presented
in Section~\ref{sec:delta-zd-strategies}.  All of the code and data discussed
in Section~\ref{sec:numerical-experiments} is open sourced, archived and
written according to best scientific principles~\cite{Wilson2014}. The data
archive can be found at~\cite{vincent_knight_2018_1297075}.

\section{Recognising Extortion}\label{sec:delta-zd-strategies}

In~\cite{Press2012}, given a match between 2 memory-one strategies, the concept
of Zero Determinant (ZD) strategies is introduced. The main result of that paper
shows that given two memory one players \(p, q\in\mathbb{R}^4\) a linear
relationship between the players' scores could be forced by one of the players.

Using the notation of~\cite{Press2012}, assuming the utilities for player \(p\)
are given by \(S_x=(R, S, T, P)\) and for player \(q\) by \(S_y=(R, T, S, P)\)
and that the stationary scores of each player is given by \(S_X\) and \(S_Y\)
respectively. The main result of~\cite{Press2012} is that if

\begin{equation}\label{eqn:linear_relationship_for_p}
    \tilde p=\alpha S_x + \beta S_y + \gamma
\end{equation}

or

\begin{equation}\label{eqn:linear_relationship_for_q}
    \tilde q=\alpha S_x + \beta S_y + \gamma
\end{equation}

where \(\tilde p = (1 - p_1, 1 - p_2, p_3, p_4)\) and
\(\tilde q = (1 - q_1, 1 - q_2, q_3, q_4)\) then:

\begin{equation}
    \alpha S_X + \beta S_Y + \gamma = 0
\end{equation}

In~\cite{Press2012} a particular type of ZD strategy is defined: extortionate
strategies. If:

\begin{equation}\label{eqn:constraint_for_extortion}
    \gamma = - P(\alpha + \beta)
\end{equation}

then the player can ensure they get a score \(\chi\) times
larger than the opponent. This extortion coefficient is given by:

\begin{equation}\label{eqn:definition_of_chi}
    \chi=\frac{-\beta}{\alpha}
\end{equation}

Thus, if (\ref{eqn:constraint_for_extortion}) holds and \(\chi >1\) a player is
said to extort their opponent.
Here, the reverse problem is considered: given a
\(p\in\mathbb{R}^4\) how does one identify \(\alpha, \beta\) if they
exist and is the strategy in fact acting in an extortionate way?

These conditions correspond to:

\begin{align}
    \tilde p_1 & = \alpha R + \beta R - P (\alpha + \beta)
            \label{eqn:condition_for_tilde_p1}\\
    \tilde p_2 & = \alpha S + \beta T - P (\alpha + \beta)
            \label{eqn:condition_for_tilde_p2}\\
    \tilde p_3 & = \alpha T + \beta S - P (\alpha + \beta)
            \label{eqn:condition_for_tilde_p3}\\
    \tilde p_4 & = \alpha P + \beta P - P (\alpha + \beta)
            \label{eqn:condition_for_tilde_p4}
\end{align}

Equation (\ref{eqn:condition_for_tilde_p4}) ensures that \(p_4=\tilde p_4=0\).
Equations (\ref{eqn:condition_for_tilde_p1}-\ref{eqn:condition_for_tilde_p3})
can be used to eliminate \(\alpha, \beta\), giving:

\begin{equation}\label{eqn:planar_definition_of_extortion}
    \tilde p_1 = \frac{(R - P)(\tilde p_2 + \tilde p_3)}{S + T - 2P}
\end{equation}

with:

\begin{equation}\label{eqn:definition_of_chi}
    \chi = \frac{\tilde p_2 (P - T) + \tilde p_3 (S - P)}
                {\tilde p_2 (P - S) + \tilde p_3 (T - P)}
\end{equation}

Given a strategy \(p\in\mathbb{R}^{4\times 1}\) equations
(\ref{eqn:condition_for_tilde_p4}), (\ref{eqn:planar_definition_of_extortion}-\ref{eqn:definition_of_chi}) can be used to check if
a strategy is extortionate. The conditions correspond to:

\begin{align}
    p_1 & = \frac{(R-P)(p_2 + p_3) - R + T + S - P}{S + T - 2P}
     \label{eqn:condition_for_p1}\\
    p_4 & = 0 \label{eqn:condition_for_p4}\\
    1 & > p_2 + p_3\label{eqn:condition_for_chi}
\end{align}

The algebraic steps necessary to prove these results are available in the
supporting materials.

All extortionate strategies reside on a triangular (\ref{eqn:condition_for_chi})
plane (\ref{eqn:condition_for_p1}) in 3 dimensions (\ref{eqn:condition_for_p4}).
Using this formulation it can be seen that a necessary (but not sufficient)
condition for an extortionate strategy is that it cooperates on average less
than 50\% of the time when in a state of disagreement with the opponent.

As an example, consider the known extortionate strategy \(p=(8 / 9, 1 / 2, 1 /
3, 0)\) from~\cite{Stewart2012} which is referred to as \texttt{Extort-2}. In
this case, for the standard values of \((R, T, S, P)\) constraint
(\ref{eqn:condition_for_p1}) corresponds to:

\begin{equation}
    p_1 = \frac{2(p_2 + p_3) + 1}{3}
\end{equation}

It is clear that in this case all constraints hold.

This approach could in fact be used to confirm that a given strategy is acting
in an extortionate manner even if it is not a memory one strategy. However, in
practice, if a closed form for \(p\) is not known, then due to measurement
and/or numerical error this would not work.

This problem can be written in the following linear algebraic form where
\(x=(\alpha, \beta)\)
and \(p^*=(\tilde p_1 - 1, tilde_2 - 1, p_3)\):

\begin{equation}\label{eqn:linear_algebraic_equation_for_p}
    Cx= p^*
\end{equation}

\(C\) corresponds to equations
(\ref{eqn:condition_for_tilde_p1}-\ref{eqn:condition_for_tilde_p3}) and is
given by:

\begin{equation}\label{eqn:definition_of_C}
    C =
    \begin{bmatrix}
        R - P & R- P \\
        S - P & T- P \\
        T - P & S- P \\
    \end{bmatrix}
\end{equation}

Note that in general, equation (\ref{eqn:linear_algebraic_equation_for_p}) will
not necessarily have a solution. From the Rouch\'{e}-Capelli theorem if there is
a solution it is unique as \(\text{rank}(C)=2\) which is the dimension of the
variable \(x\). The best fitting \(x\) is found by minimizing:

\begin{equation}\label{eqn:r_squared}
    \text{SSError} = \|C x- p^*\|_2^2 = \sum_{i=1}^{3}\left((C\bar x)_i-p_i^*\right)^2
\end{equation}

Note that \(\text{SSError}\), which is the square of the Frobenius
norm~\cite{Golub2013}, becomes a measure of how close a strategy is to being an
extortionate strategy. Suspicion
of extortion then corresponds to a threshold on \(\text{SSError}\).

By observing interactions (human or otherwise), their memory one representation
can be inferred and this approach can be used to recognise extortionate
behaviour. The notion of comparing theoretic and actual plays of the IPD is not
novel, see for example~\cite{Rand2013}. Immediately it is noted that if the
environment is noisy~\cite{Wu1995} then no strategy can be considered to be
extortionate as \(p_4>0\).

In the next section, this idea will be illustrated by observing the interactions
that take place in a computer based tournament of the IPD\@.

\section{Numerical experiments}\label{sec:numerical-experiments}

In~\cite{Stewart2012} results from a tournament with
\input{./assets/tex/number_of_stewart_plotkin_strategies/main.tex} strategies,
was presented with specific consideration given to ZD strategies. This
tournament is reproduced here using the Axelrod-Python
project~\cite{Knight2016}. To obtain a good measure of the corresponding
transition rates for each strategy all matches have been run for
\input{assets/tex/number_of_turns/main.tex} turns and every match has been
repeated \input{assets/tex/number_of_repetitions/main.tex} times. All of this
interaction data is available at~\cite{vincent_knight_2018_1297075}. A good
match between the inferred Markov chain and the state distribution of the actual
interactions has been verified. Data for this is presented in the supplementary
materials.

Figure~\ref{fig:SSError_overall_in_stewart_plotkin} shows the \(\text{SSError}\)
values for all the strategies in the tournament, as reported
in~\cite{Stewart2012} the extortionate strategy (which has an expected
\(\text{SSError}\) approximately 0) gains a large number of wins.

\begin{figure}[!htbp]
    \centering
    \includegraphics[width=.8\textwidth]{./assets/img/SSError_overall_in_stewart_plotkin/main.pdf}
    \caption{\(\text{SSError}\) and state probabilities for the strategies
        of~\cite{Stewart2012}, ordered both by number of wins and overall score.
        Note that \(P(DC)\) is not shown as it corresponds to the transpose of
        \(P(CD)\). Cooperator and Defector are omitted as they do not visit all
        the states.}
    \label{fig:SSError_overall_in_stewart_plotkin}
\end{figure}

Here, the work of~\cite{Stewart2012} is extended by investigating a tournament
with \input{assets/tex/number_of_full_strategies/main.tex}
strategies.

The results of this analysis are shown in
Figure~\ref{fig:SSError_and_probabilities_in_full}. The top ranking strategies
by number of wins seem to be extortionate (but not against all strategies) and
it can be seen that a small sub group of strategies achieve mutual defection.
All the top ranking strategies according to score achieve mutual cooperation and
do not extort each other, however they
\textbf{do} exhibit extortionate behaviour towards a number of the lower ranking
strategies.

\begin{figure}[!htbp]
    \centering
    \includegraphics[width=.8\textwidth]{./assets/img/SSError_and_probabilities_in_full/main.pdf}
    \caption{\(\text{SSError}\) for the strategies for the full tournament. Only
    strategy interactions for which \(p_4=0\) and \(\chi>1\) are displayed.}
    \label{fig:SSError_and_probabilities_in_full}
\end{figure}

\section{Conclusion}\label{sec:conclusion}

This work defines an approach to measure whether or not a player is playing a
strategy that corresponds to an extortionate strategy as defined
in~\cite{Press2012}: a mathematical model for suspicion. Indeed, all
extortionate strategies have been
 classified as lying on a triangular plane.
This rigorous classification fails to be robust to small measurement error, thus
a statistical approach is proposed.
This is done through a linear algebraic approach for approximating the solution
of a linear system. Using this, a large number of pairwise interactions is
simulated and in fact very few strategies are found to act extortionately.

The work of~\cite{Press2012}, whilst showing that a clever approach to taking
advantage of another memory one strategy exists: this is incomplete. Whilst the
elegance of this result is very attractive, just as the simplicity of the
victory of Tit For Tat in Axelrod's original tournaments was, it is incomplete.
Extortionate strategies achieve a high number of wins but they do not
achieve a high score which corresponds to the fitness landscape in an
evolutionary sense. From the large number of interactions a payoff matrix \(S\)
can be measured where \(S_{ij}\) denotes the score (using standard values of
\((R, S, T, P) = (3, 0, 5, 1)\)) of the \(i\)th strategy
against the \(j\)th strategy. Using this, the replicator equation
describes the evolution of the system based on a population density fitness
function:

\begin{equation}\label{eqn:replicator_dynamics}
    \frac{dx}{dt} = x(S-x^TS x)
\end{equation}

Equation (\ref{eqn:replicator_dynamics}) is solved numerically through an
integration technique described in~\cite{Petzold1983} and
Figure~\ref{fig:replicator_dynamics} shows the evolution of the distribution of
the system: the various strategies are ranked by scores. It is clear to see that
only the high ranking strategies survive the evolutionary process (in fact,
only \input{./assets/img/replicator_dynamics/main.tex}
have a final distribution greater than \(10 ^ {-2}\)). This confirms the
findings of~\cite{Moran1707} in which sophisticated strategies resist
evolutionary invasion of shorter memory strategies. Recalling
Figure~\ref{fig:SSError_and_probabilities_in_full} this demonstrates that:

\begin{itemize}
    \item Cooperation emerges through the evolutionary process: the high scoring
        strategies do not exhibit extortionate behaviour towards each other.
    \item Extortionate strategies do not survive the evolutionary process.
\end{itemize}

\begin{figure}[!htbp]
    \centering
    \includegraphics[width=.8\textwidth]{./assets/img/replicator_dynamics/main.pdf}
    \caption{Numerical simulation of the replicator equation
    (\ref{eqn:replicator_dynamics}): strategies are ordered by score, only the strategies with a high score survive the evolutionary process.}
    \label{fig:replicator_dynamics}
\end{figure}

This work can be used to classify plays of the IPD\@: data can be collected from
actual interactions (in lab or in the field). Furthermore, this allows for a
classification method similar to the notion of fingerprinting presented
in~\cite{Ashlock2008}. Trained strategies can potentially be classified as
extortionate or not or it could be possible to even constrain the reinforcement
learning approaches that are becoming prevalent in the literature.
Alternatively, this mathematical approach for recognising extortion could be
used in sophisticated strategies to defend against invasion. Arguably, some of
the strategies considered here exhibit this behaviour, indeed as described
in~\cite{Harper2017}, the top ranking strategies in the full tournament are
obtained using evolutionary reinforcement learning techniques, thus, suspicion
of extortionate behaviour could in fact be an evolutionary trait.

\section*{Acknowledgements}

The following open source software libraries were used in this research:

\begin{itemize}
    \item The Axelrod ~\cite{Knight2016, Knight2018} library (IPD strategies and
        tournaments).
    \item The sympy library~\cite{Meurer2017} (verification of all symbolic
        calculations).
    \item The matplotlib~\cite{Droettboom2018} library (visualisation).
    \item The pandas~\cite{Structures2010}, dask~\cite{Dask2016} and
        NumPy~\cite{Oliphant2015} libraries (data manipulation).
    \item The SciPy~\cite{Jones2001} library (numerical integration of the
        replicator equation).
\end{itemize}

This work was performed using the computational facilities of the Advanced
Research Computing @ Cardiff (ARCCA) Division, Cardiff University.

\printbibliography

\newpage
\section*{Supplementary materials}

\includepdf{assets/pdf/proof_of_form_of_extortionate_strategies/main.pdf}

\newpage

Using the pair wise interactions the transition rates \(p,
q\) can be measured and the steady state probabilities inferred and compared to
the actual probabilities of each state.
This is done numerically by computing the singular eigenvector of the
matrix \(A\) \cite{Stewart2009}:

\[
    A =
    \begin{bmatrix}
        p_1 q_1 & p_1 (1 - q_1) & (1 - p_1) q_1 & (1 -p_1) (1 - q_1) \\
        p_2 q_2 & p_2 (1 - q_2) & (1 - p_2) q_2 & (1 -p_2) (1 - q_2) \\
        p_3 q_3 & p_3 (1 - q_3) & (1 - p_3) q_3 & (1 -p_3) (1 - q_3) \\
        p_4 q_4 & p_4 (1 - q_4) & (1 - p_4) q_4 & (1 -p_4) (1 - q_4) \\
    \end{bmatrix}
\]

Figure~\ref{fig:computed_probabilities_vs_theoretic_probabilities} shows a
regression line fitted to every pairwise interaction with a reported
\(\text{SSError}\) value (pairwise interactions with missing states were
omitted). This serves to validate the approach: a part from some edge cases the
relationship is consistent.

\begin{figure}[!htbp]
    \centering
    \includegraphics[width=.8\textwidth]{./assets/img/computed_probabilities_vs_theoretic_probabilities/main.pdf}
    \caption{The
        relationship between the steady state probabilities inferred from the
        measured transitions and the actual steady state probabilities. A linear
        regression line is included validating the approach.}
    \label{fig:computed_probabilities_vs_theoretic_probabilities}
\end{figure}


\end{document}

have a final distribution greater than \(10 ^ {-2}\)). This confirms the
findings of~\cite{Moran1707} in which sophisticated strategies resist
evolutionary invasion of shorter memory strategies. Recalling
Figure~\ref{fig:SSError_and_probabilities_in_full} this demonstrates that:

\begin{itemize}
    \item Cooperation emerges through the evolutionary process: the high scoring
        strategies do not exhibit extortionate behaviour towards each other.
    \item Extortionate strategies do not survive the evolutionary process.
\end{itemize}

\begin{figure}[!htbp]
    \centering
    \includegraphics[width=.8\textwidth]{./assets/img/replicator_dynamics/main.pdf}
    \caption{Numerical simulation of the replicator equation
    (\ref{eqn:replicator_dynamics}): strategies are ordered by score, only the strategies with a high score survive the evolutionary process.}
    \label{fig:replicator_dynamics}
\end{figure}

This work can be used to classify plays of the IPD\@: data can be collected from
actual interactions (in lab or in the field). Furthermore, this allows for a
classification method similar to the notion of fingerprinting presented
in~\cite{Ashlock2008}. Trained strategies can potentially be classified as
extortionate or not or it could be possible to even constrain the reinforcement
learning approaches that are becoming prevalent in the literature.
Alternatively, this mathematical approach for recognising extortion could be
used in sophisticated strategies to defend against invasion. Arguably, some of
the strategies considered here exhibit this behaviour, indeed as described
in~\cite{Harper2017}, the top ranking strategies in the full tournament are
obtained using evolutionary reinforcement learning techniques, thus, suspicion
of extortionate behaviour could in fact be an evolutionary trait.

\section*{Acknowledgements}

The following open source software libraries were used in this research:

\begin{itemize}
    \item The Axelrod ~\cite{Knight2016, Knight2018} library (IPD strategies and
        tournaments).
    \item The sympy library~\cite{Meurer2017} (verification of all symbolic
        calculations).
    \item The matplotlib~\cite{Droettboom2018} library (visualisation).
    \item The pandas~\cite{Structures2010}, dask~\cite{Dask2016} and
        NumPy~\cite{Oliphant2015} libraries (data manipulation).
    \item The SciPy~\cite{Jones2001} library (numerical integration of the
        replicator equation).
\end{itemize}

This work was performed using the computational facilities of the Advanced
Research Computing @ Cardiff (ARCCA) Division, Cardiff University.

\printbibliography

\newpage
\section*{Supplementary materials}

\includepdf{assets/pdf/proof_of_form_of_extortionate_strategies/main.pdf}

\newpage

Using the pair wise interactions the transition rates \(p,
q\) can be measured and the steady state probabilities inferred and compared to
the actual probabilities of each state.
This is done numerically by computing the singular eigenvector of the
matrix \(A\) \cite{Stewart2009}:

\[
    A =
    \begin{bmatrix}
        p_1 q_1 & p_1 (1 - q_1) & (1 - p_1) q_1 & (1 -p_1) (1 - q_1) \\
        p_2 q_2 & p_2 (1 - q_2) & (1 - p_2) q_2 & (1 -p_2) (1 - q_2) \\
        p_3 q_3 & p_3 (1 - q_3) & (1 - p_3) q_3 & (1 -p_3) (1 - q_3) \\
        p_4 q_4 & p_4 (1 - q_4) & (1 - p_4) q_4 & (1 -p_4) (1 - q_4) \\
    \end{bmatrix}
\]

Figure~\ref{fig:computed_probabilities_vs_theoretic_probabilities} shows a
regression line fitted to every pairwise interaction with a reported
\(\text{SSError}\) value (pairwise interactions with missing states were
omitted). This serves to validate the approach: a part from some edge cases the
relationship is consistent.

\begin{figure}[!htbp]
    \centering
    \includegraphics[width=.8\textwidth]{./assets/img/computed_probabilities_vs_theoretic_probabilities/main.pdf}
    \caption{The
        relationship between the steady state probabilities inferred from the
        measured transitions and the actual steady state probabilities. A linear
        regression line is included validating the approach.}
    \label{fig:computed_probabilities_vs_theoretic_probabilities}
\end{figure}


\end{document}

strategies.

The results of this analysis are shown in
Figure~\ref{fig:SSError_and_probabilities_in_full}. The top ranking strategies
by number of wins seem to be extortionate (but not against all strategies) and
it can be seen that a small sub group of strategies achieve mutual defection.
All the top ranking strategies according to score achieve mutual cooperation and
do not extort each other, however they
\textbf{do} exhibit extortionate behaviour towards a number of the lower ranking
strategies.

\begin{figure}[!htbp]
    \centering
    \includegraphics[width=.8\textwidth]{./assets/img/SSError_and_probabilities_in_full/main.pdf}
    \caption{\(\text{SSError}\) for the strategies for the full tournament. Only
    strategy interactions for which \(p_4=0\) and \(\chi>1\) are displayed.}
    \label{fig:SSError_and_probabilities_in_full}
\end{figure}

\section{Conclusion}\label{sec:conclusion}

This work defines an approach to measure whether or not a player is playing a
strategy that corresponds to an extortionate strategy as defined
in~\cite{Press2012}: a mathematical model for suspicion. Indeed, all
extortionate strategies have been
 classified as lying on a triangular plane.
This rigorous classification fails to be robust to small measurement error, thus
a statistical approach is proposed.
This is done through a linear algebraic approach for approximating the solution
of a linear system. Using this, a large number of pairwise interactions is
simulated and in fact very few strategies are found to act extortionately.

The work of~\cite{Press2012}, whilst showing that a clever approach to taking
advantage of another memory one strategy exists: this is incomplete. Whilst the
elegance of this result is very attractive, just as the simplicity of the
victory of Tit For Tat in Axelrod's original tournaments was, it is incomplete.
Extortionate strategies achieve a high number of wins but they do not
achieve a high score which corresponds to the fitness landscape in an
evolutionary sense. From the large number of interactions a payoff matrix \(S\)
can be measured where \(S_{ij}\) denotes the score (using standard values of
\((R, S, T, P) = (3, 0, 5, 1)\)) of the \(i\)th strategy
against the \(j\)th strategy. Using this, the replicator equation
describes the evolution of the system based on a population density fitness
function:

\begin{equation}\label{eqn:replicator_dynamics}
    \frac{dx}{dt} = x(S-x^TS x)
\end{equation}

Equation (\ref{eqn:replicator_dynamics}) is solved numerically through an
integration technique described in~\cite{Petzold1983} and
Figure~\ref{fig:replicator_dynamics} shows the evolution of the distribution of
the system: the various strategies are ranked by scores. It is clear to see that
only the high ranking strategies survive the evolutionary process (in fact,
only \documentclass[a4paper]{article}

\usepackage{amsmath}
\usepackage{amssymb}
\usepackage[margin=1.5cm,
            includefoot,
            footskip=30pt]{geometry}
\usepackage{layout}
\usepackage{graphicx}
\usepackage{subcaption}

\usepackage{biblatex}
\usepackage{pdfpages}

\bibliography{main.bib}

\title{Suspicion: Recognising and evaluating the effectiveness
       of extortion in the Iterated Prisoner's Dilemma}
\author{Vincent A. Knight \and Nikoleta E. Glynatsi}
\date{\today}



\begin{document}

\maketitle

\begin{abstract}
    The Iterated Prisoner's Dilemma is a model for rational and evolutionary
    interactive behaviour. It has applications both in the study of human social
    behaviour as well as in biology.
    It is used to understand when and how a rational individual might
    accept an immediate cost to their own utility for the direct benefit of
    another.

    Much attention has been given to a class of strategies called
    Zero Determinant strategies. It has been theoretically shown that these
    strategies can ``extort'' any player.

    In this work, an approach to identify if observed strategies are playing in
    an extortionate way is described. Furthermore, experimental analysis of
    a large tournament with \documentclass[a4paper]{article}

\usepackage{amsmath}
\usepackage{amssymb}
\usepackage[margin=1.5cm,
            includefoot,
            footskip=30pt]{geometry}
\usepackage{layout}
\usepackage{graphicx}
\usepackage{subcaption}

\usepackage{biblatex}
\usepackage{pdfpages}

\bibliography{main.bib}

\title{Suspicion: Recognising and evaluating the effectiveness
       of extortion in the Iterated Prisoner's Dilemma}
\author{Vincent A. Knight \and Nikoleta E. Glynatsi}
\date{\today}



\begin{document}

\maketitle

\begin{abstract}
    The Iterated Prisoner's Dilemma is a model for rational and evolutionary
    interactive behaviour. It has applications both in the study of human social
    behaviour as well as in biology.
    It is used to understand when and how a rational individual might
    accept an immediate cost to their own utility for the direct benefit of
    another.

    Much attention has been given to a class of strategies called
    Zero Determinant strategies. It has been theoretically shown that these
    strategies can ``extort'' any player.

    In this work, an approach to identify if observed strategies are playing in
    an extortionate way is described. Furthermore, experimental analysis of
    a large tournament with \input{assets/tex/number_of_full_strategies/main.tex}
    strategies is considered. In this setting
    the most highly performing strategies do not play in an extortionate way
    against each other but do against lower performing strategies.
    This suggests that whilst the theory of Zero Determinant strategies
    indicates that memory is not of fundamental importance to the evolution of
    cooperative behaviour, this is incomplete.
\end{abstract}

\section{Introduction}\label{sec:introduction}

Agent based game theoretic models have become a stalwart of the underpinning
mathematics of interactive behaviours. One of the major pieces of work
in this area is the pair of original computer tournaments run by Robert
Axelrod~\cite{Axelrod1980, Axelrod1980a}. These tournaments pitted submitted
computer strategies against each other in plays of the Iterated Prisoner's
Dilemma. A common game where agents can choose to pay a slight cost to their
immediate utility in the hope of building a reputation. This has been used in
economic and evolutionary game theory to understand the evolution of cooperative
behaviour.

Recently, a class of strategies was described in~\cite{Press2012} that can
provably extort any given opponent. In~\cite{Hilbe2013, Moran1707} some
questions have already been asked about the true effectiveness of these
strategies in an evolutionary setting. Here another question is asked: is it
possible to recognise this extortionate behaviour? A mathematical procedure for
suspicion is presented: in the same way that the continued actions of an
extortionate individual might raise suspicion.

This work makes use of the Axelrod Python library~\cite{Knight2018, Knight2016}
with a large number of Prisoner Dilemma strategies available to give an
extensive numerical example of the ideas presented.  The approach is presented
in Section~\ref{sec:delta-zd-strategies}.  All of the code and data discussed
in Section~\ref{sec:numerical-experiments} is open sourced, archived and
written according to best scientific principles~\cite{Wilson2014}. The data
archive can be found at~\cite{vincent_knight_2018_1297075}.

\section{Recognising Extortion}\label{sec:delta-zd-strategies}

In~\cite{Press2012}, given a match between 2 memory-one strategies, the concept
of Zero Determinant (ZD) strategies is introduced. The main result of that paper
shows that given two memory one players \(p, q\in\mathbb{R}^4\) a linear
relationship between the players' scores could be forced by one of the players.

Using the notation of~\cite{Press2012}, assuming the utilities for player \(p\)
are given by \(S_x=(R, S, T, P)\) and for player \(q\) by \(S_y=(R, T, S, P)\)
and that the stationary scores of each player is given by \(S_X\) and \(S_Y\)
respectively. The main result of~\cite{Press2012} is that if

\begin{equation}\label{eqn:linear_relationship_for_p}
    \tilde p=\alpha S_x + \beta S_y + \gamma
\end{equation}

or

\begin{equation}\label{eqn:linear_relationship_for_q}
    \tilde q=\alpha S_x + \beta S_y + \gamma
\end{equation}

where \(\tilde p = (1 - p_1, 1 - p_2, p_3, p_4)\) and
\(\tilde q = (1 - q_1, 1 - q_2, q_3, q_4)\) then:

\begin{equation}
    \alpha S_X + \beta S_Y + \gamma = 0
\end{equation}

In~\cite{Press2012} a particular type of ZD strategy is defined: extortionate
strategies. If:

\begin{equation}\label{eqn:constraint_for_extortion}
    \gamma = - P(\alpha + \beta)
\end{equation}

then the player can ensure they get a score \(\chi\) times
larger than the opponent. This extortion coefficient is given by:

\begin{equation}\label{eqn:definition_of_chi}
    \chi=\frac{-\beta}{\alpha}
\end{equation}

Thus, if (\ref{eqn:constraint_for_extortion}) holds and \(\chi >1\) a player is
said to extort their opponent.
Here, the reverse problem is considered: given a
\(p\in\mathbb{R}^4\) how does one identify \(\alpha, \beta\) if they
exist and is the strategy in fact acting in an extortionate way?

These conditions correspond to:

\begin{align}
    \tilde p_1 & = \alpha R + \beta R - P (\alpha + \beta)
            \label{eqn:condition_for_tilde_p1}\\
    \tilde p_2 & = \alpha S + \beta T - P (\alpha + \beta)
            \label{eqn:condition_for_tilde_p2}\\
    \tilde p_3 & = \alpha T + \beta S - P (\alpha + \beta)
            \label{eqn:condition_for_tilde_p3}\\
    \tilde p_4 & = \alpha P + \beta P - P (\alpha + \beta)
            \label{eqn:condition_for_tilde_p4}
\end{align}

Equation (\ref{eqn:condition_for_tilde_p4}) ensures that \(p_4=\tilde p_4=0\).
Equations (\ref{eqn:condition_for_tilde_p1}-\ref{eqn:condition_for_tilde_p3})
can be used to eliminate \(\alpha, \beta\), giving:

\begin{equation}\label{eqn:planar_definition_of_extortion}
    \tilde p_1 = \frac{(R - P)(\tilde p_2 + \tilde p_3)}{S + T - 2P}
\end{equation}

with:

\begin{equation}\label{eqn:definition_of_chi}
    \chi = \frac{\tilde p_2 (P - T) + \tilde p_3 (S - P)}
                {\tilde p_2 (P - S) + \tilde p_3 (T - P)}
\end{equation}

Given a strategy \(p\in\mathbb{R}^{4\times 1}\) equations
(\ref{eqn:condition_for_tilde_p4}), (\ref{eqn:planar_definition_of_extortion}-\ref{eqn:definition_of_chi}) can be used to check if
a strategy is extortionate. The conditions correspond to:

\begin{align}
    p_1 & = \frac{(R-P)(p_2 + p_3) - R + T + S - P}{S + T - 2P}
     \label{eqn:condition_for_p1}\\
    p_4 & = 0 \label{eqn:condition_for_p4}\\
    1 & > p_2 + p_3\label{eqn:condition_for_chi}
\end{align}

The algebraic steps necessary to prove these results are available in the
supporting materials.

All extortionate strategies reside on a triangular (\ref{eqn:condition_for_chi})
plane (\ref{eqn:condition_for_p1}) in 3 dimensions (\ref{eqn:condition_for_p4}).
Using this formulation it can be seen that a necessary (but not sufficient)
condition for an extortionate strategy is that it cooperates on average less
than 50\% of the time when in a state of disagreement with the opponent.

As an example, consider the known extortionate strategy \(p=(8 / 9, 1 / 2, 1 /
3, 0)\) from~\cite{Stewart2012} which is referred to as \texttt{Extort-2}. In
this case, for the standard values of \((R, T, S, P)\) constraint
(\ref{eqn:condition_for_p1}) corresponds to:

\begin{equation}
    p_1 = \frac{2(p_2 + p_3) + 1}{3}
\end{equation}

It is clear that in this case all constraints hold.

This approach could in fact be used to confirm that a given strategy is acting
in an extortionate manner even if it is not a memory one strategy. However, in
practice, if a closed form for \(p\) is not known, then due to measurement
and/or numerical error this would not work.

This problem can be written in the following linear algebraic form where
\(x=(\alpha, \beta)\)
and \(p^*=(\tilde p_1 - 1, tilde_2 - 1, p_3)\):

\begin{equation}\label{eqn:linear_algebraic_equation_for_p}
    Cx= p^*
\end{equation}

\(C\) corresponds to equations
(\ref{eqn:condition_for_tilde_p1}-\ref{eqn:condition_for_tilde_p3}) and is
given by:

\begin{equation}\label{eqn:definition_of_C}
    C =
    \begin{bmatrix}
        R - P & R- P \\
        S - P & T- P \\
        T - P & S- P \\
    \end{bmatrix}
\end{equation}

Note that in general, equation (\ref{eqn:linear_algebraic_equation_for_p}) will
not necessarily have a solution. From the Rouch\'{e}-Capelli theorem if there is
a solution it is unique as \(\text{rank}(C)=2\) which is the dimension of the
variable \(x\). The best fitting \(x\) is found by minimizing:

\begin{equation}\label{eqn:r_squared}
    \text{SSError} = \|C x- p^*\|_2^2 = \sum_{i=1}^{3}\left((C\bar x)_i-p_i^*\right)^2
\end{equation}

Note that \(\text{SSError}\), which is the square of the Frobenius
norm~\cite{Golub2013}, becomes a measure of how close a strategy is to being an
extortionate strategy. Suspicion
of extortion then corresponds to a threshold on \(\text{SSError}\).

By observing interactions (human or otherwise), their memory one representation
can be inferred and this approach can be used to recognise extortionate
behaviour. The notion of comparing theoretic and actual plays of the IPD is not
novel, see for example~\cite{Rand2013}. Immediately it is noted that if the
environment is noisy~\cite{Wu1995} then no strategy can be considered to be
extortionate as \(p_4>0\).

In the next section, this idea will be illustrated by observing the interactions
that take place in a computer based tournament of the IPD\@.

\section{Numerical experiments}\label{sec:numerical-experiments}

In~\cite{Stewart2012} results from a tournament with
\input{./assets/tex/number_of_stewart_plotkin_strategies/main.tex} strategies,
was presented with specific consideration given to ZD strategies. This
tournament is reproduced here using the Axelrod-Python
project~\cite{Knight2016}. To obtain a good measure of the corresponding
transition rates for each strategy all matches have been run for
\input{assets/tex/number_of_turns/main.tex} turns and every match has been
repeated \input{assets/tex/number_of_repetitions/main.tex} times. All of this
interaction data is available at~\cite{vincent_knight_2018_1297075}. A good
match between the inferred Markov chain and the state distribution of the actual
interactions has been verified. Data for this is presented in the supplementary
materials.

Figure~\ref{fig:SSError_overall_in_stewart_plotkin} shows the \(\text{SSError}\)
values for all the strategies in the tournament, as reported
in~\cite{Stewart2012} the extortionate strategy (which has an expected
\(\text{SSError}\) approximately 0) gains a large number of wins.

\begin{figure}[!htbp]
    \centering
    \includegraphics[width=.8\textwidth]{./assets/img/SSError_overall_in_stewart_plotkin/main.pdf}
    \caption{\(\text{SSError}\) and state probabilities for the strategies
        of~\cite{Stewart2012}, ordered both by number of wins and overall score.
        Note that \(P(DC)\) is not shown as it corresponds to the transpose of
        \(P(CD)\). Cooperator and Defector are omitted as they do not visit all
        the states.}
    \label{fig:SSError_overall_in_stewart_plotkin}
\end{figure}

Here, the work of~\cite{Stewart2012} is extended by investigating a tournament
with \input{assets/tex/number_of_full_strategies/main.tex}
strategies.

The results of this analysis are shown in
Figure~\ref{fig:SSError_and_probabilities_in_full}. The top ranking strategies
by number of wins seem to be extortionate (but not against all strategies) and
it can be seen that a small sub group of strategies achieve mutual defection.
All the top ranking strategies according to score achieve mutual cooperation and
do not extort each other, however they
\textbf{do} exhibit extortionate behaviour towards a number of the lower ranking
strategies.

\begin{figure}[!htbp]
    \centering
    \includegraphics[width=.8\textwidth]{./assets/img/SSError_and_probabilities_in_full/main.pdf}
    \caption{\(\text{SSError}\) for the strategies for the full tournament. Only
    strategy interactions for which \(p_4=0\) and \(\chi>1\) are displayed.}
    \label{fig:SSError_and_probabilities_in_full}
\end{figure}

\section{Conclusion}\label{sec:conclusion}

This work defines an approach to measure whether or not a player is playing a
strategy that corresponds to an extortionate strategy as defined
in~\cite{Press2012}: a mathematical model for suspicion. Indeed, all
extortionate strategies have been
 classified as lying on a triangular plane.
This rigorous classification fails to be robust to small measurement error, thus
a statistical approach is proposed.
This is done through a linear algebraic approach for approximating the solution
of a linear system. Using this, a large number of pairwise interactions is
simulated and in fact very few strategies are found to act extortionately.

The work of~\cite{Press2012}, whilst showing that a clever approach to taking
advantage of another memory one strategy exists: this is incomplete. Whilst the
elegance of this result is very attractive, just as the simplicity of the
victory of Tit For Tat in Axelrod's original tournaments was, it is incomplete.
Extortionate strategies achieve a high number of wins but they do not
achieve a high score which corresponds to the fitness landscape in an
evolutionary sense. From the large number of interactions a payoff matrix \(S\)
can be measured where \(S_{ij}\) denotes the score (using standard values of
\((R, S, T, P) = (3, 0, 5, 1)\)) of the \(i\)th strategy
against the \(j\)th strategy. Using this, the replicator equation
describes the evolution of the system based on a population density fitness
function:

\begin{equation}\label{eqn:replicator_dynamics}
    \frac{dx}{dt} = x(S-x^TS x)
\end{equation}

Equation (\ref{eqn:replicator_dynamics}) is solved numerically through an
integration technique described in~\cite{Petzold1983} and
Figure~\ref{fig:replicator_dynamics} shows the evolution of the distribution of
the system: the various strategies are ranked by scores. It is clear to see that
only the high ranking strategies survive the evolutionary process (in fact,
only \input{./assets/img/replicator_dynamics/main.tex}
have a final distribution greater than \(10 ^ {-2}\)). This confirms the
findings of~\cite{Moran1707} in which sophisticated strategies resist
evolutionary invasion of shorter memory strategies. Recalling
Figure~\ref{fig:SSError_and_probabilities_in_full} this demonstrates that:

\begin{itemize}
    \item Cooperation emerges through the evolutionary process: the high scoring
        strategies do not exhibit extortionate behaviour towards each other.
    \item Extortionate strategies do not survive the evolutionary process.
\end{itemize}

\begin{figure}[!htbp]
    \centering
    \includegraphics[width=.8\textwidth]{./assets/img/replicator_dynamics/main.pdf}
    \caption{Numerical simulation of the replicator equation
    (\ref{eqn:replicator_dynamics}): strategies are ordered by score, only the strategies with a high score survive the evolutionary process.}
    \label{fig:replicator_dynamics}
\end{figure}

This work can be used to classify plays of the IPD\@: data can be collected from
actual interactions (in lab or in the field). Furthermore, this allows for a
classification method similar to the notion of fingerprinting presented
in~\cite{Ashlock2008}. Trained strategies can potentially be classified as
extortionate or not or it could be possible to even constrain the reinforcement
learning approaches that are becoming prevalent in the literature.
Alternatively, this mathematical approach for recognising extortion could be
used in sophisticated strategies to defend against invasion. Arguably, some of
the strategies considered here exhibit this behaviour, indeed as described
in~\cite{Harper2017}, the top ranking strategies in the full tournament are
obtained using evolutionary reinforcement learning techniques, thus, suspicion
of extortionate behaviour could in fact be an evolutionary trait.

\section*{Acknowledgements}

The following open source software libraries were used in this research:

\begin{itemize}
    \item The Axelrod ~\cite{Knight2016, Knight2018} library (IPD strategies and
        tournaments).
    \item The sympy library~\cite{Meurer2017} (verification of all symbolic
        calculations).
    \item The matplotlib~\cite{Droettboom2018} library (visualisation).
    \item The pandas~\cite{Structures2010}, dask~\cite{Dask2016} and
        NumPy~\cite{Oliphant2015} libraries (data manipulation).
    \item The SciPy~\cite{Jones2001} library (numerical integration of the
        replicator equation).
\end{itemize}

This work was performed using the computational facilities of the Advanced
Research Computing @ Cardiff (ARCCA) Division, Cardiff University.

\printbibliography

\newpage
\section*{Supplementary materials}

\includepdf{assets/pdf/proof_of_form_of_extortionate_strategies/main.pdf}

\newpage

Using the pair wise interactions the transition rates \(p,
q\) can be measured and the steady state probabilities inferred and compared to
the actual probabilities of each state.
This is done numerically by computing the singular eigenvector of the
matrix \(A\) \cite{Stewart2009}:

\[
    A =
    \begin{bmatrix}
        p_1 q_1 & p_1 (1 - q_1) & (1 - p_1) q_1 & (1 -p_1) (1 - q_1) \\
        p_2 q_2 & p_2 (1 - q_2) & (1 - p_2) q_2 & (1 -p_2) (1 - q_2) \\
        p_3 q_3 & p_3 (1 - q_3) & (1 - p_3) q_3 & (1 -p_3) (1 - q_3) \\
        p_4 q_4 & p_4 (1 - q_4) & (1 - p_4) q_4 & (1 -p_4) (1 - q_4) \\
    \end{bmatrix}
\]

Figure~\ref{fig:computed_probabilities_vs_theoretic_probabilities} shows a
regression line fitted to every pairwise interaction with a reported
\(\text{SSError}\) value (pairwise interactions with missing states were
omitted). This serves to validate the approach: a part from some edge cases the
relationship is consistent.

\begin{figure}[!htbp]
    \centering
    \includegraphics[width=.8\textwidth]{./assets/img/computed_probabilities_vs_theoretic_probabilities/main.pdf}
    \caption{The
        relationship between the steady state probabilities inferred from the
        measured transitions and the actual steady state probabilities. A linear
        regression line is included validating the approach.}
    \label{fig:computed_probabilities_vs_theoretic_probabilities}
\end{figure}


\end{document}

    strategies is considered. In this setting
    the most highly performing strategies do not play in an extortionate way
    against each other but do against lower performing strategies.
    This suggests that whilst the theory of Zero Determinant strategies
    indicates that memory is not of fundamental importance to the evolution of
    cooperative behaviour, this is incomplete.
\end{abstract}

\section{Introduction}\label{sec:introduction}

Agent based game theoretic models have become a stalwart of the underpinning
mathematics of interactive behaviours. One of the major pieces of work
in this area is the pair of original computer tournaments run by Robert
Axelrod~\cite{Axelrod1980, Axelrod1980a}. These tournaments pitted submitted
computer strategies against each other in plays of the Iterated Prisoner's
Dilemma. A common game where agents can choose to pay a slight cost to their
immediate utility in the hope of building a reputation. This has been used in
economic and evolutionary game theory to understand the evolution of cooperative
behaviour.

Recently, a class of strategies was described in~\cite{Press2012} that can
provably extort any given opponent. In~\cite{Hilbe2013, Moran1707} some
questions have already been asked about the true effectiveness of these
strategies in an evolutionary setting. Here another question is asked: is it
possible to recognise this extortionate behaviour? A mathematical procedure for
suspicion is presented: in the same way that the continued actions of an
extortionate individual might raise suspicion.

This work makes use of the Axelrod Python library~\cite{Knight2018, Knight2016}
with a large number of Prisoner Dilemma strategies available to give an
extensive numerical example of the ideas presented.  The approach is presented
in Section~\ref{sec:delta-zd-strategies}.  All of the code and data discussed
in Section~\ref{sec:numerical-experiments} is open sourced, archived and
written according to best scientific principles~\cite{Wilson2014}. The data
archive can be found at~\cite{vincent_knight_2018_1297075}.

\section{Recognising Extortion}\label{sec:delta-zd-strategies}

In~\cite{Press2012}, given a match between 2 memory-one strategies, the concept
of Zero Determinant (ZD) strategies is introduced. The main result of that paper
shows that given two memory one players \(p, q\in\mathbb{R}^4\) a linear
relationship between the players' scores could be forced by one of the players.

Using the notation of~\cite{Press2012}, assuming the utilities for player \(p\)
are given by \(S_x=(R, S, T, P)\) and for player \(q\) by \(S_y=(R, T, S, P)\)
and that the stationary scores of each player is given by \(S_X\) and \(S_Y\)
respectively. The main result of~\cite{Press2012} is that if

\begin{equation}\label{eqn:linear_relationship_for_p}
    \tilde p=\alpha S_x + \beta S_y + \gamma
\end{equation}

or

\begin{equation}\label{eqn:linear_relationship_for_q}
    \tilde q=\alpha S_x + \beta S_y + \gamma
\end{equation}

where \(\tilde p = (1 - p_1, 1 - p_2, p_3, p_4)\) and
\(\tilde q = (1 - q_1, 1 - q_2, q_3, q_4)\) then:

\begin{equation}
    \alpha S_X + \beta S_Y + \gamma = 0
\end{equation}

In~\cite{Press2012} a particular type of ZD strategy is defined: extortionate
strategies. If:

\begin{equation}\label{eqn:constraint_for_extortion}
    \gamma = - P(\alpha + \beta)
\end{equation}

then the player can ensure they get a score \(\chi\) times
larger than the opponent. This extortion coefficient is given by:

\begin{equation}\label{eqn:definition_of_chi}
    \chi=\frac{-\beta}{\alpha}
\end{equation}

Thus, if (\ref{eqn:constraint_for_extortion}) holds and \(\chi >1\) a player is
said to extort their opponent.
Here, the reverse problem is considered: given a
\(p\in\mathbb{R}^4\) how does one identify \(\alpha, \beta\) if they
exist and is the strategy in fact acting in an extortionate way?

These conditions correspond to:

\begin{align}
    \tilde p_1 & = \alpha R + \beta R - P (\alpha + \beta)
            \label{eqn:condition_for_tilde_p1}\\
    \tilde p_2 & = \alpha S + \beta T - P (\alpha + \beta)
            \label{eqn:condition_for_tilde_p2}\\
    \tilde p_3 & = \alpha T + \beta S - P (\alpha + \beta)
            \label{eqn:condition_for_tilde_p3}\\
    \tilde p_4 & = \alpha P + \beta P - P (\alpha + \beta)
            \label{eqn:condition_for_tilde_p4}
\end{align}

Equation (\ref{eqn:condition_for_tilde_p4}) ensures that \(p_4=\tilde p_4=0\).
Equations (\ref{eqn:condition_for_tilde_p1}-\ref{eqn:condition_for_tilde_p3})
can be used to eliminate \(\alpha, \beta\), giving:

\begin{equation}\label{eqn:planar_definition_of_extortion}
    \tilde p_1 = \frac{(R - P)(\tilde p_2 + \tilde p_3)}{S + T - 2P}
\end{equation}

with:

\begin{equation}\label{eqn:definition_of_chi}
    \chi = \frac{\tilde p_2 (P - T) + \tilde p_3 (S - P)}
                {\tilde p_2 (P - S) + \tilde p_3 (T - P)}
\end{equation}

Given a strategy \(p\in\mathbb{R}^{4\times 1}\) equations
(\ref{eqn:condition_for_tilde_p4}), (\ref{eqn:planar_definition_of_extortion}-\ref{eqn:definition_of_chi}) can be used to check if
a strategy is extortionate. The conditions correspond to:

\begin{align}
    p_1 & = \frac{(R-P)(p_2 + p_3) - R + T + S - P}{S + T - 2P}
     \label{eqn:condition_for_p1}\\
    p_4 & = 0 \label{eqn:condition_for_p4}\\
    1 & > p_2 + p_3\label{eqn:condition_for_chi}
\end{align}

The algebraic steps necessary to prove these results are available in the
supporting materials.

All extortionate strategies reside on a triangular (\ref{eqn:condition_for_chi})
plane (\ref{eqn:condition_for_p1}) in 3 dimensions (\ref{eqn:condition_for_p4}).
Using this formulation it can be seen that a necessary (but not sufficient)
condition for an extortionate strategy is that it cooperates on average less
than 50\% of the time when in a state of disagreement with the opponent.

As an example, consider the known extortionate strategy \(p=(8 / 9, 1 / 2, 1 /
3, 0)\) from~\cite{Stewart2012} which is referred to as \texttt{Extort-2}. In
this case, for the standard values of \((R, T, S, P)\) constraint
(\ref{eqn:condition_for_p1}) corresponds to:

\begin{equation}
    p_1 = \frac{2(p_2 + p_3) + 1}{3}
\end{equation}

It is clear that in this case all constraints hold.

This approach could in fact be used to confirm that a given strategy is acting
in an extortionate manner even if it is not a memory one strategy. However, in
practice, if a closed form for \(p\) is not known, then due to measurement
and/or numerical error this would not work.

This problem can be written in the following linear algebraic form where
\(x=(\alpha, \beta)\)
and \(p^*=(\tilde p_1 - 1, tilde_2 - 1, p_3)\):

\begin{equation}\label{eqn:linear_algebraic_equation_for_p}
    Cx= p^*
\end{equation}

\(C\) corresponds to equations
(\ref{eqn:condition_for_tilde_p1}-\ref{eqn:condition_for_tilde_p3}) and is
given by:

\begin{equation}\label{eqn:definition_of_C}
    C =
    \begin{bmatrix}
        R - P & R- P \\
        S - P & T- P \\
        T - P & S- P \\
    \end{bmatrix}
\end{equation}

Note that in general, equation (\ref{eqn:linear_algebraic_equation_for_p}) will
not necessarily have a solution. From the Rouch\'{e}-Capelli theorem if there is
a solution it is unique as \(\text{rank}(C)=2\) which is the dimension of the
variable \(x\). The best fitting \(x\) is found by minimizing:

\begin{equation}\label{eqn:r_squared}
    \text{SSError} = \|C x- p^*\|_2^2 = \sum_{i=1}^{3}\left((C\bar x)_i-p_i^*\right)^2
\end{equation}

Note that \(\text{SSError}\), which is the square of the Frobenius
norm~\cite{Golub2013}, becomes a measure of how close a strategy is to being an
extortionate strategy. Suspicion
of extortion then corresponds to a threshold on \(\text{SSError}\).

By observing interactions (human or otherwise), their memory one representation
can be inferred and this approach can be used to recognise extortionate
behaviour. The notion of comparing theoretic and actual plays of the IPD is not
novel, see for example~\cite{Rand2013}. Immediately it is noted that if the
environment is noisy~\cite{Wu1995} then no strategy can be considered to be
extortionate as \(p_4>0\).

In the next section, this idea will be illustrated by observing the interactions
that take place in a computer based tournament of the IPD\@.

\section{Numerical experiments}\label{sec:numerical-experiments}

In~\cite{Stewart2012} results from a tournament with
\documentclass[a4paper]{article}

\usepackage{amsmath}
\usepackage{amssymb}
\usepackage[margin=1.5cm,
            includefoot,
            footskip=30pt]{geometry}
\usepackage{layout}
\usepackage{graphicx}
\usepackage{subcaption}

\usepackage{biblatex}
\usepackage{pdfpages}

\bibliography{main.bib}

\title{Suspicion: Recognising and evaluating the effectiveness
       of extortion in the Iterated Prisoner's Dilemma}
\author{Vincent A. Knight \and Nikoleta E. Glynatsi}
\date{\today}



\begin{document}

\maketitle

\begin{abstract}
    The Iterated Prisoner's Dilemma is a model for rational and evolutionary
    interactive behaviour. It has applications both in the study of human social
    behaviour as well as in biology.
    It is used to understand when and how a rational individual might
    accept an immediate cost to their own utility for the direct benefit of
    another.

    Much attention has been given to a class of strategies called
    Zero Determinant strategies. It has been theoretically shown that these
    strategies can ``extort'' any player.

    In this work, an approach to identify if observed strategies are playing in
    an extortionate way is described. Furthermore, experimental analysis of
    a large tournament with \input{assets/tex/number_of_full_strategies/main.tex}
    strategies is considered. In this setting
    the most highly performing strategies do not play in an extortionate way
    against each other but do against lower performing strategies.
    This suggests that whilst the theory of Zero Determinant strategies
    indicates that memory is not of fundamental importance to the evolution of
    cooperative behaviour, this is incomplete.
\end{abstract}

\section{Introduction}\label{sec:introduction}

Agent based game theoretic models have become a stalwart of the underpinning
mathematics of interactive behaviours. One of the major pieces of work
in this area is the pair of original computer tournaments run by Robert
Axelrod~\cite{Axelrod1980, Axelrod1980a}. These tournaments pitted submitted
computer strategies against each other in plays of the Iterated Prisoner's
Dilemma. A common game where agents can choose to pay a slight cost to their
immediate utility in the hope of building a reputation. This has been used in
economic and evolutionary game theory to understand the evolution of cooperative
behaviour.

Recently, a class of strategies was described in~\cite{Press2012} that can
provably extort any given opponent. In~\cite{Hilbe2013, Moran1707} some
questions have already been asked about the true effectiveness of these
strategies in an evolutionary setting. Here another question is asked: is it
possible to recognise this extortionate behaviour? A mathematical procedure for
suspicion is presented: in the same way that the continued actions of an
extortionate individual might raise suspicion.

This work makes use of the Axelrod Python library~\cite{Knight2018, Knight2016}
with a large number of Prisoner Dilemma strategies available to give an
extensive numerical example of the ideas presented.  The approach is presented
in Section~\ref{sec:delta-zd-strategies}.  All of the code and data discussed
in Section~\ref{sec:numerical-experiments} is open sourced, archived and
written according to best scientific principles~\cite{Wilson2014}. The data
archive can be found at~\cite{vincent_knight_2018_1297075}.

\section{Recognising Extortion}\label{sec:delta-zd-strategies}

In~\cite{Press2012}, given a match between 2 memory-one strategies, the concept
of Zero Determinant (ZD) strategies is introduced. The main result of that paper
shows that given two memory one players \(p, q\in\mathbb{R}^4\) a linear
relationship between the players' scores could be forced by one of the players.

Using the notation of~\cite{Press2012}, assuming the utilities for player \(p\)
are given by \(S_x=(R, S, T, P)\) and for player \(q\) by \(S_y=(R, T, S, P)\)
and that the stationary scores of each player is given by \(S_X\) and \(S_Y\)
respectively. The main result of~\cite{Press2012} is that if

\begin{equation}\label{eqn:linear_relationship_for_p}
    \tilde p=\alpha S_x + \beta S_y + \gamma
\end{equation}

or

\begin{equation}\label{eqn:linear_relationship_for_q}
    \tilde q=\alpha S_x + \beta S_y + \gamma
\end{equation}

where \(\tilde p = (1 - p_1, 1 - p_2, p_3, p_4)\) and
\(\tilde q = (1 - q_1, 1 - q_2, q_3, q_4)\) then:

\begin{equation}
    \alpha S_X + \beta S_Y + \gamma = 0
\end{equation}

In~\cite{Press2012} a particular type of ZD strategy is defined: extortionate
strategies. If:

\begin{equation}\label{eqn:constraint_for_extortion}
    \gamma = - P(\alpha + \beta)
\end{equation}

then the player can ensure they get a score \(\chi\) times
larger than the opponent. This extortion coefficient is given by:

\begin{equation}\label{eqn:definition_of_chi}
    \chi=\frac{-\beta}{\alpha}
\end{equation}

Thus, if (\ref{eqn:constraint_for_extortion}) holds and \(\chi >1\) a player is
said to extort their opponent.
Here, the reverse problem is considered: given a
\(p\in\mathbb{R}^4\) how does one identify \(\alpha, \beta\) if they
exist and is the strategy in fact acting in an extortionate way?

These conditions correspond to:

\begin{align}
    \tilde p_1 & = \alpha R + \beta R - P (\alpha + \beta)
            \label{eqn:condition_for_tilde_p1}\\
    \tilde p_2 & = \alpha S + \beta T - P (\alpha + \beta)
            \label{eqn:condition_for_tilde_p2}\\
    \tilde p_3 & = \alpha T + \beta S - P (\alpha + \beta)
            \label{eqn:condition_for_tilde_p3}\\
    \tilde p_4 & = \alpha P + \beta P - P (\alpha + \beta)
            \label{eqn:condition_for_tilde_p4}
\end{align}

Equation (\ref{eqn:condition_for_tilde_p4}) ensures that \(p_4=\tilde p_4=0\).
Equations (\ref{eqn:condition_for_tilde_p1}-\ref{eqn:condition_for_tilde_p3})
can be used to eliminate \(\alpha, \beta\), giving:

\begin{equation}\label{eqn:planar_definition_of_extortion}
    \tilde p_1 = \frac{(R - P)(\tilde p_2 + \tilde p_3)}{S + T - 2P}
\end{equation}

with:

\begin{equation}\label{eqn:definition_of_chi}
    \chi = \frac{\tilde p_2 (P - T) + \tilde p_3 (S - P)}
                {\tilde p_2 (P - S) + \tilde p_3 (T - P)}
\end{equation}

Given a strategy \(p\in\mathbb{R}^{4\times 1}\) equations
(\ref{eqn:condition_for_tilde_p4}), (\ref{eqn:planar_definition_of_extortion}-\ref{eqn:definition_of_chi}) can be used to check if
a strategy is extortionate. The conditions correspond to:

\begin{align}
    p_1 & = \frac{(R-P)(p_2 + p_3) - R + T + S - P}{S + T - 2P}
     \label{eqn:condition_for_p1}\\
    p_4 & = 0 \label{eqn:condition_for_p4}\\
    1 & > p_2 + p_3\label{eqn:condition_for_chi}
\end{align}

The algebraic steps necessary to prove these results are available in the
supporting materials.

All extortionate strategies reside on a triangular (\ref{eqn:condition_for_chi})
plane (\ref{eqn:condition_for_p1}) in 3 dimensions (\ref{eqn:condition_for_p4}).
Using this formulation it can be seen that a necessary (but not sufficient)
condition for an extortionate strategy is that it cooperates on average less
than 50\% of the time when in a state of disagreement with the opponent.

As an example, consider the known extortionate strategy \(p=(8 / 9, 1 / 2, 1 /
3, 0)\) from~\cite{Stewart2012} which is referred to as \texttt{Extort-2}. In
this case, for the standard values of \((R, T, S, P)\) constraint
(\ref{eqn:condition_for_p1}) corresponds to:

\begin{equation}
    p_1 = \frac{2(p_2 + p_3) + 1}{3}
\end{equation}

It is clear that in this case all constraints hold.

This approach could in fact be used to confirm that a given strategy is acting
in an extortionate manner even if it is not a memory one strategy. However, in
practice, if a closed form for \(p\) is not known, then due to measurement
and/or numerical error this would not work.

This problem can be written in the following linear algebraic form where
\(x=(\alpha, \beta)\)
and \(p^*=(\tilde p_1 - 1, tilde_2 - 1, p_3)\):

\begin{equation}\label{eqn:linear_algebraic_equation_for_p}
    Cx= p^*
\end{equation}

\(C\) corresponds to equations
(\ref{eqn:condition_for_tilde_p1}-\ref{eqn:condition_for_tilde_p3}) and is
given by:

\begin{equation}\label{eqn:definition_of_C}
    C =
    \begin{bmatrix}
        R - P & R- P \\
        S - P & T- P \\
        T - P & S- P \\
    \end{bmatrix}
\end{equation}

Note that in general, equation (\ref{eqn:linear_algebraic_equation_for_p}) will
not necessarily have a solution. From the Rouch\'{e}-Capelli theorem if there is
a solution it is unique as \(\text{rank}(C)=2\) which is the dimension of the
variable \(x\). The best fitting \(x\) is found by minimizing:

\begin{equation}\label{eqn:r_squared}
    \text{SSError} = \|C x- p^*\|_2^2 = \sum_{i=1}^{3}\left((C\bar x)_i-p_i^*\right)^2
\end{equation}

Note that \(\text{SSError}\), which is the square of the Frobenius
norm~\cite{Golub2013}, becomes a measure of how close a strategy is to being an
extortionate strategy. Suspicion
of extortion then corresponds to a threshold on \(\text{SSError}\).

By observing interactions (human or otherwise), their memory one representation
can be inferred and this approach can be used to recognise extortionate
behaviour. The notion of comparing theoretic and actual plays of the IPD is not
novel, see for example~\cite{Rand2013}. Immediately it is noted that if the
environment is noisy~\cite{Wu1995} then no strategy can be considered to be
extortionate as \(p_4>0\).

In the next section, this idea will be illustrated by observing the interactions
that take place in a computer based tournament of the IPD\@.

\section{Numerical experiments}\label{sec:numerical-experiments}

In~\cite{Stewart2012} results from a tournament with
\input{./assets/tex/number_of_stewart_plotkin_strategies/main.tex} strategies,
was presented with specific consideration given to ZD strategies. This
tournament is reproduced here using the Axelrod-Python
project~\cite{Knight2016}. To obtain a good measure of the corresponding
transition rates for each strategy all matches have been run for
\input{assets/tex/number_of_turns/main.tex} turns and every match has been
repeated \input{assets/tex/number_of_repetitions/main.tex} times. All of this
interaction data is available at~\cite{vincent_knight_2018_1297075}. A good
match between the inferred Markov chain and the state distribution of the actual
interactions has been verified. Data for this is presented in the supplementary
materials.

Figure~\ref{fig:SSError_overall_in_stewart_plotkin} shows the \(\text{SSError}\)
values for all the strategies in the tournament, as reported
in~\cite{Stewart2012} the extortionate strategy (which has an expected
\(\text{SSError}\) approximately 0) gains a large number of wins.

\begin{figure}[!htbp]
    \centering
    \includegraphics[width=.8\textwidth]{./assets/img/SSError_overall_in_stewart_plotkin/main.pdf}
    \caption{\(\text{SSError}\) and state probabilities for the strategies
        of~\cite{Stewart2012}, ordered both by number of wins and overall score.
        Note that \(P(DC)\) is not shown as it corresponds to the transpose of
        \(P(CD)\). Cooperator and Defector are omitted as they do not visit all
        the states.}
    \label{fig:SSError_overall_in_stewart_plotkin}
\end{figure}

Here, the work of~\cite{Stewart2012} is extended by investigating a tournament
with \input{assets/tex/number_of_full_strategies/main.tex}
strategies.

The results of this analysis are shown in
Figure~\ref{fig:SSError_and_probabilities_in_full}. The top ranking strategies
by number of wins seem to be extortionate (but not against all strategies) and
it can be seen that a small sub group of strategies achieve mutual defection.
All the top ranking strategies according to score achieve mutual cooperation and
do not extort each other, however they
\textbf{do} exhibit extortionate behaviour towards a number of the lower ranking
strategies.

\begin{figure}[!htbp]
    \centering
    \includegraphics[width=.8\textwidth]{./assets/img/SSError_and_probabilities_in_full/main.pdf}
    \caption{\(\text{SSError}\) for the strategies for the full tournament. Only
    strategy interactions for which \(p_4=0\) and \(\chi>1\) are displayed.}
    \label{fig:SSError_and_probabilities_in_full}
\end{figure}

\section{Conclusion}\label{sec:conclusion}

This work defines an approach to measure whether or not a player is playing a
strategy that corresponds to an extortionate strategy as defined
in~\cite{Press2012}: a mathematical model for suspicion. Indeed, all
extortionate strategies have been
 classified as lying on a triangular plane.
This rigorous classification fails to be robust to small measurement error, thus
a statistical approach is proposed.
This is done through a linear algebraic approach for approximating the solution
of a linear system. Using this, a large number of pairwise interactions is
simulated and in fact very few strategies are found to act extortionately.

The work of~\cite{Press2012}, whilst showing that a clever approach to taking
advantage of another memory one strategy exists: this is incomplete. Whilst the
elegance of this result is very attractive, just as the simplicity of the
victory of Tit For Tat in Axelrod's original tournaments was, it is incomplete.
Extortionate strategies achieve a high number of wins but they do not
achieve a high score which corresponds to the fitness landscape in an
evolutionary sense. From the large number of interactions a payoff matrix \(S\)
can be measured where \(S_{ij}\) denotes the score (using standard values of
\((R, S, T, P) = (3, 0, 5, 1)\)) of the \(i\)th strategy
against the \(j\)th strategy. Using this, the replicator equation
describes the evolution of the system based on a population density fitness
function:

\begin{equation}\label{eqn:replicator_dynamics}
    \frac{dx}{dt} = x(S-x^TS x)
\end{equation}

Equation (\ref{eqn:replicator_dynamics}) is solved numerically through an
integration technique described in~\cite{Petzold1983} and
Figure~\ref{fig:replicator_dynamics} shows the evolution of the distribution of
the system: the various strategies are ranked by scores. It is clear to see that
only the high ranking strategies survive the evolutionary process (in fact,
only \input{./assets/img/replicator_dynamics/main.tex}
have a final distribution greater than \(10 ^ {-2}\)). This confirms the
findings of~\cite{Moran1707} in which sophisticated strategies resist
evolutionary invasion of shorter memory strategies. Recalling
Figure~\ref{fig:SSError_and_probabilities_in_full} this demonstrates that:

\begin{itemize}
    \item Cooperation emerges through the evolutionary process: the high scoring
        strategies do not exhibit extortionate behaviour towards each other.
    \item Extortionate strategies do not survive the evolutionary process.
\end{itemize}

\begin{figure}[!htbp]
    \centering
    \includegraphics[width=.8\textwidth]{./assets/img/replicator_dynamics/main.pdf}
    \caption{Numerical simulation of the replicator equation
    (\ref{eqn:replicator_dynamics}): strategies are ordered by score, only the strategies with a high score survive the evolutionary process.}
    \label{fig:replicator_dynamics}
\end{figure}

This work can be used to classify plays of the IPD\@: data can be collected from
actual interactions (in lab or in the field). Furthermore, this allows for a
classification method similar to the notion of fingerprinting presented
in~\cite{Ashlock2008}. Trained strategies can potentially be classified as
extortionate or not or it could be possible to even constrain the reinforcement
learning approaches that are becoming prevalent in the literature.
Alternatively, this mathematical approach for recognising extortion could be
used in sophisticated strategies to defend against invasion. Arguably, some of
the strategies considered here exhibit this behaviour, indeed as described
in~\cite{Harper2017}, the top ranking strategies in the full tournament are
obtained using evolutionary reinforcement learning techniques, thus, suspicion
of extortionate behaviour could in fact be an evolutionary trait.

\section*{Acknowledgements}

The following open source software libraries were used in this research:

\begin{itemize}
    \item The Axelrod ~\cite{Knight2016, Knight2018} library (IPD strategies and
        tournaments).
    \item The sympy library~\cite{Meurer2017} (verification of all symbolic
        calculations).
    \item The matplotlib~\cite{Droettboom2018} library (visualisation).
    \item The pandas~\cite{Structures2010}, dask~\cite{Dask2016} and
        NumPy~\cite{Oliphant2015} libraries (data manipulation).
    \item The SciPy~\cite{Jones2001} library (numerical integration of the
        replicator equation).
\end{itemize}

This work was performed using the computational facilities of the Advanced
Research Computing @ Cardiff (ARCCA) Division, Cardiff University.

\printbibliography

\newpage
\section*{Supplementary materials}

\includepdf{assets/pdf/proof_of_form_of_extortionate_strategies/main.pdf}

\newpage

Using the pair wise interactions the transition rates \(p,
q\) can be measured and the steady state probabilities inferred and compared to
the actual probabilities of each state.
This is done numerically by computing the singular eigenvector of the
matrix \(A\) \cite{Stewart2009}:

\[
    A =
    \begin{bmatrix}
        p_1 q_1 & p_1 (1 - q_1) & (1 - p_1) q_1 & (1 -p_1) (1 - q_1) \\
        p_2 q_2 & p_2 (1 - q_2) & (1 - p_2) q_2 & (1 -p_2) (1 - q_2) \\
        p_3 q_3 & p_3 (1 - q_3) & (1 - p_3) q_3 & (1 -p_3) (1 - q_3) \\
        p_4 q_4 & p_4 (1 - q_4) & (1 - p_4) q_4 & (1 -p_4) (1 - q_4) \\
    \end{bmatrix}
\]

Figure~\ref{fig:computed_probabilities_vs_theoretic_probabilities} shows a
regression line fitted to every pairwise interaction with a reported
\(\text{SSError}\) value (pairwise interactions with missing states were
omitted). This serves to validate the approach: a part from some edge cases the
relationship is consistent.

\begin{figure}[!htbp]
    \centering
    \includegraphics[width=.8\textwidth]{./assets/img/computed_probabilities_vs_theoretic_probabilities/main.pdf}
    \caption{The
        relationship between the steady state probabilities inferred from the
        measured transitions and the actual steady state probabilities. A linear
        regression line is included validating the approach.}
    \label{fig:computed_probabilities_vs_theoretic_probabilities}
\end{figure}


\end{document}
 strategies,
was presented with specific consideration given to ZD strategies. This
tournament is reproduced here using the Axelrod-Python
project~\cite{Knight2016}. To obtain a good measure of the corresponding
transition rates for each strategy all matches have been run for
\documentclass[a4paper]{article}

\usepackage{amsmath}
\usepackage{amssymb}
\usepackage[margin=1.5cm,
            includefoot,
            footskip=30pt]{geometry}
\usepackage{layout}
\usepackage{graphicx}
\usepackage{subcaption}

\usepackage{biblatex}
\usepackage{pdfpages}

\bibliography{main.bib}

\title{Suspicion: Recognising and evaluating the effectiveness
       of extortion in the Iterated Prisoner's Dilemma}
\author{Vincent A. Knight \and Nikoleta E. Glynatsi}
\date{\today}



\begin{document}

\maketitle

\begin{abstract}
    The Iterated Prisoner's Dilemma is a model for rational and evolutionary
    interactive behaviour. It has applications both in the study of human social
    behaviour as well as in biology.
    It is used to understand when and how a rational individual might
    accept an immediate cost to their own utility for the direct benefit of
    another.

    Much attention has been given to a class of strategies called
    Zero Determinant strategies. It has been theoretically shown that these
    strategies can ``extort'' any player.

    In this work, an approach to identify if observed strategies are playing in
    an extortionate way is described. Furthermore, experimental analysis of
    a large tournament with \input{assets/tex/number_of_full_strategies/main.tex}
    strategies is considered. In this setting
    the most highly performing strategies do not play in an extortionate way
    against each other but do against lower performing strategies.
    This suggests that whilst the theory of Zero Determinant strategies
    indicates that memory is not of fundamental importance to the evolution of
    cooperative behaviour, this is incomplete.
\end{abstract}

\section{Introduction}\label{sec:introduction}

Agent based game theoretic models have become a stalwart of the underpinning
mathematics of interactive behaviours. One of the major pieces of work
in this area is the pair of original computer tournaments run by Robert
Axelrod~\cite{Axelrod1980, Axelrod1980a}. These tournaments pitted submitted
computer strategies against each other in plays of the Iterated Prisoner's
Dilemma. A common game where agents can choose to pay a slight cost to their
immediate utility in the hope of building a reputation. This has been used in
economic and evolutionary game theory to understand the evolution of cooperative
behaviour.

Recently, a class of strategies was described in~\cite{Press2012} that can
provably extort any given opponent. In~\cite{Hilbe2013, Moran1707} some
questions have already been asked about the true effectiveness of these
strategies in an evolutionary setting. Here another question is asked: is it
possible to recognise this extortionate behaviour? A mathematical procedure for
suspicion is presented: in the same way that the continued actions of an
extortionate individual might raise suspicion.

This work makes use of the Axelrod Python library~\cite{Knight2018, Knight2016}
with a large number of Prisoner Dilemma strategies available to give an
extensive numerical example of the ideas presented.  The approach is presented
in Section~\ref{sec:delta-zd-strategies}.  All of the code and data discussed
in Section~\ref{sec:numerical-experiments} is open sourced, archived and
written according to best scientific principles~\cite{Wilson2014}. The data
archive can be found at~\cite{vincent_knight_2018_1297075}.

\section{Recognising Extortion}\label{sec:delta-zd-strategies}

In~\cite{Press2012}, given a match between 2 memory-one strategies, the concept
of Zero Determinant (ZD) strategies is introduced. The main result of that paper
shows that given two memory one players \(p, q\in\mathbb{R}^4\) a linear
relationship between the players' scores could be forced by one of the players.

Using the notation of~\cite{Press2012}, assuming the utilities for player \(p\)
are given by \(S_x=(R, S, T, P)\) and for player \(q\) by \(S_y=(R, T, S, P)\)
and that the stationary scores of each player is given by \(S_X\) and \(S_Y\)
respectively. The main result of~\cite{Press2012} is that if

\begin{equation}\label{eqn:linear_relationship_for_p}
    \tilde p=\alpha S_x + \beta S_y + \gamma
\end{equation}

or

\begin{equation}\label{eqn:linear_relationship_for_q}
    \tilde q=\alpha S_x + \beta S_y + \gamma
\end{equation}

where \(\tilde p = (1 - p_1, 1 - p_2, p_3, p_4)\) and
\(\tilde q = (1 - q_1, 1 - q_2, q_3, q_4)\) then:

\begin{equation}
    \alpha S_X + \beta S_Y + \gamma = 0
\end{equation}

In~\cite{Press2012} a particular type of ZD strategy is defined: extortionate
strategies. If:

\begin{equation}\label{eqn:constraint_for_extortion}
    \gamma = - P(\alpha + \beta)
\end{equation}

then the player can ensure they get a score \(\chi\) times
larger than the opponent. This extortion coefficient is given by:

\begin{equation}\label{eqn:definition_of_chi}
    \chi=\frac{-\beta}{\alpha}
\end{equation}

Thus, if (\ref{eqn:constraint_for_extortion}) holds and \(\chi >1\) a player is
said to extort their opponent.
Here, the reverse problem is considered: given a
\(p\in\mathbb{R}^4\) how does one identify \(\alpha, \beta\) if they
exist and is the strategy in fact acting in an extortionate way?

These conditions correspond to:

\begin{align}
    \tilde p_1 & = \alpha R + \beta R - P (\alpha + \beta)
            \label{eqn:condition_for_tilde_p1}\\
    \tilde p_2 & = \alpha S + \beta T - P (\alpha + \beta)
            \label{eqn:condition_for_tilde_p2}\\
    \tilde p_3 & = \alpha T + \beta S - P (\alpha + \beta)
            \label{eqn:condition_for_tilde_p3}\\
    \tilde p_4 & = \alpha P + \beta P - P (\alpha + \beta)
            \label{eqn:condition_for_tilde_p4}
\end{align}

Equation (\ref{eqn:condition_for_tilde_p4}) ensures that \(p_4=\tilde p_4=0\).
Equations (\ref{eqn:condition_for_tilde_p1}-\ref{eqn:condition_for_tilde_p3})
can be used to eliminate \(\alpha, \beta\), giving:

\begin{equation}\label{eqn:planar_definition_of_extortion}
    \tilde p_1 = \frac{(R - P)(\tilde p_2 + \tilde p_3)}{S + T - 2P}
\end{equation}

with:

\begin{equation}\label{eqn:definition_of_chi}
    \chi = \frac{\tilde p_2 (P - T) + \tilde p_3 (S - P)}
                {\tilde p_2 (P - S) + \tilde p_3 (T - P)}
\end{equation}

Given a strategy \(p\in\mathbb{R}^{4\times 1}\) equations
(\ref{eqn:condition_for_tilde_p4}), (\ref{eqn:planar_definition_of_extortion}-\ref{eqn:definition_of_chi}) can be used to check if
a strategy is extortionate. The conditions correspond to:

\begin{align}
    p_1 & = \frac{(R-P)(p_2 + p_3) - R + T + S - P}{S + T - 2P}
     \label{eqn:condition_for_p1}\\
    p_4 & = 0 \label{eqn:condition_for_p4}\\
    1 & > p_2 + p_3\label{eqn:condition_for_chi}
\end{align}

The algebraic steps necessary to prove these results are available in the
supporting materials.

All extortionate strategies reside on a triangular (\ref{eqn:condition_for_chi})
plane (\ref{eqn:condition_for_p1}) in 3 dimensions (\ref{eqn:condition_for_p4}).
Using this formulation it can be seen that a necessary (but not sufficient)
condition for an extortionate strategy is that it cooperates on average less
than 50\% of the time when in a state of disagreement with the opponent.

As an example, consider the known extortionate strategy \(p=(8 / 9, 1 / 2, 1 /
3, 0)\) from~\cite{Stewart2012} which is referred to as \texttt{Extort-2}. In
this case, for the standard values of \((R, T, S, P)\) constraint
(\ref{eqn:condition_for_p1}) corresponds to:

\begin{equation}
    p_1 = \frac{2(p_2 + p_3) + 1}{3}
\end{equation}

It is clear that in this case all constraints hold.

This approach could in fact be used to confirm that a given strategy is acting
in an extortionate manner even if it is not a memory one strategy. However, in
practice, if a closed form for \(p\) is not known, then due to measurement
and/or numerical error this would not work.

This problem can be written in the following linear algebraic form where
\(x=(\alpha, \beta)\)
and \(p^*=(\tilde p_1 - 1, tilde_2 - 1, p_3)\):

\begin{equation}\label{eqn:linear_algebraic_equation_for_p}
    Cx= p^*
\end{equation}

\(C\) corresponds to equations
(\ref{eqn:condition_for_tilde_p1}-\ref{eqn:condition_for_tilde_p3}) and is
given by:

\begin{equation}\label{eqn:definition_of_C}
    C =
    \begin{bmatrix}
        R - P & R- P \\
        S - P & T- P \\
        T - P & S- P \\
    \end{bmatrix}
\end{equation}

Note that in general, equation (\ref{eqn:linear_algebraic_equation_for_p}) will
not necessarily have a solution. From the Rouch\'{e}-Capelli theorem if there is
a solution it is unique as \(\text{rank}(C)=2\) which is the dimension of the
variable \(x\). The best fitting \(x\) is found by minimizing:

\begin{equation}\label{eqn:r_squared}
    \text{SSError} = \|C x- p^*\|_2^2 = \sum_{i=1}^{3}\left((C\bar x)_i-p_i^*\right)^2
\end{equation}

Note that \(\text{SSError}\), which is the square of the Frobenius
norm~\cite{Golub2013}, becomes a measure of how close a strategy is to being an
extortionate strategy. Suspicion
of extortion then corresponds to a threshold on \(\text{SSError}\).

By observing interactions (human or otherwise), their memory one representation
can be inferred and this approach can be used to recognise extortionate
behaviour. The notion of comparing theoretic and actual plays of the IPD is not
novel, see for example~\cite{Rand2013}. Immediately it is noted that if the
environment is noisy~\cite{Wu1995} then no strategy can be considered to be
extortionate as \(p_4>0\).

In the next section, this idea will be illustrated by observing the interactions
that take place in a computer based tournament of the IPD\@.

\section{Numerical experiments}\label{sec:numerical-experiments}

In~\cite{Stewart2012} results from a tournament with
\input{./assets/tex/number_of_stewart_plotkin_strategies/main.tex} strategies,
was presented with specific consideration given to ZD strategies. This
tournament is reproduced here using the Axelrod-Python
project~\cite{Knight2016}. To obtain a good measure of the corresponding
transition rates for each strategy all matches have been run for
\input{assets/tex/number_of_turns/main.tex} turns and every match has been
repeated \input{assets/tex/number_of_repetitions/main.tex} times. All of this
interaction data is available at~\cite{vincent_knight_2018_1297075}. A good
match between the inferred Markov chain and the state distribution of the actual
interactions has been verified. Data for this is presented in the supplementary
materials.

Figure~\ref{fig:SSError_overall_in_stewart_plotkin} shows the \(\text{SSError}\)
values for all the strategies in the tournament, as reported
in~\cite{Stewart2012} the extortionate strategy (which has an expected
\(\text{SSError}\) approximately 0) gains a large number of wins.

\begin{figure}[!htbp]
    \centering
    \includegraphics[width=.8\textwidth]{./assets/img/SSError_overall_in_stewart_plotkin/main.pdf}
    \caption{\(\text{SSError}\) and state probabilities for the strategies
        of~\cite{Stewart2012}, ordered both by number of wins and overall score.
        Note that \(P(DC)\) is not shown as it corresponds to the transpose of
        \(P(CD)\). Cooperator and Defector are omitted as they do not visit all
        the states.}
    \label{fig:SSError_overall_in_stewart_plotkin}
\end{figure}

Here, the work of~\cite{Stewart2012} is extended by investigating a tournament
with \input{assets/tex/number_of_full_strategies/main.tex}
strategies.

The results of this analysis are shown in
Figure~\ref{fig:SSError_and_probabilities_in_full}. The top ranking strategies
by number of wins seem to be extortionate (but not against all strategies) and
it can be seen that a small sub group of strategies achieve mutual defection.
All the top ranking strategies according to score achieve mutual cooperation and
do not extort each other, however they
\textbf{do} exhibit extortionate behaviour towards a number of the lower ranking
strategies.

\begin{figure}[!htbp]
    \centering
    \includegraphics[width=.8\textwidth]{./assets/img/SSError_and_probabilities_in_full/main.pdf}
    \caption{\(\text{SSError}\) for the strategies for the full tournament. Only
    strategy interactions for which \(p_4=0\) and \(\chi>1\) are displayed.}
    \label{fig:SSError_and_probabilities_in_full}
\end{figure}

\section{Conclusion}\label{sec:conclusion}

This work defines an approach to measure whether or not a player is playing a
strategy that corresponds to an extortionate strategy as defined
in~\cite{Press2012}: a mathematical model for suspicion. Indeed, all
extortionate strategies have been
 classified as lying on a triangular plane.
This rigorous classification fails to be robust to small measurement error, thus
a statistical approach is proposed.
This is done through a linear algebraic approach for approximating the solution
of a linear system. Using this, a large number of pairwise interactions is
simulated and in fact very few strategies are found to act extortionately.

The work of~\cite{Press2012}, whilst showing that a clever approach to taking
advantage of another memory one strategy exists: this is incomplete. Whilst the
elegance of this result is very attractive, just as the simplicity of the
victory of Tit For Tat in Axelrod's original tournaments was, it is incomplete.
Extortionate strategies achieve a high number of wins but they do not
achieve a high score which corresponds to the fitness landscape in an
evolutionary sense. From the large number of interactions a payoff matrix \(S\)
can be measured where \(S_{ij}\) denotes the score (using standard values of
\((R, S, T, P) = (3, 0, 5, 1)\)) of the \(i\)th strategy
against the \(j\)th strategy. Using this, the replicator equation
describes the evolution of the system based on a population density fitness
function:

\begin{equation}\label{eqn:replicator_dynamics}
    \frac{dx}{dt} = x(S-x^TS x)
\end{equation}

Equation (\ref{eqn:replicator_dynamics}) is solved numerically through an
integration technique described in~\cite{Petzold1983} and
Figure~\ref{fig:replicator_dynamics} shows the evolution of the distribution of
the system: the various strategies are ranked by scores. It is clear to see that
only the high ranking strategies survive the evolutionary process (in fact,
only \input{./assets/img/replicator_dynamics/main.tex}
have a final distribution greater than \(10 ^ {-2}\)). This confirms the
findings of~\cite{Moran1707} in which sophisticated strategies resist
evolutionary invasion of shorter memory strategies. Recalling
Figure~\ref{fig:SSError_and_probabilities_in_full} this demonstrates that:

\begin{itemize}
    \item Cooperation emerges through the evolutionary process: the high scoring
        strategies do not exhibit extortionate behaviour towards each other.
    \item Extortionate strategies do not survive the evolutionary process.
\end{itemize}

\begin{figure}[!htbp]
    \centering
    \includegraphics[width=.8\textwidth]{./assets/img/replicator_dynamics/main.pdf}
    \caption{Numerical simulation of the replicator equation
    (\ref{eqn:replicator_dynamics}): strategies are ordered by score, only the strategies with a high score survive the evolutionary process.}
    \label{fig:replicator_dynamics}
\end{figure}

This work can be used to classify plays of the IPD\@: data can be collected from
actual interactions (in lab or in the field). Furthermore, this allows for a
classification method similar to the notion of fingerprinting presented
in~\cite{Ashlock2008}. Trained strategies can potentially be classified as
extortionate or not or it could be possible to even constrain the reinforcement
learning approaches that are becoming prevalent in the literature.
Alternatively, this mathematical approach for recognising extortion could be
used in sophisticated strategies to defend against invasion. Arguably, some of
the strategies considered here exhibit this behaviour, indeed as described
in~\cite{Harper2017}, the top ranking strategies in the full tournament are
obtained using evolutionary reinforcement learning techniques, thus, suspicion
of extortionate behaviour could in fact be an evolutionary trait.

\section*{Acknowledgements}

The following open source software libraries were used in this research:

\begin{itemize}
    \item The Axelrod ~\cite{Knight2016, Knight2018} library (IPD strategies and
        tournaments).
    \item The sympy library~\cite{Meurer2017} (verification of all symbolic
        calculations).
    \item The matplotlib~\cite{Droettboom2018} library (visualisation).
    \item The pandas~\cite{Structures2010}, dask~\cite{Dask2016} and
        NumPy~\cite{Oliphant2015} libraries (data manipulation).
    \item The SciPy~\cite{Jones2001} library (numerical integration of the
        replicator equation).
\end{itemize}

This work was performed using the computational facilities of the Advanced
Research Computing @ Cardiff (ARCCA) Division, Cardiff University.

\printbibliography

\newpage
\section*{Supplementary materials}

\includepdf{assets/pdf/proof_of_form_of_extortionate_strategies/main.pdf}

\newpage

Using the pair wise interactions the transition rates \(p,
q\) can be measured and the steady state probabilities inferred and compared to
the actual probabilities of each state.
This is done numerically by computing the singular eigenvector of the
matrix \(A\) \cite{Stewart2009}:

\[
    A =
    \begin{bmatrix}
        p_1 q_1 & p_1 (1 - q_1) & (1 - p_1) q_1 & (1 -p_1) (1 - q_1) \\
        p_2 q_2 & p_2 (1 - q_2) & (1 - p_2) q_2 & (1 -p_2) (1 - q_2) \\
        p_3 q_3 & p_3 (1 - q_3) & (1 - p_3) q_3 & (1 -p_3) (1 - q_3) \\
        p_4 q_4 & p_4 (1 - q_4) & (1 - p_4) q_4 & (1 -p_4) (1 - q_4) \\
    \end{bmatrix}
\]

Figure~\ref{fig:computed_probabilities_vs_theoretic_probabilities} shows a
regression line fitted to every pairwise interaction with a reported
\(\text{SSError}\) value (pairwise interactions with missing states were
omitted). This serves to validate the approach: a part from some edge cases the
relationship is consistent.

\begin{figure}[!htbp]
    \centering
    \includegraphics[width=.8\textwidth]{./assets/img/computed_probabilities_vs_theoretic_probabilities/main.pdf}
    \caption{The
        relationship between the steady state probabilities inferred from the
        measured transitions and the actual steady state probabilities. A linear
        regression line is included validating the approach.}
    \label{fig:computed_probabilities_vs_theoretic_probabilities}
\end{figure}


\end{document}
 turns and every match has been
repeated \documentclass[a4paper]{article}

\usepackage{amsmath}
\usepackage{amssymb}
\usepackage[margin=1.5cm,
            includefoot,
            footskip=30pt]{geometry}
\usepackage{layout}
\usepackage{graphicx}
\usepackage{subcaption}

\usepackage{biblatex}
\usepackage{pdfpages}

\bibliography{main.bib}

\title{Suspicion: Recognising and evaluating the effectiveness
       of extortion in the Iterated Prisoner's Dilemma}
\author{Vincent A. Knight \and Nikoleta E. Glynatsi}
\date{\today}



\begin{document}

\maketitle

\begin{abstract}
    The Iterated Prisoner's Dilemma is a model for rational and evolutionary
    interactive behaviour. It has applications both in the study of human social
    behaviour as well as in biology.
    It is used to understand when and how a rational individual might
    accept an immediate cost to their own utility for the direct benefit of
    another.

    Much attention has been given to a class of strategies called
    Zero Determinant strategies. It has been theoretically shown that these
    strategies can ``extort'' any player.

    In this work, an approach to identify if observed strategies are playing in
    an extortionate way is described. Furthermore, experimental analysis of
    a large tournament with \input{assets/tex/number_of_full_strategies/main.tex}
    strategies is considered. In this setting
    the most highly performing strategies do not play in an extortionate way
    against each other but do against lower performing strategies.
    This suggests that whilst the theory of Zero Determinant strategies
    indicates that memory is not of fundamental importance to the evolution of
    cooperative behaviour, this is incomplete.
\end{abstract}

\section{Introduction}\label{sec:introduction}

Agent based game theoretic models have become a stalwart of the underpinning
mathematics of interactive behaviours. One of the major pieces of work
in this area is the pair of original computer tournaments run by Robert
Axelrod~\cite{Axelrod1980, Axelrod1980a}. These tournaments pitted submitted
computer strategies against each other in plays of the Iterated Prisoner's
Dilemma. A common game where agents can choose to pay a slight cost to their
immediate utility in the hope of building a reputation. This has been used in
economic and evolutionary game theory to understand the evolution of cooperative
behaviour.

Recently, a class of strategies was described in~\cite{Press2012} that can
provably extort any given opponent. In~\cite{Hilbe2013, Moran1707} some
questions have already been asked about the true effectiveness of these
strategies in an evolutionary setting. Here another question is asked: is it
possible to recognise this extortionate behaviour? A mathematical procedure for
suspicion is presented: in the same way that the continued actions of an
extortionate individual might raise suspicion.

This work makes use of the Axelrod Python library~\cite{Knight2018, Knight2016}
with a large number of Prisoner Dilemma strategies available to give an
extensive numerical example of the ideas presented.  The approach is presented
in Section~\ref{sec:delta-zd-strategies}.  All of the code and data discussed
in Section~\ref{sec:numerical-experiments} is open sourced, archived and
written according to best scientific principles~\cite{Wilson2014}. The data
archive can be found at~\cite{vincent_knight_2018_1297075}.

\section{Recognising Extortion}\label{sec:delta-zd-strategies}

In~\cite{Press2012}, given a match between 2 memory-one strategies, the concept
of Zero Determinant (ZD) strategies is introduced. The main result of that paper
shows that given two memory one players \(p, q\in\mathbb{R}^4\) a linear
relationship between the players' scores could be forced by one of the players.

Using the notation of~\cite{Press2012}, assuming the utilities for player \(p\)
are given by \(S_x=(R, S, T, P)\) and for player \(q\) by \(S_y=(R, T, S, P)\)
and that the stationary scores of each player is given by \(S_X\) and \(S_Y\)
respectively. The main result of~\cite{Press2012} is that if

\begin{equation}\label{eqn:linear_relationship_for_p}
    \tilde p=\alpha S_x + \beta S_y + \gamma
\end{equation}

or

\begin{equation}\label{eqn:linear_relationship_for_q}
    \tilde q=\alpha S_x + \beta S_y + \gamma
\end{equation}

where \(\tilde p = (1 - p_1, 1 - p_2, p_3, p_4)\) and
\(\tilde q = (1 - q_1, 1 - q_2, q_3, q_4)\) then:

\begin{equation}
    \alpha S_X + \beta S_Y + \gamma = 0
\end{equation}

In~\cite{Press2012} a particular type of ZD strategy is defined: extortionate
strategies. If:

\begin{equation}\label{eqn:constraint_for_extortion}
    \gamma = - P(\alpha + \beta)
\end{equation}

then the player can ensure they get a score \(\chi\) times
larger than the opponent. This extortion coefficient is given by:

\begin{equation}\label{eqn:definition_of_chi}
    \chi=\frac{-\beta}{\alpha}
\end{equation}

Thus, if (\ref{eqn:constraint_for_extortion}) holds and \(\chi >1\) a player is
said to extort their opponent.
Here, the reverse problem is considered: given a
\(p\in\mathbb{R}^4\) how does one identify \(\alpha, \beta\) if they
exist and is the strategy in fact acting in an extortionate way?

These conditions correspond to:

\begin{align}
    \tilde p_1 & = \alpha R + \beta R - P (\alpha + \beta)
            \label{eqn:condition_for_tilde_p1}\\
    \tilde p_2 & = \alpha S + \beta T - P (\alpha + \beta)
            \label{eqn:condition_for_tilde_p2}\\
    \tilde p_3 & = \alpha T + \beta S - P (\alpha + \beta)
            \label{eqn:condition_for_tilde_p3}\\
    \tilde p_4 & = \alpha P + \beta P - P (\alpha + \beta)
            \label{eqn:condition_for_tilde_p4}
\end{align}

Equation (\ref{eqn:condition_for_tilde_p4}) ensures that \(p_4=\tilde p_4=0\).
Equations (\ref{eqn:condition_for_tilde_p1}-\ref{eqn:condition_for_tilde_p3})
can be used to eliminate \(\alpha, \beta\), giving:

\begin{equation}\label{eqn:planar_definition_of_extortion}
    \tilde p_1 = \frac{(R - P)(\tilde p_2 + \tilde p_3)}{S + T - 2P}
\end{equation}

with:

\begin{equation}\label{eqn:definition_of_chi}
    \chi = \frac{\tilde p_2 (P - T) + \tilde p_3 (S - P)}
                {\tilde p_2 (P - S) + \tilde p_3 (T - P)}
\end{equation}

Given a strategy \(p\in\mathbb{R}^{4\times 1}\) equations
(\ref{eqn:condition_for_tilde_p4}), (\ref{eqn:planar_definition_of_extortion}-\ref{eqn:definition_of_chi}) can be used to check if
a strategy is extortionate. The conditions correspond to:

\begin{align}
    p_1 & = \frac{(R-P)(p_2 + p_3) - R + T + S - P}{S + T - 2P}
     \label{eqn:condition_for_p1}\\
    p_4 & = 0 \label{eqn:condition_for_p4}\\
    1 & > p_2 + p_3\label{eqn:condition_for_chi}
\end{align}

The algebraic steps necessary to prove these results are available in the
supporting materials.

All extortionate strategies reside on a triangular (\ref{eqn:condition_for_chi})
plane (\ref{eqn:condition_for_p1}) in 3 dimensions (\ref{eqn:condition_for_p4}).
Using this formulation it can be seen that a necessary (but not sufficient)
condition for an extortionate strategy is that it cooperates on average less
than 50\% of the time when in a state of disagreement with the opponent.

As an example, consider the known extortionate strategy \(p=(8 / 9, 1 / 2, 1 /
3, 0)\) from~\cite{Stewart2012} which is referred to as \texttt{Extort-2}. In
this case, for the standard values of \((R, T, S, P)\) constraint
(\ref{eqn:condition_for_p1}) corresponds to:

\begin{equation}
    p_1 = \frac{2(p_2 + p_3) + 1}{3}
\end{equation}

It is clear that in this case all constraints hold.

This approach could in fact be used to confirm that a given strategy is acting
in an extortionate manner even if it is not a memory one strategy. However, in
practice, if a closed form for \(p\) is not known, then due to measurement
and/or numerical error this would not work.

This problem can be written in the following linear algebraic form where
\(x=(\alpha, \beta)\)
and \(p^*=(\tilde p_1 - 1, tilde_2 - 1, p_3)\):

\begin{equation}\label{eqn:linear_algebraic_equation_for_p}
    Cx= p^*
\end{equation}

\(C\) corresponds to equations
(\ref{eqn:condition_for_tilde_p1}-\ref{eqn:condition_for_tilde_p3}) and is
given by:

\begin{equation}\label{eqn:definition_of_C}
    C =
    \begin{bmatrix}
        R - P & R- P \\
        S - P & T- P \\
        T - P & S- P \\
    \end{bmatrix}
\end{equation}

Note that in general, equation (\ref{eqn:linear_algebraic_equation_for_p}) will
not necessarily have a solution. From the Rouch\'{e}-Capelli theorem if there is
a solution it is unique as \(\text{rank}(C)=2\) which is the dimension of the
variable \(x\). The best fitting \(x\) is found by minimizing:

\begin{equation}\label{eqn:r_squared}
    \text{SSError} = \|C x- p^*\|_2^2 = \sum_{i=1}^{3}\left((C\bar x)_i-p_i^*\right)^2
\end{equation}

Note that \(\text{SSError}\), which is the square of the Frobenius
norm~\cite{Golub2013}, becomes a measure of how close a strategy is to being an
extortionate strategy. Suspicion
of extortion then corresponds to a threshold on \(\text{SSError}\).

By observing interactions (human or otherwise), their memory one representation
can be inferred and this approach can be used to recognise extortionate
behaviour. The notion of comparing theoretic and actual plays of the IPD is not
novel, see for example~\cite{Rand2013}. Immediately it is noted that if the
environment is noisy~\cite{Wu1995} then no strategy can be considered to be
extortionate as \(p_4>0\).

In the next section, this idea will be illustrated by observing the interactions
that take place in a computer based tournament of the IPD\@.

\section{Numerical experiments}\label{sec:numerical-experiments}

In~\cite{Stewart2012} results from a tournament with
\input{./assets/tex/number_of_stewart_plotkin_strategies/main.tex} strategies,
was presented with specific consideration given to ZD strategies. This
tournament is reproduced here using the Axelrod-Python
project~\cite{Knight2016}. To obtain a good measure of the corresponding
transition rates for each strategy all matches have been run for
\input{assets/tex/number_of_turns/main.tex} turns and every match has been
repeated \input{assets/tex/number_of_repetitions/main.tex} times. All of this
interaction data is available at~\cite{vincent_knight_2018_1297075}. A good
match between the inferred Markov chain and the state distribution of the actual
interactions has been verified. Data for this is presented in the supplementary
materials.

Figure~\ref{fig:SSError_overall_in_stewart_plotkin} shows the \(\text{SSError}\)
values for all the strategies in the tournament, as reported
in~\cite{Stewart2012} the extortionate strategy (which has an expected
\(\text{SSError}\) approximately 0) gains a large number of wins.

\begin{figure}[!htbp]
    \centering
    \includegraphics[width=.8\textwidth]{./assets/img/SSError_overall_in_stewart_plotkin/main.pdf}
    \caption{\(\text{SSError}\) and state probabilities for the strategies
        of~\cite{Stewart2012}, ordered both by number of wins and overall score.
        Note that \(P(DC)\) is not shown as it corresponds to the transpose of
        \(P(CD)\). Cooperator and Defector are omitted as they do not visit all
        the states.}
    \label{fig:SSError_overall_in_stewart_plotkin}
\end{figure}

Here, the work of~\cite{Stewart2012} is extended by investigating a tournament
with \input{assets/tex/number_of_full_strategies/main.tex}
strategies.

The results of this analysis are shown in
Figure~\ref{fig:SSError_and_probabilities_in_full}. The top ranking strategies
by number of wins seem to be extortionate (but not against all strategies) and
it can be seen that a small sub group of strategies achieve mutual defection.
All the top ranking strategies according to score achieve mutual cooperation and
do not extort each other, however they
\textbf{do} exhibit extortionate behaviour towards a number of the lower ranking
strategies.

\begin{figure}[!htbp]
    \centering
    \includegraphics[width=.8\textwidth]{./assets/img/SSError_and_probabilities_in_full/main.pdf}
    \caption{\(\text{SSError}\) for the strategies for the full tournament. Only
    strategy interactions for which \(p_4=0\) and \(\chi>1\) are displayed.}
    \label{fig:SSError_and_probabilities_in_full}
\end{figure}

\section{Conclusion}\label{sec:conclusion}

This work defines an approach to measure whether or not a player is playing a
strategy that corresponds to an extortionate strategy as defined
in~\cite{Press2012}: a mathematical model for suspicion. Indeed, all
extortionate strategies have been
 classified as lying on a triangular plane.
This rigorous classification fails to be robust to small measurement error, thus
a statistical approach is proposed.
This is done through a linear algebraic approach for approximating the solution
of a linear system. Using this, a large number of pairwise interactions is
simulated and in fact very few strategies are found to act extortionately.

The work of~\cite{Press2012}, whilst showing that a clever approach to taking
advantage of another memory one strategy exists: this is incomplete. Whilst the
elegance of this result is very attractive, just as the simplicity of the
victory of Tit For Tat in Axelrod's original tournaments was, it is incomplete.
Extortionate strategies achieve a high number of wins but they do not
achieve a high score which corresponds to the fitness landscape in an
evolutionary sense. From the large number of interactions a payoff matrix \(S\)
can be measured where \(S_{ij}\) denotes the score (using standard values of
\((R, S, T, P) = (3, 0, 5, 1)\)) of the \(i\)th strategy
against the \(j\)th strategy. Using this, the replicator equation
describes the evolution of the system based on a population density fitness
function:

\begin{equation}\label{eqn:replicator_dynamics}
    \frac{dx}{dt} = x(S-x^TS x)
\end{equation}

Equation (\ref{eqn:replicator_dynamics}) is solved numerically through an
integration technique described in~\cite{Petzold1983} and
Figure~\ref{fig:replicator_dynamics} shows the evolution of the distribution of
the system: the various strategies are ranked by scores. It is clear to see that
only the high ranking strategies survive the evolutionary process (in fact,
only \input{./assets/img/replicator_dynamics/main.tex}
have a final distribution greater than \(10 ^ {-2}\)). This confirms the
findings of~\cite{Moran1707} in which sophisticated strategies resist
evolutionary invasion of shorter memory strategies. Recalling
Figure~\ref{fig:SSError_and_probabilities_in_full} this demonstrates that:

\begin{itemize}
    \item Cooperation emerges through the evolutionary process: the high scoring
        strategies do not exhibit extortionate behaviour towards each other.
    \item Extortionate strategies do not survive the evolutionary process.
\end{itemize}

\begin{figure}[!htbp]
    \centering
    \includegraphics[width=.8\textwidth]{./assets/img/replicator_dynamics/main.pdf}
    \caption{Numerical simulation of the replicator equation
    (\ref{eqn:replicator_dynamics}): strategies are ordered by score, only the strategies with a high score survive the evolutionary process.}
    \label{fig:replicator_dynamics}
\end{figure}

This work can be used to classify plays of the IPD\@: data can be collected from
actual interactions (in lab or in the field). Furthermore, this allows for a
classification method similar to the notion of fingerprinting presented
in~\cite{Ashlock2008}. Trained strategies can potentially be classified as
extortionate or not or it could be possible to even constrain the reinforcement
learning approaches that are becoming prevalent in the literature.
Alternatively, this mathematical approach for recognising extortion could be
used in sophisticated strategies to defend against invasion. Arguably, some of
the strategies considered here exhibit this behaviour, indeed as described
in~\cite{Harper2017}, the top ranking strategies in the full tournament are
obtained using evolutionary reinforcement learning techniques, thus, suspicion
of extortionate behaviour could in fact be an evolutionary trait.

\section*{Acknowledgements}

The following open source software libraries were used in this research:

\begin{itemize}
    \item The Axelrod ~\cite{Knight2016, Knight2018} library (IPD strategies and
        tournaments).
    \item The sympy library~\cite{Meurer2017} (verification of all symbolic
        calculations).
    \item The matplotlib~\cite{Droettboom2018} library (visualisation).
    \item The pandas~\cite{Structures2010}, dask~\cite{Dask2016} and
        NumPy~\cite{Oliphant2015} libraries (data manipulation).
    \item The SciPy~\cite{Jones2001} library (numerical integration of the
        replicator equation).
\end{itemize}

This work was performed using the computational facilities of the Advanced
Research Computing @ Cardiff (ARCCA) Division, Cardiff University.

\printbibliography

\newpage
\section*{Supplementary materials}

\includepdf{assets/pdf/proof_of_form_of_extortionate_strategies/main.pdf}

\newpage

Using the pair wise interactions the transition rates \(p,
q\) can be measured and the steady state probabilities inferred and compared to
the actual probabilities of each state.
This is done numerically by computing the singular eigenvector of the
matrix \(A\) \cite{Stewart2009}:

\[
    A =
    \begin{bmatrix}
        p_1 q_1 & p_1 (1 - q_1) & (1 - p_1) q_1 & (1 -p_1) (1 - q_1) \\
        p_2 q_2 & p_2 (1 - q_2) & (1 - p_2) q_2 & (1 -p_2) (1 - q_2) \\
        p_3 q_3 & p_3 (1 - q_3) & (1 - p_3) q_3 & (1 -p_3) (1 - q_3) \\
        p_4 q_4 & p_4 (1 - q_4) & (1 - p_4) q_4 & (1 -p_4) (1 - q_4) \\
    \end{bmatrix}
\]

Figure~\ref{fig:computed_probabilities_vs_theoretic_probabilities} shows a
regression line fitted to every pairwise interaction with a reported
\(\text{SSError}\) value (pairwise interactions with missing states were
omitted). This serves to validate the approach: a part from some edge cases the
relationship is consistent.

\begin{figure}[!htbp]
    \centering
    \includegraphics[width=.8\textwidth]{./assets/img/computed_probabilities_vs_theoretic_probabilities/main.pdf}
    \caption{The
        relationship between the steady state probabilities inferred from the
        measured transitions and the actual steady state probabilities. A linear
        regression line is included validating the approach.}
    \label{fig:computed_probabilities_vs_theoretic_probabilities}
\end{figure}


\end{document}
 times. All of this
interaction data is available at~\cite{vincent_knight_2018_1297075}. A good
match between the inferred Markov chain and the state distribution of the actual
interactions has been verified. Data for this is presented in the supplementary
materials.

Figure~\ref{fig:SSError_overall_in_stewart_plotkin} shows the \(\text{SSError}\)
values for all the strategies in the tournament, as reported
in~\cite{Stewart2012} the extortionate strategy (which has an expected
\(\text{SSError}\) approximately 0) gains a large number of wins.

\begin{figure}[!htbp]
    \centering
    \includegraphics[width=.8\textwidth]{./assets/img/SSError_overall_in_stewart_plotkin/main.pdf}
    \caption{\(\text{SSError}\) and state probabilities for the strategies
        of~\cite{Stewart2012}, ordered both by number of wins and overall score.
        Note that \(P(DC)\) is not shown as it corresponds to the transpose of
        \(P(CD)\). Cooperator and Defector are omitted as they do not visit all
        the states.}
    \label{fig:SSError_overall_in_stewart_plotkin}
\end{figure}

Here, the work of~\cite{Stewart2012} is extended by investigating a tournament
with \documentclass[a4paper]{article}

\usepackage{amsmath}
\usepackage{amssymb}
\usepackage[margin=1.5cm,
            includefoot,
            footskip=30pt]{geometry}
\usepackage{layout}
\usepackage{graphicx}
\usepackage{subcaption}

\usepackage{biblatex}
\usepackage{pdfpages}

\bibliography{main.bib}

\title{Suspicion: Recognising and evaluating the effectiveness
       of extortion in the Iterated Prisoner's Dilemma}
\author{Vincent A. Knight \and Nikoleta E. Glynatsi}
\date{\today}



\begin{document}

\maketitle

\begin{abstract}
    The Iterated Prisoner's Dilemma is a model for rational and evolutionary
    interactive behaviour. It has applications both in the study of human social
    behaviour as well as in biology.
    It is used to understand when and how a rational individual might
    accept an immediate cost to their own utility for the direct benefit of
    another.

    Much attention has been given to a class of strategies called
    Zero Determinant strategies. It has been theoretically shown that these
    strategies can ``extort'' any player.

    In this work, an approach to identify if observed strategies are playing in
    an extortionate way is described. Furthermore, experimental analysis of
    a large tournament with \input{assets/tex/number_of_full_strategies/main.tex}
    strategies is considered. In this setting
    the most highly performing strategies do not play in an extortionate way
    against each other but do against lower performing strategies.
    This suggests that whilst the theory of Zero Determinant strategies
    indicates that memory is not of fundamental importance to the evolution of
    cooperative behaviour, this is incomplete.
\end{abstract}

\section{Introduction}\label{sec:introduction}

Agent based game theoretic models have become a stalwart of the underpinning
mathematics of interactive behaviours. One of the major pieces of work
in this area is the pair of original computer tournaments run by Robert
Axelrod~\cite{Axelrod1980, Axelrod1980a}. These tournaments pitted submitted
computer strategies against each other in plays of the Iterated Prisoner's
Dilemma. A common game where agents can choose to pay a slight cost to their
immediate utility in the hope of building a reputation. This has been used in
economic and evolutionary game theory to understand the evolution of cooperative
behaviour.

Recently, a class of strategies was described in~\cite{Press2012} that can
provably extort any given opponent. In~\cite{Hilbe2013, Moran1707} some
questions have already been asked about the true effectiveness of these
strategies in an evolutionary setting. Here another question is asked: is it
possible to recognise this extortionate behaviour? A mathematical procedure for
suspicion is presented: in the same way that the continued actions of an
extortionate individual might raise suspicion.

This work makes use of the Axelrod Python library~\cite{Knight2018, Knight2016}
with a large number of Prisoner Dilemma strategies available to give an
extensive numerical example of the ideas presented.  The approach is presented
in Section~\ref{sec:delta-zd-strategies}.  All of the code and data discussed
in Section~\ref{sec:numerical-experiments} is open sourced, archived and
written according to best scientific principles~\cite{Wilson2014}. The data
archive can be found at~\cite{vincent_knight_2018_1297075}.

\section{Recognising Extortion}\label{sec:delta-zd-strategies}

In~\cite{Press2012}, given a match between 2 memory-one strategies, the concept
of Zero Determinant (ZD) strategies is introduced. The main result of that paper
shows that given two memory one players \(p, q\in\mathbb{R}^4\) a linear
relationship between the players' scores could be forced by one of the players.

Using the notation of~\cite{Press2012}, assuming the utilities for player \(p\)
are given by \(S_x=(R, S, T, P)\) and for player \(q\) by \(S_y=(R, T, S, P)\)
and that the stationary scores of each player is given by \(S_X\) and \(S_Y\)
respectively. The main result of~\cite{Press2012} is that if

\begin{equation}\label{eqn:linear_relationship_for_p}
    \tilde p=\alpha S_x + \beta S_y + \gamma
\end{equation}

or

\begin{equation}\label{eqn:linear_relationship_for_q}
    \tilde q=\alpha S_x + \beta S_y + \gamma
\end{equation}

where \(\tilde p = (1 - p_1, 1 - p_2, p_3, p_4)\) and
\(\tilde q = (1 - q_1, 1 - q_2, q_3, q_4)\) then:

\begin{equation}
    \alpha S_X + \beta S_Y + \gamma = 0
\end{equation}

In~\cite{Press2012} a particular type of ZD strategy is defined: extortionate
strategies. If:

\begin{equation}\label{eqn:constraint_for_extortion}
    \gamma = - P(\alpha + \beta)
\end{equation}

then the player can ensure they get a score \(\chi\) times
larger than the opponent. This extortion coefficient is given by:

\begin{equation}\label{eqn:definition_of_chi}
    \chi=\frac{-\beta}{\alpha}
\end{equation}

Thus, if (\ref{eqn:constraint_for_extortion}) holds and \(\chi >1\) a player is
said to extort their opponent.
Here, the reverse problem is considered: given a
\(p\in\mathbb{R}^4\) how does one identify \(\alpha, \beta\) if they
exist and is the strategy in fact acting in an extortionate way?

These conditions correspond to:

\begin{align}
    \tilde p_1 & = \alpha R + \beta R - P (\alpha + \beta)
            \label{eqn:condition_for_tilde_p1}\\
    \tilde p_2 & = \alpha S + \beta T - P (\alpha + \beta)
            \label{eqn:condition_for_tilde_p2}\\
    \tilde p_3 & = \alpha T + \beta S - P (\alpha + \beta)
            \label{eqn:condition_for_tilde_p3}\\
    \tilde p_4 & = \alpha P + \beta P - P (\alpha + \beta)
            \label{eqn:condition_for_tilde_p4}
\end{align}

Equation (\ref{eqn:condition_for_tilde_p4}) ensures that \(p_4=\tilde p_4=0\).
Equations (\ref{eqn:condition_for_tilde_p1}-\ref{eqn:condition_for_tilde_p3})
can be used to eliminate \(\alpha, \beta\), giving:

\begin{equation}\label{eqn:planar_definition_of_extortion}
    \tilde p_1 = \frac{(R - P)(\tilde p_2 + \tilde p_3)}{S + T - 2P}
\end{equation}

with:

\begin{equation}\label{eqn:definition_of_chi}
    \chi = \frac{\tilde p_2 (P - T) + \tilde p_3 (S - P)}
                {\tilde p_2 (P - S) + \tilde p_3 (T - P)}
\end{equation}

Given a strategy \(p\in\mathbb{R}^{4\times 1}\) equations
(\ref{eqn:condition_for_tilde_p4}), (\ref{eqn:planar_definition_of_extortion}-\ref{eqn:definition_of_chi}) can be used to check if
a strategy is extortionate. The conditions correspond to:

\begin{align}
    p_1 & = \frac{(R-P)(p_2 + p_3) - R + T + S - P}{S + T - 2P}
     \label{eqn:condition_for_p1}\\
    p_4 & = 0 \label{eqn:condition_for_p4}\\
    1 & > p_2 + p_3\label{eqn:condition_for_chi}
\end{align}

The algebraic steps necessary to prove these results are available in the
supporting materials.

All extortionate strategies reside on a triangular (\ref{eqn:condition_for_chi})
plane (\ref{eqn:condition_for_p1}) in 3 dimensions (\ref{eqn:condition_for_p4}).
Using this formulation it can be seen that a necessary (but not sufficient)
condition for an extortionate strategy is that it cooperates on average less
than 50\% of the time when in a state of disagreement with the opponent.

As an example, consider the known extortionate strategy \(p=(8 / 9, 1 / 2, 1 /
3, 0)\) from~\cite{Stewart2012} which is referred to as \texttt{Extort-2}. In
this case, for the standard values of \((R, T, S, P)\) constraint
(\ref{eqn:condition_for_p1}) corresponds to:

\begin{equation}
    p_1 = \frac{2(p_2 + p_3) + 1}{3}
\end{equation}

It is clear that in this case all constraints hold.

This approach could in fact be used to confirm that a given strategy is acting
in an extortionate manner even if it is not a memory one strategy. However, in
practice, if a closed form for \(p\) is not known, then due to measurement
and/or numerical error this would not work.

This problem can be written in the following linear algebraic form where
\(x=(\alpha, \beta)\)
and \(p^*=(\tilde p_1 - 1, tilde_2 - 1, p_3)\):

\begin{equation}\label{eqn:linear_algebraic_equation_for_p}
    Cx= p^*
\end{equation}

\(C\) corresponds to equations
(\ref{eqn:condition_for_tilde_p1}-\ref{eqn:condition_for_tilde_p3}) and is
given by:

\begin{equation}\label{eqn:definition_of_C}
    C =
    \begin{bmatrix}
        R - P & R- P \\
        S - P & T- P \\
        T - P & S- P \\
    \end{bmatrix}
\end{equation}

Note that in general, equation (\ref{eqn:linear_algebraic_equation_for_p}) will
not necessarily have a solution. From the Rouch\'{e}-Capelli theorem if there is
a solution it is unique as \(\text{rank}(C)=2\) which is the dimension of the
variable \(x\). The best fitting \(x\) is found by minimizing:

\begin{equation}\label{eqn:r_squared}
    \text{SSError} = \|C x- p^*\|_2^2 = \sum_{i=1}^{3}\left((C\bar x)_i-p_i^*\right)^2
\end{equation}

Note that \(\text{SSError}\), which is the square of the Frobenius
norm~\cite{Golub2013}, becomes a measure of how close a strategy is to being an
extortionate strategy. Suspicion
of extortion then corresponds to a threshold on \(\text{SSError}\).

By observing interactions (human or otherwise), their memory one representation
can be inferred and this approach can be used to recognise extortionate
behaviour. The notion of comparing theoretic and actual plays of the IPD is not
novel, see for example~\cite{Rand2013}. Immediately it is noted that if the
environment is noisy~\cite{Wu1995} then no strategy can be considered to be
extortionate as \(p_4>0\).

In the next section, this idea will be illustrated by observing the interactions
that take place in a computer based tournament of the IPD\@.

\section{Numerical experiments}\label{sec:numerical-experiments}

In~\cite{Stewart2012} results from a tournament with
\input{./assets/tex/number_of_stewart_plotkin_strategies/main.tex} strategies,
was presented with specific consideration given to ZD strategies. This
tournament is reproduced here using the Axelrod-Python
project~\cite{Knight2016}. To obtain a good measure of the corresponding
transition rates for each strategy all matches have been run for
\input{assets/tex/number_of_turns/main.tex} turns and every match has been
repeated \input{assets/tex/number_of_repetitions/main.tex} times. All of this
interaction data is available at~\cite{vincent_knight_2018_1297075}. A good
match between the inferred Markov chain and the state distribution of the actual
interactions has been verified. Data for this is presented in the supplementary
materials.

Figure~\ref{fig:SSError_overall_in_stewart_plotkin} shows the \(\text{SSError}\)
values for all the strategies in the tournament, as reported
in~\cite{Stewart2012} the extortionate strategy (which has an expected
\(\text{SSError}\) approximately 0) gains a large number of wins.

\begin{figure}[!htbp]
    \centering
    \includegraphics[width=.8\textwidth]{./assets/img/SSError_overall_in_stewart_plotkin/main.pdf}
    \caption{\(\text{SSError}\) and state probabilities for the strategies
        of~\cite{Stewart2012}, ordered both by number of wins and overall score.
        Note that \(P(DC)\) is not shown as it corresponds to the transpose of
        \(P(CD)\). Cooperator and Defector are omitted as they do not visit all
        the states.}
    \label{fig:SSError_overall_in_stewart_plotkin}
\end{figure}

Here, the work of~\cite{Stewart2012} is extended by investigating a tournament
with \input{assets/tex/number_of_full_strategies/main.tex}
strategies.

The results of this analysis are shown in
Figure~\ref{fig:SSError_and_probabilities_in_full}. The top ranking strategies
by number of wins seem to be extortionate (but not against all strategies) and
it can be seen that a small sub group of strategies achieve mutual defection.
All the top ranking strategies according to score achieve mutual cooperation and
do not extort each other, however they
\textbf{do} exhibit extortionate behaviour towards a number of the lower ranking
strategies.

\begin{figure}[!htbp]
    \centering
    \includegraphics[width=.8\textwidth]{./assets/img/SSError_and_probabilities_in_full/main.pdf}
    \caption{\(\text{SSError}\) for the strategies for the full tournament. Only
    strategy interactions for which \(p_4=0\) and \(\chi>1\) are displayed.}
    \label{fig:SSError_and_probabilities_in_full}
\end{figure}

\section{Conclusion}\label{sec:conclusion}

This work defines an approach to measure whether or not a player is playing a
strategy that corresponds to an extortionate strategy as defined
in~\cite{Press2012}: a mathematical model for suspicion. Indeed, all
extortionate strategies have been
 classified as lying on a triangular plane.
This rigorous classification fails to be robust to small measurement error, thus
a statistical approach is proposed.
This is done through a linear algebraic approach for approximating the solution
of a linear system. Using this, a large number of pairwise interactions is
simulated and in fact very few strategies are found to act extortionately.

The work of~\cite{Press2012}, whilst showing that a clever approach to taking
advantage of another memory one strategy exists: this is incomplete. Whilst the
elegance of this result is very attractive, just as the simplicity of the
victory of Tit For Tat in Axelrod's original tournaments was, it is incomplete.
Extortionate strategies achieve a high number of wins but they do not
achieve a high score which corresponds to the fitness landscape in an
evolutionary sense. From the large number of interactions a payoff matrix \(S\)
can be measured where \(S_{ij}\) denotes the score (using standard values of
\((R, S, T, P) = (3, 0, 5, 1)\)) of the \(i\)th strategy
against the \(j\)th strategy. Using this, the replicator equation
describes the evolution of the system based on a population density fitness
function:

\begin{equation}\label{eqn:replicator_dynamics}
    \frac{dx}{dt} = x(S-x^TS x)
\end{equation}

Equation (\ref{eqn:replicator_dynamics}) is solved numerically through an
integration technique described in~\cite{Petzold1983} and
Figure~\ref{fig:replicator_dynamics} shows the evolution of the distribution of
the system: the various strategies are ranked by scores. It is clear to see that
only the high ranking strategies survive the evolutionary process (in fact,
only \input{./assets/img/replicator_dynamics/main.tex}
have a final distribution greater than \(10 ^ {-2}\)). This confirms the
findings of~\cite{Moran1707} in which sophisticated strategies resist
evolutionary invasion of shorter memory strategies. Recalling
Figure~\ref{fig:SSError_and_probabilities_in_full} this demonstrates that:

\begin{itemize}
    \item Cooperation emerges through the evolutionary process: the high scoring
        strategies do not exhibit extortionate behaviour towards each other.
    \item Extortionate strategies do not survive the evolutionary process.
\end{itemize}

\begin{figure}[!htbp]
    \centering
    \includegraphics[width=.8\textwidth]{./assets/img/replicator_dynamics/main.pdf}
    \caption{Numerical simulation of the replicator equation
    (\ref{eqn:replicator_dynamics}): strategies are ordered by score, only the strategies with a high score survive the evolutionary process.}
    \label{fig:replicator_dynamics}
\end{figure}

This work can be used to classify plays of the IPD\@: data can be collected from
actual interactions (in lab or in the field). Furthermore, this allows for a
classification method similar to the notion of fingerprinting presented
in~\cite{Ashlock2008}. Trained strategies can potentially be classified as
extortionate or not or it could be possible to even constrain the reinforcement
learning approaches that are becoming prevalent in the literature.
Alternatively, this mathematical approach for recognising extortion could be
used in sophisticated strategies to defend against invasion. Arguably, some of
the strategies considered here exhibit this behaviour, indeed as described
in~\cite{Harper2017}, the top ranking strategies in the full tournament are
obtained using evolutionary reinforcement learning techniques, thus, suspicion
of extortionate behaviour could in fact be an evolutionary trait.

\section*{Acknowledgements}

The following open source software libraries were used in this research:

\begin{itemize}
    \item The Axelrod ~\cite{Knight2016, Knight2018} library (IPD strategies and
        tournaments).
    \item The sympy library~\cite{Meurer2017} (verification of all symbolic
        calculations).
    \item The matplotlib~\cite{Droettboom2018} library (visualisation).
    \item The pandas~\cite{Structures2010}, dask~\cite{Dask2016} and
        NumPy~\cite{Oliphant2015} libraries (data manipulation).
    \item The SciPy~\cite{Jones2001} library (numerical integration of the
        replicator equation).
\end{itemize}

This work was performed using the computational facilities of the Advanced
Research Computing @ Cardiff (ARCCA) Division, Cardiff University.

\printbibliography

\newpage
\section*{Supplementary materials}

\includepdf{assets/pdf/proof_of_form_of_extortionate_strategies/main.pdf}

\newpage

Using the pair wise interactions the transition rates \(p,
q\) can be measured and the steady state probabilities inferred and compared to
the actual probabilities of each state.
This is done numerically by computing the singular eigenvector of the
matrix \(A\) \cite{Stewart2009}:

\[
    A =
    \begin{bmatrix}
        p_1 q_1 & p_1 (1 - q_1) & (1 - p_1) q_1 & (1 -p_1) (1 - q_1) \\
        p_2 q_2 & p_2 (1 - q_2) & (1 - p_2) q_2 & (1 -p_2) (1 - q_2) \\
        p_3 q_3 & p_3 (1 - q_3) & (1 - p_3) q_3 & (1 -p_3) (1 - q_3) \\
        p_4 q_4 & p_4 (1 - q_4) & (1 - p_4) q_4 & (1 -p_4) (1 - q_4) \\
    \end{bmatrix}
\]

Figure~\ref{fig:computed_probabilities_vs_theoretic_probabilities} shows a
regression line fitted to every pairwise interaction with a reported
\(\text{SSError}\) value (pairwise interactions with missing states were
omitted). This serves to validate the approach: a part from some edge cases the
relationship is consistent.

\begin{figure}[!htbp]
    \centering
    \includegraphics[width=.8\textwidth]{./assets/img/computed_probabilities_vs_theoretic_probabilities/main.pdf}
    \caption{The
        relationship between the steady state probabilities inferred from the
        measured transitions and the actual steady state probabilities. A linear
        regression line is included validating the approach.}
    \label{fig:computed_probabilities_vs_theoretic_probabilities}
\end{figure}


\end{document}

strategies.

The results of this analysis are shown in
Figure~\ref{fig:SSError_and_probabilities_in_full}. The top ranking strategies
by number of wins seem to be extortionate (but not against all strategies) and
it can be seen that a small sub group of strategies achieve mutual defection.
All the top ranking strategies according to score achieve mutual cooperation and
do not extort each other, however they
\textbf{do} exhibit extortionate behaviour towards a number of the lower ranking
strategies.

\begin{figure}[!htbp]
    \centering
    \includegraphics[width=.8\textwidth]{./assets/img/SSError_and_probabilities_in_full/main.pdf}
    \caption{\(\text{SSError}\) for the strategies for the full tournament. Only
    strategy interactions for which \(p_4=0\) and \(\chi>1\) are displayed.}
    \label{fig:SSError_and_probabilities_in_full}
\end{figure}

\section{Conclusion}\label{sec:conclusion}

This work defines an approach to measure whether or not a player is playing a
strategy that corresponds to an extortionate strategy as defined
in~\cite{Press2012}: a mathematical model for suspicion. Indeed, all
extortionate strategies have been
 classified as lying on a triangular plane.
This rigorous classification fails to be robust to small measurement error, thus
a statistical approach is proposed.
This is done through a linear algebraic approach for approximating the solution
of a linear system. Using this, a large number of pairwise interactions is
simulated and in fact very few strategies are found to act extortionately.

The work of~\cite{Press2012}, whilst showing that a clever approach to taking
advantage of another memory one strategy exists: this is incomplete. Whilst the
elegance of this result is very attractive, just as the simplicity of the
victory of Tit For Tat in Axelrod's original tournaments was, it is incomplete.
Extortionate strategies achieve a high number of wins but they do not
achieve a high score which corresponds to the fitness landscape in an
evolutionary sense. From the large number of interactions a payoff matrix \(S\)
can be measured where \(S_{ij}\) denotes the score (using standard values of
\((R, S, T, P) = (3, 0, 5, 1)\)) of the \(i\)th strategy
against the \(j\)th strategy. Using this, the replicator equation
describes the evolution of the system based on a population density fitness
function:

\begin{equation}\label{eqn:replicator_dynamics}
    \frac{dx}{dt} = x(S-x^TS x)
\end{equation}

Equation (\ref{eqn:replicator_dynamics}) is solved numerically through an
integration technique described in~\cite{Petzold1983} and
Figure~\ref{fig:replicator_dynamics} shows the evolution of the distribution of
the system: the various strategies are ranked by scores. It is clear to see that
only the high ranking strategies survive the evolutionary process (in fact,
only \documentclass[a4paper]{article}

\usepackage{amsmath}
\usepackage{amssymb}
\usepackage[margin=1.5cm,
            includefoot,
            footskip=30pt]{geometry}
\usepackage{layout}
\usepackage{graphicx}
\usepackage{subcaption}

\usepackage{biblatex}
\usepackage{pdfpages}

\bibliography{main.bib}

\title{Suspicion: Recognising and evaluating the effectiveness
       of extortion in the Iterated Prisoner's Dilemma}
\author{Vincent A. Knight \and Nikoleta E. Glynatsi}
\date{\today}



\begin{document}

\maketitle

\begin{abstract}
    The Iterated Prisoner's Dilemma is a model for rational and evolutionary
    interactive behaviour. It has applications both in the study of human social
    behaviour as well as in biology.
    It is used to understand when and how a rational individual might
    accept an immediate cost to their own utility for the direct benefit of
    another.

    Much attention has been given to a class of strategies called
    Zero Determinant strategies. It has been theoretically shown that these
    strategies can ``extort'' any player.

    In this work, an approach to identify if observed strategies are playing in
    an extortionate way is described. Furthermore, experimental analysis of
    a large tournament with \input{assets/tex/number_of_full_strategies/main.tex}
    strategies is considered. In this setting
    the most highly performing strategies do not play in an extortionate way
    against each other but do against lower performing strategies.
    This suggests that whilst the theory of Zero Determinant strategies
    indicates that memory is not of fundamental importance to the evolution of
    cooperative behaviour, this is incomplete.
\end{abstract}

\section{Introduction}\label{sec:introduction}

Agent based game theoretic models have become a stalwart of the underpinning
mathematics of interactive behaviours. One of the major pieces of work
in this area is the pair of original computer tournaments run by Robert
Axelrod~\cite{Axelrod1980, Axelrod1980a}. These tournaments pitted submitted
computer strategies against each other in plays of the Iterated Prisoner's
Dilemma. A common game where agents can choose to pay a slight cost to their
immediate utility in the hope of building a reputation. This has been used in
economic and evolutionary game theory to understand the evolution of cooperative
behaviour.

Recently, a class of strategies was described in~\cite{Press2012} that can
provably extort any given opponent. In~\cite{Hilbe2013, Moran1707} some
questions have already been asked about the true effectiveness of these
strategies in an evolutionary setting. Here another question is asked: is it
possible to recognise this extortionate behaviour? A mathematical procedure for
suspicion is presented: in the same way that the continued actions of an
extortionate individual might raise suspicion.

This work makes use of the Axelrod Python library~\cite{Knight2018, Knight2016}
with a large number of Prisoner Dilemma strategies available to give an
extensive numerical example of the ideas presented.  The approach is presented
in Section~\ref{sec:delta-zd-strategies}.  All of the code and data discussed
in Section~\ref{sec:numerical-experiments} is open sourced, archived and
written according to best scientific principles~\cite{Wilson2014}. The data
archive can be found at~\cite{vincent_knight_2018_1297075}.

\section{Recognising Extortion}\label{sec:delta-zd-strategies}

In~\cite{Press2012}, given a match between 2 memory-one strategies, the concept
of Zero Determinant (ZD) strategies is introduced. The main result of that paper
shows that given two memory one players \(p, q\in\mathbb{R}^4\) a linear
relationship between the players' scores could be forced by one of the players.

Using the notation of~\cite{Press2012}, assuming the utilities for player \(p\)
are given by \(S_x=(R, S, T, P)\) and for player \(q\) by \(S_y=(R, T, S, P)\)
and that the stationary scores of each player is given by \(S_X\) and \(S_Y\)
respectively. The main result of~\cite{Press2012} is that if

\begin{equation}\label{eqn:linear_relationship_for_p}
    \tilde p=\alpha S_x + \beta S_y + \gamma
\end{equation}

or

\begin{equation}\label{eqn:linear_relationship_for_q}
    \tilde q=\alpha S_x + \beta S_y + \gamma
\end{equation}

where \(\tilde p = (1 - p_1, 1 - p_2, p_3, p_4)\) and
\(\tilde q = (1 - q_1, 1 - q_2, q_3, q_4)\) then:

\begin{equation}
    \alpha S_X + \beta S_Y + \gamma = 0
\end{equation}

In~\cite{Press2012} a particular type of ZD strategy is defined: extortionate
strategies. If:

\begin{equation}\label{eqn:constraint_for_extortion}
    \gamma = - P(\alpha + \beta)
\end{equation}

then the player can ensure they get a score \(\chi\) times
larger than the opponent. This extortion coefficient is given by:

\begin{equation}\label{eqn:definition_of_chi}
    \chi=\frac{-\beta}{\alpha}
\end{equation}

Thus, if (\ref{eqn:constraint_for_extortion}) holds and \(\chi >1\) a player is
said to extort their opponent.
Here, the reverse problem is considered: given a
\(p\in\mathbb{R}^4\) how does one identify \(\alpha, \beta\) if they
exist and is the strategy in fact acting in an extortionate way?

These conditions correspond to:

\begin{align}
    \tilde p_1 & = \alpha R + \beta R - P (\alpha + \beta)
            \label{eqn:condition_for_tilde_p1}\\
    \tilde p_2 & = \alpha S + \beta T - P (\alpha + \beta)
            \label{eqn:condition_for_tilde_p2}\\
    \tilde p_3 & = \alpha T + \beta S - P (\alpha + \beta)
            \label{eqn:condition_for_tilde_p3}\\
    \tilde p_4 & = \alpha P + \beta P - P (\alpha + \beta)
            \label{eqn:condition_for_tilde_p4}
\end{align}

Equation (\ref{eqn:condition_for_tilde_p4}) ensures that \(p_4=\tilde p_4=0\).
Equations (\ref{eqn:condition_for_tilde_p1}-\ref{eqn:condition_for_tilde_p3})
can be used to eliminate \(\alpha, \beta\), giving:

\begin{equation}\label{eqn:planar_definition_of_extortion}
    \tilde p_1 = \frac{(R - P)(\tilde p_2 + \tilde p_3)}{S + T - 2P}
\end{equation}

with:

\begin{equation}\label{eqn:definition_of_chi}
    \chi = \frac{\tilde p_2 (P - T) + \tilde p_3 (S - P)}
                {\tilde p_2 (P - S) + \tilde p_3 (T - P)}
\end{equation}

Given a strategy \(p\in\mathbb{R}^{4\times 1}\) equations
(\ref{eqn:condition_for_tilde_p4}), (\ref{eqn:planar_definition_of_extortion}-\ref{eqn:definition_of_chi}) can be used to check if
a strategy is extortionate. The conditions correspond to:

\begin{align}
    p_1 & = \frac{(R-P)(p_2 + p_3) - R + T + S - P}{S + T - 2P}
     \label{eqn:condition_for_p1}\\
    p_4 & = 0 \label{eqn:condition_for_p4}\\
    1 & > p_2 + p_3\label{eqn:condition_for_chi}
\end{align}

The algebraic steps necessary to prove these results are available in the
supporting materials.

All extortionate strategies reside on a triangular (\ref{eqn:condition_for_chi})
plane (\ref{eqn:condition_for_p1}) in 3 dimensions (\ref{eqn:condition_for_p4}).
Using this formulation it can be seen that a necessary (but not sufficient)
condition for an extortionate strategy is that it cooperates on average less
than 50\% of the time when in a state of disagreement with the opponent.

As an example, consider the known extortionate strategy \(p=(8 / 9, 1 / 2, 1 /
3, 0)\) from~\cite{Stewart2012} which is referred to as \texttt{Extort-2}. In
this case, for the standard values of \((R, T, S, P)\) constraint
(\ref{eqn:condition_for_p1}) corresponds to:

\begin{equation}
    p_1 = \frac{2(p_2 + p_3) + 1}{3}
\end{equation}

It is clear that in this case all constraints hold.

This approach could in fact be used to confirm that a given strategy is acting
in an extortionate manner even if it is not a memory one strategy. However, in
practice, if a closed form for \(p\) is not known, then due to measurement
and/or numerical error this would not work.

This problem can be written in the following linear algebraic form where
\(x=(\alpha, \beta)\)
and \(p^*=(\tilde p_1 - 1, tilde_2 - 1, p_3)\):

\begin{equation}\label{eqn:linear_algebraic_equation_for_p}
    Cx= p^*
\end{equation}

\(C\) corresponds to equations
(\ref{eqn:condition_for_tilde_p1}-\ref{eqn:condition_for_tilde_p3}) and is
given by:

\begin{equation}\label{eqn:definition_of_C}
    C =
    \begin{bmatrix}
        R - P & R- P \\
        S - P & T- P \\
        T - P & S- P \\
    \end{bmatrix}
\end{equation}

Note that in general, equation (\ref{eqn:linear_algebraic_equation_for_p}) will
not necessarily have a solution. From the Rouch\'{e}-Capelli theorem if there is
a solution it is unique as \(\text{rank}(C)=2\) which is the dimension of the
variable \(x\). The best fitting \(x\) is found by minimizing:

\begin{equation}\label{eqn:r_squared}
    \text{SSError} = \|C x- p^*\|_2^2 = \sum_{i=1}^{3}\left((C\bar x)_i-p_i^*\right)^2
\end{equation}

Note that \(\text{SSError}\), which is the square of the Frobenius
norm~\cite{Golub2013}, becomes a measure of how close a strategy is to being an
extortionate strategy. Suspicion
of extortion then corresponds to a threshold on \(\text{SSError}\).

By observing interactions (human or otherwise), their memory one representation
can be inferred and this approach can be used to recognise extortionate
behaviour. The notion of comparing theoretic and actual plays of the IPD is not
novel, see for example~\cite{Rand2013}. Immediately it is noted that if the
environment is noisy~\cite{Wu1995} then no strategy can be considered to be
extortionate as \(p_4>0\).

In the next section, this idea will be illustrated by observing the interactions
that take place in a computer based tournament of the IPD\@.

\section{Numerical experiments}\label{sec:numerical-experiments}

In~\cite{Stewart2012} results from a tournament with
\input{./assets/tex/number_of_stewart_plotkin_strategies/main.tex} strategies,
was presented with specific consideration given to ZD strategies. This
tournament is reproduced here using the Axelrod-Python
project~\cite{Knight2016}. To obtain a good measure of the corresponding
transition rates for each strategy all matches have been run for
\input{assets/tex/number_of_turns/main.tex} turns and every match has been
repeated \input{assets/tex/number_of_repetitions/main.tex} times. All of this
interaction data is available at~\cite{vincent_knight_2018_1297075}. A good
match between the inferred Markov chain and the state distribution of the actual
interactions has been verified. Data for this is presented in the supplementary
materials.

Figure~\ref{fig:SSError_overall_in_stewart_plotkin} shows the \(\text{SSError}\)
values for all the strategies in the tournament, as reported
in~\cite{Stewart2012} the extortionate strategy (which has an expected
\(\text{SSError}\) approximately 0) gains a large number of wins.

\begin{figure}[!htbp]
    \centering
    \includegraphics[width=.8\textwidth]{./assets/img/SSError_overall_in_stewart_plotkin/main.pdf}
    \caption{\(\text{SSError}\) and state probabilities for the strategies
        of~\cite{Stewart2012}, ordered both by number of wins and overall score.
        Note that \(P(DC)\) is not shown as it corresponds to the transpose of
        \(P(CD)\). Cooperator and Defector are omitted as they do not visit all
        the states.}
    \label{fig:SSError_overall_in_stewart_plotkin}
\end{figure}

Here, the work of~\cite{Stewart2012} is extended by investigating a tournament
with \input{assets/tex/number_of_full_strategies/main.tex}
strategies.

The results of this analysis are shown in
Figure~\ref{fig:SSError_and_probabilities_in_full}. The top ranking strategies
by number of wins seem to be extortionate (but not against all strategies) and
it can be seen that a small sub group of strategies achieve mutual defection.
All the top ranking strategies according to score achieve mutual cooperation and
do not extort each other, however they
\textbf{do} exhibit extortionate behaviour towards a number of the lower ranking
strategies.

\begin{figure}[!htbp]
    \centering
    \includegraphics[width=.8\textwidth]{./assets/img/SSError_and_probabilities_in_full/main.pdf}
    \caption{\(\text{SSError}\) for the strategies for the full tournament. Only
    strategy interactions for which \(p_4=0\) and \(\chi>1\) are displayed.}
    \label{fig:SSError_and_probabilities_in_full}
\end{figure}

\section{Conclusion}\label{sec:conclusion}

This work defines an approach to measure whether or not a player is playing a
strategy that corresponds to an extortionate strategy as defined
in~\cite{Press2012}: a mathematical model for suspicion. Indeed, all
extortionate strategies have been
 classified as lying on a triangular plane.
This rigorous classification fails to be robust to small measurement error, thus
a statistical approach is proposed.
This is done through a linear algebraic approach for approximating the solution
of a linear system. Using this, a large number of pairwise interactions is
simulated and in fact very few strategies are found to act extortionately.

The work of~\cite{Press2012}, whilst showing that a clever approach to taking
advantage of another memory one strategy exists: this is incomplete. Whilst the
elegance of this result is very attractive, just as the simplicity of the
victory of Tit For Tat in Axelrod's original tournaments was, it is incomplete.
Extortionate strategies achieve a high number of wins but they do not
achieve a high score which corresponds to the fitness landscape in an
evolutionary sense. From the large number of interactions a payoff matrix \(S\)
can be measured where \(S_{ij}\) denotes the score (using standard values of
\((R, S, T, P) = (3, 0, 5, 1)\)) of the \(i\)th strategy
against the \(j\)th strategy. Using this, the replicator equation
describes the evolution of the system based on a population density fitness
function:

\begin{equation}\label{eqn:replicator_dynamics}
    \frac{dx}{dt} = x(S-x^TS x)
\end{equation}

Equation (\ref{eqn:replicator_dynamics}) is solved numerically through an
integration technique described in~\cite{Petzold1983} and
Figure~\ref{fig:replicator_dynamics} shows the evolution of the distribution of
the system: the various strategies are ranked by scores. It is clear to see that
only the high ranking strategies survive the evolutionary process (in fact,
only \input{./assets/img/replicator_dynamics/main.tex}
have a final distribution greater than \(10 ^ {-2}\)). This confirms the
findings of~\cite{Moran1707} in which sophisticated strategies resist
evolutionary invasion of shorter memory strategies. Recalling
Figure~\ref{fig:SSError_and_probabilities_in_full} this demonstrates that:

\begin{itemize}
    \item Cooperation emerges through the evolutionary process: the high scoring
        strategies do not exhibit extortionate behaviour towards each other.
    \item Extortionate strategies do not survive the evolutionary process.
\end{itemize}

\begin{figure}[!htbp]
    \centering
    \includegraphics[width=.8\textwidth]{./assets/img/replicator_dynamics/main.pdf}
    \caption{Numerical simulation of the replicator equation
    (\ref{eqn:replicator_dynamics}): strategies are ordered by score, only the strategies with a high score survive the evolutionary process.}
    \label{fig:replicator_dynamics}
\end{figure}

This work can be used to classify plays of the IPD\@: data can be collected from
actual interactions (in lab or in the field). Furthermore, this allows for a
classification method similar to the notion of fingerprinting presented
in~\cite{Ashlock2008}. Trained strategies can potentially be classified as
extortionate or not or it could be possible to even constrain the reinforcement
learning approaches that are becoming prevalent in the literature.
Alternatively, this mathematical approach for recognising extortion could be
used in sophisticated strategies to defend against invasion. Arguably, some of
the strategies considered here exhibit this behaviour, indeed as described
in~\cite{Harper2017}, the top ranking strategies in the full tournament are
obtained using evolutionary reinforcement learning techniques, thus, suspicion
of extortionate behaviour could in fact be an evolutionary trait.

\section*{Acknowledgements}

The following open source software libraries were used in this research:

\begin{itemize}
    \item The Axelrod ~\cite{Knight2016, Knight2018} library (IPD strategies and
        tournaments).
    \item The sympy library~\cite{Meurer2017} (verification of all symbolic
        calculations).
    \item The matplotlib~\cite{Droettboom2018} library (visualisation).
    \item The pandas~\cite{Structures2010}, dask~\cite{Dask2016} and
        NumPy~\cite{Oliphant2015} libraries (data manipulation).
    \item The SciPy~\cite{Jones2001} library (numerical integration of the
        replicator equation).
\end{itemize}

This work was performed using the computational facilities of the Advanced
Research Computing @ Cardiff (ARCCA) Division, Cardiff University.

\printbibliography

\newpage
\section*{Supplementary materials}

\includepdf{assets/pdf/proof_of_form_of_extortionate_strategies/main.pdf}

\newpage

Using the pair wise interactions the transition rates \(p,
q\) can be measured and the steady state probabilities inferred and compared to
the actual probabilities of each state.
This is done numerically by computing the singular eigenvector of the
matrix \(A\) \cite{Stewart2009}:

\[
    A =
    \begin{bmatrix}
        p_1 q_1 & p_1 (1 - q_1) & (1 - p_1) q_1 & (1 -p_1) (1 - q_1) \\
        p_2 q_2 & p_2 (1 - q_2) & (1 - p_2) q_2 & (1 -p_2) (1 - q_2) \\
        p_3 q_3 & p_3 (1 - q_3) & (1 - p_3) q_3 & (1 -p_3) (1 - q_3) \\
        p_4 q_4 & p_4 (1 - q_4) & (1 - p_4) q_4 & (1 -p_4) (1 - q_4) \\
    \end{bmatrix}
\]

Figure~\ref{fig:computed_probabilities_vs_theoretic_probabilities} shows a
regression line fitted to every pairwise interaction with a reported
\(\text{SSError}\) value (pairwise interactions with missing states were
omitted). This serves to validate the approach: a part from some edge cases the
relationship is consistent.

\begin{figure}[!htbp]
    \centering
    \includegraphics[width=.8\textwidth]{./assets/img/computed_probabilities_vs_theoretic_probabilities/main.pdf}
    \caption{The
        relationship between the steady state probabilities inferred from the
        measured transitions and the actual steady state probabilities. A linear
        regression line is included validating the approach.}
    \label{fig:computed_probabilities_vs_theoretic_probabilities}
\end{figure}


\end{document}

have a final distribution greater than \(10 ^ {-2}\)). This confirms the
findings of~\cite{Moran1707} in which sophisticated strategies resist
evolutionary invasion of shorter memory strategies. Recalling
Figure~\ref{fig:SSError_and_probabilities_in_full} this demonstrates that:

\begin{itemize}
    \item Cooperation emerges through the evolutionary process: the high scoring
        strategies do not exhibit extortionate behaviour towards each other.
    \item Extortionate strategies do not survive the evolutionary process.
\end{itemize}

\begin{figure}[!htbp]
    \centering
    \includegraphics[width=.8\textwidth]{./assets/img/replicator_dynamics/main.pdf}
    \caption{Numerical simulation of the replicator equation
    (\ref{eqn:replicator_dynamics}): strategies are ordered by score, only the strategies with a high score survive the evolutionary process.}
    \label{fig:replicator_dynamics}
\end{figure}

This work can be used to classify plays of the IPD\@: data can be collected from
actual interactions (in lab or in the field). Furthermore, this allows for a
classification method similar to the notion of fingerprinting presented
in~\cite{Ashlock2008}. Trained strategies can potentially be classified as
extortionate or not or it could be possible to even constrain the reinforcement
learning approaches that are becoming prevalent in the literature.
Alternatively, this mathematical approach for recognising extortion could be
used in sophisticated strategies to defend against invasion. Arguably, some of
the strategies considered here exhibit this behaviour, indeed as described
in~\cite{Harper2017}, the top ranking strategies in the full tournament are
obtained using evolutionary reinforcement learning techniques, thus, suspicion
of extortionate behaviour could in fact be an evolutionary trait.

\section*{Acknowledgements}

The following open source software libraries were used in this research:

\begin{itemize}
    \item The Axelrod ~\cite{Knight2016, Knight2018} library (IPD strategies and
        tournaments).
    \item The sympy library~\cite{Meurer2017} (verification of all symbolic
        calculations).
    \item The matplotlib~\cite{Droettboom2018} library (visualisation).
    \item The pandas~\cite{Structures2010}, dask~\cite{Dask2016} and
        NumPy~\cite{Oliphant2015} libraries (data manipulation).
    \item The SciPy~\cite{Jones2001} library (numerical integration of the
        replicator equation).
\end{itemize}

This work was performed using the computational facilities of the Advanced
Research Computing @ Cardiff (ARCCA) Division, Cardiff University.

\printbibliography

\newpage
\section*{Supplementary materials}

\includepdf{assets/pdf/proof_of_form_of_extortionate_strategies/main.pdf}

\newpage

Using the pair wise interactions the transition rates \(p,
q\) can be measured and the steady state probabilities inferred and compared to
the actual probabilities of each state.
This is done numerically by computing the singular eigenvector of the
matrix \(A\) \cite{Stewart2009}:

\[
    A =
    \begin{bmatrix}
        p_1 q_1 & p_1 (1 - q_1) & (1 - p_1) q_1 & (1 -p_1) (1 - q_1) \\
        p_2 q_2 & p_2 (1 - q_2) & (1 - p_2) q_2 & (1 -p_2) (1 - q_2) \\
        p_3 q_3 & p_3 (1 - q_3) & (1 - p_3) q_3 & (1 -p_3) (1 - q_3) \\
        p_4 q_4 & p_4 (1 - q_4) & (1 - p_4) q_4 & (1 -p_4) (1 - q_4) \\
    \end{bmatrix}
\]

Figure~\ref{fig:computed_probabilities_vs_theoretic_probabilities} shows a
regression line fitted to every pairwise interaction with a reported
\(\text{SSError}\) value (pairwise interactions with missing states were
omitted). This serves to validate the approach: a part from some edge cases the
relationship is consistent.

\begin{figure}[!htbp]
    \centering
    \includegraphics[width=.8\textwidth]{./assets/img/computed_probabilities_vs_theoretic_probabilities/main.pdf}
    \caption{The
        relationship between the steady state probabilities inferred from the
        measured transitions and the actual steady state probabilities. A linear
        regression line is included validating the approach.}
    \label{fig:computed_probabilities_vs_theoretic_probabilities}
\end{figure}


\end{document}

have a final distribution greater than \(10 ^ {-2}\)). This confirms the
findings of~\cite{Moran1707} in which sophisticated strategies resist
evolutionary invasion of shorter memory strategies. Recalling
Figure~\ref{fig:SSError_and_probabilities_in_full} this demonstrates that:

\begin{itemize}
    \item Cooperation emerges through the evolutionary process: the high scoring
        strategies do not exhibit extortionate behaviour towards each other.
    \item Extortionate strategies do not survive the evolutionary process.
\end{itemize}

\begin{figure}[!htbp]
    \centering
    \includegraphics[width=.8\textwidth]{./assets/img/replicator_dynamics/main.pdf}
    \caption{Numerical simulation of the replicator equation
    (\ref{eqn:replicator_dynamics}): strategies are ordered by score, only the strategies with a high score survive the evolutionary process.}
    \label{fig:replicator_dynamics}
\end{figure}

This work can be used to classify plays of the IPD\@: data can be collected from
actual interactions (in lab or in the field). Furthermore, this allows for a
classification method similar to the notion of fingerprinting presented
in~\cite{Ashlock2008}. Trained strategies can potentially be classified as
extortionate or not or it could be possible to even constrain the reinforcement
learning approaches that are becoming prevalent in the literature.
Alternatively, this mathematical approach for recognising extortion could be
used in sophisticated strategies to defend against invasion. Arguably, some of
the strategies considered here exhibit this behaviour, indeed as described
in~\cite{Harper2017}, the top ranking strategies in the full tournament are
obtained using evolutionary reinforcement learning techniques, thus, suspicion
of extortionate behaviour could in fact be an evolutionary trait.

\section*{Acknowledgements}

The following open source software libraries were used in this research:

\begin{itemize}
    \item The Axelrod ~\cite{Knight2016, Knight2018} library (IPD strategies and
        tournaments).
    \item The sympy library~\cite{Meurer2017} (verification of all symbolic
        calculations).
    \item The matplotlib~\cite{Droettboom2018} library (visualisation).
    \item The pandas~\cite{Structures2010}, dask~\cite{Dask2016} and
        NumPy~\cite{Oliphant2015} libraries (data manipulation).
    \item The SciPy~\cite{Jones2001} library (numerical integration of the
        replicator equation).
\end{itemize}

This work was performed using the computational facilities of the Advanced
Research Computing @ Cardiff (ARCCA) Division, Cardiff University.

\printbibliography

\newpage
\section*{Supplementary materials}

\includepdf{assets/pdf/proof_of_form_of_extortionate_strategies/main.pdf}

\newpage

Using the pair wise interactions the transition rates \(p,
q\) can be measured and the steady state probabilities inferred and compared to
the actual probabilities of each state.
This is done numerically by computing the singular eigenvector of the
matrix \(A\) \cite{Stewart2009}:

\[
    A =
    \begin{bmatrix}
        p_1 q_1 & p_1 (1 - q_1) & (1 - p_1) q_1 & (1 -p_1) (1 - q_1) \\
        p_2 q_2 & p_2 (1 - q_2) & (1 - p_2) q_2 & (1 -p_2) (1 - q_2) \\
        p_3 q_3 & p_3 (1 - q_3) & (1 - p_3) q_3 & (1 -p_3) (1 - q_3) \\
        p_4 q_4 & p_4 (1 - q_4) & (1 - p_4) q_4 & (1 -p_4) (1 - q_4) \\
    \end{bmatrix}
\]

Figure~\ref{fig:computed_probabilities_vs_theoretic_probabilities} shows a
regression line fitted to every pairwise interaction with a reported
\(\text{SSError}\) value (pairwise interactions with missing states were
omitted). This serves to validate the approach: a part from some edge cases the
relationship is consistent.

\begin{figure}[!htbp]
    \centering
    \includegraphics[width=.8\textwidth]{./assets/img/computed_probabilities_vs_theoretic_probabilities/main.pdf}
    \caption{The
        relationship between the steady state probabilities inferred from the
        measured transitions and the actual steady state probabilities. A linear
        regression line is included validating the approach.}
    \label{fig:computed_probabilities_vs_theoretic_probabilities}
\end{figure}


\end{document}
strategies this method
detects extortionate behaviour in well-known extortionate strategies as well
others that do not fit the algebraic definition. The highest performing
strategies in this corpus are able to exhibit selectively extortionate behavior,
cooperating with strong strategies while exploiting weaker strategies, which no
memory-one strategy can do. These strategies emerged from an evolutionary
selection process and their existence contradicts widely-repeated folklore in
the evolutionary game theory literature: complex strategies can be
extraordinarily effective, zero-determinant strategies can be outperformed by
non-zero determinant strategies, and longer memory strategies are able to
outperform short memory strategies. Moreover, while resistance to extortion is
critical for the evolution of cooperation, the extortion of weak opponents
need not prevent cooperation between stronger opponents, and this adaptability
may be crucial to maintaining cooperation in the long run.
\end{abstract}

The Iterated Prisoner's Dilemma is a model for rational and evolutionary
interactive behaviour, having applications in biology, the study of human
social behaviour, and many other domains. A standard representation  of the game
is given in equation~\ref{eqn:theipd}, where the constraints ensure a non cooperative
equilibrium.

\begin{equation}
    \begin{pmatrix}
        R & S \\
        T & P
    \end{pmatrix}
    \qquad
    T > R > P > S\text{ and }2 R > T + S
    \label{eqn:theipd}
\end{equation}


Since the introduction of
zero-determinant (ZD) strategies in \cite{Press2012}, extortionate strategies have
received considerable interest in the literature \cite{hilbe2015partners}.
These strategies ``enforce'' a difference in stationary
payouts between themselves and their opponents. The definition requires a
precise algebraic relationship between the probabilities of cooperation given
the outcome of the previous round of play and slight alterations to these
probabilities can cause a strategy to no longer satisfy the necessary equations.

In~\cite{adami2013evolutionary, Hilbe2013, hilbe2013adaptive, hilbe2015partners,
ichinose2018zero, Moran1707} the true effectiveness of these strategies in an
evolutionary setting was discussed. For example~\cite{adami2013evolutionary}
showed that ZD strategies were not evolutionarily stable. Furthermore, in that
work it was also postulated that `evolutionarily successful ZD strategies could
be designed that use longer memory to distinguish self from non-self'. In a non
evolutionary context, the work of~\cite{becks2019extortion} uses social
experiments to suggest that higher rewards promote extortionate
behaviour where statistical techniques are used to identify such behaviour.

The algebraic relationships of extortion, discussed in
Section~\ref{sec:sserror-zd-strategies}, define a subspace of
\(p\in\mathbb{R}^4\) which can be used to broaden the definition of an extortionate
strategy by requiring only that the defining four cooperation probabilities of a
memory-one strategy are close to an algebraically extortionate strategy, by the usual
technique of orthogonal projection. Moreover, given the history of play of a
strategy in an actual matchup, we can empirically observe its four
cooperation probabilities, measure the distance to the subspace of extortionate
strategies, and use this distance as a measure of the extortionality of a
strategy. This method can be applied to any strategy regardless of the memory
depth and avoids the algebraic rigidity issues.

We apply this method to the largest known corpus of strategies for the iterated
prisoner's dilemma (the Axelrod Python library~\cite{Knight2016, Knight2018})
and validate empirically that the method in fact detects extortionate strategies.
A large tournament with \documentclass[a4paper]{article}

\usepackage{amsmath}
\usepackage{amssymb}
\usepackage[margin=1.5cm,
            includefoot,
            footskip=30pt]{geometry}
\usepackage{layout}
\usepackage{graphicx}
\usepackage{subcaption}

\usepackage{biblatex}
\usepackage{pdfpages}

\bibliography{main.bib}

\title{Suspicion: Recognising and evaluating the effectiveness
       of extortion in the Iterated Prisoner's Dilemma}
\author{Vincent A. Knight \and Nikoleta E. Glynatsi}
\date{\today}



\begin{document}

\maketitle

\begin{abstract}
    The Iterated Prisoner's Dilemma is a model for rational and evolutionary
    interactive behaviour. It has applications both in the study of human social
    behaviour as well as in biology.
    It is used to understand when and how a rational individual might
    accept an immediate cost to their own utility for the direct benefit of
    another.

    Much attention has been given to a class of strategies called
    Zero Determinant strategies. It has been theoretically shown that these
    strategies can ``extort'' any player.

    In this work, an approach to identify if observed strategies are playing in
    an extortionate way is described. Furthermore, experimental analysis of
    a large tournament with \documentclass[a4paper]{article}

\usepackage{amsmath}
\usepackage{amssymb}
\usepackage[margin=1.5cm,
            includefoot,
            footskip=30pt]{geometry}
\usepackage{layout}
\usepackage{graphicx}
\usepackage{subcaption}

\usepackage{biblatex}
\usepackage{pdfpages}

\bibliography{main.bib}

\title{Suspicion: Recognising and evaluating the effectiveness
       of extortion in the Iterated Prisoner's Dilemma}
\author{Vincent A. Knight \and Nikoleta E. Glynatsi}
\date{\today}



\begin{document}

\maketitle

\begin{abstract}
    The Iterated Prisoner's Dilemma is a model for rational and evolutionary
    interactive behaviour. It has applications both in the study of human social
    behaviour as well as in biology.
    It is used to understand when and how a rational individual might
    accept an immediate cost to their own utility for the direct benefit of
    another.

    Much attention has been given to a class of strategies called
    Zero Determinant strategies. It has been theoretically shown that these
    strategies can ``extort'' any player.

    In this work, an approach to identify if observed strategies are playing in
    an extortionate way is described. Furthermore, experimental analysis of
    a large tournament with \documentclass[a4paper]{article}

\usepackage{amsmath}
\usepackage{amssymb}
\usepackage[margin=1.5cm,
            includefoot,
            footskip=30pt]{geometry}
\usepackage{layout}
\usepackage{graphicx}
\usepackage{subcaption}

\usepackage{biblatex}
\usepackage{pdfpages}

\bibliography{main.bib}

\title{Suspicion: Recognising and evaluating the effectiveness
       of extortion in the Iterated Prisoner's Dilemma}
\author{Vincent A. Knight \and Nikoleta E. Glynatsi}
\date{\today}



\begin{document}

\maketitle

\begin{abstract}
    The Iterated Prisoner's Dilemma is a model for rational and evolutionary
    interactive behaviour. It has applications both in the study of human social
    behaviour as well as in biology.
    It is used to understand when and how a rational individual might
    accept an immediate cost to their own utility for the direct benefit of
    another.

    Much attention has been given to a class of strategies called
    Zero Determinant strategies. It has been theoretically shown that these
    strategies can ``extort'' any player.

    In this work, an approach to identify if observed strategies are playing in
    an extortionate way is described. Furthermore, experimental analysis of
    a large tournament with \input{assets/tex/number_of_full_strategies/main.tex}
    strategies is considered. In this setting
    the most highly performing strategies do not play in an extortionate way
    against each other but do against lower performing strategies.
    This suggests that whilst the theory of Zero Determinant strategies
    indicates that memory is not of fundamental importance to the evolution of
    cooperative behaviour, this is incomplete.
\end{abstract}

\section{Introduction}\label{sec:introduction}

Agent based game theoretic models have become a stalwart of the underpinning
mathematics of interactive behaviours. One of the major pieces of work
in this area is the pair of original computer tournaments run by Robert
Axelrod~\cite{Axelrod1980, Axelrod1980a}. These tournaments pitted submitted
computer strategies against each other in plays of the Iterated Prisoner's
Dilemma. A common game where agents can choose to pay a slight cost to their
immediate utility in the hope of building a reputation. This has been used in
economic and evolutionary game theory to understand the evolution of cooperative
behaviour.

Recently, a class of strategies was described in~\cite{Press2012} that can
provably extort any given opponent. In~\cite{Hilbe2013, Moran1707} some
questions have already been asked about the true effectiveness of these
strategies in an evolutionary setting. Here another question is asked: is it
possible to recognise this extortionate behaviour? A mathematical procedure for
suspicion is presented: in the same way that the continued actions of an
extortionate individual might raise suspicion.

This work makes use of the Axelrod Python library~\cite{Knight2018, Knight2016}
with a large number of Prisoner Dilemma strategies available to give an
extensive numerical example of the ideas presented.  The approach is presented
in Section~\ref{sec:delta-zd-strategies}.  All of the code and data discussed
in Section~\ref{sec:numerical-experiments} is open sourced, archived and
written according to best scientific principles~\cite{Wilson2014}. The data
archive can be found at~\cite{vincent_knight_2018_1297075}.

\section{Recognising Extortion}\label{sec:delta-zd-strategies}

In~\cite{Press2012}, given a match between 2 memory-one strategies, the concept
of Zero Determinant (ZD) strategies is introduced. The main result of that paper
shows that given two memory one players \(p, q\in\mathbb{R}^4\) a linear
relationship between the players' scores could be forced by one of the players.

Using the notation of~\cite{Press2012}, assuming the utilities for player \(p\)
are given by \(S_x=(R, S, T, P)\) and for player \(q\) by \(S_y=(R, T, S, P)\)
and that the stationary scores of each player is given by \(S_X\) and \(S_Y\)
respectively. The main result of~\cite{Press2012} is that if

\begin{equation}\label{eqn:linear_relationship_for_p}
    \tilde p=\alpha S_x + \beta S_y + \gamma
\end{equation}

or

\begin{equation}\label{eqn:linear_relationship_for_q}
    \tilde q=\alpha S_x + \beta S_y + \gamma
\end{equation}

where \(\tilde p = (1 - p_1, 1 - p_2, p_3, p_4)\) and
\(\tilde q = (1 - q_1, 1 - q_2, q_3, q_4)\) then:

\begin{equation}
    \alpha S_X + \beta S_Y + \gamma = 0
\end{equation}

In~\cite{Press2012} a particular type of ZD strategy is defined: extortionate
strategies. If:

\begin{equation}\label{eqn:constraint_for_extortion}
    \gamma = - P(\alpha + \beta)
\end{equation}

then the player can ensure they get a score \(\chi\) times
larger than the opponent. This extortion coefficient is given by:

\begin{equation}\label{eqn:definition_of_chi}
    \chi=\frac{-\beta}{\alpha}
\end{equation}

Thus, if (\ref{eqn:constraint_for_extortion}) holds and \(\chi >1\) a player is
said to extort their opponent.
Here, the reverse problem is considered: given a
\(p\in\mathbb{R}^4\) how does one identify \(\alpha, \beta\) if they
exist and is the strategy in fact acting in an extortionate way?

These conditions correspond to:

\begin{align}
    \tilde p_1 & = \alpha R + \beta R - P (\alpha + \beta)
            \label{eqn:condition_for_tilde_p1}\\
    \tilde p_2 & = \alpha S + \beta T - P (\alpha + \beta)
            \label{eqn:condition_for_tilde_p2}\\
    \tilde p_3 & = \alpha T + \beta S - P (\alpha + \beta)
            \label{eqn:condition_for_tilde_p3}\\
    \tilde p_4 & = \alpha P + \beta P - P (\alpha + \beta)
            \label{eqn:condition_for_tilde_p4}
\end{align}

Equation (\ref{eqn:condition_for_tilde_p4}) ensures that \(p_4=\tilde p_4=0\).
Equations (\ref{eqn:condition_for_tilde_p1}-\ref{eqn:condition_for_tilde_p3})
can be used to eliminate \(\alpha, \beta\), giving:

\begin{equation}\label{eqn:planar_definition_of_extortion}
    \tilde p_1 = \frac{(R - P)(\tilde p_2 + \tilde p_3)}{S + T - 2P}
\end{equation}

with:

\begin{equation}\label{eqn:definition_of_chi}
    \chi = \frac{\tilde p_2 (P - T) + \tilde p_3 (S - P)}
                {\tilde p_2 (P - S) + \tilde p_3 (T - P)}
\end{equation}

Given a strategy \(p\in\mathbb{R}^{4\times 1}\) equations
(\ref{eqn:condition_for_tilde_p4}), (\ref{eqn:planar_definition_of_extortion}-\ref{eqn:definition_of_chi}) can be used to check if
a strategy is extortionate. The conditions correspond to:

\begin{align}
    p_1 & = \frac{(R-P)(p_2 + p_3) - R + T + S - P}{S + T - 2P}
     \label{eqn:condition_for_p1}\\
    p_4 & = 0 \label{eqn:condition_for_p4}\\
    1 & > p_2 + p_3\label{eqn:condition_for_chi}
\end{align}

The algebraic steps necessary to prove these results are available in the
supporting materials.

All extortionate strategies reside on a triangular (\ref{eqn:condition_for_chi})
plane (\ref{eqn:condition_for_p1}) in 3 dimensions (\ref{eqn:condition_for_p4}).
Using this formulation it can be seen that a necessary (but not sufficient)
condition for an extortionate strategy is that it cooperates on average less
than 50\% of the time when in a state of disagreement with the opponent.

As an example, consider the known extortionate strategy \(p=(8 / 9, 1 / 2, 1 /
3, 0)\) from~\cite{Stewart2012} which is referred to as \texttt{Extort-2}. In
this case, for the standard values of \((R, T, S, P)\) constraint
(\ref{eqn:condition_for_p1}) corresponds to:

\begin{equation}
    p_1 = \frac{2(p_2 + p_3) + 1}{3}
\end{equation}

It is clear that in this case all constraints hold.

This approach could in fact be used to confirm that a given strategy is acting
in an extortionate manner even if it is not a memory one strategy. However, in
practice, if a closed form for \(p\) is not known, then due to measurement
and/or numerical error this would not work.

This problem can be written in the following linear algebraic form where
\(x=(\alpha, \beta)\)
and \(p^*=(\tilde p_1 - 1, tilde_2 - 1, p_3)\):

\begin{equation}\label{eqn:linear_algebraic_equation_for_p}
    Cx= p^*
\end{equation}

\(C\) corresponds to equations
(\ref{eqn:condition_for_tilde_p1}-\ref{eqn:condition_for_tilde_p3}) and is
given by:

\begin{equation}\label{eqn:definition_of_C}
    C =
    \begin{bmatrix}
        R - P & R- P \\
        S - P & T- P \\
        T - P & S- P \\
    \end{bmatrix}
\end{equation}

Note that in general, equation (\ref{eqn:linear_algebraic_equation_for_p}) will
not necessarily have a solution. From the Rouch\'{e}-Capelli theorem if there is
a solution it is unique as \(\text{rank}(C)=2\) which is the dimension of the
variable \(x\). The best fitting \(x\) is found by minimizing:

\begin{equation}\label{eqn:r_squared}
    \text{SSError} = \|C x- p^*\|_2^2 = \sum_{i=1}^{3}\left((C\bar x)_i-p_i^*\right)^2
\end{equation}

Note that \(\text{SSError}\), which is the square of the Frobenius
norm~\cite{Golub2013}, becomes a measure of how close a strategy is to being an
extortionate strategy. Suspicion
of extortion then corresponds to a threshold on \(\text{SSError}\).

By observing interactions (human or otherwise), their memory one representation
can be inferred and this approach can be used to recognise extortionate
behaviour. The notion of comparing theoretic and actual plays of the IPD is not
novel, see for example~\cite{Rand2013}. Immediately it is noted that if the
environment is noisy~\cite{Wu1995} then no strategy can be considered to be
extortionate as \(p_4>0\).

In the next section, this idea will be illustrated by observing the interactions
that take place in a computer based tournament of the IPD\@.

\section{Numerical experiments}\label{sec:numerical-experiments}

In~\cite{Stewart2012} results from a tournament with
\input{./assets/tex/number_of_stewart_plotkin_strategies/main.tex} strategies,
was presented with specific consideration given to ZD strategies. This
tournament is reproduced here using the Axelrod-Python
project~\cite{Knight2016}. To obtain a good measure of the corresponding
transition rates for each strategy all matches have been run for
\input{assets/tex/number_of_turns/main.tex} turns and every match has been
repeated \input{assets/tex/number_of_repetitions/main.tex} times. All of this
interaction data is available at~\cite{vincent_knight_2018_1297075}. A good
match between the inferred Markov chain and the state distribution of the actual
interactions has been verified. Data for this is presented in the supplementary
materials.

Figure~\ref{fig:SSError_overall_in_stewart_plotkin} shows the \(\text{SSError}\)
values for all the strategies in the tournament, as reported
in~\cite{Stewart2012} the extortionate strategy (which has an expected
\(\text{SSError}\) approximately 0) gains a large number of wins.

\begin{figure}[!htbp]
    \centering
    \includegraphics[width=.8\textwidth]{./assets/img/SSError_overall_in_stewart_plotkin/main.pdf}
    \caption{\(\text{SSError}\) and state probabilities for the strategies
        of~\cite{Stewart2012}, ordered both by number of wins and overall score.
        Note that \(P(DC)\) is not shown as it corresponds to the transpose of
        \(P(CD)\). Cooperator and Defector are omitted as they do not visit all
        the states.}
    \label{fig:SSError_overall_in_stewart_plotkin}
\end{figure}

Here, the work of~\cite{Stewart2012} is extended by investigating a tournament
with \input{assets/tex/number_of_full_strategies/main.tex}
strategies.

The results of this analysis are shown in
Figure~\ref{fig:SSError_and_probabilities_in_full}. The top ranking strategies
by number of wins seem to be extortionate (but not against all strategies) and
it can be seen that a small sub group of strategies achieve mutual defection.
All the top ranking strategies according to score achieve mutual cooperation and
do not extort each other, however they
\textbf{do} exhibit extortionate behaviour towards a number of the lower ranking
strategies.

\begin{figure}[!htbp]
    \centering
    \includegraphics[width=.8\textwidth]{./assets/img/SSError_and_probabilities_in_full/main.pdf}
    \caption{\(\text{SSError}\) for the strategies for the full tournament. Only
    strategy interactions for which \(p_4=0\) and \(\chi>1\) are displayed.}
    \label{fig:SSError_and_probabilities_in_full}
\end{figure}

\section{Conclusion}\label{sec:conclusion}

This work defines an approach to measure whether or not a player is playing a
strategy that corresponds to an extortionate strategy as defined
in~\cite{Press2012}: a mathematical model for suspicion. Indeed, all
extortionate strategies have been
 classified as lying on a triangular plane.
This rigorous classification fails to be robust to small measurement error, thus
a statistical approach is proposed.
This is done through a linear algebraic approach for approximating the solution
of a linear system. Using this, a large number of pairwise interactions is
simulated and in fact very few strategies are found to act extortionately.

The work of~\cite{Press2012}, whilst showing that a clever approach to taking
advantage of another memory one strategy exists: this is incomplete. Whilst the
elegance of this result is very attractive, just as the simplicity of the
victory of Tit For Tat in Axelrod's original tournaments was, it is incomplete.
Extortionate strategies achieve a high number of wins but they do not
achieve a high score which corresponds to the fitness landscape in an
evolutionary sense. From the large number of interactions a payoff matrix \(S\)
can be measured where \(S_{ij}\) denotes the score (using standard values of
\((R, S, T, P) = (3, 0, 5, 1)\)) of the \(i\)th strategy
against the \(j\)th strategy. Using this, the replicator equation
describes the evolution of the system based on a population density fitness
function:

\begin{equation}\label{eqn:replicator_dynamics}
    \frac{dx}{dt} = x(S-x^TS x)
\end{equation}

Equation (\ref{eqn:replicator_dynamics}) is solved numerically through an
integration technique described in~\cite{Petzold1983} and
Figure~\ref{fig:replicator_dynamics} shows the evolution of the distribution of
the system: the various strategies are ranked by scores. It is clear to see that
only the high ranking strategies survive the evolutionary process (in fact,
only \input{./assets/img/replicator_dynamics/main.tex}
have a final distribution greater than \(10 ^ {-2}\)). This confirms the
findings of~\cite{Moran1707} in which sophisticated strategies resist
evolutionary invasion of shorter memory strategies. Recalling
Figure~\ref{fig:SSError_and_probabilities_in_full} this demonstrates that:

\begin{itemize}
    \item Cooperation emerges through the evolutionary process: the high scoring
        strategies do not exhibit extortionate behaviour towards each other.
    \item Extortionate strategies do not survive the evolutionary process.
\end{itemize}

\begin{figure}[!htbp]
    \centering
    \includegraphics[width=.8\textwidth]{./assets/img/replicator_dynamics/main.pdf}
    \caption{Numerical simulation of the replicator equation
    (\ref{eqn:replicator_dynamics}): strategies are ordered by score, only the strategies with a high score survive the evolutionary process.}
    \label{fig:replicator_dynamics}
\end{figure}

This work can be used to classify plays of the IPD\@: data can be collected from
actual interactions (in lab or in the field). Furthermore, this allows for a
classification method similar to the notion of fingerprinting presented
in~\cite{Ashlock2008}. Trained strategies can potentially be classified as
extortionate or not or it could be possible to even constrain the reinforcement
learning approaches that are becoming prevalent in the literature.
Alternatively, this mathematical approach for recognising extortion could be
used in sophisticated strategies to defend against invasion. Arguably, some of
the strategies considered here exhibit this behaviour, indeed as described
in~\cite{Harper2017}, the top ranking strategies in the full tournament are
obtained using evolutionary reinforcement learning techniques, thus, suspicion
of extortionate behaviour could in fact be an evolutionary trait.

\section*{Acknowledgements}

The following open source software libraries were used in this research:

\begin{itemize}
    \item The Axelrod ~\cite{Knight2016, Knight2018} library (IPD strategies and
        tournaments).
    \item The sympy library~\cite{Meurer2017} (verification of all symbolic
        calculations).
    \item The matplotlib~\cite{Droettboom2018} library (visualisation).
    \item The pandas~\cite{Structures2010}, dask~\cite{Dask2016} and
        NumPy~\cite{Oliphant2015} libraries (data manipulation).
    \item The SciPy~\cite{Jones2001} library (numerical integration of the
        replicator equation).
\end{itemize}

This work was performed using the computational facilities of the Advanced
Research Computing @ Cardiff (ARCCA) Division, Cardiff University.

\printbibliography

\newpage
\section*{Supplementary materials}

\includepdf{assets/pdf/proof_of_form_of_extortionate_strategies/main.pdf}

\newpage

Using the pair wise interactions the transition rates \(p,
q\) can be measured and the steady state probabilities inferred and compared to
the actual probabilities of each state.
This is done numerically by computing the singular eigenvector of the
matrix \(A\) \cite{Stewart2009}:

\[
    A =
    \begin{bmatrix}
        p_1 q_1 & p_1 (1 - q_1) & (1 - p_1) q_1 & (1 -p_1) (1 - q_1) \\
        p_2 q_2 & p_2 (1 - q_2) & (1 - p_2) q_2 & (1 -p_2) (1 - q_2) \\
        p_3 q_3 & p_3 (1 - q_3) & (1 - p_3) q_3 & (1 -p_3) (1 - q_3) \\
        p_4 q_4 & p_4 (1 - q_4) & (1 - p_4) q_4 & (1 -p_4) (1 - q_4) \\
    \end{bmatrix}
\]

Figure~\ref{fig:computed_probabilities_vs_theoretic_probabilities} shows a
regression line fitted to every pairwise interaction with a reported
\(\text{SSError}\) value (pairwise interactions with missing states were
omitted). This serves to validate the approach: a part from some edge cases the
relationship is consistent.

\begin{figure}[!htbp]
    \centering
    \includegraphics[width=.8\textwidth]{./assets/img/computed_probabilities_vs_theoretic_probabilities/main.pdf}
    \caption{The
        relationship between the steady state probabilities inferred from the
        measured transitions and the actual steady state probabilities. A linear
        regression line is included validating the approach.}
    \label{fig:computed_probabilities_vs_theoretic_probabilities}
\end{figure}


\end{document}

    strategies is considered. In this setting
    the most highly performing strategies do not play in an extortionate way
    against each other but do against lower performing strategies.
    This suggests that whilst the theory of Zero Determinant strategies
    indicates that memory is not of fundamental importance to the evolution of
    cooperative behaviour, this is incomplete.
\end{abstract}

\section{Introduction}\label{sec:introduction}

Agent based game theoretic models have become a stalwart of the underpinning
mathematics of interactive behaviours. One of the major pieces of work
in this area is the pair of original computer tournaments run by Robert
Axelrod~\cite{Axelrod1980, Axelrod1980a}. These tournaments pitted submitted
computer strategies against each other in plays of the Iterated Prisoner's
Dilemma. A common game where agents can choose to pay a slight cost to their
immediate utility in the hope of building a reputation. This has been used in
economic and evolutionary game theory to understand the evolution of cooperative
behaviour.

Recently, a class of strategies was described in~\cite{Press2012} that can
provably extort any given opponent. In~\cite{Hilbe2013, Moran1707} some
questions have already been asked about the true effectiveness of these
strategies in an evolutionary setting. Here another question is asked: is it
possible to recognise this extortionate behaviour? A mathematical procedure for
suspicion is presented: in the same way that the continued actions of an
extortionate individual might raise suspicion.

This work makes use of the Axelrod Python library~\cite{Knight2018, Knight2016}
with a large number of Prisoner Dilemma strategies available to give an
extensive numerical example of the ideas presented.  The approach is presented
in Section~\ref{sec:delta-zd-strategies}.  All of the code and data discussed
in Section~\ref{sec:numerical-experiments} is open sourced, archived and
written according to best scientific principles~\cite{Wilson2014}. The data
archive can be found at~\cite{vincent_knight_2018_1297075}.

\section{Recognising Extortion}\label{sec:delta-zd-strategies}

In~\cite{Press2012}, given a match between 2 memory-one strategies, the concept
of Zero Determinant (ZD) strategies is introduced. The main result of that paper
shows that given two memory one players \(p, q\in\mathbb{R}^4\) a linear
relationship between the players' scores could be forced by one of the players.

Using the notation of~\cite{Press2012}, assuming the utilities for player \(p\)
are given by \(S_x=(R, S, T, P)\) and for player \(q\) by \(S_y=(R, T, S, P)\)
and that the stationary scores of each player is given by \(S_X\) and \(S_Y\)
respectively. The main result of~\cite{Press2012} is that if

\begin{equation}\label{eqn:linear_relationship_for_p}
    \tilde p=\alpha S_x + \beta S_y + \gamma
\end{equation}

or

\begin{equation}\label{eqn:linear_relationship_for_q}
    \tilde q=\alpha S_x + \beta S_y + \gamma
\end{equation}

where \(\tilde p = (1 - p_1, 1 - p_2, p_3, p_4)\) and
\(\tilde q = (1 - q_1, 1 - q_2, q_3, q_4)\) then:

\begin{equation}
    \alpha S_X + \beta S_Y + \gamma = 0
\end{equation}

In~\cite{Press2012} a particular type of ZD strategy is defined: extortionate
strategies. If:

\begin{equation}\label{eqn:constraint_for_extortion}
    \gamma = - P(\alpha + \beta)
\end{equation}

then the player can ensure they get a score \(\chi\) times
larger than the opponent. This extortion coefficient is given by:

\begin{equation}\label{eqn:definition_of_chi}
    \chi=\frac{-\beta}{\alpha}
\end{equation}

Thus, if (\ref{eqn:constraint_for_extortion}) holds and \(\chi >1\) a player is
said to extort their opponent.
Here, the reverse problem is considered: given a
\(p\in\mathbb{R}^4\) how does one identify \(\alpha, \beta\) if they
exist and is the strategy in fact acting in an extortionate way?

These conditions correspond to:

\begin{align}
    \tilde p_1 & = \alpha R + \beta R - P (\alpha + \beta)
            \label{eqn:condition_for_tilde_p1}\\
    \tilde p_2 & = \alpha S + \beta T - P (\alpha + \beta)
            \label{eqn:condition_for_tilde_p2}\\
    \tilde p_3 & = \alpha T + \beta S - P (\alpha + \beta)
            \label{eqn:condition_for_tilde_p3}\\
    \tilde p_4 & = \alpha P + \beta P - P (\alpha + \beta)
            \label{eqn:condition_for_tilde_p4}
\end{align}

Equation (\ref{eqn:condition_for_tilde_p4}) ensures that \(p_4=\tilde p_4=0\).
Equations (\ref{eqn:condition_for_tilde_p1}-\ref{eqn:condition_for_tilde_p3})
can be used to eliminate \(\alpha, \beta\), giving:

\begin{equation}\label{eqn:planar_definition_of_extortion}
    \tilde p_1 = \frac{(R - P)(\tilde p_2 + \tilde p_3)}{S + T - 2P}
\end{equation}

with:

\begin{equation}\label{eqn:definition_of_chi}
    \chi = \frac{\tilde p_2 (P - T) + \tilde p_3 (S - P)}
                {\tilde p_2 (P - S) + \tilde p_3 (T - P)}
\end{equation}

Given a strategy \(p\in\mathbb{R}^{4\times 1}\) equations
(\ref{eqn:condition_for_tilde_p4}), (\ref{eqn:planar_definition_of_extortion}-\ref{eqn:definition_of_chi}) can be used to check if
a strategy is extortionate. The conditions correspond to:

\begin{align}
    p_1 & = \frac{(R-P)(p_2 + p_3) - R + T + S - P}{S + T - 2P}
     \label{eqn:condition_for_p1}\\
    p_4 & = 0 \label{eqn:condition_for_p4}\\
    1 & > p_2 + p_3\label{eqn:condition_for_chi}
\end{align}

The algebraic steps necessary to prove these results are available in the
supporting materials.

All extortionate strategies reside on a triangular (\ref{eqn:condition_for_chi})
plane (\ref{eqn:condition_for_p1}) in 3 dimensions (\ref{eqn:condition_for_p4}).
Using this formulation it can be seen that a necessary (but not sufficient)
condition for an extortionate strategy is that it cooperates on average less
than 50\% of the time when in a state of disagreement with the opponent.

As an example, consider the known extortionate strategy \(p=(8 / 9, 1 / 2, 1 /
3, 0)\) from~\cite{Stewart2012} which is referred to as \texttt{Extort-2}. In
this case, for the standard values of \((R, T, S, P)\) constraint
(\ref{eqn:condition_for_p1}) corresponds to:

\begin{equation}
    p_1 = \frac{2(p_2 + p_3) + 1}{3}
\end{equation}

It is clear that in this case all constraints hold.

This approach could in fact be used to confirm that a given strategy is acting
in an extortionate manner even if it is not a memory one strategy. However, in
practice, if a closed form for \(p\) is not known, then due to measurement
and/or numerical error this would not work.

This problem can be written in the following linear algebraic form where
\(x=(\alpha, \beta)\)
and \(p^*=(\tilde p_1 - 1, tilde_2 - 1, p_3)\):

\begin{equation}\label{eqn:linear_algebraic_equation_for_p}
    Cx= p^*
\end{equation}

\(C\) corresponds to equations
(\ref{eqn:condition_for_tilde_p1}-\ref{eqn:condition_for_tilde_p3}) and is
given by:

\begin{equation}\label{eqn:definition_of_C}
    C =
    \begin{bmatrix}
        R - P & R- P \\
        S - P & T- P \\
        T - P & S- P \\
    \end{bmatrix}
\end{equation}

Note that in general, equation (\ref{eqn:linear_algebraic_equation_for_p}) will
not necessarily have a solution. From the Rouch\'{e}-Capelli theorem if there is
a solution it is unique as \(\text{rank}(C)=2\) which is the dimension of the
variable \(x\). The best fitting \(x\) is found by minimizing:

\begin{equation}\label{eqn:r_squared}
    \text{SSError} = \|C x- p^*\|_2^2 = \sum_{i=1}^{3}\left((C\bar x)_i-p_i^*\right)^2
\end{equation}

Note that \(\text{SSError}\), which is the square of the Frobenius
norm~\cite{Golub2013}, becomes a measure of how close a strategy is to being an
extortionate strategy. Suspicion
of extortion then corresponds to a threshold on \(\text{SSError}\).

By observing interactions (human or otherwise), their memory one representation
can be inferred and this approach can be used to recognise extortionate
behaviour. The notion of comparing theoretic and actual plays of the IPD is not
novel, see for example~\cite{Rand2013}. Immediately it is noted that if the
environment is noisy~\cite{Wu1995} then no strategy can be considered to be
extortionate as \(p_4>0\).

In the next section, this idea will be illustrated by observing the interactions
that take place in a computer based tournament of the IPD\@.

\section{Numerical experiments}\label{sec:numerical-experiments}

In~\cite{Stewart2012} results from a tournament with
\documentclass[a4paper]{article}

\usepackage{amsmath}
\usepackage{amssymb}
\usepackage[margin=1.5cm,
            includefoot,
            footskip=30pt]{geometry}
\usepackage{layout}
\usepackage{graphicx}
\usepackage{subcaption}

\usepackage{biblatex}
\usepackage{pdfpages}

\bibliography{main.bib}

\title{Suspicion: Recognising and evaluating the effectiveness
       of extortion in the Iterated Prisoner's Dilemma}
\author{Vincent A. Knight \and Nikoleta E. Glynatsi}
\date{\today}



\begin{document}

\maketitle

\begin{abstract}
    The Iterated Prisoner's Dilemma is a model for rational and evolutionary
    interactive behaviour. It has applications both in the study of human social
    behaviour as well as in biology.
    It is used to understand when and how a rational individual might
    accept an immediate cost to their own utility for the direct benefit of
    another.

    Much attention has been given to a class of strategies called
    Zero Determinant strategies. It has been theoretically shown that these
    strategies can ``extort'' any player.

    In this work, an approach to identify if observed strategies are playing in
    an extortionate way is described. Furthermore, experimental analysis of
    a large tournament with \input{assets/tex/number_of_full_strategies/main.tex}
    strategies is considered. In this setting
    the most highly performing strategies do not play in an extortionate way
    against each other but do against lower performing strategies.
    This suggests that whilst the theory of Zero Determinant strategies
    indicates that memory is not of fundamental importance to the evolution of
    cooperative behaviour, this is incomplete.
\end{abstract}

\section{Introduction}\label{sec:introduction}

Agent based game theoretic models have become a stalwart of the underpinning
mathematics of interactive behaviours. One of the major pieces of work
in this area is the pair of original computer tournaments run by Robert
Axelrod~\cite{Axelrod1980, Axelrod1980a}. These tournaments pitted submitted
computer strategies against each other in plays of the Iterated Prisoner's
Dilemma. A common game where agents can choose to pay a slight cost to their
immediate utility in the hope of building a reputation. This has been used in
economic and evolutionary game theory to understand the evolution of cooperative
behaviour.

Recently, a class of strategies was described in~\cite{Press2012} that can
provably extort any given opponent. In~\cite{Hilbe2013, Moran1707} some
questions have already been asked about the true effectiveness of these
strategies in an evolutionary setting. Here another question is asked: is it
possible to recognise this extortionate behaviour? A mathematical procedure for
suspicion is presented: in the same way that the continued actions of an
extortionate individual might raise suspicion.

This work makes use of the Axelrod Python library~\cite{Knight2018, Knight2016}
with a large number of Prisoner Dilemma strategies available to give an
extensive numerical example of the ideas presented.  The approach is presented
in Section~\ref{sec:delta-zd-strategies}.  All of the code and data discussed
in Section~\ref{sec:numerical-experiments} is open sourced, archived and
written according to best scientific principles~\cite{Wilson2014}. The data
archive can be found at~\cite{vincent_knight_2018_1297075}.

\section{Recognising Extortion}\label{sec:delta-zd-strategies}

In~\cite{Press2012}, given a match between 2 memory-one strategies, the concept
of Zero Determinant (ZD) strategies is introduced. The main result of that paper
shows that given two memory one players \(p, q\in\mathbb{R}^4\) a linear
relationship between the players' scores could be forced by one of the players.

Using the notation of~\cite{Press2012}, assuming the utilities for player \(p\)
are given by \(S_x=(R, S, T, P)\) and for player \(q\) by \(S_y=(R, T, S, P)\)
and that the stationary scores of each player is given by \(S_X\) and \(S_Y\)
respectively. The main result of~\cite{Press2012} is that if

\begin{equation}\label{eqn:linear_relationship_for_p}
    \tilde p=\alpha S_x + \beta S_y + \gamma
\end{equation}

or

\begin{equation}\label{eqn:linear_relationship_for_q}
    \tilde q=\alpha S_x + \beta S_y + \gamma
\end{equation}

where \(\tilde p = (1 - p_1, 1 - p_2, p_3, p_4)\) and
\(\tilde q = (1 - q_1, 1 - q_2, q_3, q_4)\) then:

\begin{equation}
    \alpha S_X + \beta S_Y + \gamma = 0
\end{equation}

In~\cite{Press2012} a particular type of ZD strategy is defined: extortionate
strategies. If:

\begin{equation}\label{eqn:constraint_for_extortion}
    \gamma = - P(\alpha + \beta)
\end{equation}

then the player can ensure they get a score \(\chi\) times
larger than the opponent. This extortion coefficient is given by:

\begin{equation}\label{eqn:definition_of_chi}
    \chi=\frac{-\beta}{\alpha}
\end{equation}

Thus, if (\ref{eqn:constraint_for_extortion}) holds and \(\chi >1\) a player is
said to extort their opponent.
Here, the reverse problem is considered: given a
\(p\in\mathbb{R}^4\) how does one identify \(\alpha, \beta\) if they
exist and is the strategy in fact acting in an extortionate way?

These conditions correspond to:

\begin{align}
    \tilde p_1 & = \alpha R + \beta R - P (\alpha + \beta)
            \label{eqn:condition_for_tilde_p1}\\
    \tilde p_2 & = \alpha S + \beta T - P (\alpha + \beta)
            \label{eqn:condition_for_tilde_p2}\\
    \tilde p_3 & = \alpha T + \beta S - P (\alpha + \beta)
            \label{eqn:condition_for_tilde_p3}\\
    \tilde p_4 & = \alpha P + \beta P - P (\alpha + \beta)
            \label{eqn:condition_for_tilde_p4}
\end{align}

Equation (\ref{eqn:condition_for_tilde_p4}) ensures that \(p_4=\tilde p_4=0\).
Equations (\ref{eqn:condition_for_tilde_p1}-\ref{eqn:condition_for_tilde_p3})
can be used to eliminate \(\alpha, \beta\), giving:

\begin{equation}\label{eqn:planar_definition_of_extortion}
    \tilde p_1 = \frac{(R - P)(\tilde p_2 + \tilde p_3)}{S + T - 2P}
\end{equation}

with:

\begin{equation}\label{eqn:definition_of_chi}
    \chi = \frac{\tilde p_2 (P - T) + \tilde p_3 (S - P)}
                {\tilde p_2 (P - S) + \tilde p_3 (T - P)}
\end{equation}

Given a strategy \(p\in\mathbb{R}^{4\times 1}\) equations
(\ref{eqn:condition_for_tilde_p4}), (\ref{eqn:planar_definition_of_extortion}-\ref{eqn:definition_of_chi}) can be used to check if
a strategy is extortionate. The conditions correspond to:

\begin{align}
    p_1 & = \frac{(R-P)(p_2 + p_3) - R + T + S - P}{S + T - 2P}
     \label{eqn:condition_for_p1}\\
    p_4 & = 0 \label{eqn:condition_for_p4}\\
    1 & > p_2 + p_3\label{eqn:condition_for_chi}
\end{align}

The algebraic steps necessary to prove these results are available in the
supporting materials.

All extortionate strategies reside on a triangular (\ref{eqn:condition_for_chi})
plane (\ref{eqn:condition_for_p1}) in 3 dimensions (\ref{eqn:condition_for_p4}).
Using this formulation it can be seen that a necessary (but not sufficient)
condition for an extortionate strategy is that it cooperates on average less
than 50\% of the time when in a state of disagreement with the opponent.

As an example, consider the known extortionate strategy \(p=(8 / 9, 1 / 2, 1 /
3, 0)\) from~\cite{Stewart2012} which is referred to as \texttt{Extort-2}. In
this case, for the standard values of \((R, T, S, P)\) constraint
(\ref{eqn:condition_for_p1}) corresponds to:

\begin{equation}
    p_1 = \frac{2(p_2 + p_3) + 1}{3}
\end{equation}

It is clear that in this case all constraints hold.

This approach could in fact be used to confirm that a given strategy is acting
in an extortionate manner even if it is not a memory one strategy. However, in
practice, if a closed form for \(p\) is not known, then due to measurement
and/or numerical error this would not work.

This problem can be written in the following linear algebraic form where
\(x=(\alpha, \beta)\)
and \(p^*=(\tilde p_1 - 1, tilde_2 - 1, p_3)\):

\begin{equation}\label{eqn:linear_algebraic_equation_for_p}
    Cx= p^*
\end{equation}

\(C\) corresponds to equations
(\ref{eqn:condition_for_tilde_p1}-\ref{eqn:condition_for_tilde_p3}) and is
given by:

\begin{equation}\label{eqn:definition_of_C}
    C =
    \begin{bmatrix}
        R - P & R- P \\
        S - P & T- P \\
        T - P & S- P \\
    \end{bmatrix}
\end{equation}

Note that in general, equation (\ref{eqn:linear_algebraic_equation_for_p}) will
not necessarily have a solution. From the Rouch\'{e}-Capelli theorem if there is
a solution it is unique as \(\text{rank}(C)=2\) which is the dimension of the
variable \(x\). The best fitting \(x\) is found by minimizing:

\begin{equation}\label{eqn:r_squared}
    \text{SSError} = \|C x- p^*\|_2^2 = \sum_{i=1}^{3}\left((C\bar x)_i-p_i^*\right)^2
\end{equation}

Note that \(\text{SSError}\), which is the square of the Frobenius
norm~\cite{Golub2013}, becomes a measure of how close a strategy is to being an
extortionate strategy. Suspicion
of extortion then corresponds to a threshold on \(\text{SSError}\).

By observing interactions (human or otherwise), their memory one representation
can be inferred and this approach can be used to recognise extortionate
behaviour. The notion of comparing theoretic and actual plays of the IPD is not
novel, see for example~\cite{Rand2013}. Immediately it is noted that if the
environment is noisy~\cite{Wu1995} then no strategy can be considered to be
extortionate as \(p_4>0\).

In the next section, this idea will be illustrated by observing the interactions
that take place in a computer based tournament of the IPD\@.

\section{Numerical experiments}\label{sec:numerical-experiments}

In~\cite{Stewart2012} results from a tournament with
\input{./assets/tex/number_of_stewart_plotkin_strategies/main.tex} strategies,
was presented with specific consideration given to ZD strategies. This
tournament is reproduced here using the Axelrod-Python
project~\cite{Knight2016}. To obtain a good measure of the corresponding
transition rates for each strategy all matches have been run for
\input{assets/tex/number_of_turns/main.tex} turns and every match has been
repeated \input{assets/tex/number_of_repetitions/main.tex} times. All of this
interaction data is available at~\cite{vincent_knight_2018_1297075}. A good
match between the inferred Markov chain and the state distribution of the actual
interactions has been verified. Data for this is presented in the supplementary
materials.

Figure~\ref{fig:SSError_overall_in_stewart_plotkin} shows the \(\text{SSError}\)
values for all the strategies in the tournament, as reported
in~\cite{Stewart2012} the extortionate strategy (which has an expected
\(\text{SSError}\) approximately 0) gains a large number of wins.

\begin{figure}[!htbp]
    \centering
    \includegraphics[width=.8\textwidth]{./assets/img/SSError_overall_in_stewart_plotkin/main.pdf}
    \caption{\(\text{SSError}\) and state probabilities for the strategies
        of~\cite{Stewart2012}, ordered both by number of wins and overall score.
        Note that \(P(DC)\) is not shown as it corresponds to the transpose of
        \(P(CD)\). Cooperator and Defector are omitted as they do not visit all
        the states.}
    \label{fig:SSError_overall_in_stewart_plotkin}
\end{figure}

Here, the work of~\cite{Stewart2012} is extended by investigating a tournament
with \input{assets/tex/number_of_full_strategies/main.tex}
strategies.

The results of this analysis are shown in
Figure~\ref{fig:SSError_and_probabilities_in_full}. The top ranking strategies
by number of wins seem to be extortionate (but not against all strategies) and
it can be seen that a small sub group of strategies achieve mutual defection.
All the top ranking strategies according to score achieve mutual cooperation and
do not extort each other, however they
\textbf{do} exhibit extortionate behaviour towards a number of the lower ranking
strategies.

\begin{figure}[!htbp]
    \centering
    \includegraphics[width=.8\textwidth]{./assets/img/SSError_and_probabilities_in_full/main.pdf}
    \caption{\(\text{SSError}\) for the strategies for the full tournament. Only
    strategy interactions for which \(p_4=0\) and \(\chi>1\) are displayed.}
    \label{fig:SSError_and_probabilities_in_full}
\end{figure}

\section{Conclusion}\label{sec:conclusion}

This work defines an approach to measure whether or not a player is playing a
strategy that corresponds to an extortionate strategy as defined
in~\cite{Press2012}: a mathematical model for suspicion. Indeed, all
extortionate strategies have been
 classified as lying on a triangular plane.
This rigorous classification fails to be robust to small measurement error, thus
a statistical approach is proposed.
This is done through a linear algebraic approach for approximating the solution
of a linear system. Using this, a large number of pairwise interactions is
simulated and in fact very few strategies are found to act extortionately.

The work of~\cite{Press2012}, whilst showing that a clever approach to taking
advantage of another memory one strategy exists: this is incomplete. Whilst the
elegance of this result is very attractive, just as the simplicity of the
victory of Tit For Tat in Axelrod's original tournaments was, it is incomplete.
Extortionate strategies achieve a high number of wins but they do not
achieve a high score which corresponds to the fitness landscape in an
evolutionary sense. From the large number of interactions a payoff matrix \(S\)
can be measured where \(S_{ij}\) denotes the score (using standard values of
\((R, S, T, P) = (3, 0, 5, 1)\)) of the \(i\)th strategy
against the \(j\)th strategy. Using this, the replicator equation
describes the evolution of the system based on a population density fitness
function:

\begin{equation}\label{eqn:replicator_dynamics}
    \frac{dx}{dt} = x(S-x^TS x)
\end{equation}

Equation (\ref{eqn:replicator_dynamics}) is solved numerically through an
integration technique described in~\cite{Petzold1983} and
Figure~\ref{fig:replicator_dynamics} shows the evolution of the distribution of
the system: the various strategies are ranked by scores. It is clear to see that
only the high ranking strategies survive the evolutionary process (in fact,
only \input{./assets/img/replicator_dynamics/main.tex}
have a final distribution greater than \(10 ^ {-2}\)). This confirms the
findings of~\cite{Moran1707} in which sophisticated strategies resist
evolutionary invasion of shorter memory strategies. Recalling
Figure~\ref{fig:SSError_and_probabilities_in_full} this demonstrates that:

\begin{itemize}
    \item Cooperation emerges through the evolutionary process: the high scoring
        strategies do not exhibit extortionate behaviour towards each other.
    \item Extortionate strategies do not survive the evolutionary process.
\end{itemize}

\begin{figure}[!htbp]
    \centering
    \includegraphics[width=.8\textwidth]{./assets/img/replicator_dynamics/main.pdf}
    \caption{Numerical simulation of the replicator equation
    (\ref{eqn:replicator_dynamics}): strategies are ordered by score, only the strategies with a high score survive the evolutionary process.}
    \label{fig:replicator_dynamics}
\end{figure}

This work can be used to classify plays of the IPD\@: data can be collected from
actual interactions (in lab or in the field). Furthermore, this allows for a
classification method similar to the notion of fingerprinting presented
in~\cite{Ashlock2008}. Trained strategies can potentially be classified as
extortionate or not or it could be possible to even constrain the reinforcement
learning approaches that are becoming prevalent in the literature.
Alternatively, this mathematical approach for recognising extortion could be
used in sophisticated strategies to defend against invasion. Arguably, some of
the strategies considered here exhibit this behaviour, indeed as described
in~\cite{Harper2017}, the top ranking strategies in the full tournament are
obtained using evolutionary reinforcement learning techniques, thus, suspicion
of extortionate behaviour could in fact be an evolutionary trait.

\section*{Acknowledgements}

The following open source software libraries were used in this research:

\begin{itemize}
    \item The Axelrod ~\cite{Knight2016, Knight2018} library (IPD strategies and
        tournaments).
    \item The sympy library~\cite{Meurer2017} (verification of all symbolic
        calculations).
    \item The matplotlib~\cite{Droettboom2018} library (visualisation).
    \item The pandas~\cite{Structures2010}, dask~\cite{Dask2016} and
        NumPy~\cite{Oliphant2015} libraries (data manipulation).
    \item The SciPy~\cite{Jones2001} library (numerical integration of the
        replicator equation).
\end{itemize}

This work was performed using the computational facilities of the Advanced
Research Computing @ Cardiff (ARCCA) Division, Cardiff University.

\printbibliography

\newpage
\section*{Supplementary materials}

\includepdf{assets/pdf/proof_of_form_of_extortionate_strategies/main.pdf}

\newpage

Using the pair wise interactions the transition rates \(p,
q\) can be measured and the steady state probabilities inferred and compared to
the actual probabilities of each state.
This is done numerically by computing the singular eigenvector of the
matrix \(A\) \cite{Stewart2009}:

\[
    A =
    \begin{bmatrix}
        p_1 q_1 & p_1 (1 - q_1) & (1 - p_1) q_1 & (1 -p_1) (1 - q_1) \\
        p_2 q_2 & p_2 (1 - q_2) & (1 - p_2) q_2 & (1 -p_2) (1 - q_2) \\
        p_3 q_3 & p_3 (1 - q_3) & (1 - p_3) q_3 & (1 -p_3) (1 - q_3) \\
        p_4 q_4 & p_4 (1 - q_4) & (1 - p_4) q_4 & (1 -p_4) (1 - q_4) \\
    \end{bmatrix}
\]

Figure~\ref{fig:computed_probabilities_vs_theoretic_probabilities} shows a
regression line fitted to every pairwise interaction with a reported
\(\text{SSError}\) value (pairwise interactions with missing states were
omitted). This serves to validate the approach: a part from some edge cases the
relationship is consistent.

\begin{figure}[!htbp]
    \centering
    \includegraphics[width=.8\textwidth]{./assets/img/computed_probabilities_vs_theoretic_probabilities/main.pdf}
    \caption{The
        relationship between the steady state probabilities inferred from the
        measured transitions and the actual steady state probabilities. A linear
        regression line is included validating the approach.}
    \label{fig:computed_probabilities_vs_theoretic_probabilities}
\end{figure}


\end{document}
 strategies,
was presented with specific consideration given to ZD strategies. This
tournament is reproduced here using the Axelrod-Python
project~\cite{Knight2016}. To obtain a good measure of the corresponding
transition rates for each strategy all matches have been run for
\documentclass[a4paper]{article}

\usepackage{amsmath}
\usepackage{amssymb}
\usepackage[margin=1.5cm,
            includefoot,
            footskip=30pt]{geometry}
\usepackage{layout}
\usepackage{graphicx}
\usepackage{subcaption}

\usepackage{biblatex}
\usepackage{pdfpages}

\bibliography{main.bib}

\title{Suspicion: Recognising and evaluating the effectiveness
       of extortion in the Iterated Prisoner's Dilemma}
\author{Vincent A. Knight \and Nikoleta E. Glynatsi}
\date{\today}



\begin{document}

\maketitle

\begin{abstract}
    The Iterated Prisoner's Dilemma is a model for rational and evolutionary
    interactive behaviour. It has applications both in the study of human social
    behaviour as well as in biology.
    It is used to understand when and how a rational individual might
    accept an immediate cost to their own utility for the direct benefit of
    another.

    Much attention has been given to a class of strategies called
    Zero Determinant strategies. It has been theoretically shown that these
    strategies can ``extort'' any player.

    In this work, an approach to identify if observed strategies are playing in
    an extortionate way is described. Furthermore, experimental analysis of
    a large tournament with \input{assets/tex/number_of_full_strategies/main.tex}
    strategies is considered. In this setting
    the most highly performing strategies do not play in an extortionate way
    against each other but do against lower performing strategies.
    This suggests that whilst the theory of Zero Determinant strategies
    indicates that memory is not of fundamental importance to the evolution of
    cooperative behaviour, this is incomplete.
\end{abstract}

\section{Introduction}\label{sec:introduction}

Agent based game theoretic models have become a stalwart of the underpinning
mathematics of interactive behaviours. One of the major pieces of work
in this area is the pair of original computer tournaments run by Robert
Axelrod~\cite{Axelrod1980, Axelrod1980a}. These tournaments pitted submitted
computer strategies against each other in plays of the Iterated Prisoner's
Dilemma. A common game where agents can choose to pay a slight cost to their
immediate utility in the hope of building a reputation. This has been used in
economic and evolutionary game theory to understand the evolution of cooperative
behaviour.

Recently, a class of strategies was described in~\cite{Press2012} that can
provably extort any given opponent. In~\cite{Hilbe2013, Moran1707} some
questions have already been asked about the true effectiveness of these
strategies in an evolutionary setting. Here another question is asked: is it
possible to recognise this extortionate behaviour? A mathematical procedure for
suspicion is presented: in the same way that the continued actions of an
extortionate individual might raise suspicion.

This work makes use of the Axelrod Python library~\cite{Knight2018, Knight2016}
with a large number of Prisoner Dilemma strategies available to give an
extensive numerical example of the ideas presented.  The approach is presented
in Section~\ref{sec:delta-zd-strategies}.  All of the code and data discussed
in Section~\ref{sec:numerical-experiments} is open sourced, archived and
written according to best scientific principles~\cite{Wilson2014}. The data
archive can be found at~\cite{vincent_knight_2018_1297075}.

\section{Recognising Extortion}\label{sec:delta-zd-strategies}

In~\cite{Press2012}, given a match between 2 memory-one strategies, the concept
of Zero Determinant (ZD) strategies is introduced. The main result of that paper
shows that given two memory one players \(p, q\in\mathbb{R}^4\) a linear
relationship between the players' scores could be forced by one of the players.

Using the notation of~\cite{Press2012}, assuming the utilities for player \(p\)
are given by \(S_x=(R, S, T, P)\) and for player \(q\) by \(S_y=(R, T, S, P)\)
and that the stationary scores of each player is given by \(S_X\) and \(S_Y\)
respectively. The main result of~\cite{Press2012} is that if

\begin{equation}\label{eqn:linear_relationship_for_p}
    \tilde p=\alpha S_x + \beta S_y + \gamma
\end{equation}

or

\begin{equation}\label{eqn:linear_relationship_for_q}
    \tilde q=\alpha S_x + \beta S_y + \gamma
\end{equation}

where \(\tilde p = (1 - p_1, 1 - p_2, p_3, p_4)\) and
\(\tilde q = (1 - q_1, 1 - q_2, q_3, q_4)\) then:

\begin{equation}
    \alpha S_X + \beta S_Y + \gamma = 0
\end{equation}

In~\cite{Press2012} a particular type of ZD strategy is defined: extortionate
strategies. If:

\begin{equation}\label{eqn:constraint_for_extortion}
    \gamma = - P(\alpha + \beta)
\end{equation}

then the player can ensure they get a score \(\chi\) times
larger than the opponent. This extortion coefficient is given by:

\begin{equation}\label{eqn:definition_of_chi}
    \chi=\frac{-\beta}{\alpha}
\end{equation}

Thus, if (\ref{eqn:constraint_for_extortion}) holds and \(\chi >1\) a player is
said to extort their opponent.
Here, the reverse problem is considered: given a
\(p\in\mathbb{R}^4\) how does one identify \(\alpha, \beta\) if they
exist and is the strategy in fact acting in an extortionate way?

These conditions correspond to:

\begin{align}
    \tilde p_1 & = \alpha R + \beta R - P (\alpha + \beta)
            \label{eqn:condition_for_tilde_p1}\\
    \tilde p_2 & = \alpha S + \beta T - P (\alpha + \beta)
            \label{eqn:condition_for_tilde_p2}\\
    \tilde p_3 & = \alpha T + \beta S - P (\alpha + \beta)
            \label{eqn:condition_for_tilde_p3}\\
    \tilde p_4 & = \alpha P + \beta P - P (\alpha + \beta)
            \label{eqn:condition_for_tilde_p4}
\end{align}

Equation (\ref{eqn:condition_for_tilde_p4}) ensures that \(p_4=\tilde p_4=0\).
Equations (\ref{eqn:condition_for_tilde_p1}-\ref{eqn:condition_for_tilde_p3})
can be used to eliminate \(\alpha, \beta\), giving:

\begin{equation}\label{eqn:planar_definition_of_extortion}
    \tilde p_1 = \frac{(R - P)(\tilde p_2 + \tilde p_3)}{S + T - 2P}
\end{equation}

with:

\begin{equation}\label{eqn:definition_of_chi}
    \chi = \frac{\tilde p_2 (P - T) + \tilde p_3 (S - P)}
                {\tilde p_2 (P - S) + \tilde p_3 (T - P)}
\end{equation}

Given a strategy \(p\in\mathbb{R}^{4\times 1}\) equations
(\ref{eqn:condition_for_tilde_p4}), (\ref{eqn:planar_definition_of_extortion}-\ref{eqn:definition_of_chi}) can be used to check if
a strategy is extortionate. The conditions correspond to:

\begin{align}
    p_1 & = \frac{(R-P)(p_2 + p_3) - R + T + S - P}{S + T - 2P}
     \label{eqn:condition_for_p1}\\
    p_4 & = 0 \label{eqn:condition_for_p4}\\
    1 & > p_2 + p_3\label{eqn:condition_for_chi}
\end{align}

The algebraic steps necessary to prove these results are available in the
supporting materials.

All extortionate strategies reside on a triangular (\ref{eqn:condition_for_chi})
plane (\ref{eqn:condition_for_p1}) in 3 dimensions (\ref{eqn:condition_for_p4}).
Using this formulation it can be seen that a necessary (but not sufficient)
condition for an extortionate strategy is that it cooperates on average less
than 50\% of the time when in a state of disagreement with the opponent.

As an example, consider the known extortionate strategy \(p=(8 / 9, 1 / 2, 1 /
3, 0)\) from~\cite{Stewart2012} which is referred to as \texttt{Extort-2}. In
this case, for the standard values of \((R, T, S, P)\) constraint
(\ref{eqn:condition_for_p1}) corresponds to:

\begin{equation}
    p_1 = \frac{2(p_2 + p_3) + 1}{3}
\end{equation}

It is clear that in this case all constraints hold.

This approach could in fact be used to confirm that a given strategy is acting
in an extortionate manner even if it is not a memory one strategy. However, in
practice, if a closed form for \(p\) is not known, then due to measurement
and/or numerical error this would not work.

This problem can be written in the following linear algebraic form where
\(x=(\alpha, \beta)\)
and \(p^*=(\tilde p_1 - 1, tilde_2 - 1, p_3)\):

\begin{equation}\label{eqn:linear_algebraic_equation_for_p}
    Cx= p^*
\end{equation}

\(C\) corresponds to equations
(\ref{eqn:condition_for_tilde_p1}-\ref{eqn:condition_for_tilde_p3}) and is
given by:

\begin{equation}\label{eqn:definition_of_C}
    C =
    \begin{bmatrix}
        R - P & R- P \\
        S - P & T- P \\
        T - P & S- P \\
    \end{bmatrix}
\end{equation}

Note that in general, equation (\ref{eqn:linear_algebraic_equation_for_p}) will
not necessarily have a solution. From the Rouch\'{e}-Capelli theorem if there is
a solution it is unique as \(\text{rank}(C)=2\) which is the dimension of the
variable \(x\). The best fitting \(x\) is found by minimizing:

\begin{equation}\label{eqn:r_squared}
    \text{SSError} = \|C x- p^*\|_2^2 = \sum_{i=1}^{3}\left((C\bar x)_i-p_i^*\right)^2
\end{equation}

Note that \(\text{SSError}\), which is the square of the Frobenius
norm~\cite{Golub2013}, becomes a measure of how close a strategy is to being an
extortionate strategy. Suspicion
of extortion then corresponds to a threshold on \(\text{SSError}\).

By observing interactions (human or otherwise), their memory one representation
can be inferred and this approach can be used to recognise extortionate
behaviour. The notion of comparing theoretic and actual plays of the IPD is not
novel, see for example~\cite{Rand2013}. Immediately it is noted that if the
environment is noisy~\cite{Wu1995} then no strategy can be considered to be
extortionate as \(p_4>0\).

In the next section, this idea will be illustrated by observing the interactions
that take place in a computer based tournament of the IPD\@.

\section{Numerical experiments}\label{sec:numerical-experiments}

In~\cite{Stewart2012} results from a tournament with
\input{./assets/tex/number_of_stewart_plotkin_strategies/main.tex} strategies,
was presented with specific consideration given to ZD strategies. This
tournament is reproduced here using the Axelrod-Python
project~\cite{Knight2016}. To obtain a good measure of the corresponding
transition rates for each strategy all matches have been run for
\input{assets/tex/number_of_turns/main.tex} turns and every match has been
repeated \input{assets/tex/number_of_repetitions/main.tex} times. All of this
interaction data is available at~\cite{vincent_knight_2018_1297075}. A good
match between the inferred Markov chain and the state distribution of the actual
interactions has been verified. Data for this is presented in the supplementary
materials.

Figure~\ref{fig:SSError_overall_in_stewart_plotkin} shows the \(\text{SSError}\)
values for all the strategies in the tournament, as reported
in~\cite{Stewart2012} the extortionate strategy (which has an expected
\(\text{SSError}\) approximately 0) gains a large number of wins.

\begin{figure}[!htbp]
    \centering
    \includegraphics[width=.8\textwidth]{./assets/img/SSError_overall_in_stewart_plotkin/main.pdf}
    \caption{\(\text{SSError}\) and state probabilities for the strategies
        of~\cite{Stewart2012}, ordered both by number of wins and overall score.
        Note that \(P(DC)\) is not shown as it corresponds to the transpose of
        \(P(CD)\). Cooperator and Defector are omitted as they do not visit all
        the states.}
    \label{fig:SSError_overall_in_stewart_plotkin}
\end{figure}

Here, the work of~\cite{Stewart2012} is extended by investigating a tournament
with \input{assets/tex/number_of_full_strategies/main.tex}
strategies.

The results of this analysis are shown in
Figure~\ref{fig:SSError_and_probabilities_in_full}. The top ranking strategies
by number of wins seem to be extortionate (but not against all strategies) and
it can be seen that a small sub group of strategies achieve mutual defection.
All the top ranking strategies according to score achieve mutual cooperation and
do not extort each other, however they
\textbf{do} exhibit extortionate behaviour towards a number of the lower ranking
strategies.

\begin{figure}[!htbp]
    \centering
    \includegraphics[width=.8\textwidth]{./assets/img/SSError_and_probabilities_in_full/main.pdf}
    \caption{\(\text{SSError}\) for the strategies for the full tournament. Only
    strategy interactions for which \(p_4=0\) and \(\chi>1\) are displayed.}
    \label{fig:SSError_and_probabilities_in_full}
\end{figure}

\section{Conclusion}\label{sec:conclusion}

This work defines an approach to measure whether or not a player is playing a
strategy that corresponds to an extortionate strategy as defined
in~\cite{Press2012}: a mathematical model for suspicion. Indeed, all
extortionate strategies have been
 classified as lying on a triangular plane.
This rigorous classification fails to be robust to small measurement error, thus
a statistical approach is proposed.
This is done through a linear algebraic approach for approximating the solution
of a linear system. Using this, a large number of pairwise interactions is
simulated and in fact very few strategies are found to act extortionately.

The work of~\cite{Press2012}, whilst showing that a clever approach to taking
advantage of another memory one strategy exists: this is incomplete. Whilst the
elegance of this result is very attractive, just as the simplicity of the
victory of Tit For Tat in Axelrod's original tournaments was, it is incomplete.
Extortionate strategies achieve a high number of wins but they do not
achieve a high score which corresponds to the fitness landscape in an
evolutionary sense. From the large number of interactions a payoff matrix \(S\)
can be measured where \(S_{ij}\) denotes the score (using standard values of
\((R, S, T, P) = (3, 0, 5, 1)\)) of the \(i\)th strategy
against the \(j\)th strategy. Using this, the replicator equation
describes the evolution of the system based on a population density fitness
function:

\begin{equation}\label{eqn:replicator_dynamics}
    \frac{dx}{dt} = x(S-x^TS x)
\end{equation}

Equation (\ref{eqn:replicator_dynamics}) is solved numerically through an
integration technique described in~\cite{Petzold1983} and
Figure~\ref{fig:replicator_dynamics} shows the evolution of the distribution of
the system: the various strategies are ranked by scores. It is clear to see that
only the high ranking strategies survive the evolutionary process (in fact,
only \input{./assets/img/replicator_dynamics/main.tex}
have a final distribution greater than \(10 ^ {-2}\)). This confirms the
findings of~\cite{Moran1707} in which sophisticated strategies resist
evolutionary invasion of shorter memory strategies. Recalling
Figure~\ref{fig:SSError_and_probabilities_in_full} this demonstrates that:

\begin{itemize}
    \item Cooperation emerges through the evolutionary process: the high scoring
        strategies do not exhibit extortionate behaviour towards each other.
    \item Extortionate strategies do not survive the evolutionary process.
\end{itemize}

\begin{figure}[!htbp]
    \centering
    \includegraphics[width=.8\textwidth]{./assets/img/replicator_dynamics/main.pdf}
    \caption{Numerical simulation of the replicator equation
    (\ref{eqn:replicator_dynamics}): strategies are ordered by score, only the strategies with a high score survive the evolutionary process.}
    \label{fig:replicator_dynamics}
\end{figure}

This work can be used to classify plays of the IPD\@: data can be collected from
actual interactions (in lab or in the field). Furthermore, this allows for a
classification method similar to the notion of fingerprinting presented
in~\cite{Ashlock2008}. Trained strategies can potentially be classified as
extortionate or not or it could be possible to even constrain the reinforcement
learning approaches that are becoming prevalent in the literature.
Alternatively, this mathematical approach for recognising extortion could be
used in sophisticated strategies to defend against invasion. Arguably, some of
the strategies considered here exhibit this behaviour, indeed as described
in~\cite{Harper2017}, the top ranking strategies in the full tournament are
obtained using evolutionary reinforcement learning techniques, thus, suspicion
of extortionate behaviour could in fact be an evolutionary trait.

\section*{Acknowledgements}

The following open source software libraries were used in this research:

\begin{itemize}
    \item The Axelrod ~\cite{Knight2016, Knight2018} library (IPD strategies and
        tournaments).
    \item The sympy library~\cite{Meurer2017} (verification of all symbolic
        calculations).
    \item The matplotlib~\cite{Droettboom2018} library (visualisation).
    \item The pandas~\cite{Structures2010}, dask~\cite{Dask2016} and
        NumPy~\cite{Oliphant2015} libraries (data manipulation).
    \item The SciPy~\cite{Jones2001} library (numerical integration of the
        replicator equation).
\end{itemize}

This work was performed using the computational facilities of the Advanced
Research Computing @ Cardiff (ARCCA) Division, Cardiff University.

\printbibliography

\newpage
\section*{Supplementary materials}

\includepdf{assets/pdf/proof_of_form_of_extortionate_strategies/main.pdf}

\newpage

Using the pair wise interactions the transition rates \(p,
q\) can be measured and the steady state probabilities inferred and compared to
the actual probabilities of each state.
This is done numerically by computing the singular eigenvector of the
matrix \(A\) \cite{Stewart2009}:

\[
    A =
    \begin{bmatrix}
        p_1 q_1 & p_1 (1 - q_1) & (1 - p_1) q_1 & (1 -p_1) (1 - q_1) \\
        p_2 q_2 & p_2 (1 - q_2) & (1 - p_2) q_2 & (1 -p_2) (1 - q_2) \\
        p_3 q_3 & p_3 (1 - q_3) & (1 - p_3) q_3 & (1 -p_3) (1 - q_3) \\
        p_4 q_4 & p_4 (1 - q_4) & (1 - p_4) q_4 & (1 -p_4) (1 - q_4) \\
    \end{bmatrix}
\]

Figure~\ref{fig:computed_probabilities_vs_theoretic_probabilities} shows a
regression line fitted to every pairwise interaction with a reported
\(\text{SSError}\) value (pairwise interactions with missing states were
omitted). This serves to validate the approach: a part from some edge cases the
relationship is consistent.

\begin{figure}[!htbp]
    \centering
    \includegraphics[width=.8\textwidth]{./assets/img/computed_probabilities_vs_theoretic_probabilities/main.pdf}
    \caption{The
        relationship between the steady state probabilities inferred from the
        measured transitions and the actual steady state probabilities. A linear
        regression line is included validating the approach.}
    \label{fig:computed_probabilities_vs_theoretic_probabilities}
\end{figure}


\end{document}
 turns and every match has been
repeated \documentclass[a4paper]{article}

\usepackage{amsmath}
\usepackage{amssymb}
\usepackage[margin=1.5cm,
            includefoot,
            footskip=30pt]{geometry}
\usepackage{layout}
\usepackage{graphicx}
\usepackage{subcaption}

\usepackage{biblatex}
\usepackage{pdfpages}

\bibliography{main.bib}

\title{Suspicion: Recognising and evaluating the effectiveness
       of extortion in the Iterated Prisoner's Dilemma}
\author{Vincent A. Knight \and Nikoleta E. Glynatsi}
\date{\today}



\begin{document}

\maketitle

\begin{abstract}
    The Iterated Prisoner's Dilemma is a model for rational and evolutionary
    interactive behaviour. It has applications both in the study of human social
    behaviour as well as in biology.
    It is used to understand when and how a rational individual might
    accept an immediate cost to their own utility for the direct benefit of
    another.

    Much attention has been given to a class of strategies called
    Zero Determinant strategies. It has been theoretically shown that these
    strategies can ``extort'' any player.

    In this work, an approach to identify if observed strategies are playing in
    an extortionate way is described. Furthermore, experimental analysis of
    a large tournament with \input{assets/tex/number_of_full_strategies/main.tex}
    strategies is considered. In this setting
    the most highly performing strategies do not play in an extortionate way
    against each other but do against lower performing strategies.
    This suggests that whilst the theory of Zero Determinant strategies
    indicates that memory is not of fundamental importance to the evolution of
    cooperative behaviour, this is incomplete.
\end{abstract}

\section{Introduction}\label{sec:introduction}

Agent based game theoretic models have become a stalwart of the underpinning
mathematics of interactive behaviours. One of the major pieces of work
in this area is the pair of original computer tournaments run by Robert
Axelrod~\cite{Axelrod1980, Axelrod1980a}. These tournaments pitted submitted
computer strategies against each other in plays of the Iterated Prisoner's
Dilemma. A common game where agents can choose to pay a slight cost to their
immediate utility in the hope of building a reputation. This has been used in
economic and evolutionary game theory to understand the evolution of cooperative
behaviour.

Recently, a class of strategies was described in~\cite{Press2012} that can
provably extort any given opponent. In~\cite{Hilbe2013, Moran1707} some
questions have already been asked about the true effectiveness of these
strategies in an evolutionary setting. Here another question is asked: is it
possible to recognise this extortionate behaviour? A mathematical procedure for
suspicion is presented: in the same way that the continued actions of an
extortionate individual might raise suspicion.

This work makes use of the Axelrod Python library~\cite{Knight2018, Knight2016}
with a large number of Prisoner Dilemma strategies available to give an
extensive numerical example of the ideas presented.  The approach is presented
in Section~\ref{sec:delta-zd-strategies}.  All of the code and data discussed
in Section~\ref{sec:numerical-experiments} is open sourced, archived and
written according to best scientific principles~\cite{Wilson2014}. The data
archive can be found at~\cite{vincent_knight_2018_1297075}.

\section{Recognising Extortion}\label{sec:delta-zd-strategies}

In~\cite{Press2012}, given a match between 2 memory-one strategies, the concept
of Zero Determinant (ZD) strategies is introduced. The main result of that paper
shows that given two memory one players \(p, q\in\mathbb{R}^4\) a linear
relationship between the players' scores could be forced by one of the players.

Using the notation of~\cite{Press2012}, assuming the utilities for player \(p\)
are given by \(S_x=(R, S, T, P)\) and for player \(q\) by \(S_y=(R, T, S, P)\)
and that the stationary scores of each player is given by \(S_X\) and \(S_Y\)
respectively. The main result of~\cite{Press2012} is that if

\begin{equation}\label{eqn:linear_relationship_for_p}
    \tilde p=\alpha S_x + \beta S_y + \gamma
\end{equation}

or

\begin{equation}\label{eqn:linear_relationship_for_q}
    \tilde q=\alpha S_x + \beta S_y + \gamma
\end{equation}

where \(\tilde p = (1 - p_1, 1 - p_2, p_3, p_4)\) and
\(\tilde q = (1 - q_1, 1 - q_2, q_3, q_4)\) then:

\begin{equation}
    \alpha S_X + \beta S_Y + \gamma = 0
\end{equation}

In~\cite{Press2012} a particular type of ZD strategy is defined: extortionate
strategies. If:

\begin{equation}\label{eqn:constraint_for_extortion}
    \gamma = - P(\alpha + \beta)
\end{equation}

then the player can ensure they get a score \(\chi\) times
larger than the opponent. This extortion coefficient is given by:

\begin{equation}\label{eqn:definition_of_chi}
    \chi=\frac{-\beta}{\alpha}
\end{equation}

Thus, if (\ref{eqn:constraint_for_extortion}) holds and \(\chi >1\) a player is
said to extort their opponent.
Here, the reverse problem is considered: given a
\(p\in\mathbb{R}^4\) how does one identify \(\alpha, \beta\) if they
exist and is the strategy in fact acting in an extortionate way?

These conditions correspond to:

\begin{align}
    \tilde p_1 & = \alpha R + \beta R - P (\alpha + \beta)
            \label{eqn:condition_for_tilde_p1}\\
    \tilde p_2 & = \alpha S + \beta T - P (\alpha + \beta)
            \label{eqn:condition_for_tilde_p2}\\
    \tilde p_3 & = \alpha T + \beta S - P (\alpha + \beta)
            \label{eqn:condition_for_tilde_p3}\\
    \tilde p_4 & = \alpha P + \beta P - P (\alpha + \beta)
            \label{eqn:condition_for_tilde_p4}
\end{align}

Equation (\ref{eqn:condition_for_tilde_p4}) ensures that \(p_4=\tilde p_4=0\).
Equations (\ref{eqn:condition_for_tilde_p1}-\ref{eqn:condition_for_tilde_p3})
can be used to eliminate \(\alpha, \beta\), giving:

\begin{equation}\label{eqn:planar_definition_of_extortion}
    \tilde p_1 = \frac{(R - P)(\tilde p_2 + \tilde p_3)}{S + T - 2P}
\end{equation}

with:

\begin{equation}\label{eqn:definition_of_chi}
    \chi = \frac{\tilde p_2 (P - T) + \tilde p_3 (S - P)}
                {\tilde p_2 (P - S) + \tilde p_3 (T - P)}
\end{equation}

Given a strategy \(p\in\mathbb{R}^{4\times 1}\) equations
(\ref{eqn:condition_for_tilde_p4}), (\ref{eqn:planar_definition_of_extortion}-\ref{eqn:definition_of_chi}) can be used to check if
a strategy is extortionate. The conditions correspond to:

\begin{align}
    p_1 & = \frac{(R-P)(p_2 + p_3) - R + T + S - P}{S + T - 2P}
     \label{eqn:condition_for_p1}\\
    p_4 & = 0 \label{eqn:condition_for_p4}\\
    1 & > p_2 + p_3\label{eqn:condition_for_chi}
\end{align}

The algebraic steps necessary to prove these results are available in the
supporting materials.

All extortionate strategies reside on a triangular (\ref{eqn:condition_for_chi})
plane (\ref{eqn:condition_for_p1}) in 3 dimensions (\ref{eqn:condition_for_p4}).
Using this formulation it can be seen that a necessary (but not sufficient)
condition for an extortionate strategy is that it cooperates on average less
than 50\% of the time when in a state of disagreement with the opponent.

As an example, consider the known extortionate strategy \(p=(8 / 9, 1 / 2, 1 /
3, 0)\) from~\cite{Stewart2012} which is referred to as \texttt{Extort-2}. In
this case, for the standard values of \((R, T, S, P)\) constraint
(\ref{eqn:condition_for_p1}) corresponds to:

\begin{equation}
    p_1 = \frac{2(p_2 + p_3) + 1}{3}
\end{equation}

It is clear that in this case all constraints hold.

This approach could in fact be used to confirm that a given strategy is acting
in an extortionate manner even if it is not a memory one strategy. However, in
practice, if a closed form for \(p\) is not known, then due to measurement
and/or numerical error this would not work.

This problem can be written in the following linear algebraic form where
\(x=(\alpha, \beta)\)
and \(p^*=(\tilde p_1 - 1, tilde_2 - 1, p_3)\):

\begin{equation}\label{eqn:linear_algebraic_equation_for_p}
    Cx= p^*
\end{equation}

\(C\) corresponds to equations
(\ref{eqn:condition_for_tilde_p1}-\ref{eqn:condition_for_tilde_p3}) and is
given by:

\begin{equation}\label{eqn:definition_of_C}
    C =
    \begin{bmatrix}
        R - P & R- P \\
        S - P & T- P \\
        T - P & S- P \\
    \end{bmatrix}
\end{equation}

Note that in general, equation (\ref{eqn:linear_algebraic_equation_for_p}) will
not necessarily have a solution. From the Rouch\'{e}-Capelli theorem if there is
a solution it is unique as \(\text{rank}(C)=2\) which is the dimension of the
variable \(x\). The best fitting \(x\) is found by minimizing:

\begin{equation}\label{eqn:r_squared}
    \text{SSError} = \|C x- p^*\|_2^2 = \sum_{i=1}^{3}\left((C\bar x)_i-p_i^*\right)^2
\end{equation}

Note that \(\text{SSError}\), which is the square of the Frobenius
norm~\cite{Golub2013}, becomes a measure of how close a strategy is to being an
extortionate strategy. Suspicion
of extortion then corresponds to a threshold on \(\text{SSError}\).

By observing interactions (human or otherwise), their memory one representation
can be inferred and this approach can be used to recognise extortionate
behaviour. The notion of comparing theoretic and actual plays of the IPD is not
novel, see for example~\cite{Rand2013}. Immediately it is noted that if the
environment is noisy~\cite{Wu1995} then no strategy can be considered to be
extortionate as \(p_4>0\).

In the next section, this idea will be illustrated by observing the interactions
that take place in a computer based tournament of the IPD\@.

\section{Numerical experiments}\label{sec:numerical-experiments}

In~\cite{Stewart2012} results from a tournament with
\input{./assets/tex/number_of_stewart_plotkin_strategies/main.tex} strategies,
was presented with specific consideration given to ZD strategies. This
tournament is reproduced here using the Axelrod-Python
project~\cite{Knight2016}. To obtain a good measure of the corresponding
transition rates for each strategy all matches have been run for
\input{assets/tex/number_of_turns/main.tex} turns and every match has been
repeated \input{assets/tex/number_of_repetitions/main.tex} times. All of this
interaction data is available at~\cite{vincent_knight_2018_1297075}. A good
match between the inferred Markov chain and the state distribution of the actual
interactions has been verified. Data for this is presented in the supplementary
materials.

Figure~\ref{fig:SSError_overall_in_stewart_plotkin} shows the \(\text{SSError}\)
values for all the strategies in the tournament, as reported
in~\cite{Stewart2012} the extortionate strategy (which has an expected
\(\text{SSError}\) approximately 0) gains a large number of wins.

\begin{figure}[!htbp]
    \centering
    \includegraphics[width=.8\textwidth]{./assets/img/SSError_overall_in_stewart_plotkin/main.pdf}
    \caption{\(\text{SSError}\) and state probabilities for the strategies
        of~\cite{Stewart2012}, ordered both by number of wins and overall score.
        Note that \(P(DC)\) is not shown as it corresponds to the transpose of
        \(P(CD)\). Cooperator and Defector are omitted as they do not visit all
        the states.}
    \label{fig:SSError_overall_in_stewart_plotkin}
\end{figure}

Here, the work of~\cite{Stewart2012} is extended by investigating a tournament
with \input{assets/tex/number_of_full_strategies/main.tex}
strategies.

The results of this analysis are shown in
Figure~\ref{fig:SSError_and_probabilities_in_full}. The top ranking strategies
by number of wins seem to be extortionate (but not against all strategies) and
it can be seen that a small sub group of strategies achieve mutual defection.
All the top ranking strategies according to score achieve mutual cooperation and
do not extort each other, however they
\textbf{do} exhibit extortionate behaviour towards a number of the lower ranking
strategies.

\begin{figure}[!htbp]
    \centering
    \includegraphics[width=.8\textwidth]{./assets/img/SSError_and_probabilities_in_full/main.pdf}
    \caption{\(\text{SSError}\) for the strategies for the full tournament. Only
    strategy interactions for which \(p_4=0\) and \(\chi>1\) are displayed.}
    \label{fig:SSError_and_probabilities_in_full}
\end{figure}

\section{Conclusion}\label{sec:conclusion}

This work defines an approach to measure whether or not a player is playing a
strategy that corresponds to an extortionate strategy as defined
in~\cite{Press2012}: a mathematical model for suspicion. Indeed, all
extortionate strategies have been
 classified as lying on a triangular plane.
This rigorous classification fails to be robust to small measurement error, thus
a statistical approach is proposed.
This is done through a linear algebraic approach for approximating the solution
of a linear system. Using this, a large number of pairwise interactions is
simulated and in fact very few strategies are found to act extortionately.

The work of~\cite{Press2012}, whilst showing that a clever approach to taking
advantage of another memory one strategy exists: this is incomplete. Whilst the
elegance of this result is very attractive, just as the simplicity of the
victory of Tit For Tat in Axelrod's original tournaments was, it is incomplete.
Extortionate strategies achieve a high number of wins but they do not
achieve a high score which corresponds to the fitness landscape in an
evolutionary sense. From the large number of interactions a payoff matrix \(S\)
can be measured where \(S_{ij}\) denotes the score (using standard values of
\((R, S, T, P) = (3, 0, 5, 1)\)) of the \(i\)th strategy
against the \(j\)th strategy. Using this, the replicator equation
describes the evolution of the system based on a population density fitness
function:

\begin{equation}\label{eqn:replicator_dynamics}
    \frac{dx}{dt} = x(S-x^TS x)
\end{equation}

Equation (\ref{eqn:replicator_dynamics}) is solved numerically through an
integration technique described in~\cite{Petzold1983} and
Figure~\ref{fig:replicator_dynamics} shows the evolution of the distribution of
the system: the various strategies are ranked by scores. It is clear to see that
only the high ranking strategies survive the evolutionary process (in fact,
only \input{./assets/img/replicator_dynamics/main.tex}
have a final distribution greater than \(10 ^ {-2}\)). This confirms the
findings of~\cite{Moran1707} in which sophisticated strategies resist
evolutionary invasion of shorter memory strategies. Recalling
Figure~\ref{fig:SSError_and_probabilities_in_full} this demonstrates that:

\begin{itemize}
    \item Cooperation emerges through the evolutionary process: the high scoring
        strategies do not exhibit extortionate behaviour towards each other.
    \item Extortionate strategies do not survive the evolutionary process.
\end{itemize}

\begin{figure}[!htbp]
    \centering
    \includegraphics[width=.8\textwidth]{./assets/img/replicator_dynamics/main.pdf}
    \caption{Numerical simulation of the replicator equation
    (\ref{eqn:replicator_dynamics}): strategies are ordered by score, only the strategies with a high score survive the evolutionary process.}
    \label{fig:replicator_dynamics}
\end{figure}

This work can be used to classify plays of the IPD\@: data can be collected from
actual interactions (in lab or in the field). Furthermore, this allows for a
classification method similar to the notion of fingerprinting presented
in~\cite{Ashlock2008}. Trained strategies can potentially be classified as
extortionate or not or it could be possible to even constrain the reinforcement
learning approaches that are becoming prevalent in the literature.
Alternatively, this mathematical approach for recognising extortion could be
used in sophisticated strategies to defend against invasion. Arguably, some of
the strategies considered here exhibit this behaviour, indeed as described
in~\cite{Harper2017}, the top ranking strategies in the full tournament are
obtained using evolutionary reinforcement learning techniques, thus, suspicion
of extortionate behaviour could in fact be an evolutionary trait.

\section*{Acknowledgements}

The following open source software libraries were used in this research:

\begin{itemize}
    \item The Axelrod ~\cite{Knight2016, Knight2018} library (IPD strategies and
        tournaments).
    \item The sympy library~\cite{Meurer2017} (verification of all symbolic
        calculations).
    \item The matplotlib~\cite{Droettboom2018} library (visualisation).
    \item The pandas~\cite{Structures2010}, dask~\cite{Dask2016} and
        NumPy~\cite{Oliphant2015} libraries (data manipulation).
    \item The SciPy~\cite{Jones2001} library (numerical integration of the
        replicator equation).
\end{itemize}

This work was performed using the computational facilities of the Advanced
Research Computing @ Cardiff (ARCCA) Division, Cardiff University.

\printbibliography

\newpage
\section*{Supplementary materials}

\includepdf{assets/pdf/proof_of_form_of_extortionate_strategies/main.pdf}

\newpage

Using the pair wise interactions the transition rates \(p,
q\) can be measured and the steady state probabilities inferred and compared to
the actual probabilities of each state.
This is done numerically by computing the singular eigenvector of the
matrix \(A\) \cite{Stewart2009}:

\[
    A =
    \begin{bmatrix}
        p_1 q_1 & p_1 (1 - q_1) & (1 - p_1) q_1 & (1 -p_1) (1 - q_1) \\
        p_2 q_2 & p_2 (1 - q_2) & (1 - p_2) q_2 & (1 -p_2) (1 - q_2) \\
        p_3 q_3 & p_3 (1 - q_3) & (1 - p_3) q_3 & (1 -p_3) (1 - q_3) \\
        p_4 q_4 & p_4 (1 - q_4) & (1 - p_4) q_4 & (1 -p_4) (1 - q_4) \\
    \end{bmatrix}
\]

Figure~\ref{fig:computed_probabilities_vs_theoretic_probabilities} shows a
regression line fitted to every pairwise interaction with a reported
\(\text{SSError}\) value (pairwise interactions with missing states were
omitted). This serves to validate the approach: a part from some edge cases the
relationship is consistent.

\begin{figure}[!htbp]
    \centering
    \includegraphics[width=.8\textwidth]{./assets/img/computed_probabilities_vs_theoretic_probabilities/main.pdf}
    \caption{The
        relationship between the steady state probabilities inferred from the
        measured transitions and the actual steady state probabilities. A linear
        regression line is included validating the approach.}
    \label{fig:computed_probabilities_vs_theoretic_probabilities}
\end{figure}


\end{document}
 times. All of this
interaction data is available at~\cite{vincent_knight_2018_1297075}. A good
match between the inferred Markov chain and the state distribution of the actual
interactions has been verified. Data for this is presented in the supplementary
materials.

Figure~\ref{fig:SSError_overall_in_stewart_plotkin} shows the \(\text{SSError}\)
values for all the strategies in the tournament, as reported
in~\cite{Stewart2012} the extortionate strategy (which has an expected
\(\text{SSError}\) approximately 0) gains a large number of wins.

\begin{figure}[!htbp]
    \centering
    \includegraphics[width=.8\textwidth]{./assets/img/SSError_overall_in_stewart_plotkin/main.pdf}
    \caption{\(\text{SSError}\) and state probabilities for the strategies
        of~\cite{Stewart2012}, ordered both by number of wins and overall score.
        Note that \(P(DC)\) is not shown as it corresponds to the transpose of
        \(P(CD)\). Cooperator and Defector are omitted as they do not visit all
        the states.}
    \label{fig:SSError_overall_in_stewart_plotkin}
\end{figure}

Here, the work of~\cite{Stewart2012} is extended by investigating a tournament
with \documentclass[a4paper]{article}

\usepackage{amsmath}
\usepackage{amssymb}
\usepackage[margin=1.5cm,
            includefoot,
            footskip=30pt]{geometry}
\usepackage{layout}
\usepackage{graphicx}
\usepackage{subcaption}

\usepackage{biblatex}
\usepackage{pdfpages}

\bibliography{main.bib}

\title{Suspicion: Recognising and evaluating the effectiveness
       of extortion in the Iterated Prisoner's Dilemma}
\author{Vincent A. Knight \and Nikoleta E. Glynatsi}
\date{\today}



\begin{document}

\maketitle

\begin{abstract}
    The Iterated Prisoner's Dilemma is a model for rational and evolutionary
    interactive behaviour. It has applications both in the study of human social
    behaviour as well as in biology.
    It is used to understand when and how a rational individual might
    accept an immediate cost to their own utility for the direct benefit of
    another.

    Much attention has been given to a class of strategies called
    Zero Determinant strategies. It has been theoretically shown that these
    strategies can ``extort'' any player.

    In this work, an approach to identify if observed strategies are playing in
    an extortionate way is described. Furthermore, experimental analysis of
    a large tournament with \input{assets/tex/number_of_full_strategies/main.tex}
    strategies is considered. In this setting
    the most highly performing strategies do not play in an extortionate way
    against each other but do against lower performing strategies.
    This suggests that whilst the theory of Zero Determinant strategies
    indicates that memory is not of fundamental importance to the evolution of
    cooperative behaviour, this is incomplete.
\end{abstract}

\section{Introduction}\label{sec:introduction}

Agent based game theoretic models have become a stalwart of the underpinning
mathematics of interactive behaviours. One of the major pieces of work
in this area is the pair of original computer tournaments run by Robert
Axelrod~\cite{Axelrod1980, Axelrod1980a}. These tournaments pitted submitted
computer strategies against each other in plays of the Iterated Prisoner's
Dilemma. A common game where agents can choose to pay a slight cost to their
immediate utility in the hope of building a reputation. This has been used in
economic and evolutionary game theory to understand the evolution of cooperative
behaviour.

Recently, a class of strategies was described in~\cite{Press2012} that can
provably extort any given opponent. In~\cite{Hilbe2013, Moran1707} some
questions have already been asked about the true effectiveness of these
strategies in an evolutionary setting. Here another question is asked: is it
possible to recognise this extortionate behaviour? A mathematical procedure for
suspicion is presented: in the same way that the continued actions of an
extortionate individual might raise suspicion.

This work makes use of the Axelrod Python library~\cite{Knight2018, Knight2016}
with a large number of Prisoner Dilemma strategies available to give an
extensive numerical example of the ideas presented.  The approach is presented
in Section~\ref{sec:delta-zd-strategies}.  All of the code and data discussed
in Section~\ref{sec:numerical-experiments} is open sourced, archived and
written according to best scientific principles~\cite{Wilson2014}. The data
archive can be found at~\cite{vincent_knight_2018_1297075}.

\section{Recognising Extortion}\label{sec:delta-zd-strategies}

In~\cite{Press2012}, given a match between 2 memory-one strategies, the concept
of Zero Determinant (ZD) strategies is introduced. The main result of that paper
shows that given two memory one players \(p, q\in\mathbb{R}^4\) a linear
relationship between the players' scores could be forced by one of the players.

Using the notation of~\cite{Press2012}, assuming the utilities for player \(p\)
are given by \(S_x=(R, S, T, P)\) and for player \(q\) by \(S_y=(R, T, S, P)\)
and that the stationary scores of each player is given by \(S_X\) and \(S_Y\)
respectively. The main result of~\cite{Press2012} is that if

\begin{equation}\label{eqn:linear_relationship_for_p}
    \tilde p=\alpha S_x + \beta S_y + \gamma
\end{equation}

or

\begin{equation}\label{eqn:linear_relationship_for_q}
    \tilde q=\alpha S_x + \beta S_y + \gamma
\end{equation}

where \(\tilde p = (1 - p_1, 1 - p_2, p_3, p_4)\) and
\(\tilde q = (1 - q_1, 1 - q_2, q_3, q_4)\) then:

\begin{equation}
    \alpha S_X + \beta S_Y + \gamma = 0
\end{equation}

In~\cite{Press2012} a particular type of ZD strategy is defined: extortionate
strategies. If:

\begin{equation}\label{eqn:constraint_for_extortion}
    \gamma = - P(\alpha + \beta)
\end{equation}

then the player can ensure they get a score \(\chi\) times
larger than the opponent. This extortion coefficient is given by:

\begin{equation}\label{eqn:definition_of_chi}
    \chi=\frac{-\beta}{\alpha}
\end{equation}

Thus, if (\ref{eqn:constraint_for_extortion}) holds and \(\chi >1\) a player is
said to extort their opponent.
Here, the reverse problem is considered: given a
\(p\in\mathbb{R}^4\) how does one identify \(\alpha, \beta\) if they
exist and is the strategy in fact acting in an extortionate way?

These conditions correspond to:

\begin{align}
    \tilde p_1 & = \alpha R + \beta R - P (\alpha + \beta)
            \label{eqn:condition_for_tilde_p1}\\
    \tilde p_2 & = \alpha S + \beta T - P (\alpha + \beta)
            \label{eqn:condition_for_tilde_p2}\\
    \tilde p_3 & = \alpha T + \beta S - P (\alpha + \beta)
            \label{eqn:condition_for_tilde_p3}\\
    \tilde p_4 & = \alpha P + \beta P - P (\alpha + \beta)
            \label{eqn:condition_for_tilde_p4}
\end{align}

Equation (\ref{eqn:condition_for_tilde_p4}) ensures that \(p_4=\tilde p_4=0\).
Equations (\ref{eqn:condition_for_tilde_p1}-\ref{eqn:condition_for_tilde_p3})
can be used to eliminate \(\alpha, \beta\), giving:

\begin{equation}\label{eqn:planar_definition_of_extortion}
    \tilde p_1 = \frac{(R - P)(\tilde p_2 + \tilde p_3)}{S + T - 2P}
\end{equation}

with:

\begin{equation}\label{eqn:definition_of_chi}
    \chi = \frac{\tilde p_2 (P - T) + \tilde p_3 (S - P)}
                {\tilde p_2 (P - S) + \tilde p_3 (T - P)}
\end{equation}

Given a strategy \(p\in\mathbb{R}^{4\times 1}\) equations
(\ref{eqn:condition_for_tilde_p4}), (\ref{eqn:planar_definition_of_extortion}-\ref{eqn:definition_of_chi}) can be used to check if
a strategy is extortionate. The conditions correspond to:

\begin{align}
    p_1 & = \frac{(R-P)(p_2 + p_3) - R + T + S - P}{S + T - 2P}
     \label{eqn:condition_for_p1}\\
    p_4 & = 0 \label{eqn:condition_for_p4}\\
    1 & > p_2 + p_3\label{eqn:condition_for_chi}
\end{align}

The algebraic steps necessary to prove these results are available in the
supporting materials.

All extortionate strategies reside on a triangular (\ref{eqn:condition_for_chi})
plane (\ref{eqn:condition_for_p1}) in 3 dimensions (\ref{eqn:condition_for_p4}).
Using this formulation it can be seen that a necessary (but not sufficient)
condition for an extortionate strategy is that it cooperates on average less
than 50\% of the time when in a state of disagreement with the opponent.

As an example, consider the known extortionate strategy \(p=(8 / 9, 1 / 2, 1 /
3, 0)\) from~\cite{Stewart2012} which is referred to as \texttt{Extort-2}. In
this case, for the standard values of \((R, T, S, P)\) constraint
(\ref{eqn:condition_for_p1}) corresponds to:

\begin{equation}
    p_1 = \frac{2(p_2 + p_3) + 1}{3}
\end{equation}

It is clear that in this case all constraints hold.

This approach could in fact be used to confirm that a given strategy is acting
in an extortionate manner even if it is not a memory one strategy. However, in
practice, if a closed form for \(p\) is not known, then due to measurement
and/or numerical error this would not work.

This problem can be written in the following linear algebraic form where
\(x=(\alpha, \beta)\)
and \(p^*=(\tilde p_1 - 1, tilde_2 - 1, p_3)\):

\begin{equation}\label{eqn:linear_algebraic_equation_for_p}
    Cx= p^*
\end{equation}

\(C\) corresponds to equations
(\ref{eqn:condition_for_tilde_p1}-\ref{eqn:condition_for_tilde_p3}) and is
given by:

\begin{equation}\label{eqn:definition_of_C}
    C =
    \begin{bmatrix}
        R - P & R- P \\
        S - P & T- P \\
        T - P & S- P \\
    \end{bmatrix}
\end{equation}

Note that in general, equation (\ref{eqn:linear_algebraic_equation_for_p}) will
not necessarily have a solution. From the Rouch\'{e}-Capelli theorem if there is
a solution it is unique as \(\text{rank}(C)=2\) which is the dimension of the
variable \(x\). The best fitting \(x\) is found by minimizing:

\begin{equation}\label{eqn:r_squared}
    \text{SSError} = \|C x- p^*\|_2^2 = \sum_{i=1}^{3}\left((C\bar x)_i-p_i^*\right)^2
\end{equation}

Note that \(\text{SSError}\), which is the square of the Frobenius
norm~\cite{Golub2013}, becomes a measure of how close a strategy is to being an
extortionate strategy. Suspicion
of extortion then corresponds to a threshold on \(\text{SSError}\).

By observing interactions (human or otherwise), their memory one representation
can be inferred and this approach can be used to recognise extortionate
behaviour. The notion of comparing theoretic and actual plays of the IPD is not
novel, see for example~\cite{Rand2013}. Immediately it is noted that if the
environment is noisy~\cite{Wu1995} then no strategy can be considered to be
extortionate as \(p_4>0\).

In the next section, this idea will be illustrated by observing the interactions
that take place in a computer based tournament of the IPD\@.

\section{Numerical experiments}\label{sec:numerical-experiments}

In~\cite{Stewart2012} results from a tournament with
\input{./assets/tex/number_of_stewart_plotkin_strategies/main.tex} strategies,
was presented with specific consideration given to ZD strategies. This
tournament is reproduced here using the Axelrod-Python
project~\cite{Knight2016}. To obtain a good measure of the corresponding
transition rates for each strategy all matches have been run for
\input{assets/tex/number_of_turns/main.tex} turns and every match has been
repeated \input{assets/tex/number_of_repetitions/main.tex} times. All of this
interaction data is available at~\cite{vincent_knight_2018_1297075}. A good
match between the inferred Markov chain and the state distribution of the actual
interactions has been verified. Data for this is presented in the supplementary
materials.

Figure~\ref{fig:SSError_overall_in_stewart_plotkin} shows the \(\text{SSError}\)
values for all the strategies in the tournament, as reported
in~\cite{Stewart2012} the extortionate strategy (which has an expected
\(\text{SSError}\) approximately 0) gains a large number of wins.

\begin{figure}[!htbp]
    \centering
    \includegraphics[width=.8\textwidth]{./assets/img/SSError_overall_in_stewart_plotkin/main.pdf}
    \caption{\(\text{SSError}\) and state probabilities for the strategies
        of~\cite{Stewart2012}, ordered both by number of wins and overall score.
        Note that \(P(DC)\) is not shown as it corresponds to the transpose of
        \(P(CD)\). Cooperator and Defector are omitted as they do not visit all
        the states.}
    \label{fig:SSError_overall_in_stewart_plotkin}
\end{figure}

Here, the work of~\cite{Stewart2012} is extended by investigating a tournament
with \input{assets/tex/number_of_full_strategies/main.tex}
strategies.

The results of this analysis are shown in
Figure~\ref{fig:SSError_and_probabilities_in_full}. The top ranking strategies
by number of wins seem to be extortionate (but not against all strategies) and
it can be seen that a small sub group of strategies achieve mutual defection.
All the top ranking strategies according to score achieve mutual cooperation and
do not extort each other, however they
\textbf{do} exhibit extortionate behaviour towards a number of the lower ranking
strategies.

\begin{figure}[!htbp]
    \centering
    \includegraphics[width=.8\textwidth]{./assets/img/SSError_and_probabilities_in_full/main.pdf}
    \caption{\(\text{SSError}\) for the strategies for the full tournament. Only
    strategy interactions for which \(p_4=0\) and \(\chi>1\) are displayed.}
    \label{fig:SSError_and_probabilities_in_full}
\end{figure}

\section{Conclusion}\label{sec:conclusion}

This work defines an approach to measure whether or not a player is playing a
strategy that corresponds to an extortionate strategy as defined
in~\cite{Press2012}: a mathematical model for suspicion. Indeed, all
extortionate strategies have been
 classified as lying on a triangular plane.
This rigorous classification fails to be robust to small measurement error, thus
a statistical approach is proposed.
This is done through a linear algebraic approach for approximating the solution
of a linear system. Using this, a large number of pairwise interactions is
simulated and in fact very few strategies are found to act extortionately.

The work of~\cite{Press2012}, whilst showing that a clever approach to taking
advantage of another memory one strategy exists: this is incomplete. Whilst the
elegance of this result is very attractive, just as the simplicity of the
victory of Tit For Tat in Axelrod's original tournaments was, it is incomplete.
Extortionate strategies achieve a high number of wins but they do not
achieve a high score which corresponds to the fitness landscape in an
evolutionary sense. From the large number of interactions a payoff matrix \(S\)
can be measured where \(S_{ij}\) denotes the score (using standard values of
\((R, S, T, P) = (3, 0, 5, 1)\)) of the \(i\)th strategy
against the \(j\)th strategy. Using this, the replicator equation
describes the evolution of the system based on a population density fitness
function:

\begin{equation}\label{eqn:replicator_dynamics}
    \frac{dx}{dt} = x(S-x^TS x)
\end{equation}

Equation (\ref{eqn:replicator_dynamics}) is solved numerically through an
integration technique described in~\cite{Petzold1983} and
Figure~\ref{fig:replicator_dynamics} shows the evolution of the distribution of
the system: the various strategies are ranked by scores. It is clear to see that
only the high ranking strategies survive the evolutionary process (in fact,
only \input{./assets/img/replicator_dynamics/main.tex}
have a final distribution greater than \(10 ^ {-2}\)). This confirms the
findings of~\cite{Moran1707} in which sophisticated strategies resist
evolutionary invasion of shorter memory strategies. Recalling
Figure~\ref{fig:SSError_and_probabilities_in_full} this demonstrates that:

\begin{itemize}
    \item Cooperation emerges through the evolutionary process: the high scoring
        strategies do not exhibit extortionate behaviour towards each other.
    \item Extortionate strategies do not survive the evolutionary process.
\end{itemize}

\begin{figure}[!htbp]
    \centering
    \includegraphics[width=.8\textwidth]{./assets/img/replicator_dynamics/main.pdf}
    \caption{Numerical simulation of the replicator equation
    (\ref{eqn:replicator_dynamics}): strategies are ordered by score, only the strategies with a high score survive the evolutionary process.}
    \label{fig:replicator_dynamics}
\end{figure}

This work can be used to classify plays of the IPD\@: data can be collected from
actual interactions (in lab or in the field). Furthermore, this allows for a
classification method similar to the notion of fingerprinting presented
in~\cite{Ashlock2008}. Trained strategies can potentially be classified as
extortionate or not or it could be possible to even constrain the reinforcement
learning approaches that are becoming prevalent in the literature.
Alternatively, this mathematical approach for recognising extortion could be
used in sophisticated strategies to defend against invasion. Arguably, some of
the strategies considered here exhibit this behaviour, indeed as described
in~\cite{Harper2017}, the top ranking strategies in the full tournament are
obtained using evolutionary reinforcement learning techniques, thus, suspicion
of extortionate behaviour could in fact be an evolutionary trait.

\section*{Acknowledgements}

The following open source software libraries were used in this research:

\begin{itemize}
    \item The Axelrod ~\cite{Knight2016, Knight2018} library (IPD strategies and
        tournaments).
    \item The sympy library~\cite{Meurer2017} (verification of all symbolic
        calculations).
    \item The matplotlib~\cite{Droettboom2018} library (visualisation).
    \item The pandas~\cite{Structures2010}, dask~\cite{Dask2016} and
        NumPy~\cite{Oliphant2015} libraries (data manipulation).
    \item The SciPy~\cite{Jones2001} library (numerical integration of the
        replicator equation).
\end{itemize}

This work was performed using the computational facilities of the Advanced
Research Computing @ Cardiff (ARCCA) Division, Cardiff University.

\printbibliography

\newpage
\section*{Supplementary materials}

\includepdf{assets/pdf/proof_of_form_of_extortionate_strategies/main.pdf}

\newpage

Using the pair wise interactions the transition rates \(p,
q\) can be measured and the steady state probabilities inferred and compared to
the actual probabilities of each state.
This is done numerically by computing the singular eigenvector of the
matrix \(A\) \cite{Stewart2009}:

\[
    A =
    \begin{bmatrix}
        p_1 q_1 & p_1 (1 - q_1) & (1 - p_1) q_1 & (1 -p_1) (1 - q_1) \\
        p_2 q_2 & p_2 (1 - q_2) & (1 - p_2) q_2 & (1 -p_2) (1 - q_2) \\
        p_3 q_3 & p_3 (1 - q_3) & (1 - p_3) q_3 & (1 -p_3) (1 - q_3) \\
        p_4 q_4 & p_4 (1 - q_4) & (1 - p_4) q_4 & (1 -p_4) (1 - q_4) \\
    \end{bmatrix}
\]

Figure~\ref{fig:computed_probabilities_vs_theoretic_probabilities} shows a
regression line fitted to every pairwise interaction with a reported
\(\text{SSError}\) value (pairwise interactions with missing states were
omitted). This serves to validate the approach: a part from some edge cases the
relationship is consistent.

\begin{figure}[!htbp]
    \centering
    \includegraphics[width=.8\textwidth]{./assets/img/computed_probabilities_vs_theoretic_probabilities/main.pdf}
    \caption{The
        relationship between the steady state probabilities inferred from the
        measured transitions and the actual steady state probabilities. A linear
        regression line is included validating the approach.}
    \label{fig:computed_probabilities_vs_theoretic_probabilities}
\end{figure}


\end{document}

strategies.

The results of this analysis are shown in
Figure~\ref{fig:SSError_and_probabilities_in_full}. The top ranking strategies
by number of wins seem to be extortionate (but not against all strategies) and
it can be seen that a small sub group of strategies achieve mutual defection.
All the top ranking strategies according to score achieve mutual cooperation and
do not extort each other, however they
\textbf{do} exhibit extortionate behaviour towards a number of the lower ranking
strategies.

\begin{figure}[!htbp]
    \centering
    \includegraphics[width=.8\textwidth]{./assets/img/SSError_and_probabilities_in_full/main.pdf}
    \caption{\(\text{SSError}\) for the strategies for the full tournament. Only
    strategy interactions for which \(p_4=0\) and \(\chi>1\) are displayed.}
    \label{fig:SSError_and_probabilities_in_full}
\end{figure}

\section{Conclusion}\label{sec:conclusion}

This work defines an approach to measure whether or not a player is playing a
strategy that corresponds to an extortionate strategy as defined
in~\cite{Press2012}: a mathematical model for suspicion. Indeed, all
extortionate strategies have been
 classified as lying on a triangular plane.
This rigorous classification fails to be robust to small measurement error, thus
a statistical approach is proposed.
This is done through a linear algebraic approach for approximating the solution
of a linear system. Using this, a large number of pairwise interactions is
simulated and in fact very few strategies are found to act extortionately.

The work of~\cite{Press2012}, whilst showing that a clever approach to taking
advantage of another memory one strategy exists: this is incomplete. Whilst the
elegance of this result is very attractive, just as the simplicity of the
victory of Tit For Tat in Axelrod's original tournaments was, it is incomplete.
Extortionate strategies achieve a high number of wins but they do not
achieve a high score which corresponds to the fitness landscape in an
evolutionary sense. From the large number of interactions a payoff matrix \(S\)
can be measured where \(S_{ij}\) denotes the score (using standard values of
\((R, S, T, P) = (3, 0, 5, 1)\)) of the \(i\)th strategy
against the \(j\)th strategy. Using this, the replicator equation
describes the evolution of the system based on a population density fitness
function:

\begin{equation}\label{eqn:replicator_dynamics}
    \frac{dx}{dt} = x(S-x^TS x)
\end{equation}

Equation (\ref{eqn:replicator_dynamics}) is solved numerically through an
integration technique described in~\cite{Petzold1983} and
Figure~\ref{fig:replicator_dynamics} shows the evolution of the distribution of
the system: the various strategies are ranked by scores. It is clear to see that
only the high ranking strategies survive the evolutionary process (in fact,
only \documentclass[a4paper]{article}

\usepackage{amsmath}
\usepackage{amssymb}
\usepackage[margin=1.5cm,
            includefoot,
            footskip=30pt]{geometry}
\usepackage{layout}
\usepackage{graphicx}
\usepackage{subcaption}

\usepackage{biblatex}
\usepackage{pdfpages}

\bibliography{main.bib}

\title{Suspicion: Recognising and evaluating the effectiveness
       of extortion in the Iterated Prisoner's Dilemma}
\author{Vincent A. Knight \and Nikoleta E. Glynatsi}
\date{\today}



\begin{document}

\maketitle

\begin{abstract}
    The Iterated Prisoner's Dilemma is a model for rational and evolutionary
    interactive behaviour. It has applications both in the study of human social
    behaviour as well as in biology.
    It is used to understand when and how a rational individual might
    accept an immediate cost to their own utility for the direct benefit of
    another.

    Much attention has been given to a class of strategies called
    Zero Determinant strategies. It has been theoretically shown that these
    strategies can ``extort'' any player.

    In this work, an approach to identify if observed strategies are playing in
    an extortionate way is described. Furthermore, experimental analysis of
    a large tournament with \input{assets/tex/number_of_full_strategies/main.tex}
    strategies is considered. In this setting
    the most highly performing strategies do not play in an extortionate way
    against each other but do against lower performing strategies.
    This suggests that whilst the theory of Zero Determinant strategies
    indicates that memory is not of fundamental importance to the evolution of
    cooperative behaviour, this is incomplete.
\end{abstract}

\section{Introduction}\label{sec:introduction}

Agent based game theoretic models have become a stalwart of the underpinning
mathematics of interactive behaviours. One of the major pieces of work
in this area is the pair of original computer tournaments run by Robert
Axelrod~\cite{Axelrod1980, Axelrod1980a}. These tournaments pitted submitted
computer strategies against each other in plays of the Iterated Prisoner's
Dilemma. A common game where agents can choose to pay a slight cost to their
immediate utility in the hope of building a reputation. This has been used in
economic and evolutionary game theory to understand the evolution of cooperative
behaviour.

Recently, a class of strategies was described in~\cite{Press2012} that can
provably extort any given opponent. In~\cite{Hilbe2013, Moran1707} some
questions have already been asked about the true effectiveness of these
strategies in an evolutionary setting. Here another question is asked: is it
possible to recognise this extortionate behaviour? A mathematical procedure for
suspicion is presented: in the same way that the continued actions of an
extortionate individual might raise suspicion.

This work makes use of the Axelrod Python library~\cite{Knight2018, Knight2016}
with a large number of Prisoner Dilemma strategies available to give an
extensive numerical example of the ideas presented.  The approach is presented
in Section~\ref{sec:delta-zd-strategies}.  All of the code and data discussed
in Section~\ref{sec:numerical-experiments} is open sourced, archived and
written according to best scientific principles~\cite{Wilson2014}. The data
archive can be found at~\cite{vincent_knight_2018_1297075}.

\section{Recognising Extortion}\label{sec:delta-zd-strategies}

In~\cite{Press2012}, given a match between 2 memory-one strategies, the concept
of Zero Determinant (ZD) strategies is introduced. The main result of that paper
shows that given two memory one players \(p, q\in\mathbb{R}^4\) a linear
relationship between the players' scores could be forced by one of the players.

Using the notation of~\cite{Press2012}, assuming the utilities for player \(p\)
are given by \(S_x=(R, S, T, P)\) and for player \(q\) by \(S_y=(R, T, S, P)\)
and that the stationary scores of each player is given by \(S_X\) and \(S_Y\)
respectively. The main result of~\cite{Press2012} is that if

\begin{equation}\label{eqn:linear_relationship_for_p}
    \tilde p=\alpha S_x + \beta S_y + \gamma
\end{equation}

or

\begin{equation}\label{eqn:linear_relationship_for_q}
    \tilde q=\alpha S_x + \beta S_y + \gamma
\end{equation}

where \(\tilde p = (1 - p_1, 1 - p_2, p_3, p_4)\) and
\(\tilde q = (1 - q_1, 1 - q_2, q_3, q_4)\) then:

\begin{equation}
    \alpha S_X + \beta S_Y + \gamma = 0
\end{equation}

In~\cite{Press2012} a particular type of ZD strategy is defined: extortionate
strategies. If:

\begin{equation}\label{eqn:constraint_for_extortion}
    \gamma = - P(\alpha + \beta)
\end{equation}

then the player can ensure they get a score \(\chi\) times
larger than the opponent. This extortion coefficient is given by:

\begin{equation}\label{eqn:definition_of_chi}
    \chi=\frac{-\beta}{\alpha}
\end{equation}

Thus, if (\ref{eqn:constraint_for_extortion}) holds and \(\chi >1\) a player is
said to extort their opponent.
Here, the reverse problem is considered: given a
\(p\in\mathbb{R}^4\) how does one identify \(\alpha, \beta\) if they
exist and is the strategy in fact acting in an extortionate way?

These conditions correspond to:

\begin{align}
    \tilde p_1 & = \alpha R + \beta R - P (\alpha + \beta)
            \label{eqn:condition_for_tilde_p1}\\
    \tilde p_2 & = \alpha S + \beta T - P (\alpha + \beta)
            \label{eqn:condition_for_tilde_p2}\\
    \tilde p_3 & = \alpha T + \beta S - P (\alpha + \beta)
            \label{eqn:condition_for_tilde_p3}\\
    \tilde p_4 & = \alpha P + \beta P - P (\alpha + \beta)
            \label{eqn:condition_for_tilde_p4}
\end{align}

Equation (\ref{eqn:condition_for_tilde_p4}) ensures that \(p_4=\tilde p_4=0\).
Equations (\ref{eqn:condition_for_tilde_p1}-\ref{eqn:condition_for_tilde_p3})
can be used to eliminate \(\alpha, \beta\), giving:

\begin{equation}\label{eqn:planar_definition_of_extortion}
    \tilde p_1 = \frac{(R - P)(\tilde p_2 + \tilde p_3)}{S + T - 2P}
\end{equation}

with:

\begin{equation}\label{eqn:definition_of_chi}
    \chi = \frac{\tilde p_2 (P - T) + \tilde p_3 (S - P)}
                {\tilde p_2 (P - S) + \tilde p_3 (T - P)}
\end{equation}

Given a strategy \(p\in\mathbb{R}^{4\times 1}\) equations
(\ref{eqn:condition_for_tilde_p4}), (\ref{eqn:planar_definition_of_extortion}-\ref{eqn:definition_of_chi}) can be used to check if
a strategy is extortionate. The conditions correspond to:

\begin{align}
    p_1 & = \frac{(R-P)(p_2 + p_3) - R + T + S - P}{S + T - 2P}
     \label{eqn:condition_for_p1}\\
    p_4 & = 0 \label{eqn:condition_for_p4}\\
    1 & > p_2 + p_3\label{eqn:condition_for_chi}
\end{align}

The algebraic steps necessary to prove these results are available in the
supporting materials.

All extortionate strategies reside on a triangular (\ref{eqn:condition_for_chi})
plane (\ref{eqn:condition_for_p1}) in 3 dimensions (\ref{eqn:condition_for_p4}).
Using this formulation it can be seen that a necessary (but not sufficient)
condition for an extortionate strategy is that it cooperates on average less
than 50\% of the time when in a state of disagreement with the opponent.

As an example, consider the known extortionate strategy \(p=(8 / 9, 1 / 2, 1 /
3, 0)\) from~\cite{Stewart2012} which is referred to as \texttt{Extort-2}. In
this case, for the standard values of \((R, T, S, P)\) constraint
(\ref{eqn:condition_for_p1}) corresponds to:

\begin{equation}
    p_1 = \frac{2(p_2 + p_3) + 1}{3}
\end{equation}

It is clear that in this case all constraints hold.

This approach could in fact be used to confirm that a given strategy is acting
in an extortionate manner even if it is not a memory one strategy. However, in
practice, if a closed form for \(p\) is not known, then due to measurement
and/or numerical error this would not work.

This problem can be written in the following linear algebraic form where
\(x=(\alpha, \beta)\)
and \(p^*=(\tilde p_1 - 1, tilde_2 - 1, p_3)\):

\begin{equation}\label{eqn:linear_algebraic_equation_for_p}
    Cx= p^*
\end{equation}

\(C\) corresponds to equations
(\ref{eqn:condition_for_tilde_p1}-\ref{eqn:condition_for_tilde_p3}) and is
given by:

\begin{equation}\label{eqn:definition_of_C}
    C =
    \begin{bmatrix}
        R - P & R- P \\
        S - P & T- P \\
        T - P & S- P \\
    \end{bmatrix}
\end{equation}

Note that in general, equation (\ref{eqn:linear_algebraic_equation_for_p}) will
not necessarily have a solution. From the Rouch\'{e}-Capelli theorem if there is
a solution it is unique as \(\text{rank}(C)=2\) which is the dimension of the
variable \(x\). The best fitting \(x\) is found by minimizing:

\begin{equation}\label{eqn:r_squared}
    \text{SSError} = \|C x- p^*\|_2^2 = \sum_{i=1}^{3}\left((C\bar x)_i-p_i^*\right)^2
\end{equation}

Note that \(\text{SSError}\), which is the square of the Frobenius
norm~\cite{Golub2013}, becomes a measure of how close a strategy is to being an
extortionate strategy. Suspicion
of extortion then corresponds to a threshold on \(\text{SSError}\).

By observing interactions (human or otherwise), their memory one representation
can be inferred and this approach can be used to recognise extortionate
behaviour. The notion of comparing theoretic and actual plays of the IPD is not
novel, see for example~\cite{Rand2013}. Immediately it is noted that if the
environment is noisy~\cite{Wu1995} then no strategy can be considered to be
extortionate as \(p_4>0\).

In the next section, this idea will be illustrated by observing the interactions
that take place in a computer based tournament of the IPD\@.

\section{Numerical experiments}\label{sec:numerical-experiments}

In~\cite{Stewart2012} results from a tournament with
\input{./assets/tex/number_of_stewart_plotkin_strategies/main.tex} strategies,
was presented with specific consideration given to ZD strategies. This
tournament is reproduced here using the Axelrod-Python
project~\cite{Knight2016}. To obtain a good measure of the corresponding
transition rates for each strategy all matches have been run for
\input{assets/tex/number_of_turns/main.tex} turns and every match has been
repeated \input{assets/tex/number_of_repetitions/main.tex} times. All of this
interaction data is available at~\cite{vincent_knight_2018_1297075}. A good
match between the inferred Markov chain and the state distribution of the actual
interactions has been verified. Data for this is presented in the supplementary
materials.

Figure~\ref{fig:SSError_overall_in_stewart_plotkin} shows the \(\text{SSError}\)
values for all the strategies in the tournament, as reported
in~\cite{Stewart2012} the extortionate strategy (which has an expected
\(\text{SSError}\) approximately 0) gains a large number of wins.

\begin{figure}[!htbp]
    \centering
    \includegraphics[width=.8\textwidth]{./assets/img/SSError_overall_in_stewart_plotkin/main.pdf}
    \caption{\(\text{SSError}\) and state probabilities for the strategies
        of~\cite{Stewart2012}, ordered both by number of wins and overall score.
        Note that \(P(DC)\) is not shown as it corresponds to the transpose of
        \(P(CD)\). Cooperator and Defector are omitted as they do not visit all
        the states.}
    \label{fig:SSError_overall_in_stewart_plotkin}
\end{figure}

Here, the work of~\cite{Stewart2012} is extended by investigating a tournament
with \input{assets/tex/number_of_full_strategies/main.tex}
strategies.

The results of this analysis are shown in
Figure~\ref{fig:SSError_and_probabilities_in_full}. The top ranking strategies
by number of wins seem to be extortionate (but not against all strategies) and
it can be seen that a small sub group of strategies achieve mutual defection.
All the top ranking strategies according to score achieve mutual cooperation and
do not extort each other, however they
\textbf{do} exhibit extortionate behaviour towards a number of the lower ranking
strategies.

\begin{figure}[!htbp]
    \centering
    \includegraphics[width=.8\textwidth]{./assets/img/SSError_and_probabilities_in_full/main.pdf}
    \caption{\(\text{SSError}\) for the strategies for the full tournament. Only
    strategy interactions for which \(p_4=0\) and \(\chi>1\) are displayed.}
    \label{fig:SSError_and_probabilities_in_full}
\end{figure}

\section{Conclusion}\label{sec:conclusion}

This work defines an approach to measure whether or not a player is playing a
strategy that corresponds to an extortionate strategy as defined
in~\cite{Press2012}: a mathematical model for suspicion. Indeed, all
extortionate strategies have been
 classified as lying on a triangular plane.
This rigorous classification fails to be robust to small measurement error, thus
a statistical approach is proposed.
This is done through a linear algebraic approach for approximating the solution
of a linear system. Using this, a large number of pairwise interactions is
simulated and in fact very few strategies are found to act extortionately.

The work of~\cite{Press2012}, whilst showing that a clever approach to taking
advantage of another memory one strategy exists: this is incomplete. Whilst the
elegance of this result is very attractive, just as the simplicity of the
victory of Tit For Tat in Axelrod's original tournaments was, it is incomplete.
Extortionate strategies achieve a high number of wins but they do not
achieve a high score which corresponds to the fitness landscape in an
evolutionary sense. From the large number of interactions a payoff matrix \(S\)
can be measured where \(S_{ij}\) denotes the score (using standard values of
\((R, S, T, P) = (3, 0, 5, 1)\)) of the \(i\)th strategy
against the \(j\)th strategy. Using this, the replicator equation
describes the evolution of the system based on a population density fitness
function:

\begin{equation}\label{eqn:replicator_dynamics}
    \frac{dx}{dt} = x(S-x^TS x)
\end{equation}

Equation (\ref{eqn:replicator_dynamics}) is solved numerically through an
integration technique described in~\cite{Petzold1983} and
Figure~\ref{fig:replicator_dynamics} shows the evolution of the distribution of
the system: the various strategies are ranked by scores. It is clear to see that
only the high ranking strategies survive the evolutionary process (in fact,
only \input{./assets/img/replicator_dynamics/main.tex}
have a final distribution greater than \(10 ^ {-2}\)). This confirms the
findings of~\cite{Moran1707} in which sophisticated strategies resist
evolutionary invasion of shorter memory strategies. Recalling
Figure~\ref{fig:SSError_and_probabilities_in_full} this demonstrates that:

\begin{itemize}
    \item Cooperation emerges through the evolutionary process: the high scoring
        strategies do not exhibit extortionate behaviour towards each other.
    \item Extortionate strategies do not survive the evolutionary process.
\end{itemize}

\begin{figure}[!htbp]
    \centering
    \includegraphics[width=.8\textwidth]{./assets/img/replicator_dynamics/main.pdf}
    \caption{Numerical simulation of the replicator equation
    (\ref{eqn:replicator_dynamics}): strategies are ordered by score, only the strategies with a high score survive the evolutionary process.}
    \label{fig:replicator_dynamics}
\end{figure}

This work can be used to classify plays of the IPD\@: data can be collected from
actual interactions (in lab or in the field). Furthermore, this allows for a
classification method similar to the notion of fingerprinting presented
in~\cite{Ashlock2008}. Trained strategies can potentially be classified as
extortionate or not or it could be possible to even constrain the reinforcement
learning approaches that are becoming prevalent in the literature.
Alternatively, this mathematical approach for recognising extortion could be
used in sophisticated strategies to defend against invasion. Arguably, some of
the strategies considered here exhibit this behaviour, indeed as described
in~\cite{Harper2017}, the top ranking strategies in the full tournament are
obtained using evolutionary reinforcement learning techniques, thus, suspicion
of extortionate behaviour could in fact be an evolutionary trait.

\section*{Acknowledgements}

The following open source software libraries were used in this research:

\begin{itemize}
    \item The Axelrod ~\cite{Knight2016, Knight2018} library (IPD strategies and
        tournaments).
    \item The sympy library~\cite{Meurer2017} (verification of all symbolic
        calculations).
    \item The matplotlib~\cite{Droettboom2018} library (visualisation).
    \item The pandas~\cite{Structures2010}, dask~\cite{Dask2016} and
        NumPy~\cite{Oliphant2015} libraries (data manipulation).
    \item The SciPy~\cite{Jones2001} library (numerical integration of the
        replicator equation).
\end{itemize}

This work was performed using the computational facilities of the Advanced
Research Computing @ Cardiff (ARCCA) Division, Cardiff University.

\printbibliography

\newpage
\section*{Supplementary materials}

\includepdf{assets/pdf/proof_of_form_of_extortionate_strategies/main.pdf}

\newpage

Using the pair wise interactions the transition rates \(p,
q\) can be measured and the steady state probabilities inferred and compared to
the actual probabilities of each state.
This is done numerically by computing the singular eigenvector of the
matrix \(A\) \cite{Stewart2009}:

\[
    A =
    \begin{bmatrix}
        p_1 q_1 & p_1 (1 - q_1) & (1 - p_1) q_1 & (1 -p_1) (1 - q_1) \\
        p_2 q_2 & p_2 (1 - q_2) & (1 - p_2) q_2 & (1 -p_2) (1 - q_2) \\
        p_3 q_3 & p_3 (1 - q_3) & (1 - p_3) q_3 & (1 -p_3) (1 - q_3) \\
        p_4 q_4 & p_4 (1 - q_4) & (1 - p_4) q_4 & (1 -p_4) (1 - q_4) \\
    \end{bmatrix}
\]

Figure~\ref{fig:computed_probabilities_vs_theoretic_probabilities} shows a
regression line fitted to every pairwise interaction with a reported
\(\text{SSError}\) value (pairwise interactions with missing states were
omitted). This serves to validate the approach: a part from some edge cases the
relationship is consistent.

\begin{figure}[!htbp]
    \centering
    \includegraphics[width=.8\textwidth]{./assets/img/computed_probabilities_vs_theoretic_probabilities/main.pdf}
    \caption{The
        relationship between the steady state probabilities inferred from the
        measured transitions and the actual steady state probabilities. A linear
        regression line is included validating the approach.}
    \label{fig:computed_probabilities_vs_theoretic_probabilities}
\end{figure}


\end{document}

have a final distribution greater than \(10 ^ {-2}\)). This confirms the
findings of~\cite{Moran1707} in which sophisticated strategies resist
evolutionary invasion of shorter memory strategies. Recalling
Figure~\ref{fig:SSError_and_probabilities_in_full} this demonstrates that:

\begin{itemize}
    \item Cooperation emerges through the evolutionary process: the high scoring
        strategies do not exhibit extortionate behaviour towards each other.
    \item Extortionate strategies do not survive the evolutionary process.
\end{itemize}

\begin{figure}[!htbp]
    \centering
    \includegraphics[width=.8\textwidth]{./assets/img/replicator_dynamics/main.pdf}
    \caption{Numerical simulation of the replicator equation
    (\ref{eqn:replicator_dynamics}): strategies are ordered by score, only the strategies with a high score survive the evolutionary process.}
    \label{fig:replicator_dynamics}
\end{figure}

This work can be used to classify plays of the IPD\@: data can be collected from
actual interactions (in lab or in the field). Furthermore, this allows for a
classification method similar to the notion of fingerprinting presented
in~\cite{Ashlock2008}. Trained strategies can potentially be classified as
extortionate or not or it could be possible to even constrain the reinforcement
learning approaches that are becoming prevalent in the literature.
Alternatively, this mathematical approach for recognising extortion could be
used in sophisticated strategies to defend against invasion. Arguably, some of
the strategies considered here exhibit this behaviour, indeed as described
in~\cite{Harper2017}, the top ranking strategies in the full tournament are
obtained using evolutionary reinforcement learning techniques, thus, suspicion
of extortionate behaviour could in fact be an evolutionary trait.

\section*{Acknowledgements}

The following open source software libraries were used in this research:

\begin{itemize}
    \item The Axelrod ~\cite{Knight2016, Knight2018} library (IPD strategies and
        tournaments).
    \item The sympy library~\cite{Meurer2017} (verification of all symbolic
        calculations).
    \item The matplotlib~\cite{Droettboom2018} library (visualisation).
    \item The pandas~\cite{Structures2010}, dask~\cite{Dask2016} and
        NumPy~\cite{Oliphant2015} libraries (data manipulation).
    \item The SciPy~\cite{Jones2001} library (numerical integration of the
        replicator equation).
\end{itemize}

This work was performed using the computational facilities of the Advanced
Research Computing @ Cardiff (ARCCA) Division, Cardiff University.

\printbibliography

\newpage
\section*{Supplementary materials}

\includepdf{assets/pdf/proof_of_form_of_extortionate_strategies/main.pdf}

\newpage

Using the pair wise interactions the transition rates \(p,
q\) can be measured and the steady state probabilities inferred and compared to
the actual probabilities of each state.
This is done numerically by computing the singular eigenvector of the
matrix \(A\) \cite{Stewart2009}:

\[
    A =
    \begin{bmatrix}
        p_1 q_1 & p_1 (1 - q_1) & (1 - p_1) q_1 & (1 -p_1) (1 - q_1) \\
        p_2 q_2 & p_2 (1 - q_2) & (1 - p_2) q_2 & (1 -p_2) (1 - q_2) \\
        p_3 q_3 & p_3 (1 - q_3) & (1 - p_3) q_3 & (1 -p_3) (1 - q_3) \\
        p_4 q_4 & p_4 (1 - q_4) & (1 - p_4) q_4 & (1 -p_4) (1 - q_4) \\
    \end{bmatrix}
\]

Figure~\ref{fig:computed_probabilities_vs_theoretic_probabilities} shows a
regression line fitted to every pairwise interaction with a reported
\(\text{SSError}\) value (pairwise interactions with missing states were
omitted). This serves to validate the approach: a part from some edge cases the
relationship is consistent.

\begin{figure}[!htbp]
    \centering
    \includegraphics[width=.8\textwidth]{./assets/img/computed_probabilities_vs_theoretic_probabilities/main.pdf}
    \caption{The
        relationship between the steady state probabilities inferred from the
        measured transitions and the actual steady state probabilities. A linear
        regression line is included validating the approach.}
    \label{fig:computed_probabilities_vs_theoretic_probabilities}
\end{figure}


\end{document}

    strategies is considered. In this setting
    the most highly performing strategies do not play in an extortionate way
    against each other but do against lower performing strategies.
    This suggests that whilst the theory of Zero Determinant strategies
    indicates that memory is not of fundamental importance to the evolution of
    cooperative behaviour, this is incomplete.
\end{abstract}

\section{Introduction}\label{sec:introduction}

Agent based game theoretic models have become a stalwart of the underpinning
mathematics of interactive behaviours. One of the major pieces of work
in this area is the pair of original computer tournaments run by Robert
Axelrod~\cite{Axelrod1980, Axelrod1980a}. These tournaments pitted submitted
computer strategies against each other in plays of the Iterated Prisoner's
Dilemma. A common game where agents can choose to pay a slight cost to their
immediate utility in the hope of building a reputation. This has been used in
economic and evolutionary game theory to understand the evolution of cooperative
behaviour.

Recently, a class of strategies was described in~\cite{Press2012} that can
provably extort any given opponent. In~\cite{Hilbe2013, Moran1707} some
questions have already been asked about the true effectiveness of these
strategies in an evolutionary setting. Here another question is asked: is it
possible to recognise this extortionate behaviour? A mathematical procedure for
suspicion is presented: in the same way that the continued actions of an
extortionate individual might raise suspicion.

This work makes use of the Axelrod Python library~\cite{Knight2018, Knight2016}
with a large number of Prisoner Dilemma strategies available to give an
extensive numerical example of the ideas presented.  The approach is presented
in Section~\ref{sec:delta-zd-strategies}.  All of the code and data discussed
in Section~\ref{sec:numerical-experiments} is open sourced, archived and
written according to best scientific principles~\cite{Wilson2014}. The data
archive can be found at~\cite{vincent_knight_2018_1297075}.

\section{Recognising Extortion}\label{sec:delta-zd-strategies}

In~\cite{Press2012}, given a match between 2 memory-one strategies, the concept
of Zero Determinant (ZD) strategies is introduced. The main result of that paper
shows that given two memory one players \(p, q\in\mathbb{R}^4\) a linear
relationship between the players' scores could be forced by one of the players.

Using the notation of~\cite{Press2012}, assuming the utilities for player \(p\)
are given by \(S_x=(R, S, T, P)\) and for player \(q\) by \(S_y=(R, T, S, P)\)
and that the stationary scores of each player is given by \(S_X\) and \(S_Y\)
respectively. The main result of~\cite{Press2012} is that if

\begin{equation}\label{eqn:linear_relationship_for_p}
    \tilde p=\alpha S_x + \beta S_y + \gamma
\end{equation}

or

\begin{equation}\label{eqn:linear_relationship_for_q}
    \tilde q=\alpha S_x + \beta S_y + \gamma
\end{equation}

where \(\tilde p = (1 - p_1, 1 - p_2, p_3, p_4)\) and
\(\tilde q = (1 - q_1, 1 - q_2, q_3, q_4)\) then:

\begin{equation}
    \alpha S_X + \beta S_Y + \gamma = 0
\end{equation}

In~\cite{Press2012} a particular type of ZD strategy is defined: extortionate
strategies. If:

\begin{equation}\label{eqn:constraint_for_extortion}
    \gamma = - P(\alpha + \beta)
\end{equation}

then the player can ensure they get a score \(\chi\) times
larger than the opponent. This extortion coefficient is given by:

\begin{equation}\label{eqn:definition_of_chi}
    \chi=\frac{-\beta}{\alpha}
\end{equation}

Thus, if (\ref{eqn:constraint_for_extortion}) holds and \(\chi >1\) a player is
said to extort their opponent.
Here, the reverse problem is considered: given a
\(p\in\mathbb{R}^4\) how does one identify \(\alpha, \beta\) if they
exist and is the strategy in fact acting in an extortionate way?

These conditions correspond to:

\begin{align}
    \tilde p_1 & = \alpha R + \beta R - P (\alpha + \beta)
            \label{eqn:condition_for_tilde_p1}\\
    \tilde p_2 & = \alpha S + \beta T - P (\alpha + \beta)
            \label{eqn:condition_for_tilde_p2}\\
    \tilde p_3 & = \alpha T + \beta S - P (\alpha + \beta)
            \label{eqn:condition_for_tilde_p3}\\
    \tilde p_4 & = \alpha P + \beta P - P (\alpha + \beta)
            \label{eqn:condition_for_tilde_p4}
\end{align}

Equation (\ref{eqn:condition_for_tilde_p4}) ensures that \(p_4=\tilde p_4=0\).
Equations (\ref{eqn:condition_for_tilde_p1}-\ref{eqn:condition_for_tilde_p3})
can be used to eliminate \(\alpha, \beta\), giving:

\begin{equation}\label{eqn:planar_definition_of_extortion}
    \tilde p_1 = \frac{(R - P)(\tilde p_2 + \tilde p_3)}{S + T - 2P}
\end{equation}

with:

\begin{equation}\label{eqn:definition_of_chi}
    \chi = \frac{\tilde p_2 (P - T) + \tilde p_3 (S - P)}
                {\tilde p_2 (P - S) + \tilde p_3 (T - P)}
\end{equation}

Given a strategy \(p\in\mathbb{R}^{4\times 1}\) equations
(\ref{eqn:condition_for_tilde_p4}), (\ref{eqn:planar_definition_of_extortion}-\ref{eqn:definition_of_chi}) can be used to check if
a strategy is extortionate. The conditions correspond to:

\begin{align}
    p_1 & = \frac{(R-P)(p_2 + p_3) - R + T + S - P}{S + T - 2P}
     \label{eqn:condition_for_p1}\\
    p_4 & = 0 \label{eqn:condition_for_p4}\\
    1 & > p_2 + p_3\label{eqn:condition_for_chi}
\end{align}

The algebraic steps necessary to prove these results are available in the
supporting materials.

All extortionate strategies reside on a triangular (\ref{eqn:condition_for_chi})
plane (\ref{eqn:condition_for_p1}) in 3 dimensions (\ref{eqn:condition_for_p4}).
Using this formulation it can be seen that a necessary (but not sufficient)
condition for an extortionate strategy is that it cooperates on average less
than 50\% of the time when in a state of disagreement with the opponent.

As an example, consider the known extortionate strategy \(p=(8 / 9, 1 / 2, 1 /
3, 0)\) from~\cite{Stewart2012} which is referred to as \texttt{Extort-2}. In
this case, for the standard values of \((R, T, S, P)\) constraint
(\ref{eqn:condition_for_p1}) corresponds to:

\begin{equation}
    p_1 = \frac{2(p_2 + p_3) + 1}{3}
\end{equation}

It is clear that in this case all constraints hold.

This approach could in fact be used to confirm that a given strategy is acting
in an extortionate manner even if it is not a memory one strategy. However, in
practice, if a closed form for \(p\) is not known, then due to measurement
and/or numerical error this would not work.

This problem can be written in the following linear algebraic form where
\(x=(\alpha, \beta)\)
and \(p^*=(\tilde p_1 - 1, tilde_2 - 1, p_3)\):

\begin{equation}\label{eqn:linear_algebraic_equation_for_p}
    Cx= p^*
\end{equation}

\(C\) corresponds to equations
(\ref{eqn:condition_for_tilde_p1}-\ref{eqn:condition_for_tilde_p3}) and is
given by:

\begin{equation}\label{eqn:definition_of_C}
    C =
    \begin{bmatrix}
        R - P & R- P \\
        S - P & T- P \\
        T - P & S- P \\
    \end{bmatrix}
\end{equation}

Note that in general, equation (\ref{eqn:linear_algebraic_equation_for_p}) will
not necessarily have a solution. From the Rouch\'{e}-Capelli theorem if there is
a solution it is unique as \(\text{rank}(C)=2\) which is the dimension of the
variable \(x\). The best fitting \(x\) is found by minimizing:

\begin{equation}\label{eqn:r_squared}
    \text{SSError} = \|C x- p^*\|_2^2 = \sum_{i=1}^{3}\left((C\bar x)_i-p_i^*\right)^2
\end{equation}

Note that \(\text{SSError}\), which is the square of the Frobenius
norm~\cite{Golub2013}, becomes a measure of how close a strategy is to being an
extortionate strategy. Suspicion
of extortion then corresponds to a threshold on \(\text{SSError}\).

By observing interactions (human or otherwise), their memory one representation
can be inferred and this approach can be used to recognise extortionate
behaviour. The notion of comparing theoretic and actual plays of the IPD is not
novel, see for example~\cite{Rand2013}. Immediately it is noted that if the
environment is noisy~\cite{Wu1995} then no strategy can be considered to be
extortionate as \(p_4>0\).

In the next section, this idea will be illustrated by observing the interactions
that take place in a computer based tournament of the IPD\@.

\section{Numerical experiments}\label{sec:numerical-experiments}

In~\cite{Stewart2012} results from a tournament with
\documentclass[a4paper]{article}

\usepackage{amsmath}
\usepackage{amssymb}
\usepackage[margin=1.5cm,
            includefoot,
            footskip=30pt]{geometry}
\usepackage{layout}
\usepackage{graphicx}
\usepackage{subcaption}

\usepackage{biblatex}
\usepackage{pdfpages}

\bibliography{main.bib}

\title{Suspicion: Recognising and evaluating the effectiveness
       of extortion in the Iterated Prisoner's Dilemma}
\author{Vincent A. Knight \and Nikoleta E. Glynatsi}
\date{\today}



\begin{document}

\maketitle

\begin{abstract}
    The Iterated Prisoner's Dilemma is a model for rational and evolutionary
    interactive behaviour. It has applications both in the study of human social
    behaviour as well as in biology.
    It is used to understand when and how a rational individual might
    accept an immediate cost to their own utility for the direct benefit of
    another.

    Much attention has been given to a class of strategies called
    Zero Determinant strategies. It has been theoretically shown that these
    strategies can ``extort'' any player.

    In this work, an approach to identify if observed strategies are playing in
    an extortionate way is described. Furthermore, experimental analysis of
    a large tournament with \documentclass[a4paper]{article}

\usepackage{amsmath}
\usepackage{amssymb}
\usepackage[margin=1.5cm,
            includefoot,
            footskip=30pt]{geometry}
\usepackage{layout}
\usepackage{graphicx}
\usepackage{subcaption}

\usepackage{biblatex}
\usepackage{pdfpages}

\bibliography{main.bib}

\title{Suspicion: Recognising and evaluating the effectiveness
       of extortion in the Iterated Prisoner's Dilemma}
\author{Vincent A. Knight \and Nikoleta E. Glynatsi}
\date{\today}



\begin{document}

\maketitle

\begin{abstract}
    The Iterated Prisoner's Dilemma is a model for rational and evolutionary
    interactive behaviour. It has applications both in the study of human social
    behaviour as well as in biology.
    It is used to understand when and how a rational individual might
    accept an immediate cost to their own utility for the direct benefit of
    another.

    Much attention has been given to a class of strategies called
    Zero Determinant strategies. It has been theoretically shown that these
    strategies can ``extort'' any player.

    In this work, an approach to identify if observed strategies are playing in
    an extortionate way is described. Furthermore, experimental analysis of
    a large tournament with \input{assets/tex/number_of_full_strategies/main.tex}
    strategies is considered. In this setting
    the most highly performing strategies do not play in an extortionate way
    against each other but do against lower performing strategies.
    This suggests that whilst the theory of Zero Determinant strategies
    indicates that memory is not of fundamental importance to the evolution of
    cooperative behaviour, this is incomplete.
\end{abstract}

\section{Introduction}\label{sec:introduction}

Agent based game theoretic models have become a stalwart of the underpinning
mathematics of interactive behaviours. One of the major pieces of work
in this area is the pair of original computer tournaments run by Robert
Axelrod~\cite{Axelrod1980, Axelrod1980a}. These tournaments pitted submitted
computer strategies against each other in plays of the Iterated Prisoner's
Dilemma. A common game where agents can choose to pay a slight cost to their
immediate utility in the hope of building a reputation. This has been used in
economic and evolutionary game theory to understand the evolution of cooperative
behaviour.

Recently, a class of strategies was described in~\cite{Press2012} that can
provably extort any given opponent. In~\cite{Hilbe2013, Moran1707} some
questions have already been asked about the true effectiveness of these
strategies in an evolutionary setting. Here another question is asked: is it
possible to recognise this extortionate behaviour? A mathematical procedure for
suspicion is presented: in the same way that the continued actions of an
extortionate individual might raise suspicion.

This work makes use of the Axelrod Python library~\cite{Knight2018, Knight2016}
with a large number of Prisoner Dilemma strategies available to give an
extensive numerical example of the ideas presented.  The approach is presented
in Section~\ref{sec:delta-zd-strategies}.  All of the code and data discussed
in Section~\ref{sec:numerical-experiments} is open sourced, archived and
written according to best scientific principles~\cite{Wilson2014}. The data
archive can be found at~\cite{vincent_knight_2018_1297075}.

\section{Recognising Extortion}\label{sec:delta-zd-strategies}

In~\cite{Press2012}, given a match between 2 memory-one strategies, the concept
of Zero Determinant (ZD) strategies is introduced. The main result of that paper
shows that given two memory one players \(p, q\in\mathbb{R}^4\) a linear
relationship between the players' scores could be forced by one of the players.

Using the notation of~\cite{Press2012}, assuming the utilities for player \(p\)
are given by \(S_x=(R, S, T, P)\) and for player \(q\) by \(S_y=(R, T, S, P)\)
and that the stationary scores of each player is given by \(S_X\) and \(S_Y\)
respectively. The main result of~\cite{Press2012} is that if

\begin{equation}\label{eqn:linear_relationship_for_p}
    \tilde p=\alpha S_x + \beta S_y + \gamma
\end{equation}

or

\begin{equation}\label{eqn:linear_relationship_for_q}
    \tilde q=\alpha S_x + \beta S_y + \gamma
\end{equation}

where \(\tilde p = (1 - p_1, 1 - p_2, p_3, p_4)\) and
\(\tilde q = (1 - q_1, 1 - q_2, q_3, q_4)\) then:

\begin{equation}
    \alpha S_X + \beta S_Y + \gamma = 0
\end{equation}

In~\cite{Press2012} a particular type of ZD strategy is defined: extortionate
strategies. If:

\begin{equation}\label{eqn:constraint_for_extortion}
    \gamma = - P(\alpha + \beta)
\end{equation}

then the player can ensure they get a score \(\chi\) times
larger than the opponent. This extortion coefficient is given by:

\begin{equation}\label{eqn:definition_of_chi}
    \chi=\frac{-\beta}{\alpha}
\end{equation}

Thus, if (\ref{eqn:constraint_for_extortion}) holds and \(\chi >1\) a player is
said to extort their opponent.
Here, the reverse problem is considered: given a
\(p\in\mathbb{R}^4\) how does one identify \(\alpha, \beta\) if they
exist and is the strategy in fact acting in an extortionate way?

These conditions correspond to:

\begin{align}
    \tilde p_1 & = \alpha R + \beta R - P (\alpha + \beta)
            \label{eqn:condition_for_tilde_p1}\\
    \tilde p_2 & = \alpha S + \beta T - P (\alpha + \beta)
            \label{eqn:condition_for_tilde_p2}\\
    \tilde p_3 & = \alpha T + \beta S - P (\alpha + \beta)
            \label{eqn:condition_for_tilde_p3}\\
    \tilde p_4 & = \alpha P + \beta P - P (\alpha + \beta)
            \label{eqn:condition_for_tilde_p4}
\end{align}

Equation (\ref{eqn:condition_for_tilde_p4}) ensures that \(p_4=\tilde p_4=0\).
Equations (\ref{eqn:condition_for_tilde_p1}-\ref{eqn:condition_for_tilde_p3})
can be used to eliminate \(\alpha, \beta\), giving:

\begin{equation}\label{eqn:planar_definition_of_extortion}
    \tilde p_1 = \frac{(R - P)(\tilde p_2 + \tilde p_3)}{S + T - 2P}
\end{equation}

with:

\begin{equation}\label{eqn:definition_of_chi}
    \chi = \frac{\tilde p_2 (P - T) + \tilde p_3 (S - P)}
                {\tilde p_2 (P - S) + \tilde p_3 (T - P)}
\end{equation}

Given a strategy \(p\in\mathbb{R}^{4\times 1}\) equations
(\ref{eqn:condition_for_tilde_p4}), (\ref{eqn:planar_definition_of_extortion}-\ref{eqn:definition_of_chi}) can be used to check if
a strategy is extortionate. The conditions correspond to:

\begin{align}
    p_1 & = \frac{(R-P)(p_2 + p_3) - R + T + S - P}{S + T - 2P}
     \label{eqn:condition_for_p1}\\
    p_4 & = 0 \label{eqn:condition_for_p4}\\
    1 & > p_2 + p_3\label{eqn:condition_for_chi}
\end{align}

The algebraic steps necessary to prove these results are available in the
supporting materials.

All extortionate strategies reside on a triangular (\ref{eqn:condition_for_chi})
plane (\ref{eqn:condition_for_p1}) in 3 dimensions (\ref{eqn:condition_for_p4}).
Using this formulation it can be seen that a necessary (but not sufficient)
condition for an extortionate strategy is that it cooperates on average less
than 50\% of the time when in a state of disagreement with the opponent.

As an example, consider the known extortionate strategy \(p=(8 / 9, 1 / 2, 1 /
3, 0)\) from~\cite{Stewart2012} which is referred to as \texttt{Extort-2}. In
this case, for the standard values of \((R, T, S, P)\) constraint
(\ref{eqn:condition_for_p1}) corresponds to:

\begin{equation}
    p_1 = \frac{2(p_2 + p_3) + 1}{3}
\end{equation}

It is clear that in this case all constraints hold.

This approach could in fact be used to confirm that a given strategy is acting
in an extortionate manner even if it is not a memory one strategy. However, in
practice, if a closed form for \(p\) is not known, then due to measurement
and/or numerical error this would not work.

This problem can be written in the following linear algebraic form where
\(x=(\alpha, \beta)\)
and \(p^*=(\tilde p_1 - 1, tilde_2 - 1, p_3)\):

\begin{equation}\label{eqn:linear_algebraic_equation_for_p}
    Cx= p^*
\end{equation}

\(C\) corresponds to equations
(\ref{eqn:condition_for_tilde_p1}-\ref{eqn:condition_for_tilde_p3}) and is
given by:

\begin{equation}\label{eqn:definition_of_C}
    C =
    \begin{bmatrix}
        R - P & R- P \\
        S - P & T- P \\
        T - P & S- P \\
    \end{bmatrix}
\end{equation}

Note that in general, equation (\ref{eqn:linear_algebraic_equation_for_p}) will
not necessarily have a solution. From the Rouch\'{e}-Capelli theorem if there is
a solution it is unique as \(\text{rank}(C)=2\) which is the dimension of the
variable \(x\). The best fitting \(x\) is found by minimizing:

\begin{equation}\label{eqn:r_squared}
    \text{SSError} = \|C x- p^*\|_2^2 = \sum_{i=1}^{3}\left((C\bar x)_i-p_i^*\right)^2
\end{equation}

Note that \(\text{SSError}\), which is the square of the Frobenius
norm~\cite{Golub2013}, becomes a measure of how close a strategy is to being an
extortionate strategy. Suspicion
of extortion then corresponds to a threshold on \(\text{SSError}\).

By observing interactions (human or otherwise), their memory one representation
can be inferred and this approach can be used to recognise extortionate
behaviour. The notion of comparing theoretic and actual plays of the IPD is not
novel, see for example~\cite{Rand2013}. Immediately it is noted that if the
environment is noisy~\cite{Wu1995} then no strategy can be considered to be
extortionate as \(p_4>0\).

In the next section, this idea will be illustrated by observing the interactions
that take place in a computer based tournament of the IPD\@.

\section{Numerical experiments}\label{sec:numerical-experiments}

In~\cite{Stewart2012} results from a tournament with
\input{./assets/tex/number_of_stewart_plotkin_strategies/main.tex} strategies,
was presented with specific consideration given to ZD strategies. This
tournament is reproduced here using the Axelrod-Python
project~\cite{Knight2016}. To obtain a good measure of the corresponding
transition rates for each strategy all matches have been run for
\input{assets/tex/number_of_turns/main.tex} turns and every match has been
repeated \input{assets/tex/number_of_repetitions/main.tex} times. All of this
interaction data is available at~\cite{vincent_knight_2018_1297075}. A good
match between the inferred Markov chain and the state distribution of the actual
interactions has been verified. Data for this is presented in the supplementary
materials.

Figure~\ref{fig:SSError_overall_in_stewart_plotkin} shows the \(\text{SSError}\)
values for all the strategies in the tournament, as reported
in~\cite{Stewart2012} the extortionate strategy (which has an expected
\(\text{SSError}\) approximately 0) gains a large number of wins.

\begin{figure}[!htbp]
    \centering
    \includegraphics[width=.8\textwidth]{./assets/img/SSError_overall_in_stewart_plotkin/main.pdf}
    \caption{\(\text{SSError}\) and state probabilities for the strategies
        of~\cite{Stewart2012}, ordered both by number of wins and overall score.
        Note that \(P(DC)\) is not shown as it corresponds to the transpose of
        \(P(CD)\). Cooperator and Defector are omitted as they do not visit all
        the states.}
    \label{fig:SSError_overall_in_stewart_plotkin}
\end{figure}

Here, the work of~\cite{Stewart2012} is extended by investigating a tournament
with \input{assets/tex/number_of_full_strategies/main.tex}
strategies.

The results of this analysis are shown in
Figure~\ref{fig:SSError_and_probabilities_in_full}. The top ranking strategies
by number of wins seem to be extortionate (but not against all strategies) and
it can be seen that a small sub group of strategies achieve mutual defection.
All the top ranking strategies according to score achieve mutual cooperation and
do not extort each other, however they
\textbf{do} exhibit extortionate behaviour towards a number of the lower ranking
strategies.

\begin{figure}[!htbp]
    \centering
    \includegraphics[width=.8\textwidth]{./assets/img/SSError_and_probabilities_in_full/main.pdf}
    \caption{\(\text{SSError}\) for the strategies for the full tournament. Only
    strategy interactions for which \(p_4=0\) and \(\chi>1\) are displayed.}
    \label{fig:SSError_and_probabilities_in_full}
\end{figure}

\section{Conclusion}\label{sec:conclusion}

This work defines an approach to measure whether or not a player is playing a
strategy that corresponds to an extortionate strategy as defined
in~\cite{Press2012}: a mathematical model for suspicion. Indeed, all
extortionate strategies have been
 classified as lying on a triangular plane.
This rigorous classification fails to be robust to small measurement error, thus
a statistical approach is proposed.
This is done through a linear algebraic approach for approximating the solution
of a linear system. Using this, a large number of pairwise interactions is
simulated and in fact very few strategies are found to act extortionately.

The work of~\cite{Press2012}, whilst showing that a clever approach to taking
advantage of another memory one strategy exists: this is incomplete. Whilst the
elegance of this result is very attractive, just as the simplicity of the
victory of Tit For Tat in Axelrod's original tournaments was, it is incomplete.
Extortionate strategies achieve a high number of wins but they do not
achieve a high score which corresponds to the fitness landscape in an
evolutionary sense. From the large number of interactions a payoff matrix \(S\)
can be measured where \(S_{ij}\) denotes the score (using standard values of
\((R, S, T, P) = (3, 0, 5, 1)\)) of the \(i\)th strategy
against the \(j\)th strategy. Using this, the replicator equation
describes the evolution of the system based on a population density fitness
function:

\begin{equation}\label{eqn:replicator_dynamics}
    \frac{dx}{dt} = x(S-x^TS x)
\end{equation}

Equation (\ref{eqn:replicator_dynamics}) is solved numerically through an
integration technique described in~\cite{Petzold1983} and
Figure~\ref{fig:replicator_dynamics} shows the evolution of the distribution of
the system: the various strategies are ranked by scores. It is clear to see that
only the high ranking strategies survive the evolutionary process (in fact,
only \input{./assets/img/replicator_dynamics/main.tex}
have a final distribution greater than \(10 ^ {-2}\)). This confirms the
findings of~\cite{Moran1707} in which sophisticated strategies resist
evolutionary invasion of shorter memory strategies. Recalling
Figure~\ref{fig:SSError_and_probabilities_in_full} this demonstrates that:

\begin{itemize}
    \item Cooperation emerges through the evolutionary process: the high scoring
        strategies do not exhibit extortionate behaviour towards each other.
    \item Extortionate strategies do not survive the evolutionary process.
\end{itemize}

\begin{figure}[!htbp]
    \centering
    \includegraphics[width=.8\textwidth]{./assets/img/replicator_dynamics/main.pdf}
    \caption{Numerical simulation of the replicator equation
    (\ref{eqn:replicator_dynamics}): strategies are ordered by score, only the strategies with a high score survive the evolutionary process.}
    \label{fig:replicator_dynamics}
\end{figure}

This work can be used to classify plays of the IPD\@: data can be collected from
actual interactions (in lab or in the field). Furthermore, this allows for a
classification method similar to the notion of fingerprinting presented
in~\cite{Ashlock2008}. Trained strategies can potentially be classified as
extortionate or not or it could be possible to even constrain the reinforcement
learning approaches that are becoming prevalent in the literature.
Alternatively, this mathematical approach for recognising extortion could be
used in sophisticated strategies to defend against invasion. Arguably, some of
the strategies considered here exhibit this behaviour, indeed as described
in~\cite{Harper2017}, the top ranking strategies in the full tournament are
obtained using evolutionary reinforcement learning techniques, thus, suspicion
of extortionate behaviour could in fact be an evolutionary trait.

\section*{Acknowledgements}

The following open source software libraries were used in this research:

\begin{itemize}
    \item The Axelrod ~\cite{Knight2016, Knight2018} library (IPD strategies and
        tournaments).
    \item The sympy library~\cite{Meurer2017} (verification of all symbolic
        calculations).
    \item The matplotlib~\cite{Droettboom2018} library (visualisation).
    \item The pandas~\cite{Structures2010}, dask~\cite{Dask2016} and
        NumPy~\cite{Oliphant2015} libraries (data manipulation).
    \item The SciPy~\cite{Jones2001} library (numerical integration of the
        replicator equation).
\end{itemize}

This work was performed using the computational facilities of the Advanced
Research Computing @ Cardiff (ARCCA) Division, Cardiff University.

\printbibliography

\newpage
\section*{Supplementary materials}

\includepdf{assets/pdf/proof_of_form_of_extortionate_strategies/main.pdf}

\newpage

Using the pair wise interactions the transition rates \(p,
q\) can be measured and the steady state probabilities inferred and compared to
the actual probabilities of each state.
This is done numerically by computing the singular eigenvector of the
matrix \(A\) \cite{Stewart2009}:

\[
    A =
    \begin{bmatrix}
        p_1 q_1 & p_1 (1 - q_1) & (1 - p_1) q_1 & (1 -p_1) (1 - q_1) \\
        p_2 q_2 & p_2 (1 - q_2) & (1 - p_2) q_2 & (1 -p_2) (1 - q_2) \\
        p_3 q_3 & p_3 (1 - q_3) & (1 - p_3) q_3 & (1 -p_3) (1 - q_3) \\
        p_4 q_4 & p_4 (1 - q_4) & (1 - p_4) q_4 & (1 -p_4) (1 - q_4) \\
    \end{bmatrix}
\]

Figure~\ref{fig:computed_probabilities_vs_theoretic_probabilities} shows a
regression line fitted to every pairwise interaction with a reported
\(\text{SSError}\) value (pairwise interactions with missing states were
omitted). This serves to validate the approach: a part from some edge cases the
relationship is consistent.

\begin{figure}[!htbp]
    \centering
    \includegraphics[width=.8\textwidth]{./assets/img/computed_probabilities_vs_theoretic_probabilities/main.pdf}
    \caption{The
        relationship between the steady state probabilities inferred from the
        measured transitions and the actual steady state probabilities. A linear
        regression line is included validating the approach.}
    \label{fig:computed_probabilities_vs_theoretic_probabilities}
\end{figure}


\end{document}

    strategies is considered. In this setting
    the most highly performing strategies do not play in an extortionate way
    against each other but do against lower performing strategies.
    This suggests that whilst the theory of Zero Determinant strategies
    indicates that memory is not of fundamental importance to the evolution of
    cooperative behaviour, this is incomplete.
\end{abstract}

\section{Introduction}\label{sec:introduction}

Agent based game theoretic models have become a stalwart of the underpinning
mathematics of interactive behaviours. One of the major pieces of work
in this area is the pair of original computer tournaments run by Robert
Axelrod~\cite{Axelrod1980, Axelrod1980a}. These tournaments pitted submitted
computer strategies against each other in plays of the Iterated Prisoner's
Dilemma. A common game where agents can choose to pay a slight cost to their
immediate utility in the hope of building a reputation. This has been used in
economic and evolutionary game theory to understand the evolution of cooperative
behaviour.

Recently, a class of strategies was described in~\cite{Press2012} that can
provably extort any given opponent. In~\cite{Hilbe2013, Moran1707} some
questions have already been asked about the true effectiveness of these
strategies in an evolutionary setting. Here another question is asked: is it
possible to recognise this extortionate behaviour? A mathematical procedure for
suspicion is presented: in the same way that the continued actions of an
extortionate individual might raise suspicion.

This work makes use of the Axelrod Python library~\cite{Knight2018, Knight2016}
with a large number of Prisoner Dilemma strategies available to give an
extensive numerical example of the ideas presented.  The approach is presented
in Section~\ref{sec:delta-zd-strategies}.  All of the code and data discussed
in Section~\ref{sec:numerical-experiments} is open sourced, archived and
written according to best scientific principles~\cite{Wilson2014}. The data
archive can be found at~\cite{vincent_knight_2018_1297075}.

\section{Recognising Extortion}\label{sec:delta-zd-strategies}

In~\cite{Press2012}, given a match between 2 memory-one strategies, the concept
of Zero Determinant (ZD) strategies is introduced. The main result of that paper
shows that given two memory one players \(p, q\in\mathbb{R}^4\) a linear
relationship between the players' scores could be forced by one of the players.

Using the notation of~\cite{Press2012}, assuming the utilities for player \(p\)
are given by \(S_x=(R, S, T, P)\) and for player \(q\) by \(S_y=(R, T, S, P)\)
and that the stationary scores of each player is given by \(S_X\) and \(S_Y\)
respectively. The main result of~\cite{Press2012} is that if

\begin{equation}\label{eqn:linear_relationship_for_p}
    \tilde p=\alpha S_x + \beta S_y + \gamma
\end{equation}

or

\begin{equation}\label{eqn:linear_relationship_for_q}
    \tilde q=\alpha S_x + \beta S_y + \gamma
\end{equation}

where \(\tilde p = (1 - p_1, 1 - p_2, p_3, p_4)\) and
\(\tilde q = (1 - q_1, 1 - q_2, q_3, q_4)\) then:

\begin{equation}
    \alpha S_X + \beta S_Y + \gamma = 0
\end{equation}

In~\cite{Press2012} a particular type of ZD strategy is defined: extortionate
strategies. If:

\begin{equation}\label{eqn:constraint_for_extortion}
    \gamma = - P(\alpha + \beta)
\end{equation}

then the player can ensure they get a score \(\chi\) times
larger than the opponent. This extortion coefficient is given by:

\begin{equation}\label{eqn:definition_of_chi}
    \chi=\frac{-\beta}{\alpha}
\end{equation}

Thus, if (\ref{eqn:constraint_for_extortion}) holds and \(\chi >1\) a player is
said to extort their opponent.
Here, the reverse problem is considered: given a
\(p\in\mathbb{R}^4\) how does one identify \(\alpha, \beta\) if they
exist and is the strategy in fact acting in an extortionate way?

These conditions correspond to:

\begin{align}
    \tilde p_1 & = \alpha R + \beta R - P (\alpha + \beta)
            \label{eqn:condition_for_tilde_p1}\\
    \tilde p_2 & = \alpha S + \beta T - P (\alpha + \beta)
            \label{eqn:condition_for_tilde_p2}\\
    \tilde p_3 & = \alpha T + \beta S - P (\alpha + \beta)
            \label{eqn:condition_for_tilde_p3}\\
    \tilde p_4 & = \alpha P + \beta P - P (\alpha + \beta)
            \label{eqn:condition_for_tilde_p4}
\end{align}

Equation (\ref{eqn:condition_for_tilde_p4}) ensures that \(p_4=\tilde p_4=0\).
Equations (\ref{eqn:condition_for_tilde_p1}-\ref{eqn:condition_for_tilde_p3})
can be used to eliminate \(\alpha, \beta\), giving:

\begin{equation}\label{eqn:planar_definition_of_extortion}
    \tilde p_1 = \frac{(R - P)(\tilde p_2 + \tilde p_3)}{S + T - 2P}
\end{equation}

with:

\begin{equation}\label{eqn:definition_of_chi}
    \chi = \frac{\tilde p_2 (P - T) + \tilde p_3 (S - P)}
                {\tilde p_2 (P - S) + \tilde p_3 (T - P)}
\end{equation}

Given a strategy \(p\in\mathbb{R}^{4\times 1}\) equations
(\ref{eqn:condition_for_tilde_p4}), (\ref{eqn:planar_definition_of_extortion}-\ref{eqn:definition_of_chi}) can be used to check if
a strategy is extortionate. The conditions correspond to:

\begin{align}
    p_1 & = \frac{(R-P)(p_2 + p_3) - R + T + S - P}{S + T - 2P}
     \label{eqn:condition_for_p1}\\
    p_4 & = 0 \label{eqn:condition_for_p4}\\
    1 & > p_2 + p_3\label{eqn:condition_for_chi}
\end{align}

The algebraic steps necessary to prove these results are available in the
supporting materials.

All extortionate strategies reside on a triangular (\ref{eqn:condition_for_chi})
plane (\ref{eqn:condition_for_p1}) in 3 dimensions (\ref{eqn:condition_for_p4}).
Using this formulation it can be seen that a necessary (but not sufficient)
condition for an extortionate strategy is that it cooperates on average less
than 50\% of the time when in a state of disagreement with the opponent.

As an example, consider the known extortionate strategy \(p=(8 / 9, 1 / 2, 1 /
3, 0)\) from~\cite{Stewart2012} which is referred to as \texttt{Extort-2}. In
this case, for the standard values of \((R, T, S, P)\) constraint
(\ref{eqn:condition_for_p1}) corresponds to:

\begin{equation}
    p_1 = \frac{2(p_2 + p_3) + 1}{3}
\end{equation}

It is clear that in this case all constraints hold.

This approach could in fact be used to confirm that a given strategy is acting
in an extortionate manner even if it is not a memory one strategy. However, in
practice, if a closed form for \(p\) is not known, then due to measurement
and/or numerical error this would not work.

This problem can be written in the following linear algebraic form where
\(x=(\alpha, \beta)\)
and \(p^*=(\tilde p_1 - 1, tilde_2 - 1, p_3)\):

\begin{equation}\label{eqn:linear_algebraic_equation_for_p}
    Cx= p^*
\end{equation}

\(C\) corresponds to equations
(\ref{eqn:condition_for_tilde_p1}-\ref{eqn:condition_for_tilde_p3}) and is
given by:

\begin{equation}\label{eqn:definition_of_C}
    C =
    \begin{bmatrix}
        R - P & R- P \\
        S - P & T- P \\
        T - P & S- P \\
    \end{bmatrix}
\end{equation}

Note that in general, equation (\ref{eqn:linear_algebraic_equation_for_p}) will
not necessarily have a solution. From the Rouch\'{e}-Capelli theorem if there is
a solution it is unique as \(\text{rank}(C)=2\) which is the dimension of the
variable \(x\). The best fitting \(x\) is found by minimizing:

\begin{equation}\label{eqn:r_squared}
    \text{SSError} = \|C x- p^*\|_2^2 = \sum_{i=1}^{3}\left((C\bar x)_i-p_i^*\right)^2
\end{equation}

Note that \(\text{SSError}\), which is the square of the Frobenius
norm~\cite{Golub2013}, becomes a measure of how close a strategy is to being an
extortionate strategy. Suspicion
of extortion then corresponds to a threshold on \(\text{SSError}\).

By observing interactions (human or otherwise), their memory one representation
can be inferred and this approach can be used to recognise extortionate
behaviour. The notion of comparing theoretic and actual plays of the IPD is not
novel, see for example~\cite{Rand2013}. Immediately it is noted that if the
environment is noisy~\cite{Wu1995} then no strategy can be considered to be
extortionate as \(p_4>0\).

In the next section, this idea will be illustrated by observing the interactions
that take place in a computer based tournament of the IPD\@.

\section{Numerical experiments}\label{sec:numerical-experiments}

In~\cite{Stewart2012} results from a tournament with
\documentclass[a4paper]{article}

\usepackage{amsmath}
\usepackage{amssymb}
\usepackage[margin=1.5cm,
            includefoot,
            footskip=30pt]{geometry}
\usepackage{layout}
\usepackage{graphicx}
\usepackage{subcaption}

\usepackage{biblatex}
\usepackage{pdfpages}

\bibliography{main.bib}

\title{Suspicion: Recognising and evaluating the effectiveness
       of extortion in the Iterated Prisoner's Dilemma}
\author{Vincent A. Knight \and Nikoleta E. Glynatsi}
\date{\today}



\begin{document}

\maketitle

\begin{abstract}
    The Iterated Prisoner's Dilemma is a model for rational and evolutionary
    interactive behaviour. It has applications both in the study of human social
    behaviour as well as in biology.
    It is used to understand when and how a rational individual might
    accept an immediate cost to their own utility for the direct benefit of
    another.

    Much attention has been given to a class of strategies called
    Zero Determinant strategies. It has been theoretically shown that these
    strategies can ``extort'' any player.

    In this work, an approach to identify if observed strategies are playing in
    an extortionate way is described. Furthermore, experimental analysis of
    a large tournament with \input{assets/tex/number_of_full_strategies/main.tex}
    strategies is considered. In this setting
    the most highly performing strategies do not play in an extortionate way
    against each other but do against lower performing strategies.
    This suggests that whilst the theory of Zero Determinant strategies
    indicates that memory is not of fundamental importance to the evolution of
    cooperative behaviour, this is incomplete.
\end{abstract}

\section{Introduction}\label{sec:introduction}

Agent based game theoretic models have become a stalwart of the underpinning
mathematics of interactive behaviours. One of the major pieces of work
in this area is the pair of original computer tournaments run by Robert
Axelrod~\cite{Axelrod1980, Axelrod1980a}. These tournaments pitted submitted
computer strategies against each other in plays of the Iterated Prisoner's
Dilemma. A common game where agents can choose to pay a slight cost to their
immediate utility in the hope of building a reputation. This has been used in
economic and evolutionary game theory to understand the evolution of cooperative
behaviour.

Recently, a class of strategies was described in~\cite{Press2012} that can
provably extort any given opponent. In~\cite{Hilbe2013, Moran1707} some
questions have already been asked about the true effectiveness of these
strategies in an evolutionary setting. Here another question is asked: is it
possible to recognise this extortionate behaviour? A mathematical procedure for
suspicion is presented: in the same way that the continued actions of an
extortionate individual might raise suspicion.

This work makes use of the Axelrod Python library~\cite{Knight2018, Knight2016}
with a large number of Prisoner Dilemma strategies available to give an
extensive numerical example of the ideas presented.  The approach is presented
in Section~\ref{sec:delta-zd-strategies}.  All of the code and data discussed
in Section~\ref{sec:numerical-experiments} is open sourced, archived and
written according to best scientific principles~\cite{Wilson2014}. The data
archive can be found at~\cite{vincent_knight_2018_1297075}.

\section{Recognising Extortion}\label{sec:delta-zd-strategies}

In~\cite{Press2012}, given a match between 2 memory-one strategies, the concept
of Zero Determinant (ZD) strategies is introduced. The main result of that paper
shows that given two memory one players \(p, q\in\mathbb{R}^4\) a linear
relationship between the players' scores could be forced by one of the players.

Using the notation of~\cite{Press2012}, assuming the utilities for player \(p\)
are given by \(S_x=(R, S, T, P)\) and for player \(q\) by \(S_y=(R, T, S, P)\)
and that the stationary scores of each player is given by \(S_X\) and \(S_Y\)
respectively. The main result of~\cite{Press2012} is that if

\begin{equation}\label{eqn:linear_relationship_for_p}
    \tilde p=\alpha S_x + \beta S_y + \gamma
\end{equation}

or

\begin{equation}\label{eqn:linear_relationship_for_q}
    \tilde q=\alpha S_x + \beta S_y + \gamma
\end{equation}

where \(\tilde p = (1 - p_1, 1 - p_2, p_3, p_4)\) and
\(\tilde q = (1 - q_1, 1 - q_2, q_3, q_4)\) then:

\begin{equation}
    \alpha S_X + \beta S_Y + \gamma = 0
\end{equation}

In~\cite{Press2012} a particular type of ZD strategy is defined: extortionate
strategies. If:

\begin{equation}\label{eqn:constraint_for_extortion}
    \gamma = - P(\alpha + \beta)
\end{equation}

then the player can ensure they get a score \(\chi\) times
larger than the opponent. This extortion coefficient is given by:

\begin{equation}\label{eqn:definition_of_chi}
    \chi=\frac{-\beta}{\alpha}
\end{equation}

Thus, if (\ref{eqn:constraint_for_extortion}) holds and \(\chi >1\) a player is
said to extort their opponent.
Here, the reverse problem is considered: given a
\(p\in\mathbb{R}^4\) how does one identify \(\alpha, \beta\) if they
exist and is the strategy in fact acting in an extortionate way?

These conditions correspond to:

\begin{align}
    \tilde p_1 & = \alpha R + \beta R - P (\alpha + \beta)
            \label{eqn:condition_for_tilde_p1}\\
    \tilde p_2 & = \alpha S + \beta T - P (\alpha + \beta)
            \label{eqn:condition_for_tilde_p2}\\
    \tilde p_3 & = \alpha T + \beta S - P (\alpha + \beta)
            \label{eqn:condition_for_tilde_p3}\\
    \tilde p_4 & = \alpha P + \beta P - P (\alpha + \beta)
            \label{eqn:condition_for_tilde_p4}
\end{align}

Equation (\ref{eqn:condition_for_tilde_p4}) ensures that \(p_4=\tilde p_4=0\).
Equations (\ref{eqn:condition_for_tilde_p1}-\ref{eqn:condition_for_tilde_p3})
can be used to eliminate \(\alpha, \beta\), giving:

\begin{equation}\label{eqn:planar_definition_of_extortion}
    \tilde p_1 = \frac{(R - P)(\tilde p_2 + \tilde p_3)}{S + T - 2P}
\end{equation}

with:

\begin{equation}\label{eqn:definition_of_chi}
    \chi = \frac{\tilde p_2 (P - T) + \tilde p_3 (S - P)}
                {\tilde p_2 (P - S) + \tilde p_3 (T - P)}
\end{equation}

Given a strategy \(p\in\mathbb{R}^{4\times 1}\) equations
(\ref{eqn:condition_for_tilde_p4}), (\ref{eqn:planar_definition_of_extortion}-\ref{eqn:definition_of_chi}) can be used to check if
a strategy is extortionate. The conditions correspond to:

\begin{align}
    p_1 & = \frac{(R-P)(p_2 + p_3) - R + T + S - P}{S + T - 2P}
     \label{eqn:condition_for_p1}\\
    p_4 & = 0 \label{eqn:condition_for_p4}\\
    1 & > p_2 + p_3\label{eqn:condition_for_chi}
\end{align}

The algebraic steps necessary to prove these results are available in the
supporting materials.

All extortionate strategies reside on a triangular (\ref{eqn:condition_for_chi})
plane (\ref{eqn:condition_for_p1}) in 3 dimensions (\ref{eqn:condition_for_p4}).
Using this formulation it can be seen that a necessary (but not sufficient)
condition for an extortionate strategy is that it cooperates on average less
than 50\% of the time when in a state of disagreement with the opponent.

As an example, consider the known extortionate strategy \(p=(8 / 9, 1 / 2, 1 /
3, 0)\) from~\cite{Stewart2012} which is referred to as \texttt{Extort-2}. In
this case, for the standard values of \((R, T, S, P)\) constraint
(\ref{eqn:condition_for_p1}) corresponds to:

\begin{equation}
    p_1 = \frac{2(p_2 + p_3) + 1}{3}
\end{equation}

It is clear that in this case all constraints hold.

This approach could in fact be used to confirm that a given strategy is acting
in an extortionate manner even if it is not a memory one strategy. However, in
practice, if a closed form for \(p\) is not known, then due to measurement
and/or numerical error this would not work.

This problem can be written in the following linear algebraic form where
\(x=(\alpha, \beta)\)
and \(p^*=(\tilde p_1 - 1, tilde_2 - 1, p_3)\):

\begin{equation}\label{eqn:linear_algebraic_equation_for_p}
    Cx= p^*
\end{equation}

\(C\) corresponds to equations
(\ref{eqn:condition_for_tilde_p1}-\ref{eqn:condition_for_tilde_p3}) and is
given by:

\begin{equation}\label{eqn:definition_of_C}
    C =
    \begin{bmatrix}
        R - P & R- P \\
        S - P & T- P \\
        T - P & S- P \\
    \end{bmatrix}
\end{equation}

Note that in general, equation (\ref{eqn:linear_algebraic_equation_for_p}) will
not necessarily have a solution. From the Rouch\'{e}-Capelli theorem if there is
a solution it is unique as \(\text{rank}(C)=2\) which is the dimension of the
variable \(x\). The best fitting \(x\) is found by minimizing:

\begin{equation}\label{eqn:r_squared}
    \text{SSError} = \|C x- p^*\|_2^2 = \sum_{i=1}^{3}\left((C\bar x)_i-p_i^*\right)^2
\end{equation}

Note that \(\text{SSError}\), which is the square of the Frobenius
norm~\cite{Golub2013}, becomes a measure of how close a strategy is to being an
extortionate strategy. Suspicion
of extortion then corresponds to a threshold on \(\text{SSError}\).

By observing interactions (human or otherwise), their memory one representation
can be inferred and this approach can be used to recognise extortionate
behaviour. The notion of comparing theoretic and actual plays of the IPD is not
novel, see for example~\cite{Rand2013}. Immediately it is noted that if the
environment is noisy~\cite{Wu1995} then no strategy can be considered to be
extortionate as \(p_4>0\).

In the next section, this idea will be illustrated by observing the interactions
that take place in a computer based tournament of the IPD\@.

\section{Numerical experiments}\label{sec:numerical-experiments}

In~\cite{Stewart2012} results from a tournament with
\input{./assets/tex/number_of_stewart_plotkin_strategies/main.tex} strategies,
was presented with specific consideration given to ZD strategies. This
tournament is reproduced here using the Axelrod-Python
project~\cite{Knight2016}. To obtain a good measure of the corresponding
transition rates for each strategy all matches have been run for
\input{assets/tex/number_of_turns/main.tex} turns and every match has been
repeated \input{assets/tex/number_of_repetitions/main.tex} times. All of this
interaction data is available at~\cite{vincent_knight_2018_1297075}. A good
match between the inferred Markov chain and the state distribution of the actual
interactions has been verified. Data for this is presented in the supplementary
materials.

Figure~\ref{fig:SSError_overall_in_stewart_plotkin} shows the \(\text{SSError}\)
values for all the strategies in the tournament, as reported
in~\cite{Stewart2012} the extortionate strategy (which has an expected
\(\text{SSError}\) approximately 0) gains a large number of wins.

\begin{figure}[!htbp]
    \centering
    \includegraphics[width=.8\textwidth]{./assets/img/SSError_overall_in_stewart_plotkin/main.pdf}
    \caption{\(\text{SSError}\) and state probabilities for the strategies
        of~\cite{Stewart2012}, ordered both by number of wins and overall score.
        Note that \(P(DC)\) is not shown as it corresponds to the transpose of
        \(P(CD)\). Cooperator and Defector are omitted as they do not visit all
        the states.}
    \label{fig:SSError_overall_in_stewart_plotkin}
\end{figure}

Here, the work of~\cite{Stewart2012} is extended by investigating a tournament
with \input{assets/tex/number_of_full_strategies/main.tex}
strategies.

The results of this analysis are shown in
Figure~\ref{fig:SSError_and_probabilities_in_full}. The top ranking strategies
by number of wins seem to be extortionate (but not against all strategies) and
it can be seen that a small sub group of strategies achieve mutual defection.
All the top ranking strategies according to score achieve mutual cooperation and
do not extort each other, however they
\textbf{do} exhibit extortionate behaviour towards a number of the lower ranking
strategies.

\begin{figure}[!htbp]
    \centering
    \includegraphics[width=.8\textwidth]{./assets/img/SSError_and_probabilities_in_full/main.pdf}
    \caption{\(\text{SSError}\) for the strategies for the full tournament. Only
    strategy interactions for which \(p_4=0\) and \(\chi>1\) are displayed.}
    \label{fig:SSError_and_probabilities_in_full}
\end{figure}

\section{Conclusion}\label{sec:conclusion}

This work defines an approach to measure whether or not a player is playing a
strategy that corresponds to an extortionate strategy as defined
in~\cite{Press2012}: a mathematical model for suspicion. Indeed, all
extortionate strategies have been
 classified as lying on a triangular plane.
This rigorous classification fails to be robust to small measurement error, thus
a statistical approach is proposed.
This is done through a linear algebraic approach for approximating the solution
of a linear system. Using this, a large number of pairwise interactions is
simulated and in fact very few strategies are found to act extortionately.

The work of~\cite{Press2012}, whilst showing that a clever approach to taking
advantage of another memory one strategy exists: this is incomplete. Whilst the
elegance of this result is very attractive, just as the simplicity of the
victory of Tit For Tat in Axelrod's original tournaments was, it is incomplete.
Extortionate strategies achieve a high number of wins but they do not
achieve a high score which corresponds to the fitness landscape in an
evolutionary sense. From the large number of interactions a payoff matrix \(S\)
can be measured where \(S_{ij}\) denotes the score (using standard values of
\((R, S, T, P) = (3, 0, 5, 1)\)) of the \(i\)th strategy
against the \(j\)th strategy. Using this, the replicator equation
describes the evolution of the system based on a population density fitness
function:

\begin{equation}\label{eqn:replicator_dynamics}
    \frac{dx}{dt} = x(S-x^TS x)
\end{equation}

Equation (\ref{eqn:replicator_dynamics}) is solved numerically through an
integration technique described in~\cite{Petzold1983} and
Figure~\ref{fig:replicator_dynamics} shows the evolution of the distribution of
the system: the various strategies are ranked by scores. It is clear to see that
only the high ranking strategies survive the evolutionary process (in fact,
only \input{./assets/img/replicator_dynamics/main.tex}
have a final distribution greater than \(10 ^ {-2}\)). This confirms the
findings of~\cite{Moran1707} in which sophisticated strategies resist
evolutionary invasion of shorter memory strategies. Recalling
Figure~\ref{fig:SSError_and_probabilities_in_full} this demonstrates that:

\begin{itemize}
    \item Cooperation emerges through the evolutionary process: the high scoring
        strategies do not exhibit extortionate behaviour towards each other.
    \item Extortionate strategies do not survive the evolutionary process.
\end{itemize}

\begin{figure}[!htbp]
    \centering
    \includegraphics[width=.8\textwidth]{./assets/img/replicator_dynamics/main.pdf}
    \caption{Numerical simulation of the replicator equation
    (\ref{eqn:replicator_dynamics}): strategies are ordered by score, only the strategies with a high score survive the evolutionary process.}
    \label{fig:replicator_dynamics}
\end{figure}

This work can be used to classify plays of the IPD\@: data can be collected from
actual interactions (in lab or in the field). Furthermore, this allows for a
classification method similar to the notion of fingerprinting presented
in~\cite{Ashlock2008}. Trained strategies can potentially be classified as
extortionate or not or it could be possible to even constrain the reinforcement
learning approaches that are becoming prevalent in the literature.
Alternatively, this mathematical approach for recognising extortion could be
used in sophisticated strategies to defend against invasion. Arguably, some of
the strategies considered here exhibit this behaviour, indeed as described
in~\cite{Harper2017}, the top ranking strategies in the full tournament are
obtained using evolutionary reinforcement learning techniques, thus, suspicion
of extortionate behaviour could in fact be an evolutionary trait.

\section*{Acknowledgements}

The following open source software libraries were used in this research:

\begin{itemize}
    \item The Axelrod ~\cite{Knight2016, Knight2018} library (IPD strategies and
        tournaments).
    \item The sympy library~\cite{Meurer2017} (verification of all symbolic
        calculations).
    \item The matplotlib~\cite{Droettboom2018} library (visualisation).
    \item The pandas~\cite{Structures2010}, dask~\cite{Dask2016} and
        NumPy~\cite{Oliphant2015} libraries (data manipulation).
    \item The SciPy~\cite{Jones2001} library (numerical integration of the
        replicator equation).
\end{itemize}

This work was performed using the computational facilities of the Advanced
Research Computing @ Cardiff (ARCCA) Division, Cardiff University.

\printbibliography

\newpage
\section*{Supplementary materials}

\includepdf{assets/pdf/proof_of_form_of_extortionate_strategies/main.pdf}

\newpage

Using the pair wise interactions the transition rates \(p,
q\) can be measured and the steady state probabilities inferred and compared to
the actual probabilities of each state.
This is done numerically by computing the singular eigenvector of the
matrix \(A\) \cite{Stewart2009}:

\[
    A =
    \begin{bmatrix}
        p_1 q_1 & p_1 (1 - q_1) & (1 - p_1) q_1 & (1 -p_1) (1 - q_1) \\
        p_2 q_2 & p_2 (1 - q_2) & (1 - p_2) q_2 & (1 -p_2) (1 - q_2) \\
        p_3 q_3 & p_3 (1 - q_3) & (1 - p_3) q_3 & (1 -p_3) (1 - q_3) \\
        p_4 q_4 & p_4 (1 - q_4) & (1 - p_4) q_4 & (1 -p_4) (1 - q_4) \\
    \end{bmatrix}
\]

Figure~\ref{fig:computed_probabilities_vs_theoretic_probabilities} shows a
regression line fitted to every pairwise interaction with a reported
\(\text{SSError}\) value (pairwise interactions with missing states were
omitted). This serves to validate the approach: a part from some edge cases the
relationship is consistent.

\begin{figure}[!htbp]
    \centering
    \includegraphics[width=.8\textwidth]{./assets/img/computed_probabilities_vs_theoretic_probabilities/main.pdf}
    \caption{The
        relationship between the steady state probabilities inferred from the
        measured transitions and the actual steady state probabilities. A linear
        regression line is included validating the approach.}
    \label{fig:computed_probabilities_vs_theoretic_probabilities}
\end{figure}


\end{document}
 strategies,
was presented with specific consideration given to ZD strategies. This
tournament is reproduced here using the Axelrod-Python
project~\cite{Knight2016}. To obtain a good measure of the corresponding
transition rates for each strategy all matches have been run for
\documentclass[a4paper]{article}

\usepackage{amsmath}
\usepackage{amssymb}
\usepackage[margin=1.5cm,
            includefoot,
            footskip=30pt]{geometry}
\usepackage{layout}
\usepackage{graphicx}
\usepackage{subcaption}

\usepackage{biblatex}
\usepackage{pdfpages}

\bibliography{main.bib}

\title{Suspicion: Recognising and evaluating the effectiveness
       of extortion in the Iterated Prisoner's Dilemma}
\author{Vincent A. Knight \and Nikoleta E. Glynatsi}
\date{\today}



\begin{document}

\maketitle

\begin{abstract}
    The Iterated Prisoner's Dilemma is a model for rational and evolutionary
    interactive behaviour. It has applications both in the study of human social
    behaviour as well as in biology.
    It is used to understand when and how a rational individual might
    accept an immediate cost to their own utility for the direct benefit of
    another.

    Much attention has been given to a class of strategies called
    Zero Determinant strategies. It has been theoretically shown that these
    strategies can ``extort'' any player.

    In this work, an approach to identify if observed strategies are playing in
    an extortionate way is described. Furthermore, experimental analysis of
    a large tournament with \input{assets/tex/number_of_full_strategies/main.tex}
    strategies is considered. In this setting
    the most highly performing strategies do not play in an extortionate way
    against each other but do against lower performing strategies.
    This suggests that whilst the theory of Zero Determinant strategies
    indicates that memory is not of fundamental importance to the evolution of
    cooperative behaviour, this is incomplete.
\end{abstract}

\section{Introduction}\label{sec:introduction}

Agent based game theoretic models have become a stalwart of the underpinning
mathematics of interactive behaviours. One of the major pieces of work
in this area is the pair of original computer tournaments run by Robert
Axelrod~\cite{Axelrod1980, Axelrod1980a}. These tournaments pitted submitted
computer strategies against each other in plays of the Iterated Prisoner's
Dilemma. A common game where agents can choose to pay a slight cost to their
immediate utility in the hope of building a reputation. This has been used in
economic and evolutionary game theory to understand the evolution of cooperative
behaviour.

Recently, a class of strategies was described in~\cite{Press2012} that can
provably extort any given opponent. In~\cite{Hilbe2013, Moran1707} some
questions have already been asked about the true effectiveness of these
strategies in an evolutionary setting. Here another question is asked: is it
possible to recognise this extortionate behaviour? A mathematical procedure for
suspicion is presented: in the same way that the continued actions of an
extortionate individual might raise suspicion.

This work makes use of the Axelrod Python library~\cite{Knight2018, Knight2016}
with a large number of Prisoner Dilemma strategies available to give an
extensive numerical example of the ideas presented.  The approach is presented
in Section~\ref{sec:delta-zd-strategies}.  All of the code and data discussed
in Section~\ref{sec:numerical-experiments} is open sourced, archived and
written according to best scientific principles~\cite{Wilson2014}. The data
archive can be found at~\cite{vincent_knight_2018_1297075}.

\section{Recognising Extortion}\label{sec:delta-zd-strategies}

In~\cite{Press2012}, given a match between 2 memory-one strategies, the concept
of Zero Determinant (ZD) strategies is introduced. The main result of that paper
shows that given two memory one players \(p, q\in\mathbb{R}^4\) a linear
relationship between the players' scores could be forced by one of the players.

Using the notation of~\cite{Press2012}, assuming the utilities for player \(p\)
are given by \(S_x=(R, S, T, P)\) and for player \(q\) by \(S_y=(R, T, S, P)\)
and that the stationary scores of each player is given by \(S_X\) and \(S_Y\)
respectively. The main result of~\cite{Press2012} is that if

\begin{equation}\label{eqn:linear_relationship_for_p}
    \tilde p=\alpha S_x + \beta S_y + \gamma
\end{equation}

or

\begin{equation}\label{eqn:linear_relationship_for_q}
    \tilde q=\alpha S_x + \beta S_y + \gamma
\end{equation}

where \(\tilde p = (1 - p_1, 1 - p_2, p_3, p_4)\) and
\(\tilde q = (1 - q_1, 1 - q_2, q_3, q_4)\) then:

\begin{equation}
    \alpha S_X + \beta S_Y + \gamma = 0
\end{equation}

In~\cite{Press2012} a particular type of ZD strategy is defined: extortionate
strategies. If:

\begin{equation}\label{eqn:constraint_for_extortion}
    \gamma = - P(\alpha + \beta)
\end{equation}

then the player can ensure they get a score \(\chi\) times
larger than the opponent. This extortion coefficient is given by:

\begin{equation}\label{eqn:definition_of_chi}
    \chi=\frac{-\beta}{\alpha}
\end{equation}

Thus, if (\ref{eqn:constraint_for_extortion}) holds and \(\chi >1\) a player is
said to extort their opponent.
Here, the reverse problem is considered: given a
\(p\in\mathbb{R}^4\) how does one identify \(\alpha, \beta\) if they
exist and is the strategy in fact acting in an extortionate way?

These conditions correspond to:

\begin{align}
    \tilde p_1 & = \alpha R + \beta R - P (\alpha + \beta)
            \label{eqn:condition_for_tilde_p1}\\
    \tilde p_2 & = \alpha S + \beta T - P (\alpha + \beta)
            \label{eqn:condition_for_tilde_p2}\\
    \tilde p_3 & = \alpha T + \beta S - P (\alpha + \beta)
            \label{eqn:condition_for_tilde_p3}\\
    \tilde p_4 & = \alpha P + \beta P - P (\alpha + \beta)
            \label{eqn:condition_for_tilde_p4}
\end{align}

Equation (\ref{eqn:condition_for_tilde_p4}) ensures that \(p_4=\tilde p_4=0\).
Equations (\ref{eqn:condition_for_tilde_p1}-\ref{eqn:condition_for_tilde_p3})
can be used to eliminate \(\alpha, \beta\), giving:

\begin{equation}\label{eqn:planar_definition_of_extortion}
    \tilde p_1 = \frac{(R - P)(\tilde p_2 + \tilde p_3)}{S + T - 2P}
\end{equation}

with:

\begin{equation}\label{eqn:definition_of_chi}
    \chi = \frac{\tilde p_2 (P - T) + \tilde p_3 (S - P)}
                {\tilde p_2 (P - S) + \tilde p_3 (T - P)}
\end{equation}

Given a strategy \(p\in\mathbb{R}^{4\times 1}\) equations
(\ref{eqn:condition_for_tilde_p4}), (\ref{eqn:planar_definition_of_extortion}-\ref{eqn:definition_of_chi}) can be used to check if
a strategy is extortionate. The conditions correspond to:

\begin{align}
    p_1 & = \frac{(R-P)(p_2 + p_3) - R + T + S - P}{S + T - 2P}
     \label{eqn:condition_for_p1}\\
    p_4 & = 0 \label{eqn:condition_for_p4}\\
    1 & > p_2 + p_3\label{eqn:condition_for_chi}
\end{align}

The algebraic steps necessary to prove these results are available in the
supporting materials.

All extortionate strategies reside on a triangular (\ref{eqn:condition_for_chi})
plane (\ref{eqn:condition_for_p1}) in 3 dimensions (\ref{eqn:condition_for_p4}).
Using this formulation it can be seen that a necessary (but not sufficient)
condition for an extortionate strategy is that it cooperates on average less
than 50\% of the time when in a state of disagreement with the opponent.

As an example, consider the known extortionate strategy \(p=(8 / 9, 1 / 2, 1 /
3, 0)\) from~\cite{Stewart2012} which is referred to as \texttt{Extort-2}. In
this case, for the standard values of \((R, T, S, P)\) constraint
(\ref{eqn:condition_for_p1}) corresponds to:

\begin{equation}
    p_1 = \frac{2(p_2 + p_3) + 1}{3}
\end{equation}

It is clear that in this case all constraints hold.

This approach could in fact be used to confirm that a given strategy is acting
in an extortionate manner even if it is not a memory one strategy. However, in
practice, if a closed form for \(p\) is not known, then due to measurement
and/or numerical error this would not work.

This problem can be written in the following linear algebraic form where
\(x=(\alpha, \beta)\)
and \(p^*=(\tilde p_1 - 1, tilde_2 - 1, p_3)\):

\begin{equation}\label{eqn:linear_algebraic_equation_for_p}
    Cx= p^*
\end{equation}

\(C\) corresponds to equations
(\ref{eqn:condition_for_tilde_p1}-\ref{eqn:condition_for_tilde_p3}) and is
given by:

\begin{equation}\label{eqn:definition_of_C}
    C =
    \begin{bmatrix}
        R - P & R- P \\
        S - P & T- P \\
        T - P & S- P \\
    \end{bmatrix}
\end{equation}

Note that in general, equation (\ref{eqn:linear_algebraic_equation_for_p}) will
not necessarily have a solution. From the Rouch\'{e}-Capelli theorem if there is
a solution it is unique as \(\text{rank}(C)=2\) which is the dimension of the
variable \(x\). The best fitting \(x\) is found by minimizing:

\begin{equation}\label{eqn:r_squared}
    \text{SSError} = \|C x- p^*\|_2^2 = \sum_{i=1}^{3}\left((C\bar x)_i-p_i^*\right)^2
\end{equation}

Note that \(\text{SSError}\), which is the square of the Frobenius
norm~\cite{Golub2013}, becomes a measure of how close a strategy is to being an
extortionate strategy. Suspicion
of extortion then corresponds to a threshold on \(\text{SSError}\).

By observing interactions (human or otherwise), their memory one representation
can be inferred and this approach can be used to recognise extortionate
behaviour. The notion of comparing theoretic and actual plays of the IPD is not
novel, see for example~\cite{Rand2013}. Immediately it is noted that if the
environment is noisy~\cite{Wu1995} then no strategy can be considered to be
extortionate as \(p_4>0\).

In the next section, this idea will be illustrated by observing the interactions
that take place in a computer based tournament of the IPD\@.

\section{Numerical experiments}\label{sec:numerical-experiments}

In~\cite{Stewart2012} results from a tournament with
\input{./assets/tex/number_of_stewart_plotkin_strategies/main.tex} strategies,
was presented with specific consideration given to ZD strategies. This
tournament is reproduced here using the Axelrod-Python
project~\cite{Knight2016}. To obtain a good measure of the corresponding
transition rates for each strategy all matches have been run for
\input{assets/tex/number_of_turns/main.tex} turns and every match has been
repeated \input{assets/tex/number_of_repetitions/main.tex} times. All of this
interaction data is available at~\cite{vincent_knight_2018_1297075}. A good
match between the inferred Markov chain and the state distribution of the actual
interactions has been verified. Data for this is presented in the supplementary
materials.

Figure~\ref{fig:SSError_overall_in_stewart_plotkin} shows the \(\text{SSError}\)
values for all the strategies in the tournament, as reported
in~\cite{Stewart2012} the extortionate strategy (which has an expected
\(\text{SSError}\) approximately 0) gains a large number of wins.

\begin{figure}[!htbp]
    \centering
    \includegraphics[width=.8\textwidth]{./assets/img/SSError_overall_in_stewart_plotkin/main.pdf}
    \caption{\(\text{SSError}\) and state probabilities for the strategies
        of~\cite{Stewart2012}, ordered both by number of wins and overall score.
        Note that \(P(DC)\) is not shown as it corresponds to the transpose of
        \(P(CD)\). Cooperator and Defector are omitted as they do not visit all
        the states.}
    \label{fig:SSError_overall_in_stewart_plotkin}
\end{figure}

Here, the work of~\cite{Stewart2012} is extended by investigating a tournament
with \input{assets/tex/number_of_full_strategies/main.tex}
strategies.

The results of this analysis are shown in
Figure~\ref{fig:SSError_and_probabilities_in_full}. The top ranking strategies
by number of wins seem to be extortionate (but not against all strategies) and
it can be seen that a small sub group of strategies achieve mutual defection.
All the top ranking strategies according to score achieve mutual cooperation and
do not extort each other, however they
\textbf{do} exhibit extortionate behaviour towards a number of the lower ranking
strategies.

\begin{figure}[!htbp]
    \centering
    \includegraphics[width=.8\textwidth]{./assets/img/SSError_and_probabilities_in_full/main.pdf}
    \caption{\(\text{SSError}\) for the strategies for the full tournament. Only
    strategy interactions for which \(p_4=0\) and \(\chi>1\) are displayed.}
    \label{fig:SSError_and_probabilities_in_full}
\end{figure}

\section{Conclusion}\label{sec:conclusion}

This work defines an approach to measure whether or not a player is playing a
strategy that corresponds to an extortionate strategy as defined
in~\cite{Press2012}: a mathematical model for suspicion. Indeed, all
extortionate strategies have been
 classified as lying on a triangular plane.
This rigorous classification fails to be robust to small measurement error, thus
a statistical approach is proposed.
This is done through a linear algebraic approach for approximating the solution
of a linear system. Using this, a large number of pairwise interactions is
simulated and in fact very few strategies are found to act extortionately.

The work of~\cite{Press2012}, whilst showing that a clever approach to taking
advantage of another memory one strategy exists: this is incomplete. Whilst the
elegance of this result is very attractive, just as the simplicity of the
victory of Tit For Tat in Axelrod's original tournaments was, it is incomplete.
Extortionate strategies achieve a high number of wins but they do not
achieve a high score which corresponds to the fitness landscape in an
evolutionary sense. From the large number of interactions a payoff matrix \(S\)
can be measured where \(S_{ij}\) denotes the score (using standard values of
\((R, S, T, P) = (3, 0, 5, 1)\)) of the \(i\)th strategy
against the \(j\)th strategy. Using this, the replicator equation
describes the evolution of the system based on a population density fitness
function:

\begin{equation}\label{eqn:replicator_dynamics}
    \frac{dx}{dt} = x(S-x^TS x)
\end{equation}

Equation (\ref{eqn:replicator_dynamics}) is solved numerically through an
integration technique described in~\cite{Petzold1983} and
Figure~\ref{fig:replicator_dynamics} shows the evolution of the distribution of
the system: the various strategies are ranked by scores. It is clear to see that
only the high ranking strategies survive the evolutionary process (in fact,
only \input{./assets/img/replicator_dynamics/main.tex}
have a final distribution greater than \(10 ^ {-2}\)). This confirms the
findings of~\cite{Moran1707} in which sophisticated strategies resist
evolutionary invasion of shorter memory strategies. Recalling
Figure~\ref{fig:SSError_and_probabilities_in_full} this demonstrates that:

\begin{itemize}
    \item Cooperation emerges through the evolutionary process: the high scoring
        strategies do not exhibit extortionate behaviour towards each other.
    \item Extortionate strategies do not survive the evolutionary process.
\end{itemize}

\begin{figure}[!htbp]
    \centering
    \includegraphics[width=.8\textwidth]{./assets/img/replicator_dynamics/main.pdf}
    \caption{Numerical simulation of the replicator equation
    (\ref{eqn:replicator_dynamics}): strategies are ordered by score, only the strategies with a high score survive the evolutionary process.}
    \label{fig:replicator_dynamics}
\end{figure}

This work can be used to classify plays of the IPD\@: data can be collected from
actual interactions (in lab or in the field). Furthermore, this allows for a
classification method similar to the notion of fingerprinting presented
in~\cite{Ashlock2008}. Trained strategies can potentially be classified as
extortionate or not or it could be possible to even constrain the reinforcement
learning approaches that are becoming prevalent in the literature.
Alternatively, this mathematical approach for recognising extortion could be
used in sophisticated strategies to defend against invasion. Arguably, some of
the strategies considered here exhibit this behaviour, indeed as described
in~\cite{Harper2017}, the top ranking strategies in the full tournament are
obtained using evolutionary reinforcement learning techniques, thus, suspicion
of extortionate behaviour could in fact be an evolutionary trait.

\section*{Acknowledgements}

The following open source software libraries were used in this research:

\begin{itemize}
    \item The Axelrod ~\cite{Knight2016, Knight2018} library (IPD strategies and
        tournaments).
    \item The sympy library~\cite{Meurer2017} (verification of all symbolic
        calculations).
    \item The matplotlib~\cite{Droettboom2018} library (visualisation).
    \item The pandas~\cite{Structures2010}, dask~\cite{Dask2016} and
        NumPy~\cite{Oliphant2015} libraries (data manipulation).
    \item The SciPy~\cite{Jones2001} library (numerical integration of the
        replicator equation).
\end{itemize}

This work was performed using the computational facilities of the Advanced
Research Computing @ Cardiff (ARCCA) Division, Cardiff University.

\printbibliography

\newpage
\section*{Supplementary materials}

\includepdf{assets/pdf/proof_of_form_of_extortionate_strategies/main.pdf}

\newpage

Using the pair wise interactions the transition rates \(p,
q\) can be measured and the steady state probabilities inferred and compared to
the actual probabilities of each state.
This is done numerically by computing the singular eigenvector of the
matrix \(A\) \cite{Stewart2009}:

\[
    A =
    \begin{bmatrix}
        p_1 q_1 & p_1 (1 - q_1) & (1 - p_1) q_1 & (1 -p_1) (1 - q_1) \\
        p_2 q_2 & p_2 (1 - q_2) & (1 - p_2) q_2 & (1 -p_2) (1 - q_2) \\
        p_3 q_3 & p_3 (1 - q_3) & (1 - p_3) q_3 & (1 -p_3) (1 - q_3) \\
        p_4 q_4 & p_4 (1 - q_4) & (1 - p_4) q_4 & (1 -p_4) (1 - q_4) \\
    \end{bmatrix}
\]

Figure~\ref{fig:computed_probabilities_vs_theoretic_probabilities} shows a
regression line fitted to every pairwise interaction with a reported
\(\text{SSError}\) value (pairwise interactions with missing states were
omitted). This serves to validate the approach: a part from some edge cases the
relationship is consistent.

\begin{figure}[!htbp]
    \centering
    \includegraphics[width=.8\textwidth]{./assets/img/computed_probabilities_vs_theoretic_probabilities/main.pdf}
    \caption{The
        relationship between the steady state probabilities inferred from the
        measured transitions and the actual steady state probabilities. A linear
        regression line is included validating the approach.}
    \label{fig:computed_probabilities_vs_theoretic_probabilities}
\end{figure}


\end{document}
 turns and every match has been
repeated \documentclass[a4paper]{article}

\usepackage{amsmath}
\usepackage{amssymb}
\usepackage[margin=1.5cm,
            includefoot,
            footskip=30pt]{geometry}
\usepackage{layout}
\usepackage{graphicx}
\usepackage{subcaption}

\usepackage{biblatex}
\usepackage{pdfpages}

\bibliography{main.bib}

\title{Suspicion: Recognising and evaluating the effectiveness
       of extortion in the Iterated Prisoner's Dilemma}
\author{Vincent A. Knight \and Nikoleta E. Glynatsi}
\date{\today}



\begin{document}

\maketitle

\begin{abstract}
    The Iterated Prisoner's Dilemma is a model for rational and evolutionary
    interactive behaviour. It has applications both in the study of human social
    behaviour as well as in biology.
    It is used to understand when and how a rational individual might
    accept an immediate cost to their own utility for the direct benefit of
    another.

    Much attention has been given to a class of strategies called
    Zero Determinant strategies. It has been theoretically shown that these
    strategies can ``extort'' any player.

    In this work, an approach to identify if observed strategies are playing in
    an extortionate way is described. Furthermore, experimental analysis of
    a large tournament with \input{assets/tex/number_of_full_strategies/main.tex}
    strategies is considered. In this setting
    the most highly performing strategies do not play in an extortionate way
    against each other but do against lower performing strategies.
    This suggests that whilst the theory of Zero Determinant strategies
    indicates that memory is not of fundamental importance to the evolution of
    cooperative behaviour, this is incomplete.
\end{abstract}

\section{Introduction}\label{sec:introduction}

Agent based game theoretic models have become a stalwart of the underpinning
mathematics of interactive behaviours. One of the major pieces of work
in this area is the pair of original computer tournaments run by Robert
Axelrod~\cite{Axelrod1980, Axelrod1980a}. These tournaments pitted submitted
computer strategies against each other in plays of the Iterated Prisoner's
Dilemma. A common game where agents can choose to pay a slight cost to their
immediate utility in the hope of building a reputation. This has been used in
economic and evolutionary game theory to understand the evolution of cooperative
behaviour.

Recently, a class of strategies was described in~\cite{Press2012} that can
provably extort any given opponent. In~\cite{Hilbe2013, Moran1707} some
questions have already been asked about the true effectiveness of these
strategies in an evolutionary setting. Here another question is asked: is it
possible to recognise this extortionate behaviour? A mathematical procedure for
suspicion is presented: in the same way that the continued actions of an
extortionate individual might raise suspicion.

This work makes use of the Axelrod Python library~\cite{Knight2018, Knight2016}
with a large number of Prisoner Dilemma strategies available to give an
extensive numerical example of the ideas presented.  The approach is presented
in Section~\ref{sec:delta-zd-strategies}.  All of the code and data discussed
in Section~\ref{sec:numerical-experiments} is open sourced, archived and
written according to best scientific principles~\cite{Wilson2014}. The data
archive can be found at~\cite{vincent_knight_2018_1297075}.

\section{Recognising Extortion}\label{sec:delta-zd-strategies}

In~\cite{Press2012}, given a match between 2 memory-one strategies, the concept
of Zero Determinant (ZD) strategies is introduced. The main result of that paper
shows that given two memory one players \(p, q\in\mathbb{R}^4\) a linear
relationship between the players' scores could be forced by one of the players.

Using the notation of~\cite{Press2012}, assuming the utilities for player \(p\)
are given by \(S_x=(R, S, T, P)\) and for player \(q\) by \(S_y=(R, T, S, P)\)
and that the stationary scores of each player is given by \(S_X\) and \(S_Y\)
respectively. The main result of~\cite{Press2012} is that if

\begin{equation}\label{eqn:linear_relationship_for_p}
    \tilde p=\alpha S_x + \beta S_y + \gamma
\end{equation}

or

\begin{equation}\label{eqn:linear_relationship_for_q}
    \tilde q=\alpha S_x + \beta S_y + \gamma
\end{equation}

where \(\tilde p = (1 - p_1, 1 - p_2, p_3, p_4)\) and
\(\tilde q = (1 - q_1, 1 - q_2, q_3, q_4)\) then:

\begin{equation}
    \alpha S_X + \beta S_Y + \gamma = 0
\end{equation}

In~\cite{Press2012} a particular type of ZD strategy is defined: extortionate
strategies. If:

\begin{equation}\label{eqn:constraint_for_extortion}
    \gamma = - P(\alpha + \beta)
\end{equation}

then the player can ensure they get a score \(\chi\) times
larger than the opponent. This extortion coefficient is given by:

\begin{equation}\label{eqn:definition_of_chi}
    \chi=\frac{-\beta}{\alpha}
\end{equation}

Thus, if (\ref{eqn:constraint_for_extortion}) holds and \(\chi >1\) a player is
said to extort their opponent.
Here, the reverse problem is considered: given a
\(p\in\mathbb{R}^4\) how does one identify \(\alpha, \beta\) if they
exist and is the strategy in fact acting in an extortionate way?

These conditions correspond to:

\begin{align}
    \tilde p_1 & = \alpha R + \beta R - P (\alpha + \beta)
            \label{eqn:condition_for_tilde_p1}\\
    \tilde p_2 & = \alpha S + \beta T - P (\alpha + \beta)
            \label{eqn:condition_for_tilde_p2}\\
    \tilde p_3 & = \alpha T + \beta S - P (\alpha + \beta)
            \label{eqn:condition_for_tilde_p3}\\
    \tilde p_4 & = \alpha P + \beta P - P (\alpha + \beta)
            \label{eqn:condition_for_tilde_p4}
\end{align}

Equation (\ref{eqn:condition_for_tilde_p4}) ensures that \(p_4=\tilde p_4=0\).
Equations (\ref{eqn:condition_for_tilde_p1}-\ref{eqn:condition_for_tilde_p3})
can be used to eliminate \(\alpha, \beta\), giving:

\begin{equation}\label{eqn:planar_definition_of_extortion}
    \tilde p_1 = \frac{(R - P)(\tilde p_2 + \tilde p_3)}{S + T - 2P}
\end{equation}

with:

\begin{equation}\label{eqn:definition_of_chi}
    \chi = \frac{\tilde p_2 (P - T) + \tilde p_3 (S - P)}
                {\tilde p_2 (P - S) + \tilde p_3 (T - P)}
\end{equation}

Given a strategy \(p\in\mathbb{R}^{4\times 1}\) equations
(\ref{eqn:condition_for_tilde_p4}), (\ref{eqn:planar_definition_of_extortion}-\ref{eqn:definition_of_chi}) can be used to check if
a strategy is extortionate. The conditions correspond to:

\begin{align}
    p_1 & = \frac{(R-P)(p_2 + p_3) - R + T + S - P}{S + T - 2P}
     \label{eqn:condition_for_p1}\\
    p_4 & = 0 \label{eqn:condition_for_p4}\\
    1 & > p_2 + p_3\label{eqn:condition_for_chi}
\end{align}

The algebraic steps necessary to prove these results are available in the
supporting materials.

All extortionate strategies reside on a triangular (\ref{eqn:condition_for_chi})
plane (\ref{eqn:condition_for_p1}) in 3 dimensions (\ref{eqn:condition_for_p4}).
Using this formulation it can be seen that a necessary (but not sufficient)
condition for an extortionate strategy is that it cooperates on average less
than 50\% of the time when in a state of disagreement with the opponent.

As an example, consider the known extortionate strategy \(p=(8 / 9, 1 / 2, 1 /
3, 0)\) from~\cite{Stewart2012} which is referred to as \texttt{Extort-2}. In
this case, for the standard values of \((R, T, S, P)\) constraint
(\ref{eqn:condition_for_p1}) corresponds to:

\begin{equation}
    p_1 = \frac{2(p_2 + p_3) + 1}{3}
\end{equation}

It is clear that in this case all constraints hold.

This approach could in fact be used to confirm that a given strategy is acting
in an extortionate manner even if it is not a memory one strategy. However, in
practice, if a closed form for \(p\) is not known, then due to measurement
and/or numerical error this would not work.

This problem can be written in the following linear algebraic form where
\(x=(\alpha, \beta)\)
and \(p^*=(\tilde p_1 - 1, tilde_2 - 1, p_3)\):

\begin{equation}\label{eqn:linear_algebraic_equation_for_p}
    Cx= p^*
\end{equation}

\(C\) corresponds to equations
(\ref{eqn:condition_for_tilde_p1}-\ref{eqn:condition_for_tilde_p3}) and is
given by:

\begin{equation}\label{eqn:definition_of_C}
    C =
    \begin{bmatrix}
        R - P & R- P \\
        S - P & T- P \\
        T - P & S- P \\
    \end{bmatrix}
\end{equation}

Note that in general, equation (\ref{eqn:linear_algebraic_equation_for_p}) will
not necessarily have a solution. From the Rouch\'{e}-Capelli theorem if there is
a solution it is unique as \(\text{rank}(C)=2\) which is the dimension of the
variable \(x\). The best fitting \(x\) is found by minimizing:

\begin{equation}\label{eqn:r_squared}
    \text{SSError} = \|C x- p^*\|_2^2 = \sum_{i=1}^{3}\left((C\bar x)_i-p_i^*\right)^2
\end{equation}

Note that \(\text{SSError}\), which is the square of the Frobenius
norm~\cite{Golub2013}, becomes a measure of how close a strategy is to being an
extortionate strategy. Suspicion
of extortion then corresponds to a threshold on \(\text{SSError}\).

By observing interactions (human or otherwise), their memory one representation
can be inferred and this approach can be used to recognise extortionate
behaviour. The notion of comparing theoretic and actual plays of the IPD is not
novel, see for example~\cite{Rand2013}. Immediately it is noted that if the
environment is noisy~\cite{Wu1995} then no strategy can be considered to be
extortionate as \(p_4>0\).

In the next section, this idea will be illustrated by observing the interactions
that take place in a computer based tournament of the IPD\@.

\section{Numerical experiments}\label{sec:numerical-experiments}

In~\cite{Stewart2012} results from a tournament with
\input{./assets/tex/number_of_stewart_plotkin_strategies/main.tex} strategies,
was presented with specific consideration given to ZD strategies. This
tournament is reproduced here using the Axelrod-Python
project~\cite{Knight2016}. To obtain a good measure of the corresponding
transition rates for each strategy all matches have been run for
\input{assets/tex/number_of_turns/main.tex} turns and every match has been
repeated \input{assets/tex/number_of_repetitions/main.tex} times. All of this
interaction data is available at~\cite{vincent_knight_2018_1297075}. A good
match between the inferred Markov chain and the state distribution of the actual
interactions has been verified. Data for this is presented in the supplementary
materials.

Figure~\ref{fig:SSError_overall_in_stewart_plotkin} shows the \(\text{SSError}\)
values for all the strategies in the tournament, as reported
in~\cite{Stewart2012} the extortionate strategy (which has an expected
\(\text{SSError}\) approximately 0) gains a large number of wins.

\begin{figure}[!htbp]
    \centering
    \includegraphics[width=.8\textwidth]{./assets/img/SSError_overall_in_stewart_plotkin/main.pdf}
    \caption{\(\text{SSError}\) and state probabilities for the strategies
        of~\cite{Stewart2012}, ordered both by number of wins and overall score.
        Note that \(P(DC)\) is not shown as it corresponds to the transpose of
        \(P(CD)\). Cooperator and Defector are omitted as they do not visit all
        the states.}
    \label{fig:SSError_overall_in_stewart_plotkin}
\end{figure}

Here, the work of~\cite{Stewart2012} is extended by investigating a tournament
with \input{assets/tex/number_of_full_strategies/main.tex}
strategies.

The results of this analysis are shown in
Figure~\ref{fig:SSError_and_probabilities_in_full}. The top ranking strategies
by number of wins seem to be extortionate (but not against all strategies) and
it can be seen that a small sub group of strategies achieve mutual defection.
All the top ranking strategies according to score achieve mutual cooperation and
do not extort each other, however they
\textbf{do} exhibit extortionate behaviour towards a number of the lower ranking
strategies.

\begin{figure}[!htbp]
    \centering
    \includegraphics[width=.8\textwidth]{./assets/img/SSError_and_probabilities_in_full/main.pdf}
    \caption{\(\text{SSError}\) for the strategies for the full tournament. Only
    strategy interactions for which \(p_4=0\) and \(\chi>1\) are displayed.}
    \label{fig:SSError_and_probabilities_in_full}
\end{figure}

\section{Conclusion}\label{sec:conclusion}

This work defines an approach to measure whether or not a player is playing a
strategy that corresponds to an extortionate strategy as defined
in~\cite{Press2012}: a mathematical model for suspicion. Indeed, all
extortionate strategies have been
 classified as lying on a triangular plane.
This rigorous classification fails to be robust to small measurement error, thus
a statistical approach is proposed.
This is done through a linear algebraic approach for approximating the solution
of a linear system. Using this, a large number of pairwise interactions is
simulated and in fact very few strategies are found to act extortionately.

The work of~\cite{Press2012}, whilst showing that a clever approach to taking
advantage of another memory one strategy exists: this is incomplete. Whilst the
elegance of this result is very attractive, just as the simplicity of the
victory of Tit For Tat in Axelrod's original tournaments was, it is incomplete.
Extortionate strategies achieve a high number of wins but they do not
achieve a high score which corresponds to the fitness landscape in an
evolutionary sense. From the large number of interactions a payoff matrix \(S\)
can be measured where \(S_{ij}\) denotes the score (using standard values of
\((R, S, T, P) = (3, 0, 5, 1)\)) of the \(i\)th strategy
against the \(j\)th strategy. Using this, the replicator equation
describes the evolution of the system based on a population density fitness
function:

\begin{equation}\label{eqn:replicator_dynamics}
    \frac{dx}{dt} = x(S-x^TS x)
\end{equation}

Equation (\ref{eqn:replicator_dynamics}) is solved numerically through an
integration technique described in~\cite{Petzold1983} and
Figure~\ref{fig:replicator_dynamics} shows the evolution of the distribution of
the system: the various strategies are ranked by scores. It is clear to see that
only the high ranking strategies survive the evolutionary process (in fact,
only \input{./assets/img/replicator_dynamics/main.tex}
have a final distribution greater than \(10 ^ {-2}\)). This confirms the
findings of~\cite{Moran1707} in which sophisticated strategies resist
evolutionary invasion of shorter memory strategies. Recalling
Figure~\ref{fig:SSError_and_probabilities_in_full} this demonstrates that:

\begin{itemize}
    \item Cooperation emerges through the evolutionary process: the high scoring
        strategies do not exhibit extortionate behaviour towards each other.
    \item Extortionate strategies do not survive the evolutionary process.
\end{itemize}

\begin{figure}[!htbp]
    \centering
    \includegraphics[width=.8\textwidth]{./assets/img/replicator_dynamics/main.pdf}
    \caption{Numerical simulation of the replicator equation
    (\ref{eqn:replicator_dynamics}): strategies are ordered by score, only the strategies with a high score survive the evolutionary process.}
    \label{fig:replicator_dynamics}
\end{figure}

This work can be used to classify plays of the IPD\@: data can be collected from
actual interactions (in lab or in the field). Furthermore, this allows for a
classification method similar to the notion of fingerprinting presented
in~\cite{Ashlock2008}. Trained strategies can potentially be classified as
extortionate or not or it could be possible to even constrain the reinforcement
learning approaches that are becoming prevalent in the literature.
Alternatively, this mathematical approach for recognising extortion could be
used in sophisticated strategies to defend against invasion. Arguably, some of
the strategies considered here exhibit this behaviour, indeed as described
in~\cite{Harper2017}, the top ranking strategies in the full tournament are
obtained using evolutionary reinforcement learning techniques, thus, suspicion
of extortionate behaviour could in fact be an evolutionary trait.

\section*{Acknowledgements}

The following open source software libraries were used in this research:

\begin{itemize}
    \item The Axelrod ~\cite{Knight2016, Knight2018} library (IPD strategies and
        tournaments).
    \item The sympy library~\cite{Meurer2017} (verification of all symbolic
        calculations).
    \item The matplotlib~\cite{Droettboom2018} library (visualisation).
    \item The pandas~\cite{Structures2010}, dask~\cite{Dask2016} and
        NumPy~\cite{Oliphant2015} libraries (data manipulation).
    \item The SciPy~\cite{Jones2001} library (numerical integration of the
        replicator equation).
\end{itemize}

This work was performed using the computational facilities of the Advanced
Research Computing @ Cardiff (ARCCA) Division, Cardiff University.

\printbibliography

\newpage
\section*{Supplementary materials}

\includepdf{assets/pdf/proof_of_form_of_extortionate_strategies/main.pdf}

\newpage

Using the pair wise interactions the transition rates \(p,
q\) can be measured and the steady state probabilities inferred and compared to
the actual probabilities of each state.
This is done numerically by computing the singular eigenvector of the
matrix \(A\) \cite{Stewart2009}:

\[
    A =
    \begin{bmatrix}
        p_1 q_1 & p_1 (1 - q_1) & (1 - p_1) q_1 & (1 -p_1) (1 - q_1) \\
        p_2 q_2 & p_2 (1 - q_2) & (1 - p_2) q_2 & (1 -p_2) (1 - q_2) \\
        p_3 q_3 & p_3 (1 - q_3) & (1 - p_3) q_3 & (1 -p_3) (1 - q_3) \\
        p_4 q_4 & p_4 (1 - q_4) & (1 - p_4) q_4 & (1 -p_4) (1 - q_4) \\
    \end{bmatrix}
\]

Figure~\ref{fig:computed_probabilities_vs_theoretic_probabilities} shows a
regression line fitted to every pairwise interaction with a reported
\(\text{SSError}\) value (pairwise interactions with missing states were
omitted). This serves to validate the approach: a part from some edge cases the
relationship is consistent.

\begin{figure}[!htbp]
    \centering
    \includegraphics[width=.8\textwidth]{./assets/img/computed_probabilities_vs_theoretic_probabilities/main.pdf}
    \caption{The
        relationship between the steady state probabilities inferred from the
        measured transitions and the actual steady state probabilities. A linear
        regression line is included validating the approach.}
    \label{fig:computed_probabilities_vs_theoretic_probabilities}
\end{figure}


\end{document}
 times. All of this
interaction data is available at~\cite{vincent_knight_2018_1297075}. A good
match between the inferred Markov chain and the state distribution of the actual
interactions has been verified. Data for this is presented in the supplementary
materials.

Figure~\ref{fig:SSError_overall_in_stewart_plotkin} shows the \(\text{SSError}\)
values for all the strategies in the tournament, as reported
in~\cite{Stewart2012} the extortionate strategy (which has an expected
\(\text{SSError}\) approximately 0) gains a large number of wins.

\begin{figure}[!htbp]
    \centering
    \includegraphics[width=.8\textwidth]{./assets/img/SSError_overall_in_stewart_plotkin/main.pdf}
    \caption{\(\text{SSError}\) and state probabilities for the strategies
        of~\cite{Stewart2012}, ordered both by number of wins and overall score.
        Note that \(P(DC)\) is not shown as it corresponds to the transpose of
        \(P(CD)\). Cooperator and Defector are omitted as they do not visit all
        the states.}
    \label{fig:SSError_overall_in_stewart_plotkin}
\end{figure}

Here, the work of~\cite{Stewart2012} is extended by investigating a tournament
with \documentclass[a4paper]{article}

\usepackage{amsmath}
\usepackage{amssymb}
\usepackage[margin=1.5cm,
            includefoot,
            footskip=30pt]{geometry}
\usepackage{layout}
\usepackage{graphicx}
\usepackage{subcaption}

\usepackage{biblatex}
\usepackage{pdfpages}

\bibliography{main.bib}

\title{Suspicion: Recognising and evaluating the effectiveness
       of extortion in the Iterated Prisoner's Dilemma}
\author{Vincent A. Knight \and Nikoleta E. Glynatsi}
\date{\today}



\begin{document}

\maketitle

\begin{abstract}
    The Iterated Prisoner's Dilemma is a model for rational and evolutionary
    interactive behaviour. It has applications both in the study of human social
    behaviour as well as in biology.
    It is used to understand when and how a rational individual might
    accept an immediate cost to their own utility for the direct benefit of
    another.

    Much attention has been given to a class of strategies called
    Zero Determinant strategies. It has been theoretically shown that these
    strategies can ``extort'' any player.

    In this work, an approach to identify if observed strategies are playing in
    an extortionate way is described. Furthermore, experimental analysis of
    a large tournament with \input{assets/tex/number_of_full_strategies/main.tex}
    strategies is considered. In this setting
    the most highly performing strategies do not play in an extortionate way
    against each other but do against lower performing strategies.
    This suggests that whilst the theory of Zero Determinant strategies
    indicates that memory is not of fundamental importance to the evolution of
    cooperative behaviour, this is incomplete.
\end{abstract}

\section{Introduction}\label{sec:introduction}

Agent based game theoretic models have become a stalwart of the underpinning
mathematics of interactive behaviours. One of the major pieces of work
in this area is the pair of original computer tournaments run by Robert
Axelrod~\cite{Axelrod1980, Axelrod1980a}. These tournaments pitted submitted
computer strategies against each other in plays of the Iterated Prisoner's
Dilemma. A common game where agents can choose to pay a slight cost to their
immediate utility in the hope of building a reputation. This has been used in
economic and evolutionary game theory to understand the evolution of cooperative
behaviour.

Recently, a class of strategies was described in~\cite{Press2012} that can
provably extort any given opponent. In~\cite{Hilbe2013, Moran1707} some
questions have already been asked about the true effectiveness of these
strategies in an evolutionary setting. Here another question is asked: is it
possible to recognise this extortionate behaviour? A mathematical procedure for
suspicion is presented: in the same way that the continued actions of an
extortionate individual might raise suspicion.

This work makes use of the Axelrod Python library~\cite{Knight2018, Knight2016}
with a large number of Prisoner Dilemma strategies available to give an
extensive numerical example of the ideas presented.  The approach is presented
in Section~\ref{sec:delta-zd-strategies}.  All of the code and data discussed
in Section~\ref{sec:numerical-experiments} is open sourced, archived and
written according to best scientific principles~\cite{Wilson2014}. The data
archive can be found at~\cite{vincent_knight_2018_1297075}.

\section{Recognising Extortion}\label{sec:delta-zd-strategies}

In~\cite{Press2012}, given a match between 2 memory-one strategies, the concept
of Zero Determinant (ZD) strategies is introduced. The main result of that paper
shows that given two memory one players \(p, q\in\mathbb{R}^4\) a linear
relationship between the players' scores could be forced by one of the players.

Using the notation of~\cite{Press2012}, assuming the utilities for player \(p\)
are given by \(S_x=(R, S, T, P)\) and for player \(q\) by \(S_y=(R, T, S, P)\)
and that the stationary scores of each player is given by \(S_X\) and \(S_Y\)
respectively. The main result of~\cite{Press2012} is that if

\begin{equation}\label{eqn:linear_relationship_for_p}
    \tilde p=\alpha S_x + \beta S_y + \gamma
\end{equation}

or

\begin{equation}\label{eqn:linear_relationship_for_q}
    \tilde q=\alpha S_x + \beta S_y + \gamma
\end{equation}

where \(\tilde p = (1 - p_1, 1 - p_2, p_3, p_4)\) and
\(\tilde q = (1 - q_1, 1 - q_2, q_3, q_4)\) then:

\begin{equation}
    \alpha S_X + \beta S_Y + \gamma = 0
\end{equation}

In~\cite{Press2012} a particular type of ZD strategy is defined: extortionate
strategies. If:

\begin{equation}\label{eqn:constraint_for_extortion}
    \gamma = - P(\alpha + \beta)
\end{equation}

then the player can ensure they get a score \(\chi\) times
larger than the opponent. This extortion coefficient is given by:

\begin{equation}\label{eqn:definition_of_chi}
    \chi=\frac{-\beta}{\alpha}
\end{equation}

Thus, if (\ref{eqn:constraint_for_extortion}) holds and \(\chi >1\) a player is
said to extort their opponent.
Here, the reverse problem is considered: given a
\(p\in\mathbb{R}^4\) how does one identify \(\alpha, \beta\) if they
exist and is the strategy in fact acting in an extortionate way?

These conditions correspond to:

\begin{align}
    \tilde p_1 & = \alpha R + \beta R - P (\alpha + \beta)
            \label{eqn:condition_for_tilde_p1}\\
    \tilde p_2 & = \alpha S + \beta T - P (\alpha + \beta)
            \label{eqn:condition_for_tilde_p2}\\
    \tilde p_3 & = \alpha T + \beta S - P (\alpha + \beta)
            \label{eqn:condition_for_tilde_p3}\\
    \tilde p_4 & = \alpha P + \beta P - P (\alpha + \beta)
            \label{eqn:condition_for_tilde_p4}
\end{align}

Equation (\ref{eqn:condition_for_tilde_p4}) ensures that \(p_4=\tilde p_4=0\).
Equations (\ref{eqn:condition_for_tilde_p1}-\ref{eqn:condition_for_tilde_p3})
can be used to eliminate \(\alpha, \beta\), giving:

\begin{equation}\label{eqn:planar_definition_of_extortion}
    \tilde p_1 = \frac{(R - P)(\tilde p_2 + \tilde p_3)}{S + T - 2P}
\end{equation}

with:

\begin{equation}\label{eqn:definition_of_chi}
    \chi = \frac{\tilde p_2 (P - T) + \tilde p_3 (S - P)}
                {\tilde p_2 (P - S) + \tilde p_3 (T - P)}
\end{equation}

Given a strategy \(p\in\mathbb{R}^{4\times 1}\) equations
(\ref{eqn:condition_for_tilde_p4}), (\ref{eqn:planar_definition_of_extortion}-\ref{eqn:definition_of_chi}) can be used to check if
a strategy is extortionate. The conditions correspond to:

\begin{align}
    p_1 & = \frac{(R-P)(p_2 + p_3) - R + T + S - P}{S + T - 2P}
     \label{eqn:condition_for_p1}\\
    p_4 & = 0 \label{eqn:condition_for_p4}\\
    1 & > p_2 + p_3\label{eqn:condition_for_chi}
\end{align}

The algebraic steps necessary to prove these results are available in the
supporting materials.

All extortionate strategies reside on a triangular (\ref{eqn:condition_for_chi})
plane (\ref{eqn:condition_for_p1}) in 3 dimensions (\ref{eqn:condition_for_p4}).
Using this formulation it can be seen that a necessary (but not sufficient)
condition for an extortionate strategy is that it cooperates on average less
than 50\% of the time when in a state of disagreement with the opponent.

As an example, consider the known extortionate strategy \(p=(8 / 9, 1 / 2, 1 /
3, 0)\) from~\cite{Stewart2012} which is referred to as \texttt{Extort-2}. In
this case, for the standard values of \((R, T, S, P)\) constraint
(\ref{eqn:condition_for_p1}) corresponds to:

\begin{equation}
    p_1 = \frac{2(p_2 + p_3) + 1}{3}
\end{equation}

It is clear that in this case all constraints hold.

This approach could in fact be used to confirm that a given strategy is acting
in an extortionate manner even if it is not a memory one strategy. However, in
practice, if a closed form for \(p\) is not known, then due to measurement
and/or numerical error this would not work.

This problem can be written in the following linear algebraic form where
\(x=(\alpha, \beta)\)
and \(p^*=(\tilde p_1 - 1, tilde_2 - 1, p_3)\):

\begin{equation}\label{eqn:linear_algebraic_equation_for_p}
    Cx= p^*
\end{equation}

\(C\) corresponds to equations
(\ref{eqn:condition_for_tilde_p1}-\ref{eqn:condition_for_tilde_p3}) and is
given by:

\begin{equation}\label{eqn:definition_of_C}
    C =
    \begin{bmatrix}
        R - P & R- P \\
        S - P & T- P \\
        T - P & S- P \\
    \end{bmatrix}
\end{equation}

Note that in general, equation (\ref{eqn:linear_algebraic_equation_for_p}) will
not necessarily have a solution. From the Rouch\'{e}-Capelli theorem if there is
a solution it is unique as \(\text{rank}(C)=2\) which is the dimension of the
variable \(x\). The best fitting \(x\) is found by minimizing:

\begin{equation}\label{eqn:r_squared}
    \text{SSError} = \|C x- p^*\|_2^2 = \sum_{i=1}^{3}\left((C\bar x)_i-p_i^*\right)^2
\end{equation}

Note that \(\text{SSError}\), which is the square of the Frobenius
norm~\cite{Golub2013}, becomes a measure of how close a strategy is to being an
extortionate strategy. Suspicion
of extortion then corresponds to a threshold on \(\text{SSError}\).

By observing interactions (human or otherwise), their memory one representation
can be inferred and this approach can be used to recognise extortionate
behaviour. The notion of comparing theoretic and actual plays of the IPD is not
novel, see for example~\cite{Rand2013}. Immediately it is noted that if the
environment is noisy~\cite{Wu1995} then no strategy can be considered to be
extortionate as \(p_4>0\).

In the next section, this idea will be illustrated by observing the interactions
that take place in a computer based tournament of the IPD\@.

\section{Numerical experiments}\label{sec:numerical-experiments}

In~\cite{Stewart2012} results from a tournament with
\input{./assets/tex/number_of_stewart_plotkin_strategies/main.tex} strategies,
was presented with specific consideration given to ZD strategies. This
tournament is reproduced here using the Axelrod-Python
project~\cite{Knight2016}. To obtain a good measure of the corresponding
transition rates for each strategy all matches have been run for
\input{assets/tex/number_of_turns/main.tex} turns and every match has been
repeated \input{assets/tex/number_of_repetitions/main.tex} times. All of this
interaction data is available at~\cite{vincent_knight_2018_1297075}. A good
match between the inferred Markov chain and the state distribution of the actual
interactions has been verified. Data for this is presented in the supplementary
materials.

Figure~\ref{fig:SSError_overall_in_stewart_plotkin} shows the \(\text{SSError}\)
values for all the strategies in the tournament, as reported
in~\cite{Stewart2012} the extortionate strategy (which has an expected
\(\text{SSError}\) approximately 0) gains a large number of wins.

\begin{figure}[!htbp]
    \centering
    \includegraphics[width=.8\textwidth]{./assets/img/SSError_overall_in_stewart_plotkin/main.pdf}
    \caption{\(\text{SSError}\) and state probabilities for the strategies
        of~\cite{Stewart2012}, ordered both by number of wins and overall score.
        Note that \(P(DC)\) is not shown as it corresponds to the transpose of
        \(P(CD)\). Cooperator and Defector are omitted as they do not visit all
        the states.}
    \label{fig:SSError_overall_in_stewart_plotkin}
\end{figure}

Here, the work of~\cite{Stewart2012} is extended by investigating a tournament
with \input{assets/tex/number_of_full_strategies/main.tex}
strategies.

The results of this analysis are shown in
Figure~\ref{fig:SSError_and_probabilities_in_full}. The top ranking strategies
by number of wins seem to be extortionate (but not against all strategies) and
it can be seen that a small sub group of strategies achieve mutual defection.
All the top ranking strategies according to score achieve mutual cooperation and
do not extort each other, however they
\textbf{do} exhibit extortionate behaviour towards a number of the lower ranking
strategies.

\begin{figure}[!htbp]
    \centering
    \includegraphics[width=.8\textwidth]{./assets/img/SSError_and_probabilities_in_full/main.pdf}
    \caption{\(\text{SSError}\) for the strategies for the full tournament. Only
    strategy interactions for which \(p_4=0\) and \(\chi>1\) are displayed.}
    \label{fig:SSError_and_probabilities_in_full}
\end{figure}

\section{Conclusion}\label{sec:conclusion}

This work defines an approach to measure whether or not a player is playing a
strategy that corresponds to an extortionate strategy as defined
in~\cite{Press2012}: a mathematical model for suspicion. Indeed, all
extortionate strategies have been
 classified as lying on a triangular plane.
This rigorous classification fails to be robust to small measurement error, thus
a statistical approach is proposed.
This is done through a linear algebraic approach for approximating the solution
of a linear system. Using this, a large number of pairwise interactions is
simulated and in fact very few strategies are found to act extortionately.

The work of~\cite{Press2012}, whilst showing that a clever approach to taking
advantage of another memory one strategy exists: this is incomplete. Whilst the
elegance of this result is very attractive, just as the simplicity of the
victory of Tit For Tat in Axelrod's original tournaments was, it is incomplete.
Extortionate strategies achieve a high number of wins but they do not
achieve a high score which corresponds to the fitness landscape in an
evolutionary sense. From the large number of interactions a payoff matrix \(S\)
can be measured where \(S_{ij}\) denotes the score (using standard values of
\((R, S, T, P) = (3, 0, 5, 1)\)) of the \(i\)th strategy
against the \(j\)th strategy. Using this, the replicator equation
describes the evolution of the system based on a population density fitness
function:

\begin{equation}\label{eqn:replicator_dynamics}
    \frac{dx}{dt} = x(S-x^TS x)
\end{equation}

Equation (\ref{eqn:replicator_dynamics}) is solved numerically through an
integration technique described in~\cite{Petzold1983} and
Figure~\ref{fig:replicator_dynamics} shows the evolution of the distribution of
the system: the various strategies are ranked by scores. It is clear to see that
only the high ranking strategies survive the evolutionary process (in fact,
only \input{./assets/img/replicator_dynamics/main.tex}
have a final distribution greater than \(10 ^ {-2}\)). This confirms the
findings of~\cite{Moran1707} in which sophisticated strategies resist
evolutionary invasion of shorter memory strategies. Recalling
Figure~\ref{fig:SSError_and_probabilities_in_full} this demonstrates that:

\begin{itemize}
    \item Cooperation emerges through the evolutionary process: the high scoring
        strategies do not exhibit extortionate behaviour towards each other.
    \item Extortionate strategies do not survive the evolutionary process.
\end{itemize}

\begin{figure}[!htbp]
    \centering
    \includegraphics[width=.8\textwidth]{./assets/img/replicator_dynamics/main.pdf}
    \caption{Numerical simulation of the replicator equation
    (\ref{eqn:replicator_dynamics}): strategies are ordered by score, only the strategies with a high score survive the evolutionary process.}
    \label{fig:replicator_dynamics}
\end{figure}

This work can be used to classify plays of the IPD\@: data can be collected from
actual interactions (in lab or in the field). Furthermore, this allows for a
classification method similar to the notion of fingerprinting presented
in~\cite{Ashlock2008}. Trained strategies can potentially be classified as
extortionate or not or it could be possible to even constrain the reinforcement
learning approaches that are becoming prevalent in the literature.
Alternatively, this mathematical approach for recognising extortion could be
used in sophisticated strategies to defend against invasion. Arguably, some of
the strategies considered here exhibit this behaviour, indeed as described
in~\cite{Harper2017}, the top ranking strategies in the full tournament are
obtained using evolutionary reinforcement learning techniques, thus, suspicion
of extortionate behaviour could in fact be an evolutionary trait.

\section*{Acknowledgements}

The following open source software libraries were used in this research:

\begin{itemize}
    \item The Axelrod ~\cite{Knight2016, Knight2018} library (IPD strategies and
        tournaments).
    \item The sympy library~\cite{Meurer2017} (verification of all symbolic
        calculations).
    \item The matplotlib~\cite{Droettboom2018} library (visualisation).
    \item The pandas~\cite{Structures2010}, dask~\cite{Dask2016} and
        NumPy~\cite{Oliphant2015} libraries (data manipulation).
    \item The SciPy~\cite{Jones2001} library (numerical integration of the
        replicator equation).
\end{itemize}

This work was performed using the computational facilities of the Advanced
Research Computing @ Cardiff (ARCCA) Division, Cardiff University.

\printbibliography

\newpage
\section*{Supplementary materials}

\includepdf{assets/pdf/proof_of_form_of_extortionate_strategies/main.pdf}

\newpage

Using the pair wise interactions the transition rates \(p,
q\) can be measured and the steady state probabilities inferred and compared to
the actual probabilities of each state.
This is done numerically by computing the singular eigenvector of the
matrix \(A\) \cite{Stewart2009}:

\[
    A =
    \begin{bmatrix}
        p_1 q_1 & p_1 (1 - q_1) & (1 - p_1) q_1 & (1 -p_1) (1 - q_1) \\
        p_2 q_2 & p_2 (1 - q_2) & (1 - p_2) q_2 & (1 -p_2) (1 - q_2) \\
        p_3 q_3 & p_3 (1 - q_3) & (1 - p_3) q_3 & (1 -p_3) (1 - q_3) \\
        p_4 q_4 & p_4 (1 - q_4) & (1 - p_4) q_4 & (1 -p_4) (1 - q_4) \\
    \end{bmatrix}
\]

Figure~\ref{fig:computed_probabilities_vs_theoretic_probabilities} shows a
regression line fitted to every pairwise interaction with a reported
\(\text{SSError}\) value (pairwise interactions with missing states were
omitted). This serves to validate the approach: a part from some edge cases the
relationship is consistent.

\begin{figure}[!htbp]
    \centering
    \includegraphics[width=.8\textwidth]{./assets/img/computed_probabilities_vs_theoretic_probabilities/main.pdf}
    \caption{The
        relationship between the steady state probabilities inferred from the
        measured transitions and the actual steady state probabilities. A linear
        regression line is included validating the approach.}
    \label{fig:computed_probabilities_vs_theoretic_probabilities}
\end{figure}


\end{document}

strategies.

The results of this analysis are shown in
Figure~\ref{fig:SSError_and_probabilities_in_full}. The top ranking strategies
by number of wins seem to be extortionate (but not against all strategies) and
it can be seen that a small sub group of strategies achieve mutual defection.
All the top ranking strategies according to score achieve mutual cooperation and
do not extort each other, however they
\textbf{do} exhibit extortionate behaviour towards a number of the lower ranking
strategies.

\begin{figure}[!htbp]
    \centering
    \includegraphics[width=.8\textwidth]{./assets/img/SSError_and_probabilities_in_full/main.pdf}
    \caption{\(\text{SSError}\) for the strategies for the full tournament. Only
    strategy interactions for which \(p_4=0\) and \(\chi>1\) are displayed.}
    \label{fig:SSError_and_probabilities_in_full}
\end{figure}

\section{Conclusion}\label{sec:conclusion}

This work defines an approach to measure whether or not a player is playing a
strategy that corresponds to an extortionate strategy as defined
in~\cite{Press2012}: a mathematical model for suspicion. Indeed, all
extortionate strategies have been
 classified as lying on a triangular plane.
This rigorous classification fails to be robust to small measurement error, thus
a statistical approach is proposed.
This is done through a linear algebraic approach for approximating the solution
of a linear system. Using this, a large number of pairwise interactions is
simulated and in fact very few strategies are found to act extortionately.

The work of~\cite{Press2012}, whilst showing that a clever approach to taking
advantage of another memory one strategy exists: this is incomplete. Whilst the
elegance of this result is very attractive, just as the simplicity of the
victory of Tit For Tat in Axelrod's original tournaments was, it is incomplete.
Extortionate strategies achieve a high number of wins but they do not
achieve a high score which corresponds to the fitness landscape in an
evolutionary sense. From the large number of interactions a payoff matrix \(S\)
can be measured where \(S_{ij}\) denotes the score (using standard values of
\((R, S, T, P) = (3, 0, 5, 1)\)) of the \(i\)th strategy
against the \(j\)th strategy. Using this, the replicator equation
describes the evolution of the system based on a population density fitness
function:

\begin{equation}\label{eqn:replicator_dynamics}
    \frac{dx}{dt} = x(S-x^TS x)
\end{equation}

Equation (\ref{eqn:replicator_dynamics}) is solved numerically through an
integration technique described in~\cite{Petzold1983} and
Figure~\ref{fig:replicator_dynamics} shows the evolution of the distribution of
the system: the various strategies are ranked by scores. It is clear to see that
only the high ranking strategies survive the evolutionary process (in fact,
only \documentclass[a4paper]{article}

\usepackage{amsmath}
\usepackage{amssymb}
\usepackage[margin=1.5cm,
            includefoot,
            footskip=30pt]{geometry}
\usepackage{layout}
\usepackage{graphicx}
\usepackage{subcaption}

\usepackage{biblatex}
\usepackage{pdfpages}

\bibliography{main.bib}

\title{Suspicion: Recognising and evaluating the effectiveness
       of extortion in the Iterated Prisoner's Dilemma}
\author{Vincent A. Knight \and Nikoleta E. Glynatsi}
\date{\today}



\begin{document}

\maketitle

\begin{abstract}
    The Iterated Prisoner's Dilemma is a model for rational and evolutionary
    interactive behaviour. It has applications both in the study of human social
    behaviour as well as in biology.
    It is used to understand when and how a rational individual might
    accept an immediate cost to their own utility for the direct benefit of
    another.

    Much attention has been given to a class of strategies called
    Zero Determinant strategies. It has been theoretically shown that these
    strategies can ``extort'' any player.

    In this work, an approach to identify if observed strategies are playing in
    an extortionate way is described. Furthermore, experimental analysis of
    a large tournament with \input{assets/tex/number_of_full_strategies/main.tex}
    strategies is considered. In this setting
    the most highly performing strategies do not play in an extortionate way
    against each other but do against lower performing strategies.
    This suggests that whilst the theory of Zero Determinant strategies
    indicates that memory is not of fundamental importance to the evolution of
    cooperative behaviour, this is incomplete.
\end{abstract}

\section{Introduction}\label{sec:introduction}

Agent based game theoretic models have become a stalwart of the underpinning
mathematics of interactive behaviours. One of the major pieces of work
in this area is the pair of original computer tournaments run by Robert
Axelrod~\cite{Axelrod1980, Axelrod1980a}. These tournaments pitted submitted
computer strategies against each other in plays of the Iterated Prisoner's
Dilemma. A common game where agents can choose to pay a slight cost to their
immediate utility in the hope of building a reputation. This has been used in
economic and evolutionary game theory to understand the evolution of cooperative
behaviour.

Recently, a class of strategies was described in~\cite{Press2012} that can
provably extort any given opponent. In~\cite{Hilbe2013, Moran1707} some
questions have already been asked about the true effectiveness of these
strategies in an evolutionary setting. Here another question is asked: is it
possible to recognise this extortionate behaviour? A mathematical procedure for
suspicion is presented: in the same way that the continued actions of an
extortionate individual might raise suspicion.

This work makes use of the Axelrod Python library~\cite{Knight2018, Knight2016}
with a large number of Prisoner Dilemma strategies available to give an
extensive numerical example of the ideas presented.  The approach is presented
in Section~\ref{sec:delta-zd-strategies}.  All of the code and data discussed
in Section~\ref{sec:numerical-experiments} is open sourced, archived and
written according to best scientific principles~\cite{Wilson2014}. The data
archive can be found at~\cite{vincent_knight_2018_1297075}.

\section{Recognising Extortion}\label{sec:delta-zd-strategies}

In~\cite{Press2012}, given a match between 2 memory-one strategies, the concept
of Zero Determinant (ZD) strategies is introduced. The main result of that paper
shows that given two memory one players \(p, q\in\mathbb{R}^4\) a linear
relationship between the players' scores could be forced by one of the players.

Using the notation of~\cite{Press2012}, assuming the utilities for player \(p\)
are given by \(S_x=(R, S, T, P)\) and for player \(q\) by \(S_y=(R, T, S, P)\)
and that the stationary scores of each player is given by \(S_X\) and \(S_Y\)
respectively. The main result of~\cite{Press2012} is that if

\begin{equation}\label{eqn:linear_relationship_for_p}
    \tilde p=\alpha S_x + \beta S_y + \gamma
\end{equation}

or

\begin{equation}\label{eqn:linear_relationship_for_q}
    \tilde q=\alpha S_x + \beta S_y + \gamma
\end{equation}

where \(\tilde p = (1 - p_1, 1 - p_2, p_3, p_4)\) and
\(\tilde q = (1 - q_1, 1 - q_2, q_3, q_4)\) then:

\begin{equation}
    \alpha S_X + \beta S_Y + \gamma = 0
\end{equation}

In~\cite{Press2012} a particular type of ZD strategy is defined: extortionate
strategies. If:

\begin{equation}\label{eqn:constraint_for_extortion}
    \gamma = - P(\alpha + \beta)
\end{equation}

then the player can ensure they get a score \(\chi\) times
larger than the opponent. This extortion coefficient is given by:

\begin{equation}\label{eqn:definition_of_chi}
    \chi=\frac{-\beta}{\alpha}
\end{equation}

Thus, if (\ref{eqn:constraint_for_extortion}) holds and \(\chi >1\) a player is
said to extort their opponent.
Here, the reverse problem is considered: given a
\(p\in\mathbb{R}^4\) how does one identify \(\alpha, \beta\) if they
exist and is the strategy in fact acting in an extortionate way?

These conditions correspond to:

\begin{align}
    \tilde p_1 & = \alpha R + \beta R - P (\alpha + \beta)
            \label{eqn:condition_for_tilde_p1}\\
    \tilde p_2 & = \alpha S + \beta T - P (\alpha + \beta)
            \label{eqn:condition_for_tilde_p2}\\
    \tilde p_3 & = \alpha T + \beta S - P (\alpha + \beta)
            \label{eqn:condition_for_tilde_p3}\\
    \tilde p_4 & = \alpha P + \beta P - P (\alpha + \beta)
            \label{eqn:condition_for_tilde_p4}
\end{align}

Equation (\ref{eqn:condition_for_tilde_p4}) ensures that \(p_4=\tilde p_4=0\).
Equations (\ref{eqn:condition_for_tilde_p1}-\ref{eqn:condition_for_tilde_p3})
can be used to eliminate \(\alpha, \beta\), giving:

\begin{equation}\label{eqn:planar_definition_of_extortion}
    \tilde p_1 = \frac{(R - P)(\tilde p_2 + \tilde p_3)}{S + T - 2P}
\end{equation}

with:

\begin{equation}\label{eqn:definition_of_chi}
    \chi = \frac{\tilde p_2 (P - T) + \tilde p_3 (S - P)}
                {\tilde p_2 (P - S) + \tilde p_3 (T - P)}
\end{equation}

Given a strategy \(p\in\mathbb{R}^{4\times 1}\) equations
(\ref{eqn:condition_for_tilde_p4}), (\ref{eqn:planar_definition_of_extortion}-\ref{eqn:definition_of_chi}) can be used to check if
a strategy is extortionate. The conditions correspond to:

\begin{align}
    p_1 & = \frac{(R-P)(p_2 + p_3) - R + T + S - P}{S + T - 2P}
     \label{eqn:condition_for_p1}\\
    p_4 & = 0 \label{eqn:condition_for_p4}\\
    1 & > p_2 + p_3\label{eqn:condition_for_chi}
\end{align}

The algebraic steps necessary to prove these results are available in the
supporting materials.

All extortionate strategies reside on a triangular (\ref{eqn:condition_for_chi})
plane (\ref{eqn:condition_for_p1}) in 3 dimensions (\ref{eqn:condition_for_p4}).
Using this formulation it can be seen that a necessary (but not sufficient)
condition for an extortionate strategy is that it cooperates on average less
than 50\% of the time when in a state of disagreement with the opponent.

As an example, consider the known extortionate strategy \(p=(8 / 9, 1 / 2, 1 /
3, 0)\) from~\cite{Stewart2012} which is referred to as \texttt{Extort-2}. In
this case, for the standard values of \((R, T, S, P)\) constraint
(\ref{eqn:condition_for_p1}) corresponds to:

\begin{equation}
    p_1 = \frac{2(p_2 + p_3) + 1}{3}
\end{equation}

It is clear that in this case all constraints hold.

This approach could in fact be used to confirm that a given strategy is acting
in an extortionate manner even if it is not a memory one strategy. However, in
practice, if a closed form for \(p\) is not known, then due to measurement
and/or numerical error this would not work.

This problem can be written in the following linear algebraic form where
\(x=(\alpha, \beta)\)
and \(p^*=(\tilde p_1 - 1, tilde_2 - 1, p_3)\):

\begin{equation}\label{eqn:linear_algebraic_equation_for_p}
    Cx= p^*
\end{equation}

\(C\) corresponds to equations
(\ref{eqn:condition_for_tilde_p1}-\ref{eqn:condition_for_tilde_p3}) and is
given by:

\begin{equation}\label{eqn:definition_of_C}
    C =
    \begin{bmatrix}
        R - P & R- P \\
        S - P & T- P \\
        T - P & S- P \\
    \end{bmatrix}
\end{equation}

Note that in general, equation (\ref{eqn:linear_algebraic_equation_for_p}) will
not necessarily have a solution. From the Rouch\'{e}-Capelli theorem if there is
a solution it is unique as \(\text{rank}(C)=2\) which is the dimension of the
variable \(x\). The best fitting \(x\) is found by minimizing:

\begin{equation}\label{eqn:r_squared}
    \text{SSError} = \|C x- p^*\|_2^2 = \sum_{i=1}^{3}\left((C\bar x)_i-p_i^*\right)^2
\end{equation}

Note that \(\text{SSError}\), which is the square of the Frobenius
norm~\cite{Golub2013}, becomes a measure of how close a strategy is to being an
extortionate strategy. Suspicion
of extortion then corresponds to a threshold on \(\text{SSError}\).

By observing interactions (human or otherwise), their memory one representation
can be inferred and this approach can be used to recognise extortionate
behaviour. The notion of comparing theoretic and actual plays of the IPD is not
novel, see for example~\cite{Rand2013}. Immediately it is noted that if the
environment is noisy~\cite{Wu1995} then no strategy can be considered to be
extortionate as \(p_4>0\).

In the next section, this idea will be illustrated by observing the interactions
that take place in a computer based tournament of the IPD\@.

\section{Numerical experiments}\label{sec:numerical-experiments}

In~\cite{Stewart2012} results from a tournament with
\input{./assets/tex/number_of_stewart_plotkin_strategies/main.tex} strategies,
was presented with specific consideration given to ZD strategies. This
tournament is reproduced here using the Axelrod-Python
project~\cite{Knight2016}. To obtain a good measure of the corresponding
transition rates for each strategy all matches have been run for
\input{assets/tex/number_of_turns/main.tex} turns and every match has been
repeated \input{assets/tex/number_of_repetitions/main.tex} times. All of this
interaction data is available at~\cite{vincent_knight_2018_1297075}. A good
match between the inferred Markov chain and the state distribution of the actual
interactions has been verified. Data for this is presented in the supplementary
materials.

Figure~\ref{fig:SSError_overall_in_stewart_plotkin} shows the \(\text{SSError}\)
values for all the strategies in the tournament, as reported
in~\cite{Stewart2012} the extortionate strategy (which has an expected
\(\text{SSError}\) approximately 0) gains a large number of wins.

\begin{figure}[!htbp]
    \centering
    \includegraphics[width=.8\textwidth]{./assets/img/SSError_overall_in_stewart_plotkin/main.pdf}
    \caption{\(\text{SSError}\) and state probabilities for the strategies
        of~\cite{Stewart2012}, ordered both by number of wins and overall score.
        Note that \(P(DC)\) is not shown as it corresponds to the transpose of
        \(P(CD)\). Cooperator and Defector are omitted as they do not visit all
        the states.}
    \label{fig:SSError_overall_in_stewart_plotkin}
\end{figure}

Here, the work of~\cite{Stewart2012} is extended by investigating a tournament
with \input{assets/tex/number_of_full_strategies/main.tex}
strategies.

The results of this analysis are shown in
Figure~\ref{fig:SSError_and_probabilities_in_full}. The top ranking strategies
by number of wins seem to be extortionate (but not against all strategies) and
it can be seen that a small sub group of strategies achieve mutual defection.
All the top ranking strategies according to score achieve mutual cooperation and
do not extort each other, however they
\textbf{do} exhibit extortionate behaviour towards a number of the lower ranking
strategies.

\begin{figure}[!htbp]
    \centering
    \includegraphics[width=.8\textwidth]{./assets/img/SSError_and_probabilities_in_full/main.pdf}
    \caption{\(\text{SSError}\) for the strategies for the full tournament. Only
    strategy interactions for which \(p_4=0\) and \(\chi>1\) are displayed.}
    \label{fig:SSError_and_probabilities_in_full}
\end{figure}

\section{Conclusion}\label{sec:conclusion}

This work defines an approach to measure whether or not a player is playing a
strategy that corresponds to an extortionate strategy as defined
in~\cite{Press2012}: a mathematical model for suspicion. Indeed, all
extortionate strategies have been
 classified as lying on a triangular plane.
This rigorous classification fails to be robust to small measurement error, thus
a statistical approach is proposed.
This is done through a linear algebraic approach for approximating the solution
of a linear system. Using this, a large number of pairwise interactions is
simulated and in fact very few strategies are found to act extortionately.

The work of~\cite{Press2012}, whilst showing that a clever approach to taking
advantage of another memory one strategy exists: this is incomplete. Whilst the
elegance of this result is very attractive, just as the simplicity of the
victory of Tit For Tat in Axelrod's original tournaments was, it is incomplete.
Extortionate strategies achieve a high number of wins but they do not
achieve a high score which corresponds to the fitness landscape in an
evolutionary sense. From the large number of interactions a payoff matrix \(S\)
can be measured where \(S_{ij}\) denotes the score (using standard values of
\((R, S, T, P) = (3, 0, 5, 1)\)) of the \(i\)th strategy
against the \(j\)th strategy. Using this, the replicator equation
describes the evolution of the system based on a population density fitness
function:

\begin{equation}\label{eqn:replicator_dynamics}
    \frac{dx}{dt} = x(S-x^TS x)
\end{equation}

Equation (\ref{eqn:replicator_dynamics}) is solved numerically through an
integration technique described in~\cite{Petzold1983} and
Figure~\ref{fig:replicator_dynamics} shows the evolution of the distribution of
the system: the various strategies are ranked by scores. It is clear to see that
only the high ranking strategies survive the evolutionary process (in fact,
only \input{./assets/img/replicator_dynamics/main.tex}
have a final distribution greater than \(10 ^ {-2}\)). This confirms the
findings of~\cite{Moran1707} in which sophisticated strategies resist
evolutionary invasion of shorter memory strategies. Recalling
Figure~\ref{fig:SSError_and_probabilities_in_full} this demonstrates that:

\begin{itemize}
    \item Cooperation emerges through the evolutionary process: the high scoring
        strategies do not exhibit extortionate behaviour towards each other.
    \item Extortionate strategies do not survive the evolutionary process.
\end{itemize}

\begin{figure}[!htbp]
    \centering
    \includegraphics[width=.8\textwidth]{./assets/img/replicator_dynamics/main.pdf}
    \caption{Numerical simulation of the replicator equation
    (\ref{eqn:replicator_dynamics}): strategies are ordered by score, only the strategies with a high score survive the evolutionary process.}
    \label{fig:replicator_dynamics}
\end{figure}

This work can be used to classify plays of the IPD\@: data can be collected from
actual interactions (in lab or in the field). Furthermore, this allows for a
classification method similar to the notion of fingerprinting presented
in~\cite{Ashlock2008}. Trained strategies can potentially be classified as
extortionate or not or it could be possible to even constrain the reinforcement
learning approaches that are becoming prevalent in the literature.
Alternatively, this mathematical approach for recognising extortion could be
used in sophisticated strategies to defend against invasion. Arguably, some of
the strategies considered here exhibit this behaviour, indeed as described
in~\cite{Harper2017}, the top ranking strategies in the full tournament are
obtained using evolutionary reinforcement learning techniques, thus, suspicion
of extortionate behaviour could in fact be an evolutionary trait.

\section*{Acknowledgements}

The following open source software libraries were used in this research:

\begin{itemize}
    \item The Axelrod ~\cite{Knight2016, Knight2018} library (IPD strategies and
        tournaments).
    \item The sympy library~\cite{Meurer2017} (verification of all symbolic
        calculations).
    \item The matplotlib~\cite{Droettboom2018} library (visualisation).
    \item The pandas~\cite{Structures2010}, dask~\cite{Dask2016} and
        NumPy~\cite{Oliphant2015} libraries (data manipulation).
    \item The SciPy~\cite{Jones2001} library (numerical integration of the
        replicator equation).
\end{itemize}

This work was performed using the computational facilities of the Advanced
Research Computing @ Cardiff (ARCCA) Division, Cardiff University.

\printbibliography

\newpage
\section*{Supplementary materials}

\includepdf{assets/pdf/proof_of_form_of_extortionate_strategies/main.pdf}

\newpage

Using the pair wise interactions the transition rates \(p,
q\) can be measured and the steady state probabilities inferred and compared to
the actual probabilities of each state.
This is done numerically by computing the singular eigenvector of the
matrix \(A\) \cite{Stewart2009}:

\[
    A =
    \begin{bmatrix}
        p_1 q_1 & p_1 (1 - q_1) & (1 - p_1) q_1 & (1 -p_1) (1 - q_1) \\
        p_2 q_2 & p_2 (1 - q_2) & (1 - p_2) q_2 & (1 -p_2) (1 - q_2) \\
        p_3 q_3 & p_3 (1 - q_3) & (1 - p_3) q_3 & (1 -p_3) (1 - q_3) \\
        p_4 q_4 & p_4 (1 - q_4) & (1 - p_4) q_4 & (1 -p_4) (1 - q_4) \\
    \end{bmatrix}
\]

Figure~\ref{fig:computed_probabilities_vs_theoretic_probabilities} shows a
regression line fitted to every pairwise interaction with a reported
\(\text{SSError}\) value (pairwise interactions with missing states were
omitted). This serves to validate the approach: a part from some edge cases the
relationship is consistent.

\begin{figure}[!htbp]
    \centering
    \includegraphics[width=.8\textwidth]{./assets/img/computed_probabilities_vs_theoretic_probabilities/main.pdf}
    \caption{The
        relationship between the steady state probabilities inferred from the
        measured transitions and the actual steady state probabilities. A linear
        regression line is included validating the approach.}
    \label{fig:computed_probabilities_vs_theoretic_probabilities}
\end{figure}


\end{document}

have a final distribution greater than \(10 ^ {-2}\)). This confirms the
findings of~\cite{Moran1707} in which sophisticated strategies resist
evolutionary invasion of shorter memory strategies. Recalling
Figure~\ref{fig:SSError_and_probabilities_in_full} this demonstrates that:

\begin{itemize}
    \item Cooperation emerges through the evolutionary process: the high scoring
        strategies do not exhibit extortionate behaviour towards each other.
    \item Extortionate strategies do not survive the evolutionary process.
\end{itemize}

\begin{figure}[!htbp]
    \centering
    \includegraphics[width=.8\textwidth]{./assets/img/replicator_dynamics/main.pdf}
    \caption{Numerical simulation of the replicator equation
    (\ref{eqn:replicator_dynamics}): strategies are ordered by score, only the strategies with a high score survive the evolutionary process.}
    \label{fig:replicator_dynamics}
\end{figure}

This work can be used to classify plays of the IPD\@: data can be collected from
actual interactions (in lab or in the field). Furthermore, this allows for a
classification method similar to the notion of fingerprinting presented
in~\cite{Ashlock2008}. Trained strategies can potentially be classified as
extortionate or not or it could be possible to even constrain the reinforcement
learning approaches that are becoming prevalent in the literature.
Alternatively, this mathematical approach for recognising extortion could be
used in sophisticated strategies to defend against invasion. Arguably, some of
the strategies considered here exhibit this behaviour, indeed as described
in~\cite{Harper2017}, the top ranking strategies in the full tournament are
obtained using evolutionary reinforcement learning techniques, thus, suspicion
of extortionate behaviour could in fact be an evolutionary trait.

\section*{Acknowledgements}

The following open source software libraries were used in this research:

\begin{itemize}
    \item The Axelrod ~\cite{Knight2016, Knight2018} library (IPD strategies and
        tournaments).
    \item The sympy library~\cite{Meurer2017} (verification of all symbolic
        calculations).
    \item The matplotlib~\cite{Droettboom2018} library (visualisation).
    \item The pandas~\cite{Structures2010}, dask~\cite{Dask2016} and
        NumPy~\cite{Oliphant2015} libraries (data manipulation).
    \item The SciPy~\cite{Jones2001} library (numerical integration of the
        replicator equation).
\end{itemize}

This work was performed using the computational facilities of the Advanced
Research Computing @ Cardiff (ARCCA) Division, Cardiff University.

\printbibliography

\newpage
\section*{Supplementary materials}

\includepdf{assets/pdf/proof_of_form_of_extortionate_strategies/main.pdf}

\newpage

Using the pair wise interactions the transition rates \(p,
q\) can be measured and the steady state probabilities inferred and compared to
the actual probabilities of each state.
This is done numerically by computing the singular eigenvector of the
matrix \(A\) \cite{Stewart2009}:

\[
    A =
    \begin{bmatrix}
        p_1 q_1 & p_1 (1 - q_1) & (1 - p_1) q_1 & (1 -p_1) (1 - q_1) \\
        p_2 q_2 & p_2 (1 - q_2) & (1 - p_2) q_2 & (1 -p_2) (1 - q_2) \\
        p_3 q_3 & p_3 (1 - q_3) & (1 - p_3) q_3 & (1 -p_3) (1 - q_3) \\
        p_4 q_4 & p_4 (1 - q_4) & (1 - p_4) q_4 & (1 -p_4) (1 - q_4) \\
    \end{bmatrix}
\]

Figure~\ref{fig:computed_probabilities_vs_theoretic_probabilities} shows a
regression line fitted to every pairwise interaction with a reported
\(\text{SSError}\) value (pairwise interactions with missing states were
omitted). This serves to validate the approach: a part from some edge cases the
relationship is consistent.

\begin{figure}[!htbp]
    \centering
    \includegraphics[width=.8\textwidth]{./assets/img/computed_probabilities_vs_theoretic_probabilities/main.pdf}
    \caption{The
        relationship between the steady state probabilities inferred from the
        measured transitions and the actual steady state probabilities. A linear
        regression line is included validating the approach.}
    \label{fig:computed_probabilities_vs_theoretic_probabilities}
\end{figure}


\end{document}
 strategies,
was presented with specific consideration given to ZD strategies. This
tournament is reproduced here using the Axelrod-Python
project~\cite{Knight2016}. To obtain a good measure of the corresponding
transition rates for each strategy all matches have been run for
\documentclass[a4paper]{article}

\usepackage{amsmath}
\usepackage{amssymb}
\usepackage[margin=1.5cm,
            includefoot,
            footskip=30pt]{geometry}
\usepackage{layout}
\usepackage{graphicx}
\usepackage{subcaption}

\usepackage{biblatex}
\usepackage{pdfpages}

\bibliography{main.bib}

\title{Suspicion: Recognising and evaluating the effectiveness
       of extortion in the Iterated Prisoner's Dilemma}
\author{Vincent A. Knight \and Nikoleta E. Glynatsi}
\date{\today}



\begin{document}

\maketitle

\begin{abstract}
    The Iterated Prisoner's Dilemma is a model for rational and evolutionary
    interactive behaviour. It has applications both in the study of human social
    behaviour as well as in biology.
    It is used to understand when and how a rational individual might
    accept an immediate cost to their own utility for the direct benefit of
    another.

    Much attention has been given to a class of strategies called
    Zero Determinant strategies. It has been theoretically shown that these
    strategies can ``extort'' any player.

    In this work, an approach to identify if observed strategies are playing in
    an extortionate way is described. Furthermore, experimental analysis of
    a large tournament with \documentclass[a4paper]{article}

\usepackage{amsmath}
\usepackage{amssymb}
\usepackage[margin=1.5cm,
            includefoot,
            footskip=30pt]{geometry}
\usepackage{layout}
\usepackage{graphicx}
\usepackage{subcaption}

\usepackage{biblatex}
\usepackage{pdfpages}

\bibliography{main.bib}

\title{Suspicion: Recognising and evaluating the effectiveness
       of extortion in the Iterated Prisoner's Dilemma}
\author{Vincent A. Knight \and Nikoleta E. Glynatsi}
\date{\today}



\begin{document}

\maketitle

\begin{abstract}
    The Iterated Prisoner's Dilemma is a model for rational and evolutionary
    interactive behaviour. It has applications both in the study of human social
    behaviour as well as in biology.
    It is used to understand when and how a rational individual might
    accept an immediate cost to their own utility for the direct benefit of
    another.

    Much attention has been given to a class of strategies called
    Zero Determinant strategies. It has been theoretically shown that these
    strategies can ``extort'' any player.

    In this work, an approach to identify if observed strategies are playing in
    an extortionate way is described. Furthermore, experimental analysis of
    a large tournament with \input{assets/tex/number_of_full_strategies/main.tex}
    strategies is considered. In this setting
    the most highly performing strategies do not play in an extortionate way
    against each other but do against lower performing strategies.
    This suggests that whilst the theory of Zero Determinant strategies
    indicates that memory is not of fundamental importance to the evolution of
    cooperative behaviour, this is incomplete.
\end{abstract}

\section{Introduction}\label{sec:introduction}

Agent based game theoretic models have become a stalwart of the underpinning
mathematics of interactive behaviours. One of the major pieces of work
in this area is the pair of original computer tournaments run by Robert
Axelrod~\cite{Axelrod1980, Axelrod1980a}. These tournaments pitted submitted
computer strategies against each other in plays of the Iterated Prisoner's
Dilemma. A common game where agents can choose to pay a slight cost to their
immediate utility in the hope of building a reputation. This has been used in
economic and evolutionary game theory to understand the evolution of cooperative
behaviour.

Recently, a class of strategies was described in~\cite{Press2012} that can
provably extort any given opponent. In~\cite{Hilbe2013, Moran1707} some
questions have already been asked about the true effectiveness of these
strategies in an evolutionary setting. Here another question is asked: is it
possible to recognise this extortionate behaviour? A mathematical procedure for
suspicion is presented: in the same way that the continued actions of an
extortionate individual might raise suspicion.

This work makes use of the Axelrod Python library~\cite{Knight2018, Knight2016}
with a large number of Prisoner Dilemma strategies available to give an
extensive numerical example of the ideas presented.  The approach is presented
in Section~\ref{sec:delta-zd-strategies}.  All of the code and data discussed
in Section~\ref{sec:numerical-experiments} is open sourced, archived and
written according to best scientific principles~\cite{Wilson2014}. The data
archive can be found at~\cite{vincent_knight_2018_1297075}.

\section{Recognising Extortion}\label{sec:delta-zd-strategies}

In~\cite{Press2012}, given a match between 2 memory-one strategies, the concept
of Zero Determinant (ZD) strategies is introduced. The main result of that paper
shows that given two memory one players \(p, q\in\mathbb{R}^4\) a linear
relationship between the players' scores could be forced by one of the players.

Using the notation of~\cite{Press2012}, assuming the utilities for player \(p\)
are given by \(S_x=(R, S, T, P)\) and for player \(q\) by \(S_y=(R, T, S, P)\)
and that the stationary scores of each player is given by \(S_X\) and \(S_Y\)
respectively. The main result of~\cite{Press2012} is that if

\begin{equation}\label{eqn:linear_relationship_for_p}
    \tilde p=\alpha S_x + \beta S_y + \gamma
\end{equation}

or

\begin{equation}\label{eqn:linear_relationship_for_q}
    \tilde q=\alpha S_x + \beta S_y + \gamma
\end{equation}

where \(\tilde p = (1 - p_1, 1 - p_2, p_3, p_4)\) and
\(\tilde q = (1 - q_1, 1 - q_2, q_3, q_4)\) then:

\begin{equation}
    \alpha S_X + \beta S_Y + \gamma = 0
\end{equation}

In~\cite{Press2012} a particular type of ZD strategy is defined: extortionate
strategies. If:

\begin{equation}\label{eqn:constraint_for_extortion}
    \gamma = - P(\alpha + \beta)
\end{equation}

then the player can ensure they get a score \(\chi\) times
larger than the opponent. This extortion coefficient is given by:

\begin{equation}\label{eqn:definition_of_chi}
    \chi=\frac{-\beta}{\alpha}
\end{equation}

Thus, if (\ref{eqn:constraint_for_extortion}) holds and \(\chi >1\) a player is
said to extort their opponent.
Here, the reverse problem is considered: given a
\(p\in\mathbb{R}^4\) how does one identify \(\alpha, \beta\) if they
exist and is the strategy in fact acting in an extortionate way?

These conditions correspond to:

\begin{align}
    \tilde p_1 & = \alpha R + \beta R - P (\alpha + \beta)
            \label{eqn:condition_for_tilde_p1}\\
    \tilde p_2 & = \alpha S + \beta T - P (\alpha + \beta)
            \label{eqn:condition_for_tilde_p2}\\
    \tilde p_3 & = \alpha T + \beta S - P (\alpha + \beta)
            \label{eqn:condition_for_tilde_p3}\\
    \tilde p_4 & = \alpha P + \beta P - P (\alpha + \beta)
            \label{eqn:condition_for_tilde_p4}
\end{align}

Equation (\ref{eqn:condition_for_tilde_p4}) ensures that \(p_4=\tilde p_4=0\).
Equations (\ref{eqn:condition_for_tilde_p1}-\ref{eqn:condition_for_tilde_p3})
can be used to eliminate \(\alpha, \beta\), giving:

\begin{equation}\label{eqn:planar_definition_of_extortion}
    \tilde p_1 = \frac{(R - P)(\tilde p_2 + \tilde p_3)}{S + T - 2P}
\end{equation}

with:

\begin{equation}\label{eqn:definition_of_chi}
    \chi = \frac{\tilde p_2 (P - T) + \tilde p_3 (S - P)}
                {\tilde p_2 (P - S) + \tilde p_3 (T - P)}
\end{equation}

Given a strategy \(p\in\mathbb{R}^{4\times 1}\) equations
(\ref{eqn:condition_for_tilde_p4}), (\ref{eqn:planar_definition_of_extortion}-\ref{eqn:definition_of_chi}) can be used to check if
a strategy is extortionate. The conditions correspond to:

\begin{align}
    p_1 & = \frac{(R-P)(p_2 + p_3) - R + T + S - P}{S + T - 2P}
     \label{eqn:condition_for_p1}\\
    p_4 & = 0 \label{eqn:condition_for_p4}\\
    1 & > p_2 + p_3\label{eqn:condition_for_chi}
\end{align}

The algebraic steps necessary to prove these results are available in the
supporting materials.

All extortionate strategies reside on a triangular (\ref{eqn:condition_for_chi})
plane (\ref{eqn:condition_for_p1}) in 3 dimensions (\ref{eqn:condition_for_p4}).
Using this formulation it can be seen that a necessary (but not sufficient)
condition for an extortionate strategy is that it cooperates on average less
than 50\% of the time when in a state of disagreement with the opponent.

As an example, consider the known extortionate strategy \(p=(8 / 9, 1 / 2, 1 /
3, 0)\) from~\cite{Stewart2012} which is referred to as \texttt{Extort-2}. In
this case, for the standard values of \((R, T, S, P)\) constraint
(\ref{eqn:condition_for_p1}) corresponds to:

\begin{equation}
    p_1 = \frac{2(p_2 + p_3) + 1}{3}
\end{equation}

It is clear that in this case all constraints hold.

This approach could in fact be used to confirm that a given strategy is acting
in an extortionate manner even if it is not a memory one strategy. However, in
practice, if a closed form for \(p\) is not known, then due to measurement
and/or numerical error this would not work.

This problem can be written in the following linear algebraic form where
\(x=(\alpha, \beta)\)
and \(p^*=(\tilde p_1 - 1, tilde_2 - 1, p_3)\):

\begin{equation}\label{eqn:linear_algebraic_equation_for_p}
    Cx= p^*
\end{equation}

\(C\) corresponds to equations
(\ref{eqn:condition_for_tilde_p1}-\ref{eqn:condition_for_tilde_p3}) and is
given by:

\begin{equation}\label{eqn:definition_of_C}
    C =
    \begin{bmatrix}
        R - P & R- P \\
        S - P & T- P \\
        T - P & S- P \\
    \end{bmatrix}
\end{equation}

Note that in general, equation (\ref{eqn:linear_algebraic_equation_for_p}) will
not necessarily have a solution. From the Rouch\'{e}-Capelli theorem if there is
a solution it is unique as \(\text{rank}(C)=2\) which is the dimension of the
variable \(x\). The best fitting \(x\) is found by minimizing:

\begin{equation}\label{eqn:r_squared}
    \text{SSError} = \|C x- p^*\|_2^2 = \sum_{i=1}^{3}\left((C\bar x)_i-p_i^*\right)^2
\end{equation}

Note that \(\text{SSError}\), which is the square of the Frobenius
norm~\cite{Golub2013}, becomes a measure of how close a strategy is to being an
extortionate strategy. Suspicion
of extortion then corresponds to a threshold on \(\text{SSError}\).

By observing interactions (human or otherwise), their memory one representation
can be inferred and this approach can be used to recognise extortionate
behaviour. The notion of comparing theoretic and actual plays of the IPD is not
novel, see for example~\cite{Rand2013}. Immediately it is noted that if the
environment is noisy~\cite{Wu1995} then no strategy can be considered to be
extortionate as \(p_4>0\).

In the next section, this idea will be illustrated by observing the interactions
that take place in a computer based tournament of the IPD\@.

\section{Numerical experiments}\label{sec:numerical-experiments}

In~\cite{Stewart2012} results from a tournament with
\input{./assets/tex/number_of_stewart_plotkin_strategies/main.tex} strategies,
was presented with specific consideration given to ZD strategies. This
tournament is reproduced here using the Axelrod-Python
project~\cite{Knight2016}. To obtain a good measure of the corresponding
transition rates for each strategy all matches have been run for
\input{assets/tex/number_of_turns/main.tex} turns and every match has been
repeated \input{assets/tex/number_of_repetitions/main.tex} times. All of this
interaction data is available at~\cite{vincent_knight_2018_1297075}. A good
match between the inferred Markov chain and the state distribution of the actual
interactions has been verified. Data for this is presented in the supplementary
materials.

Figure~\ref{fig:SSError_overall_in_stewart_plotkin} shows the \(\text{SSError}\)
values for all the strategies in the tournament, as reported
in~\cite{Stewart2012} the extortionate strategy (which has an expected
\(\text{SSError}\) approximately 0) gains a large number of wins.

\begin{figure}[!htbp]
    \centering
    \includegraphics[width=.8\textwidth]{./assets/img/SSError_overall_in_stewart_plotkin/main.pdf}
    \caption{\(\text{SSError}\) and state probabilities for the strategies
        of~\cite{Stewart2012}, ordered both by number of wins and overall score.
        Note that \(P(DC)\) is not shown as it corresponds to the transpose of
        \(P(CD)\). Cooperator and Defector are omitted as they do not visit all
        the states.}
    \label{fig:SSError_overall_in_stewart_plotkin}
\end{figure}

Here, the work of~\cite{Stewart2012} is extended by investigating a tournament
with \input{assets/tex/number_of_full_strategies/main.tex}
strategies.

The results of this analysis are shown in
Figure~\ref{fig:SSError_and_probabilities_in_full}. The top ranking strategies
by number of wins seem to be extortionate (but not against all strategies) and
it can be seen that a small sub group of strategies achieve mutual defection.
All the top ranking strategies according to score achieve mutual cooperation and
do not extort each other, however they
\textbf{do} exhibit extortionate behaviour towards a number of the lower ranking
strategies.

\begin{figure}[!htbp]
    \centering
    \includegraphics[width=.8\textwidth]{./assets/img/SSError_and_probabilities_in_full/main.pdf}
    \caption{\(\text{SSError}\) for the strategies for the full tournament. Only
    strategy interactions for which \(p_4=0\) and \(\chi>1\) are displayed.}
    \label{fig:SSError_and_probabilities_in_full}
\end{figure}

\section{Conclusion}\label{sec:conclusion}

This work defines an approach to measure whether or not a player is playing a
strategy that corresponds to an extortionate strategy as defined
in~\cite{Press2012}: a mathematical model for suspicion. Indeed, all
extortionate strategies have been
 classified as lying on a triangular plane.
This rigorous classification fails to be robust to small measurement error, thus
a statistical approach is proposed.
This is done through a linear algebraic approach for approximating the solution
of a linear system. Using this, a large number of pairwise interactions is
simulated and in fact very few strategies are found to act extortionately.

The work of~\cite{Press2012}, whilst showing that a clever approach to taking
advantage of another memory one strategy exists: this is incomplete. Whilst the
elegance of this result is very attractive, just as the simplicity of the
victory of Tit For Tat in Axelrod's original tournaments was, it is incomplete.
Extortionate strategies achieve a high number of wins but they do not
achieve a high score which corresponds to the fitness landscape in an
evolutionary sense. From the large number of interactions a payoff matrix \(S\)
can be measured where \(S_{ij}\) denotes the score (using standard values of
\((R, S, T, P) = (3, 0, 5, 1)\)) of the \(i\)th strategy
against the \(j\)th strategy. Using this, the replicator equation
describes the evolution of the system based on a population density fitness
function:

\begin{equation}\label{eqn:replicator_dynamics}
    \frac{dx}{dt} = x(S-x^TS x)
\end{equation}

Equation (\ref{eqn:replicator_dynamics}) is solved numerically through an
integration technique described in~\cite{Petzold1983} and
Figure~\ref{fig:replicator_dynamics} shows the evolution of the distribution of
the system: the various strategies are ranked by scores. It is clear to see that
only the high ranking strategies survive the evolutionary process (in fact,
only \input{./assets/img/replicator_dynamics/main.tex}
have a final distribution greater than \(10 ^ {-2}\)). This confirms the
findings of~\cite{Moran1707} in which sophisticated strategies resist
evolutionary invasion of shorter memory strategies. Recalling
Figure~\ref{fig:SSError_and_probabilities_in_full} this demonstrates that:

\begin{itemize}
    \item Cooperation emerges through the evolutionary process: the high scoring
        strategies do not exhibit extortionate behaviour towards each other.
    \item Extortionate strategies do not survive the evolutionary process.
\end{itemize}

\begin{figure}[!htbp]
    \centering
    \includegraphics[width=.8\textwidth]{./assets/img/replicator_dynamics/main.pdf}
    \caption{Numerical simulation of the replicator equation
    (\ref{eqn:replicator_dynamics}): strategies are ordered by score, only the strategies with a high score survive the evolutionary process.}
    \label{fig:replicator_dynamics}
\end{figure}

This work can be used to classify plays of the IPD\@: data can be collected from
actual interactions (in lab or in the field). Furthermore, this allows for a
classification method similar to the notion of fingerprinting presented
in~\cite{Ashlock2008}. Trained strategies can potentially be classified as
extortionate or not or it could be possible to even constrain the reinforcement
learning approaches that are becoming prevalent in the literature.
Alternatively, this mathematical approach for recognising extortion could be
used in sophisticated strategies to defend against invasion. Arguably, some of
the strategies considered here exhibit this behaviour, indeed as described
in~\cite{Harper2017}, the top ranking strategies in the full tournament are
obtained using evolutionary reinforcement learning techniques, thus, suspicion
of extortionate behaviour could in fact be an evolutionary trait.

\section*{Acknowledgements}

The following open source software libraries were used in this research:

\begin{itemize}
    \item The Axelrod ~\cite{Knight2016, Knight2018} library (IPD strategies and
        tournaments).
    \item The sympy library~\cite{Meurer2017} (verification of all symbolic
        calculations).
    \item The matplotlib~\cite{Droettboom2018} library (visualisation).
    \item The pandas~\cite{Structures2010}, dask~\cite{Dask2016} and
        NumPy~\cite{Oliphant2015} libraries (data manipulation).
    \item The SciPy~\cite{Jones2001} library (numerical integration of the
        replicator equation).
\end{itemize}

This work was performed using the computational facilities of the Advanced
Research Computing @ Cardiff (ARCCA) Division, Cardiff University.

\printbibliography

\newpage
\section*{Supplementary materials}

\includepdf{assets/pdf/proof_of_form_of_extortionate_strategies/main.pdf}

\newpage

Using the pair wise interactions the transition rates \(p,
q\) can be measured and the steady state probabilities inferred and compared to
the actual probabilities of each state.
This is done numerically by computing the singular eigenvector of the
matrix \(A\) \cite{Stewart2009}:

\[
    A =
    \begin{bmatrix}
        p_1 q_1 & p_1 (1 - q_1) & (1 - p_1) q_1 & (1 -p_1) (1 - q_1) \\
        p_2 q_2 & p_2 (1 - q_2) & (1 - p_2) q_2 & (1 -p_2) (1 - q_2) \\
        p_3 q_3 & p_3 (1 - q_3) & (1 - p_3) q_3 & (1 -p_3) (1 - q_3) \\
        p_4 q_4 & p_4 (1 - q_4) & (1 - p_4) q_4 & (1 -p_4) (1 - q_4) \\
    \end{bmatrix}
\]

Figure~\ref{fig:computed_probabilities_vs_theoretic_probabilities} shows a
regression line fitted to every pairwise interaction with a reported
\(\text{SSError}\) value (pairwise interactions with missing states were
omitted). This serves to validate the approach: a part from some edge cases the
relationship is consistent.

\begin{figure}[!htbp]
    \centering
    \includegraphics[width=.8\textwidth]{./assets/img/computed_probabilities_vs_theoretic_probabilities/main.pdf}
    \caption{The
        relationship between the steady state probabilities inferred from the
        measured transitions and the actual steady state probabilities. A linear
        regression line is included validating the approach.}
    \label{fig:computed_probabilities_vs_theoretic_probabilities}
\end{figure}


\end{document}

    strategies is considered. In this setting
    the most highly performing strategies do not play in an extortionate way
    against each other but do against lower performing strategies.
    This suggests that whilst the theory of Zero Determinant strategies
    indicates that memory is not of fundamental importance to the evolution of
    cooperative behaviour, this is incomplete.
\end{abstract}

\section{Introduction}\label{sec:introduction}

Agent based game theoretic models have become a stalwart of the underpinning
mathematics of interactive behaviours. One of the major pieces of work
in this area is the pair of original computer tournaments run by Robert
Axelrod~\cite{Axelrod1980, Axelrod1980a}. These tournaments pitted submitted
computer strategies against each other in plays of the Iterated Prisoner's
Dilemma. A common game where agents can choose to pay a slight cost to their
immediate utility in the hope of building a reputation. This has been used in
economic and evolutionary game theory to understand the evolution of cooperative
behaviour.

Recently, a class of strategies was described in~\cite{Press2012} that can
provably extort any given opponent. In~\cite{Hilbe2013, Moran1707} some
questions have already been asked about the true effectiveness of these
strategies in an evolutionary setting. Here another question is asked: is it
possible to recognise this extortionate behaviour? A mathematical procedure for
suspicion is presented: in the same way that the continued actions of an
extortionate individual might raise suspicion.

This work makes use of the Axelrod Python library~\cite{Knight2018, Knight2016}
with a large number of Prisoner Dilemma strategies available to give an
extensive numerical example of the ideas presented.  The approach is presented
in Section~\ref{sec:delta-zd-strategies}.  All of the code and data discussed
in Section~\ref{sec:numerical-experiments} is open sourced, archived and
written according to best scientific principles~\cite{Wilson2014}. The data
archive can be found at~\cite{vincent_knight_2018_1297075}.

\section{Recognising Extortion}\label{sec:delta-zd-strategies}

In~\cite{Press2012}, given a match between 2 memory-one strategies, the concept
of Zero Determinant (ZD) strategies is introduced. The main result of that paper
shows that given two memory one players \(p, q\in\mathbb{R}^4\) a linear
relationship between the players' scores could be forced by one of the players.

Using the notation of~\cite{Press2012}, assuming the utilities for player \(p\)
are given by \(S_x=(R, S, T, P)\) and for player \(q\) by \(S_y=(R, T, S, P)\)
and that the stationary scores of each player is given by \(S_X\) and \(S_Y\)
respectively. The main result of~\cite{Press2012} is that if

\begin{equation}\label{eqn:linear_relationship_for_p}
    \tilde p=\alpha S_x + \beta S_y + \gamma
\end{equation}

or

\begin{equation}\label{eqn:linear_relationship_for_q}
    \tilde q=\alpha S_x + \beta S_y + \gamma
\end{equation}

where \(\tilde p = (1 - p_1, 1 - p_2, p_3, p_4)\) and
\(\tilde q = (1 - q_1, 1 - q_2, q_3, q_4)\) then:

\begin{equation}
    \alpha S_X + \beta S_Y + \gamma = 0
\end{equation}

In~\cite{Press2012} a particular type of ZD strategy is defined: extortionate
strategies. If:

\begin{equation}\label{eqn:constraint_for_extortion}
    \gamma = - P(\alpha + \beta)
\end{equation}

then the player can ensure they get a score \(\chi\) times
larger than the opponent. This extortion coefficient is given by:

\begin{equation}\label{eqn:definition_of_chi}
    \chi=\frac{-\beta}{\alpha}
\end{equation}

Thus, if (\ref{eqn:constraint_for_extortion}) holds and \(\chi >1\) a player is
said to extort their opponent.
Here, the reverse problem is considered: given a
\(p\in\mathbb{R}^4\) how does one identify \(\alpha, \beta\) if they
exist and is the strategy in fact acting in an extortionate way?

These conditions correspond to:

\begin{align}
    \tilde p_1 & = \alpha R + \beta R - P (\alpha + \beta)
            \label{eqn:condition_for_tilde_p1}\\
    \tilde p_2 & = \alpha S + \beta T - P (\alpha + \beta)
            \label{eqn:condition_for_tilde_p2}\\
    \tilde p_3 & = \alpha T + \beta S - P (\alpha + \beta)
            \label{eqn:condition_for_tilde_p3}\\
    \tilde p_4 & = \alpha P + \beta P - P (\alpha + \beta)
            \label{eqn:condition_for_tilde_p4}
\end{align}

Equation (\ref{eqn:condition_for_tilde_p4}) ensures that \(p_4=\tilde p_4=0\).
Equations (\ref{eqn:condition_for_tilde_p1}-\ref{eqn:condition_for_tilde_p3})
can be used to eliminate \(\alpha, \beta\), giving:

\begin{equation}\label{eqn:planar_definition_of_extortion}
    \tilde p_1 = \frac{(R - P)(\tilde p_2 + \tilde p_3)}{S + T - 2P}
\end{equation}

with:

\begin{equation}\label{eqn:definition_of_chi}
    \chi = \frac{\tilde p_2 (P - T) + \tilde p_3 (S - P)}
                {\tilde p_2 (P - S) + \tilde p_3 (T - P)}
\end{equation}

Given a strategy \(p\in\mathbb{R}^{4\times 1}\) equations
(\ref{eqn:condition_for_tilde_p4}), (\ref{eqn:planar_definition_of_extortion}-\ref{eqn:definition_of_chi}) can be used to check if
a strategy is extortionate. The conditions correspond to:

\begin{align}
    p_1 & = \frac{(R-P)(p_2 + p_3) - R + T + S - P}{S + T - 2P}
     \label{eqn:condition_for_p1}\\
    p_4 & = 0 \label{eqn:condition_for_p4}\\
    1 & > p_2 + p_3\label{eqn:condition_for_chi}
\end{align}

The algebraic steps necessary to prove these results are available in the
supporting materials.

All extortionate strategies reside on a triangular (\ref{eqn:condition_for_chi})
plane (\ref{eqn:condition_for_p1}) in 3 dimensions (\ref{eqn:condition_for_p4}).
Using this formulation it can be seen that a necessary (but not sufficient)
condition for an extortionate strategy is that it cooperates on average less
than 50\% of the time when in a state of disagreement with the opponent.

As an example, consider the known extortionate strategy \(p=(8 / 9, 1 / 2, 1 /
3, 0)\) from~\cite{Stewart2012} which is referred to as \texttt{Extort-2}. In
this case, for the standard values of \((R, T, S, P)\) constraint
(\ref{eqn:condition_for_p1}) corresponds to:

\begin{equation}
    p_1 = \frac{2(p_2 + p_3) + 1}{3}
\end{equation}

It is clear that in this case all constraints hold.

This approach could in fact be used to confirm that a given strategy is acting
in an extortionate manner even if it is not a memory one strategy. However, in
practice, if a closed form for \(p\) is not known, then due to measurement
and/or numerical error this would not work.

This problem can be written in the following linear algebraic form where
\(x=(\alpha, \beta)\)
and \(p^*=(\tilde p_1 - 1, tilde_2 - 1, p_3)\):

\begin{equation}\label{eqn:linear_algebraic_equation_for_p}
    Cx= p^*
\end{equation}

\(C\) corresponds to equations
(\ref{eqn:condition_for_tilde_p1}-\ref{eqn:condition_for_tilde_p3}) and is
given by:

\begin{equation}\label{eqn:definition_of_C}
    C =
    \begin{bmatrix}
        R - P & R- P \\
        S - P & T- P \\
        T - P & S- P \\
    \end{bmatrix}
\end{equation}

Note that in general, equation (\ref{eqn:linear_algebraic_equation_for_p}) will
not necessarily have a solution. From the Rouch\'{e}-Capelli theorem if there is
a solution it is unique as \(\text{rank}(C)=2\) which is the dimension of the
variable \(x\). The best fitting \(x\) is found by minimizing:

\begin{equation}\label{eqn:r_squared}
    \text{SSError} = \|C x- p^*\|_2^2 = \sum_{i=1}^{3}\left((C\bar x)_i-p_i^*\right)^2
\end{equation}

Note that \(\text{SSError}\), which is the square of the Frobenius
norm~\cite{Golub2013}, becomes a measure of how close a strategy is to being an
extortionate strategy. Suspicion
of extortion then corresponds to a threshold on \(\text{SSError}\).

By observing interactions (human or otherwise), their memory one representation
can be inferred and this approach can be used to recognise extortionate
behaviour. The notion of comparing theoretic and actual plays of the IPD is not
novel, see for example~\cite{Rand2013}. Immediately it is noted that if the
environment is noisy~\cite{Wu1995} then no strategy can be considered to be
extortionate as \(p_4>0\).

In the next section, this idea will be illustrated by observing the interactions
that take place in a computer based tournament of the IPD\@.

\section{Numerical experiments}\label{sec:numerical-experiments}

In~\cite{Stewart2012} results from a tournament with
\documentclass[a4paper]{article}

\usepackage{amsmath}
\usepackage{amssymb}
\usepackage[margin=1.5cm,
            includefoot,
            footskip=30pt]{geometry}
\usepackage{layout}
\usepackage{graphicx}
\usepackage{subcaption}

\usepackage{biblatex}
\usepackage{pdfpages}

\bibliography{main.bib}

\title{Suspicion: Recognising and evaluating the effectiveness
       of extortion in the Iterated Prisoner's Dilemma}
\author{Vincent A. Knight \and Nikoleta E. Glynatsi}
\date{\today}



\begin{document}

\maketitle

\begin{abstract}
    The Iterated Prisoner's Dilemma is a model for rational and evolutionary
    interactive behaviour. It has applications both in the study of human social
    behaviour as well as in biology.
    It is used to understand when and how a rational individual might
    accept an immediate cost to their own utility for the direct benefit of
    another.

    Much attention has been given to a class of strategies called
    Zero Determinant strategies. It has been theoretically shown that these
    strategies can ``extort'' any player.

    In this work, an approach to identify if observed strategies are playing in
    an extortionate way is described. Furthermore, experimental analysis of
    a large tournament with \input{assets/tex/number_of_full_strategies/main.tex}
    strategies is considered. In this setting
    the most highly performing strategies do not play in an extortionate way
    against each other but do against lower performing strategies.
    This suggests that whilst the theory of Zero Determinant strategies
    indicates that memory is not of fundamental importance to the evolution of
    cooperative behaviour, this is incomplete.
\end{abstract}

\section{Introduction}\label{sec:introduction}

Agent based game theoretic models have become a stalwart of the underpinning
mathematics of interactive behaviours. One of the major pieces of work
in this area is the pair of original computer tournaments run by Robert
Axelrod~\cite{Axelrod1980, Axelrod1980a}. These tournaments pitted submitted
computer strategies against each other in plays of the Iterated Prisoner's
Dilemma. A common game where agents can choose to pay a slight cost to their
immediate utility in the hope of building a reputation. This has been used in
economic and evolutionary game theory to understand the evolution of cooperative
behaviour.

Recently, a class of strategies was described in~\cite{Press2012} that can
provably extort any given opponent. In~\cite{Hilbe2013, Moran1707} some
questions have already been asked about the true effectiveness of these
strategies in an evolutionary setting. Here another question is asked: is it
possible to recognise this extortionate behaviour? A mathematical procedure for
suspicion is presented: in the same way that the continued actions of an
extortionate individual might raise suspicion.

This work makes use of the Axelrod Python library~\cite{Knight2018, Knight2016}
with a large number of Prisoner Dilemma strategies available to give an
extensive numerical example of the ideas presented.  The approach is presented
in Section~\ref{sec:delta-zd-strategies}.  All of the code and data discussed
in Section~\ref{sec:numerical-experiments} is open sourced, archived and
written according to best scientific principles~\cite{Wilson2014}. The data
archive can be found at~\cite{vincent_knight_2018_1297075}.

\section{Recognising Extortion}\label{sec:delta-zd-strategies}

In~\cite{Press2012}, given a match between 2 memory-one strategies, the concept
of Zero Determinant (ZD) strategies is introduced. The main result of that paper
shows that given two memory one players \(p, q\in\mathbb{R}^4\) a linear
relationship between the players' scores could be forced by one of the players.

Using the notation of~\cite{Press2012}, assuming the utilities for player \(p\)
are given by \(S_x=(R, S, T, P)\) and for player \(q\) by \(S_y=(R, T, S, P)\)
and that the stationary scores of each player is given by \(S_X\) and \(S_Y\)
respectively. The main result of~\cite{Press2012} is that if

\begin{equation}\label{eqn:linear_relationship_for_p}
    \tilde p=\alpha S_x + \beta S_y + \gamma
\end{equation}

or

\begin{equation}\label{eqn:linear_relationship_for_q}
    \tilde q=\alpha S_x + \beta S_y + \gamma
\end{equation}

where \(\tilde p = (1 - p_1, 1 - p_2, p_3, p_4)\) and
\(\tilde q = (1 - q_1, 1 - q_2, q_3, q_4)\) then:

\begin{equation}
    \alpha S_X + \beta S_Y + \gamma = 0
\end{equation}

In~\cite{Press2012} a particular type of ZD strategy is defined: extortionate
strategies. If:

\begin{equation}\label{eqn:constraint_for_extortion}
    \gamma = - P(\alpha + \beta)
\end{equation}

then the player can ensure they get a score \(\chi\) times
larger than the opponent. This extortion coefficient is given by:

\begin{equation}\label{eqn:definition_of_chi}
    \chi=\frac{-\beta}{\alpha}
\end{equation}

Thus, if (\ref{eqn:constraint_for_extortion}) holds and \(\chi >1\) a player is
said to extort their opponent.
Here, the reverse problem is considered: given a
\(p\in\mathbb{R}^4\) how does one identify \(\alpha, \beta\) if they
exist and is the strategy in fact acting in an extortionate way?

These conditions correspond to:

\begin{align}
    \tilde p_1 & = \alpha R + \beta R - P (\alpha + \beta)
            \label{eqn:condition_for_tilde_p1}\\
    \tilde p_2 & = \alpha S + \beta T - P (\alpha + \beta)
            \label{eqn:condition_for_tilde_p2}\\
    \tilde p_3 & = \alpha T + \beta S - P (\alpha + \beta)
            \label{eqn:condition_for_tilde_p3}\\
    \tilde p_4 & = \alpha P + \beta P - P (\alpha + \beta)
            \label{eqn:condition_for_tilde_p4}
\end{align}

Equation (\ref{eqn:condition_for_tilde_p4}) ensures that \(p_4=\tilde p_4=0\).
Equations (\ref{eqn:condition_for_tilde_p1}-\ref{eqn:condition_for_tilde_p3})
can be used to eliminate \(\alpha, \beta\), giving:

\begin{equation}\label{eqn:planar_definition_of_extortion}
    \tilde p_1 = \frac{(R - P)(\tilde p_2 + \tilde p_3)}{S + T - 2P}
\end{equation}

with:

\begin{equation}\label{eqn:definition_of_chi}
    \chi = \frac{\tilde p_2 (P - T) + \tilde p_3 (S - P)}
                {\tilde p_2 (P - S) + \tilde p_3 (T - P)}
\end{equation}

Given a strategy \(p\in\mathbb{R}^{4\times 1}\) equations
(\ref{eqn:condition_for_tilde_p4}), (\ref{eqn:planar_definition_of_extortion}-\ref{eqn:definition_of_chi}) can be used to check if
a strategy is extortionate. The conditions correspond to:

\begin{align}
    p_1 & = \frac{(R-P)(p_2 + p_3) - R + T + S - P}{S + T - 2P}
     \label{eqn:condition_for_p1}\\
    p_4 & = 0 \label{eqn:condition_for_p4}\\
    1 & > p_2 + p_3\label{eqn:condition_for_chi}
\end{align}

The algebraic steps necessary to prove these results are available in the
supporting materials.

All extortionate strategies reside on a triangular (\ref{eqn:condition_for_chi})
plane (\ref{eqn:condition_for_p1}) in 3 dimensions (\ref{eqn:condition_for_p4}).
Using this formulation it can be seen that a necessary (but not sufficient)
condition for an extortionate strategy is that it cooperates on average less
than 50\% of the time when in a state of disagreement with the opponent.

As an example, consider the known extortionate strategy \(p=(8 / 9, 1 / 2, 1 /
3, 0)\) from~\cite{Stewart2012} which is referred to as \texttt{Extort-2}. In
this case, for the standard values of \((R, T, S, P)\) constraint
(\ref{eqn:condition_for_p1}) corresponds to:

\begin{equation}
    p_1 = \frac{2(p_2 + p_3) + 1}{3}
\end{equation}

It is clear that in this case all constraints hold.

This approach could in fact be used to confirm that a given strategy is acting
in an extortionate manner even if it is not a memory one strategy. However, in
practice, if a closed form for \(p\) is not known, then due to measurement
and/or numerical error this would not work.

This problem can be written in the following linear algebraic form where
\(x=(\alpha, \beta)\)
and \(p^*=(\tilde p_1 - 1, tilde_2 - 1, p_3)\):

\begin{equation}\label{eqn:linear_algebraic_equation_for_p}
    Cx= p^*
\end{equation}

\(C\) corresponds to equations
(\ref{eqn:condition_for_tilde_p1}-\ref{eqn:condition_for_tilde_p3}) and is
given by:

\begin{equation}\label{eqn:definition_of_C}
    C =
    \begin{bmatrix}
        R - P & R- P \\
        S - P & T- P \\
        T - P & S- P \\
    \end{bmatrix}
\end{equation}

Note that in general, equation (\ref{eqn:linear_algebraic_equation_for_p}) will
not necessarily have a solution. From the Rouch\'{e}-Capelli theorem if there is
a solution it is unique as \(\text{rank}(C)=2\) which is the dimension of the
variable \(x\). The best fitting \(x\) is found by minimizing:

\begin{equation}\label{eqn:r_squared}
    \text{SSError} = \|C x- p^*\|_2^2 = \sum_{i=1}^{3}\left((C\bar x)_i-p_i^*\right)^2
\end{equation}

Note that \(\text{SSError}\), which is the square of the Frobenius
norm~\cite{Golub2013}, becomes a measure of how close a strategy is to being an
extortionate strategy. Suspicion
of extortion then corresponds to a threshold on \(\text{SSError}\).

By observing interactions (human or otherwise), their memory one representation
can be inferred and this approach can be used to recognise extortionate
behaviour. The notion of comparing theoretic and actual plays of the IPD is not
novel, see for example~\cite{Rand2013}. Immediately it is noted that if the
environment is noisy~\cite{Wu1995} then no strategy can be considered to be
extortionate as \(p_4>0\).

In the next section, this idea will be illustrated by observing the interactions
that take place in a computer based tournament of the IPD\@.

\section{Numerical experiments}\label{sec:numerical-experiments}

In~\cite{Stewart2012} results from a tournament with
\input{./assets/tex/number_of_stewart_plotkin_strategies/main.tex} strategies,
was presented with specific consideration given to ZD strategies. This
tournament is reproduced here using the Axelrod-Python
project~\cite{Knight2016}. To obtain a good measure of the corresponding
transition rates for each strategy all matches have been run for
\input{assets/tex/number_of_turns/main.tex} turns and every match has been
repeated \input{assets/tex/number_of_repetitions/main.tex} times. All of this
interaction data is available at~\cite{vincent_knight_2018_1297075}. A good
match between the inferred Markov chain and the state distribution of the actual
interactions has been verified. Data for this is presented in the supplementary
materials.

Figure~\ref{fig:SSError_overall_in_stewart_plotkin} shows the \(\text{SSError}\)
values for all the strategies in the tournament, as reported
in~\cite{Stewart2012} the extortionate strategy (which has an expected
\(\text{SSError}\) approximately 0) gains a large number of wins.

\begin{figure}[!htbp]
    \centering
    \includegraphics[width=.8\textwidth]{./assets/img/SSError_overall_in_stewart_plotkin/main.pdf}
    \caption{\(\text{SSError}\) and state probabilities for the strategies
        of~\cite{Stewart2012}, ordered both by number of wins and overall score.
        Note that \(P(DC)\) is not shown as it corresponds to the transpose of
        \(P(CD)\). Cooperator and Defector are omitted as they do not visit all
        the states.}
    \label{fig:SSError_overall_in_stewart_plotkin}
\end{figure}

Here, the work of~\cite{Stewart2012} is extended by investigating a tournament
with \input{assets/tex/number_of_full_strategies/main.tex}
strategies.

The results of this analysis are shown in
Figure~\ref{fig:SSError_and_probabilities_in_full}. The top ranking strategies
by number of wins seem to be extortionate (but not against all strategies) and
it can be seen that a small sub group of strategies achieve mutual defection.
All the top ranking strategies according to score achieve mutual cooperation and
do not extort each other, however they
\textbf{do} exhibit extortionate behaviour towards a number of the lower ranking
strategies.

\begin{figure}[!htbp]
    \centering
    \includegraphics[width=.8\textwidth]{./assets/img/SSError_and_probabilities_in_full/main.pdf}
    \caption{\(\text{SSError}\) for the strategies for the full tournament. Only
    strategy interactions for which \(p_4=0\) and \(\chi>1\) are displayed.}
    \label{fig:SSError_and_probabilities_in_full}
\end{figure}

\section{Conclusion}\label{sec:conclusion}

This work defines an approach to measure whether or not a player is playing a
strategy that corresponds to an extortionate strategy as defined
in~\cite{Press2012}: a mathematical model for suspicion. Indeed, all
extortionate strategies have been
 classified as lying on a triangular plane.
This rigorous classification fails to be robust to small measurement error, thus
a statistical approach is proposed.
This is done through a linear algebraic approach for approximating the solution
of a linear system. Using this, a large number of pairwise interactions is
simulated and in fact very few strategies are found to act extortionately.

The work of~\cite{Press2012}, whilst showing that a clever approach to taking
advantage of another memory one strategy exists: this is incomplete. Whilst the
elegance of this result is very attractive, just as the simplicity of the
victory of Tit For Tat in Axelrod's original tournaments was, it is incomplete.
Extortionate strategies achieve a high number of wins but they do not
achieve a high score which corresponds to the fitness landscape in an
evolutionary sense. From the large number of interactions a payoff matrix \(S\)
can be measured where \(S_{ij}\) denotes the score (using standard values of
\((R, S, T, P) = (3, 0, 5, 1)\)) of the \(i\)th strategy
against the \(j\)th strategy. Using this, the replicator equation
describes the evolution of the system based on a population density fitness
function:

\begin{equation}\label{eqn:replicator_dynamics}
    \frac{dx}{dt} = x(S-x^TS x)
\end{equation}

Equation (\ref{eqn:replicator_dynamics}) is solved numerically through an
integration technique described in~\cite{Petzold1983} and
Figure~\ref{fig:replicator_dynamics} shows the evolution of the distribution of
the system: the various strategies are ranked by scores. It is clear to see that
only the high ranking strategies survive the evolutionary process (in fact,
only \input{./assets/img/replicator_dynamics/main.tex}
have a final distribution greater than \(10 ^ {-2}\)). This confirms the
findings of~\cite{Moran1707} in which sophisticated strategies resist
evolutionary invasion of shorter memory strategies. Recalling
Figure~\ref{fig:SSError_and_probabilities_in_full} this demonstrates that:

\begin{itemize}
    \item Cooperation emerges through the evolutionary process: the high scoring
        strategies do not exhibit extortionate behaviour towards each other.
    \item Extortionate strategies do not survive the evolutionary process.
\end{itemize}

\begin{figure}[!htbp]
    \centering
    \includegraphics[width=.8\textwidth]{./assets/img/replicator_dynamics/main.pdf}
    \caption{Numerical simulation of the replicator equation
    (\ref{eqn:replicator_dynamics}): strategies are ordered by score, only the strategies with a high score survive the evolutionary process.}
    \label{fig:replicator_dynamics}
\end{figure}

This work can be used to classify plays of the IPD\@: data can be collected from
actual interactions (in lab or in the field). Furthermore, this allows for a
classification method similar to the notion of fingerprinting presented
in~\cite{Ashlock2008}. Trained strategies can potentially be classified as
extortionate or not or it could be possible to even constrain the reinforcement
learning approaches that are becoming prevalent in the literature.
Alternatively, this mathematical approach for recognising extortion could be
used in sophisticated strategies to defend against invasion. Arguably, some of
the strategies considered here exhibit this behaviour, indeed as described
in~\cite{Harper2017}, the top ranking strategies in the full tournament are
obtained using evolutionary reinforcement learning techniques, thus, suspicion
of extortionate behaviour could in fact be an evolutionary trait.

\section*{Acknowledgements}

The following open source software libraries were used in this research:

\begin{itemize}
    \item The Axelrod ~\cite{Knight2016, Knight2018} library (IPD strategies and
        tournaments).
    \item The sympy library~\cite{Meurer2017} (verification of all symbolic
        calculations).
    \item The matplotlib~\cite{Droettboom2018} library (visualisation).
    \item The pandas~\cite{Structures2010}, dask~\cite{Dask2016} and
        NumPy~\cite{Oliphant2015} libraries (data manipulation).
    \item The SciPy~\cite{Jones2001} library (numerical integration of the
        replicator equation).
\end{itemize}

This work was performed using the computational facilities of the Advanced
Research Computing @ Cardiff (ARCCA) Division, Cardiff University.

\printbibliography

\newpage
\section*{Supplementary materials}

\includepdf{assets/pdf/proof_of_form_of_extortionate_strategies/main.pdf}

\newpage

Using the pair wise interactions the transition rates \(p,
q\) can be measured and the steady state probabilities inferred and compared to
the actual probabilities of each state.
This is done numerically by computing the singular eigenvector of the
matrix \(A\) \cite{Stewart2009}:

\[
    A =
    \begin{bmatrix}
        p_1 q_1 & p_1 (1 - q_1) & (1 - p_1) q_1 & (1 -p_1) (1 - q_1) \\
        p_2 q_2 & p_2 (1 - q_2) & (1 - p_2) q_2 & (1 -p_2) (1 - q_2) \\
        p_3 q_3 & p_3 (1 - q_3) & (1 - p_3) q_3 & (1 -p_3) (1 - q_3) \\
        p_4 q_4 & p_4 (1 - q_4) & (1 - p_4) q_4 & (1 -p_4) (1 - q_4) \\
    \end{bmatrix}
\]

Figure~\ref{fig:computed_probabilities_vs_theoretic_probabilities} shows a
regression line fitted to every pairwise interaction with a reported
\(\text{SSError}\) value (pairwise interactions with missing states were
omitted). This serves to validate the approach: a part from some edge cases the
relationship is consistent.

\begin{figure}[!htbp]
    \centering
    \includegraphics[width=.8\textwidth]{./assets/img/computed_probabilities_vs_theoretic_probabilities/main.pdf}
    \caption{The
        relationship between the steady state probabilities inferred from the
        measured transitions and the actual steady state probabilities. A linear
        regression line is included validating the approach.}
    \label{fig:computed_probabilities_vs_theoretic_probabilities}
\end{figure}


\end{document}
 strategies,
was presented with specific consideration given to ZD strategies. This
tournament is reproduced here using the Axelrod-Python
project~\cite{Knight2016}. To obtain a good measure of the corresponding
transition rates for each strategy all matches have been run for
\documentclass[a4paper]{article}

\usepackage{amsmath}
\usepackage{amssymb}
\usepackage[margin=1.5cm,
            includefoot,
            footskip=30pt]{geometry}
\usepackage{layout}
\usepackage{graphicx}
\usepackage{subcaption}

\usepackage{biblatex}
\usepackage{pdfpages}

\bibliography{main.bib}

\title{Suspicion: Recognising and evaluating the effectiveness
       of extortion in the Iterated Prisoner's Dilemma}
\author{Vincent A. Knight \and Nikoleta E. Glynatsi}
\date{\today}



\begin{document}

\maketitle

\begin{abstract}
    The Iterated Prisoner's Dilemma is a model for rational and evolutionary
    interactive behaviour. It has applications both in the study of human social
    behaviour as well as in biology.
    It is used to understand when and how a rational individual might
    accept an immediate cost to their own utility for the direct benefit of
    another.

    Much attention has been given to a class of strategies called
    Zero Determinant strategies. It has been theoretically shown that these
    strategies can ``extort'' any player.

    In this work, an approach to identify if observed strategies are playing in
    an extortionate way is described. Furthermore, experimental analysis of
    a large tournament with \input{assets/tex/number_of_full_strategies/main.tex}
    strategies is considered. In this setting
    the most highly performing strategies do not play in an extortionate way
    against each other but do against lower performing strategies.
    This suggests that whilst the theory of Zero Determinant strategies
    indicates that memory is not of fundamental importance to the evolution of
    cooperative behaviour, this is incomplete.
\end{abstract}

\section{Introduction}\label{sec:introduction}

Agent based game theoretic models have become a stalwart of the underpinning
mathematics of interactive behaviours. One of the major pieces of work
in this area is the pair of original computer tournaments run by Robert
Axelrod~\cite{Axelrod1980, Axelrod1980a}. These tournaments pitted submitted
computer strategies against each other in plays of the Iterated Prisoner's
Dilemma. A common game where agents can choose to pay a slight cost to their
immediate utility in the hope of building a reputation. This has been used in
economic and evolutionary game theory to understand the evolution of cooperative
behaviour.

Recently, a class of strategies was described in~\cite{Press2012} that can
provably extort any given opponent. In~\cite{Hilbe2013, Moran1707} some
questions have already been asked about the true effectiveness of these
strategies in an evolutionary setting. Here another question is asked: is it
possible to recognise this extortionate behaviour? A mathematical procedure for
suspicion is presented: in the same way that the continued actions of an
extortionate individual might raise suspicion.

This work makes use of the Axelrod Python library~\cite{Knight2018, Knight2016}
with a large number of Prisoner Dilemma strategies available to give an
extensive numerical example of the ideas presented.  The approach is presented
in Section~\ref{sec:delta-zd-strategies}.  All of the code and data discussed
in Section~\ref{sec:numerical-experiments} is open sourced, archived and
written according to best scientific principles~\cite{Wilson2014}. The data
archive can be found at~\cite{vincent_knight_2018_1297075}.

\section{Recognising Extortion}\label{sec:delta-zd-strategies}

In~\cite{Press2012}, given a match between 2 memory-one strategies, the concept
of Zero Determinant (ZD) strategies is introduced. The main result of that paper
shows that given two memory one players \(p, q\in\mathbb{R}^4\) a linear
relationship between the players' scores could be forced by one of the players.

Using the notation of~\cite{Press2012}, assuming the utilities for player \(p\)
are given by \(S_x=(R, S, T, P)\) and for player \(q\) by \(S_y=(R, T, S, P)\)
and that the stationary scores of each player is given by \(S_X\) and \(S_Y\)
respectively. The main result of~\cite{Press2012} is that if

\begin{equation}\label{eqn:linear_relationship_for_p}
    \tilde p=\alpha S_x + \beta S_y + \gamma
\end{equation}

or

\begin{equation}\label{eqn:linear_relationship_for_q}
    \tilde q=\alpha S_x + \beta S_y + \gamma
\end{equation}

where \(\tilde p = (1 - p_1, 1 - p_2, p_3, p_4)\) and
\(\tilde q = (1 - q_1, 1 - q_2, q_3, q_4)\) then:

\begin{equation}
    \alpha S_X + \beta S_Y + \gamma = 0
\end{equation}

In~\cite{Press2012} a particular type of ZD strategy is defined: extortionate
strategies. If:

\begin{equation}\label{eqn:constraint_for_extortion}
    \gamma = - P(\alpha + \beta)
\end{equation}

then the player can ensure they get a score \(\chi\) times
larger than the opponent. This extortion coefficient is given by:

\begin{equation}\label{eqn:definition_of_chi}
    \chi=\frac{-\beta}{\alpha}
\end{equation}

Thus, if (\ref{eqn:constraint_for_extortion}) holds and \(\chi >1\) a player is
said to extort their opponent.
Here, the reverse problem is considered: given a
\(p\in\mathbb{R}^4\) how does one identify \(\alpha, \beta\) if they
exist and is the strategy in fact acting in an extortionate way?

These conditions correspond to:

\begin{align}
    \tilde p_1 & = \alpha R + \beta R - P (\alpha + \beta)
            \label{eqn:condition_for_tilde_p1}\\
    \tilde p_2 & = \alpha S + \beta T - P (\alpha + \beta)
            \label{eqn:condition_for_tilde_p2}\\
    \tilde p_3 & = \alpha T + \beta S - P (\alpha + \beta)
            \label{eqn:condition_for_tilde_p3}\\
    \tilde p_4 & = \alpha P + \beta P - P (\alpha + \beta)
            \label{eqn:condition_for_tilde_p4}
\end{align}

Equation (\ref{eqn:condition_for_tilde_p4}) ensures that \(p_4=\tilde p_4=0\).
Equations (\ref{eqn:condition_for_tilde_p1}-\ref{eqn:condition_for_tilde_p3})
can be used to eliminate \(\alpha, \beta\), giving:

\begin{equation}\label{eqn:planar_definition_of_extortion}
    \tilde p_1 = \frac{(R - P)(\tilde p_2 + \tilde p_3)}{S + T - 2P}
\end{equation}

with:

\begin{equation}\label{eqn:definition_of_chi}
    \chi = \frac{\tilde p_2 (P - T) + \tilde p_3 (S - P)}
                {\tilde p_2 (P - S) + \tilde p_3 (T - P)}
\end{equation}

Given a strategy \(p\in\mathbb{R}^{4\times 1}\) equations
(\ref{eqn:condition_for_tilde_p4}), (\ref{eqn:planar_definition_of_extortion}-\ref{eqn:definition_of_chi}) can be used to check if
a strategy is extortionate. The conditions correspond to:

\begin{align}
    p_1 & = \frac{(R-P)(p_2 + p_3) - R + T + S - P}{S + T - 2P}
     \label{eqn:condition_for_p1}\\
    p_4 & = 0 \label{eqn:condition_for_p4}\\
    1 & > p_2 + p_3\label{eqn:condition_for_chi}
\end{align}

The algebraic steps necessary to prove these results are available in the
supporting materials.

All extortionate strategies reside on a triangular (\ref{eqn:condition_for_chi})
plane (\ref{eqn:condition_for_p1}) in 3 dimensions (\ref{eqn:condition_for_p4}).
Using this formulation it can be seen that a necessary (but not sufficient)
condition for an extortionate strategy is that it cooperates on average less
than 50\% of the time when in a state of disagreement with the opponent.

As an example, consider the known extortionate strategy \(p=(8 / 9, 1 / 2, 1 /
3, 0)\) from~\cite{Stewart2012} which is referred to as \texttt{Extort-2}. In
this case, for the standard values of \((R, T, S, P)\) constraint
(\ref{eqn:condition_for_p1}) corresponds to:

\begin{equation}
    p_1 = \frac{2(p_2 + p_3) + 1}{3}
\end{equation}

It is clear that in this case all constraints hold.

This approach could in fact be used to confirm that a given strategy is acting
in an extortionate manner even if it is not a memory one strategy. However, in
practice, if a closed form for \(p\) is not known, then due to measurement
and/or numerical error this would not work.

This problem can be written in the following linear algebraic form where
\(x=(\alpha, \beta)\)
and \(p^*=(\tilde p_1 - 1, tilde_2 - 1, p_3)\):

\begin{equation}\label{eqn:linear_algebraic_equation_for_p}
    Cx= p^*
\end{equation}

\(C\) corresponds to equations
(\ref{eqn:condition_for_tilde_p1}-\ref{eqn:condition_for_tilde_p3}) and is
given by:

\begin{equation}\label{eqn:definition_of_C}
    C =
    \begin{bmatrix}
        R - P & R- P \\
        S - P & T- P \\
        T - P & S- P \\
    \end{bmatrix}
\end{equation}

Note that in general, equation (\ref{eqn:linear_algebraic_equation_for_p}) will
not necessarily have a solution. From the Rouch\'{e}-Capelli theorem if there is
a solution it is unique as \(\text{rank}(C)=2\) which is the dimension of the
variable \(x\). The best fitting \(x\) is found by minimizing:

\begin{equation}\label{eqn:r_squared}
    \text{SSError} = \|C x- p^*\|_2^2 = \sum_{i=1}^{3}\left((C\bar x)_i-p_i^*\right)^2
\end{equation}

Note that \(\text{SSError}\), which is the square of the Frobenius
norm~\cite{Golub2013}, becomes a measure of how close a strategy is to being an
extortionate strategy. Suspicion
of extortion then corresponds to a threshold on \(\text{SSError}\).

By observing interactions (human or otherwise), their memory one representation
can be inferred and this approach can be used to recognise extortionate
behaviour. The notion of comparing theoretic and actual plays of the IPD is not
novel, see for example~\cite{Rand2013}. Immediately it is noted that if the
environment is noisy~\cite{Wu1995} then no strategy can be considered to be
extortionate as \(p_4>0\).

In the next section, this idea will be illustrated by observing the interactions
that take place in a computer based tournament of the IPD\@.

\section{Numerical experiments}\label{sec:numerical-experiments}

In~\cite{Stewart2012} results from a tournament with
\input{./assets/tex/number_of_stewart_plotkin_strategies/main.tex} strategies,
was presented with specific consideration given to ZD strategies. This
tournament is reproduced here using the Axelrod-Python
project~\cite{Knight2016}. To obtain a good measure of the corresponding
transition rates for each strategy all matches have been run for
\input{assets/tex/number_of_turns/main.tex} turns and every match has been
repeated \input{assets/tex/number_of_repetitions/main.tex} times. All of this
interaction data is available at~\cite{vincent_knight_2018_1297075}. A good
match between the inferred Markov chain and the state distribution of the actual
interactions has been verified. Data for this is presented in the supplementary
materials.

Figure~\ref{fig:SSError_overall_in_stewart_plotkin} shows the \(\text{SSError}\)
values for all the strategies in the tournament, as reported
in~\cite{Stewart2012} the extortionate strategy (which has an expected
\(\text{SSError}\) approximately 0) gains a large number of wins.

\begin{figure}[!htbp]
    \centering
    \includegraphics[width=.8\textwidth]{./assets/img/SSError_overall_in_stewart_plotkin/main.pdf}
    \caption{\(\text{SSError}\) and state probabilities for the strategies
        of~\cite{Stewart2012}, ordered both by number of wins and overall score.
        Note that \(P(DC)\) is not shown as it corresponds to the transpose of
        \(P(CD)\). Cooperator and Defector are omitted as they do not visit all
        the states.}
    \label{fig:SSError_overall_in_stewart_plotkin}
\end{figure}

Here, the work of~\cite{Stewart2012} is extended by investigating a tournament
with \input{assets/tex/number_of_full_strategies/main.tex}
strategies.

The results of this analysis are shown in
Figure~\ref{fig:SSError_and_probabilities_in_full}. The top ranking strategies
by number of wins seem to be extortionate (but not against all strategies) and
it can be seen that a small sub group of strategies achieve mutual defection.
All the top ranking strategies according to score achieve mutual cooperation and
do not extort each other, however they
\textbf{do} exhibit extortionate behaviour towards a number of the lower ranking
strategies.

\begin{figure}[!htbp]
    \centering
    \includegraphics[width=.8\textwidth]{./assets/img/SSError_and_probabilities_in_full/main.pdf}
    \caption{\(\text{SSError}\) for the strategies for the full tournament. Only
    strategy interactions for which \(p_4=0\) and \(\chi>1\) are displayed.}
    \label{fig:SSError_and_probabilities_in_full}
\end{figure}

\section{Conclusion}\label{sec:conclusion}

This work defines an approach to measure whether or not a player is playing a
strategy that corresponds to an extortionate strategy as defined
in~\cite{Press2012}: a mathematical model for suspicion. Indeed, all
extortionate strategies have been
 classified as lying on a triangular plane.
This rigorous classification fails to be robust to small measurement error, thus
a statistical approach is proposed.
This is done through a linear algebraic approach for approximating the solution
of a linear system. Using this, a large number of pairwise interactions is
simulated and in fact very few strategies are found to act extortionately.

The work of~\cite{Press2012}, whilst showing that a clever approach to taking
advantage of another memory one strategy exists: this is incomplete. Whilst the
elegance of this result is very attractive, just as the simplicity of the
victory of Tit For Tat in Axelrod's original tournaments was, it is incomplete.
Extortionate strategies achieve a high number of wins but they do not
achieve a high score which corresponds to the fitness landscape in an
evolutionary sense. From the large number of interactions a payoff matrix \(S\)
can be measured where \(S_{ij}\) denotes the score (using standard values of
\((R, S, T, P) = (3, 0, 5, 1)\)) of the \(i\)th strategy
against the \(j\)th strategy. Using this, the replicator equation
describes the evolution of the system based on a population density fitness
function:

\begin{equation}\label{eqn:replicator_dynamics}
    \frac{dx}{dt} = x(S-x^TS x)
\end{equation}

Equation (\ref{eqn:replicator_dynamics}) is solved numerically through an
integration technique described in~\cite{Petzold1983} and
Figure~\ref{fig:replicator_dynamics} shows the evolution of the distribution of
the system: the various strategies are ranked by scores. It is clear to see that
only the high ranking strategies survive the evolutionary process (in fact,
only \input{./assets/img/replicator_dynamics/main.tex}
have a final distribution greater than \(10 ^ {-2}\)). This confirms the
findings of~\cite{Moran1707} in which sophisticated strategies resist
evolutionary invasion of shorter memory strategies. Recalling
Figure~\ref{fig:SSError_and_probabilities_in_full} this demonstrates that:

\begin{itemize}
    \item Cooperation emerges through the evolutionary process: the high scoring
        strategies do not exhibit extortionate behaviour towards each other.
    \item Extortionate strategies do not survive the evolutionary process.
\end{itemize}

\begin{figure}[!htbp]
    \centering
    \includegraphics[width=.8\textwidth]{./assets/img/replicator_dynamics/main.pdf}
    \caption{Numerical simulation of the replicator equation
    (\ref{eqn:replicator_dynamics}): strategies are ordered by score, only the strategies with a high score survive the evolutionary process.}
    \label{fig:replicator_dynamics}
\end{figure}

This work can be used to classify plays of the IPD\@: data can be collected from
actual interactions (in lab or in the field). Furthermore, this allows for a
classification method similar to the notion of fingerprinting presented
in~\cite{Ashlock2008}. Trained strategies can potentially be classified as
extortionate or not or it could be possible to even constrain the reinforcement
learning approaches that are becoming prevalent in the literature.
Alternatively, this mathematical approach for recognising extortion could be
used in sophisticated strategies to defend against invasion. Arguably, some of
the strategies considered here exhibit this behaviour, indeed as described
in~\cite{Harper2017}, the top ranking strategies in the full tournament are
obtained using evolutionary reinforcement learning techniques, thus, suspicion
of extortionate behaviour could in fact be an evolutionary trait.

\section*{Acknowledgements}

The following open source software libraries were used in this research:

\begin{itemize}
    \item The Axelrod ~\cite{Knight2016, Knight2018} library (IPD strategies and
        tournaments).
    \item The sympy library~\cite{Meurer2017} (verification of all symbolic
        calculations).
    \item The matplotlib~\cite{Droettboom2018} library (visualisation).
    \item The pandas~\cite{Structures2010}, dask~\cite{Dask2016} and
        NumPy~\cite{Oliphant2015} libraries (data manipulation).
    \item The SciPy~\cite{Jones2001} library (numerical integration of the
        replicator equation).
\end{itemize}

This work was performed using the computational facilities of the Advanced
Research Computing @ Cardiff (ARCCA) Division, Cardiff University.

\printbibliography

\newpage
\section*{Supplementary materials}

\includepdf{assets/pdf/proof_of_form_of_extortionate_strategies/main.pdf}

\newpage

Using the pair wise interactions the transition rates \(p,
q\) can be measured and the steady state probabilities inferred and compared to
the actual probabilities of each state.
This is done numerically by computing the singular eigenvector of the
matrix \(A\) \cite{Stewart2009}:

\[
    A =
    \begin{bmatrix}
        p_1 q_1 & p_1 (1 - q_1) & (1 - p_1) q_1 & (1 -p_1) (1 - q_1) \\
        p_2 q_2 & p_2 (1 - q_2) & (1 - p_2) q_2 & (1 -p_2) (1 - q_2) \\
        p_3 q_3 & p_3 (1 - q_3) & (1 - p_3) q_3 & (1 -p_3) (1 - q_3) \\
        p_4 q_4 & p_4 (1 - q_4) & (1 - p_4) q_4 & (1 -p_4) (1 - q_4) \\
    \end{bmatrix}
\]

Figure~\ref{fig:computed_probabilities_vs_theoretic_probabilities} shows a
regression line fitted to every pairwise interaction with a reported
\(\text{SSError}\) value (pairwise interactions with missing states were
omitted). This serves to validate the approach: a part from some edge cases the
relationship is consistent.

\begin{figure}[!htbp]
    \centering
    \includegraphics[width=.8\textwidth]{./assets/img/computed_probabilities_vs_theoretic_probabilities/main.pdf}
    \caption{The
        relationship between the steady state probabilities inferred from the
        measured transitions and the actual steady state probabilities. A linear
        regression line is included validating the approach.}
    \label{fig:computed_probabilities_vs_theoretic_probabilities}
\end{figure}


\end{document}
 turns and every match has been
repeated \documentclass[a4paper]{article}

\usepackage{amsmath}
\usepackage{amssymb}
\usepackage[margin=1.5cm,
            includefoot,
            footskip=30pt]{geometry}
\usepackage{layout}
\usepackage{graphicx}
\usepackage{subcaption}

\usepackage{biblatex}
\usepackage{pdfpages}

\bibliography{main.bib}

\title{Suspicion: Recognising and evaluating the effectiveness
       of extortion in the Iterated Prisoner's Dilemma}
\author{Vincent A. Knight \and Nikoleta E. Glynatsi}
\date{\today}



\begin{document}

\maketitle

\begin{abstract}
    The Iterated Prisoner's Dilemma is a model for rational and evolutionary
    interactive behaviour. It has applications both in the study of human social
    behaviour as well as in biology.
    It is used to understand when and how a rational individual might
    accept an immediate cost to their own utility for the direct benefit of
    another.

    Much attention has been given to a class of strategies called
    Zero Determinant strategies. It has been theoretically shown that these
    strategies can ``extort'' any player.

    In this work, an approach to identify if observed strategies are playing in
    an extortionate way is described. Furthermore, experimental analysis of
    a large tournament with \input{assets/tex/number_of_full_strategies/main.tex}
    strategies is considered. In this setting
    the most highly performing strategies do not play in an extortionate way
    against each other but do against lower performing strategies.
    This suggests that whilst the theory of Zero Determinant strategies
    indicates that memory is not of fundamental importance to the evolution of
    cooperative behaviour, this is incomplete.
\end{abstract}

\section{Introduction}\label{sec:introduction}

Agent based game theoretic models have become a stalwart of the underpinning
mathematics of interactive behaviours. One of the major pieces of work
in this area is the pair of original computer tournaments run by Robert
Axelrod~\cite{Axelrod1980, Axelrod1980a}. These tournaments pitted submitted
computer strategies against each other in plays of the Iterated Prisoner's
Dilemma. A common game where agents can choose to pay a slight cost to their
immediate utility in the hope of building a reputation. This has been used in
economic and evolutionary game theory to understand the evolution of cooperative
behaviour.

Recently, a class of strategies was described in~\cite{Press2012} that can
provably extort any given opponent. In~\cite{Hilbe2013, Moran1707} some
questions have already been asked about the true effectiveness of these
strategies in an evolutionary setting. Here another question is asked: is it
possible to recognise this extortionate behaviour? A mathematical procedure for
suspicion is presented: in the same way that the continued actions of an
extortionate individual might raise suspicion.

This work makes use of the Axelrod Python library~\cite{Knight2018, Knight2016}
with a large number of Prisoner Dilemma strategies available to give an
extensive numerical example of the ideas presented.  The approach is presented
in Section~\ref{sec:delta-zd-strategies}.  All of the code and data discussed
in Section~\ref{sec:numerical-experiments} is open sourced, archived and
written according to best scientific principles~\cite{Wilson2014}. The data
archive can be found at~\cite{vincent_knight_2018_1297075}.

\section{Recognising Extortion}\label{sec:delta-zd-strategies}

In~\cite{Press2012}, given a match between 2 memory-one strategies, the concept
of Zero Determinant (ZD) strategies is introduced. The main result of that paper
shows that given two memory one players \(p, q\in\mathbb{R}^4\) a linear
relationship between the players' scores could be forced by one of the players.

Using the notation of~\cite{Press2012}, assuming the utilities for player \(p\)
are given by \(S_x=(R, S, T, P)\) and for player \(q\) by \(S_y=(R, T, S, P)\)
and that the stationary scores of each player is given by \(S_X\) and \(S_Y\)
respectively. The main result of~\cite{Press2012} is that if

\begin{equation}\label{eqn:linear_relationship_for_p}
    \tilde p=\alpha S_x + \beta S_y + \gamma
\end{equation}

or

\begin{equation}\label{eqn:linear_relationship_for_q}
    \tilde q=\alpha S_x + \beta S_y + \gamma
\end{equation}

where \(\tilde p = (1 - p_1, 1 - p_2, p_3, p_4)\) and
\(\tilde q = (1 - q_1, 1 - q_2, q_3, q_4)\) then:

\begin{equation}
    \alpha S_X + \beta S_Y + \gamma = 0
\end{equation}

In~\cite{Press2012} a particular type of ZD strategy is defined: extortionate
strategies. If:

\begin{equation}\label{eqn:constraint_for_extortion}
    \gamma = - P(\alpha + \beta)
\end{equation}

then the player can ensure they get a score \(\chi\) times
larger than the opponent. This extortion coefficient is given by:

\begin{equation}\label{eqn:definition_of_chi}
    \chi=\frac{-\beta}{\alpha}
\end{equation}

Thus, if (\ref{eqn:constraint_for_extortion}) holds and \(\chi >1\) a player is
said to extort their opponent.
Here, the reverse problem is considered: given a
\(p\in\mathbb{R}^4\) how does one identify \(\alpha, \beta\) if they
exist and is the strategy in fact acting in an extortionate way?

These conditions correspond to:

\begin{align}
    \tilde p_1 & = \alpha R + \beta R - P (\alpha + \beta)
            \label{eqn:condition_for_tilde_p1}\\
    \tilde p_2 & = \alpha S + \beta T - P (\alpha + \beta)
            \label{eqn:condition_for_tilde_p2}\\
    \tilde p_3 & = \alpha T + \beta S - P (\alpha + \beta)
            \label{eqn:condition_for_tilde_p3}\\
    \tilde p_4 & = \alpha P + \beta P - P (\alpha + \beta)
            \label{eqn:condition_for_tilde_p4}
\end{align}

Equation (\ref{eqn:condition_for_tilde_p4}) ensures that \(p_4=\tilde p_4=0\).
Equations (\ref{eqn:condition_for_tilde_p1}-\ref{eqn:condition_for_tilde_p3})
can be used to eliminate \(\alpha, \beta\), giving:

\begin{equation}\label{eqn:planar_definition_of_extortion}
    \tilde p_1 = \frac{(R - P)(\tilde p_2 + \tilde p_3)}{S + T - 2P}
\end{equation}

with:

\begin{equation}\label{eqn:definition_of_chi}
    \chi = \frac{\tilde p_2 (P - T) + \tilde p_3 (S - P)}
                {\tilde p_2 (P - S) + \tilde p_3 (T - P)}
\end{equation}

Given a strategy \(p\in\mathbb{R}^{4\times 1}\) equations
(\ref{eqn:condition_for_tilde_p4}), (\ref{eqn:planar_definition_of_extortion}-\ref{eqn:definition_of_chi}) can be used to check if
a strategy is extortionate. The conditions correspond to:

\begin{align}
    p_1 & = \frac{(R-P)(p_2 + p_3) - R + T + S - P}{S + T - 2P}
     \label{eqn:condition_for_p1}\\
    p_4 & = 0 \label{eqn:condition_for_p4}\\
    1 & > p_2 + p_3\label{eqn:condition_for_chi}
\end{align}

The algebraic steps necessary to prove these results are available in the
supporting materials.

All extortionate strategies reside on a triangular (\ref{eqn:condition_for_chi})
plane (\ref{eqn:condition_for_p1}) in 3 dimensions (\ref{eqn:condition_for_p4}).
Using this formulation it can be seen that a necessary (but not sufficient)
condition for an extortionate strategy is that it cooperates on average less
than 50\% of the time when in a state of disagreement with the opponent.

As an example, consider the known extortionate strategy \(p=(8 / 9, 1 / 2, 1 /
3, 0)\) from~\cite{Stewart2012} which is referred to as \texttt{Extort-2}. In
this case, for the standard values of \((R, T, S, P)\) constraint
(\ref{eqn:condition_for_p1}) corresponds to:

\begin{equation}
    p_1 = \frac{2(p_2 + p_3) + 1}{3}
\end{equation}

It is clear that in this case all constraints hold.

This approach could in fact be used to confirm that a given strategy is acting
in an extortionate manner even if it is not a memory one strategy. However, in
practice, if a closed form for \(p\) is not known, then due to measurement
and/or numerical error this would not work.

This problem can be written in the following linear algebraic form where
\(x=(\alpha, \beta)\)
and \(p^*=(\tilde p_1 - 1, tilde_2 - 1, p_3)\):

\begin{equation}\label{eqn:linear_algebraic_equation_for_p}
    Cx= p^*
\end{equation}

\(C\) corresponds to equations
(\ref{eqn:condition_for_tilde_p1}-\ref{eqn:condition_for_tilde_p3}) and is
given by:

\begin{equation}\label{eqn:definition_of_C}
    C =
    \begin{bmatrix}
        R - P & R- P \\
        S - P & T- P \\
        T - P & S- P \\
    \end{bmatrix}
\end{equation}

Note that in general, equation (\ref{eqn:linear_algebraic_equation_for_p}) will
not necessarily have a solution. From the Rouch\'{e}-Capelli theorem if there is
a solution it is unique as \(\text{rank}(C)=2\) which is the dimension of the
variable \(x\). The best fitting \(x\) is found by minimizing:

\begin{equation}\label{eqn:r_squared}
    \text{SSError} = \|C x- p^*\|_2^2 = \sum_{i=1}^{3}\left((C\bar x)_i-p_i^*\right)^2
\end{equation}

Note that \(\text{SSError}\), which is the square of the Frobenius
norm~\cite{Golub2013}, becomes a measure of how close a strategy is to being an
extortionate strategy. Suspicion
of extortion then corresponds to a threshold on \(\text{SSError}\).

By observing interactions (human or otherwise), their memory one representation
can be inferred and this approach can be used to recognise extortionate
behaviour. The notion of comparing theoretic and actual plays of the IPD is not
novel, see for example~\cite{Rand2013}. Immediately it is noted that if the
environment is noisy~\cite{Wu1995} then no strategy can be considered to be
extortionate as \(p_4>0\).

In the next section, this idea will be illustrated by observing the interactions
that take place in a computer based tournament of the IPD\@.

\section{Numerical experiments}\label{sec:numerical-experiments}

In~\cite{Stewart2012} results from a tournament with
\input{./assets/tex/number_of_stewart_plotkin_strategies/main.tex} strategies,
was presented with specific consideration given to ZD strategies. This
tournament is reproduced here using the Axelrod-Python
project~\cite{Knight2016}. To obtain a good measure of the corresponding
transition rates for each strategy all matches have been run for
\input{assets/tex/number_of_turns/main.tex} turns and every match has been
repeated \input{assets/tex/number_of_repetitions/main.tex} times. All of this
interaction data is available at~\cite{vincent_knight_2018_1297075}. A good
match between the inferred Markov chain and the state distribution of the actual
interactions has been verified. Data for this is presented in the supplementary
materials.

Figure~\ref{fig:SSError_overall_in_stewart_plotkin} shows the \(\text{SSError}\)
values for all the strategies in the tournament, as reported
in~\cite{Stewart2012} the extortionate strategy (which has an expected
\(\text{SSError}\) approximately 0) gains a large number of wins.

\begin{figure}[!htbp]
    \centering
    \includegraphics[width=.8\textwidth]{./assets/img/SSError_overall_in_stewart_plotkin/main.pdf}
    \caption{\(\text{SSError}\) and state probabilities for the strategies
        of~\cite{Stewart2012}, ordered both by number of wins and overall score.
        Note that \(P(DC)\) is not shown as it corresponds to the transpose of
        \(P(CD)\). Cooperator and Defector are omitted as they do not visit all
        the states.}
    \label{fig:SSError_overall_in_stewart_plotkin}
\end{figure}

Here, the work of~\cite{Stewart2012} is extended by investigating a tournament
with \input{assets/tex/number_of_full_strategies/main.tex}
strategies.

The results of this analysis are shown in
Figure~\ref{fig:SSError_and_probabilities_in_full}. The top ranking strategies
by number of wins seem to be extortionate (but not against all strategies) and
it can be seen that a small sub group of strategies achieve mutual defection.
All the top ranking strategies according to score achieve mutual cooperation and
do not extort each other, however they
\textbf{do} exhibit extortionate behaviour towards a number of the lower ranking
strategies.

\begin{figure}[!htbp]
    \centering
    \includegraphics[width=.8\textwidth]{./assets/img/SSError_and_probabilities_in_full/main.pdf}
    \caption{\(\text{SSError}\) for the strategies for the full tournament. Only
    strategy interactions for which \(p_4=0\) and \(\chi>1\) are displayed.}
    \label{fig:SSError_and_probabilities_in_full}
\end{figure}

\section{Conclusion}\label{sec:conclusion}

This work defines an approach to measure whether or not a player is playing a
strategy that corresponds to an extortionate strategy as defined
in~\cite{Press2012}: a mathematical model for suspicion. Indeed, all
extortionate strategies have been
 classified as lying on a triangular plane.
This rigorous classification fails to be robust to small measurement error, thus
a statistical approach is proposed.
This is done through a linear algebraic approach for approximating the solution
of a linear system. Using this, a large number of pairwise interactions is
simulated and in fact very few strategies are found to act extortionately.

The work of~\cite{Press2012}, whilst showing that a clever approach to taking
advantage of another memory one strategy exists: this is incomplete. Whilst the
elegance of this result is very attractive, just as the simplicity of the
victory of Tit For Tat in Axelrod's original tournaments was, it is incomplete.
Extortionate strategies achieve a high number of wins but they do not
achieve a high score which corresponds to the fitness landscape in an
evolutionary sense. From the large number of interactions a payoff matrix \(S\)
can be measured where \(S_{ij}\) denotes the score (using standard values of
\((R, S, T, P) = (3, 0, 5, 1)\)) of the \(i\)th strategy
against the \(j\)th strategy. Using this, the replicator equation
describes the evolution of the system based on a population density fitness
function:

\begin{equation}\label{eqn:replicator_dynamics}
    \frac{dx}{dt} = x(S-x^TS x)
\end{equation}

Equation (\ref{eqn:replicator_dynamics}) is solved numerically through an
integration technique described in~\cite{Petzold1983} and
Figure~\ref{fig:replicator_dynamics} shows the evolution of the distribution of
the system: the various strategies are ranked by scores. It is clear to see that
only the high ranking strategies survive the evolutionary process (in fact,
only \input{./assets/img/replicator_dynamics/main.tex}
have a final distribution greater than \(10 ^ {-2}\)). This confirms the
findings of~\cite{Moran1707} in which sophisticated strategies resist
evolutionary invasion of shorter memory strategies. Recalling
Figure~\ref{fig:SSError_and_probabilities_in_full} this demonstrates that:

\begin{itemize}
    \item Cooperation emerges through the evolutionary process: the high scoring
        strategies do not exhibit extortionate behaviour towards each other.
    \item Extortionate strategies do not survive the evolutionary process.
\end{itemize}

\begin{figure}[!htbp]
    \centering
    \includegraphics[width=.8\textwidth]{./assets/img/replicator_dynamics/main.pdf}
    \caption{Numerical simulation of the replicator equation
    (\ref{eqn:replicator_dynamics}): strategies are ordered by score, only the strategies with a high score survive the evolutionary process.}
    \label{fig:replicator_dynamics}
\end{figure}

This work can be used to classify plays of the IPD\@: data can be collected from
actual interactions (in lab or in the field). Furthermore, this allows for a
classification method similar to the notion of fingerprinting presented
in~\cite{Ashlock2008}. Trained strategies can potentially be classified as
extortionate or not or it could be possible to even constrain the reinforcement
learning approaches that are becoming prevalent in the literature.
Alternatively, this mathematical approach for recognising extortion could be
used in sophisticated strategies to defend against invasion. Arguably, some of
the strategies considered here exhibit this behaviour, indeed as described
in~\cite{Harper2017}, the top ranking strategies in the full tournament are
obtained using evolutionary reinforcement learning techniques, thus, suspicion
of extortionate behaviour could in fact be an evolutionary trait.

\section*{Acknowledgements}

The following open source software libraries were used in this research:

\begin{itemize}
    \item The Axelrod ~\cite{Knight2016, Knight2018} library (IPD strategies and
        tournaments).
    \item The sympy library~\cite{Meurer2017} (verification of all symbolic
        calculations).
    \item The matplotlib~\cite{Droettboom2018} library (visualisation).
    \item The pandas~\cite{Structures2010}, dask~\cite{Dask2016} and
        NumPy~\cite{Oliphant2015} libraries (data manipulation).
    \item The SciPy~\cite{Jones2001} library (numerical integration of the
        replicator equation).
\end{itemize}

This work was performed using the computational facilities of the Advanced
Research Computing @ Cardiff (ARCCA) Division, Cardiff University.

\printbibliography

\newpage
\section*{Supplementary materials}

\includepdf{assets/pdf/proof_of_form_of_extortionate_strategies/main.pdf}

\newpage

Using the pair wise interactions the transition rates \(p,
q\) can be measured and the steady state probabilities inferred and compared to
the actual probabilities of each state.
This is done numerically by computing the singular eigenvector of the
matrix \(A\) \cite{Stewart2009}:

\[
    A =
    \begin{bmatrix}
        p_1 q_1 & p_1 (1 - q_1) & (1 - p_1) q_1 & (1 -p_1) (1 - q_1) \\
        p_2 q_2 & p_2 (1 - q_2) & (1 - p_2) q_2 & (1 -p_2) (1 - q_2) \\
        p_3 q_3 & p_3 (1 - q_3) & (1 - p_3) q_3 & (1 -p_3) (1 - q_3) \\
        p_4 q_4 & p_4 (1 - q_4) & (1 - p_4) q_4 & (1 -p_4) (1 - q_4) \\
    \end{bmatrix}
\]

Figure~\ref{fig:computed_probabilities_vs_theoretic_probabilities} shows a
regression line fitted to every pairwise interaction with a reported
\(\text{SSError}\) value (pairwise interactions with missing states were
omitted). This serves to validate the approach: a part from some edge cases the
relationship is consistent.

\begin{figure}[!htbp]
    \centering
    \includegraphics[width=.8\textwidth]{./assets/img/computed_probabilities_vs_theoretic_probabilities/main.pdf}
    \caption{The
        relationship between the steady state probabilities inferred from the
        measured transitions and the actual steady state probabilities. A linear
        regression line is included validating the approach.}
    \label{fig:computed_probabilities_vs_theoretic_probabilities}
\end{figure}


\end{document}
 times. All of this
interaction data is available at~\cite{vincent_knight_2018_1297075}. A good
match between the inferred Markov chain and the state distribution of the actual
interactions has been verified. Data for this is presented in the supplementary
materials.

Figure~\ref{fig:SSError_overall_in_stewart_plotkin} shows the \(\text{SSError}\)
values for all the strategies in the tournament, as reported
in~\cite{Stewart2012} the extortionate strategy (which has an expected
\(\text{SSError}\) approximately 0) gains a large number of wins.

\begin{figure}[!htbp]
    \centering
    \includegraphics[width=.8\textwidth]{./assets/img/SSError_overall_in_stewart_plotkin/main.pdf}
    \caption{\(\text{SSError}\) and state probabilities for the strategies
        of~\cite{Stewart2012}, ordered both by number of wins and overall score.
        Note that \(P(DC)\) is not shown as it corresponds to the transpose of
        \(P(CD)\). Cooperator and Defector are omitted as they do not visit all
        the states.}
    \label{fig:SSError_overall_in_stewart_plotkin}
\end{figure}

Here, the work of~\cite{Stewart2012} is extended by investigating a tournament
with \documentclass[a4paper]{article}

\usepackage{amsmath}
\usepackage{amssymb}
\usepackage[margin=1.5cm,
            includefoot,
            footskip=30pt]{geometry}
\usepackage{layout}
\usepackage{graphicx}
\usepackage{subcaption}

\usepackage{biblatex}
\usepackage{pdfpages}

\bibliography{main.bib}

\title{Suspicion: Recognising and evaluating the effectiveness
       of extortion in the Iterated Prisoner's Dilemma}
\author{Vincent A. Knight \and Nikoleta E. Glynatsi}
\date{\today}



\begin{document}

\maketitle

\begin{abstract}
    The Iterated Prisoner's Dilemma is a model for rational and evolutionary
    interactive behaviour. It has applications both in the study of human social
    behaviour as well as in biology.
    It is used to understand when and how a rational individual might
    accept an immediate cost to their own utility for the direct benefit of
    another.

    Much attention has been given to a class of strategies called
    Zero Determinant strategies. It has been theoretically shown that these
    strategies can ``extort'' any player.

    In this work, an approach to identify if observed strategies are playing in
    an extortionate way is described. Furthermore, experimental analysis of
    a large tournament with \input{assets/tex/number_of_full_strategies/main.tex}
    strategies is considered. In this setting
    the most highly performing strategies do not play in an extortionate way
    against each other but do against lower performing strategies.
    This suggests that whilst the theory of Zero Determinant strategies
    indicates that memory is not of fundamental importance to the evolution of
    cooperative behaviour, this is incomplete.
\end{abstract}

\section{Introduction}\label{sec:introduction}

Agent based game theoretic models have become a stalwart of the underpinning
mathematics of interactive behaviours. One of the major pieces of work
in this area is the pair of original computer tournaments run by Robert
Axelrod~\cite{Axelrod1980, Axelrod1980a}. These tournaments pitted submitted
computer strategies against each other in plays of the Iterated Prisoner's
Dilemma. A common game where agents can choose to pay a slight cost to their
immediate utility in the hope of building a reputation. This has been used in
economic and evolutionary game theory to understand the evolution of cooperative
behaviour.

Recently, a class of strategies was described in~\cite{Press2012} that can
provably extort any given opponent. In~\cite{Hilbe2013, Moran1707} some
questions have already been asked about the true effectiveness of these
strategies in an evolutionary setting. Here another question is asked: is it
possible to recognise this extortionate behaviour? A mathematical procedure for
suspicion is presented: in the same way that the continued actions of an
extortionate individual might raise suspicion.

This work makes use of the Axelrod Python library~\cite{Knight2018, Knight2016}
with a large number of Prisoner Dilemma strategies available to give an
extensive numerical example of the ideas presented.  The approach is presented
in Section~\ref{sec:delta-zd-strategies}.  All of the code and data discussed
in Section~\ref{sec:numerical-experiments} is open sourced, archived and
written according to best scientific principles~\cite{Wilson2014}. The data
archive can be found at~\cite{vincent_knight_2018_1297075}.

\section{Recognising Extortion}\label{sec:delta-zd-strategies}

In~\cite{Press2012}, given a match between 2 memory-one strategies, the concept
of Zero Determinant (ZD) strategies is introduced. The main result of that paper
shows that given two memory one players \(p, q\in\mathbb{R}^4\) a linear
relationship between the players' scores could be forced by one of the players.

Using the notation of~\cite{Press2012}, assuming the utilities for player \(p\)
are given by \(S_x=(R, S, T, P)\) and for player \(q\) by \(S_y=(R, T, S, P)\)
and that the stationary scores of each player is given by \(S_X\) and \(S_Y\)
respectively. The main result of~\cite{Press2012} is that if

\begin{equation}\label{eqn:linear_relationship_for_p}
    \tilde p=\alpha S_x + \beta S_y + \gamma
\end{equation}

or

\begin{equation}\label{eqn:linear_relationship_for_q}
    \tilde q=\alpha S_x + \beta S_y + \gamma
\end{equation}

where \(\tilde p = (1 - p_1, 1 - p_2, p_3, p_4)\) and
\(\tilde q = (1 - q_1, 1 - q_2, q_3, q_4)\) then:

\begin{equation}
    \alpha S_X + \beta S_Y + \gamma = 0
\end{equation}

In~\cite{Press2012} a particular type of ZD strategy is defined: extortionate
strategies. If:

\begin{equation}\label{eqn:constraint_for_extortion}
    \gamma = - P(\alpha + \beta)
\end{equation}

then the player can ensure they get a score \(\chi\) times
larger than the opponent. This extortion coefficient is given by:

\begin{equation}\label{eqn:definition_of_chi}
    \chi=\frac{-\beta}{\alpha}
\end{equation}

Thus, if (\ref{eqn:constraint_for_extortion}) holds and \(\chi >1\) a player is
said to extort their opponent.
Here, the reverse problem is considered: given a
\(p\in\mathbb{R}^4\) how does one identify \(\alpha, \beta\) if they
exist and is the strategy in fact acting in an extortionate way?

These conditions correspond to:

\begin{align}
    \tilde p_1 & = \alpha R + \beta R - P (\alpha + \beta)
            \label{eqn:condition_for_tilde_p1}\\
    \tilde p_2 & = \alpha S + \beta T - P (\alpha + \beta)
            \label{eqn:condition_for_tilde_p2}\\
    \tilde p_3 & = \alpha T + \beta S - P (\alpha + \beta)
            \label{eqn:condition_for_tilde_p3}\\
    \tilde p_4 & = \alpha P + \beta P - P (\alpha + \beta)
            \label{eqn:condition_for_tilde_p4}
\end{align}

Equation (\ref{eqn:condition_for_tilde_p4}) ensures that \(p_4=\tilde p_4=0\).
Equations (\ref{eqn:condition_for_tilde_p1}-\ref{eqn:condition_for_tilde_p3})
can be used to eliminate \(\alpha, \beta\), giving:

\begin{equation}\label{eqn:planar_definition_of_extortion}
    \tilde p_1 = \frac{(R - P)(\tilde p_2 + \tilde p_3)}{S + T - 2P}
\end{equation}

with:

\begin{equation}\label{eqn:definition_of_chi}
    \chi = \frac{\tilde p_2 (P - T) + \tilde p_3 (S - P)}
                {\tilde p_2 (P - S) + \tilde p_3 (T - P)}
\end{equation}

Given a strategy \(p\in\mathbb{R}^{4\times 1}\) equations
(\ref{eqn:condition_for_tilde_p4}), (\ref{eqn:planar_definition_of_extortion}-\ref{eqn:definition_of_chi}) can be used to check if
a strategy is extortionate. The conditions correspond to:

\begin{align}
    p_1 & = \frac{(R-P)(p_2 + p_3) - R + T + S - P}{S + T - 2P}
     \label{eqn:condition_for_p1}\\
    p_4 & = 0 \label{eqn:condition_for_p4}\\
    1 & > p_2 + p_3\label{eqn:condition_for_chi}
\end{align}

The algebraic steps necessary to prove these results are available in the
supporting materials.

All extortionate strategies reside on a triangular (\ref{eqn:condition_for_chi})
plane (\ref{eqn:condition_for_p1}) in 3 dimensions (\ref{eqn:condition_for_p4}).
Using this formulation it can be seen that a necessary (but not sufficient)
condition for an extortionate strategy is that it cooperates on average less
than 50\% of the time when in a state of disagreement with the opponent.

As an example, consider the known extortionate strategy \(p=(8 / 9, 1 / 2, 1 /
3, 0)\) from~\cite{Stewart2012} which is referred to as \texttt{Extort-2}. In
this case, for the standard values of \((R, T, S, P)\) constraint
(\ref{eqn:condition_for_p1}) corresponds to:

\begin{equation}
    p_1 = \frac{2(p_2 + p_3) + 1}{3}
\end{equation}

It is clear that in this case all constraints hold.

This approach could in fact be used to confirm that a given strategy is acting
in an extortionate manner even if it is not a memory one strategy. However, in
practice, if a closed form for \(p\) is not known, then due to measurement
and/or numerical error this would not work.

This problem can be written in the following linear algebraic form where
\(x=(\alpha, \beta)\)
and \(p^*=(\tilde p_1 - 1, tilde_2 - 1, p_3)\):

\begin{equation}\label{eqn:linear_algebraic_equation_for_p}
    Cx= p^*
\end{equation}

\(C\) corresponds to equations
(\ref{eqn:condition_for_tilde_p1}-\ref{eqn:condition_for_tilde_p3}) and is
given by:

\begin{equation}\label{eqn:definition_of_C}
    C =
    \begin{bmatrix}
        R - P & R- P \\
        S - P & T- P \\
        T - P & S- P \\
    \end{bmatrix}
\end{equation}

Note that in general, equation (\ref{eqn:linear_algebraic_equation_for_p}) will
not necessarily have a solution. From the Rouch\'{e}-Capelli theorem if there is
a solution it is unique as \(\text{rank}(C)=2\) which is the dimension of the
variable \(x\). The best fitting \(x\) is found by minimizing:

\begin{equation}\label{eqn:r_squared}
    \text{SSError} = \|C x- p^*\|_2^2 = \sum_{i=1}^{3}\left((C\bar x)_i-p_i^*\right)^2
\end{equation}

Note that \(\text{SSError}\), which is the square of the Frobenius
norm~\cite{Golub2013}, becomes a measure of how close a strategy is to being an
extortionate strategy. Suspicion
of extortion then corresponds to a threshold on \(\text{SSError}\).

By observing interactions (human or otherwise), their memory one representation
can be inferred and this approach can be used to recognise extortionate
behaviour. The notion of comparing theoretic and actual plays of the IPD is not
novel, see for example~\cite{Rand2013}. Immediately it is noted that if the
environment is noisy~\cite{Wu1995} then no strategy can be considered to be
extortionate as \(p_4>0\).

In the next section, this idea will be illustrated by observing the interactions
that take place in a computer based tournament of the IPD\@.

\section{Numerical experiments}\label{sec:numerical-experiments}

In~\cite{Stewart2012} results from a tournament with
\input{./assets/tex/number_of_stewart_plotkin_strategies/main.tex} strategies,
was presented with specific consideration given to ZD strategies. This
tournament is reproduced here using the Axelrod-Python
project~\cite{Knight2016}. To obtain a good measure of the corresponding
transition rates for each strategy all matches have been run for
\input{assets/tex/number_of_turns/main.tex} turns and every match has been
repeated \input{assets/tex/number_of_repetitions/main.tex} times. All of this
interaction data is available at~\cite{vincent_knight_2018_1297075}. A good
match between the inferred Markov chain and the state distribution of the actual
interactions has been verified. Data for this is presented in the supplementary
materials.

Figure~\ref{fig:SSError_overall_in_stewart_plotkin} shows the \(\text{SSError}\)
values for all the strategies in the tournament, as reported
in~\cite{Stewart2012} the extortionate strategy (which has an expected
\(\text{SSError}\) approximately 0) gains a large number of wins.

\begin{figure}[!htbp]
    \centering
    \includegraphics[width=.8\textwidth]{./assets/img/SSError_overall_in_stewart_plotkin/main.pdf}
    \caption{\(\text{SSError}\) and state probabilities for the strategies
        of~\cite{Stewart2012}, ordered both by number of wins and overall score.
        Note that \(P(DC)\) is not shown as it corresponds to the transpose of
        \(P(CD)\). Cooperator and Defector are omitted as they do not visit all
        the states.}
    \label{fig:SSError_overall_in_stewart_plotkin}
\end{figure}

Here, the work of~\cite{Stewart2012} is extended by investigating a tournament
with \input{assets/tex/number_of_full_strategies/main.tex}
strategies.

The results of this analysis are shown in
Figure~\ref{fig:SSError_and_probabilities_in_full}. The top ranking strategies
by number of wins seem to be extortionate (but not against all strategies) and
it can be seen that a small sub group of strategies achieve mutual defection.
All the top ranking strategies according to score achieve mutual cooperation and
do not extort each other, however they
\textbf{do} exhibit extortionate behaviour towards a number of the lower ranking
strategies.

\begin{figure}[!htbp]
    \centering
    \includegraphics[width=.8\textwidth]{./assets/img/SSError_and_probabilities_in_full/main.pdf}
    \caption{\(\text{SSError}\) for the strategies for the full tournament. Only
    strategy interactions for which \(p_4=0\) and \(\chi>1\) are displayed.}
    \label{fig:SSError_and_probabilities_in_full}
\end{figure}

\section{Conclusion}\label{sec:conclusion}

This work defines an approach to measure whether or not a player is playing a
strategy that corresponds to an extortionate strategy as defined
in~\cite{Press2012}: a mathematical model for suspicion. Indeed, all
extortionate strategies have been
 classified as lying on a triangular plane.
This rigorous classification fails to be robust to small measurement error, thus
a statistical approach is proposed.
This is done through a linear algebraic approach for approximating the solution
of a linear system. Using this, a large number of pairwise interactions is
simulated and in fact very few strategies are found to act extortionately.

The work of~\cite{Press2012}, whilst showing that a clever approach to taking
advantage of another memory one strategy exists: this is incomplete. Whilst the
elegance of this result is very attractive, just as the simplicity of the
victory of Tit For Tat in Axelrod's original tournaments was, it is incomplete.
Extortionate strategies achieve a high number of wins but they do not
achieve a high score which corresponds to the fitness landscape in an
evolutionary sense. From the large number of interactions a payoff matrix \(S\)
can be measured where \(S_{ij}\) denotes the score (using standard values of
\((R, S, T, P) = (3, 0, 5, 1)\)) of the \(i\)th strategy
against the \(j\)th strategy. Using this, the replicator equation
describes the evolution of the system based on a population density fitness
function:

\begin{equation}\label{eqn:replicator_dynamics}
    \frac{dx}{dt} = x(S-x^TS x)
\end{equation}

Equation (\ref{eqn:replicator_dynamics}) is solved numerically through an
integration technique described in~\cite{Petzold1983} and
Figure~\ref{fig:replicator_dynamics} shows the evolution of the distribution of
the system: the various strategies are ranked by scores. It is clear to see that
only the high ranking strategies survive the evolutionary process (in fact,
only \input{./assets/img/replicator_dynamics/main.tex}
have a final distribution greater than \(10 ^ {-2}\)). This confirms the
findings of~\cite{Moran1707} in which sophisticated strategies resist
evolutionary invasion of shorter memory strategies. Recalling
Figure~\ref{fig:SSError_and_probabilities_in_full} this demonstrates that:

\begin{itemize}
    \item Cooperation emerges through the evolutionary process: the high scoring
        strategies do not exhibit extortionate behaviour towards each other.
    \item Extortionate strategies do not survive the evolutionary process.
\end{itemize}

\begin{figure}[!htbp]
    \centering
    \includegraphics[width=.8\textwidth]{./assets/img/replicator_dynamics/main.pdf}
    \caption{Numerical simulation of the replicator equation
    (\ref{eqn:replicator_dynamics}): strategies are ordered by score, only the strategies with a high score survive the evolutionary process.}
    \label{fig:replicator_dynamics}
\end{figure}

This work can be used to classify plays of the IPD\@: data can be collected from
actual interactions (in lab or in the field). Furthermore, this allows for a
classification method similar to the notion of fingerprinting presented
in~\cite{Ashlock2008}. Trained strategies can potentially be classified as
extortionate or not or it could be possible to even constrain the reinforcement
learning approaches that are becoming prevalent in the literature.
Alternatively, this mathematical approach for recognising extortion could be
used in sophisticated strategies to defend against invasion. Arguably, some of
the strategies considered here exhibit this behaviour, indeed as described
in~\cite{Harper2017}, the top ranking strategies in the full tournament are
obtained using evolutionary reinforcement learning techniques, thus, suspicion
of extortionate behaviour could in fact be an evolutionary trait.

\section*{Acknowledgements}

The following open source software libraries were used in this research:

\begin{itemize}
    \item The Axelrod ~\cite{Knight2016, Knight2018} library (IPD strategies and
        tournaments).
    \item The sympy library~\cite{Meurer2017} (verification of all symbolic
        calculations).
    \item The matplotlib~\cite{Droettboom2018} library (visualisation).
    \item The pandas~\cite{Structures2010}, dask~\cite{Dask2016} and
        NumPy~\cite{Oliphant2015} libraries (data manipulation).
    \item The SciPy~\cite{Jones2001} library (numerical integration of the
        replicator equation).
\end{itemize}

This work was performed using the computational facilities of the Advanced
Research Computing @ Cardiff (ARCCA) Division, Cardiff University.

\printbibliography

\newpage
\section*{Supplementary materials}

\includepdf{assets/pdf/proof_of_form_of_extortionate_strategies/main.pdf}

\newpage

Using the pair wise interactions the transition rates \(p,
q\) can be measured and the steady state probabilities inferred and compared to
the actual probabilities of each state.
This is done numerically by computing the singular eigenvector of the
matrix \(A\) \cite{Stewart2009}:

\[
    A =
    \begin{bmatrix}
        p_1 q_1 & p_1 (1 - q_1) & (1 - p_1) q_1 & (1 -p_1) (1 - q_1) \\
        p_2 q_2 & p_2 (1 - q_2) & (1 - p_2) q_2 & (1 -p_2) (1 - q_2) \\
        p_3 q_3 & p_3 (1 - q_3) & (1 - p_3) q_3 & (1 -p_3) (1 - q_3) \\
        p_4 q_4 & p_4 (1 - q_4) & (1 - p_4) q_4 & (1 -p_4) (1 - q_4) \\
    \end{bmatrix}
\]

Figure~\ref{fig:computed_probabilities_vs_theoretic_probabilities} shows a
regression line fitted to every pairwise interaction with a reported
\(\text{SSError}\) value (pairwise interactions with missing states were
omitted). This serves to validate the approach: a part from some edge cases the
relationship is consistent.

\begin{figure}[!htbp]
    \centering
    \includegraphics[width=.8\textwidth]{./assets/img/computed_probabilities_vs_theoretic_probabilities/main.pdf}
    \caption{The
        relationship between the steady state probabilities inferred from the
        measured transitions and the actual steady state probabilities. A linear
        regression line is included validating the approach.}
    \label{fig:computed_probabilities_vs_theoretic_probabilities}
\end{figure}


\end{document}

strategies.

The results of this analysis are shown in
Figure~\ref{fig:SSError_and_probabilities_in_full}. The top ranking strategies
by number of wins seem to be extortionate (but not against all strategies) and
it can be seen that a small sub group of strategies achieve mutual defection.
All the top ranking strategies according to score achieve mutual cooperation and
do not extort each other, however they
\textbf{do} exhibit extortionate behaviour towards a number of the lower ranking
strategies.

\begin{figure}[!htbp]
    \centering
    \includegraphics[width=.8\textwidth]{./assets/img/SSError_and_probabilities_in_full/main.pdf}
    \caption{\(\text{SSError}\) for the strategies for the full tournament. Only
    strategy interactions for which \(p_4=0\) and \(\chi>1\) are displayed.}
    \label{fig:SSError_and_probabilities_in_full}
\end{figure}

\section{Conclusion}\label{sec:conclusion}

This work defines an approach to measure whether or not a player is playing a
strategy that corresponds to an extortionate strategy as defined
in~\cite{Press2012}: a mathematical model for suspicion. Indeed, all
extortionate strategies have been
 classified as lying on a triangular plane.
This rigorous classification fails to be robust to small measurement error, thus
a statistical approach is proposed.
This is done through a linear algebraic approach for approximating the solution
of a linear system. Using this, a large number of pairwise interactions is
simulated and in fact very few strategies are found to act extortionately.

The work of~\cite{Press2012}, whilst showing that a clever approach to taking
advantage of another memory one strategy exists: this is incomplete. Whilst the
elegance of this result is very attractive, just as the simplicity of the
victory of Tit For Tat in Axelrod's original tournaments was, it is incomplete.
Extortionate strategies achieve a high number of wins but they do not
achieve a high score which corresponds to the fitness landscape in an
evolutionary sense. From the large number of interactions a payoff matrix \(S\)
can be measured where \(S_{ij}\) denotes the score (using standard values of
\((R, S, T, P) = (3, 0, 5, 1)\)) of the \(i\)th strategy
against the \(j\)th strategy. Using this, the replicator equation
describes the evolution of the system based on a population density fitness
function:

\begin{equation}\label{eqn:replicator_dynamics}
    \frac{dx}{dt} = x(S-x^TS x)
\end{equation}

Equation (\ref{eqn:replicator_dynamics}) is solved numerically through an
integration technique described in~\cite{Petzold1983} and
Figure~\ref{fig:replicator_dynamics} shows the evolution of the distribution of
the system: the various strategies are ranked by scores. It is clear to see that
only the high ranking strategies survive the evolutionary process (in fact,
only \documentclass[a4paper]{article}

\usepackage{amsmath}
\usepackage{amssymb}
\usepackage[margin=1.5cm,
            includefoot,
            footskip=30pt]{geometry}
\usepackage{layout}
\usepackage{graphicx}
\usepackage{subcaption}

\usepackage{biblatex}
\usepackage{pdfpages}

\bibliography{main.bib}

\title{Suspicion: Recognising and evaluating the effectiveness
       of extortion in the Iterated Prisoner's Dilemma}
\author{Vincent A. Knight \and Nikoleta E. Glynatsi}
\date{\today}



\begin{document}

\maketitle

\begin{abstract}
    The Iterated Prisoner's Dilemma is a model for rational and evolutionary
    interactive behaviour. It has applications both in the study of human social
    behaviour as well as in biology.
    It is used to understand when and how a rational individual might
    accept an immediate cost to their own utility for the direct benefit of
    another.

    Much attention has been given to a class of strategies called
    Zero Determinant strategies. It has been theoretically shown that these
    strategies can ``extort'' any player.

    In this work, an approach to identify if observed strategies are playing in
    an extortionate way is described. Furthermore, experimental analysis of
    a large tournament with \input{assets/tex/number_of_full_strategies/main.tex}
    strategies is considered. In this setting
    the most highly performing strategies do not play in an extortionate way
    against each other but do against lower performing strategies.
    This suggests that whilst the theory of Zero Determinant strategies
    indicates that memory is not of fundamental importance to the evolution of
    cooperative behaviour, this is incomplete.
\end{abstract}

\section{Introduction}\label{sec:introduction}

Agent based game theoretic models have become a stalwart of the underpinning
mathematics of interactive behaviours. One of the major pieces of work
in this area is the pair of original computer tournaments run by Robert
Axelrod~\cite{Axelrod1980, Axelrod1980a}. These tournaments pitted submitted
computer strategies against each other in plays of the Iterated Prisoner's
Dilemma. A common game where agents can choose to pay a slight cost to their
immediate utility in the hope of building a reputation. This has been used in
economic and evolutionary game theory to understand the evolution of cooperative
behaviour.

Recently, a class of strategies was described in~\cite{Press2012} that can
provably extort any given opponent. In~\cite{Hilbe2013, Moran1707} some
questions have already been asked about the true effectiveness of these
strategies in an evolutionary setting. Here another question is asked: is it
possible to recognise this extortionate behaviour? A mathematical procedure for
suspicion is presented: in the same way that the continued actions of an
extortionate individual might raise suspicion.

This work makes use of the Axelrod Python library~\cite{Knight2018, Knight2016}
with a large number of Prisoner Dilemma strategies available to give an
extensive numerical example of the ideas presented.  The approach is presented
in Section~\ref{sec:delta-zd-strategies}.  All of the code and data discussed
in Section~\ref{sec:numerical-experiments} is open sourced, archived and
written according to best scientific principles~\cite{Wilson2014}. The data
archive can be found at~\cite{vincent_knight_2018_1297075}.

\section{Recognising Extortion}\label{sec:delta-zd-strategies}

In~\cite{Press2012}, given a match between 2 memory-one strategies, the concept
of Zero Determinant (ZD) strategies is introduced. The main result of that paper
shows that given two memory one players \(p, q\in\mathbb{R}^4\) a linear
relationship between the players' scores could be forced by one of the players.

Using the notation of~\cite{Press2012}, assuming the utilities for player \(p\)
are given by \(S_x=(R, S, T, P)\) and for player \(q\) by \(S_y=(R, T, S, P)\)
and that the stationary scores of each player is given by \(S_X\) and \(S_Y\)
respectively. The main result of~\cite{Press2012} is that if

\begin{equation}\label{eqn:linear_relationship_for_p}
    \tilde p=\alpha S_x + \beta S_y + \gamma
\end{equation}

or

\begin{equation}\label{eqn:linear_relationship_for_q}
    \tilde q=\alpha S_x + \beta S_y + \gamma
\end{equation}

where \(\tilde p = (1 - p_1, 1 - p_2, p_3, p_4)\) and
\(\tilde q = (1 - q_1, 1 - q_2, q_3, q_4)\) then:

\begin{equation}
    \alpha S_X + \beta S_Y + \gamma = 0
\end{equation}

In~\cite{Press2012} a particular type of ZD strategy is defined: extortionate
strategies. If:

\begin{equation}\label{eqn:constraint_for_extortion}
    \gamma = - P(\alpha + \beta)
\end{equation}

then the player can ensure they get a score \(\chi\) times
larger than the opponent. This extortion coefficient is given by:

\begin{equation}\label{eqn:definition_of_chi}
    \chi=\frac{-\beta}{\alpha}
\end{equation}

Thus, if (\ref{eqn:constraint_for_extortion}) holds and \(\chi >1\) a player is
said to extort their opponent.
Here, the reverse problem is considered: given a
\(p\in\mathbb{R}^4\) how does one identify \(\alpha, \beta\) if they
exist and is the strategy in fact acting in an extortionate way?

These conditions correspond to:

\begin{align}
    \tilde p_1 & = \alpha R + \beta R - P (\alpha + \beta)
            \label{eqn:condition_for_tilde_p1}\\
    \tilde p_2 & = \alpha S + \beta T - P (\alpha + \beta)
            \label{eqn:condition_for_tilde_p2}\\
    \tilde p_3 & = \alpha T + \beta S - P (\alpha + \beta)
            \label{eqn:condition_for_tilde_p3}\\
    \tilde p_4 & = \alpha P + \beta P - P (\alpha + \beta)
            \label{eqn:condition_for_tilde_p4}
\end{align}

Equation (\ref{eqn:condition_for_tilde_p4}) ensures that \(p_4=\tilde p_4=0\).
Equations (\ref{eqn:condition_for_tilde_p1}-\ref{eqn:condition_for_tilde_p3})
can be used to eliminate \(\alpha, \beta\), giving:

\begin{equation}\label{eqn:planar_definition_of_extortion}
    \tilde p_1 = \frac{(R - P)(\tilde p_2 + \tilde p_3)}{S + T - 2P}
\end{equation}

with:

\begin{equation}\label{eqn:definition_of_chi}
    \chi = \frac{\tilde p_2 (P - T) + \tilde p_3 (S - P)}
                {\tilde p_2 (P - S) + \tilde p_3 (T - P)}
\end{equation}

Given a strategy \(p\in\mathbb{R}^{4\times 1}\) equations
(\ref{eqn:condition_for_tilde_p4}), (\ref{eqn:planar_definition_of_extortion}-\ref{eqn:definition_of_chi}) can be used to check if
a strategy is extortionate. The conditions correspond to:

\begin{align}
    p_1 & = \frac{(R-P)(p_2 + p_3) - R + T + S - P}{S + T - 2P}
     \label{eqn:condition_for_p1}\\
    p_4 & = 0 \label{eqn:condition_for_p4}\\
    1 & > p_2 + p_3\label{eqn:condition_for_chi}
\end{align}

The algebraic steps necessary to prove these results are available in the
supporting materials.

All extortionate strategies reside on a triangular (\ref{eqn:condition_for_chi})
plane (\ref{eqn:condition_for_p1}) in 3 dimensions (\ref{eqn:condition_for_p4}).
Using this formulation it can be seen that a necessary (but not sufficient)
condition for an extortionate strategy is that it cooperates on average less
than 50\% of the time when in a state of disagreement with the opponent.

As an example, consider the known extortionate strategy \(p=(8 / 9, 1 / 2, 1 /
3, 0)\) from~\cite{Stewart2012} which is referred to as \texttt{Extort-2}. In
this case, for the standard values of \((R, T, S, P)\) constraint
(\ref{eqn:condition_for_p1}) corresponds to:

\begin{equation}
    p_1 = \frac{2(p_2 + p_3) + 1}{3}
\end{equation}

It is clear that in this case all constraints hold.

This approach could in fact be used to confirm that a given strategy is acting
in an extortionate manner even if it is not a memory one strategy. However, in
practice, if a closed form for \(p\) is not known, then due to measurement
and/or numerical error this would not work.

This problem can be written in the following linear algebraic form where
\(x=(\alpha, \beta)\)
and \(p^*=(\tilde p_1 - 1, tilde_2 - 1, p_3)\):

\begin{equation}\label{eqn:linear_algebraic_equation_for_p}
    Cx= p^*
\end{equation}

\(C\) corresponds to equations
(\ref{eqn:condition_for_tilde_p1}-\ref{eqn:condition_for_tilde_p3}) and is
given by:

\begin{equation}\label{eqn:definition_of_C}
    C =
    \begin{bmatrix}
        R - P & R- P \\
        S - P & T- P \\
        T - P & S- P \\
    \end{bmatrix}
\end{equation}

Note that in general, equation (\ref{eqn:linear_algebraic_equation_for_p}) will
not necessarily have a solution. From the Rouch\'{e}-Capelli theorem if there is
a solution it is unique as \(\text{rank}(C)=2\) which is the dimension of the
variable \(x\). The best fitting \(x\) is found by minimizing:

\begin{equation}\label{eqn:r_squared}
    \text{SSError} = \|C x- p^*\|_2^2 = \sum_{i=1}^{3}\left((C\bar x)_i-p_i^*\right)^2
\end{equation}

Note that \(\text{SSError}\), which is the square of the Frobenius
norm~\cite{Golub2013}, becomes a measure of how close a strategy is to being an
extortionate strategy. Suspicion
of extortion then corresponds to a threshold on \(\text{SSError}\).

By observing interactions (human or otherwise), their memory one representation
can be inferred and this approach can be used to recognise extortionate
behaviour. The notion of comparing theoretic and actual plays of the IPD is not
novel, see for example~\cite{Rand2013}. Immediately it is noted that if the
environment is noisy~\cite{Wu1995} then no strategy can be considered to be
extortionate as \(p_4>0\).

In the next section, this idea will be illustrated by observing the interactions
that take place in a computer based tournament of the IPD\@.

\section{Numerical experiments}\label{sec:numerical-experiments}

In~\cite{Stewart2012} results from a tournament with
\input{./assets/tex/number_of_stewart_plotkin_strategies/main.tex} strategies,
was presented with specific consideration given to ZD strategies. This
tournament is reproduced here using the Axelrod-Python
project~\cite{Knight2016}. To obtain a good measure of the corresponding
transition rates for each strategy all matches have been run for
\input{assets/tex/number_of_turns/main.tex} turns and every match has been
repeated \input{assets/tex/number_of_repetitions/main.tex} times. All of this
interaction data is available at~\cite{vincent_knight_2018_1297075}. A good
match between the inferred Markov chain and the state distribution of the actual
interactions has been verified. Data for this is presented in the supplementary
materials.

Figure~\ref{fig:SSError_overall_in_stewart_plotkin} shows the \(\text{SSError}\)
values for all the strategies in the tournament, as reported
in~\cite{Stewart2012} the extortionate strategy (which has an expected
\(\text{SSError}\) approximately 0) gains a large number of wins.

\begin{figure}[!htbp]
    \centering
    \includegraphics[width=.8\textwidth]{./assets/img/SSError_overall_in_stewart_plotkin/main.pdf}
    \caption{\(\text{SSError}\) and state probabilities for the strategies
        of~\cite{Stewart2012}, ordered both by number of wins and overall score.
        Note that \(P(DC)\) is not shown as it corresponds to the transpose of
        \(P(CD)\). Cooperator and Defector are omitted as they do not visit all
        the states.}
    \label{fig:SSError_overall_in_stewart_plotkin}
\end{figure}

Here, the work of~\cite{Stewart2012} is extended by investigating a tournament
with \input{assets/tex/number_of_full_strategies/main.tex}
strategies.

The results of this analysis are shown in
Figure~\ref{fig:SSError_and_probabilities_in_full}. The top ranking strategies
by number of wins seem to be extortionate (but not against all strategies) and
it can be seen that a small sub group of strategies achieve mutual defection.
All the top ranking strategies according to score achieve mutual cooperation and
do not extort each other, however they
\textbf{do} exhibit extortionate behaviour towards a number of the lower ranking
strategies.

\begin{figure}[!htbp]
    \centering
    \includegraphics[width=.8\textwidth]{./assets/img/SSError_and_probabilities_in_full/main.pdf}
    \caption{\(\text{SSError}\) for the strategies for the full tournament. Only
    strategy interactions for which \(p_4=0\) and \(\chi>1\) are displayed.}
    \label{fig:SSError_and_probabilities_in_full}
\end{figure}

\section{Conclusion}\label{sec:conclusion}

This work defines an approach to measure whether or not a player is playing a
strategy that corresponds to an extortionate strategy as defined
in~\cite{Press2012}: a mathematical model for suspicion. Indeed, all
extortionate strategies have been
 classified as lying on a triangular plane.
This rigorous classification fails to be robust to small measurement error, thus
a statistical approach is proposed.
This is done through a linear algebraic approach for approximating the solution
of a linear system. Using this, a large number of pairwise interactions is
simulated and in fact very few strategies are found to act extortionately.

The work of~\cite{Press2012}, whilst showing that a clever approach to taking
advantage of another memory one strategy exists: this is incomplete. Whilst the
elegance of this result is very attractive, just as the simplicity of the
victory of Tit For Tat in Axelrod's original tournaments was, it is incomplete.
Extortionate strategies achieve a high number of wins but they do not
achieve a high score which corresponds to the fitness landscape in an
evolutionary sense. From the large number of interactions a payoff matrix \(S\)
can be measured where \(S_{ij}\) denotes the score (using standard values of
\((R, S, T, P) = (3, 0, 5, 1)\)) of the \(i\)th strategy
against the \(j\)th strategy. Using this, the replicator equation
describes the evolution of the system based on a population density fitness
function:

\begin{equation}\label{eqn:replicator_dynamics}
    \frac{dx}{dt} = x(S-x^TS x)
\end{equation}

Equation (\ref{eqn:replicator_dynamics}) is solved numerically through an
integration technique described in~\cite{Petzold1983} and
Figure~\ref{fig:replicator_dynamics} shows the evolution of the distribution of
the system: the various strategies are ranked by scores. It is clear to see that
only the high ranking strategies survive the evolutionary process (in fact,
only \input{./assets/img/replicator_dynamics/main.tex}
have a final distribution greater than \(10 ^ {-2}\)). This confirms the
findings of~\cite{Moran1707} in which sophisticated strategies resist
evolutionary invasion of shorter memory strategies. Recalling
Figure~\ref{fig:SSError_and_probabilities_in_full} this demonstrates that:

\begin{itemize}
    \item Cooperation emerges through the evolutionary process: the high scoring
        strategies do not exhibit extortionate behaviour towards each other.
    \item Extortionate strategies do not survive the evolutionary process.
\end{itemize}

\begin{figure}[!htbp]
    \centering
    \includegraphics[width=.8\textwidth]{./assets/img/replicator_dynamics/main.pdf}
    \caption{Numerical simulation of the replicator equation
    (\ref{eqn:replicator_dynamics}): strategies are ordered by score, only the strategies with a high score survive the evolutionary process.}
    \label{fig:replicator_dynamics}
\end{figure}

This work can be used to classify plays of the IPD\@: data can be collected from
actual interactions (in lab or in the field). Furthermore, this allows for a
classification method similar to the notion of fingerprinting presented
in~\cite{Ashlock2008}. Trained strategies can potentially be classified as
extortionate or not or it could be possible to even constrain the reinforcement
learning approaches that are becoming prevalent in the literature.
Alternatively, this mathematical approach for recognising extortion could be
used in sophisticated strategies to defend against invasion. Arguably, some of
the strategies considered here exhibit this behaviour, indeed as described
in~\cite{Harper2017}, the top ranking strategies in the full tournament are
obtained using evolutionary reinforcement learning techniques, thus, suspicion
of extortionate behaviour could in fact be an evolutionary trait.

\section*{Acknowledgements}

The following open source software libraries were used in this research:

\begin{itemize}
    \item The Axelrod ~\cite{Knight2016, Knight2018} library (IPD strategies and
        tournaments).
    \item The sympy library~\cite{Meurer2017} (verification of all symbolic
        calculations).
    \item The matplotlib~\cite{Droettboom2018} library (visualisation).
    \item The pandas~\cite{Structures2010}, dask~\cite{Dask2016} and
        NumPy~\cite{Oliphant2015} libraries (data manipulation).
    \item The SciPy~\cite{Jones2001} library (numerical integration of the
        replicator equation).
\end{itemize}

This work was performed using the computational facilities of the Advanced
Research Computing @ Cardiff (ARCCA) Division, Cardiff University.

\printbibliography

\newpage
\section*{Supplementary materials}

\includepdf{assets/pdf/proof_of_form_of_extortionate_strategies/main.pdf}

\newpage

Using the pair wise interactions the transition rates \(p,
q\) can be measured and the steady state probabilities inferred and compared to
the actual probabilities of each state.
This is done numerically by computing the singular eigenvector of the
matrix \(A\) \cite{Stewart2009}:

\[
    A =
    \begin{bmatrix}
        p_1 q_1 & p_1 (1 - q_1) & (1 - p_1) q_1 & (1 -p_1) (1 - q_1) \\
        p_2 q_2 & p_2 (1 - q_2) & (1 - p_2) q_2 & (1 -p_2) (1 - q_2) \\
        p_3 q_3 & p_3 (1 - q_3) & (1 - p_3) q_3 & (1 -p_3) (1 - q_3) \\
        p_4 q_4 & p_4 (1 - q_4) & (1 - p_4) q_4 & (1 -p_4) (1 - q_4) \\
    \end{bmatrix}
\]

Figure~\ref{fig:computed_probabilities_vs_theoretic_probabilities} shows a
regression line fitted to every pairwise interaction with a reported
\(\text{SSError}\) value (pairwise interactions with missing states were
omitted). This serves to validate the approach: a part from some edge cases the
relationship is consistent.

\begin{figure}[!htbp]
    \centering
    \includegraphics[width=.8\textwidth]{./assets/img/computed_probabilities_vs_theoretic_probabilities/main.pdf}
    \caption{The
        relationship between the steady state probabilities inferred from the
        measured transitions and the actual steady state probabilities. A linear
        regression line is included validating the approach.}
    \label{fig:computed_probabilities_vs_theoretic_probabilities}
\end{figure}


\end{document}

have a final distribution greater than \(10 ^ {-2}\)). This confirms the
findings of~\cite{Moran1707} in which sophisticated strategies resist
evolutionary invasion of shorter memory strategies. Recalling
Figure~\ref{fig:SSError_and_probabilities_in_full} this demonstrates that:

\begin{itemize}
    \item Cooperation emerges through the evolutionary process: the high scoring
        strategies do not exhibit extortionate behaviour towards each other.
    \item Extortionate strategies do not survive the evolutionary process.
\end{itemize}

\begin{figure}[!htbp]
    \centering
    \includegraphics[width=.8\textwidth]{./assets/img/replicator_dynamics/main.pdf}
    \caption{Numerical simulation of the replicator equation
    (\ref{eqn:replicator_dynamics}): strategies are ordered by score, only the strategies with a high score survive the evolutionary process.}
    \label{fig:replicator_dynamics}
\end{figure}

This work can be used to classify plays of the IPD\@: data can be collected from
actual interactions (in lab or in the field). Furthermore, this allows for a
classification method similar to the notion of fingerprinting presented
in~\cite{Ashlock2008}. Trained strategies can potentially be classified as
extortionate or not or it could be possible to even constrain the reinforcement
learning approaches that are becoming prevalent in the literature.
Alternatively, this mathematical approach for recognising extortion could be
used in sophisticated strategies to defend against invasion. Arguably, some of
the strategies considered here exhibit this behaviour, indeed as described
in~\cite{Harper2017}, the top ranking strategies in the full tournament are
obtained using evolutionary reinforcement learning techniques, thus, suspicion
of extortionate behaviour could in fact be an evolutionary trait.

\section*{Acknowledgements}

The following open source software libraries were used in this research:

\begin{itemize}
    \item The Axelrod ~\cite{Knight2016, Knight2018} library (IPD strategies and
        tournaments).
    \item The sympy library~\cite{Meurer2017} (verification of all symbolic
        calculations).
    \item The matplotlib~\cite{Droettboom2018} library (visualisation).
    \item The pandas~\cite{Structures2010}, dask~\cite{Dask2016} and
        NumPy~\cite{Oliphant2015} libraries (data manipulation).
    \item The SciPy~\cite{Jones2001} library (numerical integration of the
        replicator equation).
\end{itemize}

This work was performed using the computational facilities of the Advanced
Research Computing @ Cardiff (ARCCA) Division, Cardiff University.

\printbibliography

\newpage
\section*{Supplementary materials}

\includepdf{assets/pdf/proof_of_form_of_extortionate_strategies/main.pdf}

\newpage

Using the pair wise interactions the transition rates \(p,
q\) can be measured and the steady state probabilities inferred and compared to
the actual probabilities of each state.
This is done numerically by computing the singular eigenvector of the
matrix \(A\) \cite{Stewart2009}:

\[
    A =
    \begin{bmatrix}
        p_1 q_1 & p_1 (1 - q_1) & (1 - p_1) q_1 & (1 -p_1) (1 - q_1) \\
        p_2 q_2 & p_2 (1 - q_2) & (1 - p_2) q_2 & (1 -p_2) (1 - q_2) \\
        p_3 q_3 & p_3 (1 - q_3) & (1 - p_3) q_3 & (1 -p_3) (1 - q_3) \\
        p_4 q_4 & p_4 (1 - q_4) & (1 - p_4) q_4 & (1 -p_4) (1 - q_4) \\
    \end{bmatrix}
\]

Figure~\ref{fig:computed_probabilities_vs_theoretic_probabilities} shows a
regression line fitted to every pairwise interaction with a reported
\(\text{SSError}\) value (pairwise interactions with missing states were
omitted). This serves to validate the approach: a part from some edge cases the
relationship is consistent.

\begin{figure}[!htbp]
    \centering
    \includegraphics[width=.8\textwidth]{./assets/img/computed_probabilities_vs_theoretic_probabilities/main.pdf}
    \caption{The
        relationship between the steady state probabilities inferred from the
        measured transitions and the actual steady state probabilities. A linear
        regression line is included validating the approach.}
    \label{fig:computed_probabilities_vs_theoretic_probabilities}
\end{figure}


\end{document}
 turns and every match has been
repeated \documentclass[a4paper]{article}

\usepackage{amsmath}
\usepackage{amssymb}
\usepackage[margin=1.5cm,
            includefoot,
            footskip=30pt]{geometry}
\usepackage{layout}
\usepackage{graphicx}
\usepackage{subcaption}

\usepackage{biblatex}
\usepackage{pdfpages}

\bibliography{main.bib}

\title{Suspicion: Recognising and evaluating the effectiveness
       of extortion in the Iterated Prisoner's Dilemma}
\author{Vincent A. Knight \and Nikoleta E. Glynatsi}
\date{\today}



\begin{document}

\maketitle

\begin{abstract}
    The Iterated Prisoner's Dilemma is a model for rational and evolutionary
    interactive behaviour. It has applications both in the study of human social
    behaviour as well as in biology.
    It is used to understand when and how a rational individual might
    accept an immediate cost to their own utility for the direct benefit of
    another.

    Much attention has been given to a class of strategies called
    Zero Determinant strategies. It has been theoretically shown that these
    strategies can ``extort'' any player.

    In this work, an approach to identify if observed strategies are playing in
    an extortionate way is described. Furthermore, experimental analysis of
    a large tournament with \documentclass[a4paper]{article}

\usepackage{amsmath}
\usepackage{amssymb}
\usepackage[margin=1.5cm,
            includefoot,
            footskip=30pt]{geometry}
\usepackage{layout}
\usepackage{graphicx}
\usepackage{subcaption}

\usepackage{biblatex}
\usepackage{pdfpages}

\bibliography{main.bib}

\title{Suspicion: Recognising and evaluating the effectiveness
       of extortion in the Iterated Prisoner's Dilemma}
\author{Vincent A. Knight \and Nikoleta E. Glynatsi}
\date{\today}



\begin{document}

\maketitle

\begin{abstract}
    The Iterated Prisoner's Dilemma is a model for rational and evolutionary
    interactive behaviour. It has applications both in the study of human social
    behaviour as well as in biology.
    It is used to understand when and how a rational individual might
    accept an immediate cost to their own utility for the direct benefit of
    another.

    Much attention has been given to a class of strategies called
    Zero Determinant strategies. It has been theoretically shown that these
    strategies can ``extort'' any player.

    In this work, an approach to identify if observed strategies are playing in
    an extortionate way is described. Furthermore, experimental analysis of
    a large tournament with \input{assets/tex/number_of_full_strategies/main.tex}
    strategies is considered. In this setting
    the most highly performing strategies do not play in an extortionate way
    against each other but do against lower performing strategies.
    This suggests that whilst the theory of Zero Determinant strategies
    indicates that memory is not of fundamental importance to the evolution of
    cooperative behaviour, this is incomplete.
\end{abstract}

\section{Introduction}\label{sec:introduction}

Agent based game theoretic models have become a stalwart of the underpinning
mathematics of interactive behaviours. One of the major pieces of work
in this area is the pair of original computer tournaments run by Robert
Axelrod~\cite{Axelrod1980, Axelrod1980a}. These tournaments pitted submitted
computer strategies against each other in plays of the Iterated Prisoner's
Dilemma. A common game where agents can choose to pay a slight cost to their
immediate utility in the hope of building a reputation. This has been used in
economic and evolutionary game theory to understand the evolution of cooperative
behaviour.

Recently, a class of strategies was described in~\cite{Press2012} that can
provably extort any given opponent. In~\cite{Hilbe2013, Moran1707} some
questions have already been asked about the true effectiveness of these
strategies in an evolutionary setting. Here another question is asked: is it
possible to recognise this extortionate behaviour? A mathematical procedure for
suspicion is presented: in the same way that the continued actions of an
extortionate individual might raise suspicion.

This work makes use of the Axelrod Python library~\cite{Knight2018, Knight2016}
with a large number of Prisoner Dilemma strategies available to give an
extensive numerical example of the ideas presented.  The approach is presented
in Section~\ref{sec:delta-zd-strategies}.  All of the code and data discussed
in Section~\ref{sec:numerical-experiments} is open sourced, archived and
written according to best scientific principles~\cite{Wilson2014}. The data
archive can be found at~\cite{vincent_knight_2018_1297075}.

\section{Recognising Extortion}\label{sec:delta-zd-strategies}

In~\cite{Press2012}, given a match between 2 memory-one strategies, the concept
of Zero Determinant (ZD) strategies is introduced. The main result of that paper
shows that given two memory one players \(p, q\in\mathbb{R}^4\) a linear
relationship between the players' scores could be forced by one of the players.

Using the notation of~\cite{Press2012}, assuming the utilities for player \(p\)
are given by \(S_x=(R, S, T, P)\) and for player \(q\) by \(S_y=(R, T, S, P)\)
and that the stationary scores of each player is given by \(S_X\) and \(S_Y\)
respectively. The main result of~\cite{Press2012} is that if

\begin{equation}\label{eqn:linear_relationship_for_p}
    \tilde p=\alpha S_x + \beta S_y + \gamma
\end{equation}

or

\begin{equation}\label{eqn:linear_relationship_for_q}
    \tilde q=\alpha S_x + \beta S_y + \gamma
\end{equation}

where \(\tilde p = (1 - p_1, 1 - p_2, p_3, p_4)\) and
\(\tilde q = (1 - q_1, 1 - q_2, q_3, q_4)\) then:

\begin{equation}
    \alpha S_X + \beta S_Y + \gamma = 0
\end{equation}

In~\cite{Press2012} a particular type of ZD strategy is defined: extortionate
strategies. If:

\begin{equation}\label{eqn:constraint_for_extortion}
    \gamma = - P(\alpha + \beta)
\end{equation}

then the player can ensure they get a score \(\chi\) times
larger than the opponent. This extortion coefficient is given by:

\begin{equation}\label{eqn:definition_of_chi}
    \chi=\frac{-\beta}{\alpha}
\end{equation}

Thus, if (\ref{eqn:constraint_for_extortion}) holds and \(\chi >1\) a player is
said to extort their opponent.
Here, the reverse problem is considered: given a
\(p\in\mathbb{R}^4\) how does one identify \(\alpha, \beta\) if they
exist and is the strategy in fact acting in an extortionate way?

These conditions correspond to:

\begin{align}
    \tilde p_1 & = \alpha R + \beta R - P (\alpha + \beta)
            \label{eqn:condition_for_tilde_p1}\\
    \tilde p_2 & = \alpha S + \beta T - P (\alpha + \beta)
            \label{eqn:condition_for_tilde_p2}\\
    \tilde p_3 & = \alpha T + \beta S - P (\alpha + \beta)
            \label{eqn:condition_for_tilde_p3}\\
    \tilde p_4 & = \alpha P + \beta P - P (\alpha + \beta)
            \label{eqn:condition_for_tilde_p4}
\end{align}

Equation (\ref{eqn:condition_for_tilde_p4}) ensures that \(p_4=\tilde p_4=0\).
Equations (\ref{eqn:condition_for_tilde_p1}-\ref{eqn:condition_for_tilde_p3})
can be used to eliminate \(\alpha, \beta\), giving:

\begin{equation}\label{eqn:planar_definition_of_extortion}
    \tilde p_1 = \frac{(R - P)(\tilde p_2 + \tilde p_3)}{S + T - 2P}
\end{equation}

with:

\begin{equation}\label{eqn:definition_of_chi}
    \chi = \frac{\tilde p_2 (P - T) + \tilde p_3 (S - P)}
                {\tilde p_2 (P - S) + \tilde p_3 (T - P)}
\end{equation}

Given a strategy \(p\in\mathbb{R}^{4\times 1}\) equations
(\ref{eqn:condition_for_tilde_p4}), (\ref{eqn:planar_definition_of_extortion}-\ref{eqn:definition_of_chi}) can be used to check if
a strategy is extortionate. The conditions correspond to:

\begin{align}
    p_1 & = \frac{(R-P)(p_2 + p_3) - R + T + S - P}{S + T - 2P}
     \label{eqn:condition_for_p1}\\
    p_4 & = 0 \label{eqn:condition_for_p4}\\
    1 & > p_2 + p_3\label{eqn:condition_for_chi}
\end{align}

The algebraic steps necessary to prove these results are available in the
supporting materials.

All extortionate strategies reside on a triangular (\ref{eqn:condition_for_chi})
plane (\ref{eqn:condition_for_p1}) in 3 dimensions (\ref{eqn:condition_for_p4}).
Using this formulation it can be seen that a necessary (but not sufficient)
condition for an extortionate strategy is that it cooperates on average less
than 50\% of the time when in a state of disagreement with the opponent.

As an example, consider the known extortionate strategy \(p=(8 / 9, 1 / 2, 1 /
3, 0)\) from~\cite{Stewart2012} which is referred to as \texttt{Extort-2}. In
this case, for the standard values of \((R, T, S, P)\) constraint
(\ref{eqn:condition_for_p1}) corresponds to:

\begin{equation}
    p_1 = \frac{2(p_2 + p_3) + 1}{3}
\end{equation}

It is clear that in this case all constraints hold.

This approach could in fact be used to confirm that a given strategy is acting
in an extortionate manner even if it is not a memory one strategy. However, in
practice, if a closed form for \(p\) is not known, then due to measurement
and/or numerical error this would not work.

This problem can be written in the following linear algebraic form where
\(x=(\alpha, \beta)\)
and \(p^*=(\tilde p_1 - 1, tilde_2 - 1, p_3)\):

\begin{equation}\label{eqn:linear_algebraic_equation_for_p}
    Cx= p^*
\end{equation}

\(C\) corresponds to equations
(\ref{eqn:condition_for_tilde_p1}-\ref{eqn:condition_for_tilde_p3}) and is
given by:

\begin{equation}\label{eqn:definition_of_C}
    C =
    \begin{bmatrix}
        R - P & R- P \\
        S - P & T- P \\
        T - P & S- P \\
    \end{bmatrix}
\end{equation}

Note that in general, equation (\ref{eqn:linear_algebraic_equation_for_p}) will
not necessarily have a solution. From the Rouch\'{e}-Capelli theorem if there is
a solution it is unique as \(\text{rank}(C)=2\) which is the dimension of the
variable \(x\). The best fitting \(x\) is found by minimizing:

\begin{equation}\label{eqn:r_squared}
    \text{SSError} = \|C x- p^*\|_2^2 = \sum_{i=1}^{3}\left((C\bar x)_i-p_i^*\right)^2
\end{equation}

Note that \(\text{SSError}\), which is the square of the Frobenius
norm~\cite{Golub2013}, becomes a measure of how close a strategy is to being an
extortionate strategy. Suspicion
of extortion then corresponds to a threshold on \(\text{SSError}\).

By observing interactions (human or otherwise), their memory one representation
can be inferred and this approach can be used to recognise extortionate
behaviour. The notion of comparing theoretic and actual plays of the IPD is not
novel, see for example~\cite{Rand2013}. Immediately it is noted that if the
environment is noisy~\cite{Wu1995} then no strategy can be considered to be
extortionate as \(p_4>0\).

In the next section, this idea will be illustrated by observing the interactions
that take place in a computer based tournament of the IPD\@.

\section{Numerical experiments}\label{sec:numerical-experiments}

In~\cite{Stewart2012} results from a tournament with
\input{./assets/tex/number_of_stewart_plotkin_strategies/main.tex} strategies,
was presented with specific consideration given to ZD strategies. This
tournament is reproduced here using the Axelrod-Python
project~\cite{Knight2016}. To obtain a good measure of the corresponding
transition rates for each strategy all matches have been run for
\input{assets/tex/number_of_turns/main.tex} turns and every match has been
repeated \input{assets/tex/number_of_repetitions/main.tex} times. All of this
interaction data is available at~\cite{vincent_knight_2018_1297075}. A good
match between the inferred Markov chain and the state distribution of the actual
interactions has been verified. Data for this is presented in the supplementary
materials.

Figure~\ref{fig:SSError_overall_in_stewart_plotkin} shows the \(\text{SSError}\)
values for all the strategies in the tournament, as reported
in~\cite{Stewart2012} the extortionate strategy (which has an expected
\(\text{SSError}\) approximately 0) gains a large number of wins.

\begin{figure}[!htbp]
    \centering
    \includegraphics[width=.8\textwidth]{./assets/img/SSError_overall_in_stewart_plotkin/main.pdf}
    \caption{\(\text{SSError}\) and state probabilities for the strategies
        of~\cite{Stewart2012}, ordered both by number of wins and overall score.
        Note that \(P(DC)\) is not shown as it corresponds to the transpose of
        \(P(CD)\). Cooperator and Defector are omitted as they do not visit all
        the states.}
    \label{fig:SSError_overall_in_stewart_plotkin}
\end{figure}

Here, the work of~\cite{Stewart2012} is extended by investigating a tournament
with \input{assets/tex/number_of_full_strategies/main.tex}
strategies.

The results of this analysis are shown in
Figure~\ref{fig:SSError_and_probabilities_in_full}. The top ranking strategies
by number of wins seem to be extortionate (but not against all strategies) and
it can be seen that a small sub group of strategies achieve mutual defection.
All the top ranking strategies according to score achieve mutual cooperation and
do not extort each other, however they
\textbf{do} exhibit extortionate behaviour towards a number of the lower ranking
strategies.

\begin{figure}[!htbp]
    \centering
    \includegraphics[width=.8\textwidth]{./assets/img/SSError_and_probabilities_in_full/main.pdf}
    \caption{\(\text{SSError}\) for the strategies for the full tournament. Only
    strategy interactions for which \(p_4=0\) and \(\chi>1\) are displayed.}
    \label{fig:SSError_and_probabilities_in_full}
\end{figure}

\section{Conclusion}\label{sec:conclusion}

This work defines an approach to measure whether or not a player is playing a
strategy that corresponds to an extortionate strategy as defined
in~\cite{Press2012}: a mathematical model for suspicion. Indeed, all
extortionate strategies have been
 classified as lying on a triangular plane.
This rigorous classification fails to be robust to small measurement error, thus
a statistical approach is proposed.
This is done through a linear algebraic approach for approximating the solution
of a linear system. Using this, a large number of pairwise interactions is
simulated and in fact very few strategies are found to act extortionately.

The work of~\cite{Press2012}, whilst showing that a clever approach to taking
advantage of another memory one strategy exists: this is incomplete. Whilst the
elegance of this result is very attractive, just as the simplicity of the
victory of Tit For Tat in Axelrod's original tournaments was, it is incomplete.
Extortionate strategies achieve a high number of wins but they do not
achieve a high score which corresponds to the fitness landscape in an
evolutionary sense. From the large number of interactions a payoff matrix \(S\)
can be measured where \(S_{ij}\) denotes the score (using standard values of
\((R, S, T, P) = (3, 0, 5, 1)\)) of the \(i\)th strategy
against the \(j\)th strategy. Using this, the replicator equation
describes the evolution of the system based on a population density fitness
function:

\begin{equation}\label{eqn:replicator_dynamics}
    \frac{dx}{dt} = x(S-x^TS x)
\end{equation}

Equation (\ref{eqn:replicator_dynamics}) is solved numerically through an
integration technique described in~\cite{Petzold1983} and
Figure~\ref{fig:replicator_dynamics} shows the evolution of the distribution of
the system: the various strategies are ranked by scores. It is clear to see that
only the high ranking strategies survive the evolutionary process (in fact,
only \input{./assets/img/replicator_dynamics/main.tex}
have a final distribution greater than \(10 ^ {-2}\)). This confirms the
findings of~\cite{Moran1707} in which sophisticated strategies resist
evolutionary invasion of shorter memory strategies. Recalling
Figure~\ref{fig:SSError_and_probabilities_in_full} this demonstrates that:

\begin{itemize}
    \item Cooperation emerges through the evolutionary process: the high scoring
        strategies do not exhibit extortionate behaviour towards each other.
    \item Extortionate strategies do not survive the evolutionary process.
\end{itemize}

\begin{figure}[!htbp]
    \centering
    \includegraphics[width=.8\textwidth]{./assets/img/replicator_dynamics/main.pdf}
    \caption{Numerical simulation of the replicator equation
    (\ref{eqn:replicator_dynamics}): strategies are ordered by score, only the strategies with a high score survive the evolutionary process.}
    \label{fig:replicator_dynamics}
\end{figure}

This work can be used to classify plays of the IPD\@: data can be collected from
actual interactions (in lab or in the field). Furthermore, this allows for a
classification method similar to the notion of fingerprinting presented
in~\cite{Ashlock2008}. Trained strategies can potentially be classified as
extortionate or not or it could be possible to even constrain the reinforcement
learning approaches that are becoming prevalent in the literature.
Alternatively, this mathematical approach for recognising extortion could be
used in sophisticated strategies to defend against invasion. Arguably, some of
the strategies considered here exhibit this behaviour, indeed as described
in~\cite{Harper2017}, the top ranking strategies in the full tournament are
obtained using evolutionary reinforcement learning techniques, thus, suspicion
of extortionate behaviour could in fact be an evolutionary trait.

\section*{Acknowledgements}

The following open source software libraries were used in this research:

\begin{itemize}
    \item The Axelrod ~\cite{Knight2016, Knight2018} library (IPD strategies and
        tournaments).
    \item The sympy library~\cite{Meurer2017} (verification of all symbolic
        calculations).
    \item The matplotlib~\cite{Droettboom2018} library (visualisation).
    \item The pandas~\cite{Structures2010}, dask~\cite{Dask2016} and
        NumPy~\cite{Oliphant2015} libraries (data manipulation).
    \item The SciPy~\cite{Jones2001} library (numerical integration of the
        replicator equation).
\end{itemize}

This work was performed using the computational facilities of the Advanced
Research Computing @ Cardiff (ARCCA) Division, Cardiff University.

\printbibliography

\newpage
\section*{Supplementary materials}

\includepdf{assets/pdf/proof_of_form_of_extortionate_strategies/main.pdf}

\newpage

Using the pair wise interactions the transition rates \(p,
q\) can be measured and the steady state probabilities inferred and compared to
the actual probabilities of each state.
This is done numerically by computing the singular eigenvector of the
matrix \(A\) \cite{Stewart2009}:

\[
    A =
    \begin{bmatrix}
        p_1 q_1 & p_1 (1 - q_1) & (1 - p_1) q_1 & (1 -p_1) (1 - q_1) \\
        p_2 q_2 & p_2 (1 - q_2) & (1 - p_2) q_2 & (1 -p_2) (1 - q_2) \\
        p_3 q_3 & p_3 (1 - q_3) & (1 - p_3) q_3 & (1 -p_3) (1 - q_3) \\
        p_4 q_4 & p_4 (1 - q_4) & (1 - p_4) q_4 & (1 -p_4) (1 - q_4) \\
    \end{bmatrix}
\]

Figure~\ref{fig:computed_probabilities_vs_theoretic_probabilities} shows a
regression line fitted to every pairwise interaction with a reported
\(\text{SSError}\) value (pairwise interactions with missing states were
omitted). This serves to validate the approach: a part from some edge cases the
relationship is consistent.

\begin{figure}[!htbp]
    \centering
    \includegraphics[width=.8\textwidth]{./assets/img/computed_probabilities_vs_theoretic_probabilities/main.pdf}
    \caption{The
        relationship between the steady state probabilities inferred from the
        measured transitions and the actual steady state probabilities. A linear
        regression line is included validating the approach.}
    \label{fig:computed_probabilities_vs_theoretic_probabilities}
\end{figure}


\end{document}

    strategies is considered. In this setting
    the most highly performing strategies do not play in an extortionate way
    against each other but do against lower performing strategies.
    This suggests that whilst the theory of Zero Determinant strategies
    indicates that memory is not of fundamental importance to the evolution of
    cooperative behaviour, this is incomplete.
\end{abstract}

\section{Introduction}\label{sec:introduction}

Agent based game theoretic models have become a stalwart of the underpinning
mathematics of interactive behaviours. One of the major pieces of work
in this area is the pair of original computer tournaments run by Robert
Axelrod~\cite{Axelrod1980, Axelrod1980a}. These tournaments pitted submitted
computer strategies against each other in plays of the Iterated Prisoner's
Dilemma. A common game where agents can choose to pay a slight cost to their
immediate utility in the hope of building a reputation. This has been used in
economic and evolutionary game theory to understand the evolution of cooperative
behaviour.

Recently, a class of strategies was described in~\cite{Press2012} that can
provably extort any given opponent. In~\cite{Hilbe2013, Moran1707} some
questions have already been asked about the true effectiveness of these
strategies in an evolutionary setting. Here another question is asked: is it
possible to recognise this extortionate behaviour? A mathematical procedure for
suspicion is presented: in the same way that the continued actions of an
extortionate individual might raise suspicion.

This work makes use of the Axelrod Python library~\cite{Knight2018, Knight2016}
with a large number of Prisoner Dilemma strategies available to give an
extensive numerical example of the ideas presented.  The approach is presented
in Section~\ref{sec:delta-zd-strategies}.  All of the code and data discussed
in Section~\ref{sec:numerical-experiments} is open sourced, archived and
written according to best scientific principles~\cite{Wilson2014}. The data
archive can be found at~\cite{vincent_knight_2018_1297075}.

\section{Recognising Extortion}\label{sec:delta-zd-strategies}

In~\cite{Press2012}, given a match between 2 memory-one strategies, the concept
of Zero Determinant (ZD) strategies is introduced. The main result of that paper
shows that given two memory one players \(p, q\in\mathbb{R}^4\) a linear
relationship between the players' scores could be forced by one of the players.

Using the notation of~\cite{Press2012}, assuming the utilities for player \(p\)
are given by \(S_x=(R, S, T, P)\) and for player \(q\) by \(S_y=(R, T, S, P)\)
and that the stationary scores of each player is given by \(S_X\) and \(S_Y\)
respectively. The main result of~\cite{Press2012} is that if

\begin{equation}\label{eqn:linear_relationship_for_p}
    \tilde p=\alpha S_x + \beta S_y + \gamma
\end{equation}

or

\begin{equation}\label{eqn:linear_relationship_for_q}
    \tilde q=\alpha S_x + \beta S_y + \gamma
\end{equation}

where \(\tilde p = (1 - p_1, 1 - p_2, p_3, p_4)\) and
\(\tilde q = (1 - q_1, 1 - q_2, q_3, q_4)\) then:

\begin{equation}
    \alpha S_X + \beta S_Y + \gamma = 0
\end{equation}

In~\cite{Press2012} a particular type of ZD strategy is defined: extortionate
strategies. If:

\begin{equation}\label{eqn:constraint_for_extortion}
    \gamma = - P(\alpha + \beta)
\end{equation}

then the player can ensure they get a score \(\chi\) times
larger than the opponent. This extortion coefficient is given by:

\begin{equation}\label{eqn:definition_of_chi}
    \chi=\frac{-\beta}{\alpha}
\end{equation}

Thus, if (\ref{eqn:constraint_for_extortion}) holds and \(\chi >1\) a player is
said to extort their opponent.
Here, the reverse problem is considered: given a
\(p\in\mathbb{R}^4\) how does one identify \(\alpha, \beta\) if they
exist and is the strategy in fact acting in an extortionate way?

These conditions correspond to:

\begin{align}
    \tilde p_1 & = \alpha R + \beta R - P (\alpha + \beta)
            \label{eqn:condition_for_tilde_p1}\\
    \tilde p_2 & = \alpha S + \beta T - P (\alpha + \beta)
            \label{eqn:condition_for_tilde_p2}\\
    \tilde p_3 & = \alpha T + \beta S - P (\alpha + \beta)
            \label{eqn:condition_for_tilde_p3}\\
    \tilde p_4 & = \alpha P + \beta P - P (\alpha + \beta)
            \label{eqn:condition_for_tilde_p4}
\end{align}

Equation (\ref{eqn:condition_for_tilde_p4}) ensures that \(p_4=\tilde p_4=0\).
Equations (\ref{eqn:condition_for_tilde_p1}-\ref{eqn:condition_for_tilde_p3})
can be used to eliminate \(\alpha, \beta\), giving:

\begin{equation}\label{eqn:planar_definition_of_extortion}
    \tilde p_1 = \frac{(R - P)(\tilde p_2 + \tilde p_3)}{S + T - 2P}
\end{equation}

with:

\begin{equation}\label{eqn:definition_of_chi}
    \chi = \frac{\tilde p_2 (P - T) + \tilde p_3 (S - P)}
                {\tilde p_2 (P - S) + \tilde p_3 (T - P)}
\end{equation}

Given a strategy \(p\in\mathbb{R}^{4\times 1}\) equations
(\ref{eqn:condition_for_tilde_p4}), (\ref{eqn:planar_definition_of_extortion}-\ref{eqn:definition_of_chi}) can be used to check if
a strategy is extortionate. The conditions correspond to:

\begin{align}
    p_1 & = \frac{(R-P)(p_2 + p_3) - R + T + S - P}{S + T - 2P}
     \label{eqn:condition_for_p1}\\
    p_4 & = 0 \label{eqn:condition_for_p4}\\
    1 & > p_2 + p_3\label{eqn:condition_for_chi}
\end{align}

The algebraic steps necessary to prove these results are available in the
supporting materials.

All extortionate strategies reside on a triangular (\ref{eqn:condition_for_chi})
plane (\ref{eqn:condition_for_p1}) in 3 dimensions (\ref{eqn:condition_for_p4}).
Using this formulation it can be seen that a necessary (but not sufficient)
condition for an extortionate strategy is that it cooperates on average less
than 50\% of the time when in a state of disagreement with the opponent.

As an example, consider the known extortionate strategy \(p=(8 / 9, 1 / 2, 1 /
3, 0)\) from~\cite{Stewart2012} which is referred to as \texttt{Extort-2}. In
this case, for the standard values of \((R, T, S, P)\) constraint
(\ref{eqn:condition_for_p1}) corresponds to:

\begin{equation}
    p_1 = \frac{2(p_2 + p_3) + 1}{3}
\end{equation}

It is clear that in this case all constraints hold.

This approach could in fact be used to confirm that a given strategy is acting
in an extortionate manner even if it is not a memory one strategy. However, in
practice, if a closed form for \(p\) is not known, then due to measurement
and/or numerical error this would not work.

This problem can be written in the following linear algebraic form where
\(x=(\alpha, \beta)\)
and \(p^*=(\tilde p_1 - 1, tilde_2 - 1, p_3)\):

\begin{equation}\label{eqn:linear_algebraic_equation_for_p}
    Cx= p^*
\end{equation}

\(C\) corresponds to equations
(\ref{eqn:condition_for_tilde_p1}-\ref{eqn:condition_for_tilde_p3}) and is
given by:

\begin{equation}\label{eqn:definition_of_C}
    C =
    \begin{bmatrix}
        R - P & R- P \\
        S - P & T- P \\
        T - P & S- P \\
    \end{bmatrix}
\end{equation}

Note that in general, equation (\ref{eqn:linear_algebraic_equation_for_p}) will
not necessarily have a solution. From the Rouch\'{e}-Capelli theorem if there is
a solution it is unique as \(\text{rank}(C)=2\) which is the dimension of the
variable \(x\). The best fitting \(x\) is found by minimizing:

\begin{equation}\label{eqn:r_squared}
    \text{SSError} = \|C x- p^*\|_2^2 = \sum_{i=1}^{3}\left((C\bar x)_i-p_i^*\right)^2
\end{equation}

Note that \(\text{SSError}\), which is the square of the Frobenius
norm~\cite{Golub2013}, becomes a measure of how close a strategy is to being an
extortionate strategy. Suspicion
of extortion then corresponds to a threshold on \(\text{SSError}\).

By observing interactions (human or otherwise), their memory one representation
can be inferred and this approach can be used to recognise extortionate
behaviour. The notion of comparing theoretic and actual plays of the IPD is not
novel, see for example~\cite{Rand2013}. Immediately it is noted that if the
environment is noisy~\cite{Wu1995} then no strategy can be considered to be
extortionate as \(p_4>0\).

In the next section, this idea will be illustrated by observing the interactions
that take place in a computer based tournament of the IPD\@.

\section{Numerical experiments}\label{sec:numerical-experiments}

In~\cite{Stewart2012} results from a tournament with
\documentclass[a4paper]{article}

\usepackage{amsmath}
\usepackage{amssymb}
\usepackage[margin=1.5cm,
            includefoot,
            footskip=30pt]{geometry}
\usepackage{layout}
\usepackage{graphicx}
\usepackage{subcaption}

\usepackage{biblatex}
\usepackage{pdfpages}

\bibliography{main.bib}

\title{Suspicion: Recognising and evaluating the effectiveness
       of extortion in the Iterated Prisoner's Dilemma}
\author{Vincent A. Knight \and Nikoleta E. Glynatsi}
\date{\today}



\begin{document}

\maketitle

\begin{abstract}
    The Iterated Prisoner's Dilemma is a model for rational and evolutionary
    interactive behaviour. It has applications both in the study of human social
    behaviour as well as in biology.
    It is used to understand when and how a rational individual might
    accept an immediate cost to their own utility for the direct benefit of
    another.

    Much attention has been given to a class of strategies called
    Zero Determinant strategies. It has been theoretically shown that these
    strategies can ``extort'' any player.

    In this work, an approach to identify if observed strategies are playing in
    an extortionate way is described. Furthermore, experimental analysis of
    a large tournament with \input{assets/tex/number_of_full_strategies/main.tex}
    strategies is considered. In this setting
    the most highly performing strategies do not play in an extortionate way
    against each other but do against lower performing strategies.
    This suggests that whilst the theory of Zero Determinant strategies
    indicates that memory is not of fundamental importance to the evolution of
    cooperative behaviour, this is incomplete.
\end{abstract}

\section{Introduction}\label{sec:introduction}

Agent based game theoretic models have become a stalwart of the underpinning
mathematics of interactive behaviours. One of the major pieces of work
in this area is the pair of original computer tournaments run by Robert
Axelrod~\cite{Axelrod1980, Axelrod1980a}. These tournaments pitted submitted
computer strategies against each other in plays of the Iterated Prisoner's
Dilemma. A common game where agents can choose to pay a slight cost to their
immediate utility in the hope of building a reputation. This has been used in
economic and evolutionary game theory to understand the evolution of cooperative
behaviour.

Recently, a class of strategies was described in~\cite{Press2012} that can
provably extort any given opponent. In~\cite{Hilbe2013, Moran1707} some
questions have already been asked about the true effectiveness of these
strategies in an evolutionary setting. Here another question is asked: is it
possible to recognise this extortionate behaviour? A mathematical procedure for
suspicion is presented: in the same way that the continued actions of an
extortionate individual might raise suspicion.

This work makes use of the Axelrod Python library~\cite{Knight2018, Knight2016}
with a large number of Prisoner Dilemma strategies available to give an
extensive numerical example of the ideas presented.  The approach is presented
in Section~\ref{sec:delta-zd-strategies}.  All of the code and data discussed
in Section~\ref{sec:numerical-experiments} is open sourced, archived and
written according to best scientific principles~\cite{Wilson2014}. The data
archive can be found at~\cite{vincent_knight_2018_1297075}.

\section{Recognising Extortion}\label{sec:delta-zd-strategies}

In~\cite{Press2012}, given a match between 2 memory-one strategies, the concept
of Zero Determinant (ZD) strategies is introduced. The main result of that paper
shows that given two memory one players \(p, q\in\mathbb{R}^4\) a linear
relationship between the players' scores could be forced by one of the players.

Using the notation of~\cite{Press2012}, assuming the utilities for player \(p\)
are given by \(S_x=(R, S, T, P)\) and for player \(q\) by \(S_y=(R, T, S, P)\)
and that the stationary scores of each player is given by \(S_X\) and \(S_Y\)
respectively. The main result of~\cite{Press2012} is that if

\begin{equation}\label{eqn:linear_relationship_for_p}
    \tilde p=\alpha S_x + \beta S_y + \gamma
\end{equation}

or

\begin{equation}\label{eqn:linear_relationship_for_q}
    \tilde q=\alpha S_x + \beta S_y + \gamma
\end{equation}

where \(\tilde p = (1 - p_1, 1 - p_2, p_3, p_4)\) and
\(\tilde q = (1 - q_1, 1 - q_2, q_3, q_4)\) then:

\begin{equation}
    \alpha S_X + \beta S_Y + \gamma = 0
\end{equation}

In~\cite{Press2012} a particular type of ZD strategy is defined: extortionate
strategies. If:

\begin{equation}\label{eqn:constraint_for_extortion}
    \gamma = - P(\alpha + \beta)
\end{equation}

then the player can ensure they get a score \(\chi\) times
larger than the opponent. This extortion coefficient is given by:

\begin{equation}\label{eqn:definition_of_chi}
    \chi=\frac{-\beta}{\alpha}
\end{equation}

Thus, if (\ref{eqn:constraint_for_extortion}) holds and \(\chi >1\) a player is
said to extort their opponent.
Here, the reverse problem is considered: given a
\(p\in\mathbb{R}^4\) how does one identify \(\alpha, \beta\) if they
exist and is the strategy in fact acting in an extortionate way?

These conditions correspond to:

\begin{align}
    \tilde p_1 & = \alpha R + \beta R - P (\alpha + \beta)
            \label{eqn:condition_for_tilde_p1}\\
    \tilde p_2 & = \alpha S + \beta T - P (\alpha + \beta)
            \label{eqn:condition_for_tilde_p2}\\
    \tilde p_3 & = \alpha T + \beta S - P (\alpha + \beta)
            \label{eqn:condition_for_tilde_p3}\\
    \tilde p_4 & = \alpha P + \beta P - P (\alpha + \beta)
            \label{eqn:condition_for_tilde_p4}
\end{align}

Equation (\ref{eqn:condition_for_tilde_p4}) ensures that \(p_4=\tilde p_4=0\).
Equations (\ref{eqn:condition_for_tilde_p1}-\ref{eqn:condition_for_tilde_p3})
can be used to eliminate \(\alpha, \beta\), giving:

\begin{equation}\label{eqn:planar_definition_of_extortion}
    \tilde p_1 = \frac{(R - P)(\tilde p_2 + \tilde p_3)}{S + T - 2P}
\end{equation}

with:

\begin{equation}\label{eqn:definition_of_chi}
    \chi = \frac{\tilde p_2 (P - T) + \tilde p_3 (S - P)}
                {\tilde p_2 (P - S) + \tilde p_3 (T - P)}
\end{equation}

Given a strategy \(p\in\mathbb{R}^{4\times 1}\) equations
(\ref{eqn:condition_for_tilde_p4}), (\ref{eqn:planar_definition_of_extortion}-\ref{eqn:definition_of_chi}) can be used to check if
a strategy is extortionate. The conditions correspond to:

\begin{align}
    p_1 & = \frac{(R-P)(p_2 + p_3) - R + T + S - P}{S + T - 2P}
     \label{eqn:condition_for_p1}\\
    p_4 & = 0 \label{eqn:condition_for_p4}\\
    1 & > p_2 + p_3\label{eqn:condition_for_chi}
\end{align}

The algebraic steps necessary to prove these results are available in the
supporting materials.

All extortionate strategies reside on a triangular (\ref{eqn:condition_for_chi})
plane (\ref{eqn:condition_for_p1}) in 3 dimensions (\ref{eqn:condition_for_p4}).
Using this formulation it can be seen that a necessary (but not sufficient)
condition for an extortionate strategy is that it cooperates on average less
than 50\% of the time when in a state of disagreement with the opponent.

As an example, consider the known extortionate strategy \(p=(8 / 9, 1 / 2, 1 /
3, 0)\) from~\cite{Stewart2012} which is referred to as \texttt{Extort-2}. In
this case, for the standard values of \((R, T, S, P)\) constraint
(\ref{eqn:condition_for_p1}) corresponds to:

\begin{equation}
    p_1 = \frac{2(p_2 + p_3) + 1}{3}
\end{equation}

It is clear that in this case all constraints hold.

This approach could in fact be used to confirm that a given strategy is acting
in an extortionate manner even if it is not a memory one strategy. However, in
practice, if a closed form for \(p\) is not known, then due to measurement
and/or numerical error this would not work.

This problem can be written in the following linear algebraic form where
\(x=(\alpha, \beta)\)
and \(p^*=(\tilde p_1 - 1, tilde_2 - 1, p_3)\):

\begin{equation}\label{eqn:linear_algebraic_equation_for_p}
    Cx= p^*
\end{equation}

\(C\) corresponds to equations
(\ref{eqn:condition_for_tilde_p1}-\ref{eqn:condition_for_tilde_p3}) and is
given by:

\begin{equation}\label{eqn:definition_of_C}
    C =
    \begin{bmatrix}
        R - P & R- P \\
        S - P & T- P \\
        T - P & S- P \\
    \end{bmatrix}
\end{equation}

Note that in general, equation (\ref{eqn:linear_algebraic_equation_for_p}) will
not necessarily have a solution. From the Rouch\'{e}-Capelli theorem if there is
a solution it is unique as \(\text{rank}(C)=2\) which is the dimension of the
variable \(x\). The best fitting \(x\) is found by minimizing:

\begin{equation}\label{eqn:r_squared}
    \text{SSError} = \|C x- p^*\|_2^2 = \sum_{i=1}^{3}\left((C\bar x)_i-p_i^*\right)^2
\end{equation}

Note that \(\text{SSError}\), which is the square of the Frobenius
norm~\cite{Golub2013}, becomes a measure of how close a strategy is to being an
extortionate strategy. Suspicion
of extortion then corresponds to a threshold on \(\text{SSError}\).

By observing interactions (human or otherwise), their memory one representation
can be inferred and this approach can be used to recognise extortionate
behaviour. The notion of comparing theoretic and actual plays of the IPD is not
novel, see for example~\cite{Rand2013}. Immediately it is noted that if the
environment is noisy~\cite{Wu1995} then no strategy can be considered to be
extortionate as \(p_4>0\).

In the next section, this idea will be illustrated by observing the interactions
that take place in a computer based tournament of the IPD\@.

\section{Numerical experiments}\label{sec:numerical-experiments}

In~\cite{Stewart2012} results from a tournament with
\input{./assets/tex/number_of_stewart_plotkin_strategies/main.tex} strategies,
was presented with specific consideration given to ZD strategies. This
tournament is reproduced here using the Axelrod-Python
project~\cite{Knight2016}. To obtain a good measure of the corresponding
transition rates for each strategy all matches have been run for
\input{assets/tex/number_of_turns/main.tex} turns and every match has been
repeated \input{assets/tex/number_of_repetitions/main.tex} times. All of this
interaction data is available at~\cite{vincent_knight_2018_1297075}. A good
match between the inferred Markov chain and the state distribution of the actual
interactions has been verified. Data for this is presented in the supplementary
materials.

Figure~\ref{fig:SSError_overall_in_stewart_plotkin} shows the \(\text{SSError}\)
values for all the strategies in the tournament, as reported
in~\cite{Stewart2012} the extortionate strategy (which has an expected
\(\text{SSError}\) approximately 0) gains a large number of wins.

\begin{figure}[!htbp]
    \centering
    \includegraphics[width=.8\textwidth]{./assets/img/SSError_overall_in_stewart_plotkin/main.pdf}
    \caption{\(\text{SSError}\) and state probabilities for the strategies
        of~\cite{Stewart2012}, ordered both by number of wins and overall score.
        Note that \(P(DC)\) is not shown as it corresponds to the transpose of
        \(P(CD)\). Cooperator and Defector are omitted as they do not visit all
        the states.}
    \label{fig:SSError_overall_in_stewart_plotkin}
\end{figure}

Here, the work of~\cite{Stewart2012} is extended by investigating a tournament
with \input{assets/tex/number_of_full_strategies/main.tex}
strategies.

The results of this analysis are shown in
Figure~\ref{fig:SSError_and_probabilities_in_full}. The top ranking strategies
by number of wins seem to be extortionate (but not against all strategies) and
it can be seen that a small sub group of strategies achieve mutual defection.
All the top ranking strategies according to score achieve mutual cooperation and
do not extort each other, however they
\textbf{do} exhibit extortionate behaviour towards a number of the lower ranking
strategies.

\begin{figure}[!htbp]
    \centering
    \includegraphics[width=.8\textwidth]{./assets/img/SSError_and_probabilities_in_full/main.pdf}
    \caption{\(\text{SSError}\) for the strategies for the full tournament. Only
    strategy interactions for which \(p_4=0\) and \(\chi>1\) are displayed.}
    \label{fig:SSError_and_probabilities_in_full}
\end{figure}

\section{Conclusion}\label{sec:conclusion}

This work defines an approach to measure whether or not a player is playing a
strategy that corresponds to an extortionate strategy as defined
in~\cite{Press2012}: a mathematical model for suspicion. Indeed, all
extortionate strategies have been
 classified as lying on a triangular plane.
This rigorous classification fails to be robust to small measurement error, thus
a statistical approach is proposed.
This is done through a linear algebraic approach for approximating the solution
of a linear system. Using this, a large number of pairwise interactions is
simulated and in fact very few strategies are found to act extortionately.

The work of~\cite{Press2012}, whilst showing that a clever approach to taking
advantage of another memory one strategy exists: this is incomplete. Whilst the
elegance of this result is very attractive, just as the simplicity of the
victory of Tit For Tat in Axelrod's original tournaments was, it is incomplete.
Extortionate strategies achieve a high number of wins but they do not
achieve a high score which corresponds to the fitness landscape in an
evolutionary sense. From the large number of interactions a payoff matrix \(S\)
can be measured where \(S_{ij}\) denotes the score (using standard values of
\((R, S, T, P) = (3, 0, 5, 1)\)) of the \(i\)th strategy
against the \(j\)th strategy. Using this, the replicator equation
describes the evolution of the system based on a population density fitness
function:

\begin{equation}\label{eqn:replicator_dynamics}
    \frac{dx}{dt} = x(S-x^TS x)
\end{equation}

Equation (\ref{eqn:replicator_dynamics}) is solved numerically through an
integration technique described in~\cite{Petzold1983} and
Figure~\ref{fig:replicator_dynamics} shows the evolution of the distribution of
the system: the various strategies are ranked by scores. It is clear to see that
only the high ranking strategies survive the evolutionary process (in fact,
only \input{./assets/img/replicator_dynamics/main.tex}
have a final distribution greater than \(10 ^ {-2}\)). This confirms the
findings of~\cite{Moran1707} in which sophisticated strategies resist
evolutionary invasion of shorter memory strategies. Recalling
Figure~\ref{fig:SSError_and_probabilities_in_full} this demonstrates that:

\begin{itemize}
    \item Cooperation emerges through the evolutionary process: the high scoring
        strategies do not exhibit extortionate behaviour towards each other.
    \item Extortionate strategies do not survive the evolutionary process.
\end{itemize}

\begin{figure}[!htbp]
    \centering
    \includegraphics[width=.8\textwidth]{./assets/img/replicator_dynamics/main.pdf}
    \caption{Numerical simulation of the replicator equation
    (\ref{eqn:replicator_dynamics}): strategies are ordered by score, only the strategies with a high score survive the evolutionary process.}
    \label{fig:replicator_dynamics}
\end{figure}

This work can be used to classify plays of the IPD\@: data can be collected from
actual interactions (in lab or in the field). Furthermore, this allows for a
classification method similar to the notion of fingerprinting presented
in~\cite{Ashlock2008}. Trained strategies can potentially be classified as
extortionate or not or it could be possible to even constrain the reinforcement
learning approaches that are becoming prevalent in the literature.
Alternatively, this mathematical approach for recognising extortion could be
used in sophisticated strategies to defend against invasion. Arguably, some of
the strategies considered here exhibit this behaviour, indeed as described
in~\cite{Harper2017}, the top ranking strategies in the full tournament are
obtained using evolutionary reinforcement learning techniques, thus, suspicion
of extortionate behaviour could in fact be an evolutionary trait.

\section*{Acknowledgements}

The following open source software libraries were used in this research:

\begin{itemize}
    \item The Axelrod ~\cite{Knight2016, Knight2018} library (IPD strategies and
        tournaments).
    \item The sympy library~\cite{Meurer2017} (verification of all symbolic
        calculations).
    \item The matplotlib~\cite{Droettboom2018} library (visualisation).
    \item The pandas~\cite{Structures2010}, dask~\cite{Dask2016} and
        NumPy~\cite{Oliphant2015} libraries (data manipulation).
    \item The SciPy~\cite{Jones2001} library (numerical integration of the
        replicator equation).
\end{itemize}

This work was performed using the computational facilities of the Advanced
Research Computing @ Cardiff (ARCCA) Division, Cardiff University.

\printbibliography

\newpage
\section*{Supplementary materials}

\includepdf{assets/pdf/proof_of_form_of_extortionate_strategies/main.pdf}

\newpage

Using the pair wise interactions the transition rates \(p,
q\) can be measured and the steady state probabilities inferred and compared to
the actual probabilities of each state.
This is done numerically by computing the singular eigenvector of the
matrix \(A\) \cite{Stewart2009}:

\[
    A =
    \begin{bmatrix}
        p_1 q_1 & p_1 (1 - q_1) & (1 - p_1) q_1 & (1 -p_1) (1 - q_1) \\
        p_2 q_2 & p_2 (1 - q_2) & (1 - p_2) q_2 & (1 -p_2) (1 - q_2) \\
        p_3 q_3 & p_3 (1 - q_3) & (1 - p_3) q_3 & (1 -p_3) (1 - q_3) \\
        p_4 q_4 & p_4 (1 - q_4) & (1 - p_4) q_4 & (1 -p_4) (1 - q_4) \\
    \end{bmatrix}
\]

Figure~\ref{fig:computed_probabilities_vs_theoretic_probabilities} shows a
regression line fitted to every pairwise interaction with a reported
\(\text{SSError}\) value (pairwise interactions with missing states were
omitted). This serves to validate the approach: a part from some edge cases the
relationship is consistent.

\begin{figure}[!htbp]
    \centering
    \includegraphics[width=.8\textwidth]{./assets/img/computed_probabilities_vs_theoretic_probabilities/main.pdf}
    \caption{The
        relationship between the steady state probabilities inferred from the
        measured transitions and the actual steady state probabilities. A linear
        regression line is included validating the approach.}
    \label{fig:computed_probabilities_vs_theoretic_probabilities}
\end{figure}


\end{document}
 strategies,
was presented with specific consideration given to ZD strategies. This
tournament is reproduced here using the Axelrod-Python
project~\cite{Knight2016}. To obtain a good measure of the corresponding
transition rates for each strategy all matches have been run for
\documentclass[a4paper]{article}

\usepackage{amsmath}
\usepackage{amssymb}
\usepackage[margin=1.5cm,
            includefoot,
            footskip=30pt]{geometry}
\usepackage{layout}
\usepackage{graphicx}
\usepackage{subcaption}

\usepackage{biblatex}
\usepackage{pdfpages}

\bibliography{main.bib}

\title{Suspicion: Recognising and evaluating the effectiveness
       of extortion in the Iterated Prisoner's Dilemma}
\author{Vincent A. Knight \and Nikoleta E. Glynatsi}
\date{\today}



\begin{document}

\maketitle

\begin{abstract}
    The Iterated Prisoner's Dilemma is a model for rational and evolutionary
    interactive behaviour. It has applications both in the study of human social
    behaviour as well as in biology.
    It is used to understand when and how a rational individual might
    accept an immediate cost to their own utility for the direct benefit of
    another.

    Much attention has been given to a class of strategies called
    Zero Determinant strategies. It has been theoretically shown that these
    strategies can ``extort'' any player.

    In this work, an approach to identify if observed strategies are playing in
    an extortionate way is described. Furthermore, experimental analysis of
    a large tournament with \input{assets/tex/number_of_full_strategies/main.tex}
    strategies is considered. In this setting
    the most highly performing strategies do not play in an extortionate way
    against each other but do against lower performing strategies.
    This suggests that whilst the theory of Zero Determinant strategies
    indicates that memory is not of fundamental importance to the evolution of
    cooperative behaviour, this is incomplete.
\end{abstract}

\section{Introduction}\label{sec:introduction}

Agent based game theoretic models have become a stalwart of the underpinning
mathematics of interactive behaviours. One of the major pieces of work
in this area is the pair of original computer tournaments run by Robert
Axelrod~\cite{Axelrod1980, Axelrod1980a}. These tournaments pitted submitted
computer strategies against each other in plays of the Iterated Prisoner's
Dilemma. A common game where agents can choose to pay a slight cost to their
immediate utility in the hope of building a reputation. This has been used in
economic and evolutionary game theory to understand the evolution of cooperative
behaviour.

Recently, a class of strategies was described in~\cite{Press2012} that can
provably extort any given opponent. In~\cite{Hilbe2013, Moran1707} some
questions have already been asked about the true effectiveness of these
strategies in an evolutionary setting. Here another question is asked: is it
possible to recognise this extortionate behaviour? A mathematical procedure for
suspicion is presented: in the same way that the continued actions of an
extortionate individual might raise suspicion.

This work makes use of the Axelrod Python library~\cite{Knight2018, Knight2016}
with a large number of Prisoner Dilemma strategies available to give an
extensive numerical example of the ideas presented.  The approach is presented
in Section~\ref{sec:delta-zd-strategies}.  All of the code and data discussed
in Section~\ref{sec:numerical-experiments} is open sourced, archived and
written according to best scientific principles~\cite{Wilson2014}. The data
archive can be found at~\cite{vincent_knight_2018_1297075}.

\section{Recognising Extortion}\label{sec:delta-zd-strategies}

In~\cite{Press2012}, given a match between 2 memory-one strategies, the concept
of Zero Determinant (ZD) strategies is introduced. The main result of that paper
shows that given two memory one players \(p, q\in\mathbb{R}^4\) a linear
relationship between the players' scores could be forced by one of the players.

Using the notation of~\cite{Press2012}, assuming the utilities for player \(p\)
are given by \(S_x=(R, S, T, P)\) and for player \(q\) by \(S_y=(R, T, S, P)\)
and that the stationary scores of each player is given by \(S_X\) and \(S_Y\)
respectively. The main result of~\cite{Press2012} is that if

\begin{equation}\label{eqn:linear_relationship_for_p}
    \tilde p=\alpha S_x + \beta S_y + \gamma
\end{equation}

or

\begin{equation}\label{eqn:linear_relationship_for_q}
    \tilde q=\alpha S_x + \beta S_y + \gamma
\end{equation}

where \(\tilde p = (1 - p_1, 1 - p_2, p_3, p_4)\) and
\(\tilde q = (1 - q_1, 1 - q_2, q_3, q_4)\) then:

\begin{equation}
    \alpha S_X + \beta S_Y + \gamma = 0
\end{equation}

In~\cite{Press2012} a particular type of ZD strategy is defined: extortionate
strategies. If:

\begin{equation}\label{eqn:constraint_for_extortion}
    \gamma = - P(\alpha + \beta)
\end{equation}

then the player can ensure they get a score \(\chi\) times
larger than the opponent. This extortion coefficient is given by:

\begin{equation}\label{eqn:definition_of_chi}
    \chi=\frac{-\beta}{\alpha}
\end{equation}

Thus, if (\ref{eqn:constraint_for_extortion}) holds and \(\chi >1\) a player is
said to extort their opponent.
Here, the reverse problem is considered: given a
\(p\in\mathbb{R}^4\) how does one identify \(\alpha, \beta\) if they
exist and is the strategy in fact acting in an extortionate way?

These conditions correspond to:

\begin{align}
    \tilde p_1 & = \alpha R + \beta R - P (\alpha + \beta)
            \label{eqn:condition_for_tilde_p1}\\
    \tilde p_2 & = \alpha S + \beta T - P (\alpha + \beta)
            \label{eqn:condition_for_tilde_p2}\\
    \tilde p_3 & = \alpha T + \beta S - P (\alpha + \beta)
            \label{eqn:condition_for_tilde_p3}\\
    \tilde p_4 & = \alpha P + \beta P - P (\alpha + \beta)
            \label{eqn:condition_for_tilde_p4}
\end{align}

Equation (\ref{eqn:condition_for_tilde_p4}) ensures that \(p_4=\tilde p_4=0\).
Equations (\ref{eqn:condition_for_tilde_p1}-\ref{eqn:condition_for_tilde_p3})
can be used to eliminate \(\alpha, \beta\), giving:

\begin{equation}\label{eqn:planar_definition_of_extortion}
    \tilde p_1 = \frac{(R - P)(\tilde p_2 + \tilde p_3)}{S + T - 2P}
\end{equation}

with:

\begin{equation}\label{eqn:definition_of_chi}
    \chi = \frac{\tilde p_2 (P - T) + \tilde p_3 (S - P)}
                {\tilde p_2 (P - S) + \tilde p_3 (T - P)}
\end{equation}

Given a strategy \(p\in\mathbb{R}^{4\times 1}\) equations
(\ref{eqn:condition_for_tilde_p4}), (\ref{eqn:planar_definition_of_extortion}-\ref{eqn:definition_of_chi}) can be used to check if
a strategy is extortionate. The conditions correspond to:

\begin{align}
    p_1 & = \frac{(R-P)(p_2 + p_3) - R + T + S - P}{S + T - 2P}
     \label{eqn:condition_for_p1}\\
    p_4 & = 0 \label{eqn:condition_for_p4}\\
    1 & > p_2 + p_3\label{eqn:condition_for_chi}
\end{align}

The algebraic steps necessary to prove these results are available in the
supporting materials.

All extortionate strategies reside on a triangular (\ref{eqn:condition_for_chi})
plane (\ref{eqn:condition_for_p1}) in 3 dimensions (\ref{eqn:condition_for_p4}).
Using this formulation it can be seen that a necessary (but not sufficient)
condition for an extortionate strategy is that it cooperates on average less
than 50\% of the time when in a state of disagreement with the opponent.

As an example, consider the known extortionate strategy \(p=(8 / 9, 1 / 2, 1 /
3, 0)\) from~\cite{Stewart2012} which is referred to as \texttt{Extort-2}. In
this case, for the standard values of \((R, T, S, P)\) constraint
(\ref{eqn:condition_for_p1}) corresponds to:

\begin{equation}
    p_1 = \frac{2(p_2 + p_3) + 1}{3}
\end{equation}

It is clear that in this case all constraints hold.

This approach could in fact be used to confirm that a given strategy is acting
in an extortionate manner even if it is not a memory one strategy. However, in
practice, if a closed form for \(p\) is not known, then due to measurement
and/or numerical error this would not work.

This problem can be written in the following linear algebraic form where
\(x=(\alpha, \beta)\)
and \(p^*=(\tilde p_1 - 1, tilde_2 - 1, p_3)\):

\begin{equation}\label{eqn:linear_algebraic_equation_for_p}
    Cx= p^*
\end{equation}

\(C\) corresponds to equations
(\ref{eqn:condition_for_tilde_p1}-\ref{eqn:condition_for_tilde_p3}) and is
given by:

\begin{equation}\label{eqn:definition_of_C}
    C =
    \begin{bmatrix}
        R - P & R- P \\
        S - P & T- P \\
        T - P & S- P \\
    \end{bmatrix}
\end{equation}

Note that in general, equation (\ref{eqn:linear_algebraic_equation_for_p}) will
not necessarily have a solution. From the Rouch\'{e}-Capelli theorem if there is
a solution it is unique as \(\text{rank}(C)=2\) which is the dimension of the
variable \(x\). The best fitting \(x\) is found by minimizing:

\begin{equation}\label{eqn:r_squared}
    \text{SSError} = \|C x- p^*\|_2^2 = \sum_{i=1}^{3}\left((C\bar x)_i-p_i^*\right)^2
\end{equation}

Note that \(\text{SSError}\), which is the square of the Frobenius
norm~\cite{Golub2013}, becomes a measure of how close a strategy is to being an
extortionate strategy. Suspicion
of extortion then corresponds to a threshold on \(\text{SSError}\).

By observing interactions (human or otherwise), their memory one representation
can be inferred and this approach can be used to recognise extortionate
behaviour. The notion of comparing theoretic and actual plays of the IPD is not
novel, see for example~\cite{Rand2013}. Immediately it is noted that if the
environment is noisy~\cite{Wu1995} then no strategy can be considered to be
extortionate as \(p_4>0\).

In the next section, this idea will be illustrated by observing the interactions
that take place in a computer based tournament of the IPD\@.

\section{Numerical experiments}\label{sec:numerical-experiments}

In~\cite{Stewart2012} results from a tournament with
\input{./assets/tex/number_of_stewart_plotkin_strategies/main.tex} strategies,
was presented with specific consideration given to ZD strategies. This
tournament is reproduced here using the Axelrod-Python
project~\cite{Knight2016}. To obtain a good measure of the corresponding
transition rates for each strategy all matches have been run for
\input{assets/tex/number_of_turns/main.tex} turns and every match has been
repeated \input{assets/tex/number_of_repetitions/main.tex} times. All of this
interaction data is available at~\cite{vincent_knight_2018_1297075}. A good
match between the inferred Markov chain and the state distribution of the actual
interactions has been verified. Data for this is presented in the supplementary
materials.

Figure~\ref{fig:SSError_overall_in_stewart_plotkin} shows the \(\text{SSError}\)
values for all the strategies in the tournament, as reported
in~\cite{Stewart2012} the extortionate strategy (which has an expected
\(\text{SSError}\) approximately 0) gains a large number of wins.

\begin{figure}[!htbp]
    \centering
    \includegraphics[width=.8\textwidth]{./assets/img/SSError_overall_in_stewart_plotkin/main.pdf}
    \caption{\(\text{SSError}\) and state probabilities for the strategies
        of~\cite{Stewart2012}, ordered both by number of wins and overall score.
        Note that \(P(DC)\) is not shown as it corresponds to the transpose of
        \(P(CD)\). Cooperator and Defector are omitted as they do not visit all
        the states.}
    \label{fig:SSError_overall_in_stewart_plotkin}
\end{figure}

Here, the work of~\cite{Stewart2012} is extended by investigating a tournament
with \input{assets/tex/number_of_full_strategies/main.tex}
strategies.

The results of this analysis are shown in
Figure~\ref{fig:SSError_and_probabilities_in_full}. The top ranking strategies
by number of wins seem to be extortionate (but not against all strategies) and
it can be seen that a small sub group of strategies achieve mutual defection.
All the top ranking strategies according to score achieve mutual cooperation and
do not extort each other, however they
\textbf{do} exhibit extortionate behaviour towards a number of the lower ranking
strategies.

\begin{figure}[!htbp]
    \centering
    \includegraphics[width=.8\textwidth]{./assets/img/SSError_and_probabilities_in_full/main.pdf}
    \caption{\(\text{SSError}\) for the strategies for the full tournament. Only
    strategy interactions for which \(p_4=0\) and \(\chi>1\) are displayed.}
    \label{fig:SSError_and_probabilities_in_full}
\end{figure}

\section{Conclusion}\label{sec:conclusion}

This work defines an approach to measure whether or not a player is playing a
strategy that corresponds to an extortionate strategy as defined
in~\cite{Press2012}: a mathematical model for suspicion. Indeed, all
extortionate strategies have been
 classified as lying on a triangular plane.
This rigorous classification fails to be robust to small measurement error, thus
a statistical approach is proposed.
This is done through a linear algebraic approach for approximating the solution
of a linear system. Using this, a large number of pairwise interactions is
simulated and in fact very few strategies are found to act extortionately.

The work of~\cite{Press2012}, whilst showing that a clever approach to taking
advantage of another memory one strategy exists: this is incomplete. Whilst the
elegance of this result is very attractive, just as the simplicity of the
victory of Tit For Tat in Axelrod's original tournaments was, it is incomplete.
Extortionate strategies achieve a high number of wins but they do not
achieve a high score which corresponds to the fitness landscape in an
evolutionary sense. From the large number of interactions a payoff matrix \(S\)
can be measured where \(S_{ij}\) denotes the score (using standard values of
\((R, S, T, P) = (3, 0, 5, 1)\)) of the \(i\)th strategy
against the \(j\)th strategy. Using this, the replicator equation
describes the evolution of the system based on a population density fitness
function:

\begin{equation}\label{eqn:replicator_dynamics}
    \frac{dx}{dt} = x(S-x^TS x)
\end{equation}

Equation (\ref{eqn:replicator_dynamics}) is solved numerically through an
integration technique described in~\cite{Petzold1983} and
Figure~\ref{fig:replicator_dynamics} shows the evolution of the distribution of
the system: the various strategies are ranked by scores. It is clear to see that
only the high ranking strategies survive the evolutionary process (in fact,
only \input{./assets/img/replicator_dynamics/main.tex}
have a final distribution greater than \(10 ^ {-2}\)). This confirms the
findings of~\cite{Moran1707} in which sophisticated strategies resist
evolutionary invasion of shorter memory strategies. Recalling
Figure~\ref{fig:SSError_and_probabilities_in_full} this demonstrates that:

\begin{itemize}
    \item Cooperation emerges through the evolutionary process: the high scoring
        strategies do not exhibit extortionate behaviour towards each other.
    \item Extortionate strategies do not survive the evolutionary process.
\end{itemize}

\begin{figure}[!htbp]
    \centering
    \includegraphics[width=.8\textwidth]{./assets/img/replicator_dynamics/main.pdf}
    \caption{Numerical simulation of the replicator equation
    (\ref{eqn:replicator_dynamics}): strategies are ordered by score, only the strategies with a high score survive the evolutionary process.}
    \label{fig:replicator_dynamics}
\end{figure}

This work can be used to classify plays of the IPD\@: data can be collected from
actual interactions (in lab or in the field). Furthermore, this allows for a
classification method similar to the notion of fingerprinting presented
in~\cite{Ashlock2008}. Trained strategies can potentially be classified as
extortionate or not or it could be possible to even constrain the reinforcement
learning approaches that are becoming prevalent in the literature.
Alternatively, this mathematical approach for recognising extortion could be
used in sophisticated strategies to defend against invasion. Arguably, some of
the strategies considered here exhibit this behaviour, indeed as described
in~\cite{Harper2017}, the top ranking strategies in the full tournament are
obtained using evolutionary reinforcement learning techniques, thus, suspicion
of extortionate behaviour could in fact be an evolutionary trait.

\section*{Acknowledgements}

The following open source software libraries were used in this research:

\begin{itemize}
    \item The Axelrod ~\cite{Knight2016, Knight2018} library (IPD strategies and
        tournaments).
    \item The sympy library~\cite{Meurer2017} (verification of all symbolic
        calculations).
    \item The matplotlib~\cite{Droettboom2018} library (visualisation).
    \item The pandas~\cite{Structures2010}, dask~\cite{Dask2016} and
        NumPy~\cite{Oliphant2015} libraries (data manipulation).
    \item The SciPy~\cite{Jones2001} library (numerical integration of the
        replicator equation).
\end{itemize}

This work was performed using the computational facilities of the Advanced
Research Computing @ Cardiff (ARCCA) Division, Cardiff University.

\printbibliography

\newpage
\section*{Supplementary materials}

\includepdf{assets/pdf/proof_of_form_of_extortionate_strategies/main.pdf}

\newpage

Using the pair wise interactions the transition rates \(p,
q\) can be measured and the steady state probabilities inferred and compared to
the actual probabilities of each state.
This is done numerically by computing the singular eigenvector of the
matrix \(A\) \cite{Stewart2009}:

\[
    A =
    \begin{bmatrix}
        p_1 q_1 & p_1 (1 - q_1) & (1 - p_1) q_1 & (1 -p_1) (1 - q_1) \\
        p_2 q_2 & p_2 (1 - q_2) & (1 - p_2) q_2 & (1 -p_2) (1 - q_2) \\
        p_3 q_3 & p_3 (1 - q_3) & (1 - p_3) q_3 & (1 -p_3) (1 - q_3) \\
        p_4 q_4 & p_4 (1 - q_4) & (1 - p_4) q_4 & (1 -p_4) (1 - q_4) \\
    \end{bmatrix}
\]

Figure~\ref{fig:computed_probabilities_vs_theoretic_probabilities} shows a
regression line fitted to every pairwise interaction with a reported
\(\text{SSError}\) value (pairwise interactions with missing states were
omitted). This serves to validate the approach: a part from some edge cases the
relationship is consistent.

\begin{figure}[!htbp]
    \centering
    \includegraphics[width=.8\textwidth]{./assets/img/computed_probabilities_vs_theoretic_probabilities/main.pdf}
    \caption{The
        relationship between the steady state probabilities inferred from the
        measured transitions and the actual steady state probabilities. A linear
        regression line is included validating the approach.}
    \label{fig:computed_probabilities_vs_theoretic_probabilities}
\end{figure}


\end{document}
 turns and every match has been
repeated \documentclass[a4paper]{article}

\usepackage{amsmath}
\usepackage{amssymb}
\usepackage[margin=1.5cm,
            includefoot,
            footskip=30pt]{geometry}
\usepackage{layout}
\usepackage{graphicx}
\usepackage{subcaption}

\usepackage{biblatex}
\usepackage{pdfpages}

\bibliography{main.bib}

\title{Suspicion: Recognising and evaluating the effectiveness
       of extortion in the Iterated Prisoner's Dilemma}
\author{Vincent A. Knight \and Nikoleta E. Glynatsi}
\date{\today}



\begin{document}

\maketitle

\begin{abstract}
    The Iterated Prisoner's Dilemma is a model for rational and evolutionary
    interactive behaviour. It has applications both in the study of human social
    behaviour as well as in biology.
    It is used to understand when and how a rational individual might
    accept an immediate cost to their own utility for the direct benefit of
    another.

    Much attention has been given to a class of strategies called
    Zero Determinant strategies. It has been theoretically shown that these
    strategies can ``extort'' any player.

    In this work, an approach to identify if observed strategies are playing in
    an extortionate way is described. Furthermore, experimental analysis of
    a large tournament with \input{assets/tex/number_of_full_strategies/main.tex}
    strategies is considered. In this setting
    the most highly performing strategies do not play in an extortionate way
    against each other but do against lower performing strategies.
    This suggests that whilst the theory of Zero Determinant strategies
    indicates that memory is not of fundamental importance to the evolution of
    cooperative behaviour, this is incomplete.
\end{abstract}

\section{Introduction}\label{sec:introduction}

Agent based game theoretic models have become a stalwart of the underpinning
mathematics of interactive behaviours. One of the major pieces of work
in this area is the pair of original computer tournaments run by Robert
Axelrod~\cite{Axelrod1980, Axelrod1980a}. These tournaments pitted submitted
computer strategies against each other in plays of the Iterated Prisoner's
Dilemma. A common game where agents can choose to pay a slight cost to their
immediate utility in the hope of building a reputation. This has been used in
economic and evolutionary game theory to understand the evolution of cooperative
behaviour.

Recently, a class of strategies was described in~\cite{Press2012} that can
provably extort any given opponent. In~\cite{Hilbe2013, Moran1707} some
questions have already been asked about the true effectiveness of these
strategies in an evolutionary setting. Here another question is asked: is it
possible to recognise this extortionate behaviour? A mathematical procedure for
suspicion is presented: in the same way that the continued actions of an
extortionate individual might raise suspicion.

This work makes use of the Axelrod Python library~\cite{Knight2018, Knight2016}
with a large number of Prisoner Dilemma strategies available to give an
extensive numerical example of the ideas presented.  The approach is presented
in Section~\ref{sec:delta-zd-strategies}.  All of the code and data discussed
in Section~\ref{sec:numerical-experiments} is open sourced, archived and
written according to best scientific principles~\cite{Wilson2014}. The data
archive can be found at~\cite{vincent_knight_2018_1297075}.

\section{Recognising Extortion}\label{sec:delta-zd-strategies}

In~\cite{Press2012}, given a match between 2 memory-one strategies, the concept
of Zero Determinant (ZD) strategies is introduced. The main result of that paper
shows that given two memory one players \(p, q\in\mathbb{R}^4\) a linear
relationship between the players' scores could be forced by one of the players.

Using the notation of~\cite{Press2012}, assuming the utilities for player \(p\)
are given by \(S_x=(R, S, T, P)\) and for player \(q\) by \(S_y=(R, T, S, P)\)
and that the stationary scores of each player is given by \(S_X\) and \(S_Y\)
respectively. The main result of~\cite{Press2012} is that if

\begin{equation}\label{eqn:linear_relationship_for_p}
    \tilde p=\alpha S_x + \beta S_y + \gamma
\end{equation}

or

\begin{equation}\label{eqn:linear_relationship_for_q}
    \tilde q=\alpha S_x + \beta S_y + \gamma
\end{equation}

where \(\tilde p = (1 - p_1, 1 - p_2, p_3, p_4)\) and
\(\tilde q = (1 - q_1, 1 - q_2, q_3, q_4)\) then:

\begin{equation}
    \alpha S_X + \beta S_Y + \gamma = 0
\end{equation}

In~\cite{Press2012} a particular type of ZD strategy is defined: extortionate
strategies. If:

\begin{equation}\label{eqn:constraint_for_extortion}
    \gamma = - P(\alpha + \beta)
\end{equation}

then the player can ensure they get a score \(\chi\) times
larger than the opponent. This extortion coefficient is given by:

\begin{equation}\label{eqn:definition_of_chi}
    \chi=\frac{-\beta}{\alpha}
\end{equation}

Thus, if (\ref{eqn:constraint_for_extortion}) holds and \(\chi >1\) a player is
said to extort their opponent.
Here, the reverse problem is considered: given a
\(p\in\mathbb{R}^4\) how does one identify \(\alpha, \beta\) if they
exist and is the strategy in fact acting in an extortionate way?

These conditions correspond to:

\begin{align}
    \tilde p_1 & = \alpha R + \beta R - P (\alpha + \beta)
            \label{eqn:condition_for_tilde_p1}\\
    \tilde p_2 & = \alpha S + \beta T - P (\alpha + \beta)
            \label{eqn:condition_for_tilde_p2}\\
    \tilde p_3 & = \alpha T + \beta S - P (\alpha + \beta)
            \label{eqn:condition_for_tilde_p3}\\
    \tilde p_4 & = \alpha P + \beta P - P (\alpha + \beta)
            \label{eqn:condition_for_tilde_p4}
\end{align}

Equation (\ref{eqn:condition_for_tilde_p4}) ensures that \(p_4=\tilde p_4=0\).
Equations (\ref{eqn:condition_for_tilde_p1}-\ref{eqn:condition_for_tilde_p3})
can be used to eliminate \(\alpha, \beta\), giving:

\begin{equation}\label{eqn:planar_definition_of_extortion}
    \tilde p_1 = \frac{(R - P)(\tilde p_2 + \tilde p_3)}{S + T - 2P}
\end{equation}

with:

\begin{equation}\label{eqn:definition_of_chi}
    \chi = \frac{\tilde p_2 (P - T) + \tilde p_3 (S - P)}
                {\tilde p_2 (P - S) + \tilde p_3 (T - P)}
\end{equation}

Given a strategy \(p\in\mathbb{R}^{4\times 1}\) equations
(\ref{eqn:condition_for_tilde_p4}), (\ref{eqn:planar_definition_of_extortion}-\ref{eqn:definition_of_chi}) can be used to check if
a strategy is extortionate. The conditions correspond to:

\begin{align}
    p_1 & = \frac{(R-P)(p_2 + p_3) - R + T + S - P}{S + T - 2P}
     \label{eqn:condition_for_p1}\\
    p_4 & = 0 \label{eqn:condition_for_p4}\\
    1 & > p_2 + p_3\label{eqn:condition_for_chi}
\end{align}

The algebraic steps necessary to prove these results are available in the
supporting materials.

All extortionate strategies reside on a triangular (\ref{eqn:condition_for_chi})
plane (\ref{eqn:condition_for_p1}) in 3 dimensions (\ref{eqn:condition_for_p4}).
Using this formulation it can be seen that a necessary (but not sufficient)
condition for an extortionate strategy is that it cooperates on average less
than 50\% of the time when in a state of disagreement with the opponent.

As an example, consider the known extortionate strategy \(p=(8 / 9, 1 / 2, 1 /
3, 0)\) from~\cite{Stewart2012} which is referred to as \texttt{Extort-2}. In
this case, for the standard values of \((R, T, S, P)\) constraint
(\ref{eqn:condition_for_p1}) corresponds to:

\begin{equation}
    p_1 = \frac{2(p_2 + p_3) + 1}{3}
\end{equation}

It is clear that in this case all constraints hold.

This approach could in fact be used to confirm that a given strategy is acting
in an extortionate manner even if it is not a memory one strategy. However, in
practice, if a closed form for \(p\) is not known, then due to measurement
and/or numerical error this would not work.

This problem can be written in the following linear algebraic form where
\(x=(\alpha, \beta)\)
and \(p^*=(\tilde p_1 - 1, tilde_2 - 1, p_3)\):

\begin{equation}\label{eqn:linear_algebraic_equation_for_p}
    Cx= p^*
\end{equation}

\(C\) corresponds to equations
(\ref{eqn:condition_for_tilde_p1}-\ref{eqn:condition_for_tilde_p3}) and is
given by:

\begin{equation}\label{eqn:definition_of_C}
    C =
    \begin{bmatrix}
        R - P & R- P \\
        S - P & T- P \\
        T - P & S- P \\
    \end{bmatrix}
\end{equation}

Note that in general, equation (\ref{eqn:linear_algebraic_equation_for_p}) will
not necessarily have a solution. From the Rouch\'{e}-Capelli theorem if there is
a solution it is unique as \(\text{rank}(C)=2\) which is the dimension of the
variable \(x\). The best fitting \(x\) is found by minimizing:

\begin{equation}\label{eqn:r_squared}
    \text{SSError} = \|C x- p^*\|_2^2 = \sum_{i=1}^{3}\left((C\bar x)_i-p_i^*\right)^2
\end{equation}

Note that \(\text{SSError}\), which is the square of the Frobenius
norm~\cite{Golub2013}, becomes a measure of how close a strategy is to being an
extortionate strategy. Suspicion
of extortion then corresponds to a threshold on \(\text{SSError}\).

By observing interactions (human or otherwise), their memory one representation
can be inferred and this approach can be used to recognise extortionate
behaviour. The notion of comparing theoretic and actual plays of the IPD is not
novel, see for example~\cite{Rand2013}. Immediately it is noted that if the
environment is noisy~\cite{Wu1995} then no strategy can be considered to be
extortionate as \(p_4>0\).

In the next section, this idea will be illustrated by observing the interactions
that take place in a computer based tournament of the IPD\@.

\section{Numerical experiments}\label{sec:numerical-experiments}

In~\cite{Stewart2012} results from a tournament with
\input{./assets/tex/number_of_stewart_plotkin_strategies/main.tex} strategies,
was presented with specific consideration given to ZD strategies. This
tournament is reproduced here using the Axelrod-Python
project~\cite{Knight2016}. To obtain a good measure of the corresponding
transition rates for each strategy all matches have been run for
\input{assets/tex/number_of_turns/main.tex} turns and every match has been
repeated \input{assets/tex/number_of_repetitions/main.tex} times. All of this
interaction data is available at~\cite{vincent_knight_2018_1297075}. A good
match between the inferred Markov chain and the state distribution of the actual
interactions has been verified. Data for this is presented in the supplementary
materials.

Figure~\ref{fig:SSError_overall_in_stewart_plotkin} shows the \(\text{SSError}\)
values for all the strategies in the tournament, as reported
in~\cite{Stewart2012} the extortionate strategy (which has an expected
\(\text{SSError}\) approximately 0) gains a large number of wins.

\begin{figure}[!htbp]
    \centering
    \includegraphics[width=.8\textwidth]{./assets/img/SSError_overall_in_stewart_plotkin/main.pdf}
    \caption{\(\text{SSError}\) and state probabilities for the strategies
        of~\cite{Stewart2012}, ordered both by number of wins and overall score.
        Note that \(P(DC)\) is not shown as it corresponds to the transpose of
        \(P(CD)\). Cooperator and Defector are omitted as they do not visit all
        the states.}
    \label{fig:SSError_overall_in_stewart_plotkin}
\end{figure}

Here, the work of~\cite{Stewart2012} is extended by investigating a tournament
with \input{assets/tex/number_of_full_strategies/main.tex}
strategies.

The results of this analysis are shown in
Figure~\ref{fig:SSError_and_probabilities_in_full}. The top ranking strategies
by number of wins seem to be extortionate (but not against all strategies) and
it can be seen that a small sub group of strategies achieve mutual defection.
All the top ranking strategies according to score achieve mutual cooperation and
do not extort each other, however they
\textbf{do} exhibit extortionate behaviour towards a number of the lower ranking
strategies.

\begin{figure}[!htbp]
    \centering
    \includegraphics[width=.8\textwidth]{./assets/img/SSError_and_probabilities_in_full/main.pdf}
    \caption{\(\text{SSError}\) for the strategies for the full tournament. Only
    strategy interactions for which \(p_4=0\) and \(\chi>1\) are displayed.}
    \label{fig:SSError_and_probabilities_in_full}
\end{figure}

\section{Conclusion}\label{sec:conclusion}

This work defines an approach to measure whether or not a player is playing a
strategy that corresponds to an extortionate strategy as defined
in~\cite{Press2012}: a mathematical model for suspicion. Indeed, all
extortionate strategies have been
 classified as lying on a triangular plane.
This rigorous classification fails to be robust to small measurement error, thus
a statistical approach is proposed.
This is done through a linear algebraic approach for approximating the solution
of a linear system. Using this, a large number of pairwise interactions is
simulated and in fact very few strategies are found to act extortionately.

The work of~\cite{Press2012}, whilst showing that a clever approach to taking
advantage of another memory one strategy exists: this is incomplete. Whilst the
elegance of this result is very attractive, just as the simplicity of the
victory of Tit For Tat in Axelrod's original tournaments was, it is incomplete.
Extortionate strategies achieve a high number of wins but they do not
achieve a high score which corresponds to the fitness landscape in an
evolutionary sense. From the large number of interactions a payoff matrix \(S\)
can be measured where \(S_{ij}\) denotes the score (using standard values of
\((R, S, T, P) = (3, 0, 5, 1)\)) of the \(i\)th strategy
against the \(j\)th strategy. Using this, the replicator equation
describes the evolution of the system based on a population density fitness
function:

\begin{equation}\label{eqn:replicator_dynamics}
    \frac{dx}{dt} = x(S-x^TS x)
\end{equation}

Equation (\ref{eqn:replicator_dynamics}) is solved numerically through an
integration technique described in~\cite{Petzold1983} and
Figure~\ref{fig:replicator_dynamics} shows the evolution of the distribution of
the system: the various strategies are ranked by scores. It is clear to see that
only the high ranking strategies survive the evolutionary process (in fact,
only \input{./assets/img/replicator_dynamics/main.tex}
have a final distribution greater than \(10 ^ {-2}\)). This confirms the
findings of~\cite{Moran1707} in which sophisticated strategies resist
evolutionary invasion of shorter memory strategies. Recalling
Figure~\ref{fig:SSError_and_probabilities_in_full} this demonstrates that:

\begin{itemize}
    \item Cooperation emerges through the evolutionary process: the high scoring
        strategies do not exhibit extortionate behaviour towards each other.
    \item Extortionate strategies do not survive the evolutionary process.
\end{itemize}

\begin{figure}[!htbp]
    \centering
    \includegraphics[width=.8\textwidth]{./assets/img/replicator_dynamics/main.pdf}
    \caption{Numerical simulation of the replicator equation
    (\ref{eqn:replicator_dynamics}): strategies are ordered by score, only the strategies with a high score survive the evolutionary process.}
    \label{fig:replicator_dynamics}
\end{figure}

This work can be used to classify plays of the IPD\@: data can be collected from
actual interactions (in lab or in the field). Furthermore, this allows for a
classification method similar to the notion of fingerprinting presented
in~\cite{Ashlock2008}. Trained strategies can potentially be classified as
extortionate or not or it could be possible to even constrain the reinforcement
learning approaches that are becoming prevalent in the literature.
Alternatively, this mathematical approach for recognising extortion could be
used in sophisticated strategies to defend against invasion. Arguably, some of
the strategies considered here exhibit this behaviour, indeed as described
in~\cite{Harper2017}, the top ranking strategies in the full tournament are
obtained using evolutionary reinforcement learning techniques, thus, suspicion
of extortionate behaviour could in fact be an evolutionary trait.

\section*{Acknowledgements}

The following open source software libraries were used in this research:

\begin{itemize}
    \item The Axelrod ~\cite{Knight2016, Knight2018} library (IPD strategies and
        tournaments).
    \item The sympy library~\cite{Meurer2017} (verification of all symbolic
        calculations).
    \item The matplotlib~\cite{Droettboom2018} library (visualisation).
    \item The pandas~\cite{Structures2010}, dask~\cite{Dask2016} and
        NumPy~\cite{Oliphant2015} libraries (data manipulation).
    \item The SciPy~\cite{Jones2001} library (numerical integration of the
        replicator equation).
\end{itemize}

This work was performed using the computational facilities of the Advanced
Research Computing @ Cardiff (ARCCA) Division, Cardiff University.

\printbibliography

\newpage
\section*{Supplementary materials}

\includepdf{assets/pdf/proof_of_form_of_extortionate_strategies/main.pdf}

\newpage

Using the pair wise interactions the transition rates \(p,
q\) can be measured and the steady state probabilities inferred and compared to
the actual probabilities of each state.
This is done numerically by computing the singular eigenvector of the
matrix \(A\) \cite{Stewart2009}:

\[
    A =
    \begin{bmatrix}
        p_1 q_1 & p_1 (1 - q_1) & (1 - p_1) q_1 & (1 -p_1) (1 - q_1) \\
        p_2 q_2 & p_2 (1 - q_2) & (1 - p_2) q_2 & (1 -p_2) (1 - q_2) \\
        p_3 q_3 & p_3 (1 - q_3) & (1 - p_3) q_3 & (1 -p_3) (1 - q_3) \\
        p_4 q_4 & p_4 (1 - q_4) & (1 - p_4) q_4 & (1 -p_4) (1 - q_4) \\
    \end{bmatrix}
\]

Figure~\ref{fig:computed_probabilities_vs_theoretic_probabilities} shows a
regression line fitted to every pairwise interaction with a reported
\(\text{SSError}\) value (pairwise interactions with missing states were
omitted). This serves to validate the approach: a part from some edge cases the
relationship is consistent.

\begin{figure}[!htbp]
    \centering
    \includegraphics[width=.8\textwidth]{./assets/img/computed_probabilities_vs_theoretic_probabilities/main.pdf}
    \caption{The
        relationship between the steady state probabilities inferred from the
        measured transitions and the actual steady state probabilities. A linear
        regression line is included validating the approach.}
    \label{fig:computed_probabilities_vs_theoretic_probabilities}
\end{figure}


\end{document}
 times. All of this
interaction data is available at~\cite{vincent_knight_2018_1297075}. A good
match between the inferred Markov chain and the state distribution of the actual
interactions has been verified. Data for this is presented in the supplementary
materials.

Figure~\ref{fig:SSError_overall_in_stewart_plotkin} shows the \(\text{SSError}\)
values for all the strategies in the tournament, as reported
in~\cite{Stewart2012} the extortionate strategy (which has an expected
\(\text{SSError}\) approximately 0) gains a large number of wins.

\begin{figure}[!htbp]
    \centering
    \includegraphics[width=.8\textwidth]{./assets/img/SSError_overall_in_stewart_plotkin/main.pdf}
    \caption{\(\text{SSError}\) and state probabilities for the strategies
        of~\cite{Stewart2012}, ordered both by number of wins and overall score.
        Note that \(P(DC)\) is not shown as it corresponds to the transpose of
        \(P(CD)\). Cooperator and Defector are omitted as they do not visit all
        the states.}
    \label{fig:SSError_overall_in_stewart_plotkin}
\end{figure}

Here, the work of~\cite{Stewart2012} is extended by investigating a tournament
with \documentclass[a4paper]{article}

\usepackage{amsmath}
\usepackage{amssymb}
\usepackage[margin=1.5cm,
            includefoot,
            footskip=30pt]{geometry}
\usepackage{layout}
\usepackage{graphicx}
\usepackage{subcaption}

\usepackage{biblatex}
\usepackage{pdfpages}

\bibliography{main.bib}

\title{Suspicion: Recognising and evaluating the effectiveness
       of extortion in the Iterated Prisoner's Dilemma}
\author{Vincent A. Knight \and Nikoleta E. Glynatsi}
\date{\today}



\begin{document}

\maketitle

\begin{abstract}
    The Iterated Prisoner's Dilemma is a model for rational and evolutionary
    interactive behaviour. It has applications both in the study of human social
    behaviour as well as in biology.
    It is used to understand when and how a rational individual might
    accept an immediate cost to their own utility for the direct benefit of
    another.

    Much attention has been given to a class of strategies called
    Zero Determinant strategies. It has been theoretically shown that these
    strategies can ``extort'' any player.

    In this work, an approach to identify if observed strategies are playing in
    an extortionate way is described. Furthermore, experimental analysis of
    a large tournament with \input{assets/tex/number_of_full_strategies/main.tex}
    strategies is considered. In this setting
    the most highly performing strategies do not play in an extortionate way
    against each other but do against lower performing strategies.
    This suggests that whilst the theory of Zero Determinant strategies
    indicates that memory is not of fundamental importance to the evolution of
    cooperative behaviour, this is incomplete.
\end{abstract}

\section{Introduction}\label{sec:introduction}

Agent based game theoretic models have become a stalwart of the underpinning
mathematics of interactive behaviours. One of the major pieces of work
in this area is the pair of original computer tournaments run by Robert
Axelrod~\cite{Axelrod1980, Axelrod1980a}. These tournaments pitted submitted
computer strategies against each other in plays of the Iterated Prisoner's
Dilemma. A common game where agents can choose to pay a slight cost to their
immediate utility in the hope of building a reputation. This has been used in
economic and evolutionary game theory to understand the evolution of cooperative
behaviour.

Recently, a class of strategies was described in~\cite{Press2012} that can
provably extort any given opponent. In~\cite{Hilbe2013, Moran1707} some
questions have already been asked about the true effectiveness of these
strategies in an evolutionary setting. Here another question is asked: is it
possible to recognise this extortionate behaviour? A mathematical procedure for
suspicion is presented: in the same way that the continued actions of an
extortionate individual might raise suspicion.

This work makes use of the Axelrod Python library~\cite{Knight2018, Knight2016}
with a large number of Prisoner Dilemma strategies available to give an
extensive numerical example of the ideas presented.  The approach is presented
in Section~\ref{sec:delta-zd-strategies}.  All of the code and data discussed
in Section~\ref{sec:numerical-experiments} is open sourced, archived and
written according to best scientific principles~\cite{Wilson2014}. The data
archive can be found at~\cite{vincent_knight_2018_1297075}.

\section{Recognising Extortion}\label{sec:delta-zd-strategies}

In~\cite{Press2012}, given a match between 2 memory-one strategies, the concept
of Zero Determinant (ZD) strategies is introduced. The main result of that paper
shows that given two memory one players \(p, q\in\mathbb{R}^4\) a linear
relationship between the players' scores could be forced by one of the players.

Using the notation of~\cite{Press2012}, assuming the utilities for player \(p\)
are given by \(S_x=(R, S, T, P)\) and for player \(q\) by \(S_y=(R, T, S, P)\)
and that the stationary scores of each player is given by \(S_X\) and \(S_Y\)
respectively. The main result of~\cite{Press2012} is that if

\begin{equation}\label{eqn:linear_relationship_for_p}
    \tilde p=\alpha S_x + \beta S_y + \gamma
\end{equation}

or

\begin{equation}\label{eqn:linear_relationship_for_q}
    \tilde q=\alpha S_x + \beta S_y + \gamma
\end{equation}

where \(\tilde p = (1 - p_1, 1 - p_2, p_3, p_4)\) and
\(\tilde q = (1 - q_1, 1 - q_2, q_3, q_4)\) then:

\begin{equation}
    \alpha S_X + \beta S_Y + \gamma = 0
\end{equation}

In~\cite{Press2012} a particular type of ZD strategy is defined: extortionate
strategies. If:

\begin{equation}\label{eqn:constraint_for_extortion}
    \gamma = - P(\alpha + \beta)
\end{equation}

then the player can ensure they get a score \(\chi\) times
larger than the opponent. This extortion coefficient is given by:

\begin{equation}\label{eqn:definition_of_chi}
    \chi=\frac{-\beta}{\alpha}
\end{equation}

Thus, if (\ref{eqn:constraint_for_extortion}) holds and \(\chi >1\) a player is
said to extort their opponent.
Here, the reverse problem is considered: given a
\(p\in\mathbb{R}^4\) how does one identify \(\alpha, \beta\) if they
exist and is the strategy in fact acting in an extortionate way?

These conditions correspond to:

\begin{align}
    \tilde p_1 & = \alpha R + \beta R - P (\alpha + \beta)
            \label{eqn:condition_for_tilde_p1}\\
    \tilde p_2 & = \alpha S + \beta T - P (\alpha + \beta)
            \label{eqn:condition_for_tilde_p2}\\
    \tilde p_3 & = \alpha T + \beta S - P (\alpha + \beta)
            \label{eqn:condition_for_tilde_p3}\\
    \tilde p_4 & = \alpha P + \beta P - P (\alpha + \beta)
            \label{eqn:condition_for_tilde_p4}
\end{align}

Equation (\ref{eqn:condition_for_tilde_p4}) ensures that \(p_4=\tilde p_4=0\).
Equations (\ref{eqn:condition_for_tilde_p1}-\ref{eqn:condition_for_tilde_p3})
can be used to eliminate \(\alpha, \beta\), giving:

\begin{equation}\label{eqn:planar_definition_of_extortion}
    \tilde p_1 = \frac{(R - P)(\tilde p_2 + \tilde p_3)}{S + T - 2P}
\end{equation}

with:

\begin{equation}\label{eqn:definition_of_chi}
    \chi = \frac{\tilde p_2 (P - T) + \tilde p_3 (S - P)}
                {\tilde p_2 (P - S) + \tilde p_3 (T - P)}
\end{equation}

Given a strategy \(p\in\mathbb{R}^{4\times 1}\) equations
(\ref{eqn:condition_for_tilde_p4}), (\ref{eqn:planar_definition_of_extortion}-\ref{eqn:definition_of_chi}) can be used to check if
a strategy is extortionate. The conditions correspond to:

\begin{align}
    p_1 & = \frac{(R-P)(p_2 + p_3) - R + T + S - P}{S + T - 2P}
     \label{eqn:condition_for_p1}\\
    p_4 & = 0 \label{eqn:condition_for_p4}\\
    1 & > p_2 + p_3\label{eqn:condition_for_chi}
\end{align}

The algebraic steps necessary to prove these results are available in the
supporting materials.

All extortionate strategies reside on a triangular (\ref{eqn:condition_for_chi})
plane (\ref{eqn:condition_for_p1}) in 3 dimensions (\ref{eqn:condition_for_p4}).
Using this formulation it can be seen that a necessary (but not sufficient)
condition for an extortionate strategy is that it cooperates on average less
than 50\% of the time when in a state of disagreement with the opponent.

As an example, consider the known extortionate strategy \(p=(8 / 9, 1 / 2, 1 /
3, 0)\) from~\cite{Stewart2012} which is referred to as \texttt{Extort-2}. In
this case, for the standard values of \((R, T, S, P)\) constraint
(\ref{eqn:condition_for_p1}) corresponds to:

\begin{equation}
    p_1 = \frac{2(p_2 + p_3) + 1}{3}
\end{equation}

It is clear that in this case all constraints hold.

This approach could in fact be used to confirm that a given strategy is acting
in an extortionate manner even if it is not a memory one strategy. However, in
practice, if a closed form for \(p\) is not known, then due to measurement
and/or numerical error this would not work.

This problem can be written in the following linear algebraic form where
\(x=(\alpha, \beta)\)
and \(p^*=(\tilde p_1 - 1, tilde_2 - 1, p_3)\):

\begin{equation}\label{eqn:linear_algebraic_equation_for_p}
    Cx= p^*
\end{equation}

\(C\) corresponds to equations
(\ref{eqn:condition_for_tilde_p1}-\ref{eqn:condition_for_tilde_p3}) and is
given by:

\begin{equation}\label{eqn:definition_of_C}
    C =
    \begin{bmatrix}
        R - P & R- P \\
        S - P & T- P \\
        T - P & S- P \\
    \end{bmatrix}
\end{equation}

Note that in general, equation (\ref{eqn:linear_algebraic_equation_for_p}) will
not necessarily have a solution. From the Rouch\'{e}-Capelli theorem if there is
a solution it is unique as \(\text{rank}(C)=2\) which is the dimension of the
variable \(x\). The best fitting \(x\) is found by minimizing:

\begin{equation}\label{eqn:r_squared}
    \text{SSError} = \|C x- p^*\|_2^2 = \sum_{i=1}^{3}\left((C\bar x)_i-p_i^*\right)^2
\end{equation}

Note that \(\text{SSError}\), which is the square of the Frobenius
norm~\cite{Golub2013}, becomes a measure of how close a strategy is to being an
extortionate strategy. Suspicion
of extortion then corresponds to a threshold on \(\text{SSError}\).

By observing interactions (human or otherwise), their memory one representation
can be inferred and this approach can be used to recognise extortionate
behaviour. The notion of comparing theoretic and actual plays of the IPD is not
novel, see for example~\cite{Rand2013}. Immediately it is noted that if the
environment is noisy~\cite{Wu1995} then no strategy can be considered to be
extortionate as \(p_4>0\).

In the next section, this idea will be illustrated by observing the interactions
that take place in a computer based tournament of the IPD\@.

\section{Numerical experiments}\label{sec:numerical-experiments}

In~\cite{Stewart2012} results from a tournament with
\input{./assets/tex/number_of_stewart_plotkin_strategies/main.tex} strategies,
was presented with specific consideration given to ZD strategies. This
tournament is reproduced here using the Axelrod-Python
project~\cite{Knight2016}. To obtain a good measure of the corresponding
transition rates for each strategy all matches have been run for
\input{assets/tex/number_of_turns/main.tex} turns and every match has been
repeated \input{assets/tex/number_of_repetitions/main.tex} times. All of this
interaction data is available at~\cite{vincent_knight_2018_1297075}. A good
match between the inferred Markov chain and the state distribution of the actual
interactions has been verified. Data for this is presented in the supplementary
materials.

Figure~\ref{fig:SSError_overall_in_stewart_plotkin} shows the \(\text{SSError}\)
values for all the strategies in the tournament, as reported
in~\cite{Stewart2012} the extortionate strategy (which has an expected
\(\text{SSError}\) approximately 0) gains a large number of wins.

\begin{figure}[!htbp]
    \centering
    \includegraphics[width=.8\textwidth]{./assets/img/SSError_overall_in_stewart_plotkin/main.pdf}
    \caption{\(\text{SSError}\) and state probabilities for the strategies
        of~\cite{Stewart2012}, ordered both by number of wins and overall score.
        Note that \(P(DC)\) is not shown as it corresponds to the transpose of
        \(P(CD)\). Cooperator and Defector are omitted as they do not visit all
        the states.}
    \label{fig:SSError_overall_in_stewart_plotkin}
\end{figure}

Here, the work of~\cite{Stewart2012} is extended by investigating a tournament
with \input{assets/tex/number_of_full_strategies/main.tex}
strategies.

The results of this analysis are shown in
Figure~\ref{fig:SSError_and_probabilities_in_full}. The top ranking strategies
by number of wins seem to be extortionate (but not against all strategies) and
it can be seen that a small sub group of strategies achieve mutual defection.
All the top ranking strategies according to score achieve mutual cooperation and
do not extort each other, however they
\textbf{do} exhibit extortionate behaviour towards a number of the lower ranking
strategies.

\begin{figure}[!htbp]
    \centering
    \includegraphics[width=.8\textwidth]{./assets/img/SSError_and_probabilities_in_full/main.pdf}
    \caption{\(\text{SSError}\) for the strategies for the full tournament. Only
    strategy interactions for which \(p_4=0\) and \(\chi>1\) are displayed.}
    \label{fig:SSError_and_probabilities_in_full}
\end{figure}

\section{Conclusion}\label{sec:conclusion}

This work defines an approach to measure whether or not a player is playing a
strategy that corresponds to an extortionate strategy as defined
in~\cite{Press2012}: a mathematical model for suspicion. Indeed, all
extortionate strategies have been
 classified as lying on a triangular plane.
This rigorous classification fails to be robust to small measurement error, thus
a statistical approach is proposed.
This is done through a linear algebraic approach for approximating the solution
of a linear system. Using this, a large number of pairwise interactions is
simulated and in fact very few strategies are found to act extortionately.

The work of~\cite{Press2012}, whilst showing that a clever approach to taking
advantage of another memory one strategy exists: this is incomplete. Whilst the
elegance of this result is very attractive, just as the simplicity of the
victory of Tit For Tat in Axelrod's original tournaments was, it is incomplete.
Extortionate strategies achieve a high number of wins but they do not
achieve a high score which corresponds to the fitness landscape in an
evolutionary sense. From the large number of interactions a payoff matrix \(S\)
can be measured where \(S_{ij}\) denotes the score (using standard values of
\((R, S, T, P) = (3, 0, 5, 1)\)) of the \(i\)th strategy
against the \(j\)th strategy. Using this, the replicator equation
describes the evolution of the system based on a population density fitness
function:

\begin{equation}\label{eqn:replicator_dynamics}
    \frac{dx}{dt} = x(S-x^TS x)
\end{equation}

Equation (\ref{eqn:replicator_dynamics}) is solved numerically through an
integration technique described in~\cite{Petzold1983} and
Figure~\ref{fig:replicator_dynamics} shows the evolution of the distribution of
the system: the various strategies are ranked by scores. It is clear to see that
only the high ranking strategies survive the evolutionary process (in fact,
only \input{./assets/img/replicator_dynamics/main.tex}
have a final distribution greater than \(10 ^ {-2}\)). This confirms the
findings of~\cite{Moran1707} in which sophisticated strategies resist
evolutionary invasion of shorter memory strategies. Recalling
Figure~\ref{fig:SSError_and_probabilities_in_full} this demonstrates that:

\begin{itemize}
    \item Cooperation emerges through the evolutionary process: the high scoring
        strategies do not exhibit extortionate behaviour towards each other.
    \item Extortionate strategies do not survive the evolutionary process.
\end{itemize}

\begin{figure}[!htbp]
    \centering
    \includegraphics[width=.8\textwidth]{./assets/img/replicator_dynamics/main.pdf}
    \caption{Numerical simulation of the replicator equation
    (\ref{eqn:replicator_dynamics}): strategies are ordered by score, only the strategies with a high score survive the evolutionary process.}
    \label{fig:replicator_dynamics}
\end{figure}

This work can be used to classify plays of the IPD\@: data can be collected from
actual interactions (in lab or in the field). Furthermore, this allows for a
classification method similar to the notion of fingerprinting presented
in~\cite{Ashlock2008}. Trained strategies can potentially be classified as
extortionate or not or it could be possible to even constrain the reinforcement
learning approaches that are becoming prevalent in the literature.
Alternatively, this mathematical approach for recognising extortion could be
used in sophisticated strategies to defend against invasion. Arguably, some of
the strategies considered here exhibit this behaviour, indeed as described
in~\cite{Harper2017}, the top ranking strategies in the full tournament are
obtained using evolutionary reinforcement learning techniques, thus, suspicion
of extortionate behaviour could in fact be an evolutionary trait.

\section*{Acknowledgements}

The following open source software libraries were used in this research:

\begin{itemize}
    \item The Axelrod ~\cite{Knight2016, Knight2018} library (IPD strategies and
        tournaments).
    \item The sympy library~\cite{Meurer2017} (verification of all symbolic
        calculations).
    \item The matplotlib~\cite{Droettboom2018} library (visualisation).
    \item The pandas~\cite{Structures2010}, dask~\cite{Dask2016} and
        NumPy~\cite{Oliphant2015} libraries (data manipulation).
    \item The SciPy~\cite{Jones2001} library (numerical integration of the
        replicator equation).
\end{itemize}

This work was performed using the computational facilities of the Advanced
Research Computing @ Cardiff (ARCCA) Division, Cardiff University.

\printbibliography

\newpage
\section*{Supplementary materials}

\includepdf{assets/pdf/proof_of_form_of_extortionate_strategies/main.pdf}

\newpage

Using the pair wise interactions the transition rates \(p,
q\) can be measured and the steady state probabilities inferred and compared to
the actual probabilities of each state.
This is done numerically by computing the singular eigenvector of the
matrix \(A\) \cite{Stewart2009}:

\[
    A =
    \begin{bmatrix}
        p_1 q_1 & p_1 (1 - q_1) & (1 - p_1) q_1 & (1 -p_1) (1 - q_1) \\
        p_2 q_2 & p_2 (1 - q_2) & (1 - p_2) q_2 & (1 -p_2) (1 - q_2) \\
        p_3 q_3 & p_3 (1 - q_3) & (1 - p_3) q_3 & (1 -p_3) (1 - q_3) \\
        p_4 q_4 & p_4 (1 - q_4) & (1 - p_4) q_4 & (1 -p_4) (1 - q_4) \\
    \end{bmatrix}
\]

Figure~\ref{fig:computed_probabilities_vs_theoretic_probabilities} shows a
regression line fitted to every pairwise interaction with a reported
\(\text{SSError}\) value (pairwise interactions with missing states were
omitted). This serves to validate the approach: a part from some edge cases the
relationship is consistent.

\begin{figure}[!htbp]
    \centering
    \includegraphics[width=.8\textwidth]{./assets/img/computed_probabilities_vs_theoretic_probabilities/main.pdf}
    \caption{The
        relationship between the steady state probabilities inferred from the
        measured transitions and the actual steady state probabilities. A linear
        regression line is included validating the approach.}
    \label{fig:computed_probabilities_vs_theoretic_probabilities}
\end{figure}


\end{document}

strategies.

The results of this analysis are shown in
Figure~\ref{fig:SSError_and_probabilities_in_full}. The top ranking strategies
by number of wins seem to be extortionate (but not against all strategies) and
it can be seen that a small sub group of strategies achieve mutual defection.
All the top ranking strategies according to score achieve mutual cooperation and
do not extort each other, however they
\textbf{do} exhibit extortionate behaviour towards a number of the lower ranking
strategies.

\begin{figure}[!htbp]
    \centering
    \includegraphics[width=.8\textwidth]{./assets/img/SSError_and_probabilities_in_full/main.pdf}
    \caption{\(\text{SSError}\) for the strategies for the full tournament. Only
    strategy interactions for which \(p_4=0\) and \(\chi>1\) are displayed.}
    \label{fig:SSError_and_probabilities_in_full}
\end{figure}

\section{Conclusion}\label{sec:conclusion}

This work defines an approach to measure whether or not a player is playing a
strategy that corresponds to an extortionate strategy as defined
in~\cite{Press2012}: a mathematical model for suspicion. Indeed, all
extortionate strategies have been
 classified as lying on a triangular plane.
This rigorous classification fails to be robust to small measurement error, thus
a statistical approach is proposed.
This is done through a linear algebraic approach for approximating the solution
of a linear system. Using this, a large number of pairwise interactions is
simulated and in fact very few strategies are found to act extortionately.

The work of~\cite{Press2012}, whilst showing that a clever approach to taking
advantage of another memory one strategy exists: this is incomplete. Whilst the
elegance of this result is very attractive, just as the simplicity of the
victory of Tit For Tat in Axelrod's original tournaments was, it is incomplete.
Extortionate strategies achieve a high number of wins but they do not
achieve a high score which corresponds to the fitness landscape in an
evolutionary sense. From the large number of interactions a payoff matrix \(S\)
can be measured where \(S_{ij}\) denotes the score (using standard values of
\((R, S, T, P) = (3, 0, 5, 1)\)) of the \(i\)th strategy
against the \(j\)th strategy. Using this, the replicator equation
describes the evolution of the system based on a population density fitness
function:

\begin{equation}\label{eqn:replicator_dynamics}
    \frac{dx}{dt} = x(S-x^TS x)
\end{equation}

Equation (\ref{eqn:replicator_dynamics}) is solved numerically through an
integration technique described in~\cite{Petzold1983} and
Figure~\ref{fig:replicator_dynamics} shows the evolution of the distribution of
the system: the various strategies are ranked by scores. It is clear to see that
only the high ranking strategies survive the evolutionary process (in fact,
only \documentclass[a4paper]{article}

\usepackage{amsmath}
\usepackage{amssymb}
\usepackage[margin=1.5cm,
            includefoot,
            footskip=30pt]{geometry}
\usepackage{layout}
\usepackage{graphicx}
\usepackage{subcaption}

\usepackage{biblatex}
\usepackage{pdfpages}

\bibliography{main.bib}

\title{Suspicion: Recognising and evaluating the effectiveness
       of extortion in the Iterated Prisoner's Dilemma}
\author{Vincent A. Knight \and Nikoleta E. Glynatsi}
\date{\today}



\begin{document}

\maketitle

\begin{abstract}
    The Iterated Prisoner's Dilemma is a model for rational and evolutionary
    interactive behaviour. It has applications both in the study of human social
    behaviour as well as in biology.
    It is used to understand when and how a rational individual might
    accept an immediate cost to their own utility for the direct benefit of
    another.

    Much attention has been given to a class of strategies called
    Zero Determinant strategies. It has been theoretically shown that these
    strategies can ``extort'' any player.

    In this work, an approach to identify if observed strategies are playing in
    an extortionate way is described. Furthermore, experimental analysis of
    a large tournament with \input{assets/tex/number_of_full_strategies/main.tex}
    strategies is considered. In this setting
    the most highly performing strategies do not play in an extortionate way
    against each other but do against lower performing strategies.
    This suggests that whilst the theory of Zero Determinant strategies
    indicates that memory is not of fundamental importance to the evolution of
    cooperative behaviour, this is incomplete.
\end{abstract}

\section{Introduction}\label{sec:introduction}

Agent based game theoretic models have become a stalwart of the underpinning
mathematics of interactive behaviours. One of the major pieces of work
in this area is the pair of original computer tournaments run by Robert
Axelrod~\cite{Axelrod1980, Axelrod1980a}. These tournaments pitted submitted
computer strategies against each other in plays of the Iterated Prisoner's
Dilemma. A common game where agents can choose to pay a slight cost to their
immediate utility in the hope of building a reputation. This has been used in
economic and evolutionary game theory to understand the evolution of cooperative
behaviour.

Recently, a class of strategies was described in~\cite{Press2012} that can
provably extort any given opponent. In~\cite{Hilbe2013, Moran1707} some
questions have already been asked about the true effectiveness of these
strategies in an evolutionary setting. Here another question is asked: is it
possible to recognise this extortionate behaviour? A mathematical procedure for
suspicion is presented: in the same way that the continued actions of an
extortionate individual might raise suspicion.

This work makes use of the Axelrod Python library~\cite{Knight2018, Knight2016}
with a large number of Prisoner Dilemma strategies available to give an
extensive numerical example of the ideas presented.  The approach is presented
in Section~\ref{sec:delta-zd-strategies}.  All of the code and data discussed
in Section~\ref{sec:numerical-experiments} is open sourced, archived and
written according to best scientific principles~\cite{Wilson2014}. The data
archive can be found at~\cite{vincent_knight_2018_1297075}.

\section{Recognising Extortion}\label{sec:delta-zd-strategies}

In~\cite{Press2012}, given a match between 2 memory-one strategies, the concept
of Zero Determinant (ZD) strategies is introduced. The main result of that paper
shows that given two memory one players \(p, q\in\mathbb{R}^4\) a linear
relationship between the players' scores could be forced by one of the players.

Using the notation of~\cite{Press2012}, assuming the utilities for player \(p\)
are given by \(S_x=(R, S, T, P)\) and for player \(q\) by \(S_y=(R, T, S, P)\)
and that the stationary scores of each player is given by \(S_X\) and \(S_Y\)
respectively. The main result of~\cite{Press2012} is that if

\begin{equation}\label{eqn:linear_relationship_for_p}
    \tilde p=\alpha S_x + \beta S_y + \gamma
\end{equation}

or

\begin{equation}\label{eqn:linear_relationship_for_q}
    \tilde q=\alpha S_x + \beta S_y + \gamma
\end{equation}

where \(\tilde p = (1 - p_1, 1 - p_2, p_3, p_4)\) and
\(\tilde q = (1 - q_1, 1 - q_2, q_3, q_4)\) then:

\begin{equation}
    \alpha S_X + \beta S_Y + \gamma = 0
\end{equation}

In~\cite{Press2012} a particular type of ZD strategy is defined: extortionate
strategies. If:

\begin{equation}\label{eqn:constraint_for_extortion}
    \gamma = - P(\alpha + \beta)
\end{equation}

then the player can ensure they get a score \(\chi\) times
larger than the opponent. This extortion coefficient is given by:

\begin{equation}\label{eqn:definition_of_chi}
    \chi=\frac{-\beta}{\alpha}
\end{equation}

Thus, if (\ref{eqn:constraint_for_extortion}) holds and \(\chi >1\) a player is
said to extort their opponent.
Here, the reverse problem is considered: given a
\(p\in\mathbb{R}^4\) how does one identify \(\alpha, \beta\) if they
exist and is the strategy in fact acting in an extortionate way?

These conditions correspond to:

\begin{align}
    \tilde p_1 & = \alpha R + \beta R - P (\alpha + \beta)
            \label{eqn:condition_for_tilde_p1}\\
    \tilde p_2 & = \alpha S + \beta T - P (\alpha + \beta)
            \label{eqn:condition_for_tilde_p2}\\
    \tilde p_3 & = \alpha T + \beta S - P (\alpha + \beta)
            \label{eqn:condition_for_tilde_p3}\\
    \tilde p_4 & = \alpha P + \beta P - P (\alpha + \beta)
            \label{eqn:condition_for_tilde_p4}
\end{align}

Equation (\ref{eqn:condition_for_tilde_p4}) ensures that \(p_4=\tilde p_4=0\).
Equations (\ref{eqn:condition_for_tilde_p1}-\ref{eqn:condition_for_tilde_p3})
can be used to eliminate \(\alpha, \beta\), giving:

\begin{equation}\label{eqn:planar_definition_of_extortion}
    \tilde p_1 = \frac{(R - P)(\tilde p_2 + \tilde p_3)}{S + T - 2P}
\end{equation}

with:

\begin{equation}\label{eqn:definition_of_chi}
    \chi = \frac{\tilde p_2 (P - T) + \tilde p_3 (S - P)}
                {\tilde p_2 (P - S) + \tilde p_3 (T - P)}
\end{equation}

Given a strategy \(p\in\mathbb{R}^{4\times 1}\) equations
(\ref{eqn:condition_for_tilde_p4}), (\ref{eqn:planar_definition_of_extortion}-\ref{eqn:definition_of_chi}) can be used to check if
a strategy is extortionate. The conditions correspond to:

\begin{align}
    p_1 & = \frac{(R-P)(p_2 + p_3) - R + T + S - P}{S + T - 2P}
     \label{eqn:condition_for_p1}\\
    p_4 & = 0 \label{eqn:condition_for_p4}\\
    1 & > p_2 + p_3\label{eqn:condition_for_chi}
\end{align}

The algebraic steps necessary to prove these results are available in the
supporting materials.

All extortionate strategies reside on a triangular (\ref{eqn:condition_for_chi})
plane (\ref{eqn:condition_for_p1}) in 3 dimensions (\ref{eqn:condition_for_p4}).
Using this formulation it can be seen that a necessary (but not sufficient)
condition for an extortionate strategy is that it cooperates on average less
than 50\% of the time when in a state of disagreement with the opponent.

As an example, consider the known extortionate strategy \(p=(8 / 9, 1 / 2, 1 /
3, 0)\) from~\cite{Stewart2012} which is referred to as \texttt{Extort-2}. In
this case, for the standard values of \((R, T, S, P)\) constraint
(\ref{eqn:condition_for_p1}) corresponds to:

\begin{equation}
    p_1 = \frac{2(p_2 + p_3) + 1}{3}
\end{equation}

It is clear that in this case all constraints hold.

This approach could in fact be used to confirm that a given strategy is acting
in an extortionate manner even if it is not a memory one strategy. However, in
practice, if a closed form for \(p\) is not known, then due to measurement
and/or numerical error this would not work.

This problem can be written in the following linear algebraic form where
\(x=(\alpha, \beta)\)
and \(p^*=(\tilde p_1 - 1, tilde_2 - 1, p_3)\):

\begin{equation}\label{eqn:linear_algebraic_equation_for_p}
    Cx= p^*
\end{equation}

\(C\) corresponds to equations
(\ref{eqn:condition_for_tilde_p1}-\ref{eqn:condition_for_tilde_p3}) and is
given by:

\begin{equation}\label{eqn:definition_of_C}
    C =
    \begin{bmatrix}
        R - P & R- P \\
        S - P & T- P \\
        T - P & S- P \\
    \end{bmatrix}
\end{equation}

Note that in general, equation (\ref{eqn:linear_algebraic_equation_for_p}) will
not necessarily have a solution. From the Rouch\'{e}-Capelli theorem if there is
a solution it is unique as \(\text{rank}(C)=2\) which is the dimension of the
variable \(x\). The best fitting \(x\) is found by minimizing:

\begin{equation}\label{eqn:r_squared}
    \text{SSError} = \|C x- p^*\|_2^2 = \sum_{i=1}^{3}\left((C\bar x)_i-p_i^*\right)^2
\end{equation}

Note that \(\text{SSError}\), which is the square of the Frobenius
norm~\cite{Golub2013}, becomes a measure of how close a strategy is to being an
extortionate strategy. Suspicion
of extortion then corresponds to a threshold on \(\text{SSError}\).

By observing interactions (human or otherwise), their memory one representation
can be inferred and this approach can be used to recognise extortionate
behaviour. The notion of comparing theoretic and actual plays of the IPD is not
novel, see for example~\cite{Rand2013}. Immediately it is noted that if the
environment is noisy~\cite{Wu1995} then no strategy can be considered to be
extortionate as \(p_4>0\).

In the next section, this idea will be illustrated by observing the interactions
that take place in a computer based tournament of the IPD\@.

\section{Numerical experiments}\label{sec:numerical-experiments}

In~\cite{Stewart2012} results from a tournament with
\input{./assets/tex/number_of_stewart_plotkin_strategies/main.tex} strategies,
was presented with specific consideration given to ZD strategies. This
tournament is reproduced here using the Axelrod-Python
project~\cite{Knight2016}. To obtain a good measure of the corresponding
transition rates for each strategy all matches have been run for
\input{assets/tex/number_of_turns/main.tex} turns and every match has been
repeated \input{assets/tex/number_of_repetitions/main.tex} times. All of this
interaction data is available at~\cite{vincent_knight_2018_1297075}. A good
match between the inferred Markov chain and the state distribution of the actual
interactions has been verified. Data for this is presented in the supplementary
materials.

Figure~\ref{fig:SSError_overall_in_stewart_plotkin} shows the \(\text{SSError}\)
values for all the strategies in the tournament, as reported
in~\cite{Stewart2012} the extortionate strategy (which has an expected
\(\text{SSError}\) approximately 0) gains a large number of wins.

\begin{figure}[!htbp]
    \centering
    \includegraphics[width=.8\textwidth]{./assets/img/SSError_overall_in_stewart_plotkin/main.pdf}
    \caption{\(\text{SSError}\) and state probabilities for the strategies
        of~\cite{Stewart2012}, ordered both by number of wins and overall score.
        Note that \(P(DC)\) is not shown as it corresponds to the transpose of
        \(P(CD)\). Cooperator and Defector are omitted as they do not visit all
        the states.}
    \label{fig:SSError_overall_in_stewart_plotkin}
\end{figure}

Here, the work of~\cite{Stewart2012} is extended by investigating a tournament
with \input{assets/tex/number_of_full_strategies/main.tex}
strategies.

The results of this analysis are shown in
Figure~\ref{fig:SSError_and_probabilities_in_full}. The top ranking strategies
by number of wins seem to be extortionate (but not against all strategies) and
it can be seen that a small sub group of strategies achieve mutual defection.
All the top ranking strategies according to score achieve mutual cooperation and
do not extort each other, however they
\textbf{do} exhibit extortionate behaviour towards a number of the lower ranking
strategies.

\begin{figure}[!htbp]
    \centering
    \includegraphics[width=.8\textwidth]{./assets/img/SSError_and_probabilities_in_full/main.pdf}
    \caption{\(\text{SSError}\) for the strategies for the full tournament. Only
    strategy interactions for which \(p_4=0\) and \(\chi>1\) are displayed.}
    \label{fig:SSError_and_probabilities_in_full}
\end{figure}

\section{Conclusion}\label{sec:conclusion}

This work defines an approach to measure whether or not a player is playing a
strategy that corresponds to an extortionate strategy as defined
in~\cite{Press2012}: a mathematical model for suspicion. Indeed, all
extortionate strategies have been
 classified as lying on a triangular plane.
This rigorous classification fails to be robust to small measurement error, thus
a statistical approach is proposed.
This is done through a linear algebraic approach for approximating the solution
of a linear system. Using this, a large number of pairwise interactions is
simulated and in fact very few strategies are found to act extortionately.

The work of~\cite{Press2012}, whilst showing that a clever approach to taking
advantage of another memory one strategy exists: this is incomplete. Whilst the
elegance of this result is very attractive, just as the simplicity of the
victory of Tit For Tat in Axelrod's original tournaments was, it is incomplete.
Extortionate strategies achieve a high number of wins but they do not
achieve a high score which corresponds to the fitness landscape in an
evolutionary sense. From the large number of interactions a payoff matrix \(S\)
can be measured where \(S_{ij}\) denotes the score (using standard values of
\((R, S, T, P) = (3, 0, 5, 1)\)) of the \(i\)th strategy
against the \(j\)th strategy. Using this, the replicator equation
describes the evolution of the system based on a population density fitness
function:

\begin{equation}\label{eqn:replicator_dynamics}
    \frac{dx}{dt} = x(S-x^TS x)
\end{equation}

Equation (\ref{eqn:replicator_dynamics}) is solved numerically through an
integration technique described in~\cite{Petzold1983} and
Figure~\ref{fig:replicator_dynamics} shows the evolution of the distribution of
the system: the various strategies are ranked by scores. It is clear to see that
only the high ranking strategies survive the evolutionary process (in fact,
only \input{./assets/img/replicator_dynamics/main.tex}
have a final distribution greater than \(10 ^ {-2}\)). This confirms the
findings of~\cite{Moran1707} in which sophisticated strategies resist
evolutionary invasion of shorter memory strategies. Recalling
Figure~\ref{fig:SSError_and_probabilities_in_full} this demonstrates that:

\begin{itemize}
    \item Cooperation emerges through the evolutionary process: the high scoring
        strategies do not exhibit extortionate behaviour towards each other.
    \item Extortionate strategies do not survive the evolutionary process.
\end{itemize}

\begin{figure}[!htbp]
    \centering
    \includegraphics[width=.8\textwidth]{./assets/img/replicator_dynamics/main.pdf}
    \caption{Numerical simulation of the replicator equation
    (\ref{eqn:replicator_dynamics}): strategies are ordered by score, only the strategies with a high score survive the evolutionary process.}
    \label{fig:replicator_dynamics}
\end{figure}

This work can be used to classify plays of the IPD\@: data can be collected from
actual interactions (in lab or in the field). Furthermore, this allows for a
classification method similar to the notion of fingerprinting presented
in~\cite{Ashlock2008}. Trained strategies can potentially be classified as
extortionate or not or it could be possible to even constrain the reinforcement
learning approaches that are becoming prevalent in the literature.
Alternatively, this mathematical approach for recognising extortion could be
used in sophisticated strategies to defend against invasion. Arguably, some of
the strategies considered here exhibit this behaviour, indeed as described
in~\cite{Harper2017}, the top ranking strategies in the full tournament are
obtained using evolutionary reinforcement learning techniques, thus, suspicion
of extortionate behaviour could in fact be an evolutionary trait.

\section*{Acknowledgements}

The following open source software libraries were used in this research:

\begin{itemize}
    \item The Axelrod ~\cite{Knight2016, Knight2018} library (IPD strategies and
        tournaments).
    \item The sympy library~\cite{Meurer2017} (verification of all symbolic
        calculations).
    \item The matplotlib~\cite{Droettboom2018} library (visualisation).
    \item The pandas~\cite{Structures2010}, dask~\cite{Dask2016} and
        NumPy~\cite{Oliphant2015} libraries (data manipulation).
    \item The SciPy~\cite{Jones2001} library (numerical integration of the
        replicator equation).
\end{itemize}

This work was performed using the computational facilities of the Advanced
Research Computing @ Cardiff (ARCCA) Division, Cardiff University.

\printbibliography

\newpage
\section*{Supplementary materials}

\includepdf{assets/pdf/proof_of_form_of_extortionate_strategies/main.pdf}

\newpage

Using the pair wise interactions the transition rates \(p,
q\) can be measured and the steady state probabilities inferred and compared to
the actual probabilities of each state.
This is done numerically by computing the singular eigenvector of the
matrix \(A\) \cite{Stewart2009}:

\[
    A =
    \begin{bmatrix}
        p_1 q_1 & p_1 (1 - q_1) & (1 - p_1) q_1 & (1 -p_1) (1 - q_1) \\
        p_2 q_2 & p_2 (1 - q_2) & (1 - p_2) q_2 & (1 -p_2) (1 - q_2) \\
        p_3 q_3 & p_3 (1 - q_3) & (1 - p_3) q_3 & (1 -p_3) (1 - q_3) \\
        p_4 q_4 & p_4 (1 - q_4) & (1 - p_4) q_4 & (1 -p_4) (1 - q_4) \\
    \end{bmatrix}
\]

Figure~\ref{fig:computed_probabilities_vs_theoretic_probabilities} shows a
regression line fitted to every pairwise interaction with a reported
\(\text{SSError}\) value (pairwise interactions with missing states were
omitted). This serves to validate the approach: a part from some edge cases the
relationship is consistent.

\begin{figure}[!htbp]
    \centering
    \includegraphics[width=.8\textwidth]{./assets/img/computed_probabilities_vs_theoretic_probabilities/main.pdf}
    \caption{The
        relationship between the steady state probabilities inferred from the
        measured transitions and the actual steady state probabilities. A linear
        regression line is included validating the approach.}
    \label{fig:computed_probabilities_vs_theoretic_probabilities}
\end{figure}


\end{document}

have a final distribution greater than \(10 ^ {-2}\)). This confirms the
findings of~\cite{Moran1707} in which sophisticated strategies resist
evolutionary invasion of shorter memory strategies. Recalling
Figure~\ref{fig:SSError_and_probabilities_in_full} this demonstrates that:

\begin{itemize}
    \item Cooperation emerges through the evolutionary process: the high scoring
        strategies do not exhibit extortionate behaviour towards each other.
    \item Extortionate strategies do not survive the evolutionary process.
\end{itemize}

\begin{figure}[!htbp]
    \centering
    \includegraphics[width=.8\textwidth]{./assets/img/replicator_dynamics/main.pdf}
    \caption{Numerical simulation of the replicator equation
    (\ref{eqn:replicator_dynamics}): strategies are ordered by score, only the strategies with a high score survive the evolutionary process.}
    \label{fig:replicator_dynamics}
\end{figure}

This work can be used to classify plays of the IPD\@: data can be collected from
actual interactions (in lab or in the field). Furthermore, this allows for a
classification method similar to the notion of fingerprinting presented
in~\cite{Ashlock2008}. Trained strategies can potentially be classified as
extortionate or not or it could be possible to even constrain the reinforcement
learning approaches that are becoming prevalent in the literature.
Alternatively, this mathematical approach for recognising extortion could be
used in sophisticated strategies to defend against invasion. Arguably, some of
the strategies considered here exhibit this behaviour, indeed as described
in~\cite{Harper2017}, the top ranking strategies in the full tournament are
obtained using evolutionary reinforcement learning techniques, thus, suspicion
of extortionate behaviour could in fact be an evolutionary trait.

\section*{Acknowledgements}

The following open source software libraries were used in this research:

\begin{itemize}
    \item The Axelrod ~\cite{Knight2016, Knight2018} library (IPD strategies and
        tournaments).
    \item The sympy library~\cite{Meurer2017} (verification of all symbolic
        calculations).
    \item The matplotlib~\cite{Droettboom2018} library (visualisation).
    \item The pandas~\cite{Structures2010}, dask~\cite{Dask2016} and
        NumPy~\cite{Oliphant2015} libraries (data manipulation).
    \item The SciPy~\cite{Jones2001} library (numerical integration of the
        replicator equation).
\end{itemize}

This work was performed using the computational facilities of the Advanced
Research Computing @ Cardiff (ARCCA) Division, Cardiff University.

\printbibliography

\newpage
\section*{Supplementary materials}

\includepdf{assets/pdf/proof_of_form_of_extortionate_strategies/main.pdf}

\newpage

Using the pair wise interactions the transition rates \(p,
q\) can be measured and the steady state probabilities inferred and compared to
the actual probabilities of each state.
This is done numerically by computing the singular eigenvector of the
matrix \(A\) \cite{Stewart2009}:

\[
    A =
    \begin{bmatrix}
        p_1 q_1 & p_1 (1 - q_1) & (1 - p_1) q_1 & (1 -p_1) (1 - q_1) \\
        p_2 q_2 & p_2 (1 - q_2) & (1 - p_2) q_2 & (1 -p_2) (1 - q_2) \\
        p_3 q_3 & p_3 (1 - q_3) & (1 - p_3) q_3 & (1 -p_3) (1 - q_3) \\
        p_4 q_4 & p_4 (1 - q_4) & (1 - p_4) q_4 & (1 -p_4) (1 - q_4) \\
    \end{bmatrix}
\]

Figure~\ref{fig:computed_probabilities_vs_theoretic_probabilities} shows a
regression line fitted to every pairwise interaction with a reported
\(\text{SSError}\) value (pairwise interactions with missing states were
omitted). This serves to validate the approach: a part from some edge cases the
relationship is consistent.

\begin{figure}[!htbp]
    \centering
    \includegraphics[width=.8\textwidth]{./assets/img/computed_probabilities_vs_theoretic_probabilities/main.pdf}
    \caption{The
        relationship between the steady state probabilities inferred from the
        measured transitions and the actual steady state probabilities. A linear
        regression line is included validating the approach.}
    \label{fig:computed_probabilities_vs_theoretic_probabilities}
\end{figure}


\end{document}
 times. All of this
interaction data is available at~\cite{vincent_knight_2018_1297075}. A good
match between the inferred Markov chain and the state distribution of the actual
interactions has been verified. Data for this is presented in the supplementary
materials.

Figure~\ref{fig:SSError_overall_in_stewart_plotkin} shows the \(\text{SSError}\)
values for all the strategies in the tournament, as reported
in~\cite{Stewart2012} the extortionate strategy (which has an expected
\(\text{SSError}\) approximately 0) gains a large number of wins.

\begin{figure}[!htbp]
    \centering
    \includegraphics[width=.8\textwidth]{./assets/img/SSError_overall_in_stewart_plotkin/main.pdf}
    \caption{\(\text{SSError}\) and state probabilities for the strategies
        of~\cite{Stewart2012}, ordered both by number of wins and overall score.
        Note that \(P(DC)\) is not shown as it corresponds to the transpose of
        \(P(CD)\). Cooperator and Defector are omitted as they do not visit all
        the states.}
    \label{fig:SSError_overall_in_stewart_plotkin}
\end{figure}

Here, the work of~\cite{Stewart2012} is extended by investigating a tournament
with \documentclass[a4paper]{article}

\usepackage{amsmath}
\usepackage{amssymb}
\usepackage[margin=1.5cm,
            includefoot,
            footskip=30pt]{geometry}
\usepackage{layout}
\usepackage{graphicx}
\usepackage{subcaption}

\usepackage{biblatex}
\usepackage{pdfpages}

\bibliography{main.bib}

\title{Suspicion: Recognising and evaluating the effectiveness
       of extortion in the Iterated Prisoner's Dilemma}
\author{Vincent A. Knight \and Nikoleta E. Glynatsi}
\date{\today}



\begin{document}

\maketitle

\begin{abstract}
    The Iterated Prisoner's Dilemma is a model for rational and evolutionary
    interactive behaviour. It has applications both in the study of human social
    behaviour as well as in biology.
    It is used to understand when and how a rational individual might
    accept an immediate cost to their own utility for the direct benefit of
    another.

    Much attention has been given to a class of strategies called
    Zero Determinant strategies. It has been theoretically shown that these
    strategies can ``extort'' any player.

    In this work, an approach to identify if observed strategies are playing in
    an extortionate way is described. Furthermore, experimental analysis of
    a large tournament with \documentclass[a4paper]{article}

\usepackage{amsmath}
\usepackage{amssymb}
\usepackage[margin=1.5cm,
            includefoot,
            footskip=30pt]{geometry}
\usepackage{layout}
\usepackage{graphicx}
\usepackage{subcaption}

\usepackage{biblatex}
\usepackage{pdfpages}

\bibliography{main.bib}

\title{Suspicion: Recognising and evaluating the effectiveness
       of extortion in the Iterated Prisoner's Dilemma}
\author{Vincent A. Knight \and Nikoleta E. Glynatsi}
\date{\today}



\begin{document}

\maketitle

\begin{abstract}
    The Iterated Prisoner's Dilemma is a model for rational and evolutionary
    interactive behaviour. It has applications both in the study of human social
    behaviour as well as in biology.
    It is used to understand when and how a rational individual might
    accept an immediate cost to their own utility for the direct benefit of
    another.

    Much attention has been given to a class of strategies called
    Zero Determinant strategies. It has been theoretically shown that these
    strategies can ``extort'' any player.

    In this work, an approach to identify if observed strategies are playing in
    an extortionate way is described. Furthermore, experimental analysis of
    a large tournament with \input{assets/tex/number_of_full_strategies/main.tex}
    strategies is considered. In this setting
    the most highly performing strategies do not play in an extortionate way
    against each other but do against lower performing strategies.
    This suggests that whilst the theory of Zero Determinant strategies
    indicates that memory is not of fundamental importance to the evolution of
    cooperative behaviour, this is incomplete.
\end{abstract}

\section{Introduction}\label{sec:introduction}

Agent based game theoretic models have become a stalwart of the underpinning
mathematics of interactive behaviours. One of the major pieces of work
in this area is the pair of original computer tournaments run by Robert
Axelrod~\cite{Axelrod1980, Axelrod1980a}. These tournaments pitted submitted
computer strategies against each other in plays of the Iterated Prisoner's
Dilemma. A common game where agents can choose to pay a slight cost to their
immediate utility in the hope of building a reputation. This has been used in
economic and evolutionary game theory to understand the evolution of cooperative
behaviour.

Recently, a class of strategies was described in~\cite{Press2012} that can
provably extort any given opponent. In~\cite{Hilbe2013, Moran1707} some
questions have already been asked about the true effectiveness of these
strategies in an evolutionary setting. Here another question is asked: is it
possible to recognise this extortionate behaviour? A mathematical procedure for
suspicion is presented: in the same way that the continued actions of an
extortionate individual might raise suspicion.

This work makes use of the Axelrod Python library~\cite{Knight2018, Knight2016}
with a large number of Prisoner Dilemma strategies available to give an
extensive numerical example of the ideas presented.  The approach is presented
in Section~\ref{sec:delta-zd-strategies}.  All of the code and data discussed
in Section~\ref{sec:numerical-experiments} is open sourced, archived and
written according to best scientific principles~\cite{Wilson2014}. The data
archive can be found at~\cite{vincent_knight_2018_1297075}.

\section{Recognising Extortion}\label{sec:delta-zd-strategies}

In~\cite{Press2012}, given a match between 2 memory-one strategies, the concept
of Zero Determinant (ZD) strategies is introduced. The main result of that paper
shows that given two memory one players \(p, q\in\mathbb{R}^4\) a linear
relationship between the players' scores could be forced by one of the players.

Using the notation of~\cite{Press2012}, assuming the utilities for player \(p\)
are given by \(S_x=(R, S, T, P)\) and for player \(q\) by \(S_y=(R, T, S, P)\)
and that the stationary scores of each player is given by \(S_X\) and \(S_Y\)
respectively. The main result of~\cite{Press2012} is that if

\begin{equation}\label{eqn:linear_relationship_for_p}
    \tilde p=\alpha S_x + \beta S_y + \gamma
\end{equation}

or

\begin{equation}\label{eqn:linear_relationship_for_q}
    \tilde q=\alpha S_x + \beta S_y + \gamma
\end{equation}

where \(\tilde p = (1 - p_1, 1 - p_2, p_3, p_4)\) and
\(\tilde q = (1 - q_1, 1 - q_2, q_3, q_4)\) then:

\begin{equation}
    \alpha S_X + \beta S_Y + \gamma = 0
\end{equation}

In~\cite{Press2012} a particular type of ZD strategy is defined: extortionate
strategies. If:

\begin{equation}\label{eqn:constraint_for_extortion}
    \gamma = - P(\alpha + \beta)
\end{equation}

then the player can ensure they get a score \(\chi\) times
larger than the opponent. This extortion coefficient is given by:

\begin{equation}\label{eqn:definition_of_chi}
    \chi=\frac{-\beta}{\alpha}
\end{equation}

Thus, if (\ref{eqn:constraint_for_extortion}) holds and \(\chi >1\) a player is
said to extort their opponent.
Here, the reverse problem is considered: given a
\(p\in\mathbb{R}^4\) how does one identify \(\alpha, \beta\) if they
exist and is the strategy in fact acting in an extortionate way?

These conditions correspond to:

\begin{align}
    \tilde p_1 & = \alpha R + \beta R - P (\alpha + \beta)
            \label{eqn:condition_for_tilde_p1}\\
    \tilde p_2 & = \alpha S + \beta T - P (\alpha + \beta)
            \label{eqn:condition_for_tilde_p2}\\
    \tilde p_3 & = \alpha T + \beta S - P (\alpha + \beta)
            \label{eqn:condition_for_tilde_p3}\\
    \tilde p_4 & = \alpha P + \beta P - P (\alpha + \beta)
            \label{eqn:condition_for_tilde_p4}
\end{align}

Equation (\ref{eqn:condition_for_tilde_p4}) ensures that \(p_4=\tilde p_4=0\).
Equations (\ref{eqn:condition_for_tilde_p1}-\ref{eqn:condition_for_tilde_p3})
can be used to eliminate \(\alpha, \beta\), giving:

\begin{equation}\label{eqn:planar_definition_of_extortion}
    \tilde p_1 = \frac{(R - P)(\tilde p_2 + \tilde p_3)}{S + T - 2P}
\end{equation}

with:

\begin{equation}\label{eqn:definition_of_chi}
    \chi = \frac{\tilde p_2 (P - T) + \tilde p_3 (S - P)}
                {\tilde p_2 (P - S) + \tilde p_3 (T - P)}
\end{equation}

Given a strategy \(p\in\mathbb{R}^{4\times 1}\) equations
(\ref{eqn:condition_for_tilde_p4}), (\ref{eqn:planar_definition_of_extortion}-\ref{eqn:definition_of_chi}) can be used to check if
a strategy is extortionate. The conditions correspond to:

\begin{align}
    p_1 & = \frac{(R-P)(p_2 + p_3) - R + T + S - P}{S + T - 2P}
     \label{eqn:condition_for_p1}\\
    p_4 & = 0 \label{eqn:condition_for_p4}\\
    1 & > p_2 + p_3\label{eqn:condition_for_chi}
\end{align}

The algebraic steps necessary to prove these results are available in the
supporting materials.

All extortionate strategies reside on a triangular (\ref{eqn:condition_for_chi})
plane (\ref{eqn:condition_for_p1}) in 3 dimensions (\ref{eqn:condition_for_p4}).
Using this formulation it can be seen that a necessary (but not sufficient)
condition for an extortionate strategy is that it cooperates on average less
than 50\% of the time when in a state of disagreement with the opponent.

As an example, consider the known extortionate strategy \(p=(8 / 9, 1 / 2, 1 /
3, 0)\) from~\cite{Stewart2012} which is referred to as \texttt{Extort-2}. In
this case, for the standard values of \((R, T, S, P)\) constraint
(\ref{eqn:condition_for_p1}) corresponds to:

\begin{equation}
    p_1 = \frac{2(p_2 + p_3) + 1}{3}
\end{equation}

It is clear that in this case all constraints hold.

This approach could in fact be used to confirm that a given strategy is acting
in an extortionate manner even if it is not a memory one strategy. However, in
practice, if a closed form for \(p\) is not known, then due to measurement
and/or numerical error this would not work.

This problem can be written in the following linear algebraic form where
\(x=(\alpha, \beta)\)
and \(p^*=(\tilde p_1 - 1, tilde_2 - 1, p_3)\):

\begin{equation}\label{eqn:linear_algebraic_equation_for_p}
    Cx= p^*
\end{equation}

\(C\) corresponds to equations
(\ref{eqn:condition_for_tilde_p1}-\ref{eqn:condition_for_tilde_p3}) and is
given by:

\begin{equation}\label{eqn:definition_of_C}
    C =
    \begin{bmatrix}
        R - P & R- P \\
        S - P & T- P \\
        T - P & S- P \\
    \end{bmatrix}
\end{equation}

Note that in general, equation (\ref{eqn:linear_algebraic_equation_for_p}) will
not necessarily have a solution. From the Rouch\'{e}-Capelli theorem if there is
a solution it is unique as \(\text{rank}(C)=2\) which is the dimension of the
variable \(x\). The best fitting \(x\) is found by minimizing:

\begin{equation}\label{eqn:r_squared}
    \text{SSError} = \|C x- p^*\|_2^2 = \sum_{i=1}^{3}\left((C\bar x)_i-p_i^*\right)^2
\end{equation}

Note that \(\text{SSError}\), which is the square of the Frobenius
norm~\cite{Golub2013}, becomes a measure of how close a strategy is to being an
extortionate strategy. Suspicion
of extortion then corresponds to a threshold on \(\text{SSError}\).

By observing interactions (human or otherwise), their memory one representation
can be inferred and this approach can be used to recognise extortionate
behaviour. The notion of comparing theoretic and actual plays of the IPD is not
novel, see for example~\cite{Rand2013}. Immediately it is noted that if the
environment is noisy~\cite{Wu1995} then no strategy can be considered to be
extortionate as \(p_4>0\).

In the next section, this idea will be illustrated by observing the interactions
that take place in a computer based tournament of the IPD\@.

\section{Numerical experiments}\label{sec:numerical-experiments}

In~\cite{Stewart2012} results from a tournament with
\input{./assets/tex/number_of_stewart_plotkin_strategies/main.tex} strategies,
was presented with specific consideration given to ZD strategies. This
tournament is reproduced here using the Axelrod-Python
project~\cite{Knight2016}. To obtain a good measure of the corresponding
transition rates for each strategy all matches have been run for
\input{assets/tex/number_of_turns/main.tex} turns and every match has been
repeated \input{assets/tex/number_of_repetitions/main.tex} times. All of this
interaction data is available at~\cite{vincent_knight_2018_1297075}. A good
match between the inferred Markov chain and the state distribution of the actual
interactions has been verified. Data for this is presented in the supplementary
materials.

Figure~\ref{fig:SSError_overall_in_stewart_plotkin} shows the \(\text{SSError}\)
values for all the strategies in the tournament, as reported
in~\cite{Stewart2012} the extortionate strategy (which has an expected
\(\text{SSError}\) approximately 0) gains a large number of wins.

\begin{figure}[!htbp]
    \centering
    \includegraphics[width=.8\textwidth]{./assets/img/SSError_overall_in_stewart_plotkin/main.pdf}
    \caption{\(\text{SSError}\) and state probabilities for the strategies
        of~\cite{Stewart2012}, ordered both by number of wins and overall score.
        Note that \(P(DC)\) is not shown as it corresponds to the transpose of
        \(P(CD)\). Cooperator and Defector are omitted as they do not visit all
        the states.}
    \label{fig:SSError_overall_in_stewart_plotkin}
\end{figure}

Here, the work of~\cite{Stewart2012} is extended by investigating a tournament
with \input{assets/tex/number_of_full_strategies/main.tex}
strategies.

The results of this analysis are shown in
Figure~\ref{fig:SSError_and_probabilities_in_full}. The top ranking strategies
by number of wins seem to be extortionate (but not against all strategies) and
it can be seen that a small sub group of strategies achieve mutual defection.
All the top ranking strategies according to score achieve mutual cooperation and
do not extort each other, however they
\textbf{do} exhibit extortionate behaviour towards a number of the lower ranking
strategies.

\begin{figure}[!htbp]
    \centering
    \includegraphics[width=.8\textwidth]{./assets/img/SSError_and_probabilities_in_full/main.pdf}
    \caption{\(\text{SSError}\) for the strategies for the full tournament. Only
    strategy interactions for which \(p_4=0\) and \(\chi>1\) are displayed.}
    \label{fig:SSError_and_probabilities_in_full}
\end{figure}

\section{Conclusion}\label{sec:conclusion}

This work defines an approach to measure whether or not a player is playing a
strategy that corresponds to an extortionate strategy as defined
in~\cite{Press2012}: a mathematical model for suspicion. Indeed, all
extortionate strategies have been
 classified as lying on a triangular plane.
This rigorous classification fails to be robust to small measurement error, thus
a statistical approach is proposed.
This is done through a linear algebraic approach for approximating the solution
of a linear system. Using this, a large number of pairwise interactions is
simulated and in fact very few strategies are found to act extortionately.

The work of~\cite{Press2012}, whilst showing that a clever approach to taking
advantage of another memory one strategy exists: this is incomplete. Whilst the
elegance of this result is very attractive, just as the simplicity of the
victory of Tit For Tat in Axelrod's original tournaments was, it is incomplete.
Extortionate strategies achieve a high number of wins but they do not
achieve a high score which corresponds to the fitness landscape in an
evolutionary sense. From the large number of interactions a payoff matrix \(S\)
can be measured where \(S_{ij}\) denotes the score (using standard values of
\((R, S, T, P) = (3, 0, 5, 1)\)) of the \(i\)th strategy
against the \(j\)th strategy. Using this, the replicator equation
describes the evolution of the system based on a population density fitness
function:

\begin{equation}\label{eqn:replicator_dynamics}
    \frac{dx}{dt} = x(S-x^TS x)
\end{equation}

Equation (\ref{eqn:replicator_dynamics}) is solved numerically through an
integration technique described in~\cite{Petzold1983} and
Figure~\ref{fig:replicator_dynamics} shows the evolution of the distribution of
the system: the various strategies are ranked by scores. It is clear to see that
only the high ranking strategies survive the evolutionary process (in fact,
only \input{./assets/img/replicator_dynamics/main.tex}
have a final distribution greater than \(10 ^ {-2}\)). This confirms the
findings of~\cite{Moran1707} in which sophisticated strategies resist
evolutionary invasion of shorter memory strategies. Recalling
Figure~\ref{fig:SSError_and_probabilities_in_full} this demonstrates that:

\begin{itemize}
    \item Cooperation emerges through the evolutionary process: the high scoring
        strategies do not exhibit extortionate behaviour towards each other.
    \item Extortionate strategies do not survive the evolutionary process.
\end{itemize}

\begin{figure}[!htbp]
    \centering
    \includegraphics[width=.8\textwidth]{./assets/img/replicator_dynamics/main.pdf}
    \caption{Numerical simulation of the replicator equation
    (\ref{eqn:replicator_dynamics}): strategies are ordered by score, only the strategies with a high score survive the evolutionary process.}
    \label{fig:replicator_dynamics}
\end{figure}

This work can be used to classify plays of the IPD\@: data can be collected from
actual interactions (in lab or in the field). Furthermore, this allows for a
classification method similar to the notion of fingerprinting presented
in~\cite{Ashlock2008}. Trained strategies can potentially be classified as
extortionate or not or it could be possible to even constrain the reinforcement
learning approaches that are becoming prevalent in the literature.
Alternatively, this mathematical approach for recognising extortion could be
used in sophisticated strategies to defend against invasion. Arguably, some of
the strategies considered here exhibit this behaviour, indeed as described
in~\cite{Harper2017}, the top ranking strategies in the full tournament are
obtained using evolutionary reinforcement learning techniques, thus, suspicion
of extortionate behaviour could in fact be an evolutionary trait.

\section*{Acknowledgements}

The following open source software libraries were used in this research:

\begin{itemize}
    \item The Axelrod ~\cite{Knight2016, Knight2018} library (IPD strategies and
        tournaments).
    \item The sympy library~\cite{Meurer2017} (verification of all symbolic
        calculations).
    \item The matplotlib~\cite{Droettboom2018} library (visualisation).
    \item The pandas~\cite{Structures2010}, dask~\cite{Dask2016} and
        NumPy~\cite{Oliphant2015} libraries (data manipulation).
    \item The SciPy~\cite{Jones2001} library (numerical integration of the
        replicator equation).
\end{itemize}

This work was performed using the computational facilities of the Advanced
Research Computing @ Cardiff (ARCCA) Division, Cardiff University.

\printbibliography

\newpage
\section*{Supplementary materials}

\includepdf{assets/pdf/proof_of_form_of_extortionate_strategies/main.pdf}

\newpage

Using the pair wise interactions the transition rates \(p,
q\) can be measured and the steady state probabilities inferred and compared to
the actual probabilities of each state.
This is done numerically by computing the singular eigenvector of the
matrix \(A\) \cite{Stewart2009}:

\[
    A =
    \begin{bmatrix}
        p_1 q_1 & p_1 (1 - q_1) & (1 - p_1) q_1 & (1 -p_1) (1 - q_1) \\
        p_2 q_2 & p_2 (1 - q_2) & (1 - p_2) q_2 & (1 -p_2) (1 - q_2) \\
        p_3 q_3 & p_3 (1 - q_3) & (1 - p_3) q_3 & (1 -p_3) (1 - q_3) \\
        p_4 q_4 & p_4 (1 - q_4) & (1 - p_4) q_4 & (1 -p_4) (1 - q_4) \\
    \end{bmatrix}
\]

Figure~\ref{fig:computed_probabilities_vs_theoretic_probabilities} shows a
regression line fitted to every pairwise interaction with a reported
\(\text{SSError}\) value (pairwise interactions with missing states were
omitted). This serves to validate the approach: a part from some edge cases the
relationship is consistent.

\begin{figure}[!htbp]
    \centering
    \includegraphics[width=.8\textwidth]{./assets/img/computed_probabilities_vs_theoretic_probabilities/main.pdf}
    \caption{The
        relationship between the steady state probabilities inferred from the
        measured transitions and the actual steady state probabilities. A linear
        regression line is included validating the approach.}
    \label{fig:computed_probabilities_vs_theoretic_probabilities}
\end{figure}


\end{document}

    strategies is considered. In this setting
    the most highly performing strategies do not play in an extortionate way
    against each other but do against lower performing strategies.
    This suggests that whilst the theory of Zero Determinant strategies
    indicates that memory is not of fundamental importance to the evolution of
    cooperative behaviour, this is incomplete.
\end{abstract}

\section{Introduction}\label{sec:introduction}

Agent based game theoretic models have become a stalwart of the underpinning
mathematics of interactive behaviours. One of the major pieces of work
in this area is the pair of original computer tournaments run by Robert
Axelrod~\cite{Axelrod1980, Axelrod1980a}. These tournaments pitted submitted
computer strategies against each other in plays of the Iterated Prisoner's
Dilemma. A common game where agents can choose to pay a slight cost to their
immediate utility in the hope of building a reputation. This has been used in
economic and evolutionary game theory to understand the evolution of cooperative
behaviour.

Recently, a class of strategies was described in~\cite{Press2012} that can
provably extort any given opponent. In~\cite{Hilbe2013, Moran1707} some
questions have already been asked about the true effectiveness of these
strategies in an evolutionary setting. Here another question is asked: is it
possible to recognise this extortionate behaviour? A mathematical procedure for
suspicion is presented: in the same way that the continued actions of an
extortionate individual might raise suspicion.

This work makes use of the Axelrod Python library~\cite{Knight2018, Knight2016}
with a large number of Prisoner Dilemma strategies available to give an
extensive numerical example of the ideas presented.  The approach is presented
in Section~\ref{sec:delta-zd-strategies}.  All of the code and data discussed
in Section~\ref{sec:numerical-experiments} is open sourced, archived and
written according to best scientific principles~\cite{Wilson2014}. The data
archive can be found at~\cite{vincent_knight_2018_1297075}.

\section{Recognising Extortion}\label{sec:delta-zd-strategies}

In~\cite{Press2012}, given a match between 2 memory-one strategies, the concept
of Zero Determinant (ZD) strategies is introduced. The main result of that paper
shows that given two memory one players \(p, q\in\mathbb{R}^4\) a linear
relationship between the players' scores could be forced by one of the players.

Using the notation of~\cite{Press2012}, assuming the utilities for player \(p\)
are given by \(S_x=(R, S, T, P)\) and for player \(q\) by \(S_y=(R, T, S, P)\)
and that the stationary scores of each player is given by \(S_X\) and \(S_Y\)
respectively. The main result of~\cite{Press2012} is that if

\begin{equation}\label{eqn:linear_relationship_for_p}
    \tilde p=\alpha S_x + \beta S_y + \gamma
\end{equation}

or

\begin{equation}\label{eqn:linear_relationship_for_q}
    \tilde q=\alpha S_x + \beta S_y + \gamma
\end{equation}

where \(\tilde p = (1 - p_1, 1 - p_2, p_3, p_4)\) and
\(\tilde q = (1 - q_1, 1 - q_2, q_3, q_4)\) then:

\begin{equation}
    \alpha S_X + \beta S_Y + \gamma = 0
\end{equation}

In~\cite{Press2012} a particular type of ZD strategy is defined: extortionate
strategies. If:

\begin{equation}\label{eqn:constraint_for_extortion}
    \gamma = - P(\alpha + \beta)
\end{equation}

then the player can ensure they get a score \(\chi\) times
larger than the opponent. This extortion coefficient is given by:

\begin{equation}\label{eqn:definition_of_chi}
    \chi=\frac{-\beta}{\alpha}
\end{equation}

Thus, if (\ref{eqn:constraint_for_extortion}) holds and \(\chi >1\) a player is
said to extort their opponent.
Here, the reverse problem is considered: given a
\(p\in\mathbb{R}^4\) how does one identify \(\alpha, \beta\) if they
exist and is the strategy in fact acting in an extortionate way?

These conditions correspond to:

\begin{align}
    \tilde p_1 & = \alpha R + \beta R - P (\alpha + \beta)
            \label{eqn:condition_for_tilde_p1}\\
    \tilde p_2 & = \alpha S + \beta T - P (\alpha + \beta)
            \label{eqn:condition_for_tilde_p2}\\
    \tilde p_3 & = \alpha T + \beta S - P (\alpha + \beta)
            \label{eqn:condition_for_tilde_p3}\\
    \tilde p_4 & = \alpha P + \beta P - P (\alpha + \beta)
            \label{eqn:condition_for_tilde_p4}
\end{align}

Equation (\ref{eqn:condition_for_tilde_p4}) ensures that \(p_4=\tilde p_4=0\).
Equations (\ref{eqn:condition_for_tilde_p1}-\ref{eqn:condition_for_tilde_p3})
can be used to eliminate \(\alpha, \beta\), giving:

\begin{equation}\label{eqn:planar_definition_of_extortion}
    \tilde p_1 = \frac{(R - P)(\tilde p_2 + \tilde p_3)}{S + T - 2P}
\end{equation}

with:

\begin{equation}\label{eqn:definition_of_chi}
    \chi = \frac{\tilde p_2 (P - T) + \tilde p_3 (S - P)}
                {\tilde p_2 (P - S) + \tilde p_3 (T - P)}
\end{equation}

Given a strategy \(p\in\mathbb{R}^{4\times 1}\) equations
(\ref{eqn:condition_for_tilde_p4}), (\ref{eqn:planar_definition_of_extortion}-\ref{eqn:definition_of_chi}) can be used to check if
a strategy is extortionate. The conditions correspond to:

\begin{align}
    p_1 & = \frac{(R-P)(p_2 + p_3) - R + T + S - P}{S + T - 2P}
     \label{eqn:condition_for_p1}\\
    p_4 & = 0 \label{eqn:condition_for_p4}\\
    1 & > p_2 + p_3\label{eqn:condition_for_chi}
\end{align}

The algebraic steps necessary to prove these results are available in the
supporting materials.

All extortionate strategies reside on a triangular (\ref{eqn:condition_for_chi})
plane (\ref{eqn:condition_for_p1}) in 3 dimensions (\ref{eqn:condition_for_p4}).
Using this formulation it can be seen that a necessary (but not sufficient)
condition for an extortionate strategy is that it cooperates on average less
than 50\% of the time when in a state of disagreement with the opponent.

As an example, consider the known extortionate strategy \(p=(8 / 9, 1 / 2, 1 /
3, 0)\) from~\cite{Stewart2012} which is referred to as \texttt{Extort-2}. In
this case, for the standard values of \((R, T, S, P)\) constraint
(\ref{eqn:condition_for_p1}) corresponds to:

\begin{equation}
    p_1 = \frac{2(p_2 + p_3) + 1}{3}
\end{equation}

It is clear that in this case all constraints hold.

This approach could in fact be used to confirm that a given strategy is acting
in an extortionate manner even if it is not a memory one strategy. However, in
practice, if a closed form for \(p\) is not known, then due to measurement
and/or numerical error this would not work.

This problem can be written in the following linear algebraic form where
\(x=(\alpha, \beta)\)
and \(p^*=(\tilde p_1 - 1, tilde_2 - 1, p_3)\):

\begin{equation}\label{eqn:linear_algebraic_equation_for_p}
    Cx= p^*
\end{equation}

\(C\) corresponds to equations
(\ref{eqn:condition_for_tilde_p1}-\ref{eqn:condition_for_tilde_p3}) and is
given by:

\begin{equation}\label{eqn:definition_of_C}
    C =
    \begin{bmatrix}
        R - P & R- P \\
        S - P & T- P \\
        T - P & S- P \\
    \end{bmatrix}
\end{equation}

Note that in general, equation (\ref{eqn:linear_algebraic_equation_for_p}) will
not necessarily have a solution. From the Rouch\'{e}-Capelli theorem if there is
a solution it is unique as \(\text{rank}(C)=2\) which is the dimension of the
variable \(x\). The best fitting \(x\) is found by minimizing:

\begin{equation}\label{eqn:r_squared}
    \text{SSError} = \|C x- p^*\|_2^2 = \sum_{i=1}^{3}\left((C\bar x)_i-p_i^*\right)^2
\end{equation}

Note that \(\text{SSError}\), which is the square of the Frobenius
norm~\cite{Golub2013}, becomes a measure of how close a strategy is to being an
extortionate strategy. Suspicion
of extortion then corresponds to a threshold on \(\text{SSError}\).

By observing interactions (human or otherwise), their memory one representation
can be inferred and this approach can be used to recognise extortionate
behaviour. The notion of comparing theoretic and actual plays of the IPD is not
novel, see for example~\cite{Rand2013}. Immediately it is noted that if the
environment is noisy~\cite{Wu1995} then no strategy can be considered to be
extortionate as \(p_4>0\).

In the next section, this idea will be illustrated by observing the interactions
that take place in a computer based tournament of the IPD\@.

\section{Numerical experiments}\label{sec:numerical-experiments}

In~\cite{Stewart2012} results from a tournament with
\documentclass[a4paper]{article}

\usepackage{amsmath}
\usepackage{amssymb}
\usepackage[margin=1.5cm,
            includefoot,
            footskip=30pt]{geometry}
\usepackage{layout}
\usepackage{graphicx}
\usepackage{subcaption}

\usepackage{biblatex}
\usepackage{pdfpages}

\bibliography{main.bib}

\title{Suspicion: Recognising and evaluating the effectiveness
       of extortion in the Iterated Prisoner's Dilemma}
\author{Vincent A. Knight \and Nikoleta E. Glynatsi}
\date{\today}



\begin{document}

\maketitle

\begin{abstract}
    The Iterated Prisoner's Dilemma is a model for rational and evolutionary
    interactive behaviour. It has applications both in the study of human social
    behaviour as well as in biology.
    It is used to understand when and how a rational individual might
    accept an immediate cost to their own utility for the direct benefit of
    another.

    Much attention has been given to a class of strategies called
    Zero Determinant strategies. It has been theoretically shown that these
    strategies can ``extort'' any player.

    In this work, an approach to identify if observed strategies are playing in
    an extortionate way is described. Furthermore, experimental analysis of
    a large tournament with \input{assets/tex/number_of_full_strategies/main.tex}
    strategies is considered. In this setting
    the most highly performing strategies do not play in an extortionate way
    against each other but do against lower performing strategies.
    This suggests that whilst the theory of Zero Determinant strategies
    indicates that memory is not of fundamental importance to the evolution of
    cooperative behaviour, this is incomplete.
\end{abstract}

\section{Introduction}\label{sec:introduction}

Agent based game theoretic models have become a stalwart of the underpinning
mathematics of interactive behaviours. One of the major pieces of work
in this area is the pair of original computer tournaments run by Robert
Axelrod~\cite{Axelrod1980, Axelrod1980a}. These tournaments pitted submitted
computer strategies against each other in plays of the Iterated Prisoner's
Dilemma. A common game where agents can choose to pay a slight cost to their
immediate utility in the hope of building a reputation. This has been used in
economic and evolutionary game theory to understand the evolution of cooperative
behaviour.

Recently, a class of strategies was described in~\cite{Press2012} that can
provably extort any given opponent. In~\cite{Hilbe2013, Moran1707} some
questions have already been asked about the true effectiveness of these
strategies in an evolutionary setting. Here another question is asked: is it
possible to recognise this extortionate behaviour? A mathematical procedure for
suspicion is presented: in the same way that the continued actions of an
extortionate individual might raise suspicion.

This work makes use of the Axelrod Python library~\cite{Knight2018, Knight2016}
with a large number of Prisoner Dilemma strategies available to give an
extensive numerical example of the ideas presented.  The approach is presented
in Section~\ref{sec:delta-zd-strategies}.  All of the code and data discussed
in Section~\ref{sec:numerical-experiments} is open sourced, archived and
written according to best scientific principles~\cite{Wilson2014}. The data
archive can be found at~\cite{vincent_knight_2018_1297075}.

\section{Recognising Extortion}\label{sec:delta-zd-strategies}

In~\cite{Press2012}, given a match between 2 memory-one strategies, the concept
of Zero Determinant (ZD) strategies is introduced. The main result of that paper
shows that given two memory one players \(p, q\in\mathbb{R}^4\) a linear
relationship between the players' scores could be forced by one of the players.

Using the notation of~\cite{Press2012}, assuming the utilities for player \(p\)
are given by \(S_x=(R, S, T, P)\) and for player \(q\) by \(S_y=(R, T, S, P)\)
and that the stationary scores of each player is given by \(S_X\) and \(S_Y\)
respectively. The main result of~\cite{Press2012} is that if

\begin{equation}\label{eqn:linear_relationship_for_p}
    \tilde p=\alpha S_x + \beta S_y + \gamma
\end{equation}

or

\begin{equation}\label{eqn:linear_relationship_for_q}
    \tilde q=\alpha S_x + \beta S_y + \gamma
\end{equation}

where \(\tilde p = (1 - p_1, 1 - p_2, p_3, p_4)\) and
\(\tilde q = (1 - q_1, 1 - q_2, q_3, q_4)\) then:

\begin{equation}
    \alpha S_X + \beta S_Y + \gamma = 0
\end{equation}

In~\cite{Press2012} a particular type of ZD strategy is defined: extortionate
strategies. If:

\begin{equation}\label{eqn:constraint_for_extortion}
    \gamma = - P(\alpha + \beta)
\end{equation}

then the player can ensure they get a score \(\chi\) times
larger than the opponent. This extortion coefficient is given by:

\begin{equation}\label{eqn:definition_of_chi}
    \chi=\frac{-\beta}{\alpha}
\end{equation}

Thus, if (\ref{eqn:constraint_for_extortion}) holds and \(\chi >1\) a player is
said to extort their opponent.
Here, the reverse problem is considered: given a
\(p\in\mathbb{R}^4\) how does one identify \(\alpha, \beta\) if they
exist and is the strategy in fact acting in an extortionate way?

These conditions correspond to:

\begin{align}
    \tilde p_1 & = \alpha R + \beta R - P (\alpha + \beta)
            \label{eqn:condition_for_tilde_p1}\\
    \tilde p_2 & = \alpha S + \beta T - P (\alpha + \beta)
            \label{eqn:condition_for_tilde_p2}\\
    \tilde p_3 & = \alpha T + \beta S - P (\alpha + \beta)
            \label{eqn:condition_for_tilde_p3}\\
    \tilde p_4 & = \alpha P + \beta P - P (\alpha + \beta)
            \label{eqn:condition_for_tilde_p4}
\end{align}

Equation (\ref{eqn:condition_for_tilde_p4}) ensures that \(p_4=\tilde p_4=0\).
Equations (\ref{eqn:condition_for_tilde_p1}-\ref{eqn:condition_for_tilde_p3})
can be used to eliminate \(\alpha, \beta\), giving:

\begin{equation}\label{eqn:planar_definition_of_extortion}
    \tilde p_1 = \frac{(R - P)(\tilde p_2 + \tilde p_3)}{S + T - 2P}
\end{equation}

with:

\begin{equation}\label{eqn:definition_of_chi}
    \chi = \frac{\tilde p_2 (P - T) + \tilde p_3 (S - P)}
                {\tilde p_2 (P - S) + \tilde p_3 (T - P)}
\end{equation}

Given a strategy \(p\in\mathbb{R}^{4\times 1}\) equations
(\ref{eqn:condition_for_tilde_p4}), (\ref{eqn:planar_definition_of_extortion}-\ref{eqn:definition_of_chi}) can be used to check if
a strategy is extortionate. The conditions correspond to:

\begin{align}
    p_1 & = \frac{(R-P)(p_2 + p_3) - R + T + S - P}{S + T - 2P}
     \label{eqn:condition_for_p1}\\
    p_4 & = 0 \label{eqn:condition_for_p4}\\
    1 & > p_2 + p_3\label{eqn:condition_for_chi}
\end{align}

The algebraic steps necessary to prove these results are available in the
supporting materials.

All extortionate strategies reside on a triangular (\ref{eqn:condition_for_chi})
plane (\ref{eqn:condition_for_p1}) in 3 dimensions (\ref{eqn:condition_for_p4}).
Using this formulation it can be seen that a necessary (but not sufficient)
condition for an extortionate strategy is that it cooperates on average less
than 50\% of the time when in a state of disagreement with the opponent.

As an example, consider the known extortionate strategy \(p=(8 / 9, 1 / 2, 1 /
3, 0)\) from~\cite{Stewart2012} which is referred to as \texttt{Extort-2}. In
this case, for the standard values of \((R, T, S, P)\) constraint
(\ref{eqn:condition_for_p1}) corresponds to:

\begin{equation}
    p_1 = \frac{2(p_2 + p_3) + 1}{3}
\end{equation}

It is clear that in this case all constraints hold.

This approach could in fact be used to confirm that a given strategy is acting
in an extortionate manner even if it is not a memory one strategy. However, in
practice, if a closed form for \(p\) is not known, then due to measurement
and/or numerical error this would not work.

This problem can be written in the following linear algebraic form where
\(x=(\alpha, \beta)\)
and \(p^*=(\tilde p_1 - 1, tilde_2 - 1, p_3)\):

\begin{equation}\label{eqn:linear_algebraic_equation_for_p}
    Cx= p^*
\end{equation}

\(C\) corresponds to equations
(\ref{eqn:condition_for_tilde_p1}-\ref{eqn:condition_for_tilde_p3}) and is
given by:

\begin{equation}\label{eqn:definition_of_C}
    C =
    \begin{bmatrix}
        R - P & R- P \\
        S - P & T- P \\
        T - P & S- P \\
    \end{bmatrix}
\end{equation}

Note that in general, equation (\ref{eqn:linear_algebraic_equation_for_p}) will
not necessarily have a solution. From the Rouch\'{e}-Capelli theorem if there is
a solution it is unique as \(\text{rank}(C)=2\) which is the dimension of the
variable \(x\). The best fitting \(x\) is found by minimizing:

\begin{equation}\label{eqn:r_squared}
    \text{SSError} = \|C x- p^*\|_2^2 = \sum_{i=1}^{3}\left((C\bar x)_i-p_i^*\right)^2
\end{equation}

Note that \(\text{SSError}\), which is the square of the Frobenius
norm~\cite{Golub2013}, becomes a measure of how close a strategy is to being an
extortionate strategy. Suspicion
of extortion then corresponds to a threshold on \(\text{SSError}\).

By observing interactions (human or otherwise), their memory one representation
can be inferred and this approach can be used to recognise extortionate
behaviour. The notion of comparing theoretic and actual plays of the IPD is not
novel, see for example~\cite{Rand2013}. Immediately it is noted that if the
environment is noisy~\cite{Wu1995} then no strategy can be considered to be
extortionate as \(p_4>0\).

In the next section, this idea will be illustrated by observing the interactions
that take place in a computer based tournament of the IPD\@.

\section{Numerical experiments}\label{sec:numerical-experiments}

In~\cite{Stewart2012} results from a tournament with
\input{./assets/tex/number_of_stewart_plotkin_strategies/main.tex} strategies,
was presented with specific consideration given to ZD strategies. This
tournament is reproduced here using the Axelrod-Python
project~\cite{Knight2016}. To obtain a good measure of the corresponding
transition rates for each strategy all matches have been run for
\input{assets/tex/number_of_turns/main.tex} turns and every match has been
repeated \input{assets/tex/number_of_repetitions/main.tex} times. All of this
interaction data is available at~\cite{vincent_knight_2018_1297075}. A good
match between the inferred Markov chain and the state distribution of the actual
interactions has been verified. Data for this is presented in the supplementary
materials.

Figure~\ref{fig:SSError_overall_in_stewart_plotkin} shows the \(\text{SSError}\)
values for all the strategies in the tournament, as reported
in~\cite{Stewart2012} the extortionate strategy (which has an expected
\(\text{SSError}\) approximately 0) gains a large number of wins.

\begin{figure}[!htbp]
    \centering
    \includegraphics[width=.8\textwidth]{./assets/img/SSError_overall_in_stewart_plotkin/main.pdf}
    \caption{\(\text{SSError}\) and state probabilities for the strategies
        of~\cite{Stewart2012}, ordered both by number of wins and overall score.
        Note that \(P(DC)\) is not shown as it corresponds to the transpose of
        \(P(CD)\). Cooperator and Defector are omitted as they do not visit all
        the states.}
    \label{fig:SSError_overall_in_stewart_plotkin}
\end{figure}

Here, the work of~\cite{Stewart2012} is extended by investigating a tournament
with \input{assets/tex/number_of_full_strategies/main.tex}
strategies.

The results of this analysis are shown in
Figure~\ref{fig:SSError_and_probabilities_in_full}. The top ranking strategies
by number of wins seem to be extortionate (but not against all strategies) and
it can be seen that a small sub group of strategies achieve mutual defection.
All the top ranking strategies according to score achieve mutual cooperation and
do not extort each other, however they
\textbf{do} exhibit extortionate behaviour towards a number of the lower ranking
strategies.

\begin{figure}[!htbp]
    \centering
    \includegraphics[width=.8\textwidth]{./assets/img/SSError_and_probabilities_in_full/main.pdf}
    \caption{\(\text{SSError}\) for the strategies for the full tournament. Only
    strategy interactions for which \(p_4=0\) and \(\chi>1\) are displayed.}
    \label{fig:SSError_and_probabilities_in_full}
\end{figure}

\section{Conclusion}\label{sec:conclusion}

This work defines an approach to measure whether or not a player is playing a
strategy that corresponds to an extortionate strategy as defined
in~\cite{Press2012}: a mathematical model for suspicion. Indeed, all
extortionate strategies have been
 classified as lying on a triangular plane.
This rigorous classification fails to be robust to small measurement error, thus
a statistical approach is proposed.
This is done through a linear algebraic approach for approximating the solution
of a linear system. Using this, a large number of pairwise interactions is
simulated and in fact very few strategies are found to act extortionately.

The work of~\cite{Press2012}, whilst showing that a clever approach to taking
advantage of another memory one strategy exists: this is incomplete. Whilst the
elegance of this result is very attractive, just as the simplicity of the
victory of Tit For Tat in Axelrod's original tournaments was, it is incomplete.
Extortionate strategies achieve a high number of wins but they do not
achieve a high score which corresponds to the fitness landscape in an
evolutionary sense. From the large number of interactions a payoff matrix \(S\)
can be measured where \(S_{ij}\) denotes the score (using standard values of
\((R, S, T, P) = (3, 0, 5, 1)\)) of the \(i\)th strategy
against the \(j\)th strategy. Using this, the replicator equation
describes the evolution of the system based on a population density fitness
function:

\begin{equation}\label{eqn:replicator_dynamics}
    \frac{dx}{dt} = x(S-x^TS x)
\end{equation}

Equation (\ref{eqn:replicator_dynamics}) is solved numerically through an
integration technique described in~\cite{Petzold1983} and
Figure~\ref{fig:replicator_dynamics} shows the evolution of the distribution of
the system: the various strategies are ranked by scores. It is clear to see that
only the high ranking strategies survive the evolutionary process (in fact,
only \input{./assets/img/replicator_dynamics/main.tex}
have a final distribution greater than \(10 ^ {-2}\)). This confirms the
findings of~\cite{Moran1707} in which sophisticated strategies resist
evolutionary invasion of shorter memory strategies. Recalling
Figure~\ref{fig:SSError_and_probabilities_in_full} this demonstrates that:

\begin{itemize}
    \item Cooperation emerges through the evolutionary process: the high scoring
        strategies do not exhibit extortionate behaviour towards each other.
    \item Extortionate strategies do not survive the evolutionary process.
\end{itemize}

\begin{figure}[!htbp]
    \centering
    \includegraphics[width=.8\textwidth]{./assets/img/replicator_dynamics/main.pdf}
    \caption{Numerical simulation of the replicator equation
    (\ref{eqn:replicator_dynamics}): strategies are ordered by score, only the strategies with a high score survive the evolutionary process.}
    \label{fig:replicator_dynamics}
\end{figure}

This work can be used to classify plays of the IPD\@: data can be collected from
actual interactions (in lab or in the field). Furthermore, this allows for a
classification method similar to the notion of fingerprinting presented
in~\cite{Ashlock2008}. Trained strategies can potentially be classified as
extortionate or not or it could be possible to even constrain the reinforcement
learning approaches that are becoming prevalent in the literature.
Alternatively, this mathematical approach for recognising extortion could be
used in sophisticated strategies to defend against invasion. Arguably, some of
the strategies considered here exhibit this behaviour, indeed as described
in~\cite{Harper2017}, the top ranking strategies in the full tournament are
obtained using evolutionary reinforcement learning techniques, thus, suspicion
of extortionate behaviour could in fact be an evolutionary trait.

\section*{Acknowledgements}

The following open source software libraries were used in this research:

\begin{itemize}
    \item The Axelrod ~\cite{Knight2016, Knight2018} library (IPD strategies and
        tournaments).
    \item The sympy library~\cite{Meurer2017} (verification of all symbolic
        calculations).
    \item The matplotlib~\cite{Droettboom2018} library (visualisation).
    \item The pandas~\cite{Structures2010}, dask~\cite{Dask2016} and
        NumPy~\cite{Oliphant2015} libraries (data manipulation).
    \item The SciPy~\cite{Jones2001} library (numerical integration of the
        replicator equation).
\end{itemize}

This work was performed using the computational facilities of the Advanced
Research Computing @ Cardiff (ARCCA) Division, Cardiff University.

\printbibliography

\newpage
\section*{Supplementary materials}

\includepdf{assets/pdf/proof_of_form_of_extortionate_strategies/main.pdf}

\newpage

Using the pair wise interactions the transition rates \(p,
q\) can be measured and the steady state probabilities inferred and compared to
the actual probabilities of each state.
This is done numerically by computing the singular eigenvector of the
matrix \(A\) \cite{Stewart2009}:

\[
    A =
    \begin{bmatrix}
        p_1 q_1 & p_1 (1 - q_1) & (1 - p_1) q_1 & (1 -p_1) (1 - q_1) \\
        p_2 q_2 & p_2 (1 - q_2) & (1 - p_2) q_2 & (1 -p_2) (1 - q_2) \\
        p_3 q_3 & p_3 (1 - q_3) & (1 - p_3) q_3 & (1 -p_3) (1 - q_3) \\
        p_4 q_4 & p_4 (1 - q_4) & (1 - p_4) q_4 & (1 -p_4) (1 - q_4) \\
    \end{bmatrix}
\]

Figure~\ref{fig:computed_probabilities_vs_theoretic_probabilities} shows a
regression line fitted to every pairwise interaction with a reported
\(\text{SSError}\) value (pairwise interactions with missing states were
omitted). This serves to validate the approach: a part from some edge cases the
relationship is consistent.

\begin{figure}[!htbp]
    \centering
    \includegraphics[width=.8\textwidth]{./assets/img/computed_probabilities_vs_theoretic_probabilities/main.pdf}
    \caption{The
        relationship between the steady state probabilities inferred from the
        measured transitions and the actual steady state probabilities. A linear
        regression line is included validating the approach.}
    \label{fig:computed_probabilities_vs_theoretic_probabilities}
\end{figure}


\end{document}
 strategies,
was presented with specific consideration given to ZD strategies. This
tournament is reproduced here using the Axelrod-Python
project~\cite{Knight2016}. To obtain a good measure of the corresponding
transition rates for each strategy all matches have been run for
\documentclass[a4paper]{article}

\usepackage{amsmath}
\usepackage{amssymb}
\usepackage[margin=1.5cm,
            includefoot,
            footskip=30pt]{geometry}
\usepackage{layout}
\usepackage{graphicx}
\usepackage{subcaption}

\usepackage{biblatex}
\usepackage{pdfpages}

\bibliography{main.bib}

\title{Suspicion: Recognising and evaluating the effectiveness
       of extortion in the Iterated Prisoner's Dilemma}
\author{Vincent A. Knight \and Nikoleta E. Glynatsi}
\date{\today}



\begin{document}

\maketitle

\begin{abstract}
    The Iterated Prisoner's Dilemma is a model for rational and evolutionary
    interactive behaviour. It has applications both in the study of human social
    behaviour as well as in biology.
    It is used to understand when and how a rational individual might
    accept an immediate cost to their own utility for the direct benefit of
    another.

    Much attention has been given to a class of strategies called
    Zero Determinant strategies. It has been theoretically shown that these
    strategies can ``extort'' any player.

    In this work, an approach to identify if observed strategies are playing in
    an extortionate way is described. Furthermore, experimental analysis of
    a large tournament with \input{assets/tex/number_of_full_strategies/main.tex}
    strategies is considered. In this setting
    the most highly performing strategies do not play in an extortionate way
    against each other but do against lower performing strategies.
    This suggests that whilst the theory of Zero Determinant strategies
    indicates that memory is not of fundamental importance to the evolution of
    cooperative behaviour, this is incomplete.
\end{abstract}

\section{Introduction}\label{sec:introduction}

Agent based game theoretic models have become a stalwart of the underpinning
mathematics of interactive behaviours. One of the major pieces of work
in this area is the pair of original computer tournaments run by Robert
Axelrod~\cite{Axelrod1980, Axelrod1980a}. These tournaments pitted submitted
computer strategies against each other in plays of the Iterated Prisoner's
Dilemma. A common game where agents can choose to pay a slight cost to their
immediate utility in the hope of building a reputation. This has been used in
economic and evolutionary game theory to understand the evolution of cooperative
behaviour.

Recently, a class of strategies was described in~\cite{Press2012} that can
provably extort any given opponent. In~\cite{Hilbe2013, Moran1707} some
questions have already been asked about the true effectiveness of these
strategies in an evolutionary setting. Here another question is asked: is it
possible to recognise this extortionate behaviour? A mathematical procedure for
suspicion is presented: in the same way that the continued actions of an
extortionate individual might raise suspicion.

This work makes use of the Axelrod Python library~\cite{Knight2018, Knight2016}
with a large number of Prisoner Dilemma strategies available to give an
extensive numerical example of the ideas presented.  The approach is presented
in Section~\ref{sec:delta-zd-strategies}.  All of the code and data discussed
in Section~\ref{sec:numerical-experiments} is open sourced, archived and
written according to best scientific principles~\cite{Wilson2014}. The data
archive can be found at~\cite{vincent_knight_2018_1297075}.

\section{Recognising Extortion}\label{sec:delta-zd-strategies}

In~\cite{Press2012}, given a match between 2 memory-one strategies, the concept
of Zero Determinant (ZD) strategies is introduced. The main result of that paper
shows that given two memory one players \(p, q\in\mathbb{R}^4\) a linear
relationship between the players' scores could be forced by one of the players.

Using the notation of~\cite{Press2012}, assuming the utilities for player \(p\)
are given by \(S_x=(R, S, T, P)\) and for player \(q\) by \(S_y=(R, T, S, P)\)
and that the stationary scores of each player is given by \(S_X\) and \(S_Y\)
respectively. The main result of~\cite{Press2012} is that if

\begin{equation}\label{eqn:linear_relationship_for_p}
    \tilde p=\alpha S_x + \beta S_y + \gamma
\end{equation}

or

\begin{equation}\label{eqn:linear_relationship_for_q}
    \tilde q=\alpha S_x + \beta S_y + \gamma
\end{equation}

where \(\tilde p = (1 - p_1, 1 - p_2, p_3, p_4)\) and
\(\tilde q = (1 - q_1, 1 - q_2, q_3, q_4)\) then:

\begin{equation}
    \alpha S_X + \beta S_Y + \gamma = 0
\end{equation}

In~\cite{Press2012} a particular type of ZD strategy is defined: extortionate
strategies. If:

\begin{equation}\label{eqn:constraint_for_extortion}
    \gamma = - P(\alpha + \beta)
\end{equation}

then the player can ensure they get a score \(\chi\) times
larger than the opponent. This extortion coefficient is given by:

\begin{equation}\label{eqn:definition_of_chi}
    \chi=\frac{-\beta}{\alpha}
\end{equation}

Thus, if (\ref{eqn:constraint_for_extortion}) holds and \(\chi >1\) a player is
said to extort their opponent.
Here, the reverse problem is considered: given a
\(p\in\mathbb{R}^4\) how does one identify \(\alpha, \beta\) if they
exist and is the strategy in fact acting in an extortionate way?

These conditions correspond to:

\begin{align}
    \tilde p_1 & = \alpha R + \beta R - P (\alpha + \beta)
            \label{eqn:condition_for_tilde_p1}\\
    \tilde p_2 & = \alpha S + \beta T - P (\alpha + \beta)
            \label{eqn:condition_for_tilde_p2}\\
    \tilde p_3 & = \alpha T + \beta S - P (\alpha + \beta)
            \label{eqn:condition_for_tilde_p3}\\
    \tilde p_4 & = \alpha P + \beta P - P (\alpha + \beta)
            \label{eqn:condition_for_tilde_p4}
\end{align}

Equation (\ref{eqn:condition_for_tilde_p4}) ensures that \(p_4=\tilde p_4=0\).
Equations (\ref{eqn:condition_for_tilde_p1}-\ref{eqn:condition_for_tilde_p3})
can be used to eliminate \(\alpha, \beta\), giving:

\begin{equation}\label{eqn:planar_definition_of_extortion}
    \tilde p_1 = \frac{(R - P)(\tilde p_2 + \tilde p_3)}{S + T - 2P}
\end{equation}

with:

\begin{equation}\label{eqn:definition_of_chi}
    \chi = \frac{\tilde p_2 (P - T) + \tilde p_3 (S - P)}
                {\tilde p_2 (P - S) + \tilde p_3 (T - P)}
\end{equation}

Given a strategy \(p\in\mathbb{R}^{4\times 1}\) equations
(\ref{eqn:condition_for_tilde_p4}), (\ref{eqn:planar_definition_of_extortion}-\ref{eqn:definition_of_chi}) can be used to check if
a strategy is extortionate. The conditions correspond to:

\begin{align}
    p_1 & = \frac{(R-P)(p_2 + p_3) - R + T + S - P}{S + T - 2P}
     \label{eqn:condition_for_p1}\\
    p_4 & = 0 \label{eqn:condition_for_p4}\\
    1 & > p_2 + p_3\label{eqn:condition_for_chi}
\end{align}

The algebraic steps necessary to prove these results are available in the
supporting materials.

All extortionate strategies reside on a triangular (\ref{eqn:condition_for_chi})
plane (\ref{eqn:condition_for_p1}) in 3 dimensions (\ref{eqn:condition_for_p4}).
Using this formulation it can be seen that a necessary (but not sufficient)
condition for an extortionate strategy is that it cooperates on average less
than 50\% of the time when in a state of disagreement with the opponent.

As an example, consider the known extortionate strategy \(p=(8 / 9, 1 / 2, 1 /
3, 0)\) from~\cite{Stewart2012} which is referred to as \texttt{Extort-2}. In
this case, for the standard values of \((R, T, S, P)\) constraint
(\ref{eqn:condition_for_p1}) corresponds to:

\begin{equation}
    p_1 = \frac{2(p_2 + p_3) + 1}{3}
\end{equation}

It is clear that in this case all constraints hold.

This approach could in fact be used to confirm that a given strategy is acting
in an extortionate manner even if it is not a memory one strategy. However, in
practice, if a closed form for \(p\) is not known, then due to measurement
and/or numerical error this would not work.

This problem can be written in the following linear algebraic form where
\(x=(\alpha, \beta)\)
and \(p^*=(\tilde p_1 - 1, tilde_2 - 1, p_3)\):

\begin{equation}\label{eqn:linear_algebraic_equation_for_p}
    Cx= p^*
\end{equation}

\(C\) corresponds to equations
(\ref{eqn:condition_for_tilde_p1}-\ref{eqn:condition_for_tilde_p3}) and is
given by:

\begin{equation}\label{eqn:definition_of_C}
    C =
    \begin{bmatrix}
        R - P & R- P \\
        S - P & T- P \\
        T - P & S- P \\
    \end{bmatrix}
\end{equation}

Note that in general, equation (\ref{eqn:linear_algebraic_equation_for_p}) will
not necessarily have a solution. From the Rouch\'{e}-Capelli theorem if there is
a solution it is unique as \(\text{rank}(C)=2\) which is the dimension of the
variable \(x\). The best fitting \(x\) is found by minimizing:

\begin{equation}\label{eqn:r_squared}
    \text{SSError} = \|C x- p^*\|_2^2 = \sum_{i=1}^{3}\left((C\bar x)_i-p_i^*\right)^2
\end{equation}

Note that \(\text{SSError}\), which is the square of the Frobenius
norm~\cite{Golub2013}, becomes a measure of how close a strategy is to being an
extortionate strategy. Suspicion
of extortion then corresponds to a threshold on \(\text{SSError}\).

By observing interactions (human or otherwise), their memory one representation
can be inferred and this approach can be used to recognise extortionate
behaviour. The notion of comparing theoretic and actual plays of the IPD is not
novel, see for example~\cite{Rand2013}. Immediately it is noted that if the
environment is noisy~\cite{Wu1995} then no strategy can be considered to be
extortionate as \(p_4>0\).

In the next section, this idea will be illustrated by observing the interactions
that take place in a computer based tournament of the IPD\@.

\section{Numerical experiments}\label{sec:numerical-experiments}

In~\cite{Stewart2012} results from a tournament with
\input{./assets/tex/number_of_stewart_plotkin_strategies/main.tex} strategies,
was presented with specific consideration given to ZD strategies. This
tournament is reproduced here using the Axelrod-Python
project~\cite{Knight2016}. To obtain a good measure of the corresponding
transition rates for each strategy all matches have been run for
\input{assets/tex/number_of_turns/main.tex} turns and every match has been
repeated \input{assets/tex/number_of_repetitions/main.tex} times. All of this
interaction data is available at~\cite{vincent_knight_2018_1297075}. A good
match between the inferred Markov chain and the state distribution of the actual
interactions has been verified. Data for this is presented in the supplementary
materials.

Figure~\ref{fig:SSError_overall_in_stewart_plotkin} shows the \(\text{SSError}\)
values for all the strategies in the tournament, as reported
in~\cite{Stewart2012} the extortionate strategy (which has an expected
\(\text{SSError}\) approximately 0) gains a large number of wins.

\begin{figure}[!htbp]
    \centering
    \includegraphics[width=.8\textwidth]{./assets/img/SSError_overall_in_stewart_plotkin/main.pdf}
    \caption{\(\text{SSError}\) and state probabilities for the strategies
        of~\cite{Stewart2012}, ordered both by number of wins and overall score.
        Note that \(P(DC)\) is not shown as it corresponds to the transpose of
        \(P(CD)\). Cooperator and Defector are omitted as they do not visit all
        the states.}
    \label{fig:SSError_overall_in_stewart_plotkin}
\end{figure}

Here, the work of~\cite{Stewart2012} is extended by investigating a tournament
with \input{assets/tex/number_of_full_strategies/main.tex}
strategies.

The results of this analysis are shown in
Figure~\ref{fig:SSError_and_probabilities_in_full}. The top ranking strategies
by number of wins seem to be extortionate (but not against all strategies) and
it can be seen that a small sub group of strategies achieve mutual defection.
All the top ranking strategies according to score achieve mutual cooperation and
do not extort each other, however they
\textbf{do} exhibit extortionate behaviour towards a number of the lower ranking
strategies.

\begin{figure}[!htbp]
    \centering
    \includegraphics[width=.8\textwidth]{./assets/img/SSError_and_probabilities_in_full/main.pdf}
    \caption{\(\text{SSError}\) for the strategies for the full tournament. Only
    strategy interactions for which \(p_4=0\) and \(\chi>1\) are displayed.}
    \label{fig:SSError_and_probabilities_in_full}
\end{figure}

\section{Conclusion}\label{sec:conclusion}

This work defines an approach to measure whether or not a player is playing a
strategy that corresponds to an extortionate strategy as defined
in~\cite{Press2012}: a mathematical model for suspicion. Indeed, all
extortionate strategies have been
 classified as lying on a triangular plane.
This rigorous classification fails to be robust to small measurement error, thus
a statistical approach is proposed.
This is done through a linear algebraic approach for approximating the solution
of a linear system. Using this, a large number of pairwise interactions is
simulated and in fact very few strategies are found to act extortionately.

The work of~\cite{Press2012}, whilst showing that a clever approach to taking
advantage of another memory one strategy exists: this is incomplete. Whilst the
elegance of this result is very attractive, just as the simplicity of the
victory of Tit For Tat in Axelrod's original tournaments was, it is incomplete.
Extortionate strategies achieve a high number of wins but they do not
achieve a high score which corresponds to the fitness landscape in an
evolutionary sense. From the large number of interactions a payoff matrix \(S\)
can be measured where \(S_{ij}\) denotes the score (using standard values of
\((R, S, T, P) = (3, 0, 5, 1)\)) of the \(i\)th strategy
against the \(j\)th strategy. Using this, the replicator equation
describes the evolution of the system based on a population density fitness
function:

\begin{equation}\label{eqn:replicator_dynamics}
    \frac{dx}{dt} = x(S-x^TS x)
\end{equation}

Equation (\ref{eqn:replicator_dynamics}) is solved numerically through an
integration technique described in~\cite{Petzold1983} and
Figure~\ref{fig:replicator_dynamics} shows the evolution of the distribution of
the system: the various strategies are ranked by scores. It is clear to see that
only the high ranking strategies survive the evolutionary process (in fact,
only \input{./assets/img/replicator_dynamics/main.tex}
have a final distribution greater than \(10 ^ {-2}\)). This confirms the
findings of~\cite{Moran1707} in which sophisticated strategies resist
evolutionary invasion of shorter memory strategies. Recalling
Figure~\ref{fig:SSError_and_probabilities_in_full} this demonstrates that:

\begin{itemize}
    \item Cooperation emerges through the evolutionary process: the high scoring
        strategies do not exhibit extortionate behaviour towards each other.
    \item Extortionate strategies do not survive the evolutionary process.
\end{itemize}

\begin{figure}[!htbp]
    \centering
    \includegraphics[width=.8\textwidth]{./assets/img/replicator_dynamics/main.pdf}
    \caption{Numerical simulation of the replicator equation
    (\ref{eqn:replicator_dynamics}): strategies are ordered by score, only the strategies with a high score survive the evolutionary process.}
    \label{fig:replicator_dynamics}
\end{figure}

This work can be used to classify plays of the IPD\@: data can be collected from
actual interactions (in lab or in the field). Furthermore, this allows for a
classification method similar to the notion of fingerprinting presented
in~\cite{Ashlock2008}. Trained strategies can potentially be classified as
extortionate or not or it could be possible to even constrain the reinforcement
learning approaches that are becoming prevalent in the literature.
Alternatively, this mathematical approach for recognising extortion could be
used in sophisticated strategies to defend against invasion. Arguably, some of
the strategies considered here exhibit this behaviour, indeed as described
in~\cite{Harper2017}, the top ranking strategies in the full tournament are
obtained using evolutionary reinforcement learning techniques, thus, suspicion
of extortionate behaviour could in fact be an evolutionary trait.

\section*{Acknowledgements}

The following open source software libraries were used in this research:

\begin{itemize}
    \item The Axelrod ~\cite{Knight2016, Knight2018} library (IPD strategies and
        tournaments).
    \item The sympy library~\cite{Meurer2017} (verification of all symbolic
        calculations).
    \item The matplotlib~\cite{Droettboom2018} library (visualisation).
    \item The pandas~\cite{Structures2010}, dask~\cite{Dask2016} and
        NumPy~\cite{Oliphant2015} libraries (data manipulation).
    \item The SciPy~\cite{Jones2001} library (numerical integration of the
        replicator equation).
\end{itemize}

This work was performed using the computational facilities of the Advanced
Research Computing @ Cardiff (ARCCA) Division, Cardiff University.

\printbibliography

\newpage
\section*{Supplementary materials}

\includepdf{assets/pdf/proof_of_form_of_extortionate_strategies/main.pdf}

\newpage

Using the pair wise interactions the transition rates \(p,
q\) can be measured and the steady state probabilities inferred and compared to
the actual probabilities of each state.
This is done numerically by computing the singular eigenvector of the
matrix \(A\) \cite{Stewart2009}:

\[
    A =
    \begin{bmatrix}
        p_1 q_1 & p_1 (1 - q_1) & (1 - p_1) q_1 & (1 -p_1) (1 - q_1) \\
        p_2 q_2 & p_2 (1 - q_2) & (1 - p_2) q_2 & (1 -p_2) (1 - q_2) \\
        p_3 q_3 & p_3 (1 - q_3) & (1 - p_3) q_3 & (1 -p_3) (1 - q_3) \\
        p_4 q_4 & p_4 (1 - q_4) & (1 - p_4) q_4 & (1 -p_4) (1 - q_4) \\
    \end{bmatrix}
\]

Figure~\ref{fig:computed_probabilities_vs_theoretic_probabilities} shows a
regression line fitted to every pairwise interaction with a reported
\(\text{SSError}\) value (pairwise interactions with missing states were
omitted). This serves to validate the approach: a part from some edge cases the
relationship is consistent.

\begin{figure}[!htbp]
    \centering
    \includegraphics[width=.8\textwidth]{./assets/img/computed_probabilities_vs_theoretic_probabilities/main.pdf}
    \caption{The
        relationship between the steady state probabilities inferred from the
        measured transitions and the actual steady state probabilities. A linear
        regression line is included validating the approach.}
    \label{fig:computed_probabilities_vs_theoretic_probabilities}
\end{figure}


\end{document}
 turns and every match has been
repeated \documentclass[a4paper]{article}

\usepackage{amsmath}
\usepackage{amssymb}
\usepackage[margin=1.5cm,
            includefoot,
            footskip=30pt]{geometry}
\usepackage{layout}
\usepackage{graphicx}
\usepackage{subcaption}

\usepackage{biblatex}
\usepackage{pdfpages}

\bibliography{main.bib}

\title{Suspicion: Recognising and evaluating the effectiveness
       of extortion in the Iterated Prisoner's Dilemma}
\author{Vincent A. Knight \and Nikoleta E. Glynatsi}
\date{\today}



\begin{document}

\maketitle

\begin{abstract}
    The Iterated Prisoner's Dilemma is a model for rational and evolutionary
    interactive behaviour. It has applications both in the study of human social
    behaviour as well as in biology.
    It is used to understand when and how a rational individual might
    accept an immediate cost to their own utility for the direct benefit of
    another.

    Much attention has been given to a class of strategies called
    Zero Determinant strategies. It has been theoretically shown that these
    strategies can ``extort'' any player.

    In this work, an approach to identify if observed strategies are playing in
    an extortionate way is described. Furthermore, experimental analysis of
    a large tournament with \input{assets/tex/number_of_full_strategies/main.tex}
    strategies is considered. In this setting
    the most highly performing strategies do not play in an extortionate way
    against each other but do against lower performing strategies.
    This suggests that whilst the theory of Zero Determinant strategies
    indicates that memory is not of fundamental importance to the evolution of
    cooperative behaviour, this is incomplete.
\end{abstract}

\section{Introduction}\label{sec:introduction}

Agent based game theoretic models have become a stalwart of the underpinning
mathematics of interactive behaviours. One of the major pieces of work
in this area is the pair of original computer tournaments run by Robert
Axelrod~\cite{Axelrod1980, Axelrod1980a}. These tournaments pitted submitted
computer strategies against each other in plays of the Iterated Prisoner's
Dilemma. A common game where agents can choose to pay a slight cost to their
immediate utility in the hope of building a reputation. This has been used in
economic and evolutionary game theory to understand the evolution of cooperative
behaviour.

Recently, a class of strategies was described in~\cite{Press2012} that can
provably extort any given opponent. In~\cite{Hilbe2013, Moran1707} some
questions have already been asked about the true effectiveness of these
strategies in an evolutionary setting. Here another question is asked: is it
possible to recognise this extortionate behaviour? A mathematical procedure for
suspicion is presented: in the same way that the continued actions of an
extortionate individual might raise suspicion.

This work makes use of the Axelrod Python library~\cite{Knight2018, Knight2016}
with a large number of Prisoner Dilemma strategies available to give an
extensive numerical example of the ideas presented.  The approach is presented
in Section~\ref{sec:delta-zd-strategies}.  All of the code and data discussed
in Section~\ref{sec:numerical-experiments} is open sourced, archived and
written according to best scientific principles~\cite{Wilson2014}. The data
archive can be found at~\cite{vincent_knight_2018_1297075}.

\section{Recognising Extortion}\label{sec:delta-zd-strategies}

In~\cite{Press2012}, given a match between 2 memory-one strategies, the concept
of Zero Determinant (ZD) strategies is introduced. The main result of that paper
shows that given two memory one players \(p, q\in\mathbb{R}^4\) a linear
relationship between the players' scores could be forced by one of the players.

Using the notation of~\cite{Press2012}, assuming the utilities for player \(p\)
are given by \(S_x=(R, S, T, P)\) and for player \(q\) by \(S_y=(R, T, S, P)\)
and that the stationary scores of each player is given by \(S_X\) and \(S_Y\)
respectively. The main result of~\cite{Press2012} is that if

\begin{equation}\label{eqn:linear_relationship_for_p}
    \tilde p=\alpha S_x + \beta S_y + \gamma
\end{equation}

or

\begin{equation}\label{eqn:linear_relationship_for_q}
    \tilde q=\alpha S_x + \beta S_y + \gamma
\end{equation}

where \(\tilde p = (1 - p_1, 1 - p_2, p_3, p_4)\) and
\(\tilde q = (1 - q_1, 1 - q_2, q_3, q_4)\) then:

\begin{equation}
    \alpha S_X + \beta S_Y + \gamma = 0
\end{equation}

In~\cite{Press2012} a particular type of ZD strategy is defined: extortionate
strategies. If:

\begin{equation}\label{eqn:constraint_for_extortion}
    \gamma = - P(\alpha + \beta)
\end{equation}

then the player can ensure they get a score \(\chi\) times
larger than the opponent. This extortion coefficient is given by:

\begin{equation}\label{eqn:definition_of_chi}
    \chi=\frac{-\beta}{\alpha}
\end{equation}

Thus, if (\ref{eqn:constraint_for_extortion}) holds and \(\chi >1\) a player is
said to extort their opponent.
Here, the reverse problem is considered: given a
\(p\in\mathbb{R}^4\) how does one identify \(\alpha, \beta\) if they
exist and is the strategy in fact acting in an extortionate way?

These conditions correspond to:

\begin{align}
    \tilde p_1 & = \alpha R + \beta R - P (\alpha + \beta)
            \label{eqn:condition_for_tilde_p1}\\
    \tilde p_2 & = \alpha S + \beta T - P (\alpha + \beta)
            \label{eqn:condition_for_tilde_p2}\\
    \tilde p_3 & = \alpha T + \beta S - P (\alpha + \beta)
            \label{eqn:condition_for_tilde_p3}\\
    \tilde p_4 & = \alpha P + \beta P - P (\alpha + \beta)
            \label{eqn:condition_for_tilde_p4}
\end{align}

Equation (\ref{eqn:condition_for_tilde_p4}) ensures that \(p_4=\tilde p_4=0\).
Equations (\ref{eqn:condition_for_tilde_p1}-\ref{eqn:condition_for_tilde_p3})
can be used to eliminate \(\alpha, \beta\), giving:

\begin{equation}\label{eqn:planar_definition_of_extortion}
    \tilde p_1 = \frac{(R - P)(\tilde p_2 + \tilde p_3)}{S + T - 2P}
\end{equation}

with:

\begin{equation}\label{eqn:definition_of_chi}
    \chi = \frac{\tilde p_2 (P - T) + \tilde p_3 (S - P)}
                {\tilde p_2 (P - S) + \tilde p_3 (T - P)}
\end{equation}

Given a strategy \(p\in\mathbb{R}^{4\times 1}\) equations
(\ref{eqn:condition_for_tilde_p4}), (\ref{eqn:planar_definition_of_extortion}-\ref{eqn:definition_of_chi}) can be used to check if
a strategy is extortionate. The conditions correspond to:

\begin{align}
    p_1 & = \frac{(R-P)(p_2 + p_3) - R + T + S - P}{S + T - 2P}
     \label{eqn:condition_for_p1}\\
    p_4 & = 0 \label{eqn:condition_for_p4}\\
    1 & > p_2 + p_3\label{eqn:condition_for_chi}
\end{align}

The algebraic steps necessary to prove these results are available in the
supporting materials.

All extortionate strategies reside on a triangular (\ref{eqn:condition_for_chi})
plane (\ref{eqn:condition_for_p1}) in 3 dimensions (\ref{eqn:condition_for_p4}).
Using this formulation it can be seen that a necessary (but not sufficient)
condition for an extortionate strategy is that it cooperates on average less
than 50\% of the time when in a state of disagreement with the opponent.

As an example, consider the known extortionate strategy \(p=(8 / 9, 1 / 2, 1 /
3, 0)\) from~\cite{Stewart2012} which is referred to as \texttt{Extort-2}. In
this case, for the standard values of \((R, T, S, P)\) constraint
(\ref{eqn:condition_for_p1}) corresponds to:

\begin{equation}
    p_1 = \frac{2(p_2 + p_3) + 1}{3}
\end{equation}

It is clear that in this case all constraints hold.

This approach could in fact be used to confirm that a given strategy is acting
in an extortionate manner even if it is not a memory one strategy. However, in
practice, if a closed form for \(p\) is not known, then due to measurement
and/or numerical error this would not work.

This problem can be written in the following linear algebraic form where
\(x=(\alpha, \beta)\)
and \(p^*=(\tilde p_1 - 1, tilde_2 - 1, p_3)\):

\begin{equation}\label{eqn:linear_algebraic_equation_for_p}
    Cx= p^*
\end{equation}

\(C\) corresponds to equations
(\ref{eqn:condition_for_tilde_p1}-\ref{eqn:condition_for_tilde_p3}) and is
given by:

\begin{equation}\label{eqn:definition_of_C}
    C =
    \begin{bmatrix}
        R - P & R- P \\
        S - P & T- P \\
        T - P & S- P \\
    \end{bmatrix}
\end{equation}

Note that in general, equation (\ref{eqn:linear_algebraic_equation_for_p}) will
not necessarily have a solution. From the Rouch\'{e}-Capelli theorem if there is
a solution it is unique as \(\text{rank}(C)=2\) which is the dimension of the
variable \(x\). The best fitting \(x\) is found by minimizing:

\begin{equation}\label{eqn:r_squared}
    \text{SSError} = \|C x- p^*\|_2^2 = \sum_{i=1}^{3}\left((C\bar x)_i-p_i^*\right)^2
\end{equation}

Note that \(\text{SSError}\), which is the square of the Frobenius
norm~\cite{Golub2013}, becomes a measure of how close a strategy is to being an
extortionate strategy. Suspicion
of extortion then corresponds to a threshold on \(\text{SSError}\).

By observing interactions (human or otherwise), their memory one representation
can be inferred and this approach can be used to recognise extortionate
behaviour. The notion of comparing theoretic and actual plays of the IPD is not
novel, see for example~\cite{Rand2013}. Immediately it is noted that if the
environment is noisy~\cite{Wu1995} then no strategy can be considered to be
extortionate as \(p_4>0\).

In the next section, this idea will be illustrated by observing the interactions
that take place in a computer based tournament of the IPD\@.

\section{Numerical experiments}\label{sec:numerical-experiments}

In~\cite{Stewart2012} results from a tournament with
\input{./assets/tex/number_of_stewart_plotkin_strategies/main.tex} strategies,
was presented with specific consideration given to ZD strategies. This
tournament is reproduced here using the Axelrod-Python
project~\cite{Knight2016}. To obtain a good measure of the corresponding
transition rates for each strategy all matches have been run for
\input{assets/tex/number_of_turns/main.tex} turns and every match has been
repeated \input{assets/tex/number_of_repetitions/main.tex} times. All of this
interaction data is available at~\cite{vincent_knight_2018_1297075}. A good
match between the inferred Markov chain and the state distribution of the actual
interactions has been verified. Data for this is presented in the supplementary
materials.

Figure~\ref{fig:SSError_overall_in_stewart_plotkin} shows the \(\text{SSError}\)
values for all the strategies in the tournament, as reported
in~\cite{Stewart2012} the extortionate strategy (which has an expected
\(\text{SSError}\) approximately 0) gains a large number of wins.

\begin{figure}[!htbp]
    \centering
    \includegraphics[width=.8\textwidth]{./assets/img/SSError_overall_in_stewart_plotkin/main.pdf}
    \caption{\(\text{SSError}\) and state probabilities for the strategies
        of~\cite{Stewart2012}, ordered both by number of wins and overall score.
        Note that \(P(DC)\) is not shown as it corresponds to the transpose of
        \(P(CD)\). Cooperator and Defector are omitted as they do not visit all
        the states.}
    \label{fig:SSError_overall_in_stewart_plotkin}
\end{figure}

Here, the work of~\cite{Stewart2012} is extended by investigating a tournament
with \input{assets/tex/number_of_full_strategies/main.tex}
strategies.

The results of this analysis are shown in
Figure~\ref{fig:SSError_and_probabilities_in_full}. The top ranking strategies
by number of wins seem to be extortionate (but not against all strategies) and
it can be seen that a small sub group of strategies achieve mutual defection.
All the top ranking strategies according to score achieve mutual cooperation and
do not extort each other, however they
\textbf{do} exhibit extortionate behaviour towards a number of the lower ranking
strategies.

\begin{figure}[!htbp]
    \centering
    \includegraphics[width=.8\textwidth]{./assets/img/SSError_and_probabilities_in_full/main.pdf}
    \caption{\(\text{SSError}\) for the strategies for the full tournament. Only
    strategy interactions for which \(p_4=0\) and \(\chi>1\) are displayed.}
    \label{fig:SSError_and_probabilities_in_full}
\end{figure}

\section{Conclusion}\label{sec:conclusion}

This work defines an approach to measure whether or not a player is playing a
strategy that corresponds to an extortionate strategy as defined
in~\cite{Press2012}: a mathematical model for suspicion. Indeed, all
extortionate strategies have been
 classified as lying on a triangular plane.
This rigorous classification fails to be robust to small measurement error, thus
a statistical approach is proposed.
This is done through a linear algebraic approach for approximating the solution
of a linear system. Using this, a large number of pairwise interactions is
simulated and in fact very few strategies are found to act extortionately.

The work of~\cite{Press2012}, whilst showing that a clever approach to taking
advantage of another memory one strategy exists: this is incomplete. Whilst the
elegance of this result is very attractive, just as the simplicity of the
victory of Tit For Tat in Axelrod's original tournaments was, it is incomplete.
Extortionate strategies achieve a high number of wins but they do not
achieve a high score which corresponds to the fitness landscape in an
evolutionary sense. From the large number of interactions a payoff matrix \(S\)
can be measured where \(S_{ij}\) denotes the score (using standard values of
\((R, S, T, P) = (3, 0, 5, 1)\)) of the \(i\)th strategy
against the \(j\)th strategy. Using this, the replicator equation
describes the evolution of the system based on a population density fitness
function:

\begin{equation}\label{eqn:replicator_dynamics}
    \frac{dx}{dt} = x(S-x^TS x)
\end{equation}

Equation (\ref{eqn:replicator_dynamics}) is solved numerically through an
integration technique described in~\cite{Petzold1983} and
Figure~\ref{fig:replicator_dynamics} shows the evolution of the distribution of
the system: the various strategies are ranked by scores. It is clear to see that
only the high ranking strategies survive the evolutionary process (in fact,
only \input{./assets/img/replicator_dynamics/main.tex}
have a final distribution greater than \(10 ^ {-2}\)). This confirms the
findings of~\cite{Moran1707} in which sophisticated strategies resist
evolutionary invasion of shorter memory strategies. Recalling
Figure~\ref{fig:SSError_and_probabilities_in_full} this demonstrates that:

\begin{itemize}
    \item Cooperation emerges through the evolutionary process: the high scoring
        strategies do not exhibit extortionate behaviour towards each other.
    \item Extortionate strategies do not survive the evolutionary process.
\end{itemize}

\begin{figure}[!htbp]
    \centering
    \includegraphics[width=.8\textwidth]{./assets/img/replicator_dynamics/main.pdf}
    \caption{Numerical simulation of the replicator equation
    (\ref{eqn:replicator_dynamics}): strategies are ordered by score, only the strategies with a high score survive the evolutionary process.}
    \label{fig:replicator_dynamics}
\end{figure}

This work can be used to classify plays of the IPD\@: data can be collected from
actual interactions (in lab or in the field). Furthermore, this allows for a
classification method similar to the notion of fingerprinting presented
in~\cite{Ashlock2008}. Trained strategies can potentially be classified as
extortionate or not or it could be possible to even constrain the reinforcement
learning approaches that are becoming prevalent in the literature.
Alternatively, this mathematical approach for recognising extortion could be
used in sophisticated strategies to defend against invasion. Arguably, some of
the strategies considered here exhibit this behaviour, indeed as described
in~\cite{Harper2017}, the top ranking strategies in the full tournament are
obtained using evolutionary reinforcement learning techniques, thus, suspicion
of extortionate behaviour could in fact be an evolutionary trait.

\section*{Acknowledgements}

The following open source software libraries were used in this research:

\begin{itemize}
    \item The Axelrod ~\cite{Knight2016, Knight2018} library (IPD strategies and
        tournaments).
    \item The sympy library~\cite{Meurer2017} (verification of all symbolic
        calculations).
    \item The matplotlib~\cite{Droettboom2018} library (visualisation).
    \item The pandas~\cite{Structures2010}, dask~\cite{Dask2016} and
        NumPy~\cite{Oliphant2015} libraries (data manipulation).
    \item The SciPy~\cite{Jones2001} library (numerical integration of the
        replicator equation).
\end{itemize}

This work was performed using the computational facilities of the Advanced
Research Computing @ Cardiff (ARCCA) Division, Cardiff University.

\printbibliography

\newpage
\section*{Supplementary materials}

\includepdf{assets/pdf/proof_of_form_of_extortionate_strategies/main.pdf}

\newpage

Using the pair wise interactions the transition rates \(p,
q\) can be measured and the steady state probabilities inferred and compared to
the actual probabilities of each state.
This is done numerically by computing the singular eigenvector of the
matrix \(A\) \cite{Stewart2009}:

\[
    A =
    \begin{bmatrix}
        p_1 q_1 & p_1 (1 - q_1) & (1 - p_1) q_1 & (1 -p_1) (1 - q_1) \\
        p_2 q_2 & p_2 (1 - q_2) & (1 - p_2) q_2 & (1 -p_2) (1 - q_2) \\
        p_3 q_3 & p_3 (1 - q_3) & (1 - p_3) q_3 & (1 -p_3) (1 - q_3) \\
        p_4 q_4 & p_4 (1 - q_4) & (1 - p_4) q_4 & (1 -p_4) (1 - q_4) \\
    \end{bmatrix}
\]

Figure~\ref{fig:computed_probabilities_vs_theoretic_probabilities} shows a
regression line fitted to every pairwise interaction with a reported
\(\text{SSError}\) value (pairwise interactions with missing states were
omitted). This serves to validate the approach: a part from some edge cases the
relationship is consistent.

\begin{figure}[!htbp]
    \centering
    \includegraphics[width=.8\textwidth]{./assets/img/computed_probabilities_vs_theoretic_probabilities/main.pdf}
    \caption{The
        relationship between the steady state probabilities inferred from the
        measured transitions and the actual steady state probabilities. A linear
        regression line is included validating the approach.}
    \label{fig:computed_probabilities_vs_theoretic_probabilities}
\end{figure}


\end{document}
 times. All of this
interaction data is available at~\cite{vincent_knight_2018_1297075}. A good
match between the inferred Markov chain and the state distribution of the actual
interactions has been verified. Data for this is presented in the supplementary
materials.

Figure~\ref{fig:SSError_overall_in_stewart_plotkin} shows the \(\text{SSError}\)
values for all the strategies in the tournament, as reported
in~\cite{Stewart2012} the extortionate strategy (which has an expected
\(\text{SSError}\) approximately 0) gains a large number of wins.

\begin{figure}[!htbp]
    \centering
    \includegraphics[width=.8\textwidth]{./assets/img/SSError_overall_in_stewart_plotkin/main.pdf}
    \caption{\(\text{SSError}\) and state probabilities for the strategies
        of~\cite{Stewart2012}, ordered both by number of wins and overall score.
        Note that \(P(DC)\) is not shown as it corresponds to the transpose of
        \(P(CD)\). Cooperator and Defector are omitted as they do not visit all
        the states.}
    \label{fig:SSError_overall_in_stewart_plotkin}
\end{figure}

Here, the work of~\cite{Stewart2012} is extended by investigating a tournament
with \documentclass[a4paper]{article}

\usepackage{amsmath}
\usepackage{amssymb}
\usepackage[margin=1.5cm,
            includefoot,
            footskip=30pt]{geometry}
\usepackage{layout}
\usepackage{graphicx}
\usepackage{subcaption}

\usepackage{biblatex}
\usepackage{pdfpages}

\bibliography{main.bib}

\title{Suspicion: Recognising and evaluating the effectiveness
       of extortion in the Iterated Prisoner's Dilemma}
\author{Vincent A. Knight \and Nikoleta E. Glynatsi}
\date{\today}



\begin{document}

\maketitle

\begin{abstract}
    The Iterated Prisoner's Dilemma is a model for rational and evolutionary
    interactive behaviour. It has applications both in the study of human social
    behaviour as well as in biology.
    It is used to understand when and how a rational individual might
    accept an immediate cost to their own utility for the direct benefit of
    another.

    Much attention has been given to a class of strategies called
    Zero Determinant strategies. It has been theoretically shown that these
    strategies can ``extort'' any player.

    In this work, an approach to identify if observed strategies are playing in
    an extortionate way is described. Furthermore, experimental analysis of
    a large tournament with \input{assets/tex/number_of_full_strategies/main.tex}
    strategies is considered. In this setting
    the most highly performing strategies do not play in an extortionate way
    against each other but do against lower performing strategies.
    This suggests that whilst the theory of Zero Determinant strategies
    indicates that memory is not of fundamental importance to the evolution of
    cooperative behaviour, this is incomplete.
\end{abstract}

\section{Introduction}\label{sec:introduction}

Agent based game theoretic models have become a stalwart of the underpinning
mathematics of interactive behaviours. One of the major pieces of work
in this area is the pair of original computer tournaments run by Robert
Axelrod~\cite{Axelrod1980, Axelrod1980a}. These tournaments pitted submitted
computer strategies against each other in plays of the Iterated Prisoner's
Dilemma. A common game where agents can choose to pay a slight cost to their
immediate utility in the hope of building a reputation. This has been used in
economic and evolutionary game theory to understand the evolution of cooperative
behaviour.

Recently, a class of strategies was described in~\cite{Press2012} that can
provably extort any given opponent. In~\cite{Hilbe2013, Moran1707} some
questions have already been asked about the true effectiveness of these
strategies in an evolutionary setting. Here another question is asked: is it
possible to recognise this extortionate behaviour? A mathematical procedure for
suspicion is presented: in the same way that the continued actions of an
extortionate individual might raise suspicion.

This work makes use of the Axelrod Python library~\cite{Knight2018, Knight2016}
with a large number of Prisoner Dilemma strategies available to give an
extensive numerical example of the ideas presented.  The approach is presented
in Section~\ref{sec:delta-zd-strategies}.  All of the code and data discussed
in Section~\ref{sec:numerical-experiments} is open sourced, archived and
written according to best scientific principles~\cite{Wilson2014}. The data
archive can be found at~\cite{vincent_knight_2018_1297075}.

\section{Recognising Extortion}\label{sec:delta-zd-strategies}

In~\cite{Press2012}, given a match between 2 memory-one strategies, the concept
of Zero Determinant (ZD) strategies is introduced. The main result of that paper
shows that given two memory one players \(p, q\in\mathbb{R}^4\) a linear
relationship between the players' scores could be forced by one of the players.

Using the notation of~\cite{Press2012}, assuming the utilities for player \(p\)
are given by \(S_x=(R, S, T, P)\) and for player \(q\) by \(S_y=(R, T, S, P)\)
and that the stationary scores of each player is given by \(S_X\) and \(S_Y\)
respectively. The main result of~\cite{Press2012} is that if

\begin{equation}\label{eqn:linear_relationship_for_p}
    \tilde p=\alpha S_x + \beta S_y + \gamma
\end{equation}

or

\begin{equation}\label{eqn:linear_relationship_for_q}
    \tilde q=\alpha S_x + \beta S_y + \gamma
\end{equation}

where \(\tilde p = (1 - p_1, 1 - p_2, p_3, p_4)\) and
\(\tilde q = (1 - q_1, 1 - q_2, q_3, q_4)\) then:

\begin{equation}
    \alpha S_X + \beta S_Y + \gamma = 0
\end{equation}

In~\cite{Press2012} a particular type of ZD strategy is defined: extortionate
strategies. If:

\begin{equation}\label{eqn:constraint_for_extortion}
    \gamma = - P(\alpha + \beta)
\end{equation}

then the player can ensure they get a score \(\chi\) times
larger than the opponent. This extortion coefficient is given by:

\begin{equation}\label{eqn:definition_of_chi}
    \chi=\frac{-\beta}{\alpha}
\end{equation}

Thus, if (\ref{eqn:constraint_for_extortion}) holds and \(\chi >1\) a player is
said to extort their opponent.
Here, the reverse problem is considered: given a
\(p\in\mathbb{R}^4\) how does one identify \(\alpha, \beta\) if they
exist and is the strategy in fact acting in an extortionate way?

These conditions correspond to:

\begin{align}
    \tilde p_1 & = \alpha R + \beta R - P (\alpha + \beta)
            \label{eqn:condition_for_tilde_p1}\\
    \tilde p_2 & = \alpha S + \beta T - P (\alpha + \beta)
            \label{eqn:condition_for_tilde_p2}\\
    \tilde p_3 & = \alpha T + \beta S - P (\alpha + \beta)
            \label{eqn:condition_for_tilde_p3}\\
    \tilde p_4 & = \alpha P + \beta P - P (\alpha + \beta)
            \label{eqn:condition_for_tilde_p4}
\end{align}

Equation (\ref{eqn:condition_for_tilde_p4}) ensures that \(p_4=\tilde p_4=0\).
Equations (\ref{eqn:condition_for_tilde_p1}-\ref{eqn:condition_for_tilde_p3})
can be used to eliminate \(\alpha, \beta\), giving:

\begin{equation}\label{eqn:planar_definition_of_extortion}
    \tilde p_1 = \frac{(R - P)(\tilde p_2 + \tilde p_3)}{S + T - 2P}
\end{equation}

with:

\begin{equation}\label{eqn:definition_of_chi}
    \chi = \frac{\tilde p_2 (P - T) + \tilde p_3 (S - P)}
                {\tilde p_2 (P - S) + \tilde p_3 (T - P)}
\end{equation}

Given a strategy \(p\in\mathbb{R}^{4\times 1}\) equations
(\ref{eqn:condition_for_tilde_p4}), (\ref{eqn:planar_definition_of_extortion}-\ref{eqn:definition_of_chi}) can be used to check if
a strategy is extortionate. The conditions correspond to:

\begin{align}
    p_1 & = \frac{(R-P)(p_2 + p_3) - R + T + S - P}{S + T - 2P}
     \label{eqn:condition_for_p1}\\
    p_4 & = 0 \label{eqn:condition_for_p4}\\
    1 & > p_2 + p_3\label{eqn:condition_for_chi}
\end{align}

The algebraic steps necessary to prove these results are available in the
supporting materials.

All extortionate strategies reside on a triangular (\ref{eqn:condition_for_chi})
plane (\ref{eqn:condition_for_p1}) in 3 dimensions (\ref{eqn:condition_for_p4}).
Using this formulation it can be seen that a necessary (but not sufficient)
condition for an extortionate strategy is that it cooperates on average less
than 50\% of the time when in a state of disagreement with the opponent.

As an example, consider the known extortionate strategy \(p=(8 / 9, 1 / 2, 1 /
3, 0)\) from~\cite{Stewart2012} which is referred to as \texttt{Extort-2}. In
this case, for the standard values of \((R, T, S, P)\) constraint
(\ref{eqn:condition_for_p1}) corresponds to:

\begin{equation}
    p_1 = \frac{2(p_2 + p_3) + 1}{3}
\end{equation}

It is clear that in this case all constraints hold.

This approach could in fact be used to confirm that a given strategy is acting
in an extortionate manner even if it is not a memory one strategy. However, in
practice, if a closed form for \(p\) is not known, then due to measurement
and/or numerical error this would not work.

This problem can be written in the following linear algebraic form where
\(x=(\alpha, \beta)\)
and \(p^*=(\tilde p_1 - 1, tilde_2 - 1, p_3)\):

\begin{equation}\label{eqn:linear_algebraic_equation_for_p}
    Cx= p^*
\end{equation}

\(C\) corresponds to equations
(\ref{eqn:condition_for_tilde_p1}-\ref{eqn:condition_for_tilde_p3}) and is
given by:

\begin{equation}\label{eqn:definition_of_C}
    C =
    \begin{bmatrix}
        R - P & R- P \\
        S - P & T- P \\
        T - P & S- P \\
    \end{bmatrix}
\end{equation}

Note that in general, equation (\ref{eqn:linear_algebraic_equation_for_p}) will
not necessarily have a solution. From the Rouch\'{e}-Capelli theorem if there is
a solution it is unique as \(\text{rank}(C)=2\) which is the dimension of the
variable \(x\). The best fitting \(x\) is found by minimizing:

\begin{equation}\label{eqn:r_squared}
    \text{SSError} = \|C x- p^*\|_2^2 = \sum_{i=1}^{3}\left((C\bar x)_i-p_i^*\right)^2
\end{equation}

Note that \(\text{SSError}\), which is the square of the Frobenius
norm~\cite{Golub2013}, becomes a measure of how close a strategy is to being an
extortionate strategy. Suspicion
of extortion then corresponds to a threshold on \(\text{SSError}\).

By observing interactions (human or otherwise), their memory one representation
can be inferred and this approach can be used to recognise extortionate
behaviour. The notion of comparing theoretic and actual plays of the IPD is not
novel, see for example~\cite{Rand2013}. Immediately it is noted that if the
environment is noisy~\cite{Wu1995} then no strategy can be considered to be
extortionate as \(p_4>0\).

In the next section, this idea will be illustrated by observing the interactions
that take place in a computer based tournament of the IPD\@.

\section{Numerical experiments}\label{sec:numerical-experiments}

In~\cite{Stewart2012} results from a tournament with
\input{./assets/tex/number_of_stewart_plotkin_strategies/main.tex} strategies,
was presented with specific consideration given to ZD strategies. This
tournament is reproduced here using the Axelrod-Python
project~\cite{Knight2016}. To obtain a good measure of the corresponding
transition rates for each strategy all matches have been run for
\input{assets/tex/number_of_turns/main.tex} turns and every match has been
repeated \input{assets/tex/number_of_repetitions/main.tex} times. All of this
interaction data is available at~\cite{vincent_knight_2018_1297075}. A good
match between the inferred Markov chain and the state distribution of the actual
interactions has been verified. Data for this is presented in the supplementary
materials.

Figure~\ref{fig:SSError_overall_in_stewart_plotkin} shows the \(\text{SSError}\)
values for all the strategies in the tournament, as reported
in~\cite{Stewart2012} the extortionate strategy (which has an expected
\(\text{SSError}\) approximately 0) gains a large number of wins.

\begin{figure}[!htbp]
    \centering
    \includegraphics[width=.8\textwidth]{./assets/img/SSError_overall_in_stewart_plotkin/main.pdf}
    \caption{\(\text{SSError}\) and state probabilities for the strategies
        of~\cite{Stewart2012}, ordered both by number of wins and overall score.
        Note that \(P(DC)\) is not shown as it corresponds to the transpose of
        \(P(CD)\). Cooperator and Defector are omitted as they do not visit all
        the states.}
    \label{fig:SSError_overall_in_stewart_plotkin}
\end{figure}

Here, the work of~\cite{Stewart2012} is extended by investigating a tournament
with \input{assets/tex/number_of_full_strategies/main.tex}
strategies.

The results of this analysis are shown in
Figure~\ref{fig:SSError_and_probabilities_in_full}. The top ranking strategies
by number of wins seem to be extortionate (but not against all strategies) and
it can be seen that a small sub group of strategies achieve mutual defection.
All the top ranking strategies according to score achieve mutual cooperation and
do not extort each other, however they
\textbf{do} exhibit extortionate behaviour towards a number of the lower ranking
strategies.

\begin{figure}[!htbp]
    \centering
    \includegraphics[width=.8\textwidth]{./assets/img/SSError_and_probabilities_in_full/main.pdf}
    \caption{\(\text{SSError}\) for the strategies for the full tournament. Only
    strategy interactions for which \(p_4=0\) and \(\chi>1\) are displayed.}
    \label{fig:SSError_and_probabilities_in_full}
\end{figure}

\section{Conclusion}\label{sec:conclusion}

This work defines an approach to measure whether or not a player is playing a
strategy that corresponds to an extortionate strategy as defined
in~\cite{Press2012}: a mathematical model for suspicion. Indeed, all
extortionate strategies have been
 classified as lying on a triangular plane.
This rigorous classification fails to be robust to small measurement error, thus
a statistical approach is proposed.
This is done through a linear algebraic approach for approximating the solution
of a linear system. Using this, a large number of pairwise interactions is
simulated and in fact very few strategies are found to act extortionately.

The work of~\cite{Press2012}, whilst showing that a clever approach to taking
advantage of another memory one strategy exists: this is incomplete. Whilst the
elegance of this result is very attractive, just as the simplicity of the
victory of Tit For Tat in Axelrod's original tournaments was, it is incomplete.
Extortionate strategies achieve a high number of wins but they do not
achieve a high score which corresponds to the fitness landscape in an
evolutionary sense. From the large number of interactions a payoff matrix \(S\)
can be measured where \(S_{ij}\) denotes the score (using standard values of
\((R, S, T, P) = (3, 0, 5, 1)\)) of the \(i\)th strategy
against the \(j\)th strategy. Using this, the replicator equation
describes the evolution of the system based on a population density fitness
function:

\begin{equation}\label{eqn:replicator_dynamics}
    \frac{dx}{dt} = x(S-x^TS x)
\end{equation}

Equation (\ref{eqn:replicator_dynamics}) is solved numerically through an
integration technique described in~\cite{Petzold1983} and
Figure~\ref{fig:replicator_dynamics} shows the evolution of the distribution of
the system: the various strategies are ranked by scores. It is clear to see that
only the high ranking strategies survive the evolutionary process (in fact,
only \input{./assets/img/replicator_dynamics/main.tex}
have a final distribution greater than \(10 ^ {-2}\)). This confirms the
findings of~\cite{Moran1707} in which sophisticated strategies resist
evolutionary invasion of shorter memory strategies. Recalling
Figure~\ref{fig:SSError_and_probabilities_in_full} this demonstrates that:

\begin{itemize}
    \item Cooperation emerges through the evolutionary process: the high scoring
        strategies do not exhibit extortionate behaviour towards each other.
    \item Extortionate strategies do not survive the evolutionary process.
\end{itemize}

\begin{figure}[!htbp]
    \centering
    \includegraphics[width=.8\textwidth]{./assets/img/replicator_dynamics/main.pdf}
    \caption{Numerical simulation of the replicator equation
    (\ref{eqn:replicator_dynamics}): strategies are ordered by score, only the strategies with a high score survive the evolutionary process.}
    \label{fig:replicator_dynamics}
\end{figure}

This work can be used to classify plays of the IPD\@: data can be collected from
actual interactions (in lab or in the field). Furthermore, this allows for a
classification method similar to the notion of fingerprinting presented
in~\cite{Ashlock2008}. Trained strategies can potentially be classified as
extortionate or not or it could be possible to even constrain the reinforcement
learning approaches that are becoming prevalent in the literature.
Alternatively, this mathematical approach for recognising extortion could be
used in sophisticated strategies to defend against invasion. Arguably, some of
the strategies considered here exhibit this behaviour, indeed as described
in~\cite{Harper2017}, the top ranking strategies in the full tournament are
obtained using evolutionary reinforcement learning techniques, thus, suspicion
of extortionate behaviour could in fact be an evolutionary trait.

\section*{Acknowledgements}

The following open source software libraries were used in this research:

\begin{itemize}
    \item The Axelrod ~\cite{Knight2016, Knight2018} library (IPD strategies and
        tournaments).
    \item The sympy library~\cite{Meurer2017} (verification of all symbolic
        calculations).
    \item The matplotlib~\cite{Droettboom2018} library (visualisation).
    \item The pandas~\cite{Structures2010}, dask~\cite{Dask2016} and
        NumPy~\cite{Oliphant2015} libraries (data manipulation).
    \item The SciPy~\cite{Jones2001} library (numerical integration of the
        replicator equation).
\end{itemize}

This work was performed using the computational facilities of the Advanced
Research Computing @ Cardiff (ARCCA) Division, Cardiff University.

\printbibliography

\newpage
\section*{Supplementary materials}

\includepdf{assets/pdf/proof_of_form_of_extortionate_strategies/main.pdf}

\newpage

Using the pair wise interactions the transition rates \(p,
q\) can be measured and the steady state probabilities inferred and compared to
the actual probabilities of each state.
This is done numerically by computing the singular eigenvector of the
matrix \(A\) \cite{Stewart2009}:

\[
    A =
    \begin{bmatrix}
        p_1 q_1 & p_1 (1 - q_1) & (1 - p_1) q_1 & (1 -p_1) (1 - q_1) \\
        p_2 q_2 & p_2 (1 - q_2) & (1 - p_2) q_2 & (1 -p_2) (1 - q_2) \\
        p_3 q_3 & p_3 (1 - q_3) & (1 - p_3) q_3 & (1 -p_3) (1 - q_3) \\
        p_4 q_4 & p_4 (1 - q_4) & (1 - p_4) q_4 & (1 -p_4) (1 - q_4) \\
    \end{bmatrix}
\]

Figure~\ref{fig:computed_probabilities_vs_theoretic_probabilities} shows a
regression line fitted to every pairwise interaction with a reported
\(\text{SSError}\) value (pairwise interactions with missing states were
omitted). This serves to validate the approach: a part from some edge cases the
relationship is consistent.

\begin{figure}[!htbp]
    \centering
    \includegraphics[width=.8\textwidth]{./assets/img/computed_probabilities_vs_theoretic_probabilities/main.pdf}
    \caption{The
        relationship between the steady state probabilities inferred from the
        measured transitions and the actual steady state probabilities. A linear
        regression line is included validating the approach.}
    \label{fig:computed_probabilities_vs_theoretic_probabilities}
\end{figure}


\end{document}

strategies.

The results of this analysis are shown in
Figure~\ref{fig:SSError_and_probabilities_in_full}. The top ranking strategies
by number of wins seem to be extortionate (but not against all strategies) and
it can be seen that a small sub group of strategies achieve mutual defection.
All the top ranking strategies according to score achieve mutual cooperation and
do not extort each other, however they
\textbf{do} exhibit extortionate behaviour towards a number of the lower ranking
strategies.

\begin{figure}[!htbp]
    \centering
    \includegraphics[width=.8\textwidth]{./assets/img/SSError_and_probabilities_in_full/main.pdf}
    \caption{\(\text{SSError}\) for the strategies for the full tournament. Only
    strategy interactions for which \(p_4=0\) and \(\chi>1\) are displayed.}
    \label{fig:SSError_and_probabilities_in_full}
\end{figure}

\section{Conclusion}\label{sec:conclusion}

This work defines an approach to measure whether or not a player is playing a
strategy that corresponds to an extortionate strategy as defined
in~\cite{Press2012}: a mathematical model for suspicion. Indeed, all
extortionate strategies have been
 classified as lying on a triangular plane.
This rigorous classification fails to be robust to small measurement error, thus
a statistical approach is proposed.
This is done through a linear algebraic approach for approximating the solution
of a linear system. Using this, a large number of pairwise interactions is
simulated and in fact very few strategies are found to act extortionately.

The work of~\cite{Press2012}, whilst showing that a clever approach to taking
advantage of another memory one strategy exists: this is incomplete. Whilst the
elegance of this result is very attractive, just as the simplicity of the
victory of Tit For Tat in Axelrod's original tournaments was, it is incomplete.
Extortionate strategies achieve a high number of wins but they do not
achieve a high score which corresponds to the fitness landscape in an
evolutionary sense. From the large number of interactions a payoff matrix \(S\)
can be measured where \(S_{ij}\) denotes the score (using standard values of
\((R, S, T, P) = (3, 0, 5, 1)\)) of the \(i\)th strategy
against the \(j\)th strategy. Using this, the replicator equation
describes the evolution of the system based on a population density fitness
function:

\begin{equation}\label{eqn:replicator_dynamics}
    \frac{dx}{dt} = x(S-x^TS x)
\end{equation}

Equation (\ref{eqn:replicator_dynamics}) is solved numerically through an
integration technique described in~\cite{Petzold1983} and
Figure~\ref{fig:replicator_dynamics} shows the evolution of the distribution of
the system: the various strategies are ranked by scores. It is clear to see that
only the high ranking strategies survive the evolutionary process (in fact,
only \documentclass[a4paper]{article}

\usepackage{amsmath}
\usepackage{amssymb}
\usepackage[margin=1.5cm,
            includefoot,
            footskip=30pt]{geometry}
\usepackage{layout}
\usepackage{graphicx}
\usepackage{subcaption}

\usepackage{biblatex}
\usepackage{pdfpages}

\bibliography{main.bib}

\title{Suspicion: Recognising and evaluating the effectiveness
       of extortion in the Iterated Prisoner's Dilemma}
\author{Vincent A. Knight \and Nikoleta E. Glynatsi}
\date{\today}



\begin{document}

\maketitle

\begin{abstract}
    The Iterated Prisoner's Dilemma is a model for rational and evolutionary
    interactive behaviour. It has applications both in the study of human social
    behaviour as well as in biology.
    It is used to understand when and how a rational individual might
    accept an immediate cost to their own utility for the direct benefit of
    another.

    Much attention has been given to a class of strategies called
    Zero Determinant strategies. It has been theoretically shown that these
    strategies can ``extort'' any player.

    In this work, an approach to identify if observed strategies are playing in
    an extortionate way is described. Furthermore, experimental analysis of
    a large tournament with \input{assets/tex/number_of_full_strategies/main.tex}
    strategies is considered. In this setting
    the most highly performing strategies do not play in an extortionate way
    against each other but do against lower performing strategies.
    This suggests that whilst the theory of Zero Determinant strategies
    indicates that memory is not of fundamental importance to the evolution of
    cooperative behaviour, this is incomplete.
\end{abstract}

\section{Introduction}\label{sec:introduction}

Agent based game theoretic models have become a stalwart of the underpinning
mathematics of interactive behaviours. One of the major pieces of work
in this area is the pair of original computer tournaments run by Robert
Axelrod~\cite{Axelrod1980, Axelrod1980a}. These tournaments pitted submitted
computer strategies against each other in plays of the Iterated Prisoner's
Dilemma. A common game where agents can choose to pay a slight cost to their
immediate utility in the hope of building a reputation. This has been used in
economic and evolutionary game theory to understand the evolution of cooperative
behaviour.

Recently, a class of strategies was described in~\cite{Press2012} that can
provably extort any given opponent. In~\cite{Hilbe2013, Moran1707} some
questions have already been asked about the true effectiveness of these
strategies in an evolutionary setting. Here another question is asked: is it
possible to recognise this extortionate behaviour? A mathematical procedure for
suspicion is presented: in the same way that the continued actions of an
extortionate individual might raise suspicion.

This work makes use of the Axelrod Python library~\cite{Knight2018, Knight2016}
with a large number of Prisoner Dilemma strategies available to give an
extensive numerical example of the ideas presented.  The approach is presented
in Section~\ref{sec:delta-zd-strategies}.  All of the code and data discussed
in Section~\ref{sec:numerical-experiments} is open sourced, archived and
written according to best scientific principles~\cite{Wilson2014}. The data
archive can be found at~\cite{vincent_knight_2018_1297075}.

\section{Recognising Extortion}\label{sec:delta-zd-strategies}

In~\cite{Press2012}, given a match between 2 memory-one strategies, the concept
of Zero Determinant (ZD) strategies is introduced. The main result of that paper
shows that given two memory one players \(p, q\in\mathbb{R}^4\) a linear
relationship between the players' scores could be forced by one of the players.

Using the notation of~\cite{Press2012}, assuming the utilities for player \(p\)
are given by \(S_x=(R, S, T, P)\) and for player \(q\) by \(S_y=(R, T, S, P)\)
and that the stationary scores of each player is given by \(S_X\) and \(S_Y\)
respectively. The main result of~\cite{Press2012} is that if

\begin{equation}\label{eqn:linear_relationship_for_p}
    \tilde p=\alpha S_x + \beta S_y + \gamma
\end{equation}

or

\begin{equation}\label{eqn:linear_relationship_for_q}
    \tilde q=\alpha S_x + \beta S_y + \gamma
\end{equation}

where \(\tilde p = (1 - p_1, 1 - p_2, p_3, p_4)\) and
\(\tilde q = (1 - q_1, 1 - q_2, q_3, q_4)\) then:

\begin{equation}
    \alpha S_X + \beta S_Y + \gamma = 0
\end{equation}

In~\cite{Press2012} a particular type of ZD strategy is defined: extortionate
strategies. If:

\begin{equation}\label{eqn:constraint_for_extortion}
    \gamma = - P(\alpha + \beta)
\end{equation}

then the player can ensure they get a score \(\chi\) times
larger than the opponent. This extortion coefficient is given by:

\begin{equation}\label{eqn:definition_of_chi}
    \chi=\frac{-\beta}{\alpha}
\end{equation}

Thus, if (\ref{eqn:constraint_for_extortion}) holds and \(\chi >1\) a player is
said to extort their opponent.
Here, the reverse problem is considered: given a
\(p\in\mathbb{R}^4\) how does one identify \(\alpha, \beta\) if they
exist and is the strategy in fact acting in an extortionate way?

These conditions correspond to:

\begin{align}
    \tilde p_1 & = \alpha R + \beta R - P (\alpha + \beta)
            \label{eqn:condition_for_tilde_p1}\\
    \tilde p_2 & = \alpha S + \beta T - P (\alpha + \beta)
            \label{eqn:condition_for_tilde_p2}\\
    \tilde p_3 & = \alpha T + \beta S - P (\alpha + \beta)
            \label{eqn:condition_for_tilde_p3}\\
    \tilde p_4 & = \alpha P + \beta P - P (\alpha + \beta)
            \label{eqn:condition_for_tilde_p4}
\end{align}

Equation (\ref{eqn:condition_for_tilde_p4}) ensures that \(p_4=\tilde p_4=0\).
Equations (\ref{eqn:condition_for_tilde_p1}-\ref{eqn:condition_for_tilde_p3})
can be used to eliminate \(\alpha, \beta\), giving:

\begin{equation}\label{eqn:planar_definition_of_extortion}
    \tilde p_1 = \frac{(R - P)(\tilde p_2 + \tilde p_3)}{S + T - 2P}
\end{equation}

with:

\begin{equation}\label{eqn:definition_of_chi}
    \chi = \frac{\tilde p_2 (P - T) + \tilde p_3 (S - P)}
                {\tilde p_2 (P - S) + \tilde p_3 (T - P)}
\end{equation}

Given a strategy \(p\in\mathbb{R}^{4\times 1}\) equations
(\ref{eqn:condition_for_tilde_p4}), (\ref{eqn:planar_definition_of_extortion}-\ref{eqn:definition_of_chi}) can be used to check if
a strategy is extortionate. The conditions correspond to:

\begin{align}
    p_1 & = \frac{(R-P)(p_2 + p_3) - R + T + S - P}{S + T - 2P}
     \label{eqn:condition_for_p1}\\
    p_4 & = 0 \label{eqn:condition_for_p4}\\
    1 & > p_2 + p_3\label{eqn:condition_for_chi}
\end{align}

The algebraic steps necessary to prove these results are available in the
supporting materials.

All extortionate strategies reside on a triangular (\ref{eqn:condition_for_chi})
plane (\ref{eqn:condition_for_p1}) in 3 dimensions (\ref{eqn:condition_for_p4}).
Using this formulation it can be seen that a necessary (but not sufficient)
condition for an extortionate strategy is that it cooperates on average less
than 50\% of the time when in a state of disagreement with the opponent.

As an example, consider the known extortionate strategy \(p=(8 / 9, 1 / 2, 1 /
3, 0)\) from~\cite{Stewart2012} which is referred to as \texttt{Extort-2}. In
this case, for the standard values of \((R, T, S, P)\) constraint
(\ref{eqn:condition_for_p1}) corresponds to:

\begin{equation}
    p_1 = \frac{2(p_2 + p_3) + 1}{3}
\end{equation}

It is clear that in this case all constraints hold.

This approach could in fact be used to confirm that a given strategy is acting
in an extortionate manner even if it is not a memory one strategy. However, in
practice, if a closed form for \(p\) is not known, then due to measurement
and/or numerical error this would not work.

This problem can be written in the following linear algebraic form where
\(x=(\alpha, \beta)\)
and \(p^*=(\tilde p_1 - 1, tilde_2 - 1, p_3)\):

\begin{equation}\label{eqn:linear_algebraic_equation_for_p}
    Cx= p^*
\end{equation}

\(C\) corresponds to equations
(\ref{eqn:condition_for_tilde_p1}-\ref{eqn:condition_for_tilde_p3}) and is
given by:

\begin{equation}\label{eqn:definition_of_C}
    C =
    \begin{bmatrix}
        R - P & R- P \\
        S - P & T- P \\
        T - P & S- P \\
    \end{bmatrix}
\end{equation}

Note that in general, equation (\ref{eqn:linear_algebraic_equation_for_p}) will
not necessarily have a solution. From the Rouch\'{e}-Capelli theorem if there is
a solution it is unique as \(\text{rank}(C)=2\) which is the dimension of the
variable \(x\). The best fitting \(x\) is found by minimizing:

\begin{equation}\label{eqn:r_squared}
    \text{SSError} = \|C x- p^*\|_2^2 = \sum_{i=1}^{3}\left((C\bar x)_i-p_i^*\right)^2
\end{equation}

Note that \(\text{SSError}\), which is the square of the Frobenius
norm~\cite{Golub2013}, becomes a measure of how close a strategy is to being an
extortionate strategy. Suspicion
of extortion then corresponds to a threshold on \(\text{SSError}\).

By observing interactions (human or otherwise), their memory one representation
can be inferred and this approach can be used to recognise extortionate
behaviour. The notion of comparing theoretic and actual plays of the IPD is not
novel, see for example~\cite{Rand2013}. Immediately it is noted that if the
environment is noisy~\cite{Wu1995} then no strategy can be considered to be
extortionate as \(p_4>0\).

In the next section, this idea will be illustrated by observing the interactions
that take place in a computer based tournament of the IPD\@.

\section{Numerical experiments}\label{sec:numerical-experiments}

In~\cite{Stewart2012} results from a tournament with
\input{./assets/tex/number_of_stewart_plotkin_strategies/main.tex} strategies,
was presented with specific consideration given to ZD strategies. This
tournament is reproduced here using the Axelrod-Python
project~\cite{Knight2016}. To obtain a good measure of the corresponding
transition rates for each strategy all matches have been run for
\input{assets/tex/number_of_turns/main.tex} turns and every match has been
repeated \input{assets/tex/number_of_repetitions/main.tex} times. All of this
interaction data is available at~\cite{vincent_knight_2018_1297075}. A good
match between the inferred Markov chain and the state distribution of the actual
interactions has been verified. Data for this is presented in the supplementary
materials.

Figure~\ref{fig:SSError_overall_in_stewart_plotkin} shows the \(\text{SSError}\)
values for all the strategies in the tournament, as reported
in~\cite{Stewart2012} the extortionate strategy (which has an expected
\(\text{SSError}\) approximately 0) gains a large number of wins.

\begin{figure}[!htbp]
    \centering
    \includegraphics[width=.8\textwidth]{./assets/img/SSError_overall_in_stewart_plotkin/main.pdf}
    \caption{\(\text{SSError}\) and state probabilities for the strategies
        of~\cite{Stewart2012}, ordered both by number of wins and overall score.
        Note that \(P(DC)\) is not shown as it corresponds to the transpose of
        \(P(CD)\). Cooperator and Defector are omitted as they do not visit all
        the states.}
    \label{fig:SSError_overall_in_stewart_plotkin}
\end{figure}

Here, the work of~\cite{Stewart2012} is extended by investigating a tournament
with \input{assets/tex/number_of_full_strategies/main.tex}
strategies.

The results of this analysis are shown in
Figure~\ref{fig:SSError_and_probabilities_in_full}. The top ranking strategies
by number of wins seem to be extortionate (but not against all strategies) and
it can be seen that a small sub group of strategies achieve mutual defection.
All the top ranking strategies according to score achieve mutual cooperation and
do not extort each other, however they
\textbf{do} exhibit extortionate behaviour towards a number of the lower ranking
strategies.

\begin{figure}[!htbp]
    \centering
    \includegraphics[width=.8\textwidth]{./assets/img/SSError_and_probabilities_in_full/main.pdf}
    \caption{\(\text{SSError}\) for the strategies for the full tournament. Only
    strategy interactions for which \(p_4=0\) and \(\chi>1\) are displayed.}
    \label{fig:SSError_and_probabilities_in_full}
\end{figure}

\section{Conclusion}\label{sec:conclusion}

This work defines an approach to measure whether or not a player is playing a
strategy that corresponds to an extortionate strategy as defined
in~\cite{Press2012}: a mathematical model for suspicion. Indeed, all
extortionate strategies have been
 classified as lying on a triangular plane.
This rigorous classification fails to be robust to small measurement error, thus
a statistical approach is proposed.
This is done through a linear algebraic approach for approximating the solution
of a linear system. Using this, a large number of pairwise interactions is
simulated and in fact very few strategies are found to act extortionately.

The work of~\cite{Press2012}, whilst showing that a clever approach to taking
advantage of another memory one strategy exists: this is incomplete. Whilst the
elegance of this result is very attractive, just as the simplicity of the
victory of Tit For Tat in Axelrod's original tournaments was, it is incomplete.
Extortionate strategies achieve a high number of wins but they do not
achieve a high score which corresponds to the fitness landscape in an
evolutionary sense. From the large number of interactions a payoff matrix \(S\)
can be measured where \(S_{ij}\) denotes the score (using standard values of
\((R, S, T, P) = (3, 0, 5, 1)\)) of the \(i\)th strategy
against the \(j\)th strategy. Using this, the replicator equation
describes the evolution of the system based on a population density fitness
function:

\begin{equation}\label{eqn:replicator_dynamics}
    \frac{dx}{dt} = x(S-x^TS x)
\end{equation}

Equation (\ref{eqn:replicator_dynamics}) is solved numerically through an
integration technique described in~\cite{Petzold1983} and
Figure~\ref{fig:replicator_dynamics} shows the evolution of the distribution of
the system: the various strategies are ranked by scores. It is clear to see that
only the high ranking strategies survive the evolutionary process (in fact,
only \input{./assets/img/replicator_dynamics/main.tex}
have a final distribution greater than \(10 ^ {-2}\)). This confirms the
findings of~\cite{Moran1707} in which sophisticated strategies resist
evolutionary invasion of shorter memory strategies. Recalling
Figure~\ref{fig:SSError_and_probabilities_in_full} this demonstrates that:

\begin{itemize}
    \item Cooperation emerges through the evolutionary process: the high scoring
        strategies do not exhibit extortionate behaviour towards each other.
    \item Extortionate strategies do not survive the evolutionary process.
\end{itemize}

\begin{figure}[!htbp]
    \centering
    \includegraphics[width=.8\textwidth]{./assets/img/replicator_dynamics/main.pdf}
    \caption{Numerical simulation of the replicator equation
    (\ref{eqn:replicator_dynamics}): strategies are ordered by score, only the strategies with a high score survive the evolutionary process.}
    \label{fig:replicator_dynamics}
\end{figure}

This work can be used to classify plays of the IPD\@: data can be collected from
actual interactions (in lab or in the field). Furthermore, this allows for a
classification method similar to the notion of fingerprinting presented
in~\cite{Ashlock2008}. Trained strategies can potentially be classified as
extortionate or not or it could be possible to even constrain the reinforcement
learning approaches that are becoming prevalent in the literature.
Alternatively, this mathematical approach for recognising extortion could be
used in sophisticated strategies to defend against invasion. Arguably, some of
the strategies considered here exhibit this behaviour, indeed as described
in~\cite{Harper2017}, the top ranking strategies in the full tournament are
obtained using evolutionary reinforcement learning techniques, thus, suspicion
of extortionate behaviour could in fact be an evolutionary trait.

\section*{Acknowledgements}

The following open source software libraries were used in this research:

\begin{itemize}
    \item The Axelrod ~\cite{Knight2016, Knight2018} library (IPD strategies and
        tournaments).
    \item The sympy library~\cite{Meurer2017} (verification of all symbolic
        calculations).
    \item The matplotlib~\cite{Droettboom2018} library (visualisation).
    \item The pandas~\cite{Structures2010}, dask~\cite{Dask2016} and
        NumPy~\cite{Oliphant2015} libraries (data manipulation).
    \item The SciPy~\cite{Jones2001} library (numerical integration of the
        replicator equation).
\end{itemize}

This work was performed using the computational facilities of the Advanced
Research Computing @ Cardiff (ARCCA) Division, Cardiff University.

\printbibliography

\newpage
\section*{Supplementary materials}

\includepdf{assets/pdf/proof_of_form_of_extortionate_strategies/main.pdf}

\newpage

Using the pair wise interactions the transition rates \(p,
q\) can be measured and the steady state probabilities inferred and compared to
the actual probabilities of each state.
This is done numerically by computing the singular eigenvector of the
matrix \(A\) \cite{Stewart2009}:

\[
    A =
    \begin{bmatrix}
        p_1 q_1 & p_1 (1 - q_1) & (1 - p_1) q_1 & (1 -p_1) (1 - q_1) \\
        p_2 q_2 & p_2 (1 - q_2) & (1 - p_2) q_2 & (1 -p_2) (1 - q_2) \\
        p_3 q_3 & p_3 (1 - q_3) & (1 - p_3) q_3 & (1 -p_3) (1 - q_3) \\
        p_4 q_4 & p_4 (1 - q_4) & (1 - p_4) q_4 & (1 -p_4) (1 - q_4) \\
    \end{bmatrix}
\]

Figure~\ref{fig:computed_probabilities_vs_theoretic_probabilities} shows a
regression line fitted to every pairwise interaction with a reported
\(\text{SSError}\) value (pairwise interactions with missing states were
omitted). This serves to validate the approach: a part from some edge cases the
relationship is consistent.

\begin{figure}[!htbp]
    \centering
    \includegraphics[width=.8\textwidth]{./assets/img/computed_probabilities_vs_theoretic_probabilities/main.pdf}
    \caption{The
        relationship between the steady state probabilities inferred from the
        measured transitions and the actual steady state probabilities. A linear
        regression line is included validating the approach.}
    \label{fig:computed_probabilities_vs_theoretic_probabilities}
\end{figure}


\end{document}

have a final distribution greater than \(10 ^ {-2}\)). This confirms the
findings of~\cite{Moran1707} in which sophisticated strategies resist
evolutionary invasion of shorter memory strategies. Recalling
Figure~\ref{fig:SSError_and_probabilities_in_full} this demonstrates that:

\begin{itemize}
    \item Cooperation emerges through the evolutionary process: the high scoring
        strategies do not exhibit extortionate behaviour towards each other.
    \item Extortionate strategies do not survive the evolutionary process.
\end{itemize}

\begin{figure}[!htbp]
    \centering
    \includegraphics[width=.8\textwidth]{./assets/img/replicator_dynamics/main.pdf}
    \caption{Numerical simulation of the replicator equation
    (\ref{eqn:replicator_dynamics}): strategies are ordered by score, only the strategies with a high score survive the evolutionary process.}
    \label{fig:replicator_dynamics}
\end{figure}

This work can be used to classify plays of the IPD\@: data can be collected from
actual interactions (in lab or in the field). Furthermore, this allows for a
classification method similar to the notion of fingerprinting presented
in~\cite{Ashlock2008}. Trained strategies can potentially be classified as
extortionate or not or it could be possible to even constrain the reinforcement
learning approaches that are becoming prevalent in the literature.
Alternatively, this mathematical approach for recognising extortion could be
used in sophisticated strategies to defend against invasion. Arguably, some of
the strategies considered here exhibit this behaviour, indeed as described
in~\cite{Harper2017}, the top ranking strategies in the full tournament are
obtained using evolutionary reinforcement learning techniques, thus, suspicion
of extortionate behaviour could in fact be an evolutionary trait.

\section*{Acknowledgements}

The following open source software libraries were used in this research:

\begin{itemize}
    \item The Axelrod ~\cite{Knight2016, Knight2018} library (IPD strategies and
        tournaments).
    \item The sympy library~\cite{Meurer2017} (verification of all symbolic
        calculations).
    \item The matplotlib~\cite{Droettboom2018} library (visualisation).
    \item The pandas~\cite{Structures2010}, dask~\cite{Dask2016} and
        NumPy~\cite{Oliphant2015} libraries (data manipulation).
    \item The SciPy~\cite{Jones2001} library (numerical integration of the
        replicator equation).
\end{itemize}

This work was performed using the computational facilities of the Advanced
Research Computing @ Cardiff (ARCCA) Division, Cardiff University.

\printbibliography

\newpage
\section*{Supplementary materials}

\includepdf{assets/pdf/proof_of_form_of_extortionate_strategies/main.pdf}

\newpage

Using the pair wise interactions the transition rates \(p,
q\) can be measured and the steady state probabilities inferred and compared to
the actual probabilities of each state.
This is done numerically by computing the singular eigenvector of the
matrix \(A\) \cite{Stewart2009}:

\[
    A =
    \begin{bmatrix}
        p_1 q_1 & p_1 (1 - q_1) & (1 - p_1) q_1 & (1 -p_1) (1 - q_1) \\
        p_2 q_2 & p_2 (1 - q_2) & (1 - p_2) q_2 & (1 -p_2) (1 - q_2) \\
        p_3 q_3 & p_3 (1 - q_3) & (1 - p_3) q_3 & (1 -p_3) (1 - q_3) \\
        p_4 q_4 & p_4 (1 - q_4) & (1 - p_4) q_4 & (1 -p_4) (1 - q_4) \\
    \end{bmatrix}
\]

Figure~\ref{fig:computed_probabilities_vs_theoretic_probabilities} shows a
regression line fitted to every pairwise interaction with a reported
\(\text{SSError}\) value (pairwise interactions with missing states were
omitted). This serves to validate the approach: a part from some edge cases the
relationship is consistent.

\begin{figure}[!htbp]
    \centering
    \includegraphics[width=.8\textwidth]{./assets/img/computed_probabilities_vs_theoretic_probabilities/main.pdf}
    \caption{The
        relationship between the steady state probabilities inferred from the
        measured transitions and the actual steady state probabilities. A linear
        regression line is included validating the approach.}
    \label{fig:computed_probabilities_vs_theoretic_probabilities}
\end{figure}


\end{document}

strategies.

The results of this analysis are shown in
Figure~\ref{fig:SSError_and_probabilities_in_full}. The top ranking strategies
by number of wins seem to be extortionate (but not against all strategies) and
it can be seen that a small sub group of strategies achieve mutual defection.
All the top ranking strategies according to score achieve mutual cooperation and
do not extort each other, however they
\textbf{do} exhibit extortionate behaviour towards a number of the lower ranking
strategies.

\begin{figure}[!htbp]
    \centering
    \includegraphics[width=.8\textwidth]{./assets/img/SSError_and_probabilities_in_full/main.pdf}
    \caption{\(\text{SSError}\) for the strategies for the full tournament. Only
    strategy interactions for which \(p_4=0\) and \(\chi>1\) are displayed.}
    \label{fig:SSError_and_probabilities_in_full}
\end{figure}

\section{Conclusion}\label{sec:conclusion}

This work defines an approach to measure whether or not a player is playing a
strategy that corresponds to an extortionate strategy as defined
in~\cite{Press2012}: a mathematical model for suspicion. Indeed, all
extortionate strategies have been
 classified as lying on a triangular plane.
This rigorous classification fails to be robust to small measurement error, thus
a statistical approach is proposed.
This is done through a linear algebraic approach for approximating the solution
of a linear system. Using this, a large number of pairwise interactions is
simulated and in fact very few strategies are found to act extortionately.

The work of~\cite{Press2012}, whilst showing that a clever approach to taking
advantage of another memory one strategy exists: this is incomplete. Whilst the
elegance of this result is very attractive, just as the simplicity of the
victory of Tit For Tat in Axelrod's original tournaments was, it is incomplete.
Extortionate strategies achieve a high number of wins but they do not
achieve a high score which corresponds to the fitness landscape in an
evolutionary sense. From the large number of interactions a payoff matrix \(S\)
can be measured where \(S_{ij}\) denotes the score (using standard values of
\((R, S, T, P) = (3, 0, 5, 1)\)) of the \(i\)th strategy
against the \(j\)th strategy. Using this, the replicator equation
describes the evolution of the system based on a population density fitness
function:

\begin{equation}\label{eqn:replicator_dynamics}
    \frac{dx}{dt} = x(S-x^TS x)
\end{equation}

Equation (\ref{eqn:replicator_dynamics}) is solved numerically through an
integration technique described in~\cite{Petzold1983} and
Figure~\ref{fig:replicator_dynamics} shows the evolution of the distribution of
the system: the various strategies are ranked by scores. It is clear to see that
only the high ranking strategies survive the evolutionary process (in fact,
only \documentclass[a4paper]{article}

\usepackage{amsmath}
\usepackage{amssymb}
\usepackage[margin=1.5cm,
            includefoot,
            footskip=30pt]{geometry}
\usepackage{layout}
\usepackage{graphicx}
\usepackage{subcaption}

\usepackage{biblatex}
\usepackage{pdfpages}

\bibliography{main.bib}

\title{Suspicion: Recognising and evaluating the effectiveness
       of extortion in the Iterated Prisoner's Dilemma}
\author{Vincent A. Knight \and Nikoleta E. Glynatsi}
\date{\today}



\begin{document}

\maketitle

\begin{abstract}
    The Iterated Prisoner's Dilemma is a model for rational and evolutionary
    interactive behaviour. It has applications both in the study of human social
    behaviour as well as in biology.
    It is used to understand when and how a rational individual might
    accept an immediate cost to their own utility for the direct benefit of
    another.

    Much attention has been given to a class of strategies called
    Zero Determinant strategies. It has been theoretically shown that these
    strategies can ``extort'' any player.

    In this work, an approach to identify if observed strategies are playing in
    an extortionate way is described. Furthermore, experimental analysis of
    a large tournament with \documentclass[a4paper]{article}

\usepackage{amsmath}
\usepackage{amssymb}
\usepackage[margin=1.5cm,
            includefoot,
            footskip=30pt]{geometry}
\usepackage{layout}
\usepackage{graphicx}
\usepackage{subcaption}

\usepackage{biblatex}
\usepackage{pdfpages}

\bibliography{main.bib}

\title{Suspicion: Recognising and evaluating the effectiveness
       of extortion in the Iterated Prisoner's Dilemma}
\author{Vincent A. Knight \and Nikoleta E. Glynatsi}
\date{\today}



\begin{document}

\maketitle

\begin{abstract}
    The Iterated Prisoner's Dilemma is a model for rational and evolutionary
    interactive behaviour. It has applications both in the study of human social
    behaviour as well as in biology.
    It is used to understand when and how a rational individual might
    accept an immediate cost to their own utility for the direct benefit of
    another.

    Much attention has been given to a class of strategies called
    Zero Determinant strategies. It has been theoretically shown that these
    strategies can ``extort'' any player.

    In this work, an approach to identify if observed strategies are playing in
    an extortionate way is described. Furthermore, experimental analysis of
    a large tournament with \input{assets/tex/number_of_full_strategies/main.tex}
    strategies is considered. In this setting
    the most highly performing strategies do not play in an extortionate way
    against each other but do against lower performing strategies.
    This suggests that whilst the theory of Zero Determinant strategies
    indicates that memory is not of fundamental importance to the evolution of
    cooperative behaviour, this is incomplete.
\end{abstract}

\section{Introduction}\label{sec:introduction}

Agent based game theoretic models have become a stalwart of the underpinning
mathematics of interactive behaviours. One of the major pieces of work
in this area is the pair of original computer tournaments run by Robert
Axelrod~\cite{Axelrod1980, Axelrod1980a}. These tournaments pitted submitted
computer strategies against each other in plays of the Iterated Prisoner's
Dilemma. A common game where agents can choose to pay a slight cost to their
immediate utility in the hope of building a reputation. This has been used in
economic and evolutionary game theory to understand the evolution of cooperative
behaviour.

Recently, a class of strategies was described in~\cite{Press2012} that can
provably extort any given opponent. In~\cite{Hilbe2013, Moran1707} some
questions have already been asked about the true effectiveness of these
strategies in an evolutionary setting. Here another question is asked: is it
possible to recognise this extortionate behaviour? A mathematical procedure for
suspicion is presented: in the same way that the continued actions of an
extortionate individual might raise suspicion.

This work makes use of the Axelrod Python library~\cite{Knight2018, Knight2016}
with a large number of Prisoner Dilemma strategies available to give an
extensive numerical example of the ideas presented.  The approach is presented
in Section~\ref{sec:delta-zd-strategies}.  All of the code and data discussed
in Section~\ref{sec:numerical-experiments} is open sourced, archived and
written according to best scientific principles~\cite{Wilson2014}. The data
archive can be found at~\cite{vincent_knight_2018_1297075}.

\section{Recognising Extortion}\label{sec:delta-zd-strategies}

In~\cite{Press2012}, given a match between 2 memory-one strategies, the concept
of Zero Determinant (ZD) strategies is introduced. The main result of that paper
shows that given two memory one players \(p, q\in\mathbb{R}^4\) a linear
relationship between the players' scores could be forced by one of the players.

Using the notation of~\cite{Press2012}, assuming the utilities for player \(p\)
are given by \(S_x=(R, S, T, P)\) and for player \(q\) by \(S_y=(R, T, S, P)\)
and that the stationary scores of each player is given by \(S_X\) and \(S_Y\)
respectively. The main result of~\cite{Press2012} is that if

\begin{equation}\label{eqn:linear_relationship_for_p}
    \tilde p=\alpha S_x + \beta S_y + \gamma
\end{equation}

or

\begin{equation}\label{eqn:linear_relationship_for_q}
    \tilde q=\alpha S_x + \beta S_y + \gamma
\end{equation}

where \(\tilde p = (1 - p_1, 1 - p_2, p_3, p_4)\) and
\(\tilde q = (1 - q_1, 1 - q_2, q_3, q_4)\) then:

\begin{equation}
    \alpha S_X + \beta S_Y + \gamma = 0
\end{equation}

In~\cite{Press2012} a particular type of ZD strategy is defined: extortionate
strategies. If:

\begin{equation}\label{eqn:constraint_for_extortion}
    \gamma = - P(\alpha + \beta)
\end{equation}

then the player can ensure they get a score \(\chi\) times
larger than the opponent. This extortion coefficient is given by:

\begin{equation}\label{eqn:definition_of_chi}
    \chi=\frac{-\beta}{\alpha}
\end{equation}

Thus, if (\ref{eqn:constraint_for_extortion}) holds and \(\chi >1\) a player is
said to extort their opponent.
Here, the reverse problem is considered: given a
\(p\in\mathbb{R}^4\) how does one identify \(\alpha, \beta\) if they
exist and is the strategy in fact acting in an extortionate way?

These conditions correspond to:

\begin{align}
    \tilde p_1 & = \alpha R + \beta R - P (\alpha + \beta)
            \label{eqn:condition_for_tilde_p1}\\
    \tilde p_2 & = \alpha S + \beta T - P (\alpha + \beta)
            \label{eqn:condition_for_tilde_p2}\\
    \tilde p_3 & = \alpha T + \beta S - P (\alpha + \beta)
            \label{eqn:condition_for_tilde_p3}\\
    \tilde p_4 & = \alpha P + \beta P - P (\alpha + \beta)
            \label{eqn:condition_for_tilde_p4}
\end{align}

Equation (\ref{eqn:condition_for_tilde_p4}) ensures that \(p_4=\tilde p_4=0\).
Equations (\ref{eqn:condition_for_tilde_p1}-\ref{eqn:condition_for_tilde_p3})
can be used to eliminate \(\alpha, \beta\), giving:

\begin{equation}\label{eqn:planar_definition_of_extortion}
    \tilde p_1 = \frac{(R - P)(\tilde p_2 + \tilde p_3)}{S + T - 2P}
\end{equation}

with:

\begin{equation}\label{eqn:definition_of_chi}
    \chi = \frac{\tilde p_2 (P - T) + \tilde p_3 (S - P)}
                {\tilde p_2 (P - S) + \tilde p_3 (T - P)}
\end{equation}

Given a strategy \(p\in\mathbb{R}^{4\times 1}\) equations
(\ref{eqn:condition_for_tilde_p4}), (\ref{eqn:planar_definition_of_extortion}-\ref{eqn:definition_of_chi}) can be used to check if
a strategy is extortionate. The conditions correspond to:

\begin{align}
    p_1 & = \frac{(R-P)(p_2 + p_3) - R + T + S - P}{S + T - 2P}
     \label{eqn:condition_for_p1}\\
    p_4 & = 0 \label{eqn:condition_for_p4}\\
    1 & > p_2 + p_3\label{eqn:condition_for_chi}
\end{align}

The algebraic steps necessary to prove these results are available in the
supporting materials.

All extortionate strategies reside on a triangular (\ref{eqn:condition_for_chi})
plane (\ref{eqn:condition_for_p1}) in 3 dimensions (\ref{eqn:condition_for_p4}).
Using this formulation it can be seen that a necessary (but not sufficient)
condition for an extortionate strategy is that it cooperates on average less
than 50\% of the time when in a state of disagreement with the opponent.

As an example, consider the known extortionate strategy \(p=(8 / 9, 1 / 2, 1 /
3, 0)\) from~\cite{Stewart2012} which is referred to as \texttt{Extort-2}. In
this case, for the standard values of \((R, T, S, P)\) constraint
(\ref{eqn:condition_for_p1}) corresponds to:

\begin{equation}
    p_1 = \frac{2(p_2 + p_3) + 1}{3}
\end{equation}

It is clear that in this case all constraints hold.

This approach could in fact be used to confirm that a given strategy is acting
in an extortionate manner even if it is not a memory one strategy. However, in
practice, if a closed form for \(p\) is not known, then due to measurement
and/or numerical error this would not work.

This problem can be written in the following linear algebraic form where
\(x=(\alpha, \beta)\)
and \(p^*=(\tilde p_1 - 1, tilde_2 - 1, p_3)\):

\begin{equation}\label{eqn:linear_algebraic_equation_for_p}
    Cx= p^*
\end{equation}

\(C\) corresponds to equations
(\ref{eqn:condition_for_tilde_p1}-\ref{eqn:condition_for_tilde_p3}) and is
given by:

\begin{equation}\label{eqn:definition_of_C}
    C =
    \begin{bmatrix}
        R - P & R- P \\
        S - P & T- P \\
        T - P & S- P \\
    \end{bmatrix}
\end{equation}

Note that in general, equation (\ref{eqn:linear_algebraic_equation_for_p}) will
not necessarily have a solution. From the Rouch\'{e}-Capelli theorem if there is
a solution it is unique as \(\text{rank}(C)=2\) which is the dimension of the
variable \(x\). The best fitting \(x\) is found by minimizing:

\begin{equation}\label{eqn:r_squared}
    \text{SSError} = \|C x- p^*\|_2^2 = \sum_{i=1}^{3}\left((C\bar x)_i-p_i^*\right)^2
\end{equation}

Note that \(\text{SSError}\), which is the square of the Frobenius
norm~\cite{Golub2013}, becomes a measure of how close a strategy is to being an
extortionate strategy. Suspicion
of extortion then corresponds to a threshold on \(\text{SSError}\).

By observing interactions (human or otherwise), their memory one representation
can be inferred and this approach can be used to recognise extortionate
behaviour. The notion of comparing theoretic and actual plays of the IPD is not
novel, see for example~\cite{Rand2013}. Immediately it is noted that if the
environment is noisy~\cite{Wu1995} then no strategy can be considered to be
extortionate as \(p_4>0\).

In the next section, this idea will be illustrated by observing the interactions
that take place in a computer based tournament of the IPD\@.

\section{Numerical experiments}\label{sec:numerical-experiments}

In~\cite{Stewart2012} results from a tournament with
\input{./assets/tex/number_of_stewart_plotkin_strategies/main.tex} strategies,
was presented with specific consideration given to ZD strategies. This
tournament is reproduced here using the Axelrod-Python
project~\cite{Knight2016}. To obtain a good measure of the corresponding
transition rates for each strategy all matches have been run for
\input{assets/tex/number_of_turns/main.tex} turns and every match has been
repeated \input{assets/tex/number_of_repetitions/main.tex} times. All of this
interaction data is available at~\cite{vincent_knight_2018_1297075}. A good
match between the inferred Markov chain and the state distribution of the actual
interactions has been verified. Data for this is presented in the supplementary
materials.

Figure~\ref{fig:SSError_overall_in_stewart_plotkin} shows the \(\text{SSError}\)
values for all the strategies in the tournament, as reported
in~\cite{Stewart2012} the extortionate strategy (which has an expected
\(\text{SSError}\) approximately 0) gains a large number of wins.

\begin{figure}[!htbp]
    \centering
    \includegraphics[width=.8\textwidth]{./assets/img/SSError_overall_in_stewart_plotkin/main.pdf}
    \caption{\(\text{SSError}\) and state probabilities for the strategies
        of~\cite{Stewart2012}, ordered both by number of wins and overall score.
        Note that \(P(DC)\) is not shown as it corresponds to the transpose of
        \(P(CD)\). Cooperator and Defector are omitted as they do not visit all
        the states.}
    \label{fig:SSError_overall_in_stewart_plotkin}
\end{figure}

Here, the work of~\cite{Stewart2012} is extended by investigating a tournament
with \input{assets/tex/number_of_full_strategies/main.tex}
strategies.

The results of this analysis are shown in
Figure~\ref{fig:SSError_and_probabilities_in_full}. The top ranking strategies
by number of wins seem to be extortionate (but not against all strategies) and
it can be seen that a small sub group of strategies achieve mutual defection.
All the top ranking strategies according to score achieve mutual cooperation and
do not extort each other, however they
\textbf{do} exhibit extortionate behaviour towards a number of the lower ranking
strategies.

\begin{figure}[!htbp]
    \centering
    \includegraphics[width=.8\textwidth]{./assets/img/SSError_and_probabilities_in_full/main.pdf}
    \caption{\(\text{SSError}\) for the strategies for the full tournament. Only
    strategy interactions for which \(p_4=0\) and \(\chi>1\) are displayed.}
    \label{fig:SSError_and_probabilities_in_full}
\end{figure}

\section{Conclusion}\label{sec:conclusion}

This work defines an approach to measure whether or not a player is playing a
strategy that corresponds to an extortionate strategy as defined
in~\cite{Press2012}: a mathematical model for suspicion. Indeed, all
extortionate strategies have been
 classified as lying on a triangular plane.
This rigorous classification fails to be robust to small measurement error, thus
a statistical approach is proposed.
This is done through a linear algebraic approach for approximating the solution
of a linear system. Using this, a large number of pairwise interactions is
simulated and in fact very few strategies are found to act extortionately.

The work of~\cite{Press2012}, whilst showing that a clever approach to taking
advantage of another memory one strategy exists: this is incomplete. Whilst the
elegance of this result is very attractive, just as the simplicity of the
victory of Tit For Tat in Axelrod's original tournaments was, it is incomplete.
Extortionate strategies achieve a high number of wins but they do not
achieve a high score which corresponds to the fitness landscape in an
evolutionary sense. From the large number of interactions a payoff matrix \(S\)
can be measured where \(S_{ij}\) denotes the score (using standard values of
\((R, S, T, P) = (3, 0, 5, 1)\)) of the \(i\)th strategy
against the \(j\)th strategy. Using this, the replicator equation
describes the evolution of the system based on a population density fitness
function:

\begin{equation}\label{eqn:replicator_dynamics}
    \frac{dx}{dt} = x(S-x^TS x)
\end{equation}

Equation (\ref{eqn:replicator_dynamics}) is solved numerically through an
integration technique described in~\cite{Petzold1983} and
Figure~\ref{fig:replicator_dynamics} shows the evolution of the distribution of
the system: the various strategies are ranked by scores. It is clear to see that
only the high ranking strategies survive the evolutionary process (in fact,
only \input{./assets/img/replicator_dynamics/main.tex}
have a final distribution greater than \(10 ^ {-2}\)). This confirms the
findings of~\cite{Moran1707} in which sophisticated strategies resist
evolutionary invasion of shorter memory strategies. Recalling
Figure~\ref{fig:SSError_and_probabilities_in_full} this demonstrates that:

\begin{itemize}
    \item Cooperation emerges through the evolutionary process: the high scoring
        strategies do not exhibit extortionate behaviour towards each other.
    \item Extortionate strategies do not survive the evolutionary process.
\end{itemize}

\begin{figure}[!htbp]
    \centering
    \includegraphics[width=.8\textwidth]{./assets/img/replicator_dynamics/main.pdf}
    \caption{Numerical simulation of the replicator equation
    (\ref{eqn:replicator_dynamics}): strategies are ordered by score, only the strategies with a high score survive the evolutionary process.}
    \label{fig:replicator_dynamics}
\end{figure}

This work can be used to classify plays of the IPD\@: data can be collected from
actual interactions (in lab or in the field). Furthermore, this allows for a
classification method similar to the notion of fingerprinting presented
in~\cite{Ashlock2008}. Trained strategies can potentially be classified as
extortionate or not or it could be possible to even constrain the reinforcement
learning approaches that are becoming prevalent in the literature.
Alternatively, this mathematical approach for recognising extortion could be
used in sophisticated strategies to defend against invasion. Arguably, some of
the strategies considered here exhibit this behaviour, indeed as described
in~\cite{Harper2017}, the top ranking strategies in the full tournament are
obtained using evolutionary reinforcement learning techniques, thus, suspicion
of extortionate behaviour could in fact be an evolutionary trait.

\section*{Acknowledgements}

The following open source software libraries were used in this research:

\begin{itemize}
    \item The Axelrod ~\cite{Knight2016, Knight2018} library (IPD strategies and
        tournaments).
    \item The sympy library~\cite{Meurer2017} (verification of all symbolic
        calculations).
    \item The matplotlib~\cite{Droettboom2018} library (visualisation).
    \item The pandas~\cite{Structures2010}, dask~\cite{Dask2016} and
        NumPy~\cite{Oliphant2015} libraries (data manipulation).
    \item The SciPy~\cite{Jones2001} library (numerical integration of the
        replicator equation).
\end{itemize}

This work was performed using the computational facilities of the Advanced
Research Computing @ Cardiff (ARCCA) Division, Cardiff University.

\printbibliography

\newpage
\section*{Supplementary materials}

\includepdf{assets/pdf/proof_of_form_of_extortionate_strategies/main.pdf}

\newpage

Using the pair wise interactions the transition rates \(p,
q\) can be measured and the steady state probabilities inferred and compared to
the actual probabilities of each state.
This is done numerically by computing the singular eigenvector of the
matrix \(A\) \cite{Stewart2009}:

\[
    A =
    \begin{bmatrix}
        p_1 q_1 & p_1 (1 - q_1) & (1 - p_1) q_1 & (1 -p_1) (1 - q_1) \\
        p_2 q_2 & p_2 (1 - q_2) & (1 - p_2) q_2 & (1 -p_2) (1 - q_2) \\
        p_3 q_3 & p_3 (1 - q_3) & (1 - p_3) q_3 & (1 -p_3) (1 - q_3) \\
        p_4 q_4 & p_4 (1 - q_4) & (1 - p_4) q_4 & (1 -p_4) (1 - q_4) \\
    \end{bmatrix}
\]

Figure~\ref{fig:computed_probabilities_vs_theoretic_probabilities} shows a
regression line fitted to every pairwise interaction with a reported
\(\text{SSError}\) value (pairwise interactions with missing states were
omitted). This serves to validate the approach: a part from some edge cases the
relationship is consistent.

\begin{figure}[!htbp]
    \centering
    \includegraphics[width=.8\textwidth]{./assets/img/computed_probabilities_vs_theoretic_probabilities/main.pdf}
    \caption{The
        relationship between the steady state probabilities inferred from the
        measured transitions and the actual steady state probabilities. A linear
        regression line is included validating the approach.}
    \label{fig:computed_probabilities_vs_theoretic_probabilities}
\end{figure}


\end{document}

    strategies is considered. In this setting
    the most highly performing strategies do not play in an extortionate way
    against each other but do against lower performing strategies.
    This suggests that whilst the theory of Zero Determinant strategies
    indicates that memory is not of fundamental importance to the evolution of
    cooperative behaviour, this is incomplete.
\end{abstract}

\section{Introduction}\label{sec:introduction}

Agent based game theoretic models have become a stalwart of the underpinning
mathematics of interactive behaviours. One of the major pieces of work
in this area is the pair of original computer tournaments run by Robert
Axelrod~\cite{Axelrod1980, Axelrod1980a}. These tournaments pitted submitted
computer strategies against each other in plays of the Iterated Prisoner's
Dilemma. A common game where agents can choose to pay a slight cost to their
immediate utility in the hope of building a reputation. This has been used in
economic and evolutionary game theory to understand the evolution of cooperative
behaviour.

Recently, a class of strategies was described in~\cite{Press2012} that can
provably extort any given opponent. In~\cite{Hilbe2013, Moran1707} some
questions have already been asked about the true effectiveness of these
strategies in an evolutionary setting. Here another question is asked: is it
possible to recognise this extortionate behaviour? A mathematical procedure for
suspicion is presented: in the same way that the continued actions of an
extortionate individual might raise suspicion.

This work makes use of the Axelrod Python library~\cite{Knight2018, Knight2016}
with a large number of Prisoner Dilemma strategies available to give an
extensive numerical example of the ideas presented.  The approach is presented
in Section~\ref{sec:delta-zd-strategies}.  All of the code and data discussed
in Section~\ref{sec:numerical-experiments} is open sourced, archived and
written according to best scientific principles~\cite{Wilson2014}. The data
archive can be found at~\cite{vincent_knight_2018_1297075}.

\section{Recognising Extortion}\label{sec:delta-zd-strategies}

In~\cite{Press2012}, given a match between 2 memory-one strategies, the concept
of Zero Determinant (ZD) strategies is introduced. The main result of that paper
shows that given two memory one players \(p, q\in\mathbb{R}^4\) a linear
relationship between the players' scores could be forced by one of the players.

Using the notation of~\cite{Press2012}, assuming the utilities for player \(p\)
are given by \(S_x=(R, S, T, P)\) and for player \(q\) by \(S_y=(R, T, S, P)\)
and that the stationary scores of each player is given by \(S_X\) and \(S_Y\)
respectively. The main result of~\cite{Press2012} is that if

\begin{equation}\label{eqn:linear_relationship_for_p}
    \tilde p=\alpha S_x + \beta S_y + \gamma
\end{equation}

or

\begin{equation}\label{eqn:linear_relationship_for_q}
    \tilde q=\alpha S_x + \beta S_y + \gamma
\end{equation}

where \(\tilde p = (1 - p_1, 1 - p_2, p_3, p_4)\) and
\(\tilde q = (1 - q_1, 1 - q_2, q_3, q_4)\) then:

\begin{equation}
    \alpha S_X + \beta S_Y + \gamma = 0
\end{equation}

In~\cite{Press2012} a particular type of ZD strategy is defined: extortionate
strategies. If:

\begin{equation}\label{eqn:constraint_for_extortion}
    \gamma = - P(\alpha + \beta)
\end{equation}

then the player can ensure they get a score \(\chi\) times
larger than the opponent. This extortion coefficient is given by:

\begin{equation}\label{eqn:definition_of_chi}
    \chi=\frac{-\beta}{\alpha}
\end{equation}

Thus, if (\ref{eqn:constraint_for_extortion}) holds and \(\chi >1\) a player is
said to extort their opponent.
Here, the reverse problem is considered: given a
\(p\in\mathbb{R}^4\) how does one identify \(\alpha, \beta\) if they
exist and is the strategy in fact acting in an extortionate way?

These conditions correspond to:

\begin{align}
    \tilde p_1 & = \alpha R + \beta R - P (\alpha + \beta)
            \label{eqn:condition_for_tilde_p1}\\
    \tilde p_2 & = \alpha S + \beta T - P (\alpha + \beta)
            \label{eqn:condition_for_tilde_p2}\\
    \tilde p_3 & = \alpha T + \beta S - P (\alpha + \beta)
            \label{eqn:condition_for_tilde_p3}\\
    \tilde p_4 & = \alpha P + \beta P - P (\alpha + \beta)
            \label{eqn:condition_for_tilde_p4}
\end{align}

Equation (\ref{eqn:condition_for_tilde_p4}) ensures that \(p_4=\tilde p_4=0\).
Equations (\ref{eqn:condition_for_tilde_p1}-\ref{eqn:condition_for_tilde_p3})
can be used to eliminate \(\alpha, \beta\), giving:

\begin{equation}\label{eqn:planar_definition_of_extortion}
    \tilde p_1 = \frac{(R - P)(\tilde p_2 + \tilde p_3)}{S + T - 2P}
\end{equation}

with:

\begin{equation}\label{eqn:definition_of_chi}
    \chi = \frac{\tilde p_2 (P - T) + \tilde p_3 (S - P)}
                {\tilde p_2 (P - S) + \tilde p_3 (T - P)}
\end{equation}

Given a strategy \(p\in\mathbb{R}^{4\times 1}\) equations
(\ref{eqn:condition_for_tilde_p4}), (\ref{eqn:planar_definition_of_extortion}-\ref{eqn:definition_of_chi}) can be used to check if
a strategy is extortionate. The conditions correspond to:

\begin{align}
    p_1 & = \frac{(R-P)(p_2 + p_3) - R + T + S - P}{S + T - 2P}
     \label{eqn:condition_for_p1}\\
    p_4 & = 0 \label{eqn:condition_for_p4}\\
    1 & > p_2 + p_3\label{eqn:condition_for_chi}
\end{align}

The algebraic steps necessary to prove these results are available in the
supporting materials.

All extortionate strategies reside on a triangular (\ref{eqn:condition_for_chi})
plane (\ref{eqn:condition_for_p1}) in 3 dimensions (\ref{eqn:condition_for_p4}).
Using this formulation it can be seen that a necessary (but not sufficient)
condition for an extortionate strategy is that it cooperates on average less
than 50\% of the time when in a state of disagreement with the opponent.

As an example, consider the known extortionate strategy \(p=(8 / 9, 1 / 2, 1 /
3, 0)\) from~\cite{Stewart2012} which is referred to as \texttt{Extort-2}. In
this case, for the standard values of \((R, T, S, P)\) constraint
(\ref{eqn:condition_for_p1}) corresponds to:

\begin{equation}
    p_1 = \frac{2(p_2 + p_3) + 1}{3}
\end{equation}

It is clear that in this case all constraints hold.

This approach could in fact be used to confirm that a given strategy is acting
in an extortionate manner even if it is not a memory one strategy. However, in
practice, if a closed form for \(p\) is not known, then due to measurement
and/or numerical error this would not work.

This problem can be written in the following linear algebraic form where
\(x=(\alpha, \beta)\)
and \(p^*=(\tilde p_1 - 1, tilde_2 - 1, p_3)\):

\begin{equation}\label{eqn:linear_algebraic_equation_for_p}
    Cx= p^*
\end{equation}

\(C\) corresponds to equations
(\ref{eqn:condition_for_tilde_p1}-\ref{eqn:condition_for_tilde_p3}) and is
given by:

\begin{equation}\label{eqn:definition_of_C}
    C =
    \begin{bmatrix}
        R - P & R- P \\
        S - P & T- P \\
        T - P & S- P \\
    \end{bmatrix}
\end{equation}

Note that in general, equation (\ref{eqn:linear_algebraic_equation_for_p}) will
not necessarily have a solution. From the Rouch\'{e}-Capelli theorem if there is
a solution it is unique as \(\text{rank}(C)=2\) which is the dimension of the
variable \(x\). The best fitting \(x\) is found by minimizing:

\begin{equation}\label{eqn:r_squared}
    \text{SSError} = \|C x- p^*\|_2^2 = \sum_{i=1}^{3}\left((C\bar x)_i-p_i^*\right)^2
\end{equation}

Note that \(\text{SSError}\), which is the square of the Frobenius
norm~\cite{Golub2013}, becomes a measure of how close a strategy is to being an
extortionate strategy. Suspicion
of extortion then corresponds to a threshold on \(\text{SSError}\).

By observing interactions (human or otherwise), their memory one representation
can be inferred and this approach can be used to recognise extortionate
behaviour. The notion of comparing theoretic and actual plays of the IPD is not
novel, see for example~\cite{Rand2013}. Immediately it is noted that if the
environment is noisy~\cite{Wu1995} then no strategy can be considered to be
extortionate as \(p_4>0\).

In the next section, this idea will be illustrated by observing the interactions
that take place in a computer based tournament of the IPD\@.

\section{Numerical experiments}\label{sec:numerical-experiments}

In~\cite{Stewart2012} results from a tournament with
\documentclass[a4paper]{article}

\usepackage{amsmath}
\usepackage{amssymb}
\usepackage[margin=1.5cm,
            includefoot,
            footskip=30pt]{geometry}
\usepackage{layout}
\usepackage{graphicx}
\usepackage{subcaption}

\usepackage{biblatex}
\usepackage{pdfpages}

\bibliography{main.bib}

\title{Suspicion: Recognising and evaluating the effectiveness
       of extortion in the Iterated Prisoner's Dilemma}
\author{Vincent A. Knight \and Nikoleta E. Glynatsi}
\date{\today}



\begin{document}

\maketitle

\begin{abstract}
    The Iterated Prisoner's Dilemma is a model for rational and evolutionary
    interactive behaviour. It has applications both in the study of human social
    behaviour as well as in biology.
    It is used to understand when and how a rational individual might
    accept an immediate cost to their own utility for the direct benefit of
    another.

    Much attention has been given to a class of strategies called
    Zero Determinant strategies. It has been theoretically shown that these
    strategies can ``extort'' any player.

    In this work, an approach to identify if observed strategies are playing in
    an extortionate way is described. Furthermore, experimental analysis of
    a large tournament with \input{assets/tex/number_of_full_strategies/main.tex}
    strategies is considered. In this setting
    the most highly performing strategies do not play in an extortionate way
    against each other but do against lower performing strategies.
    This suggests that whilst the theory of Zero Determinant strategies
    indicates that memory is not of fundamental importance to the evolution of
    cooperative behaviour, this is incomplete.
\end{abstract}

\section{Introduction}\label{sec:introduction}

Agent based game theoretic models have become a stalwart of the underpinning
mathematics of interactive behaviours. One of the major pieces of work
in this area is the pair of original computer tournaments run by Robert
Axelrod~\cite{Axelrod1980, Axelrod1980a}. These tournaments pitted submitted
computer strategies against each other in plays of the Iterated Prisoner's
Dilemma. A common game where agents can choose to pay a slight cost to their
immediate utility in the hope of building a reputation. This has been used in
economic and evolutionary game theory to understand the evolution of cooperative
behaviour.

Recently, a class of strategies was described in~\cite{Press2012} that can
provably extort any given opponent. In~\cite{Hilbe2013, Moran1707} some
questions have already been asked about the true effectiveness of these
strategies in an evolutionary setting. Here another question is asked: is it
possible to recognise this extortionate behaviour? A mathematical procedure for
suspicion is presented: in the same way that the continued actions of an
extortionate individual might raise suspicion.

This work makes use of the Axelrod Python library~\cite{Knight2018, Knight2016}
with a large number of Prisoner Dilemma strategies available to give an
extensive numerical example of the ideas presented.  The approach is presented
in Section~\ref{sec:delta-zd-strategies}.  All of the code and data discussed
in Section~\ref{sec:numerical-experiments} is open sourced, archived and
written according to best scientific principles~\cite{Wilson2014}. The data
archive can be found at~\cite{vincent_knight_2018_1297075}.

\section{Recognising Extortion}\label{sec:delta-zd-strategies}

In~\cite{Press2012}, given a match between 2 memory-one strategies, the concept
of Zero Determinant (ZD) strategies is introduced. The main result of that paper
shows that given two memory one players \(p, q\in\mathbb{R}^4\) a linear
relationship between the players' scores could be forced by one of the players.

Using the notation of~\cite{Press2012}, assuming the utilities for player \(p\)
are given by \(S_x=(R, S, T, P)\) and for player \(q\) by \(S_y=(R, T, S, P)\)
and that the stationary scores of each player is given by \(S_X\) and \(S_Y\)
respectively. The main result of~\cite{Press2012} is that if

\begin{equation}\label{eqn:linear_relationship_for_p}
    \tilde p=\alpha S_x + \beta S_y + \gamma
\end{equation}

or

\begin{equation}\label{eqn:linear_relationship_for_q}
    \tilde q=\alpha S_x + \beta S_y + \gamma
\end{equation}

where \(\tilde p = (1 - p_1, 1 - p_2, p_3, p_4)\) and
\(\tilde q = (1 - q_1, 1 - q_2, q_3, q_4)\) then:

\begin{equation}
    \alpha S_X + \beta S_Y + \gamma = 0
\end{equation}

In~\cite{Press2012} a particular type of ZD strategy is defined: extortionate
strategies. If:

\begin{equation}\label{eqn:constraint_for_extortion}
    \gamma = - P(\alpha + \beta)
\end{equation}

then the player can ensure they get a score \(\chi\) times
larger than the opponent. This extortion coefficient is given by:

\begin{equation}\label{eqn:definition_of_chi}
    \chi=\frac{-\beta}{\alpha}
\end{equation}

Thus, if (\ref{eqn:constraint_for_extortion}) holds and \(\chi >1\) a player is
said to extort their opponent.
Here, the reverse problem is considered: given a
\(p\in\mathbb{R}^4\) how does one identify \(\alpha, \beta\) if they
exist and is the strategy in fact acting in an extortionate way?

These conditions correspond to:

\begin{align}
    \tilde p_1 & = \alpha R + \beta R - P (\alpha + \beta)
            \label{eqn:condition_for_tilde_p1}\\
    \tilde p_2 & = \alpha S + \beta T - P (\alpha + \beta)
            \label{eqn:condition_for_tilde_p2}\\
    \tilde p_3 & = \alpha T + \beta S - P (\alpha + \beta)
            \label{eqn:condition_for_tilde_p3}\\
    \tilde p_4 & = \alpha P + \beta P - P (\alpha + \beta)
            \label{eqn:condition_for_tilde_p4}
\end{align}

Equation (\ref{eqn:condition_for_tilde_p4}) ensures that \(p_4=\tilde p_4=0\).
Equations (\ref{eqn:condition_for_tilde_p1}-\ref{eqn:condition_for_tilde_p3})
can be used to eliminate \(\alpha, \beta\), giving:

\begin{equation}\label{eqn:planar_definition_of_extortion}
    \tilde p_1 = \frac{(R - P)(\tilde p_2 + \tilde p_3)}{S + T - 2P}
\end{equation}

with:

\begin{equation}\label{eqn:definition_of_chi}
    \chi = \frac{\tilde p_2 (P - T) + \tilde p_3 (S - P)}
                {\tilde p_2 (P - S) + \tilde p_3 (T - P)}
\end{equation}

Given a strategy \(p\in\mathbb{R}^{4\times 1}\) equations
(\ref{eqn:condition_for_tilde_p4}), (\ref{eqn:planar_definition_of_extortion}-\ref{eqn:definition_of_chi}) can be used to check if
a strategy is extortionate. The conditions correspond to:

\begin{align}
    p_1 & = \frac{(R-P)(p_2 + p_3) - R + T + S - P}{S + T - 2P}
     \label{eqn:condition_for_p1}\\
    p_4 & = 0 \label{eqn:condition_for_p4}\\
    1 & > p_2 + p_3\label{eqn:condition_for_chi}
\end{align}

The algebraic steps necessary to prove these results are available in the
supporting materials.

All extortionate strategies reside on a triangular (\ref{eqn:condition_for_chi})
plane (\ref{eqn:condition_for_p1}) in 3 dimensions (\ref{eqn:condition_for_p4}).
Using this formulation it can be seen that a necessary (but not sufficient)
condition for an extortionate strategy is that it cooperates on average less
than 50\% of the time when in a state of disagreement with the opponent.

As an example, consider the known extortionate strategy \(p=(8 / 9, 1 / 2, 1 /
3, 0)\) from~\cite{Stewart2012} which is referred to as \texttt{Extort-2}. In
this case, for the standard values of \((R, T, S, P)\) constraint
(\ref{eqn:condition_for_p1}) corresponds to:

\begin{equation}
    p_1 = \frac{2(p_2 + p_3) + 1}{3}
\end{equation}

It is clear that in this case all constraints hold.

This approach could in fact be used to confirm that a given strategy is acting
in an extortionate manner even if it is not a memory one strategy. However, in
practice, if a closed form for \(p\) is not known, then due to measurement
and/or numerical error this would not work.

This problem can be written in the following linear algebraic form where
\(x=(\alpha, \beta)\)
and \(p^*=(\tilde p_1 - 1, tilde_2 - 1, p_3)\):

\begin{equation}\label{eqn:linear_algebraic_equation_for_p}
    Cx= p^*
\end{equation}

\(C\) corresponds to equations
(\ref{eqn:condition_for_tilde_p1}-\ref{eqn:condition_for_tilde_p3}) and is
given by:

\begin{equation}\label{eqn:definition_of_C}
    C =
    \begin{bmatrix}
        R - P & R- P \\
        S - P & T- P \\
        T - P & S- P \\
    \end{bmatrix}
\end{equation}

Note that in general, equation (\ref{eqn:linear_algebraic_equation_for_p}) will
not necessarily have a solution. From the Rouch\'{e}-Capelli theorem if there is
a solution it is unique as \(\text{rank}(C)=2\) which is the dimension of the
variable \(x\). The best fitting \(x\) is found by minimizing:

\begin{equation}\label{eqn:r_squared}
    \text{SSError} = \|C x- p^*\|_2^2 = \sum_{i=1}^{3}\left((C\bar x)_i-p_i^*\right)^2
\end{equation}

Note that \(\text{SSError}\), which is the square of the Frobenius
norm~\cite{Golub2013}, becomes a measure of how close a strategy is to being an
extortionate strategy. Suspicion
of extortion then corresponds to a threshold on \(\text{SSError}\).

By observing interactions (human or otherwise), their memory one representation
can be inferred and this approach can be used to recognise extortionate
behaviour. The notion of comparing theoretic and actual plays of the IPD is not
novel, see for example~\cite{Rand2013}. Immediately it is noted that if the
environment is noisy~\cite{Wu1995} then no strategy can be considered to be
extortionate as \(p_4>0\).

In the next section, this idea will be illustrated by observing the interactions
that take place in a computer based tournament of the IPD\@.

\section{Numerical experiments}\label{sec:numerical-experiments}

In~\cite{Stewart2012} results from a tournament with
\input{./assets/tex/number_of_stewart_plotkin_strategies/main.tex} strategies,
was presented with specific consideration given to ZD strategies. This
tournament is reproduced here using the Axelrod-Python
project~\cite{Knight2016}. To obtain a good measure of the corresponding
transition rates for each strategy all matches have been run for
\input{assets/tex/number_of_turns/main.tex} turns and every match has been
repeated \input{assets/tex/number_of_repetitions/main.tex} times. All of this
interaction data is available at~\cite{vincent_knight_2018_1297075}. A good
match between the inferred Markov chain and the state distribution of the actual
interactions has been verified. Data for this is presented in the supplementary
materials.

Figure~\ref{fig:SSError_overall_in_stewart_plotkin} shows the \(\text{SSError}\)
values for all the strategies in the tournament, as reported
in~\cite{Stewart2012} the extortionate strategy (which has an expected
\(\text{SSError}\) approximately 0) gains a large number of wins.

\begin{figure}[!htbp]
    \centering
    \includegraphics[width=.8\textwidth]{./assets/img/SSError_overall_in_stewart_plotkin/main.pdf}
    \caption{\(\text{SSError}\) and state probabilities for the strategies
        of~\cite{Stewart2012}, ordered both by number of wins and overall score.
        Note that \(P(DC)\) is not shown as it corresponds to the transpose of
        \(P(CD)\). Cooperator and Defector are omitted as they do not visit all
        the states.}
    \label{fig:SSError_overall_in_stewart_plotkin}
\end{figure}

Here, the work of~\cite{Stewart2012} is extended by investigating a tournament
with \input{assets/tex/number_of_full_strategies/main.tex}
strategies.

The results of this analysis are shown in
Figure~\ref{fig:SSError_and_probabilities_in_full}. The top ranking strategies
by number of wins seem to be extortionate (but not against all strategies) and
it can be seen that a small sub group of strategies achieve mutual defection.
All the top ranking strategies according to score achieve mutual cooperation and
do not extort each other, however they
\textbf{do} exhibit extortionate behaviour towards a number of the lower ranking
strategies.

\begin{figure}[!htbp]
    \centering
    \includegraphics[width=.8\textwidth]{./assets/img/SSError_and_probabilities_in_full/main.pdf}
    \caption{\(\text{SSError}\) for the strategies for the full tournament. Only
    strategy interactions for which \(p_4=0\) and \(\chi>1\) are displayed.}
    \label{fig:SSError_and_probabilities_in_full}
\end{figure}

\section{Conclusion}\label{sec:conclusion}

This work defines an approach to measure whether or not a player is playing a
strategy that corresponds to an extortionate strategy as defined
in~\cite{Press2012}: a mathematical model for suspicion. Indeed, all
extortionate strategies have been
 classified as lying on a triangular plane.
This rigorous classification fails to be robust to small measurement error, thus
a statistical approach is proposed.
This is done through a linear algebraic approach for approximating the solution
of a linear system. Using this, a large number of pairwise interactions is
simulated and in fact very few strategies are found to act extortionately.

The work of~\cite{Press2012}, whilst showing that a clever approach to taking
advantage of another memory one strategy exists: this is incomplete. Whilst the
elegance of this result is very attractive, just as the simplicity of the
victory of Tit For Tat in Axelrod's original tournaments was, it is incomplete.
Extortionate strategies achieve a high number of wins but they do not
achieve a high score which corresponds to the fitness landscape in an
evolutionary sense. From the large number of interactions a payoff matrix \(S\)
can be measured where \(S_{ij}\) denotes the score (using standard values of
\((R, S, T, P) = (3, 0, 5, 1)\)) of the \(i\)th strategy
against the \(j\)th strategy. Using this, the replicator equation
describes the evolution of the system based on a population density fitness
function:

\begin{equation}\label{eqn:replicator_dynamics}
    \frac{dx}{dt} = x(S-x^TS x)
\end{equation}

Equation (\ref{eqn:replicator_dynamics}) is solved numerically through an
integration technique described in~\cite{Petzold1983} and
Figure~\ref{fig:replicator_dynamics} shows the evolution of the distribution of
the system: the various strategies are ranked by scores. It is clear to see that
only the high ranking strategies survive the evolutionary process (in fact,
only \input{./assets/img/replicator_dynamics/main.tex}
have a final distribution greater than \(10 ^ {-2}\)). This confirms the
findings of~\cite{Moran1707} in which sophisticated strategies resist
evolutionary invasion of shorter memory strategies. Recalling
Figure~\ref{fig:SSError_and_probabilities_in_full} this demonstrates that:

\begin{itemize}
    \item Cooperation emerges through the evolutionary process: the high scoring
        strategies do not exhibit extortionate behaviour towards each other.
    \item Extortionate strategies do not survive the evolutionary process.
\end{itemize}

\begin{figure}[!htbp]
    \centering
    \includegraphics[width=.8\textwidth]{./assets/img/replicator_dynamics/main.pdf}
    \caption{Numerical simulation of the replicator equation
    (\ref{eqn:replicator_dynamics}): strategies are ordered by score, only the strategies with a high score survive the evolutionary process.}
    \label{fig:replicator_dynamics}
\end{figure}

This work can be used to classify plays of the IPD\@: data can be collected from
actual interactions (in lab or in the field). Furthermore, this allows for a
classification method similar to the notion of fingerprinting presented
in~\cite{Ashlock2008}. Trained strategies can potentially be classified as
extortionate or not or it could be possible to even constrain the reinforcement
learning approaches that are becoming prevalent in the literature.
Alternatively, this mathematical approach for recognising extortion could be
used in sophisticated strategies to defend against invasion. Arguably, some of
the strategies considered here exhibit this behaviour, indeed as described
in~\cite{Harper2017}, the top ranking strategies in the full tournament are
obtained using evolutionary reinforcement learning techniques, thus, suspicion
of extortionate behaviour could in fact be an evolutionary trait.

\section*{Acknowledgements}

The following open source software libraries were used in this research:

\begin{itemize}
    \item The Axelrod ~\cite{Knight2016, Knight2018} library (IPD strategies and
        tournaments).
    \item The sympy library~\cite{Meurer2017} (verification of all symbolic
        calculations).
    \item The matplotlib~\cite{Droettboom2018} library (visualisation).
    \item The pandas~\cite{Structures2010}, dask~\cite{Dask2016} and
        NumPy~\cite{Oliphant2015} libraries (data manipulation).
    \item The SciPy~\cite{Jones2001} library (numerical integration of the
        replicator equation).
\end{itemize}

This work was performed using the computational facilities of the Advanced
Research Computing @ Cardiff (ARCCA) Division, Cardiff University.

\printbibliography

\newpage
\section*{Supplementary materials}

\includepdf{assets/pdf/proof_of_form_of_extortionate_strategies/main.pdf}

\newpage

Using the pair wise interactions the transition rates \(p,
q\) can be measured and the steady state probabilities inferred and compared to
the actual probabilities of each state.
This is done numerically by computing the singular eigenvector of the
matrix \(A\) \cite{Stewart2009}:

\[
    A =
    \begin{bmatrix}
        p_1 q_1 & p_1 (1 - q_1) & (1 - p_1) q_1 & (1 -p_1) (1 - q_1) \\
        p_2 q_2 & p_2 (1 - q_2) & (1 - p_2) q_2 & (1 -p_2) (1 - q_2) \\
        p_3 q_3 & p_3 (1 - q_3) & (1 - p_3) q_3 & (1 -p_3) (1 - q_3) \\
        p_4 q_4 & p_4 (1 - q_4) & (1 - p_4) q_4 & (1 -p_4) (1 - q_4) \\
    \end{bmatrix}
\]

Figure~\ref{fig:computed_probabilities_vs_theoretic_probabilities} shows a
regression line fitted to every pairwise interaction with a reported
\(\text{SSError}\) value (pairwise interactions with missing states were
omitted). This serves to validate the approach: a part from some edge cases the
relationship is consistent.

\begin{figure}[!htbp]
    \centering
    \includegraphics[width=.8\textwidth]{./assets/img/computed_probabilities_vs_theoretic_probabilities/main.pdf}
    \caption{The
        relationship between the steady state probabilities inferred from the
        measured transitions and the actual steady state probabilities. A linear
        regression line is included validating the approach.}
    \label{fig:computed_probabilities_vs_theoretic_probabilities}
\end{figure}


\end{document}
 strategies,
was presented with specific consideration given to ZD strategies. This
tournament is reproduced here using the Axelrod-Python
project~\cite{Knight2016}. To obtain a good measure of the corresponding
transition rates for each strategy all matches have been run for
\documentclass[a4paper]{article}

\usepackage{amsmath}
\usepackage{amssymb}
\usepackage[margin=1.5cm,
            includefoot,
            footskip=30pt]{geometry}
\usepackage{layout}
\usepackage{graphicx}
\usepackage{subcaption}

\usepackage{biblatex}
\usepackage{pdfpages}

\bibliography{main.bib}

\title{Suspicion: Recognising and evaluating the effectiveness
       of extortion in the Iterated Prisoner's Dilemma}
\author{Vincent A. Knight \and Nikoleta E. Glynatsi}
\date{\today}



\begin{document}

\maketitle

\begin{abstract}
    The Iterated Prisoner's Dilemma is a model for rational and evolutionary
    interactive behaviour. It has applications both in the study of human social
    behaviour as well as in biology.
    It is used to understand when and how a rational individual might
    accept an immediate cost to their own utility for the direct benefit of
    another.

    Much attention has been given to a class of strategies called
    Zero Determinant strategies. It has been theoretically shown that these
    strategies can ``extort'' any player.

    In this work, an approach to identify if observed strategies are playing in
    an extortionate way is described. Furthermore, experimental analysis of
    a large tournament with \input{assets/tex/number_of_full_strategies/main.tex}
    strategies is considered. In this setting
    the most highly performing strategies do not play in an extortionate way
    against each other but do against lower performing strategies.
    This suggests that whilst the theory of Zero Determinant strategies
    indicates that memory is not of fundamental importance to the evolution of
    cooperative behaviour, this is incomplete.
\end{abstract}

\section{Introduction}\label{sec:introduction}

Agent based game theoretic models have become a stalwart of the underpinning
mathematics of interactive behaviours. One of the major pieces of work
in this area is the pair of original computer tournaments run by Robert
Axelrod~\cite{Axelrod1980, Axelrod1980a}. These tournaments pitted submitted
computer strategies against each other in plays of the Iterated Prisoner's
Dilemma. A common game where agents can choose to pay a slight cost to their
immediate utility in the hope of building a reputation. This has been used in
economic and evolutionary game theory to understand the evolution of cooperative
behaviour.

Recently, a class of strategies was described in~\cite{Press2012} that can
provably extort any given opponent. In~\cite{Hilbe2013, Moran1707} some
questions have already been asked about the true effectiveness of these
strategies in an evolutionary setting. Here another question is asked: is it
possible to recognise this extortionate behaviour? A mathematical procedure for
suspicion is presented: in the same way that the continued actions of an
extortionate individual might raise suspicion.

This work makes use of the Axelrod Python library~\cite{Knight2018, Knight2016}
with a large number of Prisoner Dilemma strategies available to give an
extensive numerical example of the ideas presented.  The approach is presented
in Section~\ref{sec:delta-zd-strategies}.  All of the code and data discussed
in Section~\ref{sec:numerical-experiments} is open sourced, archived and
written according to best scientific principles~\cite{Wilson2014}. The data
archive can be found at~\cite{vincent_knight_2018_1297075}.

\section{Recognising Extortion}\label{sec:delta-zd-strategies}

In~\cite{Press2012}, given a match between 2 memory-one strategies, the concept
of Zero Determinant (ZD) strategies is introduced. The main result of that paper
shows that given two memory one players \(p, q\in\mathbb{R}^4\) a linear
relationship between the players' scores could be forced by one of the players.

Using the notation of~\cite{Press2012}, assuming the utilities for player \(p\)
are given by \(S_x=(R, S, T, P)\) and for player \(q\) by \(S_y=(R, T, S, P)\)
and that the stationary scores of each player is given by \(S_X\) and \(S_Y\)
respectively. The main result of~\cite{Press2012} is that if

\begin{equation}\label{eqn:linear_relationship_for_p}
    \tilde p=\alpha S_x + \beta S_y + \gamma
\end{equation}

or

\begin{equation}\label{eqn:linear_relationship_for_q}
    \tilde q=\alpha S_x + \beta S_y + \gamma
\end{equation}

where \(\tilde p = (1 - p_1, 1 - p_2, p_3, p_4)\) and
\(\tilde q = (1 - q_1, 1 - q_2, q_3, q_4)\) then:

\begin{equation}
    \alpha S_X + \beta S_Y + \gamma = 0
\end{equation}

In~\cite{Press2012} a particular type of ZD strategy is defined: extortionate
strategies. If:

\begin{equation}\label{eqn:constraint_for_extortion}
    \gamma = - P(\alpha + \beta)
\end{equation}

then the player can ensure they get a score \(\chi\) times
larger than the opponent. This extortion coefficient is given by:

\begin{equation}\label{eqn:definition_of_chi}
    \chi=\frac{-\beta}{\alpha}
\end{equation}

Thus, if (\ref{eqn:constraint_for_extortion}) holds and \(\chi >1\) a player is
said to extort their opponent.
Here, the reverse problem is considered: given a
\(p\in\mathbb{R}^4\) how does one identify \(\alpha, \beta\) if they
exist and is the strategy in fact acting in an extortionate way?

These conditions correspond to:

\begin{align}
    \tilde p_1 & = \alpha R + \beta R - P (\alpha + \beta)
            \label{eqn:condition_for_tilde_p1}\\
    \tilde p_2 & = \alpha S + \beta T - P (\alpha + \beta)
            \label{eqn:condition_for_tilde_p2}\\
    \tilde p_3 & = \alpha T + \beta S - P (\alpha + \beta)
            \label{eqn:condition_for_tilde_p3}\\
    \tilde p_4 & = \alpha P + \beta P - P (\alpha + \beta)
            \label{eqn:condition_for_tilde_p4}
\end{align}

Equation (\ref{eqn:condition_for_tilde_p4}) ensures that \(p_4=\tilde p_4=0\).
Equations (\ref{eqn:condition_for_tilde_p1}-\ref{eqn:condition_for_tilde_p3})
can be used to eliminate \(\alpha, \beta\), giving:

\begin{equation}\label{eqn:planar_definition_of_extortion}
    \tilde p_1 = \frac{(R - P)(\tilde p_2 + \tilde p_3)}{S + T - 2P}
\end{equation}

with:

\begin{equation}\label{eqn:definition_of_chi}
    \chi = \frac{\tilde p_2 (P - T) + \tilde p_3 (S - P)}
                {\tilde p_2 (P - S) + \tilde p_3 (T - P)}
\end{equation}

Given a strategy \(p\in\mathbb{R}^{4\times 1}\) equations
(\ref{eqn:condition_for_tilde_p4}), (\ref{eqn:planar_definition_of_extortion}-\ref{eqn:definition_of_chi}) can be used to check if
a strategy is extortionate. The conditions correspond to:

\begin{align}
    p_1 & = \frac{(R-P)(p_2 + p_3) - R + T + S - P}{S + T - 2P}
     \label{eqn:condition_for_p1}\\
    p_4 & = 0 \label{eqn:condition_for_p4}\\
    1 & > p_2 + p_3\label{eqn:condition_for_chi}
\end{align}

The algebraic steps necessary to prove these results are available in the
supporting materials.

All extortionate strategies reside on a triangular (\ref{eqn:condition_for_chi})
plane (\ref{eqn:condition_for_p1}) in 3 dimensions (\ref{eqn:condition_for_p4}).
Using this formulation it can be seen that a necessary (but not sufficient)
condition for an extortionate strategy is that it cooperates on average less
than 50\% of the time when in a state of disagreement with the opponent.

As an example, consider the known extortionate strategy \(p=(8 / 9, 1 / 2, 1 /
3, 0)\) from~\cite{Stewart2012} which is referred to as \texttt{Extort-2}. In
this case, for the standard values of \((R, T, S, P)\) constraint
(\ref{eqn:condition_for_p1}) corresponds to:

\begin{equation}
    p_1 = \frac{2(p_2 + p_3) + 1}{3}
\end{equation}

It is clear that in this case all constraints hold.

This approach could in fact be used to confirm that a given strategy is acting
in an extortionate manner even if it is not a memory one strategy. However, in
practice, if a closed form for \(p\) is not known, then due to measurement
and/or numerical error this would not work.

This problem can be written in the following linear algebraic form where
\(x=(\alpha, \beta)\)
and \(p^*=(\tilde p_1 - 1, tilde_2 - 1, p_3)\):

\begin{equation}\label{eqn:linear_algebraic_equation_for_p}
    Cx= p^*
\end{equation}

\(C\) corresponds to equations
(\ref{eqn:condition_for_tilde_p1}-\ref{eqn:condition_for_tilde_p3}) and is
given by:

\begin{equation}\label{eqn:definition_of_C}
    C =
    \begin{bmatrix}
        R - P & R- P \\
        S - P & T- P \\
        T - P & S- P \\
    \end{bmatrix}
\end{equation}

Note that in general, equation (\ref{eqn:linear_algebraic_equation_for_p}) will
not necessarily have a solution. From the Rouch\'{e}-Capelli theorem if there is
a solution it is unique as \(\text{rank}(C)=2\) which is the dimension of the
variable \(x\). The best fitting \(x\) is found by minimizing:

\begin{equation}\label{eqn:r_squared}
    \text{SSError} = \|C x- p^*\|_2^2 = \sum_{i=1}^{3}\left((C\bar x)_i-p_i^*\right)^2
\end{equation}

Note that \(\text{SSError}\), which is the square of the Frobenius
norm~\cite{Golub2013}, becomes a measure of how close a strategy is to being an
extortionate strategy. Suspicion
of extortion then corresponds to a threshold on \(\text{SSError}\).

By observing interactions (human or otherwise), their memory one representation
can be inferred and this approach can be used to recognise extortionate
behaviour. The notion of comparing theoretic and actual plays of the IPD is not
novel, see for example~\cite{Rand2013}. Immediately it is noted that if the
environment is noisy~\cite{Wu1995} then no strategy can be considered to be
extortionate as \(p_4>0\).

In the next section, this idea will be illustrated by observing the interactions
that take place in a computer based tournament of the IPD\@.

\section{Numerical experiments}\label{sec:numerical-experiments}

In~\cite{Stewart2012} results from a tournament with
\input{./assets/tex/number_of_stewart_plotkin_strategies/main.tex} strategies,
was presented with specific consideration given to ZD strategies. This
tournament is reproduced here using the Axelrod-Python
project~\cite{Knight2016}. To obtain a good measure of the corresponding
transition rates for each strategy all matches have been run for
\input{assets/tex/number_of_turns/main.tex} turns and every match has been
repeated \input{assets/tex/number_of_repetitions/main.tex} times. All of this
interaction data is available at~\cite{vincent_knight_2018_1297075}. A good
match between the inferred Markov chain and the state distribution of the actual
interactions has been verified. Data for this is presented in the supplementary
materials.

Figure~\ref{fig:SSError_overall_in_stewart_plotkin} shows the \(\text{SSError}\)
values for all the strategies in the tournament, as reported
in~\cite{Stewart2012} the extortionate strategy (which has an expected
\(\text{SSError}\) approximately 0) gains a large number of wins.

\begin{figure}[!htbp]
    \centering
    \includegraphics[width=.8\textwidth]{./assets/img/SSError_overall_in_stewart_plotkin/main.pdf}
    \caption{\(\text{SSError}\) and state probabilities for the strategies
        of~\cite{Stewart2012}, ordered both by number of wins and overall score.
        Note that \(P(DC)\) is not shown as it corresponds to the transpose of
        \(P(CD)\). Cooperator and Defector are omitted as they do not visit all
        the states.}
    \label{fig:SSError_overall_in_stewart_plotkin}
\end{figure}

Here, the work of~\cite{Stewart2012} is extended by investigating a tournament
with \input{assets/tex/number_of_full_strategies/main.tex}
strategies.

The results of this analysis are shown in
Figure~\ref{fig:SSError_and_probabilities_in_full}. The top ranking strategies
by number of wins seem to be extortionate (but not against all strategies) and
it can be seen that a small sub group of strategies achieve mutual defection.
All the top ranking strategies according to score achieve mutual cooperation and
do not extort each other, however they
\textbf{do} exhibit extortionate behaviour towards a number of the lower ranking
strategies.

\begin{figure}[!htbp]
    \centering
    \includegraphics[width=.8\textwidth]{./assets/img/SSError_and_probabilities_in_full/main.pdf}
    \caption{\(\text{SSError}\) for the strategies for the full tournament. Only
    strategy interactions for which \(p_4=0\) and \(\chi>1\) are displayed.}
    \label{fig:SSError_and_probabilities_in_full}
\end{figure}

\section{Conclusion}\label{sec:conclusion}

This work defines an approach to measure whether or not a player is playing a
strategy that corresponds to an extortionate strategy as defined
in~\cite{Press2012}: a mathematical model for suspicion. Indeed, all
extortionate strategies have been
 classified as lying on a triangular plane.
This rigorous classification fails to be robust to small measurement error, thus
a statistical approach is proposed.
This is done through a linear algebraic approach for approximating the solution
of a linear system. Using this, a large number of pairwise interactions is
simulated and in fact very few strategies are found to act extortionately.

The work of~\cite{Press2012}, whilst showing that a clever approach to taking
advantage of another memory one strategy exists: this is incomplete. Whilst the
elegance of this result is very attractive, just as the simplicity of the
victory of Tit For Tat in Axelrod's original tournaments was, it is incomplete.
Extortionate strategies achieve a high number of wins but they do not
achieve a high score which corresponds to the fitness landscape in an
evolutionary sense. From the large number of interactions a payoff matrix \(S\)
can be measured where \(S_{ij}\) denotes the score (using standard values of
\((R, S, T, P) = (3, 0, 5, 1)\)) of the \(i\)th strategy
against the \(j\)th strategy. Using this, the replicator equation
describes the evolution of the system based on a population density fitness
function:

\begin{equation}\label{eqn:replicator_dynamics}
    \frac{dx}{dt} = x(S-x^TS x)
\end{equation}

Equation (\ref{eqn:replicator_dynamics}) is solved numerically through an
integration technique described in~\cite{Petzold1983} and
Figure~\ref{fig:replicator_dynamics} shows the evolution of the distribution of
the system: the various strategies are ranked by scores. It is clear to see that
only the high ranking strategies survive the evolutionary process (in fact,
only \input{./assets/img/replicator_dynamics/main.tex}
have a final distribution greater than \(10 ^ {-2}\)). This confirms the
findings of~\cite{Moran1707} in which sophisticated strategies resist
evolutionary invasion of shorter memory strategies. Recalling
Figure~\ref{fig:SSError_and_probabilities_in_full} this demonstrates that:

\begin{itemize}
    \item Cooperation emerges through the evolutionary process: the high scoring
        strategies do not exhibit extortionate behaviour towards each other.
    \item Extortionate strategies do not survive the evolutionary process.
\end{itemize}

\begin{figure}[!htbp]
    \centering
    \includegraphics[width=.8\textwidth]{./assets/img/replicator_dynamics/main.pdf}
    \caption{Numerical simulation of the replicator equation
    (\ref{eqn:replicator_dynamics}): strategies are ordered by score, only the strategies with a high score survive the evolutionary process.}
    \label{fig:replicator_dynamics}
\end{figure}

This work can be used to classify plays of the IPD\@: data can be collected from
actual interactions (in lab or in the field). Furthermore, this allows for a
classification method similar to the notion of fingerprinting presented
in~\cite{Ashlock2008}. Trained strategies can potentially be classified as
extortionate or not or it could be possible to even constrain the reinforcement
learning approaches that are becoming prevalent in the literature.
Alternatively, this mathematical approach for recognising extortion could be
used in sophisticated strategies to defend against invasion. Arguably, some of
the strategies considered here exhibit this behaviour, indeed as described
in~\cite{Harper2017}, the top ranking strategies in the full tournament are
obtained using evolutionary reinforcement learning techniques, thus, suspicion
of extortionate behaviour could in fact be an evolutionary trait.

\section*{Acknowledgements}

The following open source software libraries were used in this research:

\begin{itemize}
    \item The Axelrod ~\cite{Knight2016, Knight2018} library (IPD strategies and
        tournaments).
    \item The sympy library~\cite{Meurer2017} (verification of all symbolic
        calculations).
    \item The matplotlib~\cite{Droettboom2018} library (visualisation).
    \item The pandas~\cite{Structures2010}, dask~\cite{Dask2016} and
        NumPy~\cite{Oliphant2015} libraries (data manipulation).
    \item The SciPy~\cite{Jones2001} library (numerical integration of the
        replicator equation).
\end{itemize}

This work was performed using the computational facilities of the Advanced
Research Computing @ Cardiff (ARCCA) Division, Cardiff University.

\printbibliography

\newpage
\section*{Supplementary materials}

\includepdf{assets/pdf/proof_of_form_of_extortionate_strategies/main.pdf}

\newpage

Using the pair wise interactions the transition rates \(p,
q\) can be measured and the steady state probabilities inferred and compared to
the actual probabilities of each state.
This is done numerically by computing the singular eigenvector of the
matrix \(A\) \cite{Stewart2009}:

\[
    A =
    \begin{bmatrix}
        p_1 q_1 & p_1 (1 - q_1) & (1 - p_1) q_1 & (1 -p_1) (1 - q_1) \\
        p_2 q_2 & p_2 (1 - q_2) & (1 - p_2) q_2 & (1 -p_2) (1 - q_2) \\
        p_3 q_3 & p_3 (1 - q_3) & (1 - p_3) q_3 & (1 -p_3) (1 - q_3) \\
        p_4 q_4 & p_4 (1 - q_4) & (1 - p_4) q_4 & (1 -p_4) (1 - q_4) \\
    \end{bmatrix}
\]

Figure~\ref{fig:computed_probabilities_vs_theoretic_probabilities} shows a
regression line fitted to every pairwise interaction with a reported
\(\text{SSError}\) value (pairwise interactions with missing states were
omitted). This serves to validate the approach: a part from some edge cases the
relationship is consistent.

\begin{figure}[!htbp]
    \centering
    \includegraphics[width=.8\textwidth]{./assets/img/computed_probabilities_vs_theoretic_probabilities/main.pdf}
    \caption{The
        relationship between the steady state probabilities inferred from the
        measured transitions and the actual steady state probabilities. A linear
        regression line is included validating the approach.}
    \label{fig:computed_probabilities_vs_theoretic_probabilities}
\end{figure}


\end{document}
 turns and every match has been
repeated \documentclass[a4paper]{article}

\usepackage{amsmath}
\usepackage{amssymb}
\usepackage[margin=1.5cm,
            includefoot,
            footskip=30pt]{geometry}
\usepackage{layout}
\usepackage{graphicx}
\usepackage{subcaption}

\usepackage{biblatex}
\usepackage{pdfpages}

\bibliography{main.bib}

\title{Suspicion: Recognising and evaluating the effectiveness
       of extortion in the Iterated Prisoner's Dilemma}
\author{Vincent A. Knight \and Nikoleta E. Glynatsi}
\date{\today}



\begin{document}

\maketitle

\begin{abstract}
    The Iterated Prisoner's Dilemma is a model for rational and evolutionary
    interactive behaviour. It has applications both in the study of human social
    behaviour as well as in biology.
    It is used to understand when and how a rational individual might
    accept an immediate cost to their own utility for the direct benefit of
    another.

    Much attention has been given to a class of strategies called
    Zero Determinant strategies. It has been theoretically shown that these
    strategies can ``extort'' any player.

    In this work, an approach to identify if observed strategies are playing in
    an extortionate way is described. Furthermore, experimental analysis of
    a large tournament with \input{assets/tex/number_of_full_strategies/main.tex}
    strategies is considered. In this setting
    the most highly performing strategies do not play in an extortionate way
    against each other but do against lower performing strategies.
    This suggests that whilst the theory of Zero Determinant strategies
    indicates that memory is not of fundamental importance to the evolution of
    cooperative behaviour, this is incomplete.
\end{abstract}

\section{Introduction}\label{sec:introduction}

Agent based game theoretic models have become a stalwart of the underpinning
mathematics of interactive behaviours. One of the major pieces of work
in this area is the pair of original computer tournaments run by Robert
Axelrod~\cite{Axelrod1980, Axelrod1980a}. These tournaments pitted submitted
computer strategies against each other in plays of the Iterated Prisoner's
Dilemma. A common game where agents can choose to pay a slight cost to their
immediate utility in the hope of building a reputation. This has been used in
economic and evolutionary game theory to understand the evolution of cooperative
behaviour.

Recently, a class of strategies was described in~\cite{Press2012} that can
provably extort any given opponent. In~\cite{Hilbe2013, Moran1707} some
questions have already been asked about the true effectiveness of these
strategies in an evolutionary setting. Here another question is asked: is it
possible to recognise this extortionate behaviour? A mathematical procedure for
suspicion is presented: in the same way that the continued actions of an
extortionate individual might raise suspicion.

This work makes use of the Axelrod Python library~\cite{Knight2018, Knight2016}
with a large number of Prisoner Dilemma strategies available to give an
extensive numerical example of the ideas presented.  The approach is presented
in Section~\ref{sec:delta-zd-strategies}.  All of the code and data discussed
in Section~\ref{sec:numerical-experiments} is open sourced, archived and
written according to best scientific principles~\cite{Wilson2014}. The data
archive can be found at~\cite{vincent_knight_2018_1297075}.

\section{Recognising Extortion}\label{sec:delta-zd-strategies}

In~\cite{Press2012}, given a match between 2 memory-one strategies, the concept
of Zero Determinant (ZD) strategies is introduced. The main result of that paper
shows that given two memory one players \(p, q\in\mathbb{R}^4\) a linear
relationship between the players' scores could be forced by one of the players.

Using the notation of~\cite{Press2012}, assuming the utilities for player \(p\)
are given by \(S_x=(R, S, T, P)\) and for player \(q\) by \(S_y=(R, T, S, P)\)
and that the stationary scores of each player is given by \(S_X\) and \(S_Y\)
respectively. The main result of~\cite{Press2012} is that if

\begin{equation}\label{eqn:linear_relationship_for_p}
    \tilde p=\alpha S_x + \beta S_y + \gamma
\end{equation}

or

\begin{equation}\label{eqn:linear_relationship_for_q}
    \tilde q=\alpha S_x + \beta S_y + \gamma
\end{equation}

where \(\tilde p = (1 - p_1, 1 - p_2, p_3, p_4)\) and
\(\tilde q = (1 - q_1, 1 - q_2, q_3, q_4)\) then:

\begin{equation}
    \alpha S_X + \beta S_Y + \gamma = 0
\end{equation}

In~\cite{Press2012} a particular type of ZD strategy is defined: extortionate
strategies. If:

\begin{equation}\label{eqn:constraint_for_extortion}
    \gamma = - P(\alpha + \beta)
\end{equation}

then the player can ensure they get a score \(\chi\) times
larger than the opponent. This extortion coefficient is given by:

\begin{equation}\label{eqn:definition_of_chi}
    \chi=\frac{-\beta}{\alpha}
\end{equation}

Thus, if (\ref{eqn:constraint_for_extortion}) holds and \(\chi >1\) a player is
said to extort their opponent.
Here, the reverse problem is considered: given a
\(p\in\mathbb{R}^4\) how does one identify \(\alpha, \beta\) if they
exist and is the strategy in fact acting in an extortionate way?

These conditions correspond to:

\begin{align}
    \tilde p_1 & = \alpha R + \beta R - P (\alpha + \beta)
            \label{eqn:condition_for_tilde_p1}\\
    \tilde p_2 & = \alpha S + \beta T - P (\alpha + \beta)
            \label{eqn:condition_for_tilde_p2}\\
    \tilde p_3 & = \alpha T + \beta S - P (\alpha + \beta)
            \label{eqn:condition_for_tilde_p3}\\
    \tilde p_4 & = \alpha P + \beta P - P (\alpha + \beta)
            \label{eqn:condition_for_tilde_p4}
\end{align}

Equation (\ref{eqn:condition_for_tilde_p4}) ensures that \(p_4=\tilde p_4=0\).
Equations (\ref{eqn:condition_for_tilde_p1}-\ref{eqn:condition_for_tilde_p3})
can be used to eliminate \(\alpha, \beta\), giving:

\begin{equation}\label{eqn:planar_definition_of_extortion}
    \tilde p_1 = \frac{(R - P)(\tilde p_2 + \tilde p_3)}{S + T - 2P}
\end{equation}

with:

\begin{equation}\label{eqn:definition_of_chi}
    \chi = \frac{\tilde p_2 (P - T) + \tilde p_3 (S - P)}
                {\tilde p_2 (P - S) + \tilde p_3 (T - P)}
\end{equation}

Given a strategy \(p\in\mathbb{R}^{4\times 1}\) equations
(\ref{eqn:condition_for_tilde_p4}), (\ref{eqn:planar_definition_of_extortion}-\ref{eqn:definition_of_chi}) can be used to check if
a strategy is extortionate. The conditions correspond to:

\begin{align}
    p_1 & = \frac{(R-P)(p_2 + p_3) - R + T + S - P}{S + T - 2P}
     \label{eqn:condition_for_p1}\\
    p_4 & = 0 \label{eqn:condition_for_p4}\\
    1 & > p_2 + p_3\label{eqn:condition_for_chi}
\end{align}

The algebraic steps necessary to prove these results are available in the
supporting materials.

All extortionate strategies reside on a triangular (\ref{eqn:condition_for_chi})
plane (\ref{eqn:condition_for_p1}) in 3 dimensions (\ref{eqn:condition_for_p4}).
Using this formulation it can be seen that a necessary (but not sufficient)
condition for an extortionate strategy is that it cooperates on average less
than 50\% of the time when in a state of disagreement with the opponent.

As an example, consider the known extortionate strategy \(p=(8 / 9, 1 / 2, 1 /
3, 0)\) from~\cite{Stewart2012} which is referred to as \texttt{Extort-2}. In
this case, for the standard values of \((R, T, S, P)\) constraint
(\ref{eqn:condition_for_p1}) corresponds to:

\begin{equation}
    p_1 = \frac{2(p_2 + p_3) + 1}{3}
\end{equation}

It is clear that in this case all constraints hold.

This approach could in fact be used to confirm that a given strategy is acting
in an extortionate manner even if it is not a memory one strategy. However, in
practice, if a closed form for \(p\) is not known, then due to measurement
and/or numerical error this would not work.

This problem can be written in the following linear algebraic form where
\(x=(\alpha, \beta)\)
and \(p^*=(\tilde p_1 - 1, tilde_2 - 1, p_3)\):

\begin{equation}\label{eqn:linear_algebraic_equation_for_p}
    Cx= p^*
\end{equation}

\(C\) corresponds to equations
(\ref{eqn:condition_for_tilde_p1}-\ref{eqn:condition_for_tilde_p3}) and is
given by:

\begin{equation}\label{eqn:definition_of_C}
    C =
    \begin{bmatrix}
        R - P & R- P \\
        S - P & T- P \\
        T - P & S- P \\
    \end{bmatrix}
\end{equation}

Note that in general, equation (\ref{eqn:linear_algebraic_equation_for_p}) will
not necessarily have a solution. From the Rouch\'{e}-Capelli theorem if there is
a solution it is unique as \(\text{rank}(C)=2\) which is the dimension of the
variable \(x\). The best fitting \(x\) is found by minimizing:

\begin{equation}\label{eqn:r_squared}
    \text{SSError} = \|C x- p^*\|_2^2 = \sum_{i=1}^{3}\left((C\bar x)_i-p_i^*\right)^2
\end{equation}

Note that \(\text{SSError}\), which is the square of the Frobenius
norm~\cite{Golub2013}, becomes a measure of how close a strategy is to being an
extortionate strategy. Suspicion
of extortion then corresponds to a threshold on \(\text{SSError}\).

By observing interactions (human or otherwise), their memory one representation
can be inferred and this approach can be used to recognise extortionate
behaviour. The notion of comparing theoretic and actual plays of the IPD is not
novel, see for example~\cite{Rand2013}. Immediately it is noted that if the
environment is noisy~\cite{Wu1995} then no strategy can be considered to be
extortionate as \(p_4>0\).

In the next section, this idea will be illustrated by observing the interactions
that take place in a computer based tournament of the IPD\@.

\section{Numerical experiments}\label{sec:numerical-experiments}

In~\cite{Stewart2012} results from a tournament with
\input{./assets/tex/number_of_stewart_plotkin_strategies/main.tex} strategies,
was presented with specific consideration given to ZD strategies. This
tournament is reproduced here using the Axelrod-Python
project~\cite{Knight2016}. To obtain a good measure of the corresponding
transition rates for each strategy all matches have been run for
\input{assets/tex/number_of_turns/main.tex} turns and every match has been
repeated \input{assets/tex/number_of_repetitions/main.tex} times. All of this
interaction data is available at~\cite{vincent_knight_2018_1297075}. A good
match between the inferred Markov chain and the state distribution of the actual
interactions has been verified. Data for this is presented in the supplementary
materials.

Figure~\ref{fig:SSError_overall_in_stewart_plotkin} shows the \(\text{SSError}\)
values for all the strategies in the tournament, as reported
in~\cite{Stewart2012} the extortionate strategy (which has an expected
\(\text{SSError}\) approximately 0) gains a large number of wins.

\begin{figure}[!htbp]
    \centering
    \includegraphics[width=.8\textwidth]{./assets/img/SSError_overall_in_stewart_plotkin/main.pdf}
    \caption{\(\text{SSError}\) and state probabilities for the strategies
        of~\cite{Stewart2012}, ordered both by number of wins and overall score.
        Note that \(P(DC)\) is not shown as it corresponds to the transpose of
        \(P(CD)\). Cooperator and Defector are omitted as they do not visit all
        the states.}
    \label{fig:SSError_overall_in_stewart_plotkin}
\end{figure}

Here, the work of~\cite{Stewart2012} is extended by investigating a tournament
with \input{assets/tex/number_of_full_strategies/main.tex}
strategies.

The results of this analysis are shown in
Figure~\ref{fig:SSError_and_probabilities_in_full}. The top ranking strategies
by number of wins seem to be extortionate (but not against all strategies) and
it can be seen that a small sub group of strategies achieve mutual defection.
All the top ranking strategies according to score achieve mutual cooperation and
do not extort each other, however they
\textbf{do} exhibit extortionate behaviour towards a number of the lower ranking
strategies.

\begin{figure}[!htbp]
    \centering
    \includegraphics[width=.8\textwidth]{./assets/img/SSError_and_probabilities_in_full/main.pdf}
    \caption{\(\text{SSError}\) for the strategies for the full tournament. Only
    strategy interactions for which \(p_4=0\) and \(\chi>1\) are displayed.}
    \label{fig:SSError_and_probabilities_in_full}
\end{figure}

\section{Conclusion}\label{sec:conclusion}

This work defines an approach to measure whether or not a player is playing a
strategy that corresponds to an extortionate strategy as defined
in~\cite{Press2012}: a mathematical model for suspicion. Indeed, all
extortionate strategies have been
 classified as lying on a triangular plane.
This rigorous classification fails to be robust to small measurement error, thus
a statistical approach is proposed.
This is done through a linear algebraic approach for approximating the solution
of a linear system. Using this, a large number of pairwise interactions is
simulated and in fact very few strategies are found to act extortionately.

The work of~\cite{Press2012}, whilst showing that a clever approach to taking
advantage of another memory one strategy exists: this is incomplete. Whilst the
elegance of this result is very attractive, just as the simplicity of the
victory of Tit For Tat in Axelrod's original tournaments was, it is incomplete.
Extortionate strategies achieve a high number of wins but they do not
achieve a high score which corresponds to the fitness landscape in an
evolutionary sense. From the large number of interactions a payoff matrix \(S\)
can be measured where \(S_{ij}\) denotes the score (using standard values of
\((R, S, T, P) = (3, 0, 5, 1)\)) of the \(i\)th strategy
against the \(j\)th strategy. Using this, the replicator equation
describes the evolution of the system based on a population density fitness
function:

\begin{equation}\label{eqn:replicator_dynamics}
    \frac{dx}{dt} = x(S-x^TS x)
\end{equation}

Equation (\ref{eqn:replicator_dynamics}) is solved numerically through an
integration technique described in~\cite{Petzold1983} and
Figure~\ref{fig:replicator_dynamics} shows the evolution of the distribution of
the system: the various strategies are ranked by scores. It is clear to see that
only the high ranking strategies survive the evolutionary process (in fact,
only \input{./assets/img/replicator_dynamics/main.tex}
have a final distribution greater than \(10 ^ {-2}\)). This confirms the
findings of~\cite{Moran1707} in which sophisticated strategies resist
evolutionary invasion of shorter memory strategies. Recalling
Figure~\ref{fig:SSError_and_probabilities_in_full} this demonstrates that:

\begin{itemize}
    \item Cooperation emerges through the evolutionary process: the high scoring
        strategies do not exhibit extortionate behaviour towards each other.
    \item Extortionate strategies do not survive the evolutionary process.
\end{itemize}

\begin{figure}[!htbp]
    \centering
    \includegraphics[width=.8\textwidth]{./assets/img/replicator_dynamics/main.pdf}
    \caption{Numerical simulation of the replicator equation
    (\ref{eqn:replicator_dynamics}): strategies are ordered by score, only the strategies with a high score survive the evolutionary process.}
    \label{fig:replicator_dynamics}
\end{figure}

This work can be used to classify plays of the IPD\@: data can be collected from
actual interactions (in lab or in the field). Furthermore, this allows for a
classification method similar to the notion of fingerprinting presented
in~\cite{Ashlock2008}. Trained strategies can potentially be classified as
extortionate or not or it could be possible to even constrain the reinforcement
learning approaches that are becoming prevalent in the literature.
Alternatively, this mathematical approach for recognising extortion could be
used in sophisticated strategies to defend against invasion. Arguably, some of
the strategies considered here exhibit this behaviour, indeed as described
in~\cite{Harper2017}, the top ranking strategies in the full tournament are
obtained using evolutionary reinforcement learning techniques, thus, suspicion
of extortionate behaviour could in fact be an evolutionary trait.

\section*{Acknowledgements}

The following open source software libraries were used in this research:

\begin{itemize}
    \item The Axelrod ~\cite{Knight2016, Knight2018} library (IPD strategies and
        tournaments).
    \item The sympy library~\cite{Meurer2017} (verification of all symbolic
        calculations).
    \item The matplotlib~\cite{Droettboom2018} library (visualisation).
    \item The pandas~\cite{Structures2010}, dask~\cite{Dask2016} and
        NumPy~\cite{Oliphant2015} libraries (data manipulation).
    \item The SciPy~\cite{Jones2001} library (numerical integration of the
        replicator equation).
\end{itemize}

This work was performed using the computational facilities of the Advanced
Research Computing @ Cardiff (ARCCA) Division, Cardiff University.

\printbibliography

\newpage
\section*{Supplementary materials}

\includepdf{assets/pdf/proof_of_form_of_extortionate_strategies/main.pdf}

\newpage

Using the pair wise interactions the transition rates \(p,
q\) can be measured and the steady state probabilities inferred and compared to
the actual probabilities of each state.
This is done numerically by computing the singular eigenvector of the
matrix \(A\) \cite{Stewart2009}:

\[
    A =
    \begin{bmatrix}
        p_1 q_1 & p_1 (1 - q_1) & (1 - p_1) q_1 & (1 -p_1) (1 - q_1) \\
        p_2 q_2 & p_2 (1 - q_2) & (1 - p_2) q_2 & (1 -p_2) (1 - q_2) \\
        p_3 q_3 & p_3 (1 - q_3) & (1 - p_3) q_3 & (1 -p_3) (1 - q_3) \\
        p_4 q_4 & p_4 (1 - q_4) & (1 - p_4) q_4 & (1 -p_4) (1 - q_4) \\
    \end{bmatrix}
\]

Figure~\ref{fig:computed_probabilities_vs_theoretic_probabilities} shows a
regression line fitted to every pairwise interaction with a reported
\(\text{SSError}\) value (pairwise interactions with missing states were
omitted). This serves to validate the approach: a part from some edge cases the
relationship is consistent.

\begin{figure}[!htbp]
    \centering
    \includegraphics[width=.8\textwidth]{./assets/img/computed_probabilities_vs_theoretic_probabilities/main.pdf}
    \caption{The
        relationship between the steady state probabilities inferred from the
        measured transitions and the actual steady state probabilities. A linear
        regression line is included validating the approach.}
    \label{fig:computed_probabilities_vs_theoretic_probabilities}
\end{figure}


\end{document}
 times. All of this
interaction data is available at~\cite{vincent_knight_2018_1297075}. A good
match between the inferred Markov chain and the state distribution of the actual
interactions has been verified. Data for this is presented in the supplementary
materials.

Figure~\ref{fig:SSError_overall_in_stewart_plotkin} shows the \(\text{SSError}\)
values for all the strategies in the tournament, as reported
in~\cite{Stewart2012} the extortionate strategy (which has an expected
\(\text{SSError}\) approximately 0) gains a large number of wins.

\begin{figure}[!htbp]
    \centering
    \includegraphics[width=.8\textwidth]{./assets/img/SSError_overall_in_stewart_plotkin/main.pdf}
    \caption{\(\text{SSError}\) and state probabilities for the strategies
        of~\cite{Stewart2012}, ordered both by number of wins and overall score.
        Note that \(P(DC)\) is not shown as it corresponds to the transpose of
        \(P(CD)\). Cooperator and Defector are omitted as they do not visit all
        the states.}
    \label{fig:SSError_overall_in_stewart_plotkin}
\end{figure}

Here, the work of~\cite{Stewart2012} is extended by investigating a tournament
with \documentclass[a4paper]{article}

\usepackage{amsmath}
\usepackage{amssymb}
\usepackage[margin=1.5cm,
            includefoot,
            footskip=30pt]{geometry}
\usepackage{layout}
\usepackage{graphicx}
\usepackage{subcaption}

\usepackage{biblatex}
\usepackage{pdfpages}

\bibliography{main.bib}

\title{Suspicion: Recognising and evaluating the effectiveness
       of extortion in the Iterated Prisoner's Dilemma}
\author{Vincent A. Knight \and Nikoleta E. Glynatsi}
\date{\today}



\begin{document}

\maketitle

\begin{abstract}
    The Iterated Prisoner's Dilemma is a model for rational and evolutionary
    interactive behaviour. It has applications both in the study of human social
    behaviour as well as in biology.
    It is used to understand when and how a rational individual might
    accept an immediate cost to their own utility for the direct benefit of
    another.

    Much attention has been given to a class of strategies called
    Zero Determinant strategies. It has been theoretically shown that these
    strategies can ``extort'' any player.

    In this work, an approach to identify if observed strategies are playing in
    an extortionate way is described. Furthermore, experimental analysis of
    a large tournament with \input{assets/tex/number_of_full_strategies/main.tex}
    strategies is considered. In this setting
    the most highly performing strategies do not play in an extortionate way
    against each other but do against lower performing strategies.
    This suggests that whilst the theory of Zero Determinant strategies
    indicates that memory is not of fundamental importance to the evolution of
    cooperative behaviour, this is incomplete.
\end{abstract}

\section{Introduction}\label{sec:introduction}

Agent based game theoretic models have become a stalwart of the underpinning
mathematics of interactive behaviours. One of the major pieces of work
in this area is the pair of original computer tournaments run by Robert
Axelrod~\cite{Axelrod1980, Axelrod1980a}. These tournaments pitted submitted
computer strategies against each other in plays of the Iterated Prisoner's
Dilemma. A common game where agents can choose to pay a slight cost to their
immediate utility in the hope of building a reputation. This has been used in
economic and evolutionary game theory to understand the evolution of cooperative
behaviour.

Recently, a class of strategies was described in~\cite{Press2012} that can
provably extort any given opponent. In~\cite{Hilbe2013, Moran1707} some
questions have already been asked about the true effectiveness of these
strategies in an evolutionary setting. Here another question is asked: is it
possible to recognise this extortionate behaviour? A mathematical procedure for
suspicion is presented: in the same way that the continued actions of an
extortionate individual might raise suspicion.

This work makes use of the Axelrod Python library~\cite{Knight2018, Knight2016}
with a large number of Prisoner Dilemma strategies available to give an
extensive numerical example of the ideas presented.  The approach is presented
in Section~\ref{sec:delta-zd-strategies}.  All of the code and data discussed
in Section~\ref{sec:numerical-experiments} is open sourced, archived and
written according to best scientific principles~\cite{Wilson2014}. The data
archive can be found at~\cite{vincent_knight_2018_1297075}.

\section{Recognising Extortion}\label{sec:delta-zd-strategies}

In~\cite{Press2012}, given a match between 2 memory-one strategies, the concept
of Zero Determinant (ZD) strategies is introduced. The main result of that paper
shows that given two memory one players \(p, q\in\mathbb{R}^4\) a linear
relationship between the players' scores could be forced by one of the players.

Using the notation of~\cite{Press2012}, assuming the utilities for player \(p\)
are given by \(S_x=(R, S, T, P)\) and for player \(q\) by \(S_y=(R, T, S, P)\)
and that the stationary scores of each player is given by \(S_X\) and \(S_Y\)
respectively. The main result of~\cite{Press2012} is that if

\begin{equation}\label{eqn:linear_relationship_for_p}
    \tilde p=\alpha S_x + \beta S_y + \gamma
\end{equation}

or

\begin{equation}\label{eqn:linear_relationship_for_q}
    \tilde q=\alpha S_x + \beta S_y + \gamma
\end{equation}

where \(\tilde p = (1 - p_1, 1 - p_2, p_3, p_4)\) and
\(\tilde q = (1 - q_1, 1 - q_2, q_3, q_4)\) then:

\begin{equation}
    \alpha S_X + \beta S_Y + \gamma = 0
\end{equation}

In~\cite{Press2012} a particular type of ZD strategy is defined: extortionate
strategies. If:

\begin{equation}\label{eqn:constraint_for_extortion}
    \gamma = - P(\alpha + \beta)
\end{equation}

then the player can ensure they get a score \(\chi\) times
larger than the opponent. This extortion coefficient is given by:

\begin{equation}\label{eqn:definition_of_chi}
    \chi=\frac{-\beta}{\alpha}
\end{equation}

Thus, if (\ref{eqn:constraint_for_extortion}) holds and \(\chi >1\) a player is
said to extort their opponent.
Here, the reverse problem is considered: given a
\(p\in\mathbb{R}^4\) how does one identify \(\alpha, \beta\) if they
exist and is the strategy in fact acting in an extortionate way?

These conditions correspond to:

\begin{align}
    \tilde p_1 & = \alpha R + \beta R - P (\alpha + \beta)
            \label{eqn:condition_for_tilde_p1}\\
    \tilde p_2 & = \alpha S + \beta T - P (\alpha + \beta)
            \label{eqn:condition_for_tilde_p2}\\
    \tilde p_3 & = \alpha T + \beta S - P (\alpha + \beta)
            \label{eqn:condition_for_tilde_p3}\\
    \tilde p_4 & = \alpha P + \beta P - P (\alpha + \beta)
            \label{eqn:condition_for_tilde_p4}
\end{align}

Equation (\ref{eqn:condition_for_tilde_p4}) ensures that \(p_4=\tilde p_4=0\).
Equations (\ref{eqn:condition_for_tilde_p1}-\ref{eqn:condition_for_tilde_p3})
can be used to eliminate \(\alpha, \beta\), giving:

\begin{equation}\label{eqn:planar_definition_of_extortion}
    \tilde p_1 = \frac{(R - P)(\tilde p_2 + \tilde p_3)}{S + T - 2P}
\end{equation}

with:

\begin{equation}\label{eqn:definition_of_chi}
    \chi = \frac{\tilde p_2 (P - T) + \tilde p_3 (S - P)}
                {\tilde p_2 (P - S) + \tilde p_3 (T - P)}
\end{equation}

Given a strategy \(p\in\mathbb{R}^{4\times 1}\) equations
(\ref{eqn:condition_for_tilde_p4}), (\ref{eqn:planar_definition_of_extortion}-\ref{eqn:definition_of_chi}) can be used to check if
a strategy is extortionate. The conditions correspond to:

\begin{align}
    p_1 & = \frac{(R-P)(p_2 + p_3) - R + T + S - P}{S + T - 2P}
     \label{eqn:condition_for_p1}\\
    p_4 & = 0 \label{eqn:condition_for_p4}\\
    1 & > p_2 + p_3\label{eqn:condition_for_chi}
\end{align}

The algebraic steps necessary to prove these results are available in the
supporting materials.

All extortionate strategies reside on a triangular (\ref{eqn:condition_for_chi})
plane (\ref{eqn:condition_for_p1}) in 3 dimensions (\ref{eqn:condition_for_p4}).
Using this formulation it can be seen that a necessary (but not sufficient)
condition for an extortionate strategy is that it cooperates on average less
than 50\% of the time when in a state of disagreement with the opponent.

As an example, consider the known extortionate strategy \(p=(8 / 9, 1 / 2, 1 /
3, 0)\) from~\cite{Stewart2012} which is referred to as \texttt{Extort-2}. In
this case, for the standard values of \((R, T, S, P)\) constraint
(\ref{eqn:condition_for_p1}) corresponds to:

\begin{equation}
    p_1 = \frac{2(p_2 + p_3) + 1}{3}
\end{equation}

It is clear that in this case all constraints hold.

This approach could in fact be used to confirm that a given strategy is acting
in an extortionate manner even if it is not a memory one strategy. However, in
practice, if a closed form for \(p\) is not known, then due to measurement
and/or numerical error this would not work.

This problem can be written in the following linear algebraic form where
\(x=(\alpha, \beta)\)
and \(p^*=(\tilde p_1 - 1, tilde_2 - 1, p_3)\):

\begin{equation}\label{eqn:linear_algebraic_equation_for_p}
    Cx= p^*
\end{equation}

\(C\) corresponds to equations
(\ref{eqn:condition_for_tilde_p1}-\ref{eqn:condition_for_tilde_p3}) and is
given by:

\begin{equation}\label{eqn:definition_of_C}
    C =
    \begin{bmatrix}
        R - P & R- P \\
        S - P & T- P \\
        T - P & S- P \\
    \end{bmatrix}
\end{equation}

Note that in general, equation (\ref{eqn:linear_algebraic_equation_for_p}) will
not necessarily have a solution. From the Rouch\'{e}-Capelli theorem if there is
a solution it is unique as \(\text{rank}(C)=2\) which is the dimension of the
variable \(x\). The best fitting \(x\) is found by minimizing:

\begin{equation}\label{eqn:r_squared}
    \text{SSError} = \|C x- p^*\|_2^2 = \sum_{i=1}^{3}\left((C\bar x)_i-p_i^*\right)^2
\end{equation}

Note that \(\text{SSError}\), which is the square of the Frobenius
norm~\cite{Golub2013}, becomes a measure of how close a strategy is to being an
extortionate strategy. Suspicion
of extortion then corresponds to a threshold on \(\text{SSError}\).

By observing interactions (human or otherwise), their memory one representation
can be inferred and this approach can be used to recognise extortionate
behaviour. The notion of comparing theoretic and actual plays of the IPD is not
novel, see for example~\cite{Rand2013}. Immediately it is noted that if the
environment is noisy~\cite{Wu1995} then no strategy can be considered to be
extortionate as \(p_4>0\).

In the next section, this idea will be illustrated by observing the interactions
that take place in a computer based tournament of the IPD\@.

\section{Numerical experiments}\label{sec:numerical-experiments}

In~\cite{Stewart2012} results from a tournament with
\input{./assets/tex/number_of_stewart_plotkin_strategies/main.tex} strategies,
was presented with specific consideration given to ZD strategies. This
tournament is reproduced here using the Axelrod-Python
project~\cite{Knight2016}. To obtain a good measure of the corresponding
transition rates for each strategy all matches have been run for
\input{assets/tex/number_of_turns/main.tex} turns and every match has been
repeated \input{assets/tex/number_of_repetitions/main.tex} times. All of this
interaction data is available at~\cite{vincent_knight_2018_1297075}. A good
match between the inferred Markov chain and the state distribution of the actual
interactions has been verified. Data for this is presented in the supplementary
materials.

Figure~\ref{fig:SSError_overall_in_stewart_plotkin} shows the \(\text{SSError}\)
values for all the strategies in the tournament, as reported
in~\cite{Stewart2012} the extortionate strategy (which has an expected
\(\text{SSError}\) approximately 0) gains a large number of wins.

\begin{figure}[!htbp]
    \centering
    \includegraphics[width=.8\textwidth]{./assets/img/SSError_overall_in_stewart_plotkin/main.pdf}
    \caption{\(\text{SSError}\) and state probabilities for the strategies
        of~\cite{Stewart2012}, ordered both by number of wins and overall score.
        Note that \(P(DC)\) is not shown as it corresponds to the transpose of
        \(P(CD)\). Cooperator and Defector are omitted as they do not visit all
        the states.}
    \label{fig:SSError_overall_in_stewart_plotkin}
\end{figure}

Here, the work of~\cite{Stewart2012} is extended by investigating a tournament
with \input{assets/tex/number_of_full_strategies/main.tex}
strategies.

The results of this analysis are shown in
Figure~\ref{fig:SSError_and_probabilities_in_full}. The top ranking strategies
by number of wins seem to be extortionate (but not against all strategies) and
it can be seen that a small sub group of strategies achieve mutual defection.
All the top ranking strategies according to score achieve mutual cooperation and
do not extort each other, however they
\textbf{do} exhibit extortionate behaviour towards a number of the lower ranking
strategies.

\begin{figure}[!htbp]
    \centering
    \includegraphics[width=.8\textwidth]{./assets/img/SSError_and_probabilities_in_full/main.pdf}
    \caption{\(\text{SSError}\) for the strategies for the full tournament. Only
    strategy interactions for which \(p_4=0\) and \(\chi>1\) are displayed.}
    \label{fig:SSError_and_probabilities_in_full}
\end{figure}

\section{Conclusion}\label{sec:conclusion}

This work defines an approach to measure whether or not a player is playing a
strategy that corresponds to an extortionate strategy as defined
in~\cite{Press2012}: a mathematical model for suspicion. Indeed, all
extortionate strategies have been
 classified as lying on a triangular plane.
This rigorous classification fails to be robust to small measurement error, thus
a statistical approach is proposed.
This is done through a linear algebraic approach for approximating the solution
of a linear system. Using this, a large number of pairwise interactions is
simulated and in fact very few strategies are found to act extortionately.

The work of~\cite{Press2012}, whilst showing that a clever approach to taking
advantage of another memory one strategy exists: this is incomplete. Whilst the
elegance of this result is very attractive, just as the simplicity of the
victory of Tit For Tat in Axelrod's original tournaments was, it is incomplete.
Extortionate strategies achieve a high number of wins but they do not
achieve a high score which corresponds to the fitness landscape in an
evolutionary sense. From the large number of interactions a payoff matrix \(S\)
can be measured where \(S_{ij}\) denotes the score (using standard values of
\((R, S, T, P) = (3, 0, 5, 1)\)) of the \(i\)th strategy
against the \(j\)th strategy. Using this, the replicator equation
describes the evolution of the system based on a population density fitness
function:

\begin{equation}\label{eqn:replicator_dynamics}
    \frac{dx}{dt} = x(S-x^TS x)
\end{equation}

Equation (\ref{eqn:replicator_dynamics}) is solved numerically through an
integration technique described in~\cite{Petzold1983} and
Figure~\ref{fig:replicator_dynamics} shows the evolution of the distribution of
the system: the various strategies are ranked by scores. It is clear to see that
only the high ranking strategies survive the evolutionary process (in fact,
only \input{./assets/img/replicator_dynamics/main.tex}
have a final distribution greater than \(10 ^ {-2}\)). This confirms the
findings of~\cite{Moran1707} in which sophisticated strategies resist
evolutionary invasion of shorter memory strategies. Recalling
Figure~\ref{fig:SSError_and_probabilities_in_full} this demonstrates that:

\begin{itemize}
    \item Cooperation emerges through the evolutionary process: the high scoring
        strategies do not exhibit extortionate behaviour towards each other.
    \item Extortionate strategies do not survive the evolutionary process.
\end{itemize}

\begin{figure}[!htbp]
    \centering
    \includegraphics[width=.8\textwidth]{./assets/img/replicator_dynamics/main.pdf}
    \caption{Numerical simulation of the replicator equation
    (\ref{eqn:replicator_dynamics}): strategies are ordered by score, only the strategies with a high score survive the evolutionary process.}
    \label{fig:replicator_dynamics}
\end{figure}

This work can be used to classify plays of the IPD\@: data can be collected from
actual interactions (in lab or in the field). Furthermore, this allows for a
classification method similar to the notion of fingerprinting presented
in~\cite{Ashlock2008}. Trained strategies can potentially be classified as
extortionate or not or it could be possible to even constrain the reinforcement
learning approaches that are becoming prevalent in the literature.
Alternatively, this mathematical approach for recognising extortion could be
used in sophisticated strategies to defend against invasion. Arguably, some of
the strategies considered here exhibit this behaviour, indeed as described
in~\cite{Harper2017}, the top ranking strategies in the full tournament are
obtained using evolutionary reinforcement learning techniques, thus, suspicion
of extortionate behaviour could in fact be an evolutionary trait.

\section*{Acknowledgements}

The following open source software libraries were used in this research:

\begin{itemize}
    \item The Axelrod ~\cite{Knight2016, Knight2018} library (IPD strategies and
        tournaments).
    \item The sympy library~\cite{Meurer2017} (verification of all symbolic
        calculations).
    \item The matplotlib~\cite{Droettboom2018} library (visualisation).
    \item The pandas~\cite{Structures2010}, dask~\cite{Dask2016} and
        NumPy~\cite{Oliphant2015} libraries (data manipulation).
    \item The SciPy~\cite{Jones2001} library (numerical integration of the
        replicator equation).
\end{itemize}

This work was performed using the computational facilities of the Advanced
Research Computing @ Cardiff (ARCCA) Division, Cardiff University.

\printbibliography

\newpage
\section*{Supplementary materials}

\includepdf{assets/pdf/proof_of_form_of_extortionate_strategies/main.pdf}

\newpage

Using the pair wise interactions the transition rates \(p,
q\) can be measured and the steady state probabilities inferred and compared to
the actual probabilities of each state.
This is done numerically by computing the singular eigenvector of the
matrix \(A\) \cite{Stewart2009}:

\[
    A =
    \begin{bmatrix}
        p_1 q_1 & p_1 (1 - q_1) & (1 - p_1) q_1 & (1 -p_1) (1 - q_1) \\
        p_2 q_2 & p_2 (1 - q_2) & (1 - p_2) q_2 & (1 -p_2) (1 - q_2) \\
        p_3 q_3 & p_3 (1 - q_3) & (1 - p_3) q_3 & (1 -p_3) (1 - q_3) \\
        p_4 q_4 & p_4 (1 - q_4) & (1 - p_4) q_4 & (1 -p_4) (1 - q_4) \\
    \end{bmatrix}
\]

Figure~\ref{fig:computed_probabilities_vs_theoretic_probabilities} shows a
regression line fitted to every pairwise interaction with a reported
\(\text{SSError}\) value (pairwise interactions with missing states were
omitted). This serves to validate the approach: a part from some edge cases the
relationship is consistent.

\begin{figure}[!htbp]
    \centering
    \includegraphics[width=.8\textwidth]{./assets/img/computed_probabilities_vs_theoretic_probabilities/main.pdf}
    \caption{The
        relationship between the steady state probabilities inferred from the
        measured transitions and the actual steady state probabilities. A linear
        regression line is included validating the approach.}
    \label{fig:computed_probabilities_vs_theoretic_probabilities}
\end{figure}


\end{document}

strategies.

The results of this analysis are shown in
Figure~\ref{fig:SSError_and_probabilities_in_full}. The top ranking strategies
by number of wins seem to be extortionate (but not against all strategies) and
it can be seen that a small sub group of strategies achieve mutual defection.
All the top ranking strategies according to score achieve mutual cooperation and
do not extort each other, however they
\textbf{do} exhibit extortionate behaviour towards a number of the lower ranking
strategies.

\begin{figure}[!htbp]
    \centering
    \includegraphics[width=.8\textwidth]{./assets/img/SSError_and_probabilities_in_full/main.pdf}
    \caption{\(\text{SSError}\) for the strategies for the full tournament. Only
    strategy interactions for which \(p_4=0\) and \(\chi>1\) are displayed.}
    \label{fig:SSError_and_probabilities_in_full}
\end{figure}

\section{Conclusion}\label{sec:conclusion}

This work defines an approach to measure whether or not a player is playing a
strategy that corresponds to an extortionate strategy as defined
in~\cite{Press2012}: a mathematical model for suspicion. Indeed, all
extortionate strategies have been
 classified as lying on a triangular plane.
This rigorous classification fails to be robust to small measurement error, thus
a statistical approach is proposed.
This is done through a linear algebraic approach for approximating the solution
of a linear system. Using this, a large number of pairwise interactions is
simulated and in fact very few strategies are found to act extortionately.

The work of~\cite{Press2012}, whilst showing that a clever approach to taking
advantage of another memory one strategy exists: this is incomplete. Whilst the
elegance of this result is very attractive, just as the simplicity of the
victory of Tit For Tat in Axelrod's original tournaments was, it is incomplete.
Extortionate strategies achieve a high number of wins but they do not
achieve a high score which corresponds to the fitness landscape in an
evolutionary sense. From the large number of interactions a payoff matrix \(S\)
can be measured where \(S_{ij}\) denotes the score (using standard values of
\((R, S, T, P) = (3, 0, 5, 1)\)) of the \(i\)th strategy
against the \(j\)th strategy. Using this, the replicator equation
describes the evolution of the system based on a population density fitness
function:

\begin{equation}\label{eqn:replicator_dynamics}
    \frac{dx}{dt} = x(S-x^TS x)
\end{equation}

Equation (\ref{eqn:replicator_dynamics}) is solved numerically through an
integration technique described in~\cite{Petzold1983} and
Figure~\ref{fig:replicator_dynamics} shows the evolution of the distribution of
the system: the various strategies are ranked by scores. It is clear to see that
only the high ranking strategies survive the evolutionary process (in fact,
only \documentclass[a4paper]{article}

\usepackage{amsmath}
\usepackage{amssymb}
\usepackage[margin=1.5cm,
            includefoot,
            footskip=30pt]{geometry}
\usepackage{layout}
\usepackage{graphicx}
\usepackage{subcaption}

\usepackage{biblatex}
\usepackage{pdfpages}

\bibliography{main.bib}

\title{Suspicion: Recognising and evaluating the effectiveness
       of extortion in the Iterated Prisoner's Dilemma}
\author{Vincent A. Knight \and Nikoleta E. Glynatsi}
\date{\today}



\begin{document}

\maketitle

\begin{abstract}
    The Iterated Prisoner's Dilemma is a model for rational and evolutionary
    interactive behaviour. It has applications both in the study of human social
    behaviour as well as in biology.
    It is used to understand when and how a rational individual might
    accept an immediate cost to their own utility for the direct benefit of
    another.

    Much attention has been given to a class of strategies called
    Zero Determinant strategies. It has been theoretically shown that these
    strategies can ``extort'' any player.

    In this work, an approach to identify if observed strategies are playing in
    an extortionate way is described. Furthermore, experimental analysis of
    a large tournament with \input{assets/tex/number_of_full_strategies/main.tex}
    strategies is considered. In this setting
    the most highly performing strategies do not play in an extortionate way
    against each other but do against lower performing strategies.
    This suggests that whilst the theory of Zero Determinant strategies
    indicates that memory is not of fundamental importance to the evolution of
    cooperative behaviour, this is incomplete.
\end{abstract}

\section{Introduction}\label{sec:introduction}

Agent based game theoretic models have become a stalwart of the underpinning
mathematics of interactive behaviours. One of the major pieces of work
in this area is the pair of original computer tournaments run by Robert
Axelrod~\cite{Axelrod1980, Axelrod1980a}. These tournaments pitted submitted
computer strategies against each other in plays of the Iterated Prisoner's
Dilemma. A common game where agents can choose to pay a slight cost to their
immediate utility in the hope of building a reputation. This has been used in
economic and evolutionary game theory to understand the evolution of cooperative
behaviour.

Recently, a class of strategies was described in~\cite{Press2012} that can
provably extort any given opponent. In~\cite{Hilbe2013, Moran1707} some
questions have already been asked about the true effectiveness of these
strategies in an evolutionary setting. Here another question is asked: is it
possible to recognise this extortionate behaviour? A mathematical procedure for
suspicion is presented: in the same way that the continued actions of an
extortionate individual might raise suspicion.

This work makes use of the Axelrod Python library~\cite{Knight2018, Knight2016}
with a large number of Prisoner Dilemma strategies available to give an
extensive numerical example of the ideas presented.  The approach is presented
in Section~\ref{sec:delta-zd-strategies}.  All of the code and data discussed
in Section~\ref{sec:numerical-experiments} is open sourced, archived and
written according to best scientific principles~\cite{Wilson2014}. The data
archive can be found at~\cite{vincent_knight_2018_1297075}.

\section{Recognising Extortion}\label{sec:delta-zd-strategies}

In~\cite{Press2012}, given a match between 2 memory-one strategies, the concept
of Zero Determinant (ZD) strategies is introduced. The main result of that paper
shows that given two memory one players \(p, q\in\mathbb{R}^4\) a linear
relationship between the players' scores could be forced by one of the players.

Using the notation of~\cite{Press2012}, assuming the utilities for player \(p\)
are given by \(S_x=(R, S, T, P)\) and for player \(q\) by \(S_y=(R, T, S, P)\)
and that the stationary scores of each player is given by \(S_X\) and \(S_Y\)
respectively. The main result of~\cite{Press2012} is that if

\begin{equation}\label{eqn:linear_relationship_for_p}
    \tilde p=\alpha S_x + \beta S_y + \gamma
\end{equation}

or

\begin{equation}\label{eqn:linear_relationship_for_q}
    \tilde q=\alpha S_x + \beta S_y + \gamma
\end{equation}

where \(\tilde p = (1 - p_1, 1 - p_2, p_3, p_4)\) and
\(\tilde q = (1 - q_1, 1 - q_2, q_3, q_4)\) then:

\begin{equation}
    \alpha S_X + \beta S_Y + \gamma = 0
\end{equation}

In~\cite{Press2012} a particular type of ZD strategy is defined: extortionate
strategies. If:

\begin{equation}\label{eqn:constraint_for_extortion}
    \gamma = - P(\alpha + \beta)
\end{equation}

then the player can ensure they get a score \(\chi\) times
larger than the opponent. This extortion coefficient is given by:

\begin{equation}\label{eqn:definition_of_chi}
    \chi=\frac{-\beta}{\alpha}
\end{equation}

Thus, if (\ref{eqn:constraint_for_extortion}) holds and \(\chi >1\) a player is
said to extort their opponent.
Here, the reverse problem is considered: given a
\(p\in\mathbb{R}^4\) how does one identify \(\alpha, \beta\) if they
exist and is the strategy in fact acting in an extortionate way?

These conditions correspond to:

\begin{align}
    \tilde p_1 & = \alpha R + \beta R - P (\alpha + \beta)
            \label{eqn:condition_for_tilde_p1}\\
    \tilde p_2 & = \alpha S + \beta T - P (\alpha + \beta)
            \label{eqn:condition_for_tilde_p2}\\
    \tilde p_3 & = \alpha T + \beta S - P (\alpha + \beta)
            \label{eqn:condition_for_tilde_p3}\\
    \tilde p_4 & = \alpha P + \beta P - P (\alpha + \beta)
            \label{eqn:condition_for_tilde_p4}
\end{align}

Equation (\ref{eqn:condition_for_tilde_p4}) ensures that \(p_4=\tilde p_4=0\).
Equations (\ref{eqn:condition_for_tilde_p1}-\ref{eqn:condition_for_tilde_p3})
can be used to eliminate \(\alpha, \beta\), giving:

\begin{equation}\label{eqn:planar_definition_of_extortion}
    \tilde p_1 = \frac{(R - P)(\tilde p_2 + \tilde p_3)}{S + T - 2P}
\end{equation}

with:

\begin{equation}\label{eqn:definition_of_chi}
    \chi = \frac{\tilde p_2 (P - T) + \tilde p_3 (S - P)}
                {\tilde p_2 (P - S) + \tilde p_3 (T - P)}
\end{equation}

Given a strategy \(p\in\mathbb{R}^{4\times 1}\) equations
(\ref{eqn:condition_for_tilde_p4}), (\ref{eqn:planar_definition_of_extortion}-\ref{eqn:definition_of_chi}) can be used to check if
a strategy is extortionate. The conditions correspond to:

\begin{align}
    p_1 & = \frac{(R-P)(p_2 + p_3) - R + T + S - P}{S + T - 2P}
     \label{eqn:condition_for_p1}\\
    p_4 & = 0 \label{eqn:condition_for_p4}\\
    1 & > p_2 + p_3\label{eqn:condition_for_chi}
\end{align}

The algebraic steps necessary to prove these results are available in the
supporting materials.

All extortionate strategies reside on a triangular (\ref{eqn:condition_for_chi})
plane (\ref{eqn:condition_for_p1}) in 3 dimensions (\ref{eqn:condition_for_p4}).
Using this formulation it can be seen that a necessary (but not sufficient)
condition for an extortionate strategy is that it cooperates on average less
than 50\% of the time when in a state of disagreement with the opponent.

As an example, consider the known extortionate strategy \(p=(8 / 9, 1 / 2, 1 /
3, 0)\) from~\cite{Stewart2012} which is referred to as \texttt{Extort-2}. In
this case, for the standard values of \((R, T, S, P)\) constraint
(\ref{eqn:condition_for_p1}) corresponds to:

\begin{equation}
    p_1 = \frac{2(p_2 + p_3) + 1}{3}
\end{equation}

It is clear that in this case all constraints hold.

This approach could in fact be used to confirm that a given strategy is acting
in an extortionate manner even if it is not a memory one strategy. However, in
practice, if a closed form for \(p\) is not known, then due to measurement
and/or numerical error this would not work.

This problem can be written in the following linear algebraic form where
\(x=(\alpha, \beta)\)
and \(p^*=(\tilde p_1 - 1, tilde_2 - 1, p_3)\):

\begin{equation}\label{eqn:linear_algebraic_equation_for_p}
    Cx= p^*
\end{equation}

\(C\) corresponds to equations
(\ref{eqn:condition_for_tilde_p1}-\ref{eqn:condition_for_tilde_p3}) and is
given by:

\begin{equation}\label{eqn:definition_of_C}
    C =
    \begin{bmatrix}
        R - P & R- P \\
        S - P & T- P \\
        T - P & S- P \\
    \end{bmatrix}
\end{equation}

Note that in general, equation (\ref{eqn:linear_algebraic_equation_for_p}) will
not necessarily have a solution. From the Rouch\'{e}-Capelli theorem if there is
a solution it is unique as \(\text{rank}(C)=2\) which is the dimension of the
variable \(x\). The best fitting \(x\) is found by minimizing:

\begin{equation}\label{eqn:r_squared}
    \text{SSError} = \|C x- p^*\|_2^2 = \sum_{i=1}^{3}\left((C\bar x)_i-p_i^*\right)^2
\end{equation}

Note that \(\text{SSError}\), which is the square of the Frobenius
norm~\cite{Golub2013}, becomes a measure of how close a strategy is to being an
extortionate strategy. Suspicion
of extortion then corresponds to a threshold on \(\text{SSError}\).

By observing interactions (human or otherwise), their memory one representation
can be inferred and this approach can be used to recognise extortionate
behaviour. The notion of comparing theoretic and actual plays of the IPD is not
novel, see for example~\cite{Rand2013}. Immediately it is noted that if the
environment is noisy~\cite{Wu1995} then no strategy can be considered to be
extortionate as \(p_4>0\).

In the next section, this idea will be illustrated by observing the interactions
that take place in a computer based tournament of the IPD\@.

\section{Numerical experiments}\label{sec:numerical-experiments}

In~\cite{Stewart2012} results from a tournament with
\input{./assets/tex/number_of_stewart_plotkin_strategies/main.tex} strategies,
was presented with specific consideration given to ZD strategies. This
tournament is reproduced here using the Axelrod-Python
project~\cite{Knight2016}. To obtain a good measure of the corresponding
transition rates for each strategy all matches have been run for
\input{assets/tex/number_of_turns/main.tex} turns and every match has been
repeated \input{assets/tex/number_of_repetitions/main.tex} times. All of this
interaction data is available at~\cite{vincent_knight_2018_1297075}. A good
match between the inferred Markov chain and the state distribution of the actual
interactions has been verified. Data for this is presented in the supplementary
materials.

Figure~\ref{fig:SSError_overall_in_stewart_plotkin} shows the \(\text{SSError}\)
values for all the strategies in the tournament, as reported
in~\cite{Stewart2012} the extortionate strategy (which has an expected
\(\text{SSError}\) approximately 0) gains a large number of wins.

\begin{figure}[!htbp]
    \centering
    \includegraphics[width=.8\textwidth]{./assets/img/SSError_overall_in_stewart_plotkin/main.pdf}
    \caption{\(\text{SSError}\) and state probabilities for the strategies
        of~\cite{Stewart2012}, ordered both by number of wins and overall score.
        Note that \(P(DC)\) is not shown as it corresponds to the transpose of
        \(P(CD)\). Cooperator and Defector are omitted as they do not visit all
        the states.}
    \label{fig:SSError_overall_in_stewart_plotkin}
\end{figure}

Here, the work of~\cite{Stewart2012} is extended by investigating a tournament
with \input{assets/tex/number_of_full_strategies/main.tex}
strategies.

The results of this analysis are shown in
Figure~\ref{fig:SSError_and_probabilities_in_full}. The top ranking strategies
by number of wins seem to be extortionate (but not against all strategies) and
it can be seen that a small sub group of strategies achieve mutual defection.
All the top ranking strategies according to score achieve mutual cooperation and
do not extort each other, however they
\textbf{do} exhibit extortionate behaviour towards a number of the lower ranking
strategies.

\begin{figure}[!htbp]
    \centering
    \includegraphics[width=.8\textwidth]{./assets/img/SSError_and_probabilities_in_full/main.pdf}
    \caption{\(\text{SSError}\) for the strategies for the full tournament. Only
    strategy interactions for which \(p_4=0\) and \(\chi>1\) are displayed.}
    \label{fig:SSError_and_probabilities_in_full}
\end{figure}

\section{Conclusion}\label{sec:conclusion}

This work defines an approach to measure whether or not a player is playing a
strategy that corresponds to an extortionate strategy as defined
in~\cite{Press2012}: a mathematical model for suspicion. Indeed, all
extortionate strategies have been
 classified as lying on a triangular plane.
This rigorous classification fails to be robust to small measurement error, thus
a statistical approach is proposed.
This is done through a linear algebraic approach for approximating the solution
of a linear system. Using this, a large number of pairwise interactions is
simulated and in fact very few strategies are found to act extortionately.

The work of~\cite{Press2012}, whilst showing that a clever approach to taking
advantage of another memory one strategy exists: this is incomplete. Whilst the
elegance of this result is very attractive, just as the simplicity of the
victory of Tit For Tat in Axelrod's original tournaments was, it is incomplete.
Extortionate strategies achieve a high number of wins but they do not
achieve a high score which corresponds to the fitness landscape in an
evolutionary sense. From the large number of interactions a payoff matrix \(S\)
can be measured where \(S_{ij}\) denotes the score (using standard values of
\((R, S, T, P) = (3, 0, 5, 1)\)) of the \(i\)th strategy
against the \(j\)th strategy. Using this, the replicator equation
describes the evolution of the system based on a population density fitness
function:

\begin{equation}\label{eqn:replicator_dynamics}
    \frac{dx}{dt} = x(S-x^TS x)
\end{equation}

Equation (\ref{eqn:replicator_dynamics}) is solved numerically through an
integration technique described in~\cite{Petzold1983} and
Figure~\ref{fig:replicator_dynamics} shows the evolution of the distribution of
the system: the various strategies are ranked by scores. It is clear to see that
only the high ranking strategies survive the evolutionary process (in fact,
only \input{./assets/img/replicator_dynamics/main.tex}
have a final distribution greater than \(10 ^ {-2}\)). This confirms the
findings of~\cite{Moran1707} in which sophisticated strategies resist
evolutionary invasion of shorter memory strategies. Recalling
Figure~\ref{fig:SSError_and_probabilities_in_full} this demonstrates that:

\begin{itemize}
    \item Cooperation emerges through the evolutionary process: the high scoring
        strategies do not exhibit extortionate behaviour towards each other.
    \item Extortionate strategies do not survive the evolutionary process.
\end{itemize}

\begin{figure}[!htbp]
    \centering
    \includegraphics[width=.8\textwidth]{./assets/img/replicator_dynamics/main.pdf}
    \caption{Numerical simulation of the replicator equation
    (\ref{eqn:replicator_dynamics}): strategies are ordered by score, only the strategies with a high score survive the evolutionary process.}
    \label{fig:replicator_dynamics}
\end{figure}

This work can be used to classify plays of the IPD\@: data can be collected from
actual interactions (in lab or in the field). Furthermore, this allows for a
classification method similar to the notion of fingerprinting presented
in~\cite{Ashlock2008}. Trained strategies can potentially be classified as
extortionate or not or it could be possible to even constrain the reinforcement
learning approaches that are becoming prevalent in the literature.
Alternatively, this mathematical approach for recognising extortion could be
used in sophisticated strategies to defend against invasion. Arguably, some of
the strategies considered here exhibit this behaviour, indeed as described
in~\cite{Harper2017}, the top ranking strategies in the full tournament are
obtained using evolutionary reinforcement learning techniques, thus, suspicion
of extortionate behaviour could in fact be an evolutionary trait.

\section*{Acknowledgements}

The following open source software libraries were used in this research:

\begin{itemize}
    \item The Axelrod ~\cite{Knight2016, Knight2018} library (IPD strategies and
        tournaments).
    \item The sympy library~\cite{Meurer2017} (verification of all symbolic
        calculations).
    \item The matplotlib~\cite{Droettboom2018} library (visualisation).
    \item The pandas~\cite{Structures2010}, dask~\cite{Dask2016} and
        NumPy~\cite{Oliphant2015} libraries (data manipulation).
    \item The SciPy~\cite{Jones2001} library (numerical integration of the
        replicator equation).
\end{itemize}

This work was performed using the computational facilities of the Advanced
Research Computing @ Cardiff (ARCCA) Division, Cardiff University.

\printbibliography

\newpage
\section*{Supplementary materials}

\includepdf{assets/pdf/proof_of_form_of_extortionate_strategies/main.pdf}

\newpage

Using the pair wise interactions the transition rates \(p,
q\) can be measured and the steady state probabilities inferred and compared to
the actual probabilities of each state.
This is done numerically by computing the singular eigenvector of the
matrix \(A\) \cite{Stewart2009}:

\[
    A =
    \begin{bmatrix}
        p_1 q_1 & p_1 (1 - q_1) & (1 - p_1) q_1 & (1 -p_1) (1 - q_1) \\
        p_2 q_2 & p_2 (1 - q_2) & (1 - p_2) q_2 & (1 -p_2) (1 - q_2) \\
        p_3 q_3 & p_3 (1 - q_3) & (1 - p_3) q_3 & (1 -p_3) (1 - q_3) \\
        p_4 q_4 & p_4 (1 - q_4) & (1 - p_4) q_4 & (1 -p_4) (1 - q_4) \\
    \end{bmatrix}
\]

Figure~\ref{fig:computed_probabilities_vs_theoretic_probabilities} shows a
regression line fitted to every pairwise interaction with a reported
\(\text{SSError}\) value (pairwise interactions with missing states were
omitted). This serves to validate the approach: a part from some edge cases the
relationship is consistent.

\begin{figure}[!htbp]
    \centering
    \includegraphics[width=.8\textwidth]{./assets/img/computed_probabilities_vs_theoretic_probabilities/main.pdf}
    \caption{The
        relationship between the steady state probabilities inferred from the
        measured transitions and the actual steady state probabilities. A linear
        regression line is included validating the approach.}
    \label{fig:computed_probabilities_vs_theoretic_probabilities}
\end{figure}


\end{document}

have a final distribution greater than \(10 ^ {-2}\)). This confirms the
findings of~\cite{Moran1707} in which sophisticated strategies resist
evolutionary invasion of shorter memory strategies. Recalling
Figure~\ref{fig:SSError_and_probabilities_in_full} this demonstrates that:

\begin{itemize}
    \item Cooperation emerges through the evolutionary process: the high scoring
        strategies do not exhibit extortionate behaviour towards each other.
    \item Extortionate strategies do not survive the evolutionary process.
\end{itemize}

\begin{figure}[!htbp]
    \centering
    \includegraphics[width=.8\textwidth]{./assets/img/replicator_dynamics/main.pdf}
    \caption{Numerical simulation of the replicator equation
    (\ref{eqn:replicator_dynamics}): strategies are ordered by score, only the strategies with a high score survive the evolutionary process.}
    \label{fig:replicator_dynamics}
\end{figure}

This work can be used to classify plays of the IPD\@: data can be collected from
actual interactions (in lab or in the field). Furthermore, this allows for a
classification method similar to the notion of fingerprinting presented
in~\cite{Ashlock2008}. Trained strategies can potentially be classified as
extortionate or not or it could be possible to even constrain the reinforcement
learning approaches that are becoming prevalent in the literature.
Alternatively, this mathematical approach for recognising extortion could be
used in sophisticated strategies to defend against invasion. Arguably, some of
the strategies considered here exhibit this behaviour, indeed as described
in~\cite{Harper2017}, the top ranking strategies in the full tournament are
obtained using evolutionary reinforcement learning techniques, thus, suspicion
of extortionate behaviour could in fact be an evolutionary trait.

\section*{Acknowledgements}

The following open source software libraries were used in this research:

\begin{itemize}
    \item The Axelrod ~\cite{Knight2016, Knight2018} library (IPD strategies and
        tournaments).
    \item The sympy library~\cite{Meurer2017} (verification of all symbolic
        calculations).
    \item The matplotlib~\cite{Droettboom2018} library (visualisation).
    \item The pandas~\cite{Structures2010}, dask~\cite{Dask2016} and
        NumPy~\cite{Oliphant2015} libraries (data manipulation).
    \item The SciPy~\cite{Jones2001} library (numerical integration of the
        replicator equation).
\end{itemize}

This work was performed using the computational facilities of the Advanced
Research Computing @ Cardiff (ARCCA) Division, Cardiff University.

\printbibliography

\newpage
\section*{Supplementary materials}

\includepdf{assets/pdf/proof_of_form_of_extortionate_strategies/main.pdf}

\newpage

Using the pair wise interactions the transition rates \(p,
q\) can be measured and the steady state probabilities inferred and compared to
the actual probabilities of each state.
This is done numerically by computing the singular eigenvector of the
matrix \(A\) \cite{Stewart2009}:

\[
    A =
    \begin{bmatrix}
        p_1 q_1 & p_1 (1 - q_1) & (1 - p_1) q_1 & (1 -p_1) (1 - q_1) \\
        p_2 q_2 & p_2 (1 - q_2) & (1 - p_2) q_2 & (1 -p_2) (1 - q_2) \\
        p_3 q_3 & p_3 (1 - q_3) & (1 - p_3) q_3 & (1 -p_3) (1 - q_3) \\
        p_4 q_4 & p_4 (1 - q_4) & (1 - p_4) q_4 & (1 -p_4) (1 - q_4) \\
    \end{bmatrix}
\]

Figure~\ref{fig:computed_probabilities_vs_theoretic_probabilities} shows a
regression line fitted to every pairwise interaction with a reported
\(\text{SSError}\) value (pairwise interactions with missing states were
omitted). This serves to validate the approach: a part from some edge cases the
relationship is consistent.

\begin{figure}[!htbp]
    \centering
    \includegraphics[width=.8\textwidth]{./assets/img/computed_probabilities_vs_theoretic_probabilities/main.pdf}
    \caption{The
        relationship between the steady state probabilities inferred from the
        measured transitions and the actual steady state probabilities. A linear
        regression line is included validating the approach.}
    \label{fig:computed_probabilities_vs_theoretic_probabilities}
\end{figure}


\end{document}

have a final distribution greater than \(10 ^ {-2}\)). This confirms the
findings of~\cite{Moran1707} in which sophisticated strategies resist
evolutionary invasion of shorter memory strategies. Recalling
Figure~\ref{fig:SSError_and_probabilities_in_full} this demonstrates that:

\begin{itemize}
    \item Cooperation emerges through the evolutionary process: the high scoring
        strategies do not exhibit extortionate behaviour towards each other.
    \item Extortionate strategies do not survive the evolutionary process.
\end{itemize}

\begin{figure}[!htbp]
    \centering
    \includegraphics[width=.8\textwidth]{./assets/img/replicator_dynamics/main.pdf}
    \caption{Numerical simulation of the replicator equation
    (\ref{eqn:replicator_dynamics}): strategies are ordered by score, only the strategies with a high score survive the evolutionary process.}
    \label{fig:replicator_dynamics}
\end{figure}

This work can be used to classify plays of the IPD\@: data can be collected from
actual interactions (in lab or in the field). Furthermore, this allows for a
classification method similar to the notion of fingerprinting presented
in~\cite{Ashlock2008}. Trained strategies can potentially be classified as
extortionate or not or it could be possible to even constrain the reinforcement
learning approaches that are becoming prevalent in the literature.
Alternatively, this mathematical approach for recognising extortion could be
used in sophisticated strategies to defend against invasion. Arguably, some of
the strategies considered here exhibit this behaviour, indeed as described
in~\cite{Harper2017}, the top ranking strategies in the full tournament are
obtained using evolutionary reinforcement learning techniques, thus, suspicion
of extortionate behaviour could in fact be an evolutionary trait.

\section*{Acknowledgements}

The following open source software libraries were used in this research:

\begin{itemize}
    \item The Axelrod ~\cite{Knight2016, Knight2018} library (IPD strategies and
        tournaments).
    \item The sympy library~\cite{Meurer2017} (verification of all symbolic
        calculations).
    \item The matplotlib~\cite{Droettboom2018} library (visualisation).
    \item The pandas~\cite{Structures2010}, dask~\cite{Dask2016} and
        NumPy~\cite{Oliphant2015} libraries (data manipulation).
    \item The SciPy~\cite{Jones2001} library (numerical integration of the
        replicator equation).
\end{itemize}

This work was performed using the computational facilities of the Advanced
Research Computing @ Cardiff (ARCCA) Division, Cardiff University.

\printbibliography

\newpage
\section*{Supplementary materials}

\includepdf{assets/pdf/proof_of_form_of_extortionate_strategies/main.pdf}

\newpage

Using the pair wise interactions the transition rates \(p,
q\) can be measured and the steady state probabilities inferred and compared to
the actual probabilities of each state.
This is done numerically by computing the singular eigenvector of the
matrix \(A\) \cite{Stewart2009}:

\[
    A =
    \begin{bmatrix}
        p_1 q_1 & p_1 (1 - q_1) & (1 - p_1) q_1 & (1 -p_1) (1 - q_1) \\
        p_2 q_2 & p_2 (1 - q_2) & (1 - p_2) q_2 & (1 -p_2) (1 - q_2) \\
        p_3 q_3 & p_3 (1 - q_3) & (1 - p_3) q_3 & (1 -p_3) (1 - q_3) \\
        p_4 q_4 & p_4 (1 - q_4) & (1 - p_4) q_4 & (1 -p_4) (1 - q_4) \\
    \end{bmatrix}
\]

Figure~\ref{fig:computed_probabilities_vs_theoretic_probabilities} shows a
regression line fitted to every pairwise interaction with a reported
\(\text{SSError}\) value (pairwise interactions with missing states were
omitted). This serves to validate the approach: a part from some edge cases the
relationship is consistent.

\begin{figure}[!htbp]
    \centering
    \includegraphics[width=.8\textwidth]{./assets/img/computed_probabilities_vs_theoretic_probabilities/main.pdf}
    \caption{The
        relationship between the steady state probabilities inferred from the
        measured transitions and the actual steady state probabilities. A linear
        regression line is included validating the approach.}
    \label{fig:computed_probabilities_vs_theoretic_probabilities}
\end{figure}


\end{document}

strategies demonstrates that sophisticated
strategies can in fact recognise extortionate behaviour and adapt to their
opponents. Further, statistical analysis of these strategies in the context of
evolutionary dynamics demonstrates the importance of adaptability to achieve
evolutionary stability. All of the code and data discussed in
Section~\ref{sec:numerical-experiments} is open sourced, archived, and written
according to best scientific principles~\cite{Wilson2014}. The data archive can
be found at~\cite{vincent_knight_2018_1297075} and the source code was developed
at~\url{https://github.com/drvinceknight/testing_for_ZD/} and has been archived
at~\cite{vincent_knight_2019_2598534}. In
Section~\ref{sec:evolutionary-dynamics}, this large tournament is complemented
with evolutionary dynamics that offer some insight in to the
effectiveness of extortionate strategies.

Several theoretical insights emerge from this work. Infamously, extortionate
strategies do not play well with themselves. In \cite{Press2012},
Press and Dyson claim that a player with a ``theory of mind'' would
rationally chose to cooperate against an opponent that also has knowledge
of zero-determinant strategies to avoid sustained mutual defection. While not
possible for memory-one strategies, we show that this behavior is exhibited by
relatively simple longer memory strategies which previously emerged from an
evolutionary selection process. Similarly, in
\cite{adami2013evolutionary}, Adami and Hintze suggest that there may exist
strategies that are able to selectively behave extortionately to some opponents
and cooperatively to others. We show that this is indeed the case for the same
evolved strategies. It seems that humans have trouble explicitly creating such
strategies but evolution is able to do so by optimizing for total payoff in IPD
interactions. Accordingly, while resistance to extortionate behavior appears
critical to the evolution of cooperation, there is no prohibition on selectively
extorting weaker opponents, even in population dynamics, and this behavior is
evolutionarily advantageous.


\section{Methods: Recognising Extortion}\label{sec:sserror-zd-strategies}

Zero-determinant strategies are a special case of memory-one strategies,
which are defined by elements of \(\mathbb{R}^4\) mapping a state of
\({\{C, D\}}^2\), corresponding to the prior round of play, to a probability of
cooperating in the next round. A match between two such strategies creates a
Markov chain with transient states \({\{C, D\}}^2\). The main result
of~\cite{Press2012} is that given two memory-one players \(p,
q\in\mathbb{R}^4\), a linear relationship between the players' scores can, in
some cases, be forced by one of the players for specific choices of these
probabilities.

Using the notation of~\cite{Press2012}, the utilities for player \(p\)
are given by \(S_x=(R, S, T, P)\) and for player \(q\) by \(S_y=(R, T, S, P)\)
and the stationary scores of each player are given by \(S_X\) and \(S_Y\)
respectively. The main result of~\cite{Press2012} is that if

\begin{equation}\label{eqn:linear_relationship_for_p}
    \tilde p=\alpha S_x + \beta S_y + \gamma
\end{equation}

or

\begin{equation}\label{eqn:linear_relationship_for_q}
    \tilde q=\alpha S_x + \beta S_y + \gamma
\end{equation}

where \(\tilde p = (p_1 - 1, p_2 - 1, p_3, p_4)\) and
\(\tilde q = (q_1 - 1, q_3, q_2 - 1, q_4)\) then:

\begin{equation}
    \alpha S_X + \beta S_Y + \gamma = 0
\end{equation}

Extortionate strategies are defined as follows. If this relationship is
satisfied

\begin{equation}\label{eqn:constraint_for_extortion}
    \gamma = - P(\alpha + \beta)
\end{equation}

then the player can ensure \((S_X - P)=\chi(S_Y-P)\) where:

\begin{equation}\label{eqn:definition_of_chi}
    \chi=\frac{-\beta}{\alpha}
\end{equation}

\noindent Thus, if (\ref{eqn:constraint_for_extortion}) holds and \(\chi >1\) a player is
said to extort their opponent.
In Section~\label{sec:subspace_of_extortionate_strategies}, the reverse problem is considered: given a
\(p\in\mathbb{R}^4\) can one determine if the associated strategy is attempting
to act in an extortionate way?

\subsection{Subspace of Extortionate Strategies}\label{sec:subspace_of_extortionate_strategies}

Constraints (\ref{eqn:linear_relationship_for_p}) and
(\ref{eqn:constraint_for_extortion}) correspond to:

\begin{align}
    \tilde p_1 & = \alpha R + \beta R - P (\alpha + \beta)
            \label{eqn:condition_for_tilde_p1}\\
    \tilde p_2 & = \alpha S + \beta T - P (\alpha + \beta)
            \label{eqn:condition_for_tilde_p2}\\
    \tilde p_3 & = \alpha T + \beta S - P (\alpha + \beta)
            \label{eqn:condition_for_tilde_p3}\\
    \tilde p_4 & = \alpha P + \beta P - P (\alpha + \beta) = 0
            \label{eqn:condition_for_tilde_p4}
\end{align}

Equation (\ref{eqn:condition_for_tilde_p4}) ensures that \(p_4=\tilde p_4=0\).
Equations (\ref{eqn:condition_for_tilde_p1}-\ref{eqn:condition_for_tilde_p3})
can be used to eliminate \(\alpha, \beta\), giving:

\begin{equation}\label{eqn:planar_definition_of_extortion}
    \tilde p_1 = \frac{(R - P)(\tilde p_2 + \tilde p_3)}{S + T - 2P}
\end{equation}

with:

\begin{equation}\label{eqn:definition_of_chi}
    \chi = \frac{\tilde p_2 (P - T) + \tilde p_3 (S - P)}
                {\tilde p_2 (P - S) + \tilde p_3 (T - P)}
\end{equation}

Given a strategy \(p\in\mathbb{R}^{4}\) equations
(\ref{eqn:condition_for_tilde_p4}-\ref{eqn:definition_of_chi}) can be used to
check if a strategy is extortionate. The conditions correspond to:

\begin{align}
    p_1 & = \frac{(R-P)(p_2 + p_3) - R + T + S - P}{S + T - 2P}
     \label{eqn:condition_for_p1}\\
    p_4 & = 0 \label{eqn:condition_for_p4}\\
    1 & > p_2 + p_3\label{eqn:condition_for_chi}
\end{align}

The algebraic steps necessary to prove these results are available in the
supporting materials, and note that an equivalent formulation was obtained
in~\cite{adami2013evolutionary}.

All extortionate strategies reside on a triangular (\ref{eqn:condition_for_chi})
plane (\ref{eqn:condition_for_p1}) in 3 dimensions (\ref{eqn:condition_for_p4}).
Using this formulation it can be seen that a necessary (but not sufficient)
condition for an extortionate strategy is that it cooperates on average less
than 50\% of the time when in a state of disagreement with the opponent
(\ref{eqn:condition_for_chi}).

As an example, consider the known extortionate strategy \(p=(8 / 9, 1 / 2, 1 /
3, 0)\) from~\cite{Stewart2012} which is referred to as Extort-2. In
this case, for the standard values of \((R, S, T, P) = (3, 0, 5, 1)\)
constraint (\ref{eqn:condition_for_p1}) corresponds to:

\begin{equation}
    p_1 = \frac{2(p_2 + p_3) + 1}{3}
        = \frac{2(1 / 2 + 1 / 3) + 1}{3}
        = \frac{8}{9}
\end{equation}

It is clear that in this case all constraints hold. As a counterexample,
consider the strategy that cooperates 25\% of the time: \(p=(1 /4, 1 / 4, 1 / 4,
1 / 4)\) satisfies~(\ref{eqn:condition_for_chi}) but is not extortionate as:

\begin{equation}
    p_1 \ne \frac{2(p_2 + p_3) + 1}{3}
        = \frac{2(1 / 4 + 1 / 4) + 1}{3}
        = \frac{2}{3}
\end{equation}

\subsection{Measuring Extortion from the History of Play}

Not all strategies are memory-one strategies but it is possible to
measure a given \(p\) from any set of interactions between two strategies.
This approach can then be used to confirm that a given strategy is acting
in an extortionate manner even if it is not a memory-one strategy. However, in
practice, if an exact form for \(p\) is not known but measured from observed
plays of the game then measurement and/or numerical error might lead to an
extortionate strategy not being confirmed as such. \footnote{Comparing theoretic
and actual plays of the IPD is not novel, see for example~\cite{Rand2013}.}


As an example consider Table~\ref{tab:actual_plays_of_ZDextort-2} which shows
some actual plays of Extort-2 (\(p=(8 / 9, 1 / 2, 1 / 3, 0)\)) against an
alternating strategy (\(p=(0, 0, 1, 1)\)). In this particular instance the
measured value of \(p\) for the known extortionate strategy would be:
\((2/2, 1/5, 3/8, 0/4)\) which does not fit the definition of a ZD strategy.


\begin{table}[!hbtp]
    \begin{tabular}{lllllllllllllllllllll}
\toprule
Turn               &  1 &  2 &  3 &  4 &  5 &  6 &  7 &  8 &  9 &  10 &  11 &  12 &  13 &  14 &  15 &  16 &  17 &  18 &  19 &  20 \\\midrule
(8/9, 1/2, 1/3, 0) &  C &  C &  D &  D &  D &  C &  D &  D &  D &   D &   D &   C &   C &   C &   D &   D &   D &   C &   D &   D \\
Alternator         &  C &  D &  C &  D &  C &  D &  C &  D &  C &   D &   C &   D &   C &   D &   C &   D &   C &   D &   C &   D \\
\bottomrule
\end{tabular}

    \caption{A seeded play of 20 turns of two strategies.}
    \label{tab:actual_plays_of_ZDextort-2}
\end{table}


Note that measurement of behaviour might in some cases lead to missing values.
For example the strategy \(p=(8 / 9, 1 / 2, 1 / 3, 0)\) when playing against an
opponent that always cooperates will in fact never visit any state which would allow measurement
of \(p_3\) and \(p_4\). To overcome this, it is proposed that if \(s\) is a state
that is not visited then \(p_s\) is approximated using a sensible prior or
imputation. In Section~\ref{sec:numerical-experiments} the overall cooperation
rate is used. Another approach to overcoming this measurement error would be to
measure strategies in a sufficiently noisy environment.

We can measure how close a strategy is to being zero determinant using standard
linear algebraic approaches. Essentially we attempt to find \(x=(\alpha,
\beta)\) such that:

\begin{equation}\label{eqn:linear_algebraic_equation_for_p}
    Cx= \tilde p
\end{equation}

where \(C\) corresponds to equations
(\ref{eqn:condition_for_tilde_p1}-\ref{eqn:condition_for_tilde_p3}) and is
given by:

\begin{equation}\label{eqn:definition_of_C}
    C =
    \begin{bmatrix}
        R - P & R- P \\
        S - P & T- P \\
        T - P & S- P \\
        0     & 0 \\
    \end{bmatrix}
\end{equation}

Note that in general, equation (\ref{eqn:linear_algebraic_equation_for_p}) will
not necessarily have a solution. From the Rouch\'{e}-Capelli theorem if there is
a solution it is unique since \(\text{rank}(C)=2\) which is the dimension of the
variable \(x\). The best fitting \(x^*\) is defined by:

\begin{equation}\label{eqn:x_star}
    x^* = \text{argmin}_{x\in\mathbb{R}^2}\|C x- \tilde p\|_2^2
\end{equation}

Known results~\cite{kutner2004applied, rao1973linear, wakefield2013bayesian}
yield $x^*$, corresponding to the nearest extortionate strategy to the measured
\(\tilde p\). It is in fact an orthogonal projection of \(\tilde p\) on to the
plane defined by (\ref{eqn:condition_for_p1}).

\begin{equation}\label{eqn:x_star_formula}
    x^* = {\left(C^{T}C\right)}^{-1}C^{T}\tilde p
\end{equation}

The squared norm of the remaining error is referred to as sum of squared errors
of prediction (\(\SSe\)):

\begin{equation}\label{eqn:r_squared}
    \SSe = \|C x^*- \tilde p\|_2^2
\end{equation}

This gives expressions for \(\alpha, \beta\) as \(\alpha=x^*_1\) and
\(\beta=x^*_2\) thus the conditions for a strategy to be acting extortionately
becomes:

\begin{equation}
    -x^*_2 < x^*_1 \label{eqn:measured_condition_for_chi}
\end{equation}

A further known result~~\cite{kutner2004applied, rao1973linear,
wakefield2013bayesian} gives an expression for
\(\SSe\):

\begin{equation}\label{eqn:x_SSError_formula}
    \SSe = {\tilde p} ^ T \tilde p
           \tilde p C \left(C ^ T C \right) ^ {-1} C ^ T \tilde p
         = {\tilde p} ^ T \tilde p - \tilde p C x ^ *
\end{equation}

Using this approach, the memory-one representation \(p\in\mathbb{R}^4\) of any
strategy against any other can can be measured and if
(\ref{eqn:measured_condition_for_chi}) holds then (\ref{eqn:x_SSError_formula})
can be used to identify if a strategy is acting extortionately. While the
specific memory-one representation might not be one that acts extortionately, a
high \(\SSe\) does imply that a strategy is not extortionate. For a measured
\(p\), \(\SSe\) corresponds to the best fitting \(\alpha, \beta\). Suspicion of
extortion then corresponds to a threshold on \(\SSe\) and a comparison of the
measured \(\chi=\frac{-\beta}{\alpha}\).

\section{Results: Validation of approach and Numerical experiments}\label{sec:numerical-experiments}

\subsection{Validation}

To validate the method described, we use~\cite{Stewart2012} which
presents results from a tournament with
\documentclass[a4paper]{article}

\usepackage{amsmath}
\usepackage{amssymb}
\usepackage[margin=1.5cm,
            includefoot,
            footskip=30pt]{geometry}
\usepackage{layout}
\usepackage{graphicx}
\usepackage{subcaption}

\usepackage{biblatex}
\usepackage{pdfpages}

\bibliography{main.bib}

\title{Suspicion: Recognising and evaluating the effectiveness
       of extortion in the Iterated Prisoner's Dilemma}
\author{Vincent A. Knight \and Nikoleta E. Glynatsi}
\date{\today}



\begin{document}

\maketitle

\begin{abstract}
    The Iterated Prisoner's Dilemma is a model for rational and evolutionary
    interactive behaviour. It has applications both in the study of human social
    behaviour as well as in biology.
    It is used to understand when and how a rational individual might
    accept an immediate cost to their own utility for the direct benefit of
    another.

    Much attention has been given to a class of strategies called
    Zero Determinant strategies. It has been theoretically shown that these
    strategies can ``extort'' any player.

    In this work, an approach to identify if observed strategies are playing in
    an extortionate way is described. Furthermore, experimental analysis of
    a large tournament with \documentclass[a4paper]{article}

\usepackage{amsmath}
\usepackage{amssymb}
\usepackage[margin=1.5cm,
            includefoot,
            footskip=30pt]{geometry}
\usepackage{layout}
\usepackage{graphicx}
\usepackage{subcaption}

\usepackage{biblatex}
\usepackage{pdfpages}

\bibliography{main.bib}

\title{Suspicion: Recognising and evaluating the effectiveness
       of extortion in the Iterated Prisoner's Dilemma}
\author{Vincent A. Knight \and Nikoleta E. Glynatsi}
\date{\today}



\begin{document}

\maketitle

\begin{abstract}
    The Iterated Prisoner's Dilemma is a model for rational and evolutionary
    interactive behaviour. It has applications both in the study of human social
    behaviour as well as in biology.
    It is used to understand when and how a rational individual might
    accept an immediate cost to their own utility for the direct benefit of
    another.

    Much attention has been given to a class of strategies called
    Zero Determinant strategies. It has been theoretically shown that these
    strategies can ``extort'' any player.

    In this work, an approach to identify if observed strategies are playing in
    an extortionate way is described. Furthermore, experimental analysis of
    a large tournament with \documentclass[a4paper]{article}

\usepackage{amsmath}
\usepackage{amssymb}
\usepackage[margin=1.5cm,
            includefoot,
            footskip=30pt]{geometry}
\usepackage{layout}
\usepackage{graphicx}
\usepackage{subcaption}

\usepackage{biblatex}
\usepackage{pdfpages}

\bibliography{main.bib}

\title{Suspicion: Recognising and evaluating the effectiveness
       of extortion in the Iterated Prisoner's Dilemma}
\author{Vincent A. Knight \and Nikoleta E. Glynatsi}
\date{\today}



\begin{document}

\maketitle

\begin{abstract}
    The Iterated Prisoner's Dilemma is a model for rational and evolutionary
    interactive behaviour. It has applications both in the study of human social
    behaviour as well as in biology.
    It is used to understand when and how a rational individual might
    accept an immediate cost to their own utility for the direct benefit of
    another.

    Much attention has been given to a class of strategies called
    Zero Determinant strategies. It has been theoretically shown that these
    strategies can ``extort'' any player.

    In this work, an approach to identify if observed strategies are playing in
    an extortionate way is described. Furthermore, experimental analysis of
    a large tournament with \input{assets/tex/number_of_full_strategies/main.tex}
    strategies is considered. In this setting
    the most highly performing strategies do not play in an extortionate way
    against each other but do against lower performing strategies.
    This suggests that whilst the theory of Zero Determinant strategies
    indicates that memory is not of fundamental importance to the evolution of
    cooperative behaviour, this is incomplete.
\end{abstract}

\section{Introduction}\label{sec:introduction}

Agent based game theoretic models have become a stalwart of the underpinning
mathematics of interactive behaviours. One of the major pieces of work
in this area is the pair of original computer tournaments run by Robert
Axelrod~\cite{Axelrod1980, Axelrod1980a}. These tournaments pitted submitted
computer strategies against each other in plays of the Iterated Prisoner's
Dilemma. A common game where agents can choose to pay a slight cost to their
immediate utility in the hope of building a reputation. This has been used in
economic and evolutionary game theory to understand the evolution of cooperative
behaviour.

Recently, a class of strategies was described in~\cite{Press2012} that can
provably extort any given opponent. In~\cite{Hilbe2013, Moran1707} some
questions have already been asked about the true effectiveness of these
strategies in an evolutionary setting. Here another question is asked: is it
possible to recognise this extortionate behaviour? A mathematical procedure for
suspicion is presented: in the same way that the continued actions of an
extortionate individual might raise suspicion.

This work makes use of the Axelrod Python library~\cite{Knight2018, Knight2016}
with a large number of Prisoner Dilemma strategies available to give an
extensive numerical example of the ideas presented.  The approach is presented
in Section~\ref{sec:delta-zd-strategies}.  All of the code and data discussed
in Section~\ref{sec:numerical-experiments} is open sourced, archived and
written according to best scientific principles~\cite{Wilson2014}. The data
archive can be found at~\cite{vincent_knight_2018_1297075}.

\section{Recognising Extortion}\label{sec:delta-zd-strategies}

In~\cite{Press2012}, given a match between 2 memory-one strategies, the concept
of Zero Determinant (ZD) strategies is introduced. The main result of that paper
shows that given two memory one players \(p, q\in\mathbb{R}^4\) a linear
relationship between the players' scores could be forced by one of the players.

Using the notation of~\cite{Press2012}, assuming the utilities for player \(p\)
are given by \(S_x=(R, S, T, P)\) and for player \(q\) by \(S_y=(R, T, S, P)\)
and that the stationary scores of each player is given by \(S_X\) and \(S_Y\)
respectively. The main result of~\cite{Press2012} is that if

\begin{equation}\label{eqn:linear_relationship_for_p}
    \tilde p=\alpha S_x + \beta S_y + \gamma
\end{equation}

or

\begin{equation}\label{eqn:linear_relationship_for_q}
    \tilde q=\alpha S_x + \beta S_y + \gamma
\end{equation}

where \(\tilde p = (1 - p_1, 1 - p_2, p_3, p_4)\) and
\(\tilde q = (1 - q_1, 1 - q_2, q_3, q_4)\) then:

\begin{equation}
    \alpha S_X + \beta S_Y + \gamma = 0
\end{equation}

In~\cite{Press2012} a particular type of ZD strategy is defined: extortionate
strategies. If:

\begin{equation}\label{eqn:constraint_for_extortion}
    \gamma = - P(\alpha + \beta)
\end{equation}

then the player can ensure they get a score \(\chi\) times
larger than the opponent. This extortion coefficient is given by:

\begin{equation}\label{eqn:definition_of_chi}
    \chi=\frac{-\beta}{\alpha}
\end{equation}

Thus, if (\ref{eqn:constraint_for_extortion}) holds and \(\chi >1\) a player is
said to extort their opponent.
Here, the reverse problem is considered: given a
\(p\in\mathbb{R}^4\) how does one identify \(\alpha, \beta\) if they
exist and is the strategy in fact acting in an extortionate way?

These conditions correspond to:

\begin{align}
    \tilde p_1 & = \alpha R + \beta R - P (\alpha + \beta)
            \label{eqn:condition_for_tilde_p1}\\
    \tilde p_2 & = \alpha S + \beta T - P (\alpha + \beta)
            \label{eqn:condition_for_tilde_p2}\\
    \tilde p_3 & = \alpha T + \beta S - P (\alpha + \beta)
            \label{eqn:condition_for_tilde_p3}\\
    \tilde p_4 & = \alpha P + \beta P - P (\alpha + \beta)
            \label{eqn:condition_for_tilde_p4}
\end{align}

Equation (\ref{eqn:condition_for_tilde_p4}) ensures that \(p_4=\tilde p_4=0\).
Equations (\ref{eqn:condition_for_tilde_p1}-\ref{eqn:condition_for_tilde_p3})
can be used to eliminate \(\alpha, \beta\), giving:

\begin{equation}\label{eqn:planar_definition_of_extortion}
    \tilde p_1 = \frac{(R - P)(\tilde p_2 + \tilde p_3)}{S + T - 2P}
\end{equation}

with:

\begin{equation}\label{eqn:definition_of_chi}
    \chi = \frac{\tilde p_2 (P - T) + \tilde p_3 (S - P)}
                {\tilde p_2 (P - S) + \tilde p_3 (T - P)}
\end{equation}

Given a strategy \(p\in\mathbb{R}^{4\times 1}\) equations
(\ref{eqn:condition_for_tilde_p4}), (\ref{eqn:planar_definition_of_extortion}-\ref{eqn:definition_of_chi}) can be used to check if
a strategy is extortionate. The conditions correspond to:

\begin{align}
    p_1 & = \frac{(R-P)(p_2 + p_3) - R + T + S - P}{S + T - 2P}
     \label{eqn:condition_for_p1}\\
    p_4 & = 0 \label{eqn:condition_for_p4}\\
    1 & > p_2 + p_3\label{eqn:condition_for_chi}
\end{align}

The algebraic steps necessary to prove these results are available in the
supporting materials.

All extortionate strategies reside on a triangular (\ref{eqn:condition_for_chi})
plane (\ref{eqn:condition_for_p1}) in 3 dimensions (\ref{eqn:condition_for_p4}).
Using this formulation it can be seen that a necessary (but not sufficient)
condition for an extortionate strategy is that it cooperates on average less
than 50\% of the time when in a state of disagreement with the opponent.

As an example, consider the known extortionate strategy \(p=(8 / 9, 1 / 2, 1 /
3, 0)\) from~\cite{Stewart2012} which is referred to as \texttt{Extort-2}. In
this case, for the standard values of \((R, T, S, P)\) constraint
(\ref{eqn:condition_for_p1}) corresponds to:

\begin{equation}
    p_1 = \frac{2(p_2 + p_3) + 1}{3}
\end{equation}

It is clear that in this case all constraints hold.

This approach could in fact be used to confirm that a given strategy is acting
in an extortionate manner even if it is not a memory one strategy. However, in
practice, if a closed form for \(p\) is not known, then due to measurement
and/or numerical error this would not work.

This problem can be written in the following linear algebraic form where
\(x=(\alpha, \beta)\)
and \(p^*=(\tilde p_1 - 1, tilde_2 - 1, p_3)\):

\begin{equation}\label{eqn:linear_algebraic_equation_for_p}
    Cx= p^*
\end{equation}

\(C\) corresponds to equations
(\ref{eqn:condition_for_tilde_p1}-\ref{eqn:condition_for_tilde_p3}) and is
given by:

\begin{equation}\label{eqn:definition_of_C}
    C =
    \begin{bmatrix}
        R - P & R- P \\
        S - P & T- P \\
        T - P & S- P \\
    \end{bmatrix}
\end{equation}

Note that in general, equation (\ref{eqn:linear_algebraic_equation_for_p}) will
not necessarily have a solution. From the Rouch\'{e}-Capelli theorem if there is
a solution it is unique as \(\text{rank}(C)=2\) which is the dimension of the
variable \(x\). The best fitting \(x\) is found by minimizing:

\begin{equation}\label{eqn:r_squared}
    \text{SSError} = \|C x- p^*\|_2^2 = \sum_{i=1}^{3}\left((C\bar x)_i-p_i^*\right)^2
\end{equation}

Note that \(\text{SSError}\), which is the square of the Frobenius
norm~\cite{Golub2013}, becomes a measure of how close a strategy is to being an
extortionate strategy. Suspicion
of extortion then corresponds to a threshold on \(\text{SSError}\).

By observing interactions (human or otherwise), their memory one representation
can be inferred and this approach can be used to recognise extortionate
behaviour. The notion of comparing theoretic and actual plays of the IPD is not
novel, see for example~\cite{Rand2013}. Immediately it is noted that if the
environment is noisy~\cite{Wu1995} then no strategy can be considered to be
extortionate as \(p_4>0\).

In the next section, this idea will be illustrated by observing the interactions
that take place in a computer based tournament of the IPD\@.

\section{Numerical experiments}\label{sec:numerical-experiments}

In~\cite{Stewart2012} results from a tournament with
\input{./assets/tex/number_of_stewart_plotkin_strategies/main.tex} strategies,
was presented with specific consideration given to ZD strategies. This
tournament is reproduced here using the Axelrod-Python
project~\cite{Knight2016}. To obtain a good measure of the corresponding
transition rates for each strategy all matches have been run for
\input{assets/tex/number_of_turns/main.tex} turns and every match has been
repeated \input{assets/tex/number_of_repetitions/main.tex} times. All of this
interaction data is available at~\cite{vincent_knight_2018_1297075}. A good
match between the inferred Markov chain and the state distribution of the actual
interactions has been verified. Data for this is presented in the supplementary
materials.

Figure~\ref{fig:SSError_overall_in_stewart_plotkin} shows the \(\text{SSError}\)
values for all the strategies in the tournament, as reported
in~\cite{Stewart2012} the extortionate strategy (which has an expected
\(\text{SSError}\) approximately 0) gains a large number of wins.

\begin{figure}[!htbp]
    \centering
    \includegraphics[width=.8\textwidth]{./assets/img/SSError_overall_in_stewart_plotkin/main.pdf}
    \caption{\(\text{SSError}\) and state probabilities for the strategies
        of~\cite{Stewart2012}, ordered both by number of wins and overall score.
        Note that \(P(DC)\) is not shown as it corresponds to the transpose of
        \(P(CD)\). Cooperator and Defector are omitted as they do not visit all
        the states.}
    \label{fig:SSError_overall_in_stewart_plotkin}
\end{figure}

Here, the work of~\cite{Stewart2012} is extended by investigating a tournament
with \input{assets/tex/number_of_full_strategies/main.tex}
strategies.

The results of this analysis are shown in
Figure~\ref{fig:SSError_and_probabilities_in_full}. The top ranking strategies
by number of wins seem to be extortionate (but not against all strategies) and
it can be seen that a small sub group of strategies achieve mutual defection.
All the top ranking strategies according to score achieve mutual cooperation and
do not extort each other, however they
\textbf{do} exhibit extortionate behaviour towards a number of the lower ranking
strategies.

\begin{figure}[!htbp]
    \centering
    \includegraphics[width=.8\textwidth]{./assets/img/SSError_and_probabilities_in_full/main.pdf}
    \caption{\(\text{SSError}\) for the strategies for the full tournament. Only
    strategy interactions for which \(p_4=0\) and \(\chi>1\) are displayed.}
    \label{fig:SSError_and_probabilities_in_full}
\end{figure}

\section{Conclusion}\label{sec:conclusion}

This work defines an approach to measure whether or not a player is playing a
strategy that corresponds to an extortionate strategy as defined
in~\cite{Press2012}: a mathematical model for suspicion. Indeed, all
extortionate strategies have been
 classified as lying on a triangular plane.
This rigorous classification fails to be robust to small measurement error, thus
a statistical approach is proposed.
This is done through a linear algebraic approach for approximating the solution
of a linear system. Using this, a large number of pairwise interactions is
simulated and in fact very few strategies are found to act extortionately.

The work of~\cite{Press2012}, whilst showing that a clever approach to taking
advantage of another memory one strategy exists: this is incomplete. Whilst the
elegance of this result is very attractive, just as the simplicity of the
victory of Tit For Tat in Axelrod's original tournaments was, it is incomplete.
Extortionate strategies achieve a high number of wins but they do not
achieve a high score which corresponds to the fitness landscape in an
evolutionary sense. From the large number of interactions a payoff matrix \(S\)
can be measured where \(S_{ij}\) denotes the score (using standard values of
\((R, S, T, P) = (3, 0, 5, 1)\)) of the \(i\)th strategy
against the \(j\)th strategy. Using this, the replicator equation
describes the evolution of the system based on a population density fitness
function:

\begin{equation}\label{eqn:replicator_dynamics}
    \frac{dx}{dt} = x(S-x^TS x)
\end{equation}

Equation (\ref{eqn:replicator_dynamics}) is solved numerically through an
integration technique described in~\cite{Petzold1983} and
Figure~\ref{fig:replicator_dynamics} shows the evolution of the distribution of
the system: the various strategies are ranked by scores. It is clear to see that
only the high ranking strategies survive the evolutionary process (in fact,
only \input{./assets/img/replicator_dynamics/main.tex}
have a final distribution greater than \(10 ^ {-2}\)). This confirms the
findings of~\cite{Moran1707} in which sophisticated strategies resist
evolutionary invasion of shorter memory strategies. Recalling
Figure~\ref{fig:SSError_and_probabilities_in_full} this demonstrates that:

\begin{itemize}
    \item Cooperation emerges through the evolutionary process: the high scoring
        strategies do not exhibit extortionate behaviour towards each other.
    \item Extortionate strategies do not survive the evolutionary process.
\end{itemize}

\begin{figure}[!htbp]
    \centering
    \includegraphics[width=.8\textwidth]{./assets/img/replicator_dynamics/main.pdf}
    \caption{Numerical simulation of the replicator equation
    (\ref{eqn:replicator_dynamics}): strategies are ordered by score, only the strategies with a high score survive the evolutionary process.}
    \label{fig:replicator_dynamics}
\end{figure}

This work can be used to classify plays of the IPD\@: data can be collected from
actual interactions (in lab or in the field). Furthermore, this allows for a
classification method similar to the notion of fingerprinting presented
in~\cite{Ashlock2008}. Trained strategies can potentially be classified as
extortionate or not or it could be possible to even constrain the reinforcement
learning approaches that are becoming prevalent in the literature.
Alternatively, this mathematical approach for recognising extortion could be
used in sophisticated strategies to defend against invasion. Arguably, some of
the strategies considered here exhibit this behaviour, indeed as described
in~\cite{Harper2017}, the top ranking strategies in the full tournament are
obtained using evolutionary reinforcement learning techniques, thus, suspicion
of extortionate behaviour could in fact be an evolutionary trait.

\section*{Acknowledgements}

The following open source software libraries were used in this research:

\begin{itemize}
    \item The Axelrod ~\cite{Knight2016, Knight2018} library (IPD strategies and
        tournaments).
    \item The sympy library~\cite{Meurer2017} (verification of all symbolic
        calculations).
    \item The matplotlib~\cite{Droettboom2018} library (visualisation).
    \item The pandas~\cite{Structures2010}, dask~\cite{Dask2016} and
        NumPy~\cite{Oliphant2015} libraries (data manipulation).
    \item The SciPy~\cite{Jones2001} library (numerical integration of the
        replicator equation).
\end{itemize}

This work was performed using the computational facilities of the Advanced
Research Computing @ Cardiff (ARCCA) Division, Cardiff University.

\printbibliography

\newpage
\section*{Supplementary materials}

\includepdf{assets/pdf/proof_of_form_of_extortionate_strategies/main.pdf}

\newpage

Using the pair wise interactions the transition rates \(p,
q\) can be measured and the steady state probabilities inferred and compared to
the actual probabilities of each state.
This is done numerically by computing the singular eigenvector of the
matrix \(A\) \cite{Stewart2009}:

\[
    A =
    \begin{bmatrix}
        p_1 q_1 & p_1 (1 - q_1) & (1 - p_1) q_1 & (1 -p_1) (1 - q_1) \\
        p_2 q_2 & p_2 (1 - q_2) & (1 - p_2) q_2 & (1 -p_2) (1 - q_2) \\
        p_3 q_3 & p_3 (1 - q_3) & (1 - p_3) q_3 & (1 -p_3) (1 - q_3) \\
        p_4 q_4 & p_4 (1 - q_4) & (1 - p_4) q_4 & (1 -p_4) (1 - q_4) \\
    \end{bmatrix}
\]

Figure~\ref{fig:computed_probabilities_vs_theoretic_probabilities} shows a
regression line fitted to every pairwise interaction with a reported
\(\text{SSError}\) value (pairwise interactions with missing states were
omitted). This serves to validate the approach: a part from some edge cases the
relationship is consistent.

\begin{figure}[!htbp]
    \centering
    \includegraphics[width=.8\textwidth]{./assets/img/computed_probabilities_vs_theoretic_probabilities/main.pdf}
    \caption{The
        relationship between the steady state probabilities inferred from the
        measured transitions and the actual steady state probabilities. A linear
        regression line is included validating the approach.}
    \label{fig:computed_probabilities_vs_theoretic_probabilities}
\end{figure}


\end{document}

    strategies is considered. In this setting
    the most highly performing strategies do not play in an extortionate way
    against each other but do against lower performing strategies.
    This suggests that whilst the theory of Zero Determinant strategies
    indicates that memory is not of fundamental importance to the evolution of
    cooperative behaviour, this is incomplete.
\end{abstract}

\section{Introduction}\label{sec:introduction}

Agent based game theoretic models have become a stalwart of the underpinning
mathematics of interactive behaviours. One of the major pieces of work
in this area is the pair of original computer tournaments run by Robert
Axelrod~\cite{Axelrod1980, Axelrod1980a}. These tournaments pitted submitted
computer strategies against each other in plays of the Iterated Prisoner's
Dilemma. A common game where agents can choose to pay a slight cost to their
immediate utility in the hope of building a reputation. This has been used in
economic and evolutionary game theory to understand the evolution of cooperative
behaviour.

Recently, a class of strategies was described in~\cite{Press2012} that can
provably extort any given opponent. In~\cite{Hilbe2013, Moran1707} some
questions have already been asked about the true effectiveness of these
strategies in an evolutionary setting. Here another question is asked: is it
possible to recognise this extortionate behaviour? A mathematical procedure for
suspicion is presented: in the same way that the continued actions of an
extortionate individual might raise suspicion.

This work makes use of the Axelrod Python library~\cite{Knight2018, Knight2016}
with a large number of Prisoner Dilemma strategies available to give an
extensive numerical example of the ideas presented.  The approach is presented
in Section~\ref{sec:delta-zd-strategies}.  All of the code and data discussed
in Section~\ref{sec:numerical-experiments} is open sourced, archived and
written according to best scientific principles~\cite{Wilson2014}. The data
archive can be found at~\cite{vincent_knight_2018_1297075}.

\section{Recognising Extortion}\label{sec:delta-zd-strategies}

In~\cite{Press2012}, given a match between 2 memory-one strategies, the concept
of Zero Determinant (ZD) strategies is introduced. The main result of that paper
shows that given two memory one players \(p, q\in\mathbb{R}^4\) a linear
relationship between the players' scores could be forced by one of the players.

Using the notation of~\cite{Press2012}, assuming the utilities for player \(p\)
are given by \(S_x=(R, S, T, P)\) and for player \(q\) by \(S_y=(R, T, S, P)\)
and that the stationary scores of each player is given by \(S_X\) and \(S_Y\)
respectively. The main result of~\cite{Press2012} is that if

\begin{equation}\label{eqn:linear_relationship_for_p}
    \tilde p=\alpha S_x + \beta S_y + \gamma
\end{equation}

or

\begin{equation}\label{eqn:linear_relationship_for_q}
    \tilde q=\alpha S_x + \beta S_y + \gamma
\end{equation}

where \(\tilde p = (1 - p_1, 1 - p_2, p_3, p_4)\) and
\(\tilde q = (1 - q_1, 1 - q_2, q_3, q_4)\) then:

\begin{equation}
    \alpha S_X + \beta S_Y + \gamma = 0
\end{equation}

In~\cite{Press2012} a particular type of ZD strategy is defined: extortionate
strategies. If:

\begin{equation}\label{eqn:constraint_for_extortion}
    \gamma = - P(\alpha + \beta)
\end{equation}

then the player can ensure they get a score \(\chi\) times
larger than the opponent. This extortion coefficient is given by:

\begin{equation}\label{eqn:definition_of_chi}
    \chi=\frac{-\beta}{\alpha}
\end{equation}

Thus, if (\ref{eqn:constraint_for_extortion}) holds and \(\chi >1\) a player is
said to extort their opponent.
Here, the reverse problem is considered: given a
\(p\in\mathbb{R}^4\) how does one identify \(\alpha, \beta\) if they
exist and is the strategy in fact acting in an extortionate way?

These conditions correspond to:

\begin{align}
    \tilde p_1 & = \alpha R + \beta R - P (\alpha + \beta)
            \label{eqn:condition_for_tilde_p1}\\
    \tilde p_2 & = \alpha S + \beta T - P (\alpha + \beta)
            \label{eqn:condition_for_tilde_p2}\\
    \tilde p_3 & = \alpha T + \beta S - P (\alpha + \beta)
            \label{eqn:condition_for_tilde_p3}\\
    \tilde p_4 & = \alpha P + \beta P - P (\alpha + \beta)
            \label{eqn:condition_for_tilde_p4}
\end{align}

Equation (\ref{eqn:condition_for_tilde_p4}) ensures that \(p_4=\tilde p_4=0\).
Equations (\ref{eqn:condition_for_tilde_p1}-\ref{eqn:condition_for_tilde_p3})
can be used to eliminate \(\alpha, \beta\), giving:

\begin{equation}\label{eqn:planar_definition_of_extortion}
    \tilde p_1 = \frac{(R - P)(\tilde p_2 + \tilde p_3)}{S + T - 2P}
\end{equation}

with:

\begin{equation}\label{eqn:definition_of_chi}
    \chi = \frac{\tilde p_2 (P - T) + \tilde p_3 (S - P)}
                {\tilde p_2 (P - S) + \tilde p_3 (T - P)}
\end{equation}

Given a strategy \(p\in\mathbb{R}^{4\times 1}\) equations
(\ref{eqn:condition_for_tilde_p4}), (\ref{eqn:planar_definition_of_extortion}-\ref{eqn:definition_of_chi}) can be used to check if
a strategy is extortionate. The conditions correspond to:

\begin{align}
    p_1 & = \frac{(R-P)(p_2 + p_3) - R + T + S - P}{S + T - 2P}
     \label{eqn:condition_for_p1}\\
    p_4 & = 0 \label{eqn:condition_for_p4}\\
    1 & > p_2 + p_3\label{eqn:condition_for_chi}
\end{align}

The algebraic steps necessary to prove these results are available in the
supporting materials.

All extortionate strategies reside on a triangular (\ref{eqn:condition_for_chi})
plane (\ref{eqn:condition_for_p1}) in 3 dimensions (\ref{eqn:condition_for_p4}).
Using this formulation it can be seen that a necessary (but not sufficient)
condition for an extortionate strategy is that it cooperates on average less
than 50\% of the time when in a state of disagreement with the opponent.

As an example, consider the known extortionate strategy \(p=(8 / 9, 1 / 2, 1 /
3, 0)\) from~\cite{Stewart2012} which is referred to as \texttt{Extort-2}. In
this case, for the standard values of \((R, T, S, P)\) constraint
(\ref{eqn:condition_for_p1}) corresponds to:

\begin{equation}
    p_1 = \frac{2(p_2 + p_3) + 1}{3}
\end{equation}

It is clear that in this case all constraints hold.

This approach could in fact be used to confirm that a given strategy is acting
in an extortionate manner even if it is not a memory one strategy. However, in
practice, if a closed form for \(p\) is not known, then due to measurement
and/or numerical error this would not work.

This problem can be written in the following linear algebraic form where
\(x=(\alpha, \beta)\)
and \(p^*=(\tilde p_1 - 1, tilde_2 - 1, p_3)\):

\begin{equation}\label{eqn:linear_algebraic_equation_for_p}
    Cx= p^*
\end{equation}

\(C\) corresponds to equations
(\ref{eqn:condition_for_tilde_p1}-\ref{eqn:condition_for_tilde_p3}) and is
given by:

\begin{equation}\label{eqn:definition_of_C}
    C =
    \begin{bmatrix}
        R - P & R- P \\
        S - P & T- P \\
        T - P & S- P \\
    \end{bmatrix}
\end{equation}

Note that in general, equation (\ref{eqn:linear_algebraic_equation_for_p}) will
not necessarily have a solution. From the Rouch\'{e}-Capelli theorem if there is
a solution it is unique as \(\text{rank}(C)=2\) which is the dimension of the
variable \(x\). The best fitting \(x\) is found by minimizing:

\begin{equation}\label{eqn:r_squared}
    \text{SSError} = \|C x- p^*\|_2^2 = \sum_{i=1}^{3}\left((C\bar x)_i-p_i^*\right)^2
\end{equation}

Note that \(\text{SSError}\), which is the square of the Frobenius
norm~\cite{Golub2013}, becomes a measure of how close a strategy is to being an
extortionate strategy. Suspicion
of extortion then corresponds to a threshold on \(\text{SSError}\).

By observing interactions (human or otherwise), their memory one representation
can be inferred and this approach can be used to recognise extortionate
behaviour. The notion of comparing theoretic and actual plays of the IPD is not
novel, see for example~\cite{Rand2013}. Immediately it is noted that if the
environment is noisy~\cite{Wu1995} then no strategy can be considered to be
extortionate as \(p_4>0\).

In the next section, this idea will be illustrated by observing the interactions
that take place in a computer based tournament of the IPD\@.

\section{Numerical experiments}\label{sec:numerical-experiments}

In~\cite{Stewart2012} results from a tournament with
\documentclass[a4paper]{article}

\usepackage{amsmath}
\usepackage{amssymb}
\usepackage[margin=1.5cm,
            includefoot,
            footskip=30pt]{geometry}
\usepackage{layout}
\usepackage{graphicx}
\usepackage{subcaption}

\usepackage{biblatex}
\usepackage{pdfpages}

\bibliography{main.bib}

\title{Suspicion: Recognising and evaluating the effectiveness
       of extortion in the Iterated Prisoner's Dilemma}
\author{Vincent A. Knight \and Nikoleta E. Glynatsi}
\date{\today}



\begin{document}

\maketitle

\begin{abstract}
    The Iterated Prisoner's Dilemma is a model for rational and evolutionary
    interactive behaviour. It has applications both in the study of human social
    behaviour as well as in biology.
    It is used to understand when and how a rational individual might
    accept an immediate cost to their own utility for the direct benefit of
    another.

    Much attention has been given to a class of strategies called
    Zero Determinant strategies. It has been theoretically shown that these
    strategies can ``extort'' any player.

    In this work, an approach to identify if observed strategies are playing in
    an extortionate way is described. Furthermore, experimental analysis of
    a large tournament with \input{assets/tex/number_of_full_strategies/main.tex}
    strategies is considered. In this setting
    the most highly performing strategies do not play in an extortionate way
    against each other but do against lower performing strategies.
    This suggests that whilst the theory of Zero Determinant strategies
    indicates that memory is not of fundamental importance to the evolution of
    cooperative behaviour, this is incomplete.
\end{abstract}

\section{Introduction}\label{sec:introduction}

Agent based game theoretic models have become a stalwart of the underpinning
mathematics of interactive behaviours. One of the major pieces of work
in this area is the pair of original computer tournaments run by Robert
Axelrod~\cite{Axelrod1980, Axelrod1980a}. These tournaments pitted submitted
computer strategies against each other in plays of the Iterated Prisoner's
Dilemma. A common game where agents can choose to pay a slight cost to their
immediate utility in the hope of building a reputation. This has been used in
economic and evolutionary game theory to understand the evolution of cooperative
behaviour.

Recently, a class of strategies was described in~\cite{Press2012} that can
provably extort any given opponent. In~\cite{Hilbe2013, Moran1707} some
questions have already been asked about the true effectiveness of these
strategies in an evolutionary setting. Here another question is asked: is it
possible to recognise this extortionate behaviour? A mathematical procedure for
suspicion is presented: in the same way that the continued actions of an
extortionate individual might raise suspicion.

This work makes use of the Axelrod Python library~\cite{Knight2018, Knight2016}
with a large number of Prisoner Dilemma strategies available to give an
extensive numerical example of the ideas presented.  The approach is presented
in Section~\ref{sec:delta-zd-strategies}.  All of the code and data discussed
in Section~\ref{sec:numerical-experiments} is open sourced, archived and
written according to best scientific principles~\cite{Wilson2014}. The data
archive can be found at~\cite{vincent_knight_2018_1297075}.

\section{Recognising Extortion}\label{sec:delta-zd-strategies}

In~\cite{Press2012}, given a match between 2 memory-one strategies, the concept
of Zero Determinant (ZD) strategies is introduced. The main result of that paper
shows that given two memory one players \(p, q\in\mathbb{R}^4\) a linear
relationship between the players' scores could be forced by one of the players.

Using the notation of~\cite{Press2012}, assuming the utilities for player \(p\)
are given by \(S_x=(R, S, T, P)\) and for player \(q\) by \(S_y=(R, T, S, P)\)
and that the stationary scores of each player is given by \(S_X\) and \(S_Y\)
respectively. The main result of~\cite{Press2012} is that if

\begin{equation}\label{eqn:linear_relationship_for_p}
    \tilde p=\alpha S_x + \beta S_y + \gamma
\end{equation}

or

\begin{equation}\label{eqn:linear_relationship_for_q}
    \tilde q=\alpha S_x + \beta S_y + \gamma
\end{equation}

where \(\tilde p = (1 - p_1, 1 - p_2, p_3, p_4)\) and
\(\tilde q = (1 - q_1, 1 - q_2, q_3, q_4)\) then:

\begin{equation}
    \alpha S_X + \beta S_Y + \gamma = 0
\end{equation}

In~\cite{Press2012} a particular type of ZD strategy is defined: extortionate
strategies. If:

\begin{equation}\label{eqn:constraint_for_extortion}
    \gamma = - P(\alpha + \beta)
\end{equation}

then the player can ensure they get a score \(\chi\) times
larger than the opponent. This extortion coefficient is given by:

\begin{equation}\label{eqn:definition_of_chi}
    \chi=\frac{-\beta}{\alpha}
\end{equation}

Thus, if (\ref{eqn:constraint_for_extortion}) holds and \(\chi >1\) a player is
said to extort their opponent.
Here, the reverse problem is considered: given a
\(p\in\mathbb{R}^4\) how does one identify \(\alpha, \beta\) if they
exist and is the strategy in fact acting in an extortionate way?

These conditions correspond to:

\begin{align}
    \tilde p_1 & = \alpha R + \beta R - P (\alpha + \beta)
            \label{eqn:condition_for_tilde_p1}\\
    \tilde p_2 & = \alpha S + \beta T - P (\alpha + \beta)
            \label{eqn:condition_for_tilde_p2}\\
    \tilde p_3 & = \alpha T + \beta S - P (\alpha + \beta)
            \label{eqn:condition_for_tilde_p3}\\
    \tilde p_4 & = \alpha P + \beta P - P (\alpha + \beta)
            \label{eqn:condition_for_tilde_p4}
\end{align}

Equation (\ref{eqn:condition_for_tilde_p4}) ensures that \(p_4=\tilde p_4=0\).
Equations (\ref{eqn:condition_for_tilde_p1}-\ref{eqn:condition_for_tilde_p3})
can be used to eliminate \(\alpha, \beta\), giving:

\begin{equation}\label{eqn:planar_definition_of_extortion}
    \tilde p_1 = \frac{(R - P)(\tilde p_2 + \tilde p_3)}{S + T - 2P}
\end{equation}

with:

\begin{equation}\label{eqn:definition_of_chi}
    \chi = \frac{\tilde p_2 (P - T) + \tilde p_3 (S - P)}
                {\tilde p_2 (P - S) + \tilde p_3 (T - P)}
\end{equation}

Given a strategy \(p\in\mathbb{R}^{4\times 1}\) equations
(\ref{eqn:condition_for_tilde_p4}), (\ref{eqn:planar_definition_of_extortion}-\ref{eqn:definition_of_chi}) can be used to check if
a strategy is extortionate. The conditions correspond to:

\begin{align}
    p_1 & = \frac{(R-P)(p_2 + p_3) - R + T + S - P}{S + T - 2P}
     \label{eqn:condition_for_p1}\\
    p_4 & = 0 \label{eqn:condition_for_p4}\\
    1 & > p_2 + p_3\label{eqn:condition_for_chi}
\end{align}

The algebraic steps necessary to prove these results are available in the
supporting materials.

All extortionate strategies reside on a triangular (\ref{eqn:condition_for_chi})
plane (\ref{eqn:condition_for_p1}) in 3 dimensions (\ref{eqn:condition_for_p4}).
Using this formulation it can be seen that a necessary (but not sufficient)
condition for an extortionate strategy is that it cooperates on average less
than 50\% of the time when in a state of disagreement with the opponent.

As an example, consider the known extortionate strategy \(p=(8 / 9, 1 / 2, 1 /
3, 0)\) from~\cite{Stewart2012} which is referred to as \texttt{Extort-2}. In
this case, for the standard values of \((R, T, S, P)\) constraint
(\ref{eqn:condition_for_p1}) corresponds to:

\begin{equation}
    p_1 = \frac{2(p_2 + p_3) + 1}{3}
\end{equation}

It is clear that in this case all constraints hold.

This approach could in fact be used to confirm that a given strategy is acting
in an extortionate manner even if it is not a memory one strategy. However, in
practice, if a closed form for \(p\) is not known, then due to measurement
and/or numerical error this would not work.

This problem can be written in the following linear algebraic form where
\(x=(\alpha, \beta)\)
and \(p^*=(\tilde p_1 - 1, tilde_2 - 1, p_3)\):

\begin{equation}\label{eqn:linear_algebraic_equation_for_p}
    Cx= p^*
\end{equation}

\(C\) corresponds to equations
(\ref{eqn:condition_for_tilde_p1}-\ref{eqn:condition_for_tilde_p3}) and is
given by:

\begin{equation}\label{eqn:definition_of_C}
    C =
    \begin{bmatrix}
        R - P & R- P \\
        S - P & T- P \\
        T - P & S- P \\
    \end{bmatrix}
\end{equation}

Note that in general, equation (\ref{eqn:linear_algebraic_equation_for_p}) will
not necessarily have a solution. From the Rouch\'{e}-Capelli theorem if there is
a solution it is unique as \(\text{rank}(C)=2\) which is the dimension of the
variable \(x\). The best fitting \(x\) is found by minimizing:

\begin{equation}\label{eqn:r_squared}
    \text{SSError} = \|C x- p^*\|_2^2 = \sum_{i=1}^{3}\left((C\bar x)_i-p_i^*\right)^2
\end{equation}

Note that \(\text{SSError}\), which is the square of the Frobenius
norm~\cite{Golub2013}, becomes a measure of how close a strategy is to being an
extortionate strategy. Suspicion
of extortion then corresponds to a threshold on \(\text{SSError}\).

By observing interactions (human or otherwise), their memory one representation
can be inferred and this approach can be used to recognise extortionate
behaviour. The notion of comparing theoretic and actual plays of the IPD is not
novel, see for example~\cite{Rand2013}. Immediately it is noted that if the
environment is noisy~\cite{Wu1995} then no strategy can be considered to be
extortionate as \(p_4>0\).

In the next section, this idea will be illustrated by observing the interactions
that take place in a computer based tournament of the IPD\@.

\section{Numerical experiments}\label{sec:numerical-experiments}

In~\cite{Stewart2012} results from a tournament with
\input{./assets/tex/number_of_stewart_plotkin_strategies/main.tex} strategies,
was presented with specific consideration given to ZD strategies. This
tournament is reproduced here using the Axelrod-Python
project~\cite{Knight2016}. To obtain a good measure of the corresponding
transition rates for each strategy all matches have been run for
\input{assets/tex/number_of_turns/main.tex} turns and every match has been
repeated \input{assets/tex/number_of_repetitions/main.tex} times. All of this
interaction data is available at~\cite{vincent_knight_2018_1297075}. A good
match between the inferred Markov chain and the state distribution of the actual
interactions has been verified. Data for this is presented in the supplementary
materials.

Figure~\ref{fig:SSError_overall_in_stewart_plotkin} shows the \(\text{SSError}\)
values for all the strategies in the tournament, as reported
in~\cite{Stewart2012} the extortionate strategy (which has an expected
\(\text{SSError}\) approximately 0) gains a large number of wins.

\begin{figure}[!htbp]
    \centering
    \includegraphics[width=.8\textwidth]{./assets/img/SSError_overall_in_stewart_plotkin/main.pdf}
    \caption{\(\text{SSError}\) and state probabilities for the strategies
        of~\cite{Stewart2012}, ordered both by number of wins and overall score.
        Note that \(P(DC)\) is not shown as it corresponds to the transpose of
        \(P(CD)\). Cooperator and Defector are omitted as they do not visit all
        the states.}
    \label{fig:SSError_overall_in_stewart_plotkin}
\end{figure}

Here, the work of~\cite{Stewart2012} is extended by investigating a tournament
with \input{assets/tex/number_of_full_strategies/main.tex}
strategies.

The results of this analysis are shown in
Figure~\ref{fig:SSError_and_probabilities_in_full}. The top ranking strategies
by number of wins seem to be extortionate (but not against all strategies) and
it can be seen that a small sub group of strategies achieve mutual defection.
All the top ranking strategies according to score achieve mutual cooperation and
do not extort each other, however they
\textbf{do} exhibit extortionate behaviour towards a number of the lower ranking
strategies.

\begin{figure}[!htbp]
    \centering
    \includegraphics[width=.8\textwidth]{./assets/img/SSError_and_probabilities_in_full/main.pdf}
    \caption{\(\text{SSError}\) for the strategies for the full tournament. Only
    strategy interactions for which \(p_4=0\) and \(\chi>1\) are displayed.}
    \label{fig:SSError_and_probabilities_in_full}
\end{figure}

\section{Conclusion}\label{sec:conclusion}

This work defines an approach to measure whether or not a player is playing a
strategy that corresponds to an extortionate strategy as defined
in~\cite{Press2012}: a mathematical model for suspicion. Indeed, all
extortionate strategies have been
 classified as lying on a triangular plane.
This rigorous classification fails to be robust to small measurement error, thus
a statistical approach is proposed.
This is done through a linear algebraic approach for approximating the solution
of a linear system. Using this, a large number of pairwise interactions is
simulated and in fact very few strategies are found to act extortionately.

The work of~\cite{Press2012}, whilst showing that a clever approach to taking
advantage of another memory one strategy exists: this is incomplete. Whilst the
elegance of this result is very attractive, just as the simplicity of the
victory of Tit For Tat in Axelrod's original tournaments was, it is incomplete.
Extortionate strategies achieve a high number of wins but they do not
achieve a high score which corresponds to the fitness landscape in an
evolutionary sense. From the large number of interactions a payoff matrix \(S\)
can be measured where \(S_{ij}\) denotes the score (using standard values of
\((R, S, T, P) = (3, 0, 5, 1)\)) of the \(i\)th strategy
against the \(j\)th strategy. Using this, the replicator equation
describes the evolution of the system based on a population density fitness
function:

\begin{equation}\label{eqn:replicator_dynamics}
    \frac{dx}{dt} = x(S-x^TS x)
\end{equation}

Equation (\ref{eqn:replicator_dynamics}) is solved numerically through an
integration technique described in~\cite{Petzold1983} and
Figure~\ref{fig:replicator_dynamics} shows the evolution of the distribution of
the system: the various strategies are ranked by scores. It is clear to see that
only the high ranking strategies survive the evolutionary process (in fact,
only \input{./assets/img/replicator_dynamics/main.tex}
have a final distribution greater than \(10 ^ {-2}\)). This confirms the
findings of~\cite{Moran1707} in which sophisticated strategies resist
evolutionary invasion of shorter memory strategies. Recalling
Figure~\ref{fig:SSError_and_probabilities_in_full} this demonstrates that:

\begin{itemize}
    \item Cooperation emerges through the evolutionary process: the high scoring
        strategies do not exhibit extortionate behaviour towards each other.
    \item Extortionate strategies do not survive the evolutionary process.
\end{itemize}

\begin{figure}[!htbp]
    \centering
    \includegraphics[width=.8\textwidth]{./assets/img/replicator_dynamics/main.pdf}
    \caption{Numerical simulation of the replicator equation
    (\ref{eqn:replicator_dynamics}): strategies are ordered by score, only the strategies with a high score survive the evolutionary process.}
    \label{fig:replicator_dynamics}
\end{figure}

This work can be used to classify plays of the IPD\@: data can be collected from
actual interactions (in lab or in the field). Furthermore, this allows for a
classification method similar to the notion of fingerprinting presented
in~\cite{Ashlock2008}. Trained strategies can potentially be classified as
extortionate or not or it could be possible to even constrain the reinforcement
learning approaches that are becoming prevalent in the literature.
Alternatively, this mathematical approach for recognising extortion could be
used in sophisticated strategies to defend against invasion. Arguably, some of
the strategies considered here exhibit this behaviour, indeed as described
in~\cite{Harper2017}, the top ranking strategies in the full tournament are
obtained using evolutionary reinforcement learning techniques, thus, suspicion
of extortionate behaviour could in fact be an evolutionary trait.

\section*{Acknowledgements}

The following open source software libraries were used in this research:

\begin{itemize}
    \item The Axelrod ~\cite{Knight2016, Knight2018} library (IPD strategies and
        tournaments).
    \item The sympy library~\cite{Meurer2017} (verification of all symbolic
        calculations).
    \item The matplotlib~\cite{Droettboom2018} library (visualisation).
    \item The pandas~\cite{Structures2010}, dask~\cite{Dask2016} and
        NumPy~\cite{Oliphant2015} libraries (data manipulation).
    \item The SciPy~\cite{Jones2001} library (numerical integration of the
        replicator equation).
\end{itemize}

This work was performed using the computational facilities of the Advanced
Research Computing @ Cardiff (ARCCA) Division, Cardiff University.

\printbibliography

\newpage
\section*{Supplementary materials}

\includepdf{assets/pdf/proof_of_form_of_extortionate_strategies/main.pdf}

\newpage

Using the pair wise interactions the transition rates \(p,
q\) can be measured and the steady state probabilities inferred and compared to
the actual probabilities of each state.
This is done numerically by computing the singular eigenvector of the
matrix \(A\) \cite{Stewart2009}:

\[
    A =
    \begin{bmatrix}
        p_1 q_1 & p_1 (1 - q_1) & (1 - p_1) q_1 & (1 -p_1) (1 - q_1) \\
        p_2 q_2 & p_2 (1 - q_2) & (1 - p_2) q_2 & (1 -p_2) (1 - q_2) \\
        p_3 q_3 & p_3 (1 - q_3) & (1 - p_3) q_3 & (1 -p_3) (1 - q_3) \\
        p_4 q_4 & p_4 (1 - q_4) & (1 - p_4) q_4 & (1 -p_4) (1 - q_4) \\
    \end{bmatrix}
\]

Figure~\ref{fig:computed_probabilities_vs_theoretic_probabilities} shows a
regression line fitted to every pairwise interaction with a reported
\(\text{SSError}\) value (pairwise interactions with missing states were
omitted). This serves to validate the approach: a part from some edge cases the
relationship is consistent.

\begin{figure}[!htbp]
    \centering
    \includegraphics[width=.8\textwidth]{./assets/img/computed_probabilities_vs_theoretic_probabilities/main.pdf}
    \caption{The
        relationship between the steady state probabilities inferred from the
        measured transitions and the actual steady state probabilities. A linear
        regression line is included validating the approach.}
    \label{fig:computed_probabilities_vs_theoretic_probabilities}
\end{figure}


\end{document}
 strategies,
was presented with specific consideration given to ZD strategies. This
tournament is reproduced here using the Axelrod-Python
project~\cite{Knight2016}. To obtain a good measure of the corresponding
transition rates for each strategy all matches have been run for
\documentclass[a4paper]{article}

\usepackage{amsmath}
\usepackage{amssymb}
\usepackage[margin=1.5cm,
            includefoot,
            footskip=30pt]{geometry}
\usepackage{layout}
\usepackage{graphicx}
\usepackage{subcaption}

\usepackage{biblatex}
\usepackage{pdfpages}

\bibliography{main.bib}

\title{Suspicion: Recognising and evaluating the effectiveness
       of extortion in the Iterated Prisoner's Dilemma}
\author{Vincent A. Knight \and Nikoleta E. Glynatsi}
\date{\today}



\begin{document}

\maketitle

\begin{abstract}
    The Iterated Prisoner's Dilemma is a model for rational and evolutionary
    interactive behaviour. It has applications both in the study of human social
    behaviour as well as in biology.
    It is used to understand when and how a rational individual might
    accept an immediate cost to their own utility for the direct benefit of
    another.

    Much attention has been given to a class of strategies called
    Zero Determinant strategies. It has been theoretically shown that these
    strategies can ``extort'' any player.

    In this work, an approach to identify if observed strategies are playing in
    an extortionate way is described. Furthermore, experimental analysis of
    a large tournament with \input{assets/tex/number_of_full_strategies/main.tex}
    strategies is considered. In this setting
    the most highly performing strategies do not play in an extortionate way
    against each other but do against lower performing strategies.
    This suggests that whilst the theory of Zero Determinant strategies
    indicates that memory is not of fundamental importance to the evolution of
    cooperative behaviour, this is incomplete.
\end{abstract}

\section{Introduction}\label{sec:introduction}

Agent based game theoretic models have become a stalwart of the underpinning
mathematics of interactive behaviours. One of the major pieces of work
in this area is the pair of original computer tournaments run by Robert
Axelrod~\cite{Axelrod1980, Axelrod1980a}. These tournaments pitted submitted
computer strategies against each other in plays of the Iterated Prisoner's
Dilemma. A common game where agents can choose to pay a slight cost to their
immediate utility in the hope of building a reputation. This has been used in
economic and evolutionary game theory to understand the evolution of cooperative
behaviour.

Recently, a class of strategies was described in~\cite{Press2012} that can
provably extort any given opponent. In~\cite{Hilbe2013, Moran1707} some
questions have already been asked about the true effectiveness of these
strategies in an evolutionary setting. Here another question is asked: is it
possible to recognise this extortionate behaviour? A mathematical procedure for
suspicion is presented: in the same way that the continued actions of an
extortionate individual might raise suspicion.

This work makes use of the Axelrod Python library~\cite{Knight2018, Knight2016}
with a large number of Prisoner Dilemma strategies available to give an
extensive numerical example of the ideas presented.  The approach is presented
in Section~\ref{sec:delta-zd-strategies}.  All of the code and data discussed
in Section~\ref{sec:numerical-experiments} is open sourced, archived and
written according to best scientific principles~\cite{Wilson2014}. The data
archive can be found at~\cite{vincent_knight_2018_1297075}.

\section{Recognising Extortion}\label{sec:delta-zd-strategies}

In~\cite{Press2012}, given a match between 2 memory-one strategies, the concept
of Zero Determinant (ZD) strategies is introduced. The main result of that paper
shows that given two memory one players \(p, q\in\mathbb{R}^4\) a linear
relationship between the players' scores could be forced by one of the players.

Using the notation of~\cite{Press2012}, assuming the utilities for player \(p\)
are given by \(S_x=(R, S, T, P)\) and for player \(q\) by \(S_y=(R, T, S, P)\)
and that the stationary scores of each player is given by \(S_X\) and \(S_Y\)
respectively. The main result of~\cite{Press2012} is that if

\begin{equation}\label{eqn:linear_relationship_for_p}
    \tilde p=\alpha S_x + \beta S_y + \gamma
\end{equation}

or

\begin{equation}\label{eqn:linear_relationship_for_q}
    \tilde q=\alpha S_x + \beta S_y + \gamma
\end{equation}

where \(\tilde p = (1 - p_1, 1 - p_2, p_3, p_4)\) and
\(\tilde q = (1 - q_1, 1 - q_2, q_3, q_4)\) then:

\begin{equation}
    \alpha S_X + \beta S_Y + \gamma = 0
\end{equation}

In~\cite{Press2012} a particular type of ZD strategy is defined: extortionate
strategies. If:

\begin{equation}\label{eqn:constraint_for_extortion}
    \gamma = - P(\alpha + \beta)
\end{equation}

then the player can ensure they get a score \(\chi\) times
larger than the opponent. This extortion coefficient is given by:

\begin{equation}\label{eqn:definition_of_chi}
    \chi=\frac{-\beta}{\alpha}
\end{equation}

Thus, if (\ref{eqn:constraint_for_extortion}) holds and \(\chi >1\) a player is
said to extort their opponent.
Here, the reverse problem is considered: given a
\(p\in\mathbb{R}^4\) how does one identify \(\alpha, \beta\) if they
exist and is the strategy in fact acting in an extortionate way?

These conditions correspond to:

\begin{align}
    \tilde p_1 & = \alpha R + \beta R - P (\alpha + \beta)
            \label{eqn:condition_for_tilde_p1}\\
    \tilde p_2 & = \alpha S + \beta T - P (\alpha + \beta)
            \label{eqn:condition_for_tilde_p2}\\
    \tilde p_3 & = \alpha T + \beta S - P (\alpha + \beta)
            \label{eqn:condition_for_tilde_p3}\\
    \tilde p_4 & = \alpha P + \beta P - P (\alpha + \beta)
            \label{eqn:condition_for_tilde_p4}
\end{align}

Equation (\ref{eqn:condition_for_tilde_p4}) ensures that \(p_4=\tilde p_4=0\).
Equations (\ref{eqn:condition_for_tilde_p1}-\ref{eqn:condition_for_tilde_p3})
can be used to eliminate \(\alpha, \beta\), giving:

\begin{equation}\label{eqn:planar_definition_of_extortion}
    \tilde p_1 = \frac{(R - P)(\tilde p_2 + \tilde p_3)}{S + T - 2P}
\end{equation}

with:

\begin{equation}\label{eqn:definition_of_chi}
    \chi = \frac{\tilde p_2 (P - T) + \tilde p_3 (S - P)}
                {\tilde p_2 (P - S) + \tilde p_3 (T - P)}
\end{equation}

Given a strategy \(p\in\mathbb{R}^{4\times 1}\) equations
(\ref{eqn:condition_for_tilde_p4}), (\ref{eqn:planar_definition_of_extortion}-\ref{eqn:definition_of_chi}) can be used to check if
a strategy is extortionate. The conditions correspond to:

\begin{align}
    p_1 & = \frac{(R-P)(p_2 + p_3) - R + T + S - P}{S + T - 2P}
     \label{eqn:condition_for_p1}\\
    p_4 & = 0 \label{eqn:condition_for_p4}\\
    1 & > p_2 + p_3\label{eqn:condition_for_chi}
\end{align}

The algebraic steps necessary to prove these results are available in the
supporting materials.

All extortionate strategies reside on a triangular (\ref{eqn:condition_for_chi})
plane (\ref{eqn:condition_for_p1}) in 3 dimensions (\ref{eqn:condition_for_p4}).
Using this formulation it can be seen that a necessary (but not sufficient)
condition for an extortionate strategy is that it cooperates on average less
than 50\% of the time when in a state of disagreement with the opponent.

As an example, consider the known extortionate strategy \(p=(8 / 9, 1 / 2, 1 /
3, 0)\) from~\cite{Stewart2012} which is referred to as \texttt{Extort-2}. In
this case, for the standard values of \((R, T, S, P)\) constraint
(\ref{eqn:condition_for_p1}) corresponds to:

\begin{equation}
    p_1 = \frac{2(p_2 + p_3) + 1}{3}
\end{equation}

It is clear that in this case all constraints hold.

This approach could in fact be used to confirm that a given strategy is acting
in an extortionate manner even if it is not a memory one strategy. However, in
practice, if a closed form for \(p\) is not known, then due to measurement
and/or numerical error this would not work.

This problem can be written in the following linear algebraic form where
\(x=(\alpha, \beta)\)
and \(p^*=(\tilde p_1 - 1, tilde_2 - 1, p_3)\):

\begin{equation}\label{eqn:linear_algebraic_equation_for_p}
    Cx= p^*
\end{equation}

\(C\) corresponds to equations
(\ref{eqn:condition_for_tilde_p1}-\ref{eqn:condition_for_tilde_p3}) and is
given by:

\begin{equation}\label{eqn:definition_of_C}
    C =
    \begin{bmatrix}
        R - P & R- P \\
        S - P & T- P \\
        T - P & S- P \\
    \end{bmatrix}
\end{equation}

Note that in general, equation (\ref{eqn:linear_algebraic_equation_for_p}) will
not necessarily have a solution. From the Rouch\'{e}-Capelli theorem if there is
a solution it is unique as \(\text{rank}(C)=2\) which is the dimension of the
variable \(x\). The best fitting \(x\) is found by minimizing:

\begin{equation}\label{eqn:r_squared}
    \text{SSError} = \|C x- p^*\|_2^2 = \sum_{i=1}^{3}\left((C\bar x)_i-p_i^*\right)^2
\end{equation}

Note that \(\text{SSError}\), which is the square of the Frobenius
norm~\cite{Golub2013}, becomes a measure of how close a strategy is to being an
extortionate strategy. Suspicion
of extortion then corresponds to a threshold on \(\text{SSError}\).

By observing interactions (human or otherwise), their memory one representation
can be inferred and this approach can be used to recognise extortionate
behaviour. The notion of comparing theoretic and actual plays of the IPD is not
novel, see for example~\cite{Rand2013}. Immediately it is noted that if the
environment is noisy~\cite{Wu1995} then no strategy can be considered to be
extortionate as \(p_4>0\).

In the next section, this idea will be illustrated by observing the interactions
that take place in a computer based tournament of the IPD\@.

\section{Numerical experiments}\label{sec:numerical-experiments}

In~\cite{Stewart2012} results from a tournament with
\input{./assets/tex/number_of_stewart_plotkin_strategies/main.tex} strategies,
was presented with specific consideration given to ZD strategies. This
tournament is reproduced here using the Axelrod-Python
project~\cite{Knight2016}. To obtain a good measure of the corresponding
transition rates for each strategy all matches have been run for
\input{assets/tex/number_of_turns/main.tex} turns and every match has been
repeated \input{assets/tex/number_of_repetitions/main.tex} times. All of this
interaction data is available at~\cite{vincent_knight_2018_1297075}. A good
match between the inferred Markov chain and the state distribution of the actual
interactions has been verified. Data for this is presented in the supplementary
materials.

Figure~\ref{fig:SSError_overall_in_stewart_plotkin} shows the \(\text{SSError}\)
values for all the strategies in the tournament, as reported
in~\cite{Stewart2012} the extortionate strategy (which has an expected
\(\text{SSError}\) approximately 0) gains a large number of wins.

\begin{figure}[!htbp]
    \centering
    \includegraphics[width=.8\textwidth]{./assets/img/SSError_overall_in_stewart_plotkin/main.pdf}
    \caption{\(\text{SSError}\) and state probabilities for the strategies
        of~\cite{Stewart2012}, ordered both by number of wins and overall score.
        Note that \(P(DC)\) is not shown as it corresponds to the transpose of
        \(P(CD)\). Cooperator and Defector are omitted as they do not visit all
        the states.}
    \label{fig:SSError_overall_in_stewart_plotkin}
\end{figure}

Here, the work of~\cite{Stewart2012} is extended by investigating a tournament
with \input{assets/tex/number_of_full_strategies/main.tex}
strategies.

The results of this analysis are shown in
Figure~\ref{fig:SSError_and_probabilities_in_full}. The top ranking strategies
by number of wins seem to be extortionate (but not against all strategies) and
it can be seen that a small sub group of strategies achieve mutual defection.
All the top ranking strategies according to score achieve mutual cooperation and
do not extort each other, however they
\textbf{do} exhibit extortionate behaviour towards a number of the lower ranking
strategies.

\begin{figure}[!htbp]
    \centering
    \includegraphics[width=.8\textwidth]{./assets/img/SSError_and_probabilities_in_full/main.pdf}
    \caption{\(\text{SSError}\) for the strategies for the full tournament. Only
    strategy interactions for which \(p_4=0\) and \(\chi>1\) are displayed.}
    \label{fig:SSError_and_probabilities_in_full}
\end{figure}

\section{Conclusion}\label{sec:conclusion}

This work defines an approach to measure whether or not a player is playing a
strategy that corresponds to an extortionate strategy as defined
in~\cite{Press2012}: a mathematical model for suspicion. Indeed, all
extortionate strategies have been
 classified as lying on a triangular plane.
This rigorous classification fails to be robust to small measurement error, thus
a statistical approach is proposed.
This is done through a linear algebraic approach for approximating the solution
of a linear system. Using this, a large number of pairwise interactions is
simulated and in fact very few strategies are found to act extortionately.

The work of~\cite{Press2012}, whilst showing that a clever approach to taking
advantage of another memory one strategy exists: this is incomplete. Whilst the
elegance of this result is very attractive, just as the simplicity of the
victory of Tit For Tat in Axelrod's original tournaments was, it is incomplete.
Extortionate strategies achieve a high number of wins but they do not
achieve a high score which corresponds to the fitness landscape in an
evolutionary sense. From the large number of interactions a payoff matrix \(S\)
can be measured where \(S_{ij}\) denotes the score (using standard values of
\((R, S, T, P) = (3, 0, 5, 1)\)) of the \(i\)th strategy
against the \(j\)th strategy. Using this, the replicator equation
describes the evolution of the system based on a population density fitness
function:

\begin{equation}\label{eqn:replicator_dynamics}
    \frac{dx}{dt} = x(S-x^TS x)
\end{equation}

Equation (\ref{eqn:replicator_dynamics}) is solved numerically through an
integration technique described in~\cite{Petzold1983} and
Figure~\ref{fig:replicator_dynamics} shows the evolution of the distribution of
the system: the various strategies are ranked by scores. It is clear to see that
only the high ranking strategies survive the evolutionary process (in fact,
only \input{./assets/img/replicator_dynamics/main.tex}
have a final distribution greater than \(10 ^ {-2}\)). This confirms the
findings of~\cite{Moran1707} in which sophisticated strategies resist
evolutionary invasion of shorter memory strategies. Recalling
Figure~\ref{fig:SSError_and_probabilities_in_full} this demonstrates that:

\begin{itemize}
    \item Cooperation emerges through the evolutionary process: the high scoring
        strategies do not exhibit extortionate behaviour towards each other.
    \item Extortionate strategies do not survive the evolutionary process.
\end{itemize}

\begin{figure}[!htbp]
    \centering
    \includegraphics[width=.8\textwidth]{./assets/img/replicator_dynamics/main.pdf}
    \caption{Numerical simulation of the replicator equation
    (\ref{eqn:replicator_dynamics}): strategies are ordered by score, only the strategies with a high score survive the evolutionary process.}
    \label{fig:replicator_dynamics}
\end{figure}

This work can be used to classify plays of the IPD\@: data can be collected from
actual interactions (in lab or in the field). Furthermore, this allows for a
classification method similar to the notion of fingerprinting presented
in~\cite{Ashlock2008}. Trained strategies can potentially be classified as
extortionate or not or it could be possible to even constrain the reinforcement
learning approaches that are becoming prevalent in the literature.
Alternatively, this mathematical approach for recognising extortion could be
used in sophisticated strategies to defend against invasion. Arguably, some of
the strategies considered here exhibit this behaviour, indeed as described
in~\cite{Harper2017}, the top ranking strategies in the full tournament are
obtained using evolutionary reinforcement learning techniques, thus, suspicion
of extortionate behaviour could in fact be an evolutionary trait.

\section*{Acknowledgements}

The following open source software libraries were used in this research:

\begin{itemize}
    \item The Axelrod ~\cite{Knight2016, Knight2018} library (IPD strategies and
        tournaments).
    \item The sympy library~\cite{Meurer2017} (verification of all symbolic
        calculations).
    \item The matplotlib~\cite{Droettboom2018} library (visualisation).
    \item The pandas~\cite{Structures2010}, dask~\cite{Dask2016} and
        NumPy~\cite{Oliphant2015} libraries (data manipulation).
    \item The SciPy~\cite{Jones2001} library (numerical integration of the
        replicator equation).
\end{itemize}

This work was performed using the computational facilities of the Advanced
Research Computing @ Cardiff (ARCCA) Division, Cardiff University.

\printbibliography

\newpage
\section*{Supplementary materials}

\includepdf{assets/pdf/proof_of_form_of_extortionate_strategies/main.pdf}

\newpage

Using the pair wise interactions the transition rates \(p,
q\) can be measured and the steady state probabilities inferred and compared to
the actual probabilities of each state.
This is done numerically by computing the singular eigenvector of the
matrix \(A\) \cite{Stewart2009}:

\[
    A =
    \begin{bmatrix}
        p_1 q_1 & p_1 (1 - q_1) & (1 - p_1) q_1 & (1 -p_1) (1 - q_1) \\
        p_2 q_2 & p_2 (1 - q_2) & (1 - p_2) q_2 & (1 -p_2) (1 - q_2) \\
        p_3 q_3 & p_3 (1 - q_3) & (1 - p_3) q_3 & (1 -p_3) (1 - q_3) \\
        p_4 q_4 & p_4 (1 - q_4) & (1 - p_4) q_4 & (1 -p_4) (1 - q_4) \\
    \end{bmatrix}
\]

Figure~\ref{fig:computed_probabilities_vs_theoretic_probabilities} shows a
regression line fitted to every pairwise interaction with a reported
\(\text{SSError}\) value (pairwise interactions with missing states were
omitted). This serves to validate the approach: a part from some edge cases the
relationship is consistent.

\begin{figure}[!htbp]
    \centering
    \includegraphics[width=.8\textwidth]{./assets/img/computed_probabilities_vs_theoretic_probabilities/main.pdf}
    \caption{The
        relationship between the steady state probabilities inferred from the
        measured transitions and the actual steady state probabilities. A linear
        regression line is included validating the approach.}
    \label{fig:computed_probabilities_vs_theoretic_probabilities}
\end{figure}


\end{document}
 turns and every match has been
repeated \documentclass[a4paper]{article}

\usepackage{amsmath}
\usepackage{amssymb}
\usepackage[margin=1.5cm,
            includefoot,
            footskip=30pt]{geometry}
\usepackage{layout}
\usepackage{graphicx}
\usepackage{subcaption}

\usepackage{biblatex}
\usepackage{pdfpages}

\bibliography{main.bib}

\title{Suspicion: Recognising and evaluating the effectiveness
       of extortion in the Iterated Prisoner's Dilemma}
\author{Vincent A. Knight \and Nikoleta E. Glynatsi}
\date{\today}



\begin{document}

\maketitle

\begin{abstract}
    The Iterated Prisoner's Dilemma is a model for rational and evolutionary
    interactive behaviour. It has applications both in the study of human social
    behaviour as well as in biology.
    It is used to understand when and how a rational individual might
    accept an immediate cost to their own utility for the direct benefit of
    another.

    Much attention has been given to a class of strategies called
    Zero Determinant strategies. It has been theoretically shown that these
    strategies can ``extort'' any player.

    In this work, an approach to identify if observed strategies are playing in
    an extortionate way is described. Furthermore, experimental analysis of
    a large tournament with \input{assets/tex/number_of_full_strategies/main.tex}
    strategies is considered. In this setting
    the most highly performing strategies do not play in an extortionate way
    against each other but do against lower performing strategies.
    This suggests that whilst the theory of Zero Determinant strategies
    indicates that memory is not of fundamental importance to the evolution of
    cooperative behaviour, this is incomplete.
\end{abstract}

\section{Introduction}\label{sec:introduction}

Agent based game theoretic models have become a stalwart of the underpinning
mathematics of interactive behaviours. One of the major pieces of work
in this area is the pair of original computer tournaments run by Robert
Axelrod~\cite{Axelrod1980, Axelrod1980a}. These tournaments pitted submitted
computer strategies against each other in plays of the Iterated Prisoner's
Dilemma. A common game where agents can choose to pay a slight cost to their
immediate utility in the hope of building a reputation. This has been used in
economic and evolutionary game theory to understand the evolution of cooperative
behaviour.

Recently, a class of strategies was described in~\cite{Press2012} that can
provably extort any given opponent. In~\cite{Hilbe2013, Moran1707} some
questions have already been asked about the true effectiveness of these
strategies in an evolutionary setting. Here another question is asked: is it
possible to recognise this extortionate behaviour? A mathematical procedure for
suspicion is presented: in the same way that the continued actions of an
extortionate individual might raise suspicion.

This work makes use of the Axelrod Python library~\cite{Knight2018, Knight2016}
with a large number of Prisoner Dilemma strategies available to give an
extensive numerical example of the ideas presented.  The approach is presented
in Section~\ref{sec:delta-zd-strategies}.  All of the code and data discussed
in Section~\ref{sec:numerical-experiments} is open sourced, archived and
written according to best scientific principles~\cite{Wilson2014}. The data
archive can be found at~\cite{vincent_knight_2018_1297075}.

\section{Recognising Extortion}\label{sec:delta-zd-strategies}

In~\cite{Press2012}, given a match between 2 memory-one strategies, the concept
of Zero Determinant (ZD) strategies is introduced. The main result of that paper
shows that given two memory one players \(p, q\in\mathbb{R}^4\) a linear
relationship between the players' scores could be forced by one of the players.

Using the notation of~\cite{Press2012}, assuming the utilities for player \(p\)
are given by \(S_x=(R, S, T, P)\) and for player \(q\) by \(S_y=(R, T, S, P)\)
and that the stationary scores of each player is given by \(S_X\) and \(S_Y\)
respectively. The main result of~\cite{Press2012} is that if

\begin{equation}\label{eqn:linear_relationship_for_p}
    \tilde p=\alpha S_x + \beta S_y + \gamma
\end{equation}

or

\begin{equation}\label{eqn:linear_relationship_for_q}
    \tilde q=\alpha S_x + \beta S_y + \gamma
\end{equation}

where \(\tilde p = (1 - p_1, 1 - p_2, p_3, p_4)\) and
\(\tilde q = (1 - q_1, 1 - q_2, q_3, q_4)\) then:

\begin{equation}
    \alpha S_X + \beta S_Y + \gamma = 0
\end{equation}

In~\cite{Press2012} a particular type of ZD strategy is defined: extortionate
strategies. If:

\begin{equation}\label{eqn:constraint_for_extortion}
    \gamma = - P(\alpha + \beta)
\end{equation}

then the player can ensure they get a score \(\chi\) times
larger than the opponent. This extortion coefficient is given by:

\begin{equation}\label{eqn:definition_of_chi}
    \chi=\frac{-\beta}{\alpha}
\end{equation}

Thus, if (\ref{eqn:constraint_for_extortion}) holds and \(\chi >1\) a player is
said to extort their opponent.
Here, the reverse problem is considered: given a
\(p\in\mathbb{R}^4\) how does one identify \(\alpha, \beta\) if they
exist and is the strategy in fact acting in an extortionate way?

These conditions correspond to:

\begin{align}
    \tilde p_1 & = \alpha R + \beta R - P (\alpha + \beta)
            \label{eqn:condition_for_tilde_p1}\\
    \tilde p_2 & = \alpha S + \beta T - P (\alpha + \beta)
            \label{eqn:condition_for_tilde_p2}\\
    \tilde p_3 & = \alpha T + \beta S - P (\alpha + \beta)
            \label{eqn:condition_for_tilde_p3}\\
    \tilde p_4 & = \alpha P + \beta P - P (\alpha + \beta)
            \label{eqn:condition_for_tilde_p4}
\end{align}

Equation (\ref{eqn:condition_for_tilde_p4}) ensures that \(p_4=\tilde p_4=0\).
Equations (\ref{eqn:condition_for_tilde_p1}-\ref{eqn:condition_for_tilde_p3})
can be used to eliminate \(\alpha, \beta\), giving:

\begin{equation}\label{eqn:planar_definition_of_extortion}
    \tilde p_1 = \frac{(R - P)(\tilde p_2 + \tilde p_3)}{S + T - 2P}
\end{equation}

with:

\begin{equation}\label{eqn:definition_of_chi}
    \chi = \frac{\tilde p_2 (P - T) + \tilde p_3 (S - P)}
                {\tilde p_2 (P - S) + \tilde p_3 (T - P)}
\end{equation}

Given a strategy \(p\in\mathbb{R}^{4\times 1}\) equations
(\ref{eqn:condition_for_tilde_p4}), (\ref{eqn:planar_definition_of_extortion}-\ref{eqn:definition_of_chi}) can be used to check if
a strategy is extortionate. The conditions correspond to:

\begin{align}
    p_1 & = \frac{(R-P)(p_2 + p_3) - R + T + S - P}{S + T - 2P}
     \label{eqn:condition_for_p1}\\
    p_4 & = 0 \label{eqn:condition_for_p4}\\
    1 & > p_2 + p_3\label{eqn:condition_for_chi}
\end{align}

The algebraic steps necessary to prove these results are available in the
supporting materials.

All extortionate strategies reside on a triangular (\ref{eqn:condition_for_chi})
plane (\ref{eqn:condition_for_p1}) in 3 dimensions (\ref{eqn:condition_for_p4}).
Using this formulation it can be seen that a necessary (but not sufficient)
condition for an extortionate strategy is that it cooperates on average less
than 50\% of the time when in a state of disagreement with the opponent.

As an example, consider the known extortionate strategy \(p=(8 / 9, 1 / 2, 1 /
3, 0)\) from~\cite{Stewart2012} which is referred to as \texttt{Extort-2}. In
this case, for the standard values of \((R, T, S, P)\) constraint
(\ref{eqn:condition_for_p1}) corresponds to:

\begin{equation}
    p_1 = \frac{2(p_2 + p_3) + 1}{3}
\end{equation}

It is clear that in this case all constraints hold.

This approach could in fact be used to confirm that a given strategy is acting
in an extortionate manner even if it is not a memory one strategy. However, in
practice, if a closed form for \(p\) is not known, then due to measurement
and/or numerical error this would not work.

This problem can be written in the following linear algebraic form where
\(x=(\alpha, \beta)\)
and \(p^*=(\tilde p_1 - 1, tilde_2 - 1, p_3)\):

\begin{equation}\label{eqn:linear_algebraic_equation_for_p}
    Cx= p^*
\end{equation}

\(C\) corresponds to equations
(\ref{eqn:condition_for_tilde_p1}-\ref{eqn:condition_for_tilde_p3}) and is
given by:

\begin{equation}\label{eqn:definition_of_C}
    C =
    \begin{bmatrix}
        R - P & R- P \\
        S - P & T- P \\
        T - P & S- P \\
    \end{bmatrix}
\end{equation}

Note that in general, equation (\ref{eqn:linear_algebraic_equation_for_p}) will
not necessarily have a solution. From the Rouch\'{e}-Capelli theorem if there is
a solution it is unique as \(\text{rank}(C)=2\) which is the dimension of the
variable \(x\). The best fitting \(x\) is found by minimizing:

\begin{equation}\label{eqn:r_squared}
    \text{SSError} = \|C x- p^*\|_2^2 = \sum_{i=1}^{3}\left((C\bar x)_i-p_i^*\right)^2
\end{equation}

Note that \(\text{SSError}\), which is the square of the Frobenius
norm~\cite{Golub2013}, becomes a measure of how close a strategy is to being an
extortionate strategy. Suspicion
of extortion then corresponds to a threshold on \(\text{SSError}\).

By observing interactions (human or otherwise), their memory one representation
can be inferred and this approach can be used to recognise extortionate
behaviour. The notion of comparing theoretic and actual plays of the IPD is not
novel, see for example~\cite{Rand2013}. Immediately it is noted that if the
environment is noisy~\cite{Wu1995} then no strategy can be considered to be
extortionate as \(p_4>0\).

In the next section, this idea will be illustrated by observing the interactions
that take place in a computer based tournament of the IPD\@.

\section{Numerical experiments}\label{sec:numerical-experiments}

In~\cite{Stewart2012} results from a tournament with
\input{./assets/tex/number_of_stewart_plotkin_strategies/main.tex} strategies,
was presented with specific consideration given to ZD strategies. This
tournament is reproduced here using the Axelrod-Python
project~\cite{Knight2016}. To obtain a good measure of the corresponding
transition rates for each strategy all matches have been run for
\input{assets/tex/number_of_turns/main.tex} turns and every match has been
repeated \input{assets/tex/number_of_repetitions/main.tex} times. All of this
interaction data is available at~\cite{vincent_knight_2018_1297075}. A good
match between the inferred Markov chain and the state distribution of the actual
interactions has been verified. Data for this is presented in the supplementary
materials.

Figure~\ref{fig:SSError_overall_in_stewart_plotkin} shows the \(\text{SSError}\)
values for all the strategies in the tournament, as reported
in~\cite{Stewart2012} the extortionate strategy (which has an expected
\(\text{SSError}\) approximately 0) gains a large number of wins.

\begin{figure}[!htbp]
    \centering
    \includegraphics[width=.8\textwidth]{./assets/img/SSError_overall_in_stewart_plotkin/main.pdf}
    \caption{\(\text{SSError}\) and state probabilities for the strategies
        of~\cite{Stewart2012}, ordered both by number of wins and overall score.
        Note that \(P(DC)\) is not shown as it corresponds to the transpose of
        \(P(CD)\). Cooperator and Defector are omitted as they do not visit all
        the states.}
    \label{fig:SSError_overall_in_stewart_plotkin}
\end{figure}

Here, the work of~\cite{Stewart2012} is extended by investigating a tournament
with \input{assets/tex/number_of_full_strategies/main.tex}
strategies.

The results of this analysis are shown in
Figure~\ref{fig:SSError_and_probabilities_in_full}. The top ranking strategies
by number of wins seem to be extortionate (but not against all strategies) and
it can be seen that a small sub group of strategies achieve mutual defection.
All the top ranking strategies according to score achieve mutual cooperation and
do not extort each other, however they
\textbf{do} exhibit extortionate behaviour towards a number of the lower ranking
strategies.

\begin{figure}[!htbp]
    \centering
    \includegraphics[width=.8\textwidth]{./assets/img/SSError_and_probabilities_in_full/main.pdf}
    \caption{\(\text{SSError}\) for the strategies for the full tournament. Only
    strategy interactions for which \(p_4=0\) and \(\chi>1\) are displayed.}
    \label{fig:SSError_and_probabilities_in_full}
\end{figure}

\section{Conclusion}\label{sec:conclusion}

This work defines an approach to measure whether or not a player is playing a
strategy that corresponds to an extortionate strategy as defined
in~\cite{Press2012}: a mathematical model for suspicion. Indeed, all
extortionate strategies have been
 classified as lying on a triangular plane.
This rigorous classification fails to be robust to small measurement error, thus
a statistical approach is proposed.
This is done through a linear algebraic approach for approximating the solution
of a linear system. Using this, a large number of pairwise interactions is
simulated and in fact very few strategies are found to act extortionately.

The work of~\cite{Press2012}, whilst showing that a clever approach to taking
advantage of another memory one strategy exists: this is incomplete. Whilst the
elegance of this result is very attractive, just as the simplicity of the
victory of Tit For Tat in Axelrod's original tournaments was, it is incomplete.
Extortionate strategies achieve a high number of wins but they do not
achieve a high score which corresponds to the fitness landscape in an
evolutionary sense. From the large number of interactions a payoff matrix \(S\)
can be measured where \(S_{ij}\) denotes the score (using standard values of
\((R, S, T, P) = (3, 0, 5, 1)\)) of the \(i\)th strategy
against the \(j\)th strategy. Using this, the replicator equation
describes the evolution of the system based on a population density fitness
function:

\begin{equation}\label{eqn:replicator_dynamics}
    \frac{dx}{dt} = x(S-x^TS x)
\end{equation}

Equation (\ref{eqn:replicator_dynamics}) is solved numerically through an
integration technique described in~\cite{Petzold1983} and
Figure~\ref{fig:replicator_dynamics} shows the evolution of the distribution of
the system: the various strategies are ranked by scores. It is clear to see that
only the high ranking strategies survive the evolutionary process (in fact,
only \input{./assets/img/replicator_dynamics/main.tex}
have a final distribution greater than \(10 ^ {-2}\)). This confirms the
findings of~\cite{Moran1707} in which sophisticated strategies resist
evolutionary invasion of shorter memory strategies. Recalling
Figure~\ref{fig:SSError_and_probabilities_in_full} this demonstrates that:

\begin{itemize}
    \item Cooperation emerges through the evolutionary process: the high scoring
        strategies do not exhibit extortionate behaviour towards each other.
    \item Extortionate strategies do not survive the evolutionary process.
\end{itemize}

\begin{figure}[!htbp]
    \centering
    \includegraphics[width=.8\textwidth]{./assets/img/replicator_dynamics/main.pdf}
    \caption{Numerical simulation of the replicator equation
    (\ref{eqn:replicator_dynamics}): strategies are ordered by score, only the strategies with a high score survive the evolutionary process.}
    \label{fig:replicator_dynamics}
\end{figure}

This work can be used to classify plays of the IPD\@: data can be collected from
actual interactions (in lab or in the field). Furthermore, this allows for a
classification method similar to the notion of fingerprinting presented
in~\cite{Ashlock2008}. Trained strategies can potentially be classified as
extortionate or not or it could be possible to even constrain the reinforcement
learning approaches that are becoming prevalent in the literature.
Alternatively, this mathematical approach for recognising extortion could be
used in sophisticated strategies to defend against invasion. Arguably, some of
the strategies considered here exhibit this behaviour, indeed as described
in~\cite{Harper2017}, the top ranking strategies in the full tournament are
obtained using evolutionary reinforcement learning techniques, thus, suspicion
of extortionate behaviour could in fact be an evolutionary trait.

\section*{Acknowledgements}

The following open source software libraries were used in this research:

\begin{itemize}
    \item The Axelrod ~\cite{Knight2016, Knight2018} library (IPD strategies and
        tournaments).
    \item The sympy library~\cite{Meurer2017} (verification of all symbolic
        calculations).
    \item The matplotlib~\cite{Droettboom2018} library (visualisation).
    \item The pandas~\cite{Structures2010}, dask~\cite{Dask2016} and
        NumPy~\cite{Oliphant2015} libraries (data manipulation).
    \item The SciPy~\cite{Jones2001} library (numerical integration of the
        replicator equation).
\end{itemize}

This work was performed using the computational facilities of the Advanced
Research Computing @ Cardiff (ARCCA) Division, Cardiff University.

\printbibliography

\newpage
\section*{Supplementary materials}

\includepdf{assets/pdf/proof_of_form_of_extortionate_strategies/main.pdf}

\newpage

Using the pair wise interactions the transition rates \(p,
q\) can be measured and the steady state probabilities inferred and compared to
the actual probabilities of each state.
This is done numerically by computing the singular eigenvector of the
matrix \(A\) \cite{Stewart2009}:

\[
    A =
    \begin{bmatrix}
        p_1 q_1 & p_1 (1 - q_1) & (1 - p_1) q_1 & (1 -p_1) (1 - q_1) \\
        p_2 q_2 & p_2 (1 - q_2) & (1 - p_2) q_2 & (1 -p_2) (1 - q_2) \\
        p_3 q_3 & p_3 (1 - q_3) & (1 - p_3) q_3 & (1 -p_3) (1 - q_3) \\
        p_4 q_4 & p_4 (1 - q_4) & (1 - p_4) q_4 & (1 -p_4) (1 - q_4) \\
    \end{bmatrix}
\]

Figure~\ref{fig:computed_probabilities_vs_theoretic_probabilities} shows a
regression line fitted to every pairwise interaction with a reported
\(\text{SSError}\) value (pairwise interactions with missing states were
omitted). This serves to validate the approach: a part from some edge cases the
relationship is consistent.

\begin{figure}[!htbp]
    \centering
    \includegraphics[width=.8\textwidth]{./assets/img/computed_probabilities_vs_theoretic_probabilities/main.pdf}
    \caption{The
        relationship between the steady state probabilities inferred from the
        measured transitions and the actual steady state probabilities. A linear
        regression line is included validating the approach.}
    \label{fig:computed_probabilities_vs_theoretic_probabilities}
\end{figure}


\end{document}
 times. All of this
interaction data is available at~\cite{vincent_knight_2018_1297075}. A good
match between the inferred Markov chain and the state distribution of the actual
interactions has been verified. Data for this is presented in the supplementary
materials.

Figure~\ref{fig:SSError_overall_in_stewart_plotkin} shows the \(\text{SSError}\)
values for all the strategies in the tournament, as reported
in~\cite{Stewart2012} the extortionate strategy (which has an expected
\(\text{SSError}\) approximately 0) gains a large number of wins.

\begin{figure}[!htbp]
    \centering
    \includegraphics[width=.8\textwidth]{./assets/img/SSError_overall_in_stewart_plotkin/main.pdf}
    \caption{\(\text{SSError}\) and state probabilities for the strategies
        of~\cite{Stewart2012}, ordered both by number of wins and overall score.
        Note that \(P(DC)\) is not shown as it corresponds to the transpose of
        \(P(CD)\). Cooperator and Defector are omitted as they do not visit all
        the states.}
    \label{fig:SSError_overall_in_stewart_plotkin}
\end{figure}

Here, the work of~\cite{Stewart2012} is extended by investigating a tournament
with \documentclass[a4paper]{article}

\usepackage{amsmath}
\usepackage{amssymb}
\usepackage[margin=1.5cm,
            includefoot,
            footskip=30pt]{geometry}
\usepackage{layout}
\usepackage{graphicx}
\usepackage{subcaption}

\usepackage{biblatex}
\usepackage{pdfpages}

\bibliography{main.bib}

\title{Suspicion: Recognising and evaluating the effectiveness
       of extortion in the Iterated Prisoner's Dilemma}
\author{Vincent A. Knight \and Nikoleta E. Glynatsi}
\date{\today}



\begin{document}

\maketitle

\begin{abstract}
    The Iterated Prisoner's Dilemma is a model for rational and evolutionary
    interactive behaviour. It has applications both in the study of human social
    behaviour as well as in biology.
    It is used to understand when and how a rational individual might
    accept an immediate cost to their own utility for the direct benefit of
    another.

    Much attention has been given to a class of strategies called
    Zero Determinant strategies. It has been theoretically shown that these
    strategies can ``extort'' any player.

    In this work, an approach to identify if observed strategies are playing in
    an extortionate way is described. Furthermore, experimental analysis of
    a large tournament with \input{assets/tex/number_of_full_strategies/main.tex}
    strategies is considered. In this setting
    the most highly performing strategies do not play in an extortionate way
    against each other but do against lower performing strategies.
    This suggests that whilst the theory of Zero Determinant strategies
    indicates that memory is not of fundamental importance to the evolution of
    cooperative behaviour, this is incomplete.
\end{abstract}

\section{Introduction}\label{sec:introduction}

Agent based game theoretic models have become a stalwart of the underpinning
mathematics of interactive behaviours. One of the major pieces of work
in this area is the pair of original computer tournaments run by Robert
Axelrod~\cite{Axelrod1980, Axelrod1980a}. These tournaments pitted submitted
computer strategies against each other in plays of the Iterated Prisoner's
Dilemma. A common game where agents can choose to pay a slight cost to their
immediate utility in the hope of building a reputation. This has been used in
economic and evolutionary game theory to understand the evolution of cooperative
behaviour.

Recently, a class of strategies was described in~\cite{Press2012} that can
provably extort any given opponent. In~\cite{Hilbe2013, Moran1707} some
questions have already been asked about the true effectiveness of these
strategies in an evolutionary setting. Here another question is asked: is it
possible to recognise this extortionate behaviour? A mathematical procedure for
suspicion is presented: in the same way that the continued actions of an
extortionate individual might raise suspicion.

This work makes use of the Axelrod Python library~\cite{Knight2018, Knight2016}
with a large number of Prisoner Dilemma strategies available to give an
extensive numerical example of the ideas presented.  The approach is presented
in Section~\ref{sec:delta-zd-strategies}.  All of the code and data discussed
in Section~\ref{sec:numerical-experiments} is open sourced, archived and
written according to best scientific principles~\cite{Wilson2014}. The data
archive can be found at~\cite{vincent_knight_2018_1297075}.

\section{Recognising Extortion}\label{sec:delta-zd-strategies}

In~\cite{Press2012}, given a match between 2 memory-one strategies, the concept
of Zero Determinant (ZD) strategies is introduced. The main result of that paper
shows that given two memory one players \(p, q\in\mathbb{R}^4\) a linear
relationship between the players' scores could be forced by one of the players.

Using the notation of~\cite{Press2012}, assuming the utilities for player \(p\)
are given by \(S_x=(R, S, T, P)\) and for player \(q\) by \(S_y=(R, T, S, P)\)
and that the stationary scores of each player is given by \(S_X\) and \(S_Y\)
respectively. The main result of~\cite{Press2012} is that if

\begin{equation}\label{eqn:linear_relationship_for_p}
    \tilde p=\alpha S_x + \beta S_y + \gamma
\end{equation}

or

\begin{equation}\label{eqn:linear_relationship_for_q}
    \tilde q=\alpha S_x + \beta S_y + \gamma
\end{equation}

where \(\tilde p = (1 - p_1, 1 - p_2, p_3, p_4)\) and
\(\tilde q = (1 - q_1, 1 - q_2, q_3, q_4)\) then:

\begin{equation}
    \alpha S_X + \beta S_Y + \gamma = 0
\end{equation}

In~\cite{Press2012} a particular type of ZD strategy is defined: extortionate
strategies. If:

\begin{equation}\label{eqn:constraint_for_extortion}
    \gamma = - P(\alpha + \beta)
\end{equation}

then the player can ensure they get a score \(\chi\) times
larger than the opponent. This extortion coefficient is given by:

\begin{equation}\label{eqn:definition_of_chi}
    \chi=\frac{-\beta}{\alpha}
\end{equation}

Thus, if (\ref{eqn:constraint_for_extortion}) holds and \(\chi >1\) a player is
said to extort their opponent.
Here, the reverse problem is considered: given a
\(p\in\mathbb{R}^4\) how does one identify \(\alpha, \beta\) if they
exist and is the strategy in fact acting in an extortionate way?

These conditions correspond to:

\begin{align}
    \tilde p_1 & = \alpha R + \beta R - P (\alpha + \beta)
            \label{eqn:condition_for_tilde_p1}\\
    \tilde p_2 & = \alpha S + \beta T - P (\alpha + \beta)
            \label{eqn:condition_for_tilde_p2}\\
    \tilde p_3 & = \alpha T + \beta S - P (\alpha + \beta)
            \label{eqn:condition_for_tilde_p3}\\
    \tilde p_4 & = \alpha P + \beta P - P (\alpha + \beta)
            \label{eqn:condition_for_tilde_p4}
\end{align}

Equation (\ref{eqn:condition_for_tilde_p4}) ensures that \(p_4=\tilde p_4=0\).
Equations (\ref{eqn:condition_for_tilde_p1}-\ref{eqn:condition_for_tilde_p3})
can be used to eliminate \(\alpha, \beta\), giving:

\begin{equation}\label{eqn:planar_definition_of_extortion}
    \tilde p_1 = \frac{(R - P)(\tilde p_2 + \tilde p_3)}{S + T - 2P}
\end{equation}

with:

\begin{equation}\label{eqn:definition_of_chi}
    \chi = \frac{\tilde p_2 (P - T) + \tilde p_3 (S - P)}
                {\tilde p_2 (P - S) + \tilde p_3 (T - P)}
\end{equation}

Given a strategy \(p\in\mathbb{R}^{4\times 1}\) equations
(\ref{eqn:condition_for_tilde_p4}), (\ref{eqn:planar_definition_of_extortion}-\ref{eqn:definition_of_chi}) can be used to check if
a strategy is extortionate. The conditions correspond to:

\begin{align}
    p_1 & = \frac{(R-P)(p_2 + p_3) - R + T + S - P}{S + T - 2P}
     \label{eqn:condition_for_p1}\\
    p_4 & = 0 \label{eqn:condition_for_p4}\\
    1 & > p_2 + p_3\label{eqn:condition_for_chi}
\end{align}

The algebraic steps necessary to prove these results are available in the
supporting materials.

All extortionate strategies reside on a triangular (\ref{eqn:condition_for_chi})
plane (\ref{eqn:condition_for_p1}) in 3 dimensions (\ref{eqn:condition_for_p4}).
Using this formulation it can be seen that a necessary (but not sufficient)
condition for an extortionate strategy is that it cooperates on average less
than 50\% of the time when in a state of disagreement with the opponent.

As an example, consider the known extortionate strategy \(p=(8 / 9, 1 / 2, 1 /
3, 0)\) from~\cite{Stewart2012} which is referred to as \texttt{Extort-2}. In
this case, for the standard values of \((R, T, S, P)\) constraint
(\ref{eqn:condition_for_p1}) corresponds to:

\begin{equation}
    p_1 = \frac{2(p_2 + p_3) + 1}{3}
\end{equation}

It is clear that in this case all constraints hold.

This approach could in fact be used to confirm that a given strategy is acting
in an extortionate manner even if it is not a memory one strategy. However, in
practice, if a closed form for \(p\) is not known, then due to measurement
and/or numerical error this would not work.

This problem can be written in the following linear algebraic form where
\(x=(\alpha, \beta)\)
and \(p^*=(\tilde p_1 - 1, tilde_2 - 1, p_3)\):

\begin{equation}\label{eqn:linear_algebraic_equation_for_p}
    Cx= p^*
\end{equation}

\(C\) corresponds to equations
(\ref{eqn:condition_for_tilde_p1}-\ref{eqn:condition_for_tilde_p3}) and is
given by:

\begin{equation}\label{eqn:definition_of_C}
    C =
    \begin{bmatrix}
        R - P & R- P \\
        S - P & T- P \\
        T - P & S- P \\
    \end{bmatrix}
\end{equation}

Note that in general, equation (\ref{eqn:linear_algebraic_equation_for_p}) will
not necessarily have a solution. From the Rouch\'{e}-Capelli theorem if there is
a solution it is unique as \(\text{rank}(C)=2\) which is the dimension of the
variable \(x\). The best fitting \(x\) is found by minimizing:

\begin{equation}\label{eqn:r_squared}
    \text{SSError} = \|C x- p^*\|_2^2 = \sum_{i=1}^{3}\left((C\bar x)_i-p_i^*\right)^2
\end{equation}

Note that \(\text{SSError}\), which is the square of the Frobenius
norm~\cite{Golub2013}, becomes a measure of how close a strategy is to being an
extortionate strategy. Suspicion
of extortion then corresponds to a threshold on \(\text{SSError}\).

By observing interactions (human or otherwise), their memory one representation
can be inferred and this approach can be used to recognise extortionate
behaviour. The notion of comparing theoretic and actual plays of the IPD is not
novel, see for example~\cite{Rand2013}. Immediately it is noted that if the
environment is noisy~\cite{Wu1995} then no strategy can be considered to be
extortionate as \(p_4>0\).

In the next section, this idea will be illustrated by observing the interactions
that take place in a computer based tournament of the IPD\@.

\section{Numerical experiments}\label{sec:numerical-experiments}

In~\cite{Stewart2012} results from a tournament with
\input{./assets/tex/number_of_stewart_plotkin_strategies/main.tex} strategies,
was presented with specific consideration given to ZD strategies. This
tournament is reproduced here using the Axelrod-Python
project~\cite{Knight2016}. To obtain a good measure of the corresponding
transition rates for each strategy all matches have been run for
\input{assets/tex/number_of_turns/main.tex} turns and every match has been
repeated \input{assets/tex/number_of_repetitions/main.tex} times. All of this
interaction data is available at~\cite{vincent_knight_2018_1297075}. A good
match between the inferred Markov chain and the state distribution of the actual
interactions has been verified. Data for this is presented in the supplementary
materials.

Figure~\ref{fig:SSError_overall_in_stewart_plotkin} shows the \(\text{SSError}\)
values for all the strategies in the tournament, as reported
in~\cite{Stewart2012} the extortionate strategy (which has an expected
\(\text{SSError}\) approximately 0) gains a large number of wins.

\begin{figure}[!htbp]
    \centering
    \includegraphics[width=.8\textwidth]{./assets/img/SSError_overall_in_stewart_plotkin/main.pdf}
    \caption{\(\text{SSError}\) and state probabilities for the strategies
        of~\cite{Stewart2012}, ordered both by number of wins and overall score.
        Note that \(P(DC)\) is not shown as it corresponds to the transpose of
        \(P(CD)\). Cooperator and Defector are omitted as they do not visit all
        the states.}
    \label{fig:SSError_overall_in_stewart_plotkin}
\end{figure}

Here, the work of~\cite{Stewart2012} is extended by investigating a tournament
with \input{assets/tex/number_of_full_strategies/main.tex}
strategies.

The results of this analysis are shown in
Figure~\ref{fig:SSError_and_probabilities_in_full}. The top ranking strategies
by number of wins seem to be extortionate (but not against all strategies) and
it can be seen that a small sub group of strategies achieve mutual defection.
All the top ranking strategies according to score achieve mutual cooperation and
do not extort each other, however they
\textbf{do} exhibit extortionate behaviour towards a number of the lower ranking
strategies.

\begin{figure}[!htbp]
    \centering
    \includegraphics[width=.8\textwidth]{./assets/img/SSError_and_probabilities_in_full/main.pdf}
    \caption{\(\text{SSError}\) for the strategies for the full tournament. Only
    strategy interactions for which \(p_4=0\) and \(\chi>1\) are displayed.}
    \label{fig:SSError_and_probabilities_in_full}
\end{figure}

\section{Conclusion}\label{sec:conclusion}

This work defines an approach to measure whether or not a player is playing a
strategy that corresponds to an extortionate strategy as defined
in~\cite{Press2012}: a mathematical model for suspicion. Indeed, all
extortionate strategies have been
 classified as lying on a triangular plane.
This rigorous classification fails to be robust to small measurement error, thus
a statistical approach is proposed.
This is done through a linear algebraic approach for approximating the solution
of a linear system. Using this, a large number of pairwise interactions is
simulated and in fact very few strategies are found to act extortionately.

The work of~\cite{Press2012}, whilst showing that a clever approach to taking
advantage of another memory one strategy exists: this is incomplete. Whilst the
elegance of this result is very attractive, just as the simplicity of the
victory of Tit For Tat in Axelrod's original tournaments was, it is incomplete.
Extortionate strategies achieve a high number of wins but they do not
achieve a high score which corresponds to the fitness landscape in an
evolutionary sense. From the large number of interactions a payoff matrix \(S\)
can be measured where \(S_{ij}\) denotes the score (using standard values of
\((R, S, T, P) = (3, 0, 5, 1)\)) of the \(i\)th strategy
against the \(j\)th strategy. Using this, the replicator equation
describes the evolution of the system based on a population density fitness
function:

\begin{equation}\label{eqn:replicator_dynamics}
    \frac{dx}{dt} = x(S-x^TS x)
\end{equation}

Equation (\ref{eqn:replicator_dynamics}) is solved numerically through an
integration technique described in~\cite{Petzold1983} and
Figure~\ref{fig:replicator_dynamics} shows the evolution of the distribution of
the system: the various strategies are ranked by scores. It is clear to see that
only the high ranking strategies survive the evolutionary process (in fact,
only \input{./assets/img/replicator_dynamics/main.tex}
have a final distribution greater than \(10 ^ {-2}\)). This confirms the
findings of~\cite{Moran1707} in which sophisticated strategies resist
evolutionary invasion of shorter memory strategies. Recalling
Figure~\ref{fig:SSError_and_probabilities_in_full} this demonstrates that:

\begin{itemize}
    \item Cooperation emerges through the evolutionary process: the high scoring
        strategies do not exhibit extortionate behaviour towards each other.
    \item Extortionate strategies do not survive the evolutionary process.
\end{itemize}

\begin{figure}[!htbp]
    \centering
    \includegraphics[width=.8\textwidth]{./assets/img/replicator_dynamics/main.pdf}
    \caption{Numerical simulation of the replicator equation
    (\ref{eqn:replicator_dynamics}): strategies are ordered by score, only the strategies with a high score survive the evolutionary process.}
    \label{fig:replicator_dynamics}
\end{figure}

This work can be used to classify plays of the IPD\@: data can be collected from
actual interactions (in lab or in the field). Furthermore, this allows for a
classification method similar to the notion of fingerprinting presented
in~\cite{Ashlock2008}. Trained strategies can potentially be classified as
extortionate or not or it could be possible to even constrain the reinforcement
learning approaches that are becoming prevalent in the literature.
Alternatively, this mathematical approach for recognising extortion could be
used in sophisticated strategies to defend against invasion. Arguably, some of
the strategies considered here exhibit this behaviour, indeed as described
in~\cite{Harper2017}, the top ranking strategies in the full tournament are
obtained using evolutionary reinforcement learning techniques, thus, suspicion
of extortionate behaviour could in fact be an evolutionary trait.

\section*{Acknowledgements}

The following open source software libraries were used in this research:

\begin{itemize}
    \item The Axelrod ~\cite{Knight2016, Knight2018} library (IPD strategies and
        tournaments).
    \item The sympy library~\cite{Meurer2017} (verification of all symbolic
        calculations).
    \item The matplotlib~\cite{Droettboom2018} library (visualisation).
    \item The pandas~\cite{Structures2010}, dask~\cite{Dask2016} and
        NumPy~\cite{Oliphant2015} libraries (data manipulation).
    \item The SciPy~\cite{Jones2001} library (numerical integration of the
        replicator equation).
\end{itemize}

This work was performed using the computational facilities of the Advanced
Research Computing @ Cardiff (ARCCA) Division, Cardiff University.

\printbibliography

\newpage
\section*{Supplementary materials}

\includepdf{assets/pdf/proof_of_form_of_extortionate_strategies/main.pdf}

\newpage

Using the pair wise interactions the transition rates \(p,
q\) can be measured and the steady state probabilities inferred and compared to
the actual probabilities of each state.
This is done numerically by computing the singular eigenvector of the
matrix \(A\) \cite{Stewart2009}:

\[
    A =
    \begin{bmatrix}
        p_1 q_1 & p_1 (1 - q_1) & (1 - p_1) q_1 & (1 -p_1) (1 - q_1) \\
        p_2 q_2 & p_2 (1 - q_2) & (1 - p_2) q_2 & (1 -p_2) (1 - q_2) \\
        p_3 q_3 & p_3 (1 - q_3) & (1 - p_3) q_3 & (1 -p_3) (1 - q_3) \\
        p_4 q_4 & p_4 (1 - q_4) & (1 - p_4) q_4 & (1 -p_4) (1 - q_4) \\
    \end{bmatrix}
\]

Figure~\ref{fig:computed_probabilities_vs_theoretic_probabilities} shows a
regression line fitted to every pairwise interaction with a reported
\(\text{SSError}\) value (pairwise interactions with missing states were
omitted). This serves to validate the approach: a part from some edge cases the
relationship is consistent.

\begin{figure}[!htbp]
    \centering
    \includegraphics[width=.8\textwidth]{./assets/img/computed_probabilities_vs_theoretic_probabilities/main.pdf}
    \caption{The
        relationship between the steady state probabilities inferred from the
        measured transitions and the actual steady state probabilities. A linear
        regression line is included validating the approach.}
    \label{fig:computed_probabilities_vs_theoretic_probabilities}
\end{figure}


\end{document}

strategies.

The results of this analysis are shown in
Figure~\ref{fig:SSError_and_probabilities_in_full}. The top ranking strategies
by number of wins seem to be extortionate (but not against all strategies) and
it can be seen that a small sub group of strategies achieve mutual defection.
All the top ranking strategies according to score achieve mutual cooperation and
do not extort each other, however they
\textbf{do} exhibit extortionate behaviour towards a number of the lower ranking
strategies.

\begin{figure}[!htbp]
    \centering
    \includegraphics[width=.8\textwidth]{./assets/img/SSError_and_probabilities_in_full/main.pdf}
    \caption{\(\text{SSError}\) for the strategies for the full tournament. Only
    strategy interactions for which \(p_4=0\) and \(\chi>1\) are displayed.}
    \label{fig:SSError_and_probabilities_in_full}
\end{figure}

\section{Conclusion}\label{sec:conclusion}

This work defines an approach to measure whether or not a player is playing a
strategy that corresponds to an extortionate strategy as defined
in~\cite{Press2012}: a mathematical model for suspicion. Indeed, all
extortionate strategies have been
 classified as lying on a triangular plane.
This rigorous classification fails to be robust to small measurement error, thus
a statistical approach is proposed.
This is done through a linear algebraic approach for approximating the solution
of a linear system. Using this, a large number of pairwise interactions is
simulated and in fact very few strategies are found to act extortionately.

The work of~\cite{Press2012}, whilst showing that a clever approach to taking
advantage of another memory one strategy exists: this is incomplete. Whilst the
elegance of this result is very attractive, just as the simplicity of the
victory of Tit For Tat in Axelrod's original tournaments was, it is incomplete.
Extortionate strategies achieve a high number of wins but they do not
achieve a high score which corresponds to the fitness landscape in an
evolutionary sense. From the large number of interactions a payoff matrix \(S\)
can be measured where \(S_{ij}\) denotes the score (using standard values of
\((R, S, T, P) = (3, 0, 5, 1)\)) of the \(i\)th strategy
against the \(j\)th strategy. Using this, the replicator equation
describes the evolution of the system based on a population density fitness
function:

\begin{equation}\label{eqn:replicator_dynamics}
    \frac{dx}{dt} = x(S-x^TS x)
\end{equation}

Equation (\ref{eqn:replicator_dynamics}) is solved numerically through an
integration technique described in~\cite{Petzold1983} and
Figure~\ref{fig:replicator_dynamics} shows the evolution of the distribution of
the system: the various strategies are ranked by scores. It is clear to see that
only the high ranking strategies survive the evolutionary process (in fact,
only \documentclass[a4paper]{article}

\usepackage{amsmath}
\usepackage{amssymb}
\usepackage[margin=1.5cm,
            includefoot,
            footskip=30pt]{geometry}
\usepackage{layout}
\usepackage{graphicx}
\usepackage{subcaption}

\usepackage{biblatex}
\usepackage{pdfpages}

\bibliography{main.bib}

\title{Suspicion: Recognising and evaluating the effectiveness
       of extortion in the Iterated Prisoner's Dilemma}
\author{Vincent A. Knight \and Nikoleta E. Glynatsi}
\date{\today}



\begin{document}

\maketitle

\begin{abstract}
    The Iterated Prisoner's Dilemma is a model for rational and evolutionary
    interactive behaviour. It has applications both in the study of human social
    behaviour as well as in biology.
    It is used to understand when and how a rational individual might
    accept an immediate cost to their own utility for the direct benefit of
    another.

    Much attention has been given to a class of strategies called
    Zero Determinant strategies. It has been theoretically shown that these
    strategies can ``extort'' any player.

    In this work, an approach to identify if observed strategies are playing in
    an extortionate way is described. Furthermore, experimental analysis of
    a large tournament with \input{assets/tex/number_of_full_strategies/main.tex}
    strategies is considered. In this setting
    the most highly performing strategies do not play in an extortionate way
    against each other but do against lower performing strategies.
    This suggests that whilst the theory of Zero Determinant strategies
    indicates that memory is not of fundamental importance to the evolution of
    cooperative behaviour, this is incomplete.
\end{abstract}

\section{Introduction}\label{sec:introduction}

Agent based game theoretic models have become a stalwart of the underpinning
mathematics of interactive behaviours. One of the major pieces of work
in this area is the pair of original computer tournaments run by Robert
Axelrod~\cite{Axelrod1980, Axelrod1980a}. These tournaments pitted submitted
computer strategies against each other in plays of the Iterated Prisoner's
Dilemma. A common game where agents can choose to pay a slight cost to their
immediate utility in the hope of building a reputation. This has been used in
economic and evolutionary game theory to understand the evolution of cooperative
behaviour.

Recently, a class of strategies was described in~\cite{Press2012} that can
provably extort any given opponent. In~\cite{Hilbe2013, Moran1707} some
questions have already been asked about the true effectiveness of these
strategies in an evolutionary setting. Here another question is asked: is it
possible to recognise this extortionate behaviour? A mathematical procedure for
suspicion is presented: in the same way that the continued actions of an
extortionate individual might raise suspicion.

This work makes use of the Axelrod Python library~\cite{Knight2018, Knight2016}
with a large number of Prisoner Dilemma strategies available to give an
extensive numerical example of the ideas presented.  The approach is presented
in Section~\ref{sec:delta-zd-strategies}.  All of the code and data discussed
in Section~\ref{sec:numerical-experiments} is open sourced, archived and
written according to best scientific principles~\cite{Wilson2014}. The data
archive can be found at~\cite{vincent_knight_2018_1297075}.

\section{Recognising Extortion}\label{sec:delta-zd-strategies}

In~\cite{Press2012}, given a match between 2 memory-one strategies, the concept
of Zero Determinant (ZD) strategies is introduced. The main result of that paper
shows that given two memory one players \(p, q\in\mathbb{R}^4\) a linear
relationship between the players' scores could be forced by one of the players.

Using the notation of~\cite{Press2012}, assuming the utilities for player \(p\)
are given by \(S_x=(R, S, T, P)\) and for player \(q\) by \(S_y=(R, T, S, P)\)
and that the stationary scores of each player is given by \(S_X\) and \(S_Y\)
respectively. The main result of~\cite{Press2012} is that if

\begin{equation}\label{eqn:linear_relationship_for_p}
    \tilde p=\alpha S_x + \beta S_y + \gamma
\end{equation}

or

\begin{equation}\label{eqn:linear_relationship_for_q}
    \tilde q=\alpha S_x + \beta S_y + \gamma
\end{equation}

where \(\tilde p = (1 - p_1, 1 - p_2, p_3, p_4)\) and
\(\tilde q = (1 - q_1, 1 - q_2, q_3, q_4)\) then:

\begin{equation}
    \alpha S_X + \beta S_Y + \gamma = 0
\end{equation}

In~\cite{Press2012} a particular type of ZD strategy is defined: extortionate
strategies. If:

\begin{equation}\label{eqn:constraint_for_extortion}
    \gamma = - P(\alpha + \beta)
\end{equation}

then the player can ensure they get a score \(\chi\) times
larger than the opponent. This extortion coefficient is given by:

\begin{equation}\label{eqn:definition_of_chi}
    \chi=\frac{-\beta}{\alpha}
\end{equation}

Thus, if (\ref{eqn:constraint_for_extortion}) holds and \(\chi >1\) a player is
said to extort their opponent.
Here, the reverse problem is considered: given a
\(p\in\mathbb{R}^4\) how does one identify \(\alpha, \beta\) if they
exist and is the strategy in fact acting in an extortionate way?

These conditions correspond to:

\begin{align}
    \tilde p_1 & = \alpha R + \beta R - P (\alpha + \beta)
            \label{eqn:condition_for_tilde_p1}\\
    \tilde p_2 & = \alpha S + \beta T - P (\alpha + \beta)
            \label{eqn:condition_for_tilde_p2}\\
    \tilde p_3 & = \alpha T + \beta S - P (\alpha + \beta)
            \label{eqn:condition_for_tilde_p3}\\
    \tilde p_4 & = \alpha P + \beta P - P (\alpha + \beta)
            \label{eqn:condition_for_tilde_p4}
\end{align}

Equation (\ref{eqn:condition_for_tilde_p4}) ensures that \(p_4=\tilde p_4=0\).
Equations (\ref{eqn:condition_for_tilde_p1}-\ref{eqn:condition_for_tilde_p3})
can be used to eliminate \(\alpha, \beta\), giving:

\begin{equation}\label{eqn:planar_definition_of_extortion}
    \tilde p_1 = \frac{(R - P)(\tilde p_2 + \tilde p_3)}{S + T - 2P}
\end{equation}

with:

\begin{equation}\label{eqn:definition_of_chi}
    \chi = \frac{\tilde p_2 (P - T) + \tilde p_3 (S - P)}
                {\tilde p_2 (P - S) + \tilde p_3 (T - P)}
\end{equation}

Given a strategy \(p\in\mathbb{R}^{4\times 1}\) equations
(\ref{eqn:condition_for_tilde_p4}), (\ref{eqn:planar_definition_of_extortion}-\ref{eqn:definition_of_chi}) can be used to check if
a strategy is extortionate. The conditions correspond to:

\begin{align}
    p_1 & = \frac{(R-P)(p_2 + p_3) - R + T + S - P}{S + T - 2P}
     \label{eqn:condition_for_p1}\\
    p_4 & = 0 \label{eqn:condition_for_p4}\\
    1 & > p_2 + p_3\label{eqn:condition_for_chi}
\end{align}

The algebraic steps necessary to prove these results are available in the
supporting materials.

All extortionate strategies reside on a triangular (\ref{eqn:condition_for_chi})
plane (\ref{eqn:condition_for_p1}) in 3 dimensions (\ref{eqn:condition_for_p4}).
Using this formulation it can be seen that a necessary (but not sufficient)
condition for an extortionate strategy is that it cooperates on average less
than 50\% of the time when in a state of disagreement with the opponent.

As an example, consider the known extortionate strategy \(p=(8 / 9, 1 / 2, 1 /
3, 0)\) from~\cite{Stewart2012} which is referred to as \texttt{Extort-2}. In
this case, for the standard values of \((R, T, S, P)\) constraint
(\ref{eqn:condition_for_p1}) corresponds to:

\begin{equation}
    p_1 = \frac{2(p_2 + p_3) + 1}{3}
\end{equation}

It is clear that in this case all constraints hold.

This approach could in fact be used to confirm that a given strategy is acting
in an extortionate manner even if it is not a memory one strategy. However, in
practice, if a closed form for \(p\) is not known, then due to measurement
and/or numerical error this would not work.

This problem can be written in the following linear algebraic form where
\(x=(\alpha, \beta)\)
and \(p^*=(\tilde p_1 - 1, tilde_2 - 1, p_3)\):

\begin{equation}\label{eqn:linear_algebraic_equation_for_p}
    Cx= p^*
\end{equation}

\(C\) corresponds to equations
(\ref{eqn:condition_for_tilde_p1}-\ref{eqn:condition_for_tilde_p3}) and is
given by:

\begin{equation}\label{eqn:definition_of_C}
    C =
    \begin{bmatrix}
        R - P & R- P \\
        S - P & T- P \\
        T - P & S- P \\
    \end{bmatrix}
\end{equation}

Note that in general, equation (\ref{eqn:linear_algebraic_equation_for_p}) will
not necessarily have a solution. From the Rouch\'{e}-Capelli theorem if there is
a solution it is unique as \(\text{rank}(C)=2\) which is the dimension of the
variable \(x\). The best fitting \(x\) is found by minimizing:

\begin{equation}\label{eqn:r_squared}
    \text{SSError} = \|C x- p^*\|_2^2 = \sum_{i=1}^{3}\left((C\bar x)_i-p_i^*\right)^2
\end{equation}

Note that \(\text{SSError}\), which is the square of the Frobenius
norm~\cite{Golub2013}, becomes a measure of how close a strategy is to being an
extortionate strategy. Suspicion
of extortion then corresponds to a threshold on \(\text{SSError}\).

By observing interactions (human or otherwise), their memory one representation
can be inferred and this approach can be used to recognise extortionate
behaviour. The notion of comparing theoretic and actual plays of the IPD is not
novel, see for example~\cite{Rand2013}. Immediately it is noted that if the
environment is noisy~\cite{Wu1995} then no strategy can be considered to be
extortionate as \(p_4>0\).

In the next section, this idea will be illustrated by observing the interactions
that take place in a computer based tournament of the IPD\@.

\section{Numerical experiments}\label{sec:numerical-experiments}

In~\cite{Stewart2012} results from a tournament with
\input{./assets/tex/number_of_stewart_plotkin_strategies/main.tex} strategies,
was presented with specific consideration given to ZD strategies. This
tournament is reproduced here using the Axelrod-Python
project~\cite{Knight2016}. To obtain a good measure of the corresponding
transition rates for each strategy all matches have been run for
\input{assets/tex/number_of_turns/main.tex} turns and every match has been
repeated \input{assets/tex/number_of_repetitions/main.tex} times. All of this
interaction data is available at~\cite{vincent_knight_2018_1297075}. A good
match between the inferred Markov chain and the state distribution of the actual
interactions has been verified. Data for this is presented in the supplementary
materials.

Figure~\ref{fig:SSError_overall_in_stewart_plotkin} shows the \(\text{SSError}\)
values for all the strategies in the tournament, as reported
in~\cite{Stewart2012} the extortionate strategy (which has an expected
\(\text{SSError}\) approximately 0) gains a large number of wins.

\begin{figure}[!htbp]
    \centering
    \includegraphics[width=.8\textwidth]{./assets/img/SSError_overall_in_stewart_plotkin/main.pdf}
    \caption{\(\text{SSError}\) and state probabilities for the strategies
        of~\cite{Stewart2012}, ordered both by number of wins and overall score.
        Note that \(P(DC)\) is not shown as it corresponds to the transpose of
        \(P(CD)\). Cooperator and Defector are omitted as they do not visit all
        the states.}
    \label{fig:SSError_overall_in_stewart_plotkin}
\end{figure}

Here, the work of~\cite{Stewart2012} is extended by investigating a tournament
with \input{assets/tex/number_of_full_strategies/main.tex}
strategies.

The results of this analysis are shown in
Figure~\ref{fig:SSError_and_probabilities_in_full}. The top ranking strategies
by number of wins seem to be extortionate (but not against all strategies) and
it can be seen that a small sub group of strategies achieve mutual defection.
All the top ranking strategies according to score achieve mutual cooperation and
do not extort each other, however they
\textbf{do} exhibit extortionate behaviour towards a number of the lower ranking
strategies.

\begin{figure}[!htbp]
    \centering
    \includegraphics[width=.8\textwidth]{./assets/img/SSError_and_probabilities_in_full/main.pdf}
    \caption{\(\text{SSError}\) for the strategies for the full tournament. Only
    strategy interactions for which \(p_4=0\) and \(\chi>1\) are displayed.}
    \label{fig:SSError_and_probabilities_in_full}
\end{figure}

\section{Conclusion}\label{sec:conclusion}

This work defines an approach to measure whether or not a player is playing a
strategy that corresponds to an extortionate strategy as defined
in~\cite{Press2012}: a mathematical model for suspicion. Indeed, all
extortionate strategies have been
 classified as lying on a triangular plane.
This rigorous classification fails to be robust to small measurement error, thus
a statistical approach is proposed.
This is done through a linear algebraic approach for approximating the solution
of a linear system. Using this, a large number of pairwise interactions is
simulated and in fact very few strategies are found to act extortionately.

The work of~\cite{Press2012}, whilst showing that a clever approach to taking
advantage of another memory one strategy exists: this is incomplete. Whilst the
elegance of this result is very attractive, just as the simplicity of the
victory of Tit For Tat in Axelrod's original tournaments was, it is incomplete.
Extortionate strategies achieve a high number of wins but they do not
achieve a high score which corresponds to the fitness landscape in an
evolutionary sense. From the large number of interactions a payoff matrix \(S\)
can be measured where \(S_{ij}\) denotes the score (using standard values of
\((R, S, T, P) = (3, 0, 5, 1)\)) of the \(i\)th strategy
against the \(j\)th strategy. Using this, the replicator equation
describes the evolution of the system based on a population density fitness
function:

\begin{equation}\label{eqn:replicator_dynamics}
    \frac{dx}{dt} = x(S-x^TS x)
\end{equation}

Equation (\ref{eqn:replicator_dynamics}) is solved numerically through an
integration technique described in~\cite{Petzold1983} and
Figure~\ref{fig:replicator_dynamics} shows the evolution of the distribution of
the system: the various strategies are ranked by scores. It is clear to see that
only the high ranking strategies survive the evolutionary process (in fact,
only \input{./assets/img/replicator_dynamics/main.tex}
have a final distribution greater than \(10 ^ {-2}\)). This confirms the
findings of~\cite{Moran1707} in which sophisticated strategies resist
evolutionary invasion of shorter memory strategies. Recalling
Figure~\ref{fig:SSError_and_probabilities_in_full} this demonstrates that:

\begin{itemize}
    \item Cooperation emerges through the evolutionary process: the high scoring
        strategies do not exhibit extortionate behaviour towards each other.
    \item Extortionate strategies do not survive the evolutionary process.
\end{itemize}

\begin{figure}[!htbp]
    \centering
    \includegraphics[width=.8\textwidth]{./assets/img/replicator_dynamics/main.pdf}
    \caption{Numerical simulation of the replicator equation
    (\ref{eqn:replicator_dynamics}): strategies are ordered by score, only the strategies with a high score survive the evolutionary process.}
    \label{fig:replicator_dynamics}
\end{figure}

This work can be used to classify plays of the IPD\@: data can be collected from
actual interactions (in lab or in the field). Furthermore, this allows for a
classification method similar to the notion of fingerprinting presented
in~\cite{Ashlock2008}. Trained strategies can potentially be classified as
extortionate or not or it could be possible to even constrain the reinforcement
learning approaches that are becoming prevalent in the literature.
Alternatively, this mathematical approach for recognising extortion could be
used in sophisticated strategies to defend against invasion. Arguably, some of
the strategies considered here exhibit this behaviour, indeed as described
in~\cite{Harper2017}, the top ranking strategies in the full tournament are
obtained using evolutionary reinforcement learning techniques, thus, suspicion
of extortionate behaviour could in fact be an evolutionary trait.

\section*{Acknowledgements}

The following open source software libraries were used in this research:

\begin{itemize}
    \item The Axelrod ~\cite{Knight2016, Knight2018} library (IPD strategies and
        tournaments).
    \item The sympy library~\cite{Meurer2017} (verification of all symbolic
        calculations).
    \item The matplotlib~\cite{Droettboom2018} library (visualisation).
    \item The pandas~\cite{Structures2010}, dask~\cite{Dask2016} and
        NumPy~\cite{Oliphant2015} libraries (data manipulation).
    \item The SciPy~\cite{Jones2001} library (numerical integration of the
        replicator equation).
\end{itemize}

This work was performed using the computational facilities of the Advanced
Research Computing @ Cardiff (ARCCA) Division, Cardiff University.

\printbibliography

\newpage
\section*{Supplementary materials}

\includepdf{assets/pdf/proof_of_form_of_extortionate_strategies/main.pdf}

\newpage

Using the pair wise interactions the transition rates \(p,
q\) can be measured and the steady state probabilities inferred and compared to
the actual probabilities of each state.
This is done numerically by computing the singular eigenvector of the
matrix \(A\) \cite{Stewart2009}:

\[
    A =
    \begin{bmatrix}
        p_1 q_1 & p_1 (1 - q_1) & (1 - p_1) q_1 & (1 -p_1) (1 - q_1) \\
        p_2 q_2 & p_2 (1 - q_2) & (1 - p_2) q_2 & (1 -p_2) (1 - q_2) \\
        p_3 q_3 & p_3 (1 - q_3) & (1 - p_3) q_3 & (1 -p_3) (1 - q_3) \\
        p_4 q_4 & p_4 (1 - q_4) & (1 - p_4) q_4 & (1 -p_4) (1 - q_4) \\
    \end{bmatrix}
\]

Figure~\ref{fig:computed_probabilities_vs_theoretic_probabilities} shows a
regression line fitted to every pairwise interaction with a reported
\(\text{SSError}\) value (pairwise interactions with missing states were
omitted). This serves to validate the approach: a part from some edge cases the
relationship is consistent.

\begin{figure}[!htbp]
    \centering
    \includegraphics[width=.8\textwidth]{./assets/img/computed_probabilities_vs_theoretic_probabilities/main.pdf}
    \caption{The
        relationship between the steady state probabilities inferred from the
        measured transitions and the actual steady state probabilities. A linear
        regression line is included validating the approach.}
    \label{fig:computed_probabilities_vs_theoretic_probabilities}
\end{figure}


\end{document}

have a final distribution greater than \(10 ^ {-2}\)). This confirms the
findings of~\cite{Moran1707} in which sophisticated strategies resist
evolutionary invasion of shorter memory strategies. Recalling
Figure~\ref{fig:SSError_and_probabilities_in_full} this demonstrates that:

\begin{itemize}
    \item Cooperation emerges through the evolutionary process: the high scoring
        strategies do not exhibit extortionate behaviour towards each other.
    \item Extortionate strategies do not survive the evolutionary process.
\end{itemize}

\begin{figure}[!htbp]
    \centering
    \includegraphics[width=.8\textwidth]{./assets/img/replicator_dynamics/main.pdf}
    \caption{Numerical simulation of the replicator equation
    (\ref{eqn:replicator_dynamics}): strategies are ordered by score, only the strategies with a high score survive the evolutionary process.}
    \label{fig:replicator_dynamics}
\end{figure}

This work can be used to classify plays of the IPD\@: data can be collected from
actual interactions (in lab or in the field). Furthermore, this allows for a
classification method similar to the notion of fingerprinting presented
in~\cite{Ashlock2008}. Trained strategies can potentially be classified as
extortionate or not or it could be possible to even constrain the reinforcement
learning approaches that are becoming prevalent in the literature.
Alternatively, this mathematical approach for recognising extortion could be
used in sophisticated strategies to defend against invasion. Arguably, some of
the strategies considered here exhibit this behaviour, indeed as described
in~\cite{Harper2017}, the top ranking strategies in the full tournament are
obtained using evolutionary reinforcement learning techniques, thus, suspicion
of extortionate behaviour could in fact be an evolutionary trait.

\section*{Acknowledgements}

The following open source software libraries were used in this research:

\begin{itemize}
    \item The Axelrod ~\cite{Knight2016, Knight2018} library (IPD strategies and
        tournaments).
    \item The sympy library~\cite{Meurer2017} (verification of all symbolic
        calculations).
    \item The matplotlib~\cite{Droettboom2018} library (visualisation).
    \item The pandas~\cite{Structures2010}, dask~\cite{Dask2016} and
        NumPy~\cite{Oliphant2015} libraries (data manipulation).
    \item The SciPy~\cite{Jones2001} library (numerical integration of the
        replicator equation).
\end{itemize}

This work was performed using the computational facilities of the Advanced
Research Computing @ Cardiff (ARCCA) Division, Cardiff University.

\printbibliography

\newpage
\section*{Supplementary materials}

\includepdf{assets/pdf/proof_of_form_of_extortionate_strategies/main.pdf}

\newpage

Using the pair wise interactions the transition rates \(p,
q\) can be measured and the steady state probabilities inferred and compared to
the actual probabilities of each state.
This is done numerically by computing the singular eigenvector of the
matrix \(A\) \cite{Stewart2009}:

\[
    A =
    \begin{bmatrix}
        p_1 q_1 & p_1 (1 - q_1) & (1 - p_1) q_1 & (1 -p_1) (1 - q_1) \\
        p_2 q_2 & p_2 (1 - q_2) & (1 - p_2) q_2 & (1 -p_2) (1 - q_2) \\
        p_3 q_3 & p_3 (1 - q_3) & (1 - p_3) q_3 & (1 -p_3) (1 - q_3) \\
        p_4 q_4 & p_4 (1 - q_4) & (1 - p_4) q_4 & (1 -p_4) (1 - q_4) \\
    \end{bmatrix}
\]

Figure~\ref{fig:computed_probabilities_vs_theoretic_probabilities} shows a
regression line fitted to every pairwise interaction with a reported
\(\text{SSError}\) value (pairwise interactions with missing states were
omitted). This serves to validate the approach: a part from some edge cases the
relationship is consistent.

\begin{figure}[!htbp]
    \centering
    \includegraphics[width=.8\textwidth]{./assets/img/computed_probabilities_vs_theoretic_probabilities/main.pdf}
    \caption{The
        relationship between the steady state probabilities inferred from the
        measured transitions and the actual steady state probabilities. A linear
        regression line is included validating the approach.}
    \label{fig:computed_probabilities_vs_theoretic_probabilities}
\end{figure}


\end{document}

    strategies is considered. In this setting
    the most highly performing strategies do not play in an extortionate way
    against each other but do against lower performing strategies.
    This suggests that whilst the theory of Zero Determinant strategies
    indicates that memory is not of fundamental importance to the evolution of
    cooperative behaviour, this is incomplete.
\end{abstract}

\section{Introduction}\label{sec:introduction}

Agent based game theoretic models have become a stalwart of the underpinning
mathematics of interactive behaviours. One of the major pieces of work
in this area is the pair of original computer tournaments run by Robert
Axelrod~\cite{Axelrod1980, Axelrod1980a}. These tournaments pitted submitted
computer strategies against each other in plays of the Iterated Prisoner's
Dilemma. A common game where agents can choose to pay a slight cost to their
immediate utility in the hope of building a reputation. This has been used in
economic and evolutionary game theory to understand the evolution of cooperative
behaviour.

Recently, a class of strategies was described in~\cite{Press2012} that can
provably extort any given opponent. In~\cite{Hilbe2013, Moran1707} some
questions have already been asked about the true effectiveness of these
strategies in an evolutionary setting. Here another question is asked: is it
possible to recognise this extortionate behaviour? A mathematical procedure for
suspicion is presented: in the same way that the continued actions of an
extortionate individual might raise suspicion.

This work makes use of the Axelrod Python library~\cite{Knight2018, Knight2016}
with a large number of Prisoner Dilemma strategies available to give an
extensive numerical example of the ideas presented.  The approach is presented
in Section~\ref{sec:delta-zd-strategies}.  All of the code and data discussed
in Section~\ref{sec:numerical-experiments} is open sourced, archived and
written according to best scientific principles~\cite{Wilson2014}. The data
archive can be found at~\cite{vincent_knight_2018_1297075}.

\section{Recognising Extortion}\label{sec:delta-zd-strategies}

In~\cite{Press2012}, given a match between 2 memory-one strategies, the concept
of Zero Determinant (ZD) strategies is introduced. The main result of that paper
shows that given two memory one players \(p, q\in\mathbb{R}^4\) a linear
relationship between the players' scores could be forced by one of the players.

Using the notation of~\cite{Press2012}, assuming the utilities for player \(p\)
are given by \(S_x=(R, S, T, P)\) and for player \(q\) by \(S_y=(R, T, S, P)\)
and that the stationary scores of each player is given by \(S_X\) and \(S_Y\)
respectively. The main result of~\cite{Press2012} is that if

\begin{equation}\label{eqn:linear_relationship_for_p}
    \tilde p=\alpha S_x + \beta S_y + \gamma
\end{equation}

or

\begin{equation}\label{eqn:linear_relationship_for_q}
    \tilde q=\alpha S_x + \beta S_y + \gamma
\end{equation}

where \(\tilde p = (1 - p_1, 1 - p_2, p_3, p_4)\) and
\(\tilde q = (1 - q_1, 1 - q_2, q_3, q_4)\) then:

\begin{equation}
    \alpha S_X + \beta S_Y + \gamma = 0
\end{equation}

In~\cite{Press2012} a particular type of ZD strategy is defined: extortionate
strategies. If:

\begin{equation}\label{eqn:constraint_for_extortion}
    \gamma = - P(\alpha + \beta)
\end{equation}

then the player can ensure they get a score \(\chi\) times
larger than the opponent. This extortion coefficient is given by:

\begin{equation}\label{eqn:definition_of_chi}
    \chi=\frac{-\beta}{\alpha}
\end{equation}

Thus, if (\ref{eqn:constraint_for_extortion}) holds and \(\chi >1\) a player is
said to extort their opponent.
Here, the reverse problem is considered: given a
\(p\in\mathbb{R}^4\) how does one identify \(\alpha, \beta\) if they
exist and is the strategy in fact acting in an extortionate way?

These conditions correspond to:

\begin{align}
    \tilde p_1 & = \alpha R + \beta R - P (\alpha + \beta)
            \label{eqn:condition_for_tilde_p1}\\
    \tilde p_2 & = \alpha S + \beta T - P (\alpha + \beta)
            \label{eqn:condition_for_tilde_p2}\\
    \tilde p_3 & = \alpha T + \beta S - P (\alpha + \beta)
            \label{eqn:condition_for_tilde_p3}\\
    \tilde p_4 & = \alpha P + \beta P - P (\alpha + \beta)
            \label{eqn:condition_for_tilde_p4}
\end{align}

Equation (\ref{eqn:condition_for_tilde_p4}) ensures that \(p_4=\tilde p_4=0\).
Equations (\ref{eqn:condition_for_tilde_p1}-\ref{eqn:condition_for_tilde_p3})
can be used to eliminate \(\alpha, \beta\), giving:

\begin{equation}\label{eqn:planar_definition_of_extortion}
    \tilde p_1 = \frac{(R - P)(\tilde p_2 + \tilde p_3)}{S + T - 2P}
\end{equation}

with:

\begin{equation}\label{eqn:definition_of_chi}
    \chi = \frac{\tilde p_2 (P - T) + \tilde p_3 (S - P)}
                {\tilde p_2 (P - S) + \tilde p_3 (T - P)}
\end{equation}

Given a strategy \(p\in\mathbb{R}^{4\times 1}\) equations
(\ref{eqn:condition_for_tilde_p4}), (\ref{eqn:planar_definition_of_extortion}-\ref{eqn:definition_of_chi}) can be used to check if
a strategy is extortionate. The conditions correspond to:

\begin{align}
    p_1 & = \frac{(R-P)(p_2 + p_3) - R + T + S - P}{S + T - 2P}
     \label{eqn:condition_for_p1}\\
    p_4 & = 0 \label{eqn:condition_for_p4}\\
    1 & > p_2 + p_3\label{eqn:condition_for_chi}
\end{align}

The algebraic steps necessary to prove these results are available in the
supporting materials.

All extortionate strategies reside on a triangular (\ref{eqn:condition_for_chi})
plane (\ref{eqn:condition_for_p1}) in 3 dimensions (\ref{eqn:condition_for_p4}).
Using this formulation it can be seen that a necessary (but not sufficient)
condition for an extortionate strategy is that it cooperates on average less
than 50\% of the time when in a state of disagreement with the opponent.

As an example, consider the known extortionate strategy \(p=(8 / 9, 1 / 2, 1 /
3, 0)\) from~\cite{Stewart2012} which is referred to as \texttt{Extort-2}. In
this case, for the standard values of \((R, T, S, P)\) constraint
(\ref{eqn:condition_for_p1}) corresponds to:

\begin{equation}
    p_1 = \frac{2(p_2 + p_3) + 1}{3}
\end{equation}

It is clear that in this case all constraints hold.

This approach could in fact be used to confirm that a given strategy is acting
in an extortionate manner even if it is not a memory one strategy. However, in
practice, if a closed form for \(p\) is not known, then due to measurement
and/or numerical error this would not work.

This problem can be written in the following linear algebraic form where
\(x=(\alpha, \beta)\)
and \(p^*=(\tilde p_1 - 1, tilde_2 - 1, p_3)\):

\begin{equation}\label{eqn:linear_algebraic_equation_for_p}
    Cx= p^*
\end{equation}

\(C\) corresponds to equations
(\ref{eqn:condition_for_tilde_p1}-\ref{eqn:condition_for_tilde_p3}) and is
given by:

\begin{equation}\label{eqn:definition_of_C}
    C =
    \begin{bmatrix}
        R - P & R- P \\
        S - P & T- P \\
        T - P & S- P \\
    \end{bmatrix}
\end{equation}

Note that in general, equation (\ref{eqn:linear_algebraic_equation_for_p}) will
not necessarily have a solution. From the Rouch\'{e}-Capelli theorem if there is
a solution it is unique as \(\text{rank}(C)=2\) which is the dimension of the
variable \(x\). The best fitting \(x\) is found by minimizing:

\begin{equation}\label{eqn:r_squared}
    \text{SSError} = \|C x- p^*\|_2^2 = \sum_{i=1}^{3}\left((C\bar x)_i-p_i^*\right)^2
\end{equation}

Note that \(\text{SSError}\), which is the square of the Frobenius
norm~\cite{Golub2013}, becomes a measure of how close a strategy is to being an
extortionate strategy. Suspicion
of extortion then corresponds to a threshold on \(\text{SSError}\).

By observing interactions (human or otherwise), their memory one representation
can be inferred and this approach can be used to recognise extortionate
behaviour. The notion of comparing theoretic and actual plays of the IPD is not
novel, see for example~\cite{Rand2013}. Immediately it is noted that if the
environment is noisy~\cite{Wu1995} then no strategy can be considered to be
extortionate as \(p_4>0\).

In the next section, this idea will be illustrated by observing the interactions
that take place in a computer based tournament of the IPD\@.

\section{Numerical experiments}\label{sec:numerical-experiments}

In~\cite{Stewart2012} results from a tournament with
\documentclass[a4paper]{article}

\usepackage{amsmath}
\usepackage{amssymb}
\usepackage[margin=1.5cm,
            includefoot,
            footskip=30pt]{geometry}
\usepackage{layout}
\usepackage{graphicx}
\usepackage{subcaption}

\usepackage{biblatex}
\usepackage{pdfpages}

\bibliography{main.bib}

\title{Suspicion: Recognising and evaluating the effectiveness
       of extortion in the Iterated Prisoner's Dilemma}
\author{Vincent A. Knight \and Nikoleta E. Glynatsi}
\date{\today}



\begin{document}

\maketitle

\begin{abstract}
    The Iterated Prisoner's Dilemma is a model for rational and evolutionary
    interactive behaviour. It has applications both in the study of human social
    behaviour as well as in biology.
    It is used to understand when and how a rational individual might
    accept an immediate cost to their own utility for the direct benefit of
    another.

    Much attention has been given to a class of strategies called
    Zero Determinant strategies. It has been theoretically shown that these
    strategies can ``extort'' any player.

    In this work, an approach to identify if observed strategies are playing in
    an extortionate way is described. Furthermore, experimental analysis of
    a large tournament with \documentclass[a4paper]{article}

\usepackage{amsmath}
\usepackage{amssymb}
\usepackage[margin=1.5cm,
            includefoot,
            footskip=30pt]{geometry}
\usepackage{layout}
\usepackage{graphicx}
\usepackage{subcaption}

\usepackage{biblatex}
\usepackage{pdfpages}

\bibliography{main.bib}

\title{Suspicion: Recognising and evaluating the effectiveness
       of extortion in the Iterated Prisoner's Dilemma}
\author{Vincent A. Knight \and Nikoleta E. Glynatsi}
\date{\today}



\begin{document}

\maketitle

\begin{abstract}
    The Iterated Prisoner's Dilemma is a model for rational and evolutionary
    interactive behaviour. It has applications both in the study of human social
    behaviour as well as in biology.
    It is used to understand when and how a rational individual might
    accept an immediate cost to their own utility for the direct benefit of
    another.

    Much attention has been given to a class of strategies called
    Zero Determinant strategies. It has been theoretically shown that these
    strategies can ``extort'' any player.

    In this work, an approach to identify if observed strategies are playing in
    an extortionate way is described. Furthermore, experimental analysis of
    a large tournament with \input{assets/tex/number_of_full_strategies/main.tex}
    strategies is considered. In this setting
    the most highly performing strategies do not play in an extortionate way
    against each other but do against lower performing strategies.
    This suggests that whilst the theory of Zero Determinant strategies
    indicates that memory is not of fundamental importance to the evolution of
    cooperative behaviour, this is incomplete.
\end{abstract}

\section{Introduction}\label{sec:introduction}

Agent based game theoretic models have become a stalwart of the underpinning
mathematics of interactive behaviours. One of the major pieces of work
in this area is the pair of original computer tournaments run by Robert
Axelrod~\cite{Axelrod1980, Axelrod1980a}. These tournaments pitted submitted
computer strategies against each other in plays of the Iterated Prisoner's
Dilemma. A common game where agents can choose to pay a slight cost to their
immediate utility in the hope of building a reputation. This has been used in
economic and evolutionary game theory to understand the evolution of cooperative
behaviour.

Recently, a class of strategies was described in~\cite{Press2012} that can
provably extort any given opponent. In~\cite{Hilbe2013, Moran1707} some
questions have already been asked about the true effectiveness of these
strategies in an evolutionary setting. Here another question is asked: is it
possible to recognise this extortionate behaviour? A mathematical procedure for
suspicion is presented: in the same way that the continued actions of an
extortionate individual might raise suspicion.

This work makes use of the Axelrod Python library~\cite{Knight2018, Knight2016}
with a large number of Prisoner Dilemma strategies available to give an
extensive numerical example of the ideas presented.  The approach is presented
in Section~\ref{sec:delta-zd-strategies}.  All of the code and data discussed
in Section~\ref{sec:numerical-experiments} is open sourced, archived and
written according to best scientific principles~\cite{Wilson2014}. The data
archive can be found at~\cite{vincent_knight_2018_1297075}.

\section{Recognising Extortion}\label{sec:delta-zd-strategies}

In~\cite{Press2012}, given a match between 2 memory-one strategies, the concept
of Zero Determinant (ZD) strategies is introduced. The main result of that paper
shows that given two memory one players \(p, q\in\mathbb{R}^4\) a linear
relationship between the players' scores could be forced by one of the players.

Using the notation of~\cite{Press2012}, assuming the utilities for player \(p\)
are given by \(S_x=(R, S, T, P)\) and for player \(q\) by \(S_y=(R, T, S, P)\)
and that the stationary scores of each player is given by \(S_X\) and \(S_Y\)
respectively. The main result of~\cite{Press2012} is that if

\begin{equation}\label{eqn:linear_relationship_for_p}
    \tilde p=\alpha S_x + \beta S_y + \gamma
\end{equation}

or

\begin{equation}\label{eqn:linear_relationship_for_q}
    \tilde q=\alpha S_x + \beta S_y + \gamma
\end{equation}

where \(\tilde p = (1 - p_1, 1 - p_2, p_3, p_4)\) and
\(\tilde q = (1 - q_1, 1 - q_2, q_3, q_4)\) then:

\begin{equation}
    \alpha S_X + \beta S_Y + \gamma = 0
\end{equation}

In~\cite{Press2012} a particular type of ZD strategy is defined: extortionate
strategies. If:

\begin{equation}\label{eqn:constraint_for_extortion}
    \gamma = - P(\alpha + \beta)
\end{equation}

then the player can ensure they get a score \(\chi\) times
larger than the opponent. This extortion coefficient is given by:

\begin{equation}\label{eqn:definition_of_chi}
    \chi=\frac{-\beta}{\alpha}
\end{equation}

Thus, if (\ref{eqn:constraint_for_extortion}) holds and \(\chi >1\) a player is
said to extort their opponent.
Here, the reverse problem is considered: given a
\(p\in\mathbb{R}^4\) how does one identify \(\alpha, \beta\) if they
exist and is the strategy in fact acting in an extortionate way?

These conditions correspond to:

\begin{align}
    \tilde p_1 & = \alpha R + \beta R - P (\alpha + \beta)
            \label{eqn:condition_for_tilde_p1}\\
    \tilde p_2 & = \alpha S + \beta T - P (\alpha + \beta)
            \label{eqn:condition_for_tilde_p2}\\
    \tilde p_3 & = \alpha T + \beta S - P (\alpha + \beta)
            \label{eqn:condition_for_tilde_p3}\\
    \tilde p_4 & = \alpha P + \beta P - P (\alpha + \beta)
            \label{eqn:condition_for_tilde_p4}
\end{align}

Equation (\ref{eqn:condition_for_tilde_p4}) ensures that \(p_4=\tilde p_4=0\).
Equations (\ref{eqn:condition_for_tilde_p1}-\ref{eqn:condition_for_tilde_p3})
can be used to eliminate \(\alpha, \beta\), giving:

\begin{equation}\label{eqn:planar_definition_of_extortion}
    \tilde p_1 = \frac{(R - P)(\tilde p_2 + \tilde p_3)}{S + T - 2P}
\end{equation}

with:

\begin{equation}\label{eqn:definition_of_chi}
    \chi = \frac{\tilde p_2 (P - T) + \tilde p_3 (S - P)}
                {\tilde p_2 (P - S) + \tilde p_3 (T - P)}
\end{equation}

Given a strategy \(p\in\mathbb{R}^{4\times 1}\) equations
(\ref{eqn:condition_for_tilde_p4}), (\ref{eqn:planar_definition_of_extortion}-\ref{eqn:definition_of_chi}) can be used to check if
a strategy is extortionate. The conditions correspond to:

\begin{align}
    p_1 & = \frac{(R-P)(p_2 + p_3) - R + T + S - P}{S + T - 2P}
     \label{eqn:condition_for_p1}\\
    p_4 & = 0 \label{eqn:condition_for_p4}\\
    1 & > p_2 + p_3\label{eqn:condition_for_chi}
\end{align}

The algebraic steps necessary to prove these results are available in the
supporting materials.

All extortionate strategies reside on a triangular (\ref{eqn:condition_for_chi})
plane (\ref{eqn:condition_for_p1}) in 3 dimensions (\ref{eqn:condition_for_p4}).
Using this formulation it can be seen that a necessary (but not sufficient)
condition for an extortionate strategy is that it cooperates on average less
than 50\% of the time when in a state of disagreement with the opponent.

As an example, consider the known extortionate strategy \(p=(8 / 9, 1 / 2, 1 /
3, 0)\) from~\cite{Stewart2012} which is referred to as \texttt{Extort-2}. In
this case, for the standard values of \((R, T, S, P)\) constraint
(\ref{eqn:condition_for_p1}) corresponds to:

\begin{equation}
    p_1 = \frac{2(p_2 + p_3) + 1}{3}
\end{equation}

It is clear that in this case all constraints hold.

This approach could in fact be used to confirm that a given strategy is acting
in an extortionate manner even if it is not a memory one strategy. However, in
practice, if a closed form for \(p\) is not known, then due to measurement
and/or numerical error this would not work.

This problem can be written in the following linear algebraic form where
\(x=(\alpha, \beta)\)
and \(p^*=(\tilde p_1 - 1, tilde_2 - 1, p_3)\):

\begin{equation}\label{eqn:linear_algebraic_equation_for_p}
    Cx= p^*
\end{equation}

\(C\) corresponds to equations
(\ref{eqn:condition_for_tilde_p1}-\ref{eqn:condition_for_tilde_p3}) and is
given by:

\begin{equation}\label{eqn:definition_of_C}
    C =
    \begin{bmatrix}
        R - P & R- P \\
        S - P & T- P \\
        T - P & S- P \\
    \end{bmatrix}
\end{equation}

Note that in general, equation (\ref{eqn:linear_algebraic_equation_for_p}) will
not necessarily have a solution. From the Rouch\'{e}-Capelli theorem if there is
a solution it is unique as \(\text{rank}(C)=2\) which is the dimension of the
variable \(x\). The best fitting \(x\) is found by minimizing:

\begin{equation}\label{eqn:r_squared}
    \text{SSError} = \|C x- p^*\|_2^2 = \sum_{i=1}^{3}\left((C\bar x)_i-p_i^*\right)^2
\end{equation}

Note that \(\text{SSError}\), which is the square of the Frobenius
norm~\cite{Golub2013}, becomes a measure of how close a strategy is to being an
extortionate strategy. Suspicion
of extortion then corresponds to a threshold on \(\text{SSError}\).

By observing interactions (human or otherwise), their memory one representation
can be inferred and this approach can be used to recognise extortionate
behaviour. The notion of comparing theoretic and actual plays of the IPD is not
novel, see for example~\cite{Rand2013}. Immediately it is noted that if the
environment is noisy~\cite{Wu1995} then no strategy can be considered to be
extortionate as \(p_4>0\).

In the next section, this idea will be illustrated by observing the interactions
that take place in a computer based tournament of the IPD\@.

\section{Numerical experiments}\label{sec:numerical-experiments}

In~\cite{Stewart2012} results from a tournament with
\input{./assets/tex/number_of_stewart_plotkin_strategies/main.tex} strategies,
was presented with specific consideration given to ZD strategies. This
tournament is reproduced here using the Axelrod-Python
project~\cite{Knight2016}. To obtain a good measure of the corresponding
transition rates for each strategy all matches have been run for
\input{assets/tex/number_of_turns/main.tex} turns and every match has been
repeated \input{assets/tex/number_of_repetitions/main.tex} times. All of this
interaction data is available at~\cite{vincent_knight_2018_1297075}. A good
match between the inferred Markov chain and the state distribution of the actual
interactions has been verified. Data for this is presented in the supplementary
materials.

Figure~\ref{fig:SSError_overall_in_stewart_plotkin} shows the \(\text{SSError}\)
values for all the strategies in the tournament, as reported
in~\cite{Stewart2012} the extortionate strategy (which has an expected
\(\text{SSError}\) approximately 0) gains a large number of wins.

\begin{figure}[!htbp]
    \centering
    \includegraphics[width=.8\textwidth]{./assets/img/SSError_overall_in_stewart_plotkin/main.pdf}
    \caption{\(\text{SSError}\) and state probabilities for the strategies
        of~\cite{Stewart2012}, ordered both by number of wins and overall score.
        Note that \(P(DC)\) is not shown as it corresponds to the transpose of
        \(P(CD)\). Cooperator and Defector are omitted as they do not visit all
        the states.}
    \label{fig:SSError_overall_in_stewart_plotkin}
\end{figure}

Here, the work of~\cite{Stewart2012} is extended by investigating a tournament
with \input{assets/tex/number_of_full_strategies/main.tex}
strategies.

The results of this analysis are shown in
Figure~\ref{fig:SSError_and_probabilities_in_full}. The top ranking strategies
by number of wins seem to be extortionate (but not against all strategies) and
it can be seen that a small sub group of strategies achieve mutual defection.
All the top ranking strategies according to score achieve mutual cooperation and
do not extort each other, however they
\textbf{do} exhibit extortionate behaviour towards a number of the lower ranking
strategies.

\begin{figure}[!htbp]
    \centering
    \includegraphics[width=.8\textwidth]{./assets/img/SSError_and_probabilities_in_full/main.pdf}
    \caption{\(\text{SSError}\) for the strategies for the full tournament. Only
    strategy interactions for which \(p_4=0\) and \(\chi>1\) are displayed.}
    \label{fig:SSError_and_probabilities_in_full}
\end{figure}

\section{Conclusion}\label{sec:conclusion}

This work defines an approach to measure whether or not a player is playing a
strategy that corresponds to an extortionate strategy as defined
in~\cite{Press2012}: a mathematical model for suspicion. Indeed, all
extortionate strategies have been
 classified as lying on a triangular plane.
This rigorous classification fails to be robust to small measurement error, thus
a statistical approach is proposed.
This is done through a linear algebraic approach for approximating the solution
of a linear system. Using this, a large number of pairwise interactions is
simulated and in fact very few strategies are found to act extortionately.

The work of~\cite{Press2012}, whilst showing that a clever approach to taking
advantage of another memory one strategy exists: this is incomplete. Whilst the
elegance of this result is very attractive, just as the simplicity of the
victory of Tit For Tat in Axelrod's original tournaments was, it is incomplete.
Extortionate strategies achieve a high number of wins but they do not
achieve a high score which corresponds to the fitness landscape in an
evolutionary sense. From the large number of interactions a payoff matrix \(S\)
can be measured where \(S_{ij}\) denotes the score (using standard values of
\((R, S, T, P) = (3, 0, 5, 1)\)) of the \(i\)th strategy
against the \(j\)th strategy. Using this, the replicator equation
describes the evolution of the system based on a population density fitness
function:

\begin{equation}\label{eqn:replicator_dynamics}
    \frac{dx}{dt} = x(S-x^TS x)
\end{equation}

Equation (\ref{eqn:replicator_dynamics}) is solved numerically through an
integration technique described in~\cite{Petzold1983} and
Figure~\ref{fig:replicator_dynamics} shows the evolution of the distribution of
the system: the various strategies are ranked by scores. It is clear to see that
only the high ranking strategies survive the evolutionary process (in fact,
only \input{./assets/img/replicator_dynamics/main.tex}
have a final distribution greater than \(10 ^ {-2}\)). This confirms the
findings of~\cite{Moran1707} in which sophisticated strategies resist
evolutionary invasion of shorter memory strategies. Recalling
Figure~\ref{fig:SSError_and_probabilities_in_full} this demonstrates that:

\begin{itemize}
    \item Cooperation emerges through the evolutionary process: the high scoring
        strategies do not exhibit extortionate behaviour towards each other.
    \item Extortionate strategies do not survive the evolutionary process.
\end{itemize}

\begin{figure}[!htbp]
    \centering
    \includegraphics[width=.8\textwidth]{./assets/img/replicator_dynamics/main.pdf}
    \caption{Numerical simulation of the replicator equation
    (\ref{eqn:replicator_dynamics}): strategies are ordered by score, only the strategies with a high score survive the evolutionary process.}
    \label{fig:replicator_dynamics}
\end{figure}

This work can be used to classify plays of the IPD\@: data can be collected from
actual interactions (in lab or in the field). Furthermore, this allows for a
classification method similar to the notion of fingerprinting presented
in~\cite{Ashlock2008}. Trained strategies can potentially be classified as
extortionate or not or it could be possible to even constrain the reinforcement
learning approaches that are becoming prevalent in the literature.
Alternatively, this mathematical approach for recognising extortion could be
used in sophisticated strategies to defend against invasion. Arguably, some of
the strategies considered here exhibit this behaviour, indeed as described
in~\cite{Harper2017}, the top ranking strategies in the full tournament are
obtained using evolutionary reinforcement learning techniques, thus, suspicion
of extortionate behaviour could in fact be an evolutionary trait.

\section*{Acknowledgements}

The following open source software libraries were used in this research:

\begin{itemize}
    \item The Axelrod ~\cite{Knight2016, Knight2018} library (IPD strategies and
        tournaments).
    \item The sympy library~\cite{Meurer2017} (verification of all symbolic
        calculations).
    \item The matplotlib~\cite{Droettboom2018} library (visualisation).
    \item The pandas~\cite{Structures2010}, dask~\cite{Dask2016} and
        NumPy~\cite{Oliphant2015} libraries (data manipulation).
    \item The SciPy~\cite{Jones2001} library (numerical integration of the
        replicator equation).
\end{itemize}

This work was performed using the computational facilities of the Advanced
Research Computing @ Cardiff (ARCCA) Division, Cardiff University.

\printbibliography

\newpage
\section*{Supplementary materials}

\includepdf{assets/pdf/proof_of_form_of_extortionate_strategies/main.pdf}

\newpage

Using the pair wise interactions the transition rates \(p,
q\) can be measured and the steady state probabilities inferred and compared to
the actual probabilities of each state.
This is done numerically by computing the singular eigenvector of the
matrix \(A\) \cite{Stewart2009}:

\[
    A =
    \begin{bmatrix}
        p_1 q_1 & p_1 (1 - q_1) & (1 - p_1) q_1 & (1 -p_1) (1 - q_1) \\
        p_2 q_2 & p_2 (1 - q_2) & (1 - p_2) q_2 & (1 -p_2) (1 - q_2) \\
        p_3 q_3 & p_3 (1 - q_3) & (1 - p_3) q_3 & (1 -p_3) (1 - q_3) \\
        p_4 q_4 & p_4 (1 - q_4) & (1 - p_4) q_4 & (1 -p_4) (1 - q_4) \\
    \end{bmatrix}
\]

Figure~\ref{fig:computed_probabilities_vs_theoretic_probabilities} shows a
regression line fitted to every pairwise interaction with a reported
\(\text{SSError}\) value (pairwise interactions with missing states were
omitted). This serves to validate the approach: a part from some edge cases the
relationship is consistent.

\begin{figure}[!htbp]
    \centering
    \includegraphics[width=.8\textwidth]{./assets/img/computed_probabilities_vs_theoretic_probabilities/main.pdf}
    \caption{The
        relationship between the steady state probabilities inferred from the
        measured transitions and the actual steady state probabilities. A linear
        regression line is included validating the approach.}
    \label{fig:computed_probabilities_vs_theoretic_probabilities}
\end{figure}


\end{document}

    strategies is considered. In this setting
    the most highly performing strategies do not play in an extortionate way
    against each other but do against lower performing strategies.
    This suggests that whilst the theory of Zero Determinant strategies
    indicates that memory is not of fundamental importance to the evolution of
    cooperative behaviour, this is incomplete.
\end{abstract}

\section{Introduction}\label{sec:introduction}

Agent based game theoretic models have become a stalwart of the underpinning
mathematics of interactive behaviours. One of the major pieces of work
in this area is the pair of original computer tournaments run by Robert
Axelrod~\cite{Axelrod1980, Axelrod1980a}. These tournaments pitted submitted
computer strategies against each other in plays of the Iterated Prisoner's
Dilemma. A common game where agents can choose to pay a slight cost to their
immediate utility in the hope of building a reputation. This has been used in
economic and evolutionary game theory to understand the evolution of cooperative
behaviour.

Recently, a class of strategies was described in~\cite{Press2012} that can
provably extort any given opponent. In~\cite{Hilbe2013, Moran1707} some
questions have already been asked about the true effectiveness of these
strategies in an evolutionary setting. Here another question is asked: is it
possible to recognise this extortionate behaviour? A mathematical procedure for
suspicion is presented: in the same way that the continued actions of an
extortionate individual might raise suspicion.

This work makes use of the Axelrod Python library~\cite{Knight2018, Knight2016}
with a large number of Prisoner Dilemma strategies available to give an
extensive numerical example of the ideas presented.  The approach is presented
in Section~\ref{sec:delta-zd-strategies}.  All of the code and data discussed
in Section~\ref{sec:numerical-experiments} is open sourced, archived and
written according to best scientific principles~\cite{Wilson2014}. The data
archive can be found at~\cite{vincent_knight_2018_1297075}.

\section{Recognising Extortion}\label{sec:delta-zd-strategies}

In~\cite{Press2012}, given a match between 2 memory-one strategies, the concept
of Zero Determinant (ZD) strategies is introduced. The main result of that paper
shows that given two memory one players \(p, q\in\mathbb{R}^4\) a linear
relationship between the players' scores could be forced by one of the players.

Using the notation of~\cite{Press2012}, assuming the utilities for player \(p\)
are given by \(S_x=(R, S, T, P)\) and for player \(q\) by \(S_y=(R, T, S, P)\)
and that the stationary scores of each player is given by \(S_X\) and \(S_Y\)
respectively. The main result of~\cite{Press2012} is that if

\begin{equation}\label{eqn:linear_relationship_for_p}
    \tilde p=\alpha S_x + \beta S_y + \gamma
\end{equation}

or

\begin{equation}\label{eqn:linear_relationship_for_q}
    \tilde q=\alpha S_x + \beta S_y + \gamma
\end{equation}

where \(\tilde p = (1 - p_1, 1 - p_2, p_3, p_4)\) and
\(\tilde q = (1 - q_1, 1 - q_2, q_3, q_4)\) then:

\begin{equation}
    \alpha S_X + \beta S_Y + \gamma = 0
\end{equation}

In~\cite{Press2012} a particular type of ZD strategy is defined: extortionate
strategies. If:

\begin{equation}\label{eqn:constraint_for_extortion}
    \gamma = - P(\alpha + \beta)
\end{equation}

then the player can ensure they get a score \(\chi\) times
larger than the opponent. This extortion coefficient is given by:

\begin{equation}\label{eqn:definition_of_chi}
    \chi=\frac{-\beta}{\alpha}
\end{equation}

Thus, if (\ref{eqn:constraint_for_extortion}) holds and \(\chi >1\) a player is
said to extort their opponent.
Here, the reverse problem is considered: given a
\(p\in\mathbb{R}^4\) how does one identify \(\alpha, \beta\) if they
exist and is the strategy in fact acting in an extortionate way?

These conditions correspond to:

\begin{align}
    \tilde p_1 & = \alpha R + \beta R - P (\alpha + \beta)
            \label{eqn:condition_for_tilde_p1}\\
    \tilde p_2 & = \alpha S + \beta T - P (\alpha + \beta)
            \label{eqn:condition_for_tilde_p2}\\
    \tilde p_3 & = \alpha T + \beta S - P (\alpha + \beta)
            \label{eqn:condition_for_tilde_p3}\\
    \tilde p_4 & = \alpha P + \beta P - P (\alpha + \beta)
            \label{eqn:condition_for_tilde_p4}
\end{align}

Equation (\ref{eqn:condition_for_tilde_p4}) ensures that \(p_4=\tilde p_4=0\).
Equations (\ref{eqn:condition_for_tilde_p1}-\ref{eqn:condition_for_tilde_p3})
can be used to eliminate \(\alpha, \beta\), giving:

\begin{equation}\label{eqn:planar_definition_of_extortion}
    \tilde p_1 = \frac{(R - P)(\tilde p_2 + \tilde p_3)}{S + T - 2P}
\end{equation}

with:

\begin{equation}\label{eqn:definition_of_chi}
    \chi = \frac{\tilde p_2 (P - T) + \tilde p_3 (S - P)}
                {\tilde p_2 (P - S) + \tilde p_3 (T - P)}
\end{equation}

Given a strategy \(p\in\mathbb{R}^{4\times 1}\) equations
(\ref{eqn:condition_for_tilde_p4}), (\ref{eqn:planar_definition_of_extortion}-\ref{eqn:definition_of_chi}) can be used to check if
a strategy is extortionate. The conditions correspond to:

\begin{align}
    p_1 & = \frac{(R-P)(p_2 + p_3) - R + T + S - P}{S + T - 2P}
     \label{eqn:condition_for_p1}\\
    p_4 & = 0 \label{eqn:condition_for_p4}\\
    1 & > p_2 + p_3\label{eqn:condition_for_chi}
\end{align}

The algebraic steps necessary to prove these results are available in the
supporting materials.

All extortionate strategies reside on a triangular (\ref{eqn:condition_for_chi})
plane (\ref{eqn:condition_for_p1}) in 3 dimensions (\ref{eqn:condition_for_p4}).
Using this formulation it can be seen that a necessary (but not sufficient)
condition for an extortionate strategy is that it cooperates on average less
than 50\% of the time when in a state of disagreement with the opponent.

As an example, consider the known extortionate strategy \(p=(8 / 9, 1 / 2, 1 /
3, 0)\) from~\cite{Stewart2012} which is referred to as \texttt{Extort-2}. In
this case, for the standard values of \((R, T, S, P)\) constraint
(\ref{eqn:condition_for_p1}) corresponds to:

\begin{equation}
    p_1 = \frac{2(p_2 + p_3) + 1}{3}
\end{equation}

It is clear that in this case all constraints hold.

This approach could in fact be used to confirm that a given strategy is acting
in an extortionate manner even if it is not a memory one strategy. However, in
practice, if a closed form for \(p\) is not known, then due to measurement
and/or numerical error this would not work.

This problem can be written in the following linear algebraic form where
\(x=(\alpha, \beta)\)
and \(p^*=(\tilde p_1 - 1, tilde_2 - 1, p_3)\):

\begin{equation}\label{eqn:linear_algebraic_equation_for_p}
    Cx= p^*
\end{equation}

\(C\) corresponds to equations
(\ref{eqn:condition_for_tilde_p1}-\ref{eqn:condition_for_tilde_p3}) and is
given by:

\begin{equation}\label{eqn:definition_of_C}
    C =
    \begin{bmatrix}
        R - P & R- P \\
        S - P & T- P \\
        T - P & S- P \\
    \end{bmatrix}
\end{equation}

Note that in general, equation (\ref{eqn:linear_algebraic_equation_for_p}) will
not necessarily have a solution. From the Rouch\'{e}-Capelli theorem if there is
a solution it is unique as \(\text{rank}(C)=2\) which is the dimension of the
variable \(x\). The best fitting \(x\) is found by minimizing:

\begin{equation}\label{eqn:r_squared}
    \text{SSError} = \|C x- p^*\|_2^2 = \sum_{i=1}^{3}\left((C\bar x)_i-p_i^*\right)^2
\end{equation}

Note that \(\text{SSError}\), which is the square of the Frobenius
norm~\cite{Golub2013}, becomes a measure of how close a strategy is to being an
extortionate strategy. Suspicion
of extortion then corresponds to a threshold on \(\text{SSError}\).

By observing interactions (human or otherwise), their memory one representation
can be inferred and this approach can be used to recognise extortionate
behaviour. The notion of comparing theoretic and actual plays of the IPD is not
novel, see for example~\cite{Rand2013}. Immediately it is noted that if the
environment is noisy~\cite{Wu1995} then no strategy can be considered to be
extortionate as \(p_4>0\).

In the next section, this idea will be illustrated by observing the interactions
that take place in a computer based tournament of the IPD\@.

\section{Numerical experiments}\label{sec:numerical-experiments}

In~\cite{Stewart2012} results from a tournament with
\documentclass[a4paper]{article}

\usepackage{amsmath}
\usepackage{amssymb}
\usepackage[margin=1.5cm,
            includefoot,
            footskip=30pt]{geometry}
\usepackage{layout}
\usepackage{graphicx}
\usepackage{subcaption}

\usepackage{biblatex}
\usepackage{pdfpages}

\bibliography{main.bib}

\title{Suspicion: Recognising and evaluating the effectiveness
       of extortion in the Iterated Prisoner's Dilemma}
\author{Vincent A. Knight \and Nikoleta E. Glynatsi}
\date{\today}



\begin{document}

\maketitle

\begin{abstract}
    The Iterated Prisoner's Dilemma is a model for rational and evolutionary
    interactive behaviour. It has applications both in the study of human social
    behaviour as well as in biology.
    It is used to understand when and how a rational individual might
    accept an immediate cost to their own utility for the direct benefit of
    another.

    Much attention has been given to a class of strategies called
    Zero Determinant strategies. It has been theoretically shown that these
    strategies can ``extort'' any player.

    In this work, an approach to identify if observed strategies are playing in
    an extortionate way is described. Furthermore, experimental analysis of
    a large tournament with \input{assets/tex/number_of_full_strategies/main.tex}
    strategies is considered. In this setting
    the most highly performing strategies do not play in an extortionate way
    against each other but do against lower performing strategies.
    This suggests that whilst the theory of Zero Determinant strategies
    indicates that memory is not of fundamental importance to the evolution of
    cooperative behaviour, this is incomplete.
\end{abstract}

\section{Introduction}\label{sec:introduction}

Agent based game theoretic models have become a stalwart of the underpinning
mathematics of interactive behaviours. One of the major pieces of work
in this area is the pair of original computer tournaments run by Robert
Axelrod~\cite{Axelrod1980, Axelrod1980a}. These tournaments pitted submitted
computer strategies against each other in plays of the Iterated Prisoner's
Dilemma. A common game where agents can choose to pay a slight cost to their
immediate utility in the hope of building a reputation. This has been used in
economic and evolutionary game theory to understand the evolution of cooperative
behaviour.

Recently, a class of strategies was described in~\cite{Press2012} that can
provably extort any given opponent. In~\cite{Hilbe2013, Moran1707} some
questions have already been asked about the true effectiveness of these
strategies in an evolutionary setting. Here another question is asked: is it
possible to recognise this extortionate behaviour? A mathematical procedure for
suspicion is presented: in the same way that the continued actions of an
extortionate individual might raise suspicion.

This work makes use of the Axelrod Python library~\cite{Knight2018, Knight2016}
with a large number of Prisoner Dilemma strategies available to give an
extensive numerical example of the ideas presented.  The approach is presented
in Section~\ref{sec:delta-zd-strategies}.  All of the code and data discussed
in Section~\ref{sec:numerical-experiments} is open sourced, archived and
written according to best scientific principles~\cite{Wilson2014}. The data
archive can be found at~\cite{vincent_knight_2018_1297075}.

\section{Recognising Extortion}\label{sec:delta-zd-strategies}

In~\cite{Press2012}, given a match between 2 memory-one strategies, the concept
of Zero Determinant (ZD) strategies is introduced. The main result of that paper
shows that given two memory one players \(p, q\in\mathbb{R}^4\) a linear
relationship between the players' scores could be forced by one of the players.

Using the notation of~\cite{Press2012}, assuming the utilities for player \(p\)
are given by \(S_x=(R, S, T, P)\) and for player \(q\) by \(S_y=(R, T, S, P)\)
and that the stationary scores of each player is given by \(S_X\) and \(S_Y\)
respectively. The main result of~\cite{Press2012} is that if

\begin{equation}\label{eqn:linear_relationship_for_p}
    \tilde p=\alpha S_x + \beta S_y + \gamma
\end{equation}

or

\begin{equation}\label{eqn:linear_relationship_for_q}
    \tilde q=\alpha S_x + \beta S_y + \gamma
\end{equation}

where \(\tilde p = (1 - p_1, 1 - p_2, p_3, p_4)\) and
\(\tilde q = (1 - q_1, 1 - q_2, q_3, q_4)\) then:

\begin{equation}
    \alpha S_X + \beta S_Y + \gamma = 0
\end{equation}

In~\cite{Press2012} a particular type of ZD strategy is defined: extortionate
strategies. If:

\begin{equation}\label{eqn:constraint_for_extortion}
    \gamma = - P(\alpha + \beta)
\end{equation}

then the player can ensure they get a score \(\chi\) times
larger than the opponent. This extortion coefficient is given by:

\begin{equation}\label{eqn:definition_of_chi}
    \chi=\frac{-\beta}{\alpha}
\end{equation}

Thus, if (\ref{eqn:constraint_for_extortion}) holds and \(\chi >1\) a player is
said to extort their opponent.
Here, the reverse problem is considered: given a
\(p\in\mathbb{R}^4\) how does one identify \(\alpha, \beta\) if they
exist and is the strategy in fact acting in an extortionate way?

These conditions correspond to:

\begin{align}
    \tilde p_1 & = \alpha R + \beta R - P (\alpha + \beta)
            \label{eqn:condition_for_tilde_p1}\\
    \tilde p_2 & = \alpha S + \beta T - P (\alpha + \beta)
            \label{eqn:condition_for_tilde_p2}\\
    \tilde p_3 & = \alpha T + \beta S - P (\alpha + \beta)
            \label{eqn:condition_for_tilde_p3}\\
    \tilde p_4 & = \alpha P + \beta P - P (\alpha + \beta)
            \label{eqn:condition_for_tilde_p4}
\end{align}

Equation (\ref{eqn:condition_for_tilde_p4}) ensures that \(p_4=\tilde p_4=0\).
Equations (\ref{eqn:condition_for_tilde_p1}-\ref{eqn:condition_for_tilde_p3})
can be used to eliminate \(\alpha, \beta\), giving:

\begin{equation}\label{eqn:planar_definition_of_extortion}
    \tilde p_1 = \frac{(R - P)(\tilde p_2 + \tilde p_3)}{S + T - 2P}
\end{equation}

with:

\begin{equation}\label{eqn:definition_of_chi}
    \chi = \frac{\tilde p_2 (P - T) + \tilde p_3 (S - P)}
                {\tilde p_2 (P - S) + \tilde p_3 (T - P)}
\end{equation}

Given a strategy \(p\in\mathbb{R}^{4\times 1}\) equations
(\ref{eqn:condition_for_tilde_p4}), (\ref{eqn:planar_definition_of_extortion}-\ref{eqn:definition_of_chi}) can be used to check if
a strategy is extortionate. The conditions correspond to:

\begin{align}
    p_1 & = \frac{(R-P)(p_2 + p_3) - R + T + S - P}{S + T - 2P}
     \label{eqn:condition_for_p1}\\
    p_4 & = 0 \label{eqn:condition_for_p4}\\
    1 & > p_2 + p_3\label{eqn:condition_for_chi}
\end{align}

The algebraic steps necessary to prove these results are available in the
supporting materials.

All extortionate strategies reside on a triangular (\ref{eqn:condition_for_chi})
plane (\ref{eqn:condition_for_p1}) in 3 dimensions (\ref{eqn:condition_for_p4}).
Using this formulation it can be seen that a necessary (but not sufficient)
condition for an extortionate strategy is that it cooperates on average less
than 50\% of the time when in a state of disagreement with the opponent.

As an example, consider the known extortionate strategy \(p=(8 / 9, 1 / 2, 1 /
3, 0)\) from~\cite{Stewart2012} which is referred to as \texttt{Extort-2}. In
this case, for the standard values of \((R, T, S, P)\) constraint
(\ref{eqn:condition_for_p1}) corresponds to:

\begin{equation}
    p_1 = \frac{2(p_2 + p_3) + 1}{3}
\end{equation}

It is clear that in this case all constraints hold.

This approach could in fact be used to confirm that a given strategy is acting
in an extortionate manner even if it is not a memory one strategy. However, in
practice, if a closed form for \(p\) is not known, then due to measurement
and/or numerical error this would not work.

This problem can be written in the following linear algebraic form where
\(x=(\alpha, \beta)\)
and \(p^*=(\tilde p_1 - 1, tilde_2 - 1, p_3)\):

\begin{equation}\label{eqn:linear_algebraic_equation_for_p}
    Cx= p^*
\end{equation}

\(C\) corresponds to equations
(\ref{eqn:condition_for_tilde_p1}-\ref{eqn:condition_for_tilde_p3}) and is
given by:

\begin{equation}\label{eqn:definition_of_C}
    C =
    \begin{bmatrix}
        R - P & R- P \\
        S - P & T- P \\
        T - P & S- P \\
    \end{bmatrix}
\end{equation}

Note that in general, equation (\ref{eqn:linear_algebraic_equation_for_p}) will
not necessarily have a solution. From the Rouch\'{e}-Capelli theorem if there is
a solution it is unique as \(\text{rank}(C)=2\) which is the dimension of the
variable \(x\). The best fitting \(x\) is found by minimizing:

\begin{equation}\label{eqn:r_squared}
    \text{SSError} = \|C x- p^*\|_2^2 = \sum_{i=1}^{3}\left((C\bar x)_i-p_i^*\right)^2
\end{equation}

Note that \(\text{SSError}\), which is the square of the Frobenius
norm~\cite{Golub2013}, becomes a measure of how close a strategy is to being an
extortionate strategy. Suspicion
of extortion then corresponds to a threshold on \(\text{SSError}\).

By observing interactions (human or otherwise), their memory one representation
can be inferred and this approach can be used to recognise extortionate
behaviour. The notion of comparing theoretic and actual plays of the IPD is not
novel, see for example~\cite{Rand2013}. Immediately it is noted that if the
environment is noisy~\cite{Wu1995} then no strategy can be considered to be
extortionate as \(p_4>0\).

In the next section, this idea will be illustrated by observing the interactions
that take place in a computer based tournament of the IPD\@.

\section{Numerical experiments}\label{sec:numerical-experiments}

In~\cite{Stewart2012} results from a tournament with
\input{./assets/tex/number_of_stewart_plotkin_strategies/main.tex} strategies,
was presented with specific consideration given to ZD strategies. This
tournament is reproduced here using the Axelrod-Python
project~\cite{Knight2016}. To obtain a good measure of the corresponding
transition rates for each strategy all matches have been run for
\input{assets/tex/number_of_turns/main.tex} turns and every match has been
repeated \input{assets/tex/number_of_repetitions/main.tex} times. All of this
interaction data is available at~\cite{vincent_knight_2018_1297075}. A good
match between the inferred Markov chain and the state distribution of the actual
interactions has been verified. Data for this is presented in the supplementary
materials.

Figure~\ref{fig:SSError_overall_in_stewart_plotkin} shows the \(\text{SSError}\)
values for all the strategies in the tournament, as reported
in~\cite{Stewart2012} the extortionate strategy (which has an expected
\(\text{SSError}\) approximately 0) gains a large number of wins.

\begin{figure}[!htbp]
    \centering
    \includegraphics[width=.8\textwidth]{./assets/img/SSError_overall_in_stewart_plotkin/main.pdf}
    \caption{\(\text{SSError}\) and state probabilities for the strategies
        of~\cite{Stewart2012}, ordered both by number of wins and overall score.
        Note that \(P(DC)\) is not shown as it corresponds to the transpose of
        \(P(CD)\). Cooperator and Defector are omitted as they do not visit all
        the states.}
    \label{fig:SSError_overall_in_stewart_plotkin}
\end{figure}

Here, the work of~\cite{Stewart2012} is extended by investigating a tournament
with \input{assets/tex/number_of_full_strategies/main.tex}
strategies.

The results of this analysis are shown in
Figure~\ref{fig:SSError_and_probabilities_in_full}. The top ranking strategies
by number of wins seem to be extortionate (but not against all strategies) and
it can be seen that a small sub group of strategies achieve mutual defection.
All the top ranking strategies according to score achieve mutual cooperation and
do not extort each other, however they
\textbf{do} exhibit extortionate behaviour towards a number of the lower ranking
strategies.

\begin{figure}[!htbp]
    \centering
    \includegraphics[width=.8\textwidth]{./assets/img/SSError_and_probabilities_in_full/main.pdf}
    \caption{\(\text{SSError}\) for the strategies for the full tournament. Only
    strategy interactions for which \(p_4=0\) and \(\chi>1\) are displayed.}
    \label{fig:SSError_and_probabilities_in_full}
\end{figure}

\section{Conclusion}\label{sec:conclusion}

This work defines an approach to measure whether or not a player is playing a
strategy that corresponds to an extortionate strategy as defined
in~\cite{Press2012}: a mathematical model for suspicion. Indeed, all
extortionate strategies have been
 classified as lying on a triangular plane.
This rigorous classification fails to be robust to small measurement error, thus
a statistical approach is proposed.
This is done through a linear algebraic approach for approximating the solution
of a linear system. Using this, a large number of pairwise interactions is
simulated and in fact very few strategies are found to act extortionately.

The work of~\cite{Press2012}, whilst showing that a clever approach to taking
advantage of another memory one strategy exists: this is incomplete. Whilst the
elegance of this result is very attractive, just as the simplicity of the
victory of Tit For Tat in Axelrod's original tournaments was, it is incomplete.
Extortionate strategies achieve a high number of wins but they do not
achieve a high score which corresponds to the fitness landscape in an
evolutionary sense. From the large number of interactions a payoff matrix \(S\)
can be measured where \(S_{ij}\) denotes the score (using standard values of
\((R, S, T, P) = (3, 0, 5, 1)\)) of the \(i\)th strategy
against the \(j\)th strategy. Using this, the replicator equation
describes the evolution of the system based on a population density fitness
function:

\begin{equation}\label{eqn:replicator_dynamics}
    \frac{dx}{dt} = x(S-x^TS x)
\end{equation}

Equation (\ref{eqn:replicator_dynamics}) is solved numerically through an
integration technique described in~\cite{Petzold1983} and
Figure~\ref{fig:replicator_dynamics} shows the evolution of the distribution of
the system: the various strategies are ranked by scores. It is clear to see that
only the high ranking strategies survive the evolutionary process (in fact,
only \input{./assets/img/replicator_dynamics/main.tex}
have a final distribution greater than \(10 ^ {-2}\)). This confirms the
findings of~\cite{Moran1707} in which sophisticated strategies resist
evolutionary invasion of shorter memory strategies. Recalling
Figure~\ref{fig:SSError_and_probabilities_in_full} this demonstrates that:

\begin{itemize}
    \item Cooperation emerges through the evolutionary process: the high scoring
        strategies do not exhibit extortionate behaviour towards each other.
    \item Extortionate strategies do not survive the evolutionary process.
\end{itemize}

\begin{figure}[!htbp]
    \centering
    \includegraphics[width=.8\textwidth]{./assets/img/replicator_dynamics/main.pdf}
    \caption{Numerical simulation of the replicator equation
    (\ref{eqn:replicator_dynamics}): strategies are ordered by score, only the strategies with a high score survive the evolutionary process.}
    \label{fig:replicator_dynamics}
\end{figure}

This work can be used to classify plays of the IPD\@: data can be collected from
actual interactions (in lab or in the field). Furthermore, this allows for a
classification method similar to the notion of fingerprinting presented
in~\cite{Ashlock2008}. Trained strategies can potentially be classified as
extortionate or not or it could be possible to even constrain the reinforcement
learning approaches that are becoming prevalent in the literature.
Alternatively, this mathematical approach for recognising extortion could be
used in sophisticated strategies to defend against invasion. Arguably, some of
the strategies considered here exhibit this behaviour, indeed as described
in~\cite{Harper2017}, the top ranking strategies in the full tournament are
obtained using evolutionary reinforcement learning techniques, thus, suspicion
of extortionate behaviour could in fact be an evolutionary trait.

\section*{Acknowledgements}

The following open source software libraries were used in this research:

\begin{itemize}
    \item The Axelrod ~\cite{Knight2016, Knight2018} library (IPD strategies and
        tournaments).
    \item The sympy library~\cite{Meurer2017} (verification of all symbolic
        calculations).
    \item The matplotlib~\cite{Droettboom2018} library (visualisation).
    \item The pandas~\cite{Structures2010}, dask~\cite{Dask2016} and
        NumPy~\cite{Oliphant2015} libraries (data manipulation).
    \item The SciPy~\cite{Jones2001} library (numerical integration of the
        replicator equation).
\end{itemize}

This work was performed using the computational facilities of the Advanced
Research Computing @ Cardiff (ARCCA) Division, Cardiff University.

\printbibliography

\newpage
\section*{Supplementary materials}

\includepdf{assets/pdf/proof_of_form_of_extortionate_strategies/main.pdf}

\newpage

Using the pair wise interactions the transition rates \(p,
q\) can be measured and the steady state probabilities inferred and compared to
the actual probabilities of each state.
This is done numerically by computing the singular eigenvector of the
matrix \(A\) \cite{Stewart2009}:

\[
    A =
    \begin{bmatrix}
        p_1 q_1 & p_1 (1 - q_1) & (1 - p_1) q_1 & (1 -p_1) (1 - q_1) \\
        p_2 q_2 & p_2 (1 - q_2) & (1 - p_2) q_2 & (1 -p_2) (1 - q_2) \\
        p_3 q_3 & p_3 (1 - q_3) & (1 - p_3) q_3 & (1 -p_3) (1 - q_3) \\
        p_4 q_4 & p_4 (1 - q_4) & (1 - p_4) q_4 & (1 -p_4) (1 - q_4) \\
    \end{bmatrix}
\]

Figure~\ref{fig:computed_probabilities_vs_theoretic_probabilities} shows a
regression line fitted to every pairwise interaction with a reported
\(\text{SSError}\) value (pairwise interactions with missing states were
omitted). This serves to validate the approach: a part from some edge cases the
relationship is consistent.

\begin{figure}[!htbp]
    \centering
    \includegraphics[width=.8\textwidth]{./assets/img/computed_probabilities_vs_theoretic_probabilities/main.pdf}
    \caption{The
        relationship between the steady state probabilities inferred from the
        measured transitions and the actual steady state probabilities. A linear
        regression line is included validating the approach.}
    \label{fig:computed_probabilities_vs_theoretic_probabilities}
\end{figure}


\end{document}
 strategies,
was presented with specific consideration given to ZD strategies. This
tournament is reproduced here using the Axelrod-Python
project~\cite{Knight2016}. To obtain a good measure of the corresponding
transition rates for each strategy all matches have been run for
\documentclass[a4paper]{article}

\usepackage{amsmath}
\usepackage{amssymb}
\usepackage[margin=1.5cm,
            includefoot,
            footskip=30pt]{geometry}
\usepackage{layout}
\usepackage{graphicx}
\usepackage{subcaption}

\usepackage{biblatex}
\usepackage{pdfpages}

\bibliography{main.bib}

\title{Suspicion: Recognising and evaluating the effectiveness
       of extortion in the Iterated Prisoner's Dilemma}
\author{Vincent A. Knight \and Nikoleta E. Glynatsi}
\date{\today}



\begin{document}

\maketitle

\begin{abstract}
    The Iterated Prisoner's Dilemma is a model for rational and evolutionary
    interactive behaviour. It has applications both in the study of human social
    behaviour as well as in biology.
    It is used to understand when and how a rational individual might
    accept an immediate cost to their own utility for the direct benefit of
    another.

    Much attention has been given to a class of strategies called
    Zero Determinant strategies. It has been theoretically shown that these
    strategies can ``extort'' any player.

    In this work, an approach to identify if observed strategies are playing in
    an extortionate way is described. Furthermore, experimental analysis of
    a large tournament with \input{assets/tex/number_of_full_strategies/main.tex}
    strategies is considered. In this setting
    the most highly performing strategies do not play in an extortionate way
    against each other but do against lower performing strategies.
    This suggests that whilst the theory of Zero Determinant strategies
    indicates that memory is not of fundamental importance to the evolution of
    cooperative behaviour, this is incomplete.
\end{abstract}

\section{Introduction}\label{sec:introduction}

Agent based game theoretic models have become a stalwart of the underpinning
mathematics of interactive behaviours. One of the major pieces of work
in this area is the pair of original computer tournaments run by Robert
Axelrod~\cite{Axelrod1980, Axelrod1980a}. These tournaments pitted submitted
computer strategies against each other in plays of the Iterated Prisoner's
Dilemma. A common game where agents can choose to pay a slight cost to their
immediate utility in the hope of building a reputation. This has been used in
economic and evolutionary game theory to understand the evolution of cooperative
behaviour.

Recently, a class of strategies was described in~\cite{Press2012} that can
provably extort any given opponent. In~\cite{Hilbe2013, Moran1707} some
questions have already been asked about the true effectiveness of these
strategies in an evolutionary setting. Here another question is asked: is it
possible to recognise this extortionate behaviour? A mathematical procedure for
suspicion is presented: in the same way that the continued actions of an
extortionate individual might raise suspicion.

This work makes use of the Axelrod Python library~\cite{Knight2018, Knight2016}
with a large number of Prisoner Dilemma strategies available to give an
extensive numerical example of the ideas presented.  The approach is presented
in Section~\ref{sec:delta-zd-strategies}.  All of the code and data discussed
in Section~\ref{sec:numerical-experiments} is open sourced, archived and
written according to best scientific principles~\cite{Wilson2014}. The data
archive can be found at~\cite{vincent_knight_2018_1297075}.

\section{Recognising Extortion}\label{sec:delta-zd-strategies}

In~\cite{Press2012}, given a match between 2 memory-one strategies, the concept
of Zero Determinant (ZD) strategies is introduced. The main result of that paper
shows that given two memory one players \(p, q\in\mathbb{R}^4\) a linear
relationship between the players' scores could be forced by one of the players.

Using the notation of~\cite{Press2012}, assuming the utilities for player \(p\)
are given by \(S_x=(R, S, T, P)\) and for player \(q\) by \(S_y=(R, T, S, P)\)
and that the stationary scores of each player is given by \(S_X\) and \(S_Y\)
respectively. The main result of~\cite{Press2012} is that if

\begin{equation}\label{eqn:linear_relationship_for_p}
    \tilde p=\alpha S_x + \beta S_y + \gamma
\end{equation}

or

\begin{equation}\label{eqn:linear_relationship_for_q}
    \tilde q=\alpha S_x + \beta S_y + \gamma
\end{equation}

where \(\tilde p = (1 - p_1, 1 - p_2, p_3, p_4)\) and
\(\tilde q = (1 - q_1, 1 - q_2, q_3, q_4)\) then:

\begin{equation}
    \alpha S_X + \beta S_Y + \gamma = 0
\end{equation}

In~\cite{Press2012} a particular type of ZD strategy is defined: extortionate
strategies. If:

\begin{equation}\label{eqn:constraint_for_extortion}
    \gamma = - P(\alpha + \beta)
\end{equation}

then the player can ensure they get a score \(\chi\) times
larger than the opponent. This extortion coefficient is given by:

\begin{equation}\label{eqn:definition_of_chi}
    \chi=\frac{-\beta}{\alpha}
\end{equation}

Thus, if (\ref{eqn:constraint_for_extortion}) holds and \(\chi >1\) a player is
said to extort their opponent.
Here, the reverse problem is considered: given a
\(p\in\mathbb{R}^4\) how does one identify \(\alpha, \beta\) if they
exist and is the strategy in fact acting in an extortionate way?

These conditions correspond to:

\begin{align}
    \tilde p_1 & = \alpha R + \beta R - P (\alpha + \beta)
            \label{eqn:condition_for_tilde_p1}\\
    \tilde p_2 & = \alpha S + \beta T - P (\alpha + \beta)
            \label{eqn:condition_for_tilde_p2}\\
    \tilde p_3 & = \alpha T + \beta S - P (\alpha + \beta)
            \label{eqn:condition_for_tilde_p3}\\
    \tilde p_4 & = \alpha P + \beta P - P (\alpha + \beta)
            \label{eqn:condition_for_tilde_p4}
\end{align}

Equation (\ref{eqn:condition_for_tilde_p4}) ensures that \(p_4=\tilde p_4=0\).
Equations (\ref{eqn:condition_for_tilde_p1}-\ref{eqn:condition_for_tilde_p3})
can be used to eliminate \(\alpha, \beta\), giving:

\begin{equation}\label{eqn:planar_definition_of_extortion}
    \tilde p_1 = \frac{(R - P)(\tilde p_2 + \tilde p_3)}{S + T - 2P}
\end{equation}

with:

\begin{equation}\label{eqn:definition_of_chi}
    \chi = \frac{\tilde p_2 (P - T) + \tilde p_3 (S - P)}
                {\tilde p_2 (P - S) + \tilde p_3 (T - P)}
\end{equation}

Given a strategy \(p\in\mathbb{R}^{4\times 1}\) equations
(\ref{eqn:condition_for_tilde_p4}), (\ref{eqn:planar_definition_of_extortion}-\ref{eqn:definition_of_chi}) can be used to check if
a strategy is extortionate. The conditions correspond to:

\begin{align}
    p_1 & = \frac{(R-P)(p_2 + p_3) - R + T + S - P}{S + T - 2P}
     \label{eqn:condition_for_p1}\\
    p_4 & = 0 \label{eqn:condition_for_p4}\\
    1 & > p_2 + p_3\label{eqn:condition_for_chi}
\end{align}

The algebraic steps necessary to prove these results are available in the
supporting materials.

All extortionate strategies reside on a triangular (\ref{eqn:condition_for_chi})
plane (\ref{eqn:condition_for_p1}) in 3 dimensions (\ref{eqn:condition_for_p4}).
Using this formulation it can be seen that a necessary (but not sufficient)
condition for an extortionate strategy is that it cooperates on average less
than 50\% of the time when in a state of disagreement with the opponent.

As an example, consider the known extortionate strategy \(p=(8 / 9, 1 / 2, 1 /
3, 0)\) from~\cite{Stewart2012} which is referred to as \texttt{Extort-2}. In
this case, for the standard values of \((R, T, S, P)\) constraint
(\ref{eqn:condition_for_p1}) corresponds to:

\begin{equation}
    p_1 = \frac{2(p_2 + p_3) + 1}{3}
\end{equation}

It is clear that in this case all constraints hold.

This approach could in fact be used to confirm that a given strategy is acting
in an extortionate manner even if it is not a memory one strategy. However, in
practice, if a closed form for \(p\) is not known, then due to measurement
and/or numerical error this would not work.

This problem can be written in the following linear algebraic form where
\(x=(\alpha, \beta)\)
and \(p^*=(\tilde p_1 - 1, tilde_2 - 1, p_3)\):

\begin{equation}\label{eqn:linear_algebraic_equation_for_p}
    Cx= p^*
\end{equation}

\(C\) corresponds to equations
(\ref{eqn:condition_for_tilde_p1}-\ref{eqn:condition_for_tilde_p3}) and is
given by:

\begin{equation}\label{eqn:definition_of_C}
    C =
    \begin{bmatrix}
        R - P & R- P \\
        S - P & T- P \\
        T - P & S- P \\
    \end{bmatrix}
\end{equation}

Note that in general, equation (\ref{eqn:linear_algebraic_equation_for_p}) will
not necessarily have a solution. From the Rouch\'{e}-Capelli theorem if there is
a solution it is unique as \(\text{rank}(C)=2\) which is the dimension of the
variable \(x\). The best fitting \(x\) is found by minimizing:

\begin{equation}\label{eqn:r_squared}
    \text{SSError} = \|C x- p^*\|_2^2 = \sum_{i=1}^{3}\left((C\bar x)_i-p_i^*\right)^2
\end{equation}

Note that \(\text{SSError}\), which is the square of the Frobenius
norm~\cite{Golub2013}, becomes a measure of how close a strategy is to being an
extortionate strategy. Suspicion
of extortion then corresponds to a threshold on \(\text{SSError}\).

By observing interactions (human or otherwise), their memory one representation
can be inferred and this approach can be used to recognise extortionate
behaviour. The notion of comparing theoretic and actual plays of the IPD is not
novel, see for example~\cite{Rand2013}. Immediately it is noted that if the
environment is noisy~\cite{Wu1995} then no strategy can be considered to be
extortionate as \(p_4>0\).

In the next section, this idea will be illustrated by observing the interactions
that take place in a computer based tournament of the IPD\@.

\section{Numerical experiments}\label{sec:numerical-experiments}

In~\cite{Stewart2012} results from a tournament with
\input{./assets/tex/number_of_stewart_plotkin_strategies/main.tex} strategies,
was presented with specific consideration given to ZD strategies. This
tournament is reproduced here using the Axelrod-Python
project~\cite{Knight2016}. To obtain a good measure of the corresponding
transition rates for each strategy all matches have been run for
\input{assets/tex/number_of_turns/main.tex} turns and every match has been
repeated \input{assets/tex/number_of_repetitions/main.tex} times. All of this
interaction data is available at~\cite{vincent_knight_2018_1297075}. A good
match between the inferred Markov chain and the state distribution of the actual
interactions has been verified. Data for this is presented in the supplementary
materials.

Figure~\ref{fig:SSError_overall_in_stewart_plotkin} shows the \(\text{SSError}\)
values for all the strategies in the tournament, as reported
in~\cite{Stewart2012} the extortionate strategy (which has an expected
\(\text{SSError}\) approximately 0) gains a large number of wins.

\begin{figure}[!htbp]
    \centering
    \includegraphics[width=.8\textwidth]{./assets/img/SSError_overall_in_stewart_plotkin/main.pdf}
    \caption{\(\text{SSError}\) and state probabilities for the strategies
        of~\cite{Stewart2012}, ordered both by number of wins and overall score.
        Note that \(P(DC)\) is not shown as it corresponds to the transpose of
        \(P(CD)\). Cooperator and Defector are omitted as they do not visit all
        the states.}
    \label{fig:SSError_overall_in_stewart_plotkin}
\end{figure}

Here, the work of~\cite{Stewart2012} is extended by investigating a tournament
with \input{assets/tex/number_of_full_strategies/main.tex}
strategies.

The results of this analysis are shown in
Figure~\ref{fig:SSError_and_probabilities_in_full}. The top ranking strategies
by number of wins seem to be extortionate (but not against all strategies) and
it can be seen that a small sub group of strategies achieve mutual defection.
All the top ranking strategies according to score achieve mutual cooperation and
do not extort each other, however they
\textbf{do} exhibit extortionate behaviour towards a number of the lower ranking
strategies.

\begin{figure}[!htbp]
    \centering
    \includegraphics[width=.8\textwidth]{./assets/img/SSError_and_probabilities_in_full/main.pdf}
    \caption{\(\text{SSError}\) for the strategies for the full tournament. Only
    strategy interactions for which \(p_4=0\) and \(\chi>1\) are displayed.}
    \label{fig:SSError_and_probabilities_in_full}
\end{figure}

\section{Conclusion}\label{sec:conclusion}

This work defines an approach to measure whether or not a player is playing a
strategy that corresponds to an extortionate strategy as defined
in~\cite{Press2012}: a mathematical model for suspicion. Indeed, all
extortionate strategies have been
 classified as lying on a triangular plane.
This rigorous classification fails to be robust to small measurement error, thus
a statistical approach is proposed.
This is done through a linear algebraic approach for approximating the solution
of a linear system. Using this, a large number of pairwise interactions is
simulated and in fact very few strategies are found to act extortionately.

The work of~\cite{Press2012}, whilst showing that a clever approach to taking
advantage of another memory one strategy exists: this is incomplete. Whilst the
elegance of this result is very attractive, just as the simplicity of the
victory of Tit For Tat in Axelrod's original tournaments was, it is incomplete.
Extortionate strategies achieve a high number of wins but they do not
achieve a high score which corresponds to the fitness landscape in an
evolutionary sense. From the large number of interactions a payoff matrix \(S\)
can be measured where \(S_{ij}\) denotes the score (using standard values of
\((R, S, T, P) = (3, 0, 5, 1)\)) of the \(i\)th strategy
against the \(j\)th strategy. Using this, the replicator equation
describes the evolution of the system based on a population density fitness
function:

\begin{equation}\label{eqn:replicator_dynamics}
    \frac{dx}{dt} = x(S-x^TS x)
\end{equation}

Equation (\ref{eqn:replicator_dynamics}) is solved numerically through an
integration technique described in~\cite{Petzold1983} and
Figure~\ref{fig:replicator_dynamics} shows the evolution of the distribution of
the system: the various strategies are ranked by scores. It is clear to see that
only the high ranking strategies survive the evolutionary process (in fact,
only \input{./assets/img/replicator_dynamics/main.tex}
have a final distribution greater than \(10 ^ {-2}\)). This confirms the
findings of~\cite{Moran1707} in which sophisticated strategies resist
evolutionary invasion of shorter memory strategies. Recalling
Figure~\ref{fig:SSError_and_probabilities_in_full} this demonstrates that:

\begin{itemize}
    \item Cooperation emerges through the evolutionary process: the high scoring
        strategies do not exhibit extortionate behaviour towards each other.
    \item Extortionate strategies do not survive the evolutionary process.
\end{itemize}

\begin{figure}[!htbp]
    \centering
    \includegraphics[width=.8\textwidth]{./assets/img/replicator_dynamics/main.pdf}
    \caption{Numerical simulation of the replicator equation
    (\ref{eqn:replicator_dynamics}): strategies are ordered by score, only the strategies with a high score survive the evolutionary process.}
    \label{fig:replicator_dynamics}
\end{figure}

This work can be used to classify plays of the IPD\@: data can be collected from
actual interactions (in lab or in the field). Furthermore, this allows for a
classification method similar to the notion of fingerprinting presented
in~\cite{Ashlock2008}. Trained strategies can potentially be classified as
extortionate or not or it could be possible to even constrain the reinforcement
learning approaches that are becoming prevalent in the literature.
Alternatively, this mathematical approach for recognising extortion could be
used in sophisticated strategies to defend against invasion. Arguably, some of
the strategies considered here exhibit this behaviour, indeed as described
in~\cite{Harper2017}, the top ranking strategies in the full tournament are
obtained using evolutionary reinforcement learning techniques, thus, suspicion
of extortionate behaviour could in fact be an evolutionary trait.

\section*{Acknowledgements}

The following open source software libraries were used in this research:

\begin{itemize}
    \item The Axelrod ~\cite{Knight2016, Knight2018} library (IPD strategies and
        tournaments).
    \item The sympy library~\cite{Meurer2017} (verification of all symbolic
        calculations).
    \item The matplotlib~\cite{Droettboom2018} library (visualisation).
    \item The pandas~\cite{Structures2010}, dask~\cite{Dask2016} and
        NumPy~\cite{Oliphant2015} libraries (data manipulation).
    \item The SciPy~\cite{Jones2001} library (numerical integration of the
        replicator equation).
\end{itemize}

This work was performed using the computational facilities of the Advanced
Research Computing @ Cardiff (ARCCA) Division, Cardiff University.

\printbibliography

\newpage
\section*{Supplementary materials}

\includepdf{assets/pdf/proof_of_form_of_extortionate_strategies/main.pdf}

\newpage

Using the pair wise interactions the transition rates \(p,
q\) can be measured and the steady state probabilities inferred and compared to
the actual probabilities of each state.
This is done numerically by computing the singular eigenvector of the
matrix \(A\) \cite{Stewart2009}:

\[
    A =
    \begin{bmatrix}
        p_1 q_1 & p_1 (1 - q_1) & (1 - p_1) q_1 & (1 -p_1) (1 - q_1) \\
        p_2 q_2 & p_2 (1 - q_2) & (1 - p_2) q_2 & (1 -p_2) (1 - q_2) \\
        p_3 q_3 & p_3 (1 - q_3) & (1 - p_3) q_3 & (1 -p_3) (1 - q_3) \\
        p_4 q_4 & p_4 (1 - q_4) & (1 - p_4) q_4 & (1 -p_4) (1 - q_4) \\
    \end{bmatrix}
\]

Figure~\ref{fig:computed_probabilities_vs_theoretic_probabilities} shows a
regression line fitted to every pairwise interaction with a reported
\(\text{SSError}\) value (pairwise interactions with missing states were
omitted). This serves to validate the approach: a part from some edge cases the
relationship is consistent.

\begin{figure}[!htbp]
    \centering
    \includegraphics[width=.8\textwidth]{./assets/img/computed_probabilities_vs_theoretic_probabilities/main.pdf}
    \caption{The
        relationship between the steady state probabilities inferred from the
        measured transitions and the actual steady state probabilities. A linear
        regression line is included validating the approach.}
    \label{fig:computed_probabilities_vs_theoretic_probabilities}
\end{figure}


\end{document}
 turns and every match has been
repeated \documentclass[a4paper]{article}

\usepackage{amsmath}
\usepackage{amssymb}
\usepackage[margin=1.5cm,
            includefoot,
            footskip=30pt]{geometry}
\usepackage{layout}
\usepackage{graphicx}
\usepackage{subcaption}

\usepackage{biblatex}
\usepackage{pdfpages}

\bibliography{main.bib}

\title{Suspicion: Recognising and evaluating the effectiveness
       of extortion in the Iterated Prisoner's Dilemma}
\author{Vincent A. Knight \and Nikoleta E. Glynatsi}
\date{\today}



\begin{document}

\maketitle

\begin{abstract}
    The Iterated Prisoner's Dilemma is a model for rational and evolutionary
    interactive behaviour. It has applications both in the study of human social
    behaviour as well as in biology.
    It is used to understand when and how a rational individual might
    accept an immediate cost to their own utility for the direct benefit of
    another.

    Much attention has been given to a class of strategies called
    Zero Determinant strategies. It has been theoretically shown that these
    strategies can ``extort'' any player.

    In this work, an approach to identify if observed strategies are playing in
    an extortionate way is described. Furthermore, experimental analysis of
    a large tournament with \input{assets/tex/number_of_full_strategies/main.tex}
    strategies is considered. In this setting
    the most highly performing strategies do not play in an extortionate way
    against each other but do against lower performing strategies.
    This suggests that whilst the theory of Zero Determinant strategies
    indicates that memory is not of fundamental importance to the evolution of
    cooperative behaviour, this is incomplete.
\end{abstract}

\section{Introduction}\label{sec:introduction}

Agent based game theoretic models have become a stalwart of the underpinning
mathematics of interactive behaviours. One of the major pieces of work
in this area is the pair of original computer tournaments run by Robert
Axelrod~\cite{Axelrod1980, Axelrod1980a}. These tournaments pitted submitted
computer strategies against each other in plays of the Iterated Prisoner's
Dilemma. A common game where agents can choose to pay a slight cost to their
immediate utility in the hope of building a reputation. This has been used in
economic and evolutionary game theory to understand the evolution of cooperative
behaviour.

Recently, a class of strategies was described in~\cite{Press2012} that can
provably extort any given opponent. In~\cite{Hilbe2013, Moran1707} some
questions have already been asked about the true effectiveness of these
strategies in an evolutionary setting. Here another question is asked: is it
possible to recognise this extortionate behaviour? A mathematical procedure for
suspicion is presented: in the same way that the continued actions of an
extortionate individual might raise suspicion.

This work makes use of the Axelrod Python library~\cite{Knight2018, Knight2016}
with a large number of Prisoner Dilemma strategies available to give an
extensive numerical example of the ideas presented.  The approach is presented
in Section~\ref{sec:delta-zd-strategies}.  All of the code and data discussed
in Section~\ref{sec:numerical-experiments} is open sourced, archived and
written according to best scientific principles~\cite{Wilson2014}. The data
archive can be found at~\cite{vincent_knight_2018_1297075}.

\section{Recognising Extortion}\label{sec:delta-zd-strategies}

In~\cite{Press2012}, given a match between 2 memory-one strategies, the concept
of Zero Determinant (ZD) strategies is introduced. The main result of that paper
shows that given two memory one players \(p, q\in\mathbb{R}^4\) a linear
relationship between the players' scores could be forced by one of the players.

Using the notation of~\cite{Press2012}, assuming the utilities for player \(p\)
are given by \(S_x=(R, S, T, P)\) and for player \(q\) by \(S_y=(R, T, S, P)\)
and that the stationary scores of each player is given by \(S_X\) and \(S_Y\)
respectively. The main result of~\cite{Press2012} is that if

\begin{equation}\label{eqn:linear_relationship_for_p}
    \tilde p=\alpha S_x + \beta S_y + \gamma
\end{equation}

or

\begin{equation}\label{eqn:linear_relationship_for_q}
    \tilde q=\alpha S_x + \beta S_y + \gamma
\end{equation}

where \(\tilde p = (1 - p_1, 1 - p_2, p_3, p_4)\) and
\(\tilde q = (1 - q_1, 1 - q_2, q_3, q_4)\) then:

\begin{equation}
    \alpha S_X + \beta S_Y + \gamma = 0
\end{equation}

In~\cite{Press2012} a particular type of ZD strategy is defined: extortionate
strategies. If:

\begin{equation}\label{eqn:constraint_for_extortion}
    \gamma = - P(\alpha + \beta)
\end{equation}

then the player can ensure they get a score \(\chi\) times
larger than the opponent. This extortion coefficient is given by:

\begin{equation}\label{eqn:definition_of_chi}
    \chi=\frac{-\beta}{\alpha}
\end{equation}

Thus, if (\ref{eqn:constraint_for_extortion}) holds and \(\chi >1\) a player is
said to extort their opponent.
Here, the reverse problem is considered: given a
\(p\in\mathbb{R}^4\) how does one identify \(\alpha, \beta\) if they
exist and is the strategy in fact acting in an extortionate way?

These conditions correspond to:

\begin{align}
    \tilde p_1 & = \alpha R + \beta R - P (\alpha + \beta)
            \label{eqn:condition_for_tilde_p1}\\
    \tilde p_2 & = \alpha S + \beta T - P (\alpha + \beta)
            \label{eqn:condition_for_tilde_p2}\\
    \tilde p_3 & = \alpha T + \beta S - P (\alpha + \beta)
            \label{eqn:condition_for_tilde_p3}\\
    \tilde p_4 & = \alpha P + \beta P - P (\alpha + \beta)
            \label{eqn:condition_for_tilde_p4}
\end{align}

Equation (\ref{eqn:condition_for_tilde_p4}) ensures that \(p_4=\tilde p_4=0\).
Equations (\ref{eqn:condition_for_tilde_p1}-\ref{eqn:condition_for_tilde_p3})
can be used to eliminate \(\alpha, \beta\), giving:

\begin{equation}\label{eqn:planar_definition_of_extortion}
    \tilde p_1 = \frac{(R - P)(\tilde p_2 + \tilde p_3)}{S + T - 2P}
\end{equation}

with:

\begin{equation}\label{eqn:definition_of_chi}
    \chi = \frac{\tilde p_2 (P - T) + \tilde p_3 (S - P)}
                {\tilde p_2 (P - S) + \tilde p_3 (T - P)}
\end{equation}

Given a strategy \(p\in\mathbb{R}^{4\times 1}\) equations
(\ref{eqn:condition_for_tilde_p4}), (\ref{eqn:planar_definition_of_extortion}-\ref{eqn:definition_of_chi}) can be used to check if
a strategy is extortionate. The conditions correspond to:

\begin{align}
    p_1 & = \frac{(R-P)(p_2 + p_3) - R + T + S - P}{S + T - 2P}
     \label{eqn:condition_for_p1}\\
    p_4 & = 0 \label{eqn:condition_for_p4}\\
    1 & > p_2 + p_3\label{eqn:condition_for_chi}
\end{align}

The algebraic steps necessary to prove these results are available in the
supporting materials.

All extortionate strategies reside on a triangular (\ref{eqn:condition_for_chi})
plane (\ref{eqn:condition_for_p1}) in 3 dimensions (\ref{eqn:condition_for_p4}).
Using this formulation it can be seen that a necessary (but not sufficient)
condition for an extortionate strategy is that it cooperates on average less
than 50\% of the time when in a state of disagreement with the opponent.

As an example, consider the known extortionate strategy \(p=(8 / 9, 1 / 2, 1 /
3, 0)\) from~\cite{Stewart2012} which is referred to as \texttt{Extort-2}. In
this case, for the standard values of \((R, T, S, P)\) constraint
(\ref{eqn:condition_for_p1}) corresponds to:

\begin{equation}
    p_1 = \frac{2(p_2 + p_3) + 1}{3}
\end{equation}

It is clear that in this case all constraints hold.

This approach could in fact be used to confirm that a given strategy is acting
in an extortionate manner even if it is not a memory one strategy. However, in
practice, if a closed form for \(p\) is not known, then due to measurement
and/or numerical error this would not work.

This problem can be written in the following linear algebraic form where
\(x=(\alpha, \beta)\)
and \(p^*=(\tilde p_1 - 1, tilde_2 - 1, p_3)\):

\begin{equation}\label{eqn:linear_algebraic_equation_for_p}
    Cx= p^*
\end{equation}

\(C\) corresponds to equations
(\ref{eqn:condition_for_tilde_p1}-\ref{eqn:condition_for_tilde_p3}) and is
given by:

\begin{equation}\label{eqn:definition_of_C}
    C =
    \begin{bmatrix}
        R - P & R- P \\
        S - P & T- P \\
        T - P & S- P \\
    \end{bmatrix}
\end{equation}

Note that in general, equation (\ref{eqn:linear_algebraic_equation_for_p}) will
not necessarily have a solution. From the Rouch\'{e}-Capelli theorem if there is
a solution it is unique as \(\text{rank}(C)=2\) which is the dimension of the
variable \(x\). The best fitting \(x\) is found by minimizing:

\begin{equation}\label{eqn:r_squared}
    \text{SSError} = \|C x- p^*\|_2^2 = \sum_{i=1}^{3}\left((C\bar x)_i-p_i^*\right)^2
\end{equation}

Note that \(\text{SSError}\), which is the square of the Frobenius
norm~\cite{Golub2013}, becomes a measure of how close a strategy is to being an
extortionate strategy. Suspicion
of extortion then corresponds to a threshold on \(\text{SSError}\).

By observing interactions (human or otherwise), their memory one representation
can be inferred and this approach can be used to recognise extortionate
behaviour. The notion of comparing theoretic and actual plays of the IPD is not
novel, see for example~\cite{Rand2013}. Immediately it is noted that if the
environment is noisy~\cite{Wu1995} then no strategy can be considered to be
extortionate as \(p_4>0\).

In the next section, this idea will be illustrated by observing the interactions
that take place in a computer based tournament of the IPD\@.

\section{Numerical experiments}\label{sec:numerical-experiments}

In~\cite{Stewart2012} results from a tournament with
\input{./assets/tex/number_of_stewart_plotkin_strategies/main.tex} strategies,
was presented with specific consideration given to ZD strategies. This
tournament is reproduced here using the Axelrod-Python
project~\cite{Knight2016}. To obtain a good measure of the corresponding
transition rates for each strategy all matches have been run for
\input{assets/tex/number_of_turns/main.tex} turns and every match has been
repeated \input{assets/tex/number_of_repetitions/main.tex} times. All of this
interaction data is available at~\cite{vincent_knight_2018_1297075}. A good
match between the inferred Markov chain and the state distribution of the actual
interactions has been verified. Data for this is presented in the supplementary
materials.

Figure~\ref{fig:SSError_overall_in_stewart_plotkin} shows the \(\text{SSError}\)
values for all the strategies in the tournament, as reported
in~\cite{Stewart2012} the extortionate strategy (which has an expected
\(\text{SSError}\) approximately 0) gains a large number of wins.

\begin{figure}[!htbp]
    \centering
    \includegraphics[width=.8\textwidth]{./assets/img/SSError_overall_in_stewart_plotkin/main.pdf}
    \caption{\(\text{SSError}\) and state probabilities for the strategies
        of~\cite{Stewart2012}, ordered both by number of wins and overall score.
        Note that \(P(DC)\) is not shown as it corresponds to the transpose of
        \(P(CD)\). Cooperator and Defector are omitted as they do not visit all
        the states.}
    \label{fig:SSError_overall_in_stewart_plotkin}
\end{figure}

Here, the work of~\cite{Stewart2012} is extended by investigating a tournament
with \input{assets/tex/number_of_full_strategies/main.tex}
strategies.

The results of this analysis are shown in
Figure~\ref{fig:SSError_and_probabilities_in_full}. The top ranking strategies
by number of wins seem to be extortionate (but not against all strategies) and
it can be seen that a small sub group of strategies achieve mutual defection.
All the top ranking strategies according to score achieve mutual cooperation and
do not extort each other, however they
\textbf{do} exhibit extortionate behaviour towards a number of the lower ranking
strategies.

\begin{figure}[!htbp]
    \centering
    \includegraphics[width=.8\textwidth]{./assets/img/SSError_and_probabilities_in_full/main.pdf}
    \caption{\(\text{SSError}\) for the strategies for the full tournament. Only
    strategy interactions for which \(p_4=0\) and \(\chi>1\) are displayed.}
    \label{fig:SSError_and_probabilities_in_full}
\end{figure}

\section{Conclusion}\label{sec:conclusion}

This work defines an approach to measure whether or not a player is playing a
strategy that corresponds to an extortionate strategy as defined
in~\cite{Press2012}: a mathematical model for suspicion. Indeed, all
extortionate strategies have been
 classified as lying on a triangular plane.
This rigorous classification fails to be robust to small measurement error, thus
a statistical approach is proposed.
This is done through a linear algebraic approach for approximating the solution
of a linear system. Using this, a large number of pairwise interactions is
simulated and in fact very few strategies are found to act extortionately.

The work of~\cite{Press2012}, whilst showing that a clever approach to taking
advantage of another memory one strategy exists: this is incomplete. Whilst the
elegance of this result is very attractive, just as the simplicity of the
victory of Tit For Tat in Axelrod's original tournaments was, it is incomplete.
Extortionate strategies achieve a high number of wins but they do not
achieve a high score which corresponds to the fitness landscape in an
evolutionary sense. From the large number of interactions a payoff matrix \(S\)
can be measured where \(S_{ij}\) denotes the score (using standard values of
\((R, S, T, P) = (3, 0, 5, 1)\)) of the \(i\)th strategy
against the \(j\)th strategy. Using this, the replicator equation
describes the evolution of the system based on a population density fitness
function:

\begin{equation}\label{eqn:replicator_dynamics}
    \frac{dx}{dt} = x(S-x^TS x)
\end{equation}

Equation (\ref{eqn:replicator_dynamics}) is solved numerically through an
integration technique described in~\cite{Petzold1983} and
Figure~\ref{fig:replicator_dynamics} shows the evolution of the distribution of
the system: the various strategies are ranked by scores. It is clear to see that
only the high ranking strategies survive the evolutionary process (in fact,
only \input{./assets/img/replicator_dynamics/main.tex}
have a final distribution greater than \(10 ^ {-2}\)). This confirms the
findings of~\cite{Moran1707} in which sophisticated strategies resist
evolutionary invasion of shorter memory strategies. Recalling
Figure~\ref{fig:SSError_and_probabilities_in_full} this demonstrates that:

\begin{itemize}
    \item Cooperation emerges through the evolutionary process: the high scoring
        strategies do not exhibit extortionate behaviour towards each other.
    \item Extortionate strategies do not survive the evolutionary process.
\end{itemize}

\begin{figure}[!htbp]
    \centering
    \includegraphics[width=.8\textwidth]{./assets/img/replicator_dynamics/main.pdf}
    \caption{Numerical simulation of the replicator equation
    (\ref{eqn:replicator_dynamics}): strategies are ordered by score, only the strategies with a high score survive the evolutionary process.}
    \label{fig:replicator_dynamics}
\end{figure}

This work can be used to classify plays of the IPD\@: data can be collected from
actual interactions (in lab or in the field). Furthermore, this allows for a
classification method similar to the notion of fingerprinting presented
in~\cite{Ashlock2008}. Trained strategies can potentially be classified as
extortionate or not or it could be possible to even constrain the reinforcement
learning approaches that are becoming prevalent in the literature.
Alternatively, this mathematical approach for recognising extortion could be
used in sophisticated strategies to defend against invasion. Arguably, some of
the strategies considered here exhibit this behaviour, indeed as described
in~\cite{Harper2017}, the top ranking strategies in the full tournament are
obtained using evolutionary reinforcement learning techniques, thus, suspicion
of extortionate behaviour could in fact be an evolutionary trait.

\section*{Acknowledgements}

The following open source software libraries were used in this research:

\begin{itemize}
    \item The Axelrod ~\cite{Knight2016, Knight2018} library (IPD strategies and
        tournaments).
    \item The sympy library~\cite{Meurer2017} (verification of all symbolic
        calculations).
    \item The matplotlib~\cite{Droettboom2018} library (visualisation).
    \item The pandas~\cite{Structures2010}, dask~\cite{Dask2016} and
        NumPy~\cite{Oliphant2015} libraries (data manipulation).
    \item The SciPy~\cite{Jones2001} library (numerical integration of the
        replicator equation).
\end{itemize}

This work was performed using the computational facilities of the Advanced
Research Computing @ Cardiff (ARCCA) Division, Cardiff University.

\printbibliography

\newpage
\section*{Supplementary materials}

\includepdf{assets/pdf/proof_of_form_of_extortionate_strategies/main.pdf}

\newpage

Using the pair wise interactions the transition rates \(p,
q\) can be measured and the steady state probabilities inferred and compared to
the actual probabilities of each state.
This is done numerically by computing the singular eigenvector of the
matrix \(A\) \cite{Stewart2009}:

\[
    A =
    \begin{bmatrix}
        p_1 q_1 & p_1 (1 - q_1) & (1 - p_1) q_1 & (1 -p_1) (1 - q_1) \\
        p_2 q_2 & p_2 (1 - q_2) & (1 - p_2) q_2 & (1 -p_2) (1 - q_2) \\
        p_3 q_3 & p_3 (1 - q_3) & (1 - p_3) q_3 & (1 -p_3) (1 - q_3) \\
        p_4 q_4 & p_4 (1 - q_4) & (1 - p_4) q_4 & (1 -p_4) (1 - q_4) \\
    \end{bmatrix}
\]

Figure~\ref{fig:computed_probabilities_vs_theoretic_probabilities} shows a
regression line fitted to every pairwise interaction with a reported
\(\text{SSError}\) value (pairwise interactions with missing states were
omitted). This serves to validate the approach: a part from some edge cases the
relationship is consistent.

\begin{figure}[!htbp]
    \centering
    \includegraphics[width=.8\textwidth]{./assets/img/computed_probabilities_vs_theoretic_probabilities/main.pdf}
    \caption{The
        relationship between the steady state probabilities inferred from the
        measured transitions and the actual steady state probabilities. A linear
        regression line is included validating the approach.}
    \label{fig:computed_probabilities_vs_theoretic_probabilities}
\end{figure}


\end{document}
 times. All of this
interaction data is available at~\cite{vincent_knight_2018_1297075}. A good
match between the inferred Markov chain and the state distribution of the actual
interactions has been verified. Data for this is presented in the supplementary
materials.

Figure~\ref{fig:SSError_overall_in_stewart_plotkin} shows the \(\text{SSError}\)
values for all the strategies in the tournament, as reported
in~\cite{Stewart2012} the extortionate strategy (which has an expected
\(\text{SSError}\) approximately 0) gains a large number of wins.

\begin{figure}[!htbp]
    \centering
    \includegraphics[width=.8\textwidth]{./assets/img/SSError_overall_in_stewart_plotkin/main.pdf}
    \caption{\(\text{SSError}\) and state probabilities for the strategies
        of~\cite{Stewart2012}, ordered both by number of wins and overall score.
        Note that \(P(DC)\) is not shown as it corresponds to the transpose of
        \(P(CD)\). Cooperator and Defector are omitted as they do not visit all
        the states.}
    \label{fig:SSError_overall_in_stewart_plotkin}
\end{figure}

Here, the work of~\cite{Stewart2012} is extended by investigating a tournament
with \documentclass[a4paper]{article}

\usepackage{amsmath}
\usepackage{amssymb}
\usepackage[margin=1.5cm,
            includefoot,
            footskip=30pt]{geometry}
\usepackage{layout}
\usepackage{graphicx}
\usepackage{subcaption}

\usepackage{biblatex}
\usepackage{pdfpages}

\bibliography{main.bib}

\title{Suspicion: Recognising and evaluating the effectiveness
       of extortion in the Iterated Prisoner's Dilemma}
\author{Vincent A. Knight \and Nikoleta E. Glynatsi}
\date{\today}



\begin{document}

\maketitle

\begin{abstract}
    The Iterated Prisoner's Dilemma is a model for rational and evolutionary
    interactive behaviour. It has applications both in the study of human social
    behaviour as well as in biology.
    It is used to understand when and how a rational individual might
    accept an immediate cost to their own utility for the direct benefit of
    another.

    Much attention has been given to a class of strategies called
    Zero Determinant strategies. It has been theoretically shown that these
    strategies can ``extort'' any player.

    In this work, an approach to identify if observed strategies are playing in
    an extortionate way is described. Furthermore, experimental analysis of
    a large tournament with \input{assets/tex/number_of_full_strategies/main.tex}
    strategies is considered. In this setting
    the most highly performing strategies do not play in an extortionate way
    against each other but do against lower performing strategies.
    This suggests that whilst the theory of Zero Determinant strategies
    indicates that memory is not of fundamental importance to the evolution of
    cooperative behaviour, this is incomplete.
\end{abstract}

\section{Introduction}\label{sec:introduction}

Agent based game theoretic models have become a stalwart of the underpinning
mathematics of interactive behaviours. One of the major pieces of work
in this area is the pair of original computer tournaments run by Robert
Axelrod~\cite{Axelrod1980, Axelrod1980a}. These tournaments pitted submitted
computer strategies against each other in plays of the Iterated Prisoner's
Dilemma. A common game where agents can choose to pay a slight cost to their
immediate utility in the hope of building a reputation. This has been used in
economic and evolutionary game theory to understand the evolution of cooperative
behaviour.

Recently, a class of strategies was described in~\cite{Press2012} that can
provably extort any given opponent. In~\cite{Hilbe2013, Moran1707} some
questions have already been asked about the true effectiveness of these
strategies in an evolutionary setting. Here another question is asked: is it
possible to recognise this extortionate behaviour? A mathematical procedure for
suspicion is presented: in the same way that the continued actions of an
extortionate individual might raise suspicion.

This work makes use of the Axelrod Python library~\cite{Knight2018, Knight2016}
with a large number of Prisoner Dilemma strategies available to give an
extensive numerical example of the ideas presented.  The approach is presented
in Section~\ref{sec:delta-zd-strategies}.  All of the code and data discussed
in Section~\ref{sec:numerical-experiments} is open sourced, archived and
written according to best scientific principles~\cite{Wilson2014}. The data
archive can be found at~\cite{vincent_knight_2018_1297075}.

\section{Recognising Extortion}\label{sec:delta-zd-strategies}

In~\cite{Press2012}, given a match between 2 memory-one strategies, the concept
of Zero Determinant (ZD) strategies is introduced. The main result of that paper
shows that given two memory one players \(p, q\in\mathbb{R}^4\) a linear
relationship between the players' scores could be forced by one of the players.

Using the notation of~\cite{Press2012}, assuming the utilities for player \(p\)
are given by \(S_x=(R, S, T, P)\) and for player \(q\) by \(S_y=(R, T, S, P)\)
and that the stationary scores of each player is given by \(S_X\) and \(S_Y\)
respectively. The main result of~\cite{Press2012} is that if

\begin{equation}\label{eqn:linear_relationship_for_p}
    \tilde p=\alpha S_x + \beta S_y + \gamma
\end{equation}

or

\begin{equation}\label{eqn:linear_relationship_for_q}
    \tilde q=\alpha S_x + \beta S_y + \gamma
\end{equation}

where \(\tilde p = (1 - p_1, 1 - p_2, p_3, p_4)\) and
\(\tilde q = (1 - q_1, 1 - q_2, q_3, q_4)\) then:

\begin{equation}
    \alpha S_X + \beta S_Y + \gamma = 0
\end{equation}

In~\cite{Press2012} a particular type of ZD strategy is defined: extortionate
strategies. If:

\begin{equation}\label{eqn:constraint_for_extortion}
    \gamma = - P(\alpha + \beta)
\end{equation}

then the player can ensure they get a score \(\chi\) times
larger than the opponent. This extortion coefficient is given by:

\begin{equation}\label{eqn:definition_of_chi}
    \chi=\frac{-\beta}{\alpha}
\end{equation}

Thus, if (\ref{eqn:constraint_for_extortion}) holds and \(\chi >1\) a player is
said to extort their opponent.
Here, the reverse problem is considered: given a
\(p\in\mathbb{R}^4\) how does one identify \(\alpha, \beta\) if they
exist and is the strategy in fact acting in an extortionate way?

These conditions correspond to:

\begin{align}
    \tilde p_1 & = \alpha R + \beta R - P (\alpha + \beta)
            \label{eqn:condition_for_tilde_p1}\\
    \tilde p_2 & = \alpha S + \beta T - P (\alpha + \beta)
            \label{eqn:condition_for_tilde_p2}\\
    \tilde p_3 & = \alpha T + \beta S - P (\alpha + \beta)
            \label{eqn:condition_for_tilde_p3}\\
    \tilde p_4 & = \alpha P + \beta P - P (\alpha + \beta)
            \label{eqn:condition_for_tilde_p4}
\end{align}

Equation (\ref{eqn:condition_for_tilde_p4}) ensures that \(p_4=\tilde p_4=0\).
Equations (\ref{eqn:condition_for_tilde_p1}-\ref{eqn:condition_for_tilde_p3})
can be used to eliminate \(\alpha, \beta\), giving:

\begin{equation}\label{eqn:planar_definition_of_extortion}
    \tilde p_1 = \frac{(R - P)(\tilde p_2 + \tilde p_3)}{S + T - 2P}
\end{equation}

with:

\begin{equation}\label{eqn:definition_of_chi}
    \chi = \frac{\tilde p_2 (P - T) + \tilde p_3 (S - P)}
                {\tilde p_2 (P - S) + \tilde p_3 (T - P)}
\end{equation}

Given a strategy \(p\in\mathbb{R}^{4\times 1}\) equations
(\ref{eqn:condition_for_tilde_p4}), (\ref{eqn:planar_definition_of_extortion}-\ref{eqn:definition_of_chi}) can be used to check if
a strategy is extortionate. The conditions correspond to:

\begin{align}
    p_1 & = \frac{(R-P)(p_2 + p_3) - R + T + S - P}{S + T - 2P}
     \label{eqn:condition_for_p1}\\
    p_4 & = 0 \label{eqn:condition_for_p4}\\
    1 & > p_2 + p_3\label{eqn:condition_for_chi}
\end{align}

The algebraic steps necessary to prove these results are available in the
supporting materials.

All extortionate strategies reside on a triangular (\ref{eqn:condition_for_chi})
plane (\ref{eqn:condition_for_p1}) in 3 dimensions (\ref{eqn:condition_for_p4}).
Using this formulation it can be seen that a necessary (but not sufficient)
condition for an extortionate strategy is that it cooperates on average less
than 50\% of the time when in a state of disagreement with the opponent.

As an example, consider the known extortionate strategy \(p=(8 / 9, 1 / 2, 1 /
3, 0)\) from~\cite{Stewart2012} which is referred to as \texttt{Extort-2}. In
this case, for the standard values of \((R, T, S, P)\) constraint
(\ref{eqn:condition_for_p1}) corresponds to:

\begin{equation}
    p_1 = \frac{2(p_2 + p_3) + 1}{3}
\end{equation}

It is clear that in this case all constraints hold.

This approach could in fact be used to confirm that a given strategy is acting
in an extortionate manner even if it is not a memory one strategy. However, in
practice, if a closed form for \(p\) is not known, then due to measurement
and/or numerical error this would not work.

This problem can be written in the following linear algebraic form where
\(x=(\alpha, \beta)\)
and \(p^*=(\tilde p_1 - 1, tilde_2 - 1, p_3)\):

\begin{equation}\label{eqn:linear_algebraic_equation_for_p}
    Cx= p^*
\end{equation}

\(C\) corresponds to equations
(\ref{eqn:condition_for_tilde_p1}-\ref{eqn:condition_for_tilde_p3}) and is
given by:

\begin{equation}\label{eqn:definition_of_C}
    C =
    \begin{bmatrix}
        R - P & R- P \\
        S - P & T- P \\
        T - P & S- P \\
    \end{bmatrix}
\end{equation}

Note that in general, equation (\ref{eqn:linear_algebraic_equation_for_p}) will
not necessarily have a solution. From the Rouch\'{e}-Capelli theorem if there is
a solution it is unique as \(\text{rank}(C)=2\) which is the dimension of the
variable \(x\). The best fitting \(x\) is found by minimizing:

\begin{equation}\label{eqn:r_squared}
    \text{SSError} = \|C x- p^*\|_2^2 = \sum_{i=1}^{3}\left((C\bar x)_i-p_i^*\right)^2
\end{equation}

Note that \(\text{SSError}\), which is the square of the Frobenius
norm~\cite{Golub2013}, becomes a measure of how close a strategy is to being an
extortionate strategy. Suspicion
of extortion then corresponds to a threshold on \(\text{SSError}\).

By observing interactions (human or otherwise), their memory one representation
can be inferred and this approach can be used to recognise extortionate
behaviour. The notion of comparing theoretic and actual plays of the IPD is not
novel, see for example~\cite{Rand2013}. Immediately it is noted that if the
environment is noisy~\cite{Wu1995} then no strategy can be considered to be
extortionate as \(p_4>0\).

In the next section, this idea will be illustrated by observing the interactions
that take place in a computer based tournament of the IPD\@.

\section{Numerical experiments}\label{sec:numerical-experiments}

In~\cite{Stewart2012} results from a tournament with
\input{./assets/tex/number_of_stewart_plotkin_strategies/main.tex} strategies,
was presented with specific consideration given to ZD strategies. This
tournament is reproduced here using the Axelrod-Python
project~\cite{Knight2016}. To obtain a good measure of the corresponding
transition rates for each strategy all matches have been run for
\input{assets/tex/number_of_turns/main.tex} turns and every match has been
repeated \input{assets/tex/number_of_repetitions/main.tex} times. All of this
interaction data is available at~\cite{vincent_knight_2018_1297075}. A good
match between the inferred Markov chain and the state distribution of the actual
interactions has been verified. Data for this is presented in the supplementary
materials.

Figure~\ref{fig:SSError_overall_in_stewart_plotkin} shows the \(\text{SSError}\)
values for all the strategies in the tournament, as reported
in~\cite{Stewart2012} the extortionate strategy (which has an expected
\(\text{SSError}\) approximately 0) gains a large number of wins.

\begin{figure}[!htbp]
    \centering
    \includegraphics[width=.8\textwidth]{./assets/img/SSError_overall_in_stewart_plotkin/main.pdf}
    \caption{\(\text{SSError}\) and state probabilities for the strategies
        of~\cite{Stewart2012}, ordered both by number of wins and overall score.
        Note that \(P(DC)\) is not shown as it corresponds to the transpose of
        \(P(CD)\). Cooperator and Defector are omitted as they do not visit all
        the states.}
    \label{fig:SSError_overall_in_stewart_plotkin}
\end{figure}

Here, the work of~\cite{Stewart2012} is extended by investigating a tournament
with \input{assets/tex/number_of_full_strategies/main.tex}
strategies.

The results of this analysis are shown in
Figure~\ref{fig:SSError_and_probabilities_in_full}. The top ranking strategies
by number of wins seem to be extortionate (but not against all strategies) and
it can be seen that a small sub group of strategies achieve mutual defection.
All the top ranking strategies according to score achieve mutual cooperation and
do not extort each other, however they
\textbf{do} exhibit extortionate behaviour towards a number of the lower ranking
strategies.

\begin{figure}[!htbp]
    \centering
    \includegraphics[width=.8\textwidth]{./assets/img/SSError_and_probabilities_in_full/main.pdf}
    \caption{\(\text{SSError}\) for the strategies for the full tournament. Only
    strategy interactions for which \(p_4=0\) and \(\chi>1\) are displayed.}
    \label{fig:SSError_and_probabilities_in_full}
\end{figure}

\section{Conclusion}\label{sec:conclusion}

This work defines an approach to measure whether or not a player is playing a
strategy that corresponds to an extortionate strategy as defined
in~\cite{Press2012}: a mathematical model for suspicion. Indeed, all
extortionate strategies have been
 classified as lying on a triangular plane.
This rigorous classification fails to be robust to small measurement error, thus
a statistical approach is proposed.
This is done through a linear algebraic approach for approximating the solution
of a linear system. Using this, a large number of pairwise interactions is
simulated and in fact very few strategies are found to act extortionately.

The work of~\cite{Press2012}, whilst showing that a clever approach to taking
advantage of another memory one strategy exists: this is incomplete. Whilst the
elegance of this result is very attractive, just as the simplicity of the
victory of Tit For Tat in Axelrod's original tournaments was, it is incomplete.
Extortionate strategies achieve a high number of wins but they do not
achieve a high score which corresponds to the fitness landscape in an
evolutionary sense. From the large number of interactions a payoff matrix \(S\)
can be measured where \(S_{ij}\) denotes the score (using standard values of
\((R, S, T, P) = (3, 0, 5, 1)\)) of the \(i\)th strategy
against the \(j\)th strategy. Using this, the replicator equation
describes the evolution of the system based on a population density fitness
function:

\begin{equation}\label{eqn:replicator_dynamics}
    \frac{dx}{dt} = x(S-x^TS x)
\end{equation}

Equation (\ref{eqn:replicator_dynamics}) is solved numerically through an
integration technique described in~\cite{Petzold1983} and
Figure~\ref{fig:replicator_dynamics} shows the evolution of the distribution of
the system: the various strategies are ranked by scores. It is clear to see that
only the high ranking strategies survive the evolutionary process (in fact,
only \input{./assets/img/replicator_dynamics/main.tex}
have a final distribution greater than \(10 ^ {-2}\)). This confirms the
findings of~\cite{Moran1707} in which sophisticated strategies resist
evolutionary invasion of shorter memory strategies. Recalling
Figure~\ref{fig:SSError_and_probabilities_in_full} this demonstrates that:

\begin{itemize}
    \item Cooperation emerges through the evolutionary process: the high scoring
        strategies do not exhibit extortionate behaviour towards each other.
    \item Extortionate strategies do not survive the evolutionary process.
\end{itemize}

\begin{figure}[!htbp]
    \centering
    \includegraphics[width=.8\textwidth]{./assets/img/replicator_dynamics/main.pdf}
    \caption{Numerical simulation of the replicator equation
    (\ref{eqn:replicator_dynamics}): strategies are ordered by score, only the strategies with a high score survive the evolutionary process.}
    \label{fig:replicator_dynamics}
\end{figure}

This work can be used to classify plays of the IPD\@: data can be collected from
actual interactions (in lab or in the field). Furthermore, this allows for a
classification method similar to the notion of fingerprinting presented
in~\cite{Ashlock2008}. Trained strategies can potentially be classified as
extortionate or not or it could be possible to even constrain the reinforcement
learning approaches that are becoming prevalent in the literature.
Alternatively, this mathematical approach for recognising extortion could be
used in sophisticated strategies to defend against invasion. Arguably, some of
the strategies considered here exhibit this behaviour, indeed as described
in~\cite{Harper2017}, the top ranking strategies in the full tournament are
obtained using evolutionary reinforcement learning techniques, thus, suspicion
of extortionate behaviour could in fact be an evolutionary trait.

\section*{Acknowledgements}

The following open source software libraries were used in this research:

\begin{itemize}
    \item The Axelrod ~\cite{Knight2016, Knight2018} library (IPD strategies and
        tournaments).
    \item The sympy library~\cite{Meurer2017} (verification of all symbolic
        calculations).
    \item The matplotlib~\cite{Droettboom2018} library (visualisation).
    \item The pandas~\cite{Structures2010}, dask~\cite{Dask2016} and
        NumPy~\cite{Oliphant2015} libraries (data manipulation).
    \item The SciPy~\cite{Jones2001} library (numerical integration of the
        replicator equation).
\end{itemize}

This work was performed using the computational facilities of the Advanced
Research Computing @ Cardiff (ARCCA) Division, Cardiff University.

\printbibliography

\newpage
\section*{Supplementary materials}

\includepdf{assets/pdf/proof_of_form_of_extortionate_strategies/main.pdf}

\newpage

Using the pair wise interactions the transition rates \(p,
q\) can be measured and the steady state probabilities inferred and compared to
the actual probabilities of each state.
This is done numerically by computing the singular eigenvector of the
matrix \(A\) \cite{Stewart2009}:

\[
    A =
    \begin{bmatrix}
        p_1 q_1 & p_1 (1 - q_1) & (1 - p_1) q_1 & (1 -p_1) (1 - q_1) \\
        p_2 q_2 & p_2 (1 - q_2) & (1 - p_2) q_2 & (1 -p_2) (1 - q_2) \\
        p_3 q_3 & p_3 (1 - q_3) & (1 - p_3) q_3 & (1 -p_3) (1 - q_3) \\
        p_4 q_4 & p_4 (1 - q_4) & (1 - p_4) q_4 & (1 -p_4) (1 - q_4) \\
    \end{bmatrix}
\]

Figure~\ref{fig:computed_probabilities_vs_theoretic_probabilities} shows a
regression line fitted to every pairwise interaction with a reported
\(\text{SSError}\) value (pairwise interactions with missing states were
omitted). This serves to validate the approach: a part from some edge cases the
relationship is consistent.

\begin{figure}[!htbp]
    \centering
    \includegraphics[width=.8\textwidth]{./assets/img/computed_probabilities_vs_theoretic_probabilities/main.pdf}
    \caption{The
        relationship between the steady state probabilities inferred from the
        measured transitions and the actual steady state probabilities. A linear
        regression line is included validating the approach.}
    \label{fig:computed_probabilities_vs_theoretic_probabilities}
\end{figure}


\end{document}

strategies.

The results of this analysis are shown in
Figure~\ref{fig:SSError_and_probabilities_in_full}. The top ranking strategies
by number of wins seem to be extortionate (but not against all strategies) and
it can be seen that a small sub group of strategies achieve mutual defection.
All the top ranking strategies according to score achieve mutual cooperation and
do not extort each other, however they
\textbf{do} exhibit extortionate behaviour towards a number of the lower ranking
strategies.

\begin{figure}[!htbp]
    \centering
    \includegraphics[width=.8\textwidth]{./assets/img/SSError_and_probabilities_in_full/main.pdf}
    \caption{\(\text{SSError}\) for the strategies for the full tournament. Only
    strategy interactions for which \(p_4=0\) and \(\chi>1\) are displayed.}
    \label{fig:SSError_and_probabilities_in_full}
\end{figure}

\section{Conclusion}\label{sec:conclusion}

This work defines an approach to measure whether or not a player is playing a
strategy that corresponds to an extortionate strategy as defined
in~\cite{Press2012}: a mathematical model for suspicion. Indeed, all
extortionate strategies have been
 classified as lying on a triangular plane.
This rigorous classification fails to be robust to small measurement error, thus
a statistical approach is proposed.
This is done through a linear algebraic approach for approximating the solution
of a linear system. Using this, a large number of pairwise interactions is
simulated and in fact very few strategies are found to act extortionately.

The work of~\cite{Press2012}, whilst showing that a clever approach to taking
advantage of another memory one strategy exists: this is incomplete. Whilst the
elegance of this result is very attractive, just as the simplicity of the
victory of Tit For Tat in Axelrod's original tournaments was, it is incomplete.
Extortionate strategies achieve a high number of wins but they do not
achieve a high score which corresponds to the fitness landscape in an
evolutionary sense. From the large number of interactions a payoff matrix \(S\)
can be measured where \(S_{ij}\) denotes the score (using standard values of
\((R, S, T, P) = (3, 0, 5, 1)\)) of the \(i\)th strategy
against the \(j\)th strategy. Using this, the replicator equation
describes the evolution of the system based on a population density fitness
function:

\begin{equation}\label{eqn:replicator_dynamics}
    \frac{dx}{dt} = x(S-x^TS x)
\end{equation}

Equation (\ref{eqn:replicator_dynamics}) is solved numerically through an
integration technique described in~\cite{Petzold1983} and
Figure~\ref{fig:replicator_dynamics} shows the evolution of the distribution of
the system: the various strategies are ranked by scores. It is clear to see that
only the high ranking strategies survive the evolutionary process (in fact,
only \documentclass[a4paper]{article}

\usepackage{amsmath}
\usepackage{amssymb}
\usepackage[margin=1.5cm,
            includefoot,
            footskip=30pt]{geometry}
\usepackage{layout}
\usepackage{graphicx}
\usepackage{subcaption}

\usepackage{biblatex}
\usepackage{pdfpages}

\bibliography{main.bib}

\title{Suspicion: Recognising and evaluating the effectiveness
       of extortion in the Iterated Prisoner's Dilemma}
\author{Vincent A. Knight \and Nikoleta E. Glynatsi}
\date{\today}



\begin{document}

\maketitle

\begin{abstract}
    The Iterated Prisoner's Dilemma is a model for rational and evolutionary
    interactive behaviour. It has applications both in the study of human social
    behaviour as well as in biology.
    It is used to understand when and how a rational individual might
    accept an immediate cost to their own utility for the direct benefit of
    another.

    Much attention has been given to a class of strategies called
    Zero Determinant strategies. It has been theoretically shown that these
    strategies can ``extort'' any player.

    In this work, an approach to identify if observed strategies are playing in
    an extortionate way is described. Furthermore, experimental analysis of
    a large tournament with \input{assets/tex/number_of_full_strategies/main.tex}
    strategies is considered. In this setting
    the most highly performing strategies do not play in an extortionate way
    against each other but do against lower performing strategies.
    This suggests that whilst the theory of Zero Determinant strategies
    indicates that memory is not of fundamental importance to the evolution of
    cooperative behaviour, this is incomplete.
\end{abstract}

\section{Introduction}\label{sec:introduction}

Agent based game theoretic models have become a stalwart of the underpinning
mathematics of interactive behaviours. One of the major pieces of work
in this area is the pair of original computer tournaments run by Robert
Axelrod~\cite{Axelrod1980, Axelrod1980a}. These tournaments pitted submitted
computer strategies against each other in plays of the Iterated Prisoner's
Dilemma. A common game where agents can choose to pay a slight cost to their
immediate utility in the hope of building a reputation. This has been used in
economic and evolutionary game theory to understand the evolution of cooperative
behaviour.

Recently, a class of strategies was described in~\cite{Press2012} that can
provably extort any given opponent. In~\cite{Hilbe2013, Moran1707} some
questions have already been asked about the true effectiveness of these
strategies in an evolutionary setting. Here another question is asked: is it
possible to recognise this extortionate behaviour? A mathematical procedure for
suspicion is presented: in the same way that the continued actions of an
extortionate individual might raise suspicion.

This work makes use of the Axelrod Python library~\cite{Knight2018, Knight2016}
with a large number of Prisoner Dilemma strategies available to give an
extensive numerical example of the ideas presented.  The approach is presented
in Section~\ref{sec:delta-zd-strategies}.  All of the code and data discussed
in Section~\ref{sec:numerical-experiments} is open sourced, archived and
written according to best scientific principles~\cite{Wilson2014}. The data
archive can be found at~\cite{vincent_knight_2018_1297075}.

\section{Recognising Extortion}\label{sec:delta-zd-strategies}

In~\cite{Press2012}, given a match between 2 memory-one strategies, the concept
of Zero Determinant (ZD) strategies is introduced. The main result of that paper
shows that given two memory one players \(p, q\in\mathbb{R}^4\) a linear
relationship between the players' scores could be forced by one of the players.

Using the notation of~\cite{Press2012}, assuming the utilities for player \(p\)
are given by \(S_x=(R, S, T, P)\) and for player \(q\) by \(S_y=(R, T, S, P)\)
and that the stationary scores of each player is given by \(S_X\) and \(S_Y\)
respectively. The main result of~\cite{Press2012} is that if

\begin{equation}\label{eqn:linear_relationship_for_p}
    \tilde p=\alpha S_x + \beta S_y + \gamma
\end{equation}

or

\begin{equation}\label{eqn:linear_relationship_for_q}
    \tilde q=\alpha S_x + \beta S_y + \gamma
\end{equation}

where \(\tilde p = (1 - p_1, 1 - p_2, p_3, p_4)\) and
\(\tilde q = (1 - q_1, 1 - q_2, q_3, q_4)\) then:

\begin{equation}
    \alpha S_X + \beta S_Y + \gamma = 0
\end{equation}

In~\cite{Press2012} a particular type of ZD strategy is defined: extortionate
strategies. If:

\begin{equation}\label{eqn:constraint_for_extortion}
    \gamma = - P(\alpha + \beta)
\end{equation}

then the player can ensure they get a score \(\chi\) times
larger than the opponent. This extortion coefficient is given by:

\begin{equation}\label{eqn:definition_of_chi}
    \chi=\frac{-\beta}{\alpha}
\end{equation}

Thus, if (\ref{eqn:constraint_for_extortion}) holds and \(\chi >1\) a player is
said to extort their opponent.
Here, the reverse problem is considered: given a
\(p\in\mathbb{R}^4\) how does one identify \(\alpha, \beta\) if they
exist and is the strategy in fact acting in an extortionate way?

These conditions correspond to:

\begin{align}
    \tilde p_1 & = \alpha R + \beta R - P (\alpha + \beta)
            \label{eqn:condition_for_tilde_p1}\\
    \tilde p_2 & = \alpha S + \beta T - P (\alpha + \beta)
            \label{eqn:condition_for_tilde_p2}\\
    \tilde p_3 & = \alpha T + \beta S - P (\alpha + \beta)
            \label{eqn:condition_for_tilde_p3}\\
    \tilde p_4 & = \alpha P + \beta P - P (\alpha + \beta)
            \label{eqn:condition_for_tilde_p4}
\end{align}

Equation (\ref{eqn:condition_for_tilde_p4}) ensures that \(p_4=\tilde p_4=0\).
Equations (\ref{eqn:condition_for_tilde_p1}-\ref{eqn:condition_for_tilde_p3})
can be used to eliminate \(\alpha, \beta\), giving:

\begin{equation}\label{eqn:planar_definition_of_extortion}
    \tilde p_1 = \frac{(R - P)(\tilde p_2 + \tilde p_3)}{S + T - 2P}
\end{equation}

with:

\begin{equation}\label{eqn:definition_of_chi}
    \chi = \frac{\tilde p_2 (P - T) + \tilde p_3 (S - P)}
                {\tilde p_2 (P - S) + \tilde p_3 (T - P)}
\end{equation}

Given a strategy \(p\in\mathbb{R}^{4\times 1}\) equations
(\ref{eqn:condition_for_tilde_p4}), (\ref{eqn:planar_definition_of_extortion}-\ref{eqn:definition_of_chi}) can be used to check if
a strategy is extortionate. The conditions correspond to:

\begin{align}
    p_1 & = \frac{(R-P)(p_2 + p_3) - R + T + S - P}{S + T - 2P}
     \label{eqn:condition_for_p1}\\
    p_4 & = 0 \label{eqn:condition_for_p4}\\
    1 & > p_2 + p_3\label{eqn:condition_for_chi}
\end{align}

The algebraic steps necessary to prove these results are available in the
supporting materials.

All extortionate strategies reside on a triangular (\ref{eqn:condition_for_chi})
plane (\ref{eqn:condition_for_p1}) in 3 dimensions (\ref{eqn:condition_for_p4}).
Using this formulation it can be seen that a necessary (but not sufficient)
condition for an extortionate strategy is that it cooperates on average less
than 50\% of the time when in a state of disagreement with the opponent.

As an example, consider the known extortionate strategy \(p=(8 / 9, 1 / 2, 1 /
3, 0)\) from~\cite{Stewart2012} which is referred to as \texttt{Extort-2}. In
this case, for the standard values of \((R, T, S, P)\) constraint
(\ref{eqn:condition_for_p1}) corresponds to:

\begin{equation}
    p_1 = \frac{2(p_2 + p_3) + 1}{3}
\end{equation}

It is clear that in this case all constraints hold.

This approach could in fact be used to confirm that a given strategy is acting
in an extortionate manner even if it is not a memory one strategy. However, in
practice, if a closed form for \(p\) is not known, then due to measurement
and/or numerical error this would not work.

This problem can be written in the following linear algebraic form where
\(x=(\alpha, \beta)\)
and \(p^*=(\tilde p_1 - 1, tilde_2 - 1, p_3)\):

\begin{equation}\label{eqn:linear_algebraic_equation_for_p}
    Cx= p^*
\end{equation}

\(C\) corresponds to equations
(\ref{eqn:condition_for_tilde_p1}-\ref{eqn:condition_for_tilde_p3}) and is
given by:

\begin{equation}\label{eqn:definition_of_C}
    C =
    \begin{bmatrix}
        R - P & R- P \\
        S - P & T- P \\
        T - P & S- P \\
    \end{bmatrix}
\end{equation}

Note that in general, equation (\ref{eqn:linear_algebraic_equation_for_p}) will
not necessarily have a solution. From the Rouch\'{e}-Capelli theorem if there is
a solution it is unique as \(\text{rank}(C)=2\) which is the dimension of the
variable \(x\). The best fitting \(x\) is found by minimizing:

\begin{equation}\label{eqn:r_squared}
    \text{SSError} = \|C x- p^*\|_2^2 = \sum_{i=1}^{3}\left((C\bar x)_i-p_i^*\right)^2
\end{equation}

Note that \(\text{SSError}\), which is the square of the Frobenius
norm~\cite{Golub2013}, becomes a measure of how close a strategy is to being an
extortionate strategy. Suspicion
of extortion then corresponds to a threshold on \(\text{SSError}\).

By observing interactions (human or otherwise), their memory one representation
can be inferred and this approach can be used to recognise extortionate
behaviour. The notion of comparing theoretic and actual plays of the IPD is not
novel, see for example~\cite{Rand2013}. Immediately it is noted that if the
environment is noisy~\cite{Wu1995} then no strategy can be considered to be
extortionate as \(p_4>0\).

In the next section, this idea will be illustrated by observing the interactions
that take place in a computer based tournament of the IPD\@.

\section{Numerical experiments}\label{sec:numerical-experiments}

In~\cite{Stewart2012} results from a tournament with
\input{./assets/tex/number_of_stewart_plotkin_strategies/main.tex} strategies,
was presented with specific consideration given to ZD strategies. This
tournament is reproduced here using the Axelrod-Python
project~\cite{Knight2016}. To obtain a good measure of the corresponding
transition rates for each strategy all matches have been run for
\input{assets/tex/number_of_turns/main.tex} turns and every match has been
repeated \input{assets/tex/number_of_repetitions/main.tex} times. All of this
interaction data is available at~\cite{vincent_knight_2018_1297075}. A good
match between the inferred Markov chain and the state distribution of the actual
interactions has been verified. Data for this is presented in the supplementary
materials.

Figure~\ref{fig:SSError_overall_in_stewart_plotkin} shows the \(\text{SSError}\)
values for all the strategies in the tournament, as reported
in~\cite{Stewart2012} the extortionate strategy (which has an expected
\(\text{SSError}\) approximately 0) gains a large number of wins.

\begin{figure}[!htbp]
    \centering
    \includegraphics[width=.8\textwidth]{./assets/img/SSError_overall_in_stewart_plotkin/main.pdf}
    \caption{\(\text{SSError}\) and state probabilities for the strategies
        of~\cite{Stewart2012}, ordered both by number of wins and overall score.
        Note that \(P(DC)\) is not shown as it corresponds to the transpose of
        \(P(CD)\). Cooperator and Defector are omitted as they do not visit all
        the states.}
    \label{fig:SSError_overall_in_stewart_plotkin}
\end{figure}

Here, the work of~\cite{Stewart2012} is extended by investigating a tournament
with \input{assets/tex/number_of_full_strategies/main.tex}
strategies.

The results of this analysis are shown in
Figure~\ref{fig:SSError_and_probabilities_in_full}. The top ranking strategies
by number of wins seem to be extortionate (but not against all strategies) and
it can be seen that a small sub group of strategies achieve mutual defection.
All the top ranking strategies according to score achieve mutual cooperation and
do not extort each other, however they
\textbf{do} exhibit extortionate behaviour towards a number of the lower ranking
strategies.

\begin{figure}[!htbp]
    \centering
    \includegraphics[width=.8\textwidth]{./assets/img/SSError_and_probabilities_in_full/main.pdf}
    \caption{\(\text{SSError}\) for the strategies for the full tournament. Only
    strategy interactions for which \(p_4=0\) and \(\chi>1\) are displayed.}
    \label{fig:SSError_and_probabilities_in_full}
\end{figure}

\section{Conclusion}\label{sec:conclusion}

This work defines an approach to measure whether or not a player is playing a
strategy that corresponds to an extortionate strategy as defined
in~\cite{Press2012}: a mathematical model for suspicion. Indeed, all
extortionate strategies have been
 classified as lying on a triangular plane.
This rigorous classification fails to be robust to small measurement error, thus
a statistical approach is proposed.
This is done through a linear algebraic approach for approximating the solution
of a linear system. Using this, a large number of pairwise interactions is
simulated and in fact very few strategies are found to act extortionately.

The work of~\cite{Press2012}, whilst showing that a clever approach to taking
advantage of another memory one strategy exists: this is incomplete. Whilst the
elegance of this result is very attractive, just as the simplicity of the
victory of Tit For Tat in Axelrod's original tournaments was, it is incomplete.
Extortionate strategies achieve a high number of wins but they do not
achieve a high score which corresponds to the fitness landscape in an
evolutionary sense. From the large number of interactions a payoff matrix \(S\)
can be measured where \(S_{ij}\) denotes the score (using standard values of
\((R, S, T, P) = (3, 0, 5, 1)\)) of the \(i\)th strategy
against the \(j\)th strategy. Using this, the replicator equation
describes the evolution of the system based on a population density fitness
function:

\begin{equation}\label{eqn:replicator_dynamics}
    \frac{dx}{dt} = x(S-x^TS x)
\end{equation}

Equation (\ref{eqn:replicator_dynamics}) is solved numerically through an
integration technique described in~\cite{Petzold1983} and
Figure~\ref{fig:replicator_dynamics} shows the evolution of the distribution of
the system: the various strategies are ranked by scores. It is clear to see that
only the high ranking strategies survive the evolutionary process (in fact,
only \input{./assets/img/replicator_dynamics/main.tex}
have a final distribution greater than \(10 ^ {-2}\)). This confirms the
findings of~\cite{Moran1707} in which sophisticated strategies resist
evolutionary invasion of shorter memory strategies. Recalling
Figure~\ref{fig:SSError_and_probabilities_in_full} this demonstrates that:

\begin{itemize}
    \item Cooperation emerges through the evolutionary process: the high scoring
        strategies do not exhibit extortionate behaviour towards each other.
    \item Extortionate strategies do not survive the evolutionary process.
\end{itemize}

\begin{figure}[!htbp]
    \centering
    \includegraphics[width=.8\textwidth]{./assets/img/replicator_dynamics/main.pdf}
    \caption{Numerical simulation of the replicator equation
    (\ref{eqn:replicator_dynamics}): strategies are ordered by score, only the strategies with a high score survive the evolutionary process.}
    \label{fig:replicator_dynamics}
\end{figure}

This work can be used to classify plays of the IPD\@: data can be collected from
actual interactions (in lab or in the field). Furthermore, this allows for a
classification method similar to the notion of fingerprinting presented
in~\cite{Ashlock2008}. Trained strategies can potentially be classified as
extortionate or not or it could be possible to even constrain the reinforcement
learning approaches that are becoming prevalent in the literature.
Alternatively, this mathematical approach for recognising extortion could be
used in sophisticated strategies to defend against invasion. Arguably, some of
the strategies considered here exhibit this behaviour, indeed as described
in~\cite{Harper2017}, the top ranking strategies in the full tournament are
obtained using evolutionary reinforcement learning techniques, thus, suspicion
of extortionate behaviour could in fact be an evolutionary trait.

\section*{Acknowledgements}

The following open source software libraries were used in this research:

\begin{itemize}
    \item The Axelrod ~\cite{Knight2016, Knight2018} library (IPD strategies and
        tournaments).
    \item The sympy library~\cite{Meurer2017} (verification of all symbolic
        calculations).
    \item The matplotlib~\cite{Droettboom2018} library (visualisation).
    \item The pandas~\cite{Structures2010}, dask~\cite{Dask2016} and
        NumPy~\cite{Oliphant2015} libraries (data manipulation).
    \item The SciPy~\cite{Jones2001} library (numerical integration of the
        replicator equation).
\end{itemize}

This work was performed using the computational facilities of the Advanced
Research Computing @ Cardiff (ARCCA) Division, Cardiff University.

\printbibliography

\newpage
\section*{Supplementary materials}

\includepdf{assets/pdf/proof_of_form_of_extortionate_strategies/main.pdf}

\newpage

Using the pair wise interactions the transition rates \(p,
q\) can be measured and the steady state probabilities inferred and compared to
the actual probabilities of each state.
This is done numerically by computing the singular eigenvector of the
matrix \(A\) \cite{Stewart2009}:

\[
    A =
    \begin{bmatrix}
        p_1 q_1 & p_1 (1 - q_1) & (1 - p_1) q_1 & (1 -p_1) (1 - q_1) \\
        p_2 q_2 & p_2 (1 - q_2) & (1 - p_2) q_2 & (1 -p_2) (1 - q_2) \\
        p_3 q_3 & p_3 (1 - q_3) & (1 - p_3) q_3 & (1 -p_3) (1 - q_3) \\
        p_4 q_4 & p_4 (1 - q_4) & (1 - p_4) q_4 & (1 -p_4) (1 - q_4) \\
    \end{bmatrix}
\]

Figure~\ref{fig:computed_probabilities_vs_theoretic_probabilities} shows a
regression line fitted to every pairwise interaction with a reported
\(\text{SSError}\) value (pairwise interactions with missing states were
omitted). This serves to validate the approach: a part from some edge cases the
relationship is consistent.

\begin{figure}[!htbp]
    \centering
    \includegraphics[width=.8\textwidth]{./assets/img/computed_probabilities_vs_theoretic_probabilities/main.pdf}
    \caption{The
        relationship between the steady state probabilities inferred from the
        measured transitions and the actual steady state probabilities. A linear
        regression line is included validating the approach.}
    \label{fig:computed_probabilities_vs_theoretic_probabilities}
\end{figure}


\end{document}

have a final distribution greater than \(10 ^ {-2}\)). This confirms the
findings of~\cite{Moran1707} in which sophisticated strategies resist
evolutionary invasion of shorter memory strategies. Recalling
Figure~\ref{fig:SSError_and_probabilities_in_full} this demonstrates that:

\begin{itemize}
    \item Cooperation emerges through the evolutionary process: the high scoring
        strategies do not exhibit extortionate behaviour towards each other.
    \item Extortionate strategies do not survive the evolutionary process.
\end{itemize}

\begin{figure}[!htbp]
    \centering
    \includegraphics[width=.8\textwidth]{./assets/img/replicator_dynamics/main.pdf}
    \caption{Numerical simulation of the replicator equation
    (\ref{eqn:replicator_dynamics}): strategies are ordered by score, only the strategies with a high score survive the evolutionary process.}
    \label{fig:replicator_dynamics}
\end{figure}

This work can be used to classify plays of the IPD\@: data can be collected from
actual interactions (in lab or in the field). Furthermore, this allows for a
classification method similar to the notion of fingerprinting presented
in~\cite{Ashlock2008}. Trained strategies can potentially be classified as
extortionate or not or it could be possible to even constrain the reinforcement
learning approaches that are becoming prevalent in the literature.
Alternatively, this mathematical approach for recognising extortion could be
used in sophisticated strategies to defend against invasion. Arguably, some of
the strategies considered here exhibit this behaviour, indeed as described
in~\cite{Harper2017}, the top ranking strategies in the full tournament are
obtained using evolutionary reinforcement learning techniques, thus, suspicion
of extortionate behaviour could in fact be an evolutionary trait.

\section*{Acknowledgements}

The following open source software libraries were used in this research:

\begin{itemize}
    \item The Axelrod ~\cite{Knight2016, Knight2018} library (IPD strategies and
        tournaments).
    \item The sympy library~\cite{Meurer2017} (verification of all symbolic
        calculations).
    \item The matplotlib~\cite{Droettboom2018} library (visualisation).
    \item The pandas~\cite{Structures2010}, dask~\cite{Dask2016} and
        NumPy~\cite{Oliphant2015} libraries (data manipulation).
    \item The SciPy~\cite{Jones2001} library (numerical integration of the
        replicator equation).
\end{itemize}

This work was performed using the computational facilities of the Advanced
Research Computing @ Cardiff (ARCCA) Division, Cardiff University.

\printbibliography

\newpage
\section*{Supplementary materials}

\includepdf{assets/pdf/proof_of_form_of_extortionate_strategies/main.pdf}

\newpage

Using the pair wise interactions the transition rates \(p,
q\) can be measured and the steady state probabilities inferred and compared to
the actual probabilities of each state.
This is done numerically by computing the singular eigenvector of the
matrix \(A\) \cite{Stewart2009}:

\[
    A =
    \begin{bmatrix}
        p_1 q_1 & p_1 (1 - q_1) & (1 - p_1) q_1 & (1 -p_1) (1 - q_1) \\
        p_2 q_2 & p_2 (1 - q_2) & (1 - p_2) q_2 & (1 -p_2) (1 - q_2) \\
        p_3 q_3 & p_3 (1 - q_3) & (1 - p_3) q_3 & (1 -p_3) (1 - q_3) \\
        p_4 q_4 & p_4 (1 - q_4) & (1 - p_4) q_4 & (1 -p_4) (1 - q_4) \\
    \end{bmatrix}
\]

Figure~\ref{fig:computed_probabilities_vs_theoretic_probabilities} shows a
regression line fitted to every pairwise interaction with a reported
\(\text{SSError}\) value (pairwise interactions with missing states were
omitted). This serves to validate the approach: a part from some edge cases the
relationship is consistent.

\begin{figure}[!htbp]
    \centering
    \includegraphics[width=.8\textwidth]{./assets/img/computed_probabilities_vs_theoretic_probabilities/main.pdf}
    \caption{The
        relationship between the steady state probabilities inferred from the
        measured transitions and the actual steady state probabilities. A linear
        regression line is included validating the approach.}
    \label{fig:computed_probabilities_vs_theoretic_probabilities}
\end{figure}


\end{document}
 strategies,
was presented with specific consideration given to ZD strategies. This
tournament is reproduced here using the Axelrod-Python
project~\cite{Knight2016}. To obtain a good measure of the corresponding
transition rates for each strategy all matches have been run for
\documentclass[a4paper]{article}

\usepackage{amsmath}
\usepackage{amssymb}
\usepackage[margin=1.5cm,
            includefoot,
            footskip=30pt]{geometry}
\usepackage{layout}
\usepackage{graphicx}
\usepackage{subcaption}

\usepackage{biblatex}
\usepackage{pdfpages}

\bibliography{main.bib}

\title{Suspicion: Recognising and evaluating the effectiveness
       of extortion in the Iterated Prisoner's Dilemma}
\author{Vincent A. Knight \and Nikoleta E. Glynatsi}
\date{\today}



\begin{document}

\maketitle

\begin{abstract}
    The Iterated Prisoner's Dilemma is a model for rational and evolutionary
    interactive behaviour. It has applications both in the study of human social
    behaviour as well as in biology.
    It is used to understand when and how a rational individual might
    accept an immediate cost to their own utility for the direct benefit of
    another.

    Much attention has been given to a class of strategies called
    Zero Determinant strategies. It has been theoretically shown that these
    strategies can ``extort'' any player.

    In this work, an approach to identify if observed strategies are playing in
    an extortionate way is described. Furthermore, experimental analysis of
    a large tournament with \documentclass[a4paper]{article}

\usepackage{amsmath}
\usepackage{amssymb}
\usepackage[margin=1.5cm,
            includefoot,
            footskip=30pt]{geometry}
\usepackage{layout}
\usepackage{graphicx}
\usepackage{subcaption}

\usepackage{biblatex}
\usepackage{pdfpages}

\bibliography{main.bib}

\title{Suspicion: Recognising and evaluating the effectiveness
       of extortion in the Iterated Prisoner's Dilemma}
\author{Vincent A. Knight \and Nikoleta E. Glynatsi}
\date{\today}



\begin{document}

\maketitle

\begin{abstract}
    The Iterated Prisoner's Dilemma is a model for rational and evolutionary
    interactive behaviour. It has applications both in the study of human social
    behaviour as well as in biology.
    It is used to understand when and how a rational individual might
    accept an immediate cost to their own utility for the direct benefit of
    another.

    Much attention has been given to a class of strategies called
    Zero Determinant strategies. It has been theoretically shown that these
    strategies can ``extort'' any player.

    In this work, an approach to identify if observed strategies are playing in
    an extortionate way is described. Furthermore, experimental analysis of
    a large tournament with \input{assets/tex/number_of_full_strategies/main.tex}
    strategies is considered. In this setting
    the most highly performing strategies do not play in an extortionate way
    against each other but do against lower performing strategies.
    This suggests that whilst the theory of Zero Determinant strategies
    indicates that memory is not of fundamental importance to the evolution of
    cooperative behaviour, this is incomplete.
\end{abstract}

\section{Introduction}\label{sec:introduction}

Agent based game theoretic models have become a stalwart of the underpinning
mathematics of interactive behaviours. One of the major pieces of work
in this area is the pair of original computer tournaments run by Robert
Axelrod~\cite{Axelrod1980, Axelrod1980a}. These tournaments pitted submitted
computer strategies against each other in plays of the Iterated Prisoner's
Dilemma. A common game where agents can choose to pay a slight cost to their
immediate utility in the hope of building a reputation. This has been used in
economic and evolutionary game theory to understand the evolution of cooperative
behaviour.

Recently, a class of strategies was described in~\cite{Press2012} that can
provably extort any given opponent. In~\cite{Hilbe2013, Moran1707} some
questions have already been asked about the true effectiveness of these
strategies in an evolutionary setting. Here another question is asked: is it
possible to recognise this extortionate behaviour? A mathematical procedure for
suspicion is presented: in the same way that the continued actions of an
extortionate individual might raise suspicion.

This work makes use of the Axelrod Python library~\cite{Knight2018, Knight2016}
with a large number of Prisoner Dilemma strategies available to give an
extensive numerical example of the ideas presented.  The approach is presented
in Section~\ref{sec:delta-zd-strategies}.  All of the code and data discussed
in Section~\ref{sec:numerical-experiments} is open sourced, archived and
written according to best scientific principles~\cite{Wilson2014}. The data
archive can be found at~\cite{vincent_knight_2018_1297075}.

\section{Recognising Extortion}\label{sec:delta-zd-strategies}

In~\cite{Press2012}, given a match between 2 memory-one strategies, the concept
of Zero Determinant (ZD) strategies is introduced. The main result of that paper
shows that given two memory one players \(p, q\in\mathbb{R}^4\) a linear
relationship between the players' scores could be forced by one of the players.

Using the notation of~\cite{Press2012}, assuming the utilities for player \(p\)
are given by \(S_x=(R, S, T, P)\) and for player \(q\) by \(S_y=(R, T, S, P)\)
and that the stationary scores of each player is given by \(S_X\) and \(S_Y\)
respectively. The main result of~\cite{Press2012} is that if

\begin{equation}\label{eqn:linear_relationship_for_p}
    \tilde p=\alpha S_x + \beta S_y + \gamma
\end{equation}

or

\begin{equation}\label{eqn:linear_relationship_for_q}
    \tilde q=\alpha S_x + \beta S_y + \gamma
\end{equation}

where \(\tilde p = (1 - p_1, 1 - p_2, p_3, p_4)\) and
\(\tilde q = (1 - q_1, 1 - q_2, q_3, q_4)\) then:

\begin{equation}
    \alpha S_X + \beta S_Y + \gamma = 0
\end{equation}

In~\cite{Press2012} a particular type of ZD strategy is defined: extortionate
strategies. If:

\begin{equation}\label{eqn:constraint_for_extortion}
    \gamma = - P(\alpha + \beta)
\end{equation}

then the player can ensure they get a score \(\chi\) times
larger than the opponent. This extortion coefficient is given by:

\begin{equation}\label{eqn:definition_of_chi}
    \chi=\frac{-\beta}{\alpha}
\end{equation}

Thus, if (\ref{eqn:constraint_for_extortion}) holds and \(\chi >1\) a player is
said to extort their opponent.
Here, the reverse problem is considered: given a
\(p\in\mathbb{R}^4\) how does one identify \(\alpha, \beta\) if they
exist and is the strategy in fact acting in an extortionate way?

These conditions correspond to:

\begin{align}
    \tilde p_1 & = \alpha R + \beta R - P (\alpha + \beta)
            \label{eqn:condition_for_tilde_p1}\\
    \tilde p_2 & = \alpha S + \beta T - P (\alpha + \beta)
            \label{eqn:condition_for_tilde_p2}\\
    \tilde p_3 & = \alpha T + \beta S - P (\alpha + \beta)
            \label{eqn:condition_for_tilde_p3}\\
    \tilde p_4 & = \alpha P + \beta P - P (\alpha + \beta)
            \label{eqn:condition_for_tilde_p4}
\end{align}

Equation (\ref{eqn:condition_for_tilde_p4}) ensures that \(p_4=\tilde p_4=0\).
Equations (\ref{eqn:condition_for_tilde_p1}-\ref{eqn:condition_for_tilde_p3})
can be used to eliminate \(\alpha, \beta\), giving:

\begin{equation}\label{eqn:planar_definition_of_extortion}
    \tilde p_1 = \frac{(R - P)(\tilde p_2 + \tilde p_3)}{S + T - 2P}
\end{equation}

with:

\begin{equation}\label{eqn:definition_of_chi}
    \chi = \frac{\tilde p_2 (P - T) + \tilde p_3 (S - P)}
                {\tilde p_2 (P - S) + \tilde p_3 (T - P)}
\end{equation}

Given a strategy \(p\in\mathbb{R}^{4\times 1}\) equations
(\ref{eqn:condition_for_tilde_p4}), (\ref{eqn:planar_definition_of_extortion}-\ref{eqn:definition_of_chi}) can be used to check if
a strategy is extortionate. The conditions correspond to:

\begin{align}
    p_1 & = \frac{(R-P)(p_2 + p_3) - R + T + S - P}{S + T - 2P}
     \label{eqn:condition_for_p1}\\
    p_4 & = 0 \label{eqn:condition_for_p4}\\
    1 & > p_2 + p_3\label{eqn:condition_for_chi}
\end{align}

The algebraic steps necessary to prove these results are available in the
supporting materials.

All extortionate strategies reside on a triangular (\ref{eqn:condition_for_chi})
plane (\ref{eqn:condition_for_p1}) in 3 dimensions (\ref{eqn:condition_for_p4}).
Using this formulation it can be seen that a necessary (but not sufficient)
condition for an extortionate strategy is that it cooperates on average less
than 50\% of the time when in a state of disagreement with the opponent.

As an example, consider the known extortionate strategy \(p=(8 / 9, 1 / 2, 1 /
3, 0)\) from~\cite{Stewart2012} which is referred to as \texttt{Extort-2}. In
this case, for the standard values of \((R, T, S, P)\) constraint
(\ref{eqn:condition_for_p1}) corresponds to:

\begin{equation}
    p_1 = \frac{2(p_2 + p_3) + 1}{3}
\end{equation}

It is clear that in this case all constraints hold.

This approach could in fact be used to confirm that a given strategy is acting
in an extortionate manner even if it is not a memory one strategy. However, in
practice, if a closed form for \(p\) is not known, then due to measurement
and/or numerical error this would not work.

This problem can be written in the following linear algebraic form where
\(x=(\alpha, \beta)\)
and \(p^*=(\tilde p_1 - 1, tilde_2 - 1, p_3)\):

\begin{equation}\label{eqn:linear_algebraic_equation_for_p}
    Cx= p^*
\end{equation}

\(C\) corresponds to equations
(\ref{eqn:condition_for_tilde_p1}-\ref{eqn:condition_for_tilde_p3}) and is
given by:

\begin{equation}\label{eqn:definition_of_C}
    C =
    \begin{bmatrix}
        R - P & R- P \\
        S - P & T- P \\
        T - P & S- P \\
    \end{bmatrix}
\end{equation}

Note that in general, equation (\ref{eqn:linear_algebraic_equation_for_p}) will
not necessarily have a solution. From the Rouch\'{e}-Capelli theorem if there is
a solution it is unique as \(\text{rank}(C)=2\) which is the dimension of the
variable \(x\). The best fitting \(x\) is found by minimizing:

\begin{equation}\label{eqn:r_squared}
    \text{SSError} = \|C x- p^*\|_2^2 = \sum_{i=1}^{3}\left((C\bar x)_i-p_i^*\right)^2
\end{equation}

Note that \(\text{SSError}\), which is the square of the Frobenius
norm~\cite{Golub2013}, becomes a measure of how close a strategy is to being an
extortionate strategy. Suspicion
of extortion then corresponds to a threshold on \(\text{SSError}\).

By observing interactions (human or otherwise), their memory one representation
can be inferred and this approach can be used to recognise extortionate
behaviour. The notion of comparing theoretic and actual plays of the IPD is not
novel, see for example~\cite{Rand2013}. Immediately it is noted that if the
environment is noisy~\cite{Wu1995} then no strategy can be considered to be
extortionate as \(p_4>0\).

In the next section, this idea will be illustrated by observing the interactions
that take place in a computer based tournament of the IPD\@.

\section{Numerical experiments}\label{sec:numerical-experiments}

In~\cite{Stewart2012} results from a tournament with
\input{./assets/tex/number_of_stewart_plotkin_strategies/main.tex} strategies,
was presented with specific consideration given to ZD strategies. This
tournament is reproduced here using the Axelrod-Python
project~\cite{Knight2016}. To obtain a good measure of the corresponding
transition rates for each strategy all matches have been run for
\input{assets/tex/number_of_turns/main.tex} turns and every match has been
repeated \input{assets/tex/number_of_repetitions/main.tex} times. All of this
interaction data is available at~\cite{vincent_knight_2018_1297075}. A good
match between the inferred Markov chain and the state distribution of the actual
interactions has been verified. Data for this is presented in the supplementary
materials.

Figure~\ref{fig:SSError_overall_in_stewart_plotkin} shows the \(\text{SSError}\)
values for all the strategies in the tournament, as reported
in~\cite{Stewart2012} the extortionate strategy (which has an expected
\(\text{SSError}\) approximately 0) gains a large number of wins.

\begin{figure}[!htbp]
    \centering
    \includegraphics[width=.8\textwidth]{./assets/img/SSError_overall_in_stewart_plotkin/main.pdf}
    \caption{\(\text{SSError}\) and state probabilities for the strategies
        of~\cite{Stewart2012}, ordered both by number of wins and overall score.
        Note that \(P(DC)\) is not shown as it corresponds to the transpose of
        \(P(CD)\). Cooperator and Defector are omitted as they do not visit all
        the states.}
    \label{fig:SSError_overall_in_stewart_plotkin}
\end{figure}

Here, the work of~\cite{Stewart2012} is extended by investigating a tournament
with \input{assets/tex/number_of_full_strategies/main.tex}
strategies.

The results of this analysis are shown in
Figure~\ref{fig:SSError_and_probabilities_in_full}. The top ranking strategies
by number of wins seem to be extortionate (but not against all strategies) and
it can be seen that a small sub group of strategies achieve mutual defection.
All the top ranking strategies according to score achieve mutual cooperation and
do not extort each other, however they
\textbf{do} exhibit extortionate behaviour towards a number of the lower ranking
strategies.

\begin{figure}[!htbp]
    \centering
    \includegraphics[width=.8\textwidth]{./assets/img/SSError_and_probabilities_in_full/main.pdf}
    \caption{\(\text{SSError}\) for the strategies for the full tournament. Only
    strategy interactions for which \(p_4=0\) and \(\chi>1\) are displayed.}
    \label{fig:SSError_and_probabilities_in_full}
\end{figure}

\section{Conclusion}\label{sec:conclusion}

This work defines an approach to measure whether or not a player is playing a
strategy that corresponds to an extortionate strategy as defined
in~\cite{Press2012}: a mathematical model for suspicion. Indeed, all
extortionate strategies have been
 classified as lying on a triangular plane.
This rigorous classification fails to be robust to small measurement error, thus
a statistical approach is proposed.
This is done through a linear algebraic approach for approximating the solution
of a linear system. Using this, a large number of pairwise interactions is
simulated and in fact very few strategies are found to act extortionately.

The work of~\cite{Press2012}, whilst showing that a clever approach to taking
advantage of another memory one strategy exists: this is incomplete. Whilst the
elegance of this result is very attractive, just as the simplicity of the
victory of Tit For Tat in Axelrod's original tournaments was, it is incomplete.
Extortionate strategies achieve a high number of wins but they do not
achieve a high score which corresponds to the fitness landscape in an
evolutionary sense. From the large number of interactions a payoff matrix \(S\)
can be measured where \(S_{ij}\) denotes the score (using standard values of
\((R, S, T, P) = (3, 0, 5, 1)\)) of the \(i\)th strategy
against the \(j\)th strategy. Using this, the replicator equation
describes the evolution of the system based on a population density fitness
function:

\begin{equation}\label{eqn:replicator_dynamics}
    \frac{dx}{dt} = x(S-x^TS x)
\end{equation}

Equation (\ref{eqn:replicator_dynamics}) is solved numerically through an
integration technique described in~\cite{Petzold1983} and
Figure~\ref{fig:replicator_dynamics} shows the evolution of the distribution of
the system: the various strategies are ranked by scores. It is clear to see that
only the high ranking strategies survive the evolutionary process (in fact,
only \input{./assets/img/replicator_dynamics/main.tex}
have a final distribution greater than \(10 ^ {-2}\)). This confirms the
findings of~\cite{Moran1707} in which sophisticated strategies resist
evolutionary invasion of shorter memory strategies. Recalling
Figure~\ref{fig:SSError_and_probabilities_in_full} this demonstrates that:

\begin{itemize}
    \item Cooperation emerges through the evolutionary process: the high scoring
        strategies do not exhibit extortionate behaviour towards each other.
    \item Extortionate strategies do not survive the evolutionary process.
\end{itemize}

\begin{figure}[!htbp]
    \centering
    \includegraphics[width=.8\textwidth]{./assets/img/replicator_dynamics/main.pdf}
    \caption{Numerical simulation of the replicator equation
    (\ref{eqn:replicator_dynamics}): strategies are ordered by score, only the strategies with a high score survive the evolutionary process.}
    \label{fig:replicator_dynamics}
\end{figure}

This work can be used to classify plays of the IPD\@: data can be collected from
actual interactions (in lab or in the field). Furthermore, this allows for a
classification method similar to the notion of fingerprinting presented
in~\cite{Ashlock2008}. Trained strategies can potentially be classified as
extortionate or not or it could be possible to even constrain the reinforcement
learning approaches that are becoming prevalent in the literature.
Alternatively, this mathematical approach for recognising extortion could be
used in sophisticated strategies to defend against invasion. Arguably, some of
the strategies considered here exhibit this behaviour, indeed as described
in~\cite{Harper2017}, the top ranking strategies in the full tournament are
obtained using evolutionary reinforcement learning techniques, thus, suspicion
of extortionate behaviour could in fact be an evolutionary trait.

\section*{Acknowledgements}

The following open source software libraries were used in this research:

\begin{itemize}
    \item The Axelrod ~\cite{Knight2016, Knight2018} library (IPD strategies and
        tournaments).
    \item The sympy library~\cite{Meurer2017} (verification of all symbolic
        calculations).
    \item The matplotlib~\cite{Droettboom2018} library (visualisation).
    \item The pandas~\cite{Structures2010}, dask~\cite{Dask2016} and
        NumPy~\cite{Oliphant2015} libraries (data manipulation).
    \item The SciPy~\cite{Jones2001} library (numerical integration of the
        replicator equation).
\end{itemize}

This work was performed using the computational facilities of the Advanced
Research Computing @ Cardiff (ARCCA) Division, Cardiff University.

\printbibliography

\newpage
\section*{Supplementary materials}

\includepdf{assets/pdf/proof_of_form_of_extortionate_strategies/main.pdf}

\newpage

Using the pair wise interactions the transition rates \(p,
q\) can be measured and the steady state probabilities inferred and compared to
the actual probabilities of each state.
This is done numerically by computing the singular eigenvector of the
matrix \(A\) \cite{Stewart2009}:

\[
    A =
    \begin{bmatrix}
        p_1 q_1 & p_1 (1 - q_1) & (1 - p_1) q_1 & (1 -p_1) (1 - q_1) \\
        p_2 q_2 & p_2 (1 - q_2) & (1 - p_2) q_2 & (1 -p_2) (1 - q_2) \\
        p_3 q_3 & p_3 (1 - q_3) & (1 - p_3) q_3 & (1 -p_3) (1 - q_3) \\
        p_4 q_4 & p_4 (1 - q_4) & (1 - p_4) q_4 & (1 -p_4) (1 - q_4) \\
    \end{bmatrix}
\]

Figure~\ref{fig:computed_probabilities_vs_theoretic_probabilities} shows a
regression line fitted to every pairwise interaction with a reported
\(\text{SSError}\) value (pairwise interactions with missing states were
omitted). This serves to validate the approach: a part from some edge cases the
relationship is consistent.

\begin{figure}[!htbp]
    \centering
    \includegraphics[width=.8\textwidth]{./assets/img/computed_probabilities_vs_theoretic_probabilities/main.pdf}
    \caption{The
        relationship between the steady state probabilities inferred from the
        measured transitions and the actual steady state probabilities. A linear
        regression line is included validating the approach.}
    \label{fig:computed_probabilities_vs_theoretic_probabilities}
\end{figure}


\end{document}

    strategies is considered. In this setting
    the most highly performing strategies do not play in an extortionate way
    against each other but do against lower performing strategies.
    This suggests that whilst the theory of Zero Determinant strategies
    indicates that memory is not of fundamental importance to the evolution of
    cooperative behaviour, this is incomplete.
\end{abstract}

\section{Introduction}\label{sec:introduction}

Agent based game theoretic models have become a stalwart of the underpinning
mathematics of interactive behaviours. One of the major pieces of work
in this area is the pair of original computer tournaments run by Robert
Axelrod~\cite{Axelrod1980, Axelrod1980a}. These tournaments pitted submitted
computer strategies against each other in plays of the Iterated Prisoner's
Dilemma. A common game where agents can choose to pay a slight cost to their
immediate utility in the hope of building a reputation. This has been used in
economic and evolutionary game theory to understand the evolution of cooperative
behaviour.

Recently, a class of strategies was described in~\cite{Press2012} that can
provably extort any given opponent. In~\cite{Hilbe2013, Moran1707} some
questions have already been asked about the true effectiveness of these
strategies in an evolutionary setting. Here another question is asked: is it
possible to recognise this extortionate behaviour? A mathematical procedure for
suspicion is presented: in the same way that the continued actions of an
extortionate individual might raise suspicion.

This work makes use of the Axelrod Python library~\cite{Knight2018, Knight2016}
with a large number of Prisoner Dilemma strategies available to give an
extensive numerical example of the ideas presented.  The approach is presented
in Section~\ref{sec:delta-zd-strategies}.  All of the code and data discussed
in Section~\ref{sec:numerical-experiments} is open sourced, archived and
written according to best scientific principles~\cite{Wilson2014}. The data
archive can be found at~\cite{vincent_knight_2018_1297075}.

\section{Recognising Extortion}\label{sec:delta-zd-strategies}

In~\cite{Press2012}, given a match between 2 memory-one strategies, the concept
of Zero Determinant (ZD) strategies is introduced. The main result of that paper
shows that given two memory one players \(p, q\in\mathbb{R}^4\) a linear
relationship between the players' scores could be forced by one of the players.

Using the notation of~\cite{Press2012}, assuming the utilities for player \(p\)
are given by \(S_x=(R, S, T, P)\) and for player \(q\) by \(S_y=(R, T, S, P)\)
and that the stationary scores of each player is given by \(S_X\) and \(S_Y\)
respectively. The main result of~\cite{Press2012} is that if

\begin{equation}\label{eqn:linear_relationship_for_p}
    \tilde p=\alpha S_x + \beta S_y + \gamma
\end{equation}

or

\begin{equation}\label{eqn:linear_relationship_for_q}
    \tilde q=\alpha S_x + \beta S_y + \gamma
\end{equation}

where \(\tilde p = (1 - p_1, 1 - p_2, p_3, p_4)\) and
\(\tilde q = (1 - q_1, 1 - q_2, q_3, q_4)\) then:

\begin{equation}
    \alpha S_X + \beta S_Y + \gamma = 0
\end{equation}

In~\cite{Press2012} a particular type of ZD strategy is defined: extortionate
strategies. If:

\begin{equation}\label{eqn:constraint_for_extortion}
    \gamma = - P(\alpha + \beta)
\end{equation}

then the player can ensure they get a score \(\chi\) times
larger than the opponent. This extortion coefficient is given by:

\begin{equation}\label{eqn:definition_of_chi}
    \chi=\frac{-\beta}{\alpha}
\end{equation}

Thus, if (\ref{eqn:constraint_for_extortion}) holds and \(\chi >1\) a player is
said to extort their opponent.
Here, the reverse problem is considered: given a
\(p\in\mathbb{R}^4\) how does one identify \(\alpha, \beta\) if they
exist and is the strategy in fact acting in an extortionate way?

These conditions correspond to:

\begin{align}
    \tilde p_1 & = \alpha R + \beta R - P (\alpha + \beta)
            \label{eqn:condition_for_tilde_p1}\\
    \tilde p_2 & = \alpha S + \beta T - P (\alpha + \beta)
            \label{eqn:condition_for_tilde_p2}\\
    \tilde p_3 & = \alpha T + \beta S - P (\alpha + \beta)
            \label{eqn:condition_for_tilde_p3}\\
    \tilde p_4 & = \alpha P + \beta P - P (\alpha + \beta)
            \label{eqn:condition_for_tilde_p4}
\end{align}

Equation (\ref{eqn:condition_for_tilde_p4}) ensures that \(p_4=\tilde p_4=0\).
Equations (\ref{eqn:condition_for_tilde_p1}-\ref{eqn:condition_for_tilde_p3})
can be used to eliminate \(\alpha, \beta\), giving:

\begin{equation}\label{eqn:planar_definition_of_extortion}
    \tilde p_1 = \frac{(R - P)(\tilde p_2 + \tilde p_3)}{S + T - 2P}
\end{equation}

with:

\begin{equation}\label{eqn:definition_of_chi}
    \chi = \frac{\tilde p_2 (P - T) + \tilde p_3 (S - P)}
                {\tilde p_2 (P - S) + \tilde p_3 (T - P)}
\end{equation}

Given a strategy \(p\in\mathbb{R}^{4\times 1}\) equations
(\ref{eqn:condition_for_tilde_p4}), (\ref{eqn:planar_definition_of_extortion}-\ref{eqn:definition_of_chi}) can be used to check if
a strategy is extortionate. The conditions correspond to:

\begin{align}
    p_1 & = \frac{(R-P)(p_2 + p_3) - R + T + S - P}{S + T - 2P}
     \label{eqn:condition_for_p1}\\
    p_4 & = 0 \label{eqn:condition_for_p4}\\
    1 & > p_2 + p_3\label{eqn:condition_for_chi}
\end{align}

The algebraic steps necessary to prove these results are available in the
supporting materials.

All extortionate strategies reside on a triangular (\ref{eqn:condition_for_chi})
plane (\ref{eqn:condition_for_p1}) in 3 dimensions (\ref{eqn:condition_for_p4}).
Using this formulation it can be seen that a necessary (but not sufficient)
condition for an extortionate strategy is that it cooperates on average less
than 50\% of the time when in a state of disagreement with the opponent.

As an example, consider the known extortionate strategy \(p=(8 / 9, 1 / 2, 1 /
3, 0)\) from~\cite{Stewart2012} which is referred to as \texttt{Extort-2}. In
this case, for the standard values of \((R, T, S, P)\) constraint
(\ref{eqn:condition_for_p1}) corresponds to:

\begin{equation}
    p_1 = \frac{2(p_2 + p_3) + 1}{3}
\end{equation}

It is clear that in this case all constraints hold.

This approach could in fact be used to confirm that a given strategy is acting
in an extortionate manner even if it is not a memory one strategy. However, in
practice, if a closed form for \(p\) is not known, then due to measurement
and/or numerical error this would not work.

This problem can be written in the following linear algebraic form where
\(x=(\alpha, \beta)\)
and \(p^*=(\tilde p_1 - 1, tilde_2 - 1, p_3)\):

\begin{equation}\label{eqn:linear_algebraic_equation_for_p}
    Cx= p^*
\end{equation}

\(C\) corresponds to equations
(\ref{eqn:condition_for_tilde_p1}-\ref{eqn:condition_for_tilde_p3}) and is
given by:

\begin{equation}\label{eqn:definition_of_C}
    C =
    \begin{bmatrix}
        R - P & R- P \\
        S - P & T- P \\
        T - P & S- P \\
    \end{bmatrix}
\end{equation}

Note that in general, equation (\ref{eqn:linear_algebraic_equation_for_p}) will
not necessarily have a solution. From the Rouch\'{e}-Capelli theorem if there is
a solution it is unique as \(\text{rank}(C)=2\) which is the dimension of the
variable \(x\). The best fitting \(x\) is found by minimizing:

\begin{equation}\label{eqn:r_squared}
    \text{SSError} = \|C x- p^*\|_2^2 = \sum_{i=1}^{3}\left((C\bar x)_i-p_i^*\right)^2
\end{equation}

Note that \(\text{SSError}\), which is the square of the Frobenius
norm~\cite{Golub2013}, becomes a measure of how close a strategy is to being an
extortionate strategy. Suspicion
of extortion then corresponds to a threshold on \(\text{SSError}\).

By observing interactions (human or otherwise), their memory one representation
can be inferred and this approach can be used to recognise extortionate
behaviour. The notion of comparing theoretic and actual plays of the IPD is not
novel, see for example~\cite{Rand2013}. Immediately it is noted that if the
environment is noisy~\cite{Wu1995} then no strategy can be considered to be
extortionate as \(p_4>0\).

In the next section, this idea will be illustrated by observing the interactions
that take place in a computer based tournament of the IPD\@.

\section{Numerical experiments}\label{sec:numerical-experiments}

In~\cite{Stewart2012} results from a tournament with
\documentclass[a4paper]{article}

\usepackage{amsmath}
\usepackage{amssymb}
\usepackage[margin=1.5cm,
            includefoot,
            footskip=30pt]{geometry}
\usepackage{layout}
\usepackage{graphicx}
\usepackage{subcaption}

\usepackage{biblatex}
\usepackage{pdfpages}

\bibliography{main.bib}

\title{Suspicion: Recognising and evaluating the effectiveness
       of extortion in the Iterated Prisoner's Dilemma}
\author{Vincent A. Knight \and Nikoleta E. Glynatsi}
\date{\today}



\begin{document}

\maketitle

\begin{abstract}
    The Iterated Prisoner's Dilemma is a model for rational and evolutionary
    interactive behaviour. It has applications both in the study of human social
    behaviour as well as in biology.
    It is used to understand when and how a rational individual might
    accept an immediate cost to their own utility for the direct benefit of
    another.

    Much attention has been given to a class of strategies called
    Zero Determinant strategies. It has been theoretically shown that these
    strategies can ``extort'' any player.

    In this work, an approach to identify if observed strategies are playing in
    an extortionate way is described. Furthermore, experimental analysis of
    a large tournament with \input{assets/tex/number_of_full_strategies/main.tex}
    strategies is considered. In this setting
    the most highly performing strategies do not play in an extortionate way
    against each other but do against lower performing strategies.
    This suggests that whilst the theory of Zero Determinant strategies
    indicates that memory is not of fundamental importance to the evolution of
    cooperative behaviour, this is incomplete.
\end{abstract}

\section{Introduction}\label{sec:introduction}

Agent based game theoretic models have become a stalwart of the underpinning
mathematics of interactive behaviours. One of the major pieces of work
in this area is the pair of original computer tournaments run by Robert
Axelrod~\cite{Axelrod1980, Axelrod1980a}. These tournaments pitted submitted
computer strategies against each other in plays of the Iterated Prisoner's
Dilemma. A common game where agents can choose to pay a slight cost to their
immediate utility in the hope of building a reputation. This has been used in
economic and evolutionary game theory to understand the evolution of cooperative
behaviour.

Recently, a class of strategies was described in~\cite{Press2012} that can
provably extort any given opponent. In~\cite{Hilbe2013, Moran1707} some
questions have already been asked about the true effectiveness of these
strategies in an evolutionary setting. Here another question is asked: is it
possible to recognise this extortionate behaviour? A mathematical procedure for
suspicion is presented: in the same way that the continued actions of an
extortionate individual might raise suspicion.

This work makes use of the Axelrod Python library~\cite{Knight2018, Knight2016}
with a large number of Prisoner Dilemma strategies available to give an
extensive numerical example of the ideas presented.  The approach is presented
in Section~\ref{sec:delta-zd-strategies}.  All of the code and data discussed
in Section~\ref{sec:numerical-experiments} is open sourced, archived and
written according to best scientific principles~\cite{Wilson2014}. The data
archive can be found at~\cite{vincent_knight_2018_1297075}.

\section{Recognising Extortion}\label{sec:delta-zd-strategies}

In~\cite{Press2012}, given a match between 2 memory-one strategies, the concept
of Zero Determinant (ZD) strategies is introduced. The main result of that paper
shows that given two memory one players \(p, q\in\mathbb{R}^4\) a linear
relationship between the players' scores could be forced by one of the players.

Using the notation of~\cite{Press2012}, assuming the utilities for player \(p\)
are given by \(S_x=(R, S, T, P)\) and for player \(q\) by \(S_y=(R, T, S, P)\)
and that the stationary scores of each player is given by \(S_X\) and \(S_Y\)
respectively. The main result of~\cite{Press2012} is that if

\begin{equation}\label{eqn:linear_relationship_for_p}
    \tilde p=\alpha S_x + \beta S_y + \gamma
\end{equation}

or

\begin{equation}\label{eqn:linear_relationship_for_q}
    \tilde q=\alpha S_x + \beta S_y + \gamma
\end{equation}

where \(\tilde p = (1 - p_1, 1 - p_2, p_3, p_4)\) and
\(\tilde q = (1 - q_1, 1 - q_2, q_3, q_4)\) then:

\begin{equation}
    \alpha S_X + \beta S_Y + \gamma = 0
\end{equation}

In~\cite{Press2012} a particular type of ZD strategy is defined: extortionate
strategies. If:

\begin{equation}\label{eqn:constraint_for_extortion}
    \gamma = - P(\alpha + \beta)
\end{equation}

then the player can ensure they get a score \(\chi\) times
larger than the opponent. This extortion coefficient is given by:

\begin{equation}\label{eqn:definition_of_chi}
    \chi=\frac{-\beta}{\alpha}
\end{equation}

Thus, if (\ref{eqn:constraint_for_extortion}) holds and \(\chi >1\) a player is
said to extort their opponent.
Here, the reverse problem is considered: given a
\(p\in\mathbb{R}^4\) how does one identify \(\alpha, \beta\) if they
exist and is the strategy in fact acting in an extortionate way?

These conditions correspond to:

\begin{align}
    \tilde p_1 & = \alpha R + \beta R - P (\alpha + \beta)
            \label{eqn:condition_for_tilde_p1}\\
    \tilde p_2 & = \alpha S + \beta T - P (\alpha + \beta)
            \label{eqn:condition_for_tilde_p2}\\
    \tilde p_3 & = \alpha T + \beta S - P (\alpha + \beta)
            \label{eqn:condition_for_tilde_p3}\\
    \tilde p_4 & = \alpha P + \beta P - P (\alpha + \beta)
            \label{eqn:condition_for_tilde_p4}
\end{align}

Equation (\ref{eqn:condition_for_tilde_p4}) ensures that \(p_4=\tilde p_4=0\).
Equations (\ref{eqn:condition_for_tilde_p1}-\ref{eqn:condition_for_tilde_p3})
can be used to eliminate \(\alpha, \beta\), giving:

\begin{equation}\label{eqn:planar_definition_of_extortion}
    \tilde p_1 = \frac{(R - P)(\tilde p_2 + \tilde p_3)}{S + T - 2P}
\end{equation}

with:

\begin{equation}\label{eqn:definition_of_chi}
    \chi = \frac{\tilde p_2 (P - T) + \tilde p_3 (S - P)}
                {\tilde p_2 (P - S) + \tilde p_3 (T - P)}
\end{equation}

Given a strategy \(p\in\mathbb{R}^{4\times 1}\) equations
(\ref{eqn:condition_for_tilde_p4}), (\ref{eqn:planar_definition_of_extortion}-\ref{eqn:definition_of_chi}) can be used to check if
a strategy is extortionate. The conditions correspond to:

\begin{align}
    p_1 & = \frac{(R-P)(p_2 + p_3) - R + T + S - P}{S + T - 2P}
     \label{eqn:condition_for_p1}\\
    p_4 & = 0 \label{eqn:condition_for_p4}\\
    1 & > p_2 + p_3\label{eqn:condition_for_chi}
\end{align}

The algebraic steps necessary to prove these results are available in the
supporting materials.

All extortionate strategies reside on a triangular (\ref{eqn:condition_for_chi})
plane (\ref{eqn:condition_for_p1}) in 3 dimensions (\ref{eqn:condition_for_p4}).
Using this formulation it can be seen that a necessary (but not sufficient)
condition for an extortionate strategy is that it cooperates on average less
than 50\% of the time when in a state of disagreement with the opponent.

As an example, consider the known extortionate strategy \(p=(8 / 9, 1 / 2, 1 /
3, 0)\) from~\cite{Stewart2012} which is referred to as \texttt{Extort-2}. In
this case, for the standard values of \((R, T, S, P)\) constraint
(\ref{eqn:condition_for_p1}) corresponds to:

\begin{equation}
    p_1 = \frac{2(p_2 + p_3) + 1}{3}
\end{equation}

It is clear that in this case all constraints hold.

This approach could in fact be used to confirm that a given strategy is acting
in an extortionate manner even if it is not a memory one strategy. However, in
practice, if a closed form for \(p\) is not known, then due to measurement
and/or numerical error this would not work.

This problem can be written in the following linear algebraic form where
\(x=(\alpha, \beta)\)
and \(p^*=(\tilde p_1 - 1, tilde_2 - 1, p_3)\):

\begin{equation}\label{eqn:linear_algebraic_equation_for_p}
    Cx= p^*
\end{equation}

\(C\) corresponds to equations
(\ref{eqn:condition_for_tilde_p1}-\ref{eqn:condition_for_tilde_p3}) and is
given by:

\begin{equation}\label{eqn:definition_of_C}
    C =
    \begin{bmatrix}
        R - P & R- P \\
        S - P & T- P \\
        T - P & S- P \\
    \end{bmatrix}
\end{equation}

Note that in general, equation (\ref{eqn:linear_algebraic_equation_for_p}) will
not necessarily have a solution. From the Rouch\'{e}-Capelli theorem if there is
a solution it is unique as \(\text{rank}(C)=2\) which is the dimension of the
variable \(x\). The best fitting \(x\) is found by minimizing:

\begin{equation}\label{eqn:r_squared}
    \text{SSError} = \|C x- p^*\|_2^2 = \sum_{i=1}^{3}\left((C\bar x)_i-p_i^*\right)^2
\end{equation}

Note that \(\text{SSError}\), which is the square of the Frobenius
norm~\cite{Golub2013}, becomes a measure of how close a strategy is to being an
extortionate strategy. Suspicion
of extortion then corresponds to a threshold on \(\text{SSError}\).

By observing interactions (human or otherwise), their memory one representation
can be inferred and this approach can be used to recognise extortionate
behaviour. The notion of comparing theoretic and actual plays of the IPD is not
novel, see for example~\cite{Rand2013}. Immediately it is noted that if the
environment is noisy~\cite{Wu1995} then no strategy can be considered to be
extortionate as \(p_4>0\).

In the next section, this idea will be illustrated by observing the interactions
that take place in a computer based tournament of the IPD\@.

\section{Numerical experiments}\label{sec:numerical-experiments}

In~\cite{Stewart2012} results from a tournament with
\input{./assets/tex/number_of_stewart_plotkin_strategies/main.tex} strategies,
was presented with specific consideration given to ZD strategies. This
tournament is reproduced here using the Axelrod-Python
project~\cite{Knight2016}. To obtain a good measure of the corresponding
transition rates for each strategy all matches have been run for
\input{assets/tex/number_of_turns/main.tex} turns and every match has been
repeated \input{assets/tex/number_of_repetitions/main.tex} times. All of this
interaction data is available at~\cite{vincent_knight_2018_1297075}. A good
match between the inferred Markov chain and the state distribution of the actual
interactions has been verified. Data for this is presented in the supplementary
materials.

Figure~\ref{fig:SSError_overall_in_stewart_plotkin} shows the \(\text{SSError}\)
values for all the strategies in the tournament, as reported
in~\cite{Stewart2012} the extortionate strategy (which has an expected
\(\text{SSError}\) approximately 0) gains a large number of wins.

\begin{figure}[!htbp]
    \centering
    \includegraphics[width=.8\textwidth]{./assets/img/SSError_overall_in_stewart_plotkin/main.pdf}
    \caption{\(\text{SSError}\) and state probabilities for the strategies
        of~\cite{Stewart2012}, ordered both by number of wins and overall score.
        Note that \(P(DC)\) is not shown as it corresponds to the transpose of
        \(P(CD)\). Cooperator and Defector are omitted as they do not visit all
        the states.}
    \label{fig:SSError_overall_in_stewart_plotkin}
\end{figure}

Here, the work of~\cite{Stewart2012} is extended by investigating a tournament
with \input{assets/tex/number_of_full_strategies/main.tex}
strategies.

The results of this analysis are shown in
Figure~\ref{fig:SSError_and_probabilities_in_full}. The top ranking strategies
by number of wins seem to be extortionate (but not against all strategies) and
it can be seen that a small sub group of strategies achieve mutual defection.
All the top ranking strategies according to score achieve mutual cooperation and
do not extort each other, however they
\textbf{do} exhibit extortionate behaviour towards a number of the lower ranking
strategies.

\begin{figure}[!htbp]
    \centering
    \includegraphics[width=.8\textwidth]{./assets/img/SSError_and_probabilities_in_full/main.pdf}
    \caption{\(\text{SSError}\) for the strategies for the full tournament. Only
    strategy interactions for which \(p_4=0\) and \(\chi>1\) are displayed.}
    \label{fig:SSError_and_probabilities_in_full}
\end{figure}

\section{Conclusion}\label{sec:conclusion}

This work defines an approach to measure whether or not a player is playing a
strategy that corresponds to an extortionate strategy as defined
in~\cite{Press2012}: a mathematical model for suspicion. Indeed, all
extortionate strategies have been
 classified as lying on a triangular plane.
This rigorous classification fails to be robust to small measurement error, thus
a statistical approach is proposed.
This is done through a linear algebraic approach for approximating the solution
of a linear system. Using this, a large number of pairwise interactions is
simulated and in fact very few strategies are found to act extortionately.

The work of~\cite{Press2012}, whilst showing that a clever approach to taking
advantage of another memory one strategy exists: this is incomplete. Whilst the
elegance of this result is very attractive, just as the simplicity of the
victory of Tit For Tat in Axelrod's original tournaments was, it is incomplete.
Extortionate strategies achieve a high number of wins but they do not
achieve a high score which corresponds to the fitness landscape in an
evolutionary sense. From the large number of interactions a payoff matrix \(S\)
can be measured where \(S_{ij}\) denotes the score (using standard values of
\((R, S, T, P) = (3, 0, 5, 1)\)) of the \(i\)th strategy
against the \(j\)th strategy. Using this, the replicator equation
describes the evolution of the system based on a population density fitness
function:

\begin{equation}\label{eqn:replicator_dynamics}
    \frac{dx}{dt} = x(S-x^TS x)
\end{equation}

Equation (\ref{eqn:replicator_dynamics}) is solved numerically through an
integration technique described in~\cite{Petzold1983} and
Figure~\ref{fig:replicator_dynamics} shows the evolution of the distribution of
the system: the various strategies are ranked by scores. It is clear to see that
only the high ranking strategies survive the evolutionary process (in fact,
only \input{./assets/img/replicator_dynamics/main.tex}
have a final distribution greater than \(10 ^ {-2}\)). This confirms the
findings of~\cite{Moran1707} in which sophisticated strategies resist
evolutionary invasion of shorter memory strategies. Recalling
Figure~\ref{fig:SSError_and_probabilities_in_full} this demonstrates that:

\begin{itemize}
    \item Cooperation emerges through the evolutionary process: the high scoring
        strategies do not exhibit extortionate behaviour towards each other.
    \item Extortionate strategies do not survive the evolutionary process.
\end{itemize}

\begin{figure}[!htbp]
    \centering
    \includegraphics[width=.8\textwidth]{./assets/img/replicator_dynamics/main.pdf}
    \caption{Numerical simulation of the replicator equation
    (\ref{eqn:replicator_dynamics}): strategies are ordered by score, only the strategies with a high score survive the evolutionary process.}
    \label{fig:replicator_dynamics}
\end{figure}

This work can be used to classify plays of the IPD\@: data can be collected from
actual interactions (in lab or in the field). Furthermore, this allows for a
classification method similar to the notion of fingerprinting presented
in~\cite{Ashlock2008}. Trained strategies can potentially be classified as
extortionate or not or it could be possible to even constrain the reinforcement
learning approaches that are becoming prevalent in the literature.
Alternatively, this mathematical approach for recognising extortion could be
used in sophisticated strategies to defend against invasion. Arguably, some of
the strategies considered here exhibit this behaviour, indeed as described
in~\cite{Harper2017}, the top ranking strategies in the full tournament are
obtained using evolutionary reinforcement learning techniques, thus, suspicion
of extortionate behaviour could in fact be an evolutionary trait.

\section*{Acknowledgements}

The following open source software libraries were used in this research:

\begin{itemize}
    \item The Axelrod ~\cite{Knight2016, Knight2018} library (IPD strategies and
        tournaments).
    \item The sympy library~\cite{Meurer2017} (verification of all symbolic
        calculations).
    \item The matplotlib~\cite{Droettboom2018} library (visualisation).
    \item The pandas~\cite{Structures2010}, dask~\cite{Dask2016} and
        NumPy~\cite{Oliphant2015} libraries (data manipulation).
    \item The SciPy~\cite{Jones2001} library (numerical integration of the
        replicator equation).
\end{itemize}

This work was performed using the computational facilities of the Advanced
Research Computing @ Cardiff (ARCCA) Division, Cardiff University.

\printbibliography

\newpage
\section*{Supplementary materials}

\includepdf{assets/pdf/proof_of_form_of_extortionate_strategies/main.pdf}

\newpage

Using the pair wise interactions the transition rates \(p,
q\) can be measured and the steady state probabilities inferred and compared to
the actual probabilities of each state.
This is done numerically by computing the singular eigenvector of the
matrix \(A\) \cite{Stewart2009}:

\[
    A =
    \begin{bmatrix}
        p_1 q_1 & p_1 (1 - q_1) & (1 - p_1) q_1 & (1 -p_1) (1 - q_1) \\
        p_2 q_2 & p_2 (1 - q_2) & (1 - p_2) q_2 & (1 -p_2) (1 - q_2) \\
        p_3 q_3 & p_3 (1 - q_3) & (1 - p_3) q_3 & (1 -p_3) (1 - q_3) \\
        p_4 q_4 & p_4 (1 - q_4) & (1 - p_4) q_4 & (1 -p_4) (1 - q_4) \\
    \end{bmatrix}
\]

Figure~\ref{fig:computed_probabilities_vs_theoretic_probabilities} shows a
regression line fitted to every pairwise interaction with a reported
\(\text{SSError}\) value (pairwise interactions with missing states were
omitted). This serves to validate the approach: a part from some edge cases the
relationship is consistent.

\begin{figure}[!htbp]
    \centering
    \includegraphics[width=.8\textwidth]{./assets/img/computed_probabilities_vs_theoretic_probabilities/main.pdf}
    \caption{The
        relationship between the steady state probabilities inferred from the
        measured transitions and the actual steady state probabilities. A linear
        regression line is included validating the approach.}
    \label{fig:computed_probabilities_vs_theoretic_probabilities}
\end{figure}


\end{document}
 strategies,
was presented with specific consideration given to ZD strategies. This
tournament is reproduced here using the Axelrod-Python
project~\cite{Knight2016}. To obtain a good measure of the corresponding
transition rates for each strategy all matches have been run for
\documentclass[a4paper]{article}

\usepackage{amsmath}
\usepackage{amssymb}
\usepackage[margin=1.5cm,
            includefoot,
            footskip=30pt]{geometry}
\usepackage{layout}
\usepackage{graphicx}
\usepackage{subcaption}

\usepackage{biblatex}
\usepackage{pdfpages}

\bibliography{main.bib}

\title{Suspicion: Recognising and evaluating the effectiveness
       of extortion in the Iterated Prisoner's Dilemma}
\author{Vincent A. Knight \and Nikoleta E. Glynatsi}
\date{\today}



\begin{document}

\maketitle

\begin{abstract}
    The Iterated Prisoner's Dilemma is a model for rational and evolutionary
    interactive behaviour. It has applications both in the study of human social
    behaviour as well as in biology.
    It is used to understand when and how a rational individual might
    accept an immediate cost to their own utility for the direct benefit of
    another.

    Much attention has been given to a class of strategies called
    Zero Determinant strategies. It has been theoretically shown that these
    strategies can ``extort'' any player.

    In this work, an approach to identify if observed strategies are playing in
    an extortionate way is described. Furthermore, experimental analysis of
    a large tournament with \input{assets/tex/number_of_full_strategies/main.tex}
    strategies is considered. In this setting
    the most highly performing strategies do not play in an extortionate way
    against each other but do against lower performing strategies.
    This suggests that whilst the theory of Zero Determinant strategies
    indicates that memory is not of fundamental importance to the evolution of
    cooperative behaviour, this is incomplete.
\end{abstract}

\section{Introduction}\label{sec:introduction}

Agent based game theoretic models have become a stalwart of the underpinning
mathematics of interactive behaviours. One of the major pieces of work
in this area is the pair of original computer tournaments run by Robert
Axelrod~\cite{Axelrod1980, Axelrod1980a}. These tournaments pitted submitted
computer strategies against each other in plays of the Iterated Prisoner's
Dilemma. A common game where agents can choose to pay a slight cost to their
immediate utility in the hope of building a reputation. This has been used in
economic and evolutionary game theory to understand the evolution of cooperative
behaviour.

Recently, a class of strategies was described in~\cite{Press2012} that can
provably extort any given opponent. In~\cite{Hilbe2013, Moran1707} some
questions have already been asked about the true effectiveness of these
strategies in an evolutionary setting. Here another question is asked: is it
possible to recognise this extortionate behaviour? A mathematical procedure for
suspicion is presented: in the same way that the continued actions of an
extortionate individual might raise suspicion.

This work makes use of the Axelrod Python library~\cite{Knight2018, Knight2016}
with a large number of Prisoner Dilemma strategies available to give an
extensive numerical example of the ideas presented.  The approach is presented
in Section~\ref{sec:delta-zd-strategies}.  All of the code and data discussed
in Section~\ref{sec:numerical-experiments} is open sourced, archived and
written according to best scientific principles~\cite{Wilson2014}. The data
archive can be found at~\cite{vincent_knight_2018_1297075}.

\section{Recognising Extortion}\label{sec:delta-zd-strategies}

In~\cite{Press2012}, given a match between 2 memory-one strategies, the concept
of Zero Determinant (ZD) strategies is introduced. The main result of that paper
shows that given two memory one players \(p, q\in\mathbb{R}^4\) a linear
relationship between the players' scores could be forced by one of the players.

Using the notation of~\cite{Press2012}, assuming the utilities for player \(p\)
are given by \(S_x=(R, S, T, P)\) and for player \(q\) by \(S_y=(R, T, S, P)\)
and that the stationary scores of each player is given by \(S_X\) and \(S_Y\)
respectively. The main result of~\cite{Press2012} is that if

\begin{equation}\label{eqn:linear_relationship_for_p}
    \tilde p=\alpha S_x + \beta S_y + \gamma
\end{equation}

or

\begin{equation}\label{eqn:linear_relationship_for_q}
    \tilde q=\alpha S_x + \beta S_y + \gamma
\end{equation}

where \(\tilde p = (1 - p_1, 1 - p_2, p_3, p_4)\) and
\(\tilde q = (1 - q_1, 1 - q_2, q_3, q_4)\) then:

\begin{equation}
    \alpha S_X + \beta S_Y + \gamma = 0
\end{equation}

In~\cite{Press2012} a particular type of ZD strategy is defined: extortionate
strategies. If:

\begin{equation}\label{eqn:constraint_for_extortion}
    \gamma = - P(\alpha + \beta)
\end{equation}

then the player can ensure they get a score \(\chi\) times
larger than the opponent. This extortion coefficient is given by:

\begin{equation}\label{eqn:definition_of_chi}
    \chi=\frac{-\beta}{\alpha}
\end{equation}

Thus, if (\ref{eqn:constraint_for_extortion}) holds and \(\chi >1\) a player is
said to extort their opponent.
Here, the reverse problem is considered: given a
\(p\in\mathbb{R}^4\) how does one identify \(\alpha, \beta\) if they
exist and is the strategy in fact acting in an extortionate way?

These conditions correspond to:

\begin{align}
    \tilde p_1 & = \alpha R + \beta R - P (\alpha + \beta)
            \label{eqn:condition_for_tilde_p1}\\
    \tilde p_2 & = \alpha S + \beta T - P (\alpha + \beta)
            \label{eqn:condition_for_tilde_p2}\\
    \tilde p_3 & = \alpha T + \beta S - P (\alpha + \beta)
            \label{eqn:condition_for_tilde_p3}\\
    \tilde p_4 & = \alpha P + \beta P - P (\alpha + \beta)
            \label{eqn:condition_for_tilde_p4}
\end{align}

Equation (\ref{eqn:condition_for_tilde_p4}) ensures that \(p_4=\tilde p_4=0\).
Equations (\ref{eqn:condition_for_tilde_p1}-\ref{eqn:condition_for_tilde_p3})
can be used to eliminate \(\alpha, \beta\), giving:

\begin{equation}\label{eqn:planar_definition_of_extortion}
    \tilde p_1 = \frac{(R - P)(\tilde p_2 + \tilde p_3)}{S + T - 2P}
\end{equation}

with:

\begin{equation}\label{eqn:definition_of_chi}
    \chi = \frac{\tilde p_2 (P - T) + \tilde p_3 (S - P)}
                {\tilde p_2 (P - S) + \tilde p_3 (T - P)}
\end{equation}

Given a strategy \(p\in\mathbb{R}^{4\times 1}\) equations
(\ref{eqn:condition_for_tilde_p4}), (\ref{eqn:planar_definition_of_extortion}-\ref{eqn:definition_of_chi}) can be used to check if
a strategy is extortionate. The conditions correspond to:

\begin{align}
    p_1 & = \frac{(R-P)(p_2 + p_3) - R + T + S - P}{S + T - 2P}
     \label{eqn:condition_for_p1}\\
    p_4 & = 0 \label{eqn:condition_for_p4}\\
    1 & > p_2 + p_3\label{eqn:condition_for_chi}
\end{align}

The algebraic steps necessary to prove these results are available in the
supporting materials.

All extortionate strategies reside on a triangular (\ref{eqn:condition_for_chi})
plane (\ref{eqn:condition_for_p1}) in 3 dimensions (\ref{eqn:condition_for_p4}).
Using this formulation it can be seen that a necessary (but not sufficient)
condition for an extortionate strategy is that it cooperates on average less
than 50\% of the time when in a state of disagreement with the opponent.

As an example, consider the known extortionate strategy \(p=(8 / 9, 1 / 2, 1 /
3, 0)\) from~\cite{Stewart2012} which is referred to as \texttt{Extort-2}. In
this case, for the standard values of \((R, T, S, P)\) constraint
(\ref{eqn:condition_for_p1}) corresponds to:

\begin{equation}
    p_1 = \frac{2(p_2 + p_3) + 1}{3}
\end{equation}

It is clear that in this case all constraints hold.

This approach could in fact be used to confirm that a given strategy is acting
in an extortionate manner even if it is not a memory one strategy. However, in
practice, if a closed form for \(p\) is not known, then due to measurement
and/or numerical error this would not work.

This problem can be written in the following linear algebraic form where
\(x=(\alpha, \beta)\)
and \(p^*=(\tilde p_1 - 1, tilde_2 - 1, p_3)\):

\begin{equation}\label{eqn:linear_algebraic_equation_for_p}
    Cx= p^*
\end{equation}

\(C\) corresponds to equations
(\ref{eqn:condition_for_tilde_p1}-\ref{eqn:condition_for_tilde_p3}) and is
given by:

\begin{equation}\label{eqn:definition_of_C}
    C =
    \begin{bmatrix}
        R - P & R- P \\
        S - P & T- P \\
        T - P & S- P \\
    \end{bmatrix}
\end{equation}

Note that in general, equation (\ref{eqn:linear_algebraic_equation_for_p}) will
not necessarily have a solution. From the Rouch\'{e}-Capelli theorem if there is
a solution it is unique as \(\text{rank}(C)=2\) which is the dimension of the
variable \(x\). The best fitting \(x\) is found by minimizing:

\begin{equation}\label{eqn:r_squared}
    \text{SSError} = \|C x- p^*\|_2^2 = \sum_{i=1}^{3}\left((C\bar x)_i-p_i^*\right)^2
\end{equation}

Note that \(\text{SSError}\), which is the square of the Frobenius
norm~\cite{Golub2013}, becomes a measure of how close a strategy is to being an
extortionate strategy. Suspicion
of extortion then corresponds to a threshold on \(\text{SSError}\).

By observing interactions (human or otherwise), their memory one representation
can be inferred and this approach can be used to recognise extortionate
behaviour. The notion of comparing theoretic and actual plays of the IPD is not
novel, see for example~\cite{Rand2013}. Immediately it is noted that if the
environment is noisy~\cite{Wu1995} then no strategy can be considered to be
extortionate as \(p_4>0\).

In the next section, this idea will be illustrated by observing the interactions
that take place in a computer based tournament of the IPD\@.

\section{Numerical experiments}\label{sec:numerical-experiments}

In~\cite{Stewart2012} results from a tournament with
\input{./assets/tex/number_of_stewart_plotkin_strategies/main.tex} strategies,
was presented with specific consideration given to ZD strategies. This
tournament is reproduced here using the Axelrod-Python
project~\cite{Knight2016}. To obtain a good measure of the corresponding
transition rates for each strategy all matches have been run for
\input{assets/tex/number_of_turns/main.tex} turns and every match has been
repeated \input{assets/tex/number_of_repetitions/main.tex} times. All of this
interaction data is available at~\cite{vincent_knight_2018_1297075}. A good
match between the inferred Markov chain and the state distribution of the actual
interactions has been verified. Data for this is presented in the supplementary
materials.

Figure~\ref{fig:SSError_overall_in_stewart_plotkin} shows the \(\text{SSError}\)
values for all the strategies in the tournament, as reported
in~\cite{Stewart2012} the extortionate strategy (which has an expected
\(\text{SSError}\) approximately 0) gains a large number of wins.

\begin{figure}[!htbp]
    \centering
    \includegraphics[width=.8\textwidth]{./assets/img/SSError_overall_in_stewart_plotkin/main.pdf}
    \caption{\(\text{SSError}\) and state probabilities for the strategies
        of~\cite{Stewart2012}, ordered both by number of wins and overall score.
        Note that \(P(DC)\) is not shown as it corresponds to the transpose of
        \(P(CD)\). Cooperator and Defector are omitted as they do not visit all
        the states.}
    \label{fig:SSError_overall_in_stewart_plotkin}
\end{figure}

Here, the work of~\cite{Stewart2012} is extended by investigating a tournament
with \input{assets/tex/number_of_full_strategies/main.tex}
strategies.

The results of this analysis are shown in
Figure~\ref{fig:SSError_and_probabilities_in_full}. The top ranking strategies
by number of wins seem to be extortionate (but not against all strategies) and
it can be seen that a small sub group of strategies achieve mutual defection.
All the top ranking strategies according to score achieve mutual cooperation and
do not extort each other, however they
\textbf{do} exhibit extortionate behaviour towards a number of the lower ranking
strategies.

\begin{figure}[!htbp]
    \centering
    \includegraphics[width=.8\textwidth]{./assets/img/SSError_and_probabilities_in_full/main.pdf}
    \caption{\(\text{SSError}\) for the strategies for the full tournament. Only
    strategy interactions for which \(p_4=0\) and \(\chi>1\) are displayed.}
    \label{fig:SSError_and_probabilities_in_full}
\end{figure}

\section{Conclusion}\label{sec:conclusion}

This work defines an approach to measure whether or not a player is playing a
strategy that corresponds to an extortionate strategy as defined
in~\cite{Press2012}: a mathematical model for suspicion. Indeed, all
extortionate strategies have been
 classified as lying on a triangular plane.
This rigorous classification fails to be robust to small measurement error, thus
a statistical approach is proposed.
This is done through a linear algebraic approach for approximating the solution
of a linear system. Using this, a large number of pairwise interactions is
simulated and in fact very few strategies are found to act extortionately.

The work of~\cite{Press2012}, whilst showing that a clever approach to taking
advantage of another memory one strategy exists: this is incomplete. Whilst the
elegance of this result is very attractive, just as the simplicity of the
victory of Tit For Tat in Axelrod's original tournaments was, it is incomplete.
Extortionate strategies achieve a high number of wins but they do not
achieve a high score which corresponds to the fitness landscape in an
evolutionary sense. From the large number of interactions a payoff matrix \(S\)
can be measured where \(S_{ij}\) denotes the score (using standard values of
\((R, S, T, P) = (3, 0, 5, 1)\)) of the \(i\)th strategy
against the \(j\)th strategy. Using this, the replicator equation
describes the evolution of the system based on a population density fitness
function:

\begin{equation}\label{eqn:replicator_dynamics}
    \frac{dx}{dt} = x(S-x^TS x)
\end{equation}

Equation (\ref{eqn:replicator_dynamics}) is solved numerically through an
integration technique described in~\cite{Petzold1983} and
Figure~\ref{fig:replicator_dynamics} shows the evolution of the distribution of
the system: the various strategies are ranked by scores. It is clear to see that
only the high ranking strategies survive the evolutionary process (in fact,
only \input{./assets/img/replicator_dynamics/main.tex}
have a final distribution greater than \(10 ^ {-2}\)). This confirms the
findings of~\cite{Moran1707} in which sophisticated strategies resist
evolutionary invasion of shorter memory strategies. Recalling
Figure~\ref{fig:SSError_and_probabilities_in_full} this demonstrates that:

\begin{itemize}
    \item Cooperation emerges through the evolutionary process: the high scoring
        strategies do not exhibit extortionate behaviour towards each other.
    \item Extortionate strategies do not survive the evolutionary process.
\end{itemize}

\begin{figure}[!htbp]
    \centering
    \includegraphics[width=.8\textwidth]{./assets/img/replicator_dynamics/main.pdf}
    \caption{Numerical simulation of the replicator equation
    (\ref{eqn:replicator_dynamics}): strategies are ordered by score, only the strategies with a high score survive the evolutionary process.}
    \label{fig:replicator_dynamics}
\end{figure}

This work can be used to classify plays of the IPD\@: data can be collected from
actual interactions (in lab or in the field). Furthermore, this allows for a
classification method similar to the notion of fingerprinting presented
in~\cite{Ashlock2008}. Trained strategies can potentially be classified as
extortionate or not or it could be possible to even constrain the reinforcement
learning approaches that are becoming prevalent in the literature.
Alternatively, this mathematical approach for recognising extortion could be
used in sophisticated strategies to defend against invasion. Arguably, some of
the strategies considered here exhibit this behaviour, indeed as described
in~\cite{Harper2017}, the top ranking strategies in the full tournament are
obtained using evolutionary reinforcement learning techniques, thus, suspicion
of extortionate behaviour could in fact be an evolutionary trait.

\section*{Acknowledgements}

The following open source software libraries were used in this research:

\begin{itemize}
    \item The Axelrod ~\cite{Knight2016, Knight2018} library (IPD strategies and
        tournaments).
    \item The sympy library~\cite{Meurer2017} (verification of all symbolic
        calculations).
    \item The matplotlib~\cite{Droettboom2018} library (visualisation).
    \item The pandas~\cite{Structures2010}, dask~\cite{Dask2016} and
        NumPy~\cite{Oliphant2015} libraries (data manipulation).
    \item The SciPy~\cite{Jones2001} library (numerical integration of the
        replicator equation).
\end{itemize}

This work was performed using the computational facilities of the Advanced
Research Computing @ Cardiff (ARCCA) Division, Cardiff University.

\printbibliography

\newpage
\section*{Supplementary materials}

\includepdf{assets/pdf/proof_of_form_of_extortionate_strategies/main.pdf}

\newpage

Using the pair wise interactions the transition rates \(p,
q\) can be measured and the steady state probabilities inferred and compared to
the actual probabilities of each state.
This is done numerically by computing the singular eigenvector of the
matrix \(A\) \cite{Stewart2009}:

\[
    A =
    \begin{bmatrix}
        p_1 q_1 & p_1 (1 - q_1) & (1 - p_1) q_1 & (1 -p_1) (1 - q_1) \\
        p_2 q_2 & p_2 (1 - q_2) & (1 - p_2) q_2 & (1 -p_2) (1 - q_2) \\
        p_3 q_3 & p_3 (1 - q_3) & (1 - p_3) q_3 & (1 -p_3) (1 - q_3) \\
        p_4 q_4 & p_4 (1 - q_4) & (1 - p_4) q_4 & (1 -p_4) (1 - q_4) \\
    \end{bmatrix}
\]

Figure~\ref{fig:computed_probabilities_vs_theoretic_probabilities} shows a
regression line fitted to every pairwise interaction with a reported
\(\text{SSError}\) value (pairwise interactions with missing states were
omitted). This serves to validate the approach: a part from some edge cases the
relationship is consistent.

\begin{figure}[!htbp]
    \centering
    \includegraphics[width=.8\textwidth]{./assets/img/computed_probabilities_vs_theoretic_probabilities/main.pdf}
    \caption{The
        relationship between the steady state probabilities inferred from the
        measured transitions and the actual steady state probabilities. A linear
        regression line is included validating the approach.}
    \label{fig:computed_probabilities_vs_theoretic_probabilities}
\end{figure}


\end{document}
 turns and every match has been
repeated \documentclass[a4paper]{article}

\usepackage{amsmath}
\usepackage{amssymb}
\usepackage[margin=1.5cm,
            includefoot,
            footskip=30pt]{geometry}
\usepackage{layout}
\usepackage{graphicx}
\usepackage{subcaption}

\usepackage{biblatex}
\usepackage{pdfpages}

\bibliography{main.bib}

\title{Suspicion: Recognising and evaluating the effectiveness
       of extortion in the Iterated Prisoner's Dilemma}
\author{Vincent A. Knight \and Nikoleta E. Glynatsi}
\date{\today}



\begin{document}

\maketitle

\begin{abstract}
    The Iterated Prisoner's Dilemma is a model for rational and evolutionary
    interactive behaviour. It has applications both in the study of human social
    behaviour as well as in biology.
    It is used to understand when and how a rational individual might
    accept an immediate cost to their own utility for the direct benefit of
    another.

    Much attention has been given to a class of strategies called
    Zero Determinant strategies. It has been theoretically shown that these
    strategies can ``extort'' any player.

    In this work, an approach to identify if observed strategies are playing in
    an extortionate way is described. Furthermore, experimental analysis of
    a large tournament with \input{assets/tex/number_of_full_strategies/main.tex}
    strategies is considered. In this setting
    the most highly performing strategies do not play in an extortionate way
    against each other but do against lower performing strategies.
    This suggests that whilst the theory of Zero Determinant strategies
    indicates that memory is not of fundamental importance to the evolution of
    cooperative behaviour, this is incomplete.
\end{abstract}

\section{Introduction}\label{sec:introduction}

Agent based game theoretic models have become a stalwart of the underpinning
mathematics of interactive behaviours. One of the major pieces of work
in this area is the pair of original computer tournaments run by Robert
Axelrod~\cite{Axelrod1980, Axelrod1980a}. These tournaments pitted submitted
computer strategies against each other in plays of the Iterated Prisoner's
Dilemma. A common game where agents can choose to pay a slight cost to their
immediate utility in the hope of building a reputation. This has been used in
economic and evolutionary game theory to understand the evolution of cooperative
behaviour.

Recently, a class of strategies was described in~\cite{Press2012} that can
provably extort any given opponent. In~\cite{Hilbe2013, Moran1707} some
questions have already been asked about the true effectiveness of these
strategies in an evolutionary setting. Here another question is asked: is it
possible to recognise this extortionate behaviour? A mathematical procedure for
suspicion is presented: in the same way that the continued actions of an
extortionate individual might raise suspicion.

This work makes use of the Axelrod Python library~\cite{Knight2018, Knight2016}
with a large number of Prisoner Dilemma strategies available to give an
extensive numerical example of the ideas presented.  The approach is presented
in Section~\ref{sec:delta-zd-strategies}.  All of the code and data discussed
in Section~\ref{sec:numerical-experiments} is open sourced, archived and
written according to best scientific principles~\cite{Wilson2014}. The data
archive can be found at~\cite{vincent_knight_2018_1297075}.

\section{Recognising Extortion}\label{sec:delta-zd-strategies}

In~\cite{Press2012}, given a match between 2 memory-one strategies, the concept
of Zero Determinant (ZD) strategies is introduced. The main result of that paper
shows that given two memory one players \(p, q\in\mathbb{R}^4\) a linear
relationship between the players' scores could be forced by one of the players.

Using the notation of~\cite{Press2012}, assuming the utilities for player \(p\)
are given by \(S_x=(R, S, T, P)\) and for player \(q\) by \(S_y=(R, T, S, P)\)
and that the stationary scores of each player is given by \(S_X\) and \(S_Y\)
respectively. The main result of~\cite{Press2012} is that if

\begin{equation}\label{eqn:linear_relationship_for_p}
    \tilde p=\alpha S_x + \beta S_y + \gamma
\end{equation}

or

\begin{equation}\label{eqn:linear_relationship_for_q}
    \tilde q=\alpha S_x + \beta S_y + \gamma
\end{equation}

where \(\tilde p = (1 - p_1, 1 - p_2, p_3, p_4)\) and
\(\tilde q = (1 - q_1, 1 - q_2, q_3, q_4)\) then:

\begin{equation}
    \alpha S_X + \beta S_Y + \gamma = 0
\end{equation}

In~\cite{Press2012} a particular type of ZD strategy is defined: extortionate
strategies. If:

\begin{equation}\label{eqn:constraint_for_extortion}
    \gamma = - P(\alpha + \beta)
\end{equation}

then the player can ensure they get a score \(\chi\) times
larger than the opponent. This extortion coefficient is given by:

\begin{equation}\label{eqn:definition_of_chi}
    \chi=\frac{-\beta}{\alpha}
\end{equation}

Thus, if (\ref{eqn:constraint_for_extortion}) holds and \(\chi >1\) a player is
said to extort their opponent.
Here, the reverse problem is considered: given a
\(p\in\mathbb{R}^4\) how does one identify \(\alpha, \beta\) if they
exist and is the strategy in fact acting in an extortionate way?

These conditions correspond to:

\begin{align}
    \tilde p_1 & = \alpha R + \beta R - P (\alpha + \beta)
            \label{eqn:condition_for_tilde_p1}\\
    \tilde p_2 & = \alpha S + \beta T - P (\alpha + \beta)
            \label{eqn:condition_for_tilde_p2}\\
    \tilde p_3 & = \alpha T + \beta S - P (\alpha + \beta)
            \label{eqn:condition_for_tilde_p3}\\
    \tilde p_4 & = \alpha P + \beta P - P (\alpha + \beta)
            \label{eqn:condition_for_tilde_p4}
\end{align}

Equation (\ref{eqn:condition_for_tilde_p4}) ensures that \(p_4=\tilde p_4=0\).
Equations (\ref{eqn:condition_for_tilde_p1}-\ref{eqn:condition_for_tilde_p3})
can be used to eliminate \(\alpha, \beta\), giving:

\begin{equation}\label{eqn:planar_definition_of_extortion}
    \tilde p_1 = \frac{(R - P)(\tilde p_2 + \tilde p_3)}{S + T - 2P}
\end{equation}

with:

\begin{equation}\label{eqn:definition_of_chi}
    \chi = \frac{\tilde p_2 (P - T) + \tilde p_3 (S - P)}
                {\tilde p_2 (P - S) + \tilde p_3 (T - P)}
\end{equation}

Given a strategy \(p\in\mathbb{R}^{4\times 1}\) equations
(\ref{eqn:condition_for_tilde_p4}), (\ref{eqn:planar_definition_of_extortion}-\ref{eqn:definition_of_chi}) can be used to check if
a strategy is extortionate. The conditions correspond to:

\begin{align}
    p_1 & = \frac{(R-P)(p_2 + p_3) - R + T + S - P}{S + T - 2P}
     \label{eqn:condition_for_p1}\\
    p_4 & = 0 \label{eqn:condition_for_p4}\\
    1 & > p_2 + p_3\label{eqn:condition_for_chi}
\end{align}

The algebraic steps necessary to prove these results are available in the
supporting materials.

All extortionate strategies reside on a triangular (\ref{eqn:condition_for_chi})
plane (\ref{eqn:condition_for_p1}) in 3 dimensions (\ref{eqn:condition_for_p4}).
Using this formulation it can be seen that a necessary (but not sufficient)
condition for an extortionate strategy is that it cooperates on average less
than 50\% of the time when in a state of disagreement with the opponent.

As an example, consider the known extortionate strategy \(p=(8 / 9, 1 / 2, 1 /
3, 0)\) from~\cite{Stewart2012} which is referred to as \texttt{Extort-2}. In
this case, for the standard values of \((R, T, S, P)\) constraint
(\ref{eqn:condition_for_p1}) corresponds to:

\begin{equation}
    p_1 = \frac{2(p_2 + p_3) + 1}{3}
\end{equation}

It is clear that in this case all constraints hold.

This approach could in fact be used to confirm that a given strategy is acting
in an extortionate manner even if it is not a memory one strategy. However, in
practice, if a closed form for \(p\) is not known, then due to measurement
and/or numerical error this would not work.

This problem can be written in the following linear algebraic form where
\(x=(\alpha, \beta)\)
and \(p^*=(\tilde p_1 - 1, tilde_2 - 1, p_3)\):

\begin{equation}\label{eqn:linear_algebraic_equation_for_p}
    Cx= p^*
\end{equation}

\(C\) corresponds to equations
(\ref{eqn:condition_for_tilde_p1}-\ref{eqn:condition_for_tilde_p3}) and is
given by:

\begin{equation}\label{eqn:definition_of_C}
    C =
    \begin{bmatrix}
        R - P & R- P \\
        S - P & T- P \\
        T - P & S- P \\
    \end{bmatrix}
\end{equation}

Note that in general, equation (\ref{eqn:linear_algebraic_equation_for_p}) will
not necessarily have a solution. From the Rouch\'{e}-Capelli theorem if there is
a solution it is unique as \(\text{rank}(C)=2\) which is the dimension of the
variable \(x\). The best fitting \(x\) is found by minimizing:

\begin{equation}\label{eqn:r_squared}
    \text{SSError} = \|C x- p^*\|_2^2 = \sum_{i=1}^{3}\left((C\bar x)_i-p_i^*\right)^2
\end{equation}

Note that \(\text{SSError}\), which is the square of the Frobenius
norm~\cite{Golub2013}, becomes a measure of how close a strategy is to being an
extortionate strategy. Suspicion
of extortion then corresponds to a threshold on \(\text{SSError}\).

By observing interactions (human or otherwise), their memory one representation
can be inferred and this approach can be used to recognise extortionate
behaviour. The notion of comparing theoretic and actual plays of the IPD is not
novel, see for example~\cite{Rand2013}. Immediately it is noted that if the
environment is noisy~\cite{Wu1995} then no strategy can be considered to be
extortionate as \(p_4>0\).

In the next section, this idea will be illustrated by observing the interactions
that take place in a computer based tournament of the IPD\@.

\section{Numerical experiments}\label{sec:numerical-experiments}

In~\cite{Stewart2012} results from a tournament with
\input{./assets/tex/number_of_stewart_plotkin_strategies/main.tex} strategies,
was presented with specific consideration given to ZD strategies. This
tournament is reproduced here using the Axelrod-Python
project~\cite{Knight2016}. To obtain a good measure of the corresponding
transition rates for each strategy all matches have been run for
\input{assets/tex/number_of_turns/main.tex} turns and every match has been
repeated \input{assets/tex/number_of_repetitions/main.tex} times. All of this
interaction data is available at~\cite{vincent_knight_2018_1297075}. A good
match between the inferred Markov chain and the state distribution of the actual
interactions has been verified. Data for this is presented in the supplementary
materials.

Figure~\ref{fig:SSError_overall_in_stewart_plotkin} shows the \(\text{SSError}\)
values for all the strategies in the tournament, as reported
in~\cite{Stewart2012} the extortionate strategy (which has an expected
\(\text{SSError}\) approximately 0) gains a large number of wins.

\begin{figure}[!htbp]
    \centering
    \includegraphics[width=.8\textwidth]{./assets/img/SSError_overall_in_stewart_plotkin/main.pdf}
    \caption{\(\text{SSError}\) and state probabilities for the strategies
        of~\cite{Stewart2012}, ordered both by number of wins and overall score.
        Note that \(P(DC)\) is not shown as it corresponds to the transpose of
        \(P(CD)\). Cooperator and Defector are omitted as they do not visit all
        the states.}
    \label{fig:SSError_overall_in_stewart_plotkin}
\end{figure}

Here, the work of~\cite{Stewart2012} is extended by investigating a tournament
with \input{assets/tex/number_of_full_strategies/main.tex}
strategies.

The results of this analysis are shown in
Figure~\ref{fig:SSError_and_probabilities_in_full}. The top ranking strategies
by number of wins seem to be extortionate (but not against all strategies) and
it can be seen that a small sub group of strategies achieve mutual defection.
All the top ranking strategies according to score achieve mutual cooperation and
do not extort each other, however they
\textbf{do} exhibit extortionate behaviour towards a number of the lower ranking
strategies.

\begin{figure}[!htbp]
    \centering
    \includegraphics[width=.8\textwidth]{./assets/img/SSError_and_probabilities_in_full/main.pdf}
    \caption{\(\text{SSError}\) for the strategies for the full tournament. Only
    strategy interactions for which \(p_4=0\) and \(\chi>1\) are displayed.}
    \label{fig:SSError_and_probabilities_in_full}
\end{figure}

\section{Conclusion}\label{sec:conclusion}

This work defines an approach to measure whether or not a player is playing a
strategy that corresponds to an extortionate strategy as defined
in~\cite{Press2012}: a mathematical model for suspicion. Indeed, all
extortionate strategies have been
 classified as lying on a triangular plane.
This rigorous classification fails to be robust to small measurement error, thus
a statistical approach is proposed.
This is done through a linear algebraic approach for approximating the solution
of a linear system. Using this, a large number of pairwise interactions is
simulated and in fact very few strategies are found to act extortionately.

The work of~\cite{Press2012}, whilst showing that a clever approach to taking
advantage of another memory one strategy exists: this is incomplete. Whilst the
elegance of this result is very attractive, just as the simplicity of the
victory of Tit For Tat in Axelrod's original tournaments was, it is incomplete.
Extortionate strategies achieve a high number of wins but they do not
achieve a high score which corresponds to the fitness landscape in an
evolutionary sense. From the large number of interactions a payoff matrix \(S\)
can be measured where \(S_{ij}\) denotes the score (using standard values of
\((R, S, T, P) = (3, 0, 5, 1)\)) of the \(i\)th strategy
against the \(j\)th strategy. Using this, the replicator equation
describes the evolution of the system based on a population density fitness
function:

\begin{equation}\label{eqn:replicator_dynamics}
    \frac{dx}{dt} = x(S-x^TS x)
\end{equation}

Equation (\ref{eqn:replicator_dynamics}) is solved numerically through an
integration technique described in~\cite{Petzold1983} and
Figure~\ref{fig:replicator_dynamics} shows the evolution of the distribution of
the system: the various strategies are ranked by scores. It is clear to see that
only the high ranking strategies survive the evolutionary process (in fact,
only \input{./assets/img/replicator_dynamics/main.tex}
have a final distribution greater than \(10 ^ {-2}\)). This confirms the
findings of~\cite{Moran1707} in which sophisticated strategies resist
evolutionary invasion of shorter memory strategies. Recalling
Figure~\ref{fig:SSError_and_probabilities_in_full} this demonstrates that:

\begin{itemize}
    \item Cooperation emerges through the evolutionary process: the high scoring
        strategies do not exhibit extortionate behaviour towards each other.
    \item Extortionate strategies do not survive the evolutionary process.
\end{itemize}

\begin{figure}[!htbp]
    \centering
    \includegraphics[width=.8\textwidth]{./assets/img/replicator_dynamics/main.pdf}
    \caption{Numerical simulation of the replicator equation
    (\ref{eqn:replicator_dynamics}): strategies are ordered by score, only the strategies with a high score survive the evolutionary process.}
    \label{fig:replicator_dynamics}
\end{figure}

This work can be used to classify plays of the IPD\@: data can be collected from
actual interactions (in lab or in the field). Furthermore, this allows for a
classification method similar to the notion of fingerprinting presented
in~\cite{Ashlock2008}. Trained strategies can potentially be classified as
extortionate or not or it could be possible to even constrain the reinforcement
learning approaches that are becoming prevalent in the literature.
Alternatively, this mathematical approach for recognising extortion could be
used in sophisticated strategies to defend against invasion. Arguably, some of
the strategies considered here exhibit this behaviour, indeed as described
in~\cite{Harper2017}, the top ranking strategies in the full tournament are
obtained using evolutionary reinforcement learning techniques, thus, suspicion
of extortionate behaviour could in fact be an evolutionary trait.

\section*{Acknowledgements}

The following open source software libraries were used in this research:

\begin{itemize}
    \item The Axelrod ~\cite{Knight2016, Knight2018} library (IPD strategies and
        tournaments).
    \item The sympy library~\cite{Meurer2017} (verification of all symbolic
        calculations).
    \item The matplotlib~\cite{Droettboom2018} library (visualisation).
    \item The pandas~\cite{Structures2010}, dask~\cite{Dask2016} and
        NumPy~\cite{Oliphant2015} libraries (data manipulation).
    \item The SciPy~\cite{Jones2001} library (numerical integration of the
        replicator equation).
\end{itemize}

This work was performed using the computational facilities of the Advanced
Research Computing @ Cardiff (ARCCA) Division, Cardiff University.

\printbibliography

\newpage
\section*{Supplementary materials}

\includepdf{assets/pdf/proof_of_form_of_extortionate_strategies/main.pdf}

\newpage

Using the pair wise interactions the transition rates \(p,
q\) can be measured and the steady state probabilities inferred and compared to
the actual probabilities of each state.
This is done numerically by computing the singular eigenvector of the
matrix \(A\) \cite{Stewart2009}:

\[
    A =
    \begin{bmatrix}
        p_1 q_1 & p_1 (1 - q_1) & (1 - p_1) q_1 & (1 -p_1) (1 - q_1) \\
        p_2 q_2 & p_2 (1 - q_2) & (1 - p_2) q_2 & (1 -p_2) (1 - q_2) \\
        p_3 q_3 & p_3 (1 - q_3) & (1 - p_3) q_3 & (1 -p_3) (1 - q_3) \\
        p_4 q_4 & p_4 (1 - q_4) & (1 - p_4) q_4 & (1 -p_4) (1 - q_4) \\
    \end{bmatrix}
\]

Figure~\ref{fig:computed_probabilities_vs_theoretic_probabilities} shows a
regression line fitted to every pairwise interaction with a reported
\(\text{SSError}\) value (pairwise interactions with missing states were
omitted). This serves to validate the approach: a part from some edge cases the
relationship is consistent.

\begin{figure}[!htbp]
    \centering
    \includegraphics[width=.8\textwidth]{./assets/img/computed_probabilities_vs_theoretic_probabilities/main.pdf}
    \caption{The
        relationship between the steady state probabilities inferred from the
        measured transitions and the actual steady state probabilities. A linear
        regression line is included validating the approach.}
    \label{fig:computed_probabilities_vs_theoretic_probabilities}
\end{figure}


\end{document}
 times. All of this
interaction data is available at~\cite{vincent_knight_2018_1297075}. A good
match between the inferred Markov chain and the state distribution of the actual
interactions has been verified. Data for this is presented in the supplementary
materials.

Figure~\ref{fig:SSError_overall_in_stewart_plotkin} shows the \(\text{SSError}\)
values for all the strategies in the tournament, as reported
in~\cite{Stewart2012} the extortionate strategy (which has an expected
\(\text{SSError}\) approximately 0) gains a large number of wins.

\begin{figure}[!htbp]
    \centering
    \includegraphics[width=.8\textwidth]{./assets/img/SSError_overall_in_stewart_plotkin/main.pdf}
    \caption{\(\text{SSError}\) and state probabilities for the strategies
        of~\cite{Stewart2012}, ordered both by number of wins and overall score.
        Note that \(P(DC)\) is not shown as it corresponds to the transpose of
        \(P(CD)\). Cooperator and Defector are omitted as they do not visit all
        the states.}
    \label{fig:SSError_overall_in_stewart_plotkin}
\end{figure}

Here, the work of~\cite{Stewart2012} is extended by investigating a tournament
with \documentclass[a4paper]{article}

\usepackage{amsmath}
\usepackage{amssymb}
\usepackage[margin=1.5cm,
            includefoot,
            footskip=30pt]{geometry}
\usepackage{layout}
\usepackage{graphicx}
\usepackage{subcaption}

\usepackage{biblatex}
\usepackage{pdfpages}

\bibliography{main.bib}

\title{Suspicion: Recognising and evaluating the effectiveness
       of extortion in the Iterated Prisoner's Dilemma}
\author{Vincent A. Knight \and Nikoleta E. Glynatsi}
\date{\today}



\begin{document}

\maketitle

\begin{abstract}
    The Iterated Prisoner's Dilemma is a model for rational and evolutionary
    interactive behaviour. It has applications both in the study of human social
    behaviour as well as in biology.
    It is used to understand when and how a rational individual might
    accept an immediate cost to their own utility for the direct benefit of
    another.

    Much attention has been given to a class of strategies called
    Zero Determinant strategies. It has been theoretically shown that these
    strategies can ``extort'' any player.

    In this work, an approach to identify if observed strategies are playing in
    an extortionate way is described. Furthermore, experimental analysis of
    a large tournament with \input{assets/tex/number_of_full_strategies/main.tex}
    strategies is considered. In this setting
    the most highly performing strategies do not play in an extortionate way
    against each other but do against lower performing strategies.
    This suggests that whilst the theory of Zero Determinant strategies
    indicates that memory is not of fundamental importance to the evolution of
    cooperative behaviour, this is incomplete.
\end{abstract}

\section{Introduction}\label{sec:introduction}

Agent based game theoretic models have become a stalwart of the underpinning
mathematics of interactive behaviours. One of the major pieces of work
in this area is the pair of original computer tournaments run by Robert
Axelrod~\cite{Axelrod1980, Axelrod1980a}. These tournaments pitted submitted
computer strategies against each other in plays of the Iterated Prisoner's
Dilemma. A common game where agents can choose to pay a slight cost to their
immediate utility in the hope of building a reputation. This has been used in
economic and evolutionary game theory to understand the evolution of cooperative
behaviour.

Recently, a class of strategies was described in~\cite{Press2012} that can
provably extort any given opponent. In~\cite{Hilbe2013, Moran1707} some
questions have already been asked about the true effectiveness of these
strategies in an evolutionary setting. Here another question is asked: is it
possible to recognise this extortionate behaviour? A mathematical procedure for
suspicion is presented: in the same way that the continued actions of an
extortionate individual might raise suspicion.

This work makes use of the Axelrod Python library~\cite{Knight2018, Knight2016}
with a large number of Prisoner Dilemma strategies available to give an
extensive numerical example of the ideas presented.  The approach is presented
in Section~\ref{sec:delta-zd-strategies}.  All of the code and data discussed
in Section~\ref{sec:numerical-experiments} is open sourced, archived and
written according to best scientific principles~\cite{Wilson2014}. The data
archive can be found at~\cite{vincent_knight_2018_1297075}.

\section{Recognising Extortion}\label{sec:delta-zd-strategies}

In~\cite{Press2012}, given a match between 2 memory-one strategies, the concept
of Zero Determinant (ZD) strategies is introduced. The main result of that paper
shows that given two memory one players \(p, q\in\mathbb{R}^4\) a linear
relationship between the players' scores could be forced by one of the players.

Using the notation of~\cite{Press2012}, assuming the utilities for player \(p\)
are given by \(S_x=(R, S, T, P)\) and for player \(q\) by \(S_y=(R, T, S, P)\)
and that the stationary scores of each player is given by \(S_X\) and \(S_Y\)
respectively. The main result of~\cite{Press2012} is that if

\begin{equation}\label{eqn:linear_relationship_for_p}
    \tilde p=\alpha S_x + \beta S_y + \gamma
\end{equation}

or

\begin{equation}\label{eqn:linear_relationship_for_q}
    \tilde q=\alpha S_x + \beta S_y + \gamma
\end{equation}

where \(\tilde p = (1 - p_1, 1 - p_2, p_3, p_4)\) and
\(\tilde q = (1 - q_1, 1 - q_2, q_3, q_4)\) then:

\begin{equation}
    \alpha S_X + \beta S_Y + \gamma = 0
\end{equation}

In~\cite{Press2012} a particular type of ZD strategy is defined: extortionate
strategies. If:

\begin{equation}\label{eqn:constraint_for_extortion}
    \gamma = - P(\alpha + \beta)
\end{equation}

then the player can ensure they get a score \(\chi\) times
larger than the opponent. This extortion coefficient is given by:

\begin{equation}\label{eqn:definition_of_chi}
    \chi=\frac{-\beta}{\alpha}
\end{equation}

Thus, if (\ref{eqn:constraint_for_extortion}) holds and \(\chi >1\) a player is
said to extort their opponent.
Here, the reverse problem is considered: given a
\(p\in\mathbb{R}^4\) how does one identify \(\alpha, \beta\) if they
exist and is the strategy in fact acting in an extortionate way?

These conditions correspond to:

\begin{align}
    \tilde p_1 & = \alpha R + \beta R - P (\alpha + \beta)
            \label{eqn:condition_for_tilde_p1}\\
    \tilde p_2 & = \alpha S + \beta T - P (\alpha + \beta)
            \label{eqn:condition_for_tilde_p2}\\
    \tilde p_3 & = \alpha T + \beta S - P (\alpha + \beta)
            \label{eqn:condition_for_tilde_p3}\\
    \tilde p_4 & = \alpha P + \beta P - P (\alpha + \beta)
            \label{eqn:condition_for_tilde_p4}
\end{align}

Equation (\ref{eqn:condition_for_tilde_p4}) ensures that \(p_4=\tilde p_4=0\).
Equations (\ref{eqn:condition_for_tilde_p1}-\ref{eqn:condition_for_tilde_p3})
can be used to eliminate \(\alpha, \beta\), giving:

\begin{equation}\label{eqn:planar_definition_of_extortion}
    \tilde p_1 = \frac{(R - P)(\tilde p_2 + \tilde p_3)}{S + T - 2P}
\end{equation}

with:

\begin{equation}\label{eqn:definition_of_chi}
    \chi = \frac{\tilde p_2 (P - T) + \tilde p_3 (S - P)}
                {\tilde p_2 (P - S) + \tilde p_3 (T - P)}
\end{equation}

Given a strategy \(p\in\mathbb{R}^{4\times 1}\) equations
(\ref{eqn:condition_for_tilde_p4}), (\ref{eqn:planar_definition_of_extortion}-\ref{eqn:definition_of_chi}) can be used to check if
a strategy is extortionate. The conditions correspond to:

\begin{align}
    p_1 & = \frac{(R-P)(p_2 + p_3) - R + T + S - P}{S + T - 2P}
     \label{eqn:condition_for_p1}\\
    p_4 & = 0 \label{eqn:condition_for_p4}\\
    1 & > p_2 + p_3\label{eqn:condition_for_chi}
\end{align}

The algebraic steps necessary to prove these results are available in the
supporting materials.

All extortionate strategies reside on a triangular (\ref{eqn:condition_for_chi})
plane (\ref{eqn:condition_for_p1}) in 3 dimensions (\ref{eqn:condition_for_p4}).
Using this formulation it can be seen that a necessary (but not sufficient)
condition for an extortionate strategy is that it cooperates on average less
than 50\% of the time when in a state of disagreement with the opponent.

As an example, consider the known extortionate strategy \(p=(8 / 9, 1 / 2, 1 /
3, 0)\) from~\cite{Stewart2012} which is referred to as \texttt{Extort-2}. In
this case, for the standard values of \((R, T, S, P)\) constraint
(\ref{eqn:condition_for_p1}) corresponds to:

\begin{equation}
    p_1 = \frac{2(p_2 + p_3) + 1}{3}
\end{equation}

It is clear that in this case all constraints hold.

This approach could in fact be used to confirm that a given strategy is acting
in an extortionate manner even if it is not a memory one strategy. However, in
practice, if a closed form for \(p\) is not known, then due to measurement
and/or numerical error this would not work.

This problem can be written in the following linear algebraic form where
\(x=(\alpha, \beta)\)
and \(p^*=(\tilde p_1 - 1, tilde_2 - 1, p_3)\):

\begin{equation}\label{eqn:linear_algebraic_equation_for_p}
    Cx= p^*
\end{equation}

\(C\) corresponds to equations
(\ref{eqn:condition_for_tilde_p1}-\ref{eqn:condition_for_tilde_p3}) and is
given by:

\begin{equation}\label{eqn:definition_of_C}
    C =
    \begin{bmatrix}
        R - P & R- P \\
        S - P & T- P \\
        T - P & S- P \\
    \end{bmatrix}
\end{equation}

Note that in general, equation (\ref{eqn:linear_algebraic_equation_for_p}) will
not necessarily have a solution. From the Rouch\'{e}-Capelli theorem if there is
a solution it is unique as \(\text{rank}(C)=2\) which is the dimension of the
variable \(x\). The best fitting \(x\) is found by minimizing:

\begin{equation}\label{eqn:r_squared}
    \text{SSError} = \|C x- p^*\|_2^2 = \sum_{i=1}^{3}\left((C\bar x)_i-p_i^*\right)^2
\end{equation}

Note that \(\text{SSError}\), which is the square of the Frobenius
norm~\cite{Golub2013}, becomes a measure of how close a strategy is to being an
extortionate strategy. Suspicion
of extortion then corresponds to a threshold on \(\text{SSError}\).

By observing interactions (human or otherwise), their memory one representation
can be inferred and this approach can be used to recognise extortionate
behaviour. The notion of comparing theoretic and actual plays of the IPD is not
novel, see for example~\cite{Rand2013}. Immediately it is noted that if the
environment is noisy~\cite{Wu1995} then no strategy can be considered to be
extortionate as \(p_4>0\).

In the next section, this idea will be illustrated by observing the interactions
that take place in a computer based tournament of the IPD\@.

\section{Numerical experiments}\label{sec:numerical-experiments}

In~\cite{Stewart2012} results from a tournament with
\input{./assets/tex/number_of_stewart_plotkin_strategies/main.tex} strategies,
was presented with specific consideration given to ZD strategies. This
tournament is reproduced here using the Axelrod-Python
project~\cite{Knight2016}. To obtain a good measure of the corresponding
transition rates for each strategy all matches have been run for
\input{assets/tex/number_of_turns/main.tex} turns and every match has been
repeated \input{assets/tex/number_of_repetitions/main.tex} times. All of this
interaction data is available at~\cite{vincent_knight_2018_1297075}. A good
match between the inferred Markov chain and the state distribution of the actual
interactions has been verified. Data for this is presented in the supplementary
materials.

Figure~\ref{fig:SSError_overall_in_stewart_plotkin} shows the \(\text{SSError}\)
values for all the strategies in the tournament, as reported
in~\cite{Stewart2012} the extortionate strategy (which has an expected
\(\text{SSError}\) approximately 0) gains a large number of wins.

\begin{figure}[!htbp]
    \centering
    \includegraphics[width=.8\textwidth]{./assets/img/SSError_overall_in_stewart_plotkin/main.pdf}
    \caption{\(\text{SSError}\) and state probabilities for the strategies
        of~\cite{Stewart2012}, ordered both by number of wins and overall score.
        Note that \(P(DC)\) is not shown as it corresponds to the transpose of
        \(P(CD)\). Cooperator and Defector are omitted as they do not visit all
        the states.}
    \label{fig:SSError_overall_in_stewart_plotkin}
\end{figure}

Here, the work of~\cite{Stewart2012} is extended by investigating a tournament
with \input{assets/tex/number_of_full_strategies/main.tex}
strategies.

The results of this analysis are shown in
Figure~\ref{fig:SSError_and_probabilities_in_full}. The top ranking strategies
by number of wins seem to be extortionate (but not against all strategies) and
it can be seen that a small sub group of strategies achieve mutual defection.
All the top ranking strategies according to score achieve mutual cooperation and
do not extort each other, however they
\textbf{do} exhibit extortionate behaviour towards a number of the lower ranking
strategies.

\begin{figure}[!htbp]
    \centering
    \includegraphics[width=.8\textwidth]{./assets/img/SSError_and_probabilities_in_full/main.pdf}
    \caption{\(\text{SSError}\) for the strategies for the full tournament. Only
    strategy interactions for which \(p_4=0\) and \(\chi>1\) are displayed.}
    \label{fig:SSError_and_probabilities_in_full}
\end{figure}

\section{Conclusion}\label{sec:conclusion}

This work defines an approach to measure whether or not a player is playing a
strategy that corresponds to an extortionate strategy as defined
in~\cite{Press2012}: a mathematical model for suspicion. Indeed, all
extortionate strategies have been
 classified as lying on a triangular plane.
This rigorous classification fails to be robust to small measurement error, thus
a statistical approach is proposed.
This is done through a linear algebraic approach for approximating the solution
of a linear system. Using this, a large number of pairwise interactions is
simulated and in fact very few strategies are found to act extortionately.

The work of~\cite{Press2012}, whilst showing that a clever approach to taking
advantage of another memory one strategy exists: this is incomplete. Whilst the
elegance of this result is very attractive, just as the simplicity of the
victory of Tit For Tat in Axelrod's original tournaments was, it is incomplete.
Extortionate strategies achieve a high number of wins but they do not
achieve a high score which corresponds to the fitness landscape in an
evolutionary sense. From the large number of interactions a payoff matrix \(S\)
can be measured where \(S_{ij}\) denotes the score (using standard values of
\((R, S, T, P) = (3, 0, 5, 1)\)) of the \(i\)th strategy
against the \(j\)th strategy. Using this, the replicator equation
describes the evolution of the system based on a population density fitness
function:

\begin{equation}\label{eqn:replicator_dynamics}
    \frac{dx}{dt} = x(S-x^TS x)
\end{equation}

Equation (\ref{eqn:replicator_dynamics}) is solved numerically through an
integration technique described in~\cite{Petzold1983} and
Figure~\ref{fig:replicator_dynamics} shows the evolution of the distribution of
the system: the various strategies are ranked by scores. It is clear to see that
only the high ranking strategies survive the evolutionary process (in fact,
only \input{./assets/img/replicator_dynamics/main.tex}
have a final distribution greater than \(10 ^ {-2}\)). This confirms the
findings of~\cite{Moran1707} in which sophisticated strategies resist
evolutionary invasion of shorter memory strategies. Recalling
Figure~\ref{fig:SSError_and_probabilities_in_full} this demonstrates that:

\begin{itemize}
    \item Cooperation emerges through the evolutionary process: the high scoring
        strategies do not exhibit extortionate behaviour towards each other.
    \item Extortionate strategies do not survive the evolutionary process.
\end{itemize}

\begin{figure}[!htbp]
    \centering
    \includegraphics[width=.8\textwidth]{./assets/img/replicator_dynamics/main.pdf}
    \caption{Numerical simulation of the replicator equation
    (\ref{eqn:replicator_dynamics}): strategies are ordered by score, only the strategies with a high score survive the evolutionary process.}
    \label{fig:replicator_dynamics}
\end{figure}

This work can be used to classify plays of the IPD\@: data can be collected from
actual interactions (in lab or in the field). Furthermore, this allows for a
classification method similar to the notion of fingerprinting presented
in~\cite{Ashlock2008}. Trained strategies can potentially be classified as
extortionate or not or it could be possible to even constrain the reinforcement
learning approaches that are becoming prevalent in the literature.
Alternatively, this mathematical approach for recognising extortion could be
used in sophisticated strategies to defend against invasion. Arguably, some of
the strategies considered here exhibit this behaviour, indeed as described
in~\cite{Harper2017}, the top ranking strategies in the full tournament are
obtained using evolutionary reinforcement learning techniques, thus, suspicion
of extortionate behaviour could in fact be an evolutionary trait.

\section*{Acknowledgements}

The following open source software libraries were used in this research:

\begin{itemize}
    \item The Axelrod ~\cite{Knight2016, Knight2018} library (IPD strategies and
        tournaments).
    \item The sympy library~\cite{Meurer2017} (verification of all symbolic
        calculations).
    \item The matplotlib~\cite{Droettboom2018} library (visualisation).
    \item The pandas~\cite{Structures2010}, dask~\cite{Dask2016} and
        NumPy~\cite{Oliphant2015} libraries (data manipulation).
    \item The SciPy~\cite{Jones2001} library (numerical integration of the
        replicator equation).
\end{itemize}

This work was performed using the computational facilities of the Advanced
Research Computing @ Cardiff (ARCCA) Division, Cardiff University.

\printbibliography

\newpage
\section*{Supplementary materials}

\includepdf{assets/pdf/proof_of_form_of_extortionate_strategies/main.pdf}

\newpage

Using the pair wise interactions the transition rates \(p,
q\) can be measured and the steady state probabilities inferred and compared to
the actual probabilities of each state.
This is done numerically by computing the singular eigenvector of the
matrix \(A\) \cite{Stewart2009}:

\[
    A =
    \begin{bmatrix}
        p_1 q_1 & p_1 (1 - q_1) & (1 - p_1) q_1 & (1 -p_1) (1 - q_1) \\
        p_2 q_2 & p_2 (1 - q_2) & (1 - p_2) q_2 & (1 -p_2) (1 - q_2) \\
        p_3 q_3 & p_3 (1 - q_3) & (1 - p_3) q_3 & (1 -p_3) (1 - q_3) \\
        p_4 q_4 & p_4 (1 - q_4) & (1 - p_4) q_4 & (1 -p_4) (1 - q_4) \\
    \end{bmatrix}
\]

Figure~\ref{fig:computed_probabilities_vs_theoretic_probabilities} shows a
regression line fitted to every pairwise interaction with a reported
\(\text{SSError}\) value (pairwise interactions with missing states were
omitted). This serves to validate the approach: a part from some edge cases the
relationship is consistent.

\begin{figure}[!htbp]
    \centering
    \includegraphics[width=.8\textwidth]{./assets/img/computed_probabilities_vs_theoretic_probabilities/main.pdf}
    \caption{The
        relationship between the steady state probabilities inferred from the
        measured transitions and the actual steady state probabilities. A linear
        regression line is included validating the approach.}
    \label{fig:computed_probabilities_vs_theoretic_probabilities}
\end{figure}


\end{document}

strategies.

The results of this analysis are shown in
Figure~\ref{fig:SSError_and_probabilities_in_full}. The top ranking strategies
by number of wins seem to be extortionate (but not against all strategies) and
it can be seen that a small sub group of strategies achieve mutual defection.
All the top ranking strategies according to score achieve mutual cooperation and
do not extort each other, however they
\textbf{do} exhibit extortionate behaviour towards a number of the lower ranking
strategies.

\begin{figure}[!htbp]
    \centering
    \includegraphics[width=.8\textwidth]{./assets/img/SSError_and_probabilities_in_full/main.pdf}
    \caption{\(\text{SSError}\) for the strategies for the full tournament. Only
    strategy interactions for which \(p_4=0\) and \(\chi>1\) are displayed.}
    \label{fig:SSError_and_probabilities_in_full}
\end{figure}

\section{Conclusion}\label{sec:conclusion}

This work defines an approach to measure whether or not a player is playing a
strategy that corresponds to an extortionate strategy as defined
in~\cite{Press2012}: a mathematical model for suspicion. Indeed, all
extortionate strategies have been
 classified as lying on a triangular plane.
This rigorous classification fails to be robust to small measurement error, thus
a statistical approach is proposed.
This is done through a linear algebraic approach for approximating the solution
of a linear system. Using this, a large number of pairwise interactions is
simulated and in fact very few strategies are found to act extortionately.

The work of~\cite{Press2012}, whilst showing that a clever approach to taking
advantage of another memory one strategy exists: this is incomplete. Whilst the
elegance of this result is very attractive, just as the simplicity of the
victory of Tit For Tat in Axelrod's original tournaments was, it is incomplete.
Extortionate strategies achieve a high number of wins but they do not
achieve a high score which corresponds to the fitness landscape in an
evolutionary sense. From the large number of interactions a payoff matrix \(S\)
can be measured where \(S_{ij}\) denotes the score (using standard values of
\((R, S, T, P) = (3, 0, 5, 1)\)) of the \(i\)th strategy
against the \(j\)th strategy. Using this, the replicator equation
describes the evolution of the system based on a population density fitness
function:

\begin{equation}\label{eqn:replicator_dynamics}
    \frac{dx}{dt} = x(S-x^TS x)
\end{equation}

Equation (\ref{eqn:replicator_dynamics}) is solved numerically through an
integration technique described in~\cite{Petzold1983} and
Figure~\ref{fig:replicator_dynamics} shows the evolution of the distribution of
the system: the various strategies are ranked by scores. It is clear to see that
only the high ranking strategies survive the evolutionary process (in fact,
only \documentclass[a4paper]{article}

\usepackage{amsmath}
\usepackage{amssymb}
\usepackage[margin=1.5cm,
            includefoot,
            footskip=30pt]{geometry}
\usepackage{layout}
\usepackage{graphicx}
\usepackage{subcaption}

\usepackage{biblatex}
\usepackage{pdfpages}

\bibliography{main.bib}

\title{Suspicion: Recognising and evaluating the effectiveness
       of extortion in the Iterated Prisoner's Dilemma}
\author{Vincent A. Knight \and Nikoleta E. Glynatsi}
\date{\today}



\begin{document}

\maketitle

\begin{abstract}
    The Iterated Prisoner's Dilemma is a model for rational and evolutionary
    interactive behaviour. It has applications both in the study of human social
    behaviour as well as in biology.
    It is used to understand when and how a rational individual might
    accept an immediate cost to their own utility for the direct benefit of
    another.

    Much attention has been given to a class of strategies called
    Zero Determinant strategies. It has been theoretically shown that these
    strategies can ``extort'' any player.

    In this work, an approach to identify if observed strategies are playing in
    an extortionate way is described. Furthermore, experimental analysis of
    a large tournament with \input{assets/tex/number_of_full_strategies/main.tex}
    strategies is considered. In this setting
    the most highly performing strategies do not play in an extortionate way
    against each other but do against lower performing strategies.
    This suggests that whilst the theory of Zero Determinant strategies
    indicates that memory is not of fundamental importance to the evolution of
    cooperative behaviour, this is incomplete.
\end{abstract}

\section{Introduction}\label{sec:introduction}

Agent based game theoretic models have become a stalwart of the underpinning
mathematics of interactive behaviours. One of the major pieces of work
in this area is the pair of original computer tournaments run by Robert
Axelrod~\cite{Axelrod1980, Axelrod1980a}. These tournaments pitted submitted
computer strategies against each other in plays of the Iterated Prisoner's
Dilemma. A common game where agents can choose to pay a slight cost to their
immediate utility in the hope of building a reputation. This has been used in
economic and evolutionary game theory to understand the evolution of cooperative
behaviour.

Recently, a class of strategies was described in~\cite{Press2012} that can
provably extort any given opponent. In~\cite{Hilbe2013, Moran1707} some
questions have already been asked about the true effectiveness of these
strategies in an evolutionary setting. Here another question is asked: is it
possible to recognise this extortionate behaviour? A mathematical procedure for
suspicion is presented: in the same way that the continued actions of an
extortionate individual might raise suspicion.

This work makes use of the Axelrod Python library~\cite{Knight2018, Knight2016}
with a large number of Prisoner Dilemma strategies available to give an
extensive numerical example of the ideas presented.  The approach is presented
in Section~\ref{sec:delta-zd-strategies}.  All of the code and data discussed
in Section~\ref{sec:numerical-experiments} is open sourced, archived and
written according to best scientific principles~\cite{Wilson2014}. The data
archive can be found at~\cite{vincent_knight_2018_1297075}.

\section{Recognising Extortion}\label{sec:delta-zd-strategies}

In~\cite{Press2012}, given a match between 2 memory-one strategies, the concept
of Zero Determinant (ZD) strategies is introduced. The main result of that paper
shows that given two memory one players \(p, q\in\mathbb{R}^4\) a linear
relationship between the players' scores could be forced by one of the players.

Using the notation of~\cite{Press2012}, assuming the utilities for player \(p\)
are given by \(S_x=(R, S, T, P)\) and for player \(q\) by \(S_y=(R, T, S, P)\)
and that the stationary scores of each player is given by \(S_X\) and \(S_Y\)
respectively. The main result of~\cite{Press2012} is that if

\begin{equation}\label{eqn:linear_relationship_for_p}
    \tilde p=\alpha S_x + \beta S_y + \gamma
\end{equation}

or

\begin{equation}\label{eqn:linear_relationship_for_q}
    \tilde q=\alpha S_x + \beta S_y + \gamma
\end{equation}

where \(\tilde p = (1 - p_1, 1 - p_2, p_3, p_4)\) and
\(\tilde q = (1 - q_1, 1 - q_2, q_3, q_4)\) then:

\begin{equation}
    \alpha S_X + \beta S_Y + \gamma = 0
\end{equation}

In~\cite{Press2012} a particular type of ZD strategy is defined: extortionate
strategies. If:

\begin{equation}\label{eqn:constraint_for_extortion}
    \gamma = - P(\alpha + \beta)
\end{equation}

then the player can ensure they get a score \(\chi\) times
larger than the opponent. This extortion coefficient is given by:

\begin{equation}\label{eqn:definition_of_chi}
    \chi=\frac{-\beta}{\alpha}
\end{equation}

Thus, if (\ref{eqn:constraint_for_extortion}) holds and \(\chi >1\) a player is
said to extort their opponent.
Here, the reverse problem is considered: given a
\(p\in\mathbb{R}^4\) how does one identify \(\alpha, \beta\) if they
exist and is the strategy in fact acting in an extortionate way?

These conditions correspond to:

\begin{align}
    \tilde p_1 & = \alpha R + \beta R - P (\alpha + \beta)
            \label{eqn:condition_for_tilde_p1}\\
    \tilde p_2 & = \alpha S + \beta T - P (\alpha + \beta)
            \label{eqn:condition_for_tilde_p2}\\
    \tilde p_3 & = \alpha T + \beta S - P (\alpha + \beta)
            \label{eqn:condition_for_tilde_p3}\\
    \tilde p_4 & = \alpha P + \beta P - P (\alpha + \beta)
            \label{eqn:condition_for_tilde_p4}
\end{align}

Equation (\ref{eqn:condition_for_tilde_p4}) ensures that \(p_4=\tilde p_4=0\).
Equations (\ref{eqn:condition_for_tilde_p1}-\ref{eqn:condition_for_tilde_p3})
can be used to eliminate \(\alpha, \beta\), giving:

\begin{equation}\label{eqn:planar_definition_of_extortion}
    \tilde p_1 = \frac{(R - P)(\tilde p_2 + \tilde p_3)}{S + T - 2P}
\end{equation}

with:

\begin{equation}\label{eqn:definition_of_chi}
    \chi = \frac{\tilde p_2 (P - T) + \tilde p_3 (S - P)}
                {\tilde p_2 (P - S) + \tilde p_3 (T - P)}
\end{equation}

Given a strategy \(p\in\mathbb{R}^{4\times 1}\) equations
(\ref{eqn:condition_for_tilde_p4}), (\ref{eqn:planar_definition_of_extortion}-\ref{eqn:definition_of_chi}) can be used to check if
a strategy is extortionate. The conditions correspond to:

\begin{align}
    p_1 & = \frac{(R-P)(p_2 + p_3) - R + T + S - P}{S + T - 2P}
     \label{eqn:condition_for_p1}\\
    p_4 & = 0 \label{eqn:condition_for_p4}\\
    1 & > p_2 + p_3\label{eqn:condition_for_chi}
\end{align}

The algebraic steps necessary to prove these results are available in the
supporting materials.

All extortionate strategies reside on a triangular (\ref{eqn:condition_for_chi})
plane (\ref{eqn:condition_for_p1}) in 3 dimensions (\ref{eqn:condition_for_p4}).
Using this formulation it can be seen that a necessary (but not sufficient)
condition for an extortionate strategy is that it cooperates on average less
than 50\% of the time when in a state of disagreement with the opponent.

As an example, consider the known extortionate strategy \(p=(8 / 9, 1 / 2, 1 /
3, 0)\) from~\cite{Stewart2012} which is referred to as \texttt{Extort-2}. In
this case, for the standard values of \((R, T, S, P)\) constraint
(\ref{eqn:condition_for_p1}) corresponds to:

\begin{equation}
    p_1 = \frac{2(p_2 + p_3) + 1}{3}
\end{equation}

It is clear that in this case all constraints hold.

This approach could in fact be used to confirm that a given strategy is acting
in an extortionate manner even if it is not a memory one strategy. However, in
practice, if a closed form for \(p\) is not known, then due to measurement
and/or numerical error this would not work.

This problem can be written in the following linear algebraic form where
\(x=(\alpha, \beta)\)
and \(p^*=(\tilde p_1 - 1, tilde_2 - 1, p_3)\):

\begin{equation}\label{eqn:linear_algebraic_equation_for_p}
    Cx= p^*
\end{equation}

\(C\) corresponds to equations
(\ref{eqn:condition_for_tilde_p1}-\ref{eqn:condition_for_tilde_p3}) and is
given by:

\begin{equation}\label{eqn:definition_of_C}
    C =
    \begin{bmatrix}
        R - P & R- P \\
        S - P & T- P \\
        T - P & S- P \\
    \end{bmatrix}
\end{equation}

Note that in general, equation (\ref{eqn:linear_algebraic_equation_for_p}) will
not necessarily have a solution. From the Rouch\'{e}-Capelli theorem if there is
a solution it is unique as \(\text{rank}(C)=2\) which is the dimension of the
variable \(x\). The best fitting \(x\) is found by minimizing:

\begin{equation}\label{eqn:r_squared}
    \text{SSError} = \|C x- p^*\|_2^2 = \sum_{i=1}^{3}\left((C\bar x)_i-p_i^*\right)^2
\end{equation}

Note that \(\text{SSError}\), which is the square of the Frobenius
norm~\cite{Golub2013}, becomes a measure of how close a strategy is to being an
extortionate strategy. Suspicion
of extortion then corresponds to a threshold on \(\text{SSError}\).

By observing interactions (human or otherwise), their memory one representation
can be inferred and this approach can be used to recognise extortionate
behaviour. The notion of comparing theoretic and actual plays of the IPD is not
novel, see for example~\cite{Rand2013}. Immediately it is noted that if the
environment is noisy~\cite{Wu1995} then no strategy can be considered to be
extortionate as \(p_4>0\).

In the next section, this idea will be illustrated by observing the interactions
that take place in a computer based tournament of the IPD\@.

\section{Numerical experiments}\label{sec:numerical-experiments}

In~\cite{Stewart2012} results from a tournament with
\input{./assets/tex/number_of_stewart_plotkin_strategies/main.tex} strategies,
was presented with specific consideration given to ZD strategies. This
tournament is reproduced here using the Axelrod-Python
project~\cite{Knight2016}. To obtain a good measure of the corresponding
transition rates for each strategy all matches have been run for
\input{assets/tex/number_of_turns/main.tex} turns and every match has been
repeated \input{assets/tex/number_of_repetitions/main.tex} times. All of this
interaction data is available at~\cite{vincent_knight_2018_1297075}. A good
match between the inferred Markov chain and the state distribution of the actual
interactions has been verified. Data for this is presented in the supplementary
materials.

Figure~\ref{fig:SSError_overall_in_stewart_plotkin} shows the \(\text{SSError}\)
values for all the strategies in the tournament, as reported
in~\cite{Stewart2012} the extortionate strategy (which has an expected
\(\text{SSError}\) approximately 0) gains a large number of wins.

\begin{figure}[!htbp]
    \centering
    \includegraphics[width=.8\textwidth]{./assets/img/SSError_overall_in_stewart_plotkin/main.pdf}
    \caption{\(\text{SSError}\) and state probabilities for the strategies
        of~\cite{Stewart2012}, ordered both by number of wins and overall score.
        Note that \(P(DC)\) is not shown as it corresponds to the transpose of
        \(P(CD)\). Cooperator and Defector are omitted as they do not visit all
        the states.}
    \label{fig:SSError_overall_in_stewart_plotkin}
\end{figure}

Here, the work of~\cite{Stewart2012} is extended by investigating a tournament
with \input{assets/tex/number_of_full_strategies/main.tex}
strategies.

The results of this analysis are shown in
Figure~\ref{fig:SSError_and_probabilities_in_full}. The top ranking strategies
by number of wins seem to be extortionate (but not against all strategies) and
it can be seen that a small sub group of strategies achieve mutual defection.
All the top ranking strategies according to score achieve mutual cooperation and
do not extort each other, however they
\textbf{do} exhibit extortionate behaviour towards a number of the lower ranking
strategies.

\begin{figure}[!htbp]
    \centering
    \includegraphics[width=.8\textwidth]{./assets/img/SSError_and_probabilities_in_full/main.pdf}
    \caption{\(\text{SSError}\) for the strategies for the full tournament. Only
    strategy interactions for which \(p_4=0\) and \(\chi>1\) are displayed.}
    \label{fig:SSError_and_probabilities_in_full}
\end{figure}

\section{Conclusion}\label{sec:conclusion}

This work defines an approach to measure whether or not a player is playing a
strategy that corresponds to an extortionate strategy as defined
in~\cite{Press2012}: a mathematical model for suspicion. Indeed, all
extortionate strategies have been
 classified as lying on a triangular plane.
This rigorous classification fails to be robust to small measurement error, thus
a statistical approach is proposed.
This is done through a linear algebraic approach for approximating the solution
of a linear system. Using this, a large number of pairwise interactions is
simulated and in fact very few strategies are found to act extortionately.

The work of~\cite{Press2012}, whilst showing that a clever approach to taking
advantage of another memory one strategy exists: this is incomplete. Whilst the
elegance of this result is very attractive, just as the simplicity of the
victory of Tit For Tat in Axelrod's original tournaments was, it is incomplete.
Extortionate strategies achieve a high number of wins but they do not
achieve a high score which corresponds to the fitness landscape in an
evolutionary sense. From the large number of interactions a payoff matrix \(S\)
can be measured where \(S_{ij}\) denotes the score (using standard values of
\((R, S, T, P) = (3, 0, 5, 1)\)) of the \(i\)th strategy
against the \(j\)th strategy. Using this, the replicator equation
describes the evolution of the system based on a population density fitness
function:

\begin{equation}\label{eqn:replicator_dynamics}
    \frac{dx}{dt} = x(S-x^TS x)
\end{equation}

Equation (\ref{eqn:replicator_dynamics}) is solved numerically through an
integration technique described in~\cite{Petzold1983} and
Figure~\ref{fig:replicator_dynamics} shows the evolution of the distribution of
the system: the various strategies are ranked by scores. It is clear to see that
only the high ranking strategies survive the evolutionary process (in fact,
only \input{./assets/img/replicator_dynamics/main.tex}
have a final distribution greater than \(10 ^ {-2}\)). This confirms the
findings of~\cite{Moran1707} in which sophisticated strategies resist
evolutionary invasion of shorter memory strategies. Recalling
Figure~\ref{fig:SSError_and_probabilities_in_full} this demonstrates that:

\begin{itemize}
    \item Cooperation emerges through the evolutionary process: the high scoring
        strategies do not exhibit extortionate behaviour towards each other.
    \item Extortionate strategies do not survive the evolutionary process.
\end{itemize}

\begin{figure}[!htbp]
    \centering
    \includegraphics[width=.8\textwidth]{./assets/img/replicator_dynamics/main.pdf}
    \caption{Numerical simulation of the replicator equation
    (\ref{eqn:replicator_dynamics}): strategies are ordered by score, only the strategies with a high score survive the evolutionary process.}
    \label{fig:replicator_dynamics}
\end{figure}

This work can be used to classify plays of the IPD\@: data can be collected from
actual interactions (in lab or in the field). Furthermore, this allows for a
classification method similar to the notion of fingerprinting presented
in~\cite{Ashlock2008}. Trained strategies can potentially be classified as
extortionate or not or it could be possible to even constrain the reinforcement
learning approaches that are becoming prevalent in the literature.
Alternatively, this mathematical approach for recognising extortion could be
used in sophisticated strategies to defend against invasion. Arguably, some of
the strategies considered here exhibit this behaviour, indeed as described
in~\cite{Harper2017}, the top ranking strategies in the full tournament are
obtained using evolutionary reinforcement learning techniques, thus, suspicion
of extortionate behaviour could in fact be an evolutionary trait.

\section*{Acknowledgements}

The following open source software libraries were used in this research:

\begin{itemize}
    \item The Axelrod ~\cite{Knight2016, Knight2018} library (IPD strategies and
        tournaments).
    \item The sympy library~\cite{Meurer2017} (verification of all symbolic
        calculations).
    \item The matplotlib~\cite{Droettboom2018} library (visualisation).
    \item The pandas~\cite{Structures2010}, dask~\cite{Dask2016} and
        NumPy~\cite{Oliphant2015} libraries (data manipulation).
    \item The SciPy~\cite{Jones2001} library (numerical integration of the
        replicator equation).
\end{itemize}

This work was performed using the computational facilities of the Advanced
Research Computing @ Cardiff (ARCCA) Division, Cardiff University.

\printbibliography

\newpage
\section*{Supplementary materials}

\includepdf{assets/pdf/proof_of_form_of_extortionate_strategies/main.pdf}

\newpage

Using the pair wise interactions the transition rates \(p,
q\) can be measured and the steady state probabilities inferred and compared to
the actual probabilities of each state.
This is done numerically by computing the singular eigenvector of the
matrix \(A\) \cite{Stewart2009}:

\[
    A =
    \begin{bmatrix}
        p_1 q_1 & p_1 (1 - q_1) & (1 - p_1) q_1 & (1 -p_1) (1 - q_1) \\
        p_2 q_2 & p_2 (1 - q_2) & (1 - p_2) q_2 & (1 -p_2) (1 - q_2) \\
        p_3 q_3 & p_3 (1 - q_3) & (1 - p_3) q_3 & (1 -p_3) (1 - q_3) \\
        p_4 q_4 & p_4 (1 - q_4) & (1 - p_4) q_4 & (1 -p_4) (1 - q_4) \\
    \end{bmatrix}
\]

Figure~\ref{fig:computed_probabilities_vs_theoretic_probabilities} shows a
regression line fitted to every pairwise interaction with a reported
\(\text{SSError}\) value (pairwise interactions with missing states were
omitted). This serves to validate the approach: a part from some edge cases the
relationship is consistent.

\begin{figure}[!htbp]
    \centering
    \includegraphics[width=.8\textwidth]{./assets/img/computed_probabilities_vs_theoretic_probabilities/main.pdf}
    \caption{The
        relationship between the steady state probabilities inferred from the
        measured transitions and the actual steady state probabilities. A linear
        regression line is included validating the approach.}
    \label{fig:computed_probabilities_vs_theoretic_probabilities}
\end{figure}


\end{document}

have a final distribution greater than \(10 ^ {-2}\)). This confirms the
findings of~\cite{Moran1707} in which sophisticated strategies resist
evolutionary invasion of shorter memory strategies. Recalling
Figure~\ref{fig:SSError_and_probabilities_in_full} this demonstrates that:

\begin{itemize}
    \item Cooperation emerges through the evolutionary process: the high scoring
        strategies do not exhibit extortionate behaviour towards each other.
    \item Extortionate strategies do not survive the evolutionary process.
\end{itemize}

\begin{figure}[!htbp]
    \centering
    \includegraphics[width=.8\textwidth]{./assets/img/replicator_dynamics/main.pdf}
    \caption{Numerical simulation of the replicator equation
    (\ref{eqn:replicator_dynamics}): strategies are ordered by score, only the strategies with a high score survive the evolutionary process.}
    \label{fig:replicator_dynamics}
\end{figure}

This work can be used to classify plays of the IPD\@: data can be collected from
actual interactions (in lab or in the field). Furthermore, this allows for a
classification method similar to the notion of fingerprinting presented
in~\cite{Ashlock2008}. Trained strategies can potentially be classified as
extortionate or not or it could be possible to even constrain the reinforcement
learning approaches that are becoming prevalent in the literature.
Alternatively, this mathematical approach for recognising extortion could be
used in sophisticated strategies to defend against invasion. Arguably, some of
the strategies considered here exhibit this behaviour, indeed as described
in~\cite{Harper2017}, the top ranking strategies in the full tournament are
obtained using evolutionary reinforcement learning techniques, thus, suspicion
of extortionate behaviour could in fact be an evolutionary trait.

\section*{Acknowledgements}

The following open source software libraries were used in this research:

\begin{itemize}
    \item The Axelrod ~\cite{Knight2016, Knight2018} library (IPD strategies and
        tournaments).
    \item The sympy library~\cite{Meurer2017} (verification of all symbolic
        calculations).
    \item The matplotlib~\cite{Droettboom2018} library (visualisation).
    \item The pandas~\cite{Structures2010}, dask~\cite{Dask2016} and
        NumPy~\cite{Oliphant2015} libraries (data manipulation).
    \item The SciPy~\cite{Jones2001} library (numerical integration of the
        replicator equation).
\end{itemize}

This work was performed using the computational facilities of the Advanced
Research Computing @ Cardiff (ARCCA) Division, Cardiff University.

\printbibliography

\newpage
\section*{Supplementary materials}

\includepdf{assets/pdf/proof_of_form_of_extortionate_strategies/main.pdf}

\newpage

Using the pair wise interactions the transition rates \(p,
q\) can be measured and the steady state probabilities inferred and compared to
the actual probabilities of each state.
This is done numerically by computing the singular eigenvector of the
matrix \(A\) \cite{Stewart2009}:

\[
    A =
    \begin{bmatrix}
        p_1 q_1 & p_1 (1 - q_1) & (1 - p_1) q_1 & (1 -p_1) (1 - q_1) \\
        p_2 q_2 & p_2 (1 - q_2) & (1 - p_2) q_2 & (1 -p_2) (1 - q_2) \\
        p_3 q_3 & p_3 (1 - q_3) & (1 - p_3) q_3 & (1 -p_3) (1 - q_3) \\
        p_4 q_4 & p_4 (1 - q_4) & (1 - p_4) q_4 & (1 -p_4) (1 - q_4) \\
    \end{bmatrix}
\]

Figure~\ref{fig:computed_probabilities_vs_theoretic_probabilities} shows a
regression line fitted to every pairwise interaction with a reported
\(\text{SSError}\) value (pairwise interactions with missing states were
omitted). This serves to validate the approach: a part from some edge cases the
relationship is consistent.

\begin{figure}[!htbp]
    \centering
    \includegraphics[width=.8\textwidth]{./assets/img/computed_probabilities_vs_theoretic_probabilities/main.pdf}
    \caption{The
        relationship between the steady state probabilities inferred from the
        measured transitions and the actual steady state probabilities. A linear
        regression line is included validating the approach.}
    \label{fig:computed_probabilities_vs_theoretic_probabilities}
\end{figure}


\end{document}
 turns and every match has been
repeated \documentclass[a4paper]{article}

\usepackage{amsmath}
\usepackage{amssymb}
\usepackage[margin=1.5cm,
            includefoot,
            footskip=30pt]{geometry}
\usepackage{layout}
\usepackage{graphicx}
\usepackage{subcaption}

\usepackage{biblatex}
\usepackage{pdfpages}

\bibliography{main.bib}

\title{Suspicion: Recognising and evaluating the effectiveness
       of extortion in the Iterated Prisoner's Dilemma}
\author{Vincent A. Knight \and Nikoleta E. Glynatsi}
\date{\today}



\begin{document}

\maketitle

\begin{abstract}
    The Iterated Prisoner's Dilemma is a model for rational and evolutionary
    interactive behaviour. It has applications both in the study of human social
    behaviour as well as in biology.
    It is used to understand when and how a rational individual might
    accept an immediate cost to their own utility for the direct benefit of
    another.

    Much attention has been given to a class of strategies called
    Zero Determinant strategies. It has been theoretically shown that these
    strategies can ``extort'' any player.

    In this work, an approach to identify if observed strategies are playing in
    an extortionate way is described. Furthermore, experimental analysis of
    a large tournament with \documentclass[a4paper]{article}

\usepackage{amsmath}
\usepackage{amssymb}
\usepackage[margin=1.5cm,
            includefoot,
            footskip=30pt]{geometry}
\usepackage{layout}
\usepackage{graphicx}
\usepackage{subcaption}

\usepackage{biblatex}
\usepackage{pdfpages}

\bibliography{main.bib}

\title{Suspicion: Recognising and evaluating the effectiveness
       of extortion in the Iterated Prisoner's Dilemma}
\author{Vincent A. Knight \and Nikoleta E. Glynatsi}
\date{\today}



\begin{document}

\maketitle

\begin{abstract}
    The Iterated Prisoner's Dilemma is a model for rational and evolutionary
    interactive behaviour. It has applications both in the study of human social
    behaviour as well as in biology.
    It is used to understand when and how a rational individual might
    accept an immediate cost to their own utility for the direct benefit of
    another.

    Much attention has been given to a class of strategies called
    Zero Determinant strategies. It has been theoretically shown that these
    strategies can ``extort'' any player.

    In this work, an approach to identify if observed strategies are playing in
    an extortionate way is described. Furthermore, experimental analysis of
    a large tournament with \input{assets/tex/number_of_full_strategies/main.tex}
    strategies is considered. In this setting
    the most highly performing strategies do not play in an extortionate way
    against each other but do against lower performing strategies.
    This suggests that whilst the theory of Zero Determinant strategies
    indicates that memory is not of fundamental importance to the evolution of
    cooperative behaviour, this is incomplete.
\end{abstract}

\section{Introduction}\label{sec:introduction}

Agent based game theoretic models have become a stalwart of the underpinning
mathematics of interactive behaviours. One of the major pieces of work
in this area is the pair of original computer tournaments run by Robert
Axelrod~\cite{Axelrod1980, Axelrod1980a}. These tournaments pitted submitted
computer strategies against each other in plays of the Iterated Prisoner's
Dilemma. A common game where agents can choose to pay a slight cost to their
immediate utility in the hope of building a reputation. This has been used in
economic and evolutionary game theory to understand the evolution of cooperative
behaviour.

Recently, a class of strategies was described in~\cite{Press2012} that can
provably extort any given opponent. In~\cite{Hilbe2013, Moran1707} some
questions have already been asked about the true effectiveness of these
strategies in an evolutionary setting. Here another question is asked: is it
possible to recognise this extortionate behaviour? A mathematical procedure for
suspicion is presented: in the same way that the continued actions of an
extortionate individual might raise suspicion.

This work makes use of the Axelrod Python library~\cite{Knight2018, Knight2016}
with a large number of Prisoner Dilemma strategies available to give an
extensive numerical example of the ideas presented.  The approach is presented
in Section~\ref{sec:delta-zd-strategies}.  All of the code and data discussed
in Section~\ref{sec:numerical-experiments} is open sourced, archived and
written according to best scientific principles~\cite{Wilson2014}. The data
archive can be found at~\cite{vincent_knight_2018_1297075}.

\section{Recognising Extortion}\label{sec:delta-zd-strategies}

In~\cite{Press2012}, given a match between 2 memory-one strategies, the concept
of Zero Determinant (ZD) strategies is introduced. The main result of that paper
shows that given two memory one players \(p, q\in\mathbb{R}^4\) a linear
relationship between the players' scores could be forced by one of the players.

Using the notation of~\cite{Press2012}, assuming the utilities for player \(p\)
are given by \(S_x=(R, S, T, P)\) and for player \(q\) by \(S_y=(R, T, S, P)\)
and that the stationary scores of each player is given by \(S_X\) and \(S_Y\)
respectively. The main result of~\cite{Press2012} is that if

\begin{equation}\label{eqn:linear_relationship_for_p}
    \tilde p=\alpha S_x + \beta S_y + \gamma
\end{equation}

or

\begin{equation}\label{eqn:linear_relationship_for_q}
    \tilde q=\alpha S_x + \beta S_y + \gamma
\end{equation}

where \(\tilde p = (1 - p_1, 1 - p_2, p_3, p_4)\) and
\(\tilde q = (1 - q_1, 1 - q_2, q_3, q_4)\) then:

\begin{equation}
    \alpha S_X + \beta S_Y + \gamma = 0
\end{equation}

In~\cite{Press2012} a particular type of ZD strategy is defined: extortionate
strategies. If:

\begin{equation}\label{eqn:constraint_for_extortion}
    \gamma = - P(\alpha + \beta)
\end{equation}

then the player can ensure they get a score \(\chi\) times
larger than the opponent. This extortion coefficient is given by:

\begin{equation}\label{eqn:definition_of_chi}
    \chi=\frac{-\beta}{\alpha}
\end{equation}

Thus, if (\ref{eqn:constraint_for_extortion}) holds and \(\chi >1\) a player is
said to extort their opponent.
Here, the reverse problem is considered: given a
\(p\in\mathbb{R}^4\) how does one identify \(\alpha, \beta\) if they
exist and is the strategy in fact acting in an extortionate way?

These conditions correspond to:

\begin{align}
    \tilde p_1 & = \alpha R + \beta R - P (\alpha + \beta)
            \label{eqn:condition_for_tilde_p1}\\
    \tilde p_2 & = \alpha S + \beta T - P (\alpha + \beta)
            \label{eqn:condition_for_tilde_p2}\\
    \tilde p_3 & = \alpha T + \beta S - P (\alpha + \beta)
            \label{eqn:condition_for_tilde_p3}\\
    \tilde p_4 & = \alpha P + \beta P - P (\alpha + \beta)
            \label{eqn:condition_for_tilde_p4}
\end{align}

Equation (\ref{eqn:condition_for_tilde_p4}) ensures that \(p_4=\tilde p_4=0\).
Equations (\ref{eqn:condition_for_tilde_p1}-\ref{eqn:condition_for_tilde_p3})
can be used to eliminate \(\alpha, \beta\), giving:

\begin{equation}\label{eqn:planar_definition_of_extortion}
    \tilde p_1 = \frac{(R - P)(\tilde p_2 + \tilde p_3)}{S + T - 2P}
\end{equation}

with:

\begin{equation}\label{eqn:definition_of_chi}
    \chi = \frac{\tilde p_2 (P - T) + \tilde p_3 (S - P)}
                {\tilde p_2 (P - S) + \tilde p_3 (T - P)}
\end{equation}

Given a strategy \(p\in\mathbb{R}^{4\times 1}\) equations
(\ref{eqn:condition_for_tilde_p4}), (\ref{eqn:planar_definition_of_extortion}-\ref{eqn:definition_of_chi}) can be used to check if
a strategy is extortionate. The conditions correspond to:

\begin{align}
    p_1 & = \frac{(R-P)(p_2 + p_3) - R + T + S - P}{S + T - 2P}
     \label{eqn:condition_for_p1}\\
    p_4 & = 0 \label{eqn:condition_for_p4}\\
    1 & > p_2 + p_3\label{eqn:condition_for_chi}
\end{align}

The algebraic steps necessary to prove these results are available in the
supporting materials.

All extortionate strategies reside on a triangular (\ref{eqn:condition_for_chi})
plane (\ref{eqn:condition_for_p1}) in 3 dimensions (\ref{eqn:condition_for_p4}).
Using this formulation it can be seen that a necessary (but not sufficient)
condition for an extortionate strategy is that it cooperates on average less
than 50\% of the time when in a state of disagreement with the opponent.

As an example, consider the known extortionate strategy \(p=(8 / 9, 1 / 2, 1 /
3, 0)\) from~\cite{Stewart2012} which is referred to as \texttt{Extort-2}. In
this case, for the standard values of \((R, T, S, P)\) constraint
(\ref{eqn:condition_for_p1}) corresponds to:

\begin{equation}
    p_1 = \frac{2(p_2 + p_3) + 1}{3}
\end{equation}

It is clear that in this case all constraints hold.

This approach could in fact be used to confirm that a given strategy is acting
in an extortionate manner even if it is not a memory one strategy. However, in
practice, if a closed form for \(p\) is not known, then due to measurement
and/or numerical error this would not work.

This problem can be written in the following linear algebraic form where
\(x=(\alpha, \beta)\)
and \(p^*=(\tilde p_1 - 1, tilde_2 - 1, p_3)\):

\begin{equation}\label{eqn:linear_algebraic_equation_for_p}
    Cx= p^*
\end{equation}

\(C\) corresponds to equations
(\ref{eqn:condition_for_tilde_p1}-\ref{eqn:condition_for_tilde_p3}) and is
given by:

\begin{equation}\label{eqn:definition_of_C}
    C =
    \begin{bmatrix}
        R - P & R- P \\
        S - P & T- P \\
        T - P & S- P \\
    \end{bmatrix}
\end{equation}

Note that in general, equation (\ref{eqn:linear_algebraic_equation_for_p}) will
not necessarily have a solution. From the Rouch\'{e}-Capelli theorem if there is
a solution it is unique as \(\text{rank}(C)=2\) which is the dimension of the
variable \(x\). The best fitting \(x\) is found by minimizing:

\begin{equation}\label{eqn:r_squared}
    \text{SSError} = \|C x- p^*\|_2^2 = \sum_{i=1}^{3}\left((C\bar x)_i-p_i^*\right)^2
\end{equation}

Note that \(\text{SSError}\), which is the square of the Frobenius
norm~\cite{Golub2013}, becomes a measure of how close a strategy is to being an
extortionate strategy. Suspicion
of extortion then corresponds to a threshold on \(\text{SSError}\).

By observing interactions (human or otherwise), their memory one representation
can be inferred and this approach can be used to recognise extortionate
behaviour. The notion of comparing theoretic and actual plays of the IPD is not
novel, see for example~\cite{Rand2013}. Immediately it is noted that if the
environment is noisy~\cite{Wu1995} then no strategy can be considered to be
extortionate as \(p_4>0\).

In the next section, this idea will be illustrated by observing the interactions
that take place in a computer based tournament of the IPD\@.

\section{Numerical experiments}\label{sec:numerical-experiments}

In~\cite{Stewart2012} results from a tournament with
\input{./assets/tex/number_of_stewart_plotkin_strategies/main.tex} strategies,
was presented with specific consideration given to ZD strategies. This
tournament is reproduced here using the Axelrod-Python
project~\cite{Knight2016}. To obtain a good measure of the corresponding
transition rates for each strategy all matches have been run for
\input{assets/tex/number_of_turns/main.tex} turns and every match has been
repeated \input{assets/tex/number_of_repetitions/main.tex} times. All of this
interaction data is available at~\cite{vincent_knight_2018_1297075}. A good
match between the inferred Markov chain and the state distribution of the actual
interactions has been verified. Data for this is presented in the supplementary
materials.

Figure~\ref{fig:SSError_overall_in_stewart_plotkin} shows the \(\text{SSError}\)
values for all the strategies in the tournament, as reported
in~\cite{Stewart2012} the extortionate strategy (which has an expected
\(\text{SSError}\) approximately 0) gains a large number of wins.

\begin{figure}[!htbp]
    \centering
    \includegraphics[width=.8\textwidth]{./assets/img/SSError_overall_in_stewart_plotkin/main.pdf}
    \caption{\(\text{SSError}\) and state probabilities for the strategies
        of~\cite{Stewart2012}, ordered both by number of wins and overall score.
        Note that \(P(DC)\) is not shown as it corresponds to the transpose of
        \(P(CD)\). Cooperator and Defector are omitted as they do not visit all
        the states.}
    \label{fig:SSError_overall_in_stewart_plotkin}
\end{figure}

Here, the work of~\cite{Stewart2012} is extended by investigating a tournament
with \input{assets/tex/number_of_full_strategies/main.tex}
strategies.

The results of this analysis are shown in
Figure~\ref{fig:SSError_and_probabilities_in_full}. The top ranking strategies
by number of wins seem to be extortionate (but not against all strategies) and
it can be seen that a small sub group of strategies achieve mutual defection.
All the top ranking strategies according to score achieve mutual cooperation and
do not extort each other, however they
\textbf{do} exhibit extortionate behaviour towards a number of the lower ranking
strategies.

\begin{figure}[!htbp]
    \centering
    \includegraphics[width=.8\textwidth]{./assets/img/SSError_and_probabilities_in_full/main.pdf}
    \caption{\(\text{SSError}\) for the strategies for the full tournament. Only
    strategy interactions for which \(p_4=0\) and \(\chi>1\) are displayed.}
    \label{fig:SSError_and_probabilities_in_full}
\end{figure}

\section{Conclusion}\label{sec:conclusion}

This work defines an approach to measure whether or not a player is playing a
strategy that corresponds to an extortionate strategy as defined
in~\cite{Press2012}: a mathematical model for suspicion. Indeed, all
extortionate strategies have been
 classified as lying on a triangular plane.
This rigorous classification fails to be robust to small measurement error, thus
a statistical approach is proposed.
This is done through a linear algebraic approach for approximating the solution
of a linear system. Using this, a large number of pairwise interactions is
simulated and in fact very few strategies are found to act extortionately.

The work of~\cite{Press2012}, whilst showing that a clever approach to taking
advantage of another memory one strategy exists: this is incomplete. Whilst the
elegance of this result is very attractive, just as the simplicity of the
victory of Tit For Tat in Axelrod's original tournaments was, it is incomplete.
Extortionate strategies achieve a high number of wins but they do not
achieve a high score which corresponds to the fitness landscape in an
evolutionary sense. From the large number of interactions a payoff matrix \(S\)
can be measured where \(S_{ij}\) denotes the score (using standard values of
\((R, S, T, P) = (3, 0, 5, 1)\)) of the \(i\)th strategy
against the \(j\)th strategy. Using this, the replicator equation
describes the evolution of the system based on a population density fitness
function:

\begin{equation}\label{eqn:replicator_dynamics}
    \frac{dx}{dt} = x(S-x^TS x)
\end{equation}

Equation (\ref{eqn:replicator_dynamics}) is solved numerically through an
integration technique described in~\cite{Petzold1983} and
Figure~\ref{fig:replicator_dynamics} shows the evolution of the distribution of
the system: the various strategies are ranked by scores. It is clear to see that
only the high ranking strategies survive the evolutionary process (in fact,
only \input{./assets/img/replicator_dynamics/main.tex}
have a final distribution greater than \(10 ^ {-2}\)). This confirms the
findings of~\cite{Moran1707} in which sophisticated strategies resist
evolutionary invasion of shorter memory strategies. Recalling
Figure~\ref{fig:SSError_and_probabilities_in_full} this demonstrates that:

\begin{itemize}
    \item Cooperation emerges through the evolutionary process: the high scoring
        strategies do not exhibit extortionate behaviour towards each other.
    \item Extortionate strategies do not survive the evolutionary process.
\end{itemize}

\begin{figure}[!htbp]
    \centering
    \includegraphics[width=.8\textwidth]{./assets/img/replicator_dynamics/main.pdf}
    \caption{Numerical simulation of the replicator equation
    (\ref{eqn:replicator_dynamics}): strategies are ordered by score, only the strategies with a high score survive the evolutionary process.}
    \label{fig:replicator_dynamics}
\end{figure}

This work can be used to classify plays of the IPD\@: data can be collected from
actual interactions (in lab or in the field). Furthermore, this allows for a
classification method similar to the notion of fingerprinting presented
in~\cite{Ashlock2008}. Trained strategies can potentially be classified as
extortionate or not or it could be possible to even constrain the reinforcement
learning approaches that are becoming prevalent in the literature.
Alternatively, this mathematical approach for recognising extortion could be
used in sophisticated strategies to defend against invasion. Arguably, some of
the strategies considered here exhibit this behaviour, indeed as described
in~\cite{Harper2017}, the top ranking strategies in the full tournament are
obtained using evolutionary reinforcement learning techniques, thus, suspicion
of extortionate behaviour could in fact be an evolutionary trait.

\section*{Acknowledgements}

The following open source software libraries were used in this research:

\begin{itemize}
    \item The Axelrod ~\cite{Knight2016, Knight2018} library (IPD strategies and
        tournaments).
    \item The sympy library~\cite{Meurer2017} (verification of all symbolic
        calculations).
    \item The matplotlib~\cite{Droettboom2018} library (visualisation).
    \item The pandas~\cite{Structures2010}, dask~\cite{Dask2016} and
        NumPy~\cite{Oliphant2015} libraries (data manipulation).
    \item The SciPy~\cite{Jones2001} library (numerical integration of the
        replicator equation).
\end{itemize}

This work was performed using the computational facilities of the Advanced
Research Computing @ Cardiff (ARCCA) Division, Cardiff University.

\printbibliography

\newpage
\section*{Supplementary materials}

\includepdf{assets/pdf/proof_of_form_of_extortionate_strategies/main.pdf}

\newpage

Using the pair wise interactions the transition rates \(p,
q\) can be measured and the steady state probabilities inferred and compared to
the actual probabilities of each state.
This is done numerically by computing the singular eigenvector of the
matrix \(A\) \cite{Stewart2009}:

\[
    A =
    \begin{bmatrix}
        p_1 q_1 & p_1 (1 - q_1) & (1 - p_1) q_1 & (1 -p_1) (1 - q_1) \\
        p_2 q_2 & p_2 (1 - q_2) & (1 - p_2) q_2 & (1 -p_2) (1 - q_2) \\
        p_3 q_3 & p_3 (1 - q_3) & (1 - p_3) q_3 & (1 -p_3) (1 - q_3) \\
        p_4 q_4 & p_4 (1 - q_4) & (1 - p_4) q_4 & (1 -p_4) (1 - q_4) \\
    \end{bmatrix}
\]

Figure~\ref{fig:computed_probabilities_vs_theoretic_probabilities} shows a
regression line fitted to every pairwise interaction with a reported
\(\text{SSError}\) value (pairwise interactions with missing states were
omitted). This serves to validate the approach: a part from some edge cases the
relationship is consistent.

\begin{figure}[!htbp]
    \centering
    \includegraphics[width=.8\textwidth]{./assets/img/computed_probabilities_vs_theoretic_probabilities/main.pdf}
    \caption{The
        relationship between the steady state probabilities inferred from the
        measured transitions and the actual steady state probabilities. A linear
        regression line is included validating the approach.}
    \label{fig:computed_probabilities_vs_theoretic_probabilities}
\end{figure}


\end{document}

    strategies is considered. In this setting
    the most highly performing strategies do not play in an extortionate way
    against each other but do against lower performing strategies.
    This suggests that whilst the theory of Zero Determinant strategies
    indicates that memory is not of fundamental importance to the evolution of
    cooperative behaviour, this is incomplete.
\end{abstract}

\section{Introduction}\label{sec:introduction}

Agent based game theoretic models have become a stalwart of the underpinning
mathematics of interactive behaviours. One of the major pieces of work
in this area is the pair of original computer tournaments run by Robert
Axelrod~\cite{Axelrod1980, Axelrod1980a}. These tournaments pitted submitted
computer strategies against each other in plays of the Iterated Prisoner's
Dilemma. A common game where agents can choose to pay a slight cost to their
immediate utility in the hope of building a reputation. This has been used in
economic and evolutionary game theory to understand the evolution of cooperative
behaviour.

Recently, a class of strategies was described in~\cite{Press2012} that can
provably extort any given opponent. In~\cite{Hilbe2013, Moran1707} some
questions have already been asked about the true effectiveness of these
strategies in an evolutionary setting. Here another question is asked: is it
possible to recognise this extortionate behaviour? A mathematical procedure for
suspicion is presented: in the same way that the continued actions of an
extortionate individual might raise suspicion.

This work makes use of the Axelrod Python library~\cite{Knight2018, Knight2016}
with a large number of Prisoner Dilemma strategies available to give an
extensive numerical example of the ideas presented.  The approach is presented
in Section~\ref{sec:delta-zd-strategies}.  All of the code and data discussed
in Section~\ref{sec:numerical-experiments} is open sourced, archived and
written according to best scientific principles~\cite{Wilson2014}. The data
archive can be found at~\cite{vincent_knight_2018_1297075}.

\section{Recognising Extortion}\label{sec:delta-zd-strategies}

In~\cite{Press2012}, given a match between 2 memory-one strategies, the concept
of Zero Determinant (ZD) strategies is introduced. The main result of that paper
shows that given two memory one players \(p, q\in\mathbb{R}^4\) a linear
relationship between the players' scores could be forced by one of the players.

Using the notation of~\cite{Press2012}, assuming the utilities for player \(p\)
are given by \(S_x=(R, S, T, P)\) and for player \(q\) by \(S_y=(R, T, S, P)\)
and that the stationary scores of each player is given by \(S_X\) and \(S_Y\)
respectively. The main result of~\cite{Press2012} is that if

\begin{equation}\label{eqn:linear_relationship_for_p}
    \tilde p=\alpha S_x + \beta S_y + \gamma
\end{equation}

or

\begin{equation}\label{eqn:linear_relationship_for_q}
    \tilde q=\alpha S_x + \beta S_y + \gamma
\end{equation}

where \(\tilde p = (1 - p_1, 1 - p_2, p_3, p_4)\) and
\(\tilde q = (1 - q_1, 1 - q_2, q_3, q_4)\) then:

\begin{equation}
    \alpha S_X + \beta S_Y + \gamma = 0
\end{equation}

In~\cite{Press2012} a particular type of ZD strategy is defined: extortionate
strategies. If:

\begin{equation}\label{eqn:constraint_for_extortion}
    \gamma = - P(\alpha + \beta)
\end{equation}

then the player can ensure they get a score \(\chi\) times
larger than the opponent. This extortion coefficient is given by:

\begin{equation}\label{eqn:definition_of_chi}
    \chi=\frac{-\beta}{\alpha}
\end{equation}

Thus, if (\ref{eqn:constraint_for_extortion}) holds and \(\chi >1\) a player is
said to extort their opponent.
Here, the reverse problem is considered: given a
\(p\in\mathbb{R}^4\) how does one identify \(\alpha, \beta\) if they
exist and is the strategy in fact acting in an extortionate way?

These conditions correspond to:

\begin{align}
    \tilde p_1 & = \alpha R + \beta R - P (\alpha + \beta)
            \label{eqn:condition_for_tilde_p1}\\
    \tilde p_2 & = \alpha S + \beta T - P (\alpha + \beta)
            \label{eqn:condition_for_tilde_p2}\\
    \tilde p_3 & = \alpha T + \beta S - P (\alpha + \beta)
            \label{eqn:condition_for_tilde_p3}\\
    \tilde p_4 & = \alpha P + \beta P - P (\alpha + \beta)
            \label{eqn:condition_for_tilde_p4}
\end{align}

Equation (\ref{eqn:condition_for_tilde_p4}) ensures that \(p_4=\tilde p_4=0\).
Equations (\ref{eqn:condition_for_tilde_p1}-\ref{eqn:condition_for_tilde_p3})
can be used to eliminate \(\alpha, \beta\), giving:

\begin{equation}\label{eqn:planar_definition_of_extortion}
    \tilde p_1 = \frac{(R - P)(\tilde p_2 + \tilde p_3)}{S + T - 2P}
\end{equation}

with:

\begin{equation}\label{eqn:definition_of_chi}
    \chi = \frac{\tilde p_2 (P - T) + \tilde p_3 (S - P)}
                {\tilde p_2 (P - S) + \tilde p_3 (T - P)}
\end{equation}

Given a strategy \(p\in\mathbb{R}^{4\times 1}\) equations
(\ref{eqn:condition_for_tilde_p4}), (\ref{eqn:planar_definition_of_extortion}-\ref{eqn:definition_of_chi}) can be used to check if
a strategy is extortionate. The conditions correspond to:

\begin{align}
    p_1 & = \frac{(R-P)(p_2 + p_3) - R + T + S - P}{S + T - 2P}
     \label{eqn:condition_for_p1}\\
    p_4 & = 0 \label{eqn:condition_for_p4}\\
    1 & > p_2 + p_3\label{eqn:condition_for_chi}
\end{align}

The algebraic steps necessary to prove these results are available in the
supporting materials.

All extortionate strategies reside on a triangular (\ref{eqn:condition_for_chi})
plane (\ref{eqn:condition_for_p1}) in 3 dimensions (\ref{eqn:condition_for_p4}).
Using this formulation it can be seen that a necessary (but not sufficient)
condition for an extortionate strategy is that it cooperates on average less
than 50\% of the time when in a state of disagreement with the opponent.

As an example, consider the known extortionate strategy \(p=(8 / 9, 1 / 2, 1 /
3, 0)\) from~\cite{Stewart2012} which is referred to as \texttt{Extort-2}. In
this case, for the standard values of \((R, T, S, P)\) constraint
(\ref{eqn:condition_for_p1}) corresponds to:

\begin{equation}
    p_1 = \frac{2(p_2 + p_3) + 1}{3}
\end{equation}

It is clear that in this case all constraints hold.

This approach could in fact be used to confirm that a given strategy is acting
in an extortionate manner even if it is not a memory one strategy. However, in
practice, if a closed form for \(p\) is not known, then due to measurement
and/or numerical error this would not work.

This problem can be written in the following linear algebraic form where
\(x=(\alpha, \beta)\)
and \(p^*=(\tilde p_1 - 1, tilde_2 - 1, p_3)\):

\begin{equation}\label{eqn:linear_algebraic_equation_for_p}
    Cx= p^*
\end{equation}

\(C\) corresponds to equations
(\ref{eqn:condition_for_tilde_p1}-\ref{eqn:condition_for_tilde_p3}) and is
given by:

\begin{equation}\label{eqn:definition_of_C}
    C =
    \begin{bmatrix}
        R - P & R- P \\
        S - P & T- P \\
        T - P & S- P \\
    \end{bmatrix}
\end{equation}

Note that in general, equation (\ref{eqn:linear_algebraic_equation_for_p}) will
not necessarily have a solution. From the Rouch\'{e}-Capelli theorem if there is
a solution it is unique as \(\text{rank}(C)=2\) which is the dimension of the
variable \(x\). The best fitting \(x\) is found by minimizing:

\begin{equation}\label{eqn:r_squared}
    \text{SSError} = \|C x- p^*\|_2^2 = \sum_{i=1}^{3}\left((C\bar x)_i-p_i^*\right)^2
\end{equation}

Note that \(\text{SSError}\), which is the square of the Frobenius
norm~\cite{Golub2013}, becomes a measure of how close a strategy is to being an
extortionate strategy. Suspicion
of extortion then corresponds to a threshold on \(\text{SSError}\).

By observing interactions (human or otherwise), their memory one representation
can be inferred and this approach can be used to recognise extortionate
behaviour. The notion of comparing theoretic and actual plays of the IPD is not
novel, see for example~\cite{Rand2013}. Immediately it is noted that if the
environment is noisy~\cite{Wu1995} then no strategy can be considered to be
extortionate as \(p_4>0\).

In the next section, this idea will be illustrated by observing the interactions
that take place in a computer based tournament of the IPD\@.

\section{Numerical experiments}\label{sec:numerical-experiments}

In~\cite{Stewart2012} results from a tournament with
\documentclass[a4paper]{article}

\usepackage{amsmath}
\usepackage{amssymb}
\usepackage[margin=1.5cm,
            includefoot,
            footskip=30pt]{geometry}
\usepackage{layout}
\usepackage{graphicx}
\usepackage{subcaption}

\usepackage{biblatex}
\usepackage{pdfpages}

\bibliography{main.bib}

\title{Suspicion: Recognising and evaluating the effectiveness
       of extortion in the Iterated Prisoner's Dilemma}
\author{Vincent A. Knight \and Nikoleta E. Glynatsi}
\date{\today}



\begin{document}

\maketitle

\begin{abstract}
    The Iterated Prisoner's Dilemma is a model for rational and evolutionary
    interactive behaviour. It has applications both in the study of human social
    behaviour as well as in biology.
    It is used to understand when and how a rational individual might
    accept an immediate cost to their own utility for the direct benefit of
    another.

    Much attention has been given to a class of strategies called
    Zero Determinant strategies. It has been theoretically shown that these
    strategies can ``extort'' any player.

    In this work, an approach to identify if observed strategies are playing in
    an extortionate way is described. Furthermore, experimental analysis of
    a large tournament with \input{assets/tex/number_of_full_strategies/main.tex}
    strategies is considered. In this setting
    the most highly performing strategies do not play in an extortionate way
    against each other but do against lower performing strategies.
    This suggests that whilst the theory of Zero Determinant strategies
    indicates that memory is not of fundamental importance to the evolution of
    cooperative behaviour, this is incomplete.
\end{abstract}

\section{Introduction}\label{sec:introduction}

Agent based game theoretic models have become a stalwart of the underpinning
mathematics of interactive behaviours. One of the major pieces of work
in this area is the pair of original computer tournaments run by Robert
Axelrod~\cite{Axelrod1980, Axelrod1980a}. These tournaments pitted submitted
computer strategies against each other in plays of the Iterated Prisoner's
Dilemma. A common game where agents can choose to pay a slight cost to their
immediate utility in the hope of building a reputation. This has been used in
economic and evolutionary game theory to understand the evolution of cooperative
behaviour.

Recently, a class of strategies was described in~\cite{Press2012} that can
provably extort any given opponent. In~\cite{Hilbe2013, Moran1707} some
questions have already been asked about the true effectiveness of these
strategies in an evolutionary setting. Here another question is asked: is it
possible to recognise this extortionate behaviour? A mathematical procedure for
suspicion is presented: in the same way that the continued actions of an
extortionate individual might raise suspicion.

This work makes use of the Axelrod Python library~\cite{Knight2018, Knight2016}
with a large number of Prisoner Dilemma strategies available to give an
extensive numerical example of the ideas presented.  The approach is presented
in Section~\ref{sec:delta-zd-strategies}.  All of the code and data discussed
in Section~\ref{sec:numerical-experiments} is open sourced, archived and
written according to best scientific principles~\cite{Wilson2014}. The data
archive can be found at~\cite{vincent_knight_2018_1297075}.

\section{Recognising Extortion}\label{sec:delta-zd-strategies}

In~\cite{Press2012}, given a match between 2 memory-one strategies, the concept
of Zero Determinant (ZD) strategies is introduced. The main result of that paper
shows that given two memory one players \(p, q\in\mathbb{R}^4\) a linear
relationship between the players' scores could be forced by one of the players.

Using the notation of~\cite{Press2012}, assuming the utilities for player \(p\)
are given by \(S_x=(R, S, T, P)\) and for player \(q\) by \(S_y=(R, T, S, P)\)
and that the stationary scores of each player is given by \(S_X\) and \(S_Y\)
respectively. The main result of~\cite{Press2012} is that if

\begin{equation}\label{eqn:linear_relationship_for_p}
    \tilde p=\alpha S_x + \beta S_y + \gamma
\end{equation}

or

\begin{equation}\label{eqn:linear_relationship_for_q}
    \tilde q=\alpha S_x + \beta S_y + \gamma
\end{equation}

where \(\tilde p = (1 - p_1, 1 - p_2, p_3, p_4)\) and
\(\tilde q = (1 - q_1, 1 - q_2, q_3, q_4)\) then:

\begin{equation}
    \alpha S_X + \beta S_Y + \gamma = 0
\end{equation}

In~\cite{Press2012} a particular type of ZD strategy is defined: extortionate
strategies. If:

\begin{equation}\label{eqn:constraint_for_extortion}
    \gamma = - P(\alpha + \beta)
\end{equation}

then the player can ensure they get a score \(\chi\) times
larger than the opponent. This extortion coefficient is given by:

\begin{equation}\label{eqn:definition_of_chi}
    \chi=\frac{-\beta}{\alpha}
\end{equation}

Thus, if (\ref{eqn:constraint_for_extortion}) holds and \(\chi >1\) a player is
said to extort their opponent.
Here, the reverse problem is considered: given a
\(p\in\mathbb{R}^4\) how does one identify \(\alpha, \beta\) if they
exist and is the strategy in fact acting in an extortionate way?

These conditions correspond to:

\begin{align}
    \tilde p_1 & = \alpha R + \beta R - P (\alpha + \beta)
            \label{eqn:condition_for_tilde_p1}\\
    \tilde p_2 & = \alpha S + \beta T - P (\alpha + \beta)
            \label{eqn:condition_for_tilde_p2}\\
    \tilde p_3 & = \alpha T + \beta S - P (\alpha + \beta)
            \label{eqn:condition_for_tilde_p3}\\
    \tilde p_4 & = \alpha P + \beta P - P (\alpha + \beta)
            \label{eqn:condition_for_tilde_p4}
\end{align}

Equation (\ref{eqn:condition_for_tilde_p4}) ensures that \(p_4=\tilde p_4=0\).
Equations (\ref{eqn:condition_for_tilde_p1}-\ref{eqn:condition_for_tilde_p3})
can be used to eliminate \(\alpha, \beta\), giving:

\begin{equation}\label{eqn:planar_definition_of_extortion}
    \tilde p_1 = \frac{(R - P)(\tilde p_2 + \tilde p_3)}{S + T - 2P}
\end{equation}

with:

\begin{equation}\label{eqn:definition_of_chi}
    \chi = \frac{\tilde p_2 (P - T) + \tilde p_3 (S - P)}
                {\tilde p_2 (P - S) + \tilde p_3 (T - P)}
\end{equation}

Given a strategy \(p\in\mathbb{R}^{4\times 1}\) equations
(\ref{eqn:condition_for_tilde_p4}), (\ref{eqn:planar_definition_of_extortion}-\ref{eqn:definition_of_chi}) can be used to check if
a strategy is extortionate. The conditions correspond to:

\begin{align}
    p_1 & = \frac{(R-P)(p_2 + p_3) - R + T + S - P}{S + T - 2P}
     \label{eqn:condition_for_p1}\\
    p_4 & = 0 \label{eqn:condition_for_p4}\\
    1 & > p_2 + p_3\label{eqn:condition_for_chi}
\end{align}

The algebraic steps necessary to prove these results are available in the
supporting materials.

All extortionate strategies reside on a triangular (\ref{eqn:condition_for_chi})
plane (\ref{eqn:condition_for_p1}) in 3 dimensions (\ref{eqn:condition_for_p4}).
Using this formulation it can be seen that a necessary (but not sufficient)
condition for an extortionate strategy is that it cooperates on average less
than 50\% of the time when in a state of disagreement with the opponent.

As an example, consider the known extortionate strategy \(p=(8 / 9, 1 / 2, 1 /
3, 0)\) from~\cite{Stewart2012} which is referred to as \texttt{Extort-2}. In
this case, for the standard values of \((R, T, S, P)\) constraint
(\ref{eqn:condition_for_p1}) corresponds to:

\begin{equation}
    p_1 = \frac{2(p_2 + p_3) + 1}{3}
\end{equation}

It is clear that in this case all constraints hold.

This approach could in fact be used to confirm that a given strategy is acting
in an extortionate manner even if it is not a memory one strategy. However, in
practice, if a closed form for \(p\) is not known, then due to measurement
and/or numerical error this would not work.

This problem can be written in the following linear algebraic form where
\(x=(\alpha, \beta)\)
and \(p^*=(\tilde p_1 - 1, tilde_2 - 1, p_3)\):

\begin{equation}\label{eqn:linear_algebraic_equation_for_p}
    Cx= p^*
\end{equation}

\(C\) corresponds to equations
(\ref{eqn:condition_for_tilde_p1}-\ref{eqn:condition_for_tilde_p3}) and is
given by:

\begin{equation}\label{eqn:definition_of_C}
    C =
    \begin{bmatrix}
        R - P & R- P \\
        S - P & T- P \\
        T - P & S- P \\
    \end{bmatrix}
\end{equation}

Note that in general, equation (\ref{eqn:linear_algebraic_equation_for_p}) will
not necessarily have a solution. From the Rouch\'{e}-Capelli theorem if there is
a solution it is unique as \(\text{rank}(C)=2\) which is the dimension of the
variable \(x\). The best fitting \(x\) is found by minimizing:

\begin{equation}\label{eqn:r_squared}
    \text{SSError} = \|C x- p^*\|_2^2 = \sum_{i=1}^{3}\left((C\bar x)_i-p_i^*\right)^2
\end{equation}

Note that \(\text{SSError}\), which is the square of the Frobenius
norm~\cite{Golub2013}, becomes a measure of how close a strategy is to being an
extortionate strategy. Suspicion
of extortion then corresponds to a threshold on \(\text{SSError}\).

By observing interactions (human or otherwise), their memory one representation
can be inferred and this approach can be used to recognise extortionate
behaviour. The notion of comparing theoretic and actual plays of the IPD is not
novel, see for example~\cite{Rand2013}. Immediately it is noted that if the
environment is noisy~\cite{Wu1995} then no strategy can be considered to be
extortionate as \(p_4>0\).

In the next section, this idea will be illustrated by observing the interactions
that take place in a computer based tournament of the IPD\@.

\section{Numerical experiments}\label{sec:numerical-experiments}

In~\cite{Stewart2012} results from a tournament with
\input{./assets/tex/number_of_stewart_plotkin_strategies/main.tex} strategies,
was presented with specific consideration given to ZD strategies. This
tournament is reproduced here using the Axelrod-Python
project~\cite{Knight2016}. To obtain a good measure of the corresponding
transition rates for each strategy all matches have been run for
\input{assets/tex/number_of_turns/main.tex} turns and every match has been
repeated \input{assets/tex/number_of_repetitions/main.tex} times. All of this
interaction data is available at~\cite{vincent_knight_2018_1297075}. A good
match between the inferred Markov chain and the state distribution of the actual
interactions has been verified. Data for this is presented in the supplementary
materials.

Figure~\ref{fig:SSError_overall_in_stewart_plotkin} shows the \(\text{SSError}\)
values for all the strategies in the tournament, as reported
in~\cite{Stewart2012} the extortionate strategy (which has an expected
\(\text{SSError}\) approximately 0) gains a large number of wins.

\begin{figure}[!htbp]
    \centering
    \includegraphics[width=.8\textwidth]{./assets/img/SSError_overall_in_stewart_plotkin/main.pdf}
    \caption{\(\text{SSError}\) and state probabilities for the strategies
        of~\cite{Stewart2012}, ordered both by number of wins and overall score.
        Note that \(P(DC)\) is not shown as it corresponds to the transpose of
        \(P(CD)\). Cooperator and Defector are omitted as they do not visit all
        the states.}
    \label{fig:SSError_overall_in_stewart_plotkin}
\end{figure}

Here, the work of~\cite{Stewart2012} is extended by investigating a tournament
with \input{assets/tex/number_of_full_strategies/main.tex}
strategies.

The results of this analysis are shown in
Figure~\ref{fig:SSError_and_probabilities_in_full}. The top ranking strategies
by number of wins seem to be extortionate (but not against all strategies) and
it can be seen that a small sub group of strategies achieve mutual defection.
All the top ranking strategies according to score achieve mutual cooperation and
do not extort each other, however they
\textbf{do} exhibit extortionate behaviour towards a number of the lower ranking
strategies.

\begin{figure}[!htbp]
    \centering
    \includegraphics[width=.8\textwidth]{./assets/img/SSError_and_probabilities_in_full/main.pdf}
    \caption{\(\text{SSError}\) for the strategies for the full tournament. Only
    strategy interactions for which \(p_4=0\) and \(\chi>1\) are displayed.}
    \label{fig:SSError_and_probabilities_in_full}
\end{figure}

\section{Conclusion}\label{sec:conclusion}

This work defines an approach to measure whether or not a player is playing a
strategy that corresponds to an extortionate strategy as defined
in~\cite{Press2012}: a mathematical model for suspicion. Indeed, all
extortionate strategies have been
 classified as lying on a triangular plane.
This rigorous classification fails to be robust to small measurement error, thus
a statistical approach is proposed.
This is done through a linear algebraic approach for approximating the solution
of a linear system. Using this, a large number of pairwise interactions is
simulated and in fact very few strategies are found to act extortionately.

The work of~\cite{Press2012}, whilst showing that a clever approach to taking
advantage of another memory one strategy exists: this is incomplete. Whilst the
elegance of this result is very attractive, just as the simplicity of the
victory of Tit For Tat in Axelrod's original tournaments was, it is incomplete.
Extortionate strategies achieve a high number of wins but they do not
achieve a high score which corresponds to the fitness landscape in an
evolutionary sense. From the large number of interactions a payoff matrix \(S\)
can be measured where \(S_{ij}\) denotes the score (using standard values of
\((R, S, T, P) = (3, 0, 5, 1)\)) of the \(i\)th strategy
against the \(j\)th strategy. Using this, the replicator equation
describes the evolution of the system based on a population density fitness
function:

\begin{equation}\label{eqn:replicator_dynamics}
    \frac{dx}{dt} = x(S-x^TS x)
\end{equation}

Equation (\ref{eqn:replicator_dynamics}) is solved numerically through an
integration technique described in~\cite{Petzold1983} and
Figure~\ref{fig:replicator_dynamics} shows the evolution of the distribution of
the system: the various strategies are ranked by scores. It is clear to see that
only the high ranking strategies survive the evolutionary process (in fact,
only \input{./assets/img/replicator_dynamics/main.tex}
have a final distribution greater than \(10 ^ {-2}\)). This confirms the
findings of~\cite{Moran1707} in which sophisticated strategies resist
evolutionary invasion of shorter memory strategies. Recalling
Figure~\ref{fig:SSError_and_probabilities_in_full} this demonstrates that:

\begin{itemize}
    \item Cooperation emerges through the evolutionary process: the high scoring
        strategies do not exhibit extortionate behaviour towards each other.
    \item Extortionate strategies do not survive the evolutionary process.
\end{itemize}

\begin{figure}[!htbp]
    \centering
    \includegraphics[width=.8\textwidth]{./assets/img/replicator_dynamics/main.pdf}
    \caption{Numerical simulation of the replicator equation
    (\ref{eqn:replicator_dynamics}): strategies are ordered by score, only the strategies with a high score survive the evolutionary process.}
    \label{fig:replicator_dynamics}
\end{figure}

This work can be used to classify plays of the IPD\@: data can be collected from
actual interactions (in lab or in the field). Furthermore, this allows for a
classification method similar to the notion of fingerprinting presented
in~\cite{Ashlock2008}. Trained strategies can potentially be classified as
extortionate or not or it could be possible to even constrain the reinforcement
learning approaches that are becoming prevalent in the literature.
Alternatively, this mathematical approach for recognising extortion could be
used in sophisticated strategies to defend against invasion. Arguably, some of
the strategies considered here exhibit this behaviour, indeed as described
in~\cite{Harper2017}, the top ranking strategies in the full tournament are
obtained using evolutionary reinforcement learning techniques, thus, suspicion
of extortionate behaviour could in fact be an evolutionary trait.

\section*{Acknowledgements}

The following open source software libraries were used in this research:

\begin{itemize}
    \item The Axelrod ~\cite{Knight2016, Knight2018} library (IPD strategies and
        tournaments).
    \item The sympy library~\cite{Meurer2017} (verification of all symbolic
        calculations).
    \item The matplotlib~\cite{Droettboom2018} library (visualisation).
    \item The pandas~\cite{Structures2010}, dask~\cite{Dask2016} and
        NumPy~\cite{Oliphant2015} libraries (data manipulation).
    \item The SciPy~\cite{Jones2001} library (numerical integration of the
        replicator equation).
\end{itemize}

This work was performed using the computational facilities of the Advanced
Research Computing @ Cardiff (ARCCA) Division, Cardiff University.

\printbibliography

\newpage
\section*{Supplementary materials}

\includepdf{assets/pdf/proof_of_form_of_extortionate_strategies/main.pdf}

\newpage

Using the pair wise interactions the transition rates \(p,
q\) can be measured and the steady state probabilities inferred and compared to
the actual probabilities of each state.
This is done numerically by computing the singular eigenvector of the
matrix \(A\) \cite{Stewart2009}:

\[
    A =
    \begin{bmatrix}
        p_1 q_1 & p_1 (1 - q_1) & (1 - p_1) q_1 & (1 -p_1) (1 - q_1) \\
        p_2 q_2 & p_2 (1 - q_2) & (1 - p_2) q_2 & (1 -p_2) (1 - q_2) \\
        p_3 q_3 & p_3 (1 - q_3) & (1 - p_3) q_3 & (1 -p_3) (1 - q_3) \\
        p_4 q_4 & p_4 (1 - q_4) & (1 - p_4) q_4 & (1 -p_4) (1 - q_4) \\
    \end{bmatrix}
\]

Figure~\ref{fig:computed_probabilities_vs_theoretic_probabilities} shows a
regression line fitted to every pairwise interaction with a reported
\(\text{SSError}\) value (pairwise interactions with missing states were
omitted). This serves to validate the approach: a part from some edge cases the
relationship is consistent.

\begin{figure}[!htbp]
    \centering
    \includegraphics[width=.8\textwidth]{./assets/img/computed_probabilities_vs_theoretic_probabilities/main.pdf}
    \caption{The
        relationship between the steady state probabilities inferred from the
        measured transitions and the actual steady state probabilities. A linear
        regression line is included validating the approach.}
    \label{fig:computed_probabilities_vs_theoretic_probabilities}
\end{figure}


\end{document}
 strategies,
was presented with specific consideration given to ZD strategies. This
tournament is reproduced here using the Axelrod-Python
project~\cite{Knight2016}. To obtain a good measure of the corresponding
transition rates for each strategy all matches have been run for
\documentclass[a4paper]{article}

\usepackage{amsmath}
\usepackage{amssymb}
\usepackage[margin=1.5cm,
            includefoot,
            footskip=30pt]{geometry}
\usepackage{layout}
\usepackage{graphicx}
\usepackage{subcaption}

\usepackage{biblatex}
\usepackage{pdfpages}

\bibliography{main.bib}

\title{Suspicion: Recognising and evaluating the effectiveness
       of extortion in the Iterated Prisoner's Dilemma}
\author{Vincent A. Knight \and Nikoleta E. Glynatsi}
\date{\today}



\begin{document}

\maketitle

\begin{abstract}
    The Iterated Prisoner's Dilemma is a model for rational and evolutionary
    interactive behaviour. It has applications both in the study of human social
    behaviour as well as in biology.
    It is used to understand when and how a rational individual might
    accept an immediate cost to their own utility for the direct benefit of
    another.

    Much attention has been given to a class of strategies called
    Zero Determinant strategies. It has been theoretically shown that these
    strategies can ``extort'' any player.

    In this work, an approach to identify if observed strategies are playing in
    an extortionate way is described. Furthermore, experimental analysis of
    a large tournament with \input{assets/tex/number_of_full_strategies/main.tex}
    strategies is considered. In this setting
    the most highly performing strategies do not play in an extortionate way
    against each other but do against lower performing strategies.
    This suggests that whilst the theory of Zero Determinant strategies
    indicates that memory is not of fundamental importance to the evolution of
    cooperative behaviour, this is incomplete.
\end{abstract}

\section{Introduction}\label{sec:introduction}

Agent based game theoretic models have become a stalwart of the underpinning
mathematics of interactive behaviours. One of the major pieces of work
in this area is the pair of original computer tournaments run by Robert
Axelrod~\cite{Axelrod1980, Axelrod1980a}. These tournaments pitted submitted
computer strategies against each other in plays of the Iterated Prisoner's
Dilemma. A common game where agents can choose to pay a slight cost to their
immediate utility in the hope of building a reputation. This has been used in
economic and evolutionary game theory to understand the evolution of cooperative
behaviour.

Recently, a class of strategies was described in~\cite{Press2012} that can
provably extort any given opponent. In~\cite{Hilbe2013, Moran1707} some
questions have already been asked about the true effectiveness of these
strategies in an evolutionary setting. Here another question is asked: is it
possible to recognise this extortionate behaviour? A mathematical procedure for
suspicion is presented: in the same way that the continued actions of an
extortionate individual might raise suspicion.

This work makes use of the Axelrod Python library~\cite{Knight2018, Knight2016}
with a large number of Prisoner Dilemma strategies available to give an
extensive numerical example of the ideas presented.  The approach is presented
in Section~\ref{sec:delta-zd-strategies}.  All of the code and data discussed
in Section~\ref{sec:numerical-experiments} is open sourced, archived and
written according to best scientific principles~\cite{Wilson2014}. The data
archive can be found at~\cite{vincent_knight_2018_1297075}.

\section{Recognising Extortion}\label{sec:delta-zd-strategies}

In~\cite{Press2012}, given a match between 2 memory-one strategies, the concept
of Zero Determinant (ZD) strategies is introduced. The main result of that paper
shows that given two memory one players \(p, q\in\mathbb{R}^4\) a linear
relationship between the players' scores could be forced by one of the players.

Using the notation of~\cite{Press2012}, assuming the utilities for player \(p\)
are given by \(S_x=(R, S, T, P)\) and for player \(q\) by \(S_y=(R, T, S, P)\)
and that the stationary scores of each player is given by \(S_X\) and \(S_Y\)
respectively. The main result of~\cite{Press2012} is that if

\begin{equation}\label{eqn:linear_relationship_for_p}
    \tilde p=\alpha S_x + \beta S_y + \gamma
\end{equation}

or

\begin{equation}\label{eqn:linear_relationship_for_q}
    \tilde q=\alpha S_x + \beta S_y + \gamma
\end{equation}

where \(\tilde p = (1 - p_1, 1 - p_2, p_3, p_4)\) and
\(\tilde q = (1 - q_1, 1 - q_2, q_3, q_4)\) then:

\begin{equation}
    \alpha S_X + \beta S_Y + \gamma = 0
\end{equation}

In~\cite{Press2012} a particular type of ZD strategy is defined: extortionate
strategies. If:

\begin{equation}\label{eqn:constraint_for_extortion}
    \gamma = - P(\alpha + \beta)
\end{equation}

then the player can ensure they get a score \(\chi\) times
larger than the opponent. This extortion coefficient is given by:

\begin{equation}\label{eqn:definition_of_chi}
    \chi=\frac{-\beta}{\alpha}
\end{equation}

Thus, if (\ref{eqn:constraint_for_extortion}) holds and \(\chi >1\) a player is
said to extort their opponent.
Here, the reverse problem is considered: given a
\(p\in\mathbb{R}^4\) how does one identify \(\alpha, \beta\) if they
exist and is the strategy in fact acting in an extortionate way?

These conditions correspond to:

\begin{align}
    \tilde p_1 & = \alpha R + \beta R - P (\alpha + \beta)
            \label{eqn:condition_for_tilde_p1}\\
    \tilde p_2 & = \alpha S + \beta T - P (\alpha + \beta)
            \label{eqn:condition_for_tilde_p2}\\
    \tilde p_3 & = \alpha T + \beta S - P (\alpha + \beta)
            \label{eqn:condition_for_tilde_p3}\\
    \tilde p_4 & = \alpha P + \beta P - P (\alpha + \beta)
            \label{eqn:condition_for_tilde_p4}
\end{align}

Equation (\ref{eqn:condition_for_tilde_p4}) ensures that \(p_4=\tilde p_4=0\).
Equations (\ref{eqn:condition_for_tilde_p1}-\ref{eqn:condition_for_tilde_p3})
can be used to eliminate \(\alpha, \beta\), giving:

\begin{equation}\label{eqn:planar_definition_of_extortion}
    \tilde p_1 = \frac{(R - P)(\tilde p_2 + \tilde p_3)}{S + T - 2P}
\end{equation}

with:

\begin{equation}\label{eqn:definition_of_chi}
    \chi = \frac{\tilde p_2 (P - T) + \tilde p_3 (S - P)}
                {\tilde p_2 (P - S) + \tilde p_3 (T - P)}
\end{equation}

Given a strategy \(p\in\mathbb{R}^{4\times 1}\) equations
(\ref{eqn:condition_for_tilde_p4}), (\ref{eqn:planar_definition_of_extortion}-\ref{eqn:definition_of_chi}) can be used to check if
a strategy is extortionate. The conditions correspond to:

\begin{align}
    p_1 & = \frac{(R-P)(p_2 + p_3) - R + T + S - P}{S + T - 2P}
     \label{eqn:condition_for_p1}\\
    p_4 & = 0 \label{eqn:condition_for_p4}\\
    1 & > p_2 + p_3\label{eqn:condition_for_chi}
\end{align}

The algebraic steps necessary to prove these results are available in the
supporting materials.

All extortionate strategies reside on a triangular (\ref{eqn:condition_for_chi})
plane (\ref{eqn:condition_for_p1}) in 3 dimensions (\ref{eqn:condition_for_p4}).
Using this formulation it can be seen that a necessary (but not sufficient)
condition for an extortionate strategy is that it cooperates on average less
than 50\% of the time when in a state of disagreement with the opponent.

As an example, consider the known extortionate strategy \(p=(8 / 9, 1 / 2, 1 /
3, 0)\) from~\cite{Stewart2012} which is referred to as \texttt{Extort-2}. In
this case, for the standard values of \((R, T, S, P)\) constraint
(\ref{eqn:condition_for_p1}) corresponds to:

\begin{equation}
    p_1 = \frac{2(p_2 + p_3) + 1}{3}
\end{equation}

It is clear that in this case all constraints hold.

This approach could in fact be used to confirm that a given strategy is acting
in an extortionate manner even if it is not a memory one strategy. However, in
practice, if a closed form for \(p\) is not known, then due to measurement
and/or numerical error this would not work.

This problem can be written in the following linear algebraic form where
\(x=(\alpha, \beta)\)
and \(p^*=(\tilde p_1 - 1, tilde_2 - 1, p_3)\):

\begin{equation}\label{eqn:linear_algebraic_equation_for_p}
    Cx= p^*
\end{equation}

\(C\) corresponds to equations
(\ref{eqn:condition_for_tilde_p1}-\ref{eqn:condition_for_tilde_p3}) and is
given by:

\begin{equation}\label{eqn:definition_of_C}
    C =
    \begin{bmatrix}
        R - P & R- P \\
        S - P & T- P \\
        T - P & S- P \\
    \end{bmatrix}
\end{equation}

Note that in general, equation (\ref{eqn:linear_algebraic_equation_for_p}) will
not necessarily have a solution. From the Rouch\'{e}-Capelli theorem if there is
a solution it is unique as \(\text{rank}(C)=2\) which is the dimension of the
variable \(x\). The best fitting \(x\) is found by minimizing:

\begin{equation}\label{eqn:r_squared}
    \text{SSError} = \|C x- p^*\|_2^2 = \sum_{i=1}^{3}\left((C\bar x)_i-p_i^*\right)^2
\end{equation}

Note that \(\text{SSError}\), which is the square of the Frobenius
norm~\cite{Golub2013}, becomes a measure of how close a strategy is to being an
extortionate strategy. Suspicion
of extortion then corresponds to a threshold on \(\text{SSError}\).

By observing interactions (human or otherwise), their memory one representation
can be inferred and this approach can be used to recognise extortionate
behaviour. The notion of comparing theoretic and actual plays of the IPD is not
novel, see for example~\cite{Rand2013}. Immediately it is noted that if the
environment is noisy~\cite{Wu1995} then no strategy can be considered to be
extortionate as \(p_4>0\).

In the next section, this idea will be illustrated by observing the interactions
that take place in a computer based tournament of the IPD\@.

\section{Numerical experiments}\label{sec:numerical-experiments}

In~\cite{Stewart2012} results from a tournament with
\input{./assets/tex/number_of_stewart_plotkin_strategies/main.tex} strategies,
was presented with specific consideration given to ZD strategies. This
tournament is reproduced here using the Axelrod-Python
project~\cite{Knight2016}. To obtain a good measure of the corresponding
transition rates for each strategy all matches have been run for
\input{assets/tex/number_of_turns/main.tex} turns and every match has been
repeated \input{assets/tex/number_of_repetitions/main.tex} times. All of this
interaction data is available at~\cite{vincent_knight_2018_1297075}. A good
match between the inferred Markov chain and the state distribution of the actual
interactions has been verified. Data for this is presented in the supplementary
materials.

Figure~\ref{fig:SSError_overall_in_stewart_plotkin} shows the \(\text{SSError}\)
values for all the strategies in the tournament, as reported
in~\cite{Stewart2012} the extortionate strategy (which has an expected
\(\text{SSError}\) approximately 0) gains a large number of wins.

\begin{figure}[!htbp]
    \centering
    \includegraphics[width=.8\textwidth]{./assets/img/SSError_overall_in_stewart_plotkin/main.pdf}
    \caption{\(\text{SSError}\) and state probabilities for the strategies
        of~\cite{Stewart2012}, ordered both by number of wins and overall score.
        Note that \(P(DC)\) is not shown as it corresponds to the transpose of
        \(P(CD)\). Cooperator and Defector are omitted as they do not visit all
        the states.}
    \label{fig:SSError_overall_in_stewart_plotkin}
\end{figure}

Here, the work of~\cite{Stewart2012} is extended by investigating a tournament
with \input{assets/tex/number_of_full_strategies/main.tex}
strategies.

The results of this analysis are shown in
Figure~\ref{fig:SSError_and_probabilities_in_full}. The top ranking strategies
by number of wins seem to be extortionate (but not against all strategies) and
it can be seen that a small sub group of strategies achieve mutual defection.
All the top ranking strategies according to score achieve mutual cooperation and
do not extort each other, however they
\textbf{do} exhibit extortionate behaviour towards a number of the lower ranking
strategies.

\begin{figure}[!htbp]
    \centering
    \includegraphics[width=.8\textwidth]{./assets/img/SSError_and_probabilities_in_full/main.pdf}
    \caption{\(\text{SSError}\) for the strategies for the full tournament. Only
    strategy interactions for which \(p_4=0\) and \(\chi>1\) are displayed.}
    \label{fig:SSError_and_probabilities_in_full}
\end{figure}

\section{Conclusion}\label{sec:conclusion}

This work defines an approach to measure whether or not a player is playing a
strategy that corresponds to an extortionate strategy as defined
in~\cite{Press2012}: a mathematical model for suspicion. Indeed, all
extortionate strategies have been
 classified as lying on a triangular plane.
This rigorous classification fails to be robust to small measurement error, thus
a statistical approach is proposed.
This is done through a linear algebraic approach for approximating the solution
of a linear system. Using this, a large number of pairwise interactions is
simulated and in fact very few strategies are found to act extortionately.

The work of~\cite{Press2012}, whilst showing that a clever approach to taking
advantage of another memory one strategy exists: this is incomplete. Whilst the
elegance of this result is very attractive, just as the simplicity of the
victory of Tit For Tat in Axelrod's original tournaments was, it is incomplete.
Extortionate strategies achieve a high number of wins but they do not
achieve a high score which corresponds to the fitness landscape in an
evolutionary sense. From the large number of interactions a payoff matrix \(S\)
can be measured where \(S_{ij}\) denotes the score (using standard values of
\((R, S, T, P) = (3, 0, 5, 1)\)) of the \(i\)th strategy
against the \(j\)th strategy. Using this, the replicator equation
describes the evolution of the system based on a population density fitness
function:

\begin{equation}\label{eqn:replicator_dynamics}
    \frac{dx}{dt} = x(S-x^TS x)
\end{equation}

Equation (\ref{eqn:replicator_dynamics}) is solved numerically through an
integration technique described in~\cite{Petzold1983} and
Figure~\ref{fig:replicator_dynamics} shows the evolution of the distribution of
the system: the various strategies are ranked by scores. It is clear to see that
only the high ranking strategies survive the evolutionary process (in fact,
only \input{./assets/img/replicator_dynamics/main.tex}
have a final distribution greater than \(10 ^ {-2}\)). This confirms the
findings of~\cite{Moran1707} in which sophisticated strategies resist
evolutionary invasion of shorter memory strategies. Recalling
Figure~\ref{fig:SSError_and_probabilities_in_full} this demonstrates that:

\begin{itemize}
    \item Cooperation emerges through the evolutionary process: the high scoring
        strategies do not exhibit extortionate behaviour towards each other.
    \item Extortionate strategies do not survive the evolutionary process.
\end{itemize}

\begin{figure}[!htbp]
    \centering
    \includegraphics[width=.8\textwidth]{./assets/img/replicator_dynamics/main.pdf}
    \caption{Numerical simulation of the replicator equation
    (\ref{eqn:replicator_dynamics}): strategies are ordered by score, only the strategies with a high score survive the evolutionary process.}
    \label{fig:replicator_dynamics}
\end{figure}

This work can be used to classify plays of the IPD\@: data can be collected from
actual interactions (in lab or in the field). Furthermore, this allows for a
classification method similar to the notion of fingerprinting presented
in~\cite{Ashlock2008}. Trained strategies can potentially be classified as
extortionate or not or it could be possible to even constrain the reinforcement
learning approaches that are becoming prevalent in the literature.
Alternatively, this mathematical approach for recognising extortion could be
used in sophisticated strategies to defend against invasion. Arguably, some of
the strategies considered here exhibit this behaviour, indeed as described
in~\cite{Harper2017}, the top ranking strategies in the full tournament are
obtained using evolutionary reinforcement learning techniques, thus, suspicion
of extortionate behaviour could in fact be an evolutionary trait.

\section*{Acknowledgements}

The following open source software libraries were used in this research:

\begin{itemize}
    \item The Axelrod ~\cite{Knight2016, Knight2018} library (IPD strategies and
        tournaments).
    \item The sympy library~\cite{Meurer2017} (verification of all symbolic
        calculations).
    \item The matplotlib~\cite{Droettboom2018} library (visualisation).
    \item The pandas~\cite{Structures2010}, dask~\cite{Dask2016} and
        NumPy~\cite{Oliphant2015} libraries (data manipulation).
    \item The SciPy~\cite{Jones2001} library (numerical integration of the
        replicator equation).
\end{itemize}

This work was performed using the computational facilities of the Advanced
Research Computing @ Cardiff (ARCCA) Division, Cardiff University.

\printbibliography

\newpage
\section*{Supplementary materials}

\includepdf{assets/pdf/proof_of_form_of_extortionate_strategies/main.pdf}

\newpage

Using the pair wise interactions the transition rates \(p,
q\) can be measured and the steady state probabilities inferred and compared to
the actual probabilities of each state.
This is done numerically by computing the singular eigenvector of the
matrix \(A\) \cite{Stewart2009}:

\[
    A =
    \begin{bmatrix}
        p_1 q_1 & p_1 (1 - q_1) & (1 - p_1) q_1 & (1 -p_1) (1 - q_1) \\
        p_2 q_2 & p_2 (1 - q_2) & (1 - p_2) q_2 & (1 -p_2) (1 - q_2) \\
        p_3 q_3 & p_3 (1 - q_3) & (1 - p_3) q_3 & (1 -p_3) (1 - q_3) \\
        p_4 q_4 & p_4 (1 - q_4) & (1 - p_4) q_4 & (1 -p_4) (1 - q_4) \\
    \end{bmatrix}
\]

Figure~\ref{fig:computed_probabilities_vs_theoretic_probabilities} shows a
regression line fitted to every pairwise interaction with a reported
\(\text{SSError}\) value (pairwise interactions with missing states were
omitted). This serves to validate the approach: a part from some edge cases the
relationship is consistent.

\begin{figure}[!htbp]
    \centering
    \includegraphics[width=.8\textwidth]{./assets/img/computed_probabilities_vs_theoretic_probabilities/main.pdf}
    \caption{The
        relationship between the steady state probabilities inferred from the
        measured transitions and the actual steady state probabilities. A linear
        regression line is included validating the approach.}
    \label{fig:computed_probabilities_vs_theoretic_probabilities}
\end{figure}


\end{document}
 turns and every match has been
repeated \documentclass[a4paper]{article}

\usepackage{amsmath}
\usepackage{amssymb}
\usepackage[margin=1.5cm,
            includefoot,
            footskip=30pt]{geometry}
\usepackage{layout}
\usepackage{graphicx}
\usepackage{subcaption}

\usepackage{biblatex}
\usepackage{pdfpages}

\bibliography{main.bib}

\title{Suspicion: Recognising and evaluating the effectiveness
       of extortion in the Iterated Prisoner's Dilemma}
\author{Vincent A. Knight \and Nikoleta E. Glynatsi}
\date{\today}



\begin{document}

\maketitle

\begin{abstract}
    The Iterated Prisoner's Dilemma is a model for rational and evolutionary
    interactive behaviour. It has applications both in the study of human social
    behaviour as well as in biology.
    It is used to understand when and how a rational individual might
    accept an immediate cost to their own utility for the direct benefit of
    another.

    Much attention has been given to a class of strategies called
    Zero Determinant strategies. It has been theoretically shown that these
    strategies can ``extort'' any player.

    In this work, an approach to identify if observed strategies are playing in
    an extortionate way is described. Furthermore, experimental analysis of
    a large tournament with \input{assets/tex/number_of_full_strategies/main.tex}
    strategies is considered. In this setting
    the most highly performing strategies do not play in an extortionate way
    against each other but do against lower performing strategies.
    This suggests that whilst the theory of Zero Determinant strategies
    indicates that memory is not of fundamental importance to the evolution of
    cooperative behaviour, this is incomplete.
\end{abstract}

\section{Introduction}\label{sec:introduction}

Agent based game theoretic models have become a stalwart of the underpinning
mathematics of interactive behaviours. One of the major pieces of work
in this area is the pair of original computer tournaments run by Robert
Axelrod~\cite{Axelrod1980, Axelrod1980a}. These tournaments pitted submitted
computer strategies against each other in plays of the Iterated Prisoner's
Dilemma. A common game where agents can choose to pay a slight cost to their
immediate utility in the hope of building a reputation. This has been used in
economic and evolutionary game theory to understand the evolution of cooperative
behaviour.

Recently, a class of strategies was described in~\cite{Press2012} that can
provably extort any given opponent. In~\cite{Hilbe2013, Moran1707} some
questions have already been asked about the true effectiveness of these
strategies in an evolutionary setting. Here another question is asked: is it
possible to recognise this extortionate behaviour? A mathematical procedure for
suspicion is presented: in the same way that the continued actions of an
extortionate individual might raise suspicion.

This work makes use of the Axelrod Python library~\cite{Knight2018, Knight2016}
with a large number of Prisoner Dilemma strategies available to give an
extensive numerical example of the ideas presented.  The approach is presented
in Section~\ref{sec:delta-zd-strategies}.  All of the code and data discussed
in Section~\ref{sec:numerical-experiments} is open sourced, archived and
written according to best scientific principles~\cite{Wilson2014}. The data
archive can be found at~\cite{vincent_knight_2018_1297075}.

\section{Recognising Extortion}\label{sec:delta-zd-strategies}

In~\cite{Press2012}, given a match between 2 memory-one strategies, the concept
of Zero Determinant (ZD) strategies is introduced. The main result of that paper
shows that given two memory one players \(p, q\in\mathbb{R}^4\) a linear
relationship between the players' scores could be forced by one of the players.

Using the notation of~\cite{Press2012}, assuming the utilities for player \(p\)
are given by \(S_x=(R, S, T, P)\) and for player \(q\) by \(S_y=(R, T, S, P)\)
and that the stationary scores of each player is given by \(S_X\) and \(S_Y\)
respectively. The main result of~\cite{Press2012} is that if

\begin{equation}\label{eqn:linear_relationship_for_p}
    \tilde p=\alpha S_x + \beta S_y + \gamma
\end{equation}

or

\begin{equation}\label{eqn:linear_relationship_for_q}
    \tilde q=\alpha S_x + \beta S_y + \gamma
\end{equation}

where \(\tilde p = (1 - p_1, 1 - p_2, p_3, p_4)\) and
\(\tilde q = (1 - q_1, 1 - q_2, q_3, q_4)\) then:

\begin{equation}
    \alpha S_X + \beta S_Y + \gamma = 0
\end{equation}

In~\cite{Press2012} a particular type of ZD strategy is defined: extortionate
strategies. If:

\begin{equation}\label{eqn:constraint_for_extortion}
    \gamma = - P(\alpha + \beta)
\end{equation}

then the player can ensure they get a score \(\chi\) times
larger than the opponent. This extortion coefficient is given by:

\begin{equation}\label{eqn:definition_of_chi}
    \chi=\frac{-\beta}{\alpha}
\end{equation}

Thus, if (\ref{eqn:constraint_for_extortion}) holds and \(\chi >1\) a player is
said to extort their opponent.
Here, the reverse problem is considered: given a
\(p\in\mathbb{R}^4\) how does one identify \(\alpha, \beta\) if they
exist and is the strategy in fact acting in an extortionate way?

These conditions correspond to:

\begin{align}
    \tilde p_1 & = \alpha R + \beta R - P (\alpha + \beta)
            \label{eqn:condition_for_tilde_p1}\\
    \tilde p_2 & = \alpha S + \beta T - P (\alpha + \beta)
            \label{eqn:condition_for_tilde_p2}\\
    \tilde p_3 & = \alpha T + \beta S - P (\alpha + \beta)
            \label{eqn:condition_for_tilde_p3}\\
    \tilde p_4 & = \alpha P + \beta P - P (\alpha + \beta)
            \label{eqn:condition_for_tilde_p4}
\end{align}

Equation (\ref{eqn:condition_for_tilde_p4}) ensures that \(p_4=\tilde p_4=0\).
Equations (\ref{eqn:condition_for_tilde_p1}-\ref{eqn:condition_for_tilde_p3})
can be used to eliminate \(\alpha, \beta\), giving:

\begin{equation}\label{eqn:planar_definition_of_extortion}
    \tilde p_1 = \frac{(R - P)(\tilde p_2 + \tilde p_3)}{S + T - 2P}
\end{equation}

with:

\begin{equation}\label{eqn:definition_of_chi}
    \chi = \frac{\tilde p_2 (P - T) + \tilde p_3 (S - P)}
                {\tilde p_2 (P - S) + \tilde p_3 (T - P)}
\end{equation}

Given a strategy \(p\in\mathbb{R}^{4\times 1}\) equations
(\ref{eqn:condition_for_tilde_p4}), (\ref{eqn:planar_definition_of_extortion}-\ref{eqn:definition_of_chi}) can be used to check if
a strategy is extortionate. The conditions correspond to:

\begin{align}
    p_1 & = \frac{(R-P)(p_2 + p_3) - R + T + S - P}{S + T - 2P}
     \label{eqn:condition_for_p1}\\
    p_4 & = 0 \label{eqn:condition_for_p4}\\
    1 & > p_2 + p_3\label{eqn:condition_for_chi}
\end{align}

The algebraic steps necessary to prove these results are available in the
supporting materials.

All extortionate strategies reside on a triangular (\ref{eqn:condition_for_chi})
plane (\ref{eqn:condition_for_p1}) in 3 dimensions (\ref{eqn:condition_for_p4}).
Using this formulation it can be seen that a necessary (but not sufficient)
condition for an extortionate strategy is that it cooperates on average less
than 50\% of the time when in a state of disagreement with the opponent.

As an example, consider the known extortionate strategy \(p=(8 / 9, 1 / 2, 1 /
3, 0)\) from~\cite{Stewart2012} which is referred to as \texttt{Extort-2}. In
this case, for the standard values of \((R, T, S, P)\) constraint
(\ref{eqn:condition_for_p1}) corresponds to:

\begin{equation}
    p_1 = \frac{2(p_2 + p_3) + 1}{3}
\end{equation}

It is clear that in this case all constraints hold.

This approach could in fact be used to confirm that a given strategy is acting
in an extortionate manner even if it is not a memory one strategy. However, in
practice, if a closed form for \(p\) is not known, then due to measurement
and/or numerical error this would not work.

This problem can be written in the following linear algebraic form where
\(x=(\alpha, \beta)\)
and \(p^*=(\tilde p_1 - 1, tilde_2 - 1, p_3)\):

\begin{equation}\label{eqn:linear_algebraic_equation_for_p}
    Cx= p^*
\end{equation}

\(C\) corresponds to equations
(\ref{eqn:condition_for_tilde_p1}-\ref{eqn:condition_for_tilde_p3}) and is
given by:

\begin{equation}\label{eqn:definition_of_C}
    C =
    \begin{bmatrix}
        R - P & R- P \\
        S - P & T- P \\
        T - P & S- P \\
    \end{bmatrix}
\end{equation}

Note that in general, equation (\ref{eqn:linear_algebraic_equation_for_p}) will
not necessarily have a solution. From the Rouch\'{e}-Capelli theorem if there is
a solution it is unique as \(\text{rank}(C)=2\) which is the dimension of the
variable \(x\). The best fitting \(x\) is found by minimizing:

\begin{equation}\label{eqn:r_squared}
    \text{SSError} = \|C x- p^*\|_2^2 = \sum_{i=1}^{3}\left((C\bar x)_i-p_i^*\right)^2
\end{equation}

Note that \(\text{SSError}\), which is the square of the Frobenius
norm~\cite{Golub2013}, becomes a measure of how close a strategy is to being an
extortionate strategy. Suspicion
of extortion then corresponds to a threshold on \(\text{SSError}\).

By observing interactions (human or otherwise), their memory one representation
can be inferred and this approach can be used to recognise extortionate
behaviour. The notion of comparing theoretic and actual plays of the IPD is not
novel, see for example~\cite{Rand2013}. Immediately it is noted that if the
environment is noisy~\cite{Wu1995} then no strategy can be considered to be
extortionate as \(p_4>0\).

In the next section, this idea will be illustrated by observing the interactions
that take place in a computer based tournament of the IPD\@.

\section{Numerical experiments}\label{sec:numerical-experiments}

In~\cite{Stewart2012} results from a tournament with
\input{./assets/tex/number_of_stewart_plotkin_strategies/main.tex} strategies,
was presented with specific consideration given to ZD strategies. This
tournament is reproduced here using the Axelrod-Python
project~\cite{Knight2016}. To obtain a good measure of the corresponding
transition rates for each strategy all matches have been run for
\input{assets/tex/number_of_turns/main.tex} turns and every match has been
repeated \input{assets/tex/number_of_repetitions/main.tex} times. All of this
interaction data is available at~\cite{vincent_knight_2018_1297075}. A good
match between the inferred Markov chain and the state distribution of the actual
interactions has been verified. Data for this is presented in the supplementary
materials.

Figure~\ref{fig:SSError_overall_in_stewart_plotkin} shows the \(\text{SSError}\)
values for all the strategies in the tournament, as reported
in~\cite{Stewart2012} the extortionate strategy (which has an expected
\(\text{SSError}\) approximately 0) gains a large number of wins.

\begin{figure}[!htbp]
    \centering
    \includegraphics[width=.8\textwidth]{./assets/img/SSError_overall_in_stewart_plotkin/main.pdf}
    \caption{\(\text{SSError}\) and state probabilities for the strategies
        of~\cite{Stewart2012}, ordered both by number of wins and overall score.
        Note that \(P(DC)\) is not shown as it corresponds to the transpose of
        \(P(CD)\). Cooperator and Defector are omitted as they do not visit all
        the states.}
    \label{fig:SSError_overall_in_stewart_plotkin}
\end{figure}

Here, the work of~\cite{Stewart2012} is extended by investigating a tournament
with \input{assets/tex/number_of_full_strategies/main.tex}
strategies.

The results of this analysis are shown in
Figure~\ref{fig:SSError_and_probabilities_in_full}. The top ranking strategies
by number of wins seem to be extortionate (but not against all strategies) and
it can be seen that a small sub group of strategies achieve mutual defection.
All the top ranking strategies according to score achieve mutual cooperation and
do not extort each other, however they
\textbf{do} exhibit extortionate behaviour towards a number of the lower ranking
strategies.

\begin{figure}[!htbp]
    \centering
    \includegraphics[width=.8\textwidth]{./assets/img/SSError_and_probabilities_in_full/main.pdf}
    \caption{\(\text{SSError}\) for the strategies for the full tournament. Only
    strategy interactions for which \(p_4=0\) and \(\chi>1\) are displayed.}
    \label{fig:SSError_and_probabilities_in_full}
\end{figure}

\section{Conclusion}\label{sec:conclusion}

This work defines an approach to measure whether or not a player is playing a
strategy that corresponds to an extortionate strategy as defined
in~\cite{Press2012}: a mathematical model for suspicion. Indeed, all
extortionate strategies have been
 classified as lying on a triangular plane.
This rigorous classification fails to be robust to small measurement error, thus
a statistical approach is proposed.
This is done through a linear algebraic approach for approximating the solution
of a linear system. Using this, a large number of pairwise interactions is
simulated and in fact very few strategies are found to act extortionately.

The work of~\cite{Press2012}, whilst showing that a clever approach to taking
advantage of another memory one strategy exists: this is incomplete. Whilst the
elegance of this result is very attractive, just as the simplicity of the
victory of Tit For Tat in Axelrod's original tournaments was, it is incomplete.
Extortionate strategies achieve a high number of wins but they do not
achieve a high score which corresponds to the fitness landscape in an
evolutionary sense. From the large number of interactions a payoff matrix \(S\)
can be measured where \(S_{ij}\) denotes the score (using standard values of
\((R, S, T, P) = (3, 0, 5, 1)\)) of the \(i\)th strategy
against the \(j\)th strategy. Using this, the replicator equation
describes the evolution of the system based on a population density fitness
function:

\begin{equation}\label{eqn:replicator_dynamics}
    \frac{dx}{dt} = x(S-x^TS x)
\end{equation}

Equation (\ref{eqn:replicator_dynamics}) is solved numerically through an
integration technique described in~\cite{Petzold1983} and
Figure~\ref{fig:replicator_dynamics} shows the evolution of the distribution of
the system: the various strategies are ranked by scores. It is clear to see that
only the high ranking strategies survive the evolutionary process (in fact,
only \input{./assets/img/replicator_dynamics/main.tex}
have a final distribution greater than \(10 ^ {-2}\)). This confirms the
findings of~\cite{Moran1707} in which sophisticated strategies resist
evolutionary invasion of shorter memory strategies. Recalling
Figure~\ref{fig:SSError_and_probabilities_in_full} this demonstrates that:

\begin{itemize}
    \item Cooperation emerges through the evolutionary process: the high scoring
        strategies do not exhibit extortionate behaviour towards each other.
    \item Extortionate strategies do not survive the evolutionary process.
\end{itemize}

\begin{figure}[!htbp]
    \centering
    \includegraphics[width=.8\textwidth]{./assets/img/replicator_dynamics/main.pdf}
    \caption{Numerical simulation of the replicator equation
    (\ref{eqn:replicator_dynamics}): strategies are ordered by score, only the strategies with a high score survive the evolutionary process.}
    \label{fig:replicator_dynamics}
\end{figure}

This work can be used to classify plays of the IPD\@: data can be collected from
actual interactions (in lab or in the field). Furthermore, this allows for a
classification method similar to the notion of fingerprinting presented
in~\cite{Ashlock2008}. Trained strategies can potentially be classified as
extortionate or not or it could be possible to even constrain the reinforcement
learning approaches that are becoming prevalent in the literature.
Alternatively, this mathematical approach for recognising extortion could be
used in sophisticated strategies to defend against invasion. Arguably, some of
the strategies considered here exhibit this behaviour, indeed as described
in~\cite{Harper2017}, the top ranking strategies in the full tournament are
obtained using evolutionary reinforcement learning techniques, thus, suspicion
of extortionate behaviour could in fact be an evolutionary trait.

\section*{Acknowledgements}

The following open source software libraries were used in this research:

\begin{itemize}
    \item The Axelrod ~\cite{Knight2016, Knight2018} library (IPD strategies and
        tournaments).
    \item The sympy library~\cite{Meurer2017} (verification of all symbolic
        calculations).
    \item The matplotlib~\cite{Droettboom2018} library (visualisation).
    \item The pandas~\cite{Structures2010}, dask~\cite{Dask2016} and
        NumPy~\cite{Oliphant2015} libraries (data manipulation).
    \item The SciPy~\cite{Jones2001} library (numerical integration of the
        replicator equation).
\end{itemize}

This work was performed using the computational facilities of the Advanced
Research Computing @ Cardiff (ARCCA) Division, Cardiff University.

\printbibliography

\newpage
\section*{Supplementary materials}

\includepdf{assets/pdf/proof_of_form_of_extortionate_strategies/main.pdf}

\newpage

Using the pair wise interactions the transition rates \(p,
q\) can be measured and the steady state probabilities inferred and compared to
the actual probabilities of each state.
This is done numerically by computing the singular eigenvector of the
matrix \(A\) \cite{Stewart2009}:

\[
    A =
    \begin{bmatrix}
        p_1 q_1 & p_1 (1 - q_1) & (1 - p_1) q_1 & (1 -p_1) (1 - q_1) \\
        p_2 q_2 & p_2 (1 - q_2) & (1 - p_2) q_2 & (1 -p_2) (1 - q_2) \\
        p_3 q_3 & p_3 (1 - q_3) & (1 - p_3) q_3 & (1 -p_3) (1 - q_3) \\
        p_4 q_4 & p_4 (1 - q_4) & (1 - p_4) q_4 & (1 -p_4) (1 - q_4) \\
    \end{bmatrix}
\]

Figure~\ref{fig:computed_probabilities_vs_theoretic_probabilities} shows a
regression line fitted to every pairwise interaction with a reported
\(\text{SSError}\) value (pairwise interactions with missing states were
omitted). This serves to validate the approach: a part from some edge cases the
relationship is consistent.

\begin{figure}[!htbp]
    \centering
    \includegraphics[width=.8\textwidth]{./assets/img/computed_probabilities_vs_theoretic_probabilities/main.pdf}
    \caption{The
        relationship between the steady state probabilities inferred from the
        measured transitions and the actual steady state probabilities. A linear
        regression line is included validating the approach.}
    \label{fig:computed_probabilities_vs_theoretic_probabilities}
\end{figure}


\end{document}
 times. All of this
interaction data is available at~\cite{vincent_knight_2018_1297075}. A good
match between the inferred Markov chain and the state distribution of the actual
interactions has been verified. Data for this is presented in the supplementary
materials.

Figure~\ref{fig:SSError_overall_in_stewart_plotkin} shows the \(\text{SSError}\)
values for all the strategies in the tournament, as reported
in~\cite{Stewart2012} the extortionate strategy (which has an expected
\(\text{SSError}\) approximately 0) gains a large number of wins.

\begin{figure}[!htbp]
    \centering
    \includegraphics[width=.8\textwidth]{./assets/img/SSError_overall_in_stewart_plotkin/main.pdf}
    \caption{\(\text{SSError}\) and state probabilities for the strategies
        of~\cite{Stewart2012}, ordered both by number of wins and overall score.
        Note that \(P(DC)\) is not shown as it corresponds to the transpose of
        \(P(CD)\). Cooperator and Defector are omitted as they do not visit all
        the states.}
    \label{fig:SSError_overall_in_stewart_plotkin}
\end{figure}

Here, the work of~\cite{Stewart2012} is extended by investigating a tournament
with \documentclass[a4paper]{article}

\usepackage{amsmath}
\usepackage{amssymb}
\usepackage[margin=1.5cm,
            includefoot,
            footskip=30pt]{geometry}
\usepackage{layout}
\usepackage{graphicx}
\usepackage{subcaption}

\usepackage{biblatex}
\usepackage{pdfpages}

\bibliography{main.bib}

\title{Suspicion: Recognising and evaluating the effectiveness
       of extortion in the Iterated Prisoner's Dilemma}
\author{Vincent A. Knight \and Nikoleta E. Glynatsi}
\date{\today}



\begin{document}

\maketitle

\begin{abstract}
    The Iterated Prisoner's Dilemma is a model for rational and evolutionary
    interactive behaviour. It has applications both in the study of human social
    behaviour as well as in biology.
    It is used to understand when and how a rational individual might
    accept an immediate cost to their own utility for the direct benefit of
    another.

    Much attention has been given to a class of strategies called
    Zero Determinant strategies. It has been theoretically shown that these
    strategies can ``extort'' any player.

    In this work, an approach to identify if observed strategies are playing in
    an extortionate way is described. Furthermore, experimental analysis of
    a large tournament with \input{assets/tex/number_of_full_strategies/main.tex}
    strategies is considered. In this setting
    the most highly performing strategies do not play in an extortionate way
    against each other but do against lower performing strategies.
    This suggests that whilst the theory of Zero Determinant strategies
    indicates that memory is not of fundamental importance to the evolution of
    cooperative behaviour, this is incomplete.
\end{abstract}

\section{Introduction}\label{sec:introduction}

Agent based game theoretic models have become a stalwart of the underpinning
mathematics of interactive behaviours. One of the major pieces of work
in this area is the pair of original computer tournaments run by Robert
Axelrod~\cite{Axelrod1980, Axelrod1980a}. These tournaments pitted submitted
computer strategies against each other in plays of the Iterated Prisoner's
Dilemma. A common game where agents can choose to pay a slight cost to their
immediate utility in the hope of building a reputation. This has been used in
economic and evolutionary game theory to understand the evolution of cooperative
behaviour.

Recently, a class of strategies was described in~\cite{Press2012} that can
provably extort any given opponent. In~\cite{Hilbe2013, Moran1707} some
questions have already been asked about the true effectiveness of these
strategies in an evolutionary setting. Here another question is asked: is it
possible to recognise this extortionate behaviour? A mathematical procedure for
suspicion is presented: in the same way that the continued actions of an
extortionate individual might raise suspicion.

This work makes use of the Axelrod Python library~\cite{Knight2018, Knight2016}
with a large number of Prisoner Dilemma strategies available to give an
extensive numerical example of the ideas presented.  The approach is presented
in Section~\ref{sec:delta-zd-strategies}.  All of the code and data discussed
in Section~\ref{sec:numerical-experiments} is open sourced, archived and
written according to best scientific principles~\cite{Wilson2014}. The data
archive can be found at~\cite{vincent_knight_2018_1297075}.

\section{Recognising Extortion}\label{sec:delta-zd-strategies}

In~\cite{Press2012}, given a match between 2 memory-one strategies, the concept
of Zero Determinant (ZD) strategies is introduced. The main result of that paper
shows that given two memory one players \(p, q\in\mathbb{R}^4\) a linear
relationship between the players' scores could be forced by one of the players.

Using the notation of~\cite{Press2012}, assuming the utilities for player \(p\)
are given by \(S_x=(R, S, T, P)\) and for player \(q\) by \(S_y=(R, T, S, P)\)
and that the stationary scores of each player is given by \(S_X\) and \(S_Y\)
respectively. The main result of~\cite{Press2012} is that if

\begin{equation}\label{eqn:linear_relationship_for_p}
    \tilde p=\alpha S_x + \beta S_y + \gamma
\end{equation}

or

\begin{equation}\label{eqn:linear_relationship_for_q}
    \tilde q=\alpha S_x + \beta S_y + \gamma
\end{equation}

where \(\tilde p = (1 - p_1, 1 - p_2, p_3, p_4)\) and
\(\tilde q = (1 - q_1, 1 - q_2, q_3, q_4)\) then:

\begin{equation}
    \alpha S_X + \beta S_Y + \gamma = 0
\end{equation}

In~\cite{Press2012} a particular type of ZD strategy is defined: extortionate
strategies. If:

\begin{equation}\label{eqn:constraint_for_extortion}
    \gamma = - P(\alpha + \beta)
\end{equation}

then the player can ensure they get a score \(\chi\) times
larger than the opponent. This extortion coefficient is given by:

\begin{equation}\label{eqn:definition_of_chi}
    \chi=\frac{-\beta}{\alpha}
\end{equation}

Thus, if (\ref{eqn:constraint_for_extortion}) holds and \(\chi >1\) a player is
said to extort their opponent.
Here, the reverse problem is considered: given a
\(p\in\mathbb{R}^4\) how does one identify \(\alpha, \beta\) if they
exist and is the strategy in fact acting in an extortionate way?

These conditions correspond to:

\begin{align}
    \tilde p_1 & = \alpha R + \beta R - P (\alpha + \beta)
            \label{eqn:condition_for_tilde_p1}\\
    \tilde p_2 & = \alpha S + \beta T - P (\alpha + \beta)
            \label{eqn:condition_for_tilde_p2}\\
    \tilde p_3 & = \alpha T + \beta S - P (\alpha + \beta)
            \label{eqn:condition_for_tilde_p3}\\
    \tilde p_4 & = \alpha P + \beta P - P (\alpha + \beta)
            \label{eqn:condition_for_tilde_p4}
\end{align}

Equation (\ref{eqn:condition_for_tilde_p4}) ensures that \(p_4=\tilde p_4=0\).
Equations (\ref{eqn:condition_for_tilde_p1}-\ref{eqn:condition_for_tilde_p3})
can be used to eliminate \(\alpha, \beta\), giving:

\begin{equation}\label{eqn:planar_definition_of_extortion}
    \tilde p_1 = \frac{(R - P)(\tilde p_2 + \tilde p_3)}{S + T - 2P}
\end{equation}

with:

\begin{equation}\label{eqn:definition_of_chi}
    \chi = \frac{\tilde p_2 (P - T) + \tilde p_3 (S - P)}
                {\tilde p_2 (P - S) + \tilde p_3 (T - P)}
\end{equation}

Given a strategy \(p\in\mathbb{R}^{4\times 1}\) equations
(\ref{eqn:condition_for_tilde_p4}), (\ref{eqn:planar_definition_of_extortion}-\ref{eqn:definition_of_chi}) can be used to check if
a strategy is extortionate. The conditions correspond to:

\begin{align}
    p_1 & = \frac{(R-P)(p_2 + p_3) - R + T + S - P}{S + T - 2P}
     \label{eqn:condition_for_p1}\\
    p_4 & = 0 \label{eqn:condition_for_p4}\\
    1 & > p_2 + p_3\label{eqn:condition_for_chi}
\end{align}

The algebraic steps necessary to prove these results are available in the
supporting materials.

All extortionate strategies reside on a triangular (\ref{eqn:condition_for_chi})
plane (\ref{eqn:condition_for_p1}) in 3 dimensions (\ref{eqn:condition_for_p4}).
Using this formulation it can be seen that a necessary (but not sufficient)
condition for an extortionate strategy is that it cooperates on average less
than 50\% of the time when in a state of disagreement with the opponent.

As an example, consider the known extortionate strategy \(p=(8 / 9, 1 / 2, 1 /
3, 0)\) from~\cite{Stewart2012} which is referred to as \texttt{Extort-2}. In
this case, for the standard values of \((R, T, S, P)\) constraint
(\ref{eqn:condition_for_p1}) corresponds to:

\begin{equation}
    p_1 = \frac{2(p_2 + p_3) + 1}{3}
\end{equation}

It is clear that in this case all constraints hold.

This approach could in fact be used to confirm that a given strategy is acting
in an extortionate manner even if it is not a memory one strategy. However, in
practice, if a closed form for \(p\) is not known, then due to measurement
and/or numerical error this would not work.

This problem can be written in the following linear algebraic form where
\(x=(\alpha, \beta)\)
and \(p^*=(\tilde p_1 - 1, tilde_2 - 1, p_3)\):

\begin{equation}\label{eqn:linear_algebraic_equation_for_p}
    Cx= p^*
\end{equation}

\(C\) corresponds to equations
(\ref{eqn:condition_for_tilde_p1}-\ref{eqn:condition_for_tilde_p3}) and is
given by:

\begin{equation}\label{eqn:definition_of_C}
    C =
    \begin{bmatrix}
        R - P & R- P \\
        S - P & T- P \\
        T - P & S- P \\
    \end{bmatrix}
\end{equation}

Note that in general, equation (\ref{eqn:linear_algebraic_equation_for_p}) will
not necessarily have a solution. From the Rouch\'{e}-Capelli theorem if there is
a solution it is unique as \(\text{rank}(C)=2\) which is the dimension of the
variable \(x\). The best fitting \(x\) is found by minimizing:

\begin{equation}\label{eqn:r_squared}
    \text{SSError} = \|C x- p^*\|_2^2 = \sum_{i=1}^{3}\left((C\bar x)_i-p_i^*\right)^2
\end{equation}

Note that \(\text{SSError}\), which is the square of the Frobenius
norm~\cite{Golub2013}, becomes a measure of how close a strategy is to being an
extortionate strategy. Suspicion
of extortion then corresponds to a threshold on \(\text{SSError}\).

By observing interactions (human or otherwise), their memory one representation
can be inferred and this approach can be used to recognise extortionate
behaviour. The notion of comparing theoretic and actual plays of the IPD is not
novel, see for example~\cite{Rand2013}. Immediately it is noted that if the
environment is noisy~\cite{Wu1995} then no strategy can be considered to be
extortionate as \(p_4>0\).

In the next section, this idea will be illustrated by observing the interactions
that take place in a computer based tournament of the IPD\@.

\section{Numerical experiments}\label{sec:numerical-experiments}

In~\cite{Stewart2012} results from a tournament with
\input{./assets/tex/number_of_stewart_plotkin_strategies/main.tex} strategies,
was presented with specific consideration given to ZD strategies. This
tournament is reproduced here using the Axelrod-Python
project~\cite{Knight2016}. To obtain a good measure of the corresponding
transition rates for each strategy all matches have been run for
\input{assets/tex/number_of_turns/main.tex} turns and every match has been
repeated \input{assets/tex/number_of_repetitions/main.tex} times. All of this
interaction data is available at~\cite{vincent_knight_2018_1297075}. A good
match between the inferred Markov chain and the state distribution of the actual
interactions has been verified. Data for this is presented in the supplementary
materials.

Figure~\ref{fig:SSError_overall_in_stewart_plotkin} shows the \(\text{SSError}\)
values for all the strategies in the tournament, as reported
in~\cite{Stewart2012} the extortionate strategy (which has an expected
\(\text{SSError}\) approximately 0) gains a large number of wins.

\begin{figure}[!htbp]
    \centering
    \includegraphics[width=.8\textwidth]{./assets/img/SSError_overall_in_stewart_plotkin/main.pdf}
    \caption{\(\text{SSError}\) and state probabilities for the strategies
        of~\cite{Stewart2012}, ordered both by number of wins and overall score.
        Note that \(P(DC)\) is not shown as it corresponds to the transpose of
        \(P(CD)\). Cooperator and Defector are omitted as they do not visit all
        the states.}
    \label{fig:SSError_overall_in_stewart_plotkin}
\end{figure}

Here, the work of~\cite{Stewart2012} is extended by investigating a tournament
with \input{assets/tex/number_of_full_strategies/main.tex}
strategies.

The results of this analysis are shown in
Figure~\ref{fig:SSError_and_probabilities_in_full}. The top ranking strategies
by number of wins seem to be extortionate (but not against all strategies) and
it can be seen that a small sub group of strategies achieve mutual defection.
All the top ranking strategies according to score achieve mutual cooperation and
do not extort each other, however they
\textbf{do} exhibit extortionate behaviour towards a number of the lower ranking
strategies.

\begin{figure}[!htbp]
    \centering
    \includegraphics[width=.8\textwidth]{./assets/img/SSError_and_probabilities_in_full/main.pdf}
    \caption{\(\text{SSError}\) for the strategies for the full tournament. Only
    strategy interactions for which \(p_4=0\) and \(\chi>1\) are displayed.}
    \label{fig:SSError_and_probabilities_in_full}
\end{figure}

\section{Conclusion}\label{sec:conclusion}

This work defines an approach to measure whether or not a player is playing a
strategy that corresponds to an extortionate strategy as defined
in~\cite{Press2012}: a mathematical model for suspicion. Indeed, all
extortionate strategies have been
 classified as lying on a triangular plane.
This rigorous classification fails to be robust to small measurement error, thus
a statistical approach is proposed.
This is done through a linear algebraic approach for approximating the solution
of a linear system. Using this, a large number of pairwise interactions is
simulated and in fact very few strategies are found to act extortionately.

The work of~\cite{Press2012}, whilst showing that a clever approach to taking
advantage of another memory one strategy exists: this is incomplete. Whilst the
elegance of this result is very attractive, just as the simplicity of the
victory of Tit For Tat in Axelrod's original tournaments was, it is incomplete.
Extortionate strategies achieve a high number of wins but they do not
achieve a high score which corresponds to the fitness landscape in an
evolutionary sense. From the large number of interactions a payoff matrix \(S\)
can be measured where \(S_{ij}\) denotes the score (using standard values of
\((R, S, T, P) = (3, 0, 5, 1)\)) of the \(i\)th strategy
against the \(j\)th strategy. Using this, the replicator equation
describes the evolution of the system based on a population density fitness
function:

\begin{equation}\label{eqn:replicator_dynamics}
    \frac{dx}{dt} = x(S-x^TS x)
\end{equation}

Equation (\ref{eqn:replicator_dynamics}) is solved numerically through an
integration technique described in~\cite{Petzold1983} and
Figure~\ref{fig:replicator_dynamics} shows the evolution of the distribution of
the system: the various strategies are ranked by scores. It is clear to see that
only the high ranking strategies survive the evolutionary process (in fact,
only \input{./assets/img/replicator_dynamics/main.tex}
have a final distribution greater than \(10 ^ {-2}\)). This confirms the
findings of~\cite{Moran1707} in which sophisticated strategies resist
evolutionary invasion of shorter memory strategies. Recalling
Figure~\ref{fig:SSError_and_probabilities_in_full} this demonstrates that:

\begin{itemize}
    \item Cooperation emerges through the evolutionary process: the high scoring
        strategies do not exhibit extortionate behaviour towards each other.
    \item Extortionate strategies do not survive the evolutionary process.
\end{itemize}

\begin{figure}[!htbp]
    \centering
    \includegraphics[width=.8\textwidth]{./assets/img/replicator_dynamics/main.pdf}
    \caption{Numerical simulation of the replicator equation
    (\ref{eqn:replicator_dynamics}): strategies are ordered by score, only the strategies with a high score survive the evolutionary process.}
    \label{fig:replicator_dynamics}
\end{figure}

This work can be used to classify plays of the IPD\@: data can be collected from
actual interactions (in lab or in the field). Furthermore, this allows for a
classification method similar to the notion of fingerprinting presented
in~\cite{Ashlock2008}. Trained strategies can potentially be classified as
extortionate or not or it could be possible to even constrain the reinforcement
learning approaches that are becoming prevalent in the literature.
Alternatively, this mathematical approach for recognising extortion could be
used in sophisticated strategies to defend against invasion. Arguably, some of
the strategies considered here exhibit this behaviour, indeed as described
in~\cite{Harper2017}, the top ranking strategies in the full tournament are
obtained using evolutionary reinforcement learning techniques, thus, suspicion
of extortionate behaviour could in fact be an evolutionary trait.

\section*{Acknowledgements}

The following open source software libraries were used in this research:

\begin{itemize}
    \item The Axelrod ~\cite{Knight2016, Knight2018} library (IPD strategies and
        tournaments).
    \item The sympy library~\cite{Meurer2017} (verification of all symbolic
        calculations).
    \item The matplotlib~\cite{Droettboom2018} library (visualisation).
    \item The pandas~\cite{Structures2010}, dask~\cite{Dask2016} and
        NumPy~\cite{Oliphant2015} libraries (data manipulation).
    \item The SciPy~\cite{Jones2001} library (numerical integration of the
        replicator equation).
\end{itemize}

This work was performed using the computational facilities of the Advanced
Research Computing @ Cardiff (ARCCA) Division, Cardiff University.

\printbibliography

\newpage
\section*{Supplementary materials}

\includepdf{assets/pdf/proof_of_form_of_extortionate_strategies/main.pdf}

\newpage

Using the pair wise interactions the transition rates \(p,
q\) can be measured and the steady state probabilities inferred and compared to
the actual probabilities of each state.
This is done numerically by computing the singular eigenvector of the
matrix \(A\) \cite{Stewart2009}:

\[
    A =
    \begin{bmatrix}
        p_1 q_1 & p_1 (1 - q_1) & (1 - p_1) q_1 & (1 -p_1) (1 - q_1) \\
        p_2 q_2 & p_2 (1 - q_2) & (1 - p_2) q_2 & (1 -p_2) (1 - q_2) \\
        p_3 q_3 & p_3 (1 - q_3) & (1 - p_3) q_3 & (1 -p_3) (1 - q_3) \\
        p_4 q_4 & p_4 (1 - q_4) & (1 - p_4) q_4 & (1 -p_4) (1 - q_4) \\
    \end{bmatrix}
\]

Figure~\ref{fig:computed_probabilities_vs_theoretic_probabilities} shows a
regression line fitted to every pairwise interaction with a reported
\(\text{SSError}\) value (pairwise interactions with missing states were
omitted). This serves to validate the approach: a part from some edge cases the
relationship is consistent.

\begin{figure}[!htbp]
    \centering
    \includegraphics[width=.8\textwidth]{./assets/img/computed_probabilities_vs_theoretic_probabilities/main.pdf}
    \caption{The
        relationship between the steady state probabilities inferred from the
        measured transitions and the actual steady state probabilities. A linear
        regression line is included validating the approach.}
    \label{fig:computed_probabilities_vs_theoretic_probabilities}
\end{figure}


\end{document}

strategies.

The results of this analysis are shown in
Figure~\ref{fig:SSError_and_probabilities_in_full}. The top ranking strategies
by number of wins seem to be extortionate (but not against all strategies) and
it can be seen that a small sub group of strategies achieve mutual defection.
All the top ranking strategies according to score achieve mutual cooperation and
do not extort each other, however they
\textbf{do} exhibit extortionate behaviour towards a number of the lower ranking
strategies.

\begin{figure}[!htbp]
    \centering
    \includegraphics[width=.8\textwidth]{./assets/img/SSError_and_probabilities_in_full/main.pdf}
    \caption{\(\text{SSError}\) for the strategies for the full tournament. Only
    strategy interactions for which \(p_4=0\) and \(\chi>1\) are displayed.}
    \label{fig:SSError_and_probabilities_in_full}
\end{figure}

\section{Conclusion}\label{sec:conclusion}

This work defines an approach to measure whether or not a player is playing a
strategy that corresponds to an extortionate strategy as defined
in~\cite{Press2012}: a mathematical model for suspicion. Indeed, all
extortionate strategies have been
 classified as lying on a triangular plane.
This rigorous classification fails to be robust to small measurement error, thus
a statistical approach is proposed.
This is done through a linear algebraic approach for approximating the solution
of a linear system. Using this, a large number of pairwise interactions is
simulated and in fact very few strategies are found to act extortionately.

The work of~\cite{Press2012}, whilst showing that a clever approach to taking
advantage of another memory one strategy exists: this is incomplete. Whilst the
elegance of this result is very attractive, just as the simplicity of the
victory of Tit For Tat in Axelrod's original tournaments was, it is incomplete.
Extortionate strategies achieve a high number of wins but they do not
achieve a high score which corresponds to the fitness landscape in an
evolutionary sense. From the large number of interactions a payoff matrix \(S\)
can be measured where \(S_{ij}\) denotes the score (using standard values of
\((R, S, T, P) = (3, 0, 5, 1)\)) of the \(i\)th strategy
against the \(j\)th strategy. Using this, the replicator equation
describes the evolution of the system based on a population density fitness
function:

\begin{equation}\label{eqn:replicator_dynamics}
    \frac{dx}{dt} = x(S-x^TS x)
\end{equation}

Equation (\ref{eqn:replicator_dynamics}) is solved numerically through an
integration technique described in~\cite{Petzold1983} and
Figure~\ref{fig:replicator_dynamics} shows the evolution of the distribution of
the system: the various strategies are ranked by scores. It is clear to see that
only the high ranking strategies survive the evolutionary process (in fact,
only \documentclass[a4paper]{article}

\usepackage{amsmath}
\usepackage{amssymb}
\usepackage[margin=1.5cm,
            includefoot,
            footskip=30pt]{geometry}
\usepackage{layout}
\usepackage{graphicx}
\usepackage{subcaption}

\usepackage{biblatex}
\usepackage{pdfpages}

\bibliography{main.bib}

\title{Suspicion: Recognising and evaluating the effectiveness
       of extortion in the Iterated Prisoner's Dilemma}
\author{Vincent A. Knight \and Nikoleta E. Glynatsi}
\date{\today}



\begin{document}

\maketitle

\begin{abstract}
    The Iterated Prisoner's Dilemma is a model for rational and evolutionary
    interactive behaviour. It has applications both in the study of human social
    behaviour as well as in biology.
    It is used to understand when and how a rational individual might
    accept an immediate cost to their own utility for the direct benefit of
    another.

    Much attention has been given to a class of strategies called
    Zero Determinant strategies. It has been theoretically shown that these
    strategies can ``extort'' any player.

    In this work, an approach to identify if observed strategies are playing in
    an extortionate way is described. Furthermore, experimental analysis of
    a large tournament with \input{assets/tex/number_of_full_strategies/main.tex}
    strategies is considered. In this setting
    the most highly performing strategies do not play in an extortionate way
    against each other but do against lower performing strategies.
    This suggests that whilst the theory of Zero Determinant strategies
    indicates that memory is not of fundamental importance to the evolution of
    cooperative behaviour, this is incomplete.
\end{abstract}

\section{Introduction}\label{sec:introduction}

Agent based game theoretic models have become a stalwart of the underpinning
mathematics of interactive behaviours. One of the major pieces of work
in this area is the pair of original computer tournaments run by Robert
Axelrod~\cite{Axelrod1980, Axelrod1980a}. These tournaments pitted submitted
computer strategies against each other in plays of the Iterated Prisoner's
Dilemma. A common game where agents can choose to pay a slight cost to their
immediate utility in the hope of building a reputation. This has been used in
economic and evolutionary game theory to understand the evolution of cooperative
behaviour.

Recently, a class of strategies was described in~\cite{Press2012} that can
provably extort any given opponent. In~\cite{Hilbe2013, Moran1707} some
questions have already been asked about the true effectiveness of these
strategies in an evolutionary setting. Here another question is asked: is it
possible to recognise this extortionate behaviour? A mathematical procedure for
suspicion is presented: in the same way that the continued actions of an
extortionate individual might raise suspicion.

This work makes use of the Axelrod Python library~\cite{Knight2018, Knight2016}
with a large number of Prisoner Dilemma strategies available to give an
extensive numerical example of the ideas presented.  The approach is presented
in Section~\ref{sec:delta-zd-strategies}.  All of the code and data discussed
in Section~\ref{sec:numerical-experiments} is open sourced, archived and
written according to best scientific principles~\cite{Wilson2014}. The data
archive can be found at~\cite{vincent_knight_2018_1297075}.

\section{Recognising Extortion}\label{sec:delta-zd-strategies}

In~\cite{Press2012}, given a match between 2 memory-one strategies, the concept
of Zero Determinant (ZD) strategies is introduced. The main result of that paper
shows that given two memory one players \(p, q\in\mathbb{R}^4\) a linear
relationship between the players' scores could be forced by one of the players.

Using the notation of~\cite{Press2012}, assuming the utilities for player \(p\)
are given by \(S_x=(R, S, T, P)\) and for player \(q\) by \(S_y=(R, T, S, P)\)
and that the stationary scores of each player is given by \(S_X\) and \(S_Y\)
respectively. The main result of~\cite{Press2012} is that if

\begin{equation}\label{eqn:linear_relationship_for_p}
    \tilde p=\alpha S_x + \beta S_y + \gamma
\end{equation}

or

\begin{equation}\label{eqn:linear_relationship_for_q}
    \tilde q=\alpha S_x + \beta S_y + \gamma
\end{equation}

where \(\tilde p = (1 - p_1, 1 - p_2, p_3, p_4)\) and
\(\tilde q = (1 - q_1, 1 - q_2, q_3, q_4)\) then:

\begin{equation}
    \alpha S_X + \beta S_Y + \gamma = 0
\end{equation}

In~\cite{Press2012} a particular type of ZD strategy is defined: extortionate
strategies. If:

\begin{equation}\label{eqn:constraint_for_extortion}
    \gamma = - P(\alpha + \beta)
\end{equation}

then the player can ensure they get a score \(\chi\) times
larger than the opponent. This extortion coefficient is given by:

\begin{equation}\label{eqn:definition_of_chi}
    \chi=\frac{-\beta}{\alpha}
\end{equation}

Thus, if (\ref{eqn:constraint_for_extortion}) holds and \(\chi >1\) a player is
said to extort their opponent.
Here, the reverse problem is considered: given a
\(p\in\mathbb{R}^4\) how does one identify \(\alpha, \beta\) if they
exist and is the strategy in fact acting in an extortionate way?

These conditions correspond to:

\begin{align}
    \tilde p_1 & = \alpha R + \beta R - P (\alpha + \beta)
            \label{eqn:condition_for_tilde_p1}\\
    \tilde p_2 & = \alpha S + \beta T - P (\alpha + \beta)
            \label{eqn:condition_for_tilde_p2}\\
    \tilde p_3 & = \alpha T + \beta S - P (\alpha + \beta)
            \label{eqn:condition_for_tilde_p3}\\
    \tilde p_4 & = \alpha P + \beta P - P (\alpha + \beta)
            \label{eqn:condition_for_tilde_p4}
\end{align}

Equation (\ref{eqn:condition_for_tilde_p4}) ensures that \(p_4=\tilde p_4=0\).
Equations (\ref{eqn:condition_for_tilde_p1}-\ref{eqn:condition_for_tilde_p3})
can be used to eliminate \(\alpha, \beta\), giving:

\begin{equation}\label{eqn:planar_definition_of_extortion}
    \tilde p_1 = \frac{(R - P)(\tilde p_2 + \tilde p_3)}{S + T - 2P}
\end{equation}

with:

\begin{equation}\label{eqn:definition_of_chi}
    \chi = \frac{\tilde p_2 (P - T) + \tilde p_3 (S - P)}
                {\tilde p_2 (P - S) + \tilde p_3 (T - P)}
\end{equation}

Given a strategy \(p\in\mathbb{R}^{4\times 1}\) equations
(\ref{eqn:condition_for_tilde_p4}), (\ref{eqn:planar_definition_of_extortion}-\ref{eqn:definition_of_chi}) can be used to check if
a strategy is extortionate. The conditions correspond to:

\begin{align}
    p_1 & = \frac{(R-P)(p_2 + p_3) - R + T + S - P}{S + T - 2P}
     \label{eqn:condition_for_p1}\\
    p_4 & = 0 \label{eqn:condition_for_p4}\\
    1 & > p_2 + p_3\label{eqn:condition_for_chi}
\end{align}

The algebraic steps necessary to prove these results are available in the
supporting materials.

All extortionate strategies reside on a triangular (\ref{eqn:condition_for_chi})
plane (\ref{eqn:condition_for_p1}) in 3 dimensions (\ref{eqn:condition_for_p4}).
Using this formulation it can be seen that a necessary (but not sufficient)
condition for an extortionate strategy is that it cooperates on average less
than 50\% of the time when in a state of disagreement with the opponent.

As an example, consider the known extortionate strategy \(p=(8 / 9, 1 / 2, 1 /
3, 0)\) from~\cite{Stewart2012} which is referred to as \texttt{Extort-2}. In
this case, for the standard values of \((R, T, S, P)\) constraint
(\ref{eqn:condition_for_p1}) corresponds to:

\begin{equation}
    p_1 = \frac{2(p_2 + p_3) + 1}{3}
\end{equation}

It is clear that in this case all constraints hold.

This approach could in fact be used to confirm that a given strategy is acting
in an extortionate manner even if it is not a memory one strategy. However, in
practice, if a closed form for \(p\) is not known, then due to measurement
and/or numerical error this would not work.

This problem can be written in the following linear algebraic form where
\(x=(\alpha, \beta)\)
and \(p^*=(\tilde p_1 - 1, tilde_2 - 1, p_3)\):

\begin{equation}\label{eqn:linear_algebraic_equation_for_p}
    Cx= p^*
\end{equation}

\(C\) corresponds to equations
(\ref{eqn:condition_for_tilde_p1}-\ref{eqn:condition_for_tilde_p3}) and is
given by:

\begin{equation}\label{eqn:definition_of_C}
    C =
    \begin{bmatrix}
        R - P & R- P \\
        S - P & T- P \\
        T - P & S- P \\
    \end{bmatrix}
\end{equation}

Note that in general, equation (\ref{eqn:linear_algebraic_equation_for_p}) will
not necessarily have a solution. From the Rouch\'{e}-Capelli theorem if there is
a solution it is unique as \(\text{rank}(C)=2\) which is the dimension of the
variable \(x\). The best fitting \(x\) is found by minimizing:

\begin{equation}\label{eqn:r_squared}
    \text{SSError} = \|C x- p^*\|_2^2 = \sum_{i=1}^{3}\left((C\bar x)_i-p_i^*\right)^2
\end{equation}

Note that \(\text{SSError}\), which is the square of the Frobenius
norm~\cite{Golub2013}, becomes a measure of how close a strategy is to being an
extortionate strategy. Suspicion
of extortion then corresponds to a threshold on \(\text{SSError}\).

By observing interactions (human or otherwise), their memory one representation
can be inferred and this approach can be used to recognise extortionate
behaviour. The notion of comparing theoretic and actual plays of the IPD is not
novel, see for example~\cite{Rand2013}. Immediately it is noted that if the
environment is noisy~\cite{Wu1995} then no strategy can be considered to be
extortionate as \(p_4>0\).

In the next section, this idea will be illustrated by observing the interactions
that take place in a computer based tournament of the IPD\@.

\section{Numerical experiments}\label{sec:numerical-experiments}

In~\cite{Stewart2012} results from a tournament with
\input{./assets/tex/number_of_stewart_plotkin_strategies/main.tex} strategies,
was presented with specific consideration given to ZD strategies. This
tournament is reproduced here using the Axelrod-Python
project~\cite{Knight2016}. To obtain a good measure of the corresponding
transition rates for each strategy all matches have been run for
\input{assets/tex/number_of_turns/main.tex} turns and every match has been
repeated \input{assets/tex/number_of_repetitions/main.tex} times. All of this
interaction data is available at~\cite{vincent_knight_2018_1297075}. A good
match between the inferred Markov chain and the state distribution of the actual
interactions has been verified. Data for this is presented in the supplementary
materials.

Figure~\ref{fig:SSError_overall_in_stewart_plotkin} shows the \(\text{SSError}\)
values for all the strategies in the tournament, as reported
in~\cite{Stewart2012} the extortionate strategy (which has an expected
\(\text{SSError}\) approximately 0) gains a large number of wins.

\begin{figure}[!htbp]
    \centering
    \includegraphics[width=.8\textwidth]{./assets/img/SSError_overall_in_stewart_plotkin/main.pdf}
    \caption{\(\text{SSError}\) and state probabilities for the strategies
        of~\cite{Stewart2012}, ordered both by number of wins and overall score.
        Note that \(P(DC)\) is not shown as it corresponds to the transpose of
        \(P(CD)\). Cooperator and Defector are omitted as they do not visit all
        the states.}
    \label{fig:SSError_overall_in_stewart_plotkin}
\end{figure}

Here, the work of~\cite{Stewart2012} is extended by investigating a tournament
with \input{assets/tex/number_of_full_strategies/main.tex}
strategies.

The results of this analysis are shown in
Figure~\ref{fig:SSError_and_probabilities_in_full}. The top ranking strategies
by number of wins seem to be extortionate (but not against all strategies) and
it can be seen that a small sub group of strategies achieve mutual defection.
All the top ranking strategies according to score achieve mutual cooperation and
do not extort each other, however they
\textbf{do} exhibit extortionate behaviour towards a number of the lower ranking
strategies.

\begin{figure}[!htbp]
    \centering
    \includegraphics[width=.8\textwidth]{./assets/img/SSError_and_probabilities_in_full/main.pdf}
    \caption{\(\text{SSError}\) for the strategies for the full tournament. Only
    strategy interactions for which \(p_4=0\) and \(\chi>1\) are displayed.}
    \label{fig:SSError_and_probabilities_in_full}
\end{figure}

\section{Conclusion}\label{sec:conclusion}

This work defines an approach to measure whether or not a player is playing a
strategy that corresponds to an extortionate strategy as defined
in~\cite{Press2012}: a mathematical model for suspicion. Indeed, all
extortionate strategies have been
 classified as lying on a triangular plane.
This rigorous classification fails to be robust to small measurement error, thus
a statistical approach is proposed.
This is done through a linear algebraic approach for approximating the solution
of a linear system. Using this, a large number of pairwise interactions is
simulated and in fact very few strategies are found to act extortionately.

The work of~\cite{Press2012}, whilst showing that a clever approach to taking
advantage of another memory one strategy exists: this is incomplete. Whilst the
elegance of this result is very attractive, just as the simplicity of the
victory of Tit For Tat in Axelrod's original tournaments was, it is incomplete.
Extortionate strategies achieve a high number of wins but they do not
achieve a high score which corresponds to the fitness landscape in an
evolutionary sense. From the large number of interactions a payoff matrix \(S\)
can be measured where \(S_{ij}\) denotes the score (using standard values of
\((R, S, T, P) = (3, 0, 5, 1)\)) of the \(i\)th strategy
against the \(j\)th strategy. Using this, the replicator equation
describes the evolution of the system based on a population density fitness
function:

\begin{equation}\label{eqn:replicator_dynamics}
    \frac{dx}{dt} = x(S-x^TS x)
\end{equation}

Equation (\ref{eqn:replicator_dynamics}) is solved numerically through an
integration technique described in~\cite{Petzold1983} and
Figure~\ref{fig:replicator_dynamics} shows the evolution of the distribution of
the system: the various strategies are ranked by scores. It is clear to see that
only the high ranking strategies survive the evolutionary process (in fact,
only \input{./assets/img/replicator_dynamics/main.tex}
have a final distribution greater than \(10 ^ {-2}\)). This confirms the
findings of~\cite{Moran1707} in which sophisticated strategies resist
evolutionary invasion of shorter memory strategies. Recalling
Figure~\ref{fig:SSError_and_probabilities_in_full} this demonstrates that:

\begin{itemize}
    \item Cooperation emerges through the evolutionary process: the high scoring
        strategies do not exhibit extortionate behaviour towards each other.
    \item Extortionate strategies do not survive the evolutionary process.
\end{itemize}

\begin{figure}[!htbp]
    \centering
    \includegraphics[width=.8\textwidth]{./assets/img/replicator_dynamics/main.pdf}
    \caption{Numerical simulation of the replicator equation
    (\ref{eqn:replicator_dynamics}): strategies are ordered by score, only the strategies with a high score survive the evolutionary process.}
    \label{fig:replicator_dynamics}
\end{figure}

This work can be used to classify plays of the IPD\@: data can be collected from
actual interactions (in lab or in the field). Furthermore, this allows for a
classification method similar to the notion of fingerprinting presented
in~\cite{Ashlock2008}. Trained strategies can potentially be classified as
extortionate or not or it could be possible to even constrain the reinforcement
learning approaches that are becoming prevalent in the literature.
Alternatively, this mathematical approach for recognising extortion could be
used in sophisticated strategies to defend against invasion. Arguably, some of
the strategies considered here exhibit this behaviour, indeed as described
in~\cite{Harper2017}, the top ranking strategies in the full tournament are
obtained using evolutionary reinforcement learning techniques, thus, suspicion
of extortionate behaviour could in fact be an evolutionary trait.

\section*{Acknowledgements}

The following open source software libraries were used in this research:

\begin{itemize}
    \item The Axelrod ~\cite{Knight2016, Knight2018} library (IPD strategies and
        tournaments).
    \item The sympy library~\cite{Meurer2017} (verification of all symbolic
        calculations).
    \item The matplotlib~\cite{Droettboom2018} library (visualisation).
    \item The pandas~\cite{Structures2010}, dask~\cite{Dask2016} and
        NumPy~\cite{Oliphant2015} libraries (data manipulation).
    \item The SciPy~\cite{Jones2001} library (numerical integration of the
        replicator equation).
\end{itemize}

This work was performed using the computational facilities of the Advanced
Research Computing @ Cardiff (ARCCA) Division, Cardiff University.

\printbibliography

\newpage
\section*{Supplementary materials}

\includepdf{assets/pdf/proof_of_form_of_extortionate_strategies/main.pdf}

\newpage

Using the pair wise interactions the transition rates \(p,
q\) can be measured and the steady state probabilities inferred and compared to
the actual probabilities of each state.
This is done numerically by computing the singular eigenvector of the
matrix \(A\) \cite{Stewart2009}:

\[
    A =
    \begin{bmatrix}
        p_1 q_1 & p_1 (1 - q_1) & (1 - p_1) q_1 & (1 -p_1) (1 - q_1) \\
        p_2 q_2 & p_2 (1 - q_2) & (1 - p_2) q_2 & (1 -p_2) (1 - q_2) \\
        p_3 q_3 & p_3 (1 - q_3) & (1 - p_3) q_3 & (1 -p_3) (1 - q_3) \\
        p_4 q_4 & p_4 (1 - q_4) & (1 - p_4) q_4 & (1 -p_4) (1 - q_4) \\
    \end{bmatrix}
\]

Figure~\ref{fig:computed_probabilities_vs_theoretic_probabilities} shows a
regression line fitted to every pairwise interaction with a reported
\(\text{SSError}\) value (pairwise interactions with missing states were
omitted). This serves to validate the approach: a part from some edge cases the
relationship is consistent.

\begin{figure}[!htbp]
    \centering
    \includegraphics[width=.8\textwidth]{./assets/img/computed_probabilities_vs_theoretic_probabilities/main.pdf}
    \caption{The
        relationship between the steady state probabilities inferred from the
        measured transitions and the actual steady state probabilities. A linear
        regression line is included validating the approach.}
    \label{fig:computed_probabilities_vs_theoretic_probabilities}
\end{figure}


\end{document}

have a final distribution greater than \(10 ^ {-2}\)). This confirms the
findings of~\cite{Moran1707} in which sophisticated strategies resist
evolutionary invasion of shorter memory strategies. Recalling
Figure~\ref{fig:SSError_and_probabilities_in_full} this demonstrates that:

\begin{itemize}
    \item Cooperation emerges through the evolutionary process: the high scoring
        strategies do not exhibit extortionate behaviour towards each other.
    \item Extortionate strategies do not survive the evolutionary process.
\end{itemize}

\begin{figure}[!htbp]
    \centering
    \includegraphics[width=.8\textwidth]{./assets/img/replicator_dynamics/main.pdf}
    \caption{Numerical simulation of the replicator equation
    (\ref{eqn:replicator_dynamics}): strategies are ordered by score, only the strategies with a high score survive the evolutionary process.}
    \label{fig:replicator_dynamics}
\end{figure}

This work can be used to classify plays of the IPD\@: data can be collected from
actual interactions (in lab or in the field). Furthermore, this allows for a
classification method similar to the notion of fingerprinting presented
in~\cite{Ashlock2008}. Trained strategies can potentially be classified as
extortionate or not or it could be possible to even constrain the reinforcement
learning approaches that are becoming prevalent in the literature.
Alternatively, this mathematical approach for recognising extortion could be
used in sophisticated strategies to defend against invasion. Arguably, some of
the strategies considered here exhibit this behaviour, indeed as described
in~\cite{Harper2017}, the top ranking strategies in the full tournament are
obtained using evolutionary reinforcement learning techniques, thus, suspicion
of extortionate behaviour could in fact be an evolutionary trait.

\section*{Acknowledgements}

The following open source software libraries were used in this research:

\begin{itemize}
    \item The Axelrod ~\cite{Knight2016, Knight2018} library (IPD strategies and
        tournaments).
    \item The sympy library~\cite{Meurer2017} (verification of all symbolic
        calculations).
    \item The matplotlib~\cite{Droettboom2018} library (visualisation).
    \item The pandas~\cite{Structures2010}, dask~\cite{Dask2016} and
        NumPy~\cite{Oliphant2015} libraries (data manipulation).
    \item The SciPy~\cite{Jones2001} library (numerical integration of the
        replicator equation).
\end{itemize}

This work was performed using the computational facilities of the Advanced
Research Computing @ Cardiff (ARCCA) Division, Cardiff University.

\printbibliography

\newpage
\section*{Supplementary materials}

\includepdf{assets/pdf/proof_of_form_of_extortionate_strategies/main.pdf}

\newpage

Using the pair wise interactions the transition rates \(p,
q\) can be measured and the steady state probabilities inferred and compared to
the actual probabilities of each state.
This is done numerically by computing the singular eigenvector of the
matrix \(A\) \cite{Stewart2009}:

\[
    A =
    \begin{bmatrix}
        p_1 q_1 & p_1 (1 - q_1) & (1 - p_1) q_1 & (1 -p_1) (1 - q_1) \\
        p_2 q_2 & p_2 (1 - q_2) & (1 - p_2) q_2 & (1 -p_2) (1 - q_2) \\
        p_3 q_3 & p_3 (1 - q_3) & (1 - p_3) q_3 & (1 -p_3) (1 - q_3) \\
        p_4 q_4 & p_4 (1 - q_4) & (1 - p_4) q_4 & (1 -p_4) (1 - q_4) \\
    \end{bmatrix}
\]

Figure~\ref{fig:computed_probabilities_vs_theoretic_probabilities} shows a
regression line fitted to every pairwise interaction with a reported
\(\text{SSError}\) value (pairwise interactions with missing states were
omitted). This serves to validate the approach: a part from some edge cases the
relationship is consistent.

\begin{figure}[!htbp]
    \centering
    \includegraphics[width=.8\textwidth]{./assets/img/computed_probabilities_vs_theoretic_probabilities/main.pdf}
    \caption{The
        relationship between the steady state probabilities inferred from the
        measured transitions and the actual steady state probabilities. A linear
        regression line is included validating the approach.}
    \label{fig:computed_probabilities_vs_theoretic_probabilities}
\end{figure}


\end{document}
 times. All of this
interaction data is available at~\cite{vincent_knight_2018_1297075}. A good
match between the inferred Markov chain and the state distribution of the actual
interactions has been verified. Data for this is presented in the supplementary
materials.

Figure~\ref{fig:SSError_overall_in_stewart_plotkin} shows the \(\text{SSError}\)
values for all the strategies in the tournament, as reported
in~\cite{Stewart2012} the extortionate strategy (which has an expected
\(\text{SSError}\) approximately 0) gains a large number of wins.

\begin{figure}[!htbp]
    \centering
    \includegraphics[width=.8\textwidth]{./assets/img/SSError_overall_in_stewart_plotkin/main.pdf}
    \caption{\(\text{SSError}\) and state probabilities for the strategies
        of~\cite{Stewart2012}, ordered both by number of wins and overall score.
        Note that \(P(DC)\) is not shown as it corresponds to the transpose of
        \(P(CD)\). Cooperator and Defector are omitted as they do not visit all
        the states.}
    \label{fig:SSError_overall_in_stewart_plotkin}
\end{figure}

Here, the work of~\cite{Stewart2012} is extended by investigating a tournament
with \documentclass[a4paper]{article}

\usepackage{amsmath}
\usepackage{amssymb}
\usepackage[margin=1.5cm,
            includefoot,
            footskip=30pt]{geometry}
\usepackage{layout}
\usepackage{graphicx}
\usepackage{subcaption}

\usepackage{biblatex}
\usepackage{pdfpages}

\bibliography{main.bib}

\title{Suspicion: Recognising and evaluating the effectiveness
       of extortion in the Iterated Prisoner's Dilemma}
\author{Vincent A. Knight \and Nikoleta E. Glynatsi}
\date{\today}



\begin{document}

\maketitle

\begin{abstract}
    The Iterated Prisoner's Dilemma is a model for rational and evolutionary
    interactive behaviour. It has applications both in the study of human social
    behaviour as well as in biology.
    It is used to understand when and how a rational individual might
    accept an immediate cost to their own utility for the direct benefit of
    another.

    Much attention has been given to a class of strategies called
    Zero Determinant strategies. It has been theoretically shown that these
    strategies can ``extort'' any player.

    In this work, an approach to identify if observed strategies are playing in
    an extortionate way is described. Furthermore, experimental analysis of
    a large tournament with \documentclass[a4paper]{article}

\usepackage{amsmath}
\usepackage{amssymb}
\usepackage[margin=1.5cm,
            includefoot,
            footskip=30pt]{geometry}
\usepackage{layout}
\usepackage{graphicx}
\usepackage{subcaption}

\usepackage{biblatex}
\usepackage{pdfpages}

\bibliography{main.bib}

\title{Suspicion: Recognising and evaluating the effectiveness
       of extortion in the Iterated Prisoner's Dilemma}
\author{Vincent A. Knight \and Nikoleta E. Glynatsi}
\date{\today}



\begin{document}

\maketitle

\begin{abstract}
    The Iterated Prisoner's Dilemma is a model for rational and evolutionary
    interactive behaviour. It has applications both in the study of human social
    behaviour as well as in biology.
    It is used to understand when and how a rational individual might
    accept an immediate cost to their own utility for the direct benefit of
    another.

    Much attention has been given to a class of strategies called
    Zero Determinant strategies. It has been theoretically shown that these
    strategies can ``extort'' any player.

    In this work, an approach to identify if observed strategies are playing in
    an extortionate way is described. Furthermore, experimental analysis of
    a large tournament with \input{assets/tex/number_of_full_strategies/main.tex}
    strategies is considered. In this setting
    the most highly performing strategies do not play in an extortionate way
    against each other but do against lower performing strategies.
    This suggests that whilst the theory of Zero Determinant strategies
    indicates that memory is not of fundamental importance to the evolution of
    cooperative behaviour, this is incomplete.
\end{abstract}

\section{Introduction}\label{sec:introduction}

Agent based game theoretic models have become a stalwart of the underpinning
mathematics of interactive behaviours. One of the major pieces of work
in this area is the pair of original computer tournaments run by Robert
Axelrod~\cite{Axelrod1980, Axelrod1980a}. These tournaments pitted submitted
computer strategies against each other in plays of the Iterated Prisoner's
Dilemma. A common game where agents can choose to pay a slight cost to their
immediate utility in the hope of building a reputation. This has been used in
economic and evolutionary game theory to understand the evolution of cooperative
behaviour.

Recently, a class of strategies was described in~\cite{Press2012} that can
provably extort any given opponent. In~\cite{Hilbe2013, Moran1707} some
questions have already been asked about the true effectiveness of these
strategies in an evolutionary setting. Here another question is asked: is it
possible to recognise this extortionate behaviour? A mathematical procedure for
suspicion is presented: in the same way that the continued actions of an
extortionate individual might raise suspicion.

This work makes use of the Axelrod Python library~\cite{Knight2018, Knight2016}
with a large number of Prisoner Dilemma strategies available to give an
extensive numerical example of the ideas presented.  The approach is presented
in Section~\ref{sec:delta-zd-strategies}.  All of the code and data discussed
in Section~\ref{sec:numerical-experiments} is open sourced, archived and
written according to best scientific principles~\cite{Wilson2014}. The data
archive can be found at~\cite{vincent_knight_2018_1297075}.

\section{Recognising Extortion}\label{sec:delta-zd-strategies}

In~\cite{Press2012}, given a match between 2 memory-one strategies, the concept
of Zero Determinant (ZD) strategies is introduced. The main result of that paper
shows that given two memory one players \(p, q\in\mathbb{R}^4\) a linear
relationship between the players' scores could be forced by one of the players.

Using the notation of~\cite{Press2012}, assuming the utilities for player \(p\)
are given by \(S_x=(R, S, T, P)\) and for player \(q\) by \(S_y=(R, T, S, P)\)
and that the stationary scores of each player is given by \(S_X\) and \(S_Y\)
respectively. The main result of~\cite{Press2012} is that if

\begin{equation}\label{eqn:linear_relationship_for_p}
    \tilde p=\alpha S_x + \beta S_y + \gamma
\end{equation}

or

\begin{equation}\label{eqn:linear_relationship_for_q}
    \tilde q=\alpha S_x + \beta S_y + \gamma
\end{equation}

where \(\tilde p = (1 - p_1, 1 - p_2, p_3, p_4)\) and
\(\tilde q = (1 - q_1, 1 - q_2, q_3, q_4)\) then:

\begin{equation}
    \alpha S_X + \beta S_Y + \gamma = 0
\end{equation}

In~\cite{Press2012} a particular type of ZD strategy is defined: extortionate
strategies. If:

\begin{equation}\label{eqn:constraint_for_extortion}
    \gamma = - P(\alpha + \beta)
\end{equation}

then the player can ensure they get a score \(\chi\) times
larger than the opponent. This extortion coefficient is given by:

\begin{equation}\label{eqn:definition_of_chi}
    \chi=\frac{-\beta}{\alpha}
\end{equation}

Thus, if (\ref{eqn:constraint_for_extortion}) holds and \(\chi >1\) a player is
said to extort their opponent.
Here, the reverse problem is considered: given a
\(p\in\mathbb{R}^4\) how does one identify \(\alpha, \beta\) if they
exist and is the strategy in fact acting in an extortionate way?

These conditions correspond to:

\begin{align}
    \tilde p_1 & = \alpha R + \beta R - P (\alpha + \beta)
            \label{eqn:condition_for_tilde_p1}\\
    \tilde p_2 & = \alpha S + \beta T - P (\alpha + \beta)
            \label{eqn:condition_for_tilde_p2}\\
    \tilde p_3 & = \alpha T + \beta S - P (\alpha + \beta)
            \label{eqn:condition_for_tilde_p3}\\
    \tilde p_4 & = \alpha P + \beta P - P (\alpha + \beta)
            \label{eqn:condition_for_tilde_p4}
\end{align}

Equation (\ref{eqn:condition_for_tilde_p4}) ensures that \(p_4=\tilde p_4=0\).
Equations (\ref{eqn:condition_for_tilde_p1}-\ref{eqn:condition_for_tilde_p3})
can be used to eliminate \(\alpha, \beta\), giving:

\begin{equation}\label{eqn:planar_definition_of_extortion}
    \tilde p_1 = \frac{(R - P)(\tilde p_2 + \tilde p_3)}{S + T - 2P}
\end{equation}

with:

\begin{equation}\label{eqn:definition_of_chi}
    \chi = \frac{\tilde p_2 (P - T) + \tilde p_3 (S - P)}
                {\tilde p_2 (P - S) + \tilde p_3 (T - P)}
\end{equation}

Given a strategy \(p\in\mathbb{R}^{4\times 1}\) equations
(\ref{eqn:condition_for_tilde_p4}), (\ref{eqn:planar_definition_of_extortion}-\ref{eqn:definition_of_chi}) can be used to check if
a strategy is extortionate. The conditions correspond to:

\begin{align}
    p_1 & = \frac{(R-P)(p_2 + p_3) - R + T + S - P}{S + T - 2P}
     \label{eqn:condition_for_p1}\\
    p_4 & = 0 \label{eqn:condition_for_p4}\\
    1 & > p_2 + p_3\label{eqn:condition_for_chi}
\end{align}

The algebraic steps necessary to prove these results are available in the
supporting materials.

All extortionate strategies reside on a triangular (\ref{eqn:condition_for_chi})
plane (\ref{eqn:condition_for_p1}) in 3 dimensions (\ref{eqn:condition_for_p4}).
Using this formulation it can be seen that a necessary (but not sufficient)
condition for an extortionate strategy is that it cooperates on average less
than 50\% of the time when in a state of disagreement with the opponent.

As an example, consider the known extortionate strategy \(p=(8 / 9, 1 / 2, 1 /
3, 0)\) from~\cite{Stewart2012} which is referred to as \texttt{Extort-2}. In
this case, for the standard values of \((R, T, S, P)\) constraint
(\ref{eqn:condition_for_p1}) corresponds to:

\begin{equation}
    p_1 = \frac{2(p_2 + p_3) + 1}{3}
\end{equation}

It is clear that in this case all constraints hold.

This approach could in fact be used to confirm that a given strategy is acting
in an extortionate manner even if it is not a memory one strategy. However, in
practice, if a closed form for \(p\) is not known, then due to measurement
and/or numerical error this would not work.

This problem can be written in the following linear algebraic form where
\(x=(\alpha, \beta)\)
and \(p^*=(\tilde p_1 - 1, tilde_2 - 1, p_3)\):

\begin{equation}\label{eqn:linear_algebraic_equation_for_p}
    Cx= p^*
\end{equation}

\(C\) corresponds to equations
(\ref{eqn:condition_for_tilde_p1}-\ref{eqn:condition_for_tilde_p3}) and is
given by:

\begin{equation}\label{eqn:definition_of_C}
    C =
    \begin{bmatrix}
        R - P & R- P \\
        S - P & T- P \\
        T - P & S- P \\
    \end{bmatrix}
\end{equation}

Note that in general, equation (\ref{eqn:linear_algebraic_equation_for_p}) will
not necessarily have a solution. From the Rouch\'{e}-Capelli theorem if there is
a solution it is unique as \(\text{rank}(C)=2\) which is the dimension of the
variable \(x\). The best fitting \(x\) is found by minimizing:

\begin{equation}\label{eqn:r_squared}
    \text{SSError} = \|C x- p^*\|_2^2 = \sum_{i=1}^{3}\left((C\bar x)_i-p_i^*\right)^2
\end{equation}

Note that \(\text{SSError}\), which is the square of the Frobenius
norm~\cite{Golub2013}, becomes a measure of how close a strategy is to being an
extortionate strategy. Suspicion
of extortion then corresponds to a threshold on \(\text{SSError}\).

By observing interactions (human or otherwise), their memory one representation
can be inferred and this approach can be used to recognise extortionate
behaviour. The notion of comparing theoretic and actual plays of the IPD is not
novel, see for example~\cite{Rand2013}. Immediately it is noted that if the
environment is noisy~\cite{Wu1995} then no strategy can be considered to be
extortionate as \(p_4>0\).

In the next section, this idea will be illustrated by observing the interactions
that take place in a computer based tournament of the IPD\@.

\section{Numerical experiments}\label{sec:numerical-experiments}

In~\cite{Stewart2012} results from a tournament with
\input{./assets/tex/number_of_stewart_plotkin_strategies/main.tex} strategies,
was presented with specific consideration given to ZD strategies. This
tournament is reproduced here using the Axelrod-Python
project~\cite{Knight2016}. To obtain a good measure of the corresponding
transition rates for each strategy all matches have been run for
\input{assets/tex/number_of_turns/main.tex} turns and every match has been
repeated \input{assets/tex/number_of_repetitions/main.tex} times. All of this
interaction data is available at~\cite{vincent_knight_2018_1297075}. A good
match between the inferred Markov chain and the state distribution of the actual
interactions has been verified. Data for this is presented in the supplementary
materials.

Figure~\ref{fig:SSError_overall_in_stewart_plotkin} shows the \(\text{SSError}\)
values for all the strategies in the tournament, as reported
in~\cite{Stewart2012} the extortionate strategy (which has an expected
\(\text{SSError}\) approximately 0) gains a large number of wins.

\begin{figure}[!htbp]
    \centering
    \includegraphics[width=.8\textwidth]{./assets/img/SSError_overall_in_stewart_plotkin/main.pdf}
    \caption{\(\text{SSError}\) and state probabilities for the strategies
        of~\cite{Stewart2012}, ordered both by number of wins and overall score.
        Note that \(P(DC)\) is not shown as it corresponds to the transpose of
        \(P(CD)\). Cooperator and Defector are omitted as they do not visit all
        the states.}
    \label{fig:SSError_overall_in_stewart_plotkin}
\end{figure}

Here, the work of~\cite{Stewart2012} is extended by investigating a tournament
with \input{assets/tex/number_of_full_strategies/main.tex}
strategies.

The results of this analysis are shown in
Figure~\ref{fig:SSError_and_probabilities_in_full}. The top ranking strategies
by number of wins seem to be extortionate (but not against all strategies) and
it can be seen that a small sub group of strategies achieve mutual defection.
All the top ranking strategies according to score achieve mutual cooperation and
do not extort each other, however they
\textbf{do} exhibit extortionate behaviour towards a number of the lower ranking
strategies.

\begin{figure}[!htbp]
    \centering
    \includegraphics[width=.8\textwidth]{./assets/img/SSError_and_probabilities_in_full/main.pdf}
    \caption{\(\text{SSError}\) for the strategies for the full tournament. Only
    strategy interactions for which \(p_4=0\) and \(\chi>1\) are displayed.}
    \label{fig:SSError_and_probabilities_in_full}
\end{figure}

\section{Conclusion}\label{sec:conclusion}

This work defines an approach to measure whether or not a player is playing a
strategy that corresponds to an extortionate strategy as defined
in~\cite{Press2012}: a mathematical model for suspicion. Indeed, all
extortionate strategies have been
 classified as lying on a triangular plane.
This rigorous classification fails to be robust to small measurement error, thus
a statistical approach is proposed.
This is done through a linear algebraic approach for approximating the solution
of a linear system. Using this, a large number of pairwise interactions is
simulated and in fact very few strategies are found to act extortionately.

The work of~\cite{Press2012}, whilst showing that a clever approach to taking
advantage of another memory one strategy exists: this is incomplete. Whilst the
elegance of this result is very attractive, just as the simplicity of the
victory of Tit For Tat in Axelrod's original tournaments was, it is incomplete.
Extortionate strategies achieve a high number of wins but they do not
achieve a high score which corresponds to the fitness landscape in an
evolutionary sense. From the large number of interactions a payoff matrix \(S\)
can be measured where \(S_{ij}\) denotes the score (using standard values of
\((R, S, T, P) = (3, 0, 5, 1)\)) of the \(i\)th strategy
against the \(j\)th strategy. Using this, the replicator equation
describes the evolution of the system based on a population density fitness
function:

\begin{equation}\label{eqn:replicator_dynamics}
    \frac{dx}{dt} = x(S-x^TS x)
\end{equation}

Equation (\ref{eqn:replicator_dynamics}) is solved numerically through an
integration technique described in~\cite{Petzold1983} and
Figure~\ref{fig:replicator_dynamics} shows the evolution of the distribution of
the system: the various strategies are ranked by scores. It is clear to see that
only the high ranking strategies survive the evolutionary process (in fact,
only \input{./assets/img/replicator_dynamics/main.tex}
have a final distribution greater than \(10 ^ {-2}\)). This confirms the
findings of~\cite{Moran1707} in which sophisticated strategies resist
evolutionary invasion of shorter memory strategies. Recalling
Figure~\ref{fig:SSError_and_probabilities_in_full} this demonstrates that:

\begin{itemize}
    \item Cooperation emerges through the evolutionary process: the high scoring
        strategies do not exhibit extortionate behaviour towards each other.
    \item Extortionate strategies do not survive the evolutionary process.
\end{itemize}

\begin{figure}[!htbp]
    \centering
    \includegraphics[width=.8\textwidth]{./assets/img/replicator_dynamics/main.pdf}
    \caption{Numerical simulation of the replicator equation
    (\ref{eqn:replicator_dynamics}): strategies are ordered by score, only the strategies with a high score survive the evolutionary process.}
    \label{fig:replicator_dynamics}
\end{figure}

This work can be used to classify plays of the IPD\@: data can be collected from
actual interactions (in lab or in the field). Furthermore, this allows for a
classification method similar to the notion of fingerprinting presented
in~\cite{Ashlock2008}. Trained strategies can potentially be classified as
extortionate or not or it could be possible to even constrain the reinforcement
learning approaches that are becoming prevalent in the literature.
Alternatively, this mathematical approach for recognising extortion could be
used in sophisticated strategies to defend against invasion. Arguably, some of
the strategies considered here exhibit this behaviour, indeed as described
in~\cite{Harper2017}, the top ranking strategies in the full tournament are
obtained using evolutionary reinforcement learning techniques, thus, suspicion
of extortionate behaviour could in fact be an evolutionary trait.

\section*{Acknowledgements}

The following open source software libraries were used in this research:

\begin{itemize}
    \item The Axelrod ~\cite{Knight2016, Knight2018} library (IPD strategies and
        tournaments).
    \item The sympy library~\cite{Meurer2017} (verification of all symbolic
        calculations).
    \item The matplotlib~\cite{Droettboom2018} library (visualisation).
    \item The pandas~\cite{Structures2010}, dask~\cite{Dask2016} and
        NumPy~\cite{Oliphant2015} libraries (data manipulation).
    \item The SciPy~\cite{Jones2001} library (numerical integration of the
        replicator equation).
\end{itemize}

This work was performed using the computational facilities of the Advanced
Research Computing @ Cardiff (ARCCA) Division, Cardiff University.

\printbibliography

\newpage
\section*{Supplementary materials}

\includepdf{assets/pdf/proof_of_form_of_extortionate_strategies/main.pdf}

\newpage

Using the pair wise interactions the transition rates \(p,
q\) can be measured and the steady state probabilities inferred and compared to
the actual probabilities of each state.
This is done numerically by computing the singular eigenvector of the
matrix \(A\) \cite{Stewart2009}:

\[
    A =
    \begin{bmatrix}
        p_1 q_1 & p_1 (1 - q_1) & (1 - p_1) q_1 & (1 -p_1) (1 - q_1) \\
        p_2 q_2 & p_2 (1 - q_2) & (1 - p_2) q_2 & (1 -p_2) (1 - q_2) \\
        p_3 q_3 & p_3 (1 - q_3) & (1 - p_3) q_3 & (1 -p_3) (1 - q_3) \\
        p_4 q_4 & p_4 (1 - q_4) & (1 - p_4) q_4 & (1 -p_4) (1 - q_4) \\
    \end{bmatrix}
\]

Figure~\ref{fig:computed_probabilities_vs_theoretic_probabilities} shows a
regression line fitted to every pairwise interaction with a reported
\(\text{SSError}\) value (pairwise interactions with missing states were
omitted). This serves to validate the approach: a part from some edge cases the
relationship is consistent.

\begin{figure}[!htbp]
    \centering
    \includegraphics[width=.8\textwidth]{./assets/img/computed_probabilities_vs_theoretic_probabilities/main.pdf}
    \caption{The
        relationship between the steady state probabilities inferred from the
        measured transitions and the actual steady state probabilities. A linear
        regression line is included validating the approach.}
    \label{fig:computed_probabilities_vs_theoretic_probabilities}
\end{figure}


\end{document}

    strategies is considered. In this setting
    the most highly performing strategies do not play in an extortionate way
    against each other but do against lower performing strategies.
    This suggests that whilst the theory of Zero Determinant strategies
    indicates that memory is not of fundamental importance to the evolution of
    cooperative behaviour, this is incomplete.
\end{abstract}

\section{Introduction}\label{sec:introduction}

Agent based game theoretic models have become a stalwart of the underpinning
mathematics of interactive behaviours. One of the major pieces of work
in this area is the pair of original computer tournaments run by Robert
Axelrod~\cite{Axelrod1980, Axelrod1980a}. These tournaments pitted submitted
computer strategies against each other in plays of the Iterated Prisoner's
Dilemma. A common game where agents can choose to pay a slight cost to their
immediate utility in the hope of building a reputation. This has been used in
economic and evolutionary game theory to understand the evolution of cooperative
behaviour.

Recently, a class of strategies was described in~\cite{Press2012} that can
provably extort any given opponent. In~\cite{Hilbe2013, Moran1707} some
questions have already been asked about the true effectiveness of these
strategies in an evolutionary setting. Here another question is asked: is it
possible to recognise this extortionate behaviour? A mathematical procedure for
suspicion is presented: in the same way that the continued actions of an
extortionate individual might raise suspicion.

This work makes use of the Axelrod Python library~\cite{Knight2018, Knight2016}
with a large number of Prisoner Dilemma strategies available to give an
extensive numerical example of the ideas presented.  The approach is presented
in Section~\ref{sec:delta-zd-strategies}.  All of the code and data discussed
in Section~\ref{sec:numerical-experiments} is open sourced, archived and
written according to best scientific principles~\cite{Wilson2014}. The data
archive can be found at~\cite{vincent_knight_2018_1297075}.

\section{Recognising Extortion}\label{sec:delta-zd-strategies}

In~\cite{Press2012}, given a match between 2 memory-one strategies, the concept
of Zero Determinant (ZD) strategies is introduced. The main result of that paper
shows that given two memory one players \(p, q\in\mathbb{R}^4\) a linear
relationship between the players' scores could be forced by one of the players.

Using the notation of~\cite{Press2012}, assuming the utilities for player \(p\)
are given by \(S_x=(R, S, T, P)\) and for player \(q\) by \(S_y=(R, T, S, P)\)
and that the stationary scores of each player is given by \(S_X\) and \(S_Y\)
respectively. The main result of~\cite{Press2012} is that if

\begin{equation}\label{eqn:linear_relationship_for_p}
    \tilde p=\alpha S_x + \beta S_y + \gamma
\end{equation}

or

\begin{equation}\label{eqn:linear_relationship_for_q}
    \tilde q=\alpha S_x + \beta S_y + \gamma
\end{equation}

where \(\tilde p = (1 - p_1, 1 - p_2, p_3, p_4)\) and
\(\tilde q = (1 - q_1, 1 - q_2, q_3, q_4)\) then:

\begin{equation}
    \alpha S_X + \beta S_Y + \gamma = 0
\end{equation}

In~\cite{Press2012} a particular type of ZD strategy is defined: extortionate
strategies. If:

\begin{equation}\label{eqn:constraint_for_extortion}
    \gamma = - P(\alpha + \beta)
\end{equation}

then the player can ensure they get a score \(\chi\) times
larger than the opponent. This extortion coefficient is given by:

\begin{equation}\label{eqn:definition_of_chi}
    \chi=\frac{-\beta}{\alpha}
\end{equation}

Thus, if (\ref{eqn:constraint_for_extortion}) holds and \(\chi >1\) a player is
said to extort their opponent.
Here, the reverse problem is considered: given a
\(p\in\mathbb{R}^4\) how does one identify \(\alpha, \beta\) if they
exist and is the strategy in fact acting in an extortionate way?

These conditions correspond to:

\begin{align}
    \tilde p_1 & = \alpha R + \beta R - P (\alpha + \beta)
            \label{eqn:condition_for_tilde_p1}\\
    \tilde p_2 & = \alpha S + \beta T - P (\alpha + \beta)
            \label{eqn:condition_for_tilde_p2}\\
    \tilde p_3 & = \alpha T + \beta S - P (\alpha + \beta)
            \label{eqn:condition_for_tilde_p3}\\
    \tilde p_4 & = \alpha P + \beta P - P (\alpha + \beta)
            \label{eqn:condition_for_tilde_p4}
\end{align}

Equation (\ref{eqn:condition_for_tilde_p4}) ensures that \(p_4=\tilde p_4=0\).
Equations (\ref{eqn:condition_for_tilde_p1}-\ref{eqn:condition_for_tilde_p3})
can be used to eliminate \(\alpha, \beta\), giving:

\begin{equation}\label{eqn:planar_definition_of_extortion}
    \tilde p_1 = \frac{(R - P)(\tilde p_2 + \tilde p_3)}{S + T - 2P}
\end{equation}

with:

\begin{equation}\label{eqn:definition_of_chi}
    \chi = \frac{\tilde p_2 (P - T) + \tilde p_3 (S - P)}
                {\tilde p_2 (P - S) + \tilde p_3 (T - P)}
\end{equation}

Given a strategy \(p\in\mathbb{R}^{4\times 1}\) equations
(\ref{eqn:condition_for_tilde_p4}), (\ref{eqn:planar_definition_of_extortion}-\ref{eqn:definition_of_chi}) can be used to check if
a strategy is extortionate. The conditions correspond to:

\begin{align}
    p_1 & = \frac{(R-P)(p_2 + p_3) - R + T + S - P}{S + T - 2P}
     \label{eqn:condition_for_p1}\\
    p_4 & = 0 \label{eqn:condition_for_p4}\\
    1 & > p_2 + p_3\label{eqn:condition_for_chi}
\end{align}

The algebraic steps necessary to prove these results are available in the
supporting materials.

All extortionate strategies reside on a triangular (\ref{eqn:condition_for_chi})
plane (\ref{eqn:condition_for_p1}) in 3 dimensions (\ref{eqn:condition_for_p4}).
Using this formulation it can be seen that a necessary (but not sufficient)
condition for an extortionate strategy is that it cooperates on average less
than 50\% of the time when in a state of disagreement with the opponent.

As an example, consider the known extortionate strategy \(p=(8 / 9, 1 / 2, 1 /
3, 0)\) from~\cite{Stewart2012} which is referred to as \texttt{Extort-2}. In
this case, for the standard values of \((R, T, S, P)\) constraint
(\ref{eqn:condition_for_p1}) corresponds to:

\begin{equation}
    p_1 = \frac{2(p_2 + p_3) + 1}{3}
\end{equation}

It is clear that in this case all constraints hold.

This approach could in fact be used to confirm that a given strategy is acting
in an extortionate manner even if it is not a memory one strategy. However, in
practice, if a closed form for \(p\) is not known, then due to measurement
and/or numerical error this would not work.

This problem can be written in the following linear algebraic form where
\(x=(\alpha, \beta)\)
and \(p^*=(\tilde p_1 - 1, tilde_2 - 1, p_3)\):

\begin{equation}\label{eqn:linear_algebraic_equation_for_p}
    Cx= p^*
\end{equation}

\(C\) corresponds to equations
(\ref{eqn:condition_for_tilde_p1}-\ref{eqn:condition_for_tilde_p3}) and is
given by:

\begin{equation}\label{eqn:definition_of_C}
    C =
    \begin{bmatrix}
        R - P & R- P \\
        S - P & T- P \\
        T - P & S- P \\
    \end{bmatrix}
\end{equation}

Note that in general, equation (\ref{eqn:linear_algebraic_equation_for_p}) will
not necessarily have a solution. From the Rouch\'{e}-Capelli theorem if there is
a solution it is unique as \(\text{rank}(C)=2\) which is the dimension of the
variable \(x\). The best fitting \(x\) is found by minimizing:

\begin{equation}\label{eqn:r_squared}
    \text{SSError} = \|C x- p^*\|_2^2 = \sum_{i=1}^{3}\left((C\bar x)_i-p_i^*\right)^2
\end{equation}

Note that \(\text{SSError}\), which is the square of the Frobenius
norm~\cite{Golub2013}, becomes a measure of how close a strategy is to being an
extortionate strategy. Suspicion
of extortion then corresponds to a threshold on \(\text{SSError}\).

By observing interactions (human or otherwise), their memory one representation
can be inferred and this approach can be used to recognise extortionate
behaviour. The notion of comparing theoretic and actual plays of the IPD is not
novel, see for example~\cite{Rand2013}. Immediately it is noted that if the
environment is noisy~\cite{Wu1995} then no strategy can be considered to be
extortionate as \(p_4>0\).

In the next section, this idea will be illustrated by observing the interactions
that take place in a computer based tournament of the IPD\@.

\section{Numerical experiments}\label{sec:numerical-experiments}

In~\cite{Stewart2012} results from a tournament with
\documentclass[a4paper]{article}

\usepackage{amsmath}
\usepackage{amssymb}
\usepackage[margin=1.5cm,
            includefoot,
            footskip=30pt]{geometry}
\usepackage{layout}
\usepackage{graphicx}
\usepackage{subcaption}

\usepackage{biblatex}
\usepackage{pdfpages}

\bibliography{main.bib}

\title{Suspicion: Recognising and evaluating the effectiveness
       of extortion in the Iterated Prisoner's Dilemma}
\author{Vincent A. Knight \and Nikoleta E. Glynatsi}
\date{\today}



\begin{document}

\maketitle

\begin{abstract}
    The Iterated Prisoner's Dilemma is a model for rational and evolutionary
    interactive behaviour. It has applications both in the study of human social
    behaviour as well as in biology.
    It is used to understand when and how a rational individual might
    accept an immediate cost to their own utility for the direct benefit of
    another.

    Much attention has been given to a class of strategies called
    Zero Determinant strategies. It has been theoretically shown that these
    strategies can ``extort'' any player.

    In this work, an approach to identify if observed strategies are playing in
    an extortionate way is described. Furthermore, experimental analysis of
    a large tournament with \input{assets/tex/number_of_full_strategies/main.tex}
    strategies is considered. In this setting
    the most highly performing strategies do not play in an extortionate way
    against each other but do against lower performing strategies.
    This suggests that whilst the theory of Zero Determinant strategies
    indicates that memory is not of fundamental importance to the evolution of
    cooperative behaviour, this is incomplete.
\end{abstract}

\section{Introduction}\label{sec:introduction}

Agent based game theoretic models have become a stalwart of the underpinning
mathematics of interactive behaviours. One of the major pieces of work
in this area is the pair of original computer tournaments run by Robert
Axelrod~\cite{Axelrod1980, Axelrod1980a}. These tournaments pitted submitted
computer strategies against each other in plays of the Iterated Prisoner's
Dilemma. A common game where agents can choose to pay a slight cost to their
immediate utility in the hope of building a reputation. This has been used in
economic and evolutionary game theory to understand the evolution of cooperative
behaviour.

Recently, a class of strategies was described in~\cite{Press2012} that can
provably extort any given opponent. In~\cite{Hilbe2013, Moran1707} some
questions have already been asked about the true effectiveness of these
strategies in an evolutionary setting. Here another question is asked: is it
possible to recognise this extortionate behaviour? A mathematical procedure for
suspicion is presented: in the same way that the continued actions of an
extortionate individual might raise suspicion.

This work makes use of the Axelrod Python library~\cite{Knight2018, Knight2016}
with a large number of Prisoner Dilemma strategies available to give an
extensive numerical example of the ideas presented.  The approach is presented
in Section~\ref{sec:delta-zd-strategies}.  All of the code and data discussed
in Section~\ref{sec:numerical-experiments} is open sourced, archived and
written according to best scientific principles~\cite{Wilson2014}. The data
archive can be found at~\cite{vincent_knight_2018_1297075}.

\section{Recognising Extortion}\label{sec:delta-zd-strategies}

In~\cite{Press2012}, given a match between 2 memory-one strategies, the concept
of Zero Determinant (ZD) strategies is introduced. The main result of that paper
shows that given two memory one players \(p, q\in\mathbb{R}^4\) a linear
relationship between the players' scores could be forced by one of the players.

Using the notation of~\cite{Press2012}, assuming the utilities for player \(p\)
are given by \(S_x=(R, S, T, P)\) and for player \(q\) by \(S_y=(R, T, S, P)\)
and that the stationary scores of each player is given by \(S_X\) and \(S_Y\)
respectively. The main result of~\cite{Press2012} is that if

\begin{equation}\label{eqn:linear_relationship_for_p}
    \tilde p=\alpha S_x + \beta S_y + \gamma
\end{equation}

or

\begin{equation}\label{eqn:linear_relationship_for_q}
    \tilde q=\alpha S_x + \beta S_y + \gamma
\end{equation}

where \(\tilde p = (1 - p_1, 1 - p_2, p_3, p_4)\) and
\(\tilde q = (1 - q_1, 1 - q_2, q_3, q_4)\) then:

\begin{equation}
    \alpha S_X + \beta S_Y + \gamma = 0
\end{equation}

In~\cite{Press2012} a particular type of ZD strategy is defined: extortionate
strategies. If:

\begin{equation}\label{eqn:constraint_for_extortion}
    \gamma = - P(\alpha + \beta)
\end{equation}

then the player can ensure they get a score \(\chi\) times
larger than the opponent. This extortion coefficient is given by:

\begin{equation}\label{eqn:definition_of_chi}
    \chi=\frac{-\beta}{\alpha}
\end{equation}

Thus, if (\ref{eqn:constraint_for_extortion}) holds and \(\chi >1\) a player is
said to extort their opponent.
Here, the reverse problem is considered: given a
\(p\in\mathbb{R}^4\) how does one identify \(\alpha, \beta\) if they
exist and is the strategy in fact acting in an extortionate way?

These conditions correspond to:

\begin{align}
    \tilde p_1 & = \alpha R + \beta R - P (\alpha + \beta)
            \label{eqn:condition_for_tilde_p1}\\
    \tilde p_2 & = \alpha S + \beta T - P (\alpha + \beta)
            \label{eqn:condition_for_tilde_p2}\\
    \tilde p_3 & = \alpha T + \beta S - P (\alpha + \beta)
            \label{eqn:condition_for_tilde_p3}\\
    \tilde p_4 & = \alpha P + \beta P - P (\alpha + \beta)
            \label{eqn:condition_for_tilde_p4}
\end{align}

Equation (\ref{eqn:condition_for_tilde_p4}) ensures that \(p_4=\tilde p_4=0\).
Equations (\ref{eqn:condition_for_tilde_p1}-\ref{eqn:condition_for_tilde_p3})
can be used to eliminate \(\alpha, \beta\), giving:

\begin{equation}\label{eqn:planar_definition_of_extortion}
    \tilde p_1 = \frac{(R - P)(\tilde p_2 + \tilde p_3)}{S + T - 2P}
\end{equation}

with:

\begin{equation}\label{eqn:definition_of_chi}
    \chi = \frac{\tilde p_2 (P - T) + \tilde p_3 (S - P)}
                {\tilde p_2 (P - S) + \tilde p_3 (T - P)}
\end{equation}

Given a strategy \(p\in\mathbb{R}^{4\times 1}\) equations
(\ref{eqn:condition_for_tilde_p4}), (\ref{eqn:planar_definition_of_extortion}-\ref{eqn:definition_of_chi}) can be used to check if
a strategy is extortionate. The conditions correspond to:

\begin{align}
    p_1 & = \frac{(R-P)(p_2 + p_3) - R + T + S - P}{S + T - 2P}
     \label{eqn:condition_for_p1}\\
    p_4 & = 0 \label{eqn:condition_for_p4}\\
    1 & > p_2 + p_3\label{eqn:condition_for_chi}
\end{align}

The algebraic steps necessary to prove these results are available in the
supporting materials.

All extortionate strategies reside on a triangular (\ref{eqn:condition_for_chi})
plane (\ref{eqn:condition_for_p1}) in 3 dimensions (\ref{eqn:condition_for_p4}).
Using this formulation it can be seen that a necessary (but not sufficient)
condition for an extortionate strategy is that it cooperates on average less
than 50\% of the time when in a state of disagreement with the opponent.

As an example, consider the known extortionate strategy \(p=(8 / 9, 1 / 2, 1 /
3, 0)\) from~\cite{Stewart2012} which is referred to as \texttt{Extort-2}. In
this case, for the standard values of \((R, T, S, P)\) constraint
(\ref{eqn:condition_for_p1}) corresponds to:

\begin{equation}
    p_1 = \frac{2(p_2 + p_3) + 1}{3}
\end{equation}

It is clear that in this case all constraints hold.

This approach could in fact be used to confirm that a given strategy is acting
in an extortionate manner even if it is not a memory one strategy. However, in
practice, if a closed form for \(p\) is not known, then due to measurement
and/or numerical error this would not work.

This problem can be written in the following linear algebraic form where
\(x=(\alpha, \beta)\)
and \(p^*=(\tilde p_1 - 1, tilde_2 - 1, p_3)\):

\begin{equation}\label{eqn:linear_algebraic_equation_for_p}
    Cx= p^*
\end{equation}

\(C\) corresponds to equations
(\ref{eqn:condition_for_tilde_p1}-\ref{eqn:condition_for_tilde_p3}) and is
given by:

\begin{equation}\label{eqn:definition_of_C}
    C =
    \begin{bmatrix}
        R - P & R- P \\
        S - P & T- P \\
        T - P & S- P \\
    \end{bmatrix}
\end{equation}

Note that in general, equation (\ref{eqn:linear_algebraic_equation_for_p}) will
not necessarily have a solution. From the Rouch\'{e}-Capelli theorem if there is
a solution it is unique as \(\text{rank}(C)=2\) which is the dimension of the
variable \(x\). The best fitting \(x\) is found by minimizing:

\begin{equation}\label{eqn:r_squared}
    \text{SSError} = \|C x- p^*\|_2^2 = \sum_{i=1}^{3}\left((C\bar x)_i-p_i^*\right)^2
\end{equation}

Note that \(\text{SSError}\), which is the square of the Frobenius
norm~\cite{Golub2013}, becomes a measure of how close a strategy is to being an
extortionate strategy. Suspicion
of extortion then corresponds to a threshold on \(\text{SSError}\).

By observing interactions (human or otherwise), their memory one representation
can be inferred and this approach can be used to recognise extortionate
behaviour. The notion of comparing theoretic and actual plays of the IPD is not
novel, see for example~\cite{Rand2013}. Immediately it is noted that if the
environment is noisy~\cite{Wu1995} then no strategy can be considered to be
extortionate as \(p_4>0\).

In the next section, this idea will be illustrated by observing the interactions
that take place in a computer based tournament of the IPD\@.

\section{Numerical experiments}\label{sec:numerical-experiments}

In~\cite{Stewart2012} results from a tournament with
\input{./assets/tex/number_of_stewart_plotkin_strategies/main.tex} strategies,
was presented with specific consideration given to ZD strategies. This
tournament is reproduced here using the Axelrod-Python
project~\cite{Knight2016}. To obtain a good measure of the corresponding
transition rates for each strategy all matches have been run for
\input{assets/tex/number_of_turns/main.tex} turns and every match has been
repeated \input{assets/tex/number_of_repetitions/main.tex} times. All of this
interaction data is available at~\cite{vincent_knight_2018_1297075}. A good
match between the inferred Markov chain and the state distribution of the actual
interactions has been verified. Data for this is presented in the supplementary
materials.

Figure~\ref{fig:SSError_overall_in_stewart_plotkin} shows the \(\text{SSError}\)
values for all the strategies in the tournament, as reported
in~\cite{Stewart2012} the extortionate strategy (which has an expected
\(\text{SSError}\) approximately 0) gains a large number of wins.

\begin{figure}[!htbp]
    \centering
    \includegraphics[width=.8\textwidth]{./assets/img/SSError_overall_in_stewart_plotkin/main.pdf}
    \caption{\(\text{SSError}\) and state probabilities for the strategies
        of~\cite{Stewart2012}, ordered both by number of wins and overall score.
        Note that \(P(DC)\) is not shown as it corresponds to the transpose of
        \(P(CD)\). Cooperator and Defector are omitted as they do not visit all
        the states.}
    \label{fig:SSError_overall_in_stewart_plotkin}
\end{figure}

Here, the work of~\cite{Stewart2012} is extended by investigating a tournament
with \input{assets/tex/number_of_full_strategies/main.tex}
strategies.

The results of this analysis are shown in
Figure~\ref{fig:SSError_and_probabilities_in_full}. The top ranking strategies
by number of wins seem to be extortionate (but not against all strategies) and
it can be seen that a small sub group of strategies achieve mutual defection.
All the top ranking strategies according to score achieve mutual cooperation and
do not extort each other, however they
\textbf{do} exhibit extortionate behaviour towards a number of the lower ranking
strategies.

\begin{figure}[!htbp]
    \centering
    \includegraphics[width=.8\textwidth]{./assets/img/SSError_and_probabilities_in_full/main.pdf}
    \caption{\(\text{SSError}\) for the strategies for the full tournament. Only
    strategy interactions for which \(p_4=0\) and \(\chi>1\) are displayed.}
    \label{fig:SSError_and_probabilities_in_full}
\end{figure}

\section{Conclusion}\label{sec:conclusion}

This work defines an approach to measure whether or not a player is playing a
strategy that corresponds to an extortionate strategy as defined
in~\cite{Press2012}: a mathematical model for suspicion. Indeed, all
extortionate strategies have been
 classified as lying on a triangular plane.
This rigorous classification fails to be robust to small measurement error, thus
a statistical approach is proposed.
This is done through a linear algebraic approach for approximating the solution
of a linear system. Using this, a large number of pairwise interactions is
simulated and in fact very few strategies are found to act extortionately.

The work of~\cite{Press2012}, whilst showing that a clever approach to taking
advantage of another memory one strategy exists: this is incomplete. Whilst the
elegance of this result is very attractive, just as the simplicity of the
victory of Tit For Tat in Axelrod's original tournaments was, it is incomplete.
Extortionate strategies achieve a high number of wins but they do not
achieve a high score which corresponds to the fitness landscape in an
evolutionary sense. From the large number of interactions a payoff matrix \(S\)
can be measured where \(S_{ij}\) denotes the score (using standard values of
\((R, S, T, P) = (3, 0, 5, 1)\)) of the \(i\)th strategy
against the \(j\)th strategy. Using this, the replicator equation
describes the evolution of the system based on a population density fitness
function:

\begin{equation}\label{eqn:replicator_dynamics}
    \frac{dx}{dt} = x(S-x^TS x)
\end{equation}

Equation (\ref{eqn:replicator_dynamics}) is solved numerically through an
integration technique described in~\cite{Petzold1983} and
Figure~\ref{fig:replicator_dynamics} shows the evolution of the distribution of
the system: the various strategies are ranked by scores. It is clear to see that
only the high ranking strategies survive the evolutionary process (in fact,
only \input{./assets/img/replicator_dynamics/main.tex}
have a final distribution greater than \(10 ^ {-2}\)). This confirms the
findings of~\cite{Moran1707} in which sophisticated strategies resist
evolutionary invasion of shorter memory strategies. Recalling
Figure~\ref{fig:SSError_and_probabilities_in_full} this demonstrates that:

\begin{itemize}
    \item Cooperation emerges through the evolutionary process: the high scoring
        strategies do not exhibit extortionate behaviour towards each other.
    \item Extortionate strategies do not survive the evolutionary process.
\end{itemize}

\begin{figure}[!htbp]
    \centering
    \includegraphics[width=.8\textwidth]{./assets/img/replicator_dynamics/main.pdf}
    \caption{Numerical simulation of the replicator equation
    (\ref{eqn:replicator_dynamics}): strategies are ordered by score, only the strategies with a high score survive the evolutionary process.}
    \label{fig:replicator_dynamics}
\end{figure}

This work can be used to classify plays of the IPD\@: data can be collected from
actual interactions (in lab or in the field). Furthermore, this allows for a
classification method similar to the notion of fingerprinting presented
in~\cite{Ashlock2008}. Trained strategies can potentially be classified as
extortionate or not or it could be possible to even constrain the reinforcement
learning approaches that are becoming prevalent in the literature.
Alternatively, this mathematical approach for recognising extortion could be
used in sophisticated strategies to defend against invasion. Arguably, some of
the strategies considered here exhibit this behaviour, indeed as described
in~\cite{Harper2017}, the top ranking strategies in the full tournament are
obtained using evolutionary reinforcement learning techniques, thus, suspicion
of extortionate behaviour could in fact be an evolutionary trait.

\section*{Acknowledgements}

The following open source software libraries were used in this research:

\begin{itemize}
    \item The Axelrod ~\cite{Knight2016, Knight2018} library (IPD strategies and
        tournaments).
    \item The sympy library~\cite{Meurer2017} (verification of all symbolic
        calculations).
    \item The matplotlib~\cite{Droettboom2018} library (visualisation).
    \item The pandas~\cite{Structures2010}, dask~\cite{Dask2016} and
        NumPy~\cite{Oliphant2015} libraries (data manipulation).
    \item The SciPy~\cite{Jones2001} library (numerical integration of the
        replicator equation).
\end{itemize}

This work was performed using the computational facilities of the Advanced
Research Computing @ Cardiff (ARCCA) Division, Cardiff University.

\printbibliography

\newpage
\section*{Supplementary materials}

\includepdf{assets/pdf/proof_of_form_of_extortionate_strategies/main.pdf}

\newpage

Using the pair wise interactions the transition rates \(p,
q\) can be measured and the steady state probabilities inferred and compared to
the actual probabilities of each state.
This is done numerically by computing the singular eigenvector of the
matrix \(A\) \cite{Stewart2009}:

\[
    A =
    \begin{bmatrix}
        p_1 q_1 & p_1 (1 - q_1) & (1 - p_1) q_1 & (1 -p_1) (1 - q_1) \\
        p_2 q_2 & p_2 (1 - q_2) & (1 - p_2) q_2 & (1 -p_2) (1 - q_2) \\
        p_3 q_3 & p_3 (1 - q_3) & (1 - p_3) q_3 & (1 -p_3) (1 - q_3) \\
        p_4 q_4 & p_4 (1 - q_4) & (1 - p_4) q_4 & (1 -p_4) (1 - q_4) \\
    \end{bmatrix}
\]

Figure~\ref{fig:computed_probabilities_vs_theoretic_probabilities} shows a
regression line fitted to every pairwise interaction with a reported
\(\text{SSError}\) value (pairwise interactions with missing states were
omitted). This serves to validate the approach: a part from some edge cases the
relationship is consistent.

\begin{figure}[!htbp]
    \centering
    \includegraphics[width=.8\textwidth]{./assets/img/computed_probabilities_vs_theoretic_probabilities/main.pdf}
    \caption{The
        relationship between the steady state probabilities inferred from the
        measured transitions and the actual steady state probabilities. A linear
        regression line is included validating the approach.}
    \label{fig:computed_probabilities_vs_theoretic_probabilities}
\end{figure}


\end{document}
 strategies,
was presented with specific consideration given to ZD strategies. This
tournament is reproduced here using the Axelrod-Python
project~\cite{Knight2016}. To obtain a good measure of the corresponding
transition rates for each strategy all matches have been run for
\documentclass[a4paper]{article}

\usepackage{amsmath}
\usepackage{amssymb}
\usepackage[margin=1.5cm,
            includefoot,
            footskip=30pt]{geometry}
\usepackage{layout}
\usepackage{graphicx}
\usepackage{subcaption}

\usepackage{biblatex}
\usepackage{pdfpages}

\bibliography{main.bib}

\title{Suspicion: Recognising and evaluating the effectiveness
       of extortion in the Iterated Prisoner's Dilemma}
\author{Vincent A. Knight \and Nikoleta E. Glynatsi}
\date{\today}



\begin{document}

\maketitle

\begin{abstract}
    The Iterated Prisoner's Dilemma is a model for rational and evolutionary
    interactive behaviour. It has applications both in the study of human social
    behaviour as well as in biology.
    It is used to understand when and how a rational individual might
    accept an immediate cost to their own utility for the direct benefit of
    another.

    Much attention has been given to a class of strategies called
    Zero Determinant strategies. It has been theoretically shown that these
    strategies can ``extort'' any player.

    In this work, an approach to identify if observed strategies are playing in
    an extortionate way is described. Furthermore, experimental analysis of
    a large tournament with \input{assets/tex/number_of_full_strategies/main.tex}
    strategies is considered. In this setting
    the most highly performing strategies do not play in an extortionate way
    against each other but do against lower performing strategies.
    This suggests that whilst the theory of Zero Determinant strategies
    indicates that memory is not of fundamental importance to the evolution of
    cooperative behaviour, this is incomplete.
\end{abstract}

\section{Introduction}\label{sec:introduction}

Agent based game theoretic models have become a stalwart of the underpinning
mathematics of interactive behaviours. One of the major pieces of work
in this area is the pair of original computer tournaments run by Robert
Axelrod~\cite{Axelrod1980, Axelrod1980a}. These tournaments pitted submitted
computer strategies against each other in plays of the Iterated Prisoner's
Dilemma. A common game where agents can choose to pay a slight cost to their
immediate utility in the hope of building a reputation. This has been used in
economic and evolutionary game theory to understand the evolution of cooperative
behaviour.

Recently, a class of strategies was described in~\cite{Press2012} that can
provably extort any given opponent. In~\cite{Hilbe2013, Moran1707} some
questions have already been asked about the true effectiveness of these
strategies in an evolutionary setting. Here another question is asked: is it
possible to recognise this extortionate behaviour? A mathematical procedure for
suspicion is presented: in the same way that the continued actions of an
extortionate individual might raise suspicion.

This work makes use of the Axelrod Python library~\cite{Knight2018, Knight2016}
with a large number of Prisoner Dilemma strategies available to give an
extensive numerical example of the ideas presented.  The approach is presented
in Section~\ref{sec:delta-zd-strategies}.  All of the code and data discussed
in Section~\ref{sec:numerical-experiments} is open sourced, archived and
written according to best scientific principles~\cite{Wilson2014}. The data
archive can be found at~\cite{vincent_knight_2018_1297075}.

\section{Recognising Extortion}\label{sec:delta-zd-strategies}

In~\cite{Press2012}, given a match between 2 memory-one strategies, the concept
of Zero Determinant (ZD) strategies is introduced. The main result of that paper
shows that given two memory one players \(p, q\in\mathbb{R}^4\) a linear
relationship between the players' scores could be forced by one of the players.

Using the notation of~\cite{Press2012}, assuming the utilities for player \(p\)
are given by \(S_x=(R, S, T, P)\) and for player \(q\) by \(S_y=(R, T, S, P)\)
and that the stationary scores of each player is given by \(S_X\) and \(S_Y\)
respectively. The main result of~\cite{Press2012} is that if

\begin{equation}\label{eqn:linear_relationship_for_p}
    \tilde p=\alpha S_x + \beta S_y + \gamma
\end{equation}

or

\begin{equation}\label{eqn:linear_relationship_for_q}
    \tilde q=\alpha S_x + \beta S_y + \gamma
\end{equation}

where \(\tilde p = (1 - p_1, 1 - p_2, p_3, p_4)\) and
\(\tilde q = (1 - q_1, 1 - q_2, q_3, q_4)\) then:

\begin{equation}
    \alpha S_X + \beta S_Y + \gamma = 0
\end{equation}

In~\cite{Press2012} a particular type of ZD strategy is defined: extortionate
strategies. If:

\begin{equation}\label{eqn:constraint_for_extortion}
    \gamma = - P(\alpha + \beta)
\end{equation}

then the player can ensure they get a score \(\chi\) times
larger than the opponent. This extortion coefficient is given by:

\begin{equation}\label{eqn:definition_of_chi}
    \chi=\frac{-\beta}{\alpha}
\end{equation}

Thus, if (\ref{eqn:constraint_for_extortion}) holds and \(\chi >1\) a player is
said to extort their opponent.
Here, the reverse problem is considered: given a
\(p\in\mathbb{R}^4\) how does one identify \(\alpha, \beta\) if they
exist and is the strategy in fact acting in an extortionate way?

These conditions correspond to:

\begin{align}
    \tilde p_1 & = \alpha R + \beta R - P (\alpha + \beta)
            \label{eqn:condition_for_tilde_p1}\\
    \tilde p_2 & = \alpha S + \beta T - P (\alpha + \beta)
            \label{eqn:condition_for_tilde_p2}\\
    \tilde p_3 & = \alpha T + \beta S - P (\alpha + \beta)
            \label{eqn:condition_for_tilde_p3}\\
    \tilde p_4 & = \alpha P + \beta P - P (\alpha + \beta)
            \label{eqn:condition_for_tilde_p4}
\end{align}

Equation (\ref{eqn:condition_for_tilde_p4}) ensures that \(p_4=\tilde p_4=0\).
Equations (\ref{eqn:condition_for_tilde_p1}-\ref{eqn:condition_for_tilde_p3})
can be used to eliminate \(\alpha, \beta\), giving:

\begin{equation}\label{eqn:planar_definition_of_extortion}
    \tilde p_1 = \frac{(R - P)(\tilde p_2 + \tilde p_3)}{S + T - 2P}
\end{equation}

with:

\begin{equation}\label{eqn:definition_of_chi}
    \chi = \frac{\tilde p_2 (P - T) + \tilde p_3 (S - P)}
                {\tilde p_2 (P - S) + \tilde p_3 (T - P)}
\end{equation}

Given a strategy \(p\in\mathbb{R}^{4\times 1}\) equations
(\ref{eqn:condition_for_tilde_p4}), (\ref{eqn:planar_definition_of_extortion}-\ref{eqn:definition_of_chi}) can be used to check if
a strategy is extortionate. The conditions correspond to:

\begin{align}
    p_1 & = \frac{(R-P)(p_2 + p_3) - R + T + S - P}{S + T - 2P}
     \label{eqn:condition_for_p1}\\
    p_4 & = 0 \label{eqn:condition_for_p4}\\
    1 & > p_2 + p_3\label{eqn:condition_for_chi}
\end{align}

The algebraic steps necessary to prove these results are available in the
supporting materials.

All extortionate strategies reside on a triangular (\ref{eqn:condition_for_chi})
plane (\ref{eqn:condition_for_p1}) in 3 dimensions (\ref{eqn:condition_for_p4}).
Using this formulation it can be seen that a necessary (but not sufficient)
condition for an extortionate strategy is that it cooperates on average less
than 50\% of the time when in a state of disagreement with the opponent.

As an example, consider the known extortionate strategy \(p=(8 / 9, 1 / 2, 1 /
3, 0)\) from~\cite{Stewart2012} which is referred to as \texttt{Extort-2}. In
this case, for the standard values of \((R, T, S, P)\) constraint
(\ref{eqn:condition_for_p1}) corresponds to:

\begin{equation}
    p_1 = \frac{2(p_2 + p_3) + 1}{3}
\end{equation}

It is clear that in this case all constraints hold.

This approach could in fact be used to confirm that a given strategy is acting
in an extortionate manner even if it is not a memory one strategy. However, in
practice, if a closed form for \(p\) is not known, then due to measurement
and/or numerical error this would not work.

This problem can be written in the following linear algebraic form where
\(x=(\alpha, \beta)\)
and \(p^*=(\tilde p_1 - 1, tilde_2 - 1, p_3)\):

\begin{equation}\label{eqn:linear_algebraic_equation_for_p}
    Cx= p^*
\end{equation}

\(C\) corresponds to equations
(\ref{eqn:condition_for_tilde_p1}-\ref{eqn:condition_for_tilde_p3}) and is
given by:

\begin{equation}\label{eqn:definition_of_C}
    C =
    \begin{bmatrix}
        R - P & R- P \\
        S - P & T- P \\
        T - P & S- P \\
    \end{bmatrix}
\end{equation}

Note that in general, equation (\ref{eqn:linear_algebraic_equation_for_p}) will
not necessarily have a solution. From the Rouch\'{e}-Capelli theorem if there is
a solution it is unique as \(\text{rank}(C)=2\) which is the dimension of the
variable \(x\). The best fitting \(x\) is found by minimizing:

\begin{equation}\label{eqn:r_squared}
    \text{SSError} = \|C x- p^*\|_2^2 = \sum_{i=1}^{3}\left((C\bar x)_i-p_i^*\right)^2
\end{equation}

Note that \(\text{SSError}\), which is the square of the Frobenius
norm~\cite{Golub2013}, becomes a measure of how close a strategy is to being an
extortionate strategy. Suspicion
of extortion then corresponds to a threshold on \(\text{SSError}\).

By observing interactions (human or otherwise), their memory one representation
can be inferred and this approach can be used to recognise extortionate
behaviour. The notion of comparing theoretic and actual plays of the IPD is not
novel, see for example~\cite{Rand2013}. Immediately it is noted that if the
environment is noisy~\cite{Wu1995} then no strategy can be considered to be
extortionate as \(p_4>0\).

In the next section, this idea will be illustrated by observing the interactions
that take place in a computer based tournament of the IPD\@.

\section{Numerical experiments}\label{sec:numerical-experiments}

In~\cite{Stewart2012} results from a tournament with
\input{./assets/tex/number_of_stewart_plotkin_strategies/main.tex} strategies,
was presented with specific consideration given to ZD strategies. This
tournament is reproduced here using the Axelrod-Python
project~\cite{Knight2016}. To obtain a good measure of the corresponding
transition rates for each strategy all matches have been run for
\input{assets/tex/number_of_turns/main.tex} turns and every match has been
repeated \input{assets/tex/number_of_repetitions/main.tex} times. All of this
interaction data is available at~\cite{vincent_knight_2018_1297075}. A good
match between the inferred Markov chain and the state distribution of the actual
interactions has been verified. Data for this is presented in the supplementary
materials.

Figure~\ref{fig:SSError_overall_in_stewart_plotkin} shows the \(\text{SSError}\)
values for all the strategies in the tournament, as reported
in~\cite{Stewart2012} the extortionate strategy (which has an expected
\(\text{SSError}\) approximately 0) gains a large number of wins.

\begin{figure}[!htbp]
    \centering
    \includegraphics[width=.8\textwidth]{./assets/img/SSError_overall_in_stewart_plotkin/main.pdf}
    \caption{\(\text{SSError}\) and state probabilities for the strategies
        of~\cite{Stewart2012}, ordered both by number of wins and overall score.
        Note that \(P(DC)\) is not shown as it corresponds to the transpose of
        \(P(CD)\). Cooperator and Defector are omitted as they do not visit all
        the states.}
    \label{fig:SSError_overall_in_stewart_plotkin}
\end{figure}

Here, the work of~\cite{Stewart2012} is extended by investigating a tournament
with \input{assets/tex/number_of_full_strategies/main.tex}
strategies.

The results of this analysis are shown in
Figure~\ref{fig:SSError_and_probabilities_in_full}. The top ranking strategies
by number of wins seem to be extortionate (but not against all strategies) and
it can be seen that a small sub group of strategies achieve mutual defection.
All the top ranking strategies according to score achieve mutual cooperation and
do not extort each other, however they
\textbf{do} exhibit extortionate behaviour towards a number of the lower ranking
strategies.

\begin{figure}[!htbp]
    \centering
    \includegraphics[width=.8\textwidth]{./assets/img/SSError_and_probabilities_in_full/main.pdf}
    \caption{\(\text{SSError}\) for the strategies for the full tournament. Only
    strategy interactions for which \(p_4=0\) and \(\chi>1\) are displayed.}
    \label{fig:SSError_and_probabilities_in_full}
\end{figure}

\section{Conclusion}\label{sec:conclusion}

This work defines an approach to measure whether or not a player is playing a
strategy that corresponds to an extortionate strategy as defined
in~\cite{Press2012}: a mathematical model for suspicion. Indeed, all
extortionate strategies have been
 classified as lying on a triangular plane.
This rigorous classification fails to be robust to small measurement error, thus
a statistical approach is proposed.
This is done through a linear algebraic approach for approximating the solution
of a linear system. Using this, a large number of pairwise interactions is
simulated and in fact very few strategies are found to act extortionately.

The work of~\cite{Press2012}, whilst showing that a clever approach to taking
advantage of another memory one strategy exists: this is incomplete. Whilst the
elegance of this result is very attractive, just as the simplicity of the
victory of Tit For Tat in Axelrod's original tournaments was, it is incomplete.
Extortionate strategies achieve a high number of wins but they do not
achieve a high score which corresponds to the fitness landscape in an
evolutionary sense. From the large number of interactions a payoff matrix \(S\)
can be measured where \(S_{ij}\) denotes the score (using standard values of
\((R, S, T, P) = (3, 0, 5, 1)\)) of the \(i\)th strategy
against the \(j\)th strategy. Using this, the replicator equation
describes the evolution of the system based on a population density fitness
function:

\begin{equation}\label{eqn:replicator_dynamics}
    \frac{dx}{dt} = x(S-x^TS x)
\end{equation}

Equation (\ref{eqn:replicator_dynamics}) is solved numerically through an
integration technique described in~\cite{Petzold1983} and
Figure~\ref{fig:replicator_dynamics} shows the evolution of the distribution of
the system: the various strategies are ranked by scores. It is clear to see that
only the high ranking strategies survive the evolutionary process (in fact,
only \input{./assets/img/replicator_dynamics/main.tex}
have a final distribution greater than \(10 ^ {-2}\)). This confirms the
findings of~\cite{Moran1707} in which sophisticated strategies resist
evolutionary invasion of shorter memory strategies. Recalling
Figure~\ref{fig:SSError_and_probabilities_in_full} this demonstrates that:

\begin{itemize}
    \item Cooperation emerges through the evolutionary process: the high scoring
        strategies do not exhibit extortionate behaviour towards each other.
    \item Extortionate strategies do not survive the evolutionary process.
\end{itemize}

\begin{figure}[!htbp]
    \centering
    \includegraphics[width=.8\textwidth]{./assets/img/replicator_dynamics/main.pdf}
    \caption{Numerical simulation of the replicator equation
    (\ref{eqn:replicator_dynamics}): strategies are ordered by score, only the strategies with a high score survive the evolutionary process.}
    \label{fig:replicator_dynamics}
\end{figure}

This work can be used to classify plays of the IPD\@: data can be collected from
actual interactions (in lab or in the field). Furthermore, this allows for a
classification method similar to the notion of fingerprinting presented
in~\cite{Ashlock2008}. Trained strategies can potentially be classified as
extortionate or not or it could be possible to even constrain the reinforcement
learning approaches that are becoming prevalent in the literature.
Alternatively, this mathematical approach for recognising extortion could be
used in sophisticated strategies to defend against invasion. Arguably, some of
the strategies considered here exhibit this behaviour, indeed as described
in~\cite{Harper2017}, the top ranking strategies in the full tournament are
obtained using evolutionary reinforcement learning techniques, thus, suspicion
of extortionate behaviour could in fact be an evolutionary trait.

\section*{Acknowledgements}

The following open source software libraries were used in this research:

\begin{itemize}
    \item The Axelrod ~\cite{Knight2016, Knight2018} library (IPD strategies and
        tournaments).
    \item The sympy library~\cite{Meurer2017} (verification of all symbolic
        calculations).
    \item The matplotlib~\cite{Droettboom2018} library (visualisation).
    \item The pandas~\cite{Structures2010}, dask~\cite{Dask2016} and
        NumPy~\cite{Oliphant2015} libraries (data manipulation).
    \item The SciPy~\cite{Jones2001} library (numerical integration of the
        replicator equation).
\end{itemize}

This work was performed using the computational facilities of the Advanced
Research Computing @ Cardiff (ARCCA) Division, Cardiff University.

\printbibliography

\newpage
\section*{Supplementary materials}

\includepdf{assets/pdf/proof_of_form_of_extortionate_strategies/main.pdf}

\newpage

Using the pair wise interactions the transition rates \(p,
q\) can be measured and the steady state probabilities inferred and compared to
the actual probabilities of each state.
This is done numerically by computing the singular eigenvector of the
matrix \(A\) \cite{Stewart2009}:

\[
    A =
    \begin{bmatrix}
        p_1 q_1 & p_1 (1 - q_1) & (1 - p_1) q_1 & (1 -p_1) (1 - q_1) \\
        p_2 q_2 & p_2 (1 - q_2) & (1 - p_2) q_2 & (1 -p_2) (1 - q_2) \\
        p_3 q_3 & p_3 (1 - q_3) & (1 - p_3) q_3 & (1 -p_3) (1 - q_3) \\
        p_4 q_4 & p_4 (1 - q_4) & (1 - p_4) q_4 & (1 -p_4) (1 - q_4) \\
    \end{bmatrix}
\]

Figure~\ref{fig:computed_probabilities_vs_theoretic_probabilities} shows a
regression line fitted to every pairwise interaction with a reported
\(\text{SSError}\) value (pairwise interactions with missing states were
omitted). This serves to validate the approach: a part from some edge cases the
relationship is consistent.

\begin{figure}[!htbp]
    \centering
    \includegraphics[width=.8\textwidth]{./assets/img/computed_probabilities_vs_theoretic_probabilities/main.pdf}
    \caption{The
        relationship between the steady state probabilities inferred from the
        measured transitions and the actual steady state probabilities. A linear
        regression line is included validating the approach.}
    \label{fig:computed_probabilities_vs_theoretic_probabilities}
\end{figure}


\end{document}
 turns and every match has been
repeated \documentclass[a4paper]{article}

\usepackage{amsmath}
\usepackage{amssymb}
\usepackage[margin=1.5cm,
            includefoot,
            footskip=30pt]{geometry}
\usepackage{layout}
\usepackage{graphicx}
\usepackage{subcaption}

\usepackage{biblatex}
\usepackage{pdfpages}

\bibliography{main.bib}

\title{Suspicion: Recognising and evaluating the effectiveness
       of extortion in the Iterated Prisoner's Dilemma}
\author{Vincent A. Knight \and Nikoleta E. Glynatsi}
\date{\today}



\begin{document}

\maketitle

\begin{abstract}
    The Iterated Prisoner's Dilemma is a model for rational and evolutionary
    interactive behaviour. It has applications both in the study of human social
    behaviour as well as in biology.
    It is used to understand when and how a rational individual might
    accept an immediate cost to their own utility for the direct benefit of
    another.

    Much attention has been given to a class of strategies called
    Zero Determinant strategies. It has been theoretically shown that these
    strategies can ``extort'' any player.

    In this work, an approach to identify if observed strategies are playing in
    an extortionate way is described. Furthermore, experimental analysis of
    a large tournament with \input{assets/tex/number_of_full_strategies/main.tex}
    strategies is considered. In this setting
    the most highly performing strategies do not play in an extortionate way
    against each other but do against lower performing strategies.
    This suggests that whilst the theory of Zero Determinant strategies
    indicates that memory is not of fundamental importance to the evolution of
    cooperative behaviour, this is incomplete.
\end{abstract}

\section{Introduction}\label{sec:introduction}

Agent based game theoretic models have become a stalwart of the underpinning
mathematics of interactive behaviours. One of the major pieces of work
in this area is the pair of original computer tournaments run by Robert
Axelrod~\cite{Axelrod1980, Axelrod1980a}. These tournaments pitted submitted
computer strategies against each other in plays of the Iterated Prisoner's
Dilemma. A common game where agents can choose to pay a slight cost to their
immediate utility in the hope of building a reputation. This has been used in
economic and evolutionary game theory to understand the evolution of cooperative
behaviour.

Recently, a class of strategies was described in~\cite{Press2012} that can
provably extort any given opponent. In~\cite{Hilbe2013, Moran1707} some
questions have already been asked about the true effectiveness of these
strategies in an evolutionary setting. Here another question is asked: is it
possible to recognise this extortionate behaviour? A mathematical procedure for
suspicion is presented: in the same way that the continued actions of an
extortionate individual might raise suspicion.

This work makes use of the Axelrod Python library~\cite{Knight2018, Knight2016}
with a large number of Prisoner Dilemma strategies available to give an
extensive numerical example of the ideas presented.  The approach is presented
in Section~\ref{sec:delta-zd-strategies}.  All of the code and data discussed
in Section~\ref{sec:numerical-experiments} is open sourced, archived and
written according to best scientific principles~\cite{Wilson2014}. The data
archive can be found at~\cite{vincent_knight_2018_1297075}.

\section{Recognising Extortion}\label{sec:delta-zd-strategies}

In~\cite{Press2012}, given a match between 2 memory-one strategies, the concept
of Zero Determinant (ZD) strategies is introduced. The main result of that paper
shows that given two memory one players \(p, q\in\mathbb{R}^4\) a linear
relationship between the players' scores could be forced by one of the players.

Using the notation of~\cite{Press2012}, assuming the utilities for player \(p\)
are given by \(S_x=(R, S, T, P)\) and for player \(q\) by \(S_y=(R, T, S, P)\)
and that the stationary scores of each player is given by \(S_X\) and \(S_Y\)
respectively. The main result of~\cite{Press2012} is that if

\begin{equation}\label{eqn:linear_relationship_for_p}
    \tilde p=\alpha S_x + \beta S_y + \gamma
\end{equation}

or

\begin{equation}\label{eqn:linear_relationship_for_q}
    \tilde q=\alpha S_x + \beta S_y + \gamma
\end{equation}

where \(\tilde p = (1 - p_1, 1 - p_2, p_3, p_4)\) and
\(\tilde q = (1 - q_1, 1 - q_2, q_3, q_4)\) then:

\begin{equation}
    \alpha S_X + \beta S_Y + \gamma = 0
\end{equation}

In~\cite{Press2012} a particular type of ZD strategy is defined: extortionate
strategies. If:

\begin{equation}\label{eqn:constraint_for_extortion}
    \gamma = - P(\alpha + \beta)
\end{equation}

then the player can ensure they get a score \(\chi\) times
larger than the opponent. This extortion coefficient is given by:

\begin{equation}\label{eqn:definition_of_chi}
    \chi=\frac{-\beta}{\alpha}
\end{equation}

Thus, if (\ref{eqn:constraint_for_extortion}) holds and \(\chi >1\) a player is
said to extort their opponent.
Here, the reverse problem is considered: given a
\(p\in\mathbb{R}^4\) how does one identify \(\alpha, \beta\) if they
exist and is the strategy in fact acting in an extortionate way?

These conditions correspond to:

\begin{align}
    \tilde p_1 & = \alpha R + \beta R - P (\alpha + \beta)
            \label{eqn:condition_for_tilde_p1}\\
    \tilde p_2 & = \alpha S + \beta T - P (\alpha + \beta)
            \label{eqn:condition_for_tilde_p2}\\
    \tilde p_3 & = \alpha T + \beta S - P (\alpha + \beta)
            \label{eqn:condition_for_tilde_p3}\\
    \tilde p_4 & = \alpha P + \beta P - P (\alpha + \beta)
            \label{eqn:condition_for_tilde_p4}
\end{align}

Equation (\ref{eqn:condition_for_tilde_p4}) ensures that \(p_4=\tilde p_4=0\).
Equations (\ref{eqn:condition_for_tilde_p1}-\ref{eqn:condition_for_tilde_p3})
can be used to eliminate \(\alpha, \beta\), giving:

\begin{equation}\label{eqn:planar_definition_of_extortion}
    \tilde p_1 = \frac{(R - P)(\tilde p_2 + \tilde p_3)}{S + T - 2P}
\end{equation}

with:

\begin{equation}\label{eqn:definition_of_chi}
    \chi = \frac{\tilde p_2 (P - T) + \tilde p_3 (S - P)}
                {\tilde p_2 (P - S) + \tilde p_3 (T - P)}
\end{equation}

Given a strategy \(p\in\mathbb{R}^{4\times 1}\) equations
(\ref{eqn:condition_for_tilde_p4}), (\ref{eqn:planar_definition_of_extortion}-\ref{eqn:definition_of_chi}) can be used to check if
a strategy is extortionate. The conditions correspond to:

\begin{align}
    p_1 & = \frac{(R-P)(p_2 + p_3) - R + T + S - P}{S + T - 2P}
     \label{eqn:condition_for_p1}\\
    p_4 & = 0 \label{eqn:condition_for_p4}\\
    1 & > p_2 + p_3\label{eqn:condition_for_chi}
\end{align}

The algebraic steps necessary to prove these results are available in the
supporting materials.

All extortionate strategies reside on a triangular (\ref{eqn:condition_for_chi})
plane (\ref{eqn:condition_for_p1}) in 3 dimensions (\ref{eqn:condition_for_p4}).
Using this formulation it can be seen that a necessary (but not sufficient)
condition for an extortionate strategy is that it cooperates on average less
than 50\% of the time when in a state of disagreement with the opponent.

As an example, consider the known extortionate strategy \(p=(8 / 9, 1 / 2, 1 /
3, 0)\) from~\cite{Stewart2012} which is referred to as \texttt{Extort-2}. In
this case, for the standard values of \((R, T, S, P)\) constraint
(\ref{eqn:condition_for_p1}) corresponds to:

\begin{equation}
    p_1 = \frac{2(p_2 + p_3) + 1}{3}
\end{equation}

It is clear that in this case all constraints hold.

This approach could in fact be used to confirm that a given strategy is acting
in an extortionate manner even if it is not a memory one strategy. However, in
practice, if a closed form for \(p\) is not known, then due to measurement
and/or numerical error this would not work.

This problem can be written in the following linear algebraic form where
\(x=(\alpha, \beta)\)
and \(p^*=(\tilde p_1 - 1, tilde_2 - 1, p_3)\):

\begin{equation}\label{eqn:linear_algebraic_equation_for_p}
    Cx= p^*
\end{equation}

\(C\) corresponds to equations
(\ref{eqn:condition_for_tilde_p1}-\ref{eqn:condition_for_tilde_p3}) and is
given by:

\begin{equation}\label{eqn:definition_of_C}
    C =
    \begin{bmatrix}
        R - P & R- P \\
        S - P & T- P \\
        T - P & S- P \\
    \end{bmatrix}
\end{equation}

Note that in general, equation (\ref{eqn:linear_algebraic_equation_for_p}) will
not necessarily have a solution. From the Rouch\'{e}-Capelli theorem if there is
a solution it is unique as \(\text{rank}(C)=2\) which is the dimension of the
variable \(x\). The best fitting \(x\) is found by minimizing:

\begin{equation}\label{eqn:r_squared}
    \text{SSError} = \|C x- p^*\|_2^2 = \sum_{i=1}^{3}\left((C\bar x)_i-p_i^*\right)^2
\end{equation}

Note that \(\text{SSError}\), which is the square of the Frobenius
norm~\cite{Golub2013}, becomes a measure of how close a strategy is to being an
extortionate strategy. Suspicion
of extortion then corresponds to a threshold on \(\text{SSError}\).

By observing interactions (human or otherwise), their memory one representation
can be inferred and this approach can be used to recognise extortionate
behaviour. The notion of comparing theoretic and actual plays of the IPD is not
novel, see for example~\cite{Rand2013}. Immediately it is noted that if the
environment is noisy~\cite{Wu1995} then no strategy can be considered to be
extortionate as \(p_4>0\).

In the next section, this idea will be illustrated by observing the interactions
that take place in a computer based tournament of the IPD\@.

\section{Numerical experiments}\label{sec:numerical-experiments}

In~\cite{Stewart2012} results from a tournament with
\input{./assets/tex/number_of_stewart_plotkin_strategies/main.tex} strategies,
was presented with specific consideration given to ZD strategies. This
tournament is reproduced here using the Axelrod-Python
project~\cite{Knight2016}. To obtain a good measure of the corresponding
transition rates for each strategy all matches have been run for
\input{assets/tex/number_of_turns/main.tex} turns and every match has been
repeated \input{assets/tex/number_of_repetitions/main.tex} times. All of this
interaction data is available at~\cite{vincent_knight_2018_1297075}. A good
match between the inferred Markov chain and the state distribution of the actual
interactions has been verified. Data for this is presented in the supplementary
materials.

Figure~\ref{fig:SSError_overall_in_stewart_plotkin} shows the \(\text{SSError}\)
values for all the strategies in the tournament, as reported
in~\cite{Stewart2012} the extortionate strategy (which has an expected
\(\text{SSError}\) approximately 0) gains a large number of wins.

\begin{figure}[!htbp]
    \centering
    \includegraphics[width=.8\textwidth]{./assets/img/SSError_overall_in_stewart_plotkin/main.pdf}
    \caption{\(\text{SSError}\) and state probabilities for the strategies
        of~\cite{Stewart2012}, ordered both by number of wins and overall score.
        Note that \(P(DC)\) is not shown as it corresponds to the transpose of
        \(P(CD)\). Cooperator and Defector are omitted as they do not visit all
        the states.}
    \label{fig:SSError_overall_in_stewart_plotkin}
\end{figure}

Here, the work of~\cite{Stewart2012} is extended by investigating a tournament
with \input{assets/tex/number_of_full_strategies/main.tex}
strategies.

The results of this analysis are shown in
Figure~\ref{fig:SSError_and_probabilities_in_full}. The top ranking strategies
by number of wins seem to be extortionate (but not against all strategies) and
it can be seen that a small sub group of strategies achieve mutual defection.
All the top ranking strategies according to score achieve mutual cooperation and
do not extort each other, however they
\textbf{do} exhibit extortionate behaviour towards a number of the lower ranking
strategies.

\begin{figure}[!htbp]
    \centering
    \includegraphics[width=.8\textwidth]{./assets/img/SSError_and_probabilities_in_full/main.pdf}
    \caption{\(\text{SSError}\) for the strategies for the full tournament. Only
    strategy interactions for which \(p_4=0\) and \(\chi>1\) are displayed.}
    \label{fig:SSError_and_probabilities_in_full}
\end{figure}

\section{Conclusion}\label{sec:conclusion}

This work defines an approach to measure whether or not a player is playing a
strategy that corresponds to an extortionate strategy as defined
in~\cite{Press2012}: a mathematical model for suspicion. Indeed, all
extortionate strategies have been
 classified as lying on a triangular plane.
This rigorous classification fails to be robust to small measurement error, thus
a statistical approach is proposed.
This is done through a linear algebraic approach for approximating the solution
of a linear system. Using this, a large number of pairwise interactions is
simulated and in fact very few strategies are found to act extortionately.

The work of~\cite{Press2012}, whilst showing that a clever approach to taking
advantage of another memory one strategy exists: this is incomplete. Whilst the
elegance of this result is very attractive, just as the simplicity of the
victory of Tit For Tat in Axelrod's original tournaments was, it is incomplete.
Extortionate strategies achieve a high number of wins but they do not
achieve a high score which corresponds to the fitness landscape in an
evolutionary sense. From the large number of interactions a payoff matrix \(S\)
can be measured where \(S_{ij}\) denotes the score (using standard values of
\((R, S, T, P) = (3, 0, 5, 1)\)) of the \(i\)th strategy
against the \(j\)th strategy. Using this, the replicator equation
describes the evolution of the system based on a population density fitness
function:

\begin{equation}\label{eqn:replicator_dynamics}
    \frac{dx}{dt} = x(S-x^TS x)
\end{equation}

Equation (\ref{eqn:replicator_dynamics}) is solved numerically through an
integration technique described in~\cite{Petzold1983} and
Figure~\ref{fig:replicator_dynamics} shows the evolution of the distribution of
the system: the various strategies are ranked by scores. It is clear to see that
only the high ranking strategies survive the evolutionary process (in fact,
only \input{./assets/img/replicator_dynamics/main.tex}
have a final distribution greater than \(10 ^ {-2}\)). This confirms the
findings of~\cite{Moran1707} in which sophisticated strategies resist
evolutionary invasion of shorter memory strategies. Recalling
Figure~\ref{fig:SSError_and_probabilities_in_full} this demonstrates that:

\begin{itemize}
    \item Cooperation emerges through the evolutionary process: the high scoring
        strategies do not exhibit extortionate behaviour towards each other.
    \item Extortionate strategies do not survive the evolutionary process.
\end{itemize}

\begin{figure}[!htbp]
    \centering
    \includegraphics[width=.8\textwidth]{./assets/img/replicator_dynamics/main.pdf}
    \caption{Numerical simulation of the replicator equation
    (\ref{eqn:replicator_dynamics}): strategies are ordered by score, only the strategies with a high score survive the evolutionary process.}
    \label{fig:replicator_dynamics}
\end{figure}

This work can be used to classify plays of the IPD\@: data can be collected from
actual interactions (in lab or in the field). Furthermore, this allows for a
classification method similar to the notion of fingerprinting presented
in~\cite{Ashlock2008}. Trained strategies can potentially be classified as
extortionate or not or it could be possible to even constrain the reinforcement
learning approaches that are becoming prevalent in the literature.
Alternatively, this mathematical approach for recognising extortion could be
used in sophisticated strategies to defend against invasion. Arguably, some of
the strategies considered here exhibit this behaviour, indeed as described
in~\cite{Harper2017}, the top ranking strategies in the full tournament are
obtained using evolutionary reinforcement learning techniques, thus, suspicion
of extortionate behaviour could in fact be an evolutionary trait.

\section*{Acknowledgements}

The following open source software libraries were used in this research:

\begin{itemize}
    \item The Axelrod ~\cite{Knight2016, Knight2018} library (IPD strategies and
        tournaments).
    \item The sympy library~\cite{Meurer2017} (verification of all symbolic
        calculations).
    \item The matplotlib~\cite{Droettboom2018} library (visualisation).
    \item The pandas~\cite{Structures2010}, dask~\cite{Dask2016} and
        NumPy~\cite{Oliphant2015} libraries (data manipulation).
    \item The SciPy~\cite{Jones2001} library (numerical integration of the
        replicator equation).
\end{itemize}

This work was performed using the computational facilities of the Advanced
Research Computing @ Cardiff (ARCCA) Division, Cardiff University.

\printbibliography

\newpage
\section*{Supplementary materials}

\includepdf{assets/pdf/proof_of_form_of_extortionate_strategies/main.pdf}

\newpage

Using the pair wise interactions the transition rates \(p,
q\) can be measured and the steady state probabilities inferred and compared to
the actual probabilities of each state.
This is done numerically by computing the singular eigenvector of the
matrix \(A\) \cite{Stewart2009}:

\[
    A =
    \begin{bmatrix}
        p_1 q_1 & p_1 (1 - q_1) & (1 - p_1) q_1 & (1 -p_1) (1 - q_1) \\
        p_2 q_2 & p_2 (1 - q_2) & (1 - p_2) q_2 & (1 -p_2) (1 - q_2) \\
        p_3 q_3 & p_3 (1 - q_3) & (1 - p_3) q_3 & (1 -p_3) (1 - q_3) \\
        p_4 q_4 & p_4 (1 - q_4) & (1 - p_4) q_4 & (1 -p_4) (1 - q_4) \\
    \end{bmatrix}
\]

Figure~\ref{fig:computed_probabilities_vs_theoretic_probabilities} shows a
regression line fitted to every pairwise interaction with a reported
\(\text{SSError}\) value (pairwise interactions with missing states were
omitted). This serves to validate the approach: a part from some edge cases the
relationship is consistent.

\begin{figure}[!htbp]
    \centering
    \includegraphics[width=.8\textwidth]{./assets/img/computed_probabilities_vs_theoretic_probabilities/main.pdf}
    \caption{The
        relationship between the steady state probabilities inferred from the
        measured transitions and the actual steady state probabilities. A linear
        regression line is included validating the approach.}
    \label{fig:computed_probabilities_vs_theoretic_probabilities}
\end{figure}


\end{document}
 times. All of this
interaction data is available at~\cite{vincent_knight_2018_1297075}. A good
match between the inferred Markov chain and the state distribution of the actual
interactions has been verified. Data for this is presented in the supplementary
materials.

Figure~\ref{fig:SSError_overall_in_stewart_plotkin} shows the \(\text{SSError}\)
values for all the strategies in the tournament, as reported
in~\cite{Stewart2012} the extortionate strategy (which has an expected
\(\text{SSError}\) approximately 0) gains a large number of wins.

\begin{figure}[!htbp]
    \centering
    \includegraphics[width=.8\textwidth]{./assets/img/SSError_overall_in_stewart_plotkin/main.pdf}
    \caption{\(\text{SSError}\) and state probabilities for the strategies
        of~\cite{Stewart2012}, ordered both by number of wins and overall score.
        Note that \(P(DC)\) is not shown as it corresponds to the transpose of
        \(P(CD)\). Cooperator and Defector are omitted as they do not visit all
        the states.}
    \label{fig:SSError_overall_in_stewart_plotkin}
\end{figure}

Here, the work of~\cite{Stewart2012} is extended by investigating a tournament
with \documentclass[a4paper]{article}

\usepackage{amsmath}
\usepackage{amssymb}
\usepackage[margin=1.5cm,
            includefoot,
            footskip=30pt]{geometry}
\usepackage{layout}
\usepackage{graphicx}
\usepackage{subcaption}

\usepackage{biblatex}
\usepackage{pdfpages}

\bibliography{main.bib}

\title{Suspicion: Recognising and evaluating the effectiveness
       of extortion in the Iterated Prisoner's Dilemma}
\author{Vincent A. Knight \and Nikoleta E. Glynatsi}
\date{\today}



\begin{document}

\maketitle

\begin{abstract}
    The Iterated Prisoner's Dilemma is a model for rational and evolutionary
    interactive behaviour. It has applications both in the study of human social
    behaviour as well as in biology.
    It is used to understand when and how a rational individual might
    accept an immediate cost to their own utility for the direct benefit of
    another.

    Much attention has been given to a class of strategies called
    Zero Determinant strategies. It has been theoretically shown that these
    strategies can ``extort'' any player.

    In this work, an approach to identify if observed strategies are playing in
    an extortionate way is described. Furthermore, experimental analysis of
    a large tournament with \input{assets/tex/number_of_full_strategies/main.tex}
    strategies is considered. In this setting
    the most highly performing strategies do not play in an extortionate way
    against each other but do against lower performing strategies.
    This suggests that whilst the theory of Zero Determinant strategies
    indicates that memory is not of fundamental importance to the evolution of
    cooperative behaviour, this is incomplete.
\end{abstract}

\section{Introduction}\label{sec:introduction}

Agent based game theoretic models have become a stalwart of the underpinning
mathematics of interactive behaviours. One of the major pieces of work
in this area is the pair of original computer tournaments run by Robert
Axelrod~\cite{Axelrod1980, Axelrod1980a}. These tournaments pitted submitted
computer strategies against each other in plays of the Iterated Prisoner's
Dilemma. A common game where agents can choose to pay a slight cost to their
immediate utility in the hope of building a reputation. This has been used in
economic and evolutionary game theory to understand the evolution of cooperative
behaviour.

Recently, a class of strategies was described in~\cite{Press2012} that can
provably extort any given opponent. In~\cite{Hilbe2013, Moran1707} some
questions have already been asked about the true effectiveness of these
strategies in an evolutionary setting. Here another question is asked: is it
possible to recognise this extortionate behaviour? A mathematical procedure for
suspicion is presented: in the same way that the continued actions of an
extortionate individual might raise suspicion.

This work makes use of the Axelrod Python library~\cite{Knight2018, Knight2016}
with a large number of Prisoner Dilemma strategies available to give an
extensive numerical example of the ideas presented.  The approach is presented
in Section~\ref{sec:delta-zd-strategies}.  All of the code and data discussed
in Section~\ref{sec:numerical-experiments} is open sourced, archived and
written according to best scientific principles~\cite{Wilson2014}. The data
archive can be found at~\cite{vincent_knight_2018_1297075}.

\section{Recognising Extortion}\label{sec:delta-zd-strategies}

In~\cite{Press2012}, given a match between 2 memory-one strategies, the concept
of Zero Determinant (ZD) strategies is introduced. The main result of that paper
shows that given two memory one players \(p, q\in\mathbb{R}^4\) a linear
relationship between the players' scores could be forced by one of the players.

Using the notation of~\cite{Press2012}, assuming the utilities for player \(p\)
are given by \(S_x=(R, S, T, P)\) and for player \(q\) by \(S_y=(R, T, S, P)\)
and that the stationary scores of each player is given by \(S_X\) and \(S_Y\)
respectively. The main result of~\cite{Press2012} is that if

\begin{equation}\label{eqn:linear_relationship_for_p}
    \tilde p=\alpha S_x + \beta S_y + \gamma
\end{equation}

or

\begin{equation}\label{eqn:linear_relationship_for_q}
    \tilde q=\alpha S_x + \beta S_y + \gamma
\end{equation}

where \(\tilde p = (1 - p_1, 1 - p_2, p_3, p_4)\) and
\(\tilde q = (1 - q_1, 1 - q_2, q_3, q_4)\) then:

\begin{equation}
    \alpha S_X + \beta S_Y + \gamma = 0
\end{equation}

In~\cite{Press2012} a particular type of ZD strategy is defined: extortionate
strategies. If:

\begin{equation}\label{eqn:constraint_for_extortion}
    \gamma = - P(\alpha + \beta)
\end{equation}

then the player can ensure they get a score \(\chi\) times
larger than the opponent. This extortion coefficient is given by:

\begin{equation}\label{eqn:definition_of_chi}
    \chi=\frac{-\beta}{\alpha}
\end{equation}

Thus, if (\ref{eqn:constraint_for_extortion}) holds and \(\chi >1\) a player is
said to extort their opponent.
Here, the reverse problem is considered: given a
\(p\in\mathbb{R}^4\) how does one identify \(\alpha, \beta\) if they
exist and is the strategy in fact acting in an extortionate way?

These conditions correspond to:

\begin{align}
    \tilde p_1 & = \alpha R + \beta R - P (\alpha + \beta)
            \label{eqn:condition_for_tilde_p1}\\
    \tilde p_2 & = \alpha S + \beta T - P (\alpha + \beta)
            \label{eqn:condition_for_tilde_p2}\\
    \tilde p_3 & = \alpha T + \beta S - P (\alpha + \beta)
            \label{eqn:condition_for_tilde_p3}\\
    \tilde p_4 & = \alpha P + \beta P - P (\alpha + \beta)
            \label{eqn:condition_for_tilde_p4}
\end{align}

Equation (\ref{eqn:condition_for_tilde_p4}) ensures that \(p_4=\tilde p_4=0\).
Equations (\ref{eqn:condition_for_tilde_p1}-\ref{eqn:condition_for_tilde_p3})
can be used to eliminate \(\alpha, \beta\), giving:

\begin{equation}\label{eqn:planar_definition_of_extortion}
    \tilde p_1 = \frac{(R - P)(\tilde p_2 + \tilde p_3)}{S + T - 2P}
\end{equation}

with:

\begin{equation}\label{eqn:definition_of_chi}
    \chi = \frac{\tilde p_2 (P - T) + \tilde p_3 (S - P)}
                {\tilde p_2 (P - S) + \tilde p_3 (T - P)}
\end{equation}

Given a strategy \(p\in\mathbb{R}^{4\times 1}\) equations
(\ref{eqn:condition_for_tilde_p4}), (\ref{eqn:planar_definition_of_extortion}-\ref{eqn:definition_of_chi}) can be used to check if
a strategy is extortionate. The conditions correspond to:

\begin{align}
    p_1 & = \frac{(R-P)(p_2 + p_3) - R + T + S - P}{S + T - 2P}
     \label{eqn:condition_for_p1}\\
    p_4 & = 0 \label{eqn:condition_for_p4}\\
    1 & > p_2 + p_3\label{eqn:condition_for_chi}
\end{align}

The algebraic steps necessary to prove these results are available in the
supporting materials.

All extortionate strategies reside on a triangular (\ref{eqn:condition_for_chi})
plane (\ref{eqn:condition_for_p1}) in 3 dimensions (\ref{eqn:condition_for_p4}).
Using this formulation it can be seen that a necessary (but not sufficient)
condition for an extortionate strategy is that it cooperates on average less
than 50\% of the time when in a state of disagreement with the opponent.

As an example, consider the known extortionate strategy \(p=(8 / 9, 1 / 2, 1 /
3, 0)\) from~\cite{Stewart2012} which is referred to as \texttt{Extort-2}. In
this case, for the standard values of \((R, T, S, P)\) constraint
(\ref{eqn:condition_for_p1}) corresponds to:

\begin{equation}
    p_1 = \frac{2(p_2 + p_3) + 1}{3}
\end{equation}

It is clear that in this case all constraints hold.

This approach could in fact be used to confirm that a given strategy is acting
in an extortionate manner even if it is not a memory one strategy. However, in
practice, if a closed form for \(p\) is not known, then due to measurement
and/or numerical error this would not work.

This problem can be written in the following linear algebraic form where
\(x=(\alpha, \beta)\)
and \(p^*=(\tilde p_1 - 1, tilde_2 - 1, p_3)\):

\begin{equation}\label{eqn:linear_algebraic_equation_for_p}
    Cx= p^*
\end{equation}

\(C\) corresponds to equations
(\ref{eqn:condition_for_tilde_p1}-\ref{eqn:condition_for_tilde_p3}) and is
given by:

\begin{equation}\label{eqn:definition_of_C}
    C =
    \begin{bmatrix}
        R - P & R- P \\
        S - P & T- P \\
        T - P & S- P \\
    \end{bmatrix}
\end{equation}

Note that in general, equation (\ref{eqn:linear_algebraic_equation_for_p}) will
not necessarily have a solution. From the Rouch\'{e}-Capelli theorem if there is
a solution it is unique as \(\text{rank}(C)=2\) which is the dimension of the
variable \(x\). The best fitting \(x\) is found by minimizing:

\begin{equation}\label{eqn:r_squared}
    \text{SSError} = \|C x- p^*\|_2^2 = \sum_{i=1}^{3}\left((C\bar x)_i-p_i^*\right)^2
\end{equation}

Note that \(\text{SSError}\), which is the square of the Frobenius
norm~\cite{Golub2013}, becomes a measure of how close a strategy is to being an
extortionate strategy. Suspicion
of extortion then corresponds to a threshold on \(\text{SSError}\).

By observing interactions (human or otherwise), their memory one representation
can be inferred and this approach can be used to recognise extortionate
behaviour. The notion of comparing theoretic and actual plays of the IPD is not
novel, see for example~\cite{Rand2013}. Immediately it is noted that if the
environment is noisy~\cite{Wu1995} then no strategy can be considered to be
extortionate as \(p_4>0\).

In the next section, this idea will be illustrated by observing the interactions
that take place in a computer based tournament of the IPD\@.

\section{Numerical experiments}\label{sec:numerical-experiments}

In~\cite{Stewart2012} results from a tournament with
\input{./assets/tex/number_of_stewart_plotkin_strategies/main.tex} strategies,
was presented with specific consideration given to ZD strategies. This
tournament is reproduced here using the Axelrod-Python
project~\cite{Knight2016}. To obtain a good measure of the corresponding
transition rates for each strategy all matches have been run for
\input{assets/tex/number_of_turns/main.tex} turns and every match has been
repeated \input{assets/tex/number_of_repetitions/main.tex} times. All of this
interaction data is available at~\cite{vincent_knight_2018_1297075}. A good
match between the inferred Markov chain and the state distribution of the actual
interactions has been verified. Data for this is presented in the supplementary
materials.

Figure~\ref{fig:SSError_overall_in_stewart_plotkin} shows the \(\text{SSError}\)
values for all the strategies in the tournament, as reported
in~\cite{Stewart2012} the extortionate strategy (which has an expected
\(\text{SSError}\) approximately 0) gains a large number of wins.

\begin{figure}[!htbp]
    \centering
    \includegraphics[width=.8\textwidth]{./assets/img/SSError_overall_in_stewart_plotkin/main.pdf}
    \caption{\(\text{SSError}\) and state probabilities for the strategies
        of~\cite{Stewart2012}, ordered both by number of wins and overall score.
        Note that \(P(DC)\) is not shown as it corresponds to the transpose of
        \(P(CD)\). Cooperator and Defector are omitted as they do not visit all
        the states.}
    \label{fig:SSError_overall_in_stewart_plotkin}
\end{figure}

Here, the work of~\cite{Stewart2012} is extended by investigating a tournament
with \input{assets/tex/number_of_full_strategies/main.tex}
strategies.

The results of this analysis are shown in
Figure~\ref{fig:SSError_and_probabilities_in_full}. The top ranking strategies
by number of wins seem to be extortionate (but not against all strategies) and
it can be seen that a small sub group of strategies achieve mutual defection.
All the top ranking strategies according to score achieve mutual cooperation and
do not extort each other, however they
\textbf{do} exhibit extortionate behaviour towards a number of the lower ranking
strategies.

\begin{figure}[!htbp]
    \centering
    \includegraphics[width=.8\textwidth]{./assets/img/SSError_and_probabilities_in_full/main.pdf}
    \caption{\(\text{SSError}\) for the strategies for the full tournament. Only
    strategy interactions for which \(p_4=0\) and \(\chi>1\) are displayed.}
    \label{fig:SSError_and_probabilities_in_full}
\end{figure}

\section{Conclusion}\label{sec:conclusion}

This work defines an approach to measure whether or not a player is playing a
strategy that corresponds to an extortionate strategy as defined
in~\cite{Press2012}: a mathematical model for suspicion. Indeed, all
extortionate strategies have been
 classified as lying on a triangular plane.
This rigorous classification fails to be robust to small measurement error, thus
a statistical approach is proposed.
This is done through a linear algebraic approach for approximating the solution
of a linear system. Using this, a large number of pairwise interactions is
simulated and in fact very few strategies are found to act extortionately.

The work of~\cite{Press2012}, whilst showing that a clever approach to taking
advantage of another memory one strategy exists: this is incomplete. Whilst the
elegance of this result is very attractive, just as the simplicity of the
victory of Tit For Tat in Axelrod's original tournaments was, it is incomplete.
Extortionate strategies achieve a high number of wins but they do not
achieve a high score which corresponds to the fitness landscape in an
evolutionary sense. From the large number of interactions a payoff matrix \(S\)
can be measured where \(S_{ij}\) denotes the score (using standard values of
\((R, S, T, P) = (3, 0, 5, 1)\)) of the \(i\)th strategy
against the \(j\)th strategy. Using this, the replicator equation
describes the evolution of the system based on a population density fitness
function:

\begin{equation}\label{eqn:replicator_dynamics}
    \frac{dx}{dt} = x(S-x^TS x)
\end{equation}

Equation (\ref{eqn:replicator_dynamics}) is solved numerically through an
integration technique described in~\cite{Petzold1983} and
Figure~\ref{fig:replicator_dynamics} shows the evolution of the distribution of
the system: the various strategies are ranked by scores. It is clear to see that
only the high ranking strategies survive the evolutionary process (in fact,
only \input{./assets/img/replicator_dynamics/main.tex}
have a final distribution greater than \(10 ^ {-2}\)). This confirms the
findings of~\cite{Moran1707} in which sophisticated strategies resist
evolutionary invasion of shorter memory strategies. Recalling
Figure~\ref{fig:SSError_and_probabilities_in_full} this demonstrates that:

\begin{itemize}
    \item Cooperation emerges through the evolutionary process: the high scoring
        strategies do not exhibit extortionate behaviour towards each other.
    \item Extortionate strategies do not survive the evolutionary process.
\end{itemize}

\begin{figure}[!htbp]
    \centering
    \includegraphics[width=.8\textwidth]{./assets/img/replicator_dynamics/main.pdf}
    \caption{Numerical simulation of the replicator equation
    (\ref{eqn:replicator_dynamics}): strategies are ordered by score, only the strategies with a high score survive the evolutionary process.}
    \label{fig:replicator_dynamics}
\end{figure}

This work can be used to classify plays of the IPD\@: data can be collected from
actual interactions (in lab or in the field). Furthermore, this allows for a
classification method similar to the notion of fingerprinting presented
in~\cite{Ashlock2008}. Trained strategies can potentially be classified as
extortionate or not or it could be possible to even constrain the reinforcement
learning approaches that are becoming prevalent in the literature.
Alternatively, this mathematical approach for recognising extortion could be
used in sophisticated strategies to defend against invasion. Arguably, some of
the strategies considered here exhibit this behaviour, indeed as described
in~\cite{Harper2017}, the top ranking strategies in the full tournament are
obtained using evolutionary reinforcement learning techniques, thus, suspicion
of extortionate behaviour could in fact be an evolutionary trait.

\section*{Acknowledgements}

The following open source software libraries were used in this research:

\begin{itemize}
    \item The Axelrod ~\cite{Knight2016, Knight2018} library (IPD strategies and
        tournaments).
    \item The sympy library~\cite{Meurer2017} (verification of all symbolic
        calculations).
    \item The matplotlib~\cite{Droettboom2018} library (visualisation).
    \item The pandas~\cite{Structures2010}, dask~\cite{Dask2016} and
        NumPy~\cite{Oliphant2015} libraries (data manipulation).
    \item The SciPy~\cite{Jones2001} library (numerical integration of the
        replicator equation).
\end{itemize}

This work was performed using the computational facilities of the Advanced
Research Computing @ Cardiff (ARCCA) Division, Cardiff University.

\printbibliography

\newpage
\section*{Supplementary materials}

\includepdf{assets/pdf/proof_of_form_of_extortionate_strategies/main.pdf}

\newpage

Using the pair wise interactions the transition rates \(p,
q\) can be measured and the steady state probabilities inferred and compared to
the actual probabilities of each state.
This is done numerically by computing the singular eigenvector of the
matrix \(A\) \cite{Stewart2009}:

\[
    A =
    \begin{bmatrix}
        p_1 q_1 & p_1 (1 - q_1) & (1 - p_1) q_1 & (1 -p_1) (1 - q_1) \\
        p_2 q_2 & p_2 (1 - q_2) & (1 - p_2) q_2 & (1 -p_2) (1 - q_2) \\
        p_3 q_3 & p_3 (1 - q_3) & (1 - p_3) q_3 & (1 -p_3) (1 - q_3) \\
        p_4 q_4 & p_4 (1 - q_4) & (1 - p_4) q_4 & (1 -p_4) (1 - q_4) \\
    \end{bmatrix}
\]

Figure~\ref{fig:computed_probabilities_vs_theoretic_probabilities} shows a
regression line fitted to every pairwise interaction with a reported
\(\text{SSError}\) value (pairwise interactions with missing states were
omitted). This serves to validate the approach: a part from some edge cases the
relationship is consistent.

\begin{figure}[!htbp]
    \centering
    \includegraphics[width=.8\textwidth]{./assets/img/computed_probabilities_vs_theoretic_probabilities/main.pdf}
    \caption{The
        relationship between the steady state probabilities inferred from the
        measured transitions and the actual steady state probabilities. A linear
        regression line is included validating the approach.}
    \label{fig:computed_probabilities_vs_theoretic_probabilities}
\end{figure}


\end{document}

strategies.

The results of this analysis are shown in
Figure~\ref{fig:SSError_and_probabilities_in_full}. The top ranking strategies
by number of wins seem to be extortionate (but not against all strategies) and
it can be seen that a small sub group of strategies achieve mutual defection.
All the top ranking strategies according to score achieve mutual cooperation and
do not extort each other, however they
\textbf{do} exhibit extortionate behaviour towards a number of the lower ranking
strategies.

\begin{figure}[!htbp]
    \centering
    \includegraphics[width=.8\textwidth]{./assets/img/SSError_and_probabilities_in_full/main.pdf}
    \caption{\(\text{SSError}\) for the strategies for the full tournament. Only
    strategy interactions for which \(p_4=0\) and \(\chi>1\) are displayed.}
    \label{fig:SSError_and_probabilities_in_full}
\end{figure}

\section{Conclusion}\label{sec:conclusion}

This work defines an approach to measure whether or not a player is playing a
strategy that corresponds to an extortionate strategy as defined
in~\cite{Press2012}: a mathematical model for suspicion. Indeed, all
extortionate strategies have been
 classified as lying on a triangular plane.
This rigorous classification fails to be robust to small measurement error, thus
a statistical approach is proposed.
This is done through a linear algebraic approach for approximating the solution
of a linear system. Using this, a large number of pairwise interactions is
simulated and in fact very few strategies are found to act extortionately.

The work of~\cite{Press2012}, whilst showing that a clever approach to taking
advantage of another memory one strategy exists: this is incomplete. Whilst the
elegance of this result is very attractive, just as the simplicity of the
victory of Tit For Tat in Axelrod's original tournaments was, it is incomplete.
Extortionate strategies achieve a high number of wins but they do not
achieve a high score which corresponds to the fitness landscape in an
evolutionary sense. From the large number of interactions a payoff matrix \(S\)
can be measured where \(S_{ij}\) denotes the score (using standard values of
\((R, S, T, P) = (3, 0, 5, 1)\)) of the \(i\)th strategy
against the \(j\)th strategy. Using this, the replicator equation
describes the evolution of the system based on a population density fitness
function:

\begin{equation}\label{eqn:replicator_dynamics}
    \frac{dx}{dt} = x(S-x^TS x)
\end{equation}

Equation (\ref{eqn:replicator_dynamics}) is solved numerically through an
integration technique described in~\cite{Petzold1983} and
Figure~\ref{fig:replicator_dynamics} shows the evolution of the distribution of
the system: the various strategies are ranked by scores. It is clear to see that
only the high ranking strategies survive the evolutionary process (in fact,
only \documentclass[a4paper]{article}

\usepackage{amsmath}
\usepackage{amssymb}
\usepackage[margin=1.5cm,
            includefoot,
            footskip=30pt]{geometry}
\usepackage{layout}
\usepackage{graphicx}
\usepackage{subcaption}

\usepackage{biblatex}
\usepackage{pdfpages}

\bibliography{main.bib}

\title{Suspicion: Recognising and evaluating the effectiveness
       of extortion in the Iterated Prisoner's Dilemma}
\author{Vincent A. Knight \and Nikoleta E. Glynatsi}
\date{\today}



\begin{document}

\maketitle

\begin{abstract}
    The Iterated Prisoner's Dilemma is a model for rational and evolutionary
    interactive behaviour. It has applications both in the study of human social
    behaviour as well as in biology.
    It is used to understand when and how a rational individual might
    accept an immediate cost to their own utility for the direct benefit of
    another.

    Much attention has been given to a class of strategies called
    Zero Determinant strategies. It has been theoretically shown that these
    strategies can ``extort'' any player.

    In this work, an approach to identify if observed strategies are playing in
    an extortionate way is described. Furthermore, experimental analysis of
    a large tournament with \input{assets/tex/number_of_full_strategies/main.tex}
    strategies is considered. In this setting
    the most highly performing strategies do not play in an extortionate way
    against each other but do against lower performing strategies.
    This suggests that whilst the theory of Zero Determinant strategies
    indicates that memory is not of fundamental importance to the evolution of
    cooperative behaviour, this is incomplete.
\end{abstract}

\section{Introduction}\label{sec:introduction}

Agent based game theoretic models have become a stalwart of the underpinning
mathematics of interactive behaviours. One of the major pieces of work
in this area is the pair of original computer tournaments run by Robert
Axelrod~\cite{Axelrod1980, Axelrod1980a}. These tournaments pitted submitted
computer strategies against each other in plays of the Iterated Prisoner's
Dilemma. A common game where agents can choose to pay a slight cost to their
immediate utility in the hope of building a reputation. This has been used in
economic and evolutionary game theory to understand the evolution of cooperative
behaviour.

Recently, a class of strategies was described in~\cite{Press2012} that can
provably extort any given opponent. In~\cite{Hilbe2013, Moran1707} some
questions have already been asked about the true effectiveness of these
strategies in an evolutionary setting. Here another question is asked: is it
possible to recognise this extortionate behaviour? A mathematical procedure for
suspicion is presented: in the same way that the continued actions of an
extortionate individual might raise suspicion.

This work makes use of the Axelrod Python library~\cite{Knight2018, Knight2016}
with a large number of Prisoner Dilemma strategies available to give an
extensive numerical example of the ideas presented.  The approach is presented
in Section~\ref{sec:delta-zd-strategies}.  All of the code and data discussed
in Section~\ref{sec:numerical-experiments} is open sourced, archived and
written according to best scientific principles~\cite{Wilson2014}. The data
archive can be found at~\cite{vincent_knight_2018_1297075}.

\section{Recognising Extortion}\label{sec:delta-zd-strategies}

In~\cite{Press2012}, given a match between 2 memory-one strategies, the concept
of Zero Determinant (ZD) strategies is introduced. The main result of that paper
shows that given two memory one players \(p, q\in\mathbb{R}^4\) a linear
relationship between the players' scores could be forced by one of the players.

Using the notation of~\cite{Press2012}, assuming the utilities for player \(p\)
are given by \(S_x=(R, S, T, P)\) and for player \(q\) by \(S_y=(R, T, S, P)\)
and that the stationary scores of each player is given by \(S_X\) and \(S_Y\)
respectively. The main result of~\cite{Press2012} is that if

\begin{equation}\label{eqn:linear_relationship_for_p}
    \tilde p=\alpha S_x + \beta S_y + \gamma
\end{equation}

or

\begin{equation}\label{eqn:linear_relationship_for_q}
    \tilde q=\alpha S_x + \beta S_y + \gamma
\end{equation}

where \(\tilde p = (1 - p_1, 1 - p_2, p_3, p_4)\) and
\(\tilde q = (1 - q_1, 1 - q_2, q_3, q_4)\) then:

\begin{equation}
    \alpha S_X + \beta S_Y + \gamma = 0
\end{equation}

In~\cite{Press2012} a particular type of ZD strategy is defined: extortionate
strategies. If:

\begin{equation}\label{eqn:constraint_for_extortion}
    \gamma = - P(\alpha + \beta)
\end{equation}

then the player can ensure they get a score \(\chi\) times
larger than the opponent. This extortion coefficient is given by:

\begin{equation}\label{eqn:definition_of_chi}
    \chi=\frac{-\beta}{\alpha}
\end{equation}

Thus, if (\ref{eqn:constraint_for_extortion}) holds and \(\chi >1\) a player is
said to extort their opponent.
Here, the reverse problem is considered: given a
\(p\in\mathbb{R}^4\) how does one identify \(\alpha, \beta\) if they
exist and is the strategy in fact acting in an extortionate way?

These conditions correspond to:

\begin{align}
    \tilde p_1 & = \alpha R + \beta R - P (\alpha + \beta)
            \label{eqn:condition_for_tilde_p1}\\
    \tilde p_2 & = \alpha S + \beta T - P (\alpha + \beta)
            \label{eqn:condition_for_tilde_p2}\\
    \tilde p_3 & = \alpha T + \beta S - P (\alpha + \beta)
            \label{eqn:condition_for_tilde_p3}\\
    \tilde p_4 & = \alpha P + \beta P - P (\alpha + \beta)
            \label{eqn:condition_for_tilde_p4}
\end{align}

Equation (\ref{eqn:condition_for_tilde_p4}) ensures that \(p_4=\tilde p_4=0\).
Equations (\ref{eqn:condition_for_tilde_p1}-\ref{eqn:condition_for_tilde_p3})
can be used to eliminate \(\alpha, \beta\), giving:

\begin{equation}\label{eqn:planar_definition_of_extortion}
    \tilde p_1 = \frac{(R - P)(\tilde p_2 + \tilde p_3)}{S + T - 2P}
\end{equation}

with:

\begin{equation}\label{eqn:definition_of_chi}
    \chi = \frac{\tilde p_2 (P - T) + \tilde p_3 (S - P)}
                {\tilde p_2 (P - S) + \tilde p_3 (T - P)}
\end{equation}

Given a strategy \(p\in\mathbb{R}^{4\times 1}\) equations
(\ref{eqn:condition_for_tilde_p4}), (\ref{eqn:planar_definition_of_extortion}-\ref{eqn:definition_of_chi}) can be used to check if
a strategy is extortionate. The conditions correspond to:

\begin{align}
    p_1 & = \frac{(R-P)(p_2 + p_3) - R + T + S - P}{S + T - 2P}
     \label{eqn:condition_for_p1}\\
    p_4 & = 0 \label{eqn:condition_for_p4}\\
    1 & > p_2 + p_3\label{eqn:condition_for_chi}
\end{align}

The algebraic steps necessary to prove these results are available in the
supporting materials.

All extortionate strategies reside on a triangular (\ref{eqn:condition_for_chi})
plane (\ref{eqn:condition_for_p1}) in 3 dimensions (\ref{eqn:condition_for_p4}).
Using this formulation it can be seen that a necessary (but not sufficient)
condition for an extortionate strategy is that it cooperates on average less
than 50\% of the time when in a state of disagreement with the opponent.

As an example, consider the known extortionate strategy \(p=(8 / 9, 1 / 2, 1 /
3, 0)\) from~\cite{Stewart2012} which is referred to as \texttt{Extort-2}. In
this case, for the standard values of \((R, T, S, P)\) constraint
(\ref{eqn:condition_for_p1}) corresponds to:

\begin{equation}
    p_1 = \frac{2(p_2 + p_3) + 1}{3}
\end{equation}

It is clear that in this case all constraints hold.

This approach could in fact be used to confirm that a given strategy is acting
in an extortionate manner even if it is not a memory one strategy. However, in
practice, if a closed form for \(p\) is not known, then due to measurement
and/or numerical error this would not work.

This problem can be written in the following linear algebraic form where
\(x=(\alpha, \beta)\)
and \(p^*=(\tilde p_1 - 1, tilde_2 - 1, p_3)\):

\begin{equation}\label{eqn:linear_algebraic_equation_for_p}
    Cx= p^*
\end{equation}

\(C\) corresponds to equations
(\ref{eqn:condition_for_tilde_p1}-\ref{eqn:condition_for_tilde_p3}) and is
given by:

\begin{equation}\label{eqn:definition_of_C}
    C =
    \begin{bmatrix}
        R - P & R- P \\
        S - P & T- P \\
        T - P & S- P \\
    \end{bmatrix}
\end{equation}

Note that in general, equation (\ref{eqn:linear_algebraic_equation_for_p}) will
not necessarily have a solution. From the Rouch\'{e}-Capelli theorem if there is
a solution it is unique as \(\text{rank}(C)=2\) which is the dimension of the
variable \(x\). The best fitting \(x\) is found by minimizing:

\begin{equation}\label{eqn:r_squared}
    \text{SSError} = \|C x- p^*\|_2^2 = \sum_{i=1}^{3}\left((C\bar x)_i-p_i^*\right)^2
\end{equation}

Note that \(\text{SSError}\), which is the square of the Frobenius
norm~\cite{Golub2013}, becomes a measure of how close a strategy is to being an
extortionate strategy. Suspicion
of extortion then corresponds to a threshold on \(\text{SSError}\).

By observing interactions (human or otherwise), their memory one representation
can be inferred and this approach can be used to recognise extortionate
behaviour. The notion of comparing theoretic and actual plays of the IPD is not
novel, see for example~\cite{Rand2013}. Immediately it is noted that if the
environment is noisy~\cite{Wu1995} then no strategy can be considered to be
extortionate as \(p_4>0\).

In the next section, this idea will be illustrated by observing the interactions
that take place in a computer based tournament of the IPD\@.

\section{Numerical experiments}\label{sec:numerical-experiments}

In~\cite{Stewart2012} results from a tournament with
\input{./assets/tex/number_of_stewart_plotkin_strategies/main.tex} strategies,
was presented with specific consideration given to ZD strategies. This
tournament is reproduced here using the Axelrod-Python
project~\cite{Knight2016}. To obtain a good measure of the corresponding
transition rates for each strategy all matches have been run for
\input{assets/tex/number_of_turns/main.tex} turns and every match has been
repeated \input{assets/tex/number_of_repetitions/main.tex} times. All of this
interaction data is available at~\cite{vincent_knight_2018_1297075}. A good
match between the inferred Markov chain and the state distribution of the actual
interactions has been verified. Data for this is presented in the supplementary
materials.

Figure~\ref{fig:SSError_overall_in_stewart_plotkin} shows the \(\text{SSError}\)
values for all the strategies in the tournament, as reported
in~\cite{Stewart2012} the extortionate strategy (which has an expected
\(\text{SSError}\) approximately 0) gains a large number of wins.

\begin{figure}[!htbp]
    \centering
    \includegraphics[width=.8\textwidth]{./assets/img/SSError_overall_in_stewart_plotkin/main.pdf}
    \caption{\(\text{SSError}\) and state probabilities for the strategies
        of~\cite{Stewart2012}, ordered both by number of wins and overall score.
        Note that \(P(DC)\) is not shown as it corresponds to the transpose of
        \(P(CD)\). Cooperator and Defector are omitted as they do not visit all
        the states.}
    \label{fig:SSError_overall_in_stewart_plotkin}
\end{figure}

Here, the work of~\cite{Stewart2012} is extended by investigating a tournament
with \input{assets/tex/number_of_full_strategies/main.tex}
strategies.

The results of this analysis are shown in
Figure~\ref{fig:SSError_and_probabilities_in_full}. The top ranking strategies
by number of wins seem to be extortionate (but not against all strategies) and
it can be seen that a small sub group of strategies achieve mutual defection.
All the top ranking strategies according to score achieve mutual cooperation and
do not extort each other, however they
\textbf{do} exhibit extortionate behaviour towards a number of the lower ranking
strategies.

\begin{figure}[!htbp]
    \centering
    \includegraphics[width=.8\textwidth]{./assets/img/SSError_and_probabilities_in_full/main.pdf}
    \caption{\(\text{SSError}\) for the strategies for the full tournament. Only
    strategy interactions for which \(p_4=0\) and \(\chi>1\) are displayed.}
    \label{fig:SSError_and_probabilities_in_full}
\end{figure}

\section{Conclusion}\label{sec:conclusion}

This work defines an approach to measure whether or not a player is playing a
strategy that corresponds to an extortionate strategy as defined
in~\cite{Press2012}: a mathematical model for suspicion. Indeed, all
extortionate strategies have been
 classified as lying on a triangular plane.
This rigorous classification fails to be robust to small measurement error, thus
a statistical approach is proposed.
This is done through a linear algebraic approach for approximating the solution
of a linear system. Using this, a large number of pairwise interactions is
simulated and in fact very few strategies are found to act extortionately.

The work of~\cite{Press2012}, whilst showing that a clever approach to taking
advantage of another memory one strategy exists: this is incomplete. Whilst the
elegance of this result is very attractive, just as the simplicity of the
victory of Tit For Tat in Axelrod's original tournaments was, it is incomplete.
Extortionate strategies achieve a high number of wins but they do not
achieve a high score which corresponds to the fitness landscape in an
evolutionary sense. From the large number of interactions a payoff matrix \(S\)
can be measured where \(S_{ij}\) denotes the score (using standard values of
\((R, S, T, P) = (3, 0, 5, 1)\)) of the \(i\)th strategy
against the \(j\)th strategy. Using this, the replicator equation
describes the evolution of the system based on a population density fitness
function:

\begin{equation}\label{eqn:replicator_dynamics}
    \frac{dx}{dt} = x(S-x^TS x)
\end{equation}

Equation (\ref{eqn:replicator_dynamics}) is solved numerically through an
integration technique described in~\cite{Petzold1983} and
Figure~\ref{fig:replicator_dynamics} shows the evolution of the distribution of
the system: the various strategies are ranked by scores. It is clear to see that
only the high ranking strategies survive the evolutionary process (in fact,
only \input{./assets/img/replicator_dynamics/main.tex}
have a final distribution greater than \(10 ^ {-2}\)). This confirms the
findings of~\cite{Moran1707} in which sophisticated strategies resist
evolutionary invasion of shorter memory strategies. Recalling
Figure~\ref{fig:SSError_and_probabilities_in_full} this demonstrates that:

\begin{itemize}
    \item Cooperation emerges through the evolutionary process: the high scoring
        strategies do not exhibit extortionate behaviour towards each other.
    \item Extortionate strategies do not survive the evolutionary process.
\end{itemize}

\begin{figure}[!htbp]
    \centering
    \includegraphics[width=.8\textwidth]{./assets/img/replicator_dynamics/main.pdf}
    \caption{Numerical simulation of the replicator equation
    (\ref{eqn:replicator_dynamics}): strategies are ordered by score, only the strategies with a high score survive the evolutionary process.}
    \label{fig:replicator_dynamics}
\end{figure}

This work can be used to classify plays of the IPD\@: data can be collected from
actual interactions (in lab or in the field). Furthermore, this allows for a
classification method similar to the notion of fingerprinting presented
in~\cite{Ashlock2008}. Trained strategies can potentially be classified as
extortionate or not or it could be possible to even constrain the reinforcement
learning approaches that are becoming prevalent in the literature.
Alternatively, this mathematical approach for recognising extortion could be
used in sophisticated strategies to defend against invasion. Arguably, some of
the strategies considered here exhibit this behaviour, indeed as described
in~\cite{Harper2017}, the top ranking strategies in the full tournament are
obtained using evolutionary reinforcement learning techniques, thus, suspicion
of extortionate behaviour could in fact be an evolutionary trait.

\section*{Acknowledgements}

The following open source software libraries were used in this research:

\begin{itemize}
    \item The Axelrod ~\cite{Knight2016, Knight2018} library (IPD strategies and
        tournaments).
    \item The sympy library~\cite{Meurer2017} (verification of all symbolic
        calculations).
    \item The matplotlib~\cite{Droettboom2018} library (visualisation).
    \item The pandas~\cite{Structures2010}, dask~\cite{Dask2016} and
        NumPy~\cite{Oliphant2015} libraries (data manipulation).
    \item The SciPy~\cite{Jones2001} library (numerical integration of the
        replicator equation).
\end{itemize}

This work was performed using the computational facilities of the Advanced
Research Computing @ Cardiff (ARCCA) Division, Cardiff University.

\printbibliography

\newpage
\section*{Supplementary materials}

\includepdf{assets/pdf/proof_of_form_of_extortionate_strategies/main.pdf}

\newpage

Using the pair wise interactions the transition rates \(p,
q\) can be measured and the steady state probabilities inferred and compared to
the actual probabilities of each state.
This is done numerically by computing the singular eigenvector of the
matrix \(A\) \cite{Stewart2009}:

\[
    A =
    \begin{bmatrix}
        p_1 q_1 & p_1 (1 - q_1) & (1 - p_1) q_1 & (1 -p_1) (1 - q_1) \\
        p_2 q_2 & p_2 (1 - q_2) & (1 - p_2) q_2 & (1 -p_2) (1 - q_2) \\
        p_3 q_3 & p_3 (1 - q_3) & (1 - p_3) q_3 & (1 -p_3) (1 - q_3) \\
        p_4 q_4 & p_4 (1 - q_4) & (1 - p_4) q_4 & (1 -p_4) (1 - q_4) \\
    \end{bmatrix}
\]

Figure~\ref{fig:computed_probabilities_vs_theoretic_probabilities} shows a
regression line fitted to every pairwise interaction with a reported
\(\text{SSError}\) value (pairwise interactions with missing states were
omitted). This serves to validate the approach: a part from some edge cases the
relationship is consistent.

\begin{figure}[!htbp]
    \centering
    \includegraphics[width=.8\textwidth]{./assets/img/computed_probabilities_vs_theoretic_probabilities/main.pdf}
    \caption{The
        relationship between the steady state probabilities inferred from the
        measured transitions and the actual steady state probabilities. A linear
        regression line is included validating the approach.}
    \label{fig:computed_probabilities_vs_theoretic_probabilities}
\end{figure}


\end{document}

have a final distribution greater than \(10 ^ {-2}\)). This confirms the
findings of~\cite{Moran1707} in which sophisticated strategies resist
evolutionary invasion of shorter memory strategies. Recalling
Figure~\ref{fig:SSError_and_probabilities_in_full} this demonstrates that:

\begin{itemize}
    \item Cooperation emerges through the evolutionary process: the high scoring
        strategies do not exhibit extortionate behaviour towards each other.
    \item Extortionate strategies do not survive the evolutionary process.
\end{itemize}

\begin{figure}[!htbp]
    \centering
    \includegraphics[width=.8\textwidth]{./assets/img/replicator_dynamics/main.pdf}
    \caption{Numerical simulation of the replicator equation
    (\ref{eqn:replicator_dynamics}): strategies are ordered by score, only the strategies with a high score survive the evolutionary process.}
    \label{fig:replicator_dynamics}
\end{figure}

This work can be used to classify plays of the IPD\@: data can be collected from
actual interactions (in lab or in the field). Furthermore, this allows for a
classification method similar to the notion of fingerprinting presented
in~\cite{Ashlock2008}. Trained strategies can potentially be classified as
extortionate or not or it could be possible to even constrain the reinforcement
learning approaches that are becoming prevalent in the literature.
Alternatively, this mathematical approach for recognising extortion could be
used in sophisticated strategies to defend against invasion. Arguably, some of
the strategies considered here exhibit this behaviour, indeed as described
in~\cite{Harper2017}, the top ranking strategies in the full tournament are
obtained using evolutionary reinforcement learning techniques, thus, suspicion
of extortionate behaviour could in fact be an evolutionary trait.

\section*{Acknowledgements}

The following open source software libraries were used in this research:

\begin{itemize}
    \item The Axelrod ~\cite{Knight2016, Knight2018} library (IPD strategies and
        tournaments).
    \item The sympy library~\cite{Meurer2017} (verification of all symbolic
        calculations).
    \item The matplotlib~\cite{Droettboom2018} library (visualisation).
    \item The pandas~\cite{Structures2010}, dask~\cite{Dask2016} and
        NumPy~\cite{Oliphant2015} libraries (data manipulation).
    \item The SciPy~\cite{Jones2001} library (numerical integration of the
        replicator equation).
\end{itemize}

This work was performed using the computational facilities of the Advanced
Research Computing @ Cardiff (ARCCA) Division, Cardiff University.

\printbibliography

\newpage
\section*{Supplementary materials}

\includepdf{assets/pdf/proof_of_form_of_extortionate_strategies/main.pdf}

\newpage

Using the pair wise interactions the transition rates \(p,
q\) can be measured and the steady state probabilities inferred and compared to
the actual probabilities of each state.
This is done numerically by computing the singular eigenvector of the
matrix \(A\) \cite{Stewart2009}:

\[
    A =
    \begin{bmatrix}
        p_1 q_1 & p_1 (1 - q_1) & (1 - p_1) q_1 & (1 -p_1) (1 - q_1) \\
        p_2 q_2 & p_2 (1 - q_2) & (1 - p_2) q_2 & (1 -p_2) (1 - q_2) \\
        p_3 q_3 & p_3 (1 - q_3) & (1 - p_3) q_3 & (1 -p_3) (1 - q_3) \\
        p_4 q_4 & p_4 (1 - q_4) & (1 - p_4) q_4 & (1 -p_4) (1 - q_4) \\
    \end{bmatrix}
\]

Figure~\ref{fig:computed_probabilities_vs_theoretic_probabilities} shows a
regression line fitted to every pairwise interaction with a reported
\(\text{SSError}\) value (pairwise interactions with missing states were
omitted). This serves to validate the approach: a part from some edge cases the
relationship is consistent.

\begin{figure}[!htbp]
    \centering
    \includegraphics[width=.8\textwidth]{./assets/img/computed_probabilities_vs_theoretic_probabilities/main.pdf}
    \caption{The
        relationship between the steady state probabilities inferred from the
        measured transitions and the actual steady state probabilities. A linear
        regression line is included validating the approach.}
    \label{fig:computed_probabilities_vs_theoretic_probabilities}
\end{figure}


\end{document}

strategies.

The results of this analysis are shown in
Figure~\ref{fig:SSError_and_probabilities_in_full}. The top ranking strategies
by number of wins seem to be extortionate (but not against all strategies) and
it can be seen that a small sub group of strategies achieve mutual defection.
All the top ranking strategies according to score achieve mutual cooperation and
do not extort each other, however they
\textbf{do} exhibit extortionate behaviour towards a number of the lower ranking
strategies.

\begin{figure}[!htbp]
    \centering
    \includegraphics[width=.8\textwidth]{./assets/img/SSError_and_probabilities_in_full/main.pdf}
    \caption{\(\text{SSError}\) for the strategies for the full tournament. Only
    strategy interactions for which \(p_4=0\) and \(\chi>1\) are displayed.}
    \label{fig:SSError_and_probabilities_in_full}
\end{figure}

\section{Conclusion}\label{sec:conclusion}

This work defines an approach to measure whether or not a player is playing a
strategy that corresponds to an extortionate strategy as defined
in~\cite{Press2012}: a mathematical model for suspicion. Indeed, all
extortionate strategies have been
 classified as lying on a triangular plane.
This rigorous classification fails to be robust to small measurement error, thus
a statistical approach is proposed.
This is done through a linear algebraic approach for approximating the solution
of a linear system. Using this, a large number of pairwise interactions is
simulated and in fact very few strategies are found to act extortionately.

The work of~\cite{Press2012}, whilst showing that a clever approach to taking
advantage of another memory one strategy exists: this is incomplete. Whilst the
elegance of this result is very attractive, just as the simplicity of the
victory of Tit For Tat in Axelrod's original tournaments was, it is incomplete.
Extortionate strategies achieve a high number of wins but they do not
achieve a high score which corresponds to the fitness landscape in an
evolutionary sense. From the large number of interactions a payoff matrix \(S\)
can be measured where \(S_{ij}\) denotes the score (using standard values of
\((R, S, T, P) = (3, 0, 5, 1)\)) of the \(i\)th strategy
against the \(j\)th strategy. Using this, the replicator equation
describes the evolution of the system based on a population density fitness
function:

\begin{equation}\label{eqn:replicator_dynamics}
    \frac{dx}{dt} = x(S-x^TS x)
\end{equation}

Equation (\ref{eqn:replicator_dynamics}) is solved numerically through an
integration technique described in~\cite{Petzold1983} and
Figure~\ref{fig:replicator_dynamics} shows the evolution of the distribution of
the system: the various strategies are ranked by scores. It is clear to see that
only the high ranking strategies survive the evolutionary process (in fact,
only \documentclass[a4paper]{article}

\usepackage{amsmath}
\usepackage{amssymb}
\usepackage[margin=1.5cm,
            includefoot,
            footskip=30pt]{geometry}
\usepackage{layout}
\usepackage{graphicx}
\usepackage{subcaption}

\usepackage{biblatex}
\usepackage{pdfpages}

\bibliography{main.bib}

\title{Suspicion: Recognising and evaluating the effectiveness
       of extortion in the Iterated Prisoner's Dilemma}
\author{Vincent A. Knight \and Nikoleta E. Glynatsi}
\date{\today}



\begin{document}

\maketitle

\begin{abstract}
    The Iterated Prisoner's Dilemma is a model for rational and evolutionary
    interactive behaviour. It has applications both in the study of human social
    behaviour as well as in biology.
    It is used to understand when and how a rational individual might
    accept an immediate cost to their own utility for the direct benefit of
    another.

    Much attention has been given to a class of strategies called
    Zero Determinant strategies. It has been theoretically shown that these
    strategies can ``extort'' any player.

    In this work, an approach to identify if observed strategies are playing in
    an extortionate way is described. Furthermore, experimental analysis of
    a large tournament with \documentclass[a4paper]{article}

\usepackage{amsmath}
\usepackage{amssymb}
\usepackage[margin=1.5cm,
            includefoot,
            footskip=30pt]{geometry}
\usepackage{layout}
\usepackage{graphicx}
\usepackage{subcaption}

\usepackage{biblatex}
\usepackage{pdfpages}

\bibliography{main.bib}

\title{Suspicion: Recognising and evaluating the effectiveness
       of extortion in the Iterated Prisoner's Dilemma}
\author{Vincent A. Knight \and Nikoleta E. Glynatsi}
\date{\today}



\begin{document}

\maketitle

\begin{abstract}
    The Iterated Prisoner's Dilemma is a model for rational and evolutionary
    interactive behaviour. It has applications both in the study of human social
    behaviour as well as in biology.
    It is used to understand when and how a rational individual might
    accept an immediate cost to their own utility for the direct benefit of
    another.

    Much attention has been given to a class of strategies called
    Zero Determinant strategies. It has been theoretically shown that these
    strategies can ``extort'' any player.

    In this work, an approach to identify if observed strategies are playing in
    an extortionate way is described. Furthermore, experimental analysis of
    a large tournament with \input{assets/tex/number_of_full_strategies/main.tex}
    strategies is considered. In this setting
    the most highly performing strategies do not play in an extortionate way
    against each other but do against lower performing strategies.
    This suggests that whilst the theory of Zero Determinant strategies
    indicates that memory is not of fundamental importance to the evolution of
    cooperative behaviour, this is incomplete.
\end{abstract}

\section{Introduction}\label{sec:introduction}

Agent based game theoretic models have become a stalwart of the underpinning
mathematics of interactive behaviours. One of the major pieces of work
in this area is the pair of original computer tournaments run by Robert
Axelrod~\cite{Axelrod1980, Axelrod1980a}. These tournaments pitted submitted
computer strategies against each other in plays of the Iterated Prisoner's
Dilemma. A common game where agents can choose to pay a slight cost to their
immediate utility in the hope of building a reputation. This has been used in
economic and evolutionary game theory to understand the evolution of cooperative
behaviour.

Recently, a class of strategies was described in~\cite{Press2012} that can
provably extort any given opponent. In~\cite{Hilbe2013, Moran1707} some
questions have already been asked about the true effectiveness of these
strategies in an evolutionary setting. Here another question is asked: is it
possible to recognise this extortionate behaviour? A mathematical procedure for
suspicion is presented: in the same way that the continued actions of an
extortionate individual might raise suspicion.

This work makes use of the Axelrod Python library~\cite{Knight2018, Knight2016}
with a large number of Prisoner Dilemma strategies available to give an
extensive numerical example of the ideas presented.  The approach is presented
in Section~\ref{sec:delta-zd-strategies}.  All of the code and data discussed
in Section~\ref{sec:numerical-experiments} is open sourced, archived and
written according to best scientific principles~\cite{Wilson2014}. The data
archive can be found at~\cite{vincent_knight_2018_1297075}.

\section{Recognising Extortion}\label{sec:delta-zd-strategies}

In~\cite{Press2012}, given a match between 2 memory-one strategies, the concept
of Zero Determinant (ZD) strategies is introduced. The main result of that paper
shows that given two memory one players \(p, q\in\mathbb{R}^4\) a linear
relationship between the players' scores could be forced by one of the players.

Using the notation of~\cite{Press2012}, assuming the utilities for player \(p\)
are given by \(S_x=(R, S, T, P)\) and for player \(q\) by \(S_y=(R, T, S, P)\)
and that the stationary scores of each player is given by \(S_X\) and \(S_Y\)
respectively. The main result of~\cite{Press2012} is that if

\begin{equation}\label{eqn:linear_relationship_for_p}
    \tilde p=\alpha S_x + \beta S_y + \gamma
\end{equation}

or

\begin{equation}\label{eqn:linear_relationship_for_q}
    \tilde q=\alpha S_x + \beta S_y + \gamma
\end{equation}

where \(\tilde p = (1 - p_1, 1 - p_2, p_3, p_4)\) and
\(\tilde q = (1 - q_1, 1 - q_2, q_3, q_4)\) then:

\begin{equation}
    \alpha S_X + \beta S_Y + \gamma = 0
\end{equation}

In~\cite{Press2012} a particular type of ZD strategy is defined: extortionate
strategies. If:

\begin{equation}\label{eqn:constraint_for_extortion}
    \gamma = - P(\alpha + \beta)
\end{equation}

then the player can ensure they get a score \(\chi\) times
larger than the opponent. This extortion coefficient is given by:

\begin{equation}\label{eqn:definition_of_chi}
    \chi=\frac{-\beta}{\alpha}
\end{equation}

Thus, if (\ref{eqn:constraint_for_extortion}) holds and \(\chi >1\) a player is
said to extort their opponent.
Here, the reverse problem is considered: given a
\(p\in\mathbb{R}^4\) how does one identify \(\alpha, \beta\) if they
exist and is the strategy in fact acting in an extortionate way?

These conditions correspond to:

\begin{align}
    \tilde p_1 & = \alpha R + \beta R - P (\alpha + \beta)
            \label{eqn:condition_for_tilde_p1}\\
    \tilde p_2 & = \alpha S + \beta T - P (\alpha + \beta)
            \label{eqn:condition_for_tilde_p2}\\
    \tilde p_3 & = \alpha T + \beta S - P (\alpha + \beta)
            \label{eqn:condition_for_tilde_p3}\\
    \tilde p_4 & = \alpha P + \beta P - P (\alpha + \beta)
            \label{eqn:condition_for_tilde_p4}
\end{align}

Equation (\ref{eqn:condition_for_tilde_p4}) ensures that \(p_4=\tilde p_4=0\).
Equations (\ref{eqn:condition_for_tilde_p1}-\ref{eqn:condition_for_tilde_p3})
can be used to eliminate \(\alpha, \beta\), giving:

\begin{equation}\label{eqn:planar_definition_of_extortion}
    \tilde p_1 = \frac{(R - P)(\tilde p_2 + \tilde p_3)}{S + T - 2P}
\end{equation}

with:

\begin{equation}\label{eqn:definition_of_chi}
    \chi = \frac{\tilde p_2 (P - T) + \tilde p_3 (S - P)}
                {\tilde p_2 (P - S) + \tilde p_3 (T - P)}
\end{equation}

Given a strategy \(p\in\mathbb{R}^{4\times 1}\) equations
(\ref{eqn:condition_for_tilde_p4}), (\ref{eqn:planar_definition_of_extortion}-\ref{eqn:definition_of_chi}) can be used to check if
a strategy is extortionate. The conditions correspond to:

\begin{align}
    p_1 & = \frac{(R-P)(p_2 + p_3) - R + T + S - P}{S + T - 2P}
     \label{eqn:condition_for_p1}\\
    p_4 & = 0 \label{eqn:condition_for_p4}\\
    1 & > p_2 + p_3\label{eqn:condition_for_chi}
\end{align}

The algebraic steps necessary to prove these results are available in the
supporting materials.

All extortionate strategies reside on a triangular (\ref{eqn:condition_for_chi})
plane (\ref{eqn:condition_for_p1}) in 3 dimensions (\ref{eqn:condition_for_p4}).
Using this formulation it can be seen that a necessary (but not sufficient)
condition for an extortionate strategy is that it cooperates on average less
than 50\% of the time when in a state of disagreement with the opponent.

As an example, consider the known extortionate strategy \(p=(8 / 9, 1 / 2, 1 /
3, 0)\) from~\cite{Stewart2012} which is referred to as \texttt{Extort-2}. In
this case, for the standard values of \((R, T, S, P)\) constraint
(\ref{eqn:condition_for_p1}) corresponds to:

\begin{equation}
    p_1 = \frac{2(p_2 + p_3) + 1}{3}
\end{equation}

It is clear that in this case all constraints hold.

This approach could in fact be used to confirm that a given strategy is acting
in an extortionate manner even if it is not a memory one strategy. However, in
practice, if a closed form for \(p\) is not known, then due to measurement
and/or numerical error this would not work.

This problem can be written in the following linear algebraic form where
\(x=(\alpha, \beta)\)
and \(p^*=(\tilde p_1 - 1, tilde_2 - 1, p_3)\):

\begin{equation}\label{eqn:linear_algebraic_equation_for_p}
    Cx= p^*
\end{equation}

\(C\) corresponds to equations
(\ref{eqn:condition_for_tilde_p1}-\ref{eqn:condition_for_tilde_p3}) and is
given by:

\begin{equation}\label{eqn:definition_of_C}
    C =
    \begin{bmatrix}
        R - P & R- P \\
        S - P & T- P \\
        T - P & S- P \\
    \end{bmatrix}
\end{equation}

Note that in general, equation (\ref{eqn:linear_algebraic_equation_for_p}) will
not necessarily have a solution. From the Rouch\'{e}-Capelli theorem if there is
a solution it is unique as \(\text{rank}(C)=2\) which is the dimension of the
variable \(x\). The best fitting \(x\) is found by minimizing:

\begin{equation}\label{eqn:r_squared}
    \text{SSError} = \|C x- p^*\|_2^2 = \sum_{i=1}^{3}\left((C\bar x)_i-p_i^*\right)^2
\end{equation}

Note that \(\text{SSError}\), which is the square of the Frobenius
norm~\cite{Golub2013}, becomes a measure of how close a strategy is to being an
extortionate strategy. Suspicion
of extortion then corresponds to a threshold on \(\text{SSError}\).

By observing interactions (human or otherwise), their memory one representation
can be inferred and this approach can be used to recognise extortionate
behaviour. The notion of comparing theoretic and actual plays of the IPD is not
novel, see for example~\cite{Rand2013}. Immediately it is noted that if the
environment is noisy~\cite{Wu1995} then no strategy can be considered to be
extortionate as \(p_4>0\).

In the next section, this idea will be illustrated by observing the interactions
that take place in a computer based tournament of the IPD\@.

\section{Numerical experiments}\label{sec:numerical-experiments}

In~\cite{Stewart2012} results from a tournament with
\input{./assets/tex/number_of_stewart_plotkin_strategies/main.tex} strategies,
was presented with specific consideration given to ZD strategies. This
tournament is reproduced here using the Axelrod-Python
project~\cite{Knight2016}. To obtain a good measure of the corresponding
transition rates for each strategy all matches have been run for
\input{assets/tex/number_of_turns/main.tex} turns and every match has been
repeated \input{assets/tex/number_of_repetitions/main.tex} times. All of this
interaction data is available at~\cite{vincent_knight_2018_1297075}. A good
match between the inferred Markov chain and the state distribution of the actual
interactions has been verified. Data for this is presented in the supplementary
materials.

Figure~\ref{fig:SSError_overall_in_stewart_plotkin} shows the \(\text{SSError}\)
values for all the strategies in the tournament, as reported
in~\cite{Stewart2012} the extortionate strategy (which has an expected
\(\text{SSError}\) approximately 0) gains a large number of wins.

\begin{figure}[!htbp]
    \centering
    \includegraphics[width=.8\textwidth]{./assets/img/SSError_overall_in_stewart_plotkin/main.pdf}
    \caption{\(\text{SSError}\) and state probabilities for the strategies
        of~\cite{Stewart2012}, ordered both by number of wins and overall score.
        Note that \(P(DC)\) is not shown as it corresponds to the transpose of
        \(P(CD)\). Cooperator and Defector are omitted as they do not visit all
        the states.}
    \label{fig:SSError_overall_in_stewart_plotkin}
\end{figure}

Here, the work of~\cite{Stewart2012} is extended by investigating a tournament
with \input{assets/tex/number_of_full_strategies/main.tex}
strategies.

The results of this analysis are shown in
Figure~\ref{fig:SSError_and_probabilities_in_full}. The top ranking strategies
by number of wins seem to be extortionate (but not against all strategies) and
it can be seen that a small sub group of strategies achieve mutual defection.
All the top ranking strategies according to score achieve mutual cooperation and
do not extort each other, however they
\textbf{do} exhibit extortionate behaviour towards a number of the lower ranking
strategies.

\begin{figure}[!htbp]
    \centering
    \includegraphics[width=.8\textwidth]{./assets/img/SSError_and_probabilities_in_full/main.pdf}
    \caption{\(\text{SSError}\) for the strategies for the full tournament. Only
    strategy interactions for which \(p_4=0\) and \(\chi>1\) are displayed.}
    \label{fig:SSError_and_probabilities_in_full}
\end{figure}

\section{Conclusion}\label{sec:conclusion}

This work defines an approach to measure whether or not a player is playing a
strategy that corresponds to an extortionate strategy as defined
in~\cite{Press2012}: a mathematical model for suspicion. Indeed, all
extortionate strategies have been
 classified as lying on a triangular plane.
This rigorous classification fails to be robust to small measurement error, thus
a statistical approach is proposed.
This is done through a linear algebraic approach for approximating the solution
of a linear system. Using this, a large number of pairwise interactions is
simulated and in fact very few strategies are found to act extortionately.

The work of~\cite{Press2012}, whilst showing that a clever approach to taking
advantage of another memory one strategy exists: this is incomplete. Whilst the
elegance of this result is very attractive, just as the simplicity of the
victory of Tit For Tat in Axelrod's original tournaments was, it is incomplete.
Extortionate strategies achieve a high number of wins but they do not
achieve a high score which corresponds to the fitness landscape in an
evolutionary sense. From the large number of interactions a payoff matrix \(S\)
can be measured where \(S_{ij}\) denotes the score (using standard values of
\((R, S, T, P) = (3, 0, 5, 1)\)) of the \(i\)th strategy
against the \(j\)th strategy. Using this, the replicator equation
describes the evolution of the system based on a population density fitness
function:

\begin{equation}\label{eqn:replicator_dynamics}
    \frac{dx}{dt} = x(S-x^TS x)
\end{equation}

Equation (\ref{eqn:replicator_dynamics}) is solved numerically through an
integration technique described in~\cite{Petzold1983} and
Figure~\ref{fig:replicator_dynamics} shows the evolution of the distribution of
the system: the various strategies are ranked by scores. It is clear to see that
only the high ranking strategies survive the evolutionary process (in fact,
only \input{./assets/img/replicator_dynamics/main.tex}
have a final distribution greater than \(10 ^ {-2}\)). This confirms the
findings of~\cite{Moran1707} in which sophisticated strategies resist
evolutionary invasion of shorter memory strategies. Recalling
Figure~\ref{fig:SSError_and_probabilities_in_full} this demonstrates that:

\begin{itemize}
    \item Cooperation emerges through the evolutionary process: the high scoring
        strategies do not exhibit extortionate behaviour towards each other.
    \item Extortionate strategies do not survive the evolutionary process.
\end{itemize}

\begin{figure}[!htbp]
    \centering
    \includegraphics[width=.8\textwidth]{./assets/img/replicator_dynamics/main.pdf}
    \caption{Numerical simulation of the replicator equation
    (\ref{eqn:replicator_dynamics}): strategies are ordered by score, only the strategies with a high score survive the evolutionary process.}
    \label{fig:replicator_dynamics}
\end{figure}

This work can be used to classify plays of the IPD\@: data can be collected from
actual interactions (in lab or in the field). Furthermore, this allows for a
classification method similar to the notion of fingerprinting presented
in~\cite{Ashlock2008}. Trained strategies can potentially be classified as
extortionate or not or it could be possible to even constrain the reinforcement
learning approaches that are becoming prevalent in the literature.
Alternatively, this mathematical approach for recognising extortion could be
used in sophisticated strategies to defend against invasion. Arguably, some of
the strategies considered here exhibit this behaviour, indeed as described
in~\cite{Harper2017}, the top ranking strategies in the full tournament are
obtained using evolutionary reinforcement learning techniques, thus, suspicion
of extortionate behaviour could in fact be an evolutionary trait.

\section*{Acknowledgements}

The following open source software libraries were used in this research:

\begin{itemize}
    \item The Axelrod ~\cite{Knight2016, Knight2018} library (IPD strategies and
        tournaments).
    \item The sympy library~\cite{Meurer2017} (verification of all symbolic
        calculations).
    \item The matplotlib~\cite{Droettboom2018} library (visualisation).
    \item The pandas~\cite{Structures2010}, dask~\cite{Dask2016} and
        NumPy~\cite{Oliphant2015} libraries (data manipulation).
    \item The SciPy~\cite{Jones2001} library (numerical integration of the
        replicator equation).
\end{itemize}

This work was performed using the computational facilities of the Advanced
Research Computing @ Cardiff (ARCCA) Division, Cardiff University.

\printbibliography

\newpage
\section*{Supplementary materials}

\includepdf{assets/pdf/proof_of_form_of_extortionate_strategies/main.pdf}

\newpage

Using the pair wise interactions the transition rates \(p,
q\) can be measured and the steady state probabilities inferred and compared to
the actual probabilities of each state.
This is done numerically by computing the singular eigenvector of the
matrix \(A\) \cite{Stewart2009}:

\[
    A =
    \begin{bmatrix}
        p_1 q_1 & p_1 (1 - q_1) & (1 - p_1) q_1 & (1 -p_1) (1 - q_1) \\
        p_2 q_2 & p_2 (1 - q_2) & (1 - p_2) q_2 & (1 -p_2) (1 - q_2) \\
        p_3 q_3 & p_3 (1 - q_3) & (1 - p_3) q_3 & (1 -p_3) (1 - q_3) \\
        p_4 q_4 & p_4 (1 - q_4) & (1 - p_4) q_4 & (1 -p_4) (1 - q_4) \\
    \end{bmatrix}
\]

Figure~\ref{fig:computed_probabilities_vs_theoretic_probabilities} shows a
regression line fitted to every pairwise interaction with a reported
\(\text{SSError}\) value (pairwise interactions with missing states were
omitted). This serves to validate the approach: a part from some edge cases the
relationship is consistent.

\begin{figure}[!htbp]
    \centering
    \includegraphics[width=.8\textwidth]{./assets/img/computed_probabilities_vs_theoretic_probabilities/main.pdf}
    \caption{The
        relationship between the steady state probabilities inferred from the
        measured transitions and the actual steady state probabilities. A linear
        regression line is included validating the approach.}
    \label{fig:computed_probabilities_vs_theoretic_probabilities}
\end{figure}


\end{document}

    strategies is considered. In this setting
    the most highly performing strategies do not play in an extortionate way
    against each other but do against lower performing strategies.
    This suggests that whilst the theory of Zero Determinant strategies
    indicates that memory is not of fundamental importance to the evolution of
    cooperative behaviour, this is incomplete.
\end{abstract}

\section{Introduction}\label{sec:introduction}

Agent based game theoretic models have become a stalwart of the underpinning
mathematics of interactive behaviours. One of the major pieces of work
in this area is the pair of original computer tournaments run by Robert
Axelrod~\cite{Axelrod1980, Axelrod1980a}. These tournaments pitted submitted
computer strategies against each other in plays of the Iterated Prisoner's
Dilemma. A common game where agents can choose to pay a slight cost to their
immediate utility in the hope of building a reputation. This has been used in
economic and evolutionary game theory to understand the evolution of cooperative
behaviour.

Recently, a class of strategies was described in~\cite{Press2012} that can
provably extort any given opponent. In~\cite{Hilbe2013, Moran1707} some
questions have already been asked about the true effectiveness of these
strategies in an evolutionary setting. Here another question is asked: is it
possible to recognise this extortionate behaviour? A mathematical procedure for
suspicion is presented: in the same way that the continued actions of an
extortionate individual might raise suspicion.

This work makes use of the Axelrod Python library~\cite{Knight2018, Knight2016}
with a large number of Prisoner Dilemma strategies available to give an
extensive numerical example of the ideas presented.  The approach is presented
in Section~\ref{sec:delta-zd-strategies}.  All of the code and data discussed
in Section~\ref{sec:numerical-experiments} is open sourced, archived and
written according to best scientific principles~\cite{Wilson2014}. The data
archive can be found at~\cite{vincent_knight_2018_1297075}.

\section{Recognising Extortion}\label{sec:delta-zd-strategies}

In~\cite{Press2012}, given a match between 2 memory-one strategies, the concept
of Zero Determinant (ZD) strategies is introduced. The main result of that paper
shows that given two memory one players \(p, q\in\mathbb{R}^4\) a linear
relationship between the players' scores could be forced by one of the players.

Using the notation of~\cite{Press2012}, assuming the utilities for player \(p\)
are given by \(S_x=(R, S, T, P)\) and for player \(q\) by \(S_y=(R, T, S, P)\)
and that the stationary scores of each player is given by \(S_X\) and \(S_Y\)
respectively. The main result of~\cite{Press2012} is that if

\begin{equation}\label{eqn:linear_relationship_for_p}
    \tilde p=\alpha S_x + \beta S_y + \gamma
\end{equation}

or

\begin{equation}\label{eqn:linear_relationship_for_q}
    \tilde q=\alpha S_x + \beta S_y + \gamma
\end{equation}

where \(\tilde p = (1 - p_1, 1 - p_2, p_3, p_4)\) and
\(\tilde q = (1 - q_1, 1 - q_2, q_3, q_4)\) then:

\begin{equation}
    \alpha S_X + \beta S_Y + \gamma = 0
\end{equation}

In~\cite{Press2012} a particular type of ZD strategy is defined: extortionate
strategies. If:

\begin{equation}\label{eqn:constraint_for_extortion}
    \gamma = - P(\alpha + \beta)
\end{equation}

then the player can ensure they get a score \(\chi\) times
larger than the opponent. This extortion coefficient is given by:

\begin{equation}\label{eqn:definition_of_chi}
    \chi=\frac{-\beta}{\alpha}
\end{equation}

Thus, if (\ref{eqn:constraint_for_extortion}) holds and \(\chi >1\) a player is
said to extort their opponent.
Here, the reverse problem is considered: given a
\(p\in\mathbb{R}^4\) how does one identify \(\alpha, \beta\) if they
exist and is the strategy in fact acting in an extortionate way?

These conditions correspond to:

\begin{align}
    \tilde p_1 & = \alpha R + \beta R - P (\alpha + \beta)
            \label{eqn:condition_for_tilde_p1}\\
    \tilde p_2 & = \alpha S + \beta T - P (\alpha + \beta)
            \label{eqn:condition_for_tilde_p2}\\
    \tilde p_3 & = \alpha T + \beta S - P (\alpha + \beta)
            \label{eqn:condition_for_tilde_p3}\\
    \tilde p_4 & = \alpha P + \beta P - P (\alpha + \beta)
            \label{eqn:condition_for_tilde_p4}
\end{align}

Equation (\ref{eqn:condition_for_tilde_p4}) ensures that \(p_4=\tilde p_4=0\).
Equations (\ref{eqn:condition_for_tilde_p1}-\ref{eqn:condition_for_tilde_p3})
can be used to eliminate \(\alpha, \beta\), giving:

\begin{equation}\label{eqn:planar_definition_of_extortion}
    \tilde p_1 = \frac{(R - P)(\tilde p_2 + \tilde p_3)}{S + T - 2P}
\end{equation}

with:

\begin{equation}\label{eqn:definition_of_chi}
    \chi = \frac{\tilde p_2 (P - T) + \tilde p_3 (S - P)}
                {\tilde p_2 (P - S) + \tilde p_3 (T - P)}
\end{equation}

Given a strategy \(p\in\mathbb{R}^{4\times 1}\) equations
(\ref{eqn:condition_for_tilde_p4}), (\ref{eqn:planar_definition_of_extortion}-\ref{eqn:definition_of_chi}) can be used to check if
a strategy is extortionate. The conditions correspond to:

\begin{align}
    p_1 & = \frac{(R-P)(p_2 + p_3) - R + T + S - P}{S + T - 2P}
     \label{eqn:condition_for_p1}\\
    p_4 & = 0 \label{eqn:condition_for_p4}\\
    1 & > p_2 + p_3\label{eqn:condition_for_chi}
\end{align}

The algebraic steps necessary to prove these results are available in the
supporting materials.

All extortionate strategies reside on a triangular (\ref{eqn:condition_for_chi})
plane (\ref{eqn:condition_for_p1}) in 3 dimensions (\ref{eqn:condition_for_p4}).
Using this formulation it can be seen that a necessary (but not sufficient)
condition for an extortionate strategy is that it cooperates on average less
than 50\% of the time when in a state of disagreement with the opponent.

As an example, consider the known extortionate strategy \(p=(8 / 9, 1 / 2, 1 /
3, 0)\) from~\cite{Stewart2012} which is referred to as \texttt{Extort-2}. In
this case, for the standard values of \((R, T, S, P)\) constraint
(\ref{eqn:condition_for_p1}) corresponds to:

\begin{equation}
    p_1 = \frac{2(p_2 + p_3) + 1}{3}
\end{equation}

It is clear that in this case all constraints hold.

This approach could in fact be used to confirm that a given strategy is acting
in an extortionate manner even if it is not a memory one strategy. However, in
practice, if a closed form for \(p\) is not known, then due to measurement
and/or numerical error this would not work.

This problem can be written in the following linear algebraic form where
\(x=(\alpha, \beta)\)
and \(p^*=(\tilde p_1 - 1, tilde_2 - 1, p_3)\):

\begin{equation}\label{eqn:linear_algebraic_equation_for_p}
    Cx= p^*
\end{equation}

\(C\) corresponds to equations
(\ref{eqn:condition_for_tilde_p1}-\ref{eqn:condition_for_tilde_p3}) and is
given by:

\begin{equation}\label{eqn:definition_of_C}
    C =
    \begin{bmatrix}
        R - P & R- P \\
        S - P & T- P \\
        T - P & S- P \\
    \end{bmatrix}
\end{equation}

Note that in general, equation (\ref{eqn:linear_algebraic_equation_for_p}) will
not necessarily have a solution. From the Rouch\'{e}-Capelli theorem if there is
a solution it is unique as \(\text{rank}(C)=2\) which is the dimension of the
variable \(x\). The best fitting \(x\) is found by minimizing:

\begin{equation}\label{eqn:r_squared}
    \text{SSError} = \|C x- p^*\|_2^2 = \sum_{i=1}^{3}\left((C\bar x)_i-p_i^*\right)^2
\end{equation}

Note that \(\text{SSError}\), which is the square of the Frobenius
norm~\cite{Golub2013}, becomes a measure of how close a strategy is to being an
extortionate strategy. Suspicion
of extortion then corresponds to a threshold on \(\text{SSError}\).

By observing interactions (human or otherwise), their memory one representation
can be inferred and this approach can be used to recognise extortionate
behaviour. The notion of comparing theoretic and actual plays of the IPD is not
novel, see for example~\cite{Rand2013}. Immediately it is noted that if the
environment is noisy~\cite{Wu1995} then no strategy can be considered to be
extortionate as \(p_4>0\).

In the next section, this idea will be illustrated by observing the interactions
that take place in a computer based tournament of the IPD\@.

\section{Numerical experiments}\label{sec:numerical-experiments}

In~\cite{Stewart2012} results from a tournament with
\documentclass[a4paper]{article}

\usepackage{amsmath}
\usepackage{amssymb}
\usepackage[margin=1.5cm,
            includefoot,
            footskip=30pt]{geometry}
\usepackage{layout}
\usepackage{graphicx}
\usepackage{subcaption}

\usepackage{biblatex}
\usepackage{pdfpages}

\bibliography{main.bib}

\title{Suspicion: Recognising and evaluating the effectiveness
       of extortion in the Iterated Prisoner's Dilemma}
\author{Vincent A. Knight \and Nikoleta E. Glynatsi}
\date{\today}



\begin{document}

\maketitle

\begin{abstract}
    The Iterated Prisoner's Dilemma is a model for rational and evolutionary
    interactive behaviour. It has applications both in the study of human social
    behaviour as well as in biology.
    It is used to understand when and how a rational individual might
    accept an immediate cost to their own utility for the direct benefit of
    another.

    Much attention has been given to a class of strategies called
    Zero Determinant strategies. It has been theoretically shown that these
    strategies can ``extort'' any player.

    In this work, an approach to identify if observed strategies are playing in
    an extortionate way is described. Furthermore, experimental analysis of
    a large tournament with \input{assets/tex/number_of_full_strategies/main.tex}
    strategies is considered. In this setting
    the most highly performing strategies do not play in an extortionate way
    against each other but do against lower performing strategies.
    This suggests that whilst the theory of Zero Determinant strategies
    indicates that memory is not of fundamental importance to the evolution of
    cooperative behaviour, this is incomplete.
\end{abstract}

\section{Introduction}\label{sec:introduction}

Agent based game theoretic models have become a stalwart of the underpinning
mathematics of interactive behaviours. One of the major pieces of work
in this area is the pair of original computer tournaments run by Robert
Axelrod~\cite{Axelrod1980, Axelrod1980a}. These tournaments pitted submitted
computer strategies against each other in plays of the Iterated Prisoner's
Dilemma. A common game where agents can choose to pay a slight cost to their
immediate utility in the hope of building a reputation. This has been used in
economic and evolutionary game theory to understand the evolution of cooperative
behaviour.

Recently, a class of strategies was described in~\cite{Press2012} that can
provably extort any given opponent. In~\cite{Hilbe2013, Moran1707} some
questions have already been asked about the true effectiveness of these
strategies in an evolutionary setting. Here another question is asked: is it
possible to recognise this extortionate behaviour? A mathematical procedure for
suspicion is presented: in the same way that the continued actions of an
extortionate individual might raise suspicion.

This work makes use of the Axelrod Python library~\cite{Knight2018, Knight2016}
with a large number of Prisoner Dilemma strategies available to give an
extensive numerical example of the ideas presented.  The approach is presented
in Section~\ref{sec:delta-zd-strategies}.  All of the code and data discussed
in Section~\ref{sec:numerical-experiments} is open sourced, archived and
written according to best scientific principles~\cite{Wilson2014}. The data
archive can be found at~\cite{vincent_knight_2018_1297075}.

\section{Recognising Extortion}\label{sec:delta-zd-strategies}

In~\cite{Press2012}, given a match between 2 memory-one strategies, the concept
of Zero Determinant (ZD) strategies is introduced. The main result of that paper
shows that given two memory one players \(p, q\in\mathbb{R}^4\) a linear
relationship between the players' scores could be forced by one of the players.

Using the notation of~\cite{Press2012}, assuming the utilities for player \(p\)
are given by \(S_x=(R, S, T, P)\) and for player \(q\) by \(S_y=(R, T, S, P)\)
and that the stationary scores of each player is given by \(S_X\) and \(S_Y\)
respectively. The main result of~\cite{Press2012} is that if

\begin{equation}\label{eqn:linear_relationship_for_p}
    \tilde p=\alpha S_x + \beta S_y + \gamma
\end{equation}

or

\begin{equation}\label{eqn:linear_relationship_for_q}
    \tilde q=\alpha S_x + \beta S_y + \gamma
\end{equation}

where \(\tilde p = (1 - p_1, 1 - p_2, p_3, p_4)\) and
\(\tilde q = (1 - q_1, 1 - q_2, q_3, q_4)\) then:

\begin{equation}
    \alpha S_X + \beta S_Y + \gamma = 0
\end{equation}

In~\cite{Press2012} a particular type of ZD strategy is defined: extortionate
strategies. If:

\begin{equation}\label{eqn:constraint_for_extortion}
    \gamma = - P(\alpha + \beta)
\end{equation}

then the player can ensure they get a score \(\chi\) times
larger than the opponent. This extortion coefficient is given by:

\begin{equation}\label{eqn:definition_of_chi}
    \chi=\frac{-\beta}{\alpha}
\end{equation}

Thus, if (\ref{eqn:constraint_for_extortion}) holds and \(\chi >1\) a player is
said to extort their opponent.
Here, the reverse problem is considered: given a
\(p\in\mathbb{R}^4\) how does one identify \(\alpha, \beta\) if they
exist and is the strategy in fact acting in an extortionate way?

These conditions correspond to:

\begin{align}
    \tilde p_1 & = \alpha R + \beta R - P (\alpha + \beta)
            \label{eqn:condition_for_tilde_p1}\\
    \tilde p_2 & = \alpha S + \beta T - P (\alpha + \beta)
            \label{eqn:condition_for_tilde_p2}\\
    \tilde p_3 & = \alpha T + \beta S - P (\alpha + \beta)
            \label{eqn:condition_for_tilde_p3}\\
    \tilde p_4 & = \alpha P + \beta P - P (\alpha + \beta)
            \label{eqn:condition_for_tilde_p4}
\end{align}

Equation (\ref{eqn:condition_for_tilde_p4}) ensures that \(p_4=\tilde p_4=0\).
Equations (\ref{eqn:condition_for_tilde_p1}-\ref{eqn:condition_for_tilde_p3})
can be used to eliminate \(\alpha, \beta\), giving:

\begin{equation}\label{eqn:planar_definition_of_extortion}
    \tilde p_1 = \frac{(R - P)(\tilde p_2 + \tilde p_3)}{S + T - 2P}
\end{equation}

with:

\begin{equation}\label{eqn:definition_of_chi}
    \chi = \frac{\tilde p_2 (P - T) + \tilde p_3 (S - P)}
                {\tilde p_2 (P - S) + \tilde p_3 (T - P)}
\end{equation}

Given a strategy \(p\in\mathbb{R}^{4\times 1}\) equations
(\ref{eqn:condition_for_tilde_p4}), (\ref{eqn:planar_definition_of_extortion}-\ref{eqn:definition_of_chi}) can be used to check if
a strategy is extortionate. The conditions correspond to:

\begin{align}
    p_1 & = \frac{(R-P)(p_2 + p_3) - R + T + S - P}{S + T - 2P}
     \label{eqn:condition_for_p1}\\
    p_4 & = 0 \label{eqn:condition_for_p4}\\
    1 & > p_2 + p_3\label{eqn:condition_for_chi}
\end{align}

The algebraic steps necessary to prove these results are available in the
supporting materials.

All extortionate strategies reside on a triangular (\ref{eqn:condition_for_chi})
plane (\ref{eqn:condition_for_p1}) in 3 dimensions (\ref{eqn:condition_for_p4}).
Using this formulation it can be seen that a necessary (but not sufficient)
condition for an extortionate strategy is that it cooperates on average less
than 50\% of the time when in a state of disagreement with the opponent.

As an example, consider the known extortionate strategy \(p=(8 / 9, 1 / 2, 1 /
3, 0)\) from~\cite{Stewart2012} which is referred to as \texttt{Extort-2}. In
this case, for the standard values of \((R, T, S, P)\) constraint
(\ref{eqn:condition_for_p1}) corresponds to:

\begin{equation}
    p_1 = \frac{2(p_2 + p_3) + 1}{3}
\end{equation}

It is clear that in this case all constraints hold.

This approach could in fact be used to confirm that a given strategy is acting
in an extortionate manner even if it is not a memory one strategy. However, in
practice, if a closed form for \(p\) is not known, then due to measurement
and/or numerical error this would not work.

This problem can be written in the following linear algebraic form where
\(x=(\alpha, \beta)\)
and \(p^*=(\tilde p_1 - 1, tilde_2 - 1, p_3)\):

\begin{equation}\label{eqn:linear_algebraic_equation_for_p}
    Cx= p^*
\end{equation}

\(C\) corresponds to equations
(\ref{eqn:condition_for_tilde_p1}-\ref{eqn:condition_for_tilde_p3}) and is
given by:

\begin{equation}\label{eqn:definition_of_C}
    C =
    \begin{bmatrix}
        R - P & R- P \\
        S - P & T- P \\
        T - P & S- P \\
    \end{bmatrix}
\end{equation}

Note that in general, equation (\ref{eqn:linear_algebraic_equation_for_p}) will
not necessarily have a solution. From the Rouch\'{e}-Capelli theorem if there is
a solution it is unique as \(\text{rank}(C)=2\) which is the dimension of the
variable \(x\). The best fitting \(x\) is found by minimizing:

\begin{equation}\label{eqn:r_squared}
    \text{SSError} = \|C x- p^*\|_2^2 = \sum_{i=1}^{3}\left((C\bar x)_i-p_i^*\right)^2
\end{equation}

Note that \(\text{SSError}\), which is the square of the Frobenius
norm~\cite{Golub2013}, becomes a measure of how close a strategy is to being an
extortionate strategy. Suspicion
of extortion then corresponds to a threshold on \(\text{SSError}\).

By observing interactions (human or otherwise), their memory one representation
can be inferred and this approach can be used to recognise extortionate
behaviour. The notion of comparing theoretic and actual plays of the IPD is not
novel, see for example~\cite{Rand2013}. Immediately it is noted that if the
environment is noisy~\cite{Wu1995} then no strategy can be considered to be
extortionate as \(p_4>0\).

In the next section, this idea will be illustrated by observing the interactions
that take place in a computer based tournament of the IPD\@.

\section{Numerical experiments}\label{sec:numerical-experiments}

In~\cite{Stewart2012} results from a tournament with
\input{./assets/tex/number_of_stewart_plotkin_strategies/main.tex} strategies,
was presented with specific consideration given to ZD strategies. This
tournament is reproduced here using the Axelrod-Python
project~\cite{Knight2016}. To obtain a good measure of the corresponding
transition rates for each strategy all matches have been run for
\input{assets/tex/number_of_turns/main.tex} turns and every match has been
repeated \input{assets/tex/number_of_repetitions/main.tex} times. All of this
interaction data is available at~\cite{vincent_knight_2018_1297075}. A good
match between the inferred Markov chain and the state distribution of the actual
interactions has been verified. Data for this is presented in the supplementary
materials.

Figure~\ref{fig:SSError_overall_in_stewart_plotkin} shows the \(\text{SSError}\)
values for all the strategies in the tournament, as reported
in~\cite{Stewart2012} the extortionate strategy (which has an expected
\(\text{SSError}\) approximately 0) gains a large number of wins.

\begin{figure}[!htbp]
    \centering
    \includegraphics[width=.8\textwidth]{./assets/img/SSError_overall_in_stewart_plotkin/main.pdf}
    \caption{\(\text{SSError}\) and state probabilities for the strategies
        of~\cite{Stewart2012}, ordered both by number of wins and overall score.
        Note that \(P(DC)\) is not shown as it corresponds to the transpose of
        \(P(CD)\). Cooperator and Defector are omitted as they do not visit all
        the states.}
    \label{fig:SSError_overall_in_stewart_plotkin}
\end{figure}

Here, the work of~\cite{Stewart2012} is extended by investigating a tournament
with \input{assets/tex/number_of_full_strategies/main.tex}
strategies.

The results of this analysis are shown in
Figure~\ref{fig:SSError_and_probabilities_in_full}. The top ranking strategies
by number of wins seem to be extortionate (but not against all strategies) and
it can be seen that a small sub group of strategies achieve mutual defection.
All the top ranking strategies according to score achieve mutual cooperation and
do not extort each other, however they
\textbf{do} exhibit extortionate behaviour towards a number of the lower ranking
strategies.

\begin{figure}[!htbp]
    \centering
    \includegraphics[width=.8\textwidth]{./assets/img/SSError_and_probabilities_in_full/main.pdf}
    \caption{\(\text{SSError}\) for the strategies for the full tournament. Only
    strategy interactions for which \(p_4=0\) and \(\chi>1\) are displayed.}
    \label{fig:SSError_and_probabilities_in_full}
\end{figure}

\section{Conclusion}\label{sec:conclusion}

This work defines an approach to measure whether or not a player is playing a
strategy that corresponds to an extortionate strategy as defined
in~\cite{Press2012}: a mathematical model for suspicion. Indeed, all
extortionate strategies have been
 classified as lying on a triangular plane.
This rigorous classification fails to be robust to small measurement error, thus
a statistical approach is proposed.
This is done through a linear algebraic approach for approximating the solution
of a linear system. Using this, a large number of pairwise interactions is
simulated and in fact very few strategies are found to act extortionately.

The work of~\cite{Press2012}, whilst showing that a clever approach to taking
advantage of another memory one strategy exists: this is incomplete. Whilst the
elegance of this result is very attractive, just as the simplicity of the
victory of Tit For Tat in Axelrod's original tournaments was, it is incomplete.
Extortionate strategies achieve a high number of wins but they do not
achieve a high score which corresponds to the fitness landscape in an
evolutionary sense. From the large number of interactions a payoff matrix \(S\)
can be measured where \(S_{ij}\) denotes the score (using standard values of
\((R, S, T, P) = (3, 0, 5, 1)\)) of the \(i\)th strategy
against the \(j\)th strategy. Using this, the replicator equation
describes the evolution of the system based on a population density fitness
function:

\begin{equation}\label{eqn:replicator_dynamics}
    \frac{dx}{dt} = x(S-x^TS x)
\end{equation}

Equation (\ref{eqn:replicator_dynamics}) is solved numerically through an
integration technique described in~\cite{Petzold1983} and
Figure~\ref{fig:replicator_dynamics} shows the evolution of the distribution of
the system: the various strategies are ranked by scores. It is clear to see that
only the high ranking strategies survive the evolutionary process (in fact,
only \input{./assets/img/replicator_dynamics/main.tex}
have a final distribution greater than \(10 ^ {-2}\)). This confirms the
findings of~\cite{Moran1707} in which sophisticated strategies resist
evolutionary invasion of shorter memory strategies. Recalling
Figure~\ref{fig:SSError_and_probabilities_in_full} this demonstrates that:

\begin{itemize}
    \item Cooperation emerges through the evolutionary process: the high scoring
        strategies do not exhibit extortionate behaviour towards each other.
    \item Extortionate strategies do not survive the evolutionary process.
\end{itemize}

\begin{figure}[!htbp]
    \centering
    \includegraphics[width=.8\textwidth]{./assets/img/replicator_dynamics/main.pdf}
    \caption{Numerical simulation of the replicator equation
    (\ref{eqn:replicator_dynamics}): strategies are ordered by score, only the strategies with a high score survive the evolutionary process.}
    \label{fig:replicator_dynamics}
\end{figure}

This work can be used to classify plays of the IPD\@: data can be collected from
actual interactions (in lab or in the field). Furthermore, this allows for a
classification method similar to the notion of fingerprinting presented
in~\cite{Ashlock2008}. Trained strategies can potentially be classified as
extortionate or not or it could be possible to even constrain the reinforcement
learning approaches that are becoming prevalent in the literature.
Alternatively, this mathematical approach for recognising extortion could be
used in sophisticated strategies to defend against invasion. Arguably, some of
the strategies considered here exhibit this behaviour, indeed as described
in~\cite{Harper2017}, the top ranking strategies in the full tournament are
obtained using evolutionary reinforcement learning techniques, thus, suspicion
of extortionate behaviour could in fact be an evolutionary trait.

\section*{Acknowledgements}

The following open source software libraries were used in this research:

\begin{itemize}
    \item The Axelrod ~\cite{Knight2016, Knight2018} library (IPD strategies and
        tournaments).
    \item The sympy library~\cite{Meurer2017} (verification of all symbolic
        calculations).
    \item The matplotlib~\cite{Droettboom2018} library (visualisation).
    \item The pandas~\cite{Structures2010}, dask~\cite{Dask2016} and
        NumPy~\cite{Oliphant2015} libraries (data manipulation).
    \item The SciPy~\cite{Jones2001} library (numerical integration of the
        replicator equation).
\end{itemize}

This work was performed using the computational facilities of the Advanced
Research Computing @ Cardiff (ARCCA) Division, Cardiff University.

\printbibliography

\newpage
\section*{Supplementary materials}

\includepdf{assets/pdf/proof_of_form_of_extortionate_strategies/main.pdf}

\newpage

Using the pair wise interactions the transition rates \(p,
q\) can be measured and the steady state probabilities inferred and compared to
the actual probabilities of each state.
This is done numerically by computing the singular eigenvector of the
matrix \(A\) \cite{Stewart2009}:

\[
    A =
    \begin{bmatrix}
        p_1 q_1 & p_1 (1 - q_1) & (1 - p_1) q_1 & (1 -p_1) (1 - q_1) \\
        p_2 q_2 & p_2 (1 - q_2) & (1 - p_2) q_2 & (1 -p_2) (1 - q_2) \\
        p_3 q_3 & p_3 (1 - q_3) & (1 - p_3) q_3 & (1 -p_3) (1 - q_3) \\
        p_4 q_4 & p_4 (1 - q_4) & (1 - p_4) q_4 & (1 -p_4) (1 - q_4) \\
    \end{bmatrix}
\]

Figure~\ref{fig:computed_probabilities_vs_theoretic_probabilities} shows a
regression line fitted to every pairwise interaction with a reported
\(\text{SSError}\) value (pairwise interactions with missing states were
omitted). This serves to validate the approach: a part from some edge cases the
relationship is consistent.

\begin{figure}[!htbp]
    \centering
    \includegraphics[width=.8\textwidth]{./assets/img/computed_probabilities_vs_theoretic_probabilities/main.pdf}
    \caption{The
        relationship between the steady state probabilities inferred from the
        measured transitions and the actual steady state probabilities. A linear
        regression line is included validating the approach.}
    \label{fig:computed_probabilities_vs_theoretic_probabilities}
\end{figure}


\end{document}
 strategies,
was presented with specific consideration given to ZD strategies. This
tournament is reproduced here using the Axelrod-Python
project~\cite{Knight2016}. To obtain a good measure of the corresponding
transition rates for each strategy all matches have been run for
\documentclass[a4paper]{article}

\usepackage{amsmath}
\usepackage{amssymb}
\usepackage[margin=1.5cm,
            includefoot,
            footskip=30pt]{geometry}
\usepackage{layout}
\usepackage{graphicx}
\usepackage{subcaption}

\usepackage{biblatex}
\usepackage{pdfpages}

\bibliography{main.bib}

\title{Suspicion: Recognising and evaluating the effectiveness
       of extortion in the Iterated Prisoner's Dilemma}
\author{Vincent A. Knight \and Nikoleta E. Glynatsi}
\date{\today}



\begin{document}

\maketitle

\begin{abstract}
    The Iterated Prisoner's Dilemma is a model for rational and evolutionary
    interactive behaviour. It has applications both in the study of human social
    behaviour as well as in biology.
    It is used to understand when and how a rational individual might
    accept an immediate cost to their own utility for the direct benefit of
    another.

    Much attention has been given to a class of strategies called
    Zero Determinant strategies. It has been theoretically shown that these
    strategies can ``extort'' any player.

    In this work, an approach to identify if observed strategies are playing in
    an extortionate way is described. Furthermore, experimental analysis of
    a large tournament with \input{assets/tex/number_of_full_strategies/main.tex}
    strategies is considered. In this setting
    the most highly performing strategies do not play in an extortionate way
    against each other but do against lower performing strategies.
    This suggests that whilst the theory of Zero Determinant strategies
    indicates that memory is not of fundamental importance to the evolution of
    cooperative behaviour, this is incomplete.
\end{abstract}

\section{Introduction}\label{sec:introduction}

Agent based game theoretic models have become a stalwart of the underpinning
mathematics of interactive behaviours. One of the major pieces of work
in this area is the pair of original computer tournaments run by Robert
Axelrod~\cite{Axelrod1980, Axelrod1980a}. These tournaments pitted submitted
computer strategies against each other in plays of the Iterated Prisoner's
Dilemma. A common game where agents can choose to pay a slight cost to their
immediate utility in the hope of building a reputation. This has been used in
economic and evolutionary game theory to understand the evolution of cooperative
behaviour.

Recently, a class of strategies was described in~\cite{Press2012} that can
provably extort any given opponent. In~\cite{Hilbe2013, Moran1707} some
questions have already been asked about the true effectiveness of these
strategies in an evolutionary setting. Here another question is asked: is it
possible to recognise this extortionate behaviour? A mathematical procedure for
suspicion is presented: in the same way that the continued actions of an
extortionate individual might raise suspicion.

This work makes use of the Axelrod Python library~\cite{Knight2018, Knight2016}
with a large number of Prisoner Dilemma strategies available to give an
extensive numerical example of the ideas presented.  The approach is presented
in Section~\ref{sec:delta-zd-strategies}.  All of the code and data discussed
in Section~\ref{sec:numerical-experiments} is open sourced, archived and
written according to best scientific principles~\cite{Wilson2014}. The data
archive can be found at~\cite{vincent_knight_2018_1297075}.

\section{Recognising Extortion}\label{sec:delta-zd-strategies}

In~\cite{Press2012}, given a match between 2 memory-one strategies, the concept
of Zero Determinant (ZD) strategies is introduced. The main result of that paper
shows that given two memory one players \(p, q\in\mathbb{R}^4\) a linear
relationship between the players' scores could be forced by one of the players.

Using the notation of~\cite{Press2012}, assuming the utilities for player \(p\)
are given by \(S_x=(R, S, T, P)\) and for player \(q\) by \(S_y=(R, T, S, P)\)
and that the stationary scores of each player is given by \(S_X\) and \(S_Y\)
respectively. The main result of~\cite{Press2012} is that if

\begin{equation}\label{eqn:linear_relationship_for_p}
    \tilde p=\alpha S_x + \beta S_y + \gamma
\end{equation}

or

\begin{equation}\label{eqn:linear_relationship_for_q}
    \tilde q=\alpha S_x + \beta S_y + \gamma
\end{equation}

where \(\tilde p = (1 - p_1, 1 - p_2, p_3, p_4)\) and
\(\tilde q = (1 - q_1, 1 - q_2, q_3, q_4)\) then:

\begin{equation}
    \alpha S_X + \beta S_Y + \gamma = 0
\end{equation}

In~\cite{Press2012} a particular type of ZD strategy is defined: extortionate
strategies. If:

\begin{equation}\label{eqn:constraint_for_extortion}
    \gamma = - P(\alpha + \beta)
\end{equation}

then the player can ensure they get a score \(\chi\) times
larger than the opponent. This extortion coefficient is given by:

\begin{equation}\label{eqn:definition_of_chi}
    \chi=\frac{-\beta}{\alpha}
\end{equation}

Thus, if (\ref{eqn:constraint_for_extortion}) holds and \(\chi >1\) a player is
said to extort their opponent.
Here, the reverse problem is considered: given a
\(p\in\mathbb{R}^4\) how does one identify \(\alpha, \beta\) if they
exist and is the strategy in fact acting in an extortionate way?

These conditions correspond to:

\begin{align}
    \tilde p_1 & = \alpha R + \beta R - P (\alpha + \beta)
            \label{eqn:condition_for_tilde_p1}\\
    \tilde p_2 & = \alpha S + \beta T - P (\alpha + \beta)
            \label{eqn:condition_for_tilde_p2}\\
    \tilde p_3 & = \alpha T + \beta S - P (\alpha + \beta)
            \label{eqn:condition_for_tilde_p3}\\
    \tilde p_4 & = \alpha P + \beta P - P (\alpha + \beta)
            \label{eqn:condition_for_tilde_p4}
\end{align}

Equation (\ref{eqn:condition_for_tilde_p4}) ensures that \(p_4=\tilde p_4=0\).
Equations (\ref{eqn:condition_for_tilde_p1}-\ref{eqn:condition_for_tilde_p3})
can be used to eliminate \(\alpha, \beta\), giving:

\begin{equation}\label{eqn:planar_definition_of_extortion}
    \tilde p_1 = \frac{(R - P)(\tilde p_2 + \tilde p_3)}{S + T - 2P}
\end{equation}

with:

\begin{equation}\label{eqn:definition_of_chi}
    \chi = \frac{\tilde p_2 (P - T) + \tilde p_3 (S - P)}
                {\tilde p_2 (P - S) + \tilde p_3 (T - P)}
\end{equation}

Given a strategy \(p\in\mathbb{R}^{4\times 1}\) equations
(\ref{eqn:condition_for_tilde_p4}), (\ref{eqn:planar_definition_of_extortion}-\ref{eqn:definition_of_chi}) can be used to check if
a strategy is extortionate. The conditions correspond to:

\begin{align}
    p_1 & = \frac{(R-P)(p_2 + p_3) - R + T + S - P}{S + T - 2P}
     \label{eqn:condition_for_p1}\\
    p_4 & = 0 \label{eqn:condition_for_p4}\\
    1 & > p_2 + p_3\label{eqn:condition_for_chi}
\end{align}

The algebraic steps necessary to prove these results are available in the
supporting materials.

All extortionate strategies reside on a triangular (\ref{eqn:condition_for_chi})
plane (\ref{eqn:condition_for_p1}) in 3 dimensions (\ref{eqn:condition_for_p4}).
Using this formulation it can be seen that a necessary (but not sufficient)
condition for an extortionate strategy is that it cooperates on average less
than 50\% of the time when in a state of disagreement with the opponent.

As an example, consider the known extortionate strategy \(p=(8 / 9, 1 / 2, 1 /
3, 0)\) from~\cite{Stewart2012} which is referred to as \texttt{Extort-2}. In
this case, for the standard values of \((R, T, S, P)\) constraint
(\ref{eqn:condition_for_p1}) corresponds to:

\begin{equation}
    p_1 = \frac{2(p_2 + p_3) + 1}{3}
\end{equation}

It is clear that in this case all constraints hold.

This approach could in fact be used to confirm that a given strategy is acting
in an extortionate manner even if it is not a memory one strategy. However, in
practice, if a closed form for \(p\) is not known, then due to measurement
and/or numerical error this would not work.

This problem can be written in the following linear algebraic form where
\(x=(\alpha, \beta)\)
and \(p^*=(\tilde p_1 - 1, tilde_2 - 1, p_3)\):

\begin{equation}\label{eqn:linear_algebraic_equation_for_p}
    Cx= p^*
\end{equation}

\(C\) corresponds to equations
(\ref{eqn:condition_for_tilde_p1}-\ref{eqn:condition_for_tilde_p3}) and is
given by:

\begin{equation}\label{eqn:definition_of_C}
    C =
    \begin{bmatrix}
        R - P & R- P \\
        S - P & T- P \\
        T - P & S- P \\
    \end{bmatrix}
\end{equation}

Note that in general, equation (\ref{eqn:linear_algebraic_equation_for_p}) will
not necessarily have a solution. From the Rouch\'{e}-Capelli theorem if there is
a solution it is unique as \(\text{rank}(C)=2\) which is the dimension of the
variable \(x\). The best fitting \(x\) is found by minimizing:

\begin{equation}\label{eqn:r_squared}
    \text{SSError} = \|C x- p^*\|_2^2 = \sum_{i=1}^{3}\left((C\bar x)_i-p_i^*\right)^2
\end{equation}

Note that \(\text{SSError}\), which is the square of the Frobenius
norm~\cite{Golub2013}, becomes a measure of how close a strategy is to being an
extortionate strategy. Suspicion
of extortion then corresponds to a threshold on \(\text{SSError}\).

By observing interactions (human or otherwise), their memory one representation
can be inferred and this approach can be used to recognise extortionate
behaviour. The notion of comparing theoretic and actual plays of the IPD is not
novel, see for example~\cite{Rand2013}. Immediately it is noted that if the
environment is noisy~\cite{Wu1995} then no strategy can be considered to be
extortionate as \(p_4>0\).

In the next section, this idea will be illustrated by observing the interactions
that take place in a computer based tournament of the IPD\@.

\section{Numerical experiments}\label{sec:numerical-experiments}

In~\cite{Stewart2012} results from a tournament with
\input{./assets/tex/number_of_stewart_plotkin_strategies/main.tex} strategies,
was presented with specific consideration given to ZD strategies. This
tournament is reproduced here using the Axelrod-Python
project~\cite{Knight2016}. To obtain a good measure of the corresponding
transition rates for each strategy all matches have been run for
\input{assets/tex/number_of_turns/main.tex} turns and every match has been
repeated \input{assets/tex/number_of_repetitions/main.tex} times. All of this
interaction data is available at~\cite{vincent_knight_2018_1297075}. A good
match between the inferred Markov chain and the state distribution of the actual
interactions has been verified. Data for this is presented in the supplementary
materials.

Figure~\ref{fig:SSError_overall_in_stewart_plotkin} shows the \(\text{SSError}\)
values for all the strategies in the tournament, as reported
in~\cite{Stewart2012} the extortionate strategy (which has an expected
\(\text{SSError}\) approximately 0) gains a large number of wins.

\begin{figure}[!htbp]
    \centering
    \includegraphics[width=.8\textwidth]{./assets/img/SSError_overall_in_stewart_plotkin/main.pdf}
    \caption{\(\text{SSError}\) and state probabilities for the strategies
        of~\cite{Stewart2012}, ordered both by number of wins and overall score.
        Note that \(P(DC)\) is not shown as it corresponds to the transpose of
        \(P(CD)\). Cooperator and Defector are omitted as they do not visit all
        the states.}
    \label{fig:SSError_overall_in_stewart_plotkin}
\end{figure}

Here, the work of~\cite{Stewart2012} is extended by investigating a tournament
with \input{assets/tex/number_of_full_strategies/main.tex}
strategies.

The results of this analysis are shown in
Figure~\ref{fig:SSError_and_probabilities_in_full}. The top ranking strategies
by number of wins seem to be extortionate (but not against all strategies) and
it can be seen that a small sub group of strategies achieve mutual defection.
All the top ranking strategies according to score achieve mutual cooperation and
do not extort each other, however they
\textbf{do} exhibit extortionate behaviour towards a number of the lower ranking
strategies.

\begin{figure}[!htbp]
    \centering
    \includegraphics[width=.8\textwidth]{./assets/img/SSError_and_probabilities_in_full/main.pdf}
    \caption{\(\text{SSError}\) for the strategies for the full tournament. Only
    strategy interactions for which \(p_4=0\) and \(\chi>1\) are displayed.}
    \label{fig:SSError_and_probabilities_in_full}
\end{figure}

\section{Conclusion}\label{sec:conclusion}

This work defines an approach to measure whether or not a player is playing a
strategy that corresponds to an extortionate strategy as defined
in~\cite{Press2012}: a mathematical model for suspicion. Indeed, all
extortionate strategies have been
 classified as lying on a triangular plane.
This rigorous classification fails to be robust to small measurement error, thus
a statistical approach is proposed.
This is done through a linear algebraic approach for approximating the solution
of a linear system. Using this, a large number of pairwise interactions is
simulated and in fact very few strategies are found to act extortionately.

The work of~\cite{Press2012}, whilst showing that a clever approach to taking
advantage of another memory one strategy exists: this is incomplete. Whilst the
elegance of this result is very attractive, just as the simplicity of the
victory of Tit For Tat in Axelrod's original tournaments was, it is incomplete.
Extortionate strategies achieve a high number of wins but they do not
achieve a high score which corresponds to the fitness landscape in an
evolutionary sense. From the large number of interactions a payoff matrix \(S\)
can be measured where \(S_{ij}\) denotes the score (using standard values of
\((R, S, T, P) = (3, 0, 5, 1)\)) of the \(i\)th strategy
against the \(j\)th strategy. Using this, the replicator equation
describes the evolution of the system based on a population density fitness
function:

\begin{equation}\label{eqn:replicator_dynamics}
    \frac{dx}{dt} = x(S-x^TS x)
\end{equation}

Equation (\ref{eqn:replicator_dynamics}) is solved numerically through an
integration technique described in~\cite{Petzold1983} and
Figure~\ref{fig:replicator_dynamics} shows the evolution of the distribution of
the system: the various strategies are ranked by scores. It is clear to see that
only the high ranking strategies survive the evolutionary process (in fact,
only \input{./assets/img/replicator_dynamics/main.tex}
have a final distribution greater than \(10 ^ {-2}\)). This confirms the
findings of~\cite{Moran1707} in which sophisticated strategies resist
evolutionary invasion of shorter memory strategies. Recalling
Figure~\ref{fig:SSError_and_probabilities_in_full} this demonstrates that:

\begin{itemize}
    \item Cooperation emerges through the evolutionary process: the high scoring
        strategies do not exhibit extortionate behaviour towards each other.
    \item Extortionate strategies do not survive the evolutionary process.
\end{itemize}

\begin{figure}[!htbp]
    \centering
    \includegraphics[width=.8\textwidth]{./assets/img/replicator_dynamics/main.pdf}
    \caption{Numerical simulation of the replicator equation
    (\ref{eqn:replicator_dynamics}): strategies are ordered by score, only the strategies with a high score survive the evolutionary process.}
    \label{fig:replicator_dynamics}
\end{figure}

This work can be used to classify plays of the IPD\@: data can be collected from
actual interactions (in lab or in the field). Furthermore, this allows for a
classification method similar to the notion of fingerprinting presented
in~\cite{Ashlock2008}. Trained strategies can potentially be classified as
extortionate or not or it could be possible to even constrain the reinforcement
learning approaches that are becoming prevalent in the literature.
Alternatively, this mathematical approach for recognising extortion could be
used in sophisticated strategies to defend against invasion. Arguably, some of
the strategies considered here exhibit this behaviour, indeed as described
in~\cite{Harper2017}, the top ranking strategies in the full tournament are
obtained using evolutionary reinforcement learning techniques, thus, suspicion
of extortionate behaviour could in fact be an evolutionary trait.

\section*{Acknowledgements}

The following open source software libraries were used in this research:

\begin{itemize}
    \item The Axelrod ~\cite{Knight2016, Knight2018} library (IPD strategies and
        tournaments).
    \item The sympy library~\cite{Meurer2017} (verification of all symbolic
        calculations).
    \item The matplotlib~\cite{Droettboom2018} library (visualisation).
    \item The pandas~\cite{Structures2010}, dask~\cite{Dask2016} and
        NumPy~\cite{Oliphant2015} libraries (data manipulation).
    \item The SciPy~\cite{Jones2001} library (numerical integration of the
        replicator equation).
\end{itemize}

This work was performed using the computational facilities of the Advanced
Research Computing @ Cardiff (ARCCA) Division, Cardiff University.

\printbibliography

\newpage
\section*{Supplementary materials}

\includepdf{assets/pdf/proof_of_form_of_extortionate_strategies/main.pdf}

\newpage

Using the pair wise interactions the transition rates \(p,
q\) can be measured and the steady state probabilities inferred and compared to
the actual probabilities of each state.
This is done numerically by computing the singular eigenvector of the
matrix \(A\) \cite{Stewart2009}:

\[
    A =
    \begin{bmatrix}
        p_1 q_1 & p_1 (1 - q_1) & (1 - p_1) q_1 & (1 -p_1) (1 - q_1) \\
        p_2 q_2 & p_2 (1 - q_2) & (1 - p_2) q_2 & (1 -p_2) (1 - q_2) \\
        p_3 q_3 & p_3 (1 - q_3) & (1 - p_3) q_3 & (1 -p_3) (1 - q_3) \\
        p_4 q_4 & p_4 (1 - q_4) & (1 - p_4) q_4 & (1 -p_4) (1 - q_4) \\
    \end{bmatrix}
\]

Figure~\ref{fig:computed_probabilities_vs_theoretic_probabilities} shows a
regression line fitted to every pairwise interaction with a reported
\(\text{SSError}\) value (pairwise interactions with missing states were
omitted). This serves to validate the approach: a part from some edge cases the
relationship is consistent.

\begin{figure}[!htbp]
    \centering
    \includegraphics[width=.8\textwidth]{./assets/img/computed_probabilities_vs_theoretic_probabilities/main.pdf}
    \caption{The
        relationship between the steady state probabilities inferred from the
        measured transitions and the actual steady state probabilities. A linear
        regression line is included validating the approach.}
    \label{fig:computed_probabilities_vs_theoretic_probabilities}
\end{figure}


\end{document}
 turns and every match has been
repeated \documentclass[a4paper]{article}

\usepackage{amsmath}
\usepackage{amssymb}
\usepackage[margin=1.5cm,
            includefoot,
            footskip=30pt]{geometry}
\usepackage{layout}
\usepackage{graphicx}
\usepackage{subcaption}

\usepackage{biblatex}
\usepackage{pdfpages}

\bibliography{main.bib}

\title{Suspicion: Recognising and evaluating the effectiveness
       of extortion in the Iterated Prisoner's Dilemma}
\author{Vincent A. Knight \and Nikoleta E. Glynatsi}
\date{\today}



\begin{document}

\maketitle

\begin{abstract}
    The Iterated Prisoner's Dilemma is a model for rational and evolutionary
    interactive behaviour. It has applications both in the study of human social
    behaviour as well as in biology.
    It is used to understand when and how a rational individual might
    accept an immediate cost to their own utility for the direct benefit of
    another.

    Much attention has been given to a class of strategies called
    Zero Determinant strategies. It has been theoretically shown that these
    strategies can ``extort'' any player.

    In this work, an approach to identify if observed strategies are playing in
    an extortionate way is described. Furthermore, experimental analysis of
    a large tournament with \input{assets/tex/number_of_full_strategies/main.tex}
    strategies is considered. In this setting
    the most highly performing strategies do not play in an extortionate way
    against each other but do against lower performing strategies.
    This suggests that whilst the theory of Zero Determinant strategies
    indicates that memory is not of fundamental importance to the evolution of
    cooperative behaviour, this is incomplete.
\end{abstract}

\section{Introduction}\label{sec:introduction}

Agent based game theoretic models have become a stalwart of the underpinning
mathematics of interactive behaviours. One of the major pieces of work
in this area is the pair of original computer tournaments run by Robert
Axelrod~\cite{Axelrod1980, Axelrod1980a}. These tournaments pitted submitted
computer strategies against each other in plays of the Iterated Prisoner's
Dilemma. A common game where agents can choose to pay a slight cost to their
immediate utility in the hope of building a reputation. This has been used in
economic and evolutionary game theory to understand the evolution of cooperative
behaviour.

Recently, a class of strategies was described in~\cite{Press2012} that can
provably extort any given opponent. In~\cite{Hilbe2013, Moran1707} some
questions have already been asked about the true effectiveness of these
strategies in an evolutionary setting. Here another question is asked: is it
possible to recognise this extortionate behaviour? A mathematical procedure for
suspicion is presented: in the same way that the continued actions of an
extortionate individual might raise suspicion.

This work makes use of the Axelrod Python library~\cite{Knight2018, Knight2016}
with a large number of Prisoner Dilemma strategies available to give an
extensive numerical example of the ideas presented.  The approach is presented
in Section~\ref{sec:delta-zd-strategies}.  All of the code and data discussed
in Section~\ref{sec:numerical-experiments} is open sourced, archived and
written according to best scientific principles~\cite{Wilson2014}. The data
archive can be found at~\cite{vincent_knight_2018_1297075}.

\section{Recognising Extortion}\label{sec:delta-zd-strategies}

In~\cite{Press2012}, given a match between 2 memory-one strategies, the concept
of Zero Determinant (ZD) strategies is introduced. The main result of that paper
shows that given two memory one players \(p, q\in\mathbb{R}^4\) a linear
relationship between the players' scores could be forced by one of the players.

Using the notation of~\cite{Press2012}, assuming the utilities for player \(p\)
are given by \(S_x=(R, S, T, P)\) and for player \(q\) by \(S_y=(R, T, S, P)\)
and that the stationary scores of each player is given by \(S_X\) and \(S_Y\)
respectively. The main result of~\cite{Press2012} is that if

\begin{equation}\label{eqn:linear_relationship_for_p}
    \tilde p=\alpha S_x + \beta S_y + \gamma
\end{equation}

or

\begin{equation}\label{eqn:linear_relationship_for_q}
    \tilde q=\alpha S_x + \beta S_y + \gamma
\end{equation}

where \(\tilde p = (1 - p_1, 1 - p_2, p_3, p_4)\) and
\(\tilde q = (1 - q_1, 1 - q_2, q_3, q_4)\) then:

\begin{equation}
    \alpha S_X + \beta S_Y + \gamma = 0
\end{equation}

In~\cite{Press2012} a particular type of ZD strategy is defined: extortionate
strategies. If:

\begin{equation}\label{eqn:constraint_for_extortion}
    \gamma = - P(\alpha + \beta)
\end{equation}

then the player can ensure they get a score \(\chi\) times
larger than the opponent. This extortion coefficient is given by:

\begin{equation}\label{eqn:definition_of_chi}
    \chi=\frac{-\beta}{\alpha}
\end{equation}

Thus, if (\ref{eqn:constraint_for_extortion}) holds and \(\chi >1\) a player is
said to extort their opponent.
Here, the reverse problem is considered: given a
\(p\in\mathbb{R}^4\) how does one identify \(\alpha, \beta\) if they
exist and is the strategy in fact acting in an extortionate way?

These conditions correspond to:

\begin{align}
    \tilde p_1 & = \alpha R + \beta R - P (\alpha + \beta)
            \label{eqn:condition_for_tilde_p1}\\
    \tilde p_2 & = \alpha S + \beta T - P (\alpha + \beta)
            \label{eqn:condition_for_tilde_p2}\\
    \tilde p_3 & = \alpha T + \beta S - P (\alpha + \beta)
            \label{eqn:condition_for_tilde_p3}\\
    \tilde p_4 & = \alpha P + \beta P - P (\alpha + \beta)
            \label{eqn:condition_for_tilde_p4}
\end{align}

Equation (\ref{eqn:condition_for_tilde_p4}) ensures that \(p_4=\tilde p_4=0\).
Equations (\ref{eqn:condition_for_tilde_p1}-\ref{eqn:condition_for_tilde_p3})
can be used to eliminate \(\alpha, \beta\), giving:

\begin{equation}\label{eqn:planar_definition_of_extortion}
    \tilde p_1 = \frac{(R - P)(\tilde p_2 + \tilde p_3)}{S + T - 2P}
\end{equation}

with:

\begin{equation}\label{eqn:definition_of_chi}
    \chi = \frac{\tilde p_2 (P - T) + \tilde p_3 (S - P)}
                {\tilde p_2 (P - S) + \tilde p_3 (T - P)}
\end{equation}

Given a strategy \(p\in\mathbb{R}^{4\times 1}\) equations
(\ref{eqn:condition_for_tilde_p4}), (\ref{eqn:planar_definition_of_extortion}-\ref{eqn:definition_of_chi}) can be used to check if
a strategy is extortionate. The conditions correspond to:

\begin{align}
    p_1 & = \frac{(R-P)(p_2 + p_3) - R + T + S - P}{S + T - 2P}
     \label{eqn:condition_for_p1}\\
    p_4 & = 0 \label{eqn:condition_for_p4}\\
    1 & > p_2 + p_3\label{eqn:condition_for_chi}
\end{align}

The algebraic steps necessary to prove these results are available in the
supporting materials.

All extortionate strategies reside on a triangular (\ref{eqn:condition_for_chi})
plane (\ref{eqn:condition_for_p1}) in 3 dimensions (\ref{eqn:condition_for_p4}).
Using this formulation it can be seen that a necessary (but not sufficient)
condition for an extortionate strategy is that it cooperates on average less
than 50\% of the time when in a state of disagreement with the opponent.

As an example, consider the known extortionate strategy \(p=(8 / 9, 1 / 2, 1 /
3, 0)\) from~\cite{Stewart2012} which is referred to as \texttt{Extort-2}. In
this case, for the standard values of \((R, T, S, P)\) constraint
(\ref{eqn:condition_for_p1}) corresponds to:

\begin{equation}
    p_1 = \frac{2(p_2 + p_3) + 1}{3}
\end{equation}

It is clear that in this case all constraints hold.

This approach could in fact be used to confirm that a given strategy is acting
in an extortionate manner even if it is not a memory one strategy. However, in
practice, if a closed form for \(p\) is not known, then due to measurement
and/or numerical error this would not work.

This problem can be written in the following linear algebraic form where
\(x=(\alpha, \beta)\)
and \(p^*=(\tilde p_1 - 1, tilde_2 - 1, p_3)\):

\begin{equation}\label{eqn:linear_algebraic_equation_for_p}
    Cx= p^*
\end{equation}

\(C\) corresponds to equations
(\ref{eqn:condition_for_tilde_p1}-\ref{eqn:condition_for_tilde_p3}) and is
given by:

\begin{equation}\label{eqn:definition_of_C}
    C =
    \begin{bmatrix}
        R - P & R- P \\
        S - P & T- P \\
        T - P & S- P \\
    \end{bmatrix}
\end{equation}

Note that in general, equation (\ref{eqn:linear_algebraic_equation_for_p}) will
not necessarily have a solution. From the Rouch\'{e}-Capelli theorem if there is
a solution it is unique as \(\text{rank}(C)=2\) which is the dimension of the
variable \(x\). The best fitting \(x\) is found by minimizing:

\begin{equation}\label{eqn:r_squared}
    \text{SSError} = \|C x- p^*\|_2^2 = \sum_{i=1}^{3}\left((C\bar x)_i-p_i^*\right)^2
\end{equation}

Note that \(\text{SSError}\), which is the square of the Frobenius
norm~\cite{Golub2013}, becomes a measure of how close a strategy is to being an
extortionate strategy. Suspicion
of extortion then corresponds to a threshold on \(\text{SSError}\).

By observing interactions (human or otherwise), their memory one representation
can be inferred and this approach can be used to recognise extortionate
behaviour. The notion of comparing theoretic and actual plays of the IPD is not
novel, see for example~\cite{Rand2013}. Immediately it is noted that if the
environment is noisy~\cite{Wu1995} then no strategy can be considered to be
extortionate as \(p_4>0\).

In the next section, this idea will be illustrated by observing the interactions
that take place in a computer based tournament of the IPD\@.

\section{Numerical experiments}\label{sec:numerical-experiments}

In~\cite{Stewart2012} results from a tournament with
\input{./assets/tex/number_of_stewart_plotkin_strategies/main.tex} strategies,
was presented with specific consideration given to ZD strategies. This
tournament is reproduced here using the Axelrod-Python
project~\cite{Knight2016}. To obtain a good measure of the corresponding
transition rates for each strategy all matches have been run for
\input{assets/tex/number_of_turns/main.tex} turns and every match has been
repeated \input{assets/tex/number_of_repetitions/main.tex} times. All of this
interaction data is available at~\cite{vincent_knight_2018_1297075}. A good
match between the inferred Markov chain and the state distribution of the actual
interactions has been verified. Data for this is presented in the supplementary
materials.

Figure~\ref{fig:SSError_overall_in_stewart_plotkin} shows the \(\text{SSError}\)
values for all the strategies in the tournament, as reported
in~\cite{Stewart2012} the extortionate strategy (which has an expected
\(\text{SSError}\) approximately 0) gains a large number of wins.

\begin{figure}[!htbp]
    \centering
    \includegraphics[width=.8\textwidth]{./assets/img/SSError_overall_in_stewart_plotkin/main.pdf}
    \caption{\(\text{SSError}\) and state probabilities for the strategies
        of~\cite{Stewart2012}, ordered both by number of wins and overall score.
        Note that \(P(DC)\) is not shown as it corresponds to the transpose of
        \(P(CD)\). Cooperator and Defector are omitted as they do not visit all
        the states.}
    \label{fig:SSError_overall_in_stewart_plotkin}
\end{figure}

Here, the work of~\cite{Stewart2012} is extended by investigating a tournament
with \input{assets/tex/number_of_full_strategies/main.tex}
strategies.

The results of this analysis are shown in
Figure~\ref{fig:SSError_and_probabilities_in_full}. The top ranking strategies
by number of wins seem to be extortionate (but not against all strategies) and
it can be seen that a small sub group of strategies achieve mutual defection.
All the top ranking strategies according to score achieve mutual cooperation and
do not extort each other, however they
\textbf{do} exhibit extortionate behaviour towards a number of the lower ranking
strategies.

\begin{figure}[!htbp]
    \centering
    \includegraphics[width=.8\textwidth]{./assets/img/SSError_and_probabilities_in_full/main.pdf}
    \caption{\(\text{SSError}\) for the strategies for the full tournament. Only
    strategy interactions for which \(p_4=0\) and \(\chi>1\) are displayed.}
    \label{fig:SSError_and_probabilities_in_full}
\end{figure}

\section{Conclusion}\label{sec:conclusion}

This work defines an approach to measure whether or not a player is playing a
strategy that corresponds to an extortionate strategy as defined
in~\cite{Press2012}: a mathematical model for suspicion. Indeed, all
extortionate strategies have been
 classified as lying on a triangular plane.
This rigorous classification fails to be robust to small measurement error, thus
a statistical approach is proposed.
This is done through a linear algebraic approach for approximating the solution
of a linear system. Using this, a large number of pairwise interactions is
simulated and in fact very few strategies are found to act extortionately.

The work of~\cite{Press2012}, whilst showing that a clever approach to taking
advantage of another memory one strategy exists: this is incomplete. Whilst the
elegance of this result is very attractive, just as the simplicity of the
victory of Tit For Tat in Axelrod's original tournaments was, it is incomplete.
Extortionate strategies achieve a high number of wins but they do not
achieve a high score which corresponds to the fitness landscape in an
evolutionary sense. From the large number of interactions a payoff matrix \(S\)
can be measured where \(S_{ij}\) denotes the score (using standard values of
\((R, S, T, P) = (3, 0, 5, 1)\)) of the \(i\)th strategy
against the \(j\)th strategy. Using this, the replicator equation
describes the evolution of the system based on a population density fitness
function:

\begin{equation}\label{eqn:replicator_dynamics}
    \frac{dx}{dt} = x(S-x^TS x)
\end{equation}

Equation (\ref{eqn:replicator_dynamics}) is solved numerically through an
integration technique described in~\cite{Petzold1983} and
Figure~\ref{fig:replicator_dynamics} shows the evolution of the distribution of
the system: the various strategies are ranked by scores. It is clear to see that
only the high ranking strategies survive the evolutionary process (in fact,
only \input{./assets/img/replicator_dynamics/main.tex}
have a final distribution greater than \(10 ^ {-2}\)). This confirms the
findings of~\cite{Moran1707} in which sophisticated strategies resist
evolutionary invasion of shorter memory strategies. Recalling
Figure~\ref{fig:SSError_and_probabilities_in_full} this demonstrates that:

\begin{itemize}
    \item Cooperation emerges through the evolutionary process: the high scoring
        strategies do not exhibit extortionate behaviour towards each other.
    \item Extortionate strategies do not survive the evolutionary process.
\end{itemize}

\begin{figure}[!htbp]
    \centering
    \includegraphics[width=.8\textwidth]{./assets/img/replicator_dynamics/main.pdf}
    \caption{Numerical simulation of the replicator equation
    (\ref{eqn:replicator_dynamics}): strategies are ordered by score, only the strategies with a high score survive the evolutionary process.}
    \label{fig:replicator_dynamics}
\end{figure}

This work can be used to classify plays of the IPD\@: data can be collected from
actual interactions (in lab or in the field). Furthermore, this allows for a
classification method similar to the notion of fingerprinting presented
in~\cite{Ashlock2008}. Trained strategies can potentially be classified as
extortionate or not or it could be possible to even constrain the reinforcement
learning approaches that are becoming prevalent in the literature.
Alternatively, this mathematical approach for recognising extortion could be
used in sophisticated strategies to defend against invasion. Arguably, some of
the strategies considered here exhibit this behaviour, indeed as described
in~\cite{Harper2017}, the top ranking strategies in the full tournament are
obtained using evolutionary reinforcement learning techniques, thus, suspicion
of extortionate behaviour could in fact be an evolutionary trait.

\section*{Acknowledgements}

The following open source software libraries were used in this research:

\begin{itemize}
    \item The Axelrod ~\cite{Knight2016, Knight2018} library (IPD strategies and
        tournaments).
    \item The sympy library~\cite{Meurer2017} (verification of all symbolic
        calculations).
    \item The matplotlib~\cite{Droettboom2018} library (visualisation).
    \item The pandas~\cite{Structures2010}, dask~\cite{Dask2016} and
        NumPy~\cite{Oliphant2015} libraries (data manipulation).
    \item The SciPy~\cite{Jones2001} library (numerical integration of the
        replicator equation).
\end{itemize}

This work was performed using the computational facilities of the Advanced
Research Computing @ Cardiff (ARCCA) Division, Cardiff University.

\printbibliography

\newpage
\section*{Supplementary materials}

\includepdf{assets/pdf/proof_of_form_of_extortionate_strategies/main.pdf}

\newpage

Using the pair wise interactions the transition rates \(p,
q\) can be measured and the steady state probabilities inferred and compared to
the actual probabilities of each state.
This is done numerically by computing the singular eigenvector of the
matrix \(A\) \cite{Stewart2009}:

\[
    A =
    \begin{bmatrix}
        p_1 q_1 & p_1 (1 - q_1) & (1 - p_1) q_1 & (1 -p_1) (1 - q_1) \\
        p_2 q_2 & p_2 (1 - q_2) & (1 - p_2) q_2 & (1 -p_2) (1 - q_2) \\
        p_3 q_3 & p_3 (1 - q_3) & (1 - p_3) q_3 & (1 -p_3) (1 - q_3) \\
        p_4 q_4 & p_4 (1 - q_4) & (1 - p_4) q_4 & (1 -p_4) (1 - q_4) \\
    \end{bmatrix}
\]

Figure~\ref{fig:computed_probabilities_vs_theoretic_probabilities} shows a
regression line fitted to every pairwise interaction with a reported
\(\text{SSError}\) value (pairwise interactions with missing states were
omitted). This serves to validate the approach: a part from some edge cases the
relationship is consistent.

\begin{figure}[!htbp]
    \centering
    \includegraphics[width=.8\textwidth]{./assets/img/computed_probabilities_vs_theoretic_probabilities/main.pdf}
    \caption{The
        relationship between the steady state probabilities inferred from the
        measured transitions and the actual steady state probabilities. A linear
        regression line is included validating the approach.}
    \label{fig:computed_probabilities_vs_theoretic_probabilities}
\end{figure}


\end{document}
 times. All of this
interaction data is available at~\cite{vincent_knight_2018_1297075}. A good
match between the inferred Markov chain and the state distribution of the actual
interactions has been verified. Data for this is presented in the supplementary
materials.

Figure~\ref{fig:SSError_overall_in_stewart_plotkin} shows the \(\text{SSError}\)
values for all the strategies in the tournament, as reported
in~\cite{Stewart2012} the extortionate strategy (which has an expected
\(\text{SSError}\) approximately 0) gains a large number of wins.

\begin{figure}[!htbp]
    \centering
    \includegraphics[width=.8\textwidth]{./assets/img/SSError_overall_in_stewart_plotkin/main.pdf}
    \caption{\(\text{SSError}\) and state probabilities for the strategies
        of~\cite{Stewart2012}, ordered both by number of wins and overall score.
        Note that \(P(DC)\) is not shown as it corresponds to the transpose of
        \(P(CD)\). Cooperator and Defector are omitted as they do not visit all
        the states.}
    \label{fig:SSError_overall_in_stewart_plotkin}
\end{figure}

Here, the work of~\cite{Stewart2012} is extended by investigating a tournament
with \documentclass[a4paper]{article}

\usepackage{amsmath}
\usepackage{amssymb}
\usepackage[margin=1.5cm,
            includefoot,
            footskip=30pt]{geometry}
\usepackage{layout}
\usepackage{graphicx}
\usepackage{subcaption}

\usepackage{biblatex}
\usepackage{pdfpages}

\bibliography{main.bib}

\title{Suspicion: Recognising and evaluating the effectiveness
       of extortion in the Iterated Prisoner's Dilemma}
\author{Vincent A. Knight \and Nikoleta E. Glynatsi}
\date{\today}



\begin{document}

\maketitle

\begin{abstract}
    The Iterated Prisoner's Dilemma is a model for rational and evolutionary
    interactive behaviour. It has applications both in the study of human social
    behaviour as well as in biology.
    It is used to understand when and how a rational individual might
    accept an immediate cost to their own utility for the direct benefit of
    another.

    Much attention has been given to a class of strategies called
    Zero Determinant strategies. It has been theoretically shown that these
    strategies can ``extort'' any player.

    In this work, an approach to identify if observed strategies are playing in
    an extortionate way is described. Furthermore, experimental analysis of
    a large tournament with \input{assets/tex/number_of_full_strategies/main.tex}
    strategies is considered. In this setting
    the most highly performing strategies do not play in an extortionate way
    against each other but do against lower performing strategies.
    This suggests that whilst the theory of Zero Determinant strategies
    indicates that memory is not of fundamental importance to the evolution of
    cooperative behaviour, this is incomplete.
\end{abstract}

\section{Introduction}\label{sec:introduction}

Agent based game theoretic models have become a stalwart of the underpinning
mathematics of interactive behaviours. One of the major pieces of work
in this area is the pair of original computer tournaments run by Robert
Axelrod~\cite{Axelrod1980, Axelrod1980a}. These tournaments pitted submitted
computer strategies against each other in plays of the Iterated Prisoner's
Dilemma. A common game where agents can choose to pay a slight cost to their
immediate utility in the hope of building a reputation. This has been used in
economic and evolutionary game theory to understand the evolution of cooperative
behaviour.

Recently, a class of strategies was described in~\cite{Press2012} that can
provably extort any given opponent. In~\cite{Hilbe2013, Moran1707} some
questions have already been asked about the true effectiveness of these
strategies in an evolutionary setting. Here another question is asked: is it
possible to recognise this extortionate behaviour? A mathematical procedure for
suspicion is presented: in the same way that the continued actions of an
extortionate individual might raise suspicion.

This work makes use of the Axelrod Python library~\cite{Knight2018, Knight2016}
with a large number of Prisoner Dilemma strategies available to give an
extensive numerical example of the ideas presented.  The approach is presented
in Section~\ref{sec:delta-zd-strategies}.  All of the code and data discussed
in Section~\ref{sec:numerical-experiments} is open sourced, archived and
written according to best scientific principles~\cite{Wilson2014}. The data
archive can be found at~\cite{vincent_knight_2018_1297075}.

\section{Recognising Extortion}\label{sec:delta-zd-strategies}

In~\cite{Press2012}, given a match between 2 memory-one strategies, the concept
of Zero Determinant (ZD) strategies is introduced. The main result of that paper
shows that given two memory one players \(p, q\in\mathbb{R}^4\) a linear
relationship between the players' scores could be forced by one of the players.

Using the notation of~\cite{Press2012}, assuming the utilities for player \(p\)
are given by \(S_x=(R, S, T, P)\) and for player \(q\) by \(S_y=(R, T, S, P)\)
and that the stationary scores of each player is given by \(S_X\) and \(S_Y\)
respectively. The main result of~\cite{Press2012} is that if

\begin{equation}\label{eqn:linear_relationship_for_p}
    \tilde p=\alpha S_x + \beta S_y + \gamma
\end{equation}

or

\begin{equation}\label{eqn:linear_relationship_for_q}
    \tilde q=\alpha S_x + \beta S_y + \gamma
\end{equation}

where \(\tilde p = (1 - p_1, 1 - p_2, p_3, p_4)\) and
\(\tilde q = (1 - q_1, 1 - q_2, q_3, q_4)\) then:

\begin{equation}
    \alpha S_X + \beta S_Y + \gamma = 0
\end{equation}

In~\cite{Press2012} a particular type of ZD strategy is defined: extortionate
strategies. If:

\begin{equation}\label{eqn:constraint_for_extortion}
    \gamma = - P(\alpha + \beta)
\end{equation}

then the player can ensure they get a score \(\chi\) times
larger than the opponent. This extortion coefficient is given by:

\begin{equation}\label{eqn:definition_of_chi}
    \chi=\frac{-\beta}{\alpha}
\end{equation}

Thus, if (\ref{eqn:constraint_for_extortion}) holds and \(\chi >1\) a player is
said to extort their opponent.
Here, the reverse problem is considered: given a
\(p\in\mathbb{R}^4\) how does one identify \(\alpha, \beta\) if they
exist and is the strategy in fact acting in an extortionate way?

These conditions correspond to:

\begin{align}
    \tilde p_1 & = \alpha R + \beta R - P (\alpha + \beta)
            \label{eqn:condition_for_tilde_p1}\\
    \tilde p_2 & = \alpha S + \beta T - P (\alpha + \beta)
            \label{eqn:condition_for_tilde_p2}\\
    \tilde p_3 & = \alpha T + \beta S - P (\alpha + \beta)
            \label{eqn:condition_for_tilde_p3}\\
    \tilde p_4 & = \alpha P + \beta P - P (\alpha + \beta)
            \label{eqn:condition_for_tilde_p4}
\end{align}

Equation (\ref{eqn:condition_for_tilde_p4}) ensures that \(p_4=\tilde p_4=0\).
Equations (\ref{eqn:condition_for_tilde_p1}-\ref{eqn:condition_for_tilde_p3})
can be used to eliminate \(\alpha, \beta\), giving:

\begin{equation}\label{eqn:planar_definition_of_extortion}
    \tilde p_1 = \frac{(R - P)(\tilde p_2 + \tilde p_3)}{S + T - 2P}
\end{equation}

with:

\begin{equation}\label{eqn:definition_of_chi}
    \chi = \frac{\tilde p_2 (P - T) + \tilde p_3 (S - P)}
                {\tilde p_2 (P - S) + \tilde p_3 (T - P)}
\end{equation}

Given a strategy \(p\in\mathbb{R}^{4\times 1}\) equations
(\ref{eqn:condition_for_tilde_p4}), (\ref{eqn:planar_definition_of_extortion}-\ref{eqn:definition_of_chi}) can be used to check if
a strategy is extortionate. The conditions correspond to:

\begin{align}
    p_1 & = \frac{(R-P)(p_2 + p_3) - R + T + S - P}{S + T - 2P}
     \label{eqn:condition_for_p1}\\
    p_4 & = 0 \label{eqn:condition_for_p4}\\
    1 & > p_2 + p_3\label{eqn:condition_for_chi}
\end{align}

The algebraic steps necessary to prove these results are available in the
supporting materials.

All extortionate strategies reside on a triangular (\ref{eqn:condition_for_chi})
plane (\ref{eqn:condition_for_p1}) in 3 dimensions (\ref{eqn:condition_for_p4}).
Using this formulation it can be seen that a necessary (but not sufficient)
condition for an extortionate strategy is that it cooperates on average less
than 50\% of the time when in a state of disagreement with the opponent.

As an example, consider the known extortionate strategy \(p=(8 / 9, 1 / 2, 1 /
3, 0)\) from~\cite{Stewart2012} which is referred to as \texttt{Extort-2}. In
this case, for the standard values of \((R, T, S, P)\) constraint
(\ref{eqn:condition_for_p1}) corresponds to:

\begin{equation}
    p_1 = \frac{2(p_2 + p_3) + 1}{3}
\end{equation}

It is clear that in this case all constraints hold.

This approach could in fact be used to confirm that a given strategy is acting
in an extortionate manner even if it is not a memory one strategy. However, in
practice, if a closed form for \(p\) is not known, then due to measurement
and/or numerical error this would not work.

This problem can be written in the following linear algebraic form where
\(x=(\alpha, \beta)\)
and \(p^*=(\tilde p_1 - 1, tilde_2 - 1, p_3)\):

\begin{equation}\label{eqn:linear_algebraic_equation_for_p}
    Cx= p^*
\end{equation}

\(C\) corresponds to equations
(\ref{eqn:condition_for_tilde_p1}-\ref{eqn:condition_for_tilde_p3}) and is
given by:

\begin{equation}\label{eqn:definition_of_C}
    C =
    \begin{bmatrix}
        R - P & R- P \\
        S - P & T- P \\
        T - P & S- P \\
    \end{bmatrix}
\end{equation}

Note that in general, equation (\ref{eqn:linear_algebraic_equation_for_p}) will
not necessarily have a solution. From the Rouch\'{e}-Capelli theorem if there is
a solution it is unique as \(\text{rank}(C)=2\) which is the dimension of the
variable \(x\). The best fitting \(x\) is found by minimizing:

\begin{equation}\label{eqn:r_squared}
    \text{SSError} = \|C x- p^*\|_2^2 = \sum_{i=1}^{3}\left((C\bar x)_i-p_i^*\right)^2
\end{equation}

Note that \(\text{SSError}\), which is the square of the Frobenius
norm~\cite{Golub2013}, becomes a measure of how close a strategy is to being an
extortionate strategy. Suspicion
of extortion then corresponds to a threshold on \(\text{SSError}\).

By observing interactions (human or otherwise), their memory one representation
can be inferred and this approach can be used to recognise extortionate
behaviour. The notion of comparing theoretic and actual plays of the IPD is not
novel, see for example~\cite{Rand2013}. Immediately it is noted that if the
environment is noisy~\cite{Wu1995} then no strategy can be considered to be
extortionate as \(p_4>0\).

In the next section, this idea will be illustrated by observing the interactions
that take place in a computer based tournament of the IPD\@.

\section{Numerical experiments}\label{sec:numerical-experiments}

In~\cite{Stewart2012} results from a tournament with
\input{./assets/tex/number_of_stewart_plotkin_strategies/main.tex} strategies,
was presented with specific consideration given to ZD strategies. This
tournament is reproduced here using the Axelrod-Python
project~\cite{Knight2016}. To obtain a good measure of the corresponding
transition rates for each strategy all matches have been run for
\input{assets/tex/number_of_turns/main.tex} turns and every match has been
repeated \input{assets/tex/number_of_repetitions/main.tex} times. All of this
interaction data is available at~\cite{vincent_knight_2018_1297075}. A good
match between the inferred Markov chain and the state distribution of the actual
interactions has been verified. Data for this is presented in the supplementary
materials.

Figure~\ref{fig:SSError_overall_in_stewart_plotkin} shows the \(\text{SSError}\)
values for all the strategies in the tournament, as reported
in~\cite{Stewart2012} the extortionate strategy (which has an expected
\(\text{SSError}\) approximately 0) gains a large number of wins.

\begin{figure}[!htbp]
    \centering
    \includegraphics[width=.8\textwidth]{./assets/img/SSError_overall_in_stewart_plotkin/main.pdf}
    \caption{\(\text{SSError}\) and state probabilities for the strategies
        of~\cite{Stewart2012}, ordered both by number of wins and overall score.
        Note that \(P(DC)\) is not shown as it corresponds to the transpose of
        \(P(CD)\). Cooperator and Defector are omitted as they do not visit all
        the states.}
    \label{fig:SSError_overall_in_stewart_plotkin}
\end{figure}

Here, the work of~\cite{Stewart2012} is extended by investigating a tournament
with \input{assets/tex/number_of_full_strategies/main.tex}
strategies.

The results of this analysis are shown in
Figure~\ref{fig:SSError_and_probabilities_in_full}. The top ranking strategies
by number of wins seem to be extortionate (but not against all strategies) and
it can be seen that a small sub group of strategies achieve mutual defection.
All the top ranking strategies according to score achieve mutual cooperation and
do not extort each other, however they
\textbf{do} exhibit extortionate behaviour towards a number of the lower ranking
strategies.

\begin{figure}[!htbp]
    \centering
    \includegraphics[width=.8\textwidth]{./assets/img/SSError_and_probabilities_in_full/main.pdf}
    \caption{\(\text{SSError}\) for the strategies for the full tournament. Only
    strategy interactions for which \(p_4=0\) and \(\chi>1\) are displayed.}
    \label{fig:SSError_and_probabilities_in_full}
\end{figure}

\section{Conclusion}\label{sec:conclusion}

This work defines an approach to measure whether or not a player is playing a
strategy that corresponds to an extortionate strategy as defined
in~\cite{Press2012}: a mathematical model for suspicion. Indeed, all
extortionate strategies have been
 classified as lying on a triangular plane.
This rigorous classification fails to be robust to small measurement error, thus
a statistical approach is proposed.
This is done through a linear algebraic approach for approximating the solution
of a linear system. Using this, a large number of pairwise interactions is
simulated and in fact very few strategies are found to act extortionately.

The work of~\cite{Press2012}, whilst showing that a clever approach to taking
advantage of another memory one strategy exists: this is incomplete. Whilst the
elegance of this result is very attractive, just as the simplicity of the
victory of Tit For Tat in Axelrod's original tournaments was, it is incomplete.
Extortionate strategies achieve a high number of wins but they do not
achieve a high score which corresponds to the fitness landscape in an
evolutionary sense. From the large number of interactions a payoff matrix \(S\)
can be measured where \(S_{ij}\) denotes the score (using standard values of
\((R, S, T, P) = (3, 0, 5, 1)\)) of the \(i\)th strategy
against the \(j\)th strategy. Using this, the replicator equation
describes the evolution of the system based on a population density fitness
function:

\begin{equation}\label{eqn:replicator_dynamics}
    \frac{dx}{dt} = x(S-x^TS x)
\end{equation}

Equation (\ref{eqn:replicator_dynamics}) is solved numerically through an
integration technique described in~\cite{Petzold1983} and
Figure~\ref{fig:replicator_dynamics} shows the evolution of the distribution of
the system: the various strategies are ranked by scores. It is clear to see that
only the high ranking strategies survive the evolutionary process (in fact,
only \input{./assets/img/replicator_dynamics/main.tex}
have a final distribution greater than \(10 ^ {-2}\)). This confirms the
findings of~\cite{Moran1707} in which sophisticated strategies resist
evolutionary invasion of shorter memory strategies. Recalling
Figure~\ref{fig:SSError_and_probabilities_in_full} this demonstrates that:

\begin{itemize}
    \item Cooperation emerges through the evolutionary process: the high scoring
        strategies do not exhibit extortionate behaviour towards each other.
    \item Extortionate strategies do not survive the evolutionary process.
\end{itemize}

\begin{figure}[!htbp]
    \centering
    \includegraphics[width=.8\textwidth]{./assets/img/replicator_dynamics/main.pdf}
    \caption{Numerical simulation of the replicator equation
    (\ref{eqn:replicator_dynamics}): strategies are ordered by score, only the strategies with a high score survive the evolutionary process.}
    \label{fig:replicator_dynamics}
\end{figure}

This work can be used to classify plays of the IPD\@: data can be collected from
actual interactions (in lab or in the field). Furthermore, this allows for a
classification method similar to the notion of fingerprinting presented
in~\cite{Ashlock2008}. Trained strategies can potentially be classified as
extortionate or not or it could be possible to even constrain the reinforcement
learning approaches that are becoming prevalent in the literature.
Alternatively, this mathematical approach for recognising extortion could be
used in sophisticated strategies to defend against invasion. Arguably, some of
the strategies considered here exhibit this behaviour, indeed as described
in~\cite{Harper2017}, the top ranking strategies in the full tournament are
obtained using evolutionary reinforcement learning techniques, thus, suspicion
of extortionate behaviour could in fact be an evolutionary trait.

\section*{Acknowledgements}

The following open source software libraries were used in this research:

\begin{itemize}
    \item The Axelrod ~\cite{Knight2016, Knight2018} library (IPD strategies and
        tournaments).
    \item The sympy library~\cite{Meurer2017} (verification of all symbolic
        calculations).
    \item The matplotlib~\cite{Droettboom2018} library (visualisation).
    \item The pandas~\cite{Structures2010}, dask~\cite{Dask2016} and
        NumPy~\cite{Oliphant2015} libraries (data manipulation).
    \item The SciPy~\cite{Jones2001} library (numerical integration of the
        replicator equation).
\end{itemize}

This work was performed using the computational facilities of the Advanced
Research Computing @ Cardiff (ARCCA) Division, Cardiff University.

\printbibliography

\newpage
\section*{Supplementary materials}

\includepdf{assets/pdf/proof_of_form_of_extortionate_strategies/main.pdf}

\newpage

Using the pair wise interactions the transition rates \(p,
q\) can be measured and the steady state probabilities inferred and compared to
the actual probabilities of each state.
This is done numerically by computing the singular eigenvector of the
matrix \(A\) \cite{Stewart2009}:

\[
    A =
    \begin{bmatrix}
        p_1 q_1 & p_1 (1 - q_1) & (1 - p_1) q_1 & (1 -p_1) (1 - q_1) \\
        p_2 q_2 & p_2 (1 - q_2) & (1 - p_2) q_2 & (1 -p_2) (1 - q_2) \\
        p_3 q_3 & p_3 (1 - q_3) & (1 - p_3) q_3 & (1 -p_3) (1 - q_3) \\
        p_4 q_4 & p_4 (1 - q_4) & (1 - p_4) q_4 & (1 -p_4) (1 - q_4) \\
    \end{bmatrix}
\]

Figure~\ref{fig:computed_probabilities_vs_theoretic_probabilities} shows a
regression line fitted to every pairwise interaction with a reported
\(\text{SSError}\) value (pairwise interactions with missing states were
omitted). This serves to validate the approach: a part from some edge cases the
relationship is consistent.

\begin{figure}[!htbp]
    \centering
    \includegraphics[width=.8\textwidth]{./assets/img/computed_probabilities_vs_theoretic_probabilities/main.pdf}
    \caption{The
        relationship between the steady state probabilities inferred from the
        measured transitions and the actual steady state probabilities. A linear
        regression line is included validating the approach.}
    \label{fig:computed_probabilities_vs_theoretic_probabilities}
\end{figure}


\end{document}

strategies.

The results of this analysis are shown in
Figure~\ref{fig:SSError_and_probabilities_in_full}. The top ranking strategies
by number of wins seem to be extortionate (but not against all strategies) and
it can be seen that a small sub group of strategies achieve mutual defection.
All the top ranking strategies according to score achieve mutual cooperation and
do not extort each other, however they
\textbf{do} exhibit extortionate behaviour towards a number of the lower ranking
strategies.

\begin{figure}[!htbp]
    \centering
    \includegraphics[width=.8\textwidth]{./assets/img/SSError_and_probabilities_in_full/main.pdf}
    \caption{\(\text{SSError}\) for the strategies for the full tournament. Only
    strategy interactions for which \(p_4=0\) and \(\chi>1\) are displayed.}
    \label{fig:SSError_and_probabilities_in_full}
\end{figure}

\section{Conclusion}\label{sec:conclusion}

This work defines an approach to measure whether or not a player is playing a
strategy that corresponds to an extortionate strategy as defined
in~\cite{Press2012}: a mathematical model for suspicion. Indeed, all
extortionate strategies have been
 classified as lying on a triangular plane.
This rigorous classification fails to be robust to small measurement error, thus
a statistical approach is proposed.
This is done through a linear algebraic approach for approximating the solution
of a linear system. Using this, a large number of pairwise interactions is
simulated and in fact very few strategies are found to act extortionately.

The work of~\cite{Press2012}, whilst showing that a clever approach to taking
advantage of another memory one strategy exists: this is incomplete. Whilst the
elegance of this result is very attractive, just as the simplicity of the
victory of Tit For Tat in Axelrod's original tournaments was, it is incomplete.
Extortionate strategies achieve a high number of wins but they do not
achieve a high score which corresponds to the fitness landscape in an
evolutionary sense. From the large number of interactions a payoff matrix \(S\)
can be measured where \(S_{ij}\) denotes the score (using standard values of
\((R, S, T, P) = (3, 0, 5, 1)\)) of the \(i\)th strategy
against the \(j\)th strategy. Using this, the replicator equation
describes the evolution of the system based on a population density fitness
function:

\begin{equation}\label{eqn:replicator_dynamics}
    \frac{dx}{dt} = x(S-x^TS x)
\end{equation}

Equation (\ref{eqn:replicator_dynamics}) is solved numerically through an
integration technique described in~\cite{Petzold1983} and
Figure~\ref{fig:replicator_dynamics} shows the evolution of the distribution of
the system: the various strategies are ranked by scores. It is clear to see that
only the high ranking strategies survive the evolutionary process (in fact,
only \documentclass[a4paper]{article}

\usepackage{amsmath}
\usepackage{amssymb}
\usepackage[margin=1.5cm,
            includefoot,
            footskip=30pt]{geometry}
\usepackage{layout}
\usepackage{graphicx}
\usepackage{subcaption}

\usepackage{biblatex}
\usepackage{pdfpages}

\bibliography{main.bib}

\title{Suspicion: Recognising and evaluating the effectiveness
       of extortion in the Iterated Prisoner's Dilemma}
\author{Vincent A. Knight \and Nikoleta E. Glynatsi}
\date{\today}



\begin{document}

\maketitle

\begin{abstract}
    The Iterated Prisoner's Dilemma is a model for rational and evolutionary
    interactive behaviour. It has applications both in the study of human social
    behaviour as well as in biology.
    It is used to understand when and how a rational individual might
    accept an immediate cost to their own utility for the direct benefit of
    another.

    Much attention has been given to a class of strategies called
    Zero Determinant strategies. It has been theoretically shown that these
    strategies can ``extort'' any player.

    In this work, an approach to identify if observed strategies are playing in
    an extortionate way is described. Furthermore, experimental analysis of
    a large tournament with \input{assets/tex/number_of_full_strategies/main.tex}
    strategies is considered. In this setting
    the most highly performing strategies do not play in an extortionate way
    against each other but do against lower performing strategies.
    This suggests that whilst the theory of Zero Determinant strategies
    indicates that memory is not of fundamental importance to the evolution of
    cooperative behaviour, this is incomplete.
\end{abstract}

\section{Introduction}\label{sec:introduction}

Agent based game theoretic models have become a stalwart of the underpinning
mathematics of interactive behaviours. One of the major pieces of work
in this area is the pair of original computer tournaments run by Robert
Axelrod~\cite{Axelrod1980, Axelrod1980a}. These tournaments pitted submitted
computer strategies against each other in plays of the Iterated Prisoner's
Dilemma. A common game where agents can choose to pay a slight cost to their
immediate utility in the hope of building a reputation. This has been used in
economic and evolutionary game theory to understand the evolution of cooperative
behaviour.

Recently, a class of strategies was described in~\cite{Press2012} that can
provably extort any given opponent. In~\cite{Hilbe2013, Moran1707} some
questions have already been asked about the true effectiveness of these
strategies in an evolutionary setting. Here another question is asked: is it
possible to recognise this extortionate behaviour? A mathematical procedure for
suspicion is presented: in the same way that the continued actions of an
extortionate individual might raise suspicion.

This work makes use of the Axelrod Python library~\cite{Knight2018, Knight2016}
with a large number of Prisoner Dilemma strategies available to give an
extensive numerical example of the ideas presented.  The approach is presented
in Section~\ref{sec:delta-zd-strategies}.  All of the code and data discussed
in Section~\ref{sec:numerical-experiments} is open sourced, archived and
written according to best scientific principles~\cite{Wilson2014}. The data
archive can be found at~\cite{vincent_knight_2018_1297075}.

\section{Recognising Extortion}\label{sec:delta-zd-strategies}

In~\cite{Press2012}, given a match between 2 memory-one strategies, the concept
of Zero Determinant (ZD) strategies is introduced. The main result of that paper
shows that given two memory one players \(p, q\in\mathbb{R}^4\) a linear
relationship between the players' scores could be forced by one of the players.

Using the notation of~\cite{Press2012}, assuming the utilities for player \(p\)
are given by \(S_x=(R, S, T, P)\) and for player \(q\) by \(S_y=(R, T, S, P)\)
and that the stationary scores of each player is given by \(S_X\) and \(S_Y\)
respectively. The main result of~\cite{Press2012} is that if

\begin{equation}\label{eqn:linear_relationship_for_p}
    \tilde p=\alpha S_x + \beta S_y + \gamma
\end{equation}

or

\begin{equation}\label{eqn:linear_relationship_for_q}
    \tilde q=\alpha S_x + \beta S_y + \gamma
\end{equation}

where \(\tilde p = (1 - p_1, 1 - p_2, p_3, p_4)\) and
\(\tilde q = (1 - q_1, 1 - q_2, q_3, q_4)\) then:

\begin{equation}
    \alpha S_X + \beta S_Y + \gamma = 0
\end{equation}

In~\cite{Press2012} a particular type of ZD strategy is defined: extortionate
strategies. If:

\begin{equation}\label{eqn:constraint_for_extortion}
    \gamma = - P(\alpha + \beta)
\end{equation}

then the player can ensure they get a score \(\chi\) times
larger than the opponent. This extortion coefficient is given by:

\begin{equation}\label{eqn:definition_of_chi}
    \chi=\frac{-\beta}{\alpha}
\end{equation}

Thus, if (\ref{eqn:constraint_for_extortion}) holds and \(\chi >1\) a player is
said to extort their opponent.
Here, the reverse problem is considered: given a
\(p\in\mathbb{R}^4\) how does one identify \(\alpha, \beta\) if they
exist and is the strategy in fact acting in an extortionate way?

These conditions correspond to:

\begin{align}
    \tilde p_1 & = \alpha R + \beta R - P (\alpha + \beta)
            \label{eqn:condition_for_tilde_p1}\\
    \tilde p_2 & = \alpha S + \beta T - P (\alpha + \beta)
            \label{eqn:condition_for_tilde_p2}\\
    \tilde p_3 & = \alpha T + \beta S - P (\alpha + \beta)
            \label{eqn:condition_for_tilde_p3}\\
    \tilde p_4 & = \alpha P + \beta P - P (\alpha + \beta)
            \label{eqn:condition_for_tilde_p4}
\end{align}

Equation (\ref{eqn:condition_for_tilde_p4}) ensures that \(p_4=\tilde p_4=0\).
Equations (\ref{eqn:condition_for_tilde_p1}-\ref{eqn:condition_for_tilde_p3})
can be used to eliminate \(\alpha, \beta\), giving:

\begin{equation}\label{eqn:planar_definition_of_extortion}
    \tilde p_1 = \frac{(R - P)(\tilde p_2 + \tilde p_3)}{S + T - 2P}
\end{equation}

with:

\begin{equation}\label{eqn:definition_of_chi}
    \chi = \frac{\tilde p_2 (P - T) + \tilde p_3 (S - P)}
                {\tilde p_2 (P - S) + \tilde p_3 (T - P)}
\end{equation}

Given a strategy \(p\in\mathbb{R}^{4\times 1}\) equations
(\ref{eqn:condition_for_tilde_p4}), (\ref{eqn:planar_definition_of_extortion}-\ref{eqn:definition_of_chi}) can be used to check if
a strategy is extortionate. The conditions correspond to:

\begin{align}
    p_1 & = \frac{(R-P)(p_2 + p_3) - R + T + S - P}{S + T - 2P}
     \label{eqn:condition_for_p1}\\
    p_4 & = 0 \label{eqn:condition_for_p4}\\
    1 & > p_2 + p_3\label{eqn:condition_for_chi}
\end{align}

The algebraic steps necessary to prove these results are available in the
supporting materials.

All extortionate strategies reside on a triangular (\ref{eqn:condition_for_chi})
plane (\ref{eqn:condition_for_p1}) in 3 dimensions (\ref{eqn:condition_for_p4}).
Using this formulation it can be seen that a necessary (but not sufficient)
condition for an extortionate strategy is that it cooperates on average less
than 50\% of the time when in a state of disagreement with the opponent.

As an example, consider the known extortionate strategy \(p=(8 / 9, 1 / 2, 1 /
3, 0)\) from~\cite{Stewart2012} which is referred to as \texttt{Extort-2}. In
this case, for the standard values of \((R, T, S, P)\) constraint
(\ref{eqn:condition_for_p1}) corresponds to:

\begin{equation}
    p_1 = \frac{2(p_2 + p_3) + 1}{3}
\end{equation}

It is clear that in this case all constraints hold.

This approach could in fact be used to confirm that a given strategy is acting
in an extortionate manner even if it is not a memory one strategy. However, in
practice, if a closed form for \(p\) is not known, then due to measurement
and/or numerical error this would not work.

This problem can be written in the following linear algebraic form where
\(x=(\alpha, \beta)\)
and \(p^*=(\tilde p_1 - 1, tilde_2 - 1, p_3)\):

\begin{equation}\label{eqn:linear_algebraic_equation_for_p}
    Cx= p^*
\end{equation}

\(C\) corresponds to equations
(\ref{eqn:condition_for_tilde_p1}-\ref{eqn:condition_for_tilde_p3}) and is
given by:

\begin{equation}\label{eqn:definition_of_C}
    C =
    \begin{bmatrix}
        R - P & R- P \\
        S - P & T- P \\
        T - P & S- P \\
    \end{bmatrix}
\end{equation}

Note that in general, equation (\ref{eqn:linear_algebraic_equation_for_p}) will
not necessarily have a solution. From the Rouch\'{e}-Capelli theorem if there is
a solution it is unique as \(\text{rank}(C)=2\) which is the dimension of the
variable \(x\). The best fitting \(x\) is found by minimizing:

\begin{equation}\label{eqn:r_squared}
    \text{SSError} = \|C x- p^*\|_2^2 = \sum_{i=1}^{3}\left((C\bar x)_i-p_i^*\right)^2
\end{equation}

Note that \(\text{SSError}\), which is the square of the Frobenius
norm~\cite{Golub2013}, becomes a measure of how close a strategy is to being an
extortionate strategy. Suspicion
of extortion then corresponds to a threshold on \(\text{SSError}\).

By observing interactions (human or otherwise), their memory one representation
can be inferred and this approach can be used to recognise extortionate
behaviour. The notion of comparing theoretic and actual plays of the IPD is not
novel, see for example~\cite{Rand2013}. Immediately it is noted that if the
environment is noisy~\cite{Wu1995} then no strategy can be considered to be
extortionate as \(p_4>0\).

In the next section, this idea will be illustrated by observing the interactions
that take place in a computer based tournament of the IPD\@.

\section{Numerical experiments}\label{sec:numerical-experiments}

In~\cite{Stewart2012} results from a tournament with
\input{./assets/tex/number_of_stewart_plotkin_strategies/main.tex} strategies,
was presented with specific consideration given to ZD strategies. This
tournament is reproduced here using the Axelrod-Python
project~\cite{Knight2016}. To obtain a good measure of the corresponding
transition rates for each strategy all matches have been run for
\input{assets/tex/number_of_turns/main.tex} turns and every match has been
repeated \input{assets/tex/number_of_repetitions/main.tex} times. All of this
interaction data is available at~\cite{vincent_knight_2018_1297075}. A good
match between the inferred Markov chain and the state distribution of the actual
interactions has been verified. Data for this is presented in the supplementary
materials.

Figure~\ref{fig:SSError_overall_in_stewart_plotkin} shows the \(\text{SSError}\)
values for all the strategies in the tournament, as reported
in~\cite{Stewart2012} the extortionate strategy (which has an expected
\(\text{SSError}\) approximately 0) gains a large number of wins.

\begin{figure}[!htbp]
    \centering
    \includegraphics[width=.8\textwidth]{./assets/img/SSError_overall_in_stewart_plotkin/main.pdf}
    \caption{\(\text{SSError}\) and state probabilities for the strategies
        of~\cite{Stewart2012}, ordered both by number of wins and overall score.
        Note that \(P(DC)\) is not shown as it corresponds to the transpose of
        \(P(CD)\). Cooperator and Defector are omitted as they do not visit all
        the states.}
    \label{fig:SSError_overall_in_stewart_plotkin}
\end{figure}

Here, the work of~\cite{Stewart2012} is extended by investigating a tournament
with \input{assets/tex/number_of_full_strategies/main.tex}
strategies.

The results of this analysis are shown in
Figure~\ref{fig:SSError_and_probabilities_in_full}. The top ranking strategies
by number of wins seem to be extortionate (but not against all strategies) and
it can be seen that a small sub group of strategies achieve mutual defection.
All the top ranking strategies according to score achieve mutual cooperation and
do not extort each other, however they
\textbf{do} exhibit extortionate behaviour towards a number of the lower ranking
strategies.

\begin{figure}[!htbp]
    \centering
    \includegraphics[width=.8\textwidth]{./assets/img/SSError_and_probabilities_in_full/main.pdf}
    \caption{\(\text{SSError}\) for the strategies for the full tournament. Only
    strategy interactions for which \(p_4=0\) and \(\chi>1\) are displayed.}
    \label{fig:SSError_and_probabilities_in_full}
\end{figure}

\section{Conclusion}\label{sec:conclusion}

This work defines an approach to measure whether or not a player is playing a
strategy that corresponds to an extortionate strategy as defined
in~\cite{Press2012}: a mathematical model for suspicion. Indeed, all
extortionate strategies have been
 classified as lying on a triangular plane.
This rigorous classification fails to be robust to small measurement error, thus
a statistical approach is proposed.
This is done through a linear algebraic approach for approximating the solution
of a linear system. Using this, a large number of pairwise interactions is
simulated and in fact very few strategies are found to act extortionately.

The work of~\cite{Press2012}, whilst showing that a clever approach to taking
advantage of another memory one strategy exists: this is incomplete. Whilst the
elegance of this result is very attractive, just as the simplicity of the
victory of Tit For Tat in Axelrod's original tournaments was, it is incomplete.
Extortionate strategies achieve a high number of wins but they do not
achieve a high score which corresponds to the fitness landscape in an
evolutionary sense. From the large number of interactions a payoff matrix \(S\)
can be measured where \(S_{ij}\) denotes the score (using standard values of
\((R, S, T, P) = (3, 0, 5, 1)\)) of the \(i\)th strategy
against the \(j\)th strategy. Using this, the replicator equation
describes the evolution of the system based on a population density fitness
function:

\begin{equation}\label{eqn:replicator_dynamics}
    \frac{dx}{dt} = x(S-x^TS x)
\end{equation}

Equation (\ref{eqn:replicator_dynamics}) is solved numerically through an
integration technique described in~\cite{Petzold1983} and
Figure~\ref{fig:replicator_dynamics} shows the evolution of the distribution of
the system: the various strategies are ranked by scores. It is clear to see that
only the high ranking strategies survive the evolutionary process (in fact,
only \input{./assets/img/replicator_dynamics/main.tex}
have a final distribution greater than \(10 ^ {-2}\)). This confirms the
findings of~\cite{Moran1707} in which sophisticated strategies resist
evolutionary invasion of shorter memory strategies. Recalling
Figure~\ref{fig:SSError_and_probabilities_in_full} this demonstrates that:

\begin{itemize}
    \item Cooperation emerges through the evolutionary process: the high scoring
        strategies do not exhibit extortionate behaviour towards each other.
    \item Extortionate strategies do not survive the evolutionary process.
\end{itemize}

\begin{figure}[!htbp]
    \centering
    \includegraphics[width=.8\textwidth]{./assets/img/replicator_dynamics/main.pdf}
    \caption{Numerical simulation of the replicator equation
    (\ref{eqn:replicator_dynamics}): strategies are ordered by score, only the strategies with a high score survive the evolutionary process.}
    \label{fig:replicator_dynamics}
\end{figure}

This work can be used to classify plays of the IPD\@: data can be collected from
actual interactions (in lab or in the field). Furthermore, this allows for a
classification method similar to the notion of fingerprinting presented
in~\cite{Ashlock2008}. Trained strategies can potentially be classified as
extortionate or not or it could be possible to even constrain the reinforcement
learning approaches that are becoming prevalent in the literature.
Alternatively, this mathematical approach for recognising extortion could be
used in sophisticated strategies to defend against invasion. Arguably, some of
the strategies considered here exhibit this behaviour, indeed as described
in~\cite{Harper2017}, the top ranking strategies in the full tournament are
obtained using evolutionary reinforcement learning techniques, thus, suspicion
of extortionate behaviour could in fact be an evolutionary trait.

\section*{Acknowledgements}

The following open source software libraries were used in this research:

\begin{itemize}
    \item The Axelrod ~\cite{Knight2016, Knight2018} library (IPD strategies and
        tournaments).
    \item The sympy library~\cite{Meurer2017} (verification of all symbolic
        calculations).
    \item The matplotlib~\cite{Droettboom2018} library (visualisation).
    \item The pandas~\cite{Structures2010}, dask~\cite{Dask2016} and
        NumPy~\cite{Oliphant2015} libraries (data manipulation).
    \item The SciPy~\cite{Jones2001} library (numerical integration of the
        replicator equation).
\end{itemize}

This work was performed using the computational facilities of the Advanced
Research Computing @ Cardiff (ARCCA) Division, Cardiff University.

\printbibliography

\newpage
\section*{Supplementary materials}

\includepdf{assets/pdf/proof_of_form_of_extortionate_strategies/main.pdf}

\newpage

Using the pair wise interactions the transition rates \(p,
q\) can be measured and the steady state probabilities inferred and compared to
the actual probabilities of each state.
This is done numerically by computing the singular eigenvector of the
matrix \(A\) \cite{Stewart2009}:

\[
    A =
    \begin{bmatrix}
        p_1 q_1 & p_1 (1 - q_1) & (1 - p_1) q_1 & (1 -p_1) (1 - q_1) \\
        p_2 q_2 & p_2 (1 - q_2) & (1 - p_2) q_2 & (1 -p_2) (1 - q_2) \\
        p_3 q_3 & p_3 (1 - q_3) & (1 - p_3) q_3 & (1 -p_3) (1 - q_3) \\
        p_4 q_4 & p_4 (1 - q_4) & (1 - p_4) q_4 & (1 -p_4) (1 - q_4) \\
    \end{bmatrix}
\]

Figure~\ref{fig:computed_probabilities_vs_theoretic_probabilities} shows a
regression line fitted to every pairwise interaction with a reported
\(\text{SSError}\) value (pairwise interactions with missing states were
omitted). This serves to validate the approach: a part from some edge cases the
relationship is consistent.

\begin{figure}[!htbp]
    \centering
    \includegraphics[width=.8\textwidth]{./assets/img/computed_probabilities_vs_theoretic_probabilities/main.pdf}
    \caption{The
        relationship between the steady state probabilities inferred from the
        measured transitions and the actual steady state probabilities. A linear
        regression line is included validating the approach.}
    \label{fig:computed_probabilities_vs_theoretic_probabilities}
\end{figure}


\end{document}

have a final distribution greater than \(10 ^ {-2}\)). This confirms the
findings of~\cite{Moran1707} in which sophisticated strategies resist
evolutionary invasion of shorter memory strategies. Recalling
Figure~\ref{fig:SSError_and_probabilities_in_full} this demonstrates that:

\begin{itemize}
    \item Cooperation emerges through the evolutionary process: the high scoring
        strategies do not exhibit extortionate behaviour towards each other.
    \item Extortionate strategies do not survive the evolutionary process.
\end{itemize}

\begin{figure}[!htbp]
    \centering
    \includegraphics[width=.8\textwidth]{./assets/img/replicator_dynamics/main.pdf}
    \caption{Numerical simulation of the replicator equation
    (\ref{eqn:replicator_dynamics}): strategies are ordered by score, only the strategies with a high score survive the evolutionary process.}
    \label{fig:replicator_dynamics}
\end{figure}

This work can be used to classify plays of the IPD\@: data can be collected from
actual interactions (in lab or in the field). Furthermore, this allows for a
classification method similar to the notion of fingerprinting presented
in~\cite{Ashlock2008}. Trained strategies can potentially be classified as
extortionate or not or it could be possible to even constrain the reinforcement
learning approaches that are becoming prevalent in the literature.
Alternatively, this mathematical approach for recognising extortion could be
used in sophisticated strategies to defend against invasion. Arguably, some of
the strategies considered here exhibit this behaviour, indeed as described
in~\cite{Harper2017}, the top ranking strategies in the full tournament are
obtained using evolutionary reinforcement learning techniques, thus, suspicion
of extortionate behaviour could in fact be an evolutionary trait.

\section*{Acknowledgements}

The following open source software libraries were used in this research:

\begin{itemize}
    \item The Axelrod ~\cite{Knight2016, Knight2018} library (IPD strategies and
        tournaments).
    \item The sympy library~\cite{Meurer2017} (verification of all symbolic
        calculations).
    \item The matplotlib~\cite{Droettboom2018} library (visualisation).
    \item The pandas~\cite{Structures2010}, dask~\cite{Dask2016} and
        NumPy~\cite{Oliphant2015} libraries (data manipulation).
    \item The SciPy~\cite{Jones2001} library (numerical integration of the
        replicator equation).
\end{itemize}

This work was performed using the computational facilities of the Advanced
Research Computing @ Cardiff (ARCCA) Division, Cardiff University.

\printbibliography

\newpage
\section*{Supplementary materials}

\includepdf{assets/pdf/proof_of_form_of_extortionate_strategies/main.pdf}

\newpage

Using the pair wise interactions the transition rates \(p,
q\) can be measured and the steady state probabilities inferred and compared to
the actual probabilities of each state.
This is done numerically by computing the singular eigenvector of the
matrix \(A\) \cite{Stewart2009}:

\[
    A =
    \begin{bmatrix}
        p_1 q_1 & p_1 (1 - q_1) & (1 - p_1) q_1 & (1 -p_1) (1 - q_1) \\
        p_2 q_2 & p_2 (1 - q_2) & (1 - p_2) q_2 & (1 -p_2) (1 - q_2) \\
        p_3 q_3 & p_3 (1 - q_3) & (1 - p_3) q_3 & (1 -p_3) (1 - q_3) \\
        p_4 q_4 & p_4 (1 - q_4) & (1 - p_4) q_4 & (1 -p_4) (1 - q_4) \\
    \end{bmatrix}
\]

Figure~\ref{fig:computed_probabilities_vs_theoretic_probabilities} shows a
regression line fitted to every pairwise interaction with a reported
\(\text{SSError}\) value (pairwise interactions with missing states were
omitted). This serves to validate the approach: a part from some edge cases the
relationship is consistent.

\begin{figure}[!htbp]
    \centering
    \includegraphics[width=.8\textwidth]{./assets/img/computed_probabilities_vs_theoretic_probabilities/main.pdf}
    \caption{The
        relationship between the steady state probabilities inferred from the
        measured transitions and the actual steady state probabilities. A linear
        regression line is included validating the approach.}
    \label{fig:computed_probabilities_vs_theoretic_probabilities}
\end{figure}


\end{document}

have a final distribution greater than \(10 ^ {-2}\)). This confirms the
findings of~\cite{Moran1707} in which sophisticated strategies resist
evolutionary invasion of shorter memory strategies. Recalling
Figure~\ref{fig:SSError_and_probabilities_in_full} this demonstrates that:

\begin{itemize}
    \item Cooperation emerges through the evolutionary process: the high scoring
        strategies do not exhibit extortionate behaviour towards each other.
    \item Extortionate strategies do not survive the evolutionary process.
\end{itemize}

\begin{figure}[!htbp]
    \centering
    \includegraphics[width=.8\textwidth]{./assets/img/replicator_dynamics/main.pdf}
    \caption{Numerical simulation of the replicator equation
    (\ref{eqn:replicator_dynamics}): strategies are ordered by score, only the strategies with a high score survive the evolutionary process.}
    \label{fig:replicator_dynamics}
\end{figure}

This work can be used to classify plays of the IPD\@: data can be collected from
actual interactions (in lab or in the field). Furthermore, this allows for a
classification method similar to the notion of fingerprinting presented
in~\cite{Ashlock2008}. Trained strategies can potentially be classified as
extortionate or not or it could be possible to even constrain the reinforcement
learning approaches that are becoming prevalent in the literature.
Alternatively, this mathematical approach for recognising extortion could be
used in sophisticated strategies to defend against invasion. Arguably, some of
the strategies considered here exhibit this behaviour, indeed as described
in~\cite{Harper2017}, the top ranking strategies in the full tournament are
obtained using evolutionary reinforcement learning techniques, thus, suspicion
of extortionate behaviour could in fact be an evolutionary trait.

\section*{Acknowledgements}

The following open source software libraries were used in this research:

\begin{itemize}
    \item The Axelrod ~\cite{Knight2016, Knight2018} library (IPD strategies and
        tournaments).
    \item The sympy library~\cite{Meurer2017} (verification of all symbolic
        calculations).
    \item The matplotlib~\cite{Droettboom2018} library (visualisation).
    \item The pandas~\cite{Structures2010}, dask~\cite{Dask2016} and
        NumPy~\cite{Oliphant2015} libraries (data manipulation).
    \item The SciPy~\cite{Jones2001} library (numerical integration of the
        replicator equation).
\end{itemize}

This work was performed using the computational facilities of the Advanced
Research Computing @ Cardiff (ARCCA) Division, Cardiff University.

\printbibliography

\newpage
\section*{Supplementary materials}

\includepdf{assets/pdf/proof_of_form_of_extortionate_strategies/main.pdf}

\newpage

Using the pair wise interactions the transition rates \(p,
q\) can be measured and the steady state probabilities inferred and compared to
the actual probabilities of each state.
This is done numerically by computing the singular eigenvector of the
matrix \(A\) \cite{Stewart2009}:

\[
    A =
    \begin{bmatrix}
        p_1 q_1 & p_1 (1 - q_1) & (1 - p_1) q_1 & (1 -p_1) (1 - q_1) \\
        p_2 q_2 & p_2 (1 - q_2) & (1 - p_2) q_2 & (1 -p_2) (1 - q_2) \\
        p_3 q_3 & p_3 (1 - q_3) & (1 - p_3) q_3 & (1 -p_3) (1 - q_3) \\
        p_4 q_4 & p_4 (1 - q_4) & (1 - p_4) q_4 & (1 -p_4) (1 - q_4) \\
    \end{bmatrix}
\]

Figure~\ref{fig:computed_probabilities_vs_theoretic_probabilities} shows a
regression line fitted to every pairwise interaction with a reported
\(\text{SSError}\) value (pairwise interactions with missing states were
omitted). This serves to validate the approach: a part from some edge cases the
relationship is consistent.

\begin{figure}[!htbp]
    \centering
    \includegraphics[width=.8\textwidth]{./assets/img/computed_probabilities_vs_theoretic_probabilities/main.pdf}
    \caption{The
        relationship between the steady state probabilities inferred from the
        measured transitions and the actual steady state probabilities. A linear
        regression line is included validating the approach.}
    \label{fig:computed_probabilities_vs_theoretic_probabilities}
\end{figure}


\end{document}
strategies
with specific consideration given to ZD strategies. This
tournament is reproduced here using the Axelrod-Python
library~\cite{Knight2016}. To obtain a good measure of the corresponding
transition rates for each strategy all matches have been run for
\documentclass[a4paper]{article}

\usepackage{amsmath}
\usepackage{amssymb}
\usepackage[margin=1.5cm,
            includefoot,
            footskip=30pt]{geometry}
\usepackage{layout}
\usepackage{graphicx}
\usepackage{subcaption}

\usepackage{biblatex}
\usepackage{pdfpages}

\bibliography{main.bib}

\title{Suspicion: Recognising and evaluating the effectiveness
       of extortion in the Iterated Prisoner's Dilemma}
\author{Vincent A. Knight \and Nikoleta E. Glynatsi}
\date{\today}



\begin{document}

\maketitle

\begin{abstract}
    The Iterated Prisoner's Dilemma is a model for rational and evolutionary
    interactive behaviour. It has applications both in the study of human social
    behaviour as well as in biology.
    It is used to understand when and how a rational individual might
    accept an immediate cost to their own utility for the direct benefit of
    another.

    Much attention has been given to a class of strategies called
    Zero Determinant strategies. It has been theoretically shown that these
    strategies can ``extort'' any player.

    In this work, an approach to identify if observed strategies are playing in
    an extortionate way is described. Furthermore, experimental analysis of
    a large tournament with \documentclass[a4paper]{article}

\usepackage{amsmath}
\usepackage{amssymb}
\usepackage[margin=1.5cm,
            includefoot,
            footskip=30pt]{geometry}
\usepackage{layout}
\usepackage{graphicx}
\usepackage{subcaption}

\usepackage{biblatex}
\usepackage{pdfpages}

\bibliography{main.bib}

\title{Suspicion: Recognising and evaluating the effectiveness
       of extortion in the Iterated Prisoner's Dilemma}
\author{Vincent A. Knight \and Nikoleta E. Glynatsi}
\date{\today}



\begin{document}

\maketitle

\begin{abstract}
    The Iterated Prisoner's Dilemma is a model for rational and evolutionary
    interactive behaviour. It has applications both in the study of human social
    behaviour as well as in biology.
    It is used to understand when and how a rational individual might
    accept an immediate cost to their own utility for the direct benefit of
    another.

    Much attention has been given to a class of strategies called
    Zero Determinant strategies. It has been theoretically shown that these
    strategies can ``extort'' any player.

    In this work, an approach to identify if observed strategies are playing in
    an extortionate way is described. Furthermore, experimental analysis of
    a large tournament with \documentclass[a4paper]{article}

\usepackage{amsmath}
\usepackage{amssymb}
\usepackage[margin=1.5cm,
            includefoot,
            footskip=30pt]{geometry}
\usepackage{layout}
\usepackage{graphicx}
\usepackage{subcaption}

\usepackage{biblatex}
\usepackage{pdfpages}

\bibliography{main.bib}

\title{Suspicion: Recognising and evaluating the effectiveness
       of extortion in the Iterated Prisoner's Dilemma}
\author{Vincent A. Knight \and Nikoleta E. Glynatsi}
\date{\today}



\begin{document}

\maketitle

\begin{abstract}
    The Iterated Prisoner's Dilemma is a model for rational and evolutionary
    interactive behaviour. It has applications both in the study of human social
    behaviour as well as in biology.
    It is used to understand when and how a rational individual might
    accept an immediate cost to their own utility for the direct benefit of
    another.

    Much attention has been given to a class of strategies called
    Zero Determinant strategies. It has been theoretically shown that these
    strategies can ``extort'' any player.

    In this work, an approach to identify if observed strategies are playing in
    an extortionate way is described. Furthermore, experimental analysis of
    a large tournament with \input{assets/tex/number_of_full_strategies/main.tex}
    strategies is considered. In this setting
    the most highly performing strategies do not play in an extortionate way
    against each other but do against lower performing strategies.
    This suggests that whilst the theory of Zero Determinant strategies
    indicates that memory is not of fundamental importance to the evolution of
    cooperative behaviour, this is incomplete.
\end{abstract}

\section{Introduction}\label{sec:introduction}

Agent based game theoretic models have become a stalwart of the underpinning
mathematics of interactive behaviours. One of the major pieces of work
in this area is the pair of original computer tournaments run by Robert
Axelrod~\cite{Axelrod1980, Axelrod1980a}. These tournaments pitted submitted
computer strategies against each other in plays of the Iterated Prisoner's
Dilemma. A common game where agents can choose to pay a slight cost to their
immediate utility in the hope of building a reputation. This has been used in
economic and evolutionary game theory to understand the evolution of cooperative
behaviour.

Recently, a class of strategies was described in~\cite{Press2012} that can
provably extort any given opponent. In~\cite{Hilbe2013, Moran1707} some
questions have already been asked about the true effectiveness of these
strategies in an evolutionary setting. Here another question is asked: is it
possible to recognise this extortionate behaviour? A mathematical procedure for
suspicion is presented: in the same way that the continued actions of an
extortionate individual might raise suspicion.

This work makes use of the Axelrod Python library~\cite{Knight2018, Knight2016}
with a large number of Prisoner Dilemma strategies available to give an
extensive numerical example of the ideas presented.  The approach is presented
in Section~\ref{sec:delta-zd-strategies}.  All of the code and data discussed
in Section~\ref{sec:numerical-experiments} is open sourced, archived and
written according to best scientific principles~\cite{Wilson2014}. The data
archive can be found at~\cite{vincent_knight_2018_1297075}.

\section{Recognising Extortion}\label{sec:delta-zd-strategies}

In~\cite{Press2012}, given a match between 2 memory-one strategies, the concept
of Zero Determinant (ZD) strategies is introduced. The main result of that paper
shows that given two memory one players \(p, q\in\mathbb{R}^4\) a linear
relationship between the players' scores could be forced by one of the players.

Using the notation of~\cite{Press2012}, assuming the utilities for player \(p\)
are given by \(S_x=(R, S, T, P)\) and for player \(q\) by \(S_y=(R, T, S, P)\)
and that the stationary scores of each player is given by \(S_X\) and \(S_Y\)
respectively. The main result of~\cite{Press2012} is that if

\begin{equation}\label{eqn:linear_relationship_for_p}
    \tilde p=\alpha S_x + \beta S_y + \gamma
\end{equation}

or

\begin{equation}\label{eqn:linear_relationship_for_q}
    \tilde q=\alpha S_x + \beta S_y + \gamma
\end{equation}

where \(\tilde p = (1 - p_1, 1 - p_2, p_3, p_4)\) and
\(\tilde q = (1 - q_1, 1 - q_2, q_3, q_4)\) then:

\begin{equation}
    \alpha S_X + \beta S_Y + \gamma = 0
\end{equation}

In~\cite{Press2012} a particular type of ZD strategy is defined: extortionate
strategies. If:

\begin{equation}\label{eqn:constraint_for_extortion}
    \gamma = - P(\alpha + \beta)
\end{equation}

then the player can ensure they get a score \(\chi\) times
larger than the opponent. This extortion coefficient is given by:

\begin{equation}\label{eqn:definition_of_chi}
    \chi=\frac{-\beta}{\alpha}
\end{equation}

Thus, if (\ref{eqn:constraint_for_extortion}) holds and \(\chi >1\) a player is
said to extort their opponent.
Here, the reverse problem is considered: given a
\(p\in\mathbb{R}^4\) how does one identify \(\alpha, \beta\) if they
exist and is the strategy in fact acting in an extortionate way?

These conditions correspond to:

\begin{align}
    \tilde p_1 & = \alpha R + \beta R - P (\alpha + \beta)
            \label{eqn:condition_for_tilde_p1}\\
    \tilde p_2 & = \alpha S + \beta T - P (\alpha + \beta)
            \label{eqn:condition_for_tilde_p2}\\
    \tilde p_3 & = \alpha T + \beta S - P (\alpha + \beta)
            \label{eqn:condition_for_tilde_p3}\\
    \tilde p_4 & = \alpha P + \beta P - P (\alpha + \beta)
            \label{eqn:condition_for_tilde_p4}
\end{align}

Equation (\ref{eqn:condition_for_tilde_p4}) ensures that \(p_4=\tilde p_4=0\).
Equations (\ref{eqn:condition_for_tilde_p1}-\ref{eqn:condition_for_tilde_p3})
can be used to eliminate \(\alpha, \beta\), giving:

\begin{equation}\label{eqn:planar_definition_of_extortion}
    \tilde p_1 = \frac{(R - P)(\tilde p_2 + \tilde p_3)}{S + T - 2P}
\end{equation}

with:

\begin{equation}\label{eqn:definition_of_chi}
    \chi = \frac{\tilde p_2 (P - T) + \tilde p_3 (S - P)}
                {\tilde p_2 (P - S) + \tilde p_3 (T - P)}
\end{equation}

Given a strategy \(p\in\mathbb{R}^{4\times 1}\) equations
(\ref{eqn:condition_for_tilde_p4}), (\ref{eqn:planar_definition_of_extortion}-\ref{eqn:definition_of_chi}) can be used to check if
a strategy is extortionate. The conditions correspond to:

\begin{align}
    p_1 & = \frac{(R-P)(p_2 + p_3) - R + T + S - P}{S + T - 2P}
     \label{eqn:condition_for_p1}\\
    p_4 & = 0 \label{eqn:condition_for_p4}\\
    1 & > p_2 + p_3\label{eqn:condition_for_chi}
\end{align}

The algebraic steps necessary to prove these results are available in the
supporting materials.

All extortionate strategies reside on a triangular (\ref{eqn:condition_for_chi})
plane (\ref{eqn:condition_for_p1}) in 3 dimensions (\ref{eqn:condition_for_p4}).
Using this formulation it can be seen that a necessary (but not sufficient)
condition for an extortionate strategy is that it cooperates on average less
than 50\% of the time when in a state of disagreement with the opponent.

As an example, consider the known extortionate strategy \(p=(8 / 9, 1 / 2, 1 /
3, 0)\) from~\cite{Stewart2012} which is referred to as \texttt{Extort-2}. In
this case, for the standard values of \((R, T, S, P)\) constraint
(\ref{eqn:condition_for_p1}) corresponds to:

\begin{equation}
    p_1 = \frac{2(p_2 + p_3) + 1}{3}
\end{equation}

It is clear that in this case all constraints hold.

This approach could in fact be used to confirm that a given strategy is acting
in an extortionate manner even if it is not a memory one strategy. However, in
practice, if a closed form for \(p\) is not known, then due to measurement
and/or numerical error this would not work.

This problem can be written in the following linear algebraic form where
\(x=(\alpha, \beta)\)
and \(p^*=(\tilde p_1 - 1, tilde_2 - 1, p_3)\):

\begin{equation}\label{eqn:linear_algebraic_equation_for_p}
    Cx= p^*
\end{equation}

\(C\) corresponds to equations
(\ref{eqn:condition_for_tilde_p1}-\ref{eqn:condition_for_tilde_p3}) and is
given by:

\begin{equation}\label{eqn:definition_of_C}
    C =
    \begin{bmatrix}
        R - P & R- P \\
        S - P & T- P \\
        T - P & S- P \\
    \end{bmatrix}
\end{equation}

Note that in general, equation (\ref{eqn:linear_algebraic_equation_for_p}) will
not necessarily have a solution. From the Rouch\'{e}-Capelli theorem if there is
a solution it is unique as \(\text{rank}(C)=2\) which is the dimension of the
variable \(x\). The best fitting \(x\) is found by minimizing:

\begin{equation}\label{eqn:r_squared}
    \text{SSError} = \|C x- p^*\|_2^2 = \sum_{i=1}^{3}\left((C\bar x)_i-p_i^*\right)^2
\end{equation}

Note that \(\text{SSError}\), which is the square of the Frobenius
norm~\cite{Golub2013}, becomes a measure of how close a strategy is to being an
extortionate strategy. Suspicion
of extortion then corresponds to a threshold on \(\text{SSError}\).

By observing interactions (human or otherwise), their memory one representation
can be inferred and this approach can be used to recognise extortionate
behaviour. The notion of comparing theoretic and actual plays of the IPD is not
novel, see for example~\cite{Rand2013}. Immediately it is noted that if the
environment is noisy~\cite{Wu1995} then no strategy can be considered to be
extortionate as \(p_4>0\).

In the next section, this idea will be illustrated by observing the interactions
that take place in a computer based tournament of the IPD\@.

\section{Numerical experiments}\label{sec:numerical-experiments}

In~\cite{Stewart2012} results from a tournament with
\input{./assets/tex/number_of_stewart_plotkin_strategies/main.tex} strategies,
was presented with specific consideration given to ZD strategies. This
tournament is reproduced here using the Axelrod-Python
project~\cite{Knight2016}. To obtain a good measure of the corresponding
transition rates for each strategy all matches have been run for
\input{assets/tex/number_of_turns/main.tex} turns and every match has been
repeated \input{assets/tex/number_of_repetitions/main.tex} times. All of this
interaction data is available at~\cite{vincent_knight_2018_1297075}. A good
match between the inferred Markov chain and the state distribution of the actual
interactions has been verified. Data for this is presented in the supplementary
materials.

Figure~\ref{fig:SSError_overall_in_stewart_plotkin} shows the \(\text{SSError}\)
values for all the strategies in the tournament, as reported
in~\cite{Stewart2012} the extortionate strategy (which has an expected
\(\text{SSError}\) approximately 0) gains a large number of wins.

\begin{figure}[!htbp]
    \centering
    \includegraphics[width=.8\textwidth]{./assets/img/SSError_overall_in_stewart_plotkin/main.pdf}
    \caption{\(\text{SSError}\) and state probabilities for the strategies
        of~\cite{Stewart2012}, ordered both by number of wins and overall score.
        Note that \(P(DC)\) is not shown as it corresponds to the transpose of
        \(P(CD)\). Cooperator and Defector are omitted as they do not visit all
        the states.}
    \label{fig:SSError_overall_in_stewart_plotkin}
\end{figure}

Here, the work of~\cite{Stewart2012} is extended by investigating a tournament
with \input{assets/tex/number_of_full_strategies/main.tex}
strategies.

The results of this analysis are shown in
Figure~\ref{fig:SSError_and_probabilities_in_full}. The top ranking strategies
by number of wins seem to be extortionate (but not against all strategies) and
it can be seen that a small sub group of strategies achieve mutual defection.
All the top ranking strategies according to score achieve mutual cooperation and
do not extort each other, however they
\textbf{do} exhibit extortionate behaviour towards a number of the lower ranking
strategies.

\begin{figure}[!htbp]
    \centering
    \includegraphics[width=.8\textwidth]{./assets/img/SSError_and_probabilities_in_full/main.pdf}
    \caption{\(\text{SSError}\) for the strategies for the full tournament. Only
    strategy interactions for which \(p_4=0\) and \(\chi>1\) are displayed.}
    \label{fig:SSError_and_probabilities_in_full}
\end{figure}

\section{Conclusion}\label{sec:conclusion}

This work defines an approach to measure whether or not a player is playing a
strategy that corresponds to an extortionate strategy as defined
in~\cite{Press2012}: a mathematical model for suspicion. Indeed, all
extortionate strategies have been
 classified as lying on a triangular plane.
This rigorous classification fails to be robust to small measurement error, thus
a statistical approach is proposed.
This is done through a linear algebraic approach for approximating the solution
of a linear system. Using this, a large number of pairwise interactions is
simulated and in fact very few strategies are found to act extortionately.

The work of~\cite{Press2012}, whilst showing that a clever approach to taking
advantage of another memory one strategy exists: this is incomplete. Whilst the
elegance of this result is very attractive, just as the simplicity of the
victory of Tit For Tat in Axelrod's original tournaments was, it is incomplete.
Extortionate strategies achieve a high number of wins but they do not
achieve a high score which corresponds to the fitness landscape in an
evolutionary sense. From the large number of interactions a payoff matrix \(S\)
can be measured where \(S_{ij}\) denotes the score (using standard values of
\((R, S, T, P) = (3, 0, 5, 1)\)) of the \(i\)th strategy
against the \(j\)th strategy. Using this, the replicator equation
describes the evolution of the system based on a population density fitness
function:

\begin{equation}\label{eqn:replicator_dynamics}
    \frac{dx}{dt} = x(S-x^TS x)
\end{equation}

Equation (\ref{eqn:replicator_dynamics}) is solved numerically through an
integration technique described in~\cite{Petzold1983} and
Figure~\ref{fig:replicator_dynamics} shows the evolution of the distribution of
the system: the various strategies are ranked by scores. It is clear to see that
only the high ranking strategies survive the evolutionary process (in fact,
only \input{./assets/img/replicator_dynamics/main.tex}
have a final distribution greater than \(10 ^ {-2}\)). This confirms the
findings of~\cite{Moran1707} in which sophisticated strategies resist
evolutionary invasion of shorter memory strategies. Recalling
Figure~\ref{fig:SSError_and_probabilities_in_full} this demonstrates that:

\begin{itemize}
    \item Cooperation emerges through the evolutionary process: the high scoring
        strategies do not exhibit extortionate behaviour towards each other.
    \item Extortionate strategies do not survive the evolutionary process.
\end{itemize}

\begin{figure}[!htbp]
    \centering
    \includegraphics[width=.8\textwidth]{./assets/img/replicator_dynamics/main.pdf}
    \caption{Numerical simulation of the replicator equation
    (\ref{eqn:replicator_dynamics}): strategies are ordered by score, only the strategies with a high score survive the evolutionary process.}
    \label{fig:replicator_dynamics}
\end{figure}

This work can be used to classify plays of the IPD\@: data can be collected from
actual interactions (in lab or in the field). Furthermore, this allows for a
classification method similar to the notion of fingerprinting presented
in~\cite{Ashlock2008}. Trained strategies can potentially be classified as
extortionate or not or it could be possible to even constrain the reinforcement
learning approaches that are becoming prevalent in the literature.
Alternatively, this mathematical approach for recognising extortion could be
used in sophisticated strategies to defend against invasion. Arguably, some of
the strategies considered here exhibit this behaviour, indeed as described
in~\cite{Harper2017}, the top ranking strategies in the full tournament are
obtained using evolutionary reinforcement learning techniques, thus, suspicion
of extortionate behaviour could in fact be an evolutionary trait.

\section*{Acknowledgements}

The following open source software libraries were used in this research:

\begin{itemize}
    \item The Axelrod ~\cite{Knight2016, Knight2018} library (IPD strategies and
        tournaments).
    \item The sympy library~\cite{Meurer2017} (verification of all symbolic
        calculations).
    \item The matplotlib~\cite{Droettboom2018} library (visualisation).
    \item The pandas~\cite{Structures2010}, dask~\cite{Dask2016} and
        NumPy~\cite{Oliphant2015} libraries (data manipulation).
    \item The SciPy~\cite{Jones2001} library (numerical integration of the
        replicator equation).
\end{itemize}

This work was performed using the computational facilities of the Advanced
Research Computing @ Cardiff (ARCCA) Division, Cardiff University.

\printbibliography

\newpage
\section*{Supplementary materials}

\includepdf{assets/pdf/proof_of_form_of_extortionate_strategies/main.pdf}

\newpage

Using the pair wise interactions the transition rates \(p,
q\) can be measured and the steady state probabilities inferred and compared to
the actual probabilities of each state.
This is done numerically by computing the singular eigenvector of the
matrix \(A\) \cite{Stewart2009}:

\[
    A =
    \begin{bmatrix}
        p_1 q_1 & p_1 (1 - q_1) & (1 - p_1) q_1 & (1 -p_1) (1 - q_1) \\
        p_2 q_2 & p_2 (1 - q_2) & (1 - p_2) q_2 & (1 -p_2) (1 - q_2) \\
        p_3 q_3 & p_3 (1 - q_3) & (1 - p_3) q_3 & (1 -p_3) (1 - q_3) \\
        p_4 q_4 & p_4 (1 - q_4) & (1 - p_4) q_4 & (1 -p_4) (1 - q_4) \\
    \end{bmatrix}
\]

Figure~\ref{fig:computed_probabilities_vs_theoretic_probabilities} shows a
regression line fitted to every pairwise interaction with a reported
\(\text{SSError}\) value (pairwise interactions with missing states were
omitted). This serves to validate the approach: a part from some edge cases the
relationship is consistent.

\begin{figure}[!htbp]
    \centering
    \includegraphics[width=.8\textwidth]{./assets/img/computed_probabilities_vs_theoretic_probabilities/main.pdf}
    \caption{The
        relationship between the steady state probabilities inferred from the
        measured transitions and the actual steady state probabilities. A linear
        regression line is included validating the approach.}
    \label{fig:computed_probabilities_vs_theoretic_probabilities}
\end{figure}


\end{document}

    strategies is considered. In this setting
    the most highly performing strategies do not play in an extortionate way
    against each other but do against lower performing strategies.
    This suggests that whilst the theory of Zero Determinant strategies
    indicates that memory is not of fundamental importance to the evolution of
    cooperative behaviour, this is incomplete.
\end{abstract}

\section{Introduction}\label{sec:introduction}

Agent based game theoretic models have become a stalwart of the underpinning
mathematics of interactive behaviours. One of the major pieces of work
in this area is the pair of original computer tournaments run by Robert
Axelrod~\cite{Axelrod1980, Axelrod1980a}. These tournaments pitted submitted
computer strategies against each other in plays of the Iterated Prisoner's
Dilemma. A common game where agents can choose to pay a slight cost to their
immediate utility in the hope of building a reputation. This has been used in
economic and evolutionary game theory to understand the evolution of cooperative
behaviour.

Recently, a class of strategies was described in~\cite{Press2012} that can
provably extort any given opponent. In~\cite{Hilbe2013, Moran1707} some
questions have already been asked about the true effectiveness of these
strategies in an evolutionary setting. Here another question is asked: is it
possible to recognise this extortionate behaviour? A mathematical procedure for
suspicion is presented: in the same way that the continued actions of an
extortionate individual might raise suspicion.

This work makes use of the Axelrod Python library~\cite{Knight2018, Knight2016}
with a large number of Prisoner Dilemma strategies available to give an
extensive numerical example of the ideas presented.  The approach is presented
in Section~\ref{sec:delta-zd-strategies}.  All of the code and data discussed
in Section~\ref{sec:numerical-experiments} is open sourced, archived and
written according to best scientific principles~\cite{Wilson2014}. The data
archive can be found at~\cite{vincent_knight_2018_1297075}.

\section{Recognising Extortion}\label{sec:delta-zd-strategies}

In~\cite{Press2012}, given a match between 2 memory-one strategies, the concept
of Zero Determinant (ZD) strategies is introduced. The main result of that paper
shows that given two memory one players \(p, q\in\mathbb{R}^4\) a linear
relationship between the players' scores could be forced by one of the players.

Using the notation of~\cite{Press2012}, assuming the utilities for player \(p\)
are given by \(S_x=(R, S, T, P)\) and for player \(q\) by \(S_y=(R, T, S, P)\)
and that the stationary scores of each player is given by \(S_X\) and \(S_Y\)
respectively. The main result of~\cite{Press2012} is that if

\begin{equation}\label{eqn:linear_relationship_for_p}
    \tilde p=\alpha S_x + \beta S_y + \gamma
\end{equation}

or

\begin{equation}\label{eqn:linear_relationship_for_q}
    \tilde q=\alpha S_x + \beta S_y + \gamma
\end{equation}

where \(\tilde p = (1 - p_1, 1 - p_2, p_3, p_4)\) and
\(\tilde q = (1 - q_1, 1 - q_2, q_3, q_4)\) then:

\begin{equation}
    \alpha S_X + \beta S_Y + \gamma = 0
\end{equation}

In~\cite{Press2012} a particular type of ZD strategy is defined: extortionate
strategies. If:

\begin{equation}\label{eqn:constraint_for_extortion}
    \gamma = - P(\alpha + \beta)
\end{equation}

then the player can ensure they get a score \(\chi\) times
larger than the opponent. This extortion coefficient is given by:

\begin{equation}\label{eqn:definition_of_chi}
    \chi=\frac{-\beta}{\alpha}
\end{equation}

Thus, if (\ref{eqn:constraint_for_extortion}) holds and \(\chi >1\) a player is
said to extort their opponent.
Here, the reverse problem is considered: given a
\(p\in\mathbb{R}^4\) how does one identify \(\alpha, \beta\) if they
exist and is the strategy in fact acting in an extortionate way?

These conditions correspond to:

\begin{align}
    \tilde p_1 & = \alpha R + \beta R - P (\alpha + \beta)
            \label{eqn:condition_for_tilde_p1}\\
    \tilde p_2 & = \alpha S + \beta T - P (\alpha + \beta)
            \label{eqn:condition_for_tilde_p2}\\
    \tilde p_3 & = \alpha T + \beta S - P (\alpha + \beta)
            \label{eqn:condition_for_tilde_p3}\\
    \tilde p_4 & = \alpha P + \beta P - P (\alpha + \beta)
            \label{eqn:condition_for_tilde_p4}
\end{align}

Equation (\ref{eqn:condition_for_tilde_p4}) ensures that \(p_4=\tilde p_4=0\).
Equations (\ref{eqn:condition_for_tilde_p1}-\ref{eqn:condition_for_tilde_p3})
can be used to eliminate \(\alpha, \beta\), giving:

\begin{equation}\label{eqn:planar_definition_of_extortion}
    \tilde p_1 = \frac{(R - P)(\tilde p_2 + \tilde p_3)}{S + T - 2P}
\end{equation}

with:

\begin{equation}\label{eqn:definition_of_chi}
    \chi = \frac{\tilde p_2 (P - T) + \tilde p_3 (S - P)}
                {\tilde p_2 (P - S) + \tilde p_3 (T - P)}
\end{equation}

Given a strategy \(p\in\mathbb{R}^{4\times 1}\) equations
(\ref{eqn:condition_for_tilde_p4}), (\ref{eqn:planar_definition_of_extortion}-\ref{eqn:definition_of_chi}) can be used to check if
a strategy is extortionate. The conditions correspond to:

\begin{align}
    p_1 & = \frac{(R-P)(p_2 + p_3) - R + T + S - P}{S + T - 2P}
     \label{eqn:condition_for_p1}\\
    p_4 & = 0 \label{eqn:condition_for_p4}\\
    1 & > p_2 + p_3\label{eqn:condition_for_chi}
\end{align}

The algebraic steps necessary to prove these results are available in the
supporting materials.

All extortionate strategies reside on a triangular (\ref{eqn:condition_for_chi})
plane (\ref{eqn:condition_for_p1}) in 3 dimensions (\ref{eqn:condition_for_p4}).
Using this formulation it can be seen that a necessary (but not sufficient)
condition for an extortionate strategy is that it cooperates on average less
than 50\% of the time when in a state of disagreement with the opponent.

As an example, consider the known extortionate strategy \(p=(8 / 9, 1 / 2, 1 /
3, 0)\) from~\cite{Stewart2012} which is referred to as \texttt{Extort-2}. In
this case, for the standard values of \((R, T, S, P)\) constraint
(\ref{eqn:condition_for_p1}) corresponds to:

\begin{equation}
    p_1 = \frac{2(p_2 + p_3) + 1}{3}
\end{equation}

It is clear that in this case all constraints hold.

This approach could in fact be used to confirm that a given strategy is acting
in an extortionate manner even if it is not a memory one strategy. However, in
practice, if a closed form for \(p\) is not known, then due to measurement
and/or numerical error this would not work.

This problem can be written in the following linear algebraic form where
\(x=(\alpha, \beta)\)
and \(p^*=(\tilde p_1 - 1, tilde_2 - 1, p_3)\):

\begin{equation}\label{eqn:linear_algebraic_equation_for_p}
    Cx= p^*
\end{equation}

\(C\) corresponds to equations
(\ref{eqn:condition_for_tilde_p1}-\ref{eqn:condition_for_tilde_p3}) and is
given by:

\begin{equation}\label{eqn:definition_of_C}
    C =
    \begin{bmatrix}
        R - P & R- P \\
        S - P & T- P \\
        T - P & S- P \\
    \end{bmatrix}
\end{equation}

Note that in general, equation (\ref{eqn:linear_algebraic_equation_for_p}) will
not necessarily have a solution. From the Rouch\'{e}-Capelli theorem if there is
a solution it is unique as \(\text{rank}(C)=2\) which is the dimension of the
variable \(x\). The best fitting \(x\) is found by minimizing:

\begin{equation}\label{eqn:r_squared}
    \text{SSError} = \|C x- p^*\|_2^2 = \sum_{i=1}^{3}\left((C\bar x)_i-p_i^*\right)^2
\end{equation}

Note that \(\text{SSError}\), which is the square of the Frobenius
norm~\cite{Golub2013}, becomes a measure of how close a strategy is to being an
extortionate strategy. Suspicion
of extortion then corresponds to a threshold on \(\text{SSError}\).

By observing interactions (human or otherwise), their memory one representation
can be inferred and this approach can be used to recognise extortionate
behaviour. The notion of comparing theoretic and actual plays of the IPD is not
novel, see for example~\cite{Rand2013}. Immediately it is noted that if the
environment is noisy~\cite{Wu1995} then no strategy can be considered to be
extortionate as \(p_4>0\).

In the next section, this idea will be illustrated by observing the interactions
that take place in a computer based tournament of the IPD\@.

\section{Numerical experiments}\label{sec:numerical-experiments}

In~\cite{Stewart2012} results from a tournament with
\documentclass[a4paper]{article}

\usepackage{amsmath}
\usepackage{amssymb}
\usepackage[margin=1.5cm,
            includefoot,
            footskip=30pt]{geometry}
\usepackage{layout}
\usepackage{graphicx}
\usepackage{subcaption}

\usepackage{biblatex}
\usepackage{pdfpages}

\bibliography{main.bib}

\title{Suspicion: Recognising and evaluating the effectiveness
       of extortion in the Iterated Prisoner's Dilemma}
\author{Vincent A. Knight \and Nikoleta E. Glynatsi}
\date{\today}



\begin{document}

\maketitle

\begin{abstract}
    The Iterated Prisoner's Dilemma is a model for rational and evolutionary
    interactive behaviour. It has applications both in the study of human social
    behaviour as well as in biology.
    It is used to understand when and how a rational individual might
    accept an immediate cost to their own utility for the direct benefit of
    another.

    Much attention has been given to a class of strategies called
    Zero Determinant strategies. It has been theoretically shown that these
    strategies can ``extort'' any player.

    In this work, an approach to identify if observed strategies are playing in
    an extortionate way is described. Furthermore, experimental analysis of
    a large tournament with \input{assets/tex/number_of_full_strategies/main.tex}
    strategies is considered. In this setting
    the most highly performing strategies do not play in an extortionate way
    against each other but do against lower performing strategies.
    This suggests that whilst the theory of Zero Determinant strategies
    indicates that memory is not of fundamental importance to the evolution of
    cooperative behaviour, this is incomplete.
\end{abstract}

\section{Introduction}\label{sec:introduction}

Agent based game theoretic models have become a stalwart of the underpinning
mathematics of interactive behaviours. One of the major pieces of work
in this area is the pair of original computer tournaments run by Robert
Axelrod~\cite{Axelrod1980, Axelrod1980a}. These tournaments pitted submitted
computer strategies against each other in plays of the Iterated Prisoner's
Dilemma. A common game where agents can choose to pay a slight cost to their
immediate utility in the hope of building a reputation. This has been used in
economic and evolutionary game theory to understand the evolution of cooperative
behaviour.

Recently, a class of strategies was described in~\cite{Press2012} that can
provably extort any given opponent. In~\cite{Hilbe2013, Moran1707} some
questions have already been asked about the true effectiveness of these
strategies in an evolutionary setting. Here another question is asked: is it
possible to recognise this extortionate behaviour? A mathematical procedure for
suspicion is presented: in the same way that the continued actions of an
extortionate individual might raise suspicion.

This work makes use of the Axelrod Python library~\cite{Knight2018, Knight2016}
with a large number of Prisoner Dilemma strategies available to give an
extensive numerical example of the ideas presented.  The approach is presented
in Section~\ref{sec:delta-zd-strategies}.  All of the code and data discussed
in Section~\ref{sec:numerical-experiments} is open sourced, archived and
written according to best scientific principles~\cite{Wilson2014}. The data
archive can be found at~\cite{vincent_knight_2018_1297075}.

\section{Recognising Extortion}\label{sec:delta-zd-strategies}

In~\cite{Press2012}, given a match between 2 memory-one strategies, the concept
of Zero Determinant (ZD) strategies is introduced. The main result of that paper
shows that given two memory one players \(p, q\in\mathbb{R}^4\) a linear
relationship between the players' scores could be forced by one of the players.

Using the notation of~\cite{Press2012}, assuming the utilities for player \(p\)
are given by \(S_x=(R, S, T, P)\) and for player \(q\) by \(S_y=(R, T, S, P)\)
and that the stationary scores of each player is given by \(S_X\) and \(S_Y\)
respectively. The main result of~\cite{Press2012} is that if

\begin{equation}\label{eqn:linear_relationship_for_p}
    \tilde p=\alpha S_x + \beta S_y + \gamma
\end{equation}

or

\begin{equation}\label{eqn:linear_relationship_for_q}
    \tilde q=\alpha S_x + \beta S_y + \gamma
\end{equation}

where \(\tilde p = (1 - p_1, 1 - p_2, p_3, p_4)\) and
\(\tilde q = (1 - q_1, 1 - q_2, q_3, q_4)\) then:

\begin{equation}
    \alpha S_X + \beta S_Y + \gamma = 0
\end{equation}

In~\cite{Press2012} a particular type of ZD strategy is defined: extortionate
strategies. If:

\begin{equation}\label{eqn:constraint_for_extortion}
    \gamma = - P(\alpha + \beta)
\end{equation}

then the player can ensure they get a score \(\chi\) times
larger than the opponent. This extortion coefficient is given by:

\begin{equation}\label{eqn:definition_of_chi}
    \chi=\frac{-\beta}{\alpha}
\end{equation}

Thus, if (\ref{eqn:constraint_for_extortion}) holds and \(\chi >1\) a player is
said to extort their opponent.
Here, the reverse problem is considered: given a
\(p\in\mathbb{R}^4\) how does one identify \(\alpha, \beta\) if they
exist and is the strategy in fact acting in an extortionate way?

These conditions correspond to:

\begin{align}
    \tilde p_1 & = \alpha R + \beta R - P (\alpha + \beta)
            \label{eqn:condition_for_tilde_p1}\\
    \tilde p_2 & = \alpha S + \beta T - P (\alpha + \beta)
            \label{eqn:condition_for_tilde_p2}\\
    \tilde p_3 & = \alpha T + \beta S - P (\alpha + \beta)
            \label{eqn:condition_for_tilde_p3}\\
    \tilde p_4 & = \alpha P + \beta P - P (\alpha + \beta)
            \label{eqn:condition_for_tilde_p4}
\end{align}

Equation (\ref{eqn:condition_for_tilde_p4}) ensures that \(p_4=\tilde p_4=0\).
Equations (\ref{eqn:condition_for_tilde_p1}-\ref{eqn:condition_for_tilde_p3})
can be used to eliminate \(\alpha, \beta\), giving:

\begin{equation}\label{eqn:planar_definition_of_extortion}
    \tilde p_1 = \frac{(R - P)(\tilde p_2 + \tilde p_3)}{S + T - 2P}
\end{equation}

with:

\begin{equation}\label{eqn:definition_of_chi}
    \chi = \frac{\tilde p_2 (P - T) + \tilde p_3 (S - P)}
                {\tilde p_2 (P - S) + \tilde p_3 (T - P)}
\end{equation}

Given a strategy \(p\in\mathbb{R}^{4\times 1}\) equations
(\ref{eqn:condition_for_tilde_p4}), (\ref{eqn:planar_definition_of_extortion}-\ref{eqn:definition_of_chi}) can be used to check if
a strategy is extortionate. The conditions correspond to:

\begin{align}
    p_1 & = \frac{(R-P)(p_2 + p_3) - R + T + S - P}{S + T - 2P}
     \label{eqn:condition_for_p1}\\
    p_4 & = 0 \label{eqn:condition_for_p4}\\
    1 & > p_2 + p_3\label{eqn:condition_for_chi}
\end{align}

The algebraic steps necessary to prove these results are available in the
supporting materials.

All extortionate strategies reside on a triangular (\ref{eqn:condition_for_chi})
plane (\ref{eqn:condition_for_p1}) in 3 dimensions (\ref{eqn:condition_for_p4}).
Using this formulation it can be seen that a necessary (but not sufficient)
condition for an extortionate strategy is that it cooperates on average less
than 50\% of the time when in a state of disagreement with the opponent.

As an example, consider the known extortionate strategy \(p=(8 / 9, 1 / 2, 1 /
3, 0)\) from~\cite{Stewart2012} which is referred to as \texttt{Extort-2}. In
this case, for the standard values of \((R, T, S, P)\) constraint
(\ref{eqn:condition_for_p1}) corresponds to:

\begin{equation}
    p_1 = \frac{2(p_2 + p_3) + 1}{3}
\end{equation}

It is clear that in this case all constraints hold.

This approach could in fact be used to confirm that a given strategy is acting
in an extortionate manner even if it is not a memory one strategy. However, in
practice, if a closed form for \(p\) is not known, then due to measurement
and/or numerical error this would not work.

This problem can be written in the following linear algebraic form where
\(x=(\alpha, \beta)\)
and \(p^*=(\tilde p_1 - 1, tilde_2 - 1, p_3)\):

\begin{equation}\label{eqn:linear_algebraic_equation_for_p}
    Cx= p^*
\end{equation}

\(C\) corresponds to equations
(\ref{eqn:condition_for_tilde_p1}-\ref{eqn:condition_for_tilde_p3}) and is
given by:

\begin{equation}\label{eqn:definition_of_C}
    C =
    \begin{bmatrix}
        R - P & R- P \\
        S - P & T- P \\
        T - P & S- P \\
    \end{bmatrix}
\end{equation}

Note that in general, equation (\ref{eqn:linear_algebraic_equation_for_p}) will
not necessarily have a solution. From the Rouch\'{e}-Capelli theorem if there is
a solution it is unique as \(\text{rank}(C)=2\) which is the dimension of the
variable \(x\). The best fitting \(x\) is found by minimizing:

\begin{equation}\label{eqn:r_squared}
    \text{SSError} = \|C x- p^*\|_2^2 = \sum_{i=1}^{3}\left((C\bar x)_i-p_i^*\right)^2
\end{equation}

Note that \(\text{SSError}\), which is the square of the Frobenius
norm~\cite{Golub2013}, becomes a measure of how close a strategy is to being an
extortionate strategy. Suspicion
of extortion then corresponds to a threshold on \(\text{SSError}\).

By observing interactions (human or otherwise), their memory one representation
can be inferred and this approach can be used to recognise extortionate
behaviour. The notion of comparing theoretic and actual plays of the IPD is not
novel, see for example~\cite{Rand2013}. Immediately it is noted that if the
environment is noisy~\cite{Wu1995} then no strategy can be considered to be
extortionate as \(p_4>0\).

In the next section, this idea will be illustrated by observing the interactions
that take place in a computer based tournament of the IPD\@.

\section{Numerical experiments}\label{sec:numerical-experiments}

In~\cite{Stewart2012} results from a tournament with
\input{./assets/tex/number_of_stewart_plotkin_strategies/main.tex} strategies,
was presented with specific consideration given to ZD strategies. This
tournament is reproduced here using the Axelrod-Python
project~\cite{Knight2016}. To obtain a good measure of the corresponding
transition rates for each strategy all matches have been run for
\input{assets/tex/number_of_turns/main.tex} turns and every match has been
repeated \input{assets/tex/number_of_repetitions/main.tex} times. All of this
interaction data is available at~\cite{vincent_knight_2018_1297075}. A good
match between the inferred Markov chain and the state distribution of the actual
interactions has been verified. Data for this is presented in the supplementary
materials.

Figure~\ref{fig:SSError_overall_in_stewart_plotkin} shows the \(\text{SSError}\)
values for all the strategies in the tournament, as reported
in~\cite{Stewart2012} the extortionate strategy (which has an expected
\(\text{SSError}\) approximately 0) gains a large number of wins.

\begin{figure}[!htbp]
    \centering
    \includegraphics[width=.8\textwidth]{./assets/img/SSError_overall_in_stewart_plotkin/main.pdf}
    \caption{\(\text{SSError}\) and state probabilities for the strategies
        of~\cite{Stewart2012}, ordered both by number of wins and overall score.
        Note that \(P(DC)\) is not shown as it corresponds to the transpose of
        \(P(CD)\). Cooperator and Defector are omitted as they do not visit all
        the states.}
    \label{fig:SSError_overall_in_stewart_plotkin}
\end{figure}

Here, the work of~\cite{Stewart2012} is extended by investigating a tournament
with \input{assets/tex/number_of_full_strategies/main.tex}
strategies.

The results of this analysis are shown in
Figure~\ref{fig:SSError_and_probabilities_in_full}. The top ranking strategies
by number of wins seem to be extortionate (but not against all strategies) and
it can be seen that a small sub group of strategies achieve mutual defection.
All the top ranking strategies according to score achieve mutual cooperation and
do not extort each other, however they
\textbf{do} exhibit extortionate behaviour towards a number of the lower ranking
strategies.

\begin{figure}[!htbp]
    \centering
    \includegraphics[width=.8\textwidth]{./assets/img/SSError_and_probabilities_in_full/main.pdf}
    \caption{\(\text{SSError}\) for the strategies for the full tournament. Only
    strategy interactions for which \(p_4=0\) and \(\chi>1\) are displayed.}
    \label{fig:SSError_and_probabilities_in_full}
\end{figure}

\section{Conclusion}\label{sec:conclusion}

This work defines an approach to measure whether or not a player is playing a
strategy that corresponds to an extortionate strategy as defined
in~\cite{Press2012}: a mathematical model for suspicion. Indeed, all
extortionate strategies have been
 classified as lying on a triangular plane.
This rigorous classification fails to be robust to small measurement error, thus
a statistical approach is proposed.
This is done through a linear algebraic approach for approximating the solution
of a linear system. Using this, a large number of pairwise interactions is
simulated and in fact very few strategies are found to act extortionately.

The work of~\cite{Press2012}, whilst showing that a clever approach to taking
advantage of another memory one strategy exists: this is incomplete. Whilst the
elegance of this result is very attractive, just as the simplicity of the
victory of Tit For Tat in Axelrod's original tournaments was, it is incomplete.
Extortionate strategies achieve a high number of wins but they do not
achieve a high score which corresponds to the fitness landscape in an
evolutionary sense. From the large number of interactions a payoff matrix \(S\)
can be measured where \(S_{ij}\) denotes the score (using standard values of
\((R, S, T, P) = (3, 0, 5, 1)\)) of the \(i\)th strategy
against the \(j\)th strategy. Using this, the replicator equation
describes the evolution of the system based on a population density fitness
function:

\begin{equation}\label{eqn:replicator_dynamics}
    \frac{dx}{dt} = x(S-x^TS x)
\end{equation}

Equation (\ref{eqn:replicator_dynamics}) is solved numerically through an
integration technique described in~\cite{Petzold1983} and
Figure~\ref{fig:replicator_dynamics} shows the evolution of the distribution of
the system: the various strategies are ranked by scores. It is clear to see that
only the high ranking strategies survive the evolutionary process (in fact,
only \input{./assets/img/replicator_dynamics/main.tex}
have a final distribution greater than \(10 ^ {-2}\)). This confirms the
findings of~\cite{Moran1707} in which sophisticated strategies resist
evolutionary invasion of shorter memory strategies. Recalling
Figure~\ref{fig:SSError_and_probabilities_in_full} this demonstrates that:

\begin{itemize}
    \item Cooperation emerges through the evolutionary process: the high scoring
        strategies do not exhibit extortionate behaviour towards each other.
    \item Extortionate strategies do not survive the evolutionary process.
\end{itemize}

\begin{figure}[!htbp]
    \centering
    \includegraphics[width=.8\textwidth]{./assets/img/replicator_dynamics/main.pdf}
    \caption{Numerical simulation of the replicator equation
    (\ref{eqn:replicator_dynamics}): strategies are ordered by score, only the strategies with a high score survive the evolutionary process.}
    \label{fig:replicator_dynamics}
\end{figure}

This work can be used to classify plays of the IPD\@: data can be collected from
actual interactions (in lab or in the field). Furthermore, this allows for a
classification method similar to the notion of fingerprinting presented
in~\cite{Ashlock2008}. Trained strategies can potentially be classified as
extortionate or not or it could be possible to even constrain the reinforcement
learning approaches that are becoming prevalent in the literature.
Alternatively, this mathematical approach for recognising extortion could be
used in sophisticated strategies to defend against invasion. Arguably, some of
the strategies considered here exhibit this behaviour, indeed as described
in~\cite{Harper2017}, the top ranking strategies in the full tournament are
obtained using evolutionary reinforcement learning techniques, thus, suspicion
of extortionate behaviour could in fact be an evolutionary trait.

\section*{Acknowledgements}

The following open source software libraries were used in this research:

\begin{itemize}
    \item The Axelrod ~\cite{Knight2016, Knight2018} library (IPD strategies and
        tournaments).
    \item The sympy library~\cite{Meurer2017} (verification of all symbolic
        calculations).
    \item The matplotlib~\cite{Droettboom2018} library (visualisation).
    \item The pandas~\cite{Structures2010}, dask~\cite{Dask2016} and
        NumPy~\cite{Oliphant2015} libraries (data manipulation).
    \item The SciPy~\cite{Jones2001} library (numerical integration of the
        replicator equation).
\end{itemize}

This work was performed using the computational facilities of the Advanced
Research Computing @ Cardiff (ARCCA) Division, Cardiff University.

\printbibliography

\newpage
\section*{Supplementary materials}

\includepdf{assets/pdf/proof_of_form_of_extortionate_strategies/main.pdf}

\newpage

Using the pair wise interactions the transition rates \(p,
q\) can be measured and the steady state probabilities inferred and compared to
the actual probabilities of each state.
This is done numerically by computing the singular eigenvector of the
matrix \(A\) \cite{Stewart2009}:

\[
    A =
    \begin{bmatrix}
        p_1 q_1 & p_1 (1 - q_1) & (1 - p_1) q_1 & (1 -p_1) (1 - q_1) \\
        p_2 q_2 & p_2 (1 - q_2) & (1 - p_2) q_2 & (1 -p_2) (1 - q_2) \\
        p_3 q_3 & p_3 (1 - q_3) & (1 - p_3) q_3 & (1 -p_3) (1 - q_3) \\
        p_4 q_4 & p_4 (1 - q_4) & (1 - p_4) q_4 & (1 -p_4) (1 - q_4) \\
    \end{bmatrix}
\]

Figure~\ref{fig:computed_probabilities_vs_theoretic_probabilities} shows a
regression line fitted to every pairwise interaction with a reported
\(\text{SSError}\) value (pairwise interactions with missing states were
omitted). This serves to validate the approach: a part from some edge cases the
relationship is consistent.

\begin{figure}[!htbp]
    \centering
    \includegraphics[width=.8\textwidth]{./assets/img/computed_probabilities_vs_theoretic_probabilities/main.pdf}
    \caption{The
        relationship between the steady state probabilities inferred from the
        measured transitions and the actual steady state probabilities. A linear
        regression line is included validating the approach.}
    \label{fig:computed_probabilities_vs_theoretic_probabilities}
\end{figure}


\end{document}
 strategies,
was presented with specific consideration given to ZD strategies. This
tournament is reproduced here using the Axelrod-Python
project~\cite{Knight2016}. To obtain a good measure of the corresponding
transition rates for each strategy all matches have been run for
\documentclass[a4paper]{article}

\usepackage{amsmath}
\usepackage{amssymb}
\usepackage[margin=1.5cm,
            includefoot,
            footskip=30pt]{geometry}
\usepackage{layout}
\usepackage{graphicx}
\usepackage{subcaption}

\usepackage{biblatex}
\usepackage{pdfpages}

\bibliography{main.bib}

\title{Suspicion: Recognising and evaluating the effectiveness
       of extortion in the Iterated Prisoner's Dilemma}
\author{Vincent A. Knight \and Nikoleta E. Glynatsi}
\date{\today}



\begin{document}

\maketitle

\begin{abstract}
    The Iterated Prisoner's Dilemma is a model for rational and evolutionary
    interactive behaviour. It has applications both in the study of human social
    behaviour as well as in biology.
    It is used to understand when and how a rational individual might
    accept an immediate cost to their own utility for the direct benefit of
    another.

    Much attention has been given to a class of strategies called
    Zero Determinant strategies. It has been theoretically shown that these
    strategies can ``extort'' any player.

    In this work, an approach to identify if observed strategies are playing in
    an extortionate way is described. Furthermore, experimental analysis of
    a large tournament with \input{assets/tex/number_of_full_strategies/main.tex}
    strategies is considered. In this setting
    the most highly performing strategies do not play in an extortionate way
    against each other but do against lower performing strategies.
    This suggests that whilst the theory of Zero Determinant strategies
    indicates that memory is not of fundamental importance to the evolution of
    cooperative behaviour, this is incomplete.
\end{abstract}

\section{Introduction}\label{sec:introduction}

Agent based game theoretic models have become a stalwart of the underpinning
mathematics of interactive behaviours. One of the major pieces of work
in this area is the pair of original computer tournaments run by Robert
Axelrod~\cite{Axelrod1980, Axelrod1980a}. These tournaments pitted submitted
computer strategies against each other in plays of the Iterated Prisoner's
Dilemma. A common game where agents can choose to pay a slight cost to their
immediate utility in the hope of building a reputation. This has been used in
economic and evolutionary game theory to understand the evolution of cooperative
behaviour.

Recently, a class of strategies was described in~\cite{Press2012} that can
provably extort any given opponent. In~\cite{Hilbe2013, Moran1707} some
questions have already been asked about the true effectiveness of these
strategies in an evolutionary setting. Here another question is asked: is it
possible to recognise this extortionate behaviour? A mathematical procedure for
suspicion is presented: in the same way that the continued actions of an
extortionate individual might raise suspicion.

This work makes use of the Axelrod Python library~\cite{Knight2018, Knight2016}
with a large number of Prisoner Dilemma strategies available to give an
extensive numerical example of the ideas presented.  The approach is presented
in Section~\ref{sec:delta-zd-strategies}.  All of the code and data discussed
in Section~\ref{sec:numerical-experiments} is open sourced, archived and
written according to best scientific principles~\cite{Wilson2014}. The data
archive can be found at~\cite{vincent_knight_2018_1297075}.

\section{Recognising Extortion}\label{sec:delta-zd-strategies}

In~\cite{Press2012}, given a match between 2 memory-one strategies, the concept
of Zero Determinant (ZD) strategies is introduced. The main result of that paper
shows that given two memory one players \(p, q\in\mathbb{R}^4\) a linear
relationship between the players' scores could be forced by one of the players.

Using the notation of~\cite{Press2012}, assuming the utilities for player \(p\)
are given by \(S_x=(R, S, T, P)\) and for player \(q\) by \(S_y=(R, T, S, P)\)
and that the stationary scores of each player is given by \(S_X\) and \(S_Y\)
respectively. The main result of~\cite{Press2012} is that if

\begin{equation}\label{eqn:linear_relationship_for_p}
    \tilde p=\alpha S_x + \beta S_y + \gamma
\end{equation}

or

\begin{equation}\label{eqn:linear_relationship_for_q}
    \tilde q=\alpha S_x + \beta S_y + \gamma
\end{equation}

where \(\tilde p = (1 - p_1, 1 - p_2, p_3, p_4)\) and
\(\tilde q = (1 - q_1, 1 - q_2, q_3, q_4)\) then:

\begin{equation}
    \alpha S_X + \beta S_Y + \gamma = 0
\end{equation}

In~\cite{Press2012} a particular type of ZD strategy is defined: extortionate
strategies. If:

\begin{equation}\label{eqn:constraint_for_extortion}
    \gamma = - P(\alpha + \beta)
\end{equation}

then the player can ensure they get a score \(\chi\) times
larger than the opponent. This extortion coefficient is given by:

\begin{equation}\label{eqn:definition_of_chi}
    \chi=\frac{-\beta}{\alpha}
\end{equation}

Thus, if (\ref{eqn:constraint_for_extortion}) holds and \(\chi >1\) a player is
said to extort their opponent.
Here, the reverse problem is considered: given a
\(p\in\mathbb{R}^4\) how does one identify \(\alpha, \beta\) if they
exist and is the strategy in fact acting in an extortionate way?

These conditions correspond to:

\begin{align}
    \tilde p_1 & = \alpha R + \beta R - P (\alpha + \beta)
            \label{eqn:condition_for_tilde_p1}\\
    \tilde p_2 & = \alpha S + \beta T - P (\alpha + \beta)
            \label{eqn:condition_for_tilde_p2}\\
    \tilde p_3 & = \alpha T + \beta S - P (\alpha + \beta)
            \label{eqn:condition_for_tilde_p3}\\
    \tilde p_4 & = \alpha P + \beta P - P (\alpha + \beta)
            \label{eqn:condition_for_tilde_p4}
\end{align}

Equation (\ref{eqn:condition_for_tilde_p4}) ensures that \(p_4=\tilde p_4=0\).
Equations (\ref{eqn:condition_for_tilde_p1}-\ref{eqn:condition_for_tilde_p3})
can be used to eliminate \(\alpha, \beta\), giving:

\begin{equation}\label{eqn:planar_definition_of_extortion}
    \tilde p_1 = \frac{(R - P)(\tilde p_2 + \tilde p_3)}{S + T - 2P}
\end{equation}

with:

\begin{equation}\label{eqn:definition_of_chi}
    \chi = \frac{\tilde p_2 (P - T) + \tilde p_3 (S - P)}
                {\tilde p_2 (P - S) + \tilde p_3 (T - P)}
\end{equation}

Given a strategy \(p\in\mathbb{R}^{4\times 1}\) equations
(\ref{eqn:condition_for_tilde_p4}), (\ref{eqn:planar_definition_of_extortion}-\ref{eqn:definition_of_chi}) can be used to check if
a strategy is extortionate. The conditions correspond to:

\begin{align}
    p_1 & = \frac{(R-P)(p_2 + p_3) - R + T + S - P}{S + T - 2P}
     \label{eqn:condition_for_p1}\\
    p_4 & = 0 \label{eqn:condition_for_p4}\\
    1 & > p_2 + p_3\label{eqn:condition_for_chi}
\end{align}

The algebraic steps necessary to prove these results are available in the
supporting materials.

All extortionate strategies reside on a triangular (\ref{eqn:condition_for_chi})
plane (\ref{eqn:condition_for_p1}) in 3 dimensions (\ref{eqn:condition_for_p4}).
Using this formulation it can be seen that a necessary (but not sufficient)
condition for an extortionate strategy is that it cooperates on average less
than 50\% of the time when in a state of disagreement with the opponent.

As an example, consider the known extortionate strategy \(p=(8 / 9, 1 / 2, 1 /
3, 0)\) from~\cite{Stewart2012} which is referred to as \texttt{Extort-2}. In
this case, for the standard values of \((R, T, S, P)\) constraint
(\ref{eqn:condition_for_p1}) corresponds to:

\begin{equation}
    p_1 = \frac{2(p_2 + p_3) + 1}{3}
\end{equation}

It is clear that in this case all constraints hold.

This approach could in fact be used to confirm that a given strategy is acting
in an extortionate manner even if it is not a memory one strategy. However, in
practice, if a closed form for \(p\) is not known, then due to measurement
and/or numerical error this would not work.

This problem can be written in the following linear algebraic form where
\(x=(\alpha, \beta)\)
and \(p^*=(\tilde p_1 - 1, tilde_2 - 1, p_3)\):

\begin{equation}\label{eqn:linear_algebraic_equation_for_p}
    Cx= p^*
\end{equation}

\(C\) corresponds to equations
(\ref{eqn:condition_for_tilde_p1}-\ref{eqn:condition_for_tilde_p3}) and is
given by:

\begin{equation}\label{eqn:definition_of_C}
    C =
    \begin{bmatrix}
        R - P & R- P \\
        S - P & T- P \\
        T - P & S- P \\
    \end{bmatrix}
\end{equation}

Note that in general, equation (\ref{eqn:linear_algebraic_equation_for_p}) will
not necessarily have a solution. From the Rouch\'{e}-Capelli theorem if there is
a solution it is unique as \(\text{rank}(C)=2\) which is the dimension of the
variable \(x\). The best fitting \(x\) is found by minimizing:

\begin{equation}\label{eqn:r_squared}
    \text{SSError} = \|C x- p^*\|_2^2 = \sum_{i=1}^{3}\left((C\bar x)_i-p_i^*\right)^2
\end{equation}

Note that \(\text{SSError}\), which is the square of the Frobenius
norm~\cite{Golub2013}, becomes a measure of how close a strategy is to being an
extortionate strategy. Suspicion
of extortion then corresponds to a threshold on \(\text{SSError}\).

By observing interactions (human or otherwise), their memory one representation
can be inferred and this approach can be used to recognise extortionate
behaviour. The notion of comparing theoretic and actual plays of the IPD is not
novel, see for example~\cite{Rand2013}. Immediately it is noted that if the
environment is noisy~\cite{Wu1995} then no strategy can be considered to be
extortionate as \(p_4>0\).

In the next section, this idea will be illustrated by observing the interactions
that take place in a computer based tournament of the IPD\@.

\section{Numerical experiments}\label{sec:numerical-experiments}

In~\cite{Stewart2012} results from a tournament with
\input{./assets/tex/number_of_stewart_plotkin_strategies/main.tex} strategies,
was presented with specific consideration given to ZD strategies. This
tournament is reproduced here using the Axelrod-Python
project~\cite{Knight2016}. To obtain a good measure of the corresponding
transition rates for each strategy all matches have been run for
\input{assets/tex/number_of_turns/main.tex} turns and every match has been
repeated \input{assets/tex/number_of_repetitions/main.tex} times. All of this
interaction data is available at~\cite{vincent_knight_2018_1297075}. A good
match between the inferred Markov chain and the state distribution of the actual
interactions has been verified. Data for this is presented in the supplementary
materials.

Figure~\ref{fig:SSError_overall_in_stewart_plotkin} shows the \(\text{SSError}\)
values for all the strategies in the tournament, as reported
in~\cite{Stewart2012} the extortionate strategy (which has an expected
\(\text{SSError}\) approximately 0) gains a large number of wins.

\begin{figure}[!htbp]
    \centering
    \includegraphics[width=.8\textwidth]{./assets/img/SSError_overall_in_stewart_plotkin/main.pdf}
    \caption{\(\text{SSError}\) and state probabilities for the strategies
        of~\cite{Stewart2012}, ordered both by number of wins and overall score.
        Note that \(P(DC)\) is not shown as it corresponds to the transpose of
        \(P(CD)\). Cooperator and Defector are omitted as they do not visit all
        the states.}
    \label{fig:SSError_overall_in_stewart_plotkin}
\end{figure}

Here, the work of~\cite{Stewart2012} is extended by investigating a tournament
with \input{assets/tex/number_of_full_strategies/main.tex}
strategies.

The results of this analysis are shown in
Figure~\ref{fig:SSError_and_probabilities_in_full}. The top ranking strategies
by number of wins seem to be extortionate (but not against all strategies) and
it can be seen that a small sub group of strategies achieve mutual defection.
All the top ranking strategies according to score achieve mutual cooperation and
do not extort each other, however they
\textbf{do} exhibit extortionate behaviour towards a number of the lower ranking
strategies.

\begin{figure}[!htbp]
    \centering
    \includegraphics[width=.8\textwidth]{./assets/img/SSError_and_probabilities_in_full/main.pdf}
    \caption{\(\text{SSError}\) for the strategies for the full tournament. Only
    strategy interactions for which \(p_4=0\) and \(\chi>1\) are displayed.}
    \label{fig:SSError_and_probabilities_in_full}
\end{figure}

\section{Conclusion}\label{sec:conclusion}

This work defines an approach to measure whether or not a player is playing a
strategy that corresponds to an extortionate strategy as defined
in~\cite{Press2012}: a mathematical model for suspicion. Indeed, all
extortionate strategies have been
 classified as lying on a triangular plane.
This rigorous classification fails to be robust to small measurement error, thus
a statistical approach is proposed.
This is done through a linear algebraic approach for approximating the solution
of a linear system. Using this, a large number of pairwise interactions is
simulated and in fact very few strategies are found to act extortionately.

The work of~\cite{Press2012}, whilst showing that a clever approach to taking
advantage of another memory one strategy exists: this is incomplete. Whilst the
elegance of this result is very attractive, just as the simplicity of the
victory of Tit For Tat in Axelrod's original tournaments was, it is incomplete.
Extortionate strategies achieve a high number of wins but they do not
achieve a high score which corresponds to the fitness landscape in an
evolutionary sense. From the large number of interactions a payoff matrix \(S\)
can be measured where \(S_{ij}\) denotes the score (using standard values of
\((R, S, T, P) = (3, 0, 5, 1)\)) of the \(i\)th strategy
against the \(j\)th strategy. Using this, the replicator equation
describes the evolution of the system based on a population density fitness
function:

\begin{equation}\label{eqn:replicator_dynamics}
    \frac{dx}{dt} = x(S-x^TS x)
\end{equation}

Equation (\ref{eqn:replicator_dynamics}) is solved numerically through an
integration technique described in~\cite{Petzold1983} and
Figure~\ref{fig:replicator_dynamics} shows the evolution of the distribution of
the system: the various strategies are ranked by scores. It is clear to see that
only the high ranking strategies survive the evolutionary process (in fact,
only \input{./assets/img/replicator_dynamics/main.tex}
have a final distribution greater than \(10 ^ {-2}\)). This confirms the
findings of~\cite{Moran1707} in which sophisticated strategies resist
evolutionary invasion of shorter memory strategies. Recalling
Figure~\ref{fig:SSError_and_probabilities_in_full} this demonstrates that:

\begin{itemize}
    \item Cooperation emerges through the evolutionary process: the high scoring
        strategies do not exhibit extortionate behaviour towards each other.
    \item Extortionate strategies do not survive the evolutionary process.
\end{itemize}

\begin{figure}[!htbp]
    \centering
    \includegraphics[width=.8\textwidth]{./assets/img/replicator_dynamics/main.pdf}
    \caption{Numerical simulation of the replicator equation
    (\ref{eqn:replicator_dynamics}): strategies are ordered by score, only the strategies with a high score survive the evolutionary process.}
    \label{fig:replicator_dynamics}
\end{figure}

This work can be used to classify plays of the IPD\@: data can be collected from
actual interactions (in lab or in the field). Furthermore, this allows for a
classification method similar to the notion of fingerprinting presented
in~\cite{Ashlock2008}. Trained strategies can potentially be classified as
extortionate or not or it could be possible to even constrain the reinforcement
learning approaches that are becoming prevalent in the literature.
Alternatively, this mathematical approach for recognising extortion could be
used in sophisticated strategies to defend against invasion. Arguably, some of
the strategies considered here exhibit this behaviour, indeed as described
in~\cite{Harper2017}, the top ranking strategies in the full tournament are
obtained using evolutionary reinforcement learning techniques, thus, suspicion
of extortionate behaviour could in fact be an evolutionary trait.

\section*{Acknowledgements}

The following open source software libraries were used in this research:

\begin{itemize}
    \item The Axelrod ~\cite{Knight2016, Knight2018} library (IPD strategies and
        tournaments).
    \item The sympy library~\cite{Meurer2017} (verification of all symbolic
        calculations).
    \item The matplotlib~\cite{Droettboom2018} library (visualisation).
    \item The pandas~\cite{Structures2010}, dask~\cite{Dask2016} and
        NumPy~\cite{Oliphant2015} libraries (data manipulation).
    \item The SciPy~\cite{Jones2001} library (numerical integration of the
        replicator equation).
\end{itemize}

This work was performed using the computational facilities of the Advanced
Research Computing @ Cardiff (ARCCA) Division, Cardiff University.

\printbibliography

\newpage
\section*{Supplementary materials}

\includepdf{assets/pdf/proof_of_form_of_extortionate_strategies/main.pdf}

\newpage

Using the pair wise interactions the transition rates \(p,
q\) can be measured and the steady state probabilities inferred and compared to
the actual probabilities of each state.
This is done numerically by computing the singular eigenvector of the
matrix \(A\) \cite{Stewart2009}:

\[
    A =
    \begin{bmatrix}
        p_1 q_1 & p_1 (1 - q_1) & (1 - p_1) q_1 & (1 -p_1) (1 - q_1) \\
        p_2 q_2 & p_2 (1 - q_2) & (1 - p_2) q_2 & (1 -p_2) (1 - q_2) \\
        p_3 q_3 & p_3 (1 - q_3) & (1 - p_3) q_3 & (1 -p_3) (1 - q_3) \\
        p_4 q_4 & p_4 (1 - q_4) & (1 - p_4) q_4 & (1 -p_4) (1 - q_4) \\
    \end{bmatrix}
\]

Figure~\ref{fig:computed_probabilities_vs_theoretic_probabilities} shows a
regression line fitted to every pairwise interaction with a reported
\(\text{SSError}\) value (pairwise interactions with missing states were
omitted). This serves to validate the approach: a part from some edge cases the
relationship is consistent.

\begin{figure}[!htbp]
    \centering
    \includegraphics[width=.8\textwidth]{./assets/img/computed_probabilities_vs_theoretic_probabilities/main.pdf}
    \caption{The
        relationship between the steady state probabilities inferred from the
        measured transitions and the actual steady state probabilities. A linear
        regression line is included validating the approach.}
    \label{fig:computed_probabilities_vs_theoretic_probabilities}
\end{figure}


\end{document}
 turns and every match has been
repeated \documentclass[a4paper]{article}

\usepackage{amsmath}
\usepackage{amssymb}
\usepackage[margin=1.5cm,
            includefoot,
            footskip=30pt]{geometry}
\usepackage{layout}
\usepackage{graphicx}
\usepackage{subcaption}

\usepackage{biblatex}
\usepackage{pdfpages}

\bibliography{main.bib}

\title{Suspicion: Recognising and evaluating the effectiveness
       of extortion in the Iterated Prisoner's Dilemma}
\author{Vincent A. Knight \and Nikoleta E. Glynatsi}
\date{\today}



\begin{document}

\maketitle

\begin{abstract}
    The Iterated Prisoner's Dilemma is a model for rational and evolutionary
    interactive behaviour. It has applications both in the study of human social
    behaviour as well as in biology.
    It is used to understand when and how a rational individual might
    accept an immediate cost to their own utility for the direct benefit of
    another.

    Much attention has been given to a class of strategies called
    Zero Determinant strategies. It has been theoretically shown that these
    strategies can ``extort'' any player.

    In this work, an approach to identify if observed strategies are playing in
    an extortionate way is described. Furthermore, experimental analysis of
    a large tournament with \input{assets/tex/number_of_full_strategies/main.tex}
    strategies is considered. In this setting
    the most highly performing strategies do not play in an extortionate way
    against each other but do against lower performing strategies.
    This suggests that whilst the theory of Zero Determinant strategies
    indicates that memory is not of fundamental importance to the evolution of
    cooperative behaviour, this is incomplete.
\end{abstract}

\section{Introduction}\label{sec:introduction}

Agent based game theoretic models have become a stalwart of the underpinning
mathematics of interactive behaviours. One of the major pieces of work
in this area is the pair of original computer tournaments run by Robert
Axelrod~\cite{Axelrod1980, Axelrod1980a}. These tournaments pitted submitted
computer strategies against each other in plays of the Iterated Prisoner's
Dilemma. A common game where agents can choose to pay a slight cost to their
immediate utility in the hope of building a reputation. This has been used in
economic and evolutionary game theory to understand the evolution of cooperative
behaviour.

Recently, a class of strategies was described in~\cite{Press2012} that can
provably extort any given opponent. In~\cite{Hilbe2013, Moran1707} some
questions have already been asked about the true effectiveness of these
strategies in an evolutionary setting. Here another question is asked: is it
possible to recognise this extortionate behaviour? A mathematical procedure for
suspicion is presented: in the same way that the continued actions of an
extortionate individual might raise suspicion.

This work makes use of the Axelrod Python library~\cite{Knight2018, Knight2016}
with a large number of Prisoner Dilemma strategies available to give an
extensive numerical example of the ideas presented.  The approach is presented
in Section~\ref{sec:delta-zd-strategies}.  All of the code and data discussed
in Section~\ref{sec:numerical-experiments} is open sourced, archived and
written according to best scientific principles~\cite{Wilson2014}. The data
archive can be found at~\cite{vincent_knight_2018_1297075}.

\section{Recognising Extortion}\label{sec:delta-zd-strategies}

In~\cite{Press2012}, given a match between 2 memory-one strategies, the concept
of Zero Determinant (ZD) strategies is introduced. The main result of that paper
shows that given two memory one players \(p, q\in\mathbb{R}^4\) a linear
relationship between the players' scores could be forced by one of the players.

Using the notation of~\cite{Press2012}, assuming the utilities for player \(p\)
are given by \(S_x=(R, S, T, P)\) and for player \(q\) by \(S_y=(R, T, S, P)\)
and that the stationary scores of each player is given by \(S_X\) and \(S_Y\)
respectively. The main result of~\cite{Press2012} is that if

\begin{equation}\label{eqn:linear_relationship_for_p}
    \tilde p=\alpha S_x + \beta S_y + \gamma
\end{equation}

or

\begin{equation}\label{eqn:linear_relationship_for_q}
    \tilde q=\alpha S_x + \beta S_y + \gamma
\end{equation}

where \(\tilde p = (1 - p_1, 1 - p_2, p_3, p_4)\) and
\(\tilde q = (1 - q_1, 1 - q_2, q_3, q_4)\) then:

\begin{equation}
    \alpha S_X + \beta S_Y + \gamma = 0
\end{equation}

In~\cite{Press2012} a particular type of ZD strategy is defined: extortionate
strategies. If:

\begin{equation}\label{eqn:constraint_for_extortion}
    \gamma = - P(\alpha + \beta)
\end{equation}

then the player can ensure they get a score \(\chi\) times
larger than the opponent. This extortion coefficient is given by:

\begin{equation}\label{eqn:definition_of_chi}
    \chi=\frac{-\beta}{\alpha}
\end{equation}

Thus, if (\ref{eqn:constraint_for_extortion}) holds and \(\chi >1\) a player is
said to extort their opponent.
Here, the reverse problem is considered: given a
\(p\in\mathbb{R}^4\) how does one identify \(\alpha, \beta\) if they
exist and is the strategy in fact acting in an extortionate way?

These conditions correspond to:

\begin{align}
    \tilde p_1 & = \alpha R + \beta R - P (\alpha + \beta)
            \label{eqn:condition_for_tilde_p1}\\
    \tilde p_2 & = \alpha S + \beta T - P (\alpha + \beta)
            \label{eqn:condition_for_tilde_p2}\\
    \tilde p_3 & = \alpha T + \beta S - P (\alpha + \beta)
            \label{eqn:condition_for_tilde_p3}\\
    \tilde p_4 & = \alpha P + \beta P - P (\alpha + \beta)
            \label{eqn:condition_for_tilde_p4}
\end{align}

Equation (\ref{eqn:condition_for_tilde_p4}) ensures that \(p_4=\tilde p_4=0\).
Equations (\ref{eqn:condition_for_tilde_p1}-\ref{eqn:condition_for_tilde_p3})
can be used to eliminate \(\alpha, \beta\), giving:

\begin{equation}\label{eqn:planar_definition_of_extortion}
    \tilde p_1 = \frac{(R - P)(\tilde p_2 + \tilde p_3)}{S + T - 2P}
\end{equation}

with:

\begin{equation}\label{eqn:definition_of_chi}
    \chi = \frac{\tilde p_2 (P - T) + \tilde p_3 (S - P)}
                {\tilde p_2 (P - S) + \tilde p_3 (T - P)}
\end{equation}

Given a strategy \(p\in\mathbb{R}^{4\times 1}\) equations
(\ref{eqn:condition_for_tilde_p4}), (\ref{eqn:planar_definition_of_extortion}-\ref{eqn:definition_of_chi}) can be used to check if
a strategy is extortionate. The conditions correspond to:

\begin{align}
    p_1 & = \frac{(R-P)(p_2 + p_3) - R + T + S - P}{S + T - 2P}
     \label{eqn:condition_for_p1}\\
    p_4 & = 0 \label{eqn:condition_for_p4}\\
    1 & > p_2 + p_3\label{eqn:condition_for_chi}
\end{align}

The algebraic steps necessary to prove these results are available in the
supporting materials.

All extortionate strategies reside on a triangular (\ref{eqn:condition_for_chi})
plane (\ref{eqn:condition_for_p1}) in 3 dimensions (\ref{eqn:condition_for_p4}).
Using this formulation it can be seen that a necessary (but not sufficient)
condition for an extortionate strategy is that it cooperates on average less
than 50\% of the time when in a state of disagreement with the opponent.

As an example, consider the known extortionate strategy \(p=(8 / 9, 1 / 2, 1 /
3, 0)\) from~\cite{Stewart2012} which is referred to as \texttt{Extort-2}. In
this case, for the standard values of \((R, T, S, P)\) constraint
(\ref{eqn:condition_for_p1}) corresponds to:

\begin{equation}
    p_1 = \frac{2(p_2 + p_3) + 1}{3}
\end{equation}

It is clear that in this case all constraints hold.

This approach could in fact be used to confirm that a given strategy is acting
in an extortionate manner even if it is not a memory one strategy. However, in
practice, if a closed form for \(p\) is not known, then due to measurement
and/or numerical error this would not work.

This problem can be written in the following linear algebraic form where
\(x=(\alpha, \beta)\)
and \(p^*=(\tilde p_1 - 1, tilde_2 - 1, p_3)\):

\begin{equation}\label{eqn:linear_algebraic_equation_for_p}
    Cx= p^*
\end{equation}

\(C\) corresponds to equations
(\ref{eqn:condition_for_tilde_p1}-\ref{eqn:condition_for_tilde_p3}) and is
given by:

\begin{equation}\label{eqn:definition_of_C}
    C =
    \begin{bmatrix}
        R - P & R- P \\
        S - P & T- P \\
        T - P & S- P \\
    \end{bmatrix}
\end{equation}

Note that in general, equation (\ref{eqn:linear_algebraic_equation_for_p}) will
not necessarily have a solution. From the Rouch\'{e}-Capelli theorem if there is
a solution it is unique as \(\text{rank}(C)=2\) which is the dimension of the
variable \(x\). The best fitting \(x\) is found by minimizing:

\begin{equation}\label{eqn:r_squared}
    \text{SSError} = \|C x- p^*\|_2^2 = \sum_{i=1}^{3}\left((C\bar x)_i-p_i^*\right)^2
\end{equation}

Note that \(\text{SSError}\), which is the square of the Frobenius
norm~\cite{Golub2013}, becomes a measure of how close a strategy is to being an
extortionate strategy. Suspicion
of extortion then corresponds to a threshold on \(\text{SSError}\).

By observing interactions (human or otherwise), their memory one representation
can be inferred and this approach can be used to recognise extortionate
behaviour. The notion of comparing theoretic and actual plays of the IPD is not
novel, see for example~\cite{Rand2013}. Immediately it is noted that if the
environment is noisy~\cite{Wu1995} then no strategy can be considered to be
extortionate as \(p_4>0\).

In the next section, this idea will be illustrated by observing the interactions
that take place in a computer based tournament of the IPD\@.

\section{Numerical experiments}\label{sec:numerical-experiments}

In~\cite{Stewart2012} results from a tournament with
\input{./assets/tex/number_of_stewart_plotkin_strategies/main.tex} strategies,
was presented with specific consideration given to ZD strategies. This
tournament is reproduced here using the Axelrod-Python
project~\cite{Knight2016}. To obtain a good measure of the corresponding
transition rates for each strategy all matches have been run for
\input{assets/tex/number_of_turns/main.tex} turns and every match has been
repeated \input{assets/tex/number_of_repetitions/main.tex} times. All of this
interaction data is available at~\cite{vincent_knight_2018_1297075}. A good
match between the inferred Markov chain and the state distribution of the actual
interactions has been verified. Data for this is presented in the supplementary
materials.

Figure~\ref{fig:SSError_overall_in_stewart_plotkin} shows the \(\text{SSError}\)
values for all the strategies in the tournament, as reported
in~\cite{Stewart2012} the extortionate strategy (which has an expected
\(\text{SSError}\) approximately 0) gains a large number of wins.

\begin{figure}[!htbp]
    \centering
    \includegraphics[width=.8\textwidth]{./assets/img/SSError_overall_in_stewart_plotkin/main.pdf}
    \caption{\(\text{SSError}\) and state probabilities for the strategies
        of~\cite{Stewart2012}, ordered both by number of wins and overall score.
        Note that \(P(DC)\) is not shown as it corresponds to the transpose of
        \(P(CD)\). Cooperator and Defector are omitted as they do not visit all
        the states.}
    \label{fig:SSError_overall_in_stewart_plotkin}
\end{figure}

Here, the work of~\cite{Stewart2012} is extended by investigating a tournament
with \input{assets/tex/number_of_full_strategies/main.tex}
strategies.

The results of this analysis are shown in
Figure~\ref{fig:SSError_and_probabilities_in_full}. The top ranking strategies
by number of wins seem to be extortionate (but not against all strategies) and
it can be seen that a small sub group of strategies achieve mutual defection.
All the top ranking strategies according to score achieve mutual cooperation and
do not extort each other, however they
\textbf{do} exhibit extortionate behaviour towards a number of the lower ranking
strategies.

\begin{figure}[!htbp]
    \centering
    \includegraphics[width=.8\textwidth]{./assets/img/SSError_and_probabilities_in_full/main.pdf}
    \caption{\(\text{SSError}\) for the strategies for the full tournament. Only
    strategy interactions for which \(p_4=0\) and \(\chi>1\) are displayed.}
    \label{fig:SSError_and_probabilities_in_full}
\end{figure}

\section{Conclusion}\label{sec:conclusion}

This work defines an approach to measure whether or not a player is playing a
strategy that corresponds to an extortionate strategy as defined
in~\cite{Press2012}: a mathematical model for suspicion. Indeed, all
extortionate strategies have been
 classified as lying on a triangular plane.
This rigorous classification fails to be robust to small measurement error, thus
a statistical approach is proposed.
This is done through a linear algebraic approach for approximating the solution
of a linear system. Using this, a large number of pairwise interactions is
simulated and in fact very few strategies are found to act extortionately.

The work of~\cite{Press2012}, whilst showing that a clever approach to taking
advantage of another memory one strategy exists: this is incomplete. Whilst the
elegance of this result is very attractive, just as the simplicity of the
victory of Tit For Tat in Axelrod's original tournaments was, it is incomplete.
Extortionate strategies achieve a high number of wins but they do not
achieve a high score which corresponds to the fitness landscape in an
evolutionary sense. From the large number of interactions a payoff matrix \(S\)
can be measured where \(S_{ij}\) denotes the score (using standard values of
\((R, S, T, P) = (3, 0, 5, 1)\)) of the \(i\)th strategy
against the \(j\)th strategy. Using this, the replicator equation
describes the evolution of the system based on a population density fitness
function:

\begin{equation}\label{eqn:replicator_dynamics}
    \frac{dx}{dt} = x(S-x^TS x)
\end{equation}

Equation (\ref{eqn:replicator_dynamics}) is solved numerically through an
integration technique described in~\cite{Petzold1983} and
Figure~\ref{fig:replicator_dynamics} shows the evolution of the distribution of
the system: the various strategies are ranked by scores. It is clear to see that
only the high ranking strategies survive the evolutionary process (in fact,
only \input{./assets/img/replicator_dynamics/main.tex}
have a final distribution greater than \(10 ^ {-2}\)). This confirms the
findings of~\cite{Moran1707} in which sophisticated strategies resist
evolutionary invasion of shorter memory strategies. Recalling
Figure~\ref{fig:SSError_and_probabilities_in_full} this demonstrates that:

\begin{itemize}
    \item Cooperation emerges through the evolutionary process: the high scoring
        strategies do not exhibit extortionate behaviour towards each other.
    \item Extortionate strategies do not survive the evolutionary process.
\end{itemize}

\begin{figure}[!htbp]
    \centering
    \includegraphics[width=.8\textwidth]{./assets/img/replicator_dynamics/main.pdf}
    \caption{Numerical simulation of the replicator equation
    (\ref{eqn:replicator_dynamics}): strategies are ordered by score, only the strategies with a high score survive the evolutionary process.}
    \label{fig:replicator_dynamics}
\end{figure}

This work can be used to classify plays of the IPD\@: data can be collected from
actual interactions (in lab or in the field). Furthermore, this allows for a
classification method similar to the notion of fingerprinting presented
in~\cite{Ashlock2008}. Trained strategies can potentially be classified as
extortionate or not or it could be possible to even constrain the reinforcement
learning approaches that are becoming prevalent in the literature.
Alternatively, this mathematical approach for recognising extortion could be
used in sophisticated strategies to defend against invasion. Arguably, some of
the strategies considered here exhibit this behaviour, indeed as described
in~\cite{Harper2017}, the top ranking strategies in the full tournament are
obtained using evolutionary reinforcement learning techniques, thus, suspicion
of extortionate behaviour could in fact be an evolutionary trait.

\section*{Acknowledgements}

The following open source software libraries were used in this research:

\begin{itemize}
    \item The Axelrod ~\cite{Knight2016, Knight2018} library (IPD strategies and
        tournaments).
    \item The sympy library~\cite{Meurer2017} (verification of all symbolic
        calculations).
    \item The matplotlib~\cite{Droettboom2018} library (visualisation).
    \item The pandas~\cite{Structures2010}, dask~\cite{Dask2016} and
        NumPy~\cite{Oliphant2015} libraries (data manipulation).
    \item The SciPy~\cite{Jones2001} library (numerical integration of the
        replicator equation).
\end{itemize}

This work was performed using the computational facilities of the Advanced
Research Computing @ Cardiff (ARCCA) Division, Cardiff University.

\printbibliography

\newpage
\section*{Supplementary materials}

\includepdf{assets/pdf/proof_of_form_of_extortionate_strategies/main.pdf}

\newpage

Using the pair wise interactions the transition rates \(p,
q\) can be measured and the steady state probabilities inferred and compared to
the actual probabilities of each state.
This is done numerically by computing the singular eigenvector of the
matrix \(A\) \cite{Stewart2009}:

\[
    A =
    \begin{bmatrix}
        p_1 q_1 & p_1 (1 - q_1) & (1 - p_1) q_1 & (1 -p_1) (1 - q_1) \\
        p_2 q_2 & p_2 (1 - q_2) & (1 - p_2) q_2 & (1 -p_2) (1 - q_2) \\
        p_3 q_3 & p_3 (1 - q_3) & (1 - p_3) q_3 & (1 -p_3) (1 - q_3) \\
        p_4 q_4 & p_4 (1 - q_4) & (1 - p_4) q_4 & (1 -p_4) (1 - q_4) \\
    \end{bmatrix}
\]

Figure~\ref{fig:computed_probabilities_vs_theoretic_probabilities} shows a
regression line fitted to every pairwise interaction with a reported
\(\text{SSError}\) value (pairwise interactions with missing states were
omitted). This serves to validate the approach: a part from some edge cases the
relationship is consistent.

\begin{figure}[!htbp]
    \centering
    \includegraphics[width=.8\textwidth]{./assets/img/computed_probabilities_vs_theoretic_probabilities/main.pdf}
    \caption{The
        relationship between the steady state probabilities inferred from the
        measured transitions and the actual steady state probabilities. A linear
        regression line is included validating the approach.}
    \label{fig:computed_probabilities_vs_theoretic_probabilities}
\end{figure}


\end{document}
 times. All of this
interaction data is available at~\cite{vincent_knight_2018_1297075}. A good
match between the inferred Markov chain and the state distribution of the actual
interactions has been verified. Data for this is presented in the supplementary
materials.

Figure~\ref{fig:SSError_overall_in_stewart_plotkin} shows the \(\text{SSError}\)
values for all the strategies in the tournament, as reported
in~\cite{Stewart2012} the extortionate strategy (which has an expected
\(\text{SSError}\) approximately 0) gains a large number of wins.

\begin{figure}[!htbp]
    \centering
    \includegraphics[width=.8\textwidth]{./assets/img/SSError_overall_in_stewart_plotkin/main.pdf}
    \caption{\(\text{SSError}\) and state probabilities for the strategies
        of~\cite{Stewart2012}, ordered both by number of wins and overall score.
        Note that \(P(DC)\) is not shown as it corresponds to the transpose of
        \(P(CD)\). Cooperator and Defector are omitted as they do not visit all
        the states.}
    \label{fig:SSError_overall_in_stewart_plotkin}
\end{figure}

Here, the work of~\cite{Stewart2012} is extended by investigating a tournament
with \documentclass[a4paper]{article}

\usepackage{amsmath}
\usepackage{amssymb}
\usepackage[margin=1.5cm,
            includefoot,
            footskip=30pt]{geometry}
\usepackage{layout}
\usepackage{graphicx}
\usepackage{subcaption}

\usepackage{biblatex}
\usepackage{pdfpages}

\bibliography{main.bib}

\title{Suspicion: Recognising and evaluating the effectiveness
       of extortion in the Iterated Prisoner's Dilemma}
\author{Vincent A. Knight \and Nikoleta E. Glynatsi}
\date{\today}



\begin{document}

\maketitle

\begin{abstract}
    The Iterated Prisoner's Dilemma is a model for rational and evolutionary
    interactive behaviour. It has applications both in the study of human social
    behaviour as well as in biology.
    It is used to understand when and how a rational individual might
    accept an immediate cost to their own utility for the direct benefit of
    another.

    Much attention has been given to a class of strategies called
    Zero Determinant strategies. It has been theoretically shown that these
    strategies can ``extort'' any player.

    In this work, an approach to identify if observed strategies are playing in
    an extortionate way is described. Furthermore, experimental analysis of
    a large tournament with \input{assets/tex/number_of_full_strategies/main.tex}
    strategies is considered. In this setting
    the most highly performing strategies do not play in an extortionate way
    against each other but do against lower performing strategies.
    This suggests that whilst the theory of Zero Determinant strategies
    indicates that memory is not of fundamental importance to the evolution of
    cooperative behaviour, this is incomplete.
\end{abstract}

\section{Introduction}\label{sec:introduction}

Agent based game theoretic models have become a stalwart of the underpinning
mathematics of interactive behaviours. One of the major pieces of work
in this area is the pair of original computer tournaments run by Robert
Axelrod~\cite{Axelrod1980, Axelrod1980a}. These tournaments pitted submitted
computer strategies against each other in plays of the Iterated Prisoner's
Dilemma. A common game where agents can choose to pay a slight cost to their
immediate utility in the hope of building a reputation. This has been used in
economic and evolutionary game theory to understand the evolution of cooperative
behaviour.

Recently, a class of strategies was described in~\cite{Press2012} that can
provably extort any given opponent. In~\cite{Hilbe2013, Moran1707} some
questions have already been asked about the true effectiveness of these
strategies in an evolutionary setting. Here another question is asked: is it
possible to recognise this extortionate behaviour? A mathematical procedure for
suspicion is presented: in the same way that the continued actions of an
extortionate individual might raise suspicion.

This work makes use of the Axelrod Python library~\cite{Knight2018, Knight2016}
with a large number of Prisoner Dilemma strategies available to give an
extensive numerical example of the ideas presented.  The approach is presented
in Section~\ref{sec:delta-zd-strategies}.  All of the code and data discussed
in Section~\ref{sec:numerical-experiments} is open sourced, archived and
written according to best scientific principles~\cite{Wilson2014}. The data
archive can be found at~\cite{vincent_knight_2018_1297075}.

\section{Recognising Extortion}\label{sec:delta-zd-strategies}

In~\cite{Press2012}, given a match between 2 memory-one strategies, the concept
of Zero Determinant (ZD) strategies is introduced. The main result of that paper
shows that given two memory one players \(p, q\in\mathbb{R}^4\) a linear
relationship between the players' scores could be forced by one of the players.

Using the notation of~\cite{Press2012}, assuming the utilities for player \(p\)
are given by \(S_x=(R, S, T, P)\) and for player \(q\) by \(S_y=(R, T, S, P)\)
and that the stationary scores of each player is given by \(S_X\) and \(S_Y\)
respectively. The main result of~\cite{Press2012} is that if

\begin{equation}\label{eqn:linear_relationship_for_p}
    \tilde p=\alpha S_x + \beta S_y + \gamma
\end{equation}

or

\begin{equation}\label{eqn:linear_relationship_for_q}
    \tilde q=\alpha S_x + \beta S_y + \gamma
\end{equation}

where \(\tilde p = (1 - p_1, 1 - p_2, p_3, p_4)\) and
\(\tilde q = (1 - q_1, 1 - q_2, q_3, q_4)\) then:

\begin{equation}
    \alpha S_X + \beta S_Y + \gamma = 0
\end{equation}

In~\cite{Press2012} a particular type of ZD strategy is defined: extortionate
strategies. If:

\begin{equation}\label{eqn:constraint_for_extortion}
    \gamma = - P(\alpha + \beta)
\end{equation}

then the player can ensure they get a score \(\chi\) times
larger than the opponent. This extortion coefficient is given by:

\begin{equation}\label{eqn:definition_of_chi}
    \chi=\frac{-\beta}{\alpha}
\end{equation}

Thus, if (\ref{eqn:constraint_for_extortion}) holds and \(\chi >1\) a player is
said to extort their opponent.
Here, the reverse problem is considered: given a
\(p\in\mathbb{R}^4\) how does one identify \(\alpha, \beta\) if they
exist and is the strategy in fact acting in an extortionate way?

These conditions correspond to:

\begin{align}
    \tilde p_1 & = \alpha R + \beta R - P (\alpha + \beta)
            \label{eqn:condition_for_tilde_p1}\\
    \tilde p_2 & = \alpha S + \beta T - P (\alpha + \beta)
            \label{eqn:condition_for_tilde_p2}\\
    \tilde p_3 & = \alpha T + \beta S - P (\alpha + \beta)
            \label{eqn:condition_for_tilde_p3}\\
    \tilde p_4 & = \alpha P + \beta P - P (\alpha + \beta)
            \label{eqn:condition_for_tilde_p4}
\end{align}

Equation (\ref{eqn:condition_for_tilde_p4}) ensures that \(p_4=\tilde p_4=0\).
Equations (\ref{eqn:condition_for_tilde_p1}-\ref{eqn:condition_for_tilde_p3})
can be used to eliminate \(\alpha, \beta\), giving:

\begin{equation}\label{eqn:planar_definition_of_extortion}
    \tilde p_1 = \frac{(R - P)(\tilde p_2 + \tilde p_3)}{S + T - 2P}
\end{equation}

with:

\begin{equation}\label{eqn:definition_of_chi}
    \chi = \frac{\tilde p_2 (P - T) + \tilde p_3 (S - P)}
                {\tilde p_2 (P - S) + \tilde p_3 (T - P)}
\end{equation}

Given a strategy \(p\in\mathbb{R}^{4\times 1}\) equations
(\ref{eqn:condition_for_tilde_p4}), (\ref{eqn:planar_definition_of_extortion}-\ref{eqn:definition_of_chi}) can be used to check if
a strategy is extortionate. The conditions correspond to:

\begin{align}
    p_1 & = \frac{(R-P)(p_2 + p_3) - R + T + S - P}{S + T - 2P}
     \label{eqn:condition_for_p1}\\
    p_4 & = 0 \label{eqn:condition_for_p4}\\
    1 & > p_2 + p_3\label{eqn:condition_for_chi}
\end{align}

The algebraic steps necessary to prove these results are available in the
supporting materials.

All extortionate strategies reside on a triangular (\ref{eqn:condition_for_chi})
plane (\ref{eqn:condition_for_p1}) in 3 dimensions (\ref{eqn:condition_for_p4}).
Using this formulation it can be seen that a necessary (but not sufficient)
condition for an extortionate strategy is that it cooperates on average less
than 50\% of the time when in a state of disagreement with the opponent.

As an example, consider the known extortionate strategy \(p=(8 / 9, 1 / 2, 1 /
3, 0)\) from~\cite{Stewart2012} which is referred to as \texttt{Extort-2}. In
this case, for the standard values of \((R, T, S, P)\) constraint
(\ref{eqn:condition_for_p1}) corresponds to:

\begin{equation}
    p_1 = \frac{2(p_2 + p_3) + 1}{3}
\end{equation}

It is clear that in this case all constraints hold.

This approach could in fact be used to confirm that a given strategy is acting
in an extortionate manner even if it is not a memory one strategy. However, in
practice, if a closed form for \(p\) is not known, then due to measurement
and/or numerical error this would not work.

This problem can be written in the following linear algebraic form where
\(x=(\alpha, \beta)\)
and \(p^*=(\tilde p_1 - 1, tilde_2 - 1, p_3)\):

\begin{equation}\label{eqn:linear_algebraic_equation_for_p}
    Cx= p^*
\end{equation}

\(C\) corresponds to equations
(\ref{eqn:condition_for_tilde_p1}-\ref{eqn:condition_for_tilde_p3}) and is
given by:

\begin{equation}\label{eqn:definition_of_C}
    C =
    \begin{bmatrix}
        R - P & R- P \\
        S - P & T- P \\
        T - P & S- P \\
    \end{bmatrix}
\end{equation}

Note that in general, equation (\ref{eqn:linear_algebraic_equation_for_p}) will
not necessarily have a solution. From the Rouch\'{e}-Capelli theorem if there is
a solution it is unique as \(\text{rank}(C)=2\) which is the dimension of the
variable \(x\). The best fitting \(x\) is found by minimizing:

\begin{equation}\label{eqn:r_squared}
    \text{SSError} = \|C x- p^*\|_2^2 = \sum_{i=1}^{3}\left((C\bar x)_i-p_i^*\right)^2
\end{equation}

Note that \(\text{SSError}\), which is the square of the Frobenius
norm~\cite{Golub2013}, becomes a measure of how close a strategy is to being an
extortionate strategy. Suspicion
of extortion then corresponds to a threshold on \(\text{SSError}\).

By observing interactions (human or otherwise), their memory one representation
can be inferred and this approach can be used to recognise extortionate
behaviour. The notion of comparing theoretic and actual plays of the IPD is not
novel, see for example~\cite{Rand2013}. Immediately it is noted that if the
environment is noisy~\cite{Wu1995} then no strategy can be considered to be
extortionate as \(p_4>0\).

In the next section, this idea will be illustrated by observing the interactions
that take place in a computer based tournament of the IPD\@.

\section{Numerical experiments}\label{sec:numerical-experiments}

In~\cite{Stewart2012} results from a tournament with
\input{./assets/tex/number_of_stewart_plotkin_strategies/main.tex} strategies,
was presented with specific consideration given to ZD strategies. This
tournament is reproduced here using the Axelrod-Python
project~\cite{Knight2016}. To obtain a good measure of the corresponding
transition rates for each strategy all matches have been run for
\input{assets/tex/number_of_turns/main.tex} turns and every match has been
repeated \input{assets/tex/number_of_repetitions/main.tex} times. All of this
interaction data is available at~\cite{vincent_knight_2018_1297075}. A good
match between the inferred Markov chain and the state distribution of the actual
interactions has been verified. Data for this is presented in the supplementary
materials.

Figure~\ref{fig:SSError_overall_in_stewart_plotkin} shows the \(\text{SSError}\)
values for all the strategies in the tournament, as reported
in~\cite{Stewart2012} the extortionate strategy (which has an expected
\(\text{SSError}\) approximately 0) gains a large number of wins.

\begin{figure}[!htbp]
    \centering
    \includegraphics[width=.8\textwidth]{./assets/img/SSError_overall_in_stewart_plotkin/main.pdf}
    \caption{\(\text{SSError}\) and state probabilities for the strategies
        of~\cite{Stewart2012}, ordered both by number of wins and overall score.
        Note that \(P(DC)\) is not shown as it corresponds to the transpose of
        \(P(CD)\). Cooperator and Defector are omitted as they do not visit all
        the states.}
    \label{fig:SSError_overall_in_stewart_plotkin}
\end{figure}

Here, the work of~\cite{Stewart2012} is extended by investigating a tournament
with \input{assets/tex/number_of_full_strategies/main.tex}
strategies.

The results of this analysis are shown in
Figure~\ref{fig:SSError_and_probabilities_in_full}. The top ranking strategies
by number of wins seem to be extortionate (but not against all strategies) and
it can be seen that a small sub group of strategies achieve mutual defection.
All the top ranking strategies according to score achieve mutual cooperation and
do not extort each other, however they
\textbf{do} exhibit extortionate behaviour towards a number of the lower ranking
strategies.

\begin{figure}[!htbp]
    \centering
    \includegraphics[width=.8\textwidth]{./assets/img/SSError_and_probabilities_in_full/main.pdf}
    \caption{\(\text{SSError}\) for the strategies for the full tournament. Only
    strategy interactions for which \(p_4=0\) and \(\chi>1\) are displayed.}
    \label{fig:SSError_and_probabilities_in_full}
\end{figure}

\section{Conclusion}\label{sec:conclusion}

This work defines an approach to measure whether or not a player is playing a
strategy that corresponds to an extortionate strategy as defined
in~\cite{Press2012}: a mathematical model for suspicion. Indeed, all
extortionate strategies have been
 classified as lying on a triangular plane.
This rigorous classification fails to be robust to small measurement error, thus
a statistical approach is proposed.
This is done through a linear algebraic approach for approximating the solution
of a linear system. Using this, a large number of pairwise interactions is
simulated and in fact very few strategies are found to act extortionately.

The work of~\cite{Press2012}, whilst showing that a clever approach to taking
advantage of another memory one strategy exists: this is incomplete. Whilst the
elegance of this result is very attractive, just as the simplicity of the
victory of Tit For Tat in Axelrod's original tournaments was, it is incomplete.
Extortionate strategies achieve a high number of wins but they do not
achieve a high score which corresponds to the fitness landscape in an
evolutionary sense. From the large number of interactions a payoff matrix \(S\)
can be measured where \(S_{ij}\) denotes the score (using standard values of
\((R, S, T, P) = (3, 0, 5, 1)\)) of the \(i\)th strategy
against the \(j\)th strategy. Using this, the replicator equation
describes the evolution of the system based on a population density fitness
function:

\begin{equation}\label{eqn:replicator_dynamics}
    \frac{dx}{dt} = x(S-x^TS x)
\end{equation}

Equation (\ref{eqn:replicator_dynamics}) is solved numerically through an
integration technique described in~\cite{Petzold1983} and
Figure~\ref{fig:replicator_dynamics} shows the evolution of the distribution of
the system: the various strategies are ranked by scores. It is clear to see that
only the high ranking strategies survive the evolutionary process (in fact,
only \input{./assets/img/replicator_dynamics/main.tex}
have a final distribution greater than \(10 ^ {-2}\)). This confirms the
findings of~\cite{Moran1707} in which sophisticated strategies resist
evolutionary invasion of shorter memory strategies. Recalling
Figure~\ref{fig:SSError_and_probabilities_in_full} this demonstrates that:

\begin{itemize}
    \item Cooperation emerges through the evolutionary process: the high scoring
        strategies do not exhibit extortionate behaviour towards each other.
    \item Extortionate strategies do not survive the evolutionary process.
\end{itemize}

\begin{figure}[!htbp]
    \centering
    \includegraphics[width=.8\textwidth]{./assets/img/replicator_dynamics/main.pdf}
    \caption{Numerical simulation of the replicator equation
    (\ref{eqn:replicator_dynamics}): strategies are ordered by score, only the strategies with a high score survive the evolutionary process.}
    \label{fig:replicator_dynamics}
\end{figure}

This work can be used to classify plays of the IPD\@: data can be collected from
actual interactions (in lab or in the field). Furthermore, this allows for a
classification method similar to the notion of fingerprinting presented
in~\cite{Ashlock2008}. Trained strategies can potentially be classified as
extortionate or not or it could be possible to even constrain the reinforcement
learning approaches that are becoming prevalent in the literature.
Alternatively, this mathematical approach for recognising extortion could be
used in sophisticated strategies to defend against invasion. Arguably, some of
the strategies considered here exhibit this behaviour, indeed as described
in~\cite{Harper2017}, the top ranking strategies in the full tournament are
obtained using evolutionary reinforcement learning techniques, thus, suspicion
of extortionate behaviour could in fact be an evolutionary trait.

\section*{Acknowledgements}

The following open source software libraries were used in this research:

\begin{itemize}
    \item The Axelrod ~\cite{Knight2016, Knight2018} library (IPD strategies and
        tournaments).
    \item The sympy library~\cite{Meurer2017} (verification of all symbolic
        calculations).
    \item The matplotlib~\cite{Droettboom2018} library (visualisation).
    \item The pandas~\cite{Structures2010}, dask~\cite{Dask2016} and
        NumPy~\cite{Oliphant2015} libraries (data manipulation).
    \item The SciPy~\cite{Jones2001} library (numerical integration of the
        replicator equation).
\end{itemize}

This work was performed using the computational facilities of the Advanced
Research Computing @ Cardiff (ARCCA) Division, Cardiff University.

\printbibliography

\newpage
\section*{Supplementary materials}

\includepdf{assets/pdf/proof_of_form_of_extortionate_strategies/main.pdf}

\newpage

Using the pair wise interactions the transition rates \(p,
q\) can be measured and the steady state probabilities inferred and compared to
the actual probabilities of each state.
This is done numerically by computing the singular eigenvector of the
matrix \(A\) \cite{Stewart2009}:

\[
    A =
    \begin{bmatrix}
        p_1 q_1 & p_1 (1 - q_1) & (1 - p_1) q_1 & (1 -p_1) (1 - q_1) \\
        p_2 q_2 & p_2 (1 - q_2) & (1 - p_2) q_2 & (1 -p_2) (1 - q_2) \\
        p_3 q_3 & p_3 (1 - q_3) & (1 - p_3) q_3 & (1 -p_3) (1 - q_3) \\
        p_4 q_4 & p_4 (1 - q_4) & (1 - p_4) q_4 & (1 -p_4) (1 - q_4) \\
    \end{bmatrix}
\]

Figure~\ref{fig:computed_probabilities_vs_theoretic_probabilities} shows a
regression line fitted to every pairwise interaction with a reported
\(\text{SSError}\) value (pairwise interactions with missing states were
omitted). This serves to validate the approach: a part from some edge cases the
relationship is consistent.

\begin{figure}[!htbp]
    \centering
    \includegraphics[width=.8\textwidth]{./assets/img/computed_probabilities_vs_theoretic_probabilities/main.pdf}
    \caption{The
        relationship between the steady state probabilities inferred from the
        measured transitions and the actual steady state probabilities. A linear
        regression line is included validating the approach.}
    \label{fig:computed_probabilities_vs_theoretic_probabilities}
\end{figure}


\end{document}

strategies.

The results of this analysis are shown in
Figure~\ref{fig:SSError_and_probabilities_in_full}. The top ranking strategies
by number of wins seem to be extortionate (but not against all strategies) and
it can be seen that a small sub group of strategies achieve mutual defection.
All the top ranking strategies according to score achieve mutual cooperation and
do not extort each other, however they
\textbf{do} exhibit extortionate behaviour towards a number of the lower ranking
strategies.

\begin{figure}[!htbp]
    \centering
    \includegraphics[width=.8\textwidth]{./assets/img/SSError_and_probabilities_in_full/main.pdf}
    \caption{\(\text{SSError}\) for the strategies for the full tournament. Only
    strategy interactions for which \(p_4=0\) and \(\chi>1\) are displayed.}
    \label{fig:SSError_and_probabilities_in_full}
\end{figure}

\section{Conclusion}\label{sec:conclusion}

This work defines an approach to measure whether or not a player is playing a
strategy that corresponds to an extortionate strategy as defined
in~\cite{Press2012}: a mathematical model for suspicion. Indeed, all
extortionate strategies have been
 classified as lying on a triangular plane.
This rigorous classification fails to be robust to small measurement error, thus
a statistical approach is proposed.
This is done through a linear algebraic approach for approximating the solution
of a linear system. Using this, a large number of pairwise interactions is
simulated and in fact very few strategies are found to act extortionately.

The work of~\cite{Press2012}, whilst showing that a clever approach to taking
advantage of another memory one strategy exists: this is incomplete. Whilst the
elegance of this result is very attractive, just as the simplicity of the
victory of Tit For Tat in Axelrod's original tournaments was, it is incomplete.
Extortionate strategies achieve a high number of wins but they do not
achieve a high score which corresponds to the fitness landscape in an
evolutionary sense. From the large number of interactions a payoff matrix \(S\)
can be measured where \(S_{ij}\) denotes the score (using standard values of
\((R, S, T, P) = (3, 0, 5, 1)\)) of the \(i\)th strategy
against the \(j\)th strategy. Using this, the replicator equation
describes the evolution of the system based on a population density fitness
function:

\begin{equation}\label{eqn:replicator_dynamics}
    \frac{dx}{dt} = x(S-x^TS x)
\end{equation}

Equation (\ref{eqn:replicator_dynamics}) is solved numerically through an
integration technique described in~\cite{Petzold1983} and
Figure~\ref{fig:replicator_dynamics} shows the evolution of the distribution of
the system: the various strategies are ranked by scores. It is clear to see that
only the high ranking strategies survive the evolutionary process (in fact,
only \documentclass[a4paper]{article}

\usepackage{amsmath}
\usepackage{amssymb}
\usepackage[margin=1.5cm,
            includefoot,
            footskip=30pt]{geometry}
\usepackage{layout}
\usepackage{graphicx}
\usepackage{subcaption}

\usepackage{biblatex}
\usepackage{pdfpages}

\bibliography{main.bib}

\title{Suspicion: Recognising and evaluating the effectiveness
       of extortion in the Iterated Prisoner's Dilemma}
\author{Vincent A. Knight \and Nikoleta E. Glynatsi}
\date{\today}



\begin{document}

\maketitle

\begin{abstract}
    The Iterated Prisoner's Dilemma is a model for rational and evolutionary
    interactive behaviour. It has applications both in the study of human social
    behaviour as well as in biology.
    It is used to understand when and how a rational individual might
    accept an immediate cost to their own utility for the direct benefit of
    another.

    Much attention has been given to a class of strategies called
    Zero Determinant strategies. It has been theoretically shown that these
    strategies can ``extort'' any player.

    In this work, an approach to identify if observed strategies are playing in
    an extortionate way is described. Furthermore, experimental analysis of
    a large tournament with \input{assets/tex/number_of_full_strategies/main.tex}
    strategies is considered. In this setting
    the most highly performing strategies do not play in an extortionate way
    against each other but do against lower performing strategies.
    This suggests that whilst the theory of Zero Determinant strategies
    indicates that memory is not of fundamental importance to the evolution of
    cooperative behaviour, this is incomplete.
\end{abstract}

\section{Introduction}\label{sec:introduction}

Agent based game theoretic models have become a stalwart of the underpinning
mathematics of interactive behaviours. One of the major pieces of work
in this area is the pair of original computer tournaments run by Robert
Axelrod~\cite{Axelrod1980, Axelrod1980a}. These tournaments pitted submitted
computer strategies against each other in plays of the Iterated Prisoner's
Dilemma. A common game where agents can choose to pay a slight cost to their
immediate utility in the hope of building a reputation. This has been used in
economic and evolutionary game theory to understand the evolution of cooperative
behaviour.

Recently, a class of strategies was described in~\cite{Press2012} that can
provably extort any given opponent. In~\cite{Hilbe2013, Moran1707} some
questions have already been asked about the true effectiveness of these
strategies in an evolutionary setting. Here another question is asked: is it
possible to recognise this extortionate behaviour? A mathematical procedure for
suspicion is presented: in the same way that the continued actions of an
extortionate individual might raise suspicion.

This work makes use of the Axelrod Python library~\cite{Knight2018, Knight2016}
with a large number of Prisoner Dilemma strategies available to give an
extensive numerical example of the ideas presented.  The approach is presented
in Section~\ref{sec:delta-zd-strategies}.  All of the code and data discussed
in Section~\ref{sec:numerical-experiments} is open sourced, archived and
written according to best scientific principles~\cite{Wilson2014}. The data
archive can be found at~\cite{vincent_knight_2018_1297075}.

\section{Recognising Extortion}\label{sec:delta-zd-strategies}

In~\cite{Press2012}, given a match between 2 memory-one strategies, the concept
of Zero Determinant (ZD) strategies is introduced. The main result of that paper
shows that given two memory one players \(p, q\in\mathbb{R}^4\) a linear
relationship between the players' scores could be forced by one of the players.

Using the notation of~\cite{Press2012}, assuming the utilities for player \(p\)
are given by \(S_x=(R, S, T, P)\) and for player \(q\) by \(S_y=(R, T, S, P)\)
and that the stationary scores of each player is given by \(S_X\) and \(S_Y\)
respectively. The main result of~\cite{Press2012} is that if

\begin{equation}\label{eqn:linear_relationship_for_p}
    \tilde p=\alpha S_x + \beta S_y + \gamma
\end{equation}

or

\begin{equation}\label{eqn:linear_relationship_for_q}
    \tilde q=\alpha S_x + \beta S_y + \gamma
\end{equation}

where \(\tilde p = (1 - p_1, 1 - p_2, p_3, p_4)\) and
\(\tilde q = (1 - q_1, 1 - q_2, q_3, q_4)\) then:

\begin{equation}
    \alpha S_X + \beta S_Y + \gamma = 0
\end{equation}

In~\cite{Press2012} a particular type of ZD strategy is defined: extortionate
strategies. If:

\begin{equation}\label{eqn:constraint_for_extortion}
    \gamma = - P(\alpha + \beta)
\end{equation}

then the player can ensure they get a score \(\chi\) times
larger than the opponent. This extortion coefficient is given by:

\begin{equation}\label{eqn:definition_of_chi}
    \chi=\frac{-\beta}{\alpha}
\end{equation}

Thus, if (\ref{eqn:constraint_for_extortion}) holds and \(\chi >1\) a player is
said to extort their opponent.
Here, the reverse problem is considered: given a
\(p\in\mathbb{R}^4\) how does one identify \(\alpha, \beta\) if they
exist and is the strategy in fact acting in an extortionate way?

These conditions correspond to:

\begin{align}
    \tilde p_1 & = \alpha R + \beta R - P (\alpha + \beta)
            \label{eqn:condition_for_tilde_p1}\\
    \tilde p_2 & = \alpha S + \beta T - P (\alpha + \beta)
            \label{eqn:condition_for_tilde_p2}\\
    \tilde p_3 & = \alpha T + \beta S - P (\alpha + \beta)
            \label{eqn:condition_for_tilde_p3}\\
    \tilde p_4 & = \alpha P + \beta P - P (\alpha + \beta)
            \label{eqn:condition_for_tilde_p4}
\end{align}

Equation (\ref{eqn:condition_for_tilde_p4}) ensures that \(p_4=\tilde p_4=0\).
Equations (\ref{eqn:condition_for_tilde_p1}-\ref{eqn:condition_for_tilde_p3})
can be used to eliminate \(\alpha, \beta\), giving:

\begin{equation}\label{eqn:planar_definition_of_extortion}
    \tilde p_1 = \frac{(R - P)(\tilde p_2 + \tilde p_3)}{S + T - 2P}
\end{equation}

with:

\begin{equation}\label{eqn:definition_of_chi}
    \chi = \frac{\tilde p_2 (P - T) + \tilde p_3 (S - P)}
                {\tilde p_2 (P - S) + \tilde p_3 (T - P)}
\end{equation}

Given a strategy \(p\in\mathbb{R}^{4\times 1}\) equations
(\ref{eqn:condition_for_tilde_p4}), (\ref{eqn:planar_definition_of_extortion}-\ref{eqn:definition_of_chi}) can be used to check if
a strategy is extortionate. The conditions correspond to:

\begin{align}
    p_1 & = \frac{(R-P)(p_2 + p_3) - R + T + S - P}{S + T - 2P}
     \label{eqn:condition_for_p1}\\
    p_4 & = 0 \label{eqn:condition_for_p4}\\
    1 & > p_2 + p_3\label{eqn:condition_for_chi}
\end{align}

The algebraic steps necessary to prove these results are available in the
supporting materials.

All extortionate strategies reside on a triangular (\ref{eqn:condition_for_chi})
plane (\ref{eqn:condition_for_p1}) in 3 dimensions (\ref{eqn:condition_for_p4}).
Using this formulation it can be seen that a necessary (but not sufficient)
condition for an extortionate strategy is that it cooperates on average less
than 50\% of the time when in a state of disagreement with the opponent.

As an example, consider the known extortionate strategy \(p=(8 / 9, 1 / 2, 1 /
3, 0)\) from~\cite{Stewart2012} which is referred to as \texttt{Extort-2}. In
this case, for the standard values of \((R, T, S, P)\) constraint
(\ref{eqn:condition_for_p1}) corresponds to:

\begin{equation}
    p_1 = \frac{2(p_2 + p_3) + 1}{3}
\end{equation}

It is clear that in this case all constraints hold.

This approach could in fact be used to confirm that a given strategy is acting
in an extortionate manner even if it is not a memory one strategy. However, in
practice, if a closed form for \(p\) is not known, then due to measurement
and/or numerical error this would not work.

This problem can be written in the following linear algebraic form where
\(x=(\alpha, \beta)\)
and \(p^*=(\tilde p_1 - 1, tilde_2 - 1, p_3)\):

\begin{equation}\label{eqn:linear_algebraic_equation_for_p}
    Cx= p^*
\end{equation}

\(C\) corresponds to equations
(\ref{eqn:condition_for_tilde_p1}-\ref{eqn:condition_for_tilde_p3}) and is
given by:

\begin{equation}\label{eqn:definition_of_C}
    C =
    \begin{bmatrix}
        R - P & R- P \\
        S - P & T- P \\
        T - P & S- P \\
    \end{bmatrix}
\end{equation}

Note that in general, equation (\ref{eqn:linear_algebraic_equation_for_p}) will
not necessarily have a solution. From the Rouch\'{e}-Capelli theorem if there is
a solution it is unique as \(\text{rank}(C)=2\) which is the dimension of the
variable \(x\). The best fitting \(x\) is found by minimizing:

\begin{equation}\label{eqn:r_squared}
    \text{SSError} = \|C x- p^*\|_2^2 = \sum_{i=1}^{3}\left((C\bar x)_i-p_i^*\right)^2
\end{equation}

Note that \(\text{SSError}\), which is the square of the Frobenius
norm~\cite{Golub2013}, becomes a measure of how close a strategy is to being an
extortionate strategy. Suspicion
of extortion then corresponds to a threshold on \(\text{SSError}\).

By observing interactions (human or otherwise), their memory one representation
can be inferred and this approach can be used to recognise extortionate
behaviour. The notion of comparing theoretic and actual plays of the IPD is not
novel, see for example~\cite{Rand2013}. Immediately it is noted that if the
environment is noisy~\cite{Wu1995} then no strategy can be considered to be
extortionate as \(p_4>0\).

In the next section, this idea will be illustrated by observing the interactions
that take place in a computer based tournament of the IPD\@.

\section{Numerical experiments}\label{sec:numerical-experiments}

In~\cite{Stewart2012} results from a tournament with
\input{./assets/tex/number_of_stewart_plotkin_strategies/main.tex} strategies,
was presented with specific consideration given to ZD strategies. This
tournament is reproduced here using the Axelrod-Python
project~\cite{Knight2016}. To obtain a good measure of the corresponding
transition rates for each strategy all matches have been run for
\input{assets/tex/number_of_turns/main.tex} turns and every match has been
repeated \input{assets/tex/number_of_repetitions/main.tex} times. All of this
interaction data is available at~\cite{vincent_knight_2018_1297075}. A good
match between the inferred Markov chain and the state distribution of the actual
interactions has been verified. Data for this is presented in the supplementary
materials.

Figure~\ref{fig:SSError_overall_in_stewart_plotkin} shows the \(\text{SSError}\)
values for all the strategies in the tournament, as reported
in~\cite{Stewart2012} the extortionate strategy (which has an expected
\(\text{SSError}\) approximately 0) gains a large number of wins.

\begin{figure}[!htbp]
    \centering
    \includegraphics[width=.8\textwidth]{./assets/img/SSError_overall_in_stewart_plotkin/main.pdf}
    \caption{\(\text{SSError}\) and state probabilities for the strategies
        of~\cite{Stewart2012}, ordered both by number of wins and overall score.
        Note that \(P(DC)\) is not shown as it corresponds to the transpose of
        \(P(CD)\). Cooperator and Defector are omitted as they do not visit all
        the states.}
    \label{fig:SSError_overall_in_stewart_plotkin}
\end{figure}

Here, the work of~\cite{Stewart2012} is extended by investigating a tournament
with \input{assets/tex/number_of_full_strategies/main.tex}
strategies.

The results of this analysis are shown in
Figure~\ref{fig:SSError_and_probabilities_in_full}. The top ranking strategies
by number of wins seem to be extortionate (but not against all strategies) and
it can be seen that a small sub group of strategies achieve mutual defection.
All the top ranking strategies according to score achieve mutual cooperation and
do not extort each other, however they
\textbf{do} exhibit extortionate behaviour towards a number of the lower ranking
strategies.

\begin{figure}[!htbp]
    \centering
    \includegraphics[width=.8\textwidth]{./assets/img/SSError_and_probabilities_in_full/main.pdf}
    \caption{\(\text{SSError}\) for the strategies for the full tournament. Only
    strategy interactions for which \(p_4=0\) and \(\chi>1\) are displayed.}
    \label{fig:SSError_and_probabilities_in_full}
\end{figure}

\section{Conclusion}\label{sec:conclusion}

This work defines an approach to measure whether or not a player is playing a
strategy that corresponds to an extortionate strategy as defined
in~\cite{Press2012}: a mathematical model for suspicion. Indeed, all
extortionate strategies have been
 classified as lying on a triangular plane.
This rigorous classification fails to be robust to small measurement error, thus
a statistical approach is proposed.
This is done through a linear algebraic approach for approximating the solution
of a linear system. Using this, a large number of pairwise interactions is
simulated and in fact very few strategies are found to act extortionately.

The work of~\cite{Press2012}, whilst showing that a clever approach to taking
advantage of another memory one strategy exists: this is incomplete. Whilst the
elegance of this result is very attractive, just as the simplicity of the
victory of Tit For Tat in Axelrod's original tournaments was, it is incomplete.
Extortionate strategies achieve a high number of wins but they do not
achieve a high score which corresponds to the fitness landscape in an
evolutionary sense. From the large number of interactions a payoff matrix \(S\)
can be measured where \(S_{ij}\) denotes the score (using standard values of
\((R, S, T, P) = (3, 0, 5, 1)\)) of the \(i\)th strategy
against the \(j\)th strategy. Using this, the replicator equation
describes the evolution of the system based on a population density fitness
function:

\begin{equation}\label{eqn:replicator_dynamics}
    \frac{dx}{dt} = x(S-x^TS x)
\end{equation}

Equation (\ref{eqn:replicator_dynamics}) is solved numerically through an
integration technique described in~\cite{Petzold1983} and
Figure~\ref{fig:replicator_dynamics} shows the evolution of the distribution of
the system: the various strategies are ranked by scores. It is clear to see that
only the high ranking strategies survive the evolutionary process (in fact,
only \input{./assets/img/replicator_dynamics/main.tex}
have a final distribution greater than \(10 ^ {-2}\)). This confirms the
findings of~\cite{Moran1707} in which sophisticated strategies resist
evolutionary invasion of shorter memory strategies. Recalling
Figure~\ref{fig:SSError_and_probabilities_in_full} this demonstrates that:

\begin{itemize}
    \item Cooperation emerges through the evolutionary process: the high scoring
        strategies do not exhibit extortionate behaviour towards each other.
    \item Extortionate strategies do not survive the evolutionary process.
\end{itemize}

\begin{figure}[!htbp]
    \centering
    \includegraphics[width=.8\textwidth]{./assets/img/replicator_dynamics/main.pdf}
    \caption{Numerical simulation of the replicator equation
    (\ref{eqn:replicator_dynamics}): strategies are ordered by score, only the strategies with a high score survive the evolutionary process.}
    \label{fig:replicator_dynamics}
\end{figure}

This work can be used to classify plays of the IPD\@: data can be collected from
actual interactions (in lab or in the field). Furthermore, this allows for a
classification method similar to the notion of fingerprinting presented
in~\cite{Ashlock2008}. Trained strategies can potentially be classified as
extortionate or not or it could be possible to even constrain the reinforcement
learning approaches that are becoming prevalent in the literature.
Alternatively, this mathematical approach for recognising extortion could be
used in sophisticated strategies to defend against invasion. Arguably, some of
the strategies considered here exhibit this behaviour, indeed as described
in~\cite{Harper2017}, the top ranking strategies in the full tournament are
obtained using evolutionary reinforcement learning techniques, thus, suspicion
of extortionate behaviour could in fact be an evolutionary trait.

\section*{Acknowledgements}

The following open source software libraries were used in this research:

\begin{itemize}
    \item The Axelrod ~\cite{Knight2016, Knight2018} library (IPD strategies and
        tournaments).
    \item The sympy library~\cite{Meurer2017} (verification of all symbolic
        calculations).
    \item The matplotlib~\cite{Droettboom2018} library (visualisation).
    \item The pandas~\cite{Structures2010}, dask~\cite{Dask2016} and
        NumPy~\cite{Oliphant2015} libraries (data manipulation).
    \item The SciPy~\cite{Jones2001} library (numerical integration of the
        replicator equation).
\end{itemize}

This work was performed using the computational facilities of the Advanced
Research Computing @ Cardiff (ARCCA) Division, Cardiff University.

\printbibliography

\newpage
\section*{Supplementary materials}

\includepdf{assets/pdf/proof_of_form_of_extortionate_strategies/main.pdf}

\newpage

Using the pair wise interactions the transition rates \(p,
q\) can be measured and the steady state probabilities inferred and compared to
the actual probabilities of each state.
This is done numerically by computing the singular eigenvector of the
matrix \(A\) \cite{Stewart2009}:

\[
    A =
    \begin{bmatrix}
        p_1 q_1 & p_1 (1 - q_1) & (1 - p_1) q_1 & (1 -p_1) (1 - q_1) \\
        p_2 q_2 & p_2 (1 - q_2) & (1 - p_2) q_2 & (1 -p_2) (1 - q_2) \\
        p_3 q_3 & p_3 (1 - q_3) & (1 - p_3) q_3 & (1 -p_3) (1 - q_3) \\
        p_4 q_4 & p_4 (1 - q_4) & (1 - p_4) q_4 & (1 -p_4) (1 - q_4) \\
    \end{bmatrix}
\]

Figure~\ref{fig:computed_probabilities_vs_theoretic_probabilities} shows a
regression line fitted to every pairwise interaction with a reported
\(\text{SSError}\) value (pairwise interactions with missing states were
omitted). This serves to validate the approach: a part from some edge cases the
relationship is consistent.

\begin{figure}[!htbp]
    \centering
    \includegraphics[width=.8\textwidth]{./assets/img/computed_probabilities_vs_theoretic_probabilities/main.pdf}
    \caption{The
        relationship between the steady state probabilities inferred from the
        measured transitions and the actual steady state probabilities. A linear
        regression line is included validating the approach.}
    \label{fig:computed_probabilities_vs_theoretic_probabilities}
\end{figure}


\end{document}

have a final distribution greater than \(10 ^ {-2}\)). This confirms the
findings of~\cite{Moran1707} in which sophisticated strategies resist
evolutionary invasion of shorter memory strategies. Recalling
Figure~\ref{fig:SSError_and_probabilities_in_full} this demonstrates that:

\begin{itemize}
    \item Cooperation emerges through the evolutionary process: the high scoring
        strategies do not exhibit extortionate behaviour towards each other.
    \item Extortionate strategies do not survive the evolutionary process.
\end{itemize}

\begin{figure}[!htbp]
    \centering
    \includegraphics[width=.8\textwidth]{./assets/img/replicator_dynamics/main.pdf}
    \caption{Numerical simulation of the replicator equation
    (\ref{eqn:replicator_dynamics}): strategies are ordered by score, only the strategies with a high score survive the evolutionary process.}
    \label{fig:replicator_dynamics}
\end{figure}

This work can be used to classify plays of the IPD\@: data can be collected from
actual interactions (in lab or in the field). Furthermore, this allows for a
classification method similar to the notion of fingerprinting presented
in~\cite{Ashlock2008}. Trained strategies can potentially be classified as
extortionate or not or it could be possible to even constrain the reinforcement
learning approaches that are becoming prevalent in the literature.
Alternatively, this mathematical approach for recognising extortion could be
used in sophisticated strategies to defend against invasion. Arguably, some of
the strategies considered here exhibit this behaviour, indeed as described
in~\cite{Harper2017}, the top ranking strategies in the full tournament are
obtained using evolutionary reinforcement learning techniques, thus, suspicion
of extortionate behaviour could in fact be an evolutionary trait.

\section*{Acknowledgements}

The following open source software libraries were used in this research:

\begin{itemize}
    \item The Axelrod ~\cite{Knight2016, Knight2018} library (IPD strategies and
        tournaments).
    \item The sympy library~\cite{Meurer2017} (verification of all symbolic
        calculations).
    \item The matplotlib~\cite{Droettboom2018} library (visualisation).
    \item The pandas~\cite{Structures2010}, dask~\cite{Dask2016} and
        NumPy~\cite{Oliphant2015} libraries (data manipulation).
    \item The SciPy~\cite{Jones2001} library (numerical integration of the
        replicator equation).
\end{itemize}

This work was performed using the computational facilities of the Advanced
Research Computing @ Cardiff (ARCCA) Division, Cardiff University.

\printbibliography

\newpage
\section*{Supplementary materials}

\includepdf{assets/pdf/proof_of_form_of_extortionate_strategies/main.pdf}

\newpage

Using the pair wise interactions the transition rates \(p,
q\) can be measured and the steady state probabilities inferred and compared to
the actual probabilities of each state.
This is done numerically by computing the singular eigenvector of the
matrix \(A\) \cite{Stewart2009}:

\[
    A =
    \begin{bmatrix}
        p_1 q_1 & p_1 (1 - q_1) & (1 - p_1) q_1 & (1 -p_1) (1 - q_1) \\
        p_2 q_2 & p_2 (1 - q_2) & (1 - p_2) q_2 & (1 -p_2) (1 - q_2) \\
        p_3 q_3 & p_3 (1 - q_3) & (1 - p_3) q_3 & (1 -p_3) (1 - q_3) \\
        p_4 q_4 & p_4 (1 - q_4) & (1 - p_4) q_4 & (1 -p_4) (1 - q_4) \\
    \end{bmatrix}
\]

Figure~\ref{fig:computed_probabilities_vs_theoretic_probabilities} shows a
regression line fitted to every pairwise interaction with a reported
\(\text{SSError}\) value (pairwise interactions with missing states were
omitted). This serves to validate the approach: a part from some edge cases the
relationship is consistent.

\begin{figure}[!htbp]
    \centering
    \includegraphics[width=.8\textwidth]{./assets/img/computed_probabilities_vs_theoretic_probabilities/main.pdf}
    \caption{The
        relationship between the steady state probabilities inferred from the
        measured transitions and the actual steady state probabilities. A linear
        regression line is included validating the approach.}
    \label{fig:computed_probabilities_vs_theoretic_probabilities}
\end{figure}


\end{document}

    strategies is considered. In this setting
    the most highly performing strategies do not play in an extortionate way
    against each other but do against lower performing strategies.
    This suggests that whilst the theory of Zero Determinant strategies
    indicates that memory is not of fundamental importance to the evolution of
    cooperative behaviour, this is incomplete.
\end{abstract}

\section{Introduction}\label{sec:introduction}

Agent based game theoretic models have become a stalwart of the underpinning
mathematics of interactive behaviours. One of the major pieces of work
in this area is the pair of original computer tournaments run by Robert
Axelrod~\cite{Axelrod1980, Axelrod1980a}. These tournaments pitted submitted
computer strategies against each other in plays of the Iterated Prisoner's
Dilemma. A common game where agents can choose to pay a slight cost to their
immediate utility in the hope of building a reputation. This has been used in
economic and evolutionary game theory to understand the evolution of cooperative
behaviour.

Recently, a class of strategies was described in~\cite{Press2012} that can
provably extort any given opponent. In~\cite{Hilbe2013, Moran1707} some
questions have already been asked about the true effectiveness of these
strategies in an evolutionary setting. Here another question is asked: is it
possible to recognise this extortionate behaviour? A mathematical procedure for
suspicion is presented: in the same way that the continued actions of an
extortionate individual might raise suspicion.

This work makes use of the Axelrod Python library~\cite{Knight2018, Knight2016}
with a large number of Prisoner Dilemma strategies available to give an
extensive numerical example of the ideas presented.  The approach is presented
in Section~\ref{sec:delta-zd-strategies}.  All of the code and data discussed
in Section~\ref{sec:numerical-experiments} is open sourced, archived and
written according to best scientific principles~\cite{Wilson2014}. The data
archive can be found at~\cite{vincent_knight_2018_1297075}.

\section{Recognising Extortion}\label{sec:delta-zd-strategies}

In~\cite{Press2012}, given a match between 2 memory-one strategies, the concept
of Zero Determinant (ZD) strategies is introduced. The main result of that paper
shows that given two memory one players \(p, q\in\mathbb{R}^4\) a linear
relationship between the players' scores could be forced by one of the players.

Using the notation of~\cite{Press2012}, assuming the utilities for player \(p\)
are given by \(S_x=(R, S, T, P)\) and for player \(q\) by \(S_y=(R, T, S, P)\)
and that the stationary scores of each player is given by \(S_X\) and \(S_Y\)
respectively. The main result of~\cite{Press2012} is that if

\begin{equation}\label{eqn:linear_relationship_for_p}
    \tilde p=\alpha S_x + \beta S_y + \gamma
\end{equation}

or

\begin{equation}\label{eqn:linear_relationship_for_q}
    \tilde q=\alpha S_x + \beta S_y + \gamma
\end{equation}

where \(\tilde p = (1 - p_1, 1 - p_2, p_3, p_4)\) and
\(\tilde q = (1 - q_1, 1 - q_2, q_3, q_4)\) then:

\begin{equation}
    \alpha S_X + \beta S_Y + \gamma = 0
\end{equation}

In~\cite{Press2012} a particular type of ZD strategy is defined: extortionate
strategies. If:

\begin{equation}\label{eqn:constraint_for_extortion}
    \gamma = - P(\alpha + \beta)
\end{equation}

then the player can ensure they get a score \(\chi\) times
larger than the opponent. This extortion coefficient is given by:

\begin{equation}\label{eqn:definition_of_chi}
    \chi=\frac{-\beta}{\alpha}
\end{equation}

Thus, if (\ref{eqn:constraint_for_extortion}) holds and \(\chi >1\) a player is
said to extort their opponent.
Here, the reverse problem is considered: given a
\(p\in\mathbb{R}^4\) how does one identify \(\alpha, \beta\) if they
exist and is the strategy in fact acting in an extortionate way?

These conditions correspond to:

\begin{align}
    \tilde p_1 & = \alpha R + \beta R - P (\alpha + \beta)
            \label{eqn:condition_for_tilde_p1}\\
    \tilde p_2 & = \alpha S + \beta T - P (\alpha + \beta)
            \label{eqn:condition_for_tilde_p2}\\
    \tilde p_3 & = \alpha T + \beta S - P (\alpha + \beta)
            \label{eqn:condition_for_tilde_p3}\\
    \tilde p_4 & = \alpha P + \beta P - P (\alpha + \beta)
            \label{eqn:condition_for_tilde_p4}
\end{align}

Equation (\ref{eqn:condition_for_tilde_p4}) ensures that \(p_4=\tilde p_4=0\).
Equations (\ref{eqn:condition_for_tilde_p1}-\ref{eqn:condition_for_tilde_p3})
can be used to eliminate \(\alpha, \beta\), giving:

\begin{equation}\label{eqn:planar_definition_of_extortion}
    \tilde p_1 = \frac{(R - P)(\tilde p_2 + \tilde p_3)}{S + T - 2P}
\end{equation}

with:

\begin{equation}\label{eqn:definition_of_chi}
    \chi = \frac{\tilde p_2 (P - T) + \tilde p_3 (S - P)}
                {\tilde p_2 (P - S) + \tilde p_3 (T - P)}
\end{equation}

Given a strategy \(p\in\mathbb{R}^{4\times 1}\) equations
(\ref{eqn:condition_for_tilde_p4}), (\ref{eqn:planar_definition_of_extortion}-\ref{eqn:definition_of_chi}) can be used to check if
a strategy is extortionate. The conditions correspond to:

\begin{align}
    p_1 & = \frac{(R-P)(p_2 + p_3) - R + T + S - P}{S + T - 2P}
     \label{eqn:condition_for_p1}\\
    p_4 & = 0 \label{eqn:condition_for_p4}\\
    1 & > p_2 + p_3\label{eqn:condition_for_chi}
\end{align}

The algebraic steps necessary to prove these results are available in the
supporting materials.

All extortionate strategies reside on a triangular (\ref{eqn:condition_for_chi})
plane (\ref{eqn:condition_for_p1}) in 3 dimensions (\ref{eqn:condition_for_p4}).
Using this formulation it can be seen that a necessary (but not sufficient)
condition for an extortionate strategy is that it cooperates on average less
than 50\% of the time when in a state of disagreement with the opponent.

As an example, consider the known extortionate strategy \(p=(8 / 9, 1 / 2, 1 /
3, 0)\) from~\cite{Stewart2012} which is referred to as \texttt{Extort-2}. In
this case, for the standard values of \((R, T, S, P)\) constraint
(\ref{eqn:condition_for_p1}) corresponds to:

\begin{equation}
    p_1 = \frac{2(p_2 + p_3) + 1}{3}
\end{equation}

It is clear that in this case all constraints hold.

This approach could in fact be used to confirm that a given strategy is acting
in an extortionate manner even if it is not a memory one strategy. However, in
practice, if a closed form for \(p\) is not known, then due to measurement
and/or numerical error this would not work.

This problem can be written in the following linear algebraic form where
\(x=(\alpha, \beta)\)
and \(p^*=(\tilde p_1 - 1, tilde_2 - 1, p_3)\):

\begin{equation}\label{eqn:linear_algebraic_equation_for_p}
    Cx= p^*
\end{equation}

\(C\) corresponds to equations
(\ref{eqn:condition_for_tilde_p1}-\ref{eqn:condition_for_tilde_p3}) and is
given by:

\begin{equation}\label{eqn:definition_of_C}
    C =
    \begin{bmatrix}
        R - P & R- P \\
        S - P & T- P \\
        T - P & S- P \\
    \end{bmatrix}
\end{equation}

Note that in general, equation (\ref{eqn:linear_algebraic_equation_for_p}) will
not necessarily have a solution. From the Rouch\'{e}-Capelli theorem if there is
a solution it is unique as \(\text{rank}(C)=2\) which is the dimension of the
variable \(x\). The best fitting \(x\) is found by minimizing:

\begin{equation}\label{eqn:r_squared}
    \text{SSError} = \|C x- p^*\|_2^2 = \sum_{i=1}^{3}\left((C\bar x)_i-p_i^*\right)^2
\end{equation}

Note that \(\text{SSError}\), which is the square of the Frobenius
norm~\cite{Golub2013}, becomes a measure of how close a strategy is to being an
extortionate strategy. Suspicion
of extortion then corresponds to a threshold on \(\text{SSError}\).

By observing interactions (human or otherwise), their memory one representation
can be inferred and this approach can be used to recognise extortionate
behaviour. The notion of comparing theoretic and actual plays of the IPD is not
novel, see for example~\cite{Rand2013}. Immediately it is noted that if the
environment is noisy~\cite{Wu1995} then no strategy can be considered to be
extortionate as \(p_4>0\).

In the next section, this idea will be illustrated by observing the interactions
that take place in a computer based tournament of the IPD\@.

\section{Numerical experiments}\label{sec:numerical-experiments}

In~\cite{Stewart2012} results from a tournament with
\documentclass[a4paper]{article}

\usepackage{amsmath}
\usepackage{amssymb}
\usepackage[margin=1.5cm,
            includefoot,
            footskip=30pt]{geometry}
\usepackage{layout}
\usepackage{graphicx}
\usepackage{subcaption}

\usepackage{biblatex}
\usepackage{pdfpages}

\bibliography{main.bib}

\title{Suspicion: Recognising and evaluating the effectiveness
       of extortion in the Iterated Prisoner's Dilemma}
\author{Vincent A. Knight \and Nikoleta E. Glynatsi}
\date{\today}



\begin{document}

\maketitle

\begin{abstract}
    The Iterated Prisoner's Dilemma is a model for rational and evolutionary
    interactive behaviour. It has applications both in the study of human social
    behaviour as well as in biology.
    It is used to understand when and how a rational individual might
    accept an immediate cost to their own utility for the direct benefit of
    another.

    Much attention has been given to a class of strategies called
    Zero Determinant strategies. It has been theoretically shown that these
    strategies can ``extort'' any player.

    In this work, an approach to identify if observed strategies are playing in
    an extortionate way is described. Furthermore, experimental analysis of
    a large tournament with \documentclass[a4paper]{article}

\usepackage{amsmath}
\usepackage{amssymb}
\usepackage[margin=1.5cm,
            includefoot,
            footskip=30pt]{geometry}
\usepackage{layout}
\usepackage{graphicx}
\usepackage{subcaption}

\usepackage{biblatex}
\usepackage{pdfpages}

\bibliography{main.bib}

\title{Suspicion: Recognising and evaluating the effectiveness
       of extortion in the Iterated Prisoner's Dilemma}
\author{Vincent A. Knight \and Nikoleta E. Glynatsi}
\date{\today}



\begin{document}

\maketitle

\begin{abstract}
    The Iterated Prisoner's Dilemma is a model for rational and evolutionary
    interactive behaviour. It has applications both in the study of human social
    behaviour as well as in biology.
    It is used to understand when and how a rational individual might
    accept an immediate cost to their own utility for the direct benefit of
    another.

    Much attention has been given to a class of strategies called
    Zero Determinant strategies. It has been theoretically shown that these
    strategies can ``extort'' any player.

    In this work, an approach to identify if observed strategies are playing in
    an extortionate way is described. Furthermore, experimental analysis of
    a large tournament with \input{assets/tex/number_of_full_strategies/main.tex}
    strategies is considered. In this setting
    the most highly performing strategies do not play in an extortionate way
    against each other but do against lower performing strategies.
    This suggests that whilst the theory of Zero Determinant strategies
    indicates that memory is not of fundamental importance to the evolution of
    cooperative behaviour, this is incomplete.
\end{abstract}

\section{Introduction}\label{sec:introduction}

Agent based game theoretic models have become a stalwart of the underpinning
mathematics of interactive behaviours. One of the major pieces of work
in this area is the pair of original computer tournaments run by Robert
Axelrod~\cite{Axelrod1980, Axelrod1980a}. These tournaments pitted submitted
computer strategies against each other in plays of the Iterated Prisoner's
Dilemma. A common game where agents can choose to pay a slight cost to their
immediate utility in the hope of building a reputation. This has been used in
economic and evolutionary game theory to understand the evolution of cooperative
behaviour.

Recently, a class of strategies was described in~\cite{Press2012} that can
provably extort any given opponent. In~\cite{Hilbe2013, Moran1707} some
questions have already been asked about the true effectiveness of these
strategies in an evolutionary setting. Here another question is asked: is it
possible to recognise this extortionate behaviour? A mathematical procedure for
suspicion is presented: in the same way that the continued actions of an
extortionate individual might raise suspicion.

This work makes use of the Axelrod Python library~\cite{Knight2018, Knight2016}
with a large number of Prisoner Dilemma strategies available to give an
extensive numerical example of the ideas presented.  The approach is presented
in Section~\ref{sec:delta-zd-strategies}.  All of the code and data discussed
in Section~\ref{sec:numerical-experiments} is open sourced, archived and
written according to best scientific principles~\cite{Wilson2014}. The data
archive can be found at~\cite{vincent_knight_2018_1297075}.

\section{Recognising Extortion}\label{sec:delta-zd-strategies}

In~\cite{Press2012}, given a match between 2 memory-one strategies, the concept
of Zero Determinant (ZD) strategies is introduced. The main result of that paper
shows that given two memory one players \(p, q\in\mathbb{R}^4\) a linear
relationship between the players' scores could be forced by one of the players.

Using the notation of~\cite{Press2012}, assuming the utilities for player \(p\)
are given by \(S_x=(R, S, T, P)\) and for player \(q\) by \(S_y=(R, T, S, P)\)
and that the stationary scores of each player is given by \(S_X\) and \(S_Y\)
respectively. The main result of~\cite{Press2012} is that if

\begin{equation}\label{eqn:linear_relationship_for_p}
    \tilde p=\alpha S_x + \beta S_y + \gamma
\end{equation}

or

\begin{equation}\label{eqn:linear_relationship_for_q}
    \tilde q=\alpha S_x + \beta S_y + \gamma
\end{equation}

where \(\tilde p = (1 - p_1, 1 - p_2, p_3, p_4)\) and
\(\tilde q = (1 - q_1, 1 - q_2, q_3, q_4)\) then:

\begin{equation}
    \alpha S_X + \beta S_Y + \gamma = 0
\end{equation}

In~\cite{Press2012} a particular type of ZD strategy is defined: extortionate
strategies. If:

\begin{equation}\label{eqn:constraint_for_extortion}
    \gamma = - P(\alpha + \beta)
\end{equation}

then the player can ensure they get a score \(\chi\) times
larger than the opponent. This extortion coefficient is given by:

\begin{equation}\label{eqn:definition_of_chi}
    \chi=\frac{-\beta}{\alpha}
\end{equation}

Thus, if (\ref{eqn:constraint_for_extortion}) holds and \(\chi >1\) a player is
said to extort their opponent.
Here, the reverse problem is considered: given a
\(p\in\mathbb{R}^4\) how does one identify \(\alpha, \beta\) if they
exist and is the strategy in fact acting in an extortionate way?

These conditions correspond to:

\begin{align}
    \tilde p_1 & = \alpha R + \beta R - P (\alpha + \beta)
            \label{eqn:condition_for_tilde_p1}\\
    \tilde p_2 & = \alpha S + \beta T - P (\alpha + \beta)
            \label{eqn:condition_for_tilde_p2}\\
    \tilde p_3 & = \alpha T + \beta S - P (\alpha + \beta)
            \label{eqn:condition_for_tilde_p3}\\
    \tilde p_4 & = \alpha P + \beta P - P (\alpha + \beta)
            \label{eqn:condition_for_tilde_p4}
\end{align}

Equation (\ref{eqn:condition_for_tilde_p4}) ensures that \(p_4=\tilde p_4=0\).
Equations (\ref{eqn:condition_for_tilde_p1}-\ref{eqn:condition_for_tilde_p3})
can be used to eliminate \(\alpha, \beta\), giving:

\begin{equation}\label{eqn:planar_definition_of_extortion}
    \tilde p_1 = \frac{(R - P)(\tilde p_2 + \tilde p_3)}{S + T - 2P}
\end{equation}

with:

\begin{equation}\label{eqn:definition_of_chi}
    \chi = \frac{\tilde p_2 (P - T) + \tilde p_3 (S - P)}
                {\tilde p_2 (P - S) + \tilde p_3 (T - P)}
\end{equation}

Given a strategy \(p\in\mathbb{R}^{4\times 1}\) equations
(\ref{eqn:condition_for_tilde_p4}), (\ref{eqn:planar_definition_of_extortion}-\ref{eqn:definition_of_chi}) can be used to check if
a strategy is extortionate. The conditions correspond to:

\begin{align}
    p_1 & = \frac{(R-P)(p_2 + p_3) - R + T + S - P}{S + T - 2P}
     \label{eqn:condition_for_p1}\\
    p_4 & = 0 \label{eqn:condition_for_p4}\\
    1 & > p_2 + p_3\label{eqn:condition_for_chi}
\end{align}

The algebraic steps necessary to prove these results are available in the
supporting materials.

All extortionate strategies reside on a triangular (\ref{eqn:condition_for_chi})
plane (\ref{eqn:condition_for_p1}) in 3 dimensions (\ref{eqn:condition_for_p4}).
Using this formulation it can be seen that a necessary (but not sufficient)
condition for an extortionate strategy is that it cooperates on average less
than 50\% of the time when in a state of disagreement with the opponent.

As an example, consider the known extortionate strategy \(p=(8 / 9, 1 / 2, 1 /
3, 0)\) from~\cite{Stewart2012} which is referred to as \texttt{Extort-2}. In
this case, for the standard values of \((R, T, S, P)\) constraint
(\ref{eqn:condition_for_p1}) corresponds to:

\begin{equation}
    p_1 = \frac{2(p_2 + p_3) + 1}{3}
\end{equation}

It is clear that in this case all constraints hold.

This approach could in fact be used to confirm that a given strategy is acting
in an extortionate manner even if it is not a memory one strategy. However, in
practice, if a closed form for \(p\) is not known, then due to measurement
and/or numerical error this would not work.

This problem can be written in the following linear algebraic form where
\(x=(\alpha, \beta)\)
and \(p^*=(\tilde p_1 - 1, tilde_2 - 1, p_3)\):

\begin{equation}\label{eqn:linear_algebraic_equation_for_p}
    Cx= p^*
\end{equation}

\(C\) corresponds to equations
(\ref{eqn:condition_for_tilde_p1}-\ref{eqn:condition_for_tilde_p3}) and is
given by:

\begin{equation}\label{eqn:definition_of_C}
    C =
    \begin{bmatrix}
        R - P & R- P \\
        S - P & T- P \\
        T - P & S- P \\
    \end{bmatrix}
\end{equation}

Note that in general, equation (\ref{eqn:linear_algebraic_equation_for_p}) will
not necessarily have a solution. From the Rouch\'{e}-Capelli theorem if there is
a solution it is unique as \(\text{rank}(C)=2\) which is the dimension of the
variable \(x\). The best fitting \(x\) is found by minimizing:

\begin{equation}\label{eqn:r_squared}
    \text{SSError} = \|C x- p^*\|_2^2 = \sum_{i=1}^{3}\left((C\bar x)_i-p_i^*\right)^2
\end{equation}

Note that \(\text{SSError}\), which is the square of the Frobenius
norm~\cite{Golub2013}, becomes a measure of how close a strategy is to being an
extortionate strategy. Suspicion
of extortion then corresponds to a threshold on \(\text{SSError}\).

By observing interactions (human or otherwise), their memory one representation
can be inferred and this approach can be used to recognise extortionate
behaviour. The notion of comparing theoretic and actual plays of the IPD is not
novel, see for example~\cite{Rand2013}. Immediately it is noted that if the
environment is noisy~\cite{Wu1995} then no strategy can be considered to be
extortionate as \(p_4>0\).

In the next section, this idea will be illustrated by observing the interactions
that take place in a computer based tournament of the IPD\@.

\section{Numerical experiments}\label{sec:numerical-experiments}

In~\cite{Stewart2012} results from a tournament with
\input{./assets/tex/number_of_stewart_plotkin_strategies/main.tex} strategies,
was presented with specific consideration given to ZD strategies. This
tournament is reproduced here using the Axelrod-Python
project~\cite{Knight2016}. To obtain a good measure of the corresponding
transition rates for each strategy all matches have been run for
\input{assets/tex/number_of_turns/main.tex} turns and every match has been
repeated \input{assets/tex/number_of_repetitions/main.tex} times. All of this
interaction data is available at~\cite{vincent_knight_2018_1297075}. A good
match between the inferred Markov chain and the state distribution of the actual
interactions has been verified. Data for this is presented in the supplementary
materials.

Figure~\ref{fig:SSError_overall_in_stewart_plotkin} shows the \(\text{SSError}\)
values for all the strategies in the tournament, as reported
in~\cite{Stewart2012} the extortionate strategy (which has an expected
\(\text{SSError}\) approximately 0) gains a large number of wins.

\begin{figure}[!htbp]
    \centering
    \includegraphics[width=.8\textwidth]{./assets/img/SSError_overall_in_stewart_plotkin/main.pdf}
    \caption{\(\text{SSError}\) and state probabilities for the strategies
        of~\cite{Stewart2012}, ordered both by number of wins and overall score.
        Note that \(P(DC)\) is not shown as it corresponds to the transpose of
        \(P(CD)\). Cooperator and Defector are omitted as they do not visit all
        the states.}
    \label{fig:SSError_overall_in_stewart_plotkin}
\end{figure}

Here, the work of~\cite{Stewart2012} is extended by investigating a tournament
with \input{assets/tex/number_of_full_strategies/main.tex}
strategies.

The results of this analysis are shown in
Figure~\ref{fig:SSError_and_probabilities_in_full}. The top ranking strategies
by number of wins seem to be extortionate (but not against all strategies) and
it can be seen that a small sub group of strategies achieve mutual defection.
All the top ranking strategies according to score achieve mutual cooperation and
do not extort each other, however they
\textbf{do} exhibit extortionate behaviour towards a number of the lower ranking
strategies.

\begin{figure}[!htbp]
    \centering
    \includegraphics[width=.8\textwidth]{./assets/img/SSError_and_probabilities_in_full/main.pdf}
    \caption{\(\text{SSError}\) for the strategies for the full tournament. Only
    strategy interactions for which \(p_4=0\) and \(\chi>1\) are displayed.}
    \label{fig:SSError_and_probabilities_in_full}
\end{figure}

\section{Conclusion}\label{sec:conclusion}

This work defines an approach to measure whether or not a player is playing a
strategy that corresponds to an extortionate strategy as defined
in~\cite{Press2012}: a mathematical model for suspicion. Indeed, all
extortionate strategies have been
 classified as lying on a triangular plane.
This rigorous classification fails to be robust to small measurement error, thus
a statistical approach is proposed.
This is done through a linear algebraic approach for approximating the solution
of a linear system. Using this, a large number of pairwise interactions is
simulated and in fact very few strategies are found to act extortionately.

The work of~\cite{Press2012}, whilst showing that a clever approach to taking
advantage of another memory one strategy exists: this is incomplete. Whilst the
elegance of this result is very attractive, just as the simplicity of the
victory of Tit For Tat in Axelrod's original tournaments was, it is incomplete.
Extortionate strategies achieve a high number of wins but they do not
achieve a high score which corresponds to the fitness landscape in an
evolutionary sense. From the large number of interactions a payoff matrix \(S\)
can be measured where \(S_{ij}\) denotes the score (using standard values of
\((R, S, T, P) = (3, 0, 5, 1)\)) of the \(i\)th strategy
against the \(j\)th strategy. Using this, the replicator equation
describes the evolution of the system based on a population density fitness
function:

\begin{equation}\label{eqn:replicator_dynamics}
    \frac{dx}{dt} = x(S-x^TS x)
\end{equation}

Equation (\ref{eqn:replicator_dynamics}) is solved numerically through an
integration technique described in~\cite{Petzold1983} and
Figure~\ref{fig:replicator_dynamics} shows the evolution of the distribution of
the system: the various strategies are ranked by scores. It is clear to see that
only the high ranking strategies survive the evolutionary process (in fact,
only \input{./assets/img/replicator_dynamics/main.tex}
have a final distribution greater than \(10 ^ {-2}\)). This confirms the
findings of~\cite{Moran1707} in which sophisticated strategies resist
evolutionary invasion of shorter memory strategies. Recalling
Figure~\ref{fig:SSError_and_probabilities_in_full} this demonstrates that:

\begin{itemize}
    \item Cooperation emerges through the evolutionary process: the high scoring
        strategies do not exhibit extortionate behaviour towards each other.
    \item Extortionate strategies do not survive the evolutionary process.
\end{itemize}

\begin{figure}[!htbp]
    \centering
    \includegraphics[width=.8\textwidth]{./assets/img/replicator_dynamics/main.pdf}
    \caption{Numerical simulation of the replicator equation
    (\ref{eqn:replicator_dynamics}): strategies are ordered by score, only the strategies with a high score survive the evolutionary process.}
    \label{fig:replicator_dynamics}
\end{figure}

This work can be used to classify plays of the IPD\@: data can be collected from
actual interactions (in lab or in the field). Furthermore, this allows for a
classification method similar to the notion of fingerprinting presented
in~\cite{Ashlock2008}. Trained strategies can potentially be classified as
extortionate or not or it could be possible to even constrain the reinforcement
learning approaches that are becoming prevalent in the literature.
Alternatively, this mathematical approach for recognising extortion could be
used in sophisticated strategies to defend against invasion. Arguably, some of
the strategies considered here exhibit this behaviour, indeed as described
in~\cite{Harper2017}, the top ranking strategies in the full tournament are
obtained using evolutionary reinforcement learning techniques, thus, suspicion
of extortionate behaviour could in fact be an evolutionary trait.

\section*{Acknowledgements}

The following open source software libraries were used in this research:

\begin{itemize}
    \item The Axelrod ~\cite{Knight2016, Knight2018} library (IPD strategies and
        tournaments).
    \item The sympy library~\cite{Meurer2017} (verification of all symbolic
        calculations).
    \item The matplotlib~\cite{Droettboom2018} library (visualisation).
    \item The pandas~\cite{Structures2010}, dask~\cite{Dask2016} and
        NumPy~\cite{Oliphant2015} libraries (data manipulation).
    \item The SciPy~\cite{Jones2001} library (numerical integration of the
        replicator equation).
\end{itemize}

This work was performed using the computational facilities of the Advanced
Research Computing @ Cardiff (ARCCA) Division, Cardiff University.

\printbibliography

\newpage
\section*{Supplementary materials}

\includepdf{assets/pdf/proof_of_form_of_extortionate_strategies/main.pdf}

\newpage

Using the pair wise interactions the transition rates \(p,
q\) can be measured and the steady state probabilities inferred and compared to
the actual probabilities of each state.
This is done numerically by computing the singular eigenvector of the
matrix \(A\) \cite{Stewart2009}:

\[
    A =
    \begin{bmatrix}
        p_1 q_1 & p_1 (1 - q_1) & (1 - p_1) q_1 & (1 -p_1) (1 - q_1) \\
        p_2 q_2 & p_2 (1 - q_2) & (1 - p_2) q_2 & (1 -p_2) (1 - q_2) \\
        p_3 q_3 & p_3 (1 - q_3) & (1 - p_3) q_3 & (1 -p_3) (1 - q_3) \\
        p_4 q_4 & p_4 (1 - q_4) & (1 - p_4) q_4 & (1 -p_4) (1 - q_4) \\
    \end{bmatrix}
\]

Figure~\ref{fig:computed_probabilities_vs_theoretic_probabilities} shows a
regression line fitted to every pairwise interaction with a reported
\(\text{SSError}\) value (pairwise interactions with missing states were
omitted). This serves to validate the approach: a part from some edge cases the
relationship is consistent.

\begin{figure}[!htbp]
    \centering
    \includegraphics[width=.8\textwidth]{./assets/img/computed_probabilities_vs_theoretic_probabilities/main.pdf}
    \caption{The
        relationship between the steady state probabilities inferred from the
        measured transitions and the actual steady state probabilities. A linear
        regression line is included validating the approach.}
    \label{fig:computed_probabilities_vs_theoretic_probabilities}
\end{figure}


\end{document}

    strategies is considered. In this setting
    the most highly performing strategies do not play in an extortionate way
    against each other but do against lower performing strategies.
    This suggests that whilst the theory of Zero Determinant strategies
    indicates that memory is not of fundamental importance to the evolution of
    cooperative behaviour, this is incomplete.
\end{abstract}

\section{Introduction}\label{sec:introduction}

Agent based game theoretic models have become a stalwart of the underpinning
mathematics of interactive behaviours. One of the major pieces of work
in this area is the pair of original computer tournaments run by Robert
Axelrod~\cite{Axelrod1980, Axelrod1980a}. These tournaments pitted submitted
computer strategies against each other in plays of the Iterated Prisoner's
Dilemma. A common game where agents can choose to pay a slight cost to their
immediate utility in the hope of building a reputation. This has been used in
economic and evolutionary game theory to understand the evolution of cooperative
behaviour.

Recently, a class of strategies was described in~\cite{Press2012} that can
provably extort any given opponent. In~\cite{Hilbe2013, Moran1707} some
questions have already been asked about the true effectiveness of these
strategies in an evolutionary setting. Here another question is asked: is it
possible to recognise this extortionate behaviour? A mathematical procedure for
suspicion is presented: in the same way that the continued actions of an
extortionate individual might raise suspicion.

This work makes use of the Axelrod Python library~\cite{Knight2018, Knight2016}
with a large number of Prisoner Dilemma strategies available to give an
extensive numerical example of the ideas presented.  The approach is presented
in Section~\ref{sec:delta-zd-strategies}.  All of the code and data discussed
in Section~\ref{sec:numerical-experiments} is open sourced, archived and
written according to best scientific principles~\cite{Wilson2014}. The data
archive can be found at~\cite{vincent_knight_2018_1297075}.

\section{Recognising Extortion}\label{sec:delta-zd-strategies}

In~\cite{Press2012}, given a match between 2 memory-one strategies, the concept
of Zero Determinant (ZD) strategies is introduced. The main result of that paper
shows that given two memory one players \(p, q\in\mathbb{R}^4\) a linear
relationship between the players' scores could be forced by one of the players.

Using the notation of~\cite{Press2012}, assuming the utilities for player \(p\)
are given by \(S_x=(R, S, T, P)\) and for player \(q\) by \(S_y=(R, T, S, P)\)
and that the stationary scores of each player is given by \(S_X\) and \(S_Y\)
respectively. The main result of~\cite{Press2012} is that if

\begin{equation}\label{eqn:linear_relationship_for_p}
    \tilde p=\alpha S_x + \beta S_y + \gamma
\end{equation}

or

\begin{equation}\label{eqn:linear_relationship_for_q}
    \tilde q=\alpha S_x + \beta S_y + \gamma
\end{equation}

where \(\tilde p = (1 - p_1, 1 - p_2, p_3, p_4)\) and
\(\tilde q = (1 - q_1, 1 - q_2, q_3, q_4)\) then:

\begin{equation}
    \alpha S_X + \beta S_Y + \gamma = 0
\end{equation}

In~\cite{Press2012} a particular type of ZD strategy is defined: extortionate
strategies. If:

\begin{equation}\label{eqn:constraint_for_extortion}
    \gamma = - P(\alpha + \beta)
\end{equation}

then the player can ensure they get a score \(\chi\) times
larger than the opponent. This extortion coefficient is given by:

\begin{equation}\label{eqn:definition_of_chi}
    \chi=\frac{-\beta}{\alpha}
\end{equation}

Thus, if (\ref{eqn:constraint_for_extortion}) holds and \(\chi >1\) a player is
said to extort their opponent.
Here, the reverse problem is considered: given a
\(p\in\mathbb{R}^4\) how does one identify \(\alpha, \beta\) if they
exist and is the strategy in fact acting in an extortionate way?

These conditions correspond to:

\begin{align}
    \tilde p_1 & = \alpha R + \beta R - P (\alpha + \beta)
            \label{eqn:condition_for_tilde_p1}\\
    \tilde p_2 & = \alpha S + \beta T - P (\alpha + \beta)
            \label{eqn:condition_for_tilde_p2}\\
    \tilde p_3 & = \alpha T + \beta S - P (\alpha + \beta)
            \label{eqn:condition_for_tilde_p3}\\
    \tilde p_4 & = \alpha P + \beta P - P (\alpha + \beta)
            \label{eqn:condition_for_tilde_p4}
\end{align}

Equation (\ref{eqn:condition_for_tilde_p4}) ensures that \(p_4=\tilde p_4=0\).
Equations (\ref{eqn:condition_for_tilde_p1}-\ref{eqn:condition_for_tilde_p3})
can be used to eliminate \(\alpha, \beta\), giving:

\begin{equation}\label{eqn:planar_definition_of_extortion}
    \tilde p_1 = \frac{(R - P)(\tilde p_2 + \tilde p_3)}{S + T - 2P}
\end{equation}

with:

\begin{equation}\label{eqn:definition_of_chi}
    \chi = \frac{\tilde p_2 (P - T) + \tilde p_3 (S - P)}
                {\tilde p_2 (P - S) + \tilde p_3 (T - P)}
\end{equation}

Given a strategy \(p\in\mathbb{R}^{4\times 1}\) equations
(\ref{eqn:condition_for_tilde_p4}), (\ref{eqn:planar_definition_of_extortion}-\ref{eqn:definition_of_chi}) can be used to check if
a strategy is extortionate. The conditions correspond to:

\begin{align}
    p_1 & = \frac{(R-P)(p_2 + p_3) - R + T + S - P}{S + T - 2P}
     \label{eqn:condition_for_p1}\\
    p_4 & = 0 \label{eqn:condition_for_p4}\\
    1 & > p_2 + p_3\label{eqn:condition_for_chi}
\end{align}

The algebraic steps necessary to prove these results are available in the
supporting materials.

All extortionate strategies reside on a triangular (\ref{eqn:condition_for_chi})
plane (\ref{eqn:condition_for_p1}) in 3 dimensions (\ref{eqn:condition_for_p4}).
Using this formulation it can be seen that a necessary (but not sufficient)
condition for an extortionate strategy is that it cooperates on average less
than 50\% of the time when in a state of disagreement with the opponent.

As an example, consider the known extortionate strategy \(p=(8 / 9, 1 / 2, 1 /
3, 0)\) from~\cite{Stewart2012} which is referred to as \texttt{Extort-2}. In
this case, for the standard values of \((R, T, S, P)\) constraint
(\ref{eqn:condition_for_p1}) corresponds to:

\begin{equation}
    p_1 = \frac{2(p_2 + p_3) + 1}{3}
\end{equation}

It is clear that in this case all constraints hold.

This approach could in fact be used to confirm that a given strategy is acting
in an extortionate manner even if it is not a memory one strategy. However, in
practice, if a closed form for \(p\) is not known, then due to measurement
and/or numerical error this would not work.

This problem can be written in the following linear algebraic form where
\(x=(\alpha, \beta)\)
and \(p^*=(\tilde p_1 - 1, tilde_2 - 1, p_3)\):

\begin{equation}\label{eqn:linear_algebraic_equation_for_p}
    Cx= p^*
\end{equation}

\(C\) corresponds to equations
(\ref{eqn:condition_for_tilde_p1}-\ref{eqn:condition_for_tilde_p3}) and is
given by:

\begin{equation}\label{eqn:definition_of_C}
    C =
    \begin{bmatrix}
        R - P & R- P \\
        S - P & T- P \\
        T - P & S- P \\
    \end{bmatrix}
\end{equation}

Note that in general, equation (\ref{eqn:linear_algebraic_equation_for_p}) will
not necessarily have a solution. From the Rouch\'{e}-Capelli theorem if there is
a solution it is unique as \(\text{rank}(C)=2\) which is the dimension of the
variable \(x\). The best fitting \(x\) is found by minimizing:

\begin{equation}\label{eqn:r_squared}
    \text{SSError} = \|C x- p^*\|_2^2 = \sum_{i=1}^{3}\left((C\bar x)_i-p_i^*\right)^2
\end{equation}

Note that \(\text{SSError}\), which is the square of the Frobenius
norm~\cite{Golub2013}, becomes a measure of how close a strategy is to being an
extortionate strategy. Suspicion
of extortion then corresponds to a threshold on \(\text{SSError}\).

By observing interactions (human or otherwise), their memory one representation
can be inferred and this approach can be used to recognise extortionate
behaviour. The notion of comparing theoretic and actual plays of the IPD is not
novel, see for example~\cite{Rand2013}. Immediately it is noted that if the
environment is noisy~\cite{Wu1995} then no strategy can be considered to be
extortionate as \(p_4>0\).

In the next section, this idea will be illustrated by observing the interactions
that take place in a computer based tournament of the IPD\@.

\section{Numerical experiments}\label{sec:numerical-experiments}

In~\cite{Stewart2012} results from a tournament with
\documentclass[a4paper]{article}

\usepackage{amsmath}
\usepackage{amssymb}
\usepackage[margin=1.5cm,
            includefoot,
            footskip=30pt]{geometry}
\usepackage{layout}
\usepackage{graphicx}
\usepackage{subcaption}

\usepackage{biblatex}
\usepackage{pdfpages}

\bibliography{main.bib}

\title{Suspicion: Recognising and evaluating the effectiveness
       of extortion in the Iterated Prisoner's Dilemma}
\author{Vincent A. Knight \and Nikoleta E. Glynatsi}
\date{\today}



\begin{document}

\maketitle

\begin{abstract}
    The Iterated Prisoner's Dilemma is a model for rational and evolutionary
    interactive behaviour. It has applications both in the study of human social
    behaviour as well as in biology.
    It is used to understand when and how a rational individual might
    accept an immediate cost to their own utility for the direct benefit of
    another.

    Much attention has been given to a class of strategies called
    Zero Determinant strategies. It has been theoretically shown that these
    strategies can ``extort'' any player.

    In this work, an approach to identify if observed strategies are playing in
    an extortionate way is described. Furthermore, experimental analysis of
    a large tournament with \input{assets/tex/number_of_full_strategies/main.tex}
    strategies is considered. In this setting
    the most highly performing strategies do not play in an extortionate way
    against each other but do against lower performing strategies.
    This suggests that whilst the theory of Zero Determinant strategies
    indicates that memory is not of fundamental importance to the evolution of
    cooperative behaviour, this is incomplete.
\end{abstract}

\section{Introduction}\label{sec:introduction}

Agent based game theoretic models have become a stalwart of the underpinning
mathematics of interactive behaviours. One of the major pieces of work
in this area is the pair of original computer tournaments run by Robert
Axelrod~\cite{Axelrod1980, Axelrod1980a}. These tournaments pitted submitted
computer strategies against each other in plays of the Iterated Prisoner's
Dilemma. A common game where agents can choose to pay a slight cost to their
immediate utility in the hope of building a reputation. This has been used in
economic and evolutionary game theory to understand the evolution of cooperative
behaviour.

Recently, a class of strategies was described in~\cite{Press2012} that can
provably extort any given opponent. In~\cite{Hilbe2013, Moran1707} some
questions have already been asked about the true effectiveness of these
strategies in an evolutionary setting. Here another question is asked: is it
possible to recognise this extortionate behaviour? A mathematical procedure for
suspicion is presented: in the same way that the continued actions of an
extortionate individual might raise suspicion.

This work makes use of the Axelrod Python library~\cite{Knight2018, Knight2016}
with a large number of Prisoner Dilemma strategies available to give an
extensive numerical example of the ideas presented.  The approach is presented
in Section~\ref{sec:delta-zd-strategies}.  All of the code and data discussed
in Section~\ref{sec:numerical-experiments} is open sourced, archived and
written according to best scientific principles~\cite{Wilson2014}. The data
archive can be found at~\cite{vincent_knight_2018_1297075}.

\section{Recognising Extortion}\label{sec:delta-zd-strategies}

In~\cite{Press2012}, given a match between 2 memory-one strategies, the concept
of Zero Determinant (ZD) strategies is introduced. The main result of that paper
shows that given two memory one players \(p, q\in\mathbb{R}^4\) a linear
relationship between the players' scores could be forced by one of the players.

Using the notation of~\cite{Press2012}, assuming the utilities for player \(p\)
are given by \(S_x=(R, S, T, P)\) and for player \(q\) by \(S_y=(R, T, S, P)\)
and that the stationary scores of each player is given by \(S_X\) and \(S_Y\)
respectively. The main result of~\cite{Press2012} is that if

\begin{equation}\label{eqn:linear_relationship_for_p}
    \tilde p=\alpha S_x + \beta S_y + \gamma
\end{equation}

or

\begin{equation}\label{eqn:linear_relationship_for_q}
    \tilde q=\alpha S_x + \beta S_y + \gamma
\end{equation}

where \(\tilde p = (1 - p_1, 1 - p_2, p_3, p_4)\) and
\(\tilde q = (1 - q_1, 1 - q_2, q_3, q_4)\) then:

\begin{equation}
    \alpha S_X + \beta S_Y + \gamma = 0
\end{equation}

In~\cite{Press2012} a particular type of ZD strategy is defined: extortionate
strategies. If:

\begin{equation}\label{eqn:constraint_for_extortion}
    \gamma = - P(\alpha + \beta)
\end{equation}

then the player can ensure they get a score \(\chi\) times
larger than the opponent. This extortion coefficient is given by:

\begin{equation}\label{eqn:definition_of_chi}
    \chi=\frac{-\beta}{\alpha}
\end{equation}

Thus, if (\ref{eqn:constraint_for_extortion}) holds and \(\chi >1\) a player is
said to extort their opponent.
Here, the reverse problem is considered: given a
\(p\in\mathbb{R}^4\) how does one identify \(\alpha, \beta\) if they
exist and is the strategy in fact acting in an extortionate way?

These conditions correspond to:

\begin{align}
    \tilde p_1 & = \alpha R + \beta R - P (\alpha + \beta)
            \label{eqn:condition_for_tilde_p1}\\
    \tilde p_2 & = \alpha S + \beta T - P (\alpha + \beta)
            \label{eqn:condition_for_tilde_p2}\\
    \tilde p_3 & = \alpha T + \beta S - P (\alpha + \beta)
            \label{eqn:condition_for_tilde_p3}\\
    \tilde p_4 & = \alpha P + \beta P - P (\alpha + \beta)
            \label{eqn:condition_for_tilde_p4}
\end{align}

Equation (\ref{eqn:condition_for_tilde_p4}) ensures that \(p_4=\tilde p_4=0\).
Equations (\ref{eqn:condition_for_tilde_p1}-\ref{eqn:condition_for_tilde_p3})
can be used to eliminate \(\alpha, \beta\), giving:

\begin{equation}\label{eqn:planar_definition_of_extortion}
    \tilde p_1 = \frac{(R - P)(\tilde p_2 + \tilde p_3)}{S + T - 2P}
\end{equation}

with:

\begin{equation}\label{eqn:definition_of_chi}
    \chi = \frac{\tilde p_2 (P - T) + \tilde p_3 (S - P)}
                {\tilde p_2 (P - S) + \tilde p_3 (T - P)}
\end{equation}

Given a strategy \(p\in\mathbb{R}^{4\times 1}\) equations
(\ref{eqn:condition_for_tilde_p4}), (\ref{eqn:planar_definition_of_extortion}-\ref{eqn:definition_of_chi}) can be used to check if
a strategy is extortionate. The conditions correspond to:

\begin{align}
    p_1 & = \frac{(R-P)(p_2 + p_3) - R + T + S - P}{S + T - 2P}
     \label{eqn:condition_for_p1}\\
    p_4 & = 0 \label{eqn:condition_for_p4}\\
    1 & > p_2 + p_3\label{eqn:condition_for_chi}
\end{align}

The algebraic steps necessary to prove these results are available in the
supporting materials.

All extortionate strategies reside on a triangular (\ref{eqn:condition_for_chi})
plane (\ref{eqn:condition_for_p1}) in 3 dimensions (\ref{eqn:condition_for_p4}).
Using this formulation it can be seen that a necessary (but not sufficient)
condition for an extortionate strategy is that it cooperates on average less
than 50\% of the time when in a state of disagreement with the opponent.

As an example, consider the known extortionate strategy \(p=(8 / 9, 1 / 2, 1 /
3, 0)\) from~\cite{Stewart2012} which is referred to as \texttt{Extort-2}. In
this case, for the standard values of \((R, T, S, P)\) constraint
(\ref{eqn:condition_for_p1}) corresponds to:

\begin{equation}
    p_1 = \frac{2(p_2 + p_3) + 1}{3}
\end{equation}

It is clear that in this case all constraints hold.

This approach could in fact be used to confirm that a given strategy is acting
in an extortionate manner even if it is not a memory one strategy. However, in
practice, if a closed form for \(p\) is not known, then due to measurement
and/or numerical error this would not work.

This problem can be written in the following linear algebraic form where
\(x=(\alpha, \beta)\)
and \(p^*=(\tilde p_1 - 1, tilde_2 - 1, p_3)\):

\begin{equation}\label{eqn:linear_algebraic_equation_for_p}
    Cx= p^*
\end{equation}

\(C\) corresponds to equations
(\ref{eqn:condition_for_tilde_p1}-\ref{eqn:condition_for_tilde_p3}) and is
given by:

\begin{equation}\label{eqn:definition_of_C}
    C =
    \begin{bmatrix}
        R - P & R- P \\
        S - P & T- P \\
        T - P & S- P \\
    \end{bmatrix}
\end{equation}

Note that in general, equation (\ref{eqn:linear_algebraic_equation_for_p}) will
not necessarily have a solution. From the Rouch\'{e}-Capelli theorem if there is
a solution it is unique as \(\text{rank}(C)=2\) which is the dimension of the
variable \(x\). The best fitting \(x\) is found by minimizing:

\begin{equation}\label{eqn:r_squared}
    \text{SSError} = \|C x- p^*\|_2^2 = \sum_{i=1}^{3}\left((C\bar x)_i-p_i^*\right)^2
\end{equation}

Note that \(\text{SSError}\), which is the square of the Frobenius
norm~\cite{Golub2013}, becomes a measure of how close a strategy is to being an
extortionate strategy. Suspicion
of extortion then corresponds to a threshold on \(\text{SSError}\).

By observing interactions (human or otherwise), their memory one representation
can be inferred and this approach can be used to recognise extortionate
behaviour. The notion of comparing theoretic and actual plays of the IPD is not
novel, see for example~\cite{Rand2013}. Immediately it is noted that if the
environment is noisy~\cite{Wu1995} then no strategy can be considered to be
extortionate as \(p_4>0\).

In the next section, this idea will be illustrated by observing the interactions
that take place in a computer based tournament of the IPD\@.

\section{Numerical experiments}\label{sec:numerical-experiments}

In~\cite{Stewart2012} results from a tournament with
\input{./assets/tex/number_of_stewart_plotkin_strategies/main.tex} strategies,
was presented with specific consideration given to ZD strategies. This
tournament is reproduced here using the Axelrod-Python
project~\cite{Knight2016}. To obtain a good measure of the corresponding
transition rates for each strategy all matches have been run for
\input{assets/tex/number_of_turns/main.tex} turns and every match has been
repeated \input{assets/tex/number_of_repetitions/main.tex} times. All of this
interaction data is available at~\cite{vincent_knight_2018_1297075}. A good
match between the inferred Markov chain and the state distribution of the actual
interactions has been verified. Data for this is presented in the supplementary
materials.

Figure~\ref{fig:SSError_overall_in_stewart_plotkin} shows the \(\text{SSError}\)
values for all the strategies in the tournament, as reported
in~\cite{Stewart2012} the extortionate strategy (which has an expected
\(\text{SSError}\) approximately 0) gains a large number of wins.

\begin{figure}[!htbp]
    \centering
    \includegraphics[width=.8\textwidth]{./assets/img/SSError_overall_in_stewart_plotkin/main.pdf}
    \caption{\(\text{SSError}\) and state probabilities for the strategies
        of~\cite{Stewart2012}, ordered both by number of wins and overall score.
        Note that \(P(DC)\) is not shown as it corresponds to the transpose of
        \(P(CD)\). Cooperator and Defector are omitted as they do not visit all
        the states.}
    \label{fig:SSError_overall_in_stewart_plotkin}
\end{figure}

Here, the work of~\cite{Stewart2012} is extended by investigating a tournament
with \input{assets/tex/number_of_full_strategies/main.tex}
strategies.

The results of this analysis are shown in
Figure~\ref{fig:SSError_and_probabilities_in_full}. The top ranking strategies
by number of wins seem to be extortionate (but not against all strategies) and
it can be seen that a small sub group of strategies achieve mutual defection.
All the top ranking strategies according to score achieve mutual cooperation and
do not extort each other, however they
\textbf{do} exhibit extortionate behaviour towards a number of the lower ranking
strategies.

\begin{figure}[!htbp]
    \centering
    \includegraphics[width=.8\textwidth]{./assets/img/SSError_and_probabilities_in_full/main.pdf}
    \caption{\(\text{SSError}\) for the strategies for the full tournament. Only
    strategy interactions for which \(p_4=0\) and \(\chi>1\) are displayed.}
    \label{fig:SSError_and_probabilities_in_full}
\end{figure}

\section{Conclusion}\label{sec:conclusion}

This work defines an approach to measure whether or not a player is playing a
strategy that corresponds to an extortionate strategy as defined
in~\cite{Press2012}: a mathematical model for suspicion. Indeed, all
extortionate strategies have been
 classified as lying on a triangular plane.
This rigorous classification fails to be robust to small measurement error, thus
a statistical approach is proposed.
This is done through a linear algebraic approach for approximating the solution
of a linear system. Using this, a large number of pairwise interactions is
simulated and in fact very few strategies are found to act extortionately.

The work of~\cite{Press2012}, whilst showing that a clever approach to taking
advantage of another memory one strategy exists: this is incomplete. Whilst the
elegance of this result is very attractive, just as the simplicity of the
victory of Tit For Tat in Axelrod's original tournaments was, it is incomplete.
Extortionate strategies achieve a high number of wins but they do not
achieve a high score which corresponds to the fitness landscape in an
evolutionary sense. From the large number of interactions a payoff matrix \(S\)
can be measured where \(S_{ij}\) denotes the score (using standard values of
\((R, S, T, P) = (3, 0, 5, 1)\)) of the \(i\)th strategy
against the \(j\)th strategy. Using this, the replicator equation
describes the evolution of the system based on a population density fitness
function:

\begin{equation}\label{eqn:replicator_dynamics}
    \frac{dx}{dt} = x(S-x^TS x)
\end{equation}

Equation (\ref{eqn:replicator_dynamics}) is solved numerically through an
integration technique described in~\cite{Petzold1983} and
Figure~\ref{fig:replicator_dynamics} shows the evolution of the distribution of
the system: the various strategies are ranked by scores. It is clear to see that
only the high ranking strategies survive the evolutionary process (in fact,
only \input{./assets/img/replicator_dynamics/main.tex}
have a final distribution greater than \(10 ^ {-2}\)). This confirms the
findings of~\cite{Moran1707} in which sophisticated strategies resist
evolutionary invasion of shorter memory strategies. Recalling
Figure~\ref{fig:SSError_and_probabilities_in_full} this demonstrates that:

\begin{itemize}
    \item Cooperation emerges through the evolutionary process: the high scoring
        strategies do not exhibit extortionate behaviour towards each other.
    \item Extortionate strategies do not survive the evolutionary process.
\end{itemize}

\begin{figure}[!htbp]
    \centering
    \includegraphics[width=.8\textwidth]{./assets/img/replicator_dynamics/main.pdf}
    \caption{Numerical simulation of the replicator equation
    (\ref{eqn:replicator_dynamics}): strategies are ordered by score, only the strategies with a high score survive the evolutionary process.}
    \label{fig:replicator_dynamics}
\end{figure}

This work can be used to classify plays of the IPD\@: data can be collected from
actual interactions (in lab or in the field). Furthermore, this allows for a
classification method similar to the notion of fingerprinting presented
in~\cite{Ashlock2008}. Trained strategies can potentially be classified as
extortionate or not or it could be possible to even constrain the reinforcement
learning approaches that are becoming prevalent in the literature.
Alternatively, this mathematical approach for recognising extortion could be
used in sophisticated strategies to defend against invasion. Arguably, some of
the strategies considered here exhibit this behaviour, indeed as described
in~\cite{Harper2017}, the top ranking strategies in the full tournament are
obtained using evolutionary reinforcement learning techniques, thus, suspicion
of extortionate behaviour could in fact be an evolutionary trait.

\section*{Acknowledgements}

The following open source software libraries were used in this research:

\begin{itemize}
    \item The Axelrod ~\cite{Knight2016, Knight2018} library (IPD strategies and
        tournaments).
    \item The sympy library~\cite{Meurer2017} (verification of all symbolic
        calculations).
    \item The matplotlib~\cite{Droettboom2018} library (visualisation).
    \item The pandas~\cite{Structures2010}, dask~\cite{Dask2016} and
        NumPy~\cite{Oliphant2015} libraries (data manipulation).
    \item The SciPy~\cite{Jones2001} library (numerical integration of the
        replicator equation).
\end{itemize}

This work was performed using the computational facilities of the Advanced
Research Computing @ Cardiff (ARCCA) Division, Cardiff University.

\printbibliography

\newpage
\section*{Supplementary materials}

\includepdf{assets/pdf/proof_of_form_of_extortionate_strategies/main.pdf}

\newpage

Using the pair wise interactions the transition rates \(p,
q\) can be measured and the steady state probabilities inferred and compared to
the actual probabilities of each state.
This is done numerically by computing the singular eigenvector of the
matrix \(A\) \cite{Stewart2009}:

\[
    A =
    \begin{bmatrix}
        p_1 q_1 & p_1 (1 - q_1) & (1 - p_1) q_1 & (1 -p_1) (1 - q_1) \\
        p_2 q_2 & p_2 (1 - q_2) & (1 - p_2) q_2 & (1 -p_2) (1 - q_2) \\
        p_3 q_3 & p_3 (1 - q_3) & (1 - p_3) q_3 & (1 -p_3) (1 - q_3) \\
        p_4 q_4 & p_4 (1 - q_4) & (1 - p_4) q_4 & (1 -p_4) (1 - q_4) \\
    \end{bmatrix}
\]

Figure~\ref{fig:computed_probabilities_vs_theoretic_probabilities} shows a
regression line fitted to every pairwise interaction with a reported
\(\text{SSError}\) value (pairwise interactions with missing states were
omitted). This serves to validate the approach: a part from some edge cases the
relationship is consistent.

\begin{figure}[!htbp]
    \centering
    \includegraphics[width=.8\textwidth]{./assets/img/computed_probabilities_vs_theoretic_probabilities/main.pdf}
    \caption{The
        relationship between the steady state probabilities inferred from the
        measured transitions and the actual steady state probabilities. A linear
        regression line is included validating the approach.}
    \label{fig:computed_probabilities_vs_theoretic_probabilities}
\end{figure}


\end{document}
 strategies,
was presented with specific consideration given to ZD strategies. This
tournament is reproduced here using the Axelrod-Python
project~\cite{Knight2016}. To obtain a good measure of the corresponding
transition rates for each strategy all matches have been run for
\documentclass[a4paper]{article}

\usepackage{amsmath}
\usepackage{amssymb}
\usepackage[margin=1.5cm,
            includefoot,
            footskip=30pt]{geometry}
\usepackage{layout}
\usepackage{graphicx}
\usepackage{subcaption}

\usepackage{biblatex}
\usepackage{pdfpages}

\bibliography{main.bib}

\title{Suspicion: Recognising and evaluating the effectiveness
       of extortion in the Iterated Prisoner's Dilemma}
\author{Vincent A. Knight \and Nikoleta E. Glynatsi}
\date{\today}



\begin{document}

\maketitle

\begin{abstract}
    The Iterated Prisoner's Dilemma is a model for rational and evolutionary
    interactive behaviour. It has applications both in the study of human social
    behaviour as well as in biology.
    It is used to understand when and how a rational individual might
    accept an immediate cost to their own utility for the direct benefit of
    another.

    Much attention has been given to a class of strategies called
    Zero Determinant strategies. It has been theoretically shown that these
    strategies can ``extort'' any player.

    In this work, an approach to identify if observed strategies are playing in
    an extortionate way is described. Furthermore, experimental analysis of
    a large tournament with \input{assets/tex/number_of_full_strategies/main.tex}
    strategies is considered. In this setting
    the most highly performing strategies do not play in an extortionate way
    against each other but do against lower performing strategies.
    This suggests that whilst the theory of Zero Determinant strategies
    indicates that memory is not of fundamental importance to the evolution of
    cooperative behaviour, this is incomplete.
\end{abstract}

\section{Introduction}\label{sec:introduction}

Agent based game theoretic models have become a stalwart of the underpinning
mathematics of interactive behaviours. One of the major pieces of work
in this area is the pair of original computer tournaments run by Robert
Axelrod~\cite{Axelrod1980, Axelrod1980a}. These tournaments pitted submitted
computer strategies against each other in plays of the Iterated Prisoner's
Dilemma. A common game where agents can choose to pay a slight cost to their
immediate utility in the hope of building a reputation. This has been used in
economic and evolutionary game theory to understand the evolution of cooperative
behaviour.

Recently, a class of strategies was described in~\cite{Press2012} that can
provably extort any given opponent. In~\cite{Hilbe2013, Moran1707} some
questions have already been asked about the true effectiveness of these
strategies in an evolutionary setting. Here another question is asked: is it
possible to recognise this extortionate behaviour? A mathematical procedure for
suspicion is presented: in the same way that the continued actions of an
extortionate individual might raise suspicion.

This work makes use of the Axelrod Python library~\cite{Knight2018, Knight2016}
with a large number of Prisoner Dilemma strategies available to give an
extensive numerical example of the ideas presented.  The approach is presented
in Section~\ref{sec:delta-zd-strategies}.  All of the code and data discussed
in Section~\ref{sec:numerical-experiments} is open sourced, archived and
written according to best scientific principles~\cite{Wilson2014}. The data
archive can be found at~\cite{vincent_knight_2018_1297075}.

\section{Recognising Extortion}\label{sec:delta-zd-strategies}

In~\cite{Press2012}, given a match between 2 memory-one strategies, the concept
of Zero Determinant (ZD) strategies is introduced. The main result of that paper
shows that given two memory one players \(p, q\in\mathbb{R}^4\) a linear
relationship between the players' scores could be forced by one of the players.

Using the notation of~\cite{Press2012}, assuming the utilities for player \(p\)
are given by \(S_x=(R, S, T, P)\) and for player \(q\) by \(S_y=(R, T, S, P)\)
and that the stationary scores of each player is given by \(S_X\) and \(S_Y\)
respectively. The main result of~\cite{Press2012} is that if

\begin{equation}\label{eqn:linear_relationship_for_p}
    \tilde p=\alpha S_x + \beta S_y + \gamma
\end{equation}

or

\begin{equation}\label{eqn:linear_relationship_for_q}
    \tilde q=\alpha S_x + \beta S_y + \gamma
\end{equation}

where \(\tilde p = (1 - p_1, 1 - p_2, p_3, p_4)\) and
\(\tilde q = (1 - q_1, 1 - q_2, q_3, q_4)\) then:

\begin{equation}
    \alpha S_X + \beta S_Y + \gamma = 0
\end{equation}

In~\cite{Press2012} a particular type of ZD strategy is defined: extortionate
strategies. If:

\begin{equation}\label{eqn:constraint_for_extortion}
    \gamma = - P(\alpha + \beta)
\end{equation}

then the player can ensure they get a score \(\chi\) times
larger than the opponent. This extortion coefficient is given by:

\begin{equation}\label{eqn:definition_of_chi}
    \chi=\frac{-\beta}{\alpha}
\end{equation}

Thus, if (\ref{eqn:constraint_for_extortion}) holds and \(\chi >1\) a player is
said to extort their opponent.
Here, the reverse problem is considered: given a
\(p\in\mathbb{R}^4\) how does one identify \(\alpha, \beta\) if they
exist and is the strategy in fact acting in an extortionate way?

These conditions correspond to:

\begin{align}
    \tilde p_1 & = \alpha R + \beta R - P (\alpha + \beta)
            \label{eqn:condition_for_tilde_p1}\\
    \tilde p_2 & = \alpha S + \beta T - P (\alpha + \beta)
            \label{eqn:condition_for_tilde_p2}\\
    \tilde p_3 & = \alpha T + \beta S - P (\alpha + \beta)
            \label{eqn:condition_for_tilde_p3}\\
    \tilde p_4 & = \alpha P + \beta P - P (\alpha + \beta)
            \label{eqn:condition_for_tilde_p4}
\end{align}

Equation (\ref{eqn:condition_for_tilde_p4}) ensures that \(p_4=\tilde p_4=0\).
Equations (\ref{eqn:condition_for_tilde_p1}-\ref{eqn:condition_for_tilde_p3})
can be used to eliminate \(\alpha, \beta\), giving:

\begin{equation}\label{eqn:planar_definition_of_extortion}
    \tilde p_1 = \frac{(R - P)(\tilde p_2 + \tilde p_3)}{S + T - 2P}
\end{equation}

with:

\begin{equation}\label{eqn:definition_of_chi}
    \chi = \frac{\tilde p_2 (P - T) + \tilde p_3 (S - P)}
                {\tilde p_2 (P - S) + \tilde p_3 (T - P)}
\end{equation}

Given a strategy \(p\in\mathbb{R}^{4\times 1}\) equations
(\ref{eqn:condition_for_tilde_p4}), (\ref{eqn:planar_definition_of_extortion}-\ref{eqn:definition_of_chi}) can be used to check if
a strategy is extortionate. The conditions correspond to:

\begin{align}
    p_1 & = \frac{(R-P)(p_2 + p_3) - R + T + S - P}{S + T - 2P}
     \label{eqn:condition_for_p1}\\
    p_4 & = 0 \label{eqn:condition_for_p4}\\
    1 & > p_2 + p_3\label{eqn:condition_for_chi}
\end{align}

The algebraic steps necessary to prove these results are available in the
supporting materials.

All extortionate strategies reside on a triangular (\ref{eqn:condition_for_chi})
plane (\ref{eqn:condition_for_p1}) in 3 dimensions (\ref{eqn:condition_for_p4}).
Using this formulation it can be seen that a necessary (but not sufficient)
condition for an extortionate strategy is that it cooperates on average less
than 50\% of the time when in a state of disagreement with the opponent.

As an example, consider the known extortionate strategy \(p=(8 / 9, 1 / 2, 1 /
3, 0)\) from~\cite{Stewart2012} which is referred to as \texttt{Extort-2}. In
this case, for the standard values of \((R, T, S, P)\) constraint
(\ref{eqn:condition_for_p1}) corresponds to:

\begin{equation}
    p_1 = \frac{2(p_2 + p_3) + 1}{3}
\end{equation}

It is clear that in this case all constraints hold.

This approach could in fact be used to confirm that a given strategy is acting
in an extortionate manner even if it is not a memory one strategy. However, in
practice, if a closed form for \(p\) is not known, then due to measurement
and/or numerical error this would not work.

This problem can be written in the following linear algebraic form where
\(x=(\alpha, \beta)\)
and \(p^*=(\tilde p_1 - 1, tilde_2 - 1, p_3)\):

\begin{equation}\label{eqn:linear_algebraic_equation_for_p}
    Cx= p^*
\end{equation}

\(C\) corresponds to equations
(\ref{eqn:condition_for_tilde_p1}-\ref{eqn:condition_for_tilde_p3}) and is
given by:

\begin{equation}\label{eqn:definition_of_C}
    C =
    \begin{bmatrix}
        R - P & R- P \\
        S - P & T- P \\
        T - P & S- P \\
    \end{bmatrix}
\end{equation}

Note that in general, equation (\ref{eqn:linear_algebraic_equation_for_p}) will
not necessarily have a solution. From the Rouch\'{e}-Capelli theorem if there is
a solution it is unique as \(\text{rank}(C)=2\) which is the dimension of the
variable \(x\). The best fitting \(x\) is found by minimizing:

\begin{equation}\label{eqn:r_squared}
    \text{SSError} = \|C x- p^*\|_2^2 = \sum_{i=1}^{3}\left((C\bar x)_i-p_i^*\right)^2
\end{equation}

Note that \(\text{SSError}\), which is the square of the Frobenius
norm~\cite{Golub2013}, becomes a measure of how close a strategy is to being an
extortionate strategy. Suspicion
of extortion then corresponds to a threshold on \(\text{SSError}\).

By observing interactions (human or otherwise), their memory one representation
can be inferred and this approach can be used to recognise extortionate
behaviour. The notion of comparing theoretic and actual plays of the IPD is not
novel, see for example~\cite{Rand2013}. Immediately it is noted that if the
environment is noisy~\cite{Wu1995} then no strategy can be considered to be
extortionate as \(p_4>0\).

In the next section, this idea will be illustrated by observing the interactions
that take place in a computer based tournament of the IPD\@.

\section{Numerical experiments}\label{sec:numerical-experiments}

In~\cite{Stewart2012} results from a tournament with
\input{./assets/tex/number_of_stewart_plotkin_strategies/main.tex} strategies,
was presented with specific consideration given to ZD strategies. This
tournament is reproduced here using the Axelrod-Python
project~\cite{Knight2016}. To obtain a good measure of the corresponding
transition rates for each strategy all matches have been run for
\input{assets/tex/number_of_turns/main.tex} turns and every match has been
repeated \input{assets/tex/number_of_repetitions/main.tex} times. All of this
interaction data is available at~\cite{vincent_knight_2018_1297075}. A good
match between the inferred Markov chain and the state distribution of the actual
interactions has been verified. Data for this is presented in the supplementary
materials.

Figure~\ref{fig:SSError_overall_in_stewart_plotkin} shows the \(\text{SSError}\)
values for all the strategies in the tournament, as reported
in~\cite{Stewart2012} the extortionate strategy (which has an expected
\(\text{SSError}\) approximately 0) gains a large number of wins.

\begin{figure}[!htbp]
    \centering
    \includegraphics[width=.8\textwidth]{./assets/img/SSError_overall_in_stewart_plotkin/main.pdf}
    \caption{\(\text{SSError}\) and state probabilities for the strategies
        of~\cite{Stewart2012}, ordered both by number of wins and overall score.
        Note that \(P(DC)\) is not shown as it corresponds to the transpose of
        \(P(CD)\). Cooperator and Defector are omitted as they do not visit all
        the states.}
    \label{fig:SSError_overall_in_stewart_plotkin}
\end{figure}

Here, the work of~\cite{Stewart2012} is extended by investigating a tournament
with \input{assets/tex/number_of_full_strategies/main.tex}
strategies.

The results of this analysis are shown in
Figure~\ref{fig:SSError_and_probabilities_in_full}. The top ranking strategies
by number of wins seem to be extortionate (but not against all strategies) and
it can be seen that a small sub group of strategies achieve mutual defection.
All the top ranking strategies according to score achieve mutual cooperation and
do not extort each other, however they
\textbf{do} exhibit extortionate behaviour towards a number of the lower ranking
strategies.

\begin{figure}[!htbp]
    \centering
    \includegraphics[width=.8\textwidth]{./assets/img/SSError_and_probabilities_in_full/main.pdf}
    \caption{\(\text{SSError}\) for the strategies for the full tournament. Only
    strategy interactions for which \(p_4=0\) and \(\chi>1\) are displayed.}
    \label{fig:SSError_and_probabilities_in_full}
\end{figure}

\section{Conclusion}\label{sec:conclusion}

This work defines an approach to measure whether or not a player is playing a
strategy that corresponds to an extortionate strategy as defined
in~\cite{Press2012}: a mathematical model for suspicion. Indeed, all
extortionate strategies have been
 classified as lying on a triangular plane.
This rigorous classification fails to be robust to small measurement error, thus
a statistical approach is proposed.
This is done through a linear algebraic approach for approximating the solution
of a linear system. Using this, a large number of pairwise interactions is
simulated and in fact very few strategies are found to act extortionately.

The work of~\cite{Press2012}, whilst showing that a clever approach to taking
advantage of another memory one strategy exists: this is incomplete. Whilst the
elegance of this result is very attractive, just as the simplicity of the
victory of Tit For Tat in Axelrod's original tournaments was, it is incomplete.
Extortionate strategies achieve a high number of wins but they do not
achieve a high score which corresponds to the fitness landscape in an
evolutionary sense. From the large number of interactions a payoff matrix \(S\)
can be measured where \(S_{ij}\) denotes the score (using standard values of
\((R, S, T, P) = (3, 0, 5, 1)\)) of the \(i\)th strategy
against the \(j\)th strategy. Using this, the replicator equation
describes the evolution of the system based on a population density fitness
function:

\begin{equation}\label{eqn:replicator_dynamics}
    \frac{dx}{dt} = x(S-x^TS x)
\end{equation}

Equation (\ref{eqn:replicator_dynamics}) is solved numerically through an
integration technique described in~\cite{Petzold1983} and
Figure~\ref{fig:replicator_dynamics} shows the evolution of the distribution of
the system: the various strategies are ranked by scores. It is clear to see that
only the high ranking strategies survive the evolutionary process (in fact,
only \input{./assets/img/replicator_dynamics/main.tex}
have a final distribution greater than \(10 ^ {-2}\)). This confirms the
findings of~\cite{Moran1707} in which sophisticated strategies resist
evolutionary invasion of shorter memory strategies. Recalling
Figure~\ref{fig:SSError_and_probabilities_in_full} this demonstrates that:

\begin{itemize}
    \item Cooperation emerges through the evolutionary process: the high scoring
        strategies do not exhibit extortionate behaviour towards each other.
    \item Extortionate strategies do not survive the evolutionary process.
\end{itemize}

\begin{figure}[!htbp]
    \centering
    \includegraphics[width=.8\textwidth]{./assets/img/replicator_dynamics/main.pdf}
    \caption{Numerical simulation of the replicator equation
    (\ref{eqn:replicator_dynamics}): strategies are ordered by score, only the strategies with a high score survive the evolutionary process.}
    \label{fig:replicator_dynamics}
\end{figure}

This work can be used to classify plays of the IPD\@: data can be collected from
actual interactions (in lab or in the field). Furthermore, this allows for a
classification method similar to the notion of fingerprinting presented
in~\cite{Ashlock2008}. Trained strategies can potentially be classified as
extortionate or not or it could be possible to even constrain the reinforcement
learning approaches that are becoming prevalent in the literature.
Alternatively, this mathematical approach for recognising extortion could be
used in sophisticated strategies to defend against invasion. Arguably, some of
the strategies considered here exhibit this behaviour, indeed as described
in~\cite{Harper2017}, the top ranking strategies in the full tournament are
obtained using evolutionary reinforcement learning techniques, thus, suspicion
of extortionate behaviour could in fact be an evolutionary trait.

\section*{Acknowledgements}

The following open source software libraries were used in this research:

\begin{itemize}
    \item The Axelrod ~\cite{Knight2016, Knight2018} library (IPD strategies and
        tournaments).
    \item The sympy library~\cite{Meurer2017} (verification of all symbolic
        calculations).
    \item The matplotlib~\cite{Droettboom2018} library (visualisation).
    \item The pandas~\cite{Structures2010}, dask~\cite{Dask2016} and
        NumPy~\cite{Oliphant2015} libraries (data manipulation).
    \item The SciPy~\cite{Jones2001} library (numerical integration of the
        replicator equation).
\end{itemize}

This work was performed using the computational facilities of the Advanced
Research Computing @ Cardiff (ARCCA) Division, Cardiff University.

\printbibliography

\newpage
\section*{Supplementary materials}

\includepdf{assets/pdf/proof_of_form_of_extortionate_strategies/main.pdf}

\newpage

Using the pair wise interactions the transition rates \(p,
q\) can be measured and the steady state probabilities inferred and compared to
the actual probabilities of each state.
This is done numerically by computing the singular eigenvector of the
matrix \(A\) \cite{Stewart2009}:

\[
    A =
    \begin{bmatrix}
        p_1 q_1 & p_1 (1 - q_1) & (1 - p_1) q_1 & (1 -p_1) (1 - q_1) \\
        p_2 q_2 & p_2 (1 - q_2) & (1 - p_2) q_2 & (1 -p_2) (1 - q_2) \\
        p_3 q_3 & p_3 (1 - q_3) & (1 - p_3) q_3 & (1 -p_3) (1 - q_3) \\
        p_4 q_4 & p_4 (1 - q_4) & (1 - p_4) q_4 & (1 -p_4) (1 - q_4) \\
    \end{bmatrix}
\]

Figure~\ref{fig:computed_probabilities_vs_theoretic_probabilities} shows a
regression line fitted to every pairwise interaction with a reported
\(\text{SSError}\) value (pairwise interactions with missing states were
omitted). This serves to validate the approach: a part from some edge cases the
relationship is consistent.

\begin{figure}[!htbp]
    \centering
    \includegraphics[width=.8\textwidth]{./assets/img/computed_probabilities_vs_theoretic_probabilities/main.pdf}
    \caption{The
        relationship between the steady state probabilities inferred from the
        measured transitions and the actual steady state probabilities. A linear
        regression line is included validating the approach.}
    \label{fig:computed_probabilities_vs_theoretic_probabilities}
\end{figure}


\end{document}
 turns and every match has been
repeated \documentclass[a4paper]{article}

\usepackage{amsmath}
\usepackage{amssymb}
\usepackage[margin=1.5cm,
            includefoot,
            footskip=30pt]{geometry}
\usepackage{layout}
\usepackage{graphicx}
\usepackage{subcaption}

\usepackage{biblatex}
\usepackage{pdfpages}

\bibliography{main.bib}

\title{Suspicion: Recognising and evaluating the effectiveness
       of extortion in the Iterated Prisoner's Dilemma}
\author{Vincent A. Knight \and Nikoleta E. Glynatsi}
\date{\today}



\begin{document}

\maketitle

\begin{abstract}
    The Iterated Prisoner's Dilemma is a model for rational and evolutionary
    interactive behaviour. It has applications both in the study of human social
    behaviour as well as in biology.
    It is used to understand when and how a rational individual might
    accept an immediate cost to their own utility for the direct benefit of
    another.

    Much attention has been given to a class of strategies called
    Zero Determinant strategies. It has been theoretically shown that these
    strategies can ``extort'' any player.

    In this work, an approach to identify if observed strategies are playing in
    an extortionate way is described. Furthermore, experimental analysis of
    a large tournament with \input{assets/tex/number_of_full_strategies/main.tex}
    strategies is considered. In this setting
    the most highly performing strategies do not play in an extortionate way
    against each other but do against lower performing strategies.
    This suggests that whilst the theory of Zero Determinant strategies
    indicates that memory is not of fundamental importance to the evolution of
    cooperative behaviour, this is incomplete.
\end{abstract}

\section{Introduction}\label{sec:introduction}

Agent based game theoretic models have become a stalwart of the underpinning
mathematics of interactive behaviours. One of the major pieces of work
in this area is the pair of original computer tournaments run by Robert
Axelrod~\cite{Axelrod1980, Axelrod1980a}. These tournaments pitted submitted
computer strategies against each other in plays of the Iterated Prisoner's
Dilemma. A common game where agents can choose to pay a slight cost to their
immediate utility in the hope of building a reputation. This has been used in
economic and evolutionary game theory to understand the evolution of cooperative
behaviour.

Recently, a class of strategies was described in~\cite{Press2012} that can
provably extort any given opponent. In~\cite{Hilbe2013, Moran1707} some
questions have already been asked about the true effectiveness of these
strategies in an evolutionary setting. Here another question is asked: is it
possible to recognise this extortionate behaviour? A mathematical procedure for
suspicion is presented: in the same way that the continued actions of an
extortionate individual might raise suspicion.

This work makes use of the Axelrod Python library~\cite{Knight2018, Knight2016}
with a large number of Prisoner Dilemma strategies available to give an
extensive numerical example of the ideas presented.  The approach is presented
in Section~\ref{sec:delta-zd-strategies}.  All of the code and data discussed
in Section~\ref{sec:numerical-experiments} is open sourced, archived and
written according to best scientific principles~\cite{Wilson2014}. The data
archive can be found at~\cite{vincent_knight_2018_1297075}.

\section{Recognising Extortion}\label{sec:delta-zd-strategies}

In~\cite{Press2012}, given a match between 2 memory-one strategies, the concept
of Zero Determinant (ZD) strategies is introduced. The main result of that paper
shows that given two memory one players \(p, q\in\mathbb{R}^4\) a linear
relationship between the players' scores could be forced by one of the players.

Using the notation of~\cite{Press2012}, assuming the utilities for player \(p\)
are given by \(S_x=(R, S, T, P)\) and for player \(q\) by \(S_y=(R, T, S, P)\)
and that the stationary scores of each player is given by \(S_X\) and \(S_Y\)
respectively. The main result of~\cite{Press2012} is that if

\begin{equation}\label{eqn:linear_relationship_for_p}
    \tilde p=\alpha S_x + \beta S_y + \gamma
\end{equation}

or

\begin{equation}\label{eqn:linear_relationship_for_q}
    \tilde q=\alpha S_x + \beta S_y + \gamma
\end{equation}

where \(\tilde p = (1 - p_1, 1 - p_2, p_3, p_4)\) and
\(\tilde q = (1 - q_1, 1 - q_2, q_3, q_4)\) then:

\begin{equation}
    \alpha S_X + \beta S_Y + \gamma = 0
\end{equation}

In~\cite{Press2012} a particular type of ZD strategy is defined: extortionate
strategies. If:

\begin{equation}\label{eqn:constraint_for_extortion}
    \gamma = - P(\alpha + \beta)
\end{equation}

then the player can ensure they get a score \(\chi\) times
larger than the opponent. This extortion coefficient is given by:

\begin{equation}\label{eqn:definition_of_chi}
    \chi=\frac{-\beta}{\alpha}
\end{equation}

Thus, if (\ref{eqn:constraint_for_extortion}) holds and \(\chi >1\) a player is
said to extort their opponent.
Here, the reverse problem is considered: given a
\(p\in\mathbb{R}^4\) how does one identify \(\alpha, \beta\) if they
exist and is the strategy in fact acting in an extortionate way?

These conditions correspond to:

\begin{align}
    \tilde p_1 & = \alpha R + \beta R - P (\alpha + \beta)
            \label{eqn:condition_for_tilde_p1}\\
    \tilde p_2 & = \alpha S + \beta T - P (\alpha + \beta)
            \label{eqn:condition_for_tilde_p2}\\
    \tilde p_3 & = \alpha T + \beta S - P (\alpha + \beta)
            \label{eqn:condition_for_tilde_p3}\\
    \tilde p_4 & = \alpha P + \beta P - P (\alpha + \beta)
            \label{eqn:condition_for_tilde_p4}
\end{align}

Equation (\ref{eqn:condition_for_tilde_p4}) ensures that \(p_4=\tilde p_4=0\).
Equations (\ref{eqn:condition_for_tilde_p1}-\ref{eqn:condition_for_tilde_p3})
can be used to eliminate \(\alpha, \beta\), giving:

\begin{equation}\label{eqn:planar_definition_of_extortion}
    \tilde p_1 = \frac{(R - P)(\tilde p_2 + \tilde p_3)}{S + T - 2P}
\end{equation}

with:

\begin{equation}\label{eqn:definition_of_chi}
    \chi = \frac{\tilde p_2 (P - T) + \tilde p_3 (S - P)}
                {\tilde p_2 (P - S) + \tilde p_3 (T - P)}
\end{equation}

Given a strategy \(p\in\mathbb{R}^{4\times 1}\) equations
(\ref{eqn:condition_for_tilde_p4}), (\ref{eqn:planar_definition_of_extortion}-\ref{eqn:definition_of_chi}) can be used to check if
a strategy is extortionate. The conditions correspond to:

\begin{align}
    p_1 & = \frac{(R-P)(p_2 + p_3) - R + T + S - P}{S + T - 2P}
     \label{eqn:condition_for_p1}\\
    p_4 & = 0 \label{eqn:condition_for_p4}\\
    1 & > p_2 + p_3\label{eqn:condition_for_chi}
\end{align}

The algebraic steps necessary to prove these results are available in the
supporting materials.

All extortionate strategies reside on a triangular (\ref{eqn:condition_for_chi})
plane (\ref{eqn:condition_for_p1}) in 3 dimensions (\ref{eqn:condition_for_p4}).
Using this formulation it can be seen that a necessary (but not sufficient)
condition for an extortionate strategy is that it cooperates on average less
than 50\% of the time when in a state of disagreement with the opponent.

As an example, consider the known extortionate strategy \(p=(8 / 9, 1 / 2, 1 /
3, 0)\) from~\cite{Stewart2012} which is referred to as \texttt{Extort-2}. In
this case, for the standard values of \((R, T, S, P)\) constraint
(\ref{eqn:condition_for_p1}) corresponds to:

\begin{equation}
    p_1 = \frac{2(p_2 + p_3) + 1}{3}
\end{equation}

It is clear that in this case all constraints hold.

This approach could in fact be used to confirm that a given strategy is acting
in an extortionate manner even if it is not a memory one strategy. However, in
practice, if a closed form for \(p\) is not known, then due to measurement
and/or numerical error this would not work.

This problem can be written in the following linear algebraic form where
\(x=(\alpha, \beta)\)
and \(p^*=(\tilde p_1 - 1, tilde_2 - 1, p_3)\):

\begin{equation}\label{eqn:linear_algebraic_equation_for_p}
    Cx= p^*
\end{equation}

\(C\) corresponds to equations
(\ref{eqn:condition_for_tilde_p1}-\ref{eqn:condition_for_tilde_p3}) and is
given by:

\begin{equation}\label{eqn:definition_of_C}
    C =
    \begin{bmatrix}
        R - P & R- P \\
        S - P & T- P \\
        T - P & S- P \\
    \end{bmatrix}
\end{equation}

Note that in general, equation (\ref{eqn:linear_algebraic_equation_for_p}) will
not necessarily have a solution. From the Rouch\'{e}-Capelli theorem if there is
a solution it is unique as \(\text{rank}(C)=2\) which is the dimension of the
variable \(x\). The best fitting \(x\) is found by minimizing:

\begin{equation}\label{eqn:r_squared}
    \text{SSError} = \|C x- p^*\|_2^2 = \sum_{i=1}^{3}\left((C\bar x)_i-p_i^*\right)^2
\end{equation}

Note that \(\text{SSError}\), which is the square of the Frobenius
norm~\cite{Golub2013}, becomes a measure of how close a strategy is to being an
extortionate strategy. Suspicion
of extortion then corresponds to a threshold on \(\text{SSError}\).

By observing interactions (human or otherwise), their memory one representation
can be inferred and this approach can be used to recognise extortionate
behaviour. The notion of comparing theoretic and actual plays of the IPD is not
novel, see for example~\cite{Rand2013}. Immediately it is noted that if the
environment is noisy~\cite{Wu1995} then no strategy can be considered to be
extortionate as \(p_4>0\).

In the next section, this idea will be illustrated by observing the interactions
that take place in a computer based tournament of the IPD\@.

\section{Numerical experiments}\label{sec:numerical-experiments}

In~\cite{Stewart2012} results from a tournament with
\input{./assets/tex/number_of_stewart_plotkin_strategies/main.tex} strategies,
was presented with specific consideration given to ZD strategies. This
tournament is reproduced here using the Axelrod-Python
project~\cite{Knight2016}. To obtain a good measure of the corresponding
transition rates for each strategy all matches have been run for
\input{assets/tex/number_of_turns/main.tex} turns and every match has been
repeated \input{assets/tex/number_of_repetitions/main.tex} times. All of this
interaction data is available at~\cite{vincent_knight_2018_1297075}. A good
match between the inferred Markov chain and the state distribution of the actual
interactions has been verified. Data for this is presented in the supplementary
materials.

Figure~\ref{fig:SSError_overall_in_stewart_plotkin} shows the \(\text{SSError}\)
values for all the strategies in the tournament, as reported
in~\cite{Stewart2012} the extortionate strategy (which has an expected
\(\text{SSError}\) approximately 0) gains a large number of wins.

\begin{figure}[!htbp]
    \centering
    \includegraphics[width=.8\textwidth]{./assets/img/SSError_overall_in_stewart_plotkin/main.pdf}
    \caption{\(\text{SSError}\) and state probabilities for the strategies
        of~\cite{Stewart2012}, ordered both by number of wins and overall score.
        Note that \(P(DC)\) is not shown as it corresponds to the transpose of
        \(P(CD)\). Cooperator and Defector are omitted as they do not visit all
        the states.}
    \label{fig:SSError_overall_in_stewart_plotkin}
\end{figure}

Here, the work of~\cite{Stewart2012} is extended by investigating a tournament
with \input{assets/tex/number_of_full_strategies/main.tex}
strategies.

The results of this analysis are shown in
Figure~\ref{fig:SSError_and_probabilities_in_full}. The top ranking strategies
by number of wins seem to be extortionate (but not against all strategies) and
it can be seen that a small sub group of strategies achieve mutual defection.
All the top ranking strategies according to score achieve mutual cooperation and
do not extort each other, however they
\textbf{do} exhibit extortionate behaviour towards a number of the lower ranking
strategies.

\begin{figure}[!htbp]
    \centering
    \includegraphics[width=.8\textwidth]{./assets/img/SSError_and_probabilities_in_full/main.pdf}
    \caption{\(\text{SSError}\) for the strategies for the full tournament. Only
    strategy interactions for which \(p_4=0\) and \(\chi>1\) are displayed.}
    \label{fig:SSError_and_probabilities_in_full}
\end{figure}

\section{Conclusion}\label{sec:conclusion}

This work defines an approach to measure whether or not a player is playing a
strategy that corresponds to an extortionate strategy as defined
in~\cite{Press2012}: a mathematical model for suspicion. Indeed, all
extortionate strategies have been
 classified as lying on a triangular plane.
This rigorous classification fails to be robust to small measurement error, thus
a statistical approach is proposed.
This is done through a linear algebraic approach for approximating the solution
of a linear system. Using this, a large number of pairwise interactions is
simulated and in fact very few strategies are found to act extortionately.

The work of~\cite{Press2012}, whilst showing that a clever approach to taking
advantage of another memory one strategy exists: this is incomplete. Whilst the
elegance of this result is very attractive, just as the simplicity of the
victory of Tit For Tat in Axelrod's original tournaments was, it is incomplete.
Extortionate strategies achieve a high number of wins but they do not
achieve a high score which corresponds to the fitness landscape in an
evolutionary sense. From the large number of interactions a payoff matrix \(S\)
can be measured where \(S_{ij}\) denotes the score (using standard values of
\((R, S, T, P) = (3, 0, 5, 1)\)) of the \(i\)th strategy
against the \(j\)th strategy. Using this, the replicator equation
describes the evolution of the system based on a population density fitness
function:

\begin{equation}\label{eqn:replicator_dynamics}
    \frac{dx}{dt} = x(S-x^TS x)
\end{equation}

Equation (\ref{eqn:replicator_dynamics}) is solved numerically through an
integration technique described in~\cite{Petzold1983} and
Figure~\ref{fig:replicator_dynamics} shows the evolution of the distribution of
the system: the various strategies are ranked by scores. It is clear to see that
only the high ranking strategies survive the evolutionary process (in fact,
only \input{./assets/img/replicator_dynamics/main.tex}
have a final distribution greater than \(10 ^ {-2}\)). This confirms the
findings of~\cite{Moran1707} in which sophisticated strategies resist
evolutionary invasion of shorter memory strategies. Recalling
Figure~\ref{fig:SSError_and_probabilities_in_full} this demonstrates that:

\begin{itemize}
    \item Cooperation emerges through the evolutionary process: the high scoring
        strategies do not exhibit extortionate behaviour towards each other.
    \item Extortionate strategies do not survive the evolutionary process.
\end{itemize}

\begin{figure}[!htbp]
    \centering
    \includegraphics[width=.8\textwidth]{./assets/img/replicator_dynamics/main.pdf}
    \caption{Numerical simulation of the replicator equation
    (\ref{eqn:replicator_dynamics}): strategies are ordered by score, only the strategies with a high score survive the evolutionary process.}
    \label{fig:replicator_dynamics}
\end{figure}

This work can be used to classify plays of the IPD\@: data can be collected from
actual interactions (in lab or in the field). Furthermore, this allows for a
classification method similar to the notion of fingerprinting presented
in~\cite{Ashlock2008}. Trained strategies can potentially be classified as
extortionate or not or it could be possible to even constrain the reinforcement
learning approaches that are becoming prevalent in the literature.
Alternatively, this mathematical approach for recognising extortion could be
used in sophisticated strategies to defend against invasion. Arguably, some of
the strategies considered here exhibit this behaviour, indeed as described
in~\cite{Harper2017}, the top ranking strategies in the full tournament are
obtained using evolutionary reinforcement learning techniques, thus, suspicion
of extortionate behaviour could in fact be an evolutionary trait.

\section*{Acknowledgements}

The following open source software libraries were used in this research:

\begin{itemize}
    \item The Axelrod ~\cite{Knight2016, Knight2018} library (IPD strategies and
        tournaments).
    \item The sympy library~\cite{Meurer2017} (verification of all symbolic
        calculations).
    \item The matplotlib~\cite{Droettboom2018} library (visualisation).
    \item The pandas~\cite{Structures2010}, dask~\cite{Dask2016} and
        NumPy~\cite{Oliphant2015} libraries (data manipulation).
    \item The SciPy~\cite{Jones2001} library (numerical integration of the
        replicator equation).
\end{itemize}

This work was performed using the computational facilities of the Advanced
Research Computing @ Cardiff (ARCCA) Division, Cardiff University.

\printbibliography

\newpage
\section*{Supplementary materials}

\includepdf{assets/pdf/proof_of_form_of_extortionate_strategies/main.pdf}

\newpage

Using the pair wise interactions the transition rates \(p,
q\) can be measured and the steady state probabilities inferred and compared to
the actual probabilities of each state.
This is done numerically by computing the singular eigenvector of the
matrix \(A\) \cite{Stewart2009}:

\[
    A =
    \begin{bmatrix}
        p_1 q_1 & p_1 (1 - q_1) & (1 - p_1) q_1 & (1 -p_1) (1 - q_1) \\
        p_2 q_2 & p_2 (1 - q_2) & (1 - p_2) q_2 & (1 -p_2) (1 - q_2) \\
        p_3 q_3 & p_3 (1 - q_3) & (1 - p_3) q_3 & (1 -p_3) (1 - q_3) \\
        p_4 q_4 & p_4 (1 - q_4) & (1 - p_4) q_4 & (1 -p_4) (1 - q_4) \\
    \end{bmatrix}
\]

Figure~\ref{fig:computed_probabilities_vs_theoretic_probabilities} shows a
regression line fitted to every pairwise interaction with a reported
\(\text{SSError}\) value (pairwise interactions with missing states were
omitted). This serves to validate the approach: a part from some edge cases the
relationship is consistent.

\begin{figure}[!htbp]
    \centering
    \includegraphics[width=.8\textwidth]{./assets/img/computed_probabilities_vs_theoretic_probabilities/main.pdf}
    \caption{The
        relationship between the steady state probabilities inferred from the
        measured transitions and the actual steady state probabilities. A linear
        regression line is included validating the approach.}
    \label{fig:computed_probabilities_vs_theoretic_probabilities}
\end{figure}


\end{document}
 times. All of this
interaction data is available at~\cite{vincent_knight_2018_1297075}. A good
match between the inferred Markov chain and the state distribution of the actual
interactions has been verified. Data for this is presented in the supplementary
materials.

Figure~\ref{fig:SSError_overall_in_stewart_plotkin} shows the \(\text{SSError}\)
values for all the strategies in the tournament, as reported
in~\cite{Stewart2012} the extortionate strategy (which has an expected
\(\text{SSError}\) approximately 0) gains a large number of wins.

\begin{figure}[!htbp]
    \centering
    \includegraphics[width=.8\textwidth]{./assets/img/SSError_overall_in_stewart_plotkin/main.pdf}
    \caption{\(\text{SSError}\) and state probabilities for the strategies
        of~\cite{Stewart2012}, ordered both by number of wins and overall score.
        Note that \(P(DC)\) is not shown as it corresponds to the transpose of
        \(P(CD)\). Cooperator and Defector are omitted as they do not visit all
        the states.}
    \label{fig:SSError_overall_in_stewart_plotkin}
\end{figure}

Here, the work of~\cite{Stewart2012} is extended by investigating a tournament
with \documentclass[a4paper]{article}

\usepackage{amsmath}
\usepackage{amssymb}
\usepackage[margin=1.5cm,
            includefoot,
            footskip=30pt]{geometry}
\usepackage{layout}
\usepackage{graphicx}
\usepackage{subcaption}

\usepackage{biblatex}
\usepackage{pdfpages}

\bibliography{main.bib}

\title{Suspicion: Recognising and evaluating the effectiveness
       of extortion in the Iterated Prisoner's Dilemma}
\author{Vincent A. Knight \and Nikoleta E. Glynatsi}
\date{\today}



\begin{document}

\maketitle

\begin{abstract}
    The Iterated Prisoner's Dilemma is a model for rational and evolutionary
    interactive behaviour. It has applications both in the study of human social
    behaviour as well as in biology.
    It is used to understand when and how a rational individual might
    accept an immediate cost to their own utility for the direct benefit of
    another.

    Much attention has been given to a class of strategies called
    Zero Determinant strategies. It has been theoretically shown that these
    strategies can ``extort'' any player.

    In this work, an approach to identify if observed strategies are playing in
    an extortionate way is described. Furthermore, experimental analysis of
    a large tournament with \input{assets/tex/number_of_full_strategies/main.tex}
    strategies is considered. In this setting
    the most highly performing strategies do not play in an extortionate way
    against each other but do against lower performing strategies.
    This suggests that whilst the theory of Zero Determinant strategies
    indicates that memory is not of fundamental importance to the evolution of
    cooperative behaviour, this is incomplete.
\end{abstract}

\section{Introduction}\label{sec:introduction}

Agent based game theoretic models have become a stalwart of the underpinning
mathematics of interactive behaviours. One of the major pieces of work
in this area is the pair of original computer tournaments run by Robert
Axelrod~\cite{Axelrod1980, Axelrod1980a}. These tournaments pitted submitted
computer strategies against each other in plays of the Iterated Prisoner's
Dilemma. A common game where agents can choose to pay a slight cost to their
immediate utility in the hope of building a reputation. This has been used in
economic and evolutionary game theory to understand the evolution of cooperative
behaviour.

Recently, a class of strategies was described in~\cite{Press2012} that can
provably extort any given opponent. In~\cite{Hilbe2013, Moran1707} some
questions have already been asked about the true effectiveness of these
strategies in an evolutionary setting. Here another question is asked: is it
possible to recognise this extortionate behaviour? A mathematical procedure for
suspicion is presented: in the same way that the continued actions of an
extortionate individual might raise suspicion.

This work makes use of the Axelrod Python library~\cite{Knight2018, Knight2016}
with a large number of Prisoner Dilemma strategies available to give an
extensive numerical example of the ideas presented.  The approach is presented
in Section~\ref{sec:delta-zd-strategies}.  All of the code and data discussed
in Section~\ref{sec:numerical-experiments} is open sourced, archived and
written according to best scientific principles~\cite{Wilson2014}. The data
archive can be found at~\cite{vincent_knight_2018_1297075}.

\section{Recognising Extortion}\label{sec:delta-zd-strategies}

In~\cite{Press2012}, given a match between 2 memory-one strategies, the concept
of Zero Determinant (ZD) strategies is introduced. The main result of that paper
shows that given two memory one players \(p, q\in\mathbb{R}^4\) a linear
relationship between the players' scores could be forced by one of the players.

Using the notation of~\cite{Press2012}, assuming the utilities for player \(p\)
are given by \(S_x=(R, S, T, P)\) and for player \(q\) by \(S_y=(R, T, S, P)\)
and that the stationary scores of each player is given by \(S_X\) and \(S_Y\)
respectively. The main result of~\cite{Press2012} is that if

\begin{equation}\label{eqn:linear_relationship_for_p}
    \tilde p=\alpha S_x + \beta S_y + \gamma
\end{equation}

or

\begin{equation}\label{eqn:linear_relationship_for_q}
    \tilde q=\alpha S_x + \beta S_y + \gamma
\end{equation}

where \(\tilde p = (1 - p_1, 1 - p_2, p_3, p_4)\) and
\(\tilde q = (1 - q_1, 1 - q_2, q_3, q_4)\) then:

\begin{equation}
    \alpha S_X + \beta S_Y + \gamma = 0
\end{equation}

In~\cite{Press2012} a particular type of ZD strategy is defined: extortionate
strategies. If:

\begin{equation}\label{eqn:constraint_for_extortion}
    \gamma = - P(\alpha + \beta)
\end{equation}

then the player can ensure they get a score \(\chi\) times
larger than the opponent. This extortion coefficient is given by:

\begin{equation}\label{eqn:definition_of_chi}
    \chi=\frac{-\beta}{\alpha}
\end{equation}

Thus, if (\ref{eqn:constraint_for_extortion}) holds and \(\chi >1\) a player is
said to extort their opponent.
Here, the reverse problem is considered: given a
\(p\in\mathbb{R}^4\) how does one identify \(\alpha, \beta\) if they
exist and is the strategy in fact acting in an extortionate way?

These conditions correspond to:

\begin{align}
    \tilde p_1 & = \alpha R + \beta R - P (\alpha + \beta)
            \label{eqn:condition_for_tilde_p1}\\
    \tilde p_2 & = \alpha S + \beta T - P (\alpha + \beta)
            \label{eqn:condition_for_tilde_p2}\\
    \tilde p_3 & = \alpha T + \beta S - P (\alpha + \beta)
            \label{eqn:condition_for_tilde_p3}\\
    \tilde p_4 & = \alpha P + \beta P - P (\alpha + \beta)
            \label{eqn:condition_for_tilde_p4}
\end{align}

Equation (\ref{eqn:condition_for_tilde_p4}) ensures that \(p_4=\tilde p_4=0\).
Equations (\ref{eqn:condition_for_tilde_p1}-\ref{eqn:condition_for_tilde_p3})
can be used to eliminate \(\alpha, \beta\), giving:

\begin{equation}\label{eqn:planar_definition_of_extortion}
    \tilde p_1 = \frac{(R - P)(\tilde p_2 + \tilde p_3)}{S + T - 2P}
\end{equation}

with:

\begin{equation}\label{eqn:definition_of_chi}
    \chi = \frac{\tilde p_2 (P - T) + \tilde p_3 (S - P)}
                {\tilde p_2 (P - S) + \tilde p_3 (T - P)}
\end{equation}

Given a strategy \(p\in\mathbb{R}^{4\times 1}\) equations
(\ref{eqn:condition_for_tilde_p4}), (\ref{eqn:planar_definition_of_extortion}-\ref{eqn:definition_of_chi}) can be used to check if
a strategy is extortionate. The conditions correspond to:

\begin{align}
    p_1 & = \frac{(R-P)(p_2 + p_3) - R + T + S - P}{S + T - 2P}
     \label{eqn:condition_for_p1}\\
    p_4 & = 0 \label{eqn:condition_for_p4}\\
    1 & > p_2 + p_3\label{eqn:condition_for_chi}
\end{align}

The algebraic steps necessary to prove these results are available in the
supporting materials.

All extortionate strategies reside on a triangular (\ref{eqn:condition_for_chi})
plane (\ref{eqn:condition_for_p1}) in 3 dimensions (\ref{eqn:condition_for_p4}).
Using this formulation it can be seen that a necessary (but not sufficient)
condition for an extortionate strategy is that it cooperates on average less
than 50\% of the time when in a state of disagreement with the opponent.

As an example, consider the known extortionate strategy \(p=(8 / 9, 1 / 2, 1 /
3, 0)\) from~\cite{Stewart2012} which is referred to as \texttt{Extort-2}. In
this case, for the standard values of \((R, T, S, P)\) constraint
(\ref{eqn:condition_for_p1}) corresponds to:

\begin{equation}
    p_1 = \frac{2(p_2 + p_3) + 1}{3}
\end{equation}

It is clear that in this case all constraints hold.

This approach could in fact be used to confirm that a given strategy is acting
in an extortionate manner even if it is not a memory one strategy. However, in
practice, if a closed form for \(p\) is not known, then due to measurement
and/or numerical error this would not work.

This problem can be written in the following linear algebraic form where
\(x=(\alpha, \beta)\)
and \(p^*=(\tilde p_1 - 1, tilde_2 - 1, p_3)\):

\begin{equation}\label{eqn:linear_algebraic_equation_for_p}
    Cx= p^*
\end{equation}

\(C\) corresponds to equations
(\ref{eqn:condition_for_tilde_p1}-\ref{eqn:condition_for_tilde_p3}) and is
given by:

\begin{equation}\label{eqn:definition_of_C}
    C =
    \begin{bmatrix}
        R - P & R- P \\
        S - P & T- P \\
        T - P & S- P \\
    \end{bmatrix}
\end{equation}

Note that in general, equation (\ref{eqn:linear_algebraic_equation_for_p}) will
not necessarily have a solution. From the Rouch\'{e}-Capelli theorem if there is
a solution it is unique as \(\text{rank}(C)=2\) which is the dimension of the
variable \(x\). The best fitting \(x\) is found by minimizing:

\begin{equation}\label{eqn:r_squared}
    \text{SSError} = \|C x- p^*\|_2^2 = \sum_{i=1}^{3}\left((C\bar x)_i-p_i^*\right)^2
\end{equation}

Note that \(\text{SSError}\), which is the square of the Frobenius
norm~\cite{Golub2013}, becomes a measure of how close a strategy is to being an
extortionate strategy. Suspicion
of extortion then corresponds to a threshold on \(\text{SSError}\).

By observing interactions (human or otherwise), their memory one representation
can be inferred and this approach can be used to recognise extortionate
behaviour. The notion of comparing theoretic and actual plays of the IPD is not
novel, see for example~\cite{Rand2013}. Immediately it is noted that if the
environment is noisy~\cite{Wu1995} then no strategy can be considered to be
extortionate as \(p_4>0\).

In the next section, this idea will be illustrated by observing the interactions
that take place in a computer based tournament of the IPD\@.

\section{Numerical experiments}\label{sec:numerical-experiments}

In~\cite{Stewart2012} results from a tournament with
\input{./assets/tex/number_of_stewart_plotkin_strategies/main.tex} strategies,
was presented with specific consideration given to ZD strategies. This
tournament is reproduced here using the Axelrod-Python
project~\cite{Knight2016}. To obtain a good measure of the corresponding
transition rates for each strategy all matches have been run for
\input{assets/tex/number_of_turns/main.tex} turns and every match has been
repeated \input{assets/tex/number_of_repetitions/main.tex} times. All of this
interaction data is available at~\cite{vincent_knight_2018_1297075}. A good
match between the inferred Markov chain and the state distribution of the actual
interactions has been verified. Data for this is presented in the supplementary
materials.

Figure~\ref{fig:SSError_overall_in_stewart_plotkin} shows the \(\text{SSError}\)
values for all the strategies in the tournament, as reported
in~\cite{Stewart2012} the extortionate strategy (which has an expected
\(\text{SSError}\) approximately 0) gains a large number of wins.

\begin{figure}[!htbp]
    \centering
    \includegraphics[width=.8\textwidth]{./assets/img/SSError_overall_in_stewart_plotkin/main.pdf}
    \caption{\(\text{SSError}\) and state probabilities for the strategies
        of~\cite{Stewart2012}, ordered both by number of wins and overall score.
        Note that \(P(DC)\) is not shown as it corresponds to the transpose of
        \(P(CD)\). Cooperator and Defector are omitted as they do not visit all
        the states.}
    \label{fig:SSError_overall_in_stewart_plotkin}
\end{figure}

Here, the work of~\cite{Stewart2012} is extended by investigating a tournament
with \input{assets/tex/number_of_full_strategies/main.tex}
strategies.

The results of this analysis are shown in
Figure~\ref{fig:SSError_and_probabilities_in_full}. The top ranking strategies
by number of wins seem to be extortionate (but not against all strategies) and
it can be seen that a small sub group of strategies achieve mutual defection.
All the top ranking strategies according to score achieve mutual cooperation and
do not extort each other, however they
\textbf{do} exhibit extortionate behaviour towards a number of the lower ranking
strategies.

\begin{figure}[!htbp]
    \centering
    \includegraphics[width=.8\textwidth]{./assets/img/SSError_and_probabilities_in_full/main.pdf}
    \caption{\(\text{SSError}\) for the strategies for the full tournament. Only
    strategy interactions for which \(p_4=0\) and \(\chi>1\) are displayed.}
    \label{fig:SSError_and_probabilities_in_full}
\end{figure}

\section{Conclusion}\label{sec:conclusion}

This work defines an approach to measure whether or not a player is playing a
strategy that corresponds to an extortionate strategy as defined
in~\cite{Press2012}: a mathematical model for suspicion. Indeed, all
extortionate strategies have been
 classified as lying on a triangular plane.
This rigorous classification fails to be robust to small measurement error, thus
a statistical approach is proposed.
This is done through a linear algebraic approach for approximating the solution
of a linear system. Using this, a large number of pairwise interactions is
simulated and in fact very few strategies are found to act extortionately.

The work of~\cite{Press2012}, whilst showing that a clever approach to taking
advantage of another memory one strategy exists: this is incomplete. Whilst the
elegance of this result is very attractive, just as the simplicity of the
victory of Tit For Tat in Axelrod's original tournaments was, it is incomplete.
Extortionate strategies achieve a high number of wins but they do not
achieve a high score which corresponds to the fitness landscape in an
evolutionary sense. From the large number of interactions a payoff matrix \(S\)
can be measured where \(S_{ij}\) denotes the score (using standard values of
\((R, S, T, P) = (3, 0, 5, 1)\)) of the \(i\)th strategy
against the \(j\)th strategy. Using this, the replicator equation
describes the evolution of the system based on a population density fitness
function:

\begin{equation}\label{eqn:replicator_dynamics}
    \frac{dx}{dt} = x(S-x^TS x)
\end{equation}

Equation (\ref{eqn:replicator_dynamics}) is solved numerically through an
integration technique described in~\cite{Petzold1983} and
Figure~\ref{fig:replicator_dynamics} shows the evolution of the distribution of
the system: the various strategies are ranked by scores. It is clear to see that
only the high ranking strategies survive the evolutionary process (in fact,
only \input{./assets/img/replicator_dynamics/main.tex}
have a final distribution greater than \(10 ^ {-2}\)). This confirms the
findings of~\cite{Moran1707} in which sophisticated strategies resist
evolutionary invasion of shorter memory strategies. Recalling
Figure~\ref{fig:SSError_and_probabilities_in_full} this demonstrates that:

\begin{itemize}
    \item Cooperation emerges through the evolutionary process: the high scoring
        strategies do not exhibit extortionate behaviour towards each other.
    \item Extortionate strategies do not survive the evolutionary process.
\end{itemize}

\begin{figure}[!htbp]
    \centering
    \includegraphics[width=.8\textwidth]{./assets/img/replicator_dynamics/main.pdf}
    \caption{Numerical simulation of the replicator equation
    (\ref{eqn:replicator_dynamics}): strategies are ordered by score, only the strategies with a high score survive the evolutionary process.}
    \label{fig:replicator_dynamics}
\end{figure}

This work can be used to classify plays of the IPD\@: data can be collected from
actual interactions (in lab or in the field). Furthermore, this allows for a
classification method similar to the notion of fingerprinting presented
in~\cite{Ashlock2008}. Trained strategies can potentially be classified as
extortionate or not or it could be possible to even constrain the reinforcement
learning approaches that are becoming prevalent in the literature.
Alternatively, this mathematical approach for recognising extortion could be
used in sophisticated strategies to defend against invasion. Arguably, some of
the strategies considered here exhibit this behaviour, indeed as described
in~\cite{Harper2017}, the top ranking strategies in the full tournament are
obtained using evolutionary reinforcement learning techniques, thus, suspicion
of extortionate behaviour could in fact be an evolutionary trait.

\section*{Acknowledgements}

The following open source software libraries were used in this research:

\begin{itemize}
    \item The Axelrod ~\cite{Knight2016, Knight2018} library (IPD strategies and
        tournaments).
    \item The sympy library~\cite{Meurer2017} (verification of all symbolic
        calculations).
    \item The matplotlib~\cite{Droettboom2018} library (visualisation).
    \item The pandas~\cite{Structures2010}, dask~\cite{Dask2016} and
        NumPy~\cite{Oliphant2015} libraries (data manipulation).
    \item The SciPy~\cite{Jones2001} library (numerical integration of the
        replicator equation).
\end{itemize}

This work was performed using the computational facilities of the Advanced
Research Computing @ Cardiff (ARCCA) Division, Cardiff University.

\printbibliography

\newpage
\section*{Supplementary materials}

\includepdf{assets/pdf/proof_of_form_of_extortionate_strategies/main.pdf}

\newpage

Using the pair wise interactions the transition rates \(p,
q\) can be measured and the steady state probabilities inferred and compared to
the actual probabilities of each state.
This is done numerically by computing the singular eigenvector of the
matrix \(A\) \cite{Stewart2009}:

\[
    A =
    \begin{bmatrix}
        p_1 q_1 & p_1 (1 - q_1) & (1 - p_1) q_1 & (1 -p_1) (1 - q_1) \\
        p_2 q_2 & p_2 (1 - q_2) & (1 - p_2) q_2 & (1 -p_2) (1 - q_2) \\
        p_3 q_3 & p_3 (1 - q_3) & (1 - p_3) q_3 & (1 -p_3) (1 - q_3) \\
        p_4 q_4 & p_4 (1 - q_4) & (1 - p_4) q_4 & (1 -p_4) (1 - q_4) \\
    \end{bmatrix}
\]

Figure~\ref{fig:computed_probabilities_vs_theoretic_probabilities} shows a
regression line fitted to every pairwise interaction with a reported
\(\text{SSError}\) value (pairwise interactions with missing states were
omitted). This serves to validate the approach: a part from some edge cases the
relationship is consistent.

\begin{figure}[!htbp]
    \centering
    \includegraphics[width=.8\textwidth]{./assets/img/computed_probabilities_vs_theoretic_probabilities/main.pdf}
    \caption{The
        relationship between the steady state probabilities inferred from the
        measured transitions and the actual steady state probabilities. A linear
        regression line is included validating the approach.}
    \label{fig:computed_probabilities_vs_theoretic_probabilities}
\end{figure}


\end{document}

strategies.

The results of this analysis are shown in
Figure~\ref{fig:SSError_and_probabilities_in_full}. The top ranking strategies
by number of wins seem to be extortionate (but not against all strategies) and
it can be seen that a small sub group of strategies achieve mutual defection.
All the top ranking strategies according to score achieve mutual cooperation and
do not extort each other, however they
\textbf{do} exhibit extortionate behaviour towards a number of the lower ranking
strategies.

\begin{figure}[!htbp]
    \centering
    \includegraphics[width=.8\textwidth]{./assets/img/SSError_and_probabilities_in_full/main.pdf}
    \caption{\(\text{SSError}\) for the strategies for the full tournament. Only
    strategy interactions for which \(p_4=0\) and \(\chi>1\) are displayed.}
    \label{fig:SSError_and_probabilities_in_full}
\end{figure}

\section{Conclusion}\label{sec:conclusion}

This work defines an approach to measure whether or not a player is playing a
strategy that corresponds to an extortionate strategy as defined
in~\cite{Press2012}: a mathematical model for suspicion. Indeed, all
extortionate strategies have been
 classified as lying on a triangular plane.
This rigorous classification fails to be robust to small measurement error, thus
a statistical approach is proposed.
This is done through a linear algebraic approach for approximating the solution
of a linear system. Using this, a large number of pairwise interactions is
simulated and in fact very few strategies are found to act extortionately.

The work of~\cite{Press2012}, whilst showing that a clever approach to taking
advantage of another memory one strategy exists: this is incomplete. Whilst the
elegance of this result is very attractive, just as the simplicity of the
victory of Tit For Tat in Axelrod's original tournaments was, it is incomplete.
Extortionate strategies achieve a high number of wins but they do not
achieve a high score which corresponds to the fitness landscape in an
evolutionary sense. From the large number of interactions a payoff matrix \(S\)
can be measured where \(S_{ij}\) denotes the score (using standard values of
\((R, S, T, P) = (3, 0, 5, 1)\)) of the \(i\)th strategy
against the \(j\)th strategy. Using this, the replicator equation
describes the evolution of the system based on a population density fitness
function:

\begin{equation}\label{eqn:replicator_dynamics}
    \frac{dx}{dt} = x(S-x^TS x)
\end{equation}

Equation (\ref{eqn:replicator_dynamics}) is solved numerically through an
integration technique described in~\cite{Petzold1983} and
Figure~\ref{fig:replicator_dynamics} shows the evolution of the distribution of
the system: the various strategies are ranked by scores. It is clear to see that
only the high ranking strategies survive the evolutionary process (in fact,
only \documentclass[a4paper]{article}

\usepackage{amsmath}
\usepackage{amssymb}
\usepackage[margin=1.5cm,
            includefoot,
            footskip=30pt]{geometry}
\usepackage{layout}
\usepackage{graphicx}
\usepackage{subcaption}

\usepackage{biblatex}
\usepackage{pdfpages}

\bibliography{main.bib}

\title{Suspicion: Recognising and evaluating the effectiveness
       of extortion in the Iterated Prisoner's Dilemma}
\author{Vincent A. Knight \and Nikoleta E. Glynatsi}
\date{\today}



\begin{document}

\maketitle

\begin{abstract}
    The Iterated Prisoner's Dilemma is a model for rational and evolutionary
    interactive behaviour. It has applications both in the study of human social
    behaviour as well as in biology.
    It is used to understand when and how a rational individual might
    accept an immediate cost to their own utility for the direct benefit of
    another.

    Much attention has been given to a class of strategies called
    Zero Determinant strategies. It has been theoretically shown that these
    strategies can ``extort'' any player.

    In this work, an approach to identify if observed strategies are playing in
    an extortionate way is described. Furthermore, experimental analysis of
    a large tournament with \input{assets/tex/number_of_full_strategies/main.tex}
    strategies is considered. In this setting
    the most highly performing strategies do not play in an extortionate way
    against each other but do against lower performing strategies.
    This suggests that whilst the theory of Zero Determinant strategies
    indicates that memory is not of fundamental importance to the evolution of
    cooperative behaviour, this is incomplete.
\end{abstract}

\section{Introduction}\label{sec:introduction}

Agent based game theoretic models have become a stalwart of the underpinning
mathematics of interactive behaviours. One of the major pieces of work
in this area is the pair of original computer tournaments run by Robert
Axelrod~\cite{Axelrod1980, Axelrod1980a}. These tournaments pitted submitted
computer strategies against each other in plays of the Iterated Prisoner's
Dilemma. A common game where agents can choose to pay a slight cost to their
immediate utility in the hope of building a reputation. This has been used in
economic and evolutionary game theory to understand the evolution of cooperative
behaviour.

Recently, a class of strategies was described in~\cite{Press2012} that can
provably extort any given opponent. In~\cite{Hilbe2013, Moran1707} some
questions have already been asked about the true effectiveness of these
strategies in an evolutionary setting. Here another question is asked: is it
possible to recognise this extortionate behaviour? A mathematical procedure for
suspicion is presented: in the same way that the continued actions of an
extortionate individual might raise suspicion.

This work makes use of the Axelrod Python library~\cite{Knight2018, Knight2016}
with a large number of Prisoner Dilemma strategies available to give an
extensive numerical example of the ideas presented.  The approach is presented
in Section~\ref{sec:delta-zd-strategies}.  All of the code and data discussed
in Section~\ref{sec:numerical-experiments} is open sourced, archived and
written according to best scientific principles~\cite{Wilson2014}. The data
archive can be found at~\cite{vincent_knight_2018_1297075}.

\section{Recognising Extortion}\label{sec:delta-zd-strategies}

In~\cite{Press2012}, given a match between 2 memory-one strategies, the concept
of Zero Determinant (ZD) strategies is introduced. The main result of that paper
shows that given two memory one players \(p, q\in\mathbb{R}^4\) a linear
relationship between the players' scores could be forced by one of the players.

Using the notation of~\cite{Press2012}, assuming the utilities for player \(p\)
are given by \(S_x=(R, S, T, P)\) and for player \(q\) by \(S_y=(R, T, S, P)\)
and that the stationary scores of each player is given by \(S_X\) and \(S_Y\)
respectively. The main result of~\cite{Press2012} is that if

\begin{equation}\label{eqn:linear_relationship_for_p}
    \tilde p=\alpha S_x + \beta S_y + \gamma
\end{equation}

or

\begin{equation}\label{eqn:linear_relationship_for_q}
    \tilde q=\alpha S_x + \beta S_y + \gamma
\end{equation}

where \(\tilde p = (1 - p_1, 1 - p_2, p_3, p_4)\) and
\(\tilde q = (1 - q_1, 1 - q_2, q_3, q_4)\) then:

\begin{equation}
    \alpha S_X + \beta S_Y + \gamma = 0
\end{equation}

In~\cite{Press2012} a particular type of ZD strategy is defined: extortionate
strategies. If:

\begin{equation}\label{eqn:constraint_for_extortion}
    \gamma = - P(\alpha + \beta)
\end{equation}

then the player can ensure they get a score \(\chi\) times
larger than the opponent. This extortion coefficient is given by:

\begin{equation}\label{eqn:definition_of_chi}
    \chi=\frac{-\beta}{\alpha}
\end{equation}

Thus, if (\ref{eqn:constraint_for_extortion}) holds and \(\chi >1\) a player is
said to extort their opponent.
Here, the reverse problem is considered: given a
\(p\in\mathbb{R}^4\) how does one identify \(\alpha, \beta\) if they
exist and is the strategy in fact acting in an extortionate way?

These conditions correspond to:

\begin{align}
    \tilde p_1 & = \alpha R + \beta R - P (\alpha + \beta)
            \label{eqn:condition_for_tilde_p1}\\
    \tilde p_2 & = \alpha S + \beta T - P (\alpha + \beta)
            \label{eqn:condition_for_tilde_p2}\\
    \tilde p_3 & = \alpha T + \beta S - P (\alpha + \beta)
            \label{eqn:condition_for_tilde_p3}\\
    \tilde p_4 & = \alpha P + \beta P - P (\alpha + \beta)
            \label{eqn:condition_for_tilde_p4}
\end{align}

Equation (\ref{eqn:condition_for_tilde_p4}) ensures that \(p_4=\tilde p_4=0\).
Equations (\ref{eqn:condition_for_tilde_p1}-\ref{eqn:condition_for_tilde_p3})
can be used to eliminate \(\alpha, \beta\), giving:

\begin{equation}\label{eqn:planar_definition_of_extortion}
    \tilde p_1 = \frac{(R - P)(\tilde p_2 + \tilde p_3)}{S + T - 2P}
\end{equation}

with:

\begin{equation}\label{eqn:definition_of_chi}
    \chi = \frac{\tilde p_2 (P - T) + \tilde p_3 (S - P)}
                {\tilde p_2 (P - S) + \tilde p_3 (T - P)}
\end{equation}

Given a strategy \(p\in\mathbb{R}^{4\times 1}\) equations
(\ref{eqn:condition_for_tilde_p4}), (\ref{eqn:planar_definition_of_extortion}-\ref{eqn:definition_of_chi}) can be used to check if
a strategy is extortionate. The conditions correspond to:

\begin{align}
    p_1 & = \frac{(R-P)(p_2 + p_3) - R + T + S - P}{S + T - 2P}
     \label{eqn:condition_for_p1}\\
    p_4 & = 0 \label{eqn:condition_for_p4}\\
    1 & > p_2 + p_3\label{eqn:condition_for_chi}
\end{align}

The algebraic steps necessary to prove these results are available in the
supporting materials.

All extortionate strategies reside on a triangular (\ref{eqn:condition_for_chi})
plane (\ref{eqn:condition_for_p1}) in 3 dimensions (\ref{eqn:condition_for_p4}).
Using this formulation it can be seen that a necessary (but not sufficient)
condition for an extortionate strategy is that it cooperates on average less
than 50\% of the time when in a state of disagreement with the opponent.

As an example, consider the known extortionate strategy \(p=(8 / 9, 1 / 2, 1 /
3, 0)\) from~\cite{Stewart2012} which is referred to as \texttt{Extort-2}. In
this case, for the standard values of \((R, T, S, P)\) constraint
(\ref{eqn:condition_for_p1}) corresponds to:

\begin{equation}
    p_1 = \frac{2(p_2 + p_3) + 1}{3}
\end{equation}

It is clear that in this case all constraints hold.

This approach could in fact be used to confirm that a given strategy is acting
in an extortionate manner even if it is not a memory one strategy. However, in
practice, if a closed form for \(p\) is not known, then due to measurement
and/or numerical error this would not work.

This problem can be written in the following linear algebraic form where
\(x=(\alpha, \beta)\)
and \(p^*=(\tilde p_1 - 1, tilde_2 - 1, p_3)\):

\begin{equation}\label{eqn:linear_algebraic_equation_for_p}
    Cx= p^*
\end{equation}

\(C\) corresponds to equations
(\ref{eqn:condition_for_tilde_p1}-\ref{eqn:condition_for_tilde_p3}) and is
given by:

\begin{equation}\label{eqn:definition_of_C}
    C =
    \begin{bmatrix}
        R - P & R- P \\
        S - P & T- P \\
        T - P & S- P \\
    \end{bmatrix}
\end{equation}

Note that in general, equation (\ref{eqn:linear_algebraic_equation_for_p}) will
not necessarily have a solution. From the Rouch\'{e}-Capelli theorem if there is
a solution it is unique as \(\text{rank}(C)=2\) which is the dimension of the
variable \(x\). The best fitting \(x\) is found by minimizing:

\begin{equation}\label{eqn:r_squared}
    \text{SSError} = \|C x- p^*\|_2^2 = \sum_{i=1}^{3}\left((C\bar x)_i-p_i^*\right)^2
\end{equation}

Note that \(\text{SSError}\), which is the square of the Frobenius
norm~\cite{Golub2013}, becomes a measure of how close a strategy is to being an
extortionate strategy. Suspicion
of extortion then corresponds to a threshold on \(\text{SSError}\).

By observing interactions (human or otherwise), their memory one representation
can be inferred and this approach can be used to recognise extortionate
behaviour. The notion of comparing theoretic and actual plays of the IPD is not
novel, see for example~\cite{Rand2013}. Immediately it is noted that if the
environment is noisy~\cite{Wu1995} then no strategy can be considered to be
extortionate as \(p_4>0\).

In the next section, this idea will be illustrated by observing the interactions
that take place in a computer based tournament of the IPD\@.

\section{Numerical experiments}\label{sec:numerical-experiments}

In~\cite{Stewart2012} results from a tournament with
\input{./assets/tex/number_of_stewart_plotkin_strategies/main.tex} strategies,
was presented with specific consideration given to ZD strategies. This
tournament is reproduced here using the Axelrod-Python
project~\cite{Knight2016}. To obtain a good measure of the corresponding
transition rates for each strategy all matches have been run for
\input{assets/tex/number_of_turns/main.tex} turns and every match has been
repeated \input{assets/tex/number_of_repetitions/main.tex} times. All of this
interaction data is available at~\cite{vincent_knight_2018_1297075}. A good
match between the inferred Markov chain and the state distribution of the actual
interactions has been verified. Data for this is presented in the supplementary
materials.

Figure~\ref{fig:SSError_overall_in_stewart_plotkin} shows the \(\text{SSError}\)
values for all the strategies in the tournament, as reported
in~\cite{Stewart2012} the extortionate strategy (which has an expected
\(\text{SSError}\) approximately 0) gains a large number of wins.

\begin{figure}[!htbp]
    \centering
    \includegraphics[width=.8\textwidth]{./assets/img/SSError_overall_in_stewart_plotkin/main.pdf}
    \caption{\(\text{SSError}\) and state probabilities for the strategies
        of~\cite{Stewart2012}, ordered both by number of wins and overall score.
        Note that \(P(DC)\) is not shown as it corresponds to the transpose of
        \(P(CD)\). Cooperator and Defector are omitted as they do not visit all
        the states.}
    \label{fig:SSError_overall_in_stewart_plotkin}
\end{figure}

Here, the work of~\cite{Stewart2012} is extended by investigating a tournament
with \input{assets/tex/number_of_full_strategies/main.tex}
strategies.

The results of this analysis are shown in
Figure~\ref{fig:SSError_and_probabilities_in_full}. The top ranking strategies
by number of wins seem to be extortionate (but not against all strategies) and
it can be seen that a small sub group of strategies achieve mutual defection.
All the top ranking strategies according to score achieve mutual cooperation and
do not extort each other, however they
\textbf{do} exhibit extortionate behaviour towards a number of the lower ranking
strategies.

\begin{figure}[!htbp]
    \centering
    \includegraphics[width=.8\textwidth]{./assets/img/SSError_and_probabilities_in_full/main.pdf}
    \caption{\(\text{SSError}\) for the strategies for the full tournament. Only
    strategy interactions for which \(p_4=0\) and \(\chi>1\) are displayed.}
    \label{fig:SSError_and_probabilities_in_full}
\end{figure}

\section{Conclusion}\label{sec:conclusion}

This work defines an approach to measure whether or not a player is playing a
strategy that corresponds to an extortionate strategy as defined
in~\cite{Press2012}: a mathematical model for suspicion. Indeed, all
extortionate strategies have been
 classified as lying on a triangular plane.
This rigorous classification fails to be robust to small measurement error, thus
a statistical approach is proposed.
This is done through a linear algebraic approach for approximating the solution
of a linear system. Using this, a large number of pairwise interactions is
simulated and in fact very few strategies are found to act extortionately.

The work of~\cite{Press2012}, whilst showing that a clever approach to taking
advantage of another memory one strategy exists: this is incomplete. Whilst the
elegance of this result is very attractive, just as the simplicity of the
victory of Tit For Tat in Axelrod's original tournaments was, it is incomplete.
Extortionate strategies achieve a high number of wins but they do not
achieve a high score which corresponds to the fitness landscape in an
evolutionary sense. From the large number of interactions a payoff matrix \(S\)
can be measured where \(S_{ij}\) denotes the score (using standard values of
\((R, S, T, P) = (3, 0, 5, 1)\)) of the \(i\)th strategy
against the \(j\)th strategy. Using this, the replicator equation
describes the evolution of the system based on a population density fitness
function:

\begin{equation}\label{eqn:replicator_dynamics}
    \frac{dx}{dt} = x(S-x^TS x)
\end{equation}

Equation (\ref{eqn:replicator_dynamics}) is solved numerically through an
integration technique described in~\cite{Petzold1983} and
Figure~\ref{fig:replicator_dynamics} shows the evolution of the distribution of
the system: the various strategies are ranked by scores. It is clear to see that
only the high ranking strategies survive the evolutionary process (in fact,
only \input{./assets/img/replicator_dynamics/main.tex}
have a final distribution greater than \(10 ^ {-2}\)). This confirms the
findings of~\cite{Moran1707} in which sophisticated strategies resist
evolutionary invasion of shorter memory strategies. Recalling
Figure~\ref{fig:SSError_and_probabilities_in_full} this demonstrates that:

\begin{itemize}
    \item Cooperation emerges through the evolutionary process: the high scoring
        strategies do not exhibit extortionate behaviour towards each other.
    \item Extortionate strategies do not survive the evolutionary process.
\end{itemize}

\begin{figure}[!htbp]
    \centering
    \includegraphics[width=.8\textwidth]{./assets/img/replicator_dynamics/main.pdf}
    \caption{Numerical simulation of the replicator equation
    (\ref{eqn:replicator_dynamics}): strategies are ordered by score, only the strategies with a high score survive the evolutionary process.}
    \label{fig:replicator_dynamics}
\end{figure}

This work can be used to classify plays of the IPD\@: data can be collected from
actual interactions (in lab or in the field). Furthermore, this allows for a
classification method similar to the notion of fingerprinting presented
in~\cite{Ashlock2008}. Trained strategies can potentially be classified as
extortionate or not or it could be possible to even constrain the reinforcement
learning approaches that are becoming prevalent in the literature.
Alternatively, this mathematical approach for recognising extortion could be
used in sophisticated strategies to defend against invasion. Arguably, some of
the strategies considered here exhibit this behaviour, indeed as described
in~\cite{Harper2017}, the top ranking strategies in the full tournament are
obtained using evolutionary reinforcement learning techniques, thus, suspicion
of extortionate behaviour could in fact be an evolutionary trait.

\section*{Acknowledgements}

The following open source software libraries were used in this research:

\begin{itemize}
    \item The Axelrod ~\cite{Knight2016, Knight2018} library (IPD strategies and
        tournaments).
    \item The sympy library~\cite{Meurer2017} (verification of all symbolic
        calculations).
    \item The matplotlib~\cite{Droettboom2018} library (visualisation).
    \item The pandas~\cite{Structures2010}, dask~\cite{Dask2016} and
        NumPy~\cite{Oliphant2015} libraries (data manipulation).
    \item The SciPy~\cite{Jones2001} library (numerical integration of the
        replicator equation).
\end{itemize}

This work was performed using the computational facilities of the Advanced
Research Computing @ Cardiff (ARCCA) Division, Cardiff University.

\printbibliography

\newpage
\section*{Supplementary materials}

\includepdf{assets/pdf/proof_of_form_of_extortionate_strategies/main.pdf}

\newpage

Using the pair wise interactions the transition rates \(p,
q\) can be measured and the steady state probabilities inferred and compared to
the actual probabilities of each state.
This is done numerically by computing the singular eigenvector of the
matrix \(A\) \cite{Stewart2009}:

\[
    A =
    \begin{bmatrix}
        p_1 q_1 & p_1 (1 - q_1) & (1 - p_1) q_1 & (1 -p_1) (1 - q_1) \\
        p_2 q_2 & p_2 (1 - q_2) & (1 - p_2) q_2 & (1 -p_2) (1 - q_2) \\
        p_3 q_3 & p_3 (1 - q_3) & (1 - p_3) q_3 & (1 -p_3) (1 - q_3) \\
        p_4 q_4 & p_4 (1 - q_4) & (1 - p_4) q_4 & (1 -p_4) (1 - q_4) \\
    \end{bmatrix}
\]

Figure~\ref{fig:computed_probabilities_vs_theoretic_probabilities} shows a
regression line fitted to every pairwise interaction with a reported
\(\text{SSError}\) value (pairwise interactions with missing states were
omitted). This serves to validate the approach: a part from some edge cases the
relationship is consistent.

\begin{figure}[!htbp]
    \centering
    \includegraphics[width=.8\textwidth]{./assets/img/computed_probabilities_vs_theoretic_probabilities/main.pdf}
    \caption{The
        relationship between the steady state probabilities inferred from the
        measured transitions and the actual steady state probabilities. A linear
        regression line is included validating the approach.}
    \label{fig:computed_probabilities_vs_theoretic_probabilities}
\end{figure}


\end{document}

have a final distribution greater than \(10 ^ {-2}\)). This confirms the
findings of~\cite{Moran1707} in which sophisticated strategies resist
evolutionary invasion of shorter memory strategies. Recalling
Figure~\ref{fig:SSError_and_probabilities_in_full} this demonstrates that:

\begin{itemize}
    \item Cooperation emerges through the evolutionary process: the high scoring
        strategies do not exhibit extortionate behaviour towards each other.
    \item Extortionate strategies do not survive the evolutionary process.
\end{itemize}

\begin{figure}[!htbp]
    \centering
    \includegraphics[width=.8\textwidth]{./assets/img/replicator_dynamics/main.pdf}
    \caption{Numerical simulation of the replicator equation
    (\ref{eqn:replicator_dynamics}): strategies are ordered by score, only the strategies with a high score survive the evolutionary process.}
    \label{fig:replicator_dynamics}
\end{figure}

This work can be used to classify plays of the IPD\@: data can be collected from
actual interactions (in lab or in the field). Furthermore, this allows for a
classification method similar to the notion of fingerprinting presented
in~\cite{Ashlock2008}. Trained strategies can potentially be classified as
extortionate or not or it could be possible to even constrain the reinforcement
learning approaches that are becoming prevalent in the literature.
Alternatively, this mathematical approach for recognising extortion could be
used in sophisticated strategies to defend against invasion. Arguably, some of
the strategies considered here exhibit this behaviour, indeed as described
in~\cite{Harper2017}, the top ranking strategies in the full tournament are
obtained using evolutionary reinforcement learning techniques, thus, suspicion
of extortionate behaviour could in fact be an evolutionary trait.

\section*{Acknowledgements}

The following open source software libraries were used in this research:

\begin{itemize}
    \item The Axelrod ~\cite{Knight2016, Knight2018} library (IPD strategies and
        tournaments).
    \item The sympy library~\cite{Meurer2017} (verification of all symbolic
        calculations).
    \item The matplotlib~\cite{Droettboom2018} library (visualisation).
    \item The pandas~\cite{Structures2010}, dask~\cite{Dask2016} and
        NumPy~\cite{Oliphant2015} libraries (data manipulation).
    \item The SciPy~\cite{Jones2001} library (numerical integration of the
        replicator equation).
\end{itemize}

This work was performed using the computational facilities of the Advanced
Research Computing @ Cardiff (ARCCA) Division, Cardiff University.

\printbibliography

\newpage
\section*{Supplementary materials}

\includepdf{assets/pdf/proof_of_form_of_extortionate_strategies/main.pdf}

\newpage

Using the pair wise interactions the transition rates \(p,
q\) can be measured and the steady state probabilities inferred and compared to
the actual probabilities of each state.
This is done numerically by computing the singular eigenvector of the
matrix \(A\) \cite{Stewart2009}:

\[
    A =
    \begin{bmatrix}
        p_1 q_1 & p_1 (1 - q_1) & (1 - p_1) q_1 & (1 -p_1) (1 - q_1) \\
        p_2 q_2 & p_2 (1 - q_2) & (1 - p_2) q_2 & (1 -p_2) (1 - q_2) \\
        p_3 q_3 & p_3 (1 - q_3) & (1 - p_3) q_3 & (1 -p_3) (1 - q_3) \\
        p_4 q_4 & p_4 (1 - q_4) & (1 - p_4) q_4 & (1 -p_4) (1 - q_4) \\
    \end{bmatrix}
\]

Figure~\ref{fig:computed_probabilities_vs_theoretic_probabilities} shows a
regression line fitted to every pairwise interaction with a reported
\(\text{SSError}\) value (pairwise interactions with missing states were
omitted). This serves to validate the approach: a part from some edge cases the
relationship is consistent.

\begin{figure}[!htbp]
    \centering
    \includegraphics[width=.8\textwidth]{./assets/img/computed_probabilities_vs_theoretic_probabilities/main.pdf}
    \caption{The
        relationship between the steady state probabilities inferred from the
        measured transitions and the actual steady state probabilities. A linear
        regression line is included validating the approach.}
    \label{fig:computed_probabilities_vs_theoretic_probabilities}
\end{figure}


\end{document}
 strategies,
was presented with specific consideration given to ZD strategies. This
tournament is reproduced here using the Axelrod-Python
project~\cite{Knight2016}. To obtain a good measure of the corresponding
transition rates for each strategy all matches have been run for
\documentclass[a4paper]{article}

\usepackage{amsmath}
\usepackage{amssymb}
\usepackage[margin=1.5cm,
            includefoot,
            footskip=30pt]{geometry}
\usepackage{layout}
\usepackage{graphicx}
\usepackage{subcaption}

\usepackage{biblatex}
\usepackage{pdfpages}

\bibliography{main.bib}

\title{Suspicion: Recognising and evaluating the effectiveness
       of extortion in the Iterated Prisoner's Dilemma}
\author{Vincent A. Knight \and Nikoleta E. Glynatsi}
\date{\today}



\begin{document}

\maketitle

\begin{abstract}
    The Iterated Prisoner's Dilemma is a model for rational and evolutionary
    interactive behaviour. It has applications both in the study of human social
    behaviour as well as in biology.
    It is used to understand when and how a rational individual might
    accept an immediate cost to their own utility for the direct benefit of
    another.

    Much attention has been given to a class of strategies called
    Zero Determinant strategies. It has been theoretically shown that these
    strategies can ``extort'' any player.

    In this work, an approach to identify if observed strategies are playing in
    an extortionate way is described. Furthermore, experimental analysis of
    a large tournament with \documentclass[a4paper]{article}

\usepackage{amsmath}
\usepackage{amssymb}
\usepackage[margin=1.5cm,
            includefoot,
            footskip=30pt]{geometry}
\usepackage{layout}
\usepackage{graphicx}
\usepackage{subcaption}

\usepackage{biblatex}
\usepackage{pdfpages}

\bibliography{main.bib}

\title{Suspicion: Recognising and evaluating the effectiveness
       of extortion in the Iterated Prisoner's Dilemma}
\author{Vincent A. Knight \and Nikoleta E. Glynatsi}
\date{\today}



\begin{document}

\maketitle

\begin{abstract}
    The Iterated Prisoner's Dilemma is a model for rational and evolutionary
    interactive behaviour. It has applications both in the study of human social
    behaviour as well as in biology.
    It is used to understand when and how a rational individual might
    accept an immediate cost to their own utility for the direct benefit of
    another.

    Much attention has been given to a class of strategies called
    Zero Determinant strategies. It has been theoretically shown that these
    strategies can ``extort'' any player.

    In this work, an approach to identify if observed strategies are playing in
    an extortionate way is described. Furthermore, experimental analysis of
    a large tournament with \input{assets/tex/number_of_full_strategies/main.tex}
    strategies is considered. In this setting
    the most highly performing strategies do not play in an extortionate way
    against each other but do against lower performing strategies.
    This suggests that whilst the theory of Zero Determinant strategies
    indicates that memory is not of fundamental importance to the evolution of
    cooperative behaviour, this is incomplete.
\end{abstract}

\section{Introduction}\label{sec:introduction}

Agent based game theoretic models have become a stalwart of the underpinning
mathematics of interactive behaviours. One of the major pieces of work
in this area is the pair of original computer tournaments run by Robert
Axelrod~\cite{Axelrod1980, Axelrod1980a}. These tournaments pitted submitted
computer strategies against each other in plays of the Iterated Prisoner's
Dilemma. A common game where agents can choose to pay a slight cost to their
immediate utility in the hope of building a reputation. This has been used in
economic and evolutionary game theory to understand the evolution of cooperative
behaviour.

Recently, a class of strategies was described in~\cite{Press2012} that can
provably extort any given opponent. In~\cite{Hilbe2013, Moran1707} some
questions have already been asked about the true effectiveness of these
strategies in an evolutionary setting. Here another question is asked: is it
possible to recognise this extortionate behaviour? A mathematical procedure for
suspicion is presented: in the same way that the continued actions of an
extortionate individual might raise suspicion.

This work makes use of the Axelrod Python library~\cite{Knight2018, Knight2016}
with a large number of Prisoner Dilemma strategies available to give an
extensive numerical example of the ideas presented.  The approach is presented
in Section~\ref{sec:delta-zd-strategies}.  All of the code and data discussed
in Section~\ref{sec:numerical-experiments} is open sourced, archived and
written according to best scientific principles~\cite{Wilson2014}. The data
archive can be found at~\cite{vincent_knight_2018_1297075}.

\section{Recognising Extortion}\label{sec:delta-zd-strategies}

In~\cite{Press2012}, given a match between 2 memory-one strategies, the concept
of Zero Determinant (ZD) strategies is introduced. The main result of that paper
shows that given two memory one players \(p, q\in\mathbb{R}^4\) a linear
relationship between the players' scores could be forced by one of the players.

Using the notation of~\cite{Press2012}, assuming the utilities for player \(p\)
are given by \(S_x=(R, S, T, P)\) and for player \(q\) by \(S_y=(R, T, S, P)\)
and that the stationary scores of each player is given by \(S_X\) and \(S_Y\)
respectively. The main result of~\cite{Press2012} is that if

\begin{equation}\label{eqn:linear_relationship_for_p}
    \tilde p=\alpha S_x + \beta S_y + \gamma
\end{equation}

or

\begin{equation}\label{eqn:linear_relationship_for_q}
    \tilde q=\alpha S_x + \beta S_y + \gamma
\end{equation}

where \(\tilde p = (1 - p_1, 1 - p_2, p_3, p_4)\) and
\(\tilde q = (1 - q_1, 1 - q_2, q_3, q_4)\) then:

\begin{equation}
    \alpha S_X + \beta S_Y + \gamma = 0
\end{equation}

In~\cite{Press2012} a particular type of ZD strategy is defined: extortionate
strategies. If:

\begin{equation}\label{eqn:constraint_for_extortion}
    \gamma = - P(\alpha + \beta)
\end{equation}

then the player can ensure they get a score \(\chi\) times
larger than the opponent. This extortion coefficient is given by:

\begin{equation}\label{eqn:definition_of_chi}
    \chi=\frac{-\beta}{\alpha}
\end{equation}

Thus, if (\ref{eqn:constraint_for_extortion}) holds and \(\chi >1\) a player is
said to extort their opponent.
Here, the reverse problem is considered: given a
\(p\in\mathbb{R}^4\) how does one identify \(\alpha, \beta\) if they
exist and is the strategy in fact acting in an extortionate way?

These conditions correspond to:

\begin{align}
    \tilde p_1 & = \alpha R + \beta R - P (\alpha + \beta)
            \label{eqn:condition_for_tilde_p1}\\
    \tilde p_2 & = \alpha S + \beta T - P (\alpha + \beta)
            \label{eqn:condition_for_tilde_p2}\\
    \tilde p_3 & = \alpha T + \beta S - P (\alpha + \beta)
            \label{eqn:condition_for_tilde_p3}\\
    \tilde p_4 & = \alpha P + \beta P - P (\alpha + \beta)
            \label{eqn:condition_for_tilde_p4}
\end{align}

Equation (\ref{eqn:condition_for_tilde_p4}) ensures that \(p_4=\tilde p_4=0\).
Equations (\ref{eqn:condition_for_tilde_p1}-\ref{eqn:condition_for_tilde_p3})
can be used to eliminate \(\alpha, \beta\), giving:

\begin{equation}\label{eqn:planar_definition_of_extortion}
    \tilde p_1 = \frac{(R - P)(\tilde p_2 + \tilde p_3)}{S + T - 2P}
\end{equation}

with:

\begin{equation}\label{eqn:definition_of_chi}
    \chi = \frac{\tilde p_2 (P - T) + \tilde p_3 (S - P)}
                {\tilde p_2 (P - S) + \tilde p_3 (T - P)}
\end{equation}

Given a strategy \(p\in\mathbb{R}^{4\times 1}\) equations
(\ref{eqn:condition_for_tilde_p4}), (\ref{eqn:planar_definition_of_extortion}-\ref{eqn:definition_of_chi}) can be used to check if
a strategy is extortionate. The conditions correspond to:

\begin{align}
    p_1 & = \frac{(R-P)(p_2 + p_3) - R + T + S - P}{S + T - 2P}
     \label{eqn:condition_for_p1}\\
    p_4 & = 0 \label{eqn:condition_for_p4}\\
    1 & > p_2 + p_3\label{eqn:condition_for_chi}
\end{align}

The algebraic steps necessary to prove these results are available in the
supporting materials.

All extortionate strategies reside on a triangular (\ref{eqn:condition_for_chi})
plane (\ref{eqn:condition_for_p1}) in 3 dimensions (\ref{eqn:condition_for_p4}).
Using this formulation it can be seen that a necessary (but not sufficient)
condition for an extortionate strategy is that it cooperates on average less
than 50\% of the time when in a state of disagreement with the opponent.

As an example, consider the known extortionate strategy \(p=(8 / 9, 1 / 2, 1 /
3, 0)\) from~\cite{Stewart2012} which is referred to as \texttt{Extort-2}. In
this case, for the standard values of \((R, T, S, P)\) constraint
(\ref{eqn:condition_for_p1}) corresponds to:

\begin{equation}
    p_1 = \frac{2(p_2 + p_3) + 1}{3}
\end{equation}

It is clear that in this case all constraints hold.

This approach could in fact be used to confirm that a given strategy is acting
in an extortionate manner even if it is not a memory one strategy. However, in
practice, if a closed form for \(p\) is not known, then due to measurement
and/or numerical error this would not work.

This problem can be written in the following linear algebraic form where
\(x=(\alpha, \beta)\)
and \(p^*=(\tilde p_1 - 1, tilde_2 - 1, p_3)\):

\begin{equation}\label{eqn:linear_algebraic_equation_for_p}
    Cx= p^*
\end{equation}

\(C\) corresponds to equations
(\ref{eqn:condition_for_tilde_p1}-\ref{eqn:condition_for_tilde_p3}) and is
given by:

\begin{equation}\label{eqn:definition_of_C}
    C =
    \begin{bmatrix}
        R - P & R- P \\
        S - P & T- P \\
        T - P & S- P \\
    \end{bmatrix}
\end{equation}

Note that in general, equation (\ref{eqn:linear_algebraic_equation_for_p}) will
not necessarily have a solution. From the Rouch\'{e}-Capelli theorem if there is
a solution it is unique as \(\text{rank}(C)=2\) which is the dimension of the
variable \(x\). The best fitting \(x\) is found by minimizing:

\begin{equation}\label{eqn:r_squared}
    \text{SSError} = \|C x- p^*\|_2^2 = \sum_{i=1}^{3}\left((C\bar x)_i-p_i^*\right)^2
\end{equation}

Note that \(\text{SSError}\), which is the square of the Frobenius
norm~\cite{Golub2013}, becomes a measure of how close a strategy is to being an
extortionate strategy. Suspicion
of extortion then corresponds to a threshold on \(\text{SSError}\).

By observing interactions (human or otherwise), their memory one representation
can be inferred and this approach can be used to recognise extortionate
behaviour. The notion of comparing theoretic and actual plays of the IPD is not
novel, see for example~\cite{Rand2013}. Immediately it is noted that if the
environment is noisy~\cite{Wu1995} then no strategy can be considered to be
extortionate as \(p_4>0\).

In the next section, this idea will be illustrated by observing the interactions
that take place in a computer based tournament of the IPD\@.

\section{Numerical experiments}\label{sec:numerical-experiments}

In~\cite{Stewart2012} results from a tournament with
\input{./assets/tex/number_of_stewart_plotkin_strategies/main.tex} strategies,
was presented with specific consideration given to ZD strategies. This
tournament is reproduced here using the Axelrod-Python
project~\cite{Knight2016}. To obtain a good measure of the corresponding
transition rates for each strategy all matches have been run for
\input{assets/tex/number_of_turns/main.tex} turns and every match has been
repeated \input{assets/tex/number_of_repetitions/main.tex} times. All of this
interaction data is available at~\cite{vincent_knight_2018_1297075}. A good
match between the inferred Markov chain and the state distribution of the actual
interactions has been verified. Data for this is presented in the supplementary
materials.

Figure~\ref{fig:SSError_overall_in_stewart_plotkin} shows the \(\text{SSError}\)
values for all the strategies in the tournament, as reported
in~\cite{Stewart2012} the extortionate strategy (which has an expected
\(\text{SSError}\) approximately 0) gains a large number of wins.

\begin{figure}[!htbp]
    \centering
    \includegraphics[width=.8\textwidth]{./assets/img/SSError_overall_in_stewart_plotkin/main.pdf}
    \caption{\(\text{SSError}\) and state probabilities for the strategies
        of~\cite{Stewart2012}, ordered both by number of wins and overall score.
        Note that \(P(DC)\) is not shown as it corresponds to the transpose of
        \(P(CD)\). Cooperator and Defector are omitted as they do not visit all
        the states.}
    \label{fig:SSError_overall_in_stewart_plotkin}
\end{figure}

Here, the work of~\cite{Stewart2012} is extended by investigating a tournament
with \input{assets/tex/number_of_full_strategies/main.tex}
strategies.

The results of this analysis are shown in
Figure~\ref{fig:SSError_and_probabilities_in_full}. The top ranking strategies
by number of wins seem to be extortionate (but not against all strategies) and
it can be seen that a small sub group of strategies achieve mutual defection.
All the top ranking strategies according to score achieve mutual cooperation and
do not extort each other, however they
\textbf{do} exhibit extortionate behaviour towards a number of the lower ranking
strategies.

\begin{figure}[!htbp]
    \centering
    \includegraphics[width=.8\textwidth]{./assets/img/SSError_and_probabilities_in_full/main.pdf}
    \caption{\(\text{SSError}\) for the strategies for the full tournament. Only
    strategy interactions for which \(p_4=0\) and \(\chi>1\) are displayed.}
    \label{fig:SSError_and_probabilities_in_full}
\end{figure}

\section{Conclusion}\label{sec:conclusion}

This work defines an approach to measure whether or not a player is playing a
strategy that corresponds to an extortionate strategy as defined
in~\cite{Press2012}: a mathematical model for suspicion. Indeed, all
extortionate strategies have been
 classified as lying on a triangular plane.
This rigorous classification fails to be robust to small measurement error, thus
a statistical approach is proposed.
This is done through a linear algebraic approach for approximating the solution
of a linear system. Using this, a large number of pairwise interactions is
simulated and in fact very few strategies are found to act extortionately.

The work of~\cite{Press2012}, whilst showing that a clever approach to taking
advantage of another memory one strategy exists: this is incomplete. Whilst the
elegance of this result is very attractive, just as the simplicity of the
victory of Tit For Tat in Axelrod's original tournaments was, it is incomplete.
Extortionate strategies achieve a high number of wins but they do not
achieve a high score which corresponds to the fitness landscape in an
evolutionary sense. From the large number of interactions a payoff matrix \(S\)
can be measured where \(S_{ij}\) denotes the score (using standard values of
\((R, S, T, P) = (3, 0, 5, 1)\)) of the \(i\)th strategy
against the \(j\)th strategy. Using this, the replicator equation
describes the evolution of the system based on a population density fitness
function:

\begin{equation}\label{eqn:replicator_dynamics}
    \frac{dx}{dt} = x(S-x^TS x)
\end{equation}

Equation (\ref{eqn:replicator_dynamics}) is solved numerically through an
integration technique described in~\cite{Petzold1983} and
Figure~\ref{fig:replicator_dynamics} shows the evolution of the distribution of
the system: the various strategies are ranked by scores. It is clear to see that
only the high ranking strategies survive the evolutionary process (in fact,
only \input{./assets/img/replicator_dynamics/main.tex}
have a final distribution greater than \(10 ^ {-2}\)). This confirms the
findings of~\cite{Moran1707} in which sophisticated strategies resist
evolutionary invasion of shorter memory strategies. Recalling
Figure~\ref{fig:SSError_and_probabilities_in_full} this demonstrates that:

\begin{itemize}
    \item Cooperation emerges through the evolutionary process: the high scoring
        strategies do not exhibit extortionate behaviour towards each other.
    \item Extortionate strategies do not survive the evolutionary process.
\end{itemize}

\begin{figure}[!htbp]
    \centering
    \includegraphics[width=.8\textwidth]{./assets/img/replicator_dynamics/main.pdf}
    \caption{Numerical simulation of the replicator equation
    (\ref{eqn:replicator_dynamics}): strategies are ordered by score, only the strategies with a high score survive the evolutionary process.}
    \label{fig:replicator_dynamics}
\end{figure}

This work can be used to classify plays of the IPD\@: data can be collected from
actual interactions (in lab or in the field). Furthermore, this allows for a
classification method similar to the notion of fingerprinting presented
in~\cite{Ashlock2008}. Trained strategies can potentially be classified as
extortionate or not or it could be possible to even constrain the reinforcement
learning approaches that are becoming prevalent in the literature.
Alternatively, this mathematical approach for recognising extortion could be
used in sophisticated strategies to defend against invasion. Arguably, some of
the strategies considered here exhibit this behaviour, indeed as described
in~\cite{Harper2017}, the top ranking strategies in the full tournament are
obtained using evolutionary reinforcement learning techniques, thus, suspicion
of extortionate behaviour could in fact be an evolutionary trait.

\section*{Acknowledgements}

The following open source software libraries were used in this research:

\begin{itemize}
    \item The Axelrod ~\cite{Knight2016, Knight2018} library (IPD strategies and
        tournaments).
    \item The sympy library~\cite{Meurer2017} (verification of all symbolic
        calculations).
    \item The matplotlib~\cite{Droettboom2018} library (visualisation).
    \item The pandas~\cite{Structures2010}, dask~\cite{Dask2016} and
        NumPy~\cite{Oliphant2015} libraries (data manipulation).
    \item The SciPy~\cite{Jones2001} library (numerical integration of the
        replicator equation).
\end{itemize}

This work was performed using the computational facilities of the Advanced
Research Computing @ Cardiff (ARCCA) Division, Cardiff University.

\printbibliography

\newpage
\section*{Supplementary materials}

\includepdf{assets/pdf/proof_of_form_of_extortionate_strategies/main.pdf}

\newpage

Using the pair wise interactions the transition rates \(p,
q\) can be measured and the steady state probabilities inferred and compared to
the actual probabilities of each state.
This is done numerically by computing the singular eigenvector of the
matrix \(A\) \cite{Stewart2009}:

\[
    A =
    \begin{bmatrix}
        p_1 q_1 & p_1 (1 - q_1) & (1 - p_1) q_1 & (1 -p_1) (1 - q_1) \\
        p_2 q_2 & p_2 (1 - q_2) & (1 - p_2) q_2 & (1 -p_2) (1 - q_2) \\
        p_3 q_3 & p_3 (1 - q_3) & (1 - p_3) q_3 & (1 -p_3) (1 - q_3) \\
        p_4 q_4 & p_4 (1 - q_4) & (1 - p_4) q_4 & (1 -p_4) (1 - q_4) \\
    \end{bmatrix}
\]

Figure~\ref{fig:computed_probabilities_vs_theoretic_probabilities} shows a
regression line fitted to every pairwise interaction with a reported
\(\text{SSError}\) value (pairwise interactions with missing states were
omitted). This serves to validate the approach: a part from some edge cases the
relationship is consistent.

\begin{figure}[!htbp]
    \centering
    \includegraphics[width=.8\textwidth]{./assets/img/computed_probabilities_vs_theoretic_probabilities/main.pdf}
    \caption{The
        relationship between the steady state probabilities inferred from the
        measured transitions and the actual steady state probabilities. A linear
        regression line is included validating the approach.}
    \label{fig:computed_probabilities_vs_theoretic_probabilities}
\end{figure}


\end{document}

    strategies is considered. In this setting
    the most highly performing strategies do not play in an extortionate way
    against each other but do against lower performing strategies.
    This suggests that whilst the theory of Zero Determinant strategies
    indicates that memory is not of fundamental importance to the evolution of
    cooperative behaviour, this is incomplete.
\end{abstract}

\section{Introduction}\label{sec:introduction}

Agent based game theoretic models have become a stalwart of the underpinning
mathematics of interactive behaviours. One of the major pieces of work
in this area is the pair of original computer tournaments run by Robert
Axelrod~\cite{Axelrod1980, Axelrod1980a}. These tournaments pitted submitted
computer strategies against each other in plays of the Iterated Prisoner's
Dilemma. A common game where agents can choose to pay a slight cost to their
immediate utility in the hope of building a reputation. This has been used in
economic and evolutionary game theory to understand the evolution of cooperative
behaviour.

Recently, a class of strategies was described in~\cite{Press2012} that can
provably extort any given opponent. In~\cite{Hilbe2013, Moran1707} some
questions have already been asked about the true effectiveness of these
strategies in an evolutionary setting. Here another question is asked: is it
possible to recognise this extortionate behaviour? A mathematical procedure for
suspicion is presented: in the same way that the continued actions of an
extortionate individual might raise suspicion.

This work makes use of the Axelrod Python library~\cite{Knight2018, Knight2016}
with a large number of Prisoner Dilemma strategies available to give an
extensive numerical example of the ideas presented.  The approach is presented
in Section~\ref{sec:delta-zd-strategies}.  All of the code and data discussed
in Section~\ref{sec:numerical-experiments} is open sourced, archived and
written according to best scientific principles~\cite{Wilson2014}. The data
archive can be found at~\cite{vincent_knight_2018_1297075}.

\section{Recognising Extortion}\label{sec:delta-zd-strategies}

In~\cite{Press2012}, given a match between 2 memory-one strategies, the concept
of Zero Determinant (ZD) strategies is introduced. The main result of that paper
shows that given two memory one players \(p, q\in\mathbb{R}^4\) a linear
relationship between the players' scores could be forced by one of the players.

Using the notation of~\cite{Press2012}, assuming the utilities for player \(p\)
are given by \(S_x=(R, S, T, P)\) and for player \(q\) by \(S_y=(R, T, S, P)\)
and that the stationary scores of each player is given by \(S_X\) and \(S_Y\)
respectively. The main result of~\cite{Press2012} is that if

\begin{equation}\label{eqn:linear_relationship_for_p}
    \tilde p=\alpha S_x + \beta S_y + \gamma
\end{equation}

or

\begin{equation}\label{eqn:linear_relationship_for_q}
    \tilde q=\alpha S_x + \beta S_y + \gamma
\end{equation}

where \(\tilde p = (1 - p_1, 1 - p_2, p_3, p_4)\) and
\(\tilde q = (1 - q_1, 1 - q_2, q_3, q_4)\) then:

\begin{equation}
    \alpha S_X + \beta S_Y + \gamma = 0
\end{equation}

In~\cite{Press2012} a particular type of ZD strategy is defined: extortionate
strategies. If:

\begin{equation}\label{eqn:constraint_for_extortion}
    \gamma = - P(\alpha + \beta)
\end{equation}

then the player can ensure they get a score \(\chi\) times
larger than the opponent. This extortion coefficient is given by:

\begin{equation}\label{eqn:definition_of_chi}
    \chi=\frac{-\beta}{\alpha}
\end{equation}

Thus, if (\ref{eqn:constraint_for_extortion}) holds and \(\chi >1\) a player is
said to extort their opponent.
Here, the reverse problem is considered: given a
\(p\in\mathbb{R}^4\) how does one identify \(\alpha, \beta\) if they
exist and is the strategy in fact acting in an extortionate way?

These conditions correspond to:

\begin{align}
    \tilde p_1 & = \alpha R + \beta R - P (\alpha + \beta)
            \label{eqn:condition_for_tilde_p1}\\
    \tilde p_2 & = \alpha S + \beta T - P (\alpha + \beta)
            \label{eqn:condition_for_tilde_p2}\\
    \tilde p_3 & = \alpha T + \beta S - P (\alpha + \beta)
            \label{eqn:condition_for_tilde_p3}\\
    \tilde p_4 & = \alpha P + \beta P - P (\alpha + \beta)
            \label{eqn:condition_for_tilde_p4}
\end{align}

Equation (\ref{eqn:condition_for_tilde_p4}) ensures that \(p_4=\tilde p_4=0\).
Equations (\ref{eqn:condition_for_tilde_p1}-\ref{eqn:condition_for_tilde_p3})
can be used to eliminate \(\alpha, \beta\), giving:

\begin{equation}\label{eqn:planar_definition_of_extortion}
    \tilde p_1 = \frac{(R - P)(\tilde p_2 + \tilde p_3)}{S + T - 2P}
\end{equation}

with:

\begin{equation}\label{eqn:definition_of_chi}
    \chi = \frac{\tilde p_2 (P - T) + \tilde p_3 (S - P)}
                {\tilde p_2 (P - S) + \tilde p_3 (T - P)}
\end{equation}

Given a strategy \(p\in\mathbb{R}^{4\times 1}\) equations
(\ref{eqn:condition_for_tilde_p4}), (\ref{eqn:planar_definition_of_extortion}-\ref{eqn:definition_of_chi}) can be used to check if
a strategy is extortionate. The conditions correspond to:

\begin{align}
    p_1 & = \frac{(R-P)(p_2 + p_3) - R + T + S - P}{S + T - 2P}
     \label{eqn:condition_for_p1}\\
    p_4 & = 0 \label{eqn:condition_for_p4}\\
    1 & > p_2 + p_3\label{eqn:condition_for_chi}
\end{align}

The algebraic steps necessary to prove these results are available in the
supporting materials.

All extortionate strategies reside on a triangular (\ref{eqn:condition_for_chi})
plane (\ref{eqn:condition_for_p1}) in 3 dimensions (\ref{eqn:condition_for_p4}).
Using this formulation it can be seen that a necessary (but not sufficient)
condition for an extortionate strategy is that it cooperates on average less
than 50\% of the time when in a state of disagreement with the opponent.

As an example, consider the known extortionate strategy \(p=(8 / 9, 1 / 2, 1 /
3, 0)\) from~\cite{Stewart2012} which is referred to as \texttt{Extort-2}. In
this case, for the standard values of \((R, T, S, P)\) constraint
(\ref{eqn:condition_for_p1}) corresponds to:

\begin{equation}
    p_1 = \frac{2(p_2 + p_3) + 1}{3}
\end{equation}

It is clear that in this case all constraints hold.

This approach could in fact be used to confirm that a given strategy is acting
in an extortionate manner even if it is not a memory one strategy. However, in
practice, if a closed form for \(p\) is not known, then due to measurement
and/or numerical error this would not work.

This problem can be written in the following linear algebraic form where
\(x=(\alpha, \beta)\)
and \(p^*=(\tilde p_1 - 1, tilde_2 - 1, p_3)\):

\begin{equation}\label{eqn:linear_algebraic_equation_for_p}
    Cx= p^*
\end{equation}

\(C\) corresponds to equations
(\ref{eqn:condition_for_tilde_p1}-\ref{eqn:condition_for_tilde_p3}) and is
given by:

\begin{equation}\label{eqn:definition_of_C}
    C =
    \begin{bmatrix}
        R - P & R- P \\
        S - P & T- P \\
        T - P & S- P \\
    \end{bmatrix}
\end{equation}

Note that in general, equation (\ref{eqn:linear_algebraic_equation_for_p}) will
not necessarily have a solution. From the Rouch\'{e}-Capelli theorem if there is
a solution it is unique as \(\text{rank}(C)=2\) which is the dimension of the
variable \(x\). The best fitting \(x\) is found by minimizing:

\begin{equation}\label{eqn:r_squared}
    \text{SSError} = \|C x- p^*\|_2^2 = \sum_{i=1}^{3}\left((C\bar x)_i-p_i^*\right)^2
\end{equation}

Note that \(\text{SSError}\), which is the square of the Frobenius
norm~\cite{Golub2013}, becomes a measure of how close a strategy is to being an
extortionate strategy. Suspicion
of extortion then corresponds to a threshold on \(\text{SSError}\).

By observing interactions (human or otherwise), their memory one representation
can be inferred and this approach can be used to recognise extortionate
behaviour. The notion of comparing theoretic and actual plays of the IPD is not
novel, see for example~\cite{Rand2013}. Immediately it is noted that if the
environment is noisy~\cite{Wu1995} then no strategy can be considered to be
extortionate as \(p_4>0\).

In the next section, this idea will be illustrated by observing the interactions
that take place in a computer based tournament of the IPD\@.

\section{Numerical experiments}\label{sec:numerical-experiments}

In~\cite{Stewart2012} results from a tournament with
\documentclass[a4paper]{article}

\usepackage{amsmath}
\usepackage{amssymb}
\usepackage[margin=1.5cm,
            includefoot,
            footskip=30pt]{geometry}
\usepackage{layout}
\usepackage{graphicx}
\usepackage{subcaption}

\usepackage{biblatex}
\usepackage{pdfpages}

\bibliography{main.bib}

\title{Suspicion: Recognising and evaluating the effectiveness
       of extortion in the Iterated Prisoner's Dilemma}
\author{Vincent A. Knight \and Nikoleta E. Glynatsi}
\date{\today}



\begin{document}

\maketitle

\begin{abstract}
    The Iterated Prisoner's Dilemma is a model for rational and evolutionary
    interactive behaviour. It has applications both in the study of human social
    behaviour as well as in biology.
    It is used to understand when and how a rational individual might
    accept an immediate cost to their own utility for the direct benefit of
    another.

    Much attention has been given to a class of strategies called
    Zero Determinant strategies. It has been theoretically shown that these
    strategies can ``extort'' any player.

    In this work, an approach to identify if observed strategies are playing in
    an extortionate way is described. Furthermore, experimental analysis of
    a large tournament with \input{assets/tex/number_of_full_strategies/main.tex}
    strategies is considered. In this setting
    the most highly performing strategies do not play in an extortionate way
    against each other but do against lower performing strategies.
    This suggests that whilst the theory of Zero Determinant strategies
    indicates that memory is not of fundamental importance to the evolution of
    cooperative behaviour, this is incomplete.
\end{abstract}

\section{Introduction}\label{sec:introduction}

Agent based game theoretic models have become a stalwart of the underpinning
mathematics of interactive behaviours. One of the major pieces of work
in this area is the pair of original computer tournaments run by Robert
Axelrod~\cite{Axelrod1980, Axelrod1980a}. These tournaments pitted submitted
computer strategies against each other in plays of the Iterated Prisoner's
Dilemma. A common game where agents can choose to pay a slight cost to their
immediate utility in the hope of building a reputation. This has been used in
economic and evolutionary game theory to understand the evolution of cooperative
behaviour.

Recently, a class of strategies was described in~\cite{Press2012} that can
provably extort any given opponent. In~\cite{Hilbe2013, Moran1707} some
questions have already been asked about the true effectiveness of these
strategies in an evolutionary setting. Here another question is asked: is it
possible to recognise this extortionate behaviour? A mathematical procedure for
suspicion is presented: in the same way that the continued actions of an
extortionate individual might raise suspicion.

This work makes use of the Axelrod Python library~\cite{Knight2018, Knight2016}
with a large number of Prisoner Dilemma strategies available to give an
extensive numerical example of the ideas presented.  The approach is presented
in Section~\ref{sec:delta-zd-strategies}.  All of the code and data discussed
in Section~\ref{sec:numerical-experiments} is open sourced, archived and
written according to best scientific principles~\cite{Wilson2014}. The data
archive can be found at~\cite{vincent_knight_2018_1297075}.

\section{Recognising Extortion}\label{sec:delta-zd-strategies}

In~\cite{Press2012}, given a match between 2 memory-one strategies, the concept
of Zero Determinant (ZD) strategies is introduced. The main result of that paper
shows that given two memory one players \(p, q\in\mathbb{R}^4\) a linear
relationship between the players' scores could be forced by one of the players.

Using the notation of~\cite{Press2012}, assuming the utilities for player \(p\)
are given by \(S_x=(R, S, T, P)\) and for player \(q\) by \(S_y=(R, T, S, P)\)
and that the stationary scores of each player is given by \(S_X\) and \(S_Y\)
respectively. The main result of~\cite{Press2012} is that if

\begin{equation}\label{eqn:linear_relationship_for_p}
    \tilde p=\alpha S_x + \beta S_y + \gamma
\end{equation}

or

\begin{equation}\label{eqn:linear_relationship_for_q}
    \tilde q=\alpha S_x + \beta S_y + \gamma
\end{equation}

where \(\tilde p = (1 - p_1, 1 - p_2, p_3, p_4)\) and
\(\tilde q = (1 - q_1, 1 - q_2, q_3, q_4)\) then:

\begin{equation}
    \alpha S_X + \beta S_Y + \gamma = 0
\end{equation}

In~\cite{Press2012} a particular type of ZD strategy is defined: extortionate
strategies. If:

\begin{equation}\label{eqn:constraint_for_extortion}
    \gamma = - P(\alpha + \beta)
\end{equation}

then the player can ensure they get a score \(\chi\) times
larger than the opponent. This extortion coefficient is given by:

\begin{equation}\label{eqn:definition_of_chi}
    \chi=\frac{-\beta}{\alpha}
\end{equation}

Thus, if (\ref{eqn:constraint_for_extortion}) holds and \(\chi >1\) a player is
said to extort their opponent.
Here, the reverse problem is considered: given a
\(p\in\mathbb{R}^4\) how does one identify \(\alpha, \beta\) if they
exist and is the strategy in fact acting in an extortionate way?

These conditions correspond to:

\begin{align}
    \tilde p_1 & = \alpha R + \beta R - P (\alpha + \beta)
            \label{eqn:condition_for_tilde_p1}\\
    \tilde p_2 & = \alpha S + \beta T - P (\alpha + \beta)
            \label{eqn:condition_for_tilde_p2}\\
    \tilde p_3 & = \alpha T + \beta S - P (\alpha + \beta)
            \label{eqn:condition_for_tilde_p3}\\
    \tilde p_4 & = \alpha P + \beta P - P (\alpha + \beta)
            \label{eqn:condition_for_tilde_p4}
\end{align}

Equation (\ref{eqn:condition_for_tilde_p4}) ensures that \(p_4=\tilde p_4=0\).
Equations (\ref{eqn:condition_for_tilde_p1}-\ref{eqn:condition_for_tilde_p3})
can be used to eliminate \(\alpha, \beta\), giving:

\begin{equation}\label{eqn:planar_definition_of_extortion}
    \tilde p_1 = \frac{(R - P)(\tilde p_2 + \tilde p_3)}{S + T - 2P}
\end{equation}

with:

\begin{equation}\label{eqn:definition_of_chi}
    \chi = \frac{\tilde p_2 (P - T) + \tilde p_3 (S - P)}
                {\tilde p_2 (P - S) + \tilde p_3 (T - P)}
\end{equation}

Given a strategy \(p\in\mathbb{R}^{4\times 1}\) equations
(\ref{eqn:condition_for_tilde_p4}), (\ref{eqn:planar_definition_of_extortion}-\ref{eqn:definition_of_chi}) can be used to check if
a strategy is extortionate. The conditions correspond to:

\begin{align}
    p_1 & = \frac{(R-P)(p_2 + p_3) - R + T + S - P}{S + T - 2P}
     \label{eqn:condition_for_p1}\\
    p_4 & = 0 \label{eqn:condition_for_p4}\\
    1 & > p_2 + p_3\label{eqn:condition_for_chi}
\end{align}

The algebraic steps necessary to prove these results are available in the
supporting materials.

All extortionate strategies reside on a triangular (\ref{eqn:condition_for_chi})
plane (\ref{eqn:condition_for_p1}) in 3 dimensions (\ref{eqn:condition_for_p4}).
Using this formulation it can be seen that a necessary (but not sufficient)
condition for an extortionate strategy is that it cooperates on average less
than 50\% of the time when in a state of disagreement with the opponent.

As an example, consider the known extortionate strategy \(p=(8 / 9, 1 / 2, 1 /
3, 0)\) from~\cite{Stewart2012} which is referred to as \texttt{Extort-2}. In
this case, for the standard values of \((R, T, S, P)\) constraint
(\ref{eqn:condition_for_p1}) corresponds to:

\begin{equation}
    p_1 = \frac{2(p_2 + p_3) + 1}{3}
\end{equation}

It is clear that in this case all constraints hold.

This approach could in fact be used to confirm that a given strategy is acting
in an extortionate manner even if it is not a memory one strategy. However, in
practice, if a closed form for \(p\) is not known, then due to measurement
and/or numerical error this would not work.

This problem can be written in the following linear algebraic form where
\(x=(\alpha, \beta)\)
and \(p^*=(\tilde p_1 - 1, tilde_2 - 1, p_3)\):

\begin{equation}\label{eqn:linear_algebraic_equation_for_p}
    Cx= p^*
\end{equation}

\(C\) corresponds to equations
(\ref{eqn:condition_for_tilde_p1}-\ref{eqn:condition_for_tilde_p3}) and is
given by:

\begin{equation}\label{eqn:definition_of_C}
    C =
    \begin{bmatrix}
        R - P & R- P \\
        S - P & T- P \\
        T - P & S- P \\
    \end{bmatrix}
\end{equation}

Note that in general, equation (\ref{eqn:linear_algebraic_equation_for_p}) will
not necessarily have a solution. From the Rouch\'{e}-Capelli theorem if there is
a solution it is unique as \(\text{rank}(C)=2\) which is the dimension of the
variable \(x\). The best fitting \(x\) is found by minimizing:

\begin{equation}\label{eqn:r_squared}
    \text{SSError} = \|C x- p^*\|_2^2 = \sum_{i=1}^{3}\left((C\bar x)_i-p_i^*\right)^2
\end{equation}

Note that \(\text{SSError}\), which is the square of the Frobenius
norm~\cite{Golub2013}, becomes a measure of how close a strategy is to being an
extortionate strategy. Suspicion
of extortion then corresponds to a threshold on \(\text{SSError}\).

By observing interactions (human or otherwise), their memory one representation
can be inferred and this approach can be used to recognise extortionate
behaviour. The notion of comparing theoretic and actual plays of the IPD is not
novel, see for example~\cite{Rand2013}. Immediately it is noted that if the
environment is noisy~\cite{Wu1995} then no strategy can be considered to be
extortionate as \(p_4>0\).

In the next section, this idea will be illustrated by observing the interactions
that take place in a computer based tournament of the IPD\@.

\section{Numerical experiments}\label{sec:numerical-experiments}

In~\cite{Stewart2012} results from a tournament with
\input{./assets/tex/number_of_stewart_plotkin_strategies/main.tex} strategies,
was presented with specific consideration given to ZD strategies. This
tournament is reproduced here using the Axelrod-Python
project~\cite{Knight2016}. To obtain a good measure of the corresponding
transition rates for each strategy all matches have been run for
\input{assets/tex/number_of_turns/main.tex} turns and every match has been
repeated \input{assets/tex/number_of_repetitions/main.tex} times. All of this
interaction data is available at~\cite{vincent_knight_2018_1297075}. A good
match between the inferred Markov chain and the state distribution of the actual
interactions has been verified. Data for this is presented in the supplementary
materials.

Figure~\ref{fig:SSError_overall_in_stewart_plotkin} shows the \(\text{SSError}\)
values for all the strategies in the tournament, as reported
in~\cite{Stewart2012} the extortionate strategy (which has an expected
\(\text{SSError}\) approximately 0) gains a large number of wins.

\begin{figure}[!htbp]
    \centering
    \includegraphics[width=.8\textwidth]{./assets/img/SSError_overall_in_stewart_plotkin/main.pdf}
    \caption{\(\text{SSError}\) and state probabilities for the strategies
        of~\cite{Stewart2012}, ordered both by number of wins and overall score.
        Note that \(P(DC)\) is not shown as it corresponds to the transpose of
        \(P(CD)\). Cooperator and Defector are omitted as they do not visit all
        the states.}
    \label{fig:SSError_overall_in_stewart_plotkin}
\end{figure}

Here, the work of~\cite{Stewart2012} is extended by investigating a tournament
with \input{assets/tex/number_of_full_strategies/main.tex}
strategies.

The results of this analysis are shown in
Figure~\ref{fig:SSError_and_probabilities_in_full}. The top ranking strategies
by number of wins seem to be extortionate (but not against all strategies) and
it can be seen that a small sub group of strategies achieve mutual defection.
All the top ranking strategies according to score achieve mutual cooperation and
do not extort each other, however they
\textbf{do} exhibit extortionate behaviour towards a number of the lower ranking
strategies.

\begin{figure}[!htbp]
    \centering
    \includegraphics[width=.8\textwidth]{./assets/img/SSError_and_probabilities_in_full/main.pdf}
    \caption{\(\text{SSError}\) for the strategies for the full tournament. Only
    strategy interactions for which \(p_4=0\) and \(\chi>1\) are displayed.}
    \label{fig:SSError_and_probabilities_in_full}
\end{figure}

\section{Conclusion}\label{sec:conclusion}

This work defines an approach to measure whether or not a player is playing a
strategy that corresponds to an extortionate strategy as defined
in~\cite{Press2012}: a mathematical model for suspicion. Indeed, all
extortionate strategies have been
 classified as lying on a triangular plane.
This rigorous classification fails to be robust to small measurement error, thus
a statistical approach is proposed.
This is done through a linear algebraic approach for approximating the solution
of a linear system. Using this, a large number of pairwise interactions is
simulated and in fact very few strategies are found to act extortionately.

The work of~\cite{Press2012}, whilst showing that a clever approach to taking
advantage of another memory one strategy exists: this is incomplete. Whilst the
elegance of this result is very attractive, just as the simplicity of the
victory of Tit For Tat in Axelrod's original tournaments was, it is incomplete.
Extortionate strategies achieve a high number of wins but they do not
achieve a high score which corresponds to the fitness landscape in an
evolutionary sense. From the large number of interactions a payoff matrix \(S\)
can be measured where \(S_{ij}\) denotes the score (using standard values of
\((R, S, T, P) = (3, 0, 5, 1)\)) of the \(i\)th strategy
against the \(j\)th strategy. Using this, the replicator equation
describes the evolution of the system based on a population density fitness
function:

\begin{equation}\label{eqn:replicator_dynamics}
    \frac{dx}{dt} = x(S-x^TS x)
\end{equation}

Equation (\ref{eqn:replicator_dynamics}) is solved numerically through an
integration technique described in~\cite{Petzold1983} and
Figure~\ref{fig:replicator_dynamics} shows the evolution of the distribution of
the system: the various strategies are ranked by scores. It is clear to see that
only the high ranking strategies survive the evolutionary process (in fact,
only \input{./assets/img/replicator_dynamics/main.tex}
have a final distribution greater than \(10 ^ {-2}\)). This confirms the
findings of~\cite{Moran1707} in which sophisticated strategies resist
evolutionary invasion of shorter memory strategies. Recalling
Figure~\ref{fig:SSError_and_probabilities_in_full} this demonstrates that:

\begin{itemize}
    \item Cooperation emerges through the evolutionary process: the high scoring
        strategies do not exhibit extortionate behaviour towards each other.
    \item Extortionate strategies do not survive the evolutionary process.
\end{itemize}

\begin{figure}[!htbp]
    \centering
    \includegraphics[width=.8\textwidth]{./assets/img/replicator_dynamics/main.pdf}
    \caption{Numerical simulation of the replicator equation
    (\ref{eqn:replicator_dynamics}): strategies are ordered by score, only the strategies with a high score survive the evolutionary process.}
    \label{fig:replicator_dynamics}
\end{figure}

This work can be used to classify plays of the IPD\@: data can be collected from
actual interactions (in lab or in the field). Furthermore, this allows for a
classification method similar to the notion of fingerprinting presented
in~\cite{Ashlock2008}. Trained strategies can potentially be classified as
extortionate or not or it could be possible to even constrain the reinforcement
learning approaches that are becoming prevalent in the literature.
Alternatively, this mathematical approach for recognising extortion could be
used in sophisticated strategies to defend against invasion. Arguably, some of
the strategies considered here exhibit this behaviour, indeed as described
in~\cite{Harper2017}, the top ranking strategies in the full tournament are
obtained using evolutionary reinforcement learning techniques, thus, suspicion
of extortionate behaviour could in fact be an evolutionary trait.

\section*{Acknowledgements}

The following open source software libraries were used in this research:

\begin{itemize}
    \item The Axelrod ~\cite{Knight2016, Knight2018} library (IPD strategies and
        tournaments).
    \item The sympy library~\cite{Meurer2017} (verification of all symbolic
        calculations).
    \item The matplotlib~\cite{Droettboom2018} library (visualisation).
    \item The pandas~\cite{Structures2010}, dask~\cite{Dask2016} and
        NumPy~\cite{Oliphant2015} libraries (data manipulation).
    \item The SciPy~\cite{Jones2001} library (numerical integration of the
        replicator equation).
\end{itemize}

This work was performed using the computational facilities of the Advanced
Research Computing @ Cardiff (ARCCA) Division, Cardiff University.

\printbibliography

\newpage
\section*{Supplementary materials}

\includepdf{assets/pdf/proof_of_form_of_extortionate_strategies/main.pdf}

\newpage

Using the pair wise interactions the transition rates \(p,
q\) can be measured and the steady state probabilities inferred and compared to
the actual probabilities of each state.
This is done numerically by computing the singular eigenvector of the
matrix \(A\) \cite{Stewart2009}:

\[
    A =
    \begin{bmatrix}
        p_1 q_1 & p_1 (1 - q_1) & (1 - p_1) q_1 & (1 -p_1) (1 - q_1) \\
        p_2 q_2 & p_2 (1 - q_2) & (1 - p_2) q_2 & (1 -p_2) (1 - q_2) \\
        p_3 q_3 & p_3 (1 - q_3) & (1 - p_3) q_3 & (1 -p_3) (1 - q_3) \\
        p_4 q_4 & p_4 (1 - q_4) & (1 - p_4) q_4 & (1 -p_4) (1 - q_4) \\
    \end{bmatrix}
\]

Figure~\ref{fig:computed_probabilities_vs_theoretic_probabilities} shows a
regression line fitted to every pairwise interaction with a reported
\(\text{SSError}\) value (pairwise interactions with missing states were
omitted). This serves to validate the approach: a part from some edge cases the
relationship is consistent.

\begin{figure}[!htbp]
    \centering
    \includegraphics[width=.8\textwidth]{./assets/img/computed_probabilities_vs_theoretic_probabilities/main.pdf}
    \caption{The
        relationship between the steady state probabilities inferred from the
        measured transitions and the actual steady state probabilities. A linear
        regression line is included validating the approach.}
    \label{fig:computed_probabilities_vs_theoretic_probabilities}
\end{figure}


\end{document}
 strategies,
was presented with specific consideration given to ZD strategies. This
tournament is reproduced here using the Axelrod-Python
project~\cite{Knight2016}. To obtain a good measure of the corresponding
transition rates for each strategy all matches have been run for
\documentclass[a4paper]{article}

\usepackage{amsmath}
\usepackage{amssymb}
\usepackage[margin=1.5cm,
            includefoot,
            footskip=30pt]{geometry}
\usepackage{layout}
\usepackage{graphicx}
\usepackage{subcaption}

\usepackage{biblatex}
\usepackage{pdfpages}

\bibliography{main.bib}

\title{Suspicion: Recognising and evaluating the effectiveness
       of extortion in the Iterated Prisoner's Dilemma}
\author{Vincent A. Knight \and Nikoleta E. Glynatsi}
\date{\today}



\begin{document}

\maketitle

\begin{abstract}
    The Iterated Prisoner's Dilemma is a model for rational and evolutionary
    interactive behaviour. It has applications both in the study of human social
    behaviour as well as in biology.
    It is used to understand when and how a rational individual might
    accept an immediate cost to their own utility for the direct benefit of
    another.

    Much attention has been given to a class of strategies called
    Zero Determinant strategies. It has been theoretically shown that these
    strategies can ``extort'' any player.

    In this work, an approach to identify if observed strategies are playing in
    an extortionate way is described. Furthermore, experimental analysis of
    a large tournament with \input{assets/tex/number_of_full_strategies/main.tex}
    strategies is considered. In this setting
    the most highly performing strategies do not play in an extortionate way
    against each other but do against lower performing strategies.
    This suggests that whilst the theory of Zero Determinant strategies
    indicates that memory is not of fundamental importance to the evolution of
    cooperative behaviour, this is incomplete.
\end{abstract}

\section{Introduction}\label{sec:introduction}

Agent based game theoretic models have become a stalwart of the underpinning
mathematics of interactive behaviours. One of the major pieces of work
in this area is the pair of original computer tournaments run by Robert
Axelrod~\cite{Axelrod1980, Axelrod1980a}. These tournaments pitted submitted
computer strategies against each other in plays of the Iterated Prisoner's
Dilemma. A common game where agents can choose to pay a slight cost to their
immediate utility in the hope of building a reputation. This has been used in
economic and evolutionary game theory to understand the evolution of cooperative
behaviour.

Recently, a class of strategies was described in~\cite{Press2012} that can
provably extort any given opponent. In~\cite{Hilbe2013, Moran1707} some
questions have already been asked about the true effectiveness of these
strategies in an evolutionary setting. Here another question is asked: is it
possible to recognise this extortionate behaviour? A mathematical procedure for
suspicion is presented: in the same way that the continued actions of an
extortionate individual might raise suspicion.

This work makes use of the Axelrod Python library~\cite{Knight2018, Knight2016}
with a large number of Prisoner Dilemma strategies available to give an
extensive numerical example of the ideas presented.  The approach is presented
in Section~\ref{sec:delta-zd-strategies}.  All of the code and data discussed
in Section~\ref{sec:numerical-experiments} is open sourced, archived and
written according to best scientific principles~\cite{Wilson2014}. The data
archive can be found at~\cite{vincent_knight_2018_1297075}.

\section{Recognising Extortion}\label{sec:delta-zd-strategies}

In~\cite{Press2012}, given a match between 2 memory-one strategies, the concept
of Zero Determinant (ZD) strategies is introduced. The main result of that paper
shows that given two memory one players \(p, q\in\mathbb{R}^4\) a linear
relationship between the players' scores could be forced by one of the players.

Using the notation of~\cite{Press2012}, assuming the utilities for player \(p\)
are given by \(S_x=(R, S, T, P)\) and for player \(q\) by \(S_y=(R, T, S, P)\)
and that the stationary scores of each player is given by \(S_X\) and \(S_Y\)
respectively. The main result of~\cite{Press2012} is that if

\begin{equation}\label{eqn:linear_relationship_for_p}
    \tilde p=\alpha S_x + \beta S_y + \gamma
\end{equation}

or

\begin{equation}\label{eqn:linear_relationship_for_q}
    \tilde q=\alpha S_x + \beta S_y + \gamma
\end{equation}

where \(\tilde p = (1 - p_1, 1 - p_2, p_3, p_4)\) and
\(\tilde q = (1 - q_1, 1 - q_2, q_3, q_4)\) then:

\begin{equation}
    \alpha S_X + \beta S_Y + \gamma = 0
\end{equation}

In~\cite{Press2012} a particular type of ZD strategy is defined: extortionate
strategies. If:

\begin{equation}\label{eqn:constraint_for_extortion}
    \gamma = - P(\alpha + \beta)
\end{equation}

then the player can ensure they get a score \(\chi\) times
larger than the opponent. This extortion coefficient is given by:

\begin{equation}\label{eqn:definition_of_chi}
    \chi=\frac{-\beta}{\alpha}
\end{equation}

Thus, if (\ref{eqn:constraint_for_extortion}) holds and \(\chi >1\) a player is
said to extort their opponent.
Here, the reverse problem is considered: given a
\(p\in\mathbb{R}^4\) how does one identify \(\alpha, \beta\) if they
exist and is the strategy in fact acting in an extortionate way?

These conditions correspond to:

\begin{align}
    \tilde p_1 & = \alpha R + \beta R - P (\alpha + \beta)
            \label{eqn:condition_for_tilde_p1}\\
    \tilde p_2 & = \alpha S + \beta T - P (\alpha + \beta)
            \label{eqn:condition_for_tilde_p2}\\
    \tilde p_3 & = \alpha T + \beta S - P (\alpha + \beta)
            \label{eqn:condition_for_tilde_p3}\\
    \tilde p_4 & = \alpha P + \beta P - P (\alpha + \beta)
            \label{eqn:condition_for_tilde_p4}
\end{align}

Equation (\ref{eqn:condition_for_tilde_p4}) ensures that \(p_4=\tilde p_4=0\).
Equations (\ref{eqn:condition_for_tilde_p1}-\ref{eqn:condition_for_tilde_p3})
can be used to eliminate \(\alpha, \beta\), giving:

\begin{equation}\label{eqn:planar_definition_of_extortion}
    \tilde p_1 = \frac{(R - P)(\tilde p_2 + \tilde p_3)}{S + T - 2P}
\end{equation}

with:

\begin{equation}\label{eqn:definition_of_chi}
    \chi = \frac{\tilde p_2 (P - T) + \tilde p_3 (S - P)}
                {\tilde p_2 (P - S) + \tilde p_3 (T - P)}
\end{equation}

Given a strategy \(p\in\mathbb{R}^{4\times 1}\) equations
(\ref{eqn:condition_for_tilde_p4}), (\ref{eqn:planar_definition_of_extortion}-\ref{eqn:definition_of_chi}) can be used to check if
a strategy is extortionate. The conditions correspond to:

\begin{align}
    p_1 & = \frac{(R-P)(p_2 + p_3) - R + T + S - P}{S + T - 2P}
     \label{eqn:condition_for_p1}\\
    p_4 & = 0 \label{eqn:condition_for_p4}\\
    1 & > p_2 + p_3\label{eqn:condition_for_chi}
\end{align}

The algebraic steps necessary to prove these results are available in the
supporting materials.

All extortionate strategies reside on a triangular (\ref{eqn:condition_for_chi})
plane (\ref{eqn:condition_for_p1}) in 3 dimensions (\ref{eqn:condition_for_p4}).
Using this formulation it can be seen that a necessary (but not sufficient)
condition for an extortionate strategy is that it cooperates on average less
than 50\% of the time when in a state of disagreement with the opponent.

As an example, consider the known extortionate strategy \(p=(8 / 9, 1 / 2, 1 /
3, 0)\) from~\cite{Stewart2012} which is referred to as \texttt{Extort-2}. In
this case, for the standard values of \((R, T, S, P)\) constraint
(\ref{eqn:condition_for_p1}) corresponds to:

\begin{equation}
    p_1 = \frac{2(p_2 + p_3) + 1}{3}
\end{equation}

It is clear that in this case all constraints hold.

This approach could in fact be used to confirm that a given strategy is acting
in an extortionate manner even if it is not a memory one strategy. However, in
practice, if a closed form for \(p\) is not known, then due to measurement
and/or numerical error this would not work.

This problem can be written in the following linear algebraic form where
\(x=(\alpha, \beta)\)
and \(p^*=(\tilde p_1 - 1, tilde_2 - 1, p_3)\):

\begin{equation}\label{eqn:linear_algebraic_equation_for_p}
    Cx= p^*
\end{equation}

\(C\) corresponds to equations
(\ref{eqn:condition_for_tilde_p1}-\ref{eqn:condition_for_tilde_p3}) and is
given by:

\begin{equation}\label{eqn:definition_of_C}
    C =
    \begin{bmatrix}
        R - P & R- P \\
        S - P & T- P \\
        T - P & S- P \\
    \end{bmatrix}
\end{equation}

Note that in general, equation (\ref{eqn:linear_algebraic_equation_for_p}) will
not necessarily have a solution. From the Rouch\'{e}-Capelli theorem if there is
a solution it is unique as \(\text{rank}(C)=2\) which is the dimension of the
variable \(x\). The best fitting \(x\) is found by minimizing:

\begin{equation}\label{eqn:r_squared}
    \text{SSError} = \|C x- p^*\|_2^2 = \sum_{i=1}^{3}\left((C\bar x)_i-p_i^*\right)^2
\end{equation}

Note that \(\text{SSError}\), which is the square of the Frobenius
norm~\cite{Golub2013}, becomes a measure of how close a strategy is to being an
extortionate strategy. Suspicion
of extortion then corresponds to a threshold on \(\text{SSError}\).

By observing interactions (human or otherwise), their memory one representation
can be inferred and this approach can be used to recognise extortionate
behaviour. The notion of comparing theoretic and actual plays of the IPD is not
novel, see for example~\cite{Rand2013}. Immediately it is noted that if the
environment is noisy~\cite{Wu1995} then no strategy can be considered to be
extortionate as \(p_4>0\).

In the next section, this idea will be illustrated by observing the interactions
that take place in a computer based tournament of the IPD\@.

\section{Numerical experiments}\label{sec:numerical-experiments}

In~\cite{Stewart2012} results from a tournament with
\input{./assets/tex/number_of_stewart_plotkin_strategies/main.tex} strategies,
was presented with specific consideration given to ZD strategies. This
tournament is reproduced here using the Axelrod-Python
project~\cite{Knight2016}. To obtain a good measure of the corresponding
transition rates for each strategy all matches have been run for
\input{assets/tex/number_of_turns/main.tex} turns and every match has been
repeated \input{assets/tex/number_of_repetitions/main.tex} times. All of this
interaction data is available at~\cite{vincent_knight_2018_1297075}. A good
match between the inferred Markov chain and the state distribution of the actual
interactions has been verified. Data for this is presented in the supplementary
materials.

Figure~\ref{fig:SSError_overall_in_stewart_plotkin} shows the \(\text{SSError}\)
values for all the strategies in the tournament, as reported
in~\cite{Stewart2012} the extortionate strategy (which has an expected
\(\text{SSError}\) approximately 0) gains a large number of wins.

\begin{figure}[!htbp]
    \centering
    \includegraphics[width=.8\textwidth]{./assets/img/SSError_overall_in_stewart_plotkin/main.pdf}
    \caption{\(\text{SSError}\) and state probabilities for the strategies
        of~\cite{Stewart2012}, ordered both by number of wins and overall score.
        Note that \(P(DC)\) is not shown as it corresponds to the transpose of
        \(P(CD)\). Cooperator and Defector are omitted as they do not visit all
        the states.}
    \label{fig:SSError_overall_in_stewart_plotkin}
\end{figure}

Here, the work of~\cite{Stewart2012} is extended by investigating a tournament
with \input{assets/tex/number_of_full_strategies/main.tex}
strategies.

The results of this analysis are shown in
Figure~\ref{fig:SSError_and_probabilities_in_full}. The top ranking strategies
by number of wins seem to be extortionate (but not against all strategies) and
it can be seen that a small sub group of strategies achieve mutual defection.
All the top ranking strategies according to score achieve mutual cooperation and
do not extort each other, however they
\textbf{do} exhibit extortionate behaviour towards a number of the lower ranking
strategies.

\begin{figure}[!htbp]
    \centering
    \includegraphics[width=.8\textwidth]{./assets/img/SSError_and_probabilities_in_full/main.pdf}
    \caption{\(\text{SSError}\) for the strategies for the full tournament. Only
    strategy interactions for which \(p_4=0\) and \(\chi>1\) are displayed.}
    \label{fig:SSError_and_probabilities_in_full}
\end{figure}

\section{Conclusion}\label{sec:conclusion}

This work defines an approach to measure whether or not a player is playing a
strategy that corresponds to an extortionate strategy as defined
in~\cite{Press2012}: a mathematical model for suspicion. Indeed, all
extortionate strategies have been
 classified as lying on a triangular plane.
This rigorous classification fails to be robust to small measurement error, thus
a statistical approach is proposed.
This is done through a linear algebraic approach for approximating the solution
of a linear system. Using this, a large number of pairwise interactions is
simulated and in fact very few strategies are found to act extortionately.

The work of~\cite{Press2012}, whilst showing that a clever approach to taking
advantage of another memory one strategy exists: this is incomplete. Whilst the
elegance of this result is very attractive, just as the simplicity of the
victory of Tit For Tat in Axelrod's original tournaments was, it is incomplete.
Extortionate strategies achieve a high number of wins but they do not
achieve a high score which corresponds to the fitness landscape in an
evolutionary sense. From the large number of interactions a payoff matrix \(S\)
can be measured where \(S_{ij}\) denotes the score (using standard values of
\((R, S, T, P) = (3, 0, 5, 1)\)) of the \(i\)th strategy
against the \(j\)th strategy. Using this, the replicator equation
describes the evolution of the system based on a population density fitness
function:

\begin{equation}\label{eqn:replicator_dynamics}
    \frac{dx}{dt} = x(S-x^TS x)
\end{equation}

Equation (\ref{eqn:replicator_dynamics}) is solved numerically through an
integration technique described in~\cite{Petzold1983} and
Figure~\ref{fig:replicator_dynamics} shows the evolution of the distribution of
the system: the various strategies are ranked by scores. It is clear to see that
only the high ranking strategies survive the evolutionary process (in fact,
only \input{./assets/img/replicator_dynamics/main.tex}
have a final distribution greater than \(10 ^ {-2}\)). This confirms the
findings of~\cite{Moran1707} in which sophisticated strategies resist
evolutionary invasion of shorter memory strategies. Recalling
Figure~\ref{fig:SSError_and_probabilities_in_full} this demonstrates that:

\begin{itemize}
    \item Cooperation emerges through the evolutionary process: the high scoring
        strategies do not exhibit extortionate behaviour towards each other.
    \item Extortionate strategies do not survive the evolutionary process.
\end{itemize}

\begin{figure}[!htbp]
    \centering
    \includegraphics[width=.8\textwidth]{./assets/img/replicator_dynamics/main.pdf}
    \caption{Numerical simulation of the replicator equation
    (\ref{eqn:replicator_dynamics}): strategies are ordered by score, only the strategies with a high score survive the evolutionary process.}
    \label{fig:replicator_dynamics}
\end{figure}

This work can be used to classify plays of the IPD\@: data can be collected from
actual interactions (in lab or in the field). Furthermore, this allows for a
classification method similar to the notion of fingerprinting presented
in~\cite{Ashlock2008}. Trained strategies can potentially be classified as
extortionate or not or it could be possible to even constrain the reinforcement
learning approaches that are becoming prevalent in the literature.
Alternatively, this mathematical approach for recognising extortion could be
used in sophisticated strategies to defend against invasion. Arguably, some of
the strategies considered here exhibit this behaviour, indeed as described
in~\cite{Harper2017}, the top ranking strategies in the full tournament are
obtained using evolutionary reinforcement learning techniques, thus, suspicion
of extortionate behaviour could in fact be an evolutionary trait.

\section*{Acknowledgements}

The following open source software libraries were used in this research:

\begin{itemize}
    \item The Axelrod ~\cite{Knight2016, Knight2018} library (IPD strategies and
        tournaments).
    \item The sympy library~\cite{Meurer2017} (verification of all symbolic
        calculations).
    \item The matplotlib~\cite{Droettboom2018} library (visualisation).
    \item The pandas~\cite{Structures2010}, dask~\cite{Dask2016} and
        NumPy~\cite{Oliphant2015} libraries (data manipulation).
    \item The SciPy~\cite{Jones2001} library (numerical integration of the
        replicator equation).
\end{itemize}

This work was performed using the computational facilities of the Advanced
Research Computing @ Cardiff (ARCCA) Division, Cardiff University.

\printbibliography

\newpage
\section*{Supplementary materials}

\includepdf{assets/pdf/proof_of_form_of_extortionate_strategies/main.pdf}

\newpage

Using the pair wise interactions the transition rates \(p,
q\) can be measured and the steady state probabilities inferred and compared to
the actual probabilities of each state.
This is done numerically by computing the singular eigenvector of the
matrix \(A\) \cite{Stewart2009}:

\[
    A =
    \begin{bmatrix}
        p_1 q_1 & p_1 (1 - q_1) & (1 - p_1) q_1 & (1 -p_1) (1 - q_1) \\
        p_2 q_2 & p_2 (1 - q_2) & (1 - p_2) q_2 & (1 -p_2) (1 - q_2) \\
        p_3 q_3 & p_3 (1 - q_3) & (1 - p_3) q_3 & (1 -p_3) (1 - q_3) \\
        p_4 q_4 & p_4 (1 - q_4) & (1 - p_4) q_4 & (1 -p_4) (1 - q_4) \\
    \end{bmatrix}
\]

Figure~\ref{fig:computed_probabilities_vs_theoretic_probabilities} shows a
regression line fitted to every pairwise interaction with a reported
\(\text{SSError}\) value (pairwise interactions with missing states were
omitted). This serves to validate the approach: a part from some edge cases the
relationship is consistent.

\begin{figure}[!htbp]
    \centering
    \includegraphics[width=.8\textwidth]{./assets/img/computed_probabilities_vs_theoretic_probabilities/main.pdf}
    \caption{The
        relationship between the steady state probabilities inferred from the
        measured transitions and the actual steady state probabilities. A linear
        regression line is included validating the approach.}
    \label{fig:computed_probabilities_vs_theoretic_probabilities}
\end{figure}


\end{document}
 turns and every match has been
repeated \documentclass[a4paper]{article}

\usepackage{amsmath}
\usepackage{amssymb}
\usepackage[margin=1.5cm,
            includefoot,
            footskip=30pt]{geometry}
\usepackage{layout}
\usepackage{graphicx}
\usepackage{subcaption}

\usepackage{biblatex}
\usepackage{pdfpages}

\bibliography{main.bib}

\title{Suspicion: Recognising and evaluating the effectiveness
       of extortion in the Iterated Prisoner's Dilemma}
\author{Vincent A. Knight \and Nikoleta E. Glynatsi}
\date{\today}



\begin{document}

\maketitle

\begin{abstract}
    The Iterated Prisoner's Dilemma is a model for rational and evolutionary
    interactive behaviour. It has applications both in the study of human social
    behaviour as well as in biology.
    It is used to understand when and how a rational individual might
    accept an immediate cost to their own utility for the direct benefit of
    another.

    Much attention has been given to a class of strategies called
    Zero Determinant strategies. It has been theoretically shown that these
    strategies can ``extort'' any player.

    In this work, an approach to identify if observed strategies are playing in
    an extortionate way is described. Furthermore, experimental analysis of
    a large tournament with \input{assets/tex/number_of_full_strategies/main.tex}
    strategies is considered. In this setting
    the most highly performing strategies do not play in an extortionate way
    against each other but do against lower performing strategies.
    This suggests that whilst the theory of Zero Determinant strategies
    indicates that memory is not of fundamental importance to the evolution of
    cooperative behaviour, this is incomplete.
\end{abstract}

\section{Introduction}\label{sec:introduction}

Agent based game theoretic models have become a stalwart of the underpinning
mathematics of interactive behaviours. One of the major pieces of work
in this area is the pair of original computer tournaments run by Robert
Axelrod~\cite{Axelrod1980, Axelrod1980a}. These tournaments pitted submitted
computer strategies against each other in plays of the Iterated Prisoner's
Dilemma. A common game where agents can choose to pay a slight cost to their
immediate utility in the hope of building a reputation. This has been used in
economic and evolutionary game theory to understand the evolution of cooperative
behaviour.

Recently, a class of strategies was described in~\cite{Press2012} that can
provably extort any given opponent. In~\cite{Hilbe2013, Moran1707} some
questions have already been asked about the true effectiveness of these
strategies in an evolutionary setting. Here another question is asked: is it
possible to recognise this extortionate behaviour? A mathematical procedure for
suspicion is presented: in the same way that the continued actions of an
extortionate individual might raise suspicion.

This work makes use of the Axelrod Python library~\cite{Knight2018, Knight2016}
with a large number of Prisoner Dilemma strategies available to give an
extensive numerical example of the ideas presented.  The approach is presented
in Section~\ref{sec:delta-zd-strategies}.  All of the code and data discussed
in Section~\ref{sec:numerical-experiments} is open sourced, archived and
written according to best scientific principles~\cite{Wilson2014}. The data
archive can be found at~\cite{vincent_knight_2018_1297075}.

\section{Recognising Extortion}\label{sec:delta-zd-strategies}

In~\cite{Press2012}, given a match between 2 memory-one strategies, the concept
of Zero Determinant (ZD) strategies is introduced. The main result of that paper
shows that given two memory one players \(p, q\in\mathbb{R}^4\) a linear
relationship between the players' scores could be forced by one of the players.

Using the notation of~\cite{Press2012}, assuming the utilities for player \(p\)
are given by \(S_x=(R, S, T, P)\) and for player \(q\) by \(S_y=(R, T, S, P)\)
and that the stationary scores of each player is given by \(S_X\) and \(S_Y\)
respectively. The main result of~\cite{Press2012} is that if

\begin{equation}\label{eqn:linear_relationship_for_p}
    \tilde p=\alpha S_x + \beta S_y + \gamma
\end{equation}

or

\begin{equation}\label{eqn:linear_relationship_for_q}
    \tilde q=\alpha S_x + \beta S_y + \gamma
\end{equation}

where \(\tilde p = (1 - p_1, 1 - p_2, p_3, p_4)\) and
\(\tilde q = (1 - q_1, 1 - q_2, q_3, q_4)\) then:

\begin{equation}
    \alpha S_X + \beta S_Y + \gamma = 0
\end{equation}

In~\cite{Press2012} a particular type of ZD strategy is defined: extortionate
strategies. If:

\begin{equation}\label{eqn:constraint_for_extortion}
    \gamma = - P(\alpha + \beta)
\end{equation}

then the player can ensure they get a score \(\chi\) times
larger than the opponent. This extortion coefficient is given by:

\begin{equation}\label{eqn:definition_of_chi}
    \chi=\frac{-\beta}{\alpha}
\end{equation}

Thus, if (\ref{eqn:constraint_for_extortion}) holds and \(\chi >1\) a player is
said to extort their opponent.
Here, the reverse problem is considered: given a
\(p\in\mathbb{R}^4\) how does one identify \(\alpha, \beta\) if they
exist and is the strategy in fact acting in an extortionate way?

These conditions correspond to:

\begin{align}
    \tilde p_1 & = \alpha R + \beta R - P (\alpha + \beta)
            \label{eqn:condition_for_tilde_p1}\\
    \tilde p_2 & = \alpha S + \beta T - P (\alpha + \beta)
            \label{eqn:condition_for_tilde_p2}\\
    \tilde p_3 & = \alpha T + \beta S - P (\alpha + \beta)
            \label{eqn:condition_for_tilde_p3}\\
    \tilde p_4 & = \alpha P + \beta P - P (\alpha + \beta)
            \label{eqn:condition_for_tilde_p4}
\end{align}

Equation (\ref{eqn:condition_for_tilde_p4}) ensures that \(p_4=\tilde p_4=0\).
Equations (\ref{eqn:condition_for_tilde_p1}-\ref{eqn:condition_for_tilde_p3})
can be used to eliminate \(\alpha, \beta\), giving:

\begin{equation}\label{eqn:planar_definition_of_extortion}
    \tilde p_1 = \frac{(R - P)(\tilde p_2 + \tilde p_3)}{S + T - 2P}
\end{equation}

with:

\begin{equation}\label{eqn:definition_of_chi}
    \chi = \frac{\tilde p_2 (P - T) + \tilde p_3 (S - P)}
                {\tilde p_2 (P - S) + \tilde p_3 (T - P)}
\end{equation}

Given a strategy \(p\in\mathbb{R}^{4\times 1}\) equations
(\ref{eqn:condition_for_tilde_p4}), (\ref{eqn:planar_definition_of_extortion}-\ref{eqn:definition_of_chi}) can be used to check if
a strategy is extortionate. The conditions correspond to:

\begin{align}
    p_1 & = \frac{(R-P)(p_2 + p_3) - R + T + S - P}{S + T - 2P}
     \label{eqn:condition_for_p1}\\
    p_4 & = 0 \label{eqn:condition_for_p4}\\
    1 & > p_2 + p_3\label{eqn:condition_for_chi}
\end{align}

The algebraic steps necessary to prove these results are available in the
supporting materials.

All extortionate strategies reside on a triangular (\ref{eqn:condition_for_chi})
plane (\ref{eqn:condition_for_p1}) in 3 dimensions (\ref{eqn:condition_for_p4}).
Using this formulation it can be seen that a necessary (but not sufficient)
condition for an extortionate strategy is that it cooperates on average less
than 50\% of the time when in a state of disagreement with the opponent.

As an example, consider the known extortionate strategy \(p=(8 / 9, 1 / 2, 1 /
3, 0)\) from~\cite{Stewart2012} which is referred to as \texttt{Extort-2}. In
this case, for the standard values of \((R, T, S, P)\) constraint
(\ref{eqn:condition_for_p1}) corresponds to:

\begin{equation}
    p_1 = \frac{2(p_2 + p_3) + 1}{3}
\end{equation}

It is clear that in this case all constraints hold.

This approach could in fact be used to confirm that a given strategy is acting
in an extortionate manner even if it is not a memory one strategy. However, in
practice, if a closed form for \(p\) is not known, then due to measurement
and/or numerical error this would not work.

This problem can be written in the following linear algebraic form where
\(x=(\alpha, \beta)\)
and \(p^*=(\tilde p_1 - 1, tilde_2 - 1, p_3)\):

\begin{equation}\label{eqn:linear_algebraic_equation_for_p}
    Cx= p^*
\end{equation}

\(C\) corresponds to equations
(\ref{eqn:condition_for_tilde_p1}-\ref{eqn:condition_for_tilde_p3}) and is
given by:

\begin{equation}\label{eqn:definition_of_C}
    C =
    \begin{bmatrix}
        R - P & R- P \\
        S - P & T- P \\
        T - P & S- P \\
    \end{bmatrix}
\end{equation}

Note that in general, equation (\ref{eqn:linear_algebraic_equation_for_p}) will
not necessarily have a solution. From the Rouch\'{e}-Capelli theorem if there is
a solution it is unique as \(\text{rank}(C)=2\) which is the dimension of the
variable \(x\). The best fitting \(x\) is found by minimizing:

\begin{equation}\label{eqn:r_squared}
    \text{SSError} = \|C x- p^*\|_2^2 = \sum_{i=1}^{3}\left((C\bar x)_i-p_i^*\right)^2
\end{equation}

Note that \(\text{SSError}\), which is the square of the Frobenius
norm~\cite{Golub2013}, becomes a measure of how close a strategy is to being an
extortionate strategy. Suspicion
of extortion then corresponds to a threshold on \(\text{SSError}\).

By observing interactions (human or otherwise), their memory one representation
can be inferred and this approach can be used to recognise extortionate
behaviour. The notion of comparing theoretic and actual plays of the IPD is not
novel, see for example~\cite{Rand2013}. Immediately it is noted that if the
environment is noisy~\cite{Wu1995} then no strategy can be considered to be
extortionate as \(p_4>0\).

In the next section, this idea will be illustrated by observing the interactions
that take place in a computer based tournament of the IPD\@.

\section{Numerical experiments}\label{sec:numerical-experiments}

In~\cite{Stewart2012} results from a tournament with
\input{./assets/tex/number_of_stewart_plotkin_strategies/main.tex} strategies,
was presented with specific consideration given to ZD strategies. This
tournament is reproduced here using the Axelrod-Python
project~\cite{Knight2016}. To obtain a good measure of the corresponding
transition rates for each strategy all matches have been run for
\input{assets/tex/number_of_turns/main.tex} turns and every match has been
repeated \input{assets/tex/number_of_repetitions/main.tex} times. All of this
interaction data is available at~\cite{vincent_knight_2018_1297075}. A good
match between the inferred Markov chain and the state distribution of the actual
interactions has been verified. Data for this is presented in the supplementary
materials.

Figure~\ref{fig:SSError_overall_in_stewart_plotkin} shows the \(\text{SSError}\)
values for all the strategies in the tournament, as reported
in~\cite{Stewart2012} the extortionate strategy (which has an expected
\(\text{SSError}\) approximately 0) gains a large number of wins.

\begin{figure}[!htbp]
    \centering
    \includegraphics[width=.8\textwidth]{./assets/img/SSError_overall_in_stewart_plotkin/main.pdf}
    \caption{\(\text{SSError}\) and state probabilities for the strategies
        of~\cite{Stewart2012}, ordered both by number of wins and overall score.
        Note that \(P(DC)\) is not shown as it corresponds to the transpose of
        \(P(CD)\). Cooperator and Defector are omitted as they do not visit all
        the states.}
    \label{fig:SSError_overall_in_stewart_plotkin}
\end{figure}

Here, the work of~\cite{Stewart2012} is extended by investigating a tournament
with \input{assets/tex/number_of_full_strategies/main.tex}
strategies.

The results of this analysis are shown in
Figure~\ref{fig:SSError_and_probabilities_in_full}. The top ranking strategies
by number of wins seem to be extortionate (but not against all strategies) and
it can be seen that a small sub group of strategies achieve mutual defection.
All the top ranking strategies according to score achieve mutual cooperation and
do not extort each other, however they
\textbf{do} exhibit extortionate behaviour towards a number of the lower ranking
strategies.

\begin{figure}[!htbp]
    \centering
    \includegraphics[width=.8\textwidth]{./assets/img/SSError_and_probabilities_in_full/main.pdf}
    \caption{\(\text{SSError}\) for the strategies for the full tournament. Only
    strategy interactions for which \(p_4=0\) and \(\chi>1\) are displayed.}
    \label{fig:SSError_and_probabilities_in_full}
\end{figure}

\section{Conclusion}\label{sec:conclusion}

This work defines an approach to measure whether or not a player is playing a
strategy that corresponds to an extortionate strategy as defined
in~\cite{Press2012}: a mathematical model for suspicion. Indeed, all
extortionate strategies have been
 classified as lying on a triangular plane.
This rigorous classification fails to be robust to small measurement error, thus
a statistical approach is proposed.
This is done through a linear algebraic approach for approximating the solution
of a linear system. Using this, a large number of pairwise interactions is
simulated and in fact very few strategies are found to act extortionately.

The work of~\cite{Press2012}, whilst showing that a clever approach to taking
advantage of another memory one strategy exists: this is incomplete. Whilst the
elegance of this result is very attractive, just as the simplicity of the
victory of Tit For Tat in Axelrod's original tournaments was, it is incomplete.
Extortionate strategies achieve a high number of wins but they do not
achieve a high score which corresponds to the fitness landscape in an
evolutionary sense. From the large number of interactions a payoff matrix \(S\)
can be measured where \(S_{ij}\) denotes the score (using standard values of
\((R, S, T, P) = (3, 0, 5, 1)\)) of the \(i\)th strategy
against the \(j\)th strategy. Using this, the replicator equation
describes the evolution of the system based on a population density fitness
function:

\begin{equation}\label{eqn:replicator_dynamics}
    \frac{dx}{dt} = x(S-x^TS x)
\end{equation}

Equation (\ref{eqn:replicator_dynamics}) is solved numerically through an
integration technique described in~\cite{Petzold1983} and
Figure~\ref{fig:replicator_dynamics} shows the evolution of the distribution of
the system: the various strategies are ranked by scores. It is clear to see that
only the high ranking strategies survive the evolutionary process (in fact,
only \input{./assets/img/replicator_dynamics/main.tex}
have a final distribution greater than \(10 ^ {-2}\)). This confirms the
findings of~\cite{Moran1707} in which sophisticated strategies resist
evolutionary invasion of shorter memory strategies. Recalling
Figure~\ref{fig:SSError_and_probabilities_in_full} this demonstrates that:

\begin{itemize}
    \item Cooperation emerges through the evolutionary process: the high scoring
        strategies do not exhibit extortionate behaviour towards each other.
    \item Extortionate strategies do not survive the evolutionary process.
\end{itemize}

\begin{figure}[!htbp]
    \centering
    \includegraphics[width=.8\textwidth]{./assets/img/replicator_dynamics/main.pdf}
    \caption{Numerical simulation of the replicator equation
    (\ref{eqn:replicator_dynamics}): strategies are ordered by score, only the strategies with a high score survive the evolutionary process.}
    \label{fig:replicator_dynamics}
\end{figure}

This work can be used to classify plays of the IPD\@: data can be collected from
actual interactions (in lab or in the field). Furthermore, this allows for a
classification method similar to the notion of fingerprinting presented
in~\cite{Ashlock2008}. Trained strategies can potentially be classified as
extortionate or not or it could be possible to even constrain the reinforcement
learning approaches that are becoming prevalent in the literature.
Alternatively, this mathematical approach for recognising extortion could be
used in sophisticated strategies to defend against invasion. Arguably, some of
the strategies considered here exhibit this behaviour, indeed as described
in~\cite{Harper2017}, the top ranking strategies in the full tournament are
obtained using evolutionary reinforcement learning techniques, thus, suspicion
of extortionate behaviour could in fact be an evolutionary trait.

\section*{Acknowledgements}

The following open source software libraries were used in this research:

\begin{itemize}
    \item The Axelrod ~\cite{Knight2016, Knight2018} library (IPD strategies and
        tournaments).
    \item The sympy library~\cite{Meurer2017} (verification of all symbolic
        calculations).
    \item The matplotlib~\cite{Droettboom2018} library (visualisation).
    \item The pandas~\cite{Structures2010}, dask~\cite{Dask2016} and
        NumPy~\cite{Oliphant2015} libraries (data manipulation).
    \item The SciPy~\cite{Jones2001} library (numerical integration of the
        replicator equation).
\end{itemize}

This work was performed using the computational facilities of the Advanced
Research Computing @ Cardiff (ARCCA) Division, Cardiff University.

\printbibliography

\newpage
\section*{Supplementary materials}

\includepdf{assets/pdf/proof_of_form_of_extortionate_strategies/main.pdf}

\newpage

Using the pair wise interactions the transition rates \(p,
q\) can be measured and the steady state probabilities inferred and compared to
the actual probabilities of each state.
This is done numerically by computing the singular eigenvector of the
matrix \(A\) \cite{Stewart2009}:

\[
    A =
    \begin{bmatrix}
        p_1 q_1 & p_1 (1 - q_1) & (1 - p_1) q_1 & (1 -p_1) (1 - q_1) \\
        p_2 q_2 & p_2 (1 - q_2) & (1 - p_2) q_2 & (1 -p_2) (1 - q_2) \\
        p_3 q_3 & p_3 (1 - q_3) & (1 - p_3) q_3 & (1 -p_3) (1 - q_3) \\
        p_4 q_4 & p_4 (1 - q_4) & (1 - p_4) q_4 & (1 -p_4) (1 - q_4) \\
    \end{bmatrix}
\]

Figure~\ref{fig:computed_probabilities_vs_theoretic_probabilities} shows a
regression line fitted to every pairwise interaction with a reported
\(\text{SSError}\) value (pairwise interactions with missing states were
omitted). This serves to validate the approach: a part from some edge cases the
relationship is consistent.

\begin{figure}[!htbp]
    \centering
    \includegraphics[width=.8\textwidth]{./assets/img/computed_probabilities_vs_theoretic_probabilities/main.pdf}
    \caption{The
        relationship between the steady state probabilities inferred from the
        measured transitions and the actual steady state probabilities. A linear
        regression line is included validating the approach.}
    \label{fig:computed_probabilities_vs_theoretic_probabilities}
\end{figure}


\end{document}
 times. All of this
interaction data is available at~\cite{vincent_knight_2018_1297075}. A good
match between the inferred Markov chain and the state distribution of the actual
interactions has been verified. Data for this is presented in the supplementary
materials.

Figure~\ref{fig:SSError_overall_in_stewart_plotkin} shows the \(\text{SSError}\)
values for all the strategies in the tournament, as reported
in~\cite{Stewart2012} the extortionate strategy (which has an expected
\(\text{SSError}\) approximately 0) gains a large number of wins.

\begin{figure}[!htbp]
    \centering
    \includegraphics[width=.8\textwidth]{./assets/img/SSError_overall_in_stewart_plotkin/main.pdf}
    \caption{\(\text{SSError}\) and state probabilities for the strategies
        of~\cite{Stewart2012}, ordered both by number of wins and overall score.
        Note that \(P(DC)\) is not shown as it corresponds to the transpose of
        \(P(CD)\). Cooperator and Defector are omitted as they do not visit all
        the states.}
    \label{fig:SSError_overall_in_stewart_plotkin}
\end{figure}

Here, the work of~\cite{Stewart2012} is extended by investigating a tournament
with \documentclass[a4paper]{article}

\usepackage{amsmath}
\usepackage{amssymb}
\usepackage[margin=1.5cm,
            includefoot,
            footskip=30pt]{geometry}
\usepackage{layout}
\usepackage{graphicx}
\usepackage{subcaption}

\usepackage{biblatex}
\usepackage{pdfpages}

\bibliography{main.bib}

\title{Suspicion: Recognising and evaluating the effectiveness
       of extortion in the Iterated Prisoner's Dilemma}
\author{Vincent A. Knight \and Nikoleta E. Glynatsi}
\date{\today}



\begin{document}

\maketitle

\begin{abstract}
    The Iterated Prisoner's Dilemma is a model for rational and evolutionary
    interactive behaviour. It has applications both in the study of human social
    behaviour as well as in biology.
    It is used to understand when and how a rational individual might
    accept an immediate cost to their own utility for the direct benefit of
    another.

    Much attention has been given to a class of strategies called
    Zero Determinant strategies. It has been theoretically shown that these
    strategies can ``extort'' any player.

    In this work, an approach to identify if observed strategies are playing in
    an extortionate way is described. Furthermore, experimental analysis of
    a large tournament with \input{assets/tex/number_of_full_strategies/main.tex}
    strategies is considered. In this setting
    the most highly performing strategies do not play in an extortionate way
    against each other but do against lower performing strategies.
    This suggests that whilst the theory of Zero Determinant strategies
    indicates that memory is not of fundamental importance to the evolution of
    cooperative behaviour, this is incomplete.
\end{abstract}

\section{Introduction}\label{sec:introduction}

Agent based game theoretic models have become a stalwart of the underpinning
mathematics of interactive behaviours. One of the major pieces of work
in this area is the pair of original computer tournaments run by Robert
Axelrod~\cite{Axelrod1980, Axelrod1980a}. These tournaments pitted submitted
computer strategies against each other in plays of the Iterated Prisoner's
Dilemma. A common game where agents can choose to pay a slight cost to their
immediate utility in the hope of building a reputation. This has been used in
economic and evolutionary game theory to understand the evolution of cooperative
behaviour.

Recently, a class of strategies was described in~\cite{Press2012} that can
provably extort any given opponent. In~\cite{Hilbe2013, Moran1707} some
questions have already been asked about the true effectiveness of these
strategies in an evolutionary setting. Here another question is asked: is it
possible to recognise this extortionate behaviour? A mathematical procedure for
suspicion is presented: in the same way that the continued actions of an
extortionate individual might raise suspicion.

This work makes use of the Axelrod Python library~\cite{Knight2018, Knight2016}
with a large number of Prisoner Dilemma strategies available to give an
extensive numerical example of the ideas presented.  The approach is presented
in Section~\ref{sec:delta-zd-strategies}.  All of the code and data discussed
in Section~\ref{sec:numerical-experiments} is open sourced, archived and
written according to best scientific principles~\cite{Wilson2014}. The data
archive can be found at~\cite{vincent_knight_2018_1297075}.

\section{Recognising Extortion}\label{sec:delta-zd-strategies}

In~\cite{Press2012}, given a match between 2 memory-one strategies, the concept
of Zero Determinant (ZD) strategies is introduced. The main result of that paper
shows that given two memory one players \(p, q\in\mathbb{R}^4\) a linear
relationship between the players' scores could be forced by one of the players.

Using the notation of~\cite{Press2012}, assuming the utilities for player \(p\)
are given by \(S_x=(R, S, T, P)\) and for player \(q\) by \(S_y=(R, T, S, P)\)
and that the stationary scores of each player is given by \(S_X\) and \(S_Y\)
respectively. The main result of~\cite{Press2012} is that if

\begin{equation}\label{eqn:linear_relationship_for_p}
    \tilde p=\alpha S_x + \beta S_y + \gamma
\end{equation}

or

\begin{equation}\label{eqn:linear_relationship_for_q}
    \tilde q=\alpha S_x + \beta S_y + \gamma
\end{equation}

where \(\tilde p = (1 - p_1, 1 - p_2, p_3, p_4)\) and
\(\tilde q = (1 - q_1, 1 - q_2, q_3, q_4)\) then:

\begin{equation}
    \alpha S_X + \beta S_Y + \gamma = 0
\end{equation}

In~\cite{Press2012} a particular type of ZD strategy is defined: extortionate
strategies. If:

\begin{equation}\label{eqn:constraint_for_extortion}
    \gamma = - P(\alpha + \beta)
\end{equation}

then the player can ensure they get a score \(\chi\) times
larger than the opponent. This extortion coefficient is given by:

\begin{equation}\label{eqn:definition_of_chi}
    \chi=\frac{-\beta}{\alpha}
\end{equation}

Thus, if (\ref{eqn:constraint_for_extortion}) holds and \(\chi >1\) a player is
said to extort their opponent.
Here, the reverse problem is considered: given a
\(p\in\mathbb{R}^4\) how does one identify \(\alpha, \beta\) if they
exist and is the strategy in fact acting in an extortionate way?

These conditions correspond to:

\begin{align}
    \tilde p_1 & = \alpha R + \beta R - P (\alpha + \beta)
            \label{eqn:condition_for_tilde_p1}\\
    \tilde p_2 & = \alpha S + \beta T - P (\alpha + \beta)
            \label{eqn:condition_for_tilde_p2}\\
    \tilde p_3 & = \alpha T + \beta S - P (\alpha + \beta)
            \label{eqn:condition_for_tilde_p3}\\
    \tilde p_4 & = \alpha P + \beta P - P (\alpha + \beta)
            \label{eqn:condition_for_tilde_p4}
\end{align}

Equation (\ref{eqn:condition_for_tilde_p4}) ensures that \(p_4=\tilde p_4=0\).
Equations (\ref{eqn:condition_for_tilde_p1}-\ref{eqn:condition_for_tilde_p3})
can be used to eliminate \(\alpha, \beta\), giving:

\begin{equation}\label{eqn:planar_definition_of_extortion}
    \tilde p_1 = \frac{(R - P)(\tilde p_2 + \tilde p_3)}{S + T - 2P}
\end{equation}

with:

\begin{equation}\label{eqn:definition_of_chi}
    \chi = \frac{\tilde p_2 (P - T) + \tilde p_3 (S - P)}
                {\tilde p_2 (P - S) + \tilde p_3 (T - P)}
\end{equation}

Given a strategy \(p\in\mathbb{R}^{4\times 1}\) equations
(\ref{eqn:condition_for_tilde_p4}), (\ref{eqn:planar_definition_of_extortion}-\ref{eqn:definition_of_chi}) can be used to check if
a strategy is extortionate. The conditions correspond to:

\begin{align}
    p_1 & = \frac{(R-P)(p_2 + p_3) - R + T + S - P}{S + T - 2P}
     \label{eqn:condition_for_p1}\\
    p_4 & = 0 \label{eqn:condition_for_p4}\\
    1 & > p_2 + p_3\label{eqn:condition_for_chi}
\end{align}

The algebraic steps necessary to prove these results are available in the
supporting materials.

All extortionate strategies reside on a triangular (\ref{eqn:condition_for_chi})
plane (\ref{eqn:condition_for_p1}) in 3 dimensions (\ref{eqn:condition_for_p4}).
Using this formulation it can be seen that a necessary (but not sufficient)
condition for an extortionate strategy is that it cooperates on average less
than 50\% of the time when in a state of disagreement with the opponent.

As an example, consider the known extortionate strategy \(p=(8 / 9, 1 / 2, 1 /
3, 0)\) from~\cite{Stewart2012} which is referred to as \texttt{Extort-2}. In
this case, for the standard values of \((R, T, S, P)\) constraint
(\ref{eqn:condition_for_p1}) corresponds to:

\begin{equation}
    p_1 = \frac{2(p_2 + p_3) + 1}{3}
\end{equation}

It is clear that in this case all constraints hold.

This approach could in fact be used to confirm that a given strategy is acting
in an extortionate manner even if it is not a memory one strategy. However, in
practice, if a closed form for \(p\) is not known, then due to measurement
and/or numerical error this would not work.

This problem can be written in the following linear algebraic form where
\(x=(\alpha, \beta)\)
and \(p^*=(\tilde p_1 - 1, tilde_2 - 1, p_3)\):

\begin{equation}\label{eqn:linear_algebraic_equation_for_p}
    Cx= p^*
\end{equation}

\(C\) corresponds to equations
(\ref{eqn:condition_for_tilde_p1}-\ref{eqn:condition_for_tilde_p3}) and is
given by:

\begin{equation}\label{eqn:definition_of_C}
    C =
    \begin{bmatrix}
        R - P & R- P \\
        S - P & T- P \\
        T - P & S- P \\
    \end{bmatrix}
\end{equation}

Note that in general, equation (\ref{eqn:linear_algebraic_equation_for_p}) will
not necessarily have a solution. From the Rouch\'{e}-Capelli theorem if there is
a solution it is unique as \(\text{rank}(C)=2\) which is the dimension of the
variable \(x\). The best fitting \(x\) is found by minimizing:

\begin{equation}\label{eqn:r_squared}
    \text{SSError} = \|C x- p^*\|_2^2 = \sum_{i=1}^{3}\left((C\bar x)_i-p_i^*\right)^2
\end{equation}

Note that \(\text{SSError}\), which is the square of the Frobenius
norm~\cite{Golub2013}, becomes a measure of how close a strategy is to being an
extortionate strategy. Suspicion
of extortion then corresponds to a threshold on \(\text{SSError}\).

By observing interactions (human or otherwise), their memory one representation
can be inferred and this approach can be used to recognise extortionate
behaviour. The notion of comparing theoretic and actual plays of the IPD is not
novel, see for example~\cite{Rand2013}. Immediately it is noted that if the
environment is noisy~\cite{Wu1995} then no strategy can be considered to be
extortionate as \(p_4>0\).

In the next section, this idea will be illustrated by observing the interactions
that take place in a computer based tournament of the IPD\@.

\section{Numerical experiments}\label{sec:numerical-experiments}

In~\cite{Stewart2012} results from a tournament with
\input{./assets/tex/number_of_stewart_plotkin_strategies/main.tex} strategies,
was presented with specific consideration given to ZD strategies. This
tournament is reproduced here using the Axelrod-Python
project~\cite{Knight2016}. To obtain a good measure of the corresponding
transition rates for each strategy all matches have been run for
\input{assets/tex/number_of_turns/main.tex} turns and every match has been
repeated \input{assets/tex/number_of_repetitions/main.tex} times. All of this
interaction data is available at~\cite{vincent_knight_2018_1297075}. A good
match between the inferred Markov chain and the state distribution of the actual
interactions has been verified. Data for this is presented in the supplementary
materials.

Figure~\ref{fig:SSError_overall_in_stewart_plotkin} shows the \(\text{SSError}\)
values for all the strategies in the tournament, as reported
in~\cite{Stewart2012} the extortionate strategy (which has an expected
\(\text{SSError}\) approximately 0) gains a large number of wins.

\begin{figure}[!htbp]
    \centering
    \includegraphics[width=.8\textwidth]{./assets/img/SSError_overall_in_stewart_plotkin/main.pdf}
    \caption{\(\text{SSError}\) and state probabilities for the strategies
        of~\cite{Stewart2012}, ordered both by number of wins and overall score.
        Note that \(P(DC)\) is not shown as it corresponds to the transpose of
        \(P(CD)\). Cooperator and Defector are omitted as they do not visit all
        the states.}
    \label{fig:SSError_overall_in_stewart_plotkin}
\end{figure}

Here, the work of~\cite{Stewart2012} is extended by investigating a tournament
with \input{assets/tex/number_of_full_strategies/main.tex}
strategies.

The results of this analysis are shown in
Figure~\ref{fig:SSError_and_probabilities_in_full}. The top ranking strategies
by number of wins seem to be extortionate (but not against all strategies) and
it can be seen that a small sub group of strategies achieve mutual defection.
All the top ranking strategies according to score achieve mutual cooperation and
do not extort each other, however they
\textbf{do} exhibit extortionate behaviour towards a number of the lower ranking
strategies.

\begin{figure}[!htbp]
    \centering
    \includegraphics[width=.8\textwidth]{./assets/img/SSError_and_probabilities_in_full/main.pdf}
    \caption{\(\text{SSError}\) for the strategies for the full tournament. Only
    strategy interactions for which \(p_4=0\) and \(\chi>1\) are displayed.}
    \label{fig:SSError_and_probabilities_in_full}
\end{figure}

\section{Conclusion}\label{sec:conclusion}

This work defines an approach to measure whether or not a player is playing a
strategy that corresponds to an extortionate strategy as defined
in~\cite{Press2012}: a mathematical model for suspicion. Indeed, all
extortionate strategies have been
 classified as lying on a triangular plane.
This rigorous classification fails to be robust to small measurement error, thus
a statistical approach is proposed.
This is done through a linear algebraic approach for approximating the solution
of a linear system. Using this, a large number of pairwise interactions is
simulated and in fact very few strategies are found to act extortionately.

The work of~\cite{Press2012}, whilst showing that a clever approach to taking
advantage of another memory one strategy exists: this is incomplete. Whilst the
elegance of this result is very attractive, just as the simplicity of the
victory of Tit For Tat in Axelrod's original tournaments was, it is incomplete.
Extortionate strategies achieve a high number of wins but they do not
achieve a high score which corresponds to the fitness landscape in an
evolutionary sense. From the large number of interactions a payoff matrix \(S\)
can be measured where \(S_{ij}\) denotes the score (using standard values of
\((R, S, T, P) = (3, 0, 5, 1)\)) of the \(i\)th strategy
against the \(j\)th strategy. Using this, the replicator equation
describes the evolution of the system based on a population density fitness
function:

\begin{equation}\label{eqn:replicator_dynamics}
    \frac{dx}{dt} = x(S-x^TS x)
\end{equation}

Equation (\ref{eqn:replicator_dynamics}) is solved numerically through an
integration technique described in~\cite{Petzold1983} and
Figure~\ref{fig:replicator_dynamics} shows the evolution of the distribution of
the system: the various strategies are ranked by scores. It is clear to see that
only the high ranking strategies survive the evolutionary process (in fact,
only \input{./assets/img/replicator_dynamics/main.tex}
have a final distribution greater than \(10 ^ {-2}\)). This confirms the
findings of~\cite{Moran1707} in which sophisticated strategies resist
evolutionary invasion of shorter memory strategies. Recalling
Figure~\ref{fig:SSError_and_probabilities_in_full} this demonstrates that:

\begin{itemize}
    \item Cooperation emerges through the evolutionary process: the high scoring
        strategies do not exhibit extortionate behaviour towards each other.
    \item Extortionate strategies do not survive the evolutionary process.
\end{itemize}

\begin{figure}[!htbp]
    \centering
    \includegraphics[width=.8\textwidth]{./assets/img/replicator_dynamics/main.pdf}
    \caption{Numerical simulation of the replicator equation
    (\ref{eqn:replicator_dynamics}): strategies are ordered by score, only the strategies with a high score survive the evolutionary process.}
    \label{fig:replicator_dynamics}
\end{figure}

This work can be used to classify plays of the IPD\@: data can be collected from
actual interactions (in lab or in the field). Furthermore, this allows for a
classification method similar to the notion of fingerprinting presented
in~\cite{Ashlock2008}. Trained strategies can potentially be classified as
extortionate or not or it could be possible to even constrain the reinforcement
learning approaches that are becoming prevalent in the literature.
Alternatively, this mathematical approach for recognising extortion could be
used in sophisticated strategies to defend against invasion. Arguably, some of
the strategies considered here exhibit this behaviour, indeed as described
in~\cite{Harper2017}, the top ranking strategies in the full tournament are
obtained using evolutionary reinforcement learning techniques, thus, suspicion
of extortionate behaviour could in fact be an evolutionary trait.

\section*{Acknowledgements}

The following open source software libraries were used in this research:

\begin{itemize}
    \item The Axelrod ~\cite{Knight2016, Knight2018} library (IPD strategies and
        tournaments).
    \item The sympy library~\cite{Meurer2017} (verification of all symbolic
        calculations).
    \item The matplotlib~\cite{Droettboom2018} library (visualisation).
    \item The pandas~\cite{Structures2010}, dask~\cite{Dask2016} and
        NumPy~\cite{Oliphant2015} libraries (data manipulation).
    \item The SciPy~\cite{Jones2001} library (numerical integration of the
        replicator equation).
\end{itemize}

This work was performed using the computational facilities of the Advanced
Research Computing @ Cardiff (ARCCA) Division, Cardiff University.

\printbibliography

\newpage
\section*{Supplementary materials}

\includepdf{assets/pdf/proof_of_form_of_extortionate_strategies/main.pdf}

\newpage

Using the pair wise interactions the transition rates \(p,
q\) can be measured and the steady state probabilities inferred and compared to
the actual probabilities of each state.
This is done numerically by computing the singular eigenvector of the
matrix \(A\) \cite{Stewart2009}:

\[
    A =
    \begin{bmatrix}
        p_1 q_1 & p_1 (1 - q_1) & (1 - p_1) q_1 & (1 -p_1) (1 - q_1) \\
        p_2 q_2 & p_2 (1 - q_2) & (1 - p_2) q_2 & (1 -p_2) (1 - q_2) \\
        p_3 q_3 & p_3 (1 - q_3) & (1 - p_3) q_3 & (1 -p_3) (1 - q_3) \\
        p_4 q_4 & p_4 (1 - q_4) & (1 - p_4) q_4 & (1 -p_4) (1 - q_4) \\
    \end{bmatrix}
\]

Figure~\ref{fig:computed_probabilities_vs_theoretic_probabilities} shows a
regression line fitted to every pairwise interaction with a reported
\(\text{SSError}\) value (pairwise interactions with missing states were
omitted). This serves to validate the approach: a part from some edge cases the
relationship is consistent.

\begin{figure}[!htbp]
    \centering
    \includegraphics[width=.8\textwidth]{./assets/img/computed_probabilities_vs_theoretic_probabilities/main.pdf}
    \caption{The
        relationship between the steady state probabilities inferred from the
        measured transitions and the actual steady state probabilities. A linear
        regression line is included validating the approach.}
    \label{fig:computed_probabilities_vs_theoretic_probabilities}
\end{figure}


\end{document}

strategies.

The results of this analysis are shown in
Figure~\ref{fig:SSError_and_probabilities_in_full}. The top ranking strategies
by number of wins seem to be extortionate (but not against all strategies) and
it can be seen that a small sub group of strategies achieve mutual defection.
All the top ranking strategies according to score achieve mutual cooperation and
do not extort each other, however they
\textbf{do} exhibit extortionate behaviour towards a number of the lower ranking
strategies.

\begin{figure}[!htbp]
    \centering
    \includegraphics[width=.8\textwidth]{./assets/img/SSError_and_probabilities_in_full/main.pdf}
    \caption{\(\text{SSError}\) for the strategies for the full tournament. Only
    strategy interactions for which \(p_4=0\) and \(\chi>1\) are displayed.}
    \label{fig:SSError_and_probabilities_in_full}
\end{figure}

\section{Conclusion}\label{sec:conclusion}

This work defines an approach to measure whether or not a player is playing a
strategy that corresponds to an extortionate strategy as defined
in~\cite{Press2012}: a mathematical model for suspicion. Indeed, all
extortionate strategies have been
 classified as lying on a triangular plane.
This rigorous classification fails to be robust to small measurement error, thus
a statistical approach is proposed.
This is done through a linear algebraic approach for approximating the solution
of a linear system. Using this, a large number of pairwise interactions is
simulated and in fact very few strategies are found to act extortionately.

The work of~\cite{Press2012}, whilst showing that a clever approach to taking
advantage of another memory one strategy exists: this is incomplete. Whilst the
elegance of this result is very attractive, just as the simplicity of the
victory of Tit For Tat in Axelrod's original tournaments was, it is incomplete.
Extortionate strategies achieve a high number of wins but they do not
achieve a high score which corresponds to the fitness landscape in an
evolutionary sense. From the large number of interactions a payoff matrix \(S\)
can be measured where \(S_{ij}\) denotes the score (using standard values of
\((R, S, T, P) = (3, 0, 5, 1)\)) of the \(i\)th strategy
against the \(j\)th strategy. Using this, the replicator equation
describes the evolution of the system based on a population density fitness
function:

\begin{equation}\label{eqn:replicator_dynamics}
    \frac{dx}{dt} = x(S-x^TS x)
\end{equation}

Equation (\ref{eqn:replicator_dynamics}) is solved numerically through an
integration technique described in~\cite{Petzold1983} and
Figure~\ref{fig:replicator_dynamics} shows the evolution of the distribution of
the system: the various strategies are ranked by scores. It is clear to see that
only the high ranking strategies survive the evolutionary process (in fact,
only \documentclass[a4paper]{article}

\usepackage{amsmath}
\usepackage{amssymb}
\usepackage[margin=1.5cm,
            includefoot,
            footskip=30pt]{geometry}
\usepackage{layout}
\usepackage{graphicx}
\usepackage{subcaption}

\usepackage{biblatex}
\usepackage{pdfpages}

\bibliography{main.bib}

\title{Suspicion: Recognising and evaluating the effectiveness
       of extortion in the Iterated Prisoner's Dilemma}
\author{Vincent A. Knight \and Nikoleta E. Glynatsi}
\date{\today}



\begin{document}

\maketitle

\begin{abstract}
    The Iterated Prisoner's Dilemma is a model for rational and evolutionary
    interactive behaviour. It has applications both in the study of human social
    behaviour as well as in biology.
    It is used to understand when and how a rational individual might
    accept an immediate cost to their own utility for the direct benefit of
    another.

    Much attention has been given to a class of strategies called
    Zero Determinant strategies. It has been theoretically shown that these
    strategies can ``extort'' any player.

    In this work, an approach to identify if observed strategies are playing in
    an extortionate way is described. Furthermore, experimental analysis of
    a large tournament with \input{assets/tex/number_of_full_strategies/main.tex}
    strategies is considered. In this setting
    the most highly performing strategies do not play in an extortionate way
    against each other but do against lower performing strategies.
    This suggests that whilst the theory of Zero Determinant strategies
    indicates that memory is not of fundamental importance to the evolution of
    cooperative behaviour, this is incomplete.
\end{abstract}

\section{Introduction}\label{sec:introduction}

Agent based game theoretic models have become a stalwart of the underpinning
mathematics of interactive behaviours. One of the major pieces of work
in this area is the pair of original computer tournaments run by Robert
Axelrod~\cite{Axelrod1980, Axelrod1980a}. These tournaments pitted submitted
computer strategies against each other in plays of the Iterated Prisoner's
Dilemma. A common game where agents can choose to pay a slight cost to their
immediate utility in the hope of building a reputation. This has been used in
economic and evolutionary game theory to understand the evolution of cooperative
behaviour.

Recently, a class of strategies was described in~\cite{Press2012} that can
provably extort any given opponent. In~\cite{Hilbe2013, Moran1707} some
questions have already been asked about the true effectiveness of these
strategies in an evolutionary setting. Here another question is asked: is it
possible to recognise this extortionate behaviour? A mathematical procedure for
suspicion is presented: in the same way that the continued actions of an
extortionate individual might raise suspicion.

This work makes use of the Axelrod Python library~\cite{Knight2018, Knight2016}
with a large number of Prisoner Dilemma strategies available to give an
extensive numerical example of the ideas presented.  The approach is presented
in Section~\ref{sec:delta-zd-strategies}.  All of the code and data discussed
in Section~\ref{sec:numerical-experiments} is open sourced, archived and
written according to best scientific principles~\cite{Wilson2014}. The data
archive can be found at~\cite{vincent_knight_2018_1297075}.

\section{Recognising Extortion}\label{sec:delta-zd-strategies}

In~\cite{Press2012}, given a match between 2 memory-one strategies, the concept
of Zero Determinant (ZD) strategies is introduced. The main result of that paper
shows that given two memory one players \(p, q\in\mathbb{R}^4\) a linear
relationship between the players' scores could be forced by one of the players.

Using the notation of~\cite{Press2012}, assuming the utilities for player \(p\)
are given by \(S_x=(R, S, T, P)\) and for player \(q\) by \(S_y=(R, T, S, P)\)
and that the stationary scores of each player is given by \(S_X\) and \(S_Y\)
respectively. The main result of~\cite{Press2012} is that if

\begin{equation}\label{eqn:linear_relationship_for_p}
    \tilde p=\alpha S_x + \beta S_y + \gamma
\end{equation}

or

\begin{equation}\label{eqn:linear_relationship_for_q}
    \tilde q=\alpha S_x + \beta S_y + \gamma
\end{equation}

where \(\tilde p = (1 - p_1, 1 - p_2, p_3, p_4)\) and
\(\tilde q = (1 - q_1, 1 - q_2, q_3, q_4)\) then:

\begin{equation}
    \alpha S_X + \beta S_Y + \gamma = 0
\end{equation}

In~\cite{Press2012} a particular type of ZD strategy is defined: extortionate
strategies. If:

\begin{equation}\label{eqn:constraint_for_extortion}
    \gamma = - P(\alpha + \beta)
\end{equation}

then the player can ensure they get a score \(\chi\) times
larger than the opponent. This extortion coefficient is given by:

\begin{equation}\label{eqn:definition_of_chi}
    \chi=\frac{-\beta}{\alpha}
\end{equation}

Thus, if (\ref{eqn:constraint_for_extortion}) holds and \(\chi >1\) a player is
said to extort their opponent.
Here, the reverse problem is considered: given a
\(p\in\mathbb{R}^4\) how does one identify \(\alpha, \beta\) if they
exist and is the strategy in fact acting in an extortionate way?

These conditions correspond to:

\begin{align}
    \tilde p_1 & = \alpha R + \beta R - P (\alpha + \beta)
            \label{eqn:condition_for_tilde_p1}\\
    \tilde p_2 & = \alpha S + \beta T - P (\alpha + \beta)
            \label{eqn:condition_for_tilde_p2}\\
    \tilde p_3 & = \alpha T + \beta S - P (\alpha + \beta)
            \label{eqn:condition_for_tilde_p3}\\
    \tilde p_4 & = \alpha P + \beta P - P (\alpha + \beta)
            \label{eqn:condition_for_tilde_p4}
\end{align}

Equation (\ref{eqn:condition_for_tilde_p4}) ensures that \(p_4=\tilde p_4=0\).
Equations (\ref{eqn:condition_for_tilde_p1}-\ref{eqn:condition_for_tilde_p3})
can be used to eliminate \(\alpha, \beta\), giving:

\begin{equation}\label{eqn:planar_definition_of_extortion}
    \tilde p_1 = \frac{(R - P)(\tilde p_2 + \tilde p_3)}{S + T - 2P}
\end{equation}

with:

\begin{equation}\label{eqn:definition_of_chi}
    \chi = \frac{\tilde p_2 (P - T) + \tilde p_3 (S - P)}
                {\tilde p_2 (P - S) + \tilde p_3 (T - P)}
\end{equation}

Given a strategy \(p\in\mathbb{R}^{4\times 1}\) equations
(\ref{eqn:condition_for_tilde_p4}), (\ref{eqn:planar_definition_of_extortion}-\ref{eqn:definition_of_chi}) can be used to check if
a strategy is extortionate. The conditions correspond to:

\begin{align}
    p_1 & = \frac{(R-P)(p_2 + p_3) - R + T + S - P}{S + T - 2P}
     \label{eqn:condition_for_p1}\\
    p_4 & = 0 \label{eqn:condition_for_p4}\\
    1 & > p_2 + p_3\label{eqn:condition_for_chi}
\end{align}

The algebraic steps necessary to prove these results are available in the
supporting materials.

All extortionate strategies reside on a triangular (\ref{eqn:condition_for_chi})
plane (\ref{eqn:condition_for_p1}) in 3 dimensions (\ref{eqn:condition_for_p4}).
Using this formulation it can be seen that a necessary (but not sufficient)
condition for an extortionate strategy is that it cooperates on average less
than 50\% of the time when in a state of disagreement with the opponent.

As an example, consider the known extortionate strategy \(p=(8 / 9, 1 / 2, 1 /
3, 0)\) from~\cite{Stewart2012} which is referred to as \texttt{Extort-2}. In
this case, for the standard values of \((R, T, S, P)\) constraint
(\ref{eqn:condition_for_p1}) corresponds to:

\begin{equation}
    p_1 = \frac{2(p_2 + p_3) + 1}{3}
\end{equation}

It is clear that in this case all constraints hold.

This approach could in fact be used to confirm that a given strategy is acting
in an extortionate manner even if it is not a memory one strategy. However, in
practice, if a closed form for \(p\) is not known, then due to measurement
and/or numerical error this would not work.

This problem can be written in the following linear algebraic form where
\(x=(\alpha, \beta)\)
and \(p^*=(\tilde p_1 - 1, tilde_2 - 1, p_3)\):

\begin{equation}\label{eqn:linear_algebraic_equation_for_p}
    Cx= p^*
\end{equation}

\(C\) corresponds to equations
(\ref{eqn:condition_for_tilde_p1}-\ref{eqn:condition_for_tilde_p3}) and is
given by:

\begin{equation}\label{eqn:definition_of_C}
    C =
    \begin{bmatrix}
        R - P & R- P \\
        S - P & T- P \\
        T - P & S- P \\
    \end{bmatrix}
\end{equation}

Note that in general, equation (\ref{eqn:linear_algebraic_equation_for_p}) will
not necessarily have a solution. From the Rouch\'{e}-Capelli theorem if there is
a solution it is unique as \(\text{rank}(C)=2\) which is the dimension of the
variable \(x\). The best fitting \(x\) is found by minimizing:

\begin{equation}\label{eqn:r_squared}
    \text{SSError} = \|C x- p^*\|_2^2 = \sum_{i=1}^{3}\left((C\bar x)_i-p_i^*\right)^2
\end{equation}

Note that \(\text{SSError}\), which is the square of the Frobenius
norm~\cite{Golub2013}, becomes a measure of how close a strategy is to being an
extortionate strategy. Suspicion
of extortion then corresponds to a threshold on \(\text{SSError}\).

By observing interactions (human or otherwise), their memory one representation
can be inferred and this approach can be used to recognise extortionate
behaviour. The notion of comparing theoretic and actual plays of the IPD is not
novel, see for example~\cite{Rand2013}. Immediately it is noted that if the
environment is noisy~\cite{Wu1995} then no strategy can be considered to be
extortionate as \(p_4>0\).

In the next section, this idea will be illustrated by observing the interactions
that take place in a computer based tournament of the IPD\@.

\section{Numerical experiments}\label{sec:numerical-experiments}

In~\cite{Stewart2012} results from a tournament with
\input{./assets/tex/number_of_stewart_plotkin_strategies/main.tex} strategies,
was presented with specific consideration given to ZD strategies. This
tournament is reproduced here using the Axelrod-Python
project~\cite{Knight2016}. To obtain a good measure of the corresponding
transition rates for each strategy all matches have been run for
\input{assets/tex/number_of_turns/main.tex} turns and every match has been
repeated \input{assets/tex/number_of_repetitions/main.tex} times. All of this
interaction data is available at~\cite{vincent_knight_2018_1297075}. A good
match between the inferred Markov chain and the state distribution of the actual
interactions has been verified. Data for this is presented in the supplementary
materials.

Figure~\ref{fig:SSError_overall_in_stewart_plotkin} shows the \(\text{SSError}\)
values for all the strategies in the tournament, as reported
in~\cite{Stewart2012} the extortionate strategy (which has an expected
\(\text{SSError}\) approximately 0) gains a large number of wins.

\begin{figure}[!htbp]
    \centering
    \includegraphics[width=.8\textwidth]{./assets/img/SSError_overall_in_stewart_plotkin/main.pdf}
    \caption{\(\text{SSError}\) and state probabilities for the strategies
        of~\cite{Stewart2012}, ordered both by number of wins and overall score.
        Note that \(P(DC)\) is not shown as it corresponds to the transpose of
        \(P(CD)\). Cooperator and Defector are omitted as they do not visit all
        the states.}
    \label{fig:SSError_overall_in_stewart_plotkin}
\end{figure}

Here, the work of~\cite{Stewart2012} is extended by investigating a tournament
with \input{assets/tex/number_of_full_strategies/main.tex}
strategies.

The results of this analysis are shown in
Figure~\ref{fig:SSError_and_probabilities_in_full}. The top ranking strategies
by number of wins seem to be extortionate (but not against all strategies) and
it can be seen that a small sub group of strategies achieve mutual defection.
All the top ranking strategies according to score achieve mutual cooperation and
do not extort each other, however they
\textbf{do} exhibit extortionate behaviour towards a number of the lower ranking
strategies.

\begin{figure}[!htbp]
    \centering
    \includegraphics[width=.8\textwidth]{./assets/img/SSError_and_probabilities_in_full/main.pdf}
    \caption{\(\text{SSError}\) for the strategies for the full tournament. Only
    strategy interactions for which \(p_4=0\) and \(\chi>1\) are displayed.}
    \label{fig:SSError_and_probabilities_in_full}
\end{figure}

\section{Conclusion}\label{sec:conclusion}

This work defines an approach to measure whether or not a player is playing a
strategy that corresponds to an extortionate strategy as defined
in~\cite{Press2012}: a mathematical model for suspicion. Indeed, all
extortionate strategies have been
 classified as lying on a triangular plane.
This rigorous classification fails to be robust to small measurement error, thus
a statistical approach is proposed.
This is done through a linear algebraic approach for approximating the solution
of a linear system. Using this, a large number of pairwise interactions is
simulated and in fact very few strategies are found to act extortionately.

The work of~\cite{Press2012}, whilst showing that a clever approach to taking
advantage of another memory one strategy exists: this is incomplete. Whilst the
elegance of this result is very attractive, just as the simplicity of the
victory of Tit For Tat in Axelrod's original tournaments was, it is incomplete.
Extortionate strategies achieve a high number of wins but they do not
achieve a high score which corresponds to the fitness landscape in an
evolutionary sense. From the large number of interactions a payoff matrix \(S\)
can be measured where \(S_{ij}\) denotes the score (using standard values of
\((R, S, T, P) = (3, 0, 5, 1)\)) of the \(i\)th strategy
against the \(j\)th strategy. Using this, the replicator equation
describes the evolution of the system based on a population density fitness
function:

\begin{equation}\label{eqn:replicator_dynamics}
    \frac{dx}{dt} = x(S-x^TS x)
\end{equation}

Equation (\ref{eqn:replicator_dynamics}) is solved numerically through an
integration technique described in~\cite{Petzold1983} and
Figure~\ref{fig:replicator_dynamics} shows the evolution of the distribution of
the system: the various strategies are ranked by scores. It is clear to see that
only the high ranking strategies survive the evolutionary process (in fact,
only \input{./assets/img/replicator_dynamics/main.tex}
have a final distribution greater than \(10 ^ {-2}\)). This confirms the
findings of~\cite{Moran1707} in which sophisticated strategies resist
evolutionary invasion of shorter memory strategies. Recalling
Figure~\ref{fig:SSError_and_probabilities_in_full} this demonstrates that:

\begin{itemize}
    \item Cooperation emerges through the evolutionary process: the high scoring
        strategies do not exhibit extortionate behaviour towards each other.
    \item Extortionate strategies do not survive the evolutionary process.
\end{itemize}

\begin{figure}[!htbp]
    \centering
    \includegraphics[width=.8\textwidth]{./assets/img/replicator_dynamics/main.pdf}
    \caption{Numerical simulation of the replicator equation
    (\ref{eqn:replicator_dynamics}): strategies are ordered by score, only the strategies with a high score survive the evolutionary process.}
    \label{fig:replicator_dynamics}
\end{figure}

This work can be used to classify plays of the IPD\@: data can be collected from
actual interactions (in lab or in the field). Furthermore, this allows for a
classification method similar to the notion of fingerprinting presented
in~\cite{Ashlock2008}. Trained strategies can potentially be classified as
extortionate or not or it could be possible to even constrain the reinforcement
learning approaches that are becoming prevalent in the literature.
Alternatively, this mathematical approach for recognising extortion could be
used in sophisticated strategies to defend against invasion. Arguably, some of
the strategies considered here exhibit this behaviour, indeed as described
in~\cite{Harper2017}, the top ranking strategies in the full tournament are
obtained using evolutionary reinforcement learning techniques, thus, suspicion
of extortionate behaviour could in fact be an evolutionary trait.

\section*{Acknowledgements}

The following open source software libraries were used in this research:

\begin{itemize}
    \item The Axelrod ~\cite{Knight2016, Knight2018} library (IPD strategies and
        tournaments).
    \item The sympy library~\cite{Meurer2017} (verification of all symbolic
        calculations).
    \item The matplotlib~\cite{Droettboom2018} library (visualisation).
    \item The pandas~\cite{Structures2010}, dask~\cite{Dask2016} and
        NumPy~\cite{Oliphant2015} libraries (data manipulation).
    \item The SciPy~\cite{Jones2001} library (numerical integration of the
        replicator equation).
\end{itemize}

This work was performed using the computational facilities of the Advanced
Research Computing @ Cardiff (ARCCA) Division, Cardiff University.

\printbibliography

\newpage
\section*{Supplementary materials}

\includepdf{assets/pdf/proof_of_form_of_extortionate_strategies/main.pdf}

\newpage

Using the pair wise interactions the transition rates \(p,
q\) can be measured and the steady state probabilities inferred and compared to
the actual probabilities of each state.
This is done numerically by computing the singular eigenvector of the
matrix \(A\) \cite{Stewart2009}:

\[
    A =
    \begin{bmatrix}
        p_1 q_1 & p_1 (1 - q_1) & (1 - p_1) q_1 & (1 -p_1) (1 - q_1) \\
        p_2 q_2 & p_2 (1 - q_2) & (1 - p_2) q_2 & (1 -p_2) (1 - q_2) \\
        p_3 q_3 & p_3 (1 - q_3) & (1 - p_3) q_3 & (1 -p_3) (1 - q_3) \\
        p_4 q_4 & p_4 (1 - q_4) & (1 - p_4) q_4 & (1 -p_4) (1 - q_4) \\
    \end{bmatrix}
\]

Figure~\ref{fig:computed_probabilities_vs_theoretic_probabilities} shows a
regression line fitted to every pairwise interaction with a reported
\(\text{SSError}\) value (pairwise interactions with missing states were
omitted). This serves to validate the approach: a part from some edge cases the
relationship is consistent.

\begin{figure}[!htbp]
    \centering
    \includegraphics[width=.8\textwidth]{./assets/img/computed_probabilities_vs_theoretic_probabilities/main.pdf}
    \caption{The
        relationship between the steady state probabilities inferred from the
        measured transitions and the actual steady state probabilities. A linear
        regression line is included validating the approach.}
    \label{fig:computed_probabilities_vs_theoretic_probabilities}
\end{figure}


\end{document}

have a final distribution greater than \(10 ^ {-2}\)). This confirms the
findings of~\cite{Moran1707} in which sophisticated strategies resist
evolutionary invasion of shorter memory strategies. Recalling
Figure~\ref{fig:SSError_and_probabilities_in_full} this demonstrates that:

\begin{itemize}
    \item Cooperation emerges through the evolutionary process: the high scoring
        strategies do not exhibit extortionate behaviour towards each other.
    \item Extortionate strategies do not survive the evolutionary process.
\end{itemize}

\begin{figure}[!htbp]
    \centering
    \includegraphics[width=.8\textwidth]{./assets/img/replicator_dynamics/main.pdf}
    \caption{Numerical simulation of the replicator equation
    (\ref{eqn:replicator_dynamics}): strategies are ordered by score, only the strategies with a high score survive the evolutionary process.}
    \label{fig:replicator_dynamics}
\end{figure}

This work can be used to classify plays of the IPD\@: data can be collected from
actual interactions (in lab or in the field). Furthermore, this allows for a
classification method similar to the notion of fingerprinting presented
in~\cite{Ashlock2008}. Trained strategies can potentially be classified as
extortionate or not or it could be possible to even constrain the reinforcement
learning approaches that are becoming prevalent in the literature.
Alternatively, this mathematical approach for recognising extortion could be
used in sophisticated strategies to defend against invasion. Arguably, some of
the strategies considered here exhibit this behaviour, indeed as described
in~\cite{Harper2017}, the top ranking strategies in the full tournament are
obtained using evolutionary reinforcement learning techniques, thus, suspicion
of extortionate behaviour could in fact be an evolutionary trait.

\section*{Acknowledgements}

The following open source software libraries were used in this research:

\begin{itemize}
    \item The Axelrod ~\cite{Knight2016, Knight2018} library (IPD strategies and
        tournaments).
    \item The sympy library~\cite{Meurer2017} (verification of all symbolic
        calculations).
    \item The matplotlib~\cite{Droettboom2018} library (visualisation).
    \item The pandas~\cite{Structures2010}, dask~\cite{Dask2016} and
        NumPy~\cite{Oliphant2015} libraries (data manipulation).
    \item The SciPy~\cite{Jones2001} library (numerical integration of the
        replicator equation).
\end{itemize}

This work was performed using the computational facilities of the Advanced
Research Computing @ Cardiff (ARCCA) Division, Cardiff University.

\printbibliography

\newpage
\section*{Supplementary materials}

\includepdf{assets/pdf/proof_of_form_of_extortionate_strategies/main.pdf}

\newpage

Using the pair wise interactions the transition rates \(p,
q\) can be measured and the steady state probabilities inferred and compared to
the actual probabilities of each state.
This is done numerically by computing the singular eigenvector of the
matrix \(A\) \cite{Stewart2009}:

\[
    A =
    \begin{bmatrix}
        p_1 q_1 & p_1 (1 - q_1) & (1 - p_1) q_1 & (1 -p_1) (1 - q_1) \\
        p_2 q_2 & p_2 (1 - q_2) & (1 - p_2) q_2 & (1 -p_2) (1 - q_2) \\
        p_3 q_3 & p_3 (1 - q_3) & (1 - p_3) q_3 & (1 -p_3) (1 - q_3) \\
        p_4 q_4 & p_4 (1 - q_4) & (1 - p_4) q_4 & (1 -p_4) (1 - q_4) \\
    \end{bmatrix}
\]

Figure~\ref{fig:computed_probabilities_vs_theoretic_probabilities} shows a
regression line fitted to every pairwise interaction with a reported
\(\text{SSError}\) value (pairwise interactions with missing states were
omitted). This serves to validate the approach: a part from some edge cases the
relationship is consistent.

\begin{figure}[!htbp]
    \centering
    \includegraphics[width=.8\textwidth]{./assets/img/computed_probabilities_vs_theoretic_probabilities/main.pdf}
    \caption{The
        relationship between the steady state probabilities inferred from the
        measured transitions and the actual steady state probabilities. A linear
        regression line is included validating the approach.}
    \label{fig:computed_probabilities_vs_theoretic_probabilities}
\end{figure}


\end{document}
 turns and every match has been
repeated \documentclass[a4paper]{article}

\usepackage{amsmath}
\usepackage{amssymb}
\usepackage[margin=1.5cm,
            includefoot,
            footskip=30pt]{geometry}
\usepackage{layout}
\usepackage{graphicx}
\usepackage{subcaption}

\usepackage{biblatex}
\usepackage{pdfpages}

\bibliography{main.bib}

\title{Suspicion: Recognising and evaluating the effectiveness
       of extortion in the Iterated Prisoner's Dilemma}
\author{Vincent A. Knight \and Nikoleta E. Glynatsi}
\date{\today}



\begin{document}

\maketitle

\begin{abstract}
    The Iterated Prisoner's Dilemma is a model for rational and evolutionary
    interactive behaviour. It has applications both in the study of human social
    behaviour as well as in biology.
    It is used to understand when and how a rational individual might
    accept an immediate cost to their own utility for the direct benefit of
    another.

    Much attention has been given to a class of strategies called
    Zero Determinant strategies. It has been theoretically shown that these
    strategies can ``extort'' any player.

    In this work, an approach to identify if observed strategies are playing in
    an extortionate way is described. Furthermore, experimental analysis of
    a large tournament with \documentclass[a4paper]{article}

\usepackage{amsmath}
\usepackage{amssymb}
\usepackage[margin=1.5cm,
            includefoot,
            footskip=30pt]{geometry}
\usepackage{layout}
\usepackage{graphicx}
\usepackage{subcaption}

\usepackage{biblatex}
\usepackage{pdfpages}

\bibliography{main.bib}

\title{Suspicion: Recognising and evaluating the effectiveness
       of extortion in the Iterated Prisoner's Dilemma}
\author{Vincent A. Knight \and Nikoleta E. Glynatsi}
\date{\today}



\begin{document}

\maketitle

\begin{abstract}
    The Iterated Prisoner's Dilemma is a model for rational and evolutionary
    interactive behaviour. It has applications both in the study of human social
    behaviour as well as in biology.
    It is used to understand when and how a rational individual might
    accept an immediate cost to their own utility for the direct benefit of
    another.

    Much attention has been given to a class of strategies called
    Zero Determinant strategies. It has been theoretically shown that these
    strategies can ``extort'' any player.

    In this work, an approach to identify if observed strategies are playing in
    an extortionate way is described. Furthermore, experimental analysis of
    a large tournament with \input{assets/tex/number_of_full_strategies/main.tex}
    strategies is considered. In this setting
    the most highly performing strategies do not play in an extortionate way
    against each other but do against lower performing strategies.
    This suggests that whilst the theory of Zero Determinant strategies
    indicates that memory is not of fundamental importance to the evolution of
    cooperative behaviour, this is incomplete.
\end{abstract}

\section{Introduction}\label{sec:introduction}

Agent based game theoretic models have become a stalwart of the underpinning
mathematics of interactive behaviours. One of the major pieces of work
in this area is the pair of original computer tournaments run by Robert
Axelrod~\cite{Axelrod1980, Axelrod1980a}. These tournaments pitted submitted
computer strategies against each other in plays of the Iterated Prisoner's
Dilemma. A common game where agents can choose to pay a slight cost to their
immediate utility in the hope of building a reputation. This has been used in
economic and evolutionary game theory to understand the evolution of cooperative
behaviour.

Recently, a class of strategies was described in~\cite{Press2012} that can
provably extort any given opponent. In~\cite{Hilbe2013, Moran1707} some
questions have already been asked about the true effectiveness of these
strategies in an evolutionary setting. Here another question is asked: is it
possible to recognise this extortionate behaviour? A mathematical procedure for
suspicion is presented: in the same way that the continued actions of an
extortionate individual might raise suspicion.

This work makes use of the Axelrod Python library~\cite{Knight2018, Knight2016}
with a large number of Prisoner Dilemma strategies available to give an
extensive numerical example of the ideas presented.  The approach is presented
in Section~\ref{sec:delta-zd-strategies}.  All of the code and data discussed
in Section~\ref{sec:numerical-experiments} is open sourced, archived and
written according to best scientific principles~\cite{Wilson2014}. The data
archive can be found at~\cite{vincent_knight_2018_1297075}.

\section{Recognising Extortion}\label{sec:delta-zd-strategies}

In~\cite{Press2012}, given a match between 2 memory-one strategies, the concept
of Zero Determinant (ZD) strategies is introduced. The main result of that paper
shows that given two memory one players \(p, q\in\mathbb{R}^4\) a linear
relationship between the players' scores could be forced by one of the players.

Using the notation of~\cite{Press2012}, assuming the utilities for player \(p\)
are given by \(S_x=(R, S, T, P)\) and for player \(q\) by \(S_y=(R, T, S, P)\)
and that the stationary scores of each player is given by \(S_X\) and \(S_Y\)
respectively. The main result of~\cite{Press2012} is that if

\begin{equation}\label{eqn:linear_relationship_for_p}
    \tilde p=\alpha S_x + \beta S_y + \gamma
\end{equation}

or

\begin{equation}\label{eqn:linear_relationship_for_q}
    \tilde q=\alpha S_x + \beta S_y + \gamma
\end{equation}

where \(\tilde p = (1 - p_1, 1 - p_2, p_3, p_4)\) and
\(\tilde q = (1 - q_1, 1 - q_2, q_3, q_4)\) then:

\begin{equation}
    \alpha S_X + \beta S_Y + \gamma = 0
\end{equation}

In~\cite{Press2012} a particular type of ZD strategy is defined: extortionate
strategies. If:

\begin{equation}\label{eqn:constraint_for_extortion}
    \gamma = - P(\alpha + \beta)
\end{equation}

then the player can ensure they get a score \(\chi\) times
larger than the opponent. This extortion coefficient is given by:

\begin{equation}\label{eqn:definition_of_chi}
    \chi=\frac{-\beta}{\alpha}
\end{equation}

Thus, if (\ref{eqn:constraint_for_extortion}) holds and \(\chi >1\) a player is
said to extort their opponent.
Here, the reverse problem is considered: given a
\(p\in\mathbb{R}^4\) how does one identify \(\alpha, \beta\) if they
exist and is the strategy in fact acting in an extortionate way?

These conditions correspond to:

\begin{align}
    \tilde p_1 & = \alpha R + \beta R - P (\alpha + \beta)
            \label{eqn:condition_for_tilde_p1}\\
    \tilde p_2 & = \alpha S + \beta T - P (\alpha + \beta)
            \label{eqn:condition_for_tilde_p2}\\
    \tilde p_3 & = \alpha T + \beta S - P (\alpha + \beta)
            \label{eqn:condition_for_tilde_p3}\\
    \tilde p_4 & = \alpha P + \beta P - P (\alpha + \beta)
            \label{eqn:condition_for_tilde_p4}
\end{align}

Equation (\ref{eqn:condition_for_tilde_p4}) ensures that \(p_4=\tilde p_4=0\).
Equations (\ref{eqn:condition_for_tilde_p1}-\ref{eqn:condition_for_tilde_p3})
can be used to eliminate \(\alpha, \beta\), giving:

\begin{equation}\label{eqn:planar_definition_of_extortion}
    \tilde p_1 = \frac{(R - P)(\tilde p_2 + \tilde p_3)}{S + T - 2P}
\end{equation}

with:

\begin{equation}\label{eqn:definition_of_chi}
    \chi = \frac{\tilde p_2 (P - T) + \tilde p_3 (S - P)}
                {\tilde p_2 (P - S) + \tilde p_3 (T - P)}
\end{equation}

Given a strategy \(p\in\mathbb{R}^{4\times 1}\) equations
(\ref{eqn:condition_for_tilde_p4}), (\ref{eqn:planar_definition_of_extortion}-\ref{eqn:definition_of_chi}) can be used to check if
a strategy is extortionate. The conditions correspond to:

\begin{align}
    p_1 & = \frac{(R-P)(p_2 + p_3) - R + T + S - P}{S + T - 2P}
     \label{eqn:condition_for_p1}\\
    p_4 & = 0 \label{eqn:condition_for_p4}\\
    1 & > p_2 + p_3\label{eqn:condition_for_chi}
\end{align}

The algebraic steps necessary to prove these results are available in the
supporting materials.

All extortionate strategies reside on a triangular (\ref{eqn:condition_for_chi})
plane (\ref{eqn:condition_for_p1}) in 3 dimensions (\ref{eqn:condition_for_p4}).
Using this formulation it can be seen that a necessary (but not sufficient)
condition for an extortionate strategy is that it cooperates on average less
than 50\% of the time when in a state of disagreement with the opponent.

As an example, consider the known extortionate strategy \(p=(8 / 9, 1 / 2, 1 /
3, 0)\) from~\cite{Stewart2012} which is referred to as \texttt{Extort-2}. In
this case, for the standard values of \((R, T, S, P)\) constraint
(\ref{eqn:condition_for_p1}) corresponds to:

\begin{equation}
    p_1 = \frac{2(p_2 + p_3) + 1}{3}
\end{equation}

It is clear that in this case all constraints hold.

This approach could in fact be used to confirm that a given strategy is acting
in an extortionate manner even if it is not a memory one strategy. However, in
practice, if a closed form for \(p\) is not known, then due to measurement
and/or numerical error this would not work.

This problem can be written in the following linear algebraic form where
\(x=(\alpha, \beta)\)
and \(p^*=(\tilde p_1 - 1, tilde_2 - 1, p_3)\):

\begin{equation}\label{eqn:linear_algebraic_equation_for_p}
    Cx= p^*
\end{equation}

\(C\) corresponds to equations
(\ref{eqn:condition_for_tilde_p1}-\ref{eqn:condition_for_tilde_p3}) and is
given by:

\begin{equation}\label{eqn:definition_of_C}
    C =
    \begin{bmatrix}
        R - P & R- P \\
        S - P & T- P \\
        T - P & S- P \\
    \end{bmatrix}
\end{equation}

Note that in general, equation (\ref{eqn:linear_algebraic_equation_for_p}) will
not necessarily have a solution. From the Rouch\'{e}-Capelli theorem if there is
a solution it is unique as \(\text{rank}(C)=2\) which is the dimension of the
variable \(x\). The best fitting \(x\) is found by minimizing:

\begin{equation}\label{eqn:r_squared}
    \text{SSError} = \|C x- p^*\|_2^2 = \sum_{i=1}^{3}\left((C\bar x)_i-p_i^*\right)^2
\end{equation}

Note that \(\text{SSError}\), which is the square of the Frobenius
norm~\cite{Golub2013}, becomes a measure of how close a strategy is to being an
extortionate strategy. Suspicion
of extortion then corresponds to a threshold on \(\text{SSError}\).

By observing interactions (human or otherwise), their memory one representation
can be inferred and this approach can be used to recognise extortionate
behaviour. The notion of comparing theoretic and actual plays of the IPD is not
novel, see for example~\cite{Rand2013}. Immediately it is noted that if the
environment is noisy~\cite{Wu1995} then no strategy can be considered to be
extortionate as \(p_4>0\).

In the next section, this idea will be illustrated by observing the interactions
that take place in a computer based tournament of the IPD\@.

\section{Numerical experiments}\label{sec:numerical-experiments}

In~\cite{Stewart2012} results from a tournament with
\input{./assets/tex/number_of_stewart_plotkin_strategies/main.tex} strategies,
was presented with specific consideration given to ZD strategies. This
tournament is reproduced here using the Axelrod-Python
project~\cite{Knight2016}. To obtain a good measure of the corresponding
transition rates for each strategy all matches have been run for
\input{assets/tex/number_of_turns/main.tex} turns and every match has been
repeated \input{assets/tex/number_of_repetitions/main.tex} times. All of this
interaction data is available at~\cite{vincent_knight_2018_1297075}. A good
match between the inferred Markov chain and the state distribution of the actual
interactions has been verified. Data for this is presented in the supplementary
materials.

Figure~\ref{fig:SSError_overall_in_stewart_plotkin} shows the \(\text{SSError}\)
values for all the strategies in the tournament, as reported
in~\cite{Stewart2012} the extortionate strategy (which has an expected
\(\text{SSError}\) approximately 0) gains a large number of wins.

\begin{figure}[!htbp]
    \centering
    \includegraphics[width=.8\textwidth]{./assets/img/SSError_overall_in_stewart_plotkin/main.pdf}
    \caption{\(\text{SSError}\) and state probabilities for the strategies
        of~\cite{Stewart2012}, ordered both by number of wins and overall score.
        Note that \(P(DC)\) is not shown as it corresponds to the transpose of
        \(P(CD)\). Cooperator and Defector are omitted as they do not visit all
        the states.}
    \label{fig:SSError_overall_in_stewart_plotkin}
\end{figure}

Here, the work of~\cite{Stewart2012} is extended by investigating a tournament
with \input{assets/tex/number_of_full_strategies/main.tex}
strategies.

The results of this analysis are shown in
Figure~\ref{fig:SSError_and_probabilities_in_full}. The top ranking strategies
by number of wins seem to be extortionate (but not against all strategies) and
it can be seen that a small sub group of strategies achieve mutual defection.
All the top ranking strategies according to score achieve mutual cooperation and
do not extort each other, however they
\textbf{do} exhibit extortionate behaviour towards a number of the lower ranking
strategies.

\begin{figure}[!htbp]
    \centering
    \includegraphics[width=.8\textwidth]{./assets/img/SSError_and_probabilities_in_full/main.pdf}
    \caption{\(\text{SSError}\) for the strategies for the full tournament. Only
    strategy interactions for which \(p_4=0\) and \(\chi>1\) are displayed.}
    \label{fig:SSError_and_probabilities_in_full}
\end{figure}

\section{Conclusion}\label{sec:conclusion}

This work defines an approach to measure whether or not a player is playing a
strategy that corresponds to an extortionate strategy as defined
in~\cite{Press2012}: a mathematical model for suspicion. Indeed, all
extortionate strategies have been
 classified as lying on a triangular plane.
This rigorous classification fails to be robust to small measurement error, thus
a statistical approach is proposed.
This is done through a linear algebraic approach for approximating the solution
of a linear system. Using this, a large number of pairwise interactions is
simulated and in fact very few strategies are found to act extortionately.

The work of~\cite{Press2012}, whilst showing that a clever approach to taking
advantage of another memory one strategy exists: this is incomplete. Whilst the
elegance of this result is very attractive, just as the simplicity of the
victory of Tit For Tat in Axelrod's original tournaments was, it is incomplete.
Extortionate strategies achieve a high number of wins but they do not
achieve a high score which corresponds to the fitness landscape in an
evolutionary sense. From the large number of interactions a payoff matrix \(S\)
can be measured where \(S_{ij}\) denotes the score (using standard values of
\((R, S, T, P) = (3, 0, 5, 1)\)) of the \(i\)th strategy
against the \(j\)th strategy. Using this, the replicator equation
describes the evolution of the system based on a population density fitness
function:

\begin{equation}\label{eqn:replicator_dynamics}
    \frac{dx}{dt} = x(S-x^TS x)
\end{equation}

Equation (\ref{eqn:replicator_dynamics}) is solved numerically through an
integration technique described in~\cite{Petzold1983} and
Figure~\ref{fig:replicator_dynamics} shows the evolution of the distribution of
the system: the various strategies are ranked by scores. It is clear to see that
only the high ranking strategies survive the evolutionary process (in fact,
only \input{./assets/img/replicator_dynamics/main.tex}
have a final distribution greater than \(10 ^ {-2}\)). This confirms the
findings of~\cite{Moran1707} in which sophisticated strategies resist
evolutionary invasion of shorter memory strategies. Recalling
Figure~\ref{fig:SSError_and_probabilities_in_full} this demonstrates that:

\begin{itemize}
    \item Cooperation emerges through the evolutionary process: the high scoring
        strategies do not exhibit extortionate behaviour towards each other.
    \item Extortionate strategies do not survive the evolutionary process.
\end{itemize}

\begin{figure}[!htbp]
    \centering
    \includegraphics[width=.8\textwidth]{./assets/img/replicator_dynamics/main.pdf}
    \caption{Numerical simulation of the replicator equation
    (\ref{eqn:replicator_dynamics}): strategies are ordered by score, only the strategies with a high score survive the evolutionary process.}
    \label{fig:replicator_dynamics}
\end{figure}

This work can be used to classify plays of the IPD\@: data can be collected from
actual interactions (in lab or in the field). Furthermore, this allows for a
classification method similar to the notion of fingerprinting presented
in~\cite{Ashlock2008}. Trained strategies can potentially be classified as
extortionate or not or it could be possible to even constrain the reinforcement
learning approaches that are becoming prevalent in the literature.
Alternatively, this mathematical approach for recognising extortion could be
used in sophisticated strategies to defend against invasion. Arguably, some of
the strategies considered here exhibit this behaviour, indeed as described
in~\cite{Harper2017}, the top ranking strategies in the full tournament are
obtained using evolutionary reinforcement learning techniques, thus, suspicion
of extortionate behaviour could in fact be an evolutionary trait.

\section*{Acknowledgements}

The following open source software libraries were used in this research:

\begin{itemize}
    \item The Axelrod ~\cite{Knight2016, Knight2018} library (IPD strategies and
        tournaments).
    \item The sympy library~\cite{Meurer2017} (verification of all symbolic
        calculations).
    \item The matplotlib~\cite{Droettboom2018} library (visualisation).
    \item The pandas~\cite{Structures2010}, dask~\cite{Dask2016} and
        NumPy~\cite{Oliphant2015} libraries (data manipulation).
    \item The SciPy~\cite{Jones2001} library (numerical integration of the
        replicator equation).
\end{itemize}

This work was performed using the computational facilities of the Advanced
Research Computing @ Cardiff (ARCCA) Division, Cardiff University.

\printbibliography

\newpage
\section*{Supplementary materials}

\includepdf{assets/pdf/proof_of_form_of_extortionate_strategies/main.pdf}

\newpage

Using the pair wise interactions the transition rates \(p,
q\) can be measured and the steady state probabilities inferred and compared to
the actual probabilities of each state.
This is done numerically by computing the singular eigenvector of the
matrix \(A\) \cite{Stewart2009}:

\[
    A =
    \begin{bmatrix}
        p_1 q_1 & p_1 (1 - q_1) & (1 - p_1) q_1 & (1 -p_1) (1 - q_1) \\
        p_2 q_2 & p_2 (1 - q_2) & (1 - p_2) q_2 & (1 -p_2) (1 - q_2) \\
        p_3 q_3 & p_3 (1 - q_3) & (1 - p_3) q_3 & (1 -p_3) (1 - q_3) \\
        p_4 q_4 & p_4 (1 - q_4) & (1 - p_4) q_4 & (1 -p_4) (1 - q_4) \\
    \end{bmatrix}
\]

Figure~\ref{fig:computed_probabilities_vs_theoretic_probabilities} shows a
regression line fitted to every pairwise interaction with a reported
\(\text{SSError}\) value (pairwise interactions with missing states were
omitted). This serves to validate the approach: a part from some edge cases the
relationship is consistent.

\begin{figure}[!htbp]
    \centering
    \includegraphics[width=.8\textwidth]{./assets/img/computed_probabilities_vs_theoretic_probabilities/main.pdf}
    \caption{The
        relationship between the steady state probabilities inferred from the
        measured transitions and the actual steady state probabilities. A linear
        regression line is included validating the approach.}
    \label{fig:computed_probabilities_vs_theoretic_probabilities}
\end{figure}


\end{document}

    strategies is considered. In this setting
    the most highly performing strategies do not play in an extortionate way
    against each other but do against lower performing strategies.
    This suggests that whilst the theory of Zero Determinant strategies
    indicates that memory is not of fundamental importance to the evolution of
    cooperative behaviour, this is incomplete.
\end{abstract}

\section{Introduction}\label{sec:introduction}

Agent based game theoretic models have become a stalwart of the underpinning
mathematics of interactive behaviours. One of the major pieces of work
in this area is the pair of original computer tournaments run by Robert
Axelrod~\cite{Axelrod1980, Axelrod1980a}. These tournaments pitted submitted
computer strategies against each other in plays of the Iterated Prisoner's
Dilemma. A common game where agents can choose to pay a slight cost to their
immediate utility in the hope of building a reputation. This has been used in
economic and evolutionary game theory to understand the evolution of cooperative
behaviour.

Recently, a class of strategies was described in~\cite{Press2012} that can
provably extort any given opponent. In~\cite{Hilbe2013, Moran1707} some
questions have already been asked about the true effectiveness of these
strategies in an evolutionary setting. Here another question is asked: is it
possible to recognise this extortionate behaviour? A mathematical procedure for
suspicion is presented: in the same way that the continued actions of an
extortionate individual might raise suspicion.

This work makes use of the Axelrod Python library~\cite{Knight2018, Knight2016}
with a large number of Prisoner Dilemma strategies available to give an
extensive numerical example of the ideas presented.  The approach is presented
in Section~\ref{sec:delta-zd-strategies}.  All of the code and data discussed
in Section~\ref{sec:numerical-experiments} is open sourced, archived and
written according to best scientific principles~\cite{Wilson2014}. The data
archive can be found at~\cite{vincent_knight_2018_1297075}.

\section{Recognising Extortion}\label{sec:delta-zd-strategies}

In~\cite{Press2012}, given a match between 2 memory-one strategies, the concept
of Zero Determinant (ZD) strategies is introduced. The main result of that paper
shows that given two memory one players \(p, q\in\mathbb{R}^4\) a linear
relationship between the players' scores could be forced by one of the players.

Using the notation of~\cite{Press2012}, assuming the utilities for player \(p\)
are given by \(S_x=(R, S, T, P)\) and for player \(q\) by \(S_y=(R, T, S, P)\)
and that the stationary scores of each player is given by \(S_X\) and \(S_Y\)
respectively. The main result of~\cite{Press2012} is that if

\begin{equation}\label{eqn:linear_relationship_for_p}
    \tilde p=\alpha S_x + \beta S_y + \gamma
\end{equation}

or

\begin{equation}\label{eqn:linear_relationship_for_q}
    \tilde q=\alpha S_x + \beta S_y + \gamma
\end{equation}

where \(\tilde p = (1 - p_1, 1 - p_2, p_3, p_4)\) and
\(\tilde q = (1 - q_1, 1 - q_2, q_3, q_4)\) then:

\begin{equation}
    \alpha S_X + \beta S_Y + \gamma = 0
\end{equation}

In~\cite{Press2012} a particular type of ZD strategy is defined: extortionate
strategies. If:

\begin{equation}\label{eqn:constraint_for_extortion}
    \gamma = - P(\alpha + \beta)
\end{equation}

then the player can ensure they get a score \(\chi\) times
larger than the opponent. This extortion coefficient is given by:

\begin{equation}\label{eqn:definition_of_chi}
    \chi=\frac{-\beta}{\alpha}
\end{equation}

Thus, if (\ref{eqn:constraint_for_extortion}) holds and \(\chi >1\) a player is
said to extort their opponent.
Here, the reverse problem is considered: given a
\(p\in\mathbb{R}^4\) how does one identify \(\alpha, \beta\) if they
exist and is the strategy in fact acting in an extortionate way?

These conditions correspond to:

\begin{align}
    \tilde p_1 & = \alpha R + \beta R - P (\alpha + \beta)
            \label{eqn:condition_for_tilde_p1}\\
    \tilde p_2 & = \alpha S + \beta T - P (\alpha + \beta)
            \label{eqn:condition_for_tilde_p2}\\
    \tilde p_3 & = \alpha T + \beta S - P (\alpha + \beta)
            \label{eqn:condition_for_tilde_p3}\\
    \tilde p_4 & = \alpha P + \beta P - P (\alpha + \beta)
            \label{eqn:condition_for_tilde_p4}
\end{align}

Equation (\ref{eqn:condition_for_tilde_p4}) ensures that \(p_4=\tilde p_4=0\).
Equations (\ref{eqn:condition_for_tilde_p1}-\ref{eqn:condition_for_tilde_p3})
can be used to eliminate \(\alpha, \beta\), giving:

\begin{equation}\label{eqn:planar_definition_of_extortion}
    \tilde p_1 = \frac{(R - P)(\tilde p_2 + \tilde p_3)}{S + T - 2P}
\end{equation}

with:

\begin{equation}\label{eqn:definition_of_chi}
    \chi = \frac{\tilde p_2 (P - T) + \tilde p_3 (S - P)}
                {\tilde p_2 (P - S) + \tilde p_3 (T - P)}
\end{equation}

Given a strategy \(p\in\mathbb{R}^{4\times 1}\) equations
(\ref{eqn:condition_for_tilde_p4}), (\ref{eqn:planar_definition_of_extortion}-\ref{eqn:definition_of_chi}) can be used to check if
a strategy is extortionate. The conditions correspond to:

\begin{align}
    p_1 & = \frac{(R-P)(p_2 + p_3) - R + T + S - P}{S + T - 2P}
     \label{eqn:condition_for_p1}\\
    p_4 & = 0 \label{eqn:condition_for_p4}\\
    1 & > p_2 + p_3\label{eqn:condition_for_chi}
\end{align}

The algebraic steps necessary to prove these results are available in the
supporting materials.

All extortionate strategies reside on a triangular (\ref{eqn:condition_for_chi})
plane (\ref{eqn:condition_for_p1}) in 3 dimensions (\ref{eqn:condition_for_p4}).
Using this formulation it can be seen that a necessary (but not sufficient)
condition for an extortionate strategy is that it cooperates on average less
than 50\% of the time when in a state of disagreement with the opponent.

As an example, consider the known extortionate strategy \(p=(8 / 9, 1 / 2, 1 /
3, 0)\) from~\cite{Stewart2012} which is referred to as \texttt{Extort-2}. In
this case, for the standard values of \((R, T, S, P)\) constraint
(\ref{eqn:condition_for_p1}) corresponds to:

\begin{equation}
    p_1 = \frac{2(p_2 + p_3) + 1}{3}
\end{equation}

It is clear that in this case all constraints hold.

This approach could in fact be used to confirm that a given strategy is acting
in an extortionate manner even if it is not a memory one strategy. However, in
practice, if a closed form for \(p\) is not known, then due to measurement
and/or numerical error this would not work.

This problem can be written in the following linear algebraic form where
\(x=(\alpha, \beta)\)
and \(p^*=(\tilde p_1 - 1, tilde_2 - 1, p_3)\):

\begin{equation}\label{eqn:linear_algebraic_equation_for_p}
    Cx= p^*
\end{equation}

\(C\) corresponds to equations
(\ref{eqn:condition_for_tilde_p1}-\ref{eqn:condition_for_tilde_p3}) and is
given by:

\begin{equation}\label{eqn:definition_of_C}
    C =
    \begin{bmatrix}
        R - P & R- P \\
        S - P & T- P \\
        T - P & S- P \\
    \end{bmatrix}
\end{equation}

Note that in general, equation (\ref{eqn:linear_algebraic_equation_for_p}) will
not necessarily have a solution. From the Rouch\'{e}-Capelli theorem if there is
a solution it is unique as \(\text{rank}(C)=2\) which is the dimension of the
variable \(x\). The best fitting \(x\) is found by minimizing:

\begin{equation}\label{eqn:r_squared}
    \text{SSError} = \|C x- p^*\|_2^2 = \sum_{i=1}^{3}\left((C\bar x)_i-p_i^*\right)^2
\end{equation}

Note that \(\text{SSError}\), which is the square of the Frobenius
norm~\cite{Golub2013}, becomes a measure of how close a strategy is to being an
extortionate strategy. Suspicion
of extortion then corresponds to a threshold on \(\text{SSError}\).

By observing interactions (human or otherwise), their memory one representation
can be inferred and this approach can be used to recognise extortionate
behaviour. The notion of comparing theoretic and actual plays of the IPD is not
novel, see for example~\cite{Rand2013}. Immediately it is noted that if the
environment is noisy~\cite{Wu1995} then no strategy can be considered to be
extortionate as \(p_4>0\).

In the next section, this idea will be illustrated by observing the interactions
that take place in a computer based tournament of the IPD\@.

\section{Numerical experiments}\label{sec:numerical-experiments}

In~\cite{Stewart2012} results from a tournament with
\documentclass[a4paper]{article}

\usepackage{amsmath}
\usepackage{amssymb}
\usepackage[margin=1.5cm,
            includefoot,
            footskip=30pt]{geometry}
\usepackage{layout}
\usepackage{graphicx}
\usepackage{subcaption}

\usepackage{biblatex}
\usepackage{pdfpages}

\bibliography{main.bib}

\title{Suspicion: Recognising and evaluating the effectiveness
       of extortion in the Iterated Prisoner's Dilemma}
\author{Vincent A. Knight \and Nikoleta E. Glynatsi}
\date{\today}



\begin{document}

\maketitle

\begin{abstract}
    The Iterated Prisoner's Dilemma is a model for rational and evolutionary
    interactive behaviour. It has applications both in the study of human social
    behaviour as well as in biology.
    It is used to understand when and how a rational individual might
    accept an immediate cost to their own utility for the direct benefit of
    another.

    Much attention has been given to a class of strategies called
    Zero Determinant strategies. It has been theoretically shown that these
    strategies can ``extort'' any player.

    In this work, an approach to identify if observed strategies are playing in
    an extortionate way is described. Furthermore, experimental analysis of
    a large tournament with \input{assets/tex/number_of_full_strategies/main.tex}
    strategies is considered. In this setting
    the most highly performing strategies do not play in an extortionate way
    against each other but do against lower performing strategies.
    This suggests that whilst the theory of Zero Determinant strategies
    indicates that memory is not of fundamental importance to the evolution of
    cooperative behaviour, this is incomplete.
\end{abstract}

\section{Introduction}\label{sec:introduction}

Agent based game theoretic models have become a stalwart of the underpinning
mathematics of interactive behaviours. One of the major pieces of work
in this area is the pair of original computer tournaments run by Robert
Axelrod~\cite{Axelrod1980, Axelrod1980a}. These tournaments pitted submitted
computer strategies against each other in plays of the Iterated Prisoner's
Dilemma. A common game where agents can choose to pay a slight cost to their
immediate utility in the hope of building a reputation. This has been used in
economic and evolutionary game theory to understand the evolution of cooperative
behaviour.

Recently, a class of strategies was described in~\cite{Press2012} that can
provably extort any given opponent. In~\cite{Hilbe2013, Moran1707} some
questions have already been asked about the true effectiveness of these
strategies in an evolutionary setting. Here another question is asked: is it
possible to recognise this extortionate behaviour? A mathematical procedure for
suspicion is presented: in the same way that the continued actions of an
extortionate individual might raise suspicion.

This work makes use of the Axelrod Python library~\cite{Knight2018, Knight2016}
with a large number of Prisoner Dilemma strategies available to give an
extensive numerical example of the ideas presented.  The approach is presented
in Section~\ref{sec:delta-zd-strategies}.  All of the code and data discussed
in Section~\ref{sec:numerical-experiments} is open sourced, archived and
written according to best scientific principles~\cite{Wilson2014}. The data
archive can be found at~\cite{vincent_knight_2018_1297075}.

\section{Recognising Extortion}\label{sec:delta-zd-strategies}

In~\cite{Press2012}, given a match between 2 memory-one strategies, the concept
of Zero Determinant (ZD) strategies is introduced. The main result of that paper
shows that given two memory one players \(p, q\in\mathbb{R}^4\) a linear
relationship between the players' scores could be forced by one of the players.

Using the notation of~\cite{Press2012}, assuming the utilities for player \(p\)
are given by \(S_x=(R, S, T, P)\) and for player \(q\) by \(S_y=(R, T, S, P)\)
and that the stationary scores of each player is given by \(S_X\) and \(S_Y\)
respectively. The main result of~\cite{Press2012} is that if

\begin{equation}\label{eqn:linear_relationship_for_p}
    \tilde p=\alpha S_x + \beta S_y + \gamma
\end{equation}

or

\begin{equation}\label{eqn:linear_relationship_for_q}
    \tilde q=\alpha S_x + \beta S_y + \gamma
\end{equation}

where \(\tilde p = (1 - p_1, 1 - p_2, p_3, p_4)\) and
\(\tilde q = (1 - q_1, 1 - q_2, q_3, q_4)\) then:

\begin{equation}
    \alpha S_X + \beta S_Y + \gamma = 0
\end{equation}

In~\cite{Press2012} a particular type of ZD strategy is defined: extortionate
strategies. If:

\begin{equation}\label{eqn:constraint_for_extortion}
    \gamma = - P(\alpha + \beta)
\end{equation}

then the player can ensure they get a score \(\chi\) times
larger than the opponent. This extortion coefficient is given by:

\begin{equation}\label{eqn:definition_of_chi}
    \chi=\frac{-\beta}{\alpha}
\end{equation}

Thus, if (\ref{eqn:constraint_for_extortion}) holds and \(\chi >1\) a player is
said to extort their opponent.
Here, the reverse problem is considered: given a
\(p\in\mathbb{R}^4\) how does one identify \(\alpha, \beta\) if they
exist and is the strategy in fact acting in an extortionate way?

These conditions correspond to:

\begin{align}
    \tilde p_1 & = \alpha R + \beta R - P (\alpha + \beta)
            \label{eqn:condition_for_tilde_p1}\\
    \tilde p_2 & = \alpha S + \beta T - P (\alpha + \beta)
            \label{eqn:condition_for_tilde_p2}\\
    \tilde p_3 & = \alpha T + \beta S - P (\alpha + \beta)
            \label{eqn:condition_for_tilde_p3}\\
    \tilde p_4 & = \alpha P + \beta P - P (\alpha + \beta)
            \label{eqn:condition_for_tilde_p4}
\end{align}

Equation (\ref{eqn:condition_for_tilde_p4}) ensures that \(p_4=\tilde p_4=0\).
Equations (\ref{eqn:condition_for_tilde_p1}-\ref{eqn:condition_for_tilde_p3})
can be used to eliminate \(\alpha, \beta\), giving:

\begin{equation}\label{eqn:planar_definition_of_extortion}
    \tilde p_1 = \frac{(R - P)(\tilde p_2 + \tilde p_3)}{S + T - 2P}
\end{equation}

with:

\begin{equation}\label{eqn:definition_of_chi}
    \chi = \frac{\tilde p_2 (P - T) + \tilde p_3 (S - P)}
                {\tilde p_2 (P - S) + \tilde p_3 (T - P)}
\end{equation}

Given a strategy \(p\in\mathbb{R}^{4\times 1}\) equations
(\ref{eqn:condition_for_tilde_p4}), (\ref{eqn:planar_definition_of_extortion}-\ref{eqn:definition_of_chi}) can be used to check if
a strategy is extortionate. The conditions correspond to:

\begin{align}
    p_1 & = \frac{(R-P)(p_2 + p_3) - R + T + S - P}{S + T - 2P}
     \label{eqn:condition_for_p1}\\
    p_4 & = 0 \label{eqn:condition_for_p4}\\
    1 & > p_2 + p_3\label{eqn:condition_for_chi}
\end{align}

The algebraic steps necessary to prove these results are available in the
supporting materials.

All extortionate strategies reside on a triangular (\ref{eqn:condition_for_chi})
plane (\ref{eqn:condition_for_p1}) in 3 dimensions (\ref{eqn:condition_for_p4}).
Using this formulation it can be seen that a necessary (but not sufficient)
condition for an extortionate strategy is that it cooperates on average less
than 50\% of the time when in a state of disagreement with the opponent.

As an example, consider the known extortionate strategy \(p=(8 / 9, 1 / 2, 1 /
3, 0)\) from~\cite{Stewart2012} which is referred to as \texttt{Extort-2}. In
this case, for the standard values of \((R, T, S, P)\) constraint
(\ref{eqn:condition_for_p1}) corresponds to:

\begin{equation}
    p_1 = \frac{2(p_2 + p_3) + 1}{3}
\end{equation}

It is clear that in this case all constraints hold.

This approach could in fact be used to confirm that a given strategy is acting
in an extortionate manner even if it is not a memory one strategy. However, in
practice, if a closed form for \(p\) is not known, then due to measurement
and/or numerical error this would not work.

This problem can be written in the following linear algebraic form where
\(x=(\alpha, \beta)\)
and \(p^*=(\tilde p_1 - 1, tilde_2 - 1, p_3)\):

\begin{equation}\label{eqn:linear_algebraic_equation_for_p}
    Cx= p^*
\end{equation}

\(C\) corresponds to equations
(\ref{eqn:condition_for_tilde_p1}-\ref{eqn:condition_for_tilde_p3}) and is
given by:

\begin{equation}\label{eqn:definition_of_C}
    C =
    \begin{bmatrix}
        R - P & R- P \\
        S - P & T- P \\
        T - P & S- P \\
    \end{bmatrix}
\end{equation}

Note that in general, equation (\ref{eqn:linear_algebraic_equation_for_p}) will
not necessarily have a solution. From the Rouch\'{e}-Capelli theorem if there is
a solution it is unique as \(\text{rank}(C)=2\) which is the dimension of the
variable \(x\). The best fitting \(x\) is found by minimizing:

\begin{equation}\label{eqn:r_squared}
    \text{SSError} = \|C x- p^*\|_2^2 = \sum_{i=1}^{3}\left((C\bar x)_i-p_i^*\right)^2
\end{equation}

Note that \(\text{SSError}\), which is the square of the Frobenius
norm~\cite{Golub2013}, becomes a measure of how close a strategy is to being an
extortionate strategy. Suspicion
of extortion then corresponds to a threshold on \(\text{SSError}\).

By observing interactions (human or otherwise), their memory one representation
can be inferred and this approach can be used to recognise extortionate
behaviour. The notion of comparing theoretic and actual plays of the IPD is not
novel, see for example~\cite{Rand2013}. Immediately it is noted that if the
environment is noisy~\cite{Wu1995} then no strategy can be considered to be
extortionate as \(p_4>0\).

In the next section, this idea will be illustrated by observing the interactions
that take place in a computer based tournament of the IPD\@.

\section{Numerical experiments}\label{sec:numerical-experiments}

In~\cite{Stewart2012} results from a tournament with
\input{./assets/tex/number_of_stewart_plotkin_strategies/main.tex} strategies,
was presented with specific consideration given to ZD strategies. This
tournament is reproduced here using the Axelrod-Python
project~\cite{Knight2016}. To obtain a good measure of the corresponding
transition rates for each strategy all matches have been run for
\input{assets/tex/number_of_turns/main.tex} turns and every match has been
repeated \input{assets/tex/number_of_repetitions/main.tex} times. All of this
interaction data is available at~\cite{vincent_knight_2018_1297075}. A good
match between the inferred Markov chain and the state distribution of the actual
interactions has been verified. Data for this is presented in the supplementary
materials.

Figure~\ref{fig:SSError_overall_in_stewart_plotkin} shows the \(\text{SSError}\)
values for all the strategies in the tournament, as reported
in~\cite{Stewart2012} the extortionate strategy (which has an expected
\(\text{SSError}\) approximately 0) gains a large number of wins.

\begin{figure}[!htbp]
    \centering
    \includegraphics[width=.8\textwidth]{./assets/img/SSError_overall_in_stewart_plotkin/main.pdf}
    \caption{\(\text{SSError}\) and state probabilities for the strategies
        of~\cite{Stewart2012}, ordered both by number of wins and overall score.
        Note that \(P(DC)\) is not shown as it corresponds to the transpose of
        \(P(CD)\). Cooperator and Defector are omitted as they do not visit all
        the states.}
    \label{fig:SSError_overall_in_stewart_plotkin}
\end{figure}

Here, the work of~\cite{Stewart2012} is extended by investigating a tournament
with \input{assets/tex/number_of_full_strategies/main.tex}
strategies.

The results of this analysis are shown in
Figure~\ref{fig:SSError_and_probabilities_in_full}. The top ranking strategies
by number of wins seem to be extortionate (but not against all strategies) and
it can be seen that a small sub group of strategies achieve mutual defection.
All the top ranking strategies according to score achieve mutual cooperation and
do not extort each other, however they
\textbf{do} exhibit extortionate behaviour towards a number of the lower ranking
strategies.

\begin{figure}[!htbp]
    \centering
    \includegraphics[width=.8\textwidth]{./assets/img/SSError_and_probabilities_in_full/main.pdf}
    \caption{\(\text{SSError}\) for the strategies for the full tournament. Only
    strategy interactions for which \(p_4=0\) and \(\chi>1\) are displayed.}
    \label{fig:SSError_and_probabilities_in_full}
\end{figure}

\section{Conclusion}\label{sec:conclusion}

This work defines an approach to measure whether or not a player is playing a
strategy that corresponds to an extortionate strategy as defined
in~\cite{Press2012}: a mathematical model for suspicion. Indeed, all
extortionate strategies have been
 classified as lying on a triangular plane.
This rigorous classification fails to be robust to small measurement error, thus
a statistical approach is proposed.
This is done through a linear algebraic approach for approximating the solution
of a linear system. Using this, a large number of pairwise interactions is
simulated and in fact very few strategies are found to act extortionately.

The work of~\cite{Press2012}, whilst showing that a clever approach to taking
advantage of another memory one strategy exists: this is incomplete. Whilst the
elegance of this result is very attractive, just as the simplicity of the
victory of Tit For Tat in Axelrod's original tournaments was, it is incomplete.
Extortionate strategies achieve a high number of wins but they do not
achieve a high score which corresponds to the fitness landscape in an
evolutionary sense. From the large number of interactions a payoff matrix \(S\)
can be measured where \(S_{ij}\) denotes the score (using standard values of
\((R, S, T, P) = (3, 0, 5, 1)\)) of the \(i\)th strategy
against the \(j\)th strategy. Using this, the replicator equation
describes the evolution of the system based on a population density fitness
function:

\begin{equation}\label{eqn:replicator_dynamics}
    \frac{dx}{dt} = x(S-x^TS x)
\end{equation}

Equation (\ref{eqn:replicator_dynamics}) is solved numerically through an
integration technique described in~\cite{Petzold1983} and
Figure~\ref{fig:replicator_dynamics} shows the evolution of the distribution of
the system: the various strategies are ranked by scores. It is clear to see that
only the high ranking strategies survive the evolutionary process (in fact,
only \input{./assets/img/replicator_dynamics/main.tex}
have a final distribution greater than \(10 ^ {-2}\)). This confirms the
findings of~\cite{Moran1707} in which sophisticated strategies resist
evolutionary invasion of shorter memory strategies. Recalling
Figure~\ref{fig:SSError_and_probabilities_in_full} this demonstrates that:

\begin{itemize}
    \item Cooperation emerges through the evolutionary process: the high scoring
        strategies do not exhibit extortionate behaviour towards each other.
    \item Extortionate strategies do not survive the evolutionary process.
\end{itemize}

\begin{figure}[!htbp]
    \centering
    \includegraphics[width=.8\textwidth]{./assets/img/replicator_dynamics/main.pdf}
    \caption{Numerical simulation of the replicator equation
    (\ref{eqn:replicator_dynamics}): strategies are ordered by score, only the strategies with a high score survive the evolutionary process.}
    \label{fig:replicator_dynamics}
\end{figure}

This work can be used to classify plays of the IPD\@: data can be collected from
actual interactions (in lab or in the field). Furthermore, this allows for a
classification method similar to the notion of fingerprinting presented
in~\cite{Ashlock2008}. Trained strategies can potentially be classified as
extortionate or not or it could be possible to even constrain the reinforcement
learning approaches that are becoming prevalent in the literature.
Alternatively, this mathematical approach for recognising extortion could be
used in sophisticated strategies to defend against invasion. Arguably, some of
the strategies considered here exhibit this behaviour, indeed as described
in~\cite{Harper2017}, the top ranking strategies in the full tournament are
obtained using evolutionary reinforcement learning techniques, thus, suspicion
of extortionate behaviour could in fact be an evolutionary trait.

\section*{Acknowledgements}

The following open source software libraries were used in this research:

\begin{itemize}
    \item The Axelrod ~\cite{Knight2016, Knight2018} library (IPD strategies and
        tournaments).
    \item The sympy library~\cite{Meurer2017} (verification of all symbolic
        calculations).
    \item The matplotlib~\cite{Droettboom2018} library (visualisation).
    \item The pandas~\cite{Structures2010}, dask~\cite{Dask2016} and
        NumPy~\cite{Oliphant2015} libraries (data manipulation).
    \item The SciPy~\cite{Jones2001} library (numerical integration of the
        replicator equation).
\end{itemize}

This work was performed using the computational facilities of the Advanced
Research Computing @ Cardiff (ARCCA) Division, Cardiff University.

\printbibliography

\newpage
\section*{Supplementary materials}

\includepdf{assets/pdf/proof_of_form_of_extortionate_strategies/main.pdf}

\newpage

Using the pair wise interactions the transition rates \(p,
q\) can be measured and the steady state probabilities inferred and compared to
the actual probabilities of each state.
This is done numerically by computing the singular eigenvector of the
matrix \(A\) \cite{Stewart2009}:

\[
    A =
    \begin{bmatrix}
        p_1 q_1 & p_1 (1 - q_1) & (1 - p_1) q_1 & (1 -p_1) (1 - q_1) \\
        p_2 q_2 & p_2 (1 - q_2) & (1 - p_2) q_2 & (1 -p_2) (1 - q_2) \\
        p_3 q_3 & p_3 (1 - q_3) & (1 - p_3) q_3 & (1 -p_3) (1 - q_3) \\
        p_4 q_4 & p_4 (1 - q_4) & (1 - p_4) q_4 & (1 -p_4) (1 - q_4) \\
    \end{bmatrix}
\]

Figure~\ref{fig:computed_probabilities_vs_theoretic_probabilities} shows a
regression line fitted to every pairwise interaction with a reported
\(\text{SSError}\) value (pairwise interactions with missing states were
omitted). This serves to validate the approach: a part from some edge cases the
relationship is consistent.

\begin{figure}[!htbp]
    \centering
    \includegraphics[width=.8\textwidth]{./assets/img/computed_probabilities_vs_theoretic_probabilities/main.pdf}
    \caption{The
        relationship between the steady state probabilities inferred from the
        measured transitions and the actual steady state probabilities. A linear
        regression line is included validating the approach.}
    \label{fig:computed_probabilities_vs_theoretic_probabilities}
\end{figure}


\end{document}
 strategies,
was presented with specific consideration given to ZD strategies. This
tournament is reproduced here using the Axelrod-Python
project~\cite{Knight2016}. To obtain a good measure of the corresponding
transition rates for each strategy all matches have been run for
\documentclass[a4paper]{article}

\usepackage{amsmath}
\usepackage{amssymb}
\usepackage[margin=1.5cm,
            includefoot,
            footskip=30pt]{geometry}
\usepackage{layout}
\usepackage{graphicx}
\usepackage{subcaption}

\usepackage{biblatex}
\usepackage{pdfpages}

\bibliography{main.bib}

\title{Suspicion: Recognising and evaluating the effectiveness
       of extortion in the Iterated Prisoner's Dilemma}
\author{Vincent A. Knight \and Nikoleta E. Glynatsi}
\date{\today}



\begin{document}

\maketitle

\begin{abstract}
    The Iterated Prisoner's Dilemma is a model for rational and evolutionary
    interactive behaviour. It has applications both in the study of human social
    behaviour as well as in biology.
    It is used to understand when and how a rational individual might
    accept an immediate cost to their own utility for the direct benefit of
    another.

    Much attention has been given to a class of strategies called
    Zero Determinant strategies. It has been theoretically shown that these
    strategies can ``extort'' any player.

    In this work, an approach to identify if observed strategies are playing in
    an extortionate way is described. Furthermore, experimental analysis of
    a large tournament with \input{assets/tex/number_of_full_strategies/main.tex}
    strategies is considered. In this setting
    the most highly performing strategies do not play in an extortionate way
    against each other but do against lower performing strategies.
    This suggests that whilst the theory of Zero Determinant strategies
    indicates that memory is not of fundamental importance to the evolution of
    cooperative behaviour, this is incomplete.
\end{abstract}

\section{Introduction}\label{sec:introduction}

Agent based game theoretic models have become a stalwart of the underpinning
mathematics of interactive behaviours. One of the major pieces of work
in this area is the pair of original computer tournaments run by Robert
Axelrod~\cite{Axelrod1980, Axelrod1980a}. These tournaments pitted submitted
computer strategies against each other in plays of the Iterated Prisoner's
Dilemma. A common game where agents can choose to pay a slight cost to their
immediate utility in the hope of building a reputation. This has been used in
economic and evolutionary game theory to understand the evolution of cooperative
behaviour.

Recently, a class of strategies was described in~\cite{Press2012} that can
provably extort any given opponent. In~\cite{Hilbe2013, Moran1707} some
questions have already been asked about the true effectiveness of these
strategies in an evolutionary setting. Here another question is asked: is it
possible to recognise this extortionate behaviour? A mathematical procedure for
suspicion is presented: in the same way that the continued actions of an
extortionate individual might raise suspicion.

This work makes use of the Axelrod Python library~\cite{Knight2018, Knight2016}
with a large number of Prisoner Dilemma strategies available to give an
extensive numerical example of the ideas presented.  The approach is presented
in Section~\ref{sec:delta-zd-strategies}.  All of the code and data discussed
in Section~\ref{sec:numerical-experiments} is open sourced, archived and
written according to best scientific principles~\cite{Wilson2014}. The data
archive can be found at~\cite{vincent_knight_2018_1297075}.

\section{Recognising Extortion}\label{sec:delta-zd-strategies}

In~\cite{Press2012}, given a match between 2 memory-one strategies, the concept
of Zero Determinant (ZD) strategies is introduced. The main result of that paper
shows that given two memory one players \(p, q\in\mathbb{R}^4\) a linear
relationship between the players' scores could be forced by one of the players.

Using the notation of~\cite{Press2012}, assuming the utilities for player \(p\)
are given by \(S_x=(R, S, T, P)\) and for player \(q\) by \(S_y=(R, T, S, P)\)
and that the stationary scores of each player is given by \(S_X\) and \(S_Y\)
respectively. The main result of~\cite{Press2012} is that if

\begin{equation}\label{eqn:linear_relationship_for_p}
    \tilde p=\alpha S_x + \beta S_y + \gamma
\end{equation}

or

\begin{equation}\label{eqn:linear_relationship_for_q}
    \tilde q=\alpha S_x + \beta S_y + \gamma
\end{equation}

where \(\tilde p = (1 - p_1, 1 - p_2, p_3, p_4)\) and
\(\tilde q = (1 - q_1, 1 - q_2, q_3, q_4)\) then:

\begin{equation}
    \alpha S_X + \beta S_Y + \gamma = 0
\end{equation}

In~\cite{Press2012} a particular type of ZD strategy is defined: extortionate
strategies. If:

\begin{equation}\label{eqn:constraint_for_extortion}
    \gamma = - P(\alpha + \beta)
\end{equation}

then the player can ensure they get a score \(\chi\) times
larger than the opponent. This extortion coefficient is given by:

\begin{equation}\label{eqn:definition_of_chi}
    \chi=\frac{-\beta}{\alpha}
\end{equation}

Thus, if (\ref{eqn:constraint_for_extortion}) holds and \(\chi >1\) a player is
said to extort their opponent.
Here, the reverse problem is considered: given a
\(p\in\mathbb{R}^4\) how does one identify \(\alpha, \beta\) if they
exist and is the strategy in fact acting in an extortionate way?

These conditions correspond to:

\begin{align}
    \tilde p_1 & = \alpha R + \beta R - P (\alpha + \beta)
            \label{eqn:condition_for_tilde_p1}\\
    \tilde p_2 & = \alpha S + \beta T - P (\alpha + \beta)
            \label{eqn:condition_for_tilde_p2}\\
    \tilde p_3 & = \alpha T + \beta S - P (\alpha + \beta)
            \label{eqn:condition_for_tilde_p3}\\
    \tilde p_4 & = \alpha P + \beta P - P (\alpha + \beta)
            \label{eqn:condition_for_tilde_p4}
\end{align}

Equation (\ref{eqn:condition_for_tilde_p4}) ensures that \(p_4=\tilde p_4=0\).
Equations (\ref{eqn:condition_for_tilde_p1}-\ref{eqn:condition_for_tilde_p3})
can be used to eliminate \(\alpha, \beta\), giving:

\begin{equation}\label{eqn:planar_definition_of_extortion}
    \tilde p_1 = \frac{(R - P)(\tilde p_2 + \tilde p_3)}{S + T - 2P}
\end{equation}

with:

\begin{equation}\label{eqn:definition_of_chi}
    \chi = \frac{\tilde p_2 (P - T) + \tilde p_3 (S - P)}
                {\tilde p_2 (P - S) + \tilde p_3 (T - P)}
\end{equation}

Given a strategy \(p\in\mathbb{R}^{4\times 1}\) equations
(\ref{eqn:condition_for_tilde_p4}), (\ref{eqn:planar_definition_of_extortion}-\ref{eqn:definition_of_chi}) can be used to check if
a strategy is extortionate. The conditions correspond to:

\begin{align}
    p_1 & = \frac{(R-P)(p_2 + p_3) - R + T + S - P}{S + T - 2P}
     \label{eqn:condition_for_p1}\\
    p_4 & = 0 \label{eqn:condition_for_p4}\\
    1 & > p_2 + p_3\label{eqn:condition_for_chi}
\end{align}

The algebraic steps necessary to prove these results are available in the
supporting materials.

All extortionate strategies reside on a triangular (\ref{eqn:condition_for_chi})
plane (\ref{eqn:condition_for_p1}) in 3 dimensions (\ref{eqn:condition_for_p4}).
Using this formulation it can be seen that a necessary (but not sufficient)
condition for an extortionate strategy is that it cooperates on average less
than 50\% of the time when in a state of disagreement with the opponent.

As an example, consider the known extortionate strategy \(p=(8 / 9, 1 / 2, 1 /
3, 0)\) from~\cite{Stewart2012} which is referred to as \texttt{Extort-2}. In
this case, for the standard values of \((R, T, S, P)\) constraint
(\ref{eqn:condition_for_p1}) corresponds to:

\begin{equation}
    p_1 = \frac{2(p_2 + p_3) + 1}{3}
\end{equation}

It is clear that in this case all constraints hold.

This approach could in fact be used to confirm that a given strategy is acting
in an extortionate manner even if it is not a memory one strategy. However, in
practice, if a closed form for \(p\) is not known, then due to measurement
and/or numerical error this would not work.

This problem can be written in the following linear algebraic form where
\(x=(\alpha, \beta)\)
and \(p^*=(\tilde p_1 - 1, tilde_2 - 1, p_3)\):

\begin{equation}\label{eqn:linear_algebraic_equation_for_p}
    Cx= p^*
\end{equation}

\(C\) corresponds to equations
(\ref{eqn:condition_for_tilde_p1}-\ref{eqn:condition_for_tilde_p3}) and is
given by:

\begin{equation}\label{eqn:definition_of_C}
    C =
    \begin{bmatrix}
        R - P & R- P \\
        S - P & T- P \\
        T - P & S- P \\
    \end{bmatrix}
\end{equation}

Note that in general, equation (\ref{eqn:linear_algebraic_equation_for_p}) will
not necessarily have a solution. From the Rouch\'{e}-Capelli theorem if there is
a solution it is unique as \(\text{rank}(C)=2\) which is the dimension of the
variable \(x\). The best fitting \(x\) is found by minimizing:

\begin{equation}\label{eqn:r_squared}
    \text{SSError} = \|C x- p^*\|_2^2 = \sum_{i=1}^{3}\left((C\bar x)_i-p_i^*\right)^2
\end{equation}

Note that \(\text{SSError}\), which is the square of the Frobenius
norm~\cite{Golub2013}, becomes a measure of how close a strategy is to being an
extortionate strategy. Suspicion
of extortion then corresponds to a threshold on \(\text{SSError}\).

By observing interactions (human or otherwise), their memory one representation
can be inferred and this approach can be used to recognise extortionate
behaviour. The notion of comparing theoretic and actual plays of the IPD is not
novel, see for example~\cite{Rand2013}. Immediately it is noted that if the
environment is noisy~\cite{Wu1995} then no strategy can be considered to be
extortionate as \(p_4>0\).

In the next section, this idea will be illustrated by observing the interactions
that take place in a computer based tournament of the IPD\@.

\section{Numerical experiments}\label{sec:numerical-experiments}

In~\cite{Stewart2012} results from a tournament with
\input{./assets/tex/number_of_stewart_plotkin_strategies/main.tex} strategies,
was presented with specific consideration given to ZD strategies. This
tournament is reproduced here using the Axelrod-Python
project~\cite{Knight2016}. To obtain a good measure of the corresponding
transition rates for each strategy all matches have been run for
\input{assets/tex/number_of_turns/main.tex} turns and every match has been
repeated \input{assets/tex/number_of_repetitions/main.tex} times. All of this
interaction data is available at~\cite{vincent_knight_2018_1297075}. A good
match between the inferred Markov chain and the state distribution of the actual
interactions has been verified. Data for this is presented in the supplementary
materials.

Figure~\ref{fig:SSError_overall_in_stewart_plotkin} shows the \(\text{SSError}\)
values for all the strategies in the tournament, as reported
in~\cite{Stewart2012} the extortionate strategy (which has an expected
\(\text{SSError}\) approximately 0) gains a large number of wins.

\begin{figure}[!htbp]
    \centering
    \includegraphics[width=.8\textwidth]{./assets/img/SSError_overall_in_stewart_plotkin/main.pdf}
    \caption{\(\text{SSError}\) and state probabilities for the strategies
        of~\cite{Stewart2012}, ordered both by number of wins and overall score.
        Note that \(P(DC)\) is not shown as it corresponds to the transpose of
        \(P(CD)\). Cooperator and Defector are omitted as they do not visit all
        the states.}
    \label{fig:SSError_overall_in_stewart_plotkin}
\end{figure}

Here, the work of~\cite{Stewart2012} is extended by investigating a tournament
with \input{assets/tex/number_of_full_strategies/main.tex}
strategies.

The results of this analysis are shown in
Figure~\ref{fig:SSError_and_probabilities_in_full}. The top ranking strategies
by number of wins seem to be extortionate (but not against all strategies) and
it can be seen that a small sub group of strategies achieve mutual defection.
All the top ranking strategies according to score achieve mutual cooperation and
do not extort each other, however they
\textbf{do} exhibit extortionate behaviour towards a number of the lower ranking
strategies.

\begin{figure}[!htbp]
    \centering
    \includegraphics[width=.8\textwidth]{./assets/img/SSError_and_probabilities_in_full/main.pdf}
    \caption{\(\text{SSError}\) for the strategies for the full tournament. Only
    strategy interactions for which \(p_4=0\) and \(\chi>1\) are displayed.}
    \label{fig:SSError_and_probabilities_in_full}
\end{figure}

\section{Conclusion}\label{sec:conclusion}

This work defines an approach to measure whether or not a player is playing a
strategy that corresponds to an extortionate strategy as defined
in~\cite{Press2012}: a mathematical model for suspicion. Indeed, all
extortionate strategies have been
 classified as lying on a triangular plane.
This rigorous classification fails to be robust to small measurement error, thus
a statistical approach is proposed.
This is done through a linear algebraic approach for approximating the solution
of a linear system. Using this, a large number of pairwise interactions is
simulated and in fact very few strategies are found to act extortionately.

The work of~\cite{Press2012}, whilst showing that a clever approach to taking
advantage of another memory one strategy exists: this is incomplete. Whilst the
elegance of this result is very attractive, just as the simplicity of the
victory of Tit For Tat in Axelrod's original tournaments was, it is incomplete.
Extortionate strategies achieve a high number of wins but they do not
achieve a high score which corresponds to the fitness landscape in an
evolutionary sense. From the large number of interactions a payoff matrix \(S\)
can be measured where \(S_{ij}\) denotes the score (using standard values of
\((R, S, T, P) = (3, 0, 5, 1)\)) of the \(i\)th strategy
against the \(j\)th strategy. Using this, the replicator equation
describes the evolution of the system based on a population density fitness
function:

\begin{equation}\label{eqn:replicator_dynamics}
    \frac{dx}{dt} = x(S-x^TS x)
\end{equation}

Equation (\ref{eqn:replicator_dynamics}) is solved numerically through an
integration technique described in~\cite{Petzold1983} and
Figure~\ref{fig:replicator_dynamics} shows the evolution of the distribution of
the system: the various strategies are ranked by scores. It is clear to see that
only the high ranking strategies survive the evolutionary process (in fact,
only \input{./assets/img/replicator_dynamics/main.tex}
have a final distribution greater than \(10 ^ {-2}\)). This confirms the
findings of~\cite{Moran1707} in which sophisticated strategies resist
evolutionary invasion of shorter memory strategies. Recalling
Figure~\ref{fig:SSError_and_probabilities_in_full} this demonstrates that:

\begin{itemize}
    \item Cooperation emerges through the evolutionary process: the high scoring
        strategies do not exhibit extortionate behaviour towards each other.
    \item Extortionate strategies do not survive the evolutionary process.
\end{itemize}

\begin{figure}[!htbp]
    \centering
    \includegraphics[width=.8\textwidth]{./assets/img/replicator_dynamics/main.pdf}
    \caption{Numerical simulation of the replicator equation
    (\ref{eqn:replicator_dynamics}): strategies are ordered by score, only the strategies with a high score survive the evolutionary process.}
    \label{fig:replicator_dynamics}
\end{figure}

This work can be used to classify plays of the IPD\@: data can be collected from
actual interactions (in lab or in the field). Furthermore, this allows for a
classification method similar to the notion of fingerprinting presented
in~\cite{Ashlock2008}. Trained strategies can potentially be classified as
extortionate or not or it could be possible to even constrain the reinforcement
learning approaches that are becoming prevalent in the literature.
Alternatively, this mathematical approach for recognising extortion could be
used in sophisticated strategies to defend against invasion. Arguably, some of
the strategies considered here exhibit this behaviour, indeed as described
in~\cite{Harper2017}, the top ranking strategies in the full tournament are
obtained using evolutionary reinforcement learning techniques, thus, suspicion
of extortionate behaviour could in fact be an evolutionary trait.

\section*{Acknowledgements}

The following open source software libraries were used in this research:

\begin{itemize}
    \item The Axelrod ~\cite{Knight2016, Knight2018} library (IPD strategies and
        tournaments).
    \item The sympy library~\cite{Meurer2017} (verification of all symbolic
        calculations).
    \item The matplotlib~\cite{Droettboom2018} library (visualisation).
    \item The pandas~\cite{Structures2010}, dask~\cite{Dask2016} and
        NumPy~\cite{Oliphant2015} libraries (data manipulation).
    \item The SciPy~\cite{Jones2001} library (numerical integration of the
        replicator equation).
\end{itemize}

This work was performed using the computational facilities of the Advanced
Research Computing @ Cardiff (ARCCA) Division, Cardiff University.

\printbibliography

\newpage
\section*{Supplementary materials}

\includepdf{assets/pdf/proof_of_form_of_extortionate_strategies/main.pdf}

\newpage

Using the pair wise interactions the transition rates \(p,
q\) can be measured and the steady state probabilities inferred and compared to
the actual probabilities of each state.
This is done numerically by computing the singular eigenvector of the
matrix \(A\) \cite{Stewart2009}:

\[
    A =
    \begin{bmatrix}
        p_1 q_1 & p_1 (1 - q_1) & (1 - p_1) q_1 & (1 -p_1) (1 - q_1) \\
        p_2 q_2 & p_2 (1 - q_2) & (1 - p_2) q_2 & (1 -p_2) (1 - q_2) \\
        p_3 q_3 & p_3 (1 - q_3) & (1 - p_3) q_3 & (1 -p_3) (1 - q_3) \\
        p_4 q_4 & p_4 (1 - q_4) & (1 - p_4) q_4 & (1 -p_4) (1 - q_4) \\
    \end{bmatrix}
\]

Figure~\ref{fig:computed_probabilities_vs_theoretic_probabilities} shows a
regression line fitted to every pairwise interaction with a reported
\(\text{SSError}\) value (pairwise interactions with missing states were
omitted). This serves to validate the approach: a part from some edge cases the
relationship is consistent.

\begin{figure}[!htbp]
    \centering
    \includegraphics[width=.8\textwidth]{./assets/img/computed_probabilities_vs_theoretic_probabilities/main.pdf}
    \caption{The
        relationship between the steady state probabilities inferred from the
        measured transitions and the actual steady state probabilities. A linear
        regression line is included validating the approach.}
    \label{fig:computed_probabilities_vs_theoretic_probabilities}
\end{figure}


\end{document}
 turns and every match has been
repeated \documentclass[a4paper]{article}

\usepackage{amsmath}
\usepackage{amssymb}
\usepackage[margin=1.5cm,
            includefoot,
            footskip=30pt]{geometry}
\usepackage{layout}
\usepackage{graphicx}
\usepackage{subcaption}

\usepackage{biblatex}
\usepackage{pdfpages}

\bibliography{main.bib}

\title{Suspicion: Recognising and evaluating the effectiveness
       of extortion in the Iterated Prisoner's Dilemma}
\author{Vincent A. Knight \and Nikoleta E. Glynatsi}
\date{\today}



\begin{document}

\maketitle

\begin{abstract}
    The Iterated Prisoner's Dilemma is a model for rational and evolutionary
    interactive behaviour. It has applications both in the study of human social
    behaviour as well as in biology.
    It is used to understand when and how a rational individual might
    accept an immediate cost to their own utility for the direct benefit of
    another.

    Much attention has been given to a class of strategies called
    Zero Determinant strategies. It has been theoretically shown that these
    strategies can ``extort'' any player.

    In this work, an approach to identify if observed strategies are playing in
    an extortionate way is described. Furthermore, experimental analysis of
    a large tournament with \input{assets/tex/number_of_full_strategies/main.tex}
    strategies is considered. In this setting
    the most highly performing strategies do not play in an extortionate way
    against each other but do against lower performing strategies.
    This suggests that whilst the theory of Zero Determinant strategies
    indicates that memory is not of fundamental importance to the evolution of
    cooperative behaviour, this is incomplete.
\end{abstract}

\section{Introduction}\label{sec:introduction}

Agent based game theoretic models have become a stalwart of the underpinning
mathematics of interactive behaviours. One of the major pieces of work
in this area is the pair of original computer tournaments run by Robert
Axelrod~\cite{Axelrod1980, Axelrod1980a}. These tournaments pitted submitted
computer strategies against each other in plays of the Iterated Prisoner's
Dilemma. A common game where agents can choose to pay a slight cost to their
immediate utility in the hope of building a reputation. This has been used in
economic and evolutionary game theory to understand the evolution of cooperative
behaviour.

Recently, a class of strategies was described in~\cite{Press2012} that can
provably extort any given opponent. In~\cite{Hilbe2013, Moran1707} some
questions have already been asked about the true effectiveness of these
strategies in an evolutionary setting. Here another question is asked: is it
possible to recognise this extortionate behaviour? A mathematical procedure for
suspicion is presented: in the same way that the continued actions of an
extortionate individual might raise suspicion.

This work makes use of the Axelrod Python library~\cite{Knight2018, Knight2016}
with a large number of Prisoner Dilemma strategies available to give an
extensive numerical example of the ideas presented.  The approach is presented
in Section~\ref{sec:delta-zd-strategies}.  All of the code and data discussed
in Section~\ref{sec:numerical-experiments} is open sourced, archived and
written according to best scientific principles~\cite{Wilson2014}. The data
archive can be found at~\cite{vincent_knight_2018_1297075}.

\section{Recognising Extortion}\label{sec:delta-zd-strategies}

In~\cite{Press2012}, given a match between 2 memory-one strategies, the concept
of Zero Determinant (ZD) strategies is introduced. The main result of that paper
shows that given two memory one players \(p, q\in\mathbb{R}^4\) a linear
relationship between the players' scores could be forced by one of the players.

Using the notation of~\cite{Press2012}, assuming the utilities for player \(p\)
are given by \(S_x=(R, S, T, P)\) and for player \(q\) by \(S_y=(R, T, S, P)\)
and that the stationary scores of each player is given by \(S_X\) and \(S_Y\)
respectively. The main result of~\cite{Press2012} is that if

\begin{equation}\label{eqn:linear_relationship_for_p}
    \tilde p=\alpha S_x + \beta S_y + \gamma
\end{equation}

or

\begin{equation}\label{eqn:linear_relationship_for_q}
    \tilde q=\alpha S_x + \beta S_y + \gamma
\end{equation}

where \(\tilde p = (1 - p_1, 1 - p_2, p_3, p_4)\) and
\(\tilde q = (1 - q_1, 1 - q_2, q_3, q_4)\) then:

\begin{equation}
    \alpha S_X + \beta S_Y + \gamma = 0
\end{equation}

In~\cite{Press2012} a particular type of ZD strategy is defined: extortionate
strategies. If:

\begin{equation}\label{eqn:constraint_for_extortion}
    \gamma = - P(\alpha + \beta)
\end{equation}

then the player can ensure they get a score \(\chi\) times
larger than the opponent. This extortion coefficient is given by:

\begin{equation}\label{eqn:definition_of_chi}
    \chi=\frac{-\beta}{\alpha}
\end{equation}

Thus, if (\ref{eqn:constraint_for_extortion}) holds and \(\chi >1\) a player is
said to extort their opponent.
Here, the reverse problem is considered: given a
\(p\in\mathbb{R}^4\) how does one identify \(\alpha, \beta\) if they
exist and is the strategy in fact acting in an extortionate way?

These conditions correspond to:

\begin{align}
    \tilde p_1 & = \alpha R + \beta R - P (\alpha + \beta)
            \label{eqn:condition_for_tilde_p1}\\
    \tilde p_2 & = \alpha S + \beta T - P (\alpha + \beta)
            \label{eqn:condition_for_tilde_p2}\\
    \tilde p_3 & = \alpha T + \beta S - P (\alpha + \beta)
            \label{eqn:condition_for_tilde_p3}\\
    \tilde p_4 & = \alpha P + \beta P - P (\alpha + \beta)
            \label{eqn:condition_for_tilde_p4}
\end{align}

Equation (\ref{eqn:condition_for_tilde_p4}) ensures that \(p_4=\tilde p_4=0\).
Equations (\ref{eqn:condition_for_tilde_p1}-\ref{eqn:condition_for_tilde_p3})
can be used to eliminate \(\alpha, \beta\), giving:

\begin{equation}\label{eqn:planar_definition_of_extortion}
    \tilde p_1 = \frac{(R - P)(\tilde p_2 + \tilde p_3)}{S + T - 2P}
\end{equation}

with:

\begin{equation}\label{eqn:definition_of_chi}
    \chi = \frac{\tilde p_2 (P - T) + \tilde p_3 (S - P)}
                {\tilde p_2 (P - S) + \tilde p_3 (T - P)}
\end{equation}

Given a strategy \(p\in\mathbb{R}^{4\times 1}\) equations
(\ref{eqn:condition_for_tilde_p4}), (\ref{eqn:planar_definition_of_extortion}-\ref{eqn:definition_of_chi}) can be used to check if
a strategy is extortionate. The conditions correspond to:

\begin{align}
    p_1 & = \frac{(R-P)(p_2 + p_3) - R + T + S - P}{S + T - 2P}
     \label{eqn:condition_for_p1}\\
    p_4 & = 0 \label{eqn:condition_for_p4}\\
    1 & > p_2 + p_3\label{eqn:condition_for_chi}
\end{align}

The algebraic steps necessary to prove these results are available in the
supporting materials.

All extortionate strategies reside on a triangular (\ref{eqn:condition_for_chi})
plane (\ref{eqn:condition_for_p1}) in 3 dimensions (\ref{eqn:condition_for_p4}).
Using this formulation it can be seen that a necessary (but not sufficient)
condition for an extortionate strategy is that it cooperates on average less
than 50\% of the time when in a state of disagreement with the opponent.

As an example, consider the known extortionate strategy \(p=(8 / 9, 1 / 2, 1 /
3, 0)\) from~\cite{Stewart2012} which is referred to as \texttt{Extort-2}. In
this case, for the standard values of \((R, T, S, P)\) constraint
(\ref{eqn:condition_for_p1}) corresponds to:

\begin{equation}
    p_1 = \frac{2(p_2 + p_3) + 1}{3}
\end{equation}

It is clear that in this case all constraints hold.

This approach could in fact be used to confirm that a given strategy is acting
in an extortionate manner even if it is not a memory one strategy. However, in
practice, if a closed form for \(p\) is not known, then due to measurement
and/or numerical error this would not work.

This problem can be written in the following linear algebraic form where
\(x=(\alpha, \beta)\)
and \(p^*=(\tilde p_1 - 1, tilde_2 - 1, p_3)\):

\begin{equation}\label{eqn:linear_algebraic_equation_for_p}
    Cx= p^*
\end{equation}

\(C\) corresponds to equations
(\ref{eqn:condition_for_tilde_p1}-\ref{eqn:condition_for_tilde_p3}) and is
given by:

\begin{equation}\label{eqn:definition_of_C}
    C =
    \begin{bmatrix}
        R - P & R- P \\
        S - P & T- P \\
        T - P & S- P \\
    \end{bmatrix}
\end{equation}

Note that in general, equation (\ref{eqn:linear_algebraic_equation_for_p}) will
not necessarily have a solution. From the Rouch\'{e}-Capelli theorem if there is
a solution it is unique as \(\text{rank}(C)=2\) which is the dimension of the
variable \(x\). The best fitting \(x\) is found by minimizing:

\begin{equation}\label{eqn:r_squared}
    \text{SSError} = \|C x- p^*\|_2^2 = \sum_{i=1}^{3}\left((C\bar x)_i-p_i^*\right)^2
\end{equation}

Note that \(\text{SSError}\), which is the square of the Frobenius
norm~\cite{Golub2013}, becomes a measure of how close a strategy is to being an
extortionate strategy. Suspicion
of extortion then corresponds to a threshold on \(\text{SSError}\).

By observing interactions (human or otherwise), their memory one representation
can be inferred and this approach can be used to recognise extortionate
behaviour. The notion of comparing theoretic and actual plays of the IPD is not
novel, see for example~\cite{Rand2013}. Immediately it is noted that if the
environment is noisy~\cite{Wu1995} then no strategy can be considered to be
extortionate as \(p_4>0\).

In the next section, this idea will be illustrated by observing the interactions
that take place in a computer based tournament of the IPD\@.

\section{Numerical experiments}\label{sec:numerical-experiments}

In~\cite{Stewart2012} results from a tournament with
\input{./assets/tex/number_of_stewart_plotkin_strategies/main.tex} strategies,
was presented with specific consideration given to ZD strategies. This
tournament is reproduced here using the Axelrod-Python
project~\cite{Knight2016}. To obtain a good measure of the corresponding
transition rates for each strategy all matches have been run for
\input{assets/tex/number_of_turns/main.tex} turns and every match has been
repeated \input{assets/tex/number_of_repetitions/main.tex} times. All of this
interaction data is available at~\cite{vincent_knight_2018_1297075}. A good
match between the inferred Markov chain and the state distribution of the actual
interactions has been verified. Data for this is presented in the supplementary
materials.

Figure~\ref{fig:SSError_overall_in_stewart_plotkin} shows the \(\text{SSError}\)
values for all the strategies in the tournament, as reported
in~\cite{Stewart2012} the extortionate strategy (which has an expected
\(\text{SSError}\) approximately 0) gains a large number of wins.

\begin{figure}[!htbp]
    \centering
    \includegraphics[width=.8\textwidth]{./assets/img/SSError_overall_in_stewart_plotkin/main.pdf}
    \caption{\(\text{SSError}\) and state probabilities for the strategies
        of~\cite{Stewart2012}, ordered both by number of wins and overall score.
        Note that \(P(DC)\) is not shown as it corresponds to the transpose of
        \(P(CD)\). Cooperator and Defector are omitted as they do not visit all
        the states.}
    \label{fig:SSError_overall_in_stewart_plotkin}
\end{figure}

Here, the work of~\cite{Stewart2012} is extended by investigating a tournament
with \input{assets/tex/number_of_full_strategies/main.tex}
strategies.

The results of this analysis are shown in
Figure~\ref{fig:SSError_and_probabilities_in_full}. The top ranking strategies
by number of wins seem to be extortionate (but not against all strategies) and
it can be seen that a small sub group of strategies achieve mutual defection.
All the top ranking strategies according to score achieve mutual cooperation and
do not extort each other, however they
\textbf{do} exhibit extortionate behaviour towards a number of the lower ranking
strategies.

\begin{figure}[!htbp]
    \centering
    \includegraphics[width=.8\textwidth]{./assets/img/SSError_and_probabilities_in_full/main.pdf}
    \caption{\(\text{SSError}\) for the strategies for the full tournament. Only
    strategy interactions for which \(p_4=0\) and \(\chi>1\) are displayed.}
    \label{fig:SSError_and_probabilities_in_full}
\end{figure}

\section{Conclusion}\label{sec:conclusion}

This work defines an approach to measure whether or not a player is playing a
strategy that corresponds to an extortionate strategy as defined
in~\cite{Press2012}: a mathematical model for suspicion. Indeed, all
extortionate strategies have been
 classified as lying on a triangular plane.
This rigorous classification fails to be robust to small measurement error, thus
a statistical approach is proposed.
This is done through a linear algebraic approach for approximating the solution
of a linear system. Using this, a large number of pairwise interactions is
simulated and in fact very few strategies are found to act extortionately.

The work of~\cite{Press2012}, whilst showing that a clever approach to taking
advantage of another memory one strategy exists: this is incomplete. Whilst the
elegance of this result is very attractive, just as the simplicity of the
victory of Tit For Tat in Axelrod's original tournaments was, it is incomplete.
Extortionate strategies achieve a high number of wins but they do not
achieve a high score which corresponds to the fitness landscape in an
evolutionary sense. From the large number of interactions a payoff matrix \(S\)
can be measured where \(S_{ij}\) denotes the score (using standard values of
\((R, S, T, P) = (3, 0, 5, 1)\)) of the \(i\)th strategy
against the \(j\)th strategy. Using this, the replicator equation
describes the evolution of the system based on a population density fitness
function:

\begin{equation}\label{eqn:replicator_dynamics}
    \frac{dx}{dt} = x(S-x^TS x)
\end{equation}

Equation (\ref{eqn:replicator_dynamics}) is solved numerically through an
integration technique described in~\cite{Petzold1983} and
Figure~\ref{fig:replicator_dynamics} shows the evolution of the distribution of
the system: the various strategies are ranked by scores. It is clear to see that
only the high ranking strategies survive the evolutionary process (in fact,
only \input{./assets/img/replicator_dynamics/main.tex}
have a final distribution greater than \(10 ^ {-2}\)). This confirms the
findings of~\cite{Moran1707} in which sophisticated strategies resist
evolutionary invasion of shorter memory strategies. Recalling
Figure~\ref{fig:SSError_and_probabilities_in_full} this demonstrates that:

\begin{itemize}
    \item Cooperation emerges through the evolutionary process: the high scoring
        strategies do not exhibit extortionate behaviour towards each other.
    \item Extortionate strategies do not survive the evolutionary process.
\end{itemize}

\begin{figure}[!htbp]
    \centering
    \includegraphics[width=.8\textwidth]{./assets/img/replicator_dynamics/main.pdf}
    \caption{Numerical simulation of the replicator equation
    (\ref{eqn:replicator_dynamics}): strategies are ordered by score, only the strategies with a high score survive the evolutionary process.}
    \label{fig:replicator_dynamics}
\end{figure}

This work can be used to classify plays of the IPD\@: data can be collected from
actual interactions (in lab or in the field). Furthermore, this allows for a
classification method similar to the notion of fingerprinting presented
in~\cite{Ashlock2008}. Trained strategies can potentially be classified as
extortionate or not or it could be possible to even constrain the reinforcement
learning approaches that are becoming prevalent in the literature.
Alternatively, this mathematical approach for recognising extortion could be
used in sophisticated strategies to defend against invasion. Arguably, some of
the strategies considered here exhibit this behaviour, indeed as described
in~\cite{Harper2017}, the top ranking strategies in the full tournament are
obtained using evolutionary reinforcement learning techniques, thus, suspicion
of extortionate behaviour could in fact be an evolutionary trait.

\section*{Acknowledgements}

The following open source software libraries were used in this research:

\begin{itemize}
    \item The Axelrod ~\cite{Knight2016, Knight2018} library (IPD strategies and
        tournaments).
    \item The sympy library~\cite{Meurer2017} (verification of all symbolic
        calculations).
    \item The matplotlib~\cite{Droettboom2018} library (visualisation).
    \item The pandas~\cite{Structures2010}, dask~\cite{Dask2016} and
        NumPy~\cite{Oliphant2015} libraries (data manipulation).
    \item The SciPy~\cite{Jones2001} library (numerical integration of the
        replicator equation).
\end{itemize}

This work was performed using the computational facilities of the Advanced
Research Computing @ Cardiff (ARCCA) Division, Cardiff University.

\printbibliography

\newpage
\section*{Supplementary materials}

\includepdf{assets/pdf/proof_of_form_of_extortionate_strategies/main.pdf}

\newpage

Using the pair wise interactions the transition rates \(p,
q\) can be measured and the steady state probabilities inferred and compared to
the actual probabilities of each state.
This is done numerically by computing the singular eigenvector of the
matrix \(A\) \cite{Stewart2009}:

\[
    A =
    \begin{bmatrix}
        p_1 q_1 & p_1 (1 - q_1) & (1 - p_1) q_1 & (1 -p_1) (1 - q_1) \\
        p_2 q_2 & p_2 (1 - q_2) & (1 - p_2) q_2 & (1 -p_2) (1 - q_2) \\
        p_3 q_3 & p_3 (1 - q_3) & (1 - p_3) q_3 & (1 -p_3) (1 - q_3) \\
        p_4 q_4 & p_4 (1 - q_4) & (1 - p_4) q_4 & (1 -p_4) (1 - q_4) \\
    \end{bmatrix}
\]

Figure~\ref{fig:computed_probabilities_vs_theoretic_probabilities} shows a
regression line fitted to every pairwise interaction with a reported
\(\text{SSError}\) value (pairwise interactions with missing states were
omitted). This serves to validate the approach: a part from some edge cases the
relationship is consistent.

\begin{figure}[!htbp]
    \centering
    \includegraphics[width=.8\textwidth]{./assets/img/computed_probabilities_vs_theoretic_probabilities/main.pdf}
    \caption{The
        relationship between the steady state probabilities inferred from the
        measured transitions and the actual steady state probabilities. A linear
        regression line is included validating the approach.}
    \label{fig:computed_probabilities_vs_theoretic_probabilities}
\end{figure}


\end{document}
 times. All of this
interaction data is available at~\cite{vincent_knight_2018_1297075}. A good
match between the inferred Markov chain and the state distribution of the actual
interactions has been verified. Data for this is presented in the supplementary
materials.

Figure~\ref{fig:SSError_overall_in_stewart_plotkin} shows the \(\text{SSError}\)
values for all the strategies in the tournament, as reported
in~\cite{Stewart2012} the extortionate strategy (which has an expected
\(\text{SSError}\) approximately 0) gains a large number of wins.

\begin{figure}[!htbp]
    \centering
    \includegraphics[width=.8\textwidth]{./assets/img/SSError_overall_in_stewart_plotkin/main.pdf}
    \caption{\(\text{SSError}\) and state probabilities for the strategies
        of~\cite{Stewart2012}, ordered both by number of wins and overall score.
        Note that \(P(DC)\) is not shown as it corresponds to the transpose of
        \(P(CD)\). Cooperator and Defector are omitted as they do not visit all
        the states.}
    \label{fig:SSError_overall_in_stewart_plotkin}
\end{figure}

Here, the work of~\cite{Stewart2012} is extended by investigating a tournament
with \documentclass[a4paper]{article}

\usepackage{amsmath}
\usepackage{amssymb}
\usepackage[margin=1.5cm,
            includefoot,
            footskip=30pt]{geometry}
\usepackage{layout}
\usepackage{graphicx}
\usepackage{subcaption}

\usepackage{biblatex}
\usepackage{pdfpages}

\bibliography{main.bib}

\title{Suspicion: Recognising and evaluating the effectiveness
       of extortion in the Iterated Prisoner's Dilemma}
\author{Vincent A. Knight \and Nikoleta E. Glynatsi}
\date{\today}



\begin{document}

\maketitle

\begin{abstract}
    The Iterated Prisoner's Dilemma is a model for rational and evolutionary
    interactive behaviour. It has applications both in the study of human social
    behaviour as well as in biology.
    It is used to understand when and how a rational individual might
    accept an immediate cost to their own utility for the direct benefit of
    another.

    Much attention has been given to a class of strategies called
    Zero Determinant strategies. It has been theoretically shown that these
    strategies can ``extort'' any player.

    In this work, an approach to identify if observed strategies are playing in
    an extortionate way is described. Furthermore, experimental analysis of
    a large tournament with \input{assets/tex/number_of_full_strategies/main.tex}
    strategies is considered. In this setting
    the most highly performing strategies do not play in an extortionate way
    against each other but do against lower performing strategies.
    This suggests that whilst the theory of Zero Determinant strategies
    indicates that memory is not of fundamental importance to the evolution of
    cooperative behaviour, this is incomplete.
\end{abstract}

\section{Introduction}\label{sec:introduction}

Agent based game theoretic models have become a stalwart of the underpinning
mathematics of interactive behaviours. One of the major pieces of work
in this area is the pair of original computer tournaments run by Robert
Axelrod~\cite{Axelrod1980, Axelrod1980a}. These tournaments pitted submitted
computer strategies against each other in plays of the Iterated Prisoner's
Dilemma. A common game where agents can choose to pay a slight cost to their
immediate utility in the hope of building a reputation. This has been used in
economic and evolutionary game theory to understand the evolution of cooperative
behaviour.

Recently, a class of strategies was described in~\cite{Press2012} that can
provably extort any given opponent. In~\cite{Hilbe2013, Moran1707} some
questions have already been asked about the true effectiveness of these
strategies in an evolutionary setting. Here another question is asked: is it
possible to recognise this extortionate behaviour? A mathematical procedure for
suspicion is presented: in the same way that the continued actions of an
extortionate individual might raise suspicion.

This work makes use of the Axelrod Python library~\cite{Knight2018, Knight2016}
with a large number of Prisoner Dilemma strategies available to give an
extensive numerical example of the ideas presented.  The approach is presented
in Section~\ref{sec:delta-zd-strategies}.  All of the code and data discussed
in Section~\ref{sec:numerical-experiments} is open sourced, archived and
written according to best scientific principles~\cite{Wilson2014}. The data
archive can be found at~\cite{vincent_knight_2018_1297075}.

\section{Recognising Extortion}\label{sec:delta-zd-strategies}

In~\cite{Press2012}, given a match between 2 memory-one strategies, the concept
of Zero Determinant (ZD) strategies is introduced. The main result of that paper
shows that given two memory one players \(p, q\in\mathbb{R}^4\) a linear
relationship between the players' scores could be forced by one of the players.

Using the notation of~\cite{Press2012}, assuming the utilities for player \(p\)
are given by \(S_x=(R, S, T, P)\) and for player \(q\) by \(S_y=(R, T, S, P)\)
and that the stationary scores of each player is given by \(S_X\) and \(S_Y\)
respectively. The main result of~\cite{Press2012} is that if

\begin{equation}\label{eqn:linear_relationship_for_p}
    \tilde p=\alpha S_x + \beta S_y + \gamma
\end{equation}

or

\begin{equation}\label{eqn:linear_relationship_for_q}
    \tilde q=\alpha S_x + \beta S_y + \gamma
\end{equation}

where \(\tilde p = (1 - p_1, 1 - p_2, p_3, p_4)\) and
\(\tilde q = (1 - q_1, 1 - q_2, q_3, q_4)\) then:

\begin{equation}
    \alpha S_X + \beta S_Y + \gamma = 0
\end{equation}

In~\cite{Press2012} a particular type of ZD strategy is defined: extortionate
strategies. If:

\begin{equation}\label{eqn:constraint_for_extortion}
    \gamma = - P(\alpha + \beta)
\end{equation}

then the player can ensure they get a score \(\chi\) times
larger than the opponent. This extortion coefficient is given by:

\begin{equation}\label{eqn:definition_of_chi}
    \chi=\frac{-\beta}{\alpha}
\end{equation}

Thus, if (\ref{eqn:constraint_for_extortion}) holds and \(\chi >1\) a player is
said to extort their opponent.
Here, the reverse problem is considered: given a
\(p\in\mathbb{R}^4\) how does one identify \(\alpha, \beta\) if they
exist and is the strategy in fact acting in an extortionate way?

These conditions correspond to:

\begin{align}
    \tilde p_1 & = \alpha R + \beta R - P (\alpha + \beta)
            \label{eqn:condition_for_tilde_p1}\\
    \tilde p_2 & = \alpha S + \beta T - P (\alpha + \beta)
            \label{eqn:condition_for_tilde_p2}\\
    \tilde p_3 & = \alpha T + \beta S - P (\alpha + \beta)
            \label{eqn:condition_for_tilde_p3}\\
    \tilde p_4 & = \alpha P + \beta P - P (\alpha + \beta)
            \label{eqn:condition_for_tilde_p4}
\end{align}

Equation (\ref{eqn:condition_for_tilde_p4}) ensures that \(p_4=\tilde p_4=0\).
Equations (\ref{eqn:condition_for_tilde_p1}-\ref{eqn:condition_for_tilde_p3})
can be used to eliminate \(\alpha, \beta\), giving:

\begin{equation}\label{eqn:planar_definition_of_extortion}
    \tilde p_1 = \frac{(R - P)(\tilde p_2 + \tilde p_3)}{S + T - 2P}
\end{equation}

with:

\begin{equation}\label{eqn:definition_of_chi}
    \chi = \frac{\tilde p_2 (P - T) + \tilde p_3 (S - P)}
                {\tilde p_2 (P - S) + \tilde p_3 (T - P)}
\end{equation}

Given a strategy \(p\in\mathbb{R}^{4\times 1}\) equations
(\ref{eqn:condition_for_tilde_p4}), (\ref{eqn:planar_definition_of_extortion}-\ref{eqn:definition_of_chi}) can be used to check if
a strategy is extortionate. The conditions correspond to:

\begin{align}
    p_1 & = \frac{(R-P)(p_2 + p_3) - R + T + S - P}{S + T - 2P}
     \label{eqn:condition_for_p1}\\
    p_4 & = 0 \label{eqn:condition_for_p4}\\
    1 & > p_2 + p_3\label{eqn:condition_for_chi}
\end{align}

The algebraic steps necessary to prove these results are available in the
supporting materials.

All extortionate strategies reside on a triangular (\ref{eqn:condition_for_chi})
plane (\ref{eqn:condition_for_p1}) in 3 dimensions (\ref{eqn:condition_for_p4}).
Using this formulation it can be seen that a necessary (but not sufficient)
condition for an extortionate strategy is that it cooperates on average less
than 50\% of the time when in a state of disagreement with the opponent.

As an example, consider the known extortionate strategy \(p=(8 / 9, 1 / 2, 1 /
3, 0)\) from~\cite{Stewart2012} which is referred to as \texttt{Extort-2}. In
this case, for the standard values of \((R, T, S, P)\) constraint
(\ref{eqn:condition_for_p1}) corresponds to:

\begin{equation}
    p_1 = \frac{2(p_2 + p_3) + 1}{3}
\end{equation}

It is clear that in this case all constraints hold.

This approach could in fact be used to confirm that a given strategy is acting
in an extortionate manner even if it is not a memory one strategy. However, in
practice, if a closed form for \(p\) is not known, then due to measurement
and/or numerical error this would not work.

This problem can be written in the following linear algebraic form where
\(x=(\alpha, \beta)\)
and \(p^*=(\tilde p_1 - 1, tilde_2 - 1, p_3)\):

\begin{equation}\label{eqn:linear_algebraic_equation_for_p}
    Cx= p^*
\end{equation}

\(C\) corresponds to equations
(\ref{eqn:condition_for_tilde_p1}-\ref{eqn:condition_for_tilde_p3}) and is
given by:

\begin{equation}\label{eqn:definition_of_C}
    C =
    \begin{bmatrix}
        R - P & R- P \\
        S - P & T- P \\
        T - P & S- P \\
    \end{bmatrix}
\end{equation}

Note that in general, equation (\ref{eqn:linear_algebraic_equation_for_p}) will
not necessarily have a solution. From the Rouch\'{e}-Capelli theorem if there is
a solution it is unique as \(\text{rank}(C)=2\) which is the dimension of the
variable \(x\). The best fitting \(x\) is found by minimizing:

\begin{equation}\label{eqn:r_squared}
    \text{SSError} = \|C x- p^*\|_2^2 = \sum_{i=1}^{3}\left((C\bar x)_i-p_i^*\right)^2
\end{equation}

Note that \(\text{SSError}\), which is the square of the Frobenius
norm~\cite{Golub2013}, becomes a measure of how close a strategy is to being an
extortionate strategy. Suspicion
of extortion then corresponds to a threshold on \(\text{SSError}\).

By observing interactions (human or otherwise), their memory one representation
can be inferred and this approach can be used to recognise extortionate
behaviour. The notion of comparing theoretic and actual plays of the IPD is not
novel, see for example~\cite{Rand2013}. Immediately it is noted that if the
environment is noisy~\cite{Wu1995} then no strategy can be considered to be
extortionate as \(p_4>0\).

In the next section, this idea will be illustrated by observing the interactions
that take place in a computer based tournament of the IPD\@.

\section{Numerical experiments}\label{sec:numerical-experiments}

In~\cite{Stewart2012} results from a tournament with
\input{./assets/tex/number_of_stewart_plotkin_strategies/main.tex} strategies,
was presented with specific consideration given to ZD strategies. This
tournament is reproduced here using the Axelrod-Python
project~\cite{Knight2016}. To obtain a good measure of the corresponding
transition rates for each strategy all matches have been run for
\input{assets/tex/number_of_turns/main.tex} turns and every match has been
repeated \input{assets/tex/number_of_repetitions/main.tex} times. All of this
interaction data is available at~\cite{vincent_knight_2018_1297075}. A good
match between the inferred Markov chain and the state distribution of the actual
interactions has been verified. Data for this is presented in the supplementary
materials.

Figure~\ref{fig:SSError_overall_in_stewart_plotkin} shows the \(\text{SSError}\)
values for all the strategies in the tournament, as reported
in~\cite{Stewart2012} the extortionate strategy (which has an expected
\(\text{SSError}\) approximately 0) gains a large number of wins.

\begin{figure}[!htbp]
    \centering
    \includegraphics[width=.8\textwidth]{./assets/img/SSError_overall_in_stewart_plotkin/main.pdf}
    \caption{\(\text{SSError}\) and state probabilities for the strategies
        of~\cite{Stewart2012}, ordered both by number of wins and overall score.
        Note that \(P(DC)\) is not shown as it corresponds to the transpose of
        \(P(CD)\). Cooperator and Defector are omitted as they do not visit all
        the states.}
    \label{fig:SSError_overall_in_stewart_plotkin}
\end{figure}

Here, the work of~\cite{Stewart2012} is extended by investigating a tournament
with \input{assets/tex/number_of_full_strategies/main.tex}
strategies.

The results of this analysis are shown in
Figure~\ref{fig:SSError_and_probabilities_in_full}. The top ranking strategies
by number of wins seem to be extortionate (but not against all strategies) and
it can be seen that a small sub group of strategies achieve mutual defection.
All the top ranking strategies according to score achieve mutual cooperation and
do not extort each other, however they
\textbf{do} exhibit extortionate behaviour towards a number of the lower ranking
strategies.

\begin{figure}[!htbp]
    \centering
    \includegraphics[width=.8\textwidth]{./assets/img/SSError_and_probabilities_in_full/main.pdf}
    \caption{\(\text{SSError}\) for the strategies for the full tournament. Only
    strategy interactions for which \(p_4=0\) and \(\chi>1\) are displayed.}
    \label{fig:SSError_and_probabilities_in_full}
\end{figure}

\section{Conclusion}\label{sec:conclusion}

This work defines an approach to measure whether or not a player is playing a
strategy that corresponds to an extortionate strategy as defined
in~\cite{Press2012}: a mathematical model for suspicion. Indeed, all
extortionate strategies have been
 classified as lying on a triangular plane.
This rigorous classification fails to be robust to small measurement error, thus
a statistical approach is proposed.
This is done through a linear algebraic approach for approximating the solution
of a linear system. Using this, a large number of pairwise interactions is
simulated and in fact very few strategies are found to act extortionately.

The work of~\cite{Press2012}, whilst showing that a clever approach to taking
advantage of another memory one strategy exists: this is incomplete. Whilst the
elegance of this result is very attractive, just as the simplicity of the
victory of Tit For Tat in Axelrod's original tournaments was, it is incomplete.
Extortionate strategies achieve a high number of wins but they do not
achieve a high score which corresponds to the fitness landscape in an
evolutionary sense. From the large number of interactions a payoff matrix \(S\)
can be measured where \(S_{ij}\) denotes the score (using standard values of
\((R, S, T, P) = (3, 0, 5, 1)\)) of the \(i\)th strategy
against the \(j\)th strategy. Using this, the replicator equation
describes the evolution of the system based on a population density fitness
function:

\begin{equation}\label{eqn:replicator_dynamics}
    \frac{dx}{dt} = x(S-x^TS x)
\end{equation}

Equation (\ref{eqn:replicator_dynamics}) is solved numerically through an
integration technique described in~\cite{Petzold1983} and
Figure~\ref{fig:replicator_dynamics} shows the evolution of the distribution of
the system: the various strategies are ranked by scores. It is clear to see that
only the high ranking strategies survive the evolutionary process (in fact,
only \input{./assets/img/replicator_dynamics/main.tex}
have a final distribution greater than \(10 ^ {-2}\)). This confirms the
findings of~\cite{Moran1707} in which sophisticated strategies resist
evolutionary invasion of shorter memory strategies. Recalling
Figure~\ref{fig:SSError_and_probabilities_in_full} this demonstrates that:

\begin{itemize}
    \item Cooperation emerges through the evolutionary process: the high scoring
        strategies do not exhibit extortionate behaviour towards each other.
    \item Extortionate strategies do not survive the evolutionary process.
\end{itemize}

\begin{figure}[!htbp]
    \centering
    \includegraphics[width=.8\textwidth]{./assets/img/replicator_dynamics/main.pdf}
    \caption{Numerical simulation of the replicator equation
    (\ref{eqn:replicator_dynamics}): strategies are ordered by score, only the strategies with a high score survive the evolutionary process.}
    \label{fig:replicator_dynamics}
\end{figure}

This work can be used to classify plays of the IPD\@: data can be collected from
actual interactions (in lab or in the field). Furthermore, this allows for a
classification method similar to the notion of fingerprinting presented
in~\cite{Ashlock2008}. Trained strategies can potentially be classified as
extortionate or not or it could be possible to even constrain the reinforcement
learning approaches that are becoming prevalent in the literature.
Alternatively, this mathematical approach for recognising extortion could be
used in sophisticated strategies to defend against invasion. Arguably, some of
the strategies considered here exhibit this behaviour, indeed as described
in~\cite{Harper2017}, the top ranking strategies in the full tournament are
obtained using evolutionary reinforcement learning techniques, thus, suspicion
of extortionate behaviour could in fact be an evolutionary trait.

\section*{Acknowledgements}

The following open source software libraries were used in this research:

\begin{itemize}
    \item The Axelrod ~\cite{Knight2016, Knight2018} library (IPD strategies and
        tournaments).
    \item The sympy library~\cite{Meurer2017} (verification of all symbolic
        calculations).
    \item The matplotlib~\cite{Droettboom2018} library (visualisation).
    \item The pandas~\cite{Structures2010}, dask~\cite{Dask2016} and
        NumPy~\cite{Oliphant2015} libraries (data manipulation).
    \item The SciPy~\cite{Jones2001} library (numerical integration of the
        replicator equation).
\end{itemize}

This work was performed using the computational facilities of the Advanced
Research Computing @ Cardiff (ARCCA) Division, Cardiff University.

\printbibliography

\newpage
\section*{Supplementary materials}

\includepdf{assets/pdf/proof_of_form_of_extortionate_strategies/main.pdf}

\newpage

Using the pair wise interactions the transition rates \(p,
q\) can be measured and the steady state probabilities inferred and compared to
the actual probabilities of each state.
This is done numerically by computing the singular eigenvector of the
matrix \(A\) \cite{Stewart2009}:

\[
    A =
    \begin{bmatrix}
        p_1 q_1 & p_1 (1 - q_1) & (1 - p_1) q_1 & (1 -p_1) (1 - q_1) \\
        p_2 q_2 & p_2 (1 - q_2) & (1 - p_2) q_2 & (1 -p_2) (1 - q_2) \\
        p_3 q_3 & p_3 (1 - q_3) & (1 - p_3) q_3 & (1 -p_3) (1 - q_3) \\
        p_4 q_4 & p_4 (1 - q_4) & (1 - p_4) q_4 & (1 -p_4) (1 - q_4) \\
    \end{bmatrix}
\]

Figure~\ref{fig:computed_probabilities_vs_theoretic_probabilities} shows a
regression line fitted to every pairwise interaction with a reported
\(\text{SSError}\) value (pairwise interactions with missing states were
omitted). This serves to validate the approach: a part from some edge cases the
relationship is consistent.

\begin{figure}[!htbp]
    \centering
    \includegraphics[width=.8\textwidth]{./assets/img/computed_probabilities_vs_theoretic_probabilities/main.pdf}
    \caption{The
        relationship between the steady state probabilities inferred from the
        measured transitions and the actual steady state probabilities. A linear
        regression line is included validating the approach.}
    \label{fig:computed_probabilities_vs_theoretic_probabilities}
\end{figure}


\end{document}

strategies.

The results of this analysis are shown in
Figure~\ref{fig:SSError_and_probabilities_in_full}. The top ranking strategies
by number of wins seem to be extortionate (but not against all strategies) and
it can be seen that a small sub group of strategies achieve mutual defection.
All the top ranking strategies according to score achieve mutual cooperation and
do not extort each other, however they
\textbf{do} exhibit extortionate behaviour towards a number of the lower ranking
strategies.

\begin{figure}[!htbp]
    \centering
    \includegraphics[width=.8\textwidth]{./assets/img/SSError_and_probabilities_in_full/main.pdf}
    \caption{\(\text{SSError}\) for the strategies for the full tournament. Only
    strategy interactions for which \(p_4=0\) and \(\chi>1\) are displayed.}
    \label{fig:SSError_and_probabilities_in_full}
\end{figure}

\section{Conclusion}\label{sec:conclusion}

This work defines an approach to measure whether or not a player is playing a
strategy that corresponds to an extortionate strategy as defined
in~\cite{Press2012}: a mathematical model for suspicion. Indeed, all
extortionate strategies have been
 classified as lying on a triangular plane.
This rigorous classification fails to be robust to small measurement error, thus
a statistical approach is proposed.
This is done through a linear algebraic approach for approximating the solution
of a linear system. Using this, a large number of pairwise interactions is
simulated and in fact very few strategies are found to act extortionately.

The work of~\cite{Press2012}, whilst showing that a clever approach to taking
advantage of another memory one strategy exists: this is incomplete. Whilst the
elegance of this result is very attractive, just as the simplicity of the
victory of Tit For Tat in Axelrod's original tournaments was, it is incomplete.
Extortionate strategies achieve a high number of wins but they do not
achieve a high score which corresponds to the fitness landscape in an
evolutionary sense. From the large number of interactions a payoff matrix \(S\)
can be measured where \(S_{ij}\) denotes the score (using standard values of
\((R, S, T, P) = (3, 0, 5, 1)\)) of the \(i\)th strategy
against the \(j\)th strategy. Using this, the replicator equation
describes the evolution of the system based on a population density fitness
function:

\begin{equation}\label{eqn:replicator_dynamics}
    \frac{dx}{dt} = x(S-x^TS x)
\end{equation}

Equation (\ref{eqn:replicator_dynamics}) is solved numerically through an
integration technique described in~\cite{Petzold1983} and
Figure~\ref{fig:replicator_dynamics} shows the evolution of the distribution of
the system: the various strategies are ranked by scores. It is clear to see that
only the high ranking strategies survive the evolutionary process (in fact,
only \documentclass[a4paper]{article}

\usepackage{amsmath}
\usepackage{amssymb}
\usepackage[margin=1.5cm,
            includefoot,
            footskip=30pt]{geometry}
\usepackage{layout}
\usepackage{graphicx}
\usepackage{subcaption}

\usepackage{biblatex}
\usepackage{pdfpages}

\bibliography{main.bib}

\title{Suspicion: Recognising and evaluating the effectiveness
       of extortion in the Iterated Prisoner's Dilemma}
\author{Vincent A. Knight \and Nikoleta E. Glynatsi}
\date{\today}



\begin{document}

\maketitle

\begin{abstract}
    The Iterated Prisoner's Dilemma is a model for rational and evolutionary
    interactive behaviour. It has applications both in the study of human social
    behaviour as well as in biology.
    It is used to understand when and how a rational individual might
    accept an immediate cost to their own utility for the direct benefit of
    another.

    Much attention has been given to a class of strategies called
    Zero Determinant strategies. It has been theoretically shown that these
    strategies can ``extort'' any player.

    In this work, an approach to identify if observed strategies are playing in
    an extortionate way is described. Furthermore, experimental analysis of
    a large tournament with \input{assets/tex/number_of_full_strategies/main.tex}
    strategies is considered. In this setting
    the most highly performing strategies do not play in an extortionate way
    against each other but do against lower performing strategies.
    This suggests that whilst the theory of Zero Determinant strategies
    indicates that memory is not of fundamental importance to the evolution of
    cooperative behaviour, this is incomplete.
\end{abstract}

\section{Introduction}\label{sec:introduction}

Agent based game theoretic models have become a stalwart of the underpinning
mathematics of interactive behaviours. One of the major pieces of work
in this area is the pair of original computer tournaments run by Robert
Axelrod~\cite{Axelrod1980, Axelrod1980a}. These tournaments pitted submitted
computer strategies against each other in plays of the Iterated Prisoner's
Dilemma. A common game where agents can choose to pay a slight cost to their
immediate utility in the hope of building a reputation. This has been used in
economic and evolutionary game theory to understand the evolution of cooperative
behaviour.

Recently, a class of strategies was described in~\cite{Press2012} that can
provably extort any given opponent. In~\cite{Hilbe2013, Moran1707} some
questions have already been asked about the true effectiveness of these
strategies in an evolutionary setting. Here another question is asked: is it
possible to recognise this extortionate behaviour? A mathematical procedure for
suspicion is presented: in the same way that the continued actions of an
extortionate individual might raise suspicion.

This work makes use of the Axelrod Python library~\cite{Knight2018, Knight2016}
with a large number of Prisoner Dilemma strategies available to give an
extensive numerical example of the ideas presented.  The approach is presented
in Section~\ref{sec:delta-zd-strategies}.  All of the code and data discussed
in Section~\ref{sec:numerical-experiments} is open sourced, archived and
written according to best scientific principles~\cite{Wilson2014}. The data
archive can be found at~\cite{vincent_knight_2018_1297075}.

\section{Recognising Extortion}\label{sec:delta-zd-strategies}

In~\cite{Press2012}, given a match between 2 memory-one strategies, the concept
of Zero Determinant (ZD) strategies is introduced. The main result of that paper
shows that given two memory one players \(p, q\in\mathbb{R}^4\) a linear
relationship between the players' scores could be forced by one of the players.

Using the notation of~\cite{Press2012}, assuming the utilities for player \(p\)
are given by \(S_x=(R, S, T, P)\) and for player \(q\) by \(S_y=(R, T, S, P)\)
and that the stationary scores of each player is given by \(S_X\) and \(S_Y\)
respectively. The main result of~\cite{Press2012} is that if

\begin{equation}\label{eqn:linear_relationship_for_p}
    \tilde p=\alpha S_x + \beta S_y + \gamma
\end{equation}

or

\begin{equation}\label{eqn:linear_relationship_for_q}
    \tilde q=\alpha S_x + \beta S_y + \gamma
\end{equation}

where \(\tilde p = (1 - p_1, 1 - p_2, p_3, p_4)\) and
\(\tilde q = (1 - q_1, 1 - q_2, q_3, q_4)\) then:

\begin{equation}
    \alpha S_X + \beta S_Y + \gamma = 0
\end{equation}

In~\cite{Press2012} a particular type of ZD strategy is defined: extortionate
strategies. If:

\begin{equation}\label{eqn:constraint_for_extortion}
    \gamma = - P(\alpha + \beta)
\end{equation}

then the player can ensure they get a score \(\chi\) times
larger than the opponent. This extortion coefficient is given by:

\begin{equation}\label{eqn:definition_of_chi}
    \chi=\frac{-\beta}{\alpha}
\end{equation}

Thus, if (\ref{eqn:constraint_for_extortion}) holds and \(\chi >1\) a player is
said to extort their opponent.
Here, the reverse problem is considered: given a
\(p\in\mathbb{R}^4\) how does one identify \(\alpha, \beta\) if they
exist and is the strategy in fact acting in an extortionate way?

These conditions correspond to:

\begin{align}
    \tilde p_1 & = \alpha R + \beta R - P (\alpha + \beta)
            \label{eqn:condition_for_tilde_p1}\\
    \tilde p_2 & = \alpha S + \beta T - P (\alpha + \beta)
            \label{eqn:condition_for_tilde_p2}\\
    \tilde p_3 & = \alpha T + \beta S - P (\alpha + \beta)
            \label{eqn:condition_for_tilde_p3}\\
    \tilde p_4 & = \alpha P + \beta P - P (\alpha + \beta)
            \label{eqn:condition_for_tilde_p4}
\end{align}

Equation (\ref{eqn:condition_for_tilde_p4}) ensures that \(p_4=\tilde p_4=0\).
Equations (\ref{eqn:condition_for_tilde_p1}-\ref{eqn:condition_for_tilde_p3})
can be used to eliminate \(\alpha, \beta\), giving:

\begin{equation}\label{eqn:planar_definition_of_extortion}
    \tilde p_1 = \frac{(R - P)(\tilde p_2 + \tilde p_3)}{S + T - 2P}
\end{equation}

with:

\begin{equation}\label{eqn:definition_of_chi}
    \chi = \frac{\tilde p_2 (P - T) + \tilde p_3 (S - P)}
                {\tilde p_2 (P - S) + \tilde p_3 (T - P)}
\end{equation}

Given a strategy \(p\in\mathbb{R}^{4\times 1}\) equations
(\ref{eqn:condition_for_tilde_p4}), (\ref{eqn:planar_definition_of_extortion}-\ref{eqn:definition_of_chi}) can be used to check if
a strategy is extortionate. The conditions correspond to:

\begin{align}
    p_1 & = \frac{(R-P)(p_2 + p_3) - R + T + S - P}{S + T - 2P}
     \label{eqn:condition_for_p1}\\
    p_4 & = 0 \label{eqn:condition_for_p4}\\
    1 & > p_2 + p_3\label{eqn:condition_for_chi}
\end{align}

The algebraic steps necessary to prove these results are available in the
supporting materials.

All extortionate strategies reside on a triangular (\ref{eqn:condition_for_chi})
plane (\ref{eqn:condition_for_p1}) in 3 dimensions (\ref{eqn:condition_for_p4}).
Using this formulation it can be seen that a necessary (but not sufficient)
condition for an extortionate strategy is that it cooperates on average less
than 50\% of the time when in a state of disagreement with the opponent.

As an example, consider the known extortionate strategy \(p=(8 / 9, 1 / 2, 1 /
3, 0)\) from~\cite{Stewart2012} which is referred to as \texttt{Extort-2}. In
this case, for the standard values of \((R, T, S, P)\) constraint
(\ref{eqn:condition_for_p1}) corresponds to:

\begin{equation}
    p_1 = \frac{2(p_2 + p_3) + 1}{3}
\end{equation}

It is clear that in this case all constraints hold.

This approach could in fact be used to confirm that a given strategy is acting
in an extortionate manner even if it is not a memory one strategy. However, in
practice, if a closed form for \(p\) is not known, then due to measurement
and/or numerical error this would not work.

This problem can be written in the following linear algebraic form where
\(x=(\alpha, \beta)\)
and \(p^*=(\tilde p_1 - 1, tilde_2 - 1, p_3)\):

\begin{equation}\label{eqn:linear_algebraic_equation_for_p}
    Cx= p^*
\end{equation}

\(C\) corresponds to equations
(\ref{eqn:condition_for_tilde_p1}-\ref{eqn:condition_for_tilde_p3}) and is
given by:

\begin{equation}\label{eqn:definition_of_C}
    C =
    \begin{bmatrix}
        R - P & R- P \\
        S - P & T- P \\
        T - P & S- P \\
    \end{bmatrix}
\end{equation}

Note that in general, equation (\ref{eqn:linear_algebraic_equation_for_p}) will
not necessarily have a solution. From the Rouch\'{e}-Capelli theorem if there is
a solution it is unique as \(\text{rank}(C)=2\) which is the dimension of the
variable \(x\). The best fitting \(x\) is found by minimizing:

\begin{equation}\label{eqn:r_squared}
    \text{SSError} = \|C x- p^*\|_2^2 = \sum_{i=1}^{3}\left((C\bar x)_i-p_i^*\right)^2
\end{equation}

Note that \(\text{SSError}\), which is the square of the Frobenius
norm~\cite{Golub2013}, becomes a measure of how close a strategy is to being an
extortionate strategy. Suspicion
of extortion then corresponds to a threshold on \(\text{SSError}\).

By observing interactions (human or otherwise), their memory one representation
can be inferred and this approach can be used to recognise extortionate
behaviour. The notion of comparing theoretic and actual plays of the IPD is not
novel, see for example~\cite{Rand2013}. Immediately it is noted that if the
environment is noisy~\cite{Wu1995} then no strategy can be considered to be
extortionate as \(p_4>0\).

In the next section, this idea will be illustrated by observing the interactions
that take place in a computer based tournament of the IPD\@.

\section{Numerical experiments}\label{sec:numerical-experiments}

In~\cite{Stewart2012} results from a tournament with
\input{./assets/tex/number_of_stewart_plotkin_strategies/main.tex} strategies,
was presented with specific consideration given to ZD strategies. This
tournament is reproduced here using the Axelrod-Python
project~\cite{Knight2016}. To obtain a good measure of the corresponding
transition rates for each strategy all matches have been run for
\input{assets/tex/number_of_turns/main.tex} turns and every match has been
repeated \input{assets/tex/number_of_repetitions/main.tex} times. All of this
interaction data is available at~\cite{vincent_knight_2018_1297075}. A good
match between the inferred Markov chain and the state distribution of the actual
interactions has been verified. Data for this is presented in the supplementary
materials.

Figure~\ref{fig:SSError_overall_in_stewart_plotkin} shows the \(\text{SSError}\)
values for all the strategies in the tournament, as reported
in~\cite{Stewart2012} the extortionate strategy (which has an expected
\(\text{SSError}\) approximately 0) gains a large number of wins.

\begin{figure}[!htbp]
    \centering
    \includegraphics[width=.8\textwidth]{./assets/img/SSError_overall_in_stewart_plotkin/main.pdf}
    \caption{\(\text{SSError}\) and state probabilities for the strategies
        of~\cite{Stewart2012}, ordered both by number of wins and overall score.
        Note that \(P(DC)\) is not shown as it corresponds to the transpose of
        \(P(CD)\). Cooperator and Defector are omitted as they do not visit all
        the states.}
    \label{fig:SSError_overall_in_stewart_plotkin}
\end{figure}

Here, the work of~\cite{Stewart2012} is extended by investigating a tournament
with \input{assets/tex/number_of_full_strategies/main.tex}
strategies.

The results of this analysis are shown in
Figure~\ref{fig:SSError_and_probabilities_in_full}. The top ranking strategies
by number of wins seem to be extortionate (but not against all strategies) and
it can be seen that a small sub group of strategies achieve mutual defection.
All the top ranking strategies according to score achieve mutual cooperation and
do not extort each other, however they
\textbf{do} exhibit extortionate behaviour towards a number of the lower ranking
strategies.

\begin{figure}[!htbp]
    \centering
    \includegraphics[width=.8\textwidth]{./assets/img/SSError_and_probabilities_in_full/main.pdf}
    \caption{\(\text{SSError}\) for the strategies for the full tournament. Only
    strategy interactions for which \(p_4=0\) and \(\chi>1\) are displayed.}
    \label{fig:SSError_and_probabilities_in_full}
\end{figure}

\section{Conclusion}\label{sec:conclusion}

This work defines an approach to measure whether or not a player is playing a
strategy that corresponds to an extortionate strategy as defined
in~\cite{Press2012}: a mathematical model for suspicion. Indeed, all
extortionate strategies have been
 classified as lying on a triangular plane.
This rigorous classification fails to be robust to small measurement error, thus
a statistical approach is proposed.
This is done through a linear algebraic approach for approximating the solution
of a linear system. Using this, a large number of pairwise interactions is
simulated and in fact very few strategies are found to act extortionately.

The work of~\cite{Press2012}, whilst showing that a clever approach to taking
advantage of another memory one strategy exists: this is incomplete. Whilst the
elegance of this result is very attractive, just as the simplicity of the
victory of Tit For Tat in Axelrod's original tournaments was, it is incomplete.
Extortionate strategies achieve a high number of wins but they do not
achieve a high score which corresponds to the fitness landscape in an
evolutionary sense. From the large number of interactions a payoff matrix \(S\)
can be measured where \(S_{ij}\) denotes the score (using standard values of
\((R, S, T, P) = (3, 0, 5, 1)\)) of the \(i\)th strategy
against the \(j\)th strategy. Using this, the replicator equation
describes the evolution of the system based on a population density fitness
function:

\begin{equation}\label{eqn:replicator_dynamics}
    \frac{dx}{dt} = x(S-x^TS x)
\end{equation}

Equation (\ref{eqn:replicator_dynamics}) is solved numerically through an
integration technique described in~\cite{Petzold1983} and
Figure~\ref{fig:replicator_dynamics} shows the evolution of the distribution of
the system: the various strategies are ranked by scores. It is clear to see that
only the high ranking strategies survive the evolutionary process (in fact,
only \input{./assets/img/replicator_dynamics/main.tex}
have a final distribution greater than \(10 ^ {-2}\)). This confirms the
findings of~\cite{Moran1707} in which sophisticated strategies resist
evolutionary invasion of shorter memory strategies. Recalling
Figure~\ref{fig:SSError_and_probabilities_in_full} this demonstrates that:

\begin{itemize}
    \item Cooperation emerges through the evolutionary process: the high scoring
        strategies do not exhibit extortionate behaviour towards each other.
    \item Extortionate strategies do not survive the evolutionary process.
\end{itemize}

\begin{figure}[!htbp]
    \centering
    \includegraphics[width=.8\textwidth]{./assets/img/replicator_dynamics/main.pdf}
    \caption{Numerical simulation of the replicator equation
    (\ref{eqn:replicator_dynamics}): strategies are ordered by score, only the strategies with a high score survive the evolutionary process.}
    \label{fig:replicator_dynamics}
\end{figure}

This work can be used to classify plays of the IPD\@: data can be collected from
actual interactions (in lab or in the field). Furthermore, this allows for a
classification method similar to the notion of fingerprinting presented
in~\cite{Ashlock2008}. Trained strategies can potentially be classified as
extortionate or not or it could be possible to even constrain the reinforcement
learning approaches that are becoming prevalent in the literature.
Alternatively, this mathematical approach for recognising extortion could be
used in sophisticated strategies to defend against invasion. Arguably, some of
the strategies considered here exhibit this behaviour, indeed as described
in~\cite{Harper2017}, the top ranking strategies in the full tournament are
obtained using evolutionary reinforcement learning techniques, thus, suspicion
of extortionate behaviour could in fact be an evolutionary trait.

\section*{Acknowledgements}

The following open source software libraries were used in this research:

\begin{itemize}
    \item The Axelrod ~\cite{Knight2016, Knight2018} library (IPD strategies and
        tournaments).
    \item The sympy library~\cite{Meurer2017} (verification of all symbolic
        calculations).
    \item The matplotlib~\cite{Droettboom2018} library (visualisation).
    \item The pandas~\cite{Structures2010}, dask~\cite{Dask2016} and
        NumPy~\cite{Oliphant2015} libraries (data manipulation).
    \item The SciPy~\cite{Jones2001} library (numerical integration of the
        replicator equation).
\end{itemize}

This work was performed using the computational facilities of the Advanced
Research Computing @ Cardiff (ARCCA) Division, Cardiff University.

\printbibliography

\newpage
\section*{Supplementary materials}

\includepdf{assets/pdf/proof_of_form_of_extortionate_strategies/main.pdf}

\newpage

Using the pair wise interactions the transition rates \(p,
q\) can be measured and the steady state probabilities inferred and compared to
the actual probabilities of each state.
This is done numerically by computing the singular eigenvector of the
matrix \(A\) \cite{Stewart2009}:

\[
    A =
    \begin{bmatrix}
        p_1 q_1 & p_1 (1 - q_1) & (1 - p_1) q_1 & (1 -p_1) (1 - q_1) \\
        p_2 q_2 & p_2 (1 - q_2) & (1 - p_2) q_2 & (1 -p_2) (1 - q_2) \\
        p_3 q_3 & p_3 (1 - q_3) & (1 - p_3) q_3 & (1 -p_3) (1 - q_3) \\
        p_4 q_4 & p_4 (1 - q_4) & (1 - p_4) q_4 & (1 -p_4) (1 - q_4) \\
    \end{bmatrix}
\]

Figure~\ref{fig:computed_probabilities_vs_theoretic_probabilities} shows a
regression line fitted to every pairwise interaction with a reported
\(\text{SSError}\) value (pairwise interactions with missing states were
omitted). This serves to validate the approach: a part from some edge cases the
relationship is consistent.

\begin{figure}[!htbp]
    \centering
    \includegraphics[width=.8\textwidth]{./assets/img/computed_probabilities_vs_theoretic_probabilities/main.pdf}
    \caption{The
        relationship between the steady state probabilities inferred from the
        measured transitions and the actual steady state probabilities. A linear
        regression line is included validating the approach.}
    \label{fig:computed_probabilities_vs_theoretic_probabilities}
\end{figure}


\end{document}

have a final distribution greater than \(10 ^ {-2}\)). This confirms the
findings of~\cite{Moran1707} in which sophisticated strategies resist
evolutionary invasion of shorter memory strategies. Recalling
Figure~\ref{fig:SSError_and_probabilities_in_full} this demonstrates that:

\begin{itemize}
    \item Cooperation emerges through the evolutionary process: the high scoring
        strategies do not exhibit extortionate behaviour towards each other.
    \item Extortionate strategies do not survive the evolutionary process.
\end{itemize}

\begin{figure}[!htbp]
    \centering
    \includegraphics[width=.8\textwidth]{./assets/img/replicator_dynamics/main.pdf}
    \caption{Numerical simulation of the replicator equation
    (\ref{eqn:replicator_dynamics}): strategies are ordered by score, only the strategies with a high score survive the evolutionary process.}
    \label{fig:replicator_dynamics}
\end{figure}

This work can be used to classify plays of the IPD\@: data can be collected from
actual interactions (in lab or in the field). Furthermore, this allows for a
classification method similar to the notion of fingerprinting presented
in~\cite{Ashlock2008}. Trained strategies can potentially be classified as
extortionate or not or it could be possible to even constrain the reinforcement
learning approaches that are becoming prevalent in the literature.
Alternatively, this mathematical approach for recognising extortion could be
used in sophisticated strategies to defend against invasion. Arguably, some of
the strategies considered here exhibit this behaviour, indeed as described
in~\cite{Harper2017}, the top ranking strategies in the full tournament are
obtained using evolutionary reinforcement learning techniques, thus, suspicion
of extortionate behaviour could in fact be an evolutionary trait.

\section*{Acknowledgements}

The following open source software libraries were used in this research:

\begin{itemize}
    \item The Axelrod ~\cite{Knight2016, Knight2018} library (IPD strategies and
        tournaments).
    \item The sympy library~\cite{Meurer2017} (verification of all symbolic
        calculations).
    \item The matplotlib~\cite{Droettboom2018} library (visualisation).
    \item The pandas~\cite{Structures2010}, dask~\cite{Dask2016} and
        NumPy~\cite{Oliphant2015} libraries (data manipulation).
    \item The SciPy~\cite{Jones2001} library (numerical integration of the
        replicator equation).
\end{itemize}

This work was performed using the computational facilities of the Advanced
Research Computing @ Cardiff (ARCCA) Division, Cardiff University.

\printbibliography

\newpage
\section*{Supplementary materials}

\includepdf{assets/pdf/proof_of_form_of_extortionate_strategies/main.pdf}

\newpage

Using the pair wise interactions the transition rates \(p,
q\) can be measured and the steady state probabilities inferred and compared to
the actual probabilities of each state.
This is done numerically by computing the singular eigenvector of the
matrix \(A\) \cite{Stewart2009}:

\[
    A =
    \begin{bmatrix}
        p_1 q_1 & p_1 (1 - q_1) & (1 - p_1) q_1 & (1 -p_1) (1 - q_1) \\
        p_2 q_2 & p_2 (1 - q_2) & (1 - p_2) q_2 & (1 -p_2) (1 - q_2) \\
        p_3 q_3 & p_3 (1 - q_3) & (1 - p_3) q_3 & (1 -p_3) (1 - q_3) \\
        p_4 q_4 & p_4 (1 - q_4) & (1 - p_4) q_4 & (1 -p_4) (1 - q_4) \\
    \end{bmatrix}
\]

Figure~\ref{fig:computed_probabilities_vs_theoretic_probabilities} shows a
regression line fitted to every pairwise interaction with a reported
\(\text{SSError}\) value (pairwise interactions with missing states were
omitted). This serves to validate the approach: a part from some edge cases the
relationship is consistent.

\begin{figure}[!htbp]
    \centering
    \includegraphics[width=.8\textwidth]{./assets/img/computed_probabilities_vs_theoretic_probabilities/main.pdf}
    \caption{The
        relationship between the steady state probabilities inferred from the
        measured transitions and the actual steady state probabilities. A linear
        regression line is included validating the approach.}
    \label{fig:computed_probabilities_vs_theoretic_probabilities}
\end{figure}


\end{document}
 times. All of this
interaction data is available at~\cite{vincent_knight_2018_1297075}. A good
match between the inferred Markov chain and the state distribution of the actual
interactions has been verified. Data for this is presented in the supplementary
materials.

Figure~\ref{fig:SSError_overall_in_stewart_plotkin} shows the \(\text{SSError}\)
values for all the strategies in the tournament, as reported
in~\cite{Stewart2012} the extortionate strategy (which has an expected
\(\text{SSError}\) approximately 0) gains a large number of wins.

\begin{figure}[!htbp]
    \centering
    \includegraphics[width=.8\textwidth]{./assets/img/SSError_overall_in_stewart_plotkin/main.pdf}
    \caption{\(\text{SSError}\) and state probabilities for the strategies
        of~\cite{Stewart2012}, ordered both by number of wins and overall score.
        Note that \(P(DC)\) is not shown as it corresponds to the transpose of
        \(P(CD)\). Cooperator and Defector are omitted as they do not visit all
        the states.}
    \label{fig:SSError_overall_in_stewart_plotkin}
\end{figure}

Here, the work of~\cite{Stewart2012} is extended by investigating a tournament
with \documentclass[a4paper]{article}

\usepackage{amsmath}
\usepackage{amssymb}
\usepackage[margin=1.5cm,
            includefoot,
            footskip=30pt]{geometry}
\usepackage{layout}
\usepackage{graphicx}
\usepackage{subcaption}

\usepackage{biblatex}
\usepackage{pdfpages}

\bibliography{main.bib}

\title{Suspicion: Recognising and evaluating the effectiveness
       of extortion in the Iterated Prisoner's Dilemma}
\author{Vincent A. Knight \and Nikoleta E. Glynatsi}
\date{\today}



\begin{document}

\maketitle

\begin{abstract}
    The Iterated Prisoner's Dilemma is a model for rational and evolutionary
    interactive behaviour. It has applications both in the study of human social
    behaviour as well as in biology.
    It is used to understand when and how a rational individual might
    accept an immediate cost to their own utility for the direct benefit of
    another.

    Much attention has been given to a class of strategies called
    Zero Determinant strategies. It has been theoretically shown that these
    strategies can ``extort'' any player.

    In this work, an approach to identify if observed strategies are playing in
    an extortionate way is described. Furthermore, experimental analysis of
    a large tournament with \documentclass[a4paper]{article}

\usepackage{amsmath}
\usepackage{amssymb}
\usepackage[margin=1.5cm,
            includefoot,
            footskip=30pt]{geometry}
\usepackage{layout}
\usepackage{graphicx}
\usepackage{subcaption}

\usepackage{biblatex}
\usepackage{pdfpages}

\bibliography{main.bib}

\title{Suspicion: Recognising and evaluating the effectiveness
       of extortion in the Iterated Prisoner's Dilemma}
\author{Vincent A. Knight \and Nikoleta E. Glynatsi}
\date{\today}



\begin{document}

\maketitle

\begin{abstract}
    The Iterated Prisoner's Dilemma is a model for rational and evolutionary
    interactive behaviour. It has applications both in the study of human social
    behaviour as well as in biology.
    It is used to understand when and how a rational individual might
    accept an immediate cost to their own utility for the direct benefit of
    another.

    Much attention has been given to a class of strategies called
    Zero Determinant strategies. It has been theoretically shown that these
    strategies can ``extort'' any player.

    In this work, an approach to identify if observed strategies are playing in
    an extortionate way is described. Furthermore, experimental analysis of
    a large tournament with \input{assets/tex/number_of_full_strategies/main.tex}
    strategies is considered. In this setting
    the most highly performing strategies do not play in an extortionate way
    against each other but do against lower performing strategies.
    This suggests that whilst the theory of Zero Determinant strategies
    indicates that memory is not of fundamental importance to the evolution of
    cooperative behaviour, this is incomplete.
\end{abstract}

\section{Introduction}\label{sec:introduction}

Agent based game theoretic models have become a stalwart of the underpinning
mathematics of interactive behaviours. One of the major pieces of work
in this area is the pair of original computer tournaments run by Robert
Axelrod~\cite{Axelrod1980, Axelrod1980a}. These tournaments pitted submitted
computer strategies against each other in plays of the Iterated Prisoner's
Dilemma. A common game where agents can choose to pay a slight cost to their
immediate utility in the hope of building a reputation. This has been used in
economic and evolutionary game theory to understand the evolution of cooperative
behaviour.

Recently, a class of strategies was described in~\cite{Press2012} that can
provably extort any given opponent. In~\cite{Hilbe2013, Moran1707} some
questions have already been asked about the true effectiveness of these
strategies in an evolutionary setting. Here another question is asked: is it
possible to recognise this extortionate behaviour? A mathematical procedure for
suspicion is presented: in the same way that the continued actions of an
extortionate individual might raise suspicion.

This work makes use of the Axelrod Python library~\cite{Knight2018, Knight2016}
with a large number of Prisoner Dilemma strategies available to give an
extensive numerical example of the ideas presented.  The approach is presented
in Section~\ref{sec:delta-zd-strategies}.  All of the code and data discussed
in Section~\ref{sec:numerical-experiments} is open sourced, archived and
written according to best scientific principles~\cite{Wilson2014}. The data
archive can be found at~\cite{vincent_knight_2018_1297075}.

\section{Recognising Extortion}\label{sec:delta-zd-strategies}

In~\cite{Press2012}, given a match between 2 memory-one strategies, the concept
of Zero Determinant (ZD) strategies is introduced. The main result of that paper
shows that given two memory one players \(p, q\in\mathbb{R}^4\) a linear
relationship between the players' scores could be forced by one of the players.

Using the notation of~\cite{Press2012}, assuming the utilities for player \(p\)
are given by \(S_x=(R, S, T, P)\) and for player \(q\) by \(S_y=(R, T, S, P)\)
and that the stationary scores of each player is given by \(S_X\) and \(S_Y\)
respectively. The main result of~\cite{Press2012} is that if

\begin{equation}\label{eqn:linear_relationship_for_p}
    \tilde p=\alpha S_x + \beta S_y + \gamma
\end{equation}

or

\begin{equation}\label{eqn:linear_relationship_for_q}
    \tilde q=\alpha S_x + \beta S_y + \gamma
\end{equation}

where \(\tilde p = (1 - p_1, 1 - p_2, p_3, p_4)\) and
\(\tilde q = (1 - q_1, 1 - q_2, q_3, q_4)\) then:

\begin{equation}
    \alpha S_X + \beta S_Y + \gamma = 0
\end{equation}

In~\cite{Press2012} a particular type of ZD strategy is defined: extortionate
strategies. If:

\begin{equation}\label{eqn:constraint_for_extortion}
    \gamma = - P(\alpha + \beta)
\end{equation}

then the player can ensure they get a score \(\chi\) times
larger than the opponent. This extortion coefficient is given by:

\begin{equation}\label{eqn:definition_of_chi}
    \chi=\frac{-\beta}{\alpha}
\end{equation}

Thus, if (\ref{eqn:constraint_for_extortion}) holds and \(\chi >1\) a player is
said to extort their opponent.
Here, the reverse problem is considered: given a
\(p\in\mathbb{R}^4\) how does one identify \(\alpha, \beta\) if they
exist and is the strategy in fact acting in an extortionate way?

These conditions correspond to:

\begin{align}
    \tilde p_1 & = \alpha R + \beta R - P (\alpha + \beta)
            \label{eqn:condition_for_tilde_p1}\\
    \tilde p_2 & = \alpha S + \beta T - P (\alpha + \beta)
            \label{eqn:condition_for_tilde_p2}\\
    \tilde p_3 & = \alpha T + \beta S - P (\alpha + \beta)
            \label{eqn:condition_for_tilde_p3}\\
    \tilde p_4 & = \alpha P + \beta P - P (\alpha + \beta)
            \label{eqn:condition_for_tilde_p4}
\end{align}

Equation (\ref{eqn:condition_for_tilde_p4}) ensures that \(p_4=\tilde p_4=0\).
Equations (\ref{eqn:condition_for_tilde_p1}-\ref{eqn:condition_for_tilde_p3})
can be used to eliminate \(\alpha, \beta\), giving:

\begin{equation}\label{eqn:planar_definition_of_extortion}
    \tilde p_1 = \frac{(R - P)(\tilde p_2 + \tilde p_3)}{S + T - 2P}
\end{equation}

with:

\begin{equation}\label{eqn:definition_of_chi}
    \chi = \frac{\tilde p_2 (P - T) + \tilde p_3 (S - P)}
                {\tilde p_2 (P - S) + \tilde p_3 (T - P)}
\end{equation}

Given a strategy \(p\in\mathbb{R}^{4\times 1}\) equations
(\ref{eqn:condition_for_tilde_p4}), (\ref{eqn:planar_definition_of_extortion}-\ref{eqn:definition_of_chi}) can be used to check if
a strategy is extortionate. The conditions correspond to:

\begin{align}
    p_1 & = \frac{(R-P)(p_2 + p_3) - R + T + S - P}{S + T - 2P}
     \label{eqn:condition_for_p1}\\
    p_4 & = 0 \label{eqn:condition_for_p4}\\
    1 & > p_2 + p_3\label{eqn:condition_for_chi}
\end{align}

The algebraic steps necessary to prove these results are available in the
supporting materials.

All extortionate strategies reside on a triangular (\ref{eqn:condition_for_chi})
plane (\ref{eqn:condition_for_p1}) in 3 dimensions (\ref{eqn:condition_for_p4}).
Using this formulation it can be seen that a necessary (but not sufficient)
condition for an extortionate strategy is that it cooperates on average less
than 50\% of the time when in a state of disagreement with the opponent.

As an example, consider the known extortionate strategy \(p=(8 / 9, 1 / 2, 1 /
3, 0)\) from~\cite{Stewart2012} which is referred to as \texttt{Extort-2}. In
this case, for the standard values of \((R, T, S, P)\) constraint
(\ref{eqn:condition_for_p1}) corresponds to:

\begin{equation}
    p_1 = \frac{2(p_2 + p_3) + 1}{3}
\end{equation}

It is clear that in this case all constraints hold.

This approach could in fact be used to confirm that a given strategy is acting
in an extortionate manner even if it is not a memory one strategy. However, in
practice, if a closed form for \(p\) is not known, then due to measurement
and/or numerical error this would not work.

This problem can be written in the following linear algebraic form where
\(x=(\alpha, \beta)\)
and \(p^*=(\tilde p_1 - 1, tilde_2 - 1, p_3)\):

\begin{equation}\label{eqn:linear_algebraic_equation_for_p}
    Cx= p^*
\end{equation}

\(C\) corresponds to equations
(\ref{eqn:condition_for_tilde_p1}-\ref{eqn:condition_for_tilde_p3}) and is
given by:

\begin{equation}\label{eqn:definition_of_C}
    C =
    \begin{bmatrix}
        R - P & R- P \\
        S - P & T- P \\
        T - P & S- P \\
    \end{bmatrix}
\end{equation}

Note that in general, equation (\ref{eqn:linear_algebraic_equation_for_p}) will
not necessarily have a solution. From the Rouch\'{e}-Capelli theorem if there is
a solution it is unique as \(\text{rank}(C)=2\) which is the dimension of the
variable \(x\). The best fitting \(x\) is found by minimizing:

\begin{equation}\label{eqn:r_squared}
    \text{SSError} = \|C x- p^*\|_2^2 = \sum_{i=1}^{3}\left((C\bar x)_i-p_i^*\right)^2
\end{equation}

Note that \(\text{SSError}\), which is the square of the Frobenius
norm~\cite{Golub2013}, becomes a measure of how close a strategy is to being an
extortionate strategy. Suspicion
of extortion then corresponds to a threshold on \(\text{SSError}\).

By observing interactions (human or otherwise), their memory one representation
can be inferred and this approach can be used to recognise extortionate
behaviour. The notion of comparing theoretic and actual plays of the IPD is not
novel, see for example~\cite{Rand2013}. Immediately it is noted that if the
environment is noisy~\cite{Wu1995} then no strategy can be considered to be
extortionate as \(p_4>0\).

In the next section, this idea will be illustrated by observing the interactions
that take place in a computer based tournament of the IPD\@.

\section{Numerical experiments}\label{sec:numerical-experiments}

In~\cite{Stewart2012} results from a tournament with
\input{./assets/tex/number_of_stewart_plotkin_strategies/main.tex} strategies,
was presented with specific consideration given to ZD strategies. This
tournament is reproduced here using the Axelrod-Python
project~\cite{Knight2016}. To obtain a good measure of the corresponding
transition rates for each strategy all matches have been run for
\input{assets/tex/number_of_turns/main.tex} turns and every match has been
repeated \input{assets/tex/number_of_repetitions/main.tex} times. All of this
interaction data is available at~\cite{vincent_knight_2018_1297075}. A good
match between the inferred Markov chain and the state distribution of the actual
interactions has been verified. Data for this is presented in the supplementary
materials.

Figure~\ref{fig:SSError_overall_in_stewart_plotkin} shows the \(\text{SSError}\)
values for all the strategies in the tournament, as reported
in~\cite{Stewart2012} the extortionate strategy (which has an expected
\(\text{SSError}\) approximately 0) gains a large number of wins.

\begin{figure}[!htbp]
    \centering
    \includegraphics[width=.8\textwidth]{./assets/img/SSError_overall_in_stewart_plotkin/main.pdf}
    \caption{\(\text{SSError}\) and state probabilities for the strategies
        of~\cite{Stewart2012}, ordered both by number of wins and overall score.
        Note that \(P(DC)\) is not shown as it corresponds to the transpose of
        \(P(CD)\). Cooperator and Defector are omitted as they do not visit all
        the states.}
    \label{fig:SSError_overall_in_stewart_plotkin}
\end{figure}

Here, the work of~\cite{Stewart2012} is extended by investigating a tournament
with \input{assets/tex/number_of_full_strategies/main.tex}
strategies.

The results of this analysis are shown in
Figure~\ref{fig:SSError_and_probabilities_in_full}. The top ranking strategies
by number of wins seem to be extortionate (but not against all strategies) and
it can be seen that a small sub group of strategies achieve mutual defection.
All the top ranking strategies according to score achieve mutual cooperation and
do not extort each other, however they
\textbf{do} exhibit extortionate behaviour towards a number of the lower ranking
strategies.

\begin{figure}[!htbp]
    \centering
    \includegraphics[width=.8\textwidth]{./assets/img/SSError_and_probabilities_in_full/main.pdf}
    \caption{\(\text{SSError}\) for the strategies for the full tournament. Only
    strategy interactions for which \(p_4=0\) and \(\chi>1\) are displayed.}
    \label{fig:SSError_and_probabilities_in_full}
\end{figure}

\section{Conclusion}\label{sec:conclusion}

This work defines an approach to measure whether or not a player is playing a
strategy that corresponds to an extortionate strategy as defined
in~\cite{Press2012}: a mathematical model for suspicion. Indeed, all
extortionate strategies have been
 classified as lying on a triangular plane.
This rigorous classification fails to be robust to small measurement error, thus
a statistical approach is proposed.
This is done through a linear algebraic approach for approximating the solution
of a linear system. Using this, a large number of pairwise interactions is
simulated and in fact very few strategies are found to act extortionately.

The work of~\cite{Press2012}, whilst showing that a clever approach to taking
advantage of another memory one strategy exists: this is incomplete. Whilst the
elegance of this result is very attractive, just as the simplicity of the
victory of Tit For Tat in Axelrod's original tournaments was, it is incomplete.
Extortionate strategies achieve a high number of wins but they do not
achieve a high score which corresponds to the fitness landscape in an
evolutionary sense. From the large number of interactions a payoff matrix \(S\)
can be measured where \(S_{ij}\) denotes the score (using standard values of
\((R, S, T, P) = (3, 0, 5, 1)\)) of the \(i\)th strategy
against the \(j\)th strategy. Using this, the replicator equation
describes the evolution of the system based on a population density fitness
function:

\begin{equation}\label{eqn:replicator_dynamics}
    \frac{dx}{dt} = x(S-x^TS x)
\end{equation}

Equation (\ref{eqn:replicator_dynamics}) is solved numerically through an
integration technique described in~\cite{Petzold1983} and
Figure~\ref{fig:replicator_dynamics} shows the evolution of the distribution of
the system: the various strategies are ranked by scores. It is clear to see that
only the high ranking strategies survive the evolutionary process (in fact,
only \input{./assets/img/replicator_dynamics/main.tex}
have a final distribution greater than \(10 ^ {-2}\)). This confirms the
findings of~\cite{Moran1707} in which sophisticated strategies resist
evolutionary invasion of shorter memory strategies. Recalling
Figure~\ref{fig:SSError_and_probabilities_in_full} this demonstrates that:

\begin{itemize}
    \item Cooperation emerges through the evolutionary process: the high scoring
        strategies do not exhibit extortionate behaviour towards each other.
    \item Extortionate strategies do not survive the evolutionary process.
\end{itemize}

\begin{figure}[!htbp]
    \centering
    \includegraphics[width=.8\textwidth]{./assets/img/replicator_dynamics/main.pdf}
    \caption{Numerical simulation of the replicator equation
    (\ref{eqn:replicator_dynamics}): strategies are ordered by score, only the strategies with a high score survive the evolutionary process.}
    \label{fig:replicator_dynamics}
\end{figure}

This work can be used to classify plays of the IPD\@: data can be collected from
actual interactions (in lab or in the field). Furthermore, this allows for a
classification method similar to the notion of fingerprinting presented
in~\cite{Ashlock2008}. Trained strategies can potentially be classified as
extortionate or not or it could be possible to even constrain the reinforcement
learning approaches that are becoming prevalent in the literature.
Alternatively, this mathematical approach for recognising extortion could be
used in sophisticated strategies to defend against invasion. Arguably, some of
the strategies considered here exhibit this behaviour, indeed as described
in~\cite{Harper2017}, the top ranking strategies in the full tournament are
obtained using evolutionary reinforcement learning techniques, thus, suspicion
of extortionate behaviour could in fact be an evolutionary trait.

\section*{Acknowledgements}

The following open source software libraries were used in this research:

\begin{itemize}
    \item The Axelrod ~\cite{Knight2016, Knight2018} library (IPD strategies and
        tournaments).
    \item The sympy library~\cite{Meurer2017} (verification of all symbolic
        calculations).
    \item The matplotlib~\cite{Droettboom2018} library (visualisation).
    \item The pandas~\cite{Structures2010}, dask~\cite{Dask2016} and
        NumPy~\cite{Oliphant2015} libraries (data manipulation).
    \item The SciPy~\cite{Jones2001} library (numerical integration of the
        replicator equation).
\end{itemize}

This work was performed using the computational facilities of the Advanced
Research Computing @ Cardiff (ARCCA) Division, Cardiff University.

\printbibliography

\newpage
\section*{Supplementary materials}

\includepdf{assets/pdf/proof_of_form_of_extortionate_strategies/main.pdf}

\newpage

Using the pair wise interactions the transition rates \(p,
q\) can be measured and the steady state probabilities inferred and compared to
the actual probabilities of each state.
This is done numerically by computing the singular eigenvector of the
matrix \(A\) \cite{Stewart2009}:

\[
    A =
    \begin{bmatrix}
        p_1 q_1 & p_1 (1 - q_1) & (1 - p_1) q_1 & (1 -p_1) (1 - q_1) \\
        p_2 q_2 & p_2 (1 - q_2) & (1 - p_2) q_2 & (1 -p_2) (1 - q_2) \\
        p_3 q_3 & p_3 (1 - q_3) & (1 - p_3) q_3 & (1 -p_3) (1 - q_3) \\
        p_4 q_4 & p_4 (1 - q_4) & (1 - p_4) q_4 & (1 -p_4) (1 - q_4) \\
    \end{bmatrix}
\]

Figure~\ref{fig:computed_probabilities_vs_theoretic_probabilities} shows a
regression line fitted to every pairwise interaction with a reported
\(\text{SSError}\) value (pairwise interactions with missing states were
omitted). This serves to validate the approach: a part from some edge cases the
relationship is consistent.

\begin{figure}[!htbp]
    \centering
    \includegraphics[width=.8\textwidth]{./assets/img/computed_probabilities_vs_theoretic_probabilities/main.pdf}
    \caption{The
        relationship between the steady state probabilities inferred from the
        measured transitions and the actual steady state probabilities. A linear
        regression line is included validating the approach.}
    \label{fig:computed_probabilities_vs_theoretic_probabilities}
\end{figure}


\end{document}

    strategies is considered. In this setting
    the most highly performing strategies do not play in an extortionate way
    against each other but do against lower performing strategies.
    This suggests that whilst the theory of Zero Determinant strategies
    indicates that memory is not of fundamental importance to the evolution of
    cooperative behaviour, this is incomplete.
\end{abstract}

\section{Introduction}\label{sec:introduction}

Agent based game theoretic models have become a stalwart of the underpinning
mathematics of interactive behaviours. One of the major pieces of work
in this area is the pair of original computer tournaments run by Robert
Axelrod~\cite{Axelrod1980, Axelrod1980a}. These tournaments pitted submitted
computer strategies against each other in plays of the Iterated Prisoner's
Dilemma. A common game where agents can choose to pay a slight cost to their
immediate utility in the hope of building a reputation. This has been used in
economic and evolutionary game theory to understand the evolution of cooperative
behaviour.

Recently, a class of strategies was described in~\cite{Press2012} that can
provably extort any given opponent. In~\cite{Hilbe2013, Moran1707} some
questions have already been asked about the true effectiveness of these
strategies in an evolutionary setting. Here another question is asked: is it
possible to recognise this extortionate behaviour? A mathematical procedure for
suspicion is presented: in the same way that the continued actions of an
extortionate individual might raise suspicion.

This work makes use of the Axelrod Python library~\cite{Knight2018, Knight2016}
with a large number of Prisoner Dilemma strategies available to give an
extensive numerical example of the ideas presented.  The approach is presented
in Section~\ref{sec:delta-zd-strategies}.  All of the code and data discussed
in Section~\ref{sec:numerical-experiments} is open sourced, archived and
written according to best scientific principles~\cite{Wilson2014}. The data
archive can be found at~\cite{vincent_knight_2018_1297075}.

\section{Recognising Extortion}\label{sec:delta-zd-strategies}

In~\cite{Press2012}, given a match between 2 memory-one strategies, the concept
of Zero Determinant (ZD) strategies is introduced. The main result of that paper
shows that given two memory one players \(p, q\in\mathbb{R}^4\) a linear
relationship between the players' scores could be forced by one of the players.

Using the notation of~\cite{Press2012}, assuming the utilities for player \(p\)
are given by \(S_x=(R, S, T, P)\) and for player \(q\) by \(S_y=(R, T, S, P)\)
and that the stationary scores of each player is given by \(S_X\) and \(S_Y\)
respectively. The main result of~\cite{Press2012} is that if

\begin{equation}\label{eqn:linear_relationship_for_p}
    \tilde p=\alpha S_x + \beta S_y + \gamma
\end{equation}

or

\begin{equation}\label{eqn:linear_relationship_for_q}
    \tilde q=\alpha S_x + \beta S_y + \gamma
\end{equation}

where \(\tilde p = (1 - p_1, 1 - p_2, p_3, p_4)\) and
\(\tilde q = (1 - q_1, 1 - q_2, q_3, q_4)\) then:

\begin{equation}
    \alpha S_X + \beta S_Y + \gamma = 0
\end{equation}

In~\cite{Press2012} a particular type of ZD strategy is defined: extortionate
strategies. If:

\begin{equation}\label{eqn:constraint_for_extortion}
    \gamma = - P(\alpha + \beta)
\end{equation}

then the player can ensure they get a score \(\chi\) times
larger than the opponent. This extortion coefficient is given by:

\begin{equation}\label{eqn:definition_of_chi}
    \chi=\frac{-\beta}{\alpha}
\end{equation}

Thus, if (\ref{eqn:constraint_for_extortion}) holds and \(\chi >1\) a player is
said to extort their opponent.
Here, the reverse problem is considered: given a
\(p\in\mathbb{R}^4\) how does one identify \(\alpha, \beta\) if they
exist and is the strategy in fact acting in an extortionate way?

These conditions correspond to:

\begin{align}
    \tilde p_1 & = \alpha R + \beta R - P (\alpha + \beta)
            \label{eqn:condition_for_tilde_p1}\\
    \tilde p_2 & = \alpha S + \beta T - P (\alpha + \beta)
            \label{eqn:condition_for_tilde_p2}\\
    \tilde p_3 & = \alpha T + \beta S - P (\alpha + \beta)
            \label{eqn:condition_for_tilde_p3}\\
    \tilde p_4 & = \alpha P + \beta P - P (\alpha + \beta)
            \label{eqn:condition_for_tilde_p4}
\end{align}

Equation (\ref{eqn:condition_for_tilde_p4}) ensures that \(p_4=\tilde p_4=0\).
Equations (\ref{eqn:condition_for_tilde_p1}-\ref{eqn:condition_for_tilde_p3})
can be used to eliminate \(\alpha, \beta\), giving:

\begin{equation}\label{eqn:planar_definition_of_extortion}
    \tilde p_1 = \frac{(R - P)(\tilde p_2 + \tilde p_3)}{S + T - 2P}
\end{equation}

with:

\begin{equation}\label{eqn:definition_of_chi}
    \chi = \frac{\tilde p_2 (P - T) + \tilde p_3 (S - P)}
                {\tilde p_2 (P - S) + \tilde p_3 (T - P)}
\end{equation}

Given a strategy \(p\in\mathbb{R}^{4\times 1}\) equations
(\ref{eqn:condition_for_tilde_p4}), (\ref{eqn:planar_definition_of_extortion}-\ref{eqn:definition_of_chi}) can be used to check if
a strategy is extortionate. The conditions correspond to:

\begin{align}
    p_1 & = \frac{(R-P)(p_2 + p_3) - R + T + S - P}{S + T - 2P}
     \label{eqn:condition_for_p1}\\
    p_4 & = 0 \label{eqn:condition_for_p4}\\
    1 & > p_2 + p_3\label{eqn:condition_for_chi}
\end{align}

The algebraic steps necessary to prove these results are available in the
supporting materials.

All extortionate strategies reside on a triangular (\ref{eqn:condition_for_chi})
plane (\ref{eqn:condition_for_p1}) in 3 dimensions (\ref{eqn:condition_for_p4}).
Using this formulation it can be seen that a necessary (but not sufficient)
condition for an extortionate strategy is that it cooperates on average less
than 50\% of the time when in a state of disagreement with the opponent.

As an example, consider the known extortionate strategy \(p=(8 / 9, 1 / 2, 1 /
3, 0)\) from~\cite{Stewart2012} which is referred to as \texttt{Extort-2}. In
this case, for the standard values of \((R, T, S, P)\) constraint
(\ref{eqn:condition_for_p1}) corresponds to:

\begin{equation}
    p_1 = \frac{2(p_2 + p_3) + 1}{3}
\end{equation}

It is clear that in this case all constraints hold.

This approach could in fact be used to confirm that a given strategy is acting
in an extortionate manner even if it is not a memory one strategy. However, in
practice, if a closed form for \(p\) is not known, then due to measurement
and/or numerical error this would not work.

This problem can be written in the following linear algebraic form where
\(x=(\alpha, \beta)\)
and \(p^*=(\tilde p_1 - 1, tilde_2 - 1, p_3)\):

\begin{equation}\label{eqn:linear_algebraic_equation_for_p}
    Cx= p^*
\end{equation}

\(C\) corresponds to equations
(\ref{eqn:condition_for_tilde_p1}-\ref{eqn:condition_for_tilde_p3}) and is
given by:

\begin{equation}\label{eqn:definition_of_C}
    C =
    \begin{bmatrix}
        R - P & R- P \\
        S - P & T- P \\
        T - P & S- P \\
    \end{bmatrix}
\end{equation}

Note that in general, equation (\ref{eqn:linear_algebraic_equation_for_p}) will
not necessarily have a solution. From the Rouch\'{e}-Capelli theorem if there is
a solution it is unique as \(\text{rank}(C)=2\) which is the dimension of the
variable \(x\). The best fitting \(x\) is found by minimizing:

\begin{equation}\label{eqn:r_squared}
    \text{SSError} = \|C x- p^*\|_2^2 = \sum_{i=1}^{3}\left((C\bar x)_i-p_i^*\right)^2
\end{equation}

Note that \(\text{SSError}\), which is the square of the Frobenius
norm~\cite{Golub2013}, becomes a measure of how close a strategy is to being an
extortionate strategy. Suspicion
of extortion then corresponds to a threshold on \(\text{SSError}\).

By observing interactions (human or otherwise), their memory one representation
can be inferred and this approach can be used to recognise extortionate
behaviour. The notion of comparing theoretic and actual plays of the IPD is not
novel, see for example~\cite{Rand2013}. Immediately it is noted that if the
environment is noisy~\cite{Wu1995} then no strategy can be considered to be
extortionate as \(p_4>0\).

In the next section, this idea will be illustrated by observing the interactions
that take place in a computer based tournament of the IPD\@.

\section{Numerical experiments}\label{sec:numerical-experiments}

In~\cite{Stewart2012} results from a tournament with
\documentclass[a4paper]{article}

\usepackage{amsmath}
\usepackage{amssymb}
\usepackage[margin=1.5cm,
            includefoot,
            footskip=30pt]{geometry}
\usepackage{layout}
\usepackage{graphicx}
\usepackage{subcaption}

\usepackage{biblatex}
\usepackage{pdfpages}

\bibliography{main.bib}

\title{Suspicion: Recognising and evaluating the effectiveness
       of extortion in the Iterated Prisoner's Dilemma}
\author{Vincent A. Knight \and Nikoleta E. Glynatsi}
\date{\today}



\begin{document}

\maketitle

\begin{abstract}
    The Iterated Prisoner's Dilemma is a model for rational and evolutionary
    interactive behaviour. It has applications both in the study of human social
    behaviour as well as in biology.
    It is used to understand when and how a rational individual might
    accept an immediate cost to their own utility for the direct benefit of
    another.

    Much attention has been given to a class of strategies called
    Zero Determinant strategies. It has been theoretically shown that these
    strategies can ``extort'' any player.

    In this work, an approach to identify if observed strategies are playing in
    an extortionate way is described. Furthermore, experimental analysis of
    a large tournament with \input{assets/tex/number_of_full_strategies/main.tex}
    strategies is considered. In this setting
    the most highly performing strategies do not play in an extortionate way
    against each other but do against lower performing strategies.
    This suggests that whilst the theory of Zero Determinant strategies
    indicates that memory is not of fundamental importance to the evolution of
    cooperative behaviour, this is incomplete.
\end{abstract}

\section{Introduction}\label{sec:introduction}

Agent based game theoretic models have become a stalwart of the underpinning
mathematics of interactive behaviours. One of the major pieces of work
in this area is the pair of original computer tournaments run by Robert
Axelrod~\cite{Axelrod1980, Axelrod1980a}. These tournaments pitted submitted
computer strategies against each other in plays of the Iterated Prisoner's
Dilemma. A common game where agents can choose to pay a slight cost to their
immediate utility in the hope of building a reputation. This has been used in
economic and evolutionary game theory to understand the evolution of cooperative
behaviour.

Recently, a class of strategies was described in~\cite{Press2012} that can
provably extort any given opponent. In~\cite{Hilbe2013, Moran1707} some
questions have already been asked about the true effectiveness of these
strategies in an evolutionary setting. Here another question is asked: is it
possible to recognise this extortionate behaviour? A mathematical procedure for
suspicion is presented: in the same way that the continued actions of an
extortionate individual might raise suspicion.

This work makes use of the Axelrod Python library~\cite{Knight2018, Knight2016}
with a large number of Prisoner Dilemma strategies available to give an
extensive numerical example of the ideas presented.  The approach is presented
in Section~\ref{sec:delta-zd-strategies}.  All of the code and data discussed
in Section~\ref{sec:numerical-experiments} is open sourced, archived and
written according to best scientific principles~\cite{Wilson2014}. The data
archive can be found at~\cite{vincent_knight_2018_1297075}.

\section{Recognising Extortion}\label{sec:delta-zd-strategies}

In~\cite{Press2012}, given a match between 2 memory-one strategies, the concept
of Zero Determinant (ZD) strategies is introduced. The main result of that paper
shows that given two memory one players \(p, q\in\mathbb{R}^4\) a linear
relationship between the players' scores could be forced by one of the players.

Using the notation of~\cite{Press2012}, assuming the utilities for player \(p\)
are given by \(S_x=(R, S, T, P)\) and for player \(q\) by \(S_y=(R, T, S, P)\)
and that the stationary scores of each player is given by \(S_X\) and \(S_Y\)
respectively. The main result of~\cite{Press2012} is that if

\begin{equation}\label{eqn:linear_relationship_for_p}
    \tilde p=\alpha S_x + \beta S_y + \gamma
\end{equation}

or

\begin{equation}\label{eqn:linear_relationship_for_q}
    \tilde q=\alpha S_x + \beta S_y + \gamma
\end{equation}

where \(\tilde p = (1 - p_1, 1 - p_2, p_3, p_4)\) and
\(\tilde q = (1 - q_1, 1 - q_2, q_3, q_4)\) then:

\begin{equation}
    \alpha S_X + \beta S_Y + \gamma = 0
\end{equation}

In~\cite{Press2012} a particular type of ZD strategy is defined: extortionate
strategies. If:

\begin{equation}\label{eqn:constraint_for_extortion}
    \gamma = - P(\alpha + \beta)
\end{equation}

then the player can ensure they get a score \(\chi\) times
larger than the opponent. This extortion coefficient is given by:

\begin{equation}\label{eqn:definition_of_chi}
    \chi=\frac{-\beta}{\alpha}
\end{equation}

Thus, if (\ref{eqn:constraint_for_extortion}) holds and \(\chi >1\) a player is
said to extort their opponent.
Here, the reverse problem is considered: given a
\(p\in\mathbb{R}^4\) how does one identify \(\alpha, \beta\) if they
exist and is the strategy in fact acting in an extortionate way?

These conditions correspond to:

\begin{align}
    \tilde p_1 & = \alpha R + \beta R - P (\alpha + \beta)
            \label{eqn:condition_for_tilde_p1}\\
    \tilde p_2 & = \alpha S + \beta T - P (\alpha + \beta)
            \label{eqn:condition_for_tilde_p2}\\
    \tilde p_3 & = \alpha T + \beta S - P (\alpha + \beta)
            \label{eqn:condition_for_tilde_p3}\\
    \tilde p_4 & = \alpha P + \beta P - P (\alpha + \beta)
            \label{eqn:condition_for_tilde_p4}
\end{align}

Equation (\ref{eqn:condition_for_tilde_p4}) ensures that \(p_4=\tilde p_4=0\).
Equations (\ref{eqn:condition_for_tilde_p1}-\ref{eqn:condition_for_tilde_p3})
can be used to eliminate \(\alpha, \beta\), giving:

\begin{equation}\label{eqn:planar_definition_of_extortion}
    \tilde p_1 = \frac{(R - P)(\tilde p_2 + \tilde p_3)}{S + T - 2P}
\end{equation}

with:

\begin{equation}\label{eqn:definition_of_chi}
    \chi = \frac{\tilde p_2 (P - T) + \tilde p_3 (S - P)}
                {\tilde p_2 (P - S) + \tilde p_3 (T - P)}
\end{equation}

Given a strategy \(p\in\mathbb{R}^{4\times 1}\) equations
(\ref{eqn:condition_for_tilde_p4}), (\ref{eqn:planar_definition_of_extortion}-\ref{eqn:definition_of_chi}) can be used to check if
a strategy is extortionate. The conditions correspond to:

\begin{align}
    p_1 & = \frac{(R-P)(p_2 + p_3) - R + T + S - P}{S + T - 2P}
     \label{eqn:condition_for_p1}\\
    p_4 & = 0 \label{eqn:condition_for_p4}\\
    1 & > p_2 + p_3\label{eqn:condition_for_chi}
\end{align}

The algebraic steps necessary to prove these results are available in the
supporting materials.

All extortionate strategies reside on a triangular (\ref{eqn:condition_for_chi})
plane (\ref{eqn:condition_for_p1}) in 3 dimensions (\ref{eqn:condition_for_p4}).
Using this formulation it can be seen that a necessary (but not sufficient)
condition for an extortionate strategy is that it cooperates on average less
than 50\% of the time when in a state of disagreement with the opponent.

As an example, consider the known extortionate strategy \(p=(8 / 9, 1 / 2, 1 /
3, 0)\) from~\cite{Stewart2012} which is referred to as \texttt{Extort-2}. In
this case, for the standard values of \((R, T, S, P)\) constraint
(\ref{eqn:condition_for_p1}) corresponds to:

\begin{equation}
    p_1 = \frac{2(p_2 + p_3) + 1}{3}
\end{equation}

It is clear that in this case all constraints hold.

This approach could in fact be used to confirm that a given strategy is acting
in an extortionate manner even if it is not a memory one strategy. However, in
practice, if a closed form for \(p\) is not known, then due to measurement
and/or numerical error this would not work.

This problem can be written in the following linear algebraic form where
\(x=(\alpha, \beta)\)
and \(p^*=(\tilde p_1 - 1, tilde_2 - 1, p_3)\):

\begin{equation}\label{eqn:linear_algebraic_equation_for_p}
    Cx= p^*
\end{equation}

\(C\) corresponds to equations
(\ref{eqn:condition_for_tilde_p1}-\ref{eqn:condition_for_tilde_p3}) and is
given by:

\begin{equation}\label{eqn:definition_of_C}
    C =
    \begin{bmatrix}
        R - P & R- P \\
        S - P & T- P \\
        T - P & S- P \\
    \end{bmatrix}
\end{equation}

Note that in general, equation (\ref{eqn:linear_algebraic_equation_for_p}) will
not necessarily have a solution. From the Rouch\'{e}-Capelli theorem if there is
a solution it is unique as \(\text{rank}(C)=2\) which is the dimension of the
variable \(x\). The best fitting \(x\) is found by minimizing:

\begin{equation}\label{eqn:r_squared}
    \text{SSError} = \|C x- p^*\|_2^2 = \sum_{i=1}^{3}\left((C\bar x)_i-p_i^*\right)^2
\end{equation}

Note that \(\text{SSError}\), which is the square of the Frobenius
norm~\cite{Golub2013}, becomes a measure of how close a strategy is to being an
extortionate strategy. Suspicion
of extortion then corresponds to a threshold on \(\text{SSError}\).

By observing interactions (human or otherwise), their memory one representation
can be inferred and this approach can be used to recognise extortionate
behaviour. The notion of comparing theoretic and actual plays of the IPD is not
novel, see for example~\cite{Rand2013}. Immediately it is noted that if the
environment is noisy~\cite{Wu1995} then no strategy can be considered to be
extortionate as \(p_4>0\).

In the next section, this idea will be illustrated by observing the interactions
that take place in a computer based tournament of the IPD\@.

\section{Numerical experiments}\label{sec:numerical-experiments}

In~\cite{Stewart2012} results from a tournament with
\input{./assets/tex/number_of_stewart_plotkin_strategies/main.tex} strategies,
was presented with specific consideration given to ZD strategies. This
tournament is reproduced here using the Axelrod-Python
project~\cite{Knight2016}. To obtain a good measure of the corresponding
transition rates for each strategy all matches have been run for
\input{assets/tex/number_of_turns/main.tex} turns and every match has been
repeated \input{assets/tex/number_of_repetitions/main.tex} times. All of this
interaction data is available at~\cite{vincent_knight_2018_1297075}. A good
match between the inferred Markov chain and the state distribution of the actual
interactions has been verified. Data for this is presented in the supplementary
materials.

Figure~\ref{fig:SSError_overall_in_stewart_plotkin} shows the \(\text{SSError}\)
values for all the strategies in the tournament, as reported
in~\cite{Stewart2012} the extortionate strategy (which has an expected
\(\text{SSError}\) approximately 0) gains a large number of wins.

\begin{figure}[!htbp]
    \centering
    \includegraphics[width=.8\textwidth]{./assets/img/SSError_overall_in_stewart_plotkin/main.pdf}
    \caption{\(\text{SSError}\) and state probabilities for the strategies
        of~\cite{Stewart2012}, ordered both by number of wins and overall score.
        Note that \(P(DC)\) is not shown as it corresponds to the transpose of
        \(P(CD)\). Cooperator and Defector are omitted as they do not visit all
        the states.}
    \label{fig:SSError_overall_in_stewart_plotkin}
\end{figure}

Here, the work of~\cite{Stewart2012} is extended by investigating a tournament
with \input{assets/tex/number_of_full_strategies/main.tex}
strategies.

The results of this analysis are shown in
Figure~\ref{fig:SSError_and_probabilities_in_full}. The top ranking strategies
by number of wins seem to be extortionate (but not against all strategies) and
it can be seen that a small sub group of strategies achieve mutual defection.
All the top ranking strategies according to score achieve mutual cooperation and
do not extort each other, however they
\textbf{do} exhibit extortionate behaviour towards a number of the lower ranking
strategies.

\begin{figure}[!htbp]
    \centering
    \includegraphics[width=.8\textwidth]{./assets/img/SSError_and_probabilities_in_full/main.pdf}
    \caption{\(\text{SSError}\) for the strategies for the full tournament. Only
    strategy interactions for which \(p_4=0\) and \(\chi>1\) are displayed.}
    \label{fig:SSError_and_probabilities_in_full}
\end{figure}

\section{Conclusion}\label{sec:conclusion}

This work defines an approach to measure whether or not a player is playing a
strategy that corresponds to an extortionate strategy as defined
in~\cite{Press2012}: a mathematical model for suspicion. Indeed, all
extortionate strategies have been
 classified as lying on a triangular plane.
This rigorous classification fails to be robust to small measurement error, thus
a statistical approach is proposed.
This is done through a linear algebraic approach for approximating the solution
of a linear system. Using this, a large number of pairwise interactions is
simulated and in fact very few strategies are found to act extortionately.

The work of~\cite{Press2012}, whilst showing that a clever approach to taking
advantage of another memory one strategy exists: this is incomplete. Whilst the
elegance of this result is very attractive, just as the simplicity of the
victory of Tit For Tat in Axelrod's original tournaments was, it is incomplete.
Extortionate strategies achieve a high number of wins but they do not
achieve a high score which corresponds to the fitness landscape in an
evolutionary sense. From the large number of interactions a payoff matrix \(S\)
can be measured where \(S_{ij}\) denotes the score (using standard values of
\((R, S, T, P) = (3, 0, 5, 1)\)) of the \(i\)th strategy
against the \(j\)th strategy. Using this, the replicator equation
describes the evolution of the system based on a population density fitness
function:

\begin{equation}\label{eqn:replicator_dynamics}
    \frac{dx}{dt} = x(S-x^TS x)
\end{equation}

Equation (\ref{eqn:replicator_dynamics}) is solved numerically through an
integration technique described in~\cite{Petzold1983} and
Figure~\ref{fig:replicator_dynamics} shows the evolution of the distribution of
the system: the various strategies are ranked by scores. It is clear to see that
only the high ranking strategies survive the evolutionary process (in fact,
only \input{./assets/img/replicator_dynamics/main.tex}
have a final distribution greater than \(10 ^ {-2}\)). This confirms the
findings of~\cite{Moran1707} in which sophisticated strategies resist
evolutionary invasion of shorter memory strategies. Recalling
Figure~\ref{fig:SSError_and_probabilities_in_full} this demonstrates that:

\begin{itemize}
    \item Cooperation emerges through the evolutionary process: the high scoring
        strategies do not exhibit extortionate behaviour towards each other.
    \item Extortionate strategies do not survive the evolutionary process.
\end{itemize}

\begin{figure}[!htbp]
    \centering
    \includegraphics[width=.8\textwidth]{./assets/img/replicator_dynamics/main.pdf}
    \caption{Numerical simulation of the replicator equation
    (\ref{eqn:replicator_dynamics}): strategies are ordered by score, only the strategies with a high score survive the evolutionary process.}
    \label{fig:replicator_dynamics}
\end{figure}

This work can be used to classify plays of the IPD\@: data can be collected from
actual interactions (in lab or in the field). Furthermore, this allows for a
classification method similar to the notion of fingerprinting presented
in~\cite{Ashlock2008}. Trained strategies can potentially be classified as
extortionate or not or it could be possible to even constrain the reinforcement
learning approaches that are becoming prevalent in the literature.
Alternatively, this mathematical approach for recognising extortion could be
used in sophisticated strategies to defend against invasion. Arguably, some of
the strategies considered here exhibit this behaviour, indeed as described
in~\cite{Harper2017}, the top ranking strategies in the full tournament are
obtained using evolutionary reinforcement learning techniques, thus, suspicion
of extortionate behaviour could in fact be an evolutionary trait.

\section*{Acknowledgements}

The following open source software libraries were used in this research:

\begin{itemize}
    \item The Axelrod ~\cite{Knight2016, Knight2018} library (IPD strategies and
        tournaments).
    \item The sympy library~\cite{Meurer2017} (verification of all symbolic
        calculations).
    \item The matplotlib~\cite{Droettboom2018} library (visualisation).
    \item The pandas~\cite{Structures2010}, dask~\cite{Dask2016} and
        NumPy~\cite{Oliphant2015} libraries (data manipulation).
    \item The SciPy~\cite{Jones2001} library (numerical integration of the
        replicator equation).
\end{itemize}

This work was performed using the computational facilities of the Advanced
Research Computing @ Cardiff (ARCCA) Division, Cardiff University.

\printbibliography

\newpage
\section*{Supplementary materials}

\includepdf{assets/pdf/proof_of_form_of_extortionate_strategies/main.pdf}

\newpage

Using the pair wise interactions the transition rates \(p,
q\) can be measured and the steady state probabilities inferred and compared to
the actual probabilities of each state.
This is done numerically by computing the singular eigenvector of the
matrix \(A\) \cite{Stewart2009}:

\[
    A =
    \begin{bmatrix}
        p_1 q_1 & p_1 (1 - q_1) & (1 - p_1) q_1 & (1 -p_1) (1 - q_1) \\
        p_2 q_2 & p_2 (1 - q_2) & (1 - p_2) q_2 & (1 -p_2) (1 - q_2) \\
        p_3 q_3 & p_3 (1 - q_3) & (1 - p_3) q_3 & (1 -p_3) (1 - q_3) \\
        p_4 q_4 & p_4 (1 - q_4) & (1 - p_4) q_4 & (1 -p_4) (1 - q_4) \\
    \end{bmatrix}
\]

Figure~\ref{fig:computed_probabilities_vs_theoretic_probabilities} shows a
regression line fitted to every pairwise interaction with a reported
\(\text{SSError}\) value (pairwise interactions with missing states were
omitted). This serves to validate the approach: a part from some edge cases the
relationship is consistent.

\begin{figure}[!htbp]
    \centering
    \includegraphics[width=.8\textwidth]{./assets/img/computed_probabilities_vs_theoretic_probabilities/main.pdf}
    \caption{The
        relationship between the steady state probabilities inferred from the
        measured transitions and the actual steady state probabilities. A linear
        regression line is included validating the approach.}
    \label{fig:computed_probabilities_vs_theoretic_probabilities}
\end{figure}


\end{document}
 strategies,
was presented with specific consideration given to ZD strategies. This
tournament is reproduced here using the Axelrod-Python
project~\cite{Knight2016}. To obtain a good measure of the corresponding
transition rates for each strategy all matches have been run for
\documentclass[a4paper]{article}

\usepackage{amsmath}
\usepackage{amssymb}
\usepackage[margin=1.5cm,
            includefoot,
            footskip=30pt]{geometry}
\usepackage{layout}
\usepackage{graphicx}
\usepackage{subcaption}

\usepackage{biblatex}
\usepackage{pdfpages}

\bibliography{main.bib}

\title{Suspicion: Recognising and evaluating the effectiveness
       of extortion in the Iterated Prisoner's Dilemma}
\author{Vincent A. Knight \and Nikoleta E. Glynatsi}
\date{\today}



\begin{document}

\maketitle

\begin{abstract}
    The Iterated Prisoner's Dilemma is a model for rational and evolutionary
    interactive behaviour. It has applications both in the study of human social
    behaviour as well as in biology.
    It is used to understand when and how a rational individual might
    accept an immediate cost to their own utility for the direct benefit of
    another.

    Much attention has been given to a class of strategies called
    Zero Determinant strategies. It has been theoretically shown that these
    strategies can ``extort'' any player.

    In this work, an approach to identify if observed strategies are playing in
    an extortionate way is described. Furthermore, experimental analysis of
    a large tournament with \input{assets/tex/number_of_full_strategies/main.tex}
    strategies is considered. In this setting
    the most highly performing strategies do not play in an extortionate way
    against each other but do against lower performing strategies.
    This suggests that whilst the theory of Zero Determinant strategies
    indicates that memory is not of fundamental importance to the evolution of
    cooperative behaviour, this is incomplete.
\end{abstract}

\section{Introduction}\label{sec:introduction}

Agent based game theoretic models have become a stalwart of the underpinning
mathematics of interactive behaviours. One of the major pieces of work
in this area is the pair of original computer tournaments run by Robert
Axelrod~\cite{Axelrod1980, Axelrod1980a}. These tournaments pitted submitted
computer strategies against each other in plays of the Iterated Prisoner's
Dilemma. A common game where agents can choose to pay a slight cost to their
immediate utility in the hope of building a reputation. This has been used in
economic and evolutionary game theory to understand the evolution of cooperative
behaviour.

Recently, a class of strategies was described in~\cite{Press2012} that can
provably extort any given opponent. In~\cite{Hilbe2013, Moran1707} some
questions have already been asked about the true effectiveness of these
strategies in an evolutionary setting. Here another question is asked: is it
possible to recognise this extortionate behaviour? A mathematical procedure for
suspicion is presented: in the same way that the continued actions of an
extortionate individual might raise suspicion.

This work makes use of the Axelrod Python library~\cite{Knight2018, Knight2016}
with a large number of Prisoner Dilemma strategies available to give an
extensive numerical example of the ideas presented.  The approach is presented
in Section~\ref{sec:delta-zd-strategies}.  All of the code and data discussed
in Section~\ref{sec:numerical-experiments} is open sourced, archived and
written according to best scientific principles~\cite{Wilson2014}. The data
archive can be found at~\cite{vincent_knight_2018_1297075}.

\section{Recognising Extortion}\label{sec:delta-zd-strategies}

In~\cite{Press2012}, given a match between 2 memory-one strategies, the concept
of Zero Determinant (ZD) strategies is introduced. The main result of that paper
shows that given two memory one players \(p, q\in\mathbb{R}^4\) a linear
relationship between the players' scores could be forced by one of the players.

Using the notation of~\cite{Press2012}, assuming the utilities for player \(p\)
are given by \(S_x=(R, S, T, P)\) and for player \(q\) by \(S_y=(R, T, S, P)\)
and that the stationary scores of each player is given by \(S_X\) and \(S_Y\)
respectively. The main result of~\cite{Press2012} is that if

\begin{equation}\label{eqn:linear_relationship_for_p}
    \tilde p=\alpha S_x + \beta S_y + \gamma
\end{equation}

or

\begin{equation}\label{eqn:linear_relationship_for_q}
    \tilde q=\alpha S_x + \beta S_y + \gamma
\end{equation}

where \(\tilde p = (1 - p_1, 1 - p_2, p_3, p_4)\) and
\(\tilde q = (1 - q_1, 1 - q_2, q_3, q_4)\) then:

\begin{equation}
    \alpha S_X + \beta S_Y + \gamma = 0
\end{equation}

In~\cite{Press2012} a particular type of ZD strategy is defined: extortionate
strategies. If:

\begin{equation}\label{eqn:constraint_for_extortion}
    \gamma = - P(\alpha + \beta)
\end{equation}

then the player can ensure they get a score \(\chi\) times
larger than the opponent. This extortion coefficient is given by:

\begin{equation}\label{eqn:definition_of_chi}
    \chi=\frac{-\beta}{\alpha}
\end{equation}

Thus, if (\ref{eqn:constraint_for_extortion}) holds and \(\chi >1\) a player is
said to extort their opponent.
Here, the reverse problem is considered: given a
\(p\in\mathbb{R}^4\) how does one identify \(\alpha, \beta\) if they
exist and is the strategy in fact acting in an extortionate way?

These conditions correspond to:

\begin{align}
    \tilde p_1 & = \alpha R + \beta R - P (\alpha + \beta)
            \label{eqn:condition_for_tilde_p1}\\
    \tilde p_2 & = \alpha S + \beta T - P (\alpha + \beta)
            \label{eqn:condition_for_tilde_p2}\\
    \tilde p_3 & = \alpha T + \beta S - P (\alpha + \beta)
            \label{eqn:condition_for_tilde_p3}\\
    \tilde p_4 & = \alpha P + \beta P - P (\alpha + \beta)
            \label{eqn:condition_for_tilde_p4}
\end{align}

Equation (\ref{eqn:condition_for_tilde_p4}) ensures that \(p_4=\tilde p_4=0\).
Equations (\ref{eqn:condition_for_tilde_p1}-\ref{eqn:condition_for_tilde_p3})
can be used to eliminate \(\alpha, \beta\), giving:

\begin{equation}\label{eqn:planar_definition_of_extortion}
    \tilde p_1 = \frac{(R - P)(\tilde p_2 + \tilde p_3)}{S + T - 2P}
\end{equation}

with:

\begin{equation}\label{eqn:definition_of_chi}
    \chi = \frac{\tilde p_2 (P - T) + \tilde p_3 (S - P)}
                {\tilde p_2 (P - S) + \tilde p_3 (T - P)}
\end{equation}

Given a strategy \(p\in\mathbb{R}^{4\times 1}\) equations
(\ref{eqn:condition_for_tilde_p4}), (\ref{eqn:planar_definition_of_extortion}-\ref{eqn:definition_of_chi}) can be used to check if
a strategy is extortionate. The conditions correspond to:

\begin{align}
    p_1 & = \frac{(R-P)(p_2 + p_3) - R + T + S - P}{S + T - 2P}
     \label{eqn:condition_for_p1}\\
    p_4 & = 0 \label{eqn:condition_for_p4}\\
    1 & > p_2 + p_3\label{eqn:condition_for_chi}
\end{align}

The algebraic steps necessary to prove these results are available in the
supporting materials.

All extortionate strategies reside on a triangular (\ref{eqn:condition_for_chi})
plane (\ref{eqn:condition_for_p1}) in 3 dimensions (\ref{eqn:condition_for_p4}).
Using this formulation it can be seen that a necessary (but not sufficient)
condition for an extortionate strategy is that it cooperates on average less
than 50\% of the time when in a state of disagreement with the opponent.

As an example, consider the known extortionate strategy \(p=(8 / 9, 1 / 2, 1 /
3, 0)\) from~\cite{Stewart2012} which is referred to as \texttt{Extort-2}. In
this case, for the standard values of \((R, T, S, P)\) constraint
(\ref{eqn:condition_for_p1}) corresponds to:

\begin{equation}
    p_1 = \frac{2(p_2 + p_3) + 1}{3}
\end{equation}

It is clear that in this case all constraints hold.

This approach could in fact be used to confirm that a given strategy is acting
in an extortionate manner even if it is not a memory one strategy. However, in
practice, if a closed form for \(p\) is not known, then due to measurement
and/or numerical error this would not work.

This problem can be written in the following linear algebraic form where
\(x=(\alpha, \beta)\)
and \(p^*=(\tilde p_1 - 1, tilde_2 - 1, p_3)\):

\begin{equation}\label{eqn:linear_algebraic_equation_for_p}
    Cx= p^*
\end{equation}

\(C\) corresponds to equations
(\ref{eqn:condition_for_tilde_p1}-\ref{eqn:condition_for_tilde_p3}) and is
given by:

\begin{equation}\label{eqn:definition_of_C}
    C =
    \begin{bmatrix}
        R - P & R- P \\
        S - P & T- P \\
        T - P & S- P \\
    \end{bmatrix}
\end{equation}

Note that in general, equation (\ref{eqn:linear_algebraic_equation_for_p}) will
not necessarily have a solution. From the Rouch\'{e}-Capelli theorem if there is
a solution it is unique as \(\text{rank}(C)=2\) which is the dimension of the
variable \(x\). The best fitting \(x\) is found by minimizing:

\begin{equation}\label{eqn:r_squared}
    \text{SSError} = \|C x- p^*\|_2^2 = \sum_{i=1}^{3}\left((C\bar x)_i-p_i^*\right)^2
\end{equation}

Note that \(\text{SSError}\), which is the square of the Frobenius
norm~\cite{Golub2013}, becomes a measure of how close a strategy is to being an
extortionate strategy. Suspicion
of extortion then corresponds to a threshold on \(\text{SSError}\).

By observing interactions (human or otherwise), their memory one representation
can be inferred and this approach can be used to recognise extortionate
behaviour. The notion of comparing theoretic and actual plays of the IPD is not
novel, see for example~\cite{Rand2013}. Immediately it is noted that if the
environment is noisy~\cite{Wu1995} then no strategy can be considered to be
extortionate as \(p_4>0\).

In the next section, this idea will be illustrated by observing the interactions
that take place in a computer based tournament of the IPD\@.

\section{Numerical experiments}\label{sec:numerical-experiments}

In~\cite{Stewart2012} results from a tournament with
\input{./assets/tex/number_of_stewart_plotkin_strategies/main.tex} strategies,
was presented with specific consideration given to ZD strategies. This
tournament is reproduced here using the Axelrod-Python
project~\cite{Knight2016}. To obtain a good measure of the corresponding
transition rates for each strategy all matches have been run for
\input{assets/tex/number_of_turns/main.tex} turns and every match has been
repeated \input{assets/tex/number_of_repetitions/main.tex} times. All of this
interaction data is available at~\cite{vincent_knight_2018_1297075}. A good
match between the inferred Markov chain and the state distribution of the actual
interactions has been verified. Data for this is presented in the supplementary
materials.

Figure~\ref{fig:SSError_overall_in_stewart_plotkin} shows the \(\text{SSError}\)
values for all the strategies in the tournament, as reported
in~\cite{Stewart2012} the extortionate strategy (which has an expected
\(\text{SSError}\) approximately 0) gains a large number of wins.

\begin{figure}[!htbp]
    \centering
    \includegraphics[width=.8\textwidth]{./assets/img/SSError_overall_in_stewart_plotkin/main.pdf}
    \caption{\(\text{SSError}\) and state probabilities for the strategies
        of~\cite{Stewart2012}, ordered both by number of wins and overall score.
        Note that \(P(DC)\) is not shown as it corresponds to the transpose of
        \(P(CD)\). Cooperator and Defector are omitted as they do not visit all
        the states.}
    \label{fig:SSError_overall_in_stewart_plotkin}
\end{figure}

Here, the work of~\cite{Stewart2012} is extended by investigating a tournament
with \input{assets/tex/number_of_full_strategies/main.tex}
strategies.

The results of this analysis are shown in
Figure~\ref{fig:SSError_and_probabilities_in_full}. The top ranking strategies
by number of wins seem to be extortionate (but not against all strategies) and
it can be seen that a small sub group of strategies achieve mutual defection.
All the top ranking strategies according to score achieve mutual cooperation and
do not extort each other, however they
\textbf{do} exhibit extortionate behaviour towards a number of the lower ranking
strategies.

\begin{figure}[!htbp]
    \centering
    \includegraphics[width=.8\textwidth]{./assets/img/SSError_and_probabilities_in_full/main.pdf}
    \caption{\(\text{SSError}\) for the strategies for the full tournament. Only
    strategy interactions for which \(p_4=0\) and \(\chi>1\) are displayed.}
    \label{fig:SSError_and_probabilities_in_full}
\end{figure}

\section{Conclusion}\label{sec:conclusion}

This work defines an approach to measure whether or not a player is playing a
strategy that corresponds to an extortionate strategy as defined
in~\cite{Press2012}: a mathematical model for suspicion. Indeed, all
extortionate strategies have been
 classified as lying on a triangular plane.
This rigorous classification fails to be robust to small measurement error, thus
a statistical approach is proposed.
This is done through a linear algebraic approach for approximating the solution
of a linear system. Using this, a large number of pairwise interactions is
simulated and in fact very few strategies are found to act extortionately.

The work of~\cite{Press2012}, whilst showing that a clever approach to taking
advantage of another memory one strategy exists: this is incomplete. Whilst the
elegance of this result is very attractive, just as the simplicity of the
victory of Tit For Tat in Axelrod's original tournaments was, it is incomplete.
Extortionate strategies achieve a high number of wins but they do not
achieve a high score which corresponds to the fitness landscape in an
evolutionary sense. From the large number of interactions a payoff matrix \(S\)
can be measured where \(S_{ij}\) denotes the score (using standard values of
\((R, S, T, P) = (3, 0, 5, 1)\)) of the \(i\)th strategy
against the \(j\)th strategy. Using this, the replicator equation
describes the evolution of the system based on a population density fitness
function:

\begin{equation}\label{eqn:replicator_dynamics}
    \frac{dx}{dt} = x(S-x^TS x)
\end{equation}

Equation (\ref{eqn:replicator_dynamics}) is solved numerically through an
integration technique described in~\cite{Petzold1983} and
Figure~\ref{fig:replicator_dynamics} shows the evolution of the distribution of
the system: the various strategies are ranked by scores. It is clear to see that
only the high ranking strategies survive the evolutionary process (in fact,
only \input{./assets/img/replicator_dynamics/main.tex}
have a final distribution greater than \(10 ^ {-2}\)). This confirms the
findings of~\cite{Moran1707} in which sophisticated strategies resist
evolutionary invasion of shorter memory strategies. Recalling
Figure~\ref{fig:SSError_and_probabilities_in_full} this demonstrates that:

\begin{itemize}
    \item Cooperation emerges through the evolutionary process: the high scoring
        strategies do not exhibit extortionate behaviour towards each other.
    \item Extortionate strategies do not survive the evolutionary process.
\end{itemize}

\begin{figure}[!htbp]
    \centering
    \includegraphics[width=.8\textwidth]{./assets/img/replicator_dynamics/main.pdf}
    \caption{Numerical simulation of the replicator equation
    (\ref{eqn:replicator_dynamics}): strategies are ordered by score, only the strategies with a high score survive the evolutionary process.}
    \label{fig:replicator_dynamics}
\end{figure}

This work can be used to classify plays of the IPD\@: data can be collected from
actual interactions (in lab or in the field). Furthermore, this allows for a
classification method similar to the notion of fingerprinting presented
in~\cite{Ashlock2008}. Trained strategies can potentially be classified as
extortionate or not or it could be possible to even constrain the reinforcement
learning approaches that are becoming prevalent in the literature.
Alternatively, this mathematical approach for recognising extortion could be
used in sophisticated strategies to defend against invasion. Arguably, some of
the strategies considered here exhibit this behaviour, indeed as described
in~\cite{Harper2017}, the top ranking strategies in the full tournament are
obtained using evolutionary reinforcement learning techniques, thus, suspicion
of extortionate behaviour could in fact be an evolutionary trait.

\section*{Acknowledgements}

The following open source software libraries were used in this research:

\begin{itemize}
    \item The Axelrod ~\cite{Knight2016, Knight2018} library (IPD strategies and
        tournaments).
    \item The sympy library~\cite{Meurer2017} (verification of all symbolic
        calculations).
    \item The matplotlib~\cite{Droettboom2018} library (visualisation).
    \item The pandas~\cite{Structures2010}, dask~\cite{Dask2016} and
        NumPy~\cite{Oliphant2015} libraries (data manipulation).
    \item The SciPy~\cite{Jones2001} library (numerical integration of the
        replicator equation).
\end{itemize}

This work was performed using the computational facilities of the Advanced
Research Computing @ Cardiff (ARCCA) Division, Cardiff University.

\printbibliography

\newpage
\section*{Supplementary materials}

\includepdf{assets/pdf/proof_of_form_of_extortionate_strategies/main.pdf}

\newpage

Using the pair wise interactions the transition rates \(p,
q\) can be measured and the steady state probabilities inferred and compared to
the actual probabilities of each state.
This is done numerically by computing the singular eigenvector of the
matrix \(A\) \cite{Stewart2009}:

\[
    A =
    \begin{bmatrix}
        p_1 q_1 & p_1 (1 - q_1) & (1 - p_1) q_1 & (1 -p_1) (1 - q_1) \\
        p_2 q_2 & p_2 (1 - q_2) & (1 - p_2) q_2 & (1 -p_2) (1 - q_2) \\
        p_3 q_3 & p_3 (1 - q_3) & (1 - p_3) q_3 & (1 -p_3) (1 - q_3) \\
        p_4 q_4 & p_4 (1 - q_4) & (1 - p_4) q_4 & (1 -p_4) (1 - q_4) \\
    \end{bmatrix}
\]

Figure~\ref{fig:computed_probabilities_vs_theoretic_probabilities} shows a
regression line fitted to every pairwise interaction with a reported
\(\text{SSError}\) value (pairwise interactions with missing states were
omitted). This serves to validate the approach: a part from some edge cases the
relationship is consistent.

\begin{figure}[!htbp]
    \centering
    \includegraphics[width=.8\textwidth]{./assets/img/computed_probabilities_vs_theoretic_probabilities/main.pdf}
    \caption{The
        relationship between the steady state probabilities inferred from the
        measured transitions and the actual steady state probabilities. A linear
        regression line is included validating the approach.}
    \label{fig:computed_probabilities_vs_theoretic_probabilities}
\end{figure}


\end{document}
 turns and every match has been
repeated \documentclass[a4paper]{article}

\usepackage{amsmath}
\usepackage{amssymb}
\usepackage[margin=1.5cm,
            includefoot,
            footskip=30pt]{geometry}
\usepackage{layout}
\usepackage{graphicx}
\usepackage{subcaption}

\usepackage{biblatex}
\usepackage{pdfpages}

\bibliography{main.bib}

\title{Suspicion: Recognising and evaluating the effectiveness
       of extortion in the Iterated Prisoner's Dilemma}
\author{Vincent A. Knight \and Nikoleta E. Glynatsi}
\date{\today}



\begin{document}

\maketitle

\begin{abstract}
    The Iterated Prisoner's Dilemma is a model for rational and evolutionary
    interactive behaviour. It has applications both in the study of human social
    behaviour as well as in biology.
    It is used to understand when and how a rational individual might
    accept an immediate cost to their own utility for the direct benefit of
    another.

    Much attention has been given to a class of strategies called
    Zero Determinant strategies. It has been theoretically shown that these
    strategies can ``extort'' any player.

    In this work, an approach to identify if observed strategies are playing in
    an extortionate way is described. Furthermore, experimental analysis of
    a large tournament with \input{assets/tex/number_of_full_strategies/main.tex}
    strategies is considered. In this setting
    the most highly performing strategies do not play in an extortionate way
    against each other but do against lower performing strategies.
    This suggests that whilst the theory of Zero Determinant strategies
    indicates that memory is not of fundamental importance to the evolution of
    cooperative behaviour, this is incomplete.
\end{abstract}

\section{Introduction}\label{sec:introduction}

Agent based game theoretic models have become a stalwart of the underpinning
mathematics of interactive behaviours. One of the major pieces of work
in this area is the pair of original computer tournaments run by Robert
Axelrod~\cite{Axelrod1980, Axelrod1980a}. These tournaments pitted submitted
computer strategies against each other in plays of the Iterated Prisoner's
Dilemma. A common game where agents can choose to pay a slight cost to their
immediate utility in the hope of building a reputation. This has been used in
economic and evolutionary game theory to understand the evolution of cooperative
behaviour.

Recently, a class of strategies was described in~\cite{Press2012} that can
provably extort any given opponent. In~\cite{Hilbe2013, Moran1707} some
questions have already been asked about the true effectiveness of these
strategies in an evolutionary setting. Here another question is asked: is it
possible to recognise this extortionate behaviour? A mathematical procedure for
suspicion is presented: in the same way that the continued actions of an
extortionate individual might raise suspicion.

This work makes use of the Axelrod Python library~\cite{Knight2018, Knight2016}
with a large number of Prisoner Dilemma strategies available to give an
extensive numerical example of the ideas presented.  The approach is presented
in Section~\ref{sec:delta-zd-strategies}.  All of the code and data discussed
in Section~\ref{sec:numerical-experiments} is open sourced, archived and
written according to best scientific principles~\cite{Wilson2014}. The data
archive can be found at~\cite{vincent_knight_2018_1297075}.

\section{Recognising Extortion}\label{sec:delta-zd-strategies}

In~\cite{Press2012}, given a match between 2 memory-one strategies, the concept
of Zero Determinant (ZD) strategies is introduced. The main result of that paper
shows that given two memory one players \(p, q\in\mathbb{R}^4\) a linear
relationship between the players' scores could be forced by one of the players.

Using the notation of~\cite{Press2012}, assuming the utilities for player \(p\)
are given by \(S_x=(R, S, T, P)\) and for player \(q\) by \(S_y=(R, T, S, P)\)
and that the stationary scores of each player is given by \(S_X\) and \(S_Y\)
respectively. The main result of~\cite{Press2012} is that if

\begin{equation}\label{eqn:linear_relationship_for_p}
    \tilde p=\alpha S_x + \beta S_y + \gamma
\end{equation}

or

\begin{equation}\label{eqn:linear_relationship_for_q}
    \tilde q=\alpha S_x + \beta S_y + \gamma
\end{equation}

where \(\tilde p = (1 - p_1, 1 - p_2, p_3, p_4)\) and
\(\tilde q = (1 - q_1, 1 - q_2, q_3, q_4)\) then:

\begin{equation}
    \alpha S_X + \beta S_Y + \gamma = 0
\end{equation}

In~\cite{Press2012} a particular type of ZD strategy is defined: extortionate
strategies. If:

\begin{equation}\label{eqn:constraint_for_extortion}
    \gamma = - P(\alpha + \beta)
\end{equation}

then the player can ensure they get a score \(\chi\) times
larger than the opponent. This extortion coefficient is given by:

\begin{equation}\label{eqn:definition_of_chi}
    \chi=\frac{-\beta}{\alpha}
\end{equation}

Thus, if (\ref{eqn:constraint_for_extortion}) holds and \(\chi >1\) a player is
said to extort their opponent.
Here, the reverse problem is considered: given a
\(p\in\mathbb{R}^4\) how does one identify \(\alpha, \beta\) if they
exist and is the strategy in fact acting in an extortionate way?

These conditions correspond to:

\begin{align}
    \tilde p_1 & = \alpha R + \beta R - P (\alpha + \beta)
            \label{eqn:condition_for_tilde_p1}\\
    \tilde p_2 & = \alpha S + \beta T - P (\alpha + \beta)
            \label{eqn:condition_for_tilde_p2}\\
    \tilde p_3 & = \alpha T + \beta S - P (\alpha + \beta)
            \label{eqn:condition_for_tilde_p3}\\
    \tilde p_4 & = \alpha P + \beta P - P (\alpha + \beta)
            \label{eqn:condition_for_tilde_p4}
\end{align}

Equation (\ref{eqn:condition_for_tilde_p4}) ensures that \(p_4=\tilde p_4=0\).
Equations (\ref{eqn:condition_for_tilde_p1}-\ref{eqn:condition_for_tilde_p3})
can be used to eliminate \(\alpha, \beta\), giving:

\begin{equation}\label{eqn:planar_definition_of_extortion}
    \tilde p_1 = \frac{(R - P)(\tilde p_2 + \tilde p_3)}{S + T - 2P}
\end{equation}

with:

\begin{equation}\label{eqn:definition_of_chi}
    \chi = \frac{\tilde p_2 (P - T) + \tilde p_3 (S - P)}
                {\tilde p_2 (P - S) + \tilde p_3 (T - P)}
\end{equation}

Given a strategy \(p\in\mathbb{R}^{4\times 1}\) equations
(\ref{eqn:condition_for_tilde_p4}), (\ref{eqn:planar_definition_of_extortion}-\ref{eqn:definition_of_chi}) can be used to check if
a strategy is extortionate. The conditions correspond to:

\begin{align}
    p_1 & = \frac{(R-P)(p_2 + p_3) - R + T + S - P}{S + T - 2P}
     \label{eqn:condition_for_p1}\\
    p_4 & = 0 \label{eqn:condition_for_p4}\\
    1 & > p_2 + p_3\label{eqn:condition_for_chi}
\end{align}

The algebraic steps necessary to prove these results are available in the
supporting materials.

All extortionate strategies reside on a triangular (\ref{eqn:condition_for_chi})
plane (\ref{eqn:condition_for_p1}) in 3 dimensions (\ref{eqn:condition_for_p4}).
Using this formulation it can be seen that a necessary (but not sufficient)
condition for an extortionate strategy is that it cooperates on average less
than 50\% of the time when in a state of disagreement with the opponent.

As an example, consider the known extortionate strategy \(p=(8 / 9, 1 / 2, 1 /
3, 0)\) from~\cite{Stewart2012} which is referred to as \texttt{Extort-2}. In
this case, for the standard values of \((R, T, S, P)\) constraint
(\ref{eqn:condition_for_p1}) corresponds to:

\begin{equation}
    p_1 = \frac{2(p_2 + p_3) + 1}{3}
\end{equation}

It is clear that in this case all constraints hold.

This approach could in fact be used to confirm that a given strategy is acting
in an extortionate manner even if it is not a memory one strategy. However, in
practice, if a closed form for \(p\) is not known, then due to measurement
and/or numerical error this would not work.

This problem can be written in the following linear algebraic form where
\(x=(\alpha, \beta)\)
and \(p^*=(\tilde p_1 - 1, tilde_2 - 1, p_3)\):

\begin{equation}\label{eqn:linear_algebraic_equation_for_p}
    Cx= p^*
\end{equation}

\(C\) corresponds to equations
(\ref{eqn:condition_for_tilde_p1}-\ref{eqn:condition_for_tilde_p3}) and is
given by:

\begin{equation}\label{eqn:definition_of_C}
    C =
    \begin{bmatrix}
        R - P & R- P \\
        S - P & T- P \\
        T - P & S- P \\
    \end{bmatrix}
\end{equation}

Note that in general, equation (\ref{eqn:linear_algebraic_equation_for_p}) will
not necessarily have a solution. From the Rouch\'{e}-Capelli theorem if there is
a solution it is unique as \(\text{rank}(C)=2\) which is the dimension of the
variable \(x\). The best fitting \(x\) is found by minimizing:

\begin{equation}\label{eqn:r_squared}
    \text{SSError} = \|C x- p^*\|_2^2 = \sum_{i=1}^{3}\left((C\bar x)_i-p_i^*\right)^2
\end{equation}

Note that \(\text{SSError}\), which is the square of the Frobenius
norm~\cite{Golub2013}, becomes a measure of how close a strategy is to being an
extortionate strategy. Suspicion
of extortion then corresponds to a threshold on \(\text{SSError}\).

By observing interactions (human or otherwise), their memory one representation
can be inferred and this approach can be used to recognise extortionate
behaviour. The notion of comparing theoretic and actual plays of the IPD is not
novel, see for example~\cite{Rand2013}. Immediately it is noted that if the
environment is noisy~\cite{Wu1995} then no strategy can be considered to be
extortionate as \(p_4>0\).

In the next section, this idea will be illustrated by observing the interactions
that take place in a computer based tournament of the IPD\@.

\section{Numerical experiments}\label{sec:numerical-experiments}

In~\cite{Stewart2012} results from a tournament with
\input{./assets/tex/number_of_stewart_plotkin_strategies/main.tex} strategies,
was presented with specific consideration given to ZD strategies. This
tournament is reproduced here using the Axelrod-Python
project~\cite{Knight2016}. To obtain a good measure of the corresponding
transition rates for each strategy all matches have been run for
\input{assets/tex/number_of_turns/main.tex} turns and every match has been
repeated \input{assets/tex/number_of_repetitions/main.tex} times. All of this
interaction data is available at~\cite{vincent_knight_2018_1297075}. A good
match between the inferred Markov chain and the state distribution of the actual
interactions has been verified. Data for this is presented in the supplementary
materials.

Figure~\ref{fig:SSError_overall_in_stewart_plotkin} shows the \(\text{SSError}\)
values for all the strategies in the tournament, as reported
in~\cite{Stewart2012} the extortionate strategy (which has an expected
\(\text{SSError}\) approximately 0) gains a large number of wins.

\begin{figure}[!htbp]
    \centering
    \includegraphics[width=.8\textwidth]{./assets/img/SSError_overall_in_stewart_plotkin/main.pdf}
    \caption{\(\text{SSError}\) and state probabilities for the strategies
        of~\cite{Stewart2012}, ordered both by number of wins and overall score.
        Note that \(P(DC)\) is not shown as it corresponds to the transpose of
        \(P(CD)\). Cooperator and Defector are omitted as they do not visit all
        the states.}
    \label{fig:SSError_overall_in_stewart_plotkin}
\end{figure}

Here, the work of~\cite{Stewart2012} is extended by investigating a tournament
with \input{assets/tex/number_of_full_strategies/main.tex}
strategies.

The results of this analysis are shown in
Figure~\ref{fig:SSError_and_probabilities_in_full}. The top ranking strategies
by number of wins seem to be extortionate (but not against all strategies) and
it can be seen that a small sub group of strategies achieve mutual defection.
All the top ranking strategies according to score achieve mutual cooperation and
do not extort each other, however they
\textbf{do} exhibit extortionate behaviour towards a number of the lower ranking
strategies.

\begin{figure}[!htbp]
    \centering
    \includegraphics[width=.8\textwidth]{./assets/img/SSError_and_probabilities_in_full/main.pdf}
    \caption{\(\text{SSError}\) for the strategies for the full tournament. Only
    strategy interactions for which \(p_4=0\) and \(\chi>1\) are displayed.}
    \label{fig:SSError_and_probabilities_in_full}
\end{figure}

\section{Conclusion}\label{sec:conclusion}

This work defines an approach to measure whether or not a player is playing a
strategy that corresponds to an extortionate strategy as defined
in~\cite{Press2012}: a mathematical model for suspicion. Indeed, all
extortionate strategies have been
 classified as lying on a triangular plane.
This rigorous classification fails to be robust to small measurement error, thus
a statistical approach is proposed.
This is done through a linear algebraic approach for approximating the solution
of a linear system. Using this, a large number of pairwise interactions is
simulated and in fact very few strategies are found to act extortionately.

The work of~\cite{Press2012}, whilst showing that a clever approach to taking
advantage of another memory one strategy exists: this is incomplete. Whilst the
elegance of this result is very attractive, just as the simplicity of the
victory of Tit For Tat in Axelrod's original tournaments was, it is incomplete.
Extortionate strategies achieve a high number of wins but they do not
achieve a high score which corresponds to the fitness landscape in an
evolutionary sense. From the large number of interactions a payoff matrix \(S\)
can be measured where \(S_{ij}\) denotes the score (using standard values of
\((R, S, T, P) = (3, 0, 5, 1)\)) of the \(i\)th strategy
against the \(j\)th strategy. Using this, the replicator equation
describes the evolution of the system based on a population density fitness
function:

\begin{equation}\label{eqn:replicator_dynamics}
    \frac{dx}{dt} = x(S-x^TS x)
\end{equation}

Equation (\ref{eqn:replicator_dynamics}) is solved numerically through an
integration technique described in~\cite{Petzold1983} and
Figure~\ref{fig:replicator_dynamics} shows the evolution of the distribution of
the system: the various strategies are ranked by scores. It is clear to see that
only the high ranking strategies survive the evolutionary process (in fact,
only \input{./assets/img/replicator_dynamics/main.tex}
have a final distribution greater than \(10 ^ {-2}\)). This confirms the
findings of~\cite{Moran1707} in which sophisticated strategies resist
evolutionary invasion of shorter memory strategies. Recalling
Figure~\ref{fig:SSError_and_probabilities_in_full} this demonstrates that:

\begin{itemize}
    \item Cooperation emerges through the evolutionary process: the high scoring
        strategies do not exhibit extortionate behaviour towards each other.
    \item Extortionate strategies do not survive the evolutionary process.
\end{itemize}

\begin{figure}[!htbp]
    \centering
    \includegraphics[width=.8\textwidth]{./assets/img/replicator_dynamics/main.pdf}
    \caption{Numerical simulation of the replicator equation
    (\ref{eqn:replicator_dynamics}): strategies are ordered by score, only the strategies with a high score survive the evolutionary process.}
    \label{fig:replicator_dynamics}
\end{figure}

This work can be used to classify plays of the IPD\@: data can be collected from
actual interactions (in lab or in the field). Furthermore, this allows for a
classification method similar to the notion of fingerprinting presented
in~\cite{Ashlock2008}. Trained strategies can potentially be classified as
extortionate or not or it could be possible to even constrain the reinforcement
learning approaches that are becoming prevalent in the literature.
Alternatively, this mathematical approach for recognising extortion could be
used in sophisticated strategies to defend against invasion. Arguably, some of
the strategies considered here exhibit this behaviour, indeed as described
in~\cite{Harper2017}, the top ranking strategies in the full tournament are
obtained using evolutionary reinforcement learning techniques, thus, suspicion
of extortionate behaviour could in fact be an evolutionary trait.

\section*{Acknowledgements}

The following open source software libraries were used in this research:

\begin{itemize}
    \item The Axelrod ~\cite{Knight2016, Knight2018} library (IPD strategies and
        tournaments).
    \item The sympy library~\cite{Meurer2017} (verification of all symbolic
        calculations).
    \item The matplotlib~\cite{Droettboom2018} library (visualisation).
    \item The pandas~\cite{Structures2010}, dask~\cite{Dask2016} and
        NumPy~\cite{Oliphant2015} libraries (data manipulation).
    \item The SciPy~\cite{Jones2001} library (numerical integration of the
        replicator equation).
\end{itemize}

This work was performed using the computational facilities of the Advanced
Research Computing @ Cardiff (ARCCA) Division, Cardiff University.

\printbibliography

\newpage
\section*{Supplementary materials}

\includepdf{assets/pdf/proof_of_form_of_extortionate_strategies/main.pdf}

\newpage

Using the pair wise interactions the transition rates \(p,
q\) can be measured and the steady state probabilities inferred and compared to
the actual probabilities of each state.
This is done numerically by computing the singular eigenvector of the
matrix \(A\) \cite{Stewart2009}:

\[
    A =
    \begin{bmatrix}
        p_1 q_1 & p_1 (1 - q_1) & (1 - p_1) q_1 & (1 -p_1) (1 - q_1) \\
        p_2 q_2 & p_2 (1 - q_2) & (1 - p_2) q_2 & (1 -p_2) (1 - q_2) \\
        p_3 q_3 & p_3 (1 - q_3) & (1 - p_3) q_3 & (1 -p_3) (1 - q_3) \\
        p_4 q_4 & p_4 (1 - q_4) & (1 - p_4) q_4 & (1 -p_4) (1 - q_4) \\
    \end{bmatrix}
\]

Figure~\ref{fig:computed_probabilities_vs_theoretic_probabilities} shows a
regression line fitted to every pairwise interaction with a reported
\(\text{SSError}\) value (pairwise interactions with missing states were
omitted). This serves to validate the approach: a part from some edge cases the
relationship is consistent.

\begin{figure}[!htbp]
    \centering
    \includegraphics[width=.8\textwidth]{./assets/img/computed_probabilities_vs_theoretic_probabilities/main.pdf}
    \caption{The
        relationship between the steady state probabilities inferred from the
        measured transitions and the actual steady state probabilities. A linear
        regression line is included validating the approach.}
    \label{fig:computed_probabilities_vs_theoretic_probabilities}
\end{figure}


\end{document}
 times. All of this
interaction data is available at~\cite{vincent_knight_2018_1297075}. A good
match between the inferred Markov chain and the state distribution of the actual
interactions has been verified. Data for this is presented in the supplementary
materials.

Figure~\ref{fig:SSError_overall_in_stewart_plotkin} shows the \(\text{SSError}\)
values for all the strategies in the tournament, as reported
in~\cite{Stewart2012} the extortionate strategy (which has an expected
\(\text{SSError}\) approximately 0) gains a large number of wins.

\begin{figure}[!htbp]
    \centering
    \includegraphics[width=.8\textwidth]{./assets/img/SSError_overall_in_stewart_plotkin/main.pdf}
    \caption{\(\text{SSError}\) and state probabilities for the strategies
        of~\cite{Stewart2012}, ordered both by number of wins and overall score.
        Note that \(P(DC)\) is not shown as it corresponds to the transpose of
        \(P(CD)\). Cooperator and Defector are omitted as they do not visit all
        the states.}
    \label{fig:SSError_overall_in_stewart_plotkin}
\end{figure}

Here, the work of~\cite{Stewart2012} is extended by investigating a tournament
with \documentclass[a4paper]{article}

\usepackage{amsmath}
\usepackage{amssymb}
\usepackage[margin=1.5cm,
            includefoot,
            footskip=30pt]{geometry}
\usepackage{layout}
\usepackage{graphicx}
\usepackage{subcaption}

\usepackage{biblatex}
\usepackage{pdfpages}

\bibliography{main.bib}

\title{Suspicion: Recognising and evaluating the effectiveness
       of extortion in the Iterated Prisoner's Dilemma}
\author{Vincent A. Knight \and Nikoleta E. Glynatsi}
\date{\today}



\begin{document}

\maketitle

\begin{abstract}
    The Iterated Prisoner's Dilemma is a model for rational and evolutionary
    interactive behaviour. It has applications both in the study of human social
    behaviour as well as in biology.
    It is used to understand when and how a rational individual might
    accept an immediate cost to their own utility for the direct benefit of
    another.

    Much attention has been given to a class of strategies called
    Zero Determinant strategies. It has been theoretically shown that these
    strategies can ``extort'' any player.

    In this work, an approach to identify if observed strategies are playing in
    an extortionate way is described. Furthermore, experimental analysis of
    a large tournament with \input{assets/tex/number_of_full_strategies/main.tex}
    strategies is considered. In this setting
    the most highly performing strategies do not play in an extortionate way
    against each other but do against lower performing strategies.
    This suggests that whilst the theory of Zero Determinant strategies
    indicates that memory is not of fundamental importance to the evolution of
    cooperative behaviour, this is incomplete.
\end{abstract}

\section{Introduction}\label{sec:introduction}

Agent based game theoretic models have become a stalwart of the underpinning
mathematics of interactive behaviours. One of the major pieces of work
in this area is the pair of original computer tournaments run by Robert
Axelrod~\cite{Axelrod1980, Axelrod1980a}. These tournaments pitted submitted
computer strategies against each other in plays of the Iterated Prisoner's
Dilemma. A common game where agents can choose to pay a slight cost to their
immediate utility in the hope of building a reputation. This has been used in
economic and evolutionary game theory to understand the evolution of cooperative
behaviour.

Recently, a class of strategies was described in~\cite{Press2012} that can
provably extort any given opponent. In~\cite{Hilbe2013, Moran1707} some
questions have already been asked about the true effectiveness of these
strategies in an evolutionary setting. Here another question is asked: is it
possible to recognise this extortionate behaviour? A mathematical procedure for
suspicion is presented: in the same way that the continued actions of an
extortionate individual might raise suspicion.

This work makes use of the Axelrod Python library~\cite{Knight2018, Knight2016}
with a large number of Prisoner Dilemma strategies available to give an
extensive numerical example of the ideas presented.  The approach is presented
in Section~\ref{sec:delta-zd-strategies}.  All of the code and data discussed
in Section~\ref{sec:numerical-experiments} is open sourced, archived and
written according to best scientific principles~\cite{Wilson2014}. The data
archive can be found at~\cite{vincent_knight_2018_1297075}.

\section{Recognising Extortion}\label{sec:delta-zd-strategies}

In~\cite{Press2012}, given a match between 2 memory-one strategies, the concept
of Zero Determinant (ZD) strategies is introduced. The main result of that paper
shows that given two memory one players \(p, q\in\mathbb{R}^4\) a linear
relationship between the players' scores could be forced by one of the players.

Using the notation of~\cite{Press2012}, assuming the utilities for player \(p\)
are given by \(S_x=(R, S, T, P)\) and for player \(q\) by \(S_y=(R, T, S, P)\)
and that the stationary scores of each player is given by \(S_X\) and \(S_Y\)
respectively. The main result of~\cite{Press2012} is that if

\begin{equation}\label{eqn:linear_relationship_for_p}
    \tilde p=\alpha S_x + \beta S_y + \gamma
\end{equation}

or

\begin{equation}\label{eqn:linear_relationship_for_q}
    \tilde q=\alpha S_x + \beta S_y + \gamma
\end{equation}

where \(\tilde p = (1 - p_1, 1 - p_2, p_3, p_4)\) and
\(\tilde q = (1 - q_1, 1 - q_2, q_3, q_4)\) then:

\begin{equation}
    \alpha S_X + \beta S_Y + \gamma = 0
\end{equation}

In~\cite{Press2012} a particular type of ZD strategy is defined: extortionate
strategies. If:

\begin{equation}\label{eqn:constraint_for_extortion}
    \gamma = - P(\alpha + \beta)
\end{equation}

then the player can ensure they get a score \(\chi\) times
larger than the opponent. This extortion coefficient is given by:

\begin{equation}\label{eqn:definition_of_chi}
    \chi=\frac{-\beta}{\alpha}
\end{equation}

Thus, if (\ref{eqn:constraint_for_extortion}) holds and \(\chi >1\) a player is
said to extort their opponent.
Here, the reverse problem is considered: given a
\(p\in\mathbb{R}^4\) how does one identify \(\alpha, \beta\) if they
exist and is the strategy in fact acting in an extortionate way?

These conditions correspond to:

\begin{align}
    \tilde p_1 & = \alpha R + \beta R - P (\alpha + \beta)
            \label{eqn:condition_for_tilde_p1}\\
    \tilde p_2 & = \alpha S + \beta T - P (\alpha + \beta)
            \label{eqn:condition_for_tilde_p2}\\
    \tilde p_3 & = \alpha T + \beta S - P (\alpha + \beta)
            \label{eqn:condition_for_tilde_p3}\\
    \tilde p_4 & = \alpha P + \beta P - P (\alpha + \beta)
            \label{eqn:condition_for_tilde_p4}
\end{align}

Equation (\ref{eqn:condition_for_tilde_p4}) ensures that \(p_4=\tilde p_4=0\).
Equations (\ref{eqn:condition_for_tilde_p1}-\ref{eqn:condition_for_tilde_p3})
can be used to eliminate \(\alpha, \beta\), giving:

\begin{equation}\label{eqn:planar_definition_of_extortion}
    \tilde p_1 = \frac{(R - P)(\tilde p_2 + \tilde p_3)}{S + T - 2P}
\end{equation}

with:

\begin{equation}\label{eqn:definition_of_chi}
    \chi = \frac{\tilde p_2 (P - T) + \tilde p_3 (S - P)}
                {\tilde p_2 (P - S) + \tilde p_3 (T - P)}
\end{equation}

Given a strategy \(p\in\mathbb{R}^{4\times 1}\) equations
(\ref{eqn:condition_for_tilde_p4}), (\ref{eqn:planar_definition_of_extortion}-\ref{eqn:definition_of_chi}) can be used to check if
a strategy is extortionate. The conditions correspond to:

\begin{align}
    p_1 & = \frac{(R-P)(p_2 + p_3) - R + T + S - P}{S + T - 2P}
     \label{eqn:condition_for_p1}\\
    p_4 & = 0 \label{eqn:condition_for_p4}\\
    1 & > p_2 + p_3\label{eqn:condition_for_chi}
\end{align}

The algebraic steps necessary to prove these results are available in the
supporting materials.

All extortionate strategies reside on a triangular (\ref{eqn:condition_for_chi})
plane (\ref{eqn:condition_for_p1}) in 3 dimensions (\ref{eqn:condition_for_p4}).
Using this formulation it can be seen that a necessary (but not sufficient)
condition for an extortionate strategy is that it cooperates on average less
than 50\% of the time when in a state of disagreement with the opponent.

As an example, consider the known extortionate strategy \(p=(8 / 9, 1 / 2, 1 /
3, 0)\) from~\cite{Stewart2012} which is referred to as \texttt{Extort-2}. In
this case, for the standard values of \((R, T, S, P)\) constraint
(\ref{eqn:condition_for_p1}) corresponds to:

\begin{equation}
    p_1 = \frac{2(p_2 + p_3) + 1}{3}
\end{equation}

It is clear that in this case all constraints hold.

This approach could in fact be used to confirm that a given strategy is acting
in an extortionate manner even if it is not a memory one strategy. However, in
practice, if a closed form for \(p\) is not known, then due to measurement
and/or numerical error this would not work.

This problem can be written in the following linear algebraic form where
\(x=(\alpha, \beta)\)
and \(p^*=(\tilde p_1 - 1, tilde_2 - 1, p_3)\):

\begin{equation}\label{eqn:linear_algebraic_equation_for_p}
    Cx= p^*
\end{equation}

\(C\) corresponds to equations
(\ref{eqn:condition_for_tilde_p1}-\ref{eqn:condition_for_tilde_p3}) and is
given by:

\begin{equation}\label{eqn:definition_of_C}
    C =
    \begin{bmatrix}
        R - P & R- P \\
        S - P & T- P \\
        T - P & S- P \\
    \end{bmatrix}
\end{equation}

Note that in general, equation (\ref{eqn:linear_algebraic_equation_for_p}) will
not necessarily have a solution. From the Rouch\'{e}-Capelli theorem if there is
a solution it is unique as \(\text{rank}(C)=2\) which is the dimension of the
variable \(x\). The best fitting \(x\) is found by minimizing:

\begin{equation}\label{eqn:r_squared}
    \text{SSError} = \|C x- p^*\|_2^2 = \sum_{i=1}^{3}\left((C\bar x)_i-p_i^*\right)^2
\end{equation}

Note that \(\text{SSError}\), which is the square of the Frobenius
norm~\cite{Golub2013}, becomes a measure of how close a strategy is to being an
extortionate strategy. Suspicion
of extortion then corresponds to a threshold on \(\text{SSError}\).

By observing interactions (human or otherwise), their memory one representation
can be inferred and this approach can be used to recognise extortionate
behaviour. The notion of comparing theoretic and actual plays of the IPD is not
novel, see for example~\cite{Rand2013}. Immediately it is noted that if the
environment is noisy~\cite{Wu1995} then no strategy can be considered to be
extortionate as \(p_4>0\).

In the next section, this idea will be illustrated by observing the interactions
that take place in a computer based tournament of the IPD\@.

\section{Numerical experiments}\label{sec:numerical-experiments}

In~\cite{Stewart2012} results from a tournament with
\input{./assets/tex/number_of_stewart_plotkin_strategies/main.tex} strategies,
was presented with specific consideration given to ZD strategies. This
tournament is reproduced here using the Axelrod-Python
project~\cite{Knight2016}. To obtain a good measure of the corresponding
transition rates for each strategy all matches have been run for
\input{assets/tex/number_of_turns/main.tex} turns and every match has been
repeated \input{assets/tex/number_of_repetitions/main.tex} times. All of this
interaction data is available at~\cite{vincent_knight_2018_1297075}. A good
match between the inferred Markov chain and the state distribution of the actual
interactions has been verified. Data for this is presented in the supplementary
materials.

Figure~\ref{fig:SSError_overall_in_stewart_plotkin} shows the \(\text{SSError}\)
values for all the strategies in the tournament, as reported
in~\cite{Stewart2012} the extortionate strategy (which has an expected
\(\text{SSError}\) approximately 0) gains a large number of wins.

\begin{figure}[!htbp]
    \centering
    \includegraphics[width=.8\textwidth]{./assets/img/SSError_overall_in_stewart_plotkin/main.pdf}
    \caption{\(\text{SSError}\) and state probabilities for the strategies
        of~\cite{Stewart2012}, ordered both by number of wins and overall score.
        Note that \(P(DC)\) is not shown as it corresponds to the transpose of
        \(P(CD)\). Cooperator and Defector are omitted as they do not visit all
        the states.}
    \label{fig:SSError_overall_in_stewart_plotkin}
\end{figure}

Here, the work of~\cite{Stewart2012} is extended by investigating a tournament
with \input{assets/tex/number_of_full_strategies/main.tex}
strategies.

The results of this analysis are shown in
Figure~\ref{fig:SSError_and_probabilities_in_full}. The top ranking strategies
by number of wins seem to be extortionate (but not against all strategies) and
it can be seen that a small sub group of strategies achieve mutual defection.
All the top ranking strategies according to score achieve mutual cooperation and
do not extort each other, however they
\textbf{do} exhibit extortionate behaviour towards a number of the lower ranking
strategies.

\begin{figure}[!htbp]
    \centering
    \includegraphics[width=.8\textwidth]{./assets/img/SSError_and_probabilities_in_full/main.pdf}
    \caption{\(\text{SSError}\) for the strategies for the full tournament. Only
    strategy interactions for which \(p_4=0\) and \(\chi>1\) are displayed.}
    \label{fig:SSError_and_probabilities_in_full}
\end{figure}

\section{Conclusion}\label{sec:conclusion}

This work defines an approach to measure whether or not a player is playing a
strategy that corresponds to an extortionate strategy as defined
in~\cite{Press2012}: a mathematical model for suspicion. Indeed, all
extortionate strategies have been
 classified as lying on a triangular plane.
This rigorous classification fails to be robust to small measurement error, thus
a statistical approach is proposed.
This is done through a linear algebraic approach for approximating the solution
of a linear system. Using this, a large number of pairwise interactions is
simulated and in fact very few strategies are found to act extortionately.

The work of~\cite{Press2012}, whilst showing that a clever approach to taking
advantage of another memory one strategy exists: this is incomplete. Whilst the
elegance of this result is very attractive, just as the simplicity of the
victory of Tit For Tat in Axelrod's original tournaments was, it is incomplete.
Extortionate strategies achieve a high number of wins but they do not
achieve a high score which corresponds to the fitness landscape in an
evolutionary sense. From the large number of interactions a payoff matrix \(S\)
can be measured where \(S_{ij}\) denotes the score (using standard values of
\((R, S, T, P) = (3, 0, 5, 1)\)) of the \(i\)th strategy
against the \(j\)th strategy. Using this, the replicator equation
describes the evolution of the system based on a population density fitness
function:

\begin{equation}\label{eqn:replicator_dynamics}
    \frac{dx}{dt} = x(S-x^TS x)
\end{equation}

Equation (\ref{eqn:replicator_dynamics}) is solved numerically through an
integration technique described in~\cite{Petzold1983} and
Figure~\ref{fig:replicator_dynamics} shows the evolution of the distribution of
the system: the various strategies are ranked by scores. It is clear to see that
only the high ranking strategies survive the evolutionary process (in fact,
only \input{./assets/img/replicator_dynamics/main.tex}
have a final distribution greater than \(10 ^ {-2}\)). This confirms the
findings of~\cite{Moran1707} in which sophisticated strategies resist
evolutionary invasion of shorter memory strategies. Recalling
Figure~\ref{fig:SSError_and_probabilities_in_full} this demonstrates that:

\begin{itemize}
    \item Cooperation emerges through the evolutionary process: the high scoring
        strategies do not exhibit extortionate behaviour towards each other.
    \item Extortionate strategies do not survive the evolutionary process.
\end{itemize}

\begin{figure}[!htbp]
    \centering
    \includegraphics[width=.8\textwidth]{./assets/img/replicator_dynamics/main.pdf}
    \caption{Numerical simulation of the replicator equation
    (\ref{eqn:replicator_dynamics}): strategies are ordered by score, only the strategies with a high score survive the evolutionary process.}
    \label{fig:replicator_dynamics}
\end{figure}

This work can be used to classify plays of the IPD\@: data can be collected from
actual interactions (in lab or in the field). Furthermore, this allows for a
classification method similar to the notion of fingerprinting presented
in~\cite{Ashlock2008}. Trained strategies can potentially be classified as
extortionate or not or it could be possible to even constrain the reinforcement
learning approaches that are becoming prevalent in the literature.
Alternatively, this mathematical approach for recognising extortion could be
used in sophisticated strategies to defend against invasion. Arguably, some of
the strategies considered here exhibit this behaviour, indeed as described
in~\cite{Harper2017}, the top ranking strategies in the full tournament are
obtained using evolutionary reinforcement learning techniques, thus, suspicion
of extortionate behaviour could in fact be an evolutionary trait.

\section*{Acknowledgements}

The following open source software libraries were used in this research:

\begin{itemize}
    \item The Axelrod ~\cite{Knight2016, Knight2018} library (IPD strategies and
        tournaments).
    \item The sympy library~\cite{Meurer2017} (verification of all symbolic
        calculations).
    \item The matplotlib~\cite{Droettboom2018} library (visualisation).
    \item The pandas~\cite{Structures2010}, dask~\cite{Dask2016} and
        NumPy~\cite{Oliphant2015} libraries (data manipulation).
    \item The SciPy~\cite{Jones2001} library (numerical integration of the
        replicator equation).
\end{itemize}

This work was performed using the computational facilities of the Advanced
Research Computing @ Cardiff (ARCCA) Division, Cardiff University.

\printbibliography

\newpage
\section*{Supplementary materials}

\includepdf{assets/pdf/proof_of_form_of_extortionate_strategies/main.pdf}

\newpage

Using the pair wise interactions the transition rates \(p,
q\) can be measured and the steady state probabilities inferred and compared to
the actual probabilities of each state.
This is done numerically by computing the singular eigenvector of the
matrix \(A\) \cite{Stewart2009}:

\[
    A =
    \begin{bmatrix}
        p_1 q_1 & p_1 (1 - q_1) & (1 - p_1) q_1 & (1 -p_1) (1 - q_1) \\
        p_2 q_2 & p_2 (1 - q_2) & (1 - p_2) q_2 & (1 -p_2) (1 - q_2) \\
        p_3 q_3 & p_3 (1 - q_3) & (1 - p_3) q_3 & (1 -p_3) (1 - q_3) \\
        p_4 q_4 & p_4 (1 - q_4) & (1 - p_4) q_4 & (1 -p_4) (1 - q_4) \\
    \end{bmatrix}
\]

Figure~\ref{fig:computed_probabilities_vs_theoretic_probabilities} shows a
regression line fitted to every pairwise interaction with a reported
\(\text{SSError}\) value (pairwise interactions with missing states were
omitted). This serves to validate the approach: a part from some edge cases the
relationship is consistent.

\begin{figure}[!htbp]
    \centering
    \includegraphics[width=.8\textwidth]{./assets/img/computed_probabilities_vs_theoretic_probabilities/main.pdf}
    \caption{The
        relationship between the steady state probabilities inferred from the
        measured transitions and the actual steady state probabilities. A linear
        regression line is included validating the approach.}
    \label{fig:computed_probabilities_vs_theoretic_probabilities}
\end{figure}


\end{document}

strategies.

The results of this analysis are shown in
Figure~\ref{fig:SSError_and_probabilities_in_full}. The top ranking strategies
by number of wins seem to be extortionate (but not against all strategies) and
it can be seen that a small sub group of strategies achieve mutual defection.
All the top ranking strategies according to score achieve mutual cooperation and
do not extort each other, however they
\textbf{do} exhibit extortionate behaviour towards a number of the lower ranking
strategies.

\begin{figure}[!htbp]
    \centering
    \includegraphics[width=.8\textwidth]{./assets/img/SSError_and_probabilities_in_full/main.pdf}
    \caption{\(\text{SSError}\) for the strategies for the full tournament. Only
    strategy interactions for which \(p_4=0\) and \(\chi>1\) are displayed.}
    \label{fig:SSError_and_probabilities_in_full}
\end{figure}

\section{Conclusion}\label{sec:conclusion}

This work defines an approach to measure whether or not a player is playing a
strategy that corresponds to an extortionate strategy as defined
in~\cite{Press2012}: a mathematical model for suspicion. Indeed, all
extortionate strategies have been
 classified as lying on a triangular plane.
This rigorous classification fails to be robust to small measurement error, thus
a statistical approach is proposed.
This is done through a linear algebraic approach for approximating the solution
of a linear system. Using this, a large number of pairwise interactions is
simulated and in fact very few strategies are found to act extortionately.

The work of~\cite{Press2012}, whilst showing that a clever approach to taking
advantage of another memory one strategy exists: this is incomplete. Whilst the
elegance of this result is very attractive, just as the simplicity of the
victory of Tit For Tat in Axelrod's original tournaments was, it is incomplete.
Extortionate strategies achieve a high number of wins but they do not
achieve a high score which corresponds to the fitness landscape in an
evolutionary sense. From the large number of interactions a payoff matrix \(S\)
can be measured where \(S_{ij}\) denotes the score (using standard values of
\((R, S, T, P) = (3, 0, 5, 1)\)) of the \(i\)th strategy
against the \(j\)th strategy. Using this, the replicator equation
describes the evolution of the system based on a population density fitness
function:

\begin{equation}\label{eqn:replicator_dynamics}
    \frac{dx}{dt} = x(S-x^TS x)
\end{equation}

Equation (\ref{eqn:replicator_dynamics}) is solved numerically through an
integration technique described in~\cite{Petzold1983} and
Figure~\ref{fig:replicator_dynamics} shows the evolution of the distribution of
the system: the various strategies are ranked by scores. It is clear to see that
only the high ranking strategies survive the evolutionary process (in fact,
only \documentclass[a4paper]{article}

\usepackage{amsmath}
\usepackage{amssymb}
\usepackage[margin=1.5cm,
            includefoot,
            footskip=30pt]{geometry}
\usepackage{layout}
\usepackage{graphicx}
\usepackage{subcaption}

\usepackage{biblatex}
\usepackage{pdfpages}

\bibliography{main.bib}

\title{Suspicion: Recognising and evaluating the effectiveness
       of extortion in the Iterated Prisoner's Dilemma}
\author{Vincent A. Knight \and Nikoleta E. Glynatsi}
\date{\today}



\begin{document}

\maketitle

\begin{abstract}
    The Iterated Prisoner's Dilemma is a model for rational and evolutionary
    interactive behaviour. It has applications both in the study of human social
    behaviour as well as in biology.
    It is used to understand when and how a rational individual might
    accept an immediate cost to their own utility for the direct benefit of
    another.

    Much attention has been given to a class of strategies called
    Zero Determinant strategies. It has been theoretically shown that these
    strategies can ``extort'' any player.

    In this work, an approach to identify if observed strategies are playing in
    an extortionate way is described. Furthermore, experimental analysis of
    a large tournament with \input{assets/tex/number_of_full_strategies/main.tex}
    strategies is considered. In this setting
    the most highly performing strategies do not play in an extortionate way
    against each other but do against lower performing strategies.
    This suggests that whilst the theory of Zero Determinant strategies
    indicates that memory is not of fundamental importance to the evolution of
    cooperative behaviour, this is incomplete.
\end{abstract}

\section{Introduction}\label{sec:introduction}

Agent based game theoretic models have become a stalwart of the underpinning
mathematics of interactive behaviours. One of the major pieces of work
in this area is the pair of original computer tournaments run by Robert
Axelrod~\cite{Axelrod1980, Axelrod1980a}. These tournaments pitted submitted
computer strategies against each other in plays of the Iterated Prisoner's
Dilemma. A common game where agents can choose to pay a slight cost to their
immediate utility in the hope of building a reputation. This has been used in
economic and evolutionary game theory to understand the evolution of cooperative
behaviour.

Recently, a class of strategies was described in~\cite{Press2012} that can
provably extort any given opponent. In~\cite{Hilbe2013, Moran1707} some
questions have already been asked about the true effectiveness of these
strategies in an evolutionary setting. Here another question is asked: is it
possible to recognise this extortionate behaviour? A mathematical procedure for
suspicion is presented: in the same way that the continued actions of an
extortionate individual might raise suspicion.

This work makes use of the Axelrod Python library~\cite{Knight2018, Knight2016}
with a large number of Prisoner Dilemma strategies available to give an
extensive numerical example of the ideas presented.  The approach is presented
in Section~\ref{sec:delta-zd-strategies}.  All of the code and data discussed
in Section~\ref{sec:numerical-experiments} is open sourced, archived and
written according to best scientific principles~\cite{Wilson2014}. The data
archive can be found at~\cite{vincent_knight_2018_1297075}.

\section{Recognising Extortion}\label{sec:delta-zd-strategies}

In~\cite{Press2012}, given a match between 2 memory-one strategies, the concept
of Zero Determinant (ZD) strategies is introduced. The main result of that paper
shows that given two memory one players \(p, q\in\mathbb{R}^4\) a linear
relationship between the players' scores could be forced by one of the players.

Using the notation of~\cite{Press2012}, assuming the utilities for player \(p\)
are given by \(S_x=(R, S, T, P)\) and for player \(q\) by \(S_y=(R, T, S, P)\)
and that the stationary scores of each player is given by \(S_X\) and \(S_Y\)
respectively. The main result of~\cite{Press2012} is that if

\begin{equation}\label{eqn:linear_relationship_for_p}
    \tilde p=\alpha S_x + \beta S_y + \gamma
\end{equation}

or

\begin{equation}\label{eqn:linear_relationship_for_q}
    \tilde q=\alpha S_x + \beta S_y + \gamma
\end{equation}

where \(\tilde p = (1 - p_1, 1 - p_2, p_3, p_4)\) and
\(\tilde q = (1 - q_1, 1 - q_2, q_3, q_4)\) then:

\begin{equation}
    \alpha S_X + \beta S_Y + \gamma = 0
\end{equation}

In~\cite{Press2012} a particular type of ZD strategy is defined: extortionate
strategies. If:

\begin{equation}\label{eqn:constraint_for_extortion}
    \gamma = - P(\alpha + \beta)
\end{equation}

then the player can ensure they get a score \(\chi\) times
larger than the opponent. This extortion coefficient is given by:

\begin{equation}\label{eqn:definition_of_chi}
    \chi=\frac{-\beta}{\alpha}
\end{equation}

Thus, if (\ref{eqn:constraint_for_extortion}) holds and \(\chi >1\) a player is
said to extort their opponent.
Here, the reverse problem is considered: given a
\(p\in\mathbb{R}^4\) how does one identify \(\alpha, \beta\) if they
exist and is the strategy in fact acting in an extortionate way?

These conditions correspond to:

\begin{align}
    \tilde p_1 & = \alpha R + \beta R - P (\alpha + \beta)
            \label{eqn:condition_for_tilde_p1}\\
    \tilde p_2 & = \alpha S + \beta T - P (\alpha + \beta)
            \label{eqn:condition_for_tilde_p2}\\
    \tilde p_3 & = \alpha T + \beta S - P (\alpha + \beta)
            \label{eqn:condition_for_tilde_p3}\\
    \tilde p_4 & = \alpha P + \beta P - P (\alpha + \beta)
            \label{eqn:condition_for_tilde_p4}
\end{align}

Equation (\ref{eqn:condition_for_tilde_p4}) ensures that \(p_4=\tilde p_4=0\).
Equations (\ref{eqn:condition_for_tilde_p1}-\ref{eqn:condition_for_tilde_p3})
can be used to eliminate \(\alpha, \beta\), giving:

\begin{equation}\label{eqn:planar_definition_of_extortion}
    \tilde p_1 = \frac{(R - P)(\tilde p_2 + \tilde p_3)}{S + T - 2P}
\end{equation}

with:

\begin{equation}\label{eqn:definition_of_chi}
    \chi = \frac{\tilde p_2 (P - T) + \tilde p_3 (S - P)}
                {\tilde p_2 (P - S) + \tilde p_3 (T - P)}
\end{equation}

Given a strategy \(p\in\mathbb{R}^{4\times 1}\) equations
(\ref{eqn:condition_for_tilde_p4}), (\ref{eqn:planar_definition_of_extortion}-\ref{eqn:definition_of_chi}) can be used to check if
a strategy is extortionate. The conditions correspond to:

\begin{align}
    p_1 & = \frac{(R-P)(p_2 + p_3) - R + T + S - P}{S + T - 2P}
     \label{eqn:condition_for_p1}\\
    p_4 & = 0 \label{eqn:condition_for_p4}\\
    1 & > p_2 + p_3\label{eqn:condition_for_chi}
\end{align}

The algebraic steps necessary to prove these results are available in the
supporting materials.

All extortionate strategies reside on a triangular (\ref{eqn:condition_for_chi})
plane (\ref{eqn:condition_for_p1}) in 3 dimensions (\ref{eqn:condition_for_p4}).
Using this formulation it can be seen that a necessary (but not sufficient)
condition for an extortionate strategy is that it cooperates on average less
than 50\% of the time when in a state of disagreement with the opponent.

As an example, consider the known extortionate strategy \(p=(8 / 9, 1 / 2, 1 /
3, 0)\) from~\cite{Stewart2012} which is referred to as \texttt{Extort-2}. In
this case, for the standard values of \((R, T, S, P)\) constraint
(\ref{eqn:condition_for_p1}) corresponds to:

\begin{equation}
    p_1 = \frac{2(p_2 + p_3) + 1}{3}
\end{equation}

It is clear that in this case all constraints hold.

This approach could in fact be used to confirm that a given strategy is acting
in an extortionate manner even if it is not a memory one strategy. However, in
practice, if a closed form for \(p\) is not known, then due to measurement
and/or numerical error this would not work.

This problem can be written in the following linear algebraic form where
\(x=(\alpha, \beta)\)
and \(p^*=(\tilde p_1 - 1, tilde_2 - 1, p_3)\):

\begin{equation}\label{eqn:linear_algebraic_equation_for_p}
    Cx= p^*
\end{equation}

\(C\) corresponds to equations
(\ref{eqn:condition_for_tilde_p1}-\ref{eqn:condition_for_tilde_p3}) and is
given by:

\begin{equation}\label{eqn:definition_of_C}
    C =
    \begin{bmatrix}
        R - P & R- P \\
        S - P & T- P \\
        T - P & S- P \\
    \end{bmatrix}
\end{equation}

Note that in general, equation (\ref{eqn:linear_algebraic_equation_for_p}) will
not necessarily have a solution. From the Rouch\'{e}-Capelli theorem if there is
a solution it is unique as \(\text{rank}(C)=2\) which is the dimension of the
variable \(x\). The best fitting \(x\) is found by minimizing:

\begin{equation}\label{eqn:r_squared}
    \text{SSError} = \|C x- p^*\|_2^2 = \sum_{i=1}^{3}\left((C\bar x)_i-p_i^*\right)^2
\end{equation}

Note that \(\text{SSError}\), which is the square of the Frobenius
norm~\cite{Golub2013}, becomes a measure of how close a strategy is to being an
extortionate strategy. Suspicion
of extortion then corresponds to a threshold on \(\text{SSError}\).

By observing interactions (human or otherwise), their memory one representation
can be inferred and this approach can be used to recognise extortionate
behaviour. The notion of comparing theoretic and actual plays of the IPD is not
novel, see for example~\cite{Rand2013}. Immediately it is noted that if the
environment is noisy~\cite{Wu1995} then no strategy can be considered to be
extortionate as \(p_4>0\).

In the next section, this idea will be illustrated by observing the interactions
that take place in a computer based tournament of the IPD\@.

\section{Numerical experiments}\label{sec:numerical-experiments}

In~\cite{Stewart2012} results from a tournament with
\input{./assets/tex/number_of_stewart_plotkin_strategies/main.tex} strategies,
was presented with specific consideration given to ZD strategies. This
tournament is reproduced here using the Axelrod-Python
project~\cite{Knight2016}. To obtain a good measure of the corresponding
transition rates for each strategy all matches have been run for
\input{assets/tex/number_of_turns/main.tex} turns and every match has been
repeated \input{assets/tex/number_of_repetitions/main.tex} times. All of this
interaction data is available at~\cite{vincent_knight_2018_1297075}. A good
match between the inferred Markov chain and the state distribution of the actual
interactions has been verified. Data for this is presented in the supplementary
materials.

Figure~\ref{fig:SSError_overall_in_stewart_plotkin} shows the \(\text{SSError}\)
values for all the strategies in the tournament, as reported
in~\cite{Stewart2012} the extortionate strategy (which has an expected
\(\text{SSError}\) approximately 0) gains a large number of wins.

\begin{figure}[!htbp]
    \centering
    \includegraphics[width=.8\textwidth]{./assets/img/SSError_overall_in_stewart_plotkin/main.pdf}
    \caption{\(\text{SSError}\) and state probabilities for the strategies
        of~\cite{Stewart2012}, ordered both by number of wins and overall score.
        Note that \(P(DC)\) is not shown as it corresponds to the transpose of
        \(P(CD)\). Cooperator and Defector are omitted as they do not visit all
        the states.}
    \label{fig:SSError_overall_in_stewart_plotkin}
\end{figure}

Here, the work of~\cite{Stewart2012} is extended by investigating a tournament
with \input{assets/tex/number_of_full_strategies/main.tex}
strategies.

The results of this analysis are shown in
Figure~\ref{fig:SSError_and_probabilities_in_full}. The top ranking strategies
by number of wins seem to be extortionate (but not against all strategies) and
it can be seen that a small sub group of strategies achieve mutual defection.
All the top ranking strategies according to score achieve mutual cooperation and
do not extort each other, however they
\textbf{do} exhibit extortionate behaviour towards a number of the lower ranking
strategies.

\begin{figure}[!htbp]
    \centering
    \includegraphics[width=.8\textwidth]{./assets/img/SSError_and_probabilities_in_full/main.pdf}
    \caption{\(\text{SSError}\) for the strategies for the full tournament. Only
    strategy interactions for which \(p_4=0\) and \(\chi>1\) are displayed.}
    \label{fig:SSError_and_probabilities_in_full}
\end{figure}

\section{Conclusion}\label{sec:conclusion}

This work defines an approach to measure whether or not a player is playing a
strategy that corresponds to an extortionate strategy as defined
in~\cite{Press2012}: a mathematical model for suspicion. Indeed, all
extortionate strategies have been
 classified as lying on a triangular plane.
This rigorous classification fails to be robust to small measurement error, thus
a statistical approach is proposed.
This is done through a linear algebraic approach for approximating the solution
of a linear system. Using this, a large number of pairwise interactions is
simulated and in fact very few strategies are found to act extortionately.

The work of~\cite{Press2012}, whilst showing that a clever approach to taking
advantage of another memory one strategy exists: this is incomplete. Whilst the
elegance of this result is very attractive, just as the simplicity of the
victory of Tit For Tat in Axelrod's original tournaments was, it is incomplete.
Extortionate strategies achieve a high number of wins but they do not
achieve a high score which corresponds to the fitness landscape in an
evolutionary sense. From the large number of interactions a payoff matrix \(S\)
can be measured where \(S_{ij}\) denotes the score (using standard values of
\((R, S, T, P) = (3, 0, 5, 1)\)) of the \(i\)th strategy
against the \(j\)th strategy. Using this, the replicator equation
describes the evolution of the system based on a population density fitness
function:

\begin{equation}\label{eqn:replicator_dynamics}
    \frac{dx}{dt} = x(S-x^TS x)
\end{equation}

Equation (\ref{eqn:replicator_dynamics}) is solved numerically through an
integration technique described in~\cite{Petzold1983} and
Figure~\ref{fig:replicator_dynamics} shows the evolution of the distribution of
the system: the various strategies are ranked by scores. It is clear to see that
only the high ranking strategies survive the evolutionary process (in fact,
only \input{./assets/img/replicator_dynamics/main.tex}
have a final distribution greater than \(10 ^ {-2}\)). This confirms the
findings of~\cite{Moran1707} in which sophisticated strategies resist
evolutionary invasion of shorter memory strategies. Recalling
Figure~\ref{fig:SSError_and_probabilities_in_full} this demonstrates that:

\begin{itemize}
    \item Cooperation emerges through the evolutionary process: the high scoring
        strategies do not exhibit extortionate behaviour towards each other.
    \item Extortionate strategies do not survive the evolutionary process.
\end{itemize}

\begin{figure}[!htbp]
    \centering
    \includegraphics[width=.8\textwidth]{./assets/img/replicator_dynamics/main.pdf}
    \caption{Numerical simulation of the replicator equation
    (\ref{eqn:replicator_dynamics}): strategies are ordered by score, only the strategies with a high score survive the evolutionary process.}
    \label{fig:replicator_dynamics}
\end{figure}

This work can be used to classify plays of the IPD\@: data can be collected from
actual interactions (in lab or in the field). Furthermore, this allows for a
classification method similar to the notion of fingerprinting presented
in~\cite{Ashlock2008}. Trained strategies can potentially be classified as
extortionate or not or it could be possible to even constrain the reinforcement
learning approaches that are becoming prevalent in the literature.
Alternatively, this mathematical approach for recognising extortion could be
used in sophisticated strategies to defend against invasion. Arguably, some of
the strategies considered here exhibit this behaviour, indeed as described
in~\cite{Harper2017}, the top ranking strategies in the full tournament are
obtained using evolutionary reinforcement learning techniques, thus, suspicion
of extortionate behaviour could in fact be an evolutionary trait.

\section*{Acknowledgements}

The following open source software libraries were used in this research:

\begin{itemize}
    \item The Axelrod ~\cite{Knight2016, Knight2018} library (IPD strategies and
        tournaments).
    \item The sympy library~\cite{Meurer2017} (verification of all symbolic
        calculations).
    \item The matplotlib~\cite{Droettboom2018} library (visualisation).
    \item The pandas~\cite{Structures2010}, dask~\cite{Dask2016} and
        NumPy~\cite{Oliphant2015} libraries (data manipulation).
    \item The SciPy~\cite{Jones2001} library (numerical integration of the
        replicator equation).
\end{itemize}

This work was performed using the computational facilities of the Advanced
Research Computing @ Cardiff (ARCCA) Division, Cardiff University.

\printbibliography

\newpage
\section*{Supplementary materials}

\includepdf{assets/pdf/proof_of_form_of_extortionate_strategies/main.pdf}

\newpage

Using the pair wise interactions the transition rates \(p,
q\) can be measured and the steady state probabilities inferred and compared to
the actual probabilities of each state.
This is done numerically by computing the singular eigenvector of the
matrix \(A\) \cite{Stewart2009}:

\[
    A =
    \begin{bmatrix}
        p_1 q_1 & p_1 (1 - q_1) & (1 - p_1) q_1 & (1 -p_1) (1 - q_1) \\
        p_2 q_2 & p_2 (1 - q_2) & (1 - p_2) q_2 & (1 -p_2) (1 - q_2) \\
        p_3 q_3 & p_3 (1 - q_3) & (1 - p_3) q_3 & (1 -p_3) (1 - q_3) \\
        p_4 q_4 & p_4 (1 - q_4) & (1 - p_4) q_4 & (1 -p_4) (1 - q_4) \\
    \end{bmatrix}
\]

Figure~\ref{fig:computed_probabilities_vs_theoretic_probabilities} shows a
regression line fitted to every pairwise interaction with a reported
\(\text{SSError}\) value (pairwise interactions with missing states were
omitted). This serves to validate the approach: a part from some edge cases the
relationship is consistent.

\begin{figure}[!htbp]
    \centering
    \includegraphics[width=.8\textwidth]{./assets/img/computed_probabilities_vs_theoretic_probabilities/main.pdf}
    \caption{The
        relationship between the steady state probabilities inferred from the
        measured transitions and the actual steady state probabilities. A linear
        regression line is included validating the approach.}
    \label{fig:computed_probabilities_vs_theoretic_probabilities}
\end{figure}


\end{document}

have a final distribution greater than \(10 ^ {-2}\)). This confirms the
findings of~\cite{Moran1707} in which sophisticated strategies resist
evolutionary invasion of shorter memory strategies. Recalling
Figure~\ref{fig:SSError_and_probabilities_in_full} this demonstrates that:

\begin{itemize}
    \item Cooperation emerges through the evolutionary process: the high scoring
        strategies do not exhibit extortionate behaviour towards each other.
    \item Extortionate strategies do not survive the evolutionary process.
\end{itemize}

\begin{figure}[!htbp]
    \centering
    \includegraphics[width=.8\textwidth]{./assets/img/replicator_dynamics/main.pdf}
    \caption{Numerical simulation of the replicator equation
    (\ref{eqn:replicator_dynamics}): strategies are ordered by score, only the strategies with a high score survive the evolutionary process.}
    \label{fig:replicator_dynamics}
\end{figure}

This work can be used to classify plays of the IPD\@: data can be collected from
actual interactions (in lab or in the field). Furthermore, this allows for a
classification method similar to the notion of fingerprinting presented
in~\cite{Ashlock2008}. Trained strategies can potentially be classified as
extortionate or not or it could be possible to even constrain the reinforcement
learning approaches that are becoming prevalent in the literature.
Alternatively, this mathematical approach for recognising extortion could be
used in sophisticated strategies to defend against invasion. Arguably, some of
the strategies considered here exhibit this behaviour, indeed as described
in~\cite{Harper2017}, the top ranking strategies in the full tournament are
obtained using evolutionary reinforcement learning techniques, thus, suspicion
of extortionate behaviour could in fact be an evolutionary trait.

\section*{Acknowledgements}

The following open source software libraries were used in this research:

\begin{itemize}
    \item The Axelrod ~\cite{Knight2016, Knight2018} library (IPD strategies and
        tournaments).
    \item The sympy library~\cite{Meurer2017} (verification of all symbolic
        calculations).
    \item The matplotlib~\cite{Droettboom2018} library (visualisation).
    \item The pandas~\cite{Structures2010}, dask~\cite{Dask2016} and
        NumPy~\cite{Oliphant2015} libraries (data manipulation).
    \item The SciPy~\cite{Jones2001} library (numerical integration of the
        replicator equation).
\end{itemize}

This work was performed using the computational facilities of the Advanced
Research Computing @ Cardiff (ARCCA) Division, Cardiff University.

\printbibliography

\newpage
\section*{Supplementary materials}

\includepdf{assets/pdf/proof_of_form_of_extortionate_strategies/main.pdf}

\newpage

Using the pair wise interactions the transition rates \(p,
q\) can be measured and the steady state probabilities inferred and compared to
the actual probabilities of each state.
This is done numerically by computing the singular eigenvector of the
matrix \(A\) \cite{Stewart2009}:

\[
    A =
    \begin{bmatrix}
        p_1 q_1 & p_1 (1 - q_1) & (1 - p_1) q_1 & (1 -p_1) (1 - q_1) \\
        p_2 q_2 & p_2 (1 - q_2) & (1 - p_2) q_2 & (1 -p_2) (1 - q_2) \\
        p_3 q_3 & p_3 (1 - q_3) & (1 - p_3) q_3 & (1 -p_3) (1 - q_3) \\
        p_4 q_4 & p_4 (1 - q_4) & (1 - p_4) q_4 & (1 -p_4) (1 - q_4) \\
    \end{bmatrix}
\]

Figure~\ref{fig:computed_probabilities_vs_theoretic_probabilities} shows a
regression line fitted to every pairwise interaction with a reported
\(\text{SSError}\) value (pairwise interactions with missing states were
omitted). This serves to validate the approach: a part from some edge cases the
relationship is consistent.

\begin{figure}[!htbp]
    \centering
    \includegraphics[width=.8\textwidth]{./assets/img/computed_probabilities_vs_theoretic_probabilities/main.pdf}
    \caption{The
        relationship between the steady state probabilities inferred from the
        measured transitions and the actual steady state probabilities. A linear
        regression line is included validating the approach.}
    \label{fig:computed_probabilities_vs_theoretic_probabilities}
\end{figure}


\end{document}

strategies.

The results of this analysis are shown in
Figure~\ref{fig:SSError_and_probabilities_in_full}. The top ranking strategies
by number of wins seem to be extortionate (but not against all strategies) and
it can be seen that a small sub group of strategies achieve mutual defection.
All the top ranking strategies according to score achieve mutual cooperation and
do not extort each other, however they
\textbf{do} exhibit extortionate behaviour towards a number of the lower ranking
strategies.

\begin{figure}[!htbp]
    \centering
    \includegraphics[width=.8\textwidth]{./assets/img/SSError_and_probabilities_in_full/main.pdf}
    \caption{\(\text{SSError}\) for the strategies for the full tournament. Only
    strategy interactions for which \(p_4=0\) and \(\chi>1\) are displayed.}
    \label{fig:SSError_and_probabilities_in_full}
\end{figure}

\section{Conclusion}\label{sec:conclusion}

This work defines an approach to measure whether or not a player is playing a
strategy that corresponds to an extortionate strategy as defined
in~\cite{Press2012}: a mathematical model for suspicion. Indeed, all
extortionate strategies have been
 classified as lying on a triangular plane.
This rigorous classification fails to be robust to small measurement error, thus
a statistical approach is proposed.
This is done through a linear algebraic approach for approximating the solution
of a linear system. Using this, a large number of pairwise interactions is
simulated and in fact very few strategies are found to act extortionately.

The work of~\cite{Press2012}, whilst showing that a clever approach to taking
advantage of another memory one strategy exists: this is incomplete. Whilst the
elegance of this result is very attractive, just as the simplicity of the
victory of Tit For Tat in Axelrod's original tournaments was, it is incomplete.
Extortionate strategies achieve a high number of wins but they do not
achieve a high score which corresponds to the fitness landscape in an
evolutionary sense. From the large number of interactions a payoff matrix \(S\)
can be measured where \(S_{ij}\) denotes the score (using standard values of
\((R, S, T, P) = (3, 0, 5, 1)\)) of the \(i\)th strategy
against the \(j\)th strategy. Using this, the replicator equation
describes the evolution of the system based on a population density fitness
function:

\begin{equation}\label{eqn:replicator_dynamics}
    \frac{dx}{dt} = x(S-x^TS x)
\end{equation}

Equation (\ref{eqn:replicator_dynamics}) is solved numerically through an
integration technique described in~\cite{Petzold1983} and
Figure~\ref{fig:replicator_dynamics} shows the evolution of the distribution of
the system: the various strategies are ranked by scores. It is clear to see that
only the high ranking strategies survive the evolutionary process (in fact,
only \documentclass[a4paper]{article}

\usepackage{amsmath}
\usepackage{amssymb}
\usepackage[margin=1.5cm,
            includefoot,
            footskip=30pt]{geometry}
\usepackage{layout}
\usepackage{graphicx}
\usepackage{subcaption}

\usepackage{biblatex}
\usepackage{pdfpages}

\bibliography{main.bib}

\title{Suspicion: Recognising and evaluating the effectiveness
       of extortion in the Iterated Prisoner's Dilemma}
\author{Vincent A. Knight \and Nikoleta E. Glynatsi}
\date{\today}



\begin{document}

\maketitle

\begin{abstract}
    The Iterated Prisoner's Dilemma is a model for rational and evolutionary
    interactive behaviour. It has applications both in the study of human social
    behaviour as well as in biology.
    It is used to understand when and how a rational individual might
    accept an immediate cost to their own utility for the direct benefit of
    another.

    Much attention has been given to a class of strategies called
    Zero Determinant strategies. It has been theoretically shown that these
    strategies can ``extort'' any player.

    In this work, an approach to identify if observed strategies are playing in
    an extortionate way is described. Furthermore, experimental analysis of
    a large tournament with \documentclass[a4paper]{article}

\usepackage{amsmath}
\usepackage{amssymb}
\usepackage[margin=1.5cm,
            includefoot,
            footskip=30pt]{geometry}
\usepackage{layout}
\usepackage{graphicx}
\usepackage{subcaption}

\usepackage{biblatex}
\usepackage{pdfpages}

\bibliography{main.bib}

\title{Suspicion: Recognising and evaluating the effectiveness
       of extortion in the Iterated Prisoner's Dilemma}
\author{Vincent A. Knight \and Nikoleta E. Glynatsi}
\date{\today}



\begin{document}

\maketitle

\begin{abstract}
    The Iterated Prisoner's Dilemma is a model for rational and evolutionary
    interactive behaviour. It has applications both in the study of human social
    behaviour as well as in biology.
    It is used to understand when and how a rational individual might
    accept an immediate cost to their own utility for the direct benefit of
    another.

    Much attention has been given to a class of strategies called
    Zero Determinant strategies. It has been theoretically shown that these
    strategies can ``extort'' any player.

    In this work, an approach to identify if observed strategies are playing in
    an extortionate way is described. Furthermore, experimental analysis of
    a large tournament with \input{assets/tex/number_of_full_strategies/main.tex}
    strategies is considered. In this setting
    the most highly performing strategies do not play in an extortionate way
    against each other but do against lower performing strategies.
    This suggests that whilst the theory of Zero Determinant strategies
    indicates that memory is not of fundamental importance to the evolution of
    cooperative behaviour, this is incomplete.
\end{abstract}

\section{Introduction}\label{sec:introduction}

Agent based game theoretic models have become a stalwart of the underpinning
mathematics of interactive behaviours. One of the major pieces of work
in this area is the pair of original computer tournaments run by Robert
Axelrod~\cite{Axelrod1980, Axelrod1980a}. These tournaments pitted submitted
computer strategies against each other in plays of the Iterated Prisoner's
Dilemma. A common game where agents can choose to pay a slight cost to their
immediate utility in the hope of building a reputation. This has been used in
economic and evolutionary game theory to understand the evolution of cooperative
behaviour.

Recently, a class of strategies was described in~\cite{Press2012} that can
provably extort any given opponent. In~\cite{Hilbe2013, Moran1707} some
questions have already been asked about the true effectiveness of these
strategies in an evolutionary setting. Here another question is asked: is it
possible to recognise this extortionate behaviour? A mathematical procedure for
suspicion is presented: in the same way that the continued actions of an
extortionate individual might raise suspicion.

This work makes use of the Axelrod Python library~\cite{Knight2018, Knight2016}
with a large number of Prisoner Dilemma strategies available to give an
extensive numerical example of the ideas presented.  The approach is presented
in Section~\ref{sec:delta-zd-strategies}.  All of the code and data discussed
in Section~\ref{sec:numerical-experiments} is open sourced, archived and
written according to best scientific principles~\cite{Wilson2014}. The data
archive can be found at~\cite{vincent_knight_2018_1297075}.

\section{Recognising Extortion}\label{sec:delta-zd-strategies}

In~\cite{Press2012}, given a match between 2 memory-one strategies, the concept
of Zero Determinant (ZD) strategies is introduced. The main result of that paper
shows that given two memory one players \(p, q\in\mathbb{R}^4\) a linear
relationship between the players' scores could be forced by one of the players.

Using the notation of~\cite{Press2012}, assuming the utilities for player \(p\)
are given by \(S_x=(R, S, T, P)\) and for player \(q\) by \(S_y=(R, T, S, P)\)
and that the stationary scores of each player is given by \(S_X\) and \(S_Y\)
respectively. The main result of~\cite{Press2012} is that if

\begin{equation}\label{eqn:linear_relationship_for_p}
    \tilde p=\alpha S_x + \beta S_y + \gamma
\end{equation}

or

\begin{equation}\label{eqn:linear_relationship_for_q}
    \tilde q=\alpha S_x + \beta S_y + \gamma
\end{equation}

where \(\tilde p = (1 - p_1, 1 - p_2, p_3, p_4)\) and
\(\tilde q = (1 - q_1, 1 - q_2, q_3, q_4)\) then:

\begin{equation}
    \alpha S_X + \beta S_Y + \gamma = 0
\end{equation}

In~\cite{Press2012} a particular type of ZD strategy is defined: extortionate
strategies. If:

\begin{equation}\label{eqn:constraint_for_extortion}
    \gamma = - P(\alpha + \beta)
\end{equation}

then the player can ensure they get a score \(\chi\) times
larger than the opponent. This extortion coefficient is given by:

\begin{equation}\label{eqn:definition_of_chi}
    \chi=\frac{-\beta}{\alpha}
\end{equation}

Thus, if (\ref{eqn:constraint_for_extortion}) holds and \(\chi >1\) a player is
said to extort their opponent.
Here, the reverse problem is considered: given a
\(p\in\mathbb{R}^4\) how does one identify \(\alpha, \beta\) if they
exist and is the strategy in fact acting in an extortionate way?

These conditions correspond to:

\begin{align}
    \tilde p_1 & = \alpha R + \beta R - P (\alpha + \beta)
            \label{eqn:condition_for_tilde_p1}\\
    \tilde p_2 & = \alpha S + \beta T - P (\alpha + \beta)
            \label{eqn:condition_for_tilde_p2}\\
    \tilde p_3 & = \alpha T + \beta S - P (\alpha + \beta)
            \label{eqn:condition_for_tilde_p3}\\
    \tilde p_4 & = \alpha P + \beta P - P (\alpha + \beta)
            \label{eqn:condition_for_tilde_p4}
\end{align}

Equation (\ref{eqn:condition_for_tilde_p4}) ensures that \(p_4=\tilde p_4=0\).
Equations (\ref{eqn:condition_for_tilde_p1}-\ref{eqn:condition_for_tilde_p3})
can be used to eliminate \(\alpha, \beta\), giving:

\begin{equation}\label{eqn:planar_definition_of_extortion}
    \tilde p_1 = \frac{(R - P)(\tilde p_2 + \tilde p_3)}{S + T - 2P}
\end{equation}

with:

\begin{equation}\label{eqn:definition_of_chi}
    \chi = \frac{\tilde p_2 (P - T) + \tilde p_3 (S - P)}
                {\tilde p_2 (P - S) + \tilde p_3 (T - P)}
\end{equation}

Given a strategy \(p\in\mathbb{R}^{4\times 1}\) equations
(\ref{eqn:condition_for_tilde_p4}), (\ref{eqn:planar_definition_of_extortion}-\ref{eqn:definition_of_chi}) can be used to check if
a strategy is extortionate. The conditions correspond to:

\begin{align}
    p_1 & = \frac{(R-P)(p_2 + p_3) - R + T + S - P}{S + T - 2P}
     \label{eqn:condition_for_p1}\\
    p_4 & = 0 \label{eqn:condition_for_p4}\\
    1 & > p_2 + p_3\label{eqn:condition_for_chi}
\end{align}

The algebraic steps necessary to prove these results are available in the
supporting materials.

All extortionate strategies reside on a triangular (\ref{eqn:condition_for_chi})
plane (\ref{eqn:condition_for_p1}) in 3 dimensions (\ref{eqn:condition_for_p4}).
Using this formulation it can be seen that a necessary (but not sufficient)
condition for an extortionate strategy is that it cooperates on average less
than 50\% of the time when in a state of disagreement with the opponent.

As an example, consider the known extortionate strategy \(p=(8 / 9, 1 / 2, 1 /
3, 0)\) from~\cite{Stewart2012} which is referred to as \texttt{Extort-2}. In
this case, for the standard values of \((R, T, S, P)\) constraint
(\ref{eqn:condition_for_p1}) corresponds to:

\begin{equation}
    p_1 = \frac{2(p_2 + p_3) + 1}{3}
\end{equation}

It is clear that in this case all constraints hold.

This approach could in fact be used to confirm that a given strategy is acting
in an extortionate manner even if it is not a memory one strategy. However, in
practice, if a closed form for \(p\) is not known, then due to measurement
and/or numerical error this would not work.

This problem can be written in the following linear algebraic form where
\(x=(\alpha, \beta)\)
and \(p^*=(\tilde p_1 - 1, tilde_2 - 1, p_3)\):

\begin{equation}\label{eqn:linear_algebraic_equation_for_p}
    Cx= p^*
\end{equation}

\(C\) corresponds to equations
(\ref{eqn:condition_for_tilde_p1}-\ref{eqn:condition_for_tilde_p3}) and is
given by:

\begin{equation}\label{eqn:definition_of_C}
    C =
    \begin{bmatrix}
        R - P & R- P \\
        S - P & T- P \\
        T - P & S- P \\
    \end{bmatrix}
\end{equation}

Note that in general, equation (\ref{eqn:linear_algebraic_equation_for_p}) will
not necessarily have a solution. From the Rouch\'{e}-Capelli theorem if there is
a solution it is unique as \(\text{rank}(C)=2\) which is the dimension of the
variable \(x\). The best fitting \(x\) is found by minimizing:

\begin{equation}\label{eqn:r_squared}
    \text{SSError} = \|C x- p^*\|_2^2 = \sum_{i=1}^{3}\left((C\bar x)_i-p_i^*\right)^2
\end{equation}

Note that \(\text{SSError}\), which is the square of the Frobenius
norm~\cite{Golub2013}, becomes a measure of how close a strategy is to being an
extortionate strategy. Suspicion
of extortion then corresponds to a threshold on \(\text{SSError}\).

By observing interactions (human or otherwise), their memory one representation
can be inferred and this approach can be used to recognise extortionate
behaviour. The notion of comparing theoretic and actual plays of the IPD is not
novel, see for example~\cite{Rand2013}. Immediately it is noted that if the
environment is noisy~\cite{Wu1995} then no strategy can be considered to be
extortionate as \(p_4>0\).

In the next section, this idea will be illustrated by observing the interactions
that take place in a computer based tournament of the IPD\@.

\section{Numerical experiments}\label{sec:numerical-experiments}

In~\cite{Stewart2012} results from a tournament with
\input{./assets/tex/number_of_stewart_plotkin_strategies/main.tex} strategies,
was presented with specific consideration given to ZD strategies. This
tournament is reproduced here using the Axelrod-Python
project~\cite{Knight2016}. To obtain a good measure of the corresponding
transition rates for each strategy all matches have been run for
\input{assets/tex/number_of_turns/main.tex} turns and every match has been
repeated \input{assets/tex/number_of_repetitions/main.tex} times. All of this
interaction data is available at~\cite{vincent_knight_2018_1297075}. A good
match between the inferred Markov chain and the state distribution of the actual
interactions has been verified. Data for this is presented in the supplementary
materials.

Figure~\ref{fig:SSError_overall_in_stewart_plotkin} shows the \(\text{SSError}\)
values for all the strategies in the tournament, as reported
in~\cite{Stewart2012} the extortionate strategy (which has an expected
\(\text{SSError}\) approximately 0) gains a large number of wins.

\begin{figure}[!htbp]
    \centering
    \includegraphics[width=.8\textwidth]{./assets/img/SSError_overall_in_stewart_plotkin/main.pdf}
    \caption{\(\text{SSError}\) and state probabilities for the strategies
        of~\cite{Stewart2012}, ordered both by number of wins and overall score.
        Note that \(P(DC)\) is not shown as it corresponds to the transpose of
        \(P(CD)\). Cooperator and Defector are omitted as they do not visit all
        the states.}
    \label{fig:SSError_overall_in_stewart_plotkin}
\end{figure}

Here, the work of~\cite{Stewart2012} is extended by investigating a tournament
with \input{assets/tex/number_of_full_strategies/main.tex}
strategies.

The results of this analysis are shown in
Figure~\ref{fig:SSError_and_probabilities_in_full}. The top ranking strategies
by number of wins seem to be extortionate (but not against all strategies) and
it can be seen that a small sub group of strategies achieve mutual defection.
All the top ranking strategies according to score achieve mutual cooperation and
do not extort each other, however they
\textbf{do} exhibit extortionate behaviour towards a number of the lower ranking
strategies.

\begin{figure}[!htbp]
    \centering
    \includegraphics[width=.8\textwidth]{./assets/img/SSError_and_probabilities_in_full/main.pdf}
    \caption{\(\text{SSError}\) for the strategies for the full tournament. Only
    strategy interactions for which \(p_4=0\) and \(\chi>1\) are displayed.}
    \label{fig:SSError_and_probabilities_in_full}
\end{figure}

\section{Conclusion}\label{sec:conclusion}

This work defines an approach to measure whether or not a player is playing a
strategy that corresponds to an extortionate strategy as defined
in~\cite{Press2012}: a mathematical model for suspicion. Indeed, all
extortionate strategies have been
 classified as lying on a triangular plane.
This rigorous classification fails to be robust to small measurement error, thus
a statistical approach is proposed.
This is done through a linear algebraic approach for approximating the solution
of a linear system. Using this, a large number of pairwise interactions is
simulated and in fact very few strategies are found to act extortionately.

The work of~\cite{Press2012}, whilst showing that a clever approach to taking
advantage of another memory one strategy exists: this is incomplete. Whilst the
elegance of this result is very attractive, just as the simplicity of the
victory of Tit For Tat in Axelrod's original tournaments was, it is incomplete.
Extortionate strategies achieve a high number of wins but they do not
achieve a high score which corresponds to the fitness landscape in an
evolutionary sense. From the large number of interactions a payoff matrix \(S\)
can be measured where \(S_{ij}\) denotes the score (using standard values of
\((R, S, T, P) = (3, 0, 5, 1)\)) of the \(i\)th strategy
against the \(j\)th strategy. Using this, the replicator equation
describes the evolution of the system based on a population density fitness
function:

\begin{equation}\label{eqn:replicator_dynamics}
    \frac{dx}{dt} = x(S-x^TS x)
\end{equation}

Equation (\ref{eqn:replicator_dynamics}) is solved numerically through an
integration technique described in~\cite{Petzold1983} and
Figure~\ref{fig:replicator_dynamics} shows the evolution of the distribution of
the system: the various strategies are ranked by scores. It is clear to see that
only the high ranking strategies survive the evolutionary process (in fact,
only \input{./assets/img/replicator_dynamics/main.tex}
have a final distribution greater than \(10 ^ {-2}\)). This confirms the
findings of~\cite{Moran1707} in which sophisticated strategies resist
evolutionary invasion of shorter memory strategies. Recalling
Figure~\ref{fig:SSError_and_probabilities_in_full} this demonstrates that:

\begin{itemize}
    \item Cooperation emerges through the evolutionary process: the high scoring
        strategies do not exhibit extortionate behaviour towards each other.
    \item Extortionate strategies do not survive the evolutionary process.
\end{itemize}

\begin{figure}[!htbp]
    \centering
    \includegraphics[width=.8\textwidth]{./assets/img/replicator_dynamics/main.pdf}
    \caption{Numerical simulation of the replicator equation
    (\ref{eqn:replicator_dynamics}): strategies are ordered by score, only the strategies with a high score survive the evolutionary process.}
    \label{fig:replicator_dynamics}
\end{figure}

This work can be used to classify plays of the IPD\@: data can be collected from
actual interactions (in lab or in the field). Furthermore, this allows for a
classification method similar to the notion of fingerprinting presented
in~\cite{Ashlock2008}. Trained strategies can potentially be classified as
extortionate or not or it could be possible to even constrain the reinforcement
learning approaches that are becoming prevalent in the literature.
Alternatively, this mathematical approach for recognising extortion could be
used in sophisticated strategies to defend against invasion. Arguably, some of
the strategies considered here exhibit this behaviour, indeed as described
in~\cite{Harper2017}, the top ranking strategies in the full tournament are
obtained using evolutionary reinforcement learning techniques, thus, suspicion
of extortionate behaviour could in fact be an evolutionary trait.

\section*{Acknowledgements}

The following open source software libraries were used in this research:

\begin{itemize}
    \item The Axelrod ~\cite{Knight2016, Knight2018} library (IPD strategies and
        tournaments).
    \item The sympy library~\cite{Meurer2017} (verification of all symbolic
        calculations).
    \item The matplotlib~\cite{Droettboom2018} library (visualisation).
    \item The pandas~\cite{Structures2010}, dask~\cite{Dask2016} and
        NumPy~\cite{Oliphant2015} libraries (data manipulation).
    \item The SciPy~\cite{Jones2001} library (numerical integration of the
        replicator equation).
\end{itemize}

This work was performed using the computational facilities of the Advanced
Research Computing @ Cardiff (ARCCA) Division, Cardiff University.

\printbibliography

\newpage
\section*{Supplementary materials}

\includepdf{assets/pdf/proof_of_form_of_extortionate_strategies/main.pdf}

\newpage

Using the pair wise interactions the transition rates \(p,
q\) can be measured and the steady state probabilities inferred and compared to
the actual probabilities of each state.
This is done numerically by computing the singular eigenvector of the
matrix \(A\) \cite{Stewart2009}:

\[
    A =
    \begin{bmatrix}
        p_1 q_1 & p_1 (1 - q_1) & (1 - p_1) q_1 & (1 -p_1) (1 - q_1) \\
        p_2 q_2 & p_2 (1 - q_2) & (1 - p_2) q_2 & (1 -p_2) (1 - q_2) \\
        p_3 q_3 & p_3 (1 - q_3) & (1 - p_3) q_3 & (1 -p_3) (1 - q_3) \\
        p_4 q_4 & p_4 (1 - q_4) & (1 - p_4) q_4 & (1 -p_4) (1 - q_4) \\
    \end{bmatrix}
\]

Figure~\ref{fig:computed_probabilities_vs_theoretic_probabilities} shows a
regression line fitted to every pairwise interaction with a reported
\(\text{SSError}\) value (pairwise interactions with missing states were
omitted). This serves to validate the approach: a part from some edge cases the
relationship is consistent.

\begin{figure}[!htbp]
    \centering
    \includegraphics[width=.8\textwidth]{./assets/img/computed_probabilities_vs_theoretic_probabilities/main.pdf}
    \caption{The
        relationship between the steady state probabilities inferred from the
        measured transitions and the actual steady state probabilities. A linear
        regression line is included validating the approach.}
    \label{fig:computed_probabilities_vs_theoretic_probabilities}
\end{figure}


\end{document}

    strategies is considered. In this setting
    the most highly performing strategies do not play in an extortionate way
    against each other but do against lower performing strategies.
    This suggests that whilst the theory of Zero Determinant strategies
    indicates that memory is not of fundamental importance to the evolution of
    cooperative behaviour, this is incomplete.
\end{abstract}

\section{Introduction}\label{sec:introduction}

Agent based game theoretic models have become a stalwart of the underpinning
mathematics of interactive behaviours. One of the major pieces of work
in this area is the pair of original computer tournaments run by Robert
Axelrod~\cite{Axelrod1980, Axelrod1980a}. These tournaments pitted submitted
computer strategies against each other in plays of the Iterated Prisoner's
Dilemma. A common game where agents can choose to pay a slight cost to their
immediate utility in the hope of building a reputation. This has been used in
economic and evolutionary game theory to understand the evolution of cooperative
behaviour.

Recently, a class of strategies was described in~\cite{Press2012} that can
provably extort any given opponent. In~\cite{Hilbe2013, Moran1707} some
questions have already been asked about the true effectiveness of these
strategies in an evolutionary setting. Here another question is asked: is it
possible to recognise this extortionate behaviour? A mathematical procedure for
suspicion is presented: in the same way that the continued actions of an
extortionate individual might raise suspicion.

This work makes use of the Axelrod Python library~\cite{Knight2018, Knight2016}
with a large number of Prisoner Dilemma strategies available to give an
extensive numerical example of the ideas presented.  The approach is presented
in Section~\ref{sec:delta-zd-strategies}.  All of the code and data discussed
in Section~\ref{sec:numerical-experiments} is open sourced, archived and
written according to best scientific principles~\cite{Wilson2014}. The data
archive can be found at~\cite{vincent_knight_2018_1297075}.

\section{Recognising Extortion}\label{sec:delta-zd-strategies}

In~\cite{Press2012}, given a match between 2 memory-one strategies, the concept
of Zero Determinant (ZD) strategies is introduced. The main result of that paper
shows that given two memory one players \(p, q\in\mathbb{R}^4\) a linear
relationship between the players' scores could be forced by one of the players.

Using the notation of~\cite{Press2012}, assuming the utilities for player \(p\)
are given by \(S_x=(R, S, T, P)\) and for player \(q\) by \(S_y=(R, T, S, P)\)
and that the stationary scores of each player is given by \(S_X\) and \(S_Y\)
respectively. The main result of~\cite{Press2012} is that if

\begin{equation}\label{eqn:linear_relationship_for_p}
    \tilde p=\alpha S_x + \beta S_y + \gamma
\end{equation}

or

\begin{equation}\label{eqn:linear_relationship_for_q}
    \tilde q=\alpha S_x + \beta S_y + \gamma
\end{equation}

where \(\tilde p = (1 - p_1, 1 - p_2, p_3, p_4)\) and
\(\tilde q = (1 - q_1, 1 - q_2, q_3, q_4)\) then:

\begin{equation}
    \alpha S_X + \beta S_Y + \gamma = 0
\end{equation}

In~\cite{Press2012} a particular type of ZD strategy is defined: extortionate
strategies. If:

\begin{equation}\label{eqn:constraint_for_extortion}
    \gamma = - P(\alpha + \beta)
\end{equation}

then the player can ensure they get a score \(\chi\) times
larger than the opponent. This extortion coefficient is given by:

\begin{equation}\label{eqn:definition_of_chi}
    \chi=\frac{-\beta}{\alpha}
\end{equation}

Thus, if (\ref{eqn:constraint_for_extortion}) holds and \(\chi >1\) a player is
said to extort their opponent.
Here, the reverse problem is considered: given a
\(p\in\mathbb{R}^4\) how does one identify \(\alpha, \beta\) if they
exist and is the strategy in fact acting in an extortionate way?

These conditions correspond to:

\begin{align}
    \tilde p_1 & = \alpha R + \beta R - P (\alpha + \beta)
            \label{eqn:condition_for_tilde_p1}\\
    \tilde p_2 & = \alpha S + \beta T - P (\alpha + \beta)
            \label{eqn:condition_for_tilde_p2}\\
    \tilde p_3 & = \alpha T + \beta S - P (\alpha + \beta)
            \label{eqn:condition_for_tilde_p3}\\
    \tilde p_4 & = \alpha P + \beta P - P (\alpha + \beta)
            \label{eqn:condition_for_tilde_p4}
\end{align}

Equation (\ref{eqn:condition_for_tilde_p4}) ensures that \(p_4=\tilde p_4=0\).
Equations (\ref{eqn:condition_for_tilde_p1}-\ref{eqn:condition_for_tilde_p3})
can be used to eliminate \(\alpha, \beta\), giving:

\begin{equation}\label{eqn:planar_definition_of_extortion}
    \tilde p_1 = \frac{(R - P)(\tilde p_2 + \tilde p_3)}{S + T - 2P}
\end{equation}

with:

\begin{equation}\label{eqn:definition_of_chi}
    \chi = \frac{\tilde p_2 (P - T) + \tilde p_3 (S - P)}
                {\tilde p_2 (P - S) + \tilde p_3 (T - P)}
\end{equation}

Given a strategy \(p\in\mathbb{R}^{4\times 1}\) equations
(\ref{eqn:condition_for_tilde_p4}), (\ref{eqn:planar_definition_of_extortion}-\ref{eqn:definition_of_chi}) can be used to check if
a strategy is extortionate. The conditions correspond to:

\begin{align}
    p_1 & = \frac{(R-P)(p_2 + p_3) - R + T + S - P}{S + T - 2P}
     \label{eqn:condition_for_p1}\\
    p_4 & = 0 \label{eqn:condition_for_p4}\\
    1 & > p_2 + p_3\label{eqn:condition_for_chi}
\end{align}

The algebraic steps necessary to prove these results are available in the
supporting materials.

All extortionate strategies reside on a triangular (\ref{eqn:condition_for_chi})
plane (\ref{eqn:condition_for_p1}) in 3 dimensions (\ref{eqn:condition_for_p4}).
Using this formulation it can be seen that a necessary (but not sufficient)
condition for an extortionate strategy is that it cooperates on average less
than 50\% of the time when in a state of disagreement with the opponent.

As an example, consider the known extortionate strategy \(p=(8 / 9, 1 / 2, 1 /
3, 0)\) from~\cite{Stewart2012} which is referred to as \texttt{Extort-2}. In
this case, for the standard values of \((R, T, S, P)\) constraint
(\ref{eqn:condition_for_p1}) corresponds to:

\begin{equation}
    p_1 = \frac{2(p_2 + p_3) + 1}{3}
\end{equation}

It is clear that in this case all constraints hold.

This approach could in fact be used to confirm that a given strategy is acting
in an extortionate manner even if it is not a memory one strategy. However, in
practice, if a closed form for \(p\) is not known, then due to measurement
and/or numerical error this would not work.

This problem can be written in the following linear algebraic form where
\(x=(\alpha, \beta)\)
and \(p^*=(\tilde p_1 - 1, tilde_2 - 1, p_3)\):

\begin{equation}\label{eqn:linear_algebraic_equation_for_p}
    Cx= p^*
\end{equation}

\(C\) corresponds to equations
(\ref{eqn:condition_for_tilde_p1}-\ref{eqn:condition_for_tilde_p3}) and is
given by:

\begin{equation}\label{eqn:definition_of_C}
    C =
    \begin{bmatrix}
        R - P & R- P \\
        S - P & T- P \\
        T - P & S- P \\
    \end{bmatrix}
\end{equation}

Note that in general, equation (\ref{eqn:linear_algebraic_equation_for_p}) will
not necessarily have a solution. From the Rouch\'{e}-Capelli theorem if there is
a solution it is unique as \(\text{rank}(C)=2\) which is the dimension of the
variable \(x\). The best fitting \(x\) is found by minimizing:

\begin{equation}\label{eqn:r_squared}
    \text{SSError} = \|C x- p^*\|_2^2 = \sum_{i=1}^{3}\left((C\bar x)_i-p_i^*\right)^2
\end{equation}

Note that \(\text{SSError}\), which is the square of the Frobenius
norm~\cite{Golub2013}, becomes a measure of how close a strategy is to being an
extortionate strategy. Suspicion
of extortion then corresponds to a threshold on \(\text{SSError}\).

By observing interactions (human or otherwise), their memory one representation
can be inferred and this approach can be used to recognise extortionate
behaviour. The notion of comparing theoretic and actual plays of the IPD is not
novel, see for example~\cite{Rand2013}. Immediately it is noted that if the
environment is noisy~\cite{Wu1995} then no strategy can be considered to be
extortionate as \(p_4>0\).

In the next section, this idea will be illustrated by observing the interactions
that take place in a computer based tournament of the IPD\@.

\section{Numerical experiments}\label{sec:numerical-experiments}

In~\cite{Stewart2012} results from a tournament with
\documentclass[a4paper]{article}

\usepackage{amsmath}
\usepackage{amssymb}
\usepackage[margin=1.5cm,
            includefoot,
            footskip=30pt]{geometry}
\usepackage{layout}
\usepackage{graphicx}
\usepackage{subcaption}

\usepackage{biblatex}
\usepackage{pdfpages}

\bibliography{main.bib}

\title{Suspicion: Recognising and evaluating the effectiveness
       of extortion in the Iterated Prisoner's Dilemma}
\author{Vincent A. Knight \and Nikoleta E. Glynatsi}
\date{\today}



\begin{document}

\maketitle

\begin{abstract}
    The Iterated Prisoner's Dilemma is a model for rational and evolutionary
    interactive behaviour. It has applications both in the study of human social
    behaviour as well as in biology.
    It is used to understand when and how a rational individual might
    accept an immediate cost to their own utility for the direct benefit of
    another.

    Much attention has been given to a class of strategies called
    Zero Determinant strategies. It has been theoretically shown that these
    strategies can ``extort'' any player.

    In this work, an approach to identify if observed strategies are playing in
    an extortionate way is described. Furthermore, experimental analysis of
    a large tournament with \input{assets/tex/number_of_full_strategies/main.tex}
    strategies is considered. In this setting
    the most highly performing strategies do not play in an extortionate way
    against each other but do against lower performing strategies.
    This suggests that whilst the theory of Zero Determinant strategies
    indicates that memory is not of fundamental importance to the evolution of
    cooperative behaviour, this is incomplete.
\end{abstract}

\section{Introduction}\label{sec:introduction}

Agent based game theoretic models have become a stalwart of the underpinning
mathematics of interactive behaviours. One of the major pieces of work
in this area is the pair of original computer tournaments run by Robert
Axelrod~\cite{Axelrod1980, Axelrod1980a}. These tournaments pitted submitted
computer strategies against each other in plays of the Iterated Prisoner's
Dilemma. A common game where agents can choose to pay a slight cost to their
immediate utility in the hope of building a reputation. This has been used in
economic and evolutionary game theory to understand the evolution of cooperative
behaviour.

Recently, a class of strategies was described in~\cite{Press2012} that can
provably extort any given opponent. In~\cite{Hilbe2013, Moran1707} some
questions have already been asked about the true effectiveness of these
strategies in an evolutionary setting. Here another question is asked: is it
possible to recognise this extortionate behaviour? A mathematical procedure for
suspicion is presented: in the same way that the continued actions of an
extortionate individual might raise suspicion.

This work makes use of the Axelrod Python library~\cite{Knight2018, Knight2016}
with a large number of Prisoner Dilemma strategies available to give an
extensive numerical example of the ideas presented.  The approach is presented
in Section~\ref{sec:delta-zd-strategies}.  All of the code and data discussed
in Section~\ref{sec:numerical-experiments} is open sourced, archived and
written according to best scientific principles~\cite{Wilson2014}. The data
archive can be found at~\cite{vincent_knight_2018_1297075}.

\section{Recognising Extortion}\label{sec:delta-zd-strategies}

In~\cite{Press2012}, given a match between 2 memory-one strategies, the concept
of Zero Determinant (ZD) strategies is introduced. The main result of that paper
shows that given two memory one players \(p, q\in\mathbb{R}^4\) a linear
relationship between the players' scores could be forced by one of the players.

Using the notation of~\cite{Press2012}, assuming the utilities for player \(p\)
are given by \(S_x=(R, S, T, P)\) and for player \(q\) by \(S_y=(R, T, S, P)\)
and that the stationary scores of each player is given by \(S_X\) and \(S_Y\)
respectively. The main result of~\cite{Press2012} is that if

\begin{equation}\label{eqn:linear_relationship_for_p}
    \tilde p=\alpha S_x + \beta S_y + \gamma
\end{equation}

or

\begin{equation}\label{eqn:linear_relationship_for_q}
    \tilde q=\alpha S_x + \beta S_y + \gamma
\end{equation}

where \(\tilde p = (1 - p_1, 1 - p_2, p_3, p_4)\) and
\(\tilde q = (1 - q_1, 1 - q_2, q_3, q_4)\) then:

\begin{equation}
    \alpha S_X + \beta S_Y + \gamma = 0
\end{equation}

In~\cite{Press2012} a particular type of ZD strategy is defined: extortionate
strategies. If:

\begin{equation}\label{eqn:constraint_for_extortion}
    \gamma = - P(\alpha + \beta)
\end{equation}

then the player can ensure they get a score \(\chi\) times
larger than the opponent. This extortion coefficient is given by:

\begin{equation}\label{eqn:definition_of_chi}
    \chi=\frac{-\beta}{\alpha}
\end{equation}

Thus, if (\ref{eqn:constraint_for_extortion}) holds and \(\chi >1\) a player is
said to extort their opponent.
Here, the reverse problem is considered: given a
\(p\in\mathbb{R}^4\) how does one identify \(\alpha, \beta\) if they
exist and is the strategy in fact acting in an extortionate way?

These conditions correspond to:

\begin{align}
    \tilde p_1 & = \alpha R + \beta R - P (\alpha + \beta)
            \label{eqn:condition_for_tilde_p1}\\
    \tilde p_2 & = \alpha S + \beta T - P (\alpha + \beta)
            \label{eqn:condition_for_tilde_p2}\\
    \tilde p_3 & = \alpha T + \beta S - P (\alpha + \beta)
            \label{eqn:condition_for_tilde_p3}\\
    \tilde p_4 & = \alpha P + \beta P - P (\alpha + \beta)
            \label{eqn:condition_for_tilde_p4}
\end{align}

Equation (\ref{eqn:condition_for_tilde_p4}) ensures that \(p_4=\tilde p_4=0\).
Equations (\ref{eqn:condition_for_tilde_p1}-\ref{eqn:condition_for_tilde_p3})
can be used to eliminate \(\alpha, \beta\), giving:

\begin{equation}\label{eqn:planar_definition_of_extortion}
    \tilde p_1 = \frac{(R - P)(\tilde p_2 + \tilde p_3)}{S + T - 2P}
\end{equation}

with:

\begin{equation}\label{eqn:definition_of_chi}
    \chi = \frac{\tilde p_2 (P - T) + \tilde p_3 (S - P)}
                {\tilde p_2 (P - S) + \tilde p_3 (T - P)}
\end{equation}

Given a strategy \(p\in\mathbb{R}^{4\times 1}\) equations
(\ref{eqn:condition_for_tilde_p4}), (\ref{eqn:planar_definition_of_extortion}-\ref{eqn:definition_of_chi}) can be used to check if
a strategy is extortionate. The conditions correspond to:

\begin{align}
    p_1 & = \frac{(R-P)(p_2 + p_3) - R + T + S - P}{S + T - 2P}
     \label{eqn:condition_for_p1}\\
    p_4 & = 0 \label{eqn:condition_for_p4}\\
    1 & > p_2 + p_3\label{eqn:condition_for_chi}
\end{align}

The algebraic steps necessary to prove these results are available in the
supporting materials.

All extortionate strategies reside on a triangular (\ref{eqn:condition_for_chi})
plane (\ref{eqn:condition_for_p1}) in 3 dimensions (\ref{eqn:condition_for_p4}).
Using this formulation it can be seen that a necessary (but not sufficient)
condition for an extortionate strategy is that it cooperates on average less
than 50\% of the time when in a state of disagreement with the opponent.

As an example, consider the known extortionate strategy \(p=(8 / 9, 1 / 2, 1 /
3, 0)\) from~\cite{Stewart2012} which is referred to as \texttt{Extort-2}. In
this case, for the standard values of \((R, T, S, P)\) constraint
(\ref{eqn:condition_for_p1}) corresponds to:

\begin{equation}
    p_1 = \frac{2(p_2 + p_3) + 1}{3}
\end{equation}

It is clear that in this case all constraints hold.

This approach could in fact be used to confirm that a given strategy is acting
in an extortionate manner even if it is not a memory one strategy. However, in
practice, if a closed form for \(p\) is not known, then due to measurement
and/or numerical error this would not work.

This problem can be written in the following linear algebraic form where
\(x=(\alpha, \beta)\)
and \(p^*=(\tilde p_1 - 1, tilde_2 - 1, p_3)\):

\begin{equation}\label{eqn:linear_algebraic_equation_for_p}
    Cx= p^*
\end{equation}

\(C\) corresponds to equations
(\ref{eqn:condition_for_tilde_p1}-\ref{eqn:condition_for_tilde_p3}) and is
given by:

\begin{equation}\label{eqn:definition_of_C}
    C =
    \begin{bmatrix}
        R - P & R- P \\
        S - P & T- P \\
        T - P & S- P \\
    \end{bmatrix}
\end{equation}

Note that in general, equation (\ref{eqn:linear_algebraic_equation_for_p}) will
not necessarily have a solution. From the Rouch\'{e}-Capelli theorem if there is
a solution it is unique as \(\text{rank}(C)=2\) which is the dimension of the
variable \(x\). The best fitting \(x\) is found by minimizing:

\begin{equation}\label{eqn:r_squared}
    \text{SSError} = \|C x- p^*\|_2^2 = \sum_{i=1}^{3}\left((C\bar x)_i-p_i^*\right)^2
\end{equation}

Note that \(\text{SSError}\), which is the square of the Frobenius
norm~\cite{Golub2013}, becomes a measure of how close a strategy is to being an
extortionate strategy. Suspicion
of extortion then corresponds to a threshold on \(\text{SSError}\).

By observing interactions (human or otherwise), their memory one representation
can be inferred and this approach can be used to recognise extortionate
behaviour. The notion of comparing theoretic and actual plays of the IPD is not
novel, see for example~\cite{Rand2013}. Immediately it is noted that if the
environment is noisy~\cite{Wu1995} then no strategy can be considered to be
extortionate as \(p_4>0\).

In the next section, this idea will be illustrated by observing the interactions
that take place in a computer based tournament of the IPD\@.

\section{Numerical experiments}\label{sec:numerical-experiments}

In~\cite{Stewart2012} results from a tournament with
\input{./assets/tex/number_of_stewart_plotkin_strategies/main.tex} strategies,
was presented with specific consideration given to ZD strategies. This
tournament is reproduced here using the Axelrod-Python
project~\cite{Knight2016}. To obtain a good measure of the corresponding
transition rates for each strategy all matches have been run for
\input{assets/tex/number_of_turns/main.tex} turns and every match has been
repeated \input{assets/tex/number_of_repetitions/main.tex} times. All of this
interaction data is available at~\cite{vincent_knight_2018_1297075}. A good
match between the inferred Markov chain and the state distribution of the actual
interactions has been verified. Data for this is presented in the supplementary
materials.

Figure~\ref{fig:SSError_overall_in_stewart_plotkin} shows the \(\text{SSError}\)
values for all the strategies in the tournament, as reported
in~\cite{Stewart2012} the extortionate strategy (which has an expected
\(\text{SSError}\) approximately 0) gains a large number of wins.

\begin{figure}[!htbp]
    \centering
    \includegraphics[width=.8\textwidth]{./assets/img/SSError_overall_in_stewart_plotkin/main.pdf}
    \caption{\(\text{SSError}\) and state probabilities for the strategies
        of~\cite{Stewart2012}, ordered both by number of wins and overall score.
        Note that \(P(DC)\) is not shown as it corresponds to the transpose of
        \(P(CD)\). Cooperator and Defector are omitted as they do not visit all
        the states.}
    \label{fig:SSError_overall_in_stewart_plotkin}
\end{figure}

Here, the work of~\cite{Stewart2012} is extended by investigating a tournament
with \input{assets/tex/number_of_full_strategies/main.tex}
strategies.

The results of this analysis are shown in
Figure~\ref{fig:SSError_and_probabilities_in_full}. The top ranking strategies
by number of wins seem to be extortionate (but not against all strategies) and
it can be seen that a small sub group of strategies achieve mutual defection.
All the top ranking strategies according to score achieve mutual cooperation and
do not extort each other, however they
\textbf{do} exhibit extortionate behaviour towards a number of the lower ranking
strategies.

\begin{figure}[!htbp]
    \centering
    \includegraphics[width=.8\textwidth]{./assets/img/SSError_and_probabilities_in_full/main.pdf}
    \caption{\(\text{SSError}\) for the strategies for the full tournament. Only
    strategy interactions for which \(p_4=0\) and \(\chi>1\) are displayed.}
    \label{fig:SSError_and_probabilities_in_full}
\end{figure}

\section{Conclusion}\label{sec:conclusion}

This work defines an approach to measure whether or not a player is playing a
strategy that corresponds to an extortionate strategy as defined
in~\cite{Press2012}: a mathematical model for suspicion. Indeed, all
extortionate strategies have been
 classified as lying on a triangular plane.
This rigorous classification fails to be robust to small measurement error, thus
a statistical approach is proposed.
This is done through a linear algebraic approach for approximating the solution
of a linear system. Using this, a large number of pairwise interactions is
simulated and in fact very few strategies are found to act extortionately.

The work of~\cite{Press2012}, whilst showing that a clever approach to taking
advantage of another memory one strategy exists: this is incomplete. Whilst the
elegance of this result is very attractive, just as the simplicity of the
victory of Tit For Tat in Axelrod's original tournaments was, it is incomplete.
Extortionate strategies achieve a high number of wins but they do not
achieve a high score which corresponds to the fitness landscape in an
evolutionary sense. From the large number of interactions a payoff matrix \(S\)
can be measured where \(S_{ij}\) denotes the score (using standard values of
\((R, S, T, P) = (3, 0, 5, 1)\)) of the \(i\)th strategy
against the \(j\)th strategy. Using this, the replicator equation
describes the evolution of the system based on a population density fitness
function:

\begin{equation}\label{eqn:replicator_dynamics}
    \frac{dx}{dt} = x(S-x^TS x)
\end{equation}

Equation (\ref{eqn:replicator_dynamics}) is solved numerically through an
integration technique described in~\cite{Petzold1983} and
Figure~\ref{fig:replicator_dynamics} shows the evolution of the distribution of
the system: the various strategies are ranked by scores. It is clear to see that
only the high ranking strategies survive the evolutionary process (in fact,
only \input{./assets/img/replicator_dynamics/main.tex}
have a final distribution greater than \(10 ^ {-2}\)). This confirms the
findings of~\cite{Moran1707} in which sophisticated strategies resist
evolutionary invasion of shorter memory strategies. Recalling
Figure~\ref{fig:SSError_and_probabilities_in_full} this demonstrates that:

\begin{itemize}
    \item Cooperation emerges through the evolutionary process: the high scoring
        strategies do not exhibit extortionate behaviour towards each other.
    \item Extortionate strategies do not survive the evolutionary process.
\end{itemize}

\begin{figure}[!htbp]
    \centering
    \includegraphics[width=.8\textwidth]{./assets/img/replicator_dynamics/main.pdf}
    \caption{Numerical simulation of the replicator equation
    (\ref{eqn:replicator_dynamics}): strategies are ordered by score, only the strategies with a high score survive the evolutionary process.}
    \label{fig:replicator_dynamics}
\end{figure}

This work can be used to classify plays of the IPD\@: data can be collected from
actual interactions (in lab or in the field). Furthermore, this allows for a
classification method similar to the notion of fingerprinting presented
in~\cite{Ashlock2008}. Trained strategies can potentially be classified as
extortionate or not or it could be possible to even constrain the reinforcement
learning approaches that are becoming prevalent in the literature.
Alternatively, this mathematical approach for recognising extortion could be
used in sophisticated strategies to defend against invasion. Arguably, some of
the strategies considered here exhibit this behaviour, indeed as described
in~\cite{Harper2017}, the top ranking strategies in the full tournament are
obtained using evolutionary reinforcement learning techniques, thus, suspicion
of extortionate behaviour could in fact be an evolutionary trait.

\section*{Acknowledgements}

The following open source software libraries were used in this research:

\begin{itemize}
    \item The Axelrod ~\cite{Knight2016, Knight2018} library (IPD strategies and
        tournaments).
    \item The sympy library~\cite{Meurer2017} (verification of all symbolic
        calculations).
    \item The matplotlib~\cite{Droettboom2018} library (visualisation).
    \item The pandas~\cite{Structures2010}, dask~\cite{Dask2016} and
        NumPy~\cite{Oliphant2015} libraries (data manipulation).
    \item The SciPy~\cite{Jones2001} library (numerical integration of the
        replicator equation).
\end{itemize}

This work was performed using the computational facilities of the Advanced
Research Computing @ Cardiff (ARCCA) Division, Cardiff University.

\printbibliography

\newpage
\section*{Supplementary materials}

\includepdf{assets/pdf/proof_of_form_of_extortionate_strategies/main.pdf}

\newpage

Using the pair wise interactions the transition rates \(p,
q\) can be measured and the steady state probabilities inferred and compared to
the actual probabilities of each state.
This is done numerically by computing the singular eigenvector of the
matrix \(A\) \cite{Stewart2009}:

\[
    A =
    \begin{bmatrix}
        p_1 q_1 & p_1 (1 - q_1) & (1 - p_1) q_1 & (1 -p_1) (1 - q_1) \\
        p_2 q_2 & p_2 (1 - q_2) & (1 - p_2) q_2 & (1 -p_2) (1 - q_2) \\
        p_3 q_3 & p_3 (1 - q_3) & (1 - p_3) q_3 & (1 -p_3) (1 - q_3) \\
        p_4 q_4 & p_4 (1 - q_4) & (1 - p_4) q_4 & (1 -p_4) (1 - q_4) \\
    \end{bmatrix}
\]

Figure~\ref{fig:computed_probabilities_vs_theoretic_probabilities} shows a
regression line fitted to every pairwise interaction with a reported
\(\text{SSError}\) value (pairwise interactions with missing states were
omitted). This serves to validate the approach: a part from some edge cases the
relationship is consistent.

\begin{figure}[!htbp]
    \centering
    \includegraphics[width=.8\textwidth]{./assets/img/computed_probabilities_vs_theoretic_probabilities/main.pdf}
    \caption{The
        relationship between the steady state probabilities inferred from the
        measured transitions and the actual steady state probabilities. A linear
        regression line is included validating the approach.}
    \label{fig:computed_probabilities_vs_theoretic_probabilities}
\end{figure}


\end{document}
 strategies,
was presented with specific consideration given to ZD strategies. This
tournament is reproduced here using the Axelrod-Python
project~\cite{Knight2016}. To obtain a good measure of the corresponding
transition rates for each strategy all matches have been run for
\documentclass[a4paper]{article}

\usepackage{amsmath}
\usepackage{amssymb}
\usepackage[margin=1.5cm,
            includefoot,
            footskip=30pt]{geometry}
\usepackage{layout}
\usepackage{graphicx}
\usepackage{subcaption}

\usepackage{biblatex}
\usepackage{pdfpages}

\bibliography{main.bib}

\title{Suspicion: Recognising and evaluating the effectiveness
       of extortion in the Iterated Prisoner's Dilemma}
\author{Vincent A. Knight \and Nikoleta E. Glynatsi}
\date{\today}



\begin{document}

\maketitle

\begin{abstract}
    The Iterated Prisoner's Dilemma is a model for rational and evolutionary
    interactive behaviour. It has applications both in the study of human social
    behaviour as well as in biology.
    It is used to understand when and how a rational individual might
    accept an immediate cost to their own utility for the direct benefit of
    another.

    Much attention has been given to a class of strategies called
    Zero Determinant strategies. It has been theoretically shown that these
    strategies can ``extort'' any player.

    In this work, an approach to identify if observed strategies are playing in
    an extortionate way is described. Furthermore, experimental analysis of
    a large tournament with \input{assets/tex/number_of_full_strategies/main.tex}
    strategies is considered. In this setting
    the most highly performing strategies do not play in an extortionate way
    against each other but do against lower performing strategies.
    This suggests that whilst the theory of Zero Determinant strategies
    indicates that memory is not of fundamental importance to the evolution of
    cooperative behaviour, this is incomplete.
\end{abstract}

\section{Introduction}\label{sec:introduction}

Agent based game theoretic models have become a stalwart of the underpinning
mathematics of interactive behaviours. One of the major pieces of work
in this area is the pair of original computer tournaments run by Robert
Axelrod~\cite{Axelrod1980, Axelrod1980a}. These tournaments pitted submitted
computer strategies against each other in plays of the Iterated Prisoner's
Dilemma. A common game where agents can choose to pay a slight cost to their
immediate utility in the hope of building a reputation. This has been used in
economic and evolutionary game theory to understand the evolution of cooperative
behaviour.

Recently, a class of strategies was described in~\cite{Press2012} that can
provably extort any given opponent. In~\cite{Hilbe2013, Moran1707} some
questions have already been asked about the true effectiveness of these
strategies in an evolutionary setting. Here another question is asked: is it
possible to recognise this extortionate behaviour? A mathematical procedure for
suspicion is presented: in the same way that the continued actions of an
extortionate individual might raise suspicion.

This work makes use of the Axelrod Python library~\cite{Knight2018, Knight2016}
with a large number of Prisoner Dilemma strategies available to give an
extensive numerical example of the ideas presented.  The approach is presented
in Section~\ref{sec:delta-zd-strategies}.  All of the code and data discussed
in Section~\ref{sec:numerical-experiments} is open sourced, archived and
written according to best scientific principles~\cite{Wilson2014}. The data
archive can be found at~\cite{vincent_knight_2018_1297075}.

\section{Recognising Extortion}\label{sec:delta-zd-strategies}

In~\cite{Press2012}, given a match between 2 memory-one strategies, the concept
of Zero Determinant (ZD) strategies is introduced. The main result of that paper
shows that given two memory one players \(p, q\in\mathbb{R}^4\) a linear
relationship between the players' scores could be forced by one of the players.

Using the notation of~\cite{Press2012}, assuming the utilities for player \(p\)
are given by \(S_x=(R, S, T, P)\) and for player \(q\) by \(S_y=(R, T, S, P)\)
and that the stationary scores of each player is given by \(S_X\) and \(S_Y\)
respectively. The main result of~\cite{Press2012} is that if

\begin{equation}\label{eqn:linear_relationship_for_p}
    \tilde p=\alpha S_x + \beta S_y + \gamma
\end{equation}

or

\begin{equation}\label{eqn:linear_relationship_for_q}
    \tilde q=\alpha S_x + \beta S_y + \gamma
\end{equation}

where \(\tilde p = (1 - p_1, 1 - p_2, p_3, p_4)\) and
\(\tilde q = (1 - q_1, 1 - q_2, q_3, q_4)\) then:

\begin{equation}
    \alpha S_X + \beta S_Y + \gamma = 0
\end{equation}

In~\cite{Press2012} a particular type of ZD strategy is defined: extortionate
strategies. If:

\begin{equation}\label{eqn:constraint_for_extortion}
    \gamma = - P(\alpha + \beta)
\end{equation}

then the player can ensure they get a score \(\chi\) times
larger than the opponent. This extortion coefficient is given by:

\begin{equation}\label{eqn:definition_of_chi}
    \chi=\frac{-\beta}{\alpha}
\end{equation}

Thus, if (\ref{eqn:constraint_for_extortion}) holds and \(\chi >1\) a player is
said to extort their opponent.
Here, the reverse problem is considered: given a
\(p\in\mathbb{R}^4\) how does one identify \(\alpha, \beta\) if they
exist and is the strategy in fact acting in an extortionate way?

These conditions correspond to:

\begin{align}
    \tilde p_1 & = \alpha R + \beta R - P (\alpha + \beta)
            \label{eqn:condition_for_tilde_p1}\\
    \tilde p_2 & = \alpha S + \beta T - P (\alpha + \beta)
            \label{eqn:condition_for_tilde_p2}\\
    \tilde p_3 & = \alpha T + \beta S - P (\alpha + \beta)
            \label{eqn:condition_for_tilde_p3}\\
    \tilde p_4 & = \alpha P + \beta P - P (\alpha + \beta)
            \label{eqn:condition_for_tilde_p4}
\end{align}

Equation (\ref{eqn:condition_for_tilde_p4}) ensures that \(p_4=\tilde p_4=0\).
Equations (\ref{eqn:condition_for_tilde_p1}-\ref{eqn:condition_for_tilde_p3})
can be used to eliminate \(\alpha, \beta\), giving:

\begin{equation}\label{eqn:planar_definition_of_extortion}
    \tilde p_1 = \frac{(R - P)(\tilde p_2 + \tilde p_3)}{S + T - 2P}
\end{equation}

with:

\begin{equation}\label{eqn:definition_of_chi}
    \chi = \frac{\tilde p_2 (P - T) + \tilde p_3 (S - P)}
                {\tilde p_2 (P - S) + \tilde p_3 (T - P)}
\end{equation}

Given a strategy \(p\in\mathbb{R}^{4\times 1}\) equations
(\ref{eqn:condition_for_tilde_p4}), (\ref{eqn:planar_definition_of_extortion}-\ref{eqn:definition_of_chi}) can be used to check if
a strategy is extortionate. The conditions correspond to:

\begin{align}
    p_1 & = \frac{(R-P)(p_2 + p_3) - R + T + S - P}{S + T - 2P}
     \label{eqn:condition_for_p1}\\
    p_4 & = 0 \label{eqn:condition_for_p4}\\
    1 & > p_2 + p_3\label{eqn:condition_for_chi}
\end{align}

The algebraic steps necessary to prove these results are available in the
supporting materials.

All extortionate strategies reside on a triangular (\ref{eqn:condition_for_chi})
plane (\ref{eqn:condition_for_p1}) in 3 dimensions (\ref{eqn:condition_for_p4}).
Using this formulation it can be seen that a necessary (but not sufficient)
condition for an extortionate strategy is that it cooperates on average less
than 50\% of the time when in a state of disagreement with the opponent.

As an example, consider the known extortionate strategy \(p=(8 / 9, 1 / 2, 1 /
3, 0)\) from~\cite{Stewart2012} which is referred to as \texttt{Extort-2}. In
this case, for the standard values of \((R, T, S, P)\) constraint
(\ref{eqn:condition_for_p1}) corresponds to:

\begin{equation}
    p_1 = \frac{2(p_2 + p_3) + 1}{3}
\end{equation}

It is clear that in this case all constraints hold.

This approach could in fact be used to confirm that a given strategy is acting
in an extortionate manner even if it is not a memory one strategy. However, in
practice, if a closed form for \(p\) is not known, then due to measurement
and/or numerical error this would not work.

This problem can be written in the following linear algebraic form where
\(x=(\alpha, \beta)\)
and \(p^*=(\tilde p_1 - 1, tilde_2 - 1, p_3)\):

\begin{equation}\label{eqn:linear_algebraic_equation_for_p}
    Cx= p^*
\end{equation}

\(C\) corresponds to equations
(\ref{eqn:condition_for_tilde_p1}-\ref{eqn:condition_for_tilde_p3}) and is
given by:

\begin{equation}\label{eqn:definition_of_C}
    C =
    \begin{bmatrix}
        R - P & R- P \\
        S - P & T- P \\
        T - P & S- P \\
    \end{bmatrix}
\end{equation}

Note that in general, equation (\ref{eqn:linear_algebraic_equation_for_p}) will
not necessarily have a solution. From the Rouch\'{e}-Capelli theorem if there is
a solution it is unique as \(\text{rank}(C)=2\) which is the dimension of the
variable \(x\). The best fitting \(x\) is found by minimizing:

\begin{equation}\label{eqn:r_squared}
    \text{SSError} = \|C x- p^*\|_2^2 = \sum_{i=1}^{3}\left((C\bar x)_i-p_i^*\right)^2
\end{equation}

Note that \(\text{SSError}\), which is the square of the Frobenius
norm~\cite{Golub2013}, becomes a measure of how close a strategy is to being an
extortionate strategy. Suspicion
of extortion then corresponds to a threshold on \(\text{SSError}\).

By observing interactions (human or otherwise), their memory one representation
can be inferred and this approach can be used to recognise extortionate
behaviour. The notion of comparing theoretic and actual plays of the IPD is not
novel, see for example~\cite{Rand2013}. Immediately it is noted that if the
environment is noisy~\cite{Wu1995} then no strategy can be considered to be
extortionate as \(p_4>0\).

In the next section, this idea will be illustrated by observing the interactions
that take place in a computer based tournament of the IPD\@.

\section{Numerical experiments}\label{sec:numerical-experiments}

In~\cite{Stewart2012} results from a tournament with
\input{./assets/tex/number_of_stewart_plotkin_strategies/main.tex} strategies,
was presented with specific consideration given to ZD strategies. This
tournament is reproduced here using the Axelrod-Python
project~\cite{Knight2016}. To obtain a good measure of the corresponding
transition rates for each strategy all matches have been run for
\input{assets/tex/number_of_turns/main.tex} turns and every match has been
repeated \input{assets/tex/number_of_repetitions/main.tex} times. All of this
interaction data is available at~\cite{vincent_knight_2018_1297075}. A good
match between the inferred Markov chain and the state distribution of the actual
interactions has been verified. Data for this is presented in the supplementary
materials.

Figure~\ref{fig:SSError_overall_in_stewart_plotkin} shows the \(\text{SSError}\)
values for all the strategies in the tournament, as reported
in~\cite{Stewart2012} the extortionate strategy (which has an expected
\(\text{SSError}\) approximately 0) gains a large number of wins.

\begin{figure}[!htbp]
    \centering
    \includegraphics[width=.8\textwidth]{./assets/img/SSError_overall_in_stewart_plotkin/main.pdf}
    \caption{\(\text{SSError}\) and state probabilities for the strategies
        of~\cite{Stewart2012}, ordered both by number of wins and overall score.
        Note that \(P(DC)\) is not shown as it corresponds to the transpose of
        \(P(CD)\). Cooperator and Defector are omitted as they do not visit all
        the states.}
    \label{fig:SSError_overall_in_stewart_plotkin}
\end{figure}

Here, the work of~\cite{Stewart2012} is extended by investigating a tournament
with \input{assets/tex/number_of_full_strategies/main.tex}
strategies.

The results of this analysis are shown in
Figure~\ref{fig:SSError_and_probabilities_in_full}. The top ranking strategies
by number of wins seem to be extortionate (but not against all strategies) and
it can be seen that a small sub group of strategies achieve mutual defection.
All the top ranking strategies according to score achieve mutual cooperation and
do not extort each other, however they
\textbf{do} exhibit extortionate behaviour towards a number of the lower ranking
strategies.

\begin{figure}[!htbp]
    \centering
    \includegraphics[width=.8\textwidth]{./assets/img/SSError_and_probabilities_in_full/main.pdf}
    \caption{\(\text{SSError}\) for the strategies for the full tournament. Only
    strategy interactions for which \(p_4=0\) and \(\chi>1\) are displayed.}
    \label{fig:SSError_and_probabilities_in_full}
\end{figure}

\section{Conclusion}\label{sec:conclusion}

This work defines an approach to measure whether or not a player is playing a
strategy that corresponds to an extortionate strategy as defined
in~\cite{Press2012}: a mathematical model for suspicion. Indeed, all
extortionate strategies have been
 classified as lying on a triangular plane.
This rigorous classification fails to be robust to small measurement error, thus
a statistical approach is proposed.
This is done through a linear algebraic approach for approximating the solution
of a linear system. Using this, a large number of pairwise interactions is
simulated and in fact very few strategies are found to act extortionately.

The work of~\cite{Press2012}, whilst showing that a clever approach to taking
advantage of another memory one strategy exists: this is incomplete. Whilst the
elegance of this result is very attractive, just as the simplicity of the
victory of Tit For Tat in Axelrod's original tournaments was, it is incomplete.
Extortionate strategies achieve a high number of wins but they do not
achieve a high score which corresponds to the fitness landscape in an
evolutionary sense. From the large number of interactions a payoff matrix \(S\)
can be measured where \(S_{ij}\) denotes the score (using standard values of
\((R, S, T, P) = (3, 0, 5, 1)\)) of the \(i\)th strategy
against the \(j\)th strategy. Using this, the replicator equation
describes the evolution of the system based on a population density fitness
function:

\begin{equation}\label{eqn:replicator_dynamics}
    \frac{dx}{dt} = x(S-x^TS x)
\end{equation}

Equation (\ref{eqn:replicator_dynamics}) is solved numerically through an
integration technique described in~\cite{Petzold1983} and
Figure~\ref{fig:replicator_dynamics} shows the evolution of the distribution of
the system: the various strategies are ranked by scores. It is clear to see that
only the high ranking strategies survive the evolutionary process (in fact,
only \input{./assets/img/replicator_dynamics/main.tex}
have a final distribution greater than \(10 ^ {-2}\)). This confirms the
findings of~\cite{Moran1707} in which sophisticated strategies resist
evolutionary invasion of shorter memory strategies. Recalling
Figure~\ref{fig:SSError_and_probabilities_in_full} this demonstrates that:

\begin{itemize}
    \item Cooperation emerges through the evolutionary process: the high scoring
        strategies do not exhibit extortionate behaviour towards each other.
    \item Extortionate strategies do not survive the evolutionary process.
\end{itemize}

\begin{figure}[!htbp]
    \centering
    \includegraphics[width=.8\textwidth]{./assets/img/replicator_dynamics/main.pdf}
    \caption{Numerical simulation of the replicator equation
    (\ref{eqn:replicator_dynamics}): strategies are ordered by score, only the strategies with a high score survive the evolutionary process.}
    \label{fig:replicator_dynamics}
\end{figure}

This work can be used to classify plays of the IPD\@: data can be collected from
actual interactions (in lab or in the field). Furthermore, this allows for a
classification method similar to the notion of fingerprinting presented
in~\cite{Ashlock2008}. Trained strategies can potentially be classified as
extortionate or not or it could be possible to even constrain the reinforcement
learning approaches that are becoming prevalent in the literature.
Alternatively, this mathematical approach for recognising extortion could be
used in sophisticated strategies to defend against invasion. Arguably, some of
the strategies considered here exhibit this behaviour, indeed as described
in~\cite{Harper2017}, the top ranking strategies in the full tournament are
obtained using evolutionary reinforcement learning techniques, thus, suspicion
of extortionate behaviour could in fact be an evolutionary trait.

\section*{Acknowledgements}

The following open source software libraries were used in this research:

\begin{itemize}
    \item The Axelrod ~\cite{Knight2016, Knight2018} library (IPD strategies and
        tournaments).
    \item The sympy library~\cite{Meurer2017} (verification of all symbolic
        calculations).
    \item The matplotlib~\cite{Droettboom2018} library (visualisation).
    \item The pandas~\cite{Structures2010}, dask~\cite{Dask2016} and
        NumPy~\cite{Oliphant2015} libraries (data manipulation).
    \item The SciPy~\cite{Jones2001} library (numerical integration of the
        replicator equation).
\end{itemize}

This work was performed using the computational facilities of the Advanced
Research Computing @ Cardiff (ARCCA) Division, Cardiff University.

\printbibliography

\newpage
\section*{Supplementary materials}

\includepdf{assets/pdf/proof_of_form_of_extortionate_strategies/main.pdf}

\newpage

Using the pair wise interactions the transition rates \(p,
q\) can be measured and the steady state probabilities inferred and compared to
the actual probabilities of each state.
This is done numerically by computing the singular eigenvector of the
matrix \(A\) \cite{Stewart2009}:

\[
    A =
    \begin{bmatrix}
        p_1 q_1 & p_1 (1 - q_1) & (1 - p_1) q_1 & (1 -p_1) (1 - q_1) \\
        p_2 q_2 & p_2 (1 - q_2) & (1 - p_2) q_2 & (1 -p_2) (1 - q_2) \\
        p_3 q_3 & p_3 (1 - q_3) & (1 - p_3) q_3 & (1 -p_3) (1 - q_3) \\
        p_4 q_4 & p_4 (1 - q_4) & (1 - p_4) q_4 & (1 -p_4) (1 - q_4) \\
    \end{bmatrix}
\]

Figure~\ref{fig:computed_probabilities_vs_theoretic_probabilities} shows a
regression line fitted to every pairwise interaction with a reported
\(\text{SSError}\) value (pairwise interactions with missing states were
omitted). This serves to validate the approach: a part from some edge cases the
relationship is consistent.

\begin{figure}[!htbp]
    \centering
    \includegraphics[width=.8\textwidth]{./assets/img/computed_probabilities_vs_theoretic_probabilities/main.pdf}
    \caption{The
        relationship between the steady state probabilities inferred from the
        measured transitions and the actual steady state probabilities. A linear
        regression line is included validating the approach.}
    \label{fig:computed_probabilities_vs_theoretic_probabilities}
\end{figure}


\end{document}
 turns and every match has been
repeated \documentclass[a4paper]{article}

\usepackage{amsmath}
\usepackage{amssymb}
\usepackage[margin=1.5cm,
            includefoot,
            footskip=30pt]{geometry}
\usepackage{layout}
\usepackage{graphicx}
\usepackage{subcaption}

\usepackage{biblatex}
\usepackage{pdfpages}

\bibliography{main.bib}

\title{Suspicion: Recognising and evaluating the effectiveness
       of extortion in the Iterated Prisoner's Dilemma}
\author{Vincent A. Knight \and Nikoleta E. Glynatsi}
\date{\today}



\begin{document}

\maketitle

\begin{abstract}
    The Iterated Prisoner's Dilemma is a model for rational and evolutionary
    interactive behaviour. It has applications both in the study of human social
    behaviour as well as in biology.
    It is used to understand when and how a rational individual might
    accept an immediate cost to their own utility for the direct benefit of
    another.

    Much attention has been given to a class of strategies called
    Zero Determinant strategies. It has been theoretically shown that these
    strategies can ``extort'' any player.

    In this work, an approach to identify if observed strategies are playing in
    an extortionate way is described. Furthermore, experimental analysis of
    a large tournament with \input{assets/tex/number_of_full_strategies/main.tex}
    strategies is considered. In this setting
    the most highly performing strategies do not play in an extortionate way
    against each other but do against lower performing strategies.
    This suggests that whilst the theory of Zero Determinant strategies
    indicates that memory is not of fundamental importance to the evolution of
    cooperative behaviour, this is incomplete.
\end{abstract}

\section{Introduction}\label{sec:introduction}

Agent based game theoretic models have become a stalwart of the underpinning
mathematics of interactive behaviours. One of the major pieces of work
in this area is the pair of original computer tournaments run by Robert
Axelrod~\cite{Axelrod1980, Axelrod1980a}. These tournaments pitted submitted
computer strategies against each other in plays of the Iterated Prisoner's
Dilemma. A common game where agents can choose to pay a slight cost to their
immediate utility in the hope of building a reputation. This has been used in
economic and evolutionary game theory to understand the evolution of cooperative
behaviour.

Recently, a class of strategies was described in~\cite{Press2012} that can
provably extort any given opponent. In~\cite{Hilbe2013, Moran1707} some
questions have already been asked about the true effectiveness of these
strategies in an evolutionary setting. Here another question is asked: is it
possible to recognise this extortionate behaviour? A mathematical procedure for
suspicion is presented: in the same way that the continued actions of an
extortionate individual might raise suspicion.

This work makes use of the Axelrod Python library~\cite{Knight2018, Knight2016}
with a large number of Prisoner Dilemma strategies available to give an
extensive numerical example of the ideas presented.  The approach is presented
in Section~\ref{sec:delta-zd-strategies}.  All of the code and data discussed
in Section~\ref{sec:numerical-experiments} is open sourced, archived and
written according to best scientific principles~\cite{Wilson2014}. The data
archive can be found at~\cite{vincent_knight_2018_1297075}.

\section{Recognising Extortion}\label{sec:delta-zd-strategies}

In~\cite{Press2012}, given a match between 2 memory-one strategies, the concept
of Zero Determinant (ZD) strategies is introduced. The main result of that paper
shows that given two memory one players \(p, q\in\mathbb{R}^4\) a linear
relationship between the players' scores could be forced by one of the players.

Using the notation of~\cite{Press2012}, assuming the utilities for player \(p\)
are given by \(S_x=(R, S, T, P)\) and for player \(q\) by \(S_y=(R, T, S, P)\)
and that the stationary scores of each player is given by \(S_X\) and \(S_Y\)
respectively. The main result of~\cite{Press2012} is that if

\begin{equation}\label{eqn:linear_relationship_for_p}
    \tilde p=\alpha S_x + \beta S_y + \gamma
\end{equation}

or

\begin{equation}\label{eqn:linear_relationship_for_q}
    \tilde q=\alpha S_x + \beta S_y + \gamma
\end{equation}

where \(\tilde p = (1 - p_1, 1 - p_2, p_3, p_4)\) and
\(\tilde q = (1 - q_1, 1 - q_2, q_3, q_4)\) then:

\begin{equation}
    \alpha S_X + \beta S_Y + \gamma = 0
\end{equation}

In~\cite{Press2012} a particular type of ZD strategy is defined: extortionate
strategies. If:

\begin{equation}\label{eqn:constraint_for_extortion}
    \gamma = - P(\alpha + \beta)
\end{equation}

then the player can ensure they get a score \(\chi\) times
larger than the opponent. This extortion coefficient is given by:

\begin{equation}\label{eqn:definition_of_chi}
    \chi=\frac{-\beta}{\alpha}
\end{equation}

Thus, if (\ref{eqn:constraint_for_extortion}) holds and \(\chi >1\) a player is
said to extort their opponent.
Here, the reverse problem is considered: given a
\(p\in\mathbb{R}^4\) how does one identify \(\alpha, \beta\) if they
exist and is the strategy in fact acting in an extortionate way?

These conditions correspond to:

\begin{align}
    \tilde p_1 & = \alpha R + \beta R - P (\alpha + \beta)
            \label{eqn:condition_for_tilde_p1}\\
    \tilde p_2 & = \alpha S + \beta T - P (\alpha + \beta)
            \label{eqn:condition_for_tilde_p2}\\
    \tilde p_3 & = \alpha T + \beta S - P (\alpha + \beta)
            \label{eqn:condition_for_tilde_p3}\\
    \tilde p_4 & = \alpha P + \beta P - P (\alpha + \beta)
            \label{eqn:condition_for_tilde_p4}
\end{align}

Equation (\ref{eqn:condition_for_tilde_p4}) ensures that \(p_4=\tilde p_4=0\).
Equations (\ref{eqn:condition_for_tilde_p1}-\ref{eqn:condition_for_tilde_p3})
can be used to eliminate \(\alpha, \beta\), giving:

\begin{equation}\label{eqn:planar_definition_of_extortion}
    \tilde p_1 = \frac{(R - P)(\tilde p_2 + \tilde p_3)}{S + T - 2P}
\end{equation}

with:

\begin{equation}\label{eqn:definition_of_chi}
    \chi = \frac{\tilde p_2 (P - T) + \tilde p_3 (S - P)}
                {\tilde p_2 (P - S) + \tilde p_3 (T - P)}
\end{equation}

Given a strategy \(p\in\mathbb{R}^{4\times 1}\) equations
(\ref{eqn:condition_for_tilde_p4}), (\ref{eqn:planar_definition_of_extortion}-\ref{eqn:definition_of_chi}) can be used to check if
a strategy is extortionate. The conditions correspond to:

\begin{align}
    p_1 & = \frac{(R-P)(p_2 + p_3) - R + T + S - P}{S + T - 2P}
     \label{eqn:condition_for_p1}\\
    p_4 & = 0 \label{eqn:condition_for_p4}\\
    1 & > p_2 + p_3\label{eqn:condition_for_chi}
\end{align}

The algebraic steps necessary to prove these results are available in the
supporting materials.

All extortionate strategies reside on a triangular (\ref{eqn:condition_for_chi})
plane (\ref{eqn:condition_for_p1}) in 3 dimensions (\ref{eqn:condition_for_p4}).
Using this formulation it can be seen that a necessary (but not sufficient)
condition for an extortionate strategy is that it cooperates on average less
than 50\% of the time when in a state of disagreement with the opponent.

As an example, consider the known extortionate strategy \(p=(8 / 9, 1 / 2, 1 /
3, 0)\) from~\cite{Stewart2012} which is referred to as \texttt{Extort-2}. In
this case, for the standard values of \((R, T, S, P)\) constraint
(\ref{eqn:condition_for_p1}) corresponds to:

\begin{equation}
    p_1 = \frac{2(p_2 + p_3) + 1}{3}
\end{equation}

It is clear that in this case all constraints hold.

This approach could in fact be used to confirm that a given strategy is acting
in an extortionate manner even if it is not a memory one strategy. However, in
practice, if a closed form for \(p\) is not known, then due to measurement
and/or numerical error this would not work.

This problem can be written in the following linear algebraic form where
\(x=(\alpha, \beta)\)
and \(p^*=(\tilde p_1 - 1, tilde_2 - 1, p_3)\):

\begin{equation}\label{eqn:linear_algebraic_equation_for_p}
    Cx= p^*
\end{equation}

\(C\) corresponds to equations
(\ref{eqn:condition_for_tilde_p1}-\ref{eqn:condition_for_tilde_p3}) and is
given by:

\begin{equation}\label{eqn:definition_of_C}
    C =
    \begin{bmatrix}
        R - P & R- P \\
        S - P & T- P \\
        T - P & S- P \\
    \end{bmatrix}
\end{equation}

Note that in general, equation (\ref{eqn:linear_algebraic_equation_for_p}) will
not necessarily have a solution. From the Rouch\'{e}-Capelli theorem if there is
a solution it is unique as \(\text{rank}(C)=2\) which is the dimension of the
variable \(x\). The best fitting \(x\) is found by minimizing:

\begin{equation}\label{eqn:r_squared}
    \text{SSError} = \|C x- p^*\|_2^2 = \sum_{i=1}^{3}\left((C\bar x)_i-p_i^*\right)^2
\end{equation}

Note that \(\text{SSError}\), which is the square of the Frobenius
norm~\cite{Golub2013}, becomes a measure of how close a strategy is to being an
extortionate strategy. Suspicion
of extortion then corresponds to a threshold on \(\text{SSError}\).

By observing interactions (human or otherwise), their memory one representation
can be inferred and this approach can be used to recognise extortionate
behaviour. The notion of comparing theoretic and actual plays of the IPD is not
novel, see for example~\cite{Rand2013}. Immediately it is noted that if the
environment is noisy~\cite{Wu1995} then no strategy can be considered to be
extortionate as \(p_4>0\).

In the next section, this idea will be illustrated by observing the interactions
that take place in a computer based tournament of the IPD\@.

\section{Numerical experiments}\label{sec:numerical-experiments}

In~\cite{Stewart2012} results from a tournament with
\input{./assets/tex/number_of_stewart_plotkin_strategies/main.tex} strategies,
was presented with specific consideration given to ZD strategies. This
tournament is reproduced here using the Axelrod-Python
project~\cite{Knight2016}. To obtain a good measure of the corresponding
transition rates for each strategy all matches have been run for
\input{assets/tex/number_of_turns/main.tex} turns and every match has been
repeated \input{assets/tex/number_of_repetitions/main.tex} times. All of this
interaction data is available at~\cite{vincent_knight_2018_1297075}. A good
match between the inferred Markov chain and the state distribution of the actual
interactions has been verified. Data for this is presented in the supplementary
materials.

Figure~\ref{fig:SSError_overall_in_stewart_plotkin} shows the \(\text{SSError}\)
values for all the strategies in the tournament, as reported
in~\cite{Stewart2012} the extortionate strategy (which has an expected
\(\text{SSError}\) approximately 0) gains a large number of wins.

\begin{figure}[!htbp]
    \centering
    \includegraphics[width=.8\textwidth]{./assets/img/SSError_overall_in_stewart_plotkin/main.pdf}
    \caption{\(\text{SSError}\) and state probabilities for the strategies
        of~\cite{Stewart2012}, ordered both by number of wins and overall score.
        Note that \(P(DC)\) is not shown as it corresponds to the transpose of
        \(P(CD)\). Cooperator and Defector are omitted as they do not visit all
        the states.}
    \label{fig:SSError_overall_in_stewart_plotkin}
\end{figure}

Here, the work of~\cite{Stewart2012} is extended by investigating a tournament
with \input{assets/tex/number_of_full_strategies/main.tex}
strategies.

The results of this analysis are shown in
Figure~\ref{fig:SSError_and_probabilities_in_full}. The top ranking strategies
by number of wins seem to be extortionate (but not against all strategies) and
it can be seen that a small sub group of strategies achieve mutual defection.
All the top ranking strategies according to score achieve mutual cooperation and
do not extort each other, however they
\textbf{do} exhibit extortionate behaviour towards a number of the lower ranking
strategies.

\begin{figure}[!htbp]
    \centering
    \includegraphics[width=.8\textwidth]{./assets/img/SSError_and_probabilities_in_full/main.pdf}
    \caption{\(\text{SSError}\) for the strategies for the full tournament. Only
    strategy interactions for which \(p_4=0\) and \(\chi>1\) are displayed.}
    \label{fig:SSError_and_probabilities_in_full}
\end{figure}

\section{Conclusion}\label{sec:conclusion}

This work defines an approach to measure whether or not a player is playing a
strategy that corresponds to an extortionate strategy as defined
in~\cite{Press2012}: a mathematical model for suspicion. Indeed, all
extortionate strategies have been
 classified as lying on a triangular plane.
This rigorous classification fails to be robust to small measurement error, thus
a statistical approach is proposed.
This is done through a linear algebraic approach for approximating the solution
of a linear system. Using this, a large number of pairwise interactions is
simulated and in fact very few strategies are found to act extortionately.

The work of~\cite{Press2012}, whilst showing that a clever approach to taking
advantage of another memory one strategy exists: this is incomplete. Whilst the
elegance of this result is very attractive, just as the simplicity of the
victory of Tit For Tat in Axelrod's original tournaments was, it is incomplete.
Extortionate strategies achieve a high number of wins but they do not
achieve a high score which corresponds to the fitness landscape in an
evolutionary sense. From the large number of interactions a payoff matrix \(S\)
can be measured where \(S_{ij}\) denotes the score (using standard values of
\((R, S, T, P) = (3, 0, 5, 1)\)) of the \(i\)th strategy
against the \(j\)th strategy. Using this, the replicator equation
describes the evolution of the system based on a population density fitness
function:

\begin{equation}\label{eqn:replicator_dynamics}
    \frac{dx}{dt} = x(S-x^TS x)
\end{equation}

Equation (\ref{eqn:replicator_dynamics}) is solved numerically through an
integration technique described in~\cite{Petzold1983} and
Figure~\ref{fig:replicator_dynamics} shows the evolution of the distribution of
the system: the various strategies are ranked by scores. It is clear to see that
only the high ranking strategies survive the evolutionary process (in fact,
only \input{./assets/img/replicator_dynamics/main.tex}
have a final distribution greater than \(10 ^ {-2}\)). This confirms the
findings of~\cite{Moran1707} in which sophisticated strategies resist
evolutionary invasion of shorter memory strategies. Recalling
Figure~\ref{fig:SSError_and_probabilities_in_full} this demonstrates that:

\begin{itemize}
    \item Cooperation emerges through the evolutionary process: the high scoring
        strategies do not exhibit extortionate behaviour towards each other.
    \item Extortionate strategies do not survive the evolutionary process.
\end{itemize}

\begin{figure}[!htbp]
    \centering
    \includegraphics[width=.8\textwidth]{./assets/img/replicator_dynamics/main.pdf}
    \caption{Numerical simulation of the replicator equation
    (\ref{eqn:replicator_dynamics}): strategies are ordered by score, only the strategies with a high score survive the evolutionary process.}
    \label{fig:replicator_dynamics}
\end{figure}

This work can be used to classify plays of the IPD\@: data can be collected from
actual interactions (in lab or in the field). Furthermore, this allows for a
classification method similar to the notion of fingerprinting presented
in~\cite{Ashlock2008}. Trained strategies can potentially be classified as
extortionate or not or it could be possible to even constrain the reinforcement
learning approaches that are becoming prevalent in the literature.
Alternatively, this mathematical approach for recognising extortion could be
used in sophisticated strategies to defend against invasion. Arguably, some of
the strategies considered here exhibit this behaviour, indeed as described
in~\cite{Harper2017}, the top ranking strategies in the full tournament are
obtained using evolutionary reinforcement learning techniques, thus, suspicion
of extortionate behaviour could in fact be an evolutionary trait.

\section*{Acknowledgements}

The following open source software libraries were used in this research:

\begin{itemize}
    \item The Axelrod ~\cite{Knight2016, Knight2018} library (IPD strategies and
        tournaments).
    \item The sympy library~\cite{Meurer2017} (verification of all symbolic
        calculations).
    \item The matplotlib~\cite{Droettboom2018} library (visualisation).
    \item The pandas~\cite{Structures2010}, dask~\cite{Dask2016} and
        NumPy~\cite{Oliphant2015} libraries (data manipulation).
    \item The SciPy~\cite{Jones2001} library (numerical integration of the
        replicator equation).
\end{itemize}

This work was performed using the computational facilities of the Advanced
Research Computing @ Cardiff (ARCCA) Division, Cardiff University.

\printbibliography

\newpage
\section*{Supplementary materials}

\includepdf{assets/pdf/proof_of_form_of_extortionate_strategies/main.pdf}

\newpage

Using the pair wise interactions the transition rates \(p,
q\) can be measured and the steady state probabilities inferred and compared to
the actual probabilities of each state.
This is done numerically by computing the singular eigenvector of the
matrix \(A\) \cite{Stewart2009}:

\[
    A =
    \begin{bmatrix}
        p_1 q_1 & p_1 (1 - q_1) & (1 - p_1) q_1 & (1 -p_1) (1 - q_1) \\
        p_2 q_2 & p_2 (1 - q_2) & (1 - p_2) q_2 & (1 -p_2) (1 - q_2) \\
        p_3 q_3 & p_3 (1 - q_3) & (1 - p_3) q_3 & (1 -p_3) (1 - q_3) \\
        p_4 q_4 & p_4 (1 - q_4) & (1 - p_4) q_4 & (1 -p_4) (1 - q_4) \\
    \end{bmatrix}
\]

Figure~\ref{fig:computed_probabilities_vs_theoretic_probabilities} shows a
regression line fitted to every pairwise interaction with a reported
\(\text{SSError}\) value (pairwise interactions with missing states were
omitted). This serves to validate the approach: a part from some edge cases the
relationship is consistent.

\begin{figure}[!htbp]
    \centering
    \includegraphics[width=.8\textwidth]{./assets/img/computed_probabilities_vs_theoretic_probabilities/main.pdf}
    \caption{The
        relationship between the steady state probabilities inferred from the
        measured transitions and the actual steady state probabilities. A linear
        regression line is included validating the approach.}
    \label{fig:computed_probabilities_vs_theoretic_probabilities}
\end{figure}


\end{document}
 times. All of this
interaction data is available at~\cite{vincent_knight_2018_1297075}. A good
match between the inferred Markov chain and the state distribution of the actual
interactions has been verified. Data for this is presented in the supplementary
materials.

Figure~\ref{fig:SSError_overall_in_stewart_plotkin} shows the \(\text{SSError}\)
values for all the strategies in the tournament, as reported
in~\cite{Stewart2012} the extortionate strategy (which has an expected
\(\text{SSError}\) approximately 0) gains a large number of wins.

\begin{figure}[!htbp]
    \centering
    \includegraphics[width=.8\textwidth]{./assets/img/SSError_overall_in_stewart_plotkin/main.pdf}
    \caption{\(\text{SSError}\) and state probabilities for the strategies
        of~\cite{Stewart2012}, ordered both by number of wins and overall score.
        Note that \(P(DC)\) is not shown as it corresponds to the transpose of
        \(P(CD)\). Cooperator and Defector are omitted as they do not visit all
        the states.}
    \label{fig:SSError_overall_in_stewart_plotkin}
\end{figure}

Here, the work of~\cite{Stewart2012} is extended by investigating a tournament
with \documentclass[a4paper]{article}

\usepackage{amsmath}
\usepackage{amssymb}
\usepackage[margin=1.5cm,
            includefoot,
            footskip=30pt]{geometry}
\usepackage{layout}
\usepackage{graphicx}
\usepackage{subcaption}

\usepackage{biblatex}
\usepackage{pdfpages}

\bibliography{main.bib}

\title{Suspicion: Recognising and evaluating the effectiveness
       of extortion in the Iterated Prisoner's Dilemma}
\author{Vincent A. Knight \and Nikoleta E. Glynatsi}
\date{\today}



\begin{document}

\maketitle

\begin{abstract}
    The Iterated Prisoner's Dilemma is a model for rational and evolutionary
    interactive behaviour. It has applications both in the study of human social
    behaviour as well as in biology.
    It is used to understand when and how a rational individual might
    accept an immediate cost to their own utility for the direct benefit of
    another.

    Much attention has been given to a class of strategies called
    Zero Determinant strategies. It has been theoretically shown that these
    strategies can ``extort'' any player.

    In this work, an approach to identify if observed strategies are playing in
    an extortionate way is described. Furthermore, experimental analysis of
    a large tournament with \input{assets/tex/number_of_full_strategies/main.tex}
    strategies is considered. In this setting
    the most highly performing strategies do not play in an extortionate way
    against each other but do against lower performing strategies.
    This suggests that whilst the theory of Zero Determinant strategies
    indicates that memory is not of fundamental importance to the evolution of
    cooperative behaviour, this is incomplete.
\end{abstract}

\section{Introduction}\label{sec:introduction}

Agent based game theoretic models have become a stalwart of the underpinning
mathematics of interactive behaviours. One of the major pieces of work
in this area is the pair of original computer tournaments run by Robert
Axelrod~\cite{Axelrod1980, Axelrod1980a}. These tournaments pitted submitted
computer strategies against each other in plays of the Iterated Prisoner's
Dilemma. A common game where agents can choose to pay a slight cost to their
immediate utility in the hope of building a reputation. This has been used in
economic and evolutionary game theory to understand the evolution of cooperative
behaviour.

Recently, a class of strategies was described in~\cite{Press2012} that can
provably extort any given opponent. In~\cite{Hilbe2013, Moran1707} some
questions have already been asked about the true effectiveness of these
strategies in an evolutionary setting. Here another question is asked: is it
possible to recognise this extortionate behaviour? A mathematical procedure for
suspicion is presented: in the same way that the continued actions of an
extortionate individual might raise suspicion.

This work makes use of the Axelrod Python library~\cite{Knight2018, Knight2016}
with a large number of Prisoner Dilemma strategies available to give an
extensive numerical example of the ideas presented.  The approach is presented
in Section~\ref{sec:delta-zd-strategies}.  All of the code and data discussed
in Section~\ref{sec:numerical-experiments} is open sourced, archived and
written according to best scientific principles~\cite{Wilson2014}. The data
archive can be found at~\cite{vincent_knight_2018_1297075}.

\section{Recognising Extortion}\label{sec:delta-zd-strategies}

In~\cite{Press2012}, given a match between 2 memory-one strategies, the concept
of Zero Determinant (ZD) strategies is introduced. The main result of that paper
shows that given two memory one players \(p, q\in\mathbb{R}^4\) a linear
relationship between the players' scores could be forced by one of the players.

Using the notation of~\cite{Press2012}, assuming the utilities for player \(p\)
are given by \(S_x=(R, S, T, P)\) and for player \(q\) by \(S_y=(R, T, S, P)\)
and that the stationary scores of each player is given by \(S_X\) and \(S_Y\)
respectively. The main result of~\cite{Press2012} is that if

\begin{equation}\label{eqn:linear_relationship_for_p}
    \tilde p=\alpha S_x + \beta S_y + \gamma
\end{equation}

or

\begin{equation}\label{eqn:linear_relationship_for_q}
    \tilde q=\alpha S_x + \beta S_y + \gamma
\end{equation}

where \(\tilde p = (1 - p_1, 1 - p_2, p_3, p_4)\) and
\(\tilde q = (1 - q_1, 1 - q_2, q_3, q_4)\) then:

\begin{equation}
    \alpha S_X + \beta S_Y + \gamma = 0
\end{equation}

In~\cite{Press2012} a particular type of ZD strategy is defined: extortionate
strategies. If:

\begin{equation}\label{eqn:constraint_for_extortion}
    \gamma = - P(\alpha + \beta)
\end{equation}

then the player can ensure they get a score \(\chi\) times
larger than the opponent. This extortion coefficient is given by:

\begin{equation}\label{eqn:definition_of_chi}
    \chi=\frac{-\beta}{\alpha}
\end{equation}

Thus, if (\ref{eqn:constraint_for_extortion}) holds and \(\chi >1\) a player is
said to extort their opponent.
Here, the reverse problem is considered: given a
\(p\in\mathbb{R}^4\) how does one identify \(\alpha, \beta\) if they
exist and is the strategy in fact acting in an extortionate way?

These conditions correspond to:

\begin{align}
    \tilde p_1 & = \alpha R + \beta R - P (\alpha + \beta)
            \label{eqn:condition_for_tilde_p1}\\
    \tilde p_2 & = \alpha S + \beta T - P (\alpha + \beta)
            \label{eqn:condition_for_tilde_p2}\\
    \tilde p_3 & = \alpha T + \beta S - P (\alpha + \beta)
            \label{eqn:condition_for_tilde_p3}\\
    \tilde p_4 & = \alpha P + \beta P - P (\alpha + \beta)
            \label{eqn:condition_for_tilde_p4}
\end{align}

Equation (\ref{eqn:condition_for_tilde_p4}) ensures that \(p_4=\tilde p_4=0\).
Equations (\ref{eqn:condition_for_tilde_p1}-\ref{eqn:condition_for_tilde_p3})
can be used to eliminate \(\alpha, \beta\), giving:

\begin{equation}\label{eqn:planar_definition_of_extortion}
    \tilde p_1 = \frac{(R - P)(\tilde p_2 + \tilde p_3)}{S + T - 2P}
\end{equation}

with:

\begin{equation}\label{eqn:definition_of_chi}
    \chi = \frac{\tilde p_2 (P - T) + \tilde p_3 (S - P)}
                {\tilde p_2 (P - S) + \tilde p_3 (T - P)}
\end{equation}

Given a strategy \(p\in\mathbb{R}^{4\times 1}\) equations
(\ref{eqn:condition_for_tilde_p4}), (\ref{eqn:planar_definition_of_extortion}-\ref{eqn:definition_of_chi}) can be used to check if
a strategy is extortionate. The conditions correspond to:

\begin{align}
    p_1 & = \frac{(R-P)(p_2 + p_3) - R + T + S - P}{S + T - 2P}
     \label{eqn:condition_for_p1}\\
    p_4 & = 0 \label{eqn:condition_for_p4}\\
    1 & > p_2 + p_3\label{eqn:condition_for_chi}
\end{align}

The algebraic steps necessary to prove these results are available in the
supporting materials.

All extortionate strategies reside on a triangular (\ref{eqn:condition_for_chi})
plane (\ref{eqn:condition_for_p1}) in 3 dimensions (\ref{eqn:condition_for_p4}).
Using this formulation it can be seen that a necessary (but not sufficient)
condition for an extortionate strategy is that it cooperates on average less
than 50\% of the time when in a state of disagreement with the opponent.

As an example, consider the known extortionate strategy \(p=(8 / 9, 1 / 2, 1 /
3, 0)\) from~\cite{Stewart2012} which is referred to as \texttt{Extort-2}. In
this case, for the standard values of \((R, T, S, P)\) constraint
(\ref{eqn:condition_for_p1}) corresponds to:

\begin{equation}
    p_1 = \frac{2(p_2 + p_3) + 1}{3}
\end{equation}

It is clear that in this case all constraints hold.

This approach could in fact be used to confirm that a given strategy is acting
in an extortionate manner even if it is not a memory one strategy. However, in
practice, if a closed form for \(p\) is not known, then due to measurement
and/or numerical error this would not work.

This problem can be written in the following linear algebraic form where
\(x=(\alpha, \beta)\)
and \(p^*=(\tilde p_1 - 1, tilde_2 - 1, p_3)\):

\begin{equation}\label{eqn:linear_algebraic_equation_for_p}
    Cx= p^*
\end{equation}

\(C\) corresponds to equations
(\ref{eqn:condition_for_tilde_p1}-\ref{eqn:condition_for_tilde_p3}) and is
given by:

\begin{equation}\label{eqn:definition_of_C}
    C =
    \begin{bmatrix}
        R - P & R- P \\
        S - P & T- P \\
        T - P & S- P \\
    \end{bmatrix}
\end{equation}

Note that in general, equation (\ref{eqn:linear_algebraic_equation_for_p}) will
not necessarily have a solution. From the Rouch\'{e}-Capelli theorem if there is
a solution it is unique as \(\text{rank}(C)=2\) which is the dimension of the
variable \(x\). The best fitting \(x\) is found by minimizing:

\begin{equation}\label{eqn:r_squared}
    \text{SSError} = \|C x- p^*\|_2^2 = \sum_{i=1}^{3}\left((C\bar x)_i-p_i^*\right)^2
\end{equation}

Note that \(\text{SSError}\), which is the square of the Frobenius
norm~\cite{Golub2013}, becomes a measure of how close a strategy is to being an
extortionate strategy. Suspicion
of extortion then corresponds to a threshold on \(\text{SSError}\).

By observing interactions (human or otherwise), their memory one representation
can be inferred and this approach can be used to recognise extortionate
behaviour. The notion of comparing theoretic and actual plays of the IPD is not
novel, see for example~\cite{Rand2013}. Immediately it is noted that if the
environment is noisy~\cite{Wu1995} then no strategy can be considered to be
extortionate as \(p_4>0\).

In the next section, this idea will be illustrated by observing the interactions
that take place in a computer based tournament of the IPD\@.

\section{Numerical experiments}\label{sec:numerical-experiments}

In~\cite{Stewart2012} results from a tournament with
\input{./assets/tex/number_of_stewart_plotkin_strategies/main.tex} strategies,
was presented with specific consideration given to ZD strategies. This
tournament is reproduced here using the Axelrod-Python
project~\cite{Knight2016}. To obtain a good measure of the corresponding
transition rates for each strategy all matches have been run for
\input{assets/tex/number_of_turns/main.tex} turns and every match has been
repeated \input{assets/tex/number_of_repetitions/main.tex} times. All of this
interaction data is available at~\cite{vincent_knight_2018_1297075}. A good
match between the inferred Markov chain and the state distribution of the actual
interactions has been verified. Data for this is presented in the supplementary
materials.

Figure~\ref{fig:SSError_overall_in_stewart_plotkin} shows the \(\text{SSError}\)
values for all the strategies in the tournament, as reported
in~\cite{Stewart2012} the extortionate strategy (which has an expected
\(\text{SSError}\) approximately 0) gains a large number of wins.

\begin{figure}[!htbp]
    \centering
    \includegraphics[width=.8\textwidth]{./assets/img/SSError_overall_in_stewart_plotkin/main.pdf}
    \caption{\(\text{SSError}\) and state probabilities for the strategies
        of~\cite{Stewart2012}, ordered both by number of wins and overall score.
        Note that \(P(DC)\) is not shown as it corresponds to the transpose of
        \(P(CD)\). Cooperator and Defector are omitted as they do not visit all
        the states.}
    \label{fig:SSError_overall_in_stewart_plotkin}
\end{figure}

Here, the work of~\cite{Stewart2012} is extended by investigating a tournament
with \input{assets/tex/number_of_full_strategies/main.tex}
strategies.

The results of this analysis are shown in
Figure~\ref{fig:SSError_and_probabilities_in_full}. The top ranking strategies
by number of wins seem to be extortionate (but not against all strategies) and
it can be seen that a small sub group of strategies achieve mutual defection.
All the top ranking strategies according to score achieve mutual cooperation and
do not extort each other, however they
\textbf{do} exhibit extortionate behaviour towards a number of the lower ranking
strategies.

\begin{figure}[!htbp]
    \centering
    \includegraphics[width=.8\textwidth]{./assets/img/SSError_and_probabilities_in_full/main.pdf}
    \caption{\(\text{SSError}\) for the strategies for the full tournament. Only
    strategy interactions for which \(p_4=0\) and \(\chi>1\) are displayed.}
    \label{fig:SSError_and_probabilities_in_full}
\end{figure}

\section{Conclusion}\label{sec:conclusion}

This work defines an approach to measure whether or not a player is playing a
strategy that corresponds to an extortionate strategy as defined
in~\cite{Press2012}: a mathematical model for suspicion. Indeed, all
extortionate strategies have been
 classified as lying on a triangular plane.
This rigorous classification fails to be robust to small measurement error, thus
a statistical approach is proposed.
This is done through a linear algebraic approach for approximating the solution
of a linear system. Using this, a large number of pairwise interactions is
simulated and in fact very few strategies are found to act extortionately.

The work of~\cite{Press2012}, whilst showing that a clever approach to taking
advantage of another memory one strategy exists: this is incomplete. Whilst the
elegance of this result is very attractive, just as the simplicity of the
victory of Tit For Tat in Axelrod's original tournaments was, it is incomplete.
Extortionate strategies achieve a high number of wins but they do not
achieve a high score which corresponds to the fitness landscape in an
evolutionary sense. From the large number of interactions a payoff matrix \(S\)
can be measured where \(S_{ij}\) denotes the score (using standard values of
\((R, S, T, P) = (3, 0, 5, 1)\)) of the \(i\)th strategy
against the \(j\)th strategy. Using this, the replicator equation
describes the evolution of the system based on a population density fitness
function:

\begin{equation}\label{eqn:replicator_dynamics}
    \frac{dx}{dt} = x(S-x^TS x)
\end{equation}

Equation (\ref{eqn:replicator_dynamics}) is solved numerically through an
integration technique described in~\cite{Petzold1983} and
Figure~\ref{fig:replicator_dynamics} shows the evolution of the distribution of
the system: the various strategies are ranked by scores. It is clear to see that
only the high ranking strategies survive the evolutionary process (in fact,
only \input{./assets/img/replicator_dynamics/main.tex}
have a final distribution greater than \(10 ^ {-2}\)). This confirms the
findings of~\cite{Moran1707} in which sophisticated strategies resist
evolutionary invasion of shorter memory strategies. Recalling
Figure~\ref{fig:SSError_and_probabilities_in_full} this demonstrates that:

\begin{itemize}
    \item Cooperation emerges through the evolutionary process: the high scoring
        strategies do not exhibit extortionate behaviour towards each other.
    \item Extortionate strategies do not survive the evolutionary process.
\end{itemize}

\begin{figure}[!htbp]
    \centering
    \includegraphics[width=.8\textwidth]{./assets/img/replicator_dynamics/main.pdf}
    \caption{Numerical simulation of the replicator equation
    (\ref{eqn:replicator_dynamics}): strategies are ordered by score, only the strategies with a high score survive the evolutionary process.}
    \label{fig:replicator_dynamics}
\end{figure}

This work can be used to classify plays of the IPD\@: data can be collected from
actual interactions (in lab or in the field). Furthermore, this allows for a
classification method similar to the notion of fingerprinting presented
in~\cite{Ashlock2008}. Trained strategies can potentially be classified as
extortionate or not or it could be possible to even constrain the reinforcement
learning approaches that are becoming prevalent in the literature.
Alternatively, this mathematical approach for recognising extortion could be
used in sophisticated strategies to defend against invasion. Arguably, some of
the strategies considered here exhibit this behaviour, indeed as described
in~\cite{Harper2017}, the top ranking strategies in the full tournament are
obtained using evolutionary reinforcement learning techniques, thus, suspicion
of extortionate behaviour could in fact be an evolutionary trait.

\section*{Acknowledgements}

The following open source software libraries were used in this research:

\begin{itemize}
    \item The Axelrod ~\cite{Knight2016, Knight2018} library (IPD strategies and
        tournaments).
    \item The sympy library~\cite{Meurer2017} (verification of all symbolic
        calculations).
    \item The matplotlib~\cite{Droettboom2018} library (visualisation).
    \item The pandas~\cite{Structures2010}, dask~\cite{Dask2016} and
        NumPy~\cite{Oliphant2015} libraries (data manipulation).
    \item The SciPy~\cite{Jones2001} library (numerical integration of the
        replicator equation).
\end{itemize}

This work was performed using the computational facilities of the Advanced
Research Computing @ Cardiff (ARCCA) Division, Cardiff University.

\printbibliography

\newpage
\section*{Supplementary materials}

\includepdf{assets/pdf/proof_of_form_of_extortionate_strategies/main.pdf}

\newpage

Using the pair wise interactions the transition rates \(p,
q\) can be measured and the steady state probabilities inferred and compared to
the actual probabilities of each state.
This is done numerically by computing the singular eigenvector of the
matrix \(A\) \cite{Stewart2009}:

\[
    A =
    \begin{bmatrix}
        p_1 q_1 & p_1 (1 - q_1) & (1 - p_1) q_1 & (1 -p_1) (1 - q_1) \\
        p_2 q_2 & p_2 (1 - q_2) & (1 - p_2) q_2 & (1 -p_2) (1 - q_2) \\
        p_3 q_3 & p_3 (1 - q_3) & (1 - p_3) q_3 & (1 -p_3) (1 - q_3) \\
        p_4 q_4 & p_4 (1 - q_4) & (1 - p_4) q_4 & (1 -p_4) (1 - q_4) \\
    \end{bmatrix}
\]

Figure~\ref{fig:computed_probabilities_vs_theoretic_probabilities} shows a
regression line fitted to every pairwise interaction with a reported
\(\text{SSError}\) value (pairwise interactions with missing states were
omitted). This serves to validate the approach: a part from some edge cases the
relationship is consistent.

\begin{figure}[!htbp]
    \centering
    \includegraphics[width=.8\textwidth]{./assets/img/computed_probabilities_vs_theoretic_probabilities/main.pdf}
    \caption{The
        relationship between the steady state probabilities inferred from the
        measured transitions and the actual steady state probabilities. A linear
        regression line is included validating the approach.}
    \label{fig:computed_probabilities_vs_theoretic_probabilities}
\end{figure}


\end{document}

strategies.

The results of this analysis are shown in
Figure~\ref{fig:SSError_and_probabilities_in_full}. The top ranking strategies
by number of wins seem to be extortionate (but not against all strategies) and
it can be seen that a small sub group of strategies achieve mutual defection.
All the top ranking strategies according to score achieve mutual cooperation and
do not extort each other, however they
\textbf{do} exhibit extortionate behaviour towards a number of the lower ranking
strategies.

\begin{figure}[!htbp]
    \centering
    \includegraphics[width=.8\textwidth]{./assets/img/SSError_and_probabilities_in_full/main.pdf}
    \caption{\(\text{SSError}\) for the strategies for the full tournament. Only
    strategy interactions for which \(p_4=0\) and \(\chi>1\) are displayed.}
    \label{fig:SSError_and_probabilities_in_full}
\end{figure}

\section{Conclusion}\label{sec:conclusion}

This work defines an approach to measure whether or not a player is playing a
strategy that corresponds to an extortionate strategy as defined
in~\cite{Press2012}: a mathematical model for suspicion. Indeed, all
extortionate strategies have been
 classified as lying on a triangular plane.
This rigorous classification fails to be robust to small measurement error, thus
a statistical approach is proposed.
This is done through a linear algebraic approach for approximating the solution
of a linear system. Using this, a large number of pairwise interactions is
simulated and in fact very few strategies are found to act extortionately.

The work of~\cite{Press2012}, whilst showing that a clever approach to taking
advantage of another memory one strategy exists: this is incomplete. Whilst the
elegance of this result is very attractive, just as the simplicity of the
victory of Tit For Tat in Axelrod's original tournaments was, it is incomplete.
Extortionate strategies achieve a high number of wins but they do not
achieve a high score which corresponds to the fitness landscape in an
evolutionary sense. From the large number of interactions a payoff matrix \(S\)
can be measured where \(S_{ij}\) denotes the score (using standard values of
\((R, S, T, P) = (3, 0, 5, 1)\)) of the \(i\)th strategy
against the \(j\)th strategy. Using this, the replicator equation
describes the evolution of the system based on a population density fitness
function:

\begin{equation}\label{eqn:replicator_dynamics}
    \frac{dx}{dt} = x(S-x^TS x)
\end{equation}

Equation (\ref{eqn:replicator_dynamics}) is solved numerically through an
integration technique described in~\cite{Petzold1983} and
Figure~\ref{fig:replicator_dynamics} shows the evolution of the distribution of
the system: the various strategies are ranked by scores. It is clear to see that
only the high ranking strategies survive the evolutionary process (in fact,
only \documentclass[a4paper]{article}

\usepackage{amsmath}
\usepackage{amssymb}
\usepackage[margin=1.5cm,
            includefoot,
            footskip=30pt]{geometry}
\usepackage{layout}
\usepackage{graphicx}
\usepackage{subcaption}

\usepackage{biblatex}
\usepackage{pdfpages}

\bibliography{main.bib}

\title{Suspicion: Recognising and evaluating the effectiveness
       of extortion in the Iterated Prisoner's Dilemma}
\author{Vincent A. Knight \and Nikoleta E. Glynatsi}
\date{\today}



\begin{document}

\maketitle

\begin{abstract}
    The Iterated Prisoner's Dilemma is a model for rational and evolutionary
    interactive behaviour. It has applications both in the study of human social
    behaviour as well as in biology.
    It is used to understand when and how a rational individual might
    accept an immediate cost to their own utility for the direct benefit of
    another.

    Much attention has been given to a class of strategies called
    Zero Determinant strategies. It has been theoretically shown that these
    strategies can ``extort'' any player.

    In this work, an approach to identify if observed strategies are playing in
    an extortionate way is described. Furthermore, experimental analysis of
    a large tournament with \input{assets/tex/number_of_full_strategies/main.tex}
    strategies is considered. In this setting
    the most highly performing strategies do not play in an extortionate way
    against each other but do against lower performing strategies.
    This suggests that whilst the theory of Zero Determinant strategies
    indicates that memory is not of fundamental importance to the evolution of
    cooperative behaviour, this is incomplete.
\end{abstract}

\section{Introduction}\label{sec:introduction}

Agent based game theoretic models have become a stalwart of the underpinning
mathematics of interactive behaviours. One of the major pieces of work
in this area is the pair of original computer tournaments run by Robert
Axelrod~\cite{Axelrod1980, Axelrod1980a}. These tournaments pitted submitted
computer strategies against each other in plays of the Iterated Prisoner's
Dilemma. A common game where agents can choose to pay a slight cost to their
immediate utility in the hope of building a reputation. This has been used in
economic and evolutionary game theory to understand the evolution of cooperative
behaviour.

Recently, a class of strategies was described in~\cite{Press2012} that can
provably extort any given opponent. In~\cite{Hilbe2013, Moran1707} some
questions have already been asked about the true effectiveness of these
strategies in an evolutionary setting. Here another question is asked: is it
possible to recognise this extortionate behaviour? A mathematical procedure for
suspicion is presented: in the same way that the continued actions of an
extortionate individual might raise suspicion.

This work makes use of the Axelrod Python library~\cite{Knight2018, Knight2016}
with a large number of Prisoner Dilemma strategies available to give an
extensive numerical example of the ideas presented.  The approach is presented
in Section~\ref{sec:delta-zd-strategies}.  All of the code and data discussed
in Section~\ref{sec:numerical-experiments} is open sourced, archived and
written according to best scientific principles~\cite{Wilson2014}. The data
archive can be found at~\cite{vincent_knight_2018_1297075}.

\section{Recognising Extortion}\label{sec:delta-zd-strategies}

In~\cite{Press2012}, given a match between 2 memory-one strategies, the concept
of Zero Determinant (ZD) strategies is introduced. The main result of that paper
shows that given two memory one players \(p, q\in\mathbb{R}^4\) a linear
relationship between the players' scores could be forced by one of the players.

Using the notation of~\cite{Press2012}, assuming the utilities for player \(p\)
are given by \(S_x=(R, S, T, P)\) and for player \(q\) by \(S_y=(R, T, S, P)\)
and that the stationary scores of each player is given by \(S_X\) and \(S_Y\)
respectively. The main result of~\cite{Press2012} is that if

\begin{equation}\label{eqn:linear_relationship_for_p}
    \tilde p=\alpha S_x + \beta S_y + \gamma
\end{equation}

or

\begin{equation}\label{eqn:linear_relationship_for_q}
    \tilde q=\alpha S_x + \beta S_y + \gamma
\end{equation}

where \(\tilde p = (1 - p_1, 1 - p_2, p_3, p_4)\) and
\(\tilde q = (1 - q_1, 1 - q_2, q_3, q_4)\) then:

\begin{equation}
    \alpha S_X + \beta S_Y + \gamma = 0
\end{equation}

In~\cite{Press2012} a particular type of ZD strategy is defined: extortionate
strategies. If:

\begin{equation}\label{eqn:constraint_for_extortion}
    \gamma = - P(\alpha + \beta)
\end{equation}

then the player can ensure they get a score \(\chi\) times
larger than the opponent. This extortion coefficient is given by:

\begin{equation}\label{eqn:definition_of_chi}
    \chi=\frac{-\beta}{\alpha}
\end{equation}

Thus, if (\ref{eqn:constraint_for_extortion}) holds and \(\chi >1\) a player is
said to extort their opponent.
Here, the reverse problem is considered: given a
\(p\in\mathbb{R}^4\) how does one identify \(\alpha, \beta\) if they
exist and is the strategy in fact acting in an extortionate way?

These conditions correspond to:

\begin{align}
    \tilde p_1 & = \alpha R + \beta R - P (\alpha + \beta)
            \label{eqn:condition_for_tilde_p1}\\
    \tilde p_2 & = \alpha S + \beta T - P (\alpha + \beta)
            \label{eqn:condition_for_tilde_p2}\\
    \tilde p_3 & = \alpha T + \beta S - P (\alpha + \beta)
            \label{eqn:condition_for_tilde_p3}\\
    \tilde p_4 & = \alpha P + \beta P - P (\alpha + \beta)
            \label{eqn:condition_for_tilde_p4}
\end{align}

Equation (\ref{eqn:condition_for_tilde_p4}) ensures that \(p_4=\tilde p_4=0\).
Equations (\ref{eqn:condition_for_tilde_p1}-\ref{eqn:condition_for_tilde_p3})
can be used to eliminate \(\alpha, \beta\), giving:

\begin{equation}\label{eqn:planar_definition_of_extortion}
    \tilde p_1 = \frac{(R - P)(\tilde p_2 + \tilde p_3)}{S + T - 2P}
\end{equation}

with:

\begin{equation}\label{eqn:definition_of_chi}
    \chi = \frac{\tilde p_2 (P - T) + \tilde p_3 (S - P)}
                {\tilde p_2 (P - S) + \tilde p_3 (T - P)}
\end{equation}

Given a strategy \(p\in\mathbb{R}^{4\times 1}\) equations
(\ref{eqn:condition_for_tilde_p4}), (\ref{eqn:planar_definition_of_extortion}-\ref{eqn:definition_of_chi}) can be used to check if
a strategy is extortionate. The conditions correspond to:

\begin{align}
    p_1 & = \frac{(R-P)(p_2 + p_3) - R + T + S - P}{S + T - 2P}
     \label{eqn:condition_for_p1}\\
    p_4 & = 0 \label{eqn:condition_for_p4}\\
    1 & > p_2 + p_3\label{eqn:condition_for_chi}
\end{align}

The algebraic steps necessary to prove these results are available in the
supporting materials.

All extortionate strategies reside on a triangular (\ref{eqn:condition_for_chi})
plane (\ref{eqn:condition_for_p1}) in 3 dimensions (\ref{eqn:condition_for_p4}).
Using this formulation it can be seen that a necessary (but not sufficient)
condition for an extortionate strategy is that it cooperates on average less
than 50\% of the time when in a state of disagreement with the opponent.

As an example, consider the known extortionate strategy \(p=(8 / 9, 1 / 2, 1 /
3, 0)\) from~\cite{Stewart2012} which is referred to as \texttt{Extort-2}. In
this case, for the standard values of \((R, T, S, P)\) constraint
(\ref{eqn:condition_for_p1}) corresponds to:

\begin{equation}
    p_1 = \frac{2(p_2 + p_3) + 1}{3}
\end{equation}

It is clear that in this case all constraints hold.

This approach could in fact be used to confirm that a given strategy is acting
in an extortionate manner even if it is not a memory one strategy. However, in
practice, if a closed form for \(p\) is not known, then due to measurement
and/or numerical error this would not work.

This problem can be written in the following linear algebraic form where
\(x=(\alpha, \beta)\)
and \(p^*=(\tilde p_1 - 1, tilde_2 - 1, p_3)\):

\begin{equation}\label{eqn:linear_algebraic_equation_for_p}
    Cx= p^*
\end{equation}

\(C\) corresponds to equations
(\ref{eqn:condition_for_tilde_p1}-\ref{eqn:condition_for_tilde_p3}) and is
given by:

\begin{equation}\label{eqn:definition_of_C}
    C =
    \begin{bmatrix}
        R - P & R- P \\
        S - P & T- P \\
        T - P & S- P \\
    \end{bmatrix}
\end{equation}

Note that in general, equation (\ref{eqn:linear_algebraic_equation_for_p}) will
not necessarily have a solution. From the Rouch\'{e}-Capelli theorem if there is
a solution it is unique as \(\text{rank}(C)=2\) which is the dimension of the
variable \(x\). The best fitting \(x\) is found by minimizing:

\begin{equation}\label{eqn:r_squared}
    \text{SSError} = \|C x- p^*\|_2^2 = \sum_{i=1}^{3}\left((C\bar x)_i-p_i^*\right)^2
\end{equation}

Note that \(\text{SSError}\), which is the square of the Frobenius
norm~\cite{Golub2013}, becomes a measure of how close a strategy is to being an
extortionate strategy. Suspicion
of extortion then corresponds to a threshold on \(\text{SSError}\).

By observing interactions (human or otherwise), their memory one representation
can be inferred and this approach can be used to recognise extortionate
behaviour. The notion of comparing theoretic and actual plays of the IPD is not
novel, see for example~\cite{Rand2013}. Immediately it is noted that if the
environment is noisy~\cite{Wu1995} then no strategy can be considered to be
extortionate as \(p_4>0\).

In the next section, this idea will be illustrated by observing the interactions
that take place in a computer based tournament of the IPD\@.

\section{Numerical experiments}\label{sec:numerical-experiments}

In~\cite{Stewart2012} results from a tournament with
\input{./assets/tex/number_of_stewart_plotkin_strategies/main.tex} strategies,
was presented with specific consideration given to ZD strategies. This
tournament is reproduced here using the Axelrod-Python
project~\cite{Knight2016}. To obtain a good measure of the corresponding
transition rates for each strategy all matches have been run for
\input{assets/tex/number_of_turns/main.tex} turns and every match has been
repeated \input{assets/tex/number_of_repetitions/main.tex} times. All of this
interaction data is available at~\cite{vincent_knight_2018_1297075}. A good
match between the inferred Markov chain and the state distribution of the actual
interactions has been verified. Data for this is presented in the supplementary
materials.

Figure~\ref{fig:SSError_overall_in_stewart_plotkin} shows the \(\text{SSError}\)
values for all the strategies in the tournament, as reported
in~\cite{Stewart2012} the extortionate strategy (which has an expected
\(\text{SSError}\) approximately 0) gains a large number of wins.

\begin{figure}[!htbp]
    \centering
    \includegraphics[width=.8\textwidth]{./assets/img/SSError_overall_in_stewart_plotkin/main.pdf}
    \caption{\(\text{SSError}\) and state probabilities for the strategies
        of~\cite{Stewart2012}, ordered both by number of wins and overall score.
        Note that \(P(DC)\) is not shown as it corresponds to the transpose of
        \(P(CD)\). Cooperator and Defector are omitted as they do not visit all
        the states.}
    \label{fig:SSError_overall_in_stewart_plotkin}
\end{figure}

Here, the work of~\cite{Stewart2012} is extended by investigating a tournament
with \input{assets/tex/number_of_full_strategies/main.tex}
strategies.

The results of this analysis are shown in
Figure~\ref{fig:SSError_and_probabilities_in_full}. The top ranking strategies
by number of wins seem to be extortionate (but not against all strategies) and
it can be seen that a small sub group of strategies achieve mutual defection.
All the top ranking strategies according to score achieve mutual cooperation and
do not extort each other, however they
\textbf{do} exhibit extortionate behaviour towards a number of the lower ranking
strategies.

\begin{figure}[!htbp]
    \centering
    \includegraphics[width=.8\textwidth]{./assets/img/SSError_and_probabilities_in_full/main.pdf}
    \caption{\(\text{SSError}\) for the strategies for the full tournament. Only
    strategy interactions for which \(p_4=0\) and \(\chi>1\) are displayed.}
    \label{fig:SSError_and_probabilities_in_full}
\end{figure}

\section{Conclusion}\label{sec:conclusion}

This work defines an approach to measure whether or not a player is playing a
strategy that corresponds to an extortionate strategy as defined
in~\cite{Press2012}: a mathematical model for suspicion. Indeed, all
extortionate strategies have been
 classified as lying on a triangular plane.
This rigorous classification fails to be robust to small measurement error, thus
a statistical approach is proposed.
This is done through a linear algebraic approach for approximating the solution
of a linear system. Using this, a large number of pairwise interactions is
simulated and in fact very few strategies are found to act extortionately.

The work of~\cite{Press2012}, whilst showing that a clever approach to taking
advantage of another memory one strategy exists: this is incomplete. Whilst the
elegance of this result is very attractive, just as the simplicity of the
victory of Tit For Tat in Axelrod's original tournaments was, it is incomplete.
Extortionate strategies achieve a high number of wins but they do not
achieve a high score which corresponds to the fitness landscape in an
evolutionary sense. From the large number of interactions a payoff matrix \(S\)
can be measured where \(S_{ij}\) denotes the score (using standard values of
\((R, S, T, P) = (3, 0, 5, 1)\)) of the \(i\)th strategy
against the \(j\)th strategy. Using this, the replicator equation
describes the evolution of the system based on a population density fitness
function:

\begin{equation}\label{eqn:replicator_dynamics}
    \frac{dx}{dt} = x(S-x^TS x)
\end{equation}

Equation (\ref{eqn:replicator_dynamics}) is solved numerically through an
integration technique described in~\cite{Petzold1983} and
Figure~\ref{fig:replicator_dynamics} shows the evolution of the distribution of
the system: the various strategies are ranked by scores. It is clear to see that
only the high ranking strategies survive the evolutionary process (in fact,
only \input{./assets/img/replicator_dynamics/main.tex}
have a final distribution greater than \(10 ^ {-2}\)). This confirms the
findings of~\cite{Moran1707} in which sophisticated strategies resist
evolutionary invasion of shorter memory strategies. Recalling
Figure~\ref{fig:SSError_and_probabilities_in_full} this demonstrates that:

\begin{itemize}
    \item Cooperation emerges through the evolutionary process: the high scoring
        strategies do not exhibit extortionate behaviour towards each other.
    \item Extortionate strategies do not survive the evolutionary process.
\end{itemize}

\begin{figure}[!htbp]
    \centering
    \includegraphics[width=.8\textwidth]{./assets/img/replicator_dynamics/main.pdf}
    \caption{Numerical simulation of the replicator equation
    (\ref{eqn:replicator_dynamics}): strategies are ordered by score, only the strategies with a high score survive the evolutionary process.}
    \label{fig:replicator_dynamics}
\end{figure}

This work can be used to classify plays of the IPD\@: data can be collected from
actual interactions (in lab or in the field). Furthermore, this allows for a
classification method similar to the notion of fingerprinting presented
in~\cite{Ashlock2008}. Trained strategies can potentially be classified as
extortionate or not or it could be possible to even constrain the reinforcement
learning approaches that are becoming prevalent in the literature.
Alternatively, this mathematical approach for recognising extortion could be
used in sophisticated strategies to defend against invasion. Arguably, some of
the strategies considered here exhibit this behaviour, indeed as described
in~\cite{Harper2017}, the top ranking strategies in the full tournament are
obtained using evolutionary reinforcement learning techniques, thus, suspicion
of extortionate behaviour could in fact be an evolutionary trait.

\section*{Acknowledgements}

The following open source software libraries were used in this research:

\begin{itemize}
    \item The Axelrod ~\cite{Knight2016, Knight2018} library (IPD strategies and
        tournaments).
    \item The sympy library~\cite{Meurer2017} (verification of all symbolic
        calculations).
    \item The matplotlib~\cite{Droettboom2018} library (visualisation).
    \item The pandas~\cite{Structures2010}, dask~\cite{Dask2016} and
        NumPy~\cite{Oliphant2015} libraries (data manipulation).
    \item The SciPy~\cite{Jones2001} library (numerical integration of the
        replicator equation).
\end{itemize}

This work was performed using the computational facilities of the Advanced
Research Computing @ Cardiff (ARCCA) Division, Cardiff University.

\printbibliography

\newpage
\section*{Supplementary materials}

\includepdf{assets/pdf/proof_of_form_of_extortionate_strategies/main.pdf}

\newpage

Using the pair wise interactions the transition rates \(p,
q\) can be measured and the steady state probabilities inferred and compared to
the actual probabilities of each state.
This is done numerically by computing the singular eigenvector of the
matrix \(A\) \cite{Stewart2009}:

\[
    A =
    \begin{bmatrix}
        p_1 q_1 & p_1 (1 - q_1) & (1 - p_1) q_1 & (1 -p_1) (1 - q_1) \\
        p_2 q_2 & p_2 (1 - q_2) & (1 - p_2) q_2 & (1 -p_2) (1 - q_2) \\
        p_3 q_3 & p_3 (1 - q_3) & (1 - p_3) q_3 & (1 -p_3) (1 - q_3) \\
        p_4 q_4 & p_4 (1 - q_4) & (1 - p_4) q_4 & (1 -p_4) (1 - q_4) \\
    \end{bmatrix}
\]

Figure~\ref{fig:computed_probabilities_vs_theoretic_probabilities} shows a
regression line fitted to every pairwise interaction with a reported
\(\text{SSError}\) value (pairwise interactions with missing states were
omitted). This serves to validate the approach: a part from some edge cases the
relationship is consistent.

\begin{figure}[!htbp]
    \centering
    \includegraphics[width=.8\textwidth]{./assets/img/computed_probabilities_vs_theoretic_probabilities/main.pdf}
    \caption{The
        relationship between the steady state probabilities inferred from the
        measured transitions and the actual steady state probabilities. A linear
        regression line is included validating the approach.}
    \label{fig:computed_probabilities_vs_theoretic_probabilities}
\end{figure}


\end{document}

have a final distribution greater than \(10 ^ {-2}\)). This confirms the
findings of~\cite{Moran1707} in which sophisticated strategies resist
evolutionary invasion of shorter memory strategies. Recalling
Figure~\ref{fig:SSError_and_probabilities_in_full} this demonstrates that:

\begin{itemize}
    \item Cooperation emerges through the evolutionary process: the high scoring
        strategies do not exhibit extortionate behaviour towards each other.
    \item Extortionate strategies do not survive the evolutionary process.
\end{itemize}

\begin{figure}[!htbp]
    \centering
    \includegraphics[width=.8\textwidth]{./assets/img/replicator_dynamics/main.pdf}
    \caption{Numerical simulation of the replicator equation
    (\ref{eqn:replicator_dynamics}): strategies are ordered by score, only the strategies with a high score survive the evolutionary process.}
    \label{fig:replicator_dynamics}
\end{figure}

This work can be used to classify plays of the IPD\@: data can be collected from
actual interactions (in lab or in the field). Furthermore, this allows for a
classification method similar to the notion of fingerprinting presented
in~\cite{Ashlock2008}. Trained strategies can potentially be classified as
extortionate or not or it could be possible to even constrain the reinforcement
learning approaches that are becoming prevalent in the literature.
Alternatively, this mathematical approach for recognising extortion could be
used in sophisticated strategies to defend against invasion. Arguably, some of
the strategies considered here exhibit this behaviour, indeed as described
in~\cite{Harper2017}, the top ranking strategies in the full tournament are
obtained using evolutionary reinforcement learning techniques, thus, suspicion
of extortionate behaviour could in fact be an evolutionary trait.

\section*{Acknowledgements}

The following open source software libraries were used in this research:

\begin{itemize}
    \item The Axelrod ~\cite{Knight2016, Knight2018} library (IPD strategies and
        tournaments).
    \item The sympy library~\cite{Meurer2017} (verification of all symbolic
        calculations).
    \item The matplotlib~\cite{Droettboom2018} library (visualisation).
    \item The pandas~\cite{Structures2010}, dask~\cite{Dask2016} and
        NumPy~\cite{Oliphant2015} libraries (data manipulation).
    \item The SciPy~\cite{Jones2001} library (numerical integration of the
        replicator equation).
\end{itemize}

This work was performed using the computational facilities of the Advanced
Research Computing @ Cardiff (ARCCA) Division, Cardiff University.

\printbibliography

\newpage
\section*{Supplementary materials}

\includepdf{assets/pdf/proof_of_form_of_extortionate_strategies/main.pdf}

\newpage

Using the pair wise interactions the transition rates \(p,
q\) can be measured and the steady state probabilities inferred and compared to
the actual probabilities of each state.
This is done numerically by computing the singular eigenvector of the
matrix \(A\) \cite{Stewart2009}:

\[
    A =
    \begin{bmatrix}
        p_1 q_1 & p_1 (1 - q_1) & (1 - p_1) q_1 & (1 -p_1) (1 - q_1) \\
        p_2 q_2 & p_2 (1 - q_2) & (1 - p_2) q_2 & (1 -p_2) (1 - q_2) \\
        p_3 q_3 & p_3 (1 - q_3) & (1 - p_3) q_3 & (1 -p_3) (1 - q_3) \\
        p_4 q_4 & p_4 (1 - q_4) & (1 - p_4) q_4 & (1 -p_4) (1 - q_4) \\
    \end{bmatrix}
\]

Figure~\ref{fig:computed_probabilities_vs_theoretic_probabilities} shows a
regression line fitted to every pairwise interaction with a reported
\(\text{SSError}\) value (pairwise interactions with missing states were
omitted). This serves to validate the approach: a part from some edge cases the
relationship is consistent.

\begin{figure}[!htbp]
    \centering
    \includegraphics[width=.8\textwidth]{./assets/img/computed_probabilities_vs_theoretic_probabilities/main.pdf}
    \caption{The
        relationship between the steady state probabilities inferred from the
        measured transitions and the actual steady state probabilities. A linear
        regression line is included validating the approach.}
    \label{fig:computed_probabilities_vs_theoretic_probabilities}
\end{figure}


\end{document}

have a final distribution greater than \(10 ^ {-2}\)). This confirms the
findings of~\cite{Moran1707} in which sophisticated strategies resist
evolutionary invasion of shorter memory strategies. Recalling
Figure~\ref{fig:SSError_and_probabilities_in_full} this demonstrates that:

\begin{itemize}
    \item Cooperation emerges through the evolutionary process: the high scoring
        strategies do not exhibit extortionate behaviour towards each other.
    \item Extortionate strategies do not survive the evolutionary process.
\end{itemize}

\begin{figure}[!htbp]
    \centering
    \includegraphics[width=.8\textwidth]{./assets/img/replicator_dynamics/main.pdf}
    \caption{Numerical simulation of the replicator equation
    (\ref{eqn:replicator_dynamics}): strategies are ordered by score, only the strategies with a high score survive the evolutionary process.}
    \label{fig:replicator_dynamics}
\end{figure}

This work can be used to classify plays of the IPD\@: data can be collected from
actual interactions (in lab or in the field). Furthermore, this allows for a
classification method similar to the notion of fingerprinting presented
in~\cite{Ashlock2008}. Trained strategies can potentially be classified as
extortionate or not or it could be possible to even constrain the reinforcement
learning approaches that are becoming prevalent in the literature.
Alternatively, this mathematical approach for recognising extortion could be
used in sophisticated strategies to defend against invasion. Arguably, some of
the strategies considered here exhibit this behaviour, indeed as described
in~\cite{Harper2017}, the top ranking strategies in the full tournament are
obtained using evolutionary reinforcement learning techniques, thus, suspicion
of extortionate behaviour could in fact be an evolutionary trait.

\section*{Acknowledgements}

The following open source software libraries were used in this research:

\begin{itemize}
    \item The Axelrod ~\cite{Knight2016, Knight2018} library (IPD strategies and
        tournaments).
    \item The sympy library~\cite{Meurer2017} (verification of all symbolic
        calculations).
    \item The matplotlib~\cite{Droettboom2018} library (visualisation).
    \item The pandas~\cite{Structures2010}, dask~\cite{Dask2016} and
        NumPy~\cite{Oliphant2015} libraries (data manipulation).
    \item The SciPy~\cite{Jones2001} library (numerical integration of the
        replicator equation).
\end{itemize}

This work was performed using the computational facilities of the Advanced
Research Computing @ Cardiff (ARCCA) Division, Cardiff University.

\printbibliography

\newpage
\section*{Supplementary materials}

\includepdf{assets/pdf/proof_of_form_of_extortionate_strategies/main.pdf}

\newpage

Using the pair wise interactions the transition rates \(p,
q\) can be measured and the steady state probabilities inferred and compared to
the actual probabilities of each state.
This is done numerically by computing the singular eigenvector of the
matrix \(A\) \cite{Stewart2009}:

\[
    A =
    \begin{bmatrix}
        p_1 q_1 & p_1 (1 - q_1) & (1 - p_1) q_1 & (1 -p_1) (1 - q_1) \\
        p_2 q_2 & p_2 (1 - q_2) & (1 - p_2) q_2 & (1 -p_2) (1 - q_2) \\
        p_3 q_3 & p_3 (1 - q_3) & (1 - p_3) q_3 & (1 -p_3) (1 - q_3) \\
        p_4 q_4 & p_4 (1 - q_4) & (1 - p_4) q_4 & (1 -p_4) (1 - q_4) \\
    \end{bmatrix}
\]

Figure~\ref{fig:computed_probabilities_vs_theoretic_probabilities} shows a
regression line fitted to every pairwise interaction with a reported
\(\text{SSError}\) value (pairwise interactions with missing states were
omitted). This serves to validate the approach: a part from some edge cases the
relationship is consistent.

\begin{figure}[!htbp]
    \centering
    \includegraphics[width=.8\textwidth]{./assets/img/computed_probabilities_vs_theoretic_probabilities/main.pdf}
    \caption{The
        relationship between the steady state probabilities inferred from the
        measured transitions and the actual steady state probabilities. A linear
        regression line is included validating the approach.}
    \label{fig:computed_probabilities_vs_theoretic_probabilities}
\end{figure}


\end{document}
turns and every match has been
repeated \documentclass[a4paper]{article}

\usepackage{amsmath}
\usepackage{amssymb}
\usepackage[margin=1.5cm,
            includefoot,
            footskip=30pt]{geometry}
\usepackage{layout}
\usepackage{graphicx}
\usepackage{subcaption}

\usepackage{biblatex}
\usepackage{pdfpages}

\bibliography{main.bib}

\title{Suspicion: Recognising and evaluating the effectiveness
       of extortion in the Iterated Prisoner's Dilemma}
\author{Vincent A. Knight \and Nikoleta E. Glynatsi}
\date{\today}



\begin{document}

\maketitle

\begin{abstract}
    The Iterated Prisoner's Dilemma is a model for rational and evolutionary
    interactive behaviour. It has applications both in the study of human social
    behaviour as well as in biology.
    It is used to understand when and how a rational individual might
    accept an immediate cost to their own utility for the direct benefit of
    another.

    Much attention has been given to a class of strategies called
    Zero Determinant strategies. It has been theoretically shown that these
    strategies can ``extort'' any player.

    In this work, an approach to identify if observed strategies are playing in
    an extortionate way is described. Furthermore, experimental analysis of
    a large tournament with \documentclass[a4paper]{article}

\usepackage{amsmath}
\usepackage{amssymb}
\usepackage[margin=1.5cm,
            includefoot,
            footskip=30pt]{geometry}
\usepackage{layout}
\usepackage{graphicx}
\usepackage{subcaption}

\usepackage{biblatex}
\usepackage{pdfpages}

\bibliography{main.bib}

\title{Suspicion: Recognising and evaluating the effectiveness
       of extortion in the Iterated Prisoner's Dilemma}
\author{Vincent A. Knight \and Nikoleta E. Glynatsi}
\date{\today}



\begin{document}

\maketitle

\begin{abstract}
    The Iterated Prisoner's Dilemma is a model for rational and evolutionary
    interactive behaviour. It has applications both in the study of human social
    behaviour as well as in biology.
    It is used to understand when and how a rational individual might
    accept an immediate cost to their own utility for the direct benefit of
    another.

    Much attention has been given to a class of strategies called
    Zero Determinant strategies. It has been theoretically shown that these
    strategies can ``extort'' any player.

    In this work, an approach to identify if observed strategies are playing in
    an extortionate way is described. Furthermore, experimental analysis of
    a large tournament with \documentclass[a4paper]{article}

\usepackage{amsmath}
\usepackage{amssymb}
\usepackage[margin=1.5cm,
            includefoot,
            footskip=30pt]{geometry}
\usepackage{layout}
\usepackage{graphicx}
\usepackage{subcaption}

\usepackage{biblatex}
\usepackage{pdfpages}

\bibliography{main.bib}

\title{Suspicion: Recognising and evaluating the effectiveness
       of extortion in the Iterated Prisoner's Dilemma}
\author{Vincent A. Knight \and Nikoleta E. Glynatsi}
\date{\today}



\begin{document}

\maketitle

\begin{abstract}
    The Iterated Prisoner's Dilemma is a model for rational and evolutionary
    interactive behaviour. It has applications both in the study of human social
    behaviour as well as in biology.
    It is used to understand when and how a rational individual might
    accept an immediate cost to their own utility for the direct benefit of
    another.

    Much attention has been given to a class of strategies called
    Zero Determinant strategies. It has been theoretically shown that these
    strategies can ``extort'' any player.

    In this work, an approach to identify if observed strategies are playing in
    an extortionate way is described. Furthermore, experimental analysis of
    a large tournament with \input{assets/tex/number_of_full_strategies/main.tex}
    strategies is considered. In this setting
    the most highly performing strategies do not play in an extortionate way
    against each other but do against lower performing strategies.
    This suggests that whilst the theory of Zero Determinant strategies
    indicates that memory is not of fundamental importance to the evolution of
    cooperative behaviour, this is incomplete.
\end{abstract}

\section{Introduction}\label{sec:introduction}

Agent based game theoretic models have become a stalwart of the underpinning
mathematics of interactive behaviours. One of the major pieces of work
in this area is the pair of original computer tournaments run by Robert
Axelrod~\cite{Axelrod1980, Axelrod1980a}. These tournaments pitted submitted
computer strategies against each other in plays of the Iterated Prisoner's
Dilemma. A common game where agents can choose to pay a slight cost to their
immediate utility in the hope of building a reputation. This has been used in
economic and evolutionary game theory to understand the evolution of cooperative
behaviour.

Recently, a class of strategies was described in~\cite{Press2012} that can
provably extort any given opponent. In~\cite{Hilbe2013, Moran1707} some
questions have already been asked about the true effectiveness of these
strategies in an evolutionary setting. Here another question is asked: is it
possible to recognise this extortionate behaviour? A mathematical procedure for
suspicion is presented: in the same way that the continued actions of an
extortionate individual might raise suspicion.

This work makes use of the Axelrod Python library~\cite{Knight2018, Knight2016}
with a large number of Prisoner Dilemma strategies available to give an
extensive numerical example of the ideas presented.  The approach is presented
in Section~\ref{sec:delta-zd-strategies}.  All of the code and data discussed
in Section~\ref{sec:numerical-experiments} is open sourced, archived and
written according to best scientific principles~\cite{Wilson2014}. The data
archive can be found at~\cite{vincent_knight_2018_1297075}.

\section{Recognising Extortion}\label{sec:delta-zd-strategies}

In~\cite{Press2012}, given a match between 2 memory-one strategies, the concept
of Zero Determinant (ZD) strategies is introduced. The main result of that paper
shows that given two memory one players \(p, q\in\mathbb{R}^4\) a linear
relationship between the players' scores could be forced by one of the players.

Using the notation of~\cite{Press2012}, assuming the utilities for player \(p\)
are given by \(S_x=(R, S, T, P)\) and for player \(q\) by \(S_y=(R, T, S, P)\)
and that the stationary scores of each player is given by \(S_X\) and \(S_Y\)
respectively. The main result of~\cite{Press2012} is that if

\begin{equation}\label{eqn:linear_relationship_for_p}
    \tilde p=\alpha S_x + \beta S_y + \gamma
\end{equation}

or

\begin{equation}\label{eqn:linear_relationship_for_q}
    \tilde q=\alpha S_x + \beta S_y + \gamma
\end{equation}

where \(\tilde p = (1 - p_1, 1 - p_2, p_3, p_4)\) and
\(\tilde q = (1 - q_1, 1 - q_2, q_3, q_4)\) then:

\begin{equation}
    \alpha S_X + \beta S_Y + \gamma = 0
\end{equation}

In~\cite{Press2012} a particular type of ZD strategy is defined: extortionate
strategies. If:

\begin{equation}\label{eqn:constraint_for_extortion}
    \gamma = - P(\alpha + \beta)
\end{equation}

then the player can ensure they get a score \(\chi\) times
larger than the opponent. This extortion coefficient is given by:

\begin{equation}\label{eqn:definition_of_chi}
    \chi=\frac{-\beta}{\alpha}
\end{equation}

Thus, if (\ref{eqn:constraint_for_extortion}) holds and \(\chi >1\) a player is
said to extort their opponent.
Here, the reverse problem is considered: given a
\(p\in\mathbb{R}^4\) how does one identify \(\alpha, \beta\) if they
exist and is the strategy in fact acting in an extortionate way?

These conditions correspond to:

\begin{align}
    \tilde p_1 & = \alpha R + \beta R - P (\alpha + \beta)
            \label{eqn:condition_for_tilde_p1}\\
    \tilde p_2 & = \alpha S + \beta T - P (\alpha + \beta)
            \label{eqn:condition_for_tilde_p2}\\
    \tilde p_3 & = \alpha T + \beta S - P (\alpha + \beta)
            \label{eqn:condition_for_tilde_p3}\\
    \tilde p_4 & = \alpha P + \beta P - P (\alpha + \beta)
            \label{eqn:condition_for_tilde_p4}
\end{align}

Equation (\ref{eqn:condition_for_tilde_p4}) ensures that \(p_4=\tilde p_4=0\).
Equations (\ref{eqn:condition_for_tilde_p1}-\ref{eqn:condition_for_tilde_p3})
can be used to eliminate \(\alpha, \beta\), giving:

\begin{equation}\label{eqn:planar_definition_of_extortion}
    \tilde p_1 = \frac{(R - P)(\tilde p_2 + \tilde p_3)}{S + T - 2P}
\end{equation}

with:

\begin{equation}\label{eqn:definition_of_chi}
    \chi = \frac{\tilde p_2 (P - T) + \tilde p_3 (S - P)}
                {\tilde p_2 (P - S) + \tilde p_3 (T - P)}
\end{equation}

Given a strategy \(p\in\mathbb{R}^{4\times 1}\) equations
(\ref{eqn:condition_for_tilde_p4}), (\ref{eqn:planar_definition_of_extortion}-\ref{eqn:definition_of_chi}) can be used to check if
a strategy is extortionate. The conditions correspond to:

\begin{align}
    p_1 & = \frac{(R-P)(p_2 + p_3) - R + T + S - P}{S + T - 2P}
     \label{eqn:condition_for_p1}\\
    p_4 & = 0 \label{eqn:condition_for_p4}\\
    1 & > p_2 + p_3\label{eqn:condition_for_chi}
\end{align}

The algebraic steps necessary to prove these results are available in the
supporting materials.

All extortionate strategies reside on a triangular (\ref{eqn:condition_for_chi})
plane (\ref{eqn:condition_for_p1}) in 3 dimensions (\ref{eqn:condition_for_p4}).
Using this formulation it can be seen that a necessary (but not sufficient)
condition for an extortionate strategy is that it cooperates on average less
than 50\% of the time when in a state of disagreement with the opponent.

As an example, consider the known extortionate strategy \(p=(8 / 9, 1 / 2, 1 /
3, 0)\) from~\cite{Stewart2012} which is referred to as \texttt{Extort-2}. In
this case, for the standard values of \((R, T, S, P)\) constraint
(\ref{eqn:condition_for_p1}) corresponds to:

\begin{equation}
    p_1 = \frac{2(p_2 + p_3) + 1}{3}
\end{equation}

It is clear that in this case all constraints hold.

This approach could in fact be used to confirm that a given strategy is acting
in an extortionate manner even if it is not a memory one strategy. However, in
practice, if a closed form for \(p\) is not known, then due to measurement
and/or numerical error this would not work.

This problem can be written in the following linear algebraic form where
\(x=(\alpha, \beta)\)
and \(p^*=(\tilde p_1 - 1, tilde_2 - 1, p_3)\):

\begin{equation}\label{eqn:linear_algebraic_equation_for_p}
    Cx= p^*
\end{equation}

\(C\) corresponds to equations
(\ref{eqn:condition_for_tilde_p1}-\ref{eqn:condition_for_tilde_p3}) and is
given by:

\begin{equation}\label{eqn:definition_of_C}
    C =
    \begin{bmatrix}
        R - P & R- P \\
        S - P & T- P \\
        T - P & S- P \\
    \end{bmatrix}
\end{equation}

Note that in general, equation (\ref{eqn:linear_algebraic_equation_for_p}) will
not necessarily have a solution. From the Rouch\'{e}-Capelli theorem if there is
a solution it is unique as \(\text{rank}(C)=2\) which is the dimension of the
variable \(x\). The best fitting \(x\) is found by minimizing:

\begin{equation}\label{eqn:r_squared}
    \text{SSError} = \|C x- p^*\|_2^2 = \sum_{i=1}^{3}\left((C\bar x)_i-p_i^*\right)^2
\end{equation}

Note that \(\text{SSError}\), which is the square of the Frobenius
norm~\cite{Golub2013}, becomes a measure of how close a strategy is to being an
extortionate strategy. Suspicion
of extortion then corresponds to a threshold on \(\text{SSError}\).

By observing interactions (human or otherwise), their memory one representation
can be inferred and this approach can be used to recognise extortionate
behaviour. The notion of comparing theoretic and actual plays of the IPD is not
novel, see for example~\cite{Rand2013}. Immediately it is noted that if the
environment is noisy~\cite{Wu1995} then no strategy can be considered to be
extortionate as \(p_4>0\).

In the next section, this idea will be illustrated by observing the interactions
that take place in a computer based tournament of the IPD\@.

\section{Numerical experiments}\label{sec:numerical-experiments}

In~\cite{Stewart2012} results from a tournament with
\input{./assets/tex/number_of_stewart_plotkin_strategies/main.tex} strategies,
was presented with specific consideration given to ZD strategies. This
tournament is reproduced here using the Axelrod-Python
project~\cite{Knight2016}. To obtain a good measure of the corresponding
transition rates for each strategy all matches have been run for
\input{assets/tex/number_of_turns/main.tex} turns and every match has been
repeated \input{assets/tex/number_of_repetitions/main.tex} times. All of this
interaction data is available at~\cite{vincent_knight_2018_1297075}. A good
match between the inferred Markov chain and the state distribution of the actual
interactions has been verified. Data for this is presented in the supplementary
materials.

Figure~\ref{fig:SSError_overall_in_stewart_plotkin} shows the \(\text{SSError}\)
values for all the strategies in the tournament, as reported
in~\cite{Stewart2012} the extortionate strategy (which has an expected
\(\text{SSError}\) approximately 0) gains a large number of wins.

\begin{figure}[!htbp]
    \centering
    \includegraphics[width=.8\textwidth]{./assets/img/SSError_overall_in_stewart_plotkin/main.pdf}
    \caption{\(\text{SSError}\) and state probabilities for the strategies
        of~\cite{Stewart2012}, ordered both by number of wins and overall score.
        Note that \(P(DC)\) is not shown as it corresponds to the transpose of
        \(P(CD)\). Cooperator and Defector are omitted as they do not visit all
        the states.}
    \label{fig:SSError_overall_in_stewart_plotkin}
\end{figure}

Here, the work of~\cite{Stewart2012} is extended by investigating a tournament
with \input{assets/tex/number_of_full_strategies/main.tex}
strategies.

The results of this analysis are shown in
Figure~\ref{fig:SSError_and_probabilities_in_full}. The top ranking strategies
by number of wins seem to be extortionate (but not against all strategies) and
it can be seen that a small sub group of strategies achieve mutual defection.
All the top ranking strategies according to score achieve mutual cooperation and
do not extort each other, however they
\textbf{do} exhibit extortionate behaviour towards a number of the lower ranking
strategies.

\begin{figure}[!htbp]
    \centering
    \includegraphics[width=.8\textwidth]{./assets/img/SSError_and_probabilities_in_full/main.pdf}
    \caption{\(\text{SSError}\) for the strategies for the full tournament. Only
    strategy interactions for which \(p_4=0\) and \(\chi>1\) are displayed.}
    \label{fig:SSError_and_probabilities_in_full}
\end{figure}

\section{Conclusion}\label{sec:conclusion}

This work defines an approach to measure whether or not a player is playing a
strategy that corresponds to an extortionate strategy as defined
in~\cite{Press2012}: a mathematical model for suspicion. Indeed, all
extortionate strategies have been
 classified as lying on a triangular plane.
This rigorous classification fails to be robust to small measurement error, thus
a statistical approach is proposed.
This is done through a linear algebraic approach for approximating the solution
of a linear system. Using this, a large number of pairwise interactions is
simulated and in fact very few strategies are found to act extortionately.

The work of~\cite{Press2012}, whilst showing that a clever approach to taking
advantage of another memory one strategy exists: this is incomplete. Whilst the
elegance of this result is very attractive, just as the simplicity of the
victory of Tit For Tat in Axelrod's original tournaments was, it is incomplete.
Extortionate strategies achieve a high number of wins but they do not
achieve a high score which corresponds to the fitness landscape in an
evolutionary sense. From the large number of interactions a payoff matrix \(S\)
can be measured where \(S_{ij}\) denotes the score (using standard values of
\((R, S, T, P) = (3, 0, 5, 1)\)) of the \(i\)th strategy
against the \(j\)th strategy. Using this, the replicator equation
describes the evolution of the system based on a population density fitness
function:

\begin{equation}\label{eqn:replicator_dynamics}
    \frac{dx}{dt} = x(S-x^TS x)
\end{equation}

Equation (\ref{eqn:replicator_dynamics}) is solved numerically through an
integration technique described in~\cite{Petzold1983} and
Figure~\ref{fig:replicator_dynamics} shows the evolution of the distribution of
the system: the various strategies are ranked by scores. It is clear to see that
only the high ranking strategies survive the evolutionary process (in fact,
only \input{./assets/img/replicator_dynamics/main.tex}
have a final distribution greater than \(10 ^ {-2}\)). This confirms the
findings of~\cite{Moran1707} in which sophisticated strategies resist
evolutionary invasion of shorter memory strategies. Recalling
Figure~\ref{fig:SSError_and_probabilities_in_full} this demonstrates that:

\begin{itemize}
    \item Cooperation emerges through the evolutionary process: the high scoring
        strategies do not exhibit extortionate behaviour towards each other.
    \item Extortionate strategies do not survive the evolutionary process.
\end{itemize}

\begin{figure}[!htbp]
    \centering
    \includegraphics[width=.8\textwidth]{./assets/img/replicator_dynamics/main.pdf}
    \caption{Numerical simulation of the replicator equation
    (\ref{eqn:replicator_dynamics}): strategies are ordered by score, only the strategies with a high score survive the evolutionary process.}
    \label{fig:replicator_dynamics}
\end{figure}

This work can be used to classify plays of the IPD\@: data can be collected from
actual interactions (in lab or in the field). Furthermore, this allows for a
classification method similar to the notion of fingerprinting presented
in~\cite{Ashlock2008}. Trained strategies can potentially be classified as
extortionate or not or it could be possible to even constrain the reinforcement
learning approaches that are becoming prevalent in the literature.
Alternatively, this mathematical approach for recognising extortion could be
used in sophisticated strategies to defend against invasion. Arguably, some of
the strategies considered here exhibit this behaviour, indeed as described
in~\cite{Harper2017}, the top ranking strategies in the full tournament are
obtained using evolutionary reinforcement learning techniques, thus, suspicion
of extortionate behaviour could in fact be an evolutionary trait.

\section*{Acknowledgements}

The following open source software libraries were used in this research:

\begin{itemize}
    \item The Axelrod ~\cite{Knight2016, Knight2018} library (IPD strategies and
        tournaments).
    \item The sympy library~\cite{Meurer2017} (verification of all symbolic
        calculations).
    \item The matplotlib~\cite{Droettboom2018} library (visualisation).
    \item The pandas~\cite{Structures2010}, dask~\cite{Dask2016} and
        NumPy~\cite{Oliphant2015} libraries (data manipulation).
    \item The SciPy~\cite{Jones2001} library (numerical integration of the
        replicator equation).
\end{itemize}

This work was performed using the computational facilities of the Advanced
Research Computing @ Cardiff (ARCCA) Division, Cardiff University.

\printbibliography

\newpage
\section*{Supplementary materials}

\includepdf{assets/pdf/proof_of_form_of_extortionate_strategies/main.pdf}

\newpage

Using the pair wise interactions the transition rates \(p,
q\) can be measured and the steady state probabilities inferred and compared to
the actual probabilities of each state.
This is done numerically by computing the singular eigenvector of the
matrix \(A\) \cite{Stewart2009}:

\[
    A =
    \begin{bmatrix}
        p_1 q_1 & p_1 (1 - q_1) & (1 - p_1) q_1 & (1 -p_1) (1 - q_1) \\
        p_2 q_2 & p_2 (1 - q_2) & (1 - p_2) q_2 & (1 -p_2) (1 - q_2) \\
        p_3 q_3 & p_3 (1 - q_3) & (1 - p_3) q_3 & (1 -p_3) (1 - q_3) \\
        p_4 q_4 & p_4 (1 - q_4) & (1 - p_4) q_4 & (1 -p_4) (1 - q_4) \\
    \end{bmatrix}
\]

Figure~\ref{fig:computed_probabilities_vs_theoretic_probabilities} shows a
regression line fitted to every pairwise interaction with a reported
\(\text{SSError}\) value (pairwise interactions with missing states were
omitted). This serves to validate the approach: a part from some edge cases the
relationship is consistent.

\begin{figure}[!htbp]
    \centering
    \includegraphics[width=.8\textwidth]{./assets/img/computed_probabilities_vs_theoretic_probabilities/main.pdf}
    \caption{The
        relationship between the steady state probabilities inferred from the
        measured transitions and the actual steady state probabilities. A linear
        regression line is included validating the approach.}
    \label{fig:computed_probabilities_vs_theoretic_probabilities}
\end{figure}


\end{document}

    strategies is considered. In this setting
    the most highly performing strategies do not play in an extortionate way
    against each other but do against lower performing strategies.
    This suggests that whilst the theory of Zero Determinant strategies
    indicates that memory is not of fundamental importance to the evolution of
    cooperative behaviour, this is incomplete.
\end{abstract}

\section{Introduction}\label{sec:introduction}

Agent based game theoretic models have become a stalwart of the underpinning
mathematics of interactive behaviours. One of the major pieces of work
in this area is the pair of original computer tournaments run by Robert
Axelrod~\cite{Axelrod1980, Axelrod1980a}. These tournaments pitted submitted
computer strategies against each other in plays of the Iterated Prisoner's
Dilemma. A common game where agents can choose to pay a slight cost to their
immediate utility in the hope of building a reputation. This has been used in
economic and evolutionary game theory to understand the evolution of cooperative
behaviour.

Recently, a class of strategies was described in~\cite{Press2012} that can
provably extort any given opponent. In~\cite{Hilbe2013, Moran1707} some
questions have already been asked about the true effectiveness of these
strategies in an evolutionary setting. Here another question is asked: is it
possible to recognise this extortionate behaviour? A mathematical procedure for
suspicion is presented: in the same way that the continued actions of an
extortionate individual might raise suspicion.

This work makes use of the Axelrod Python library~\cite{Knight2018, Knight2016}
with a large number of Prisoner Dilemma strategies available to give an
extensive numerical example of the ideas presented.  The approach is presented
in Section~\ref{sec:delta-zd-strategies}.  All of the code and data discussed
in Section~\ref{sec:numerical-experiments} is open sourced, archived and
written according to best scientific principles~\cite{Wilson2014}. The data
archive can be found at~\cite{vincent_knight_2018_1297075}.

\section{Recognising Extortion}\label{sec:delta-zd-strategies}

In~\cite{Press2012}, given a match between 2 memory-one strategies, the concept
of Zero Determinant (ZD) strategies is introduced. The main result of that paper
shows that given two memory one players \(p, q\in\mathbb{R}^4\) a linear
relationship between the players' scores could be forced by one of the players.

Using the notation of~\cite{Press2012}, assuming the utilities for player \(p\)
are given by \(S_x=(R, S, T, P)\) and for player \(q\) by \(S_y=(R, T, S, P)\)
and that the stationary scores of each player is given by \(S_X\) and \(S_Y\)
respectively. The main result of~\cite{Press2012} is that if

\begin{equation}\label{eqn:linear_relationship_for_p}
    \tilde p=\alpha S_x + \beta S_y + \gamma
\end{equation}

or

\begin{equation}\label{eqn:linear_relationship_for_q}
    \tilde q=\alpha S_x + \beta S_y + \gamma
\end{equation}

where \(\tilde p = (1 - p_1, 1 - p_2, p_3, p_4)\) and
\(\tilde q = (1 - q_1, 1 - q_2, q_3, q_4)\) then:

\begin{equation}
    \alpha S_X + \beta S_Y + \gamma = 0
\end{equation}

In~\cite{Press2012} a particular type of ZD strategy is defined: extortionate
strategies. If:

\begin{equation}\label{eqn:constraint_for_extortion}
    \gamma = - P(\alpha + \beta)
\end{equation}

then the player can ensure they get a score \(\chi\) times
larger than the opponent. This extortion coefficient is given by:

\begin{equation}\label{eqn:definition_of_chi}
    \chi=\frac{-\beta}{\alpha}
\end{equation}

Thus, if (\ref{eqn:constraint_for_extortion}) holds and \(\chi >1\) a player is
said to extort their opponent.
Here, the reverse problem is considered: given a
\(p\in\mathbb{R}^4\) how does one identify \(\alpha, \beta\) if they
exist and is the strategy in fact acting in an extortionate way?

These conditions correspond to:

\begin{align}
    \tilde p_1 & = \alpha R + \beta R - P (\alpha + \beta)
            \label{eqn:condition_for_tilde_p1}\\
    \tilde p_2 & = \alpha S + \beta T - P (\alpha + \beta)
            \label{eqn:condition_for_tilde_p2}\\
    \tilde p_3 & = \alpha T + \beta S - P (\alpha + \beta)
            \label{eqn:condition_for_tilde_p3}\\
    \tilde p_4 & = \alpha P + \beta P - P (\alpha + \beta)
            \label{eqn:condition_for_tilde_p4}
\end{align}

Equation (\ref{eqn:condition_for_tilde_p4}) ensures that \(p_4=\tilde p_4=0\).
Equations (\ref{eqn:condition_for_tilde_p1}-\ref{eqn:condition_for_tilde_p3})
can be used to eliminate \(\alpha, \beta\), giving:

\begin{equation}\label{eqn:planar_definition_of_extortion}
    \tilde p_1 = \frac{(R - P)(\tilde p_2 + \tilde p_3)}{S + T - 2P}
\end{equation}

with:

\begin{equation}\label{eqn:definition_of_chi}
    \chi = \frac{\tilde p_2 (P - T) + \tilde p_3 (S - P)}
                {\tilde p_2 (P - S) + \tilde p_3 (T - P)}
\end{equation}

Given a strategy \(p\in\mathbb{R}^{4\times 1}\) equations
(\ref{eqn:condition_for_tilde_p4}), (\ref{eqn:planar_definition_of_extortion}-\ref{eqn:definition_of_chi}) can be used to check if
a strategy is extortionate. The conditions correspond to:

\begin{align}
    p_1 & = \frac{(R-P)(p_2 + p_3) - R + T + S - P}{S + T - 2P}
     \label{eqn:condition_for_p1}\\
    p_4 & = 0 \label{eqn:condition_for_p4}\\
    1 & > p_2 + p_3\label{eqn:condition_for_chi}
\end{align}

The algebraic steps necessary to prove these results are available in the
supporting materials.

All extortionate strategies reside on a triangular (\ref{eqn:condition_for_chi})
plane (\ref{eqn:condition_for_p1}) in 3 dimensions (\ref{eqn:condition_for_p4}).
Using this formulation it can be seen that a necessary (but not sufficient)
condition for an extortionate strategy is that it cooperates on average less
than 50\% of the time when in a state of disagreement with the opponent.

As an example, consider the known extortionate strategy \(p=(8 / 9, 1 / 2, 1 /
3, 0)\) from~\cite{Stewart2012} which is referred to as \texttt{Extort-2}. In
this case, for the standard values of \((R, T, S, P)\) constraint
(\ref{eqn:condition_for_p1}) corresponds to:

\begin{equation}
    p_1 = \frac{2(p_2 + p_3) + 1}{3}
\end{equation}

It is clear that in this case all constraints hold.

This approach could in fact be used to confirm that a given strategy is acting
in an extortionate manner even if it is not a memory one strategy. However, in
practice, if a closed form for \(p\) is not known, then due to measurement
and/or numerical error this would not work.

This problem can be written in the following linear algebraic form where
\(x=(\alpha, \beta)\)
and \(p^*=(\tilde p_1 - 1, tilde_2 - 1, p_3)\):

\begin{equation}\label{eqn:linear_algebraic_equation_for_p}
    Cx= p^*
\end{equation}

\(C\) corresponds to equations
(\ref{eqn:condition_for_tilde_p1}-\ref{eqn:condition_for_tilde_p3}) and is
given by:

\begin{equation}\label{eqn:definition_of_C}
    C =
    \begin{bmatrix}
        R - P & R- P \\
        S - P & T- P \\
        T - P & S- P \\
    \end{bmatrix}
\end{equation}

Note that in general, equation (\ref{eqn:linear_algebraic_equation_for_p}) will
not necessarily have a solution. From the Rouch\'{e}-Capelli theorem if there is
a solution it is unique as \(\text{rank}(C)=2\) which is the dimension of the
variable \(x\). The best fitting \(x\) is found by minimizing:

\begin{equation}\label{eqn:r_squared}
    \text{SSError} = \|C x- p^*\|_2^2 = \sum_{i=1}^{3}\left((C\bar x)_i-p_i^*\right)^2
\end{equation}

Note that \(\text{SSError}\), which is the square of the Frobenius
norm~\cite{Golub2013}, becomes a measure of how close a strategy is to being an
extortionate strategy. Suspicion
of extortion then corresponds to a threshold on \(\text{SSError}\).

By observing interactions (human or otherwise), their memory one representation
can be inferred and this approach can be used to recognise extortionate
behaviour. The notion of comparing theoretic and actual plays of the IPD is not
novel, see for example~\cite{Rand2013}. Immediately it is noted that if the
environment is noisy~\cite{Wu1995} then no strategy can be considered to be
extortionate as \(p_4>0\).

In the next section, this idea will be illustrated by observing the interactions
that take place in a computer based tournament of the IPD\@.

\section{Numerical experiments}\label{sec:numerical-experiments}

In~\cite{Stewart2012} results from a tournament with
\documentclass[a4paper]{article}

\usepackage{amsmath}
\usepackage{amssymb}
\usepackage[margin=1.5cm,
            includefoot,
            footskip=30pt]{geometry}
\usepackage{layout}
\usepackage{graphicx}
\usepackage{subcaption}

\usepackage{biblatex}
\usepackage{pdfpages}

\bibliography{main.bib}

\title{Suspicion: Recognising and evaluating the effectiveness
       of extortion in the Iterated Prisoner's Dilemma}
\author{Vincent A. Knight \and Nikoleta E. Glynatsi}
\date{\today}



\begin{document}

\maketitle

\begin{abstract}
    The Iterated Prisoner's Dilemma is a model for rational and evolutionary
    interactive behaviour. It has applications both in the study of human social
    behaviour as well as in biology.
    It is used to understand when and how a rational individual might
    accept an immediate cost to their own utility for the direct benefit of
    another.

    Much attention has been given to a class of strategies called
    Zero Determinant strategies. It has been theoretically shown that these
    strategies can ``extort'' any player.

    In this work, an approach to identify if observed strategies are playing in
    an extortionate way is described. Furthermore, experimental analysis of
    a large tournament with \input{assets/tex/number_of_full_strategies/main.tex}
    strategies is considered. In this setting
    the most highly performing strategies do not play in an extortionate way
    against each other but do against lower performing strategies.
    This suggests that whilst the theory of Zero Determinant strategies
    indicates that memory is not of fundamental importance to the evolution of
    cooperative behaviour, this is incomplete.
\end{abstract}

\section{Introduction}\label{sec:introduction}

Agent based game theoretic models have become a stalwart of the underpinning
mathematics of interactive behaviours. One of the major pieces of work
in this area is the pair of original computer tournaments run by Robert
Axelrod~\cite{Axelrod1980, Axelrod1980a}. These tournaments pitted submitted
computer strategies against each other in plays of the Iterated Prisoner's
Dilemma. A common game where agents can choose to pay a slight cost to their
immediate utility in the hope of building a reputation. This has been used in
economic and evolutionary game theory to understand the evolution of cooperative
behaviour.

Recently, a class of strategies was described in~\cite{Press2012} that can
provably extort any given opponent. In~\cite{Hilbe2013, Moran1707} some
questions have already been asked about the true effectiveness of these
strategies in an evolutionary setting. Here another question is asked: is it
possible to recognise this extortionate behaviour? A mathematical procedure for
suspicion is presented: in the same way that the continued actions of an
extortionate individual might raise suspicion.

This work makes use of the Axelrod Python library~\cite{Knight2018, Knight2016}
with a large number of Prisoner Dilemma strategies available to give an
extensive numerical example of the ideas presented.  The approach is presented
in Section~\ref{sec:delta-zd-strategies}.  All of the code and data discussed
in Section~\ref{sec:numerical-experiments} is open sourced, archived and
written according to best scientific principles~\cite{Wilson2014}. The data
archive can be found at~\cite{vincent_knight_2018_1297075}.

\section{Recognising Extortion}\label{sec:delta-zd-strategies}

In~\cite{Press2012}, given a match between 2 memory-one strategies, the concept
of Zero Determinant (ZD) strategies is introduced. The main result of that paper
shows that given two memory one players \(p, q\in\mathbb{R}^4\) a linear
relationship between the players' scores could be forced by one of the players.

Using the notation of~\cite{Press2012}, assuming the utilities for player \(p\)
are given by \(S_x=(R, S, T, P)\) and for player \(q\) by \(S_y=(R, T, S, P)\)
and that the stationary scores of each player is given by \(S_X\) and \(S_Y\)
respectively. The main result of~\cite{Press2012} is that if

\begin{equation}\label{eqn:linear_relationship_for_p}
    \tilde p=\alpha S_x + \beta S_y + \gamma
\end{equation}

or

\begin{equation}\label{eqn:linear_relationship_for_q}
    \tilde q=\alpha S_x + \beta S_y + \gamma
\end{equation}

where \(\tilde p = (1 - p_1, 1 - p_2, p_3, p_4)\) and
\(\tilde q = (1 - q_1, 1 - q_2, q_3, q_4)\) then:

\begin{equation}
    \alpha S_X + \beta S_Y + \gamma = 0
\end{equation}

In~\cite{Press2012} a particular type of ZD strategy is defined: extortionate
strategies. If:

\begin{equation}\label{eqn:constraint_for_extortion}
    \gamma = - P(\alpha + \beta)
\end{equation}

then the player can ensure they get a score \(\chi\) times
larger than the opponent. This extortion coefficient is given by:

\begin{equation}\label{eqn:definition_of_chi}
    \chi=\frac{-\beta}{\alpha}
\end{equation}

Thus, if (\ref{eqn:constraint_for_extortion}) holds and \(\chi >1\) a player is
said to extort their opponent.
Here, the reverse problem is considered: given a
\(p\in\mathbb{R}^4\) how does one identify \(\alpha, \beta\) if they
exist and is the strategy in fact acting in an extortionate way?

These conditions correspond to:

\begin{align}
    \tilde p_1 & = \alpha R + \beta R - P (\alpha + \beta)
            \label{eqn:condition_for_tilde_p1}\\
    \tilde p_2 & = \alpha S + \beta T - P (\alpha + \beta)
            \label{eqn:condition_for_tilde_p2}\\
    \tilde p_3 & = \alpha T + \beta S - P (\alpha + \beta)
            \label{eqn:condition_for_tilde_p3}\\
    \tilde p_4 & = \alpha P + \beta P - P (\alpha + \beta)
            \label{eqn:condition_for_tilde_p4}
\end{align}

Equation (\ref{eqn:condition_for_tilde_p4}) ensures that \(p_4=\tilde p_4=0\).
Equations (\ref{eqn:condition_for_tilde_p1}-\ref{eqn:condition_for_tilde_p3})
can be used to eliminate \(\alpha, \beta\), giving:

\begin{equation}\label{eqn:planar_definition_of_extortion}
    \tilde p_1 = \frac{(R - P)(\tilde p_2 + \tilde p_3)}{S + T - 2P}
\end{equation}

with:

\begin{equation}\label{eqn:definition_of_chi}
    \chi = \frac{\tilde p_2 (P - T) + \tilde p_3 (S - P)}
                {\tilde p_2 (P - S) + \tilde p_3 (T - P)}
\end{equation}

Given a strategy \(p\in\mathbb{R}^{4\times 1}\) equations
(\ref{eqn:condition_for_tilde_p4}), (\ref{eqn:planar_definition_of_extortion}-\ref{eqn:definition_of_chi}) can be used to check if
a strategy is extortionate. The conditions correspond to:

\begin{align}
    p_1 & = \frac{(R-P)(p_2 + p_3) - R + T + S - P}{S + T - 2P}
     \label{eqn:condition_for_p1}\\
    p_4 & = 0 \label{eqn:condition_for_p4}\\
    1 & > p_2 + p_3\label{eqn:condition_for_chi}
\end{align}

The algebraic steps necessary to prove these results are available in the
supporting materials.

All extortionate strategies reside on a triangular (\ref{eqn:condition_for_chi})
plane (\ref{eqn:condition_for_p1}) in 3 dimensions (\ref{eqn:condition_for_p4}).
Using this formulation it can be seen that a necessary (but not sufficient)
condition for an extortionate strategy is that it cooperates on average less
than 50\% of the time when in a state of disagreement with the opponent.

As an example, consider the known extortionate strategy \(p=(8 / 9, 1 / 2, 1 /
3, 0)\) from~\cite{Stewart2012} which is referred to as \texttt{Extort-2}. In
this case, for the standard values of \((R, T, S, P)\) constraint
(\ref{eqn:condition_for_p1}) corresponds to:

\begin{equation}
    p_1 = \frac{2(p_2 + p_3) + 1}{3}
\end{equation}

It is clear that in this case all constraints hold.

This approach could in fact be used to confirm that a given strategy is acting
in an extortionate manner even if it is not a memory one strategy. However, in
practice, if a closed form for \(p\) is not known, then due to measurement
and/or numerical error this would not work.

This problem can be written in the following linear algebraic form where
\(x=(\alpha, \beta)\)
and \(p^*=(\tilde p_1 - 1, tilde_2 - 1, p_3)\):

\begin{equation}\label{eqn:linear_algebraic_equation_for_p}
    Cx= p^*
\end{equation}

\(C\) corresponds to equations
(\ref{eqn:condition_for_tilde_p1}-\ref{eqn:condition_for_tilde_p3}) and is
given by:

\begin{equation}\label{eqn:definition_of_C}
    C =
    \begin{bmatrix}
        R - P & R- P \\
        S - P & T- P \\
        T - P & S- P \\
    \end{bmatrix}
\end{equation}

Note that in general, equation (\ref{eqn:linear_algebraic_equation_for_p}) will
not necessarily have a solution. From the Rouch\'{e}-Capelli theorem if there is
a solution it is unique as \(\text{rank}(C)=2\) which is the dimension of the
variable \(x\). The best fitting \(x\) is found by minimizing:

\begin{equation}\label{eqn:r_squared}
    \text{SSError} = \|C x- p^*\|_2^2 = \sum_{i=1}^{3}\left((C\bar x)_i-p_i^*\right)^2
\end{equation}

Note that \(\text{SSError}\), which is the square of the Frobenius
norm~\cite{Golub2013}, becomes a measure of how close a strategy is to being an
extortionate strategy. Suspicion
of extortion then corresponds to a threshold on \(\text{SSError}\).

By observing interactions (human or otherwise), their memory one representation
can be inferred and this approach can be used to recognise extortionate
behaviour. The notion of comparing theoretic and actual plays of the IPD is not
novel, see for example~\cite{Rand2013}. Immediately it is noted that if the
environment is noisy~\cite{Wu1995} then no strategy can be considered to be
extortionate as \(p_4>0\).

In the next section, this idea will be illustrated by observing the interactions
that take place in a computer based tournament of the IPD\@.

\section{Numerical experiments}\label{sec:numerical-experiments}

In~\cite{Stewart2012} results from a tournament with
\input{./assets/tex/number_of_stewart_plotkin_strategies/main.tex} strategies,
was presented with specific consideration given to ZD strategies. This
tournament is reproduced here using the Axelrod-Python
project~\cite{Knight2016}. To obtain a good measure of the corresponding
transition rates for each strategy all matches have been run for
\input{assets/tex/number_of_turns/main.tex} turns and every match has been
repeated \input{assets/tex/number_of_repetitions/main.tex} times. All of this
interaction data is available at~\cite{vincent_knight_2018_1297075}. A good
match between the inferred Markov chain and the state distribution of the actual
interactions has been verified. Data for this is presented in the supplementary
materials.

Figure~\ref{fig:SSError_overall_in_stewart_plotkin} shows the \(\text{SSError}\)
values for all the strategies in the tournament, as reported
in~\cite{Stewart2012} the extortionate strategy (which has an expected
\(\text{SSError}\) approximately 0) gains a large number of wins.

\begin{figure}[!htbp]
    \centering
    \includegraphics[width=.8\textwidth]{./assets/img/SSError_overall_in_stewart_plotkin/main.pdf}
    \caption{\(\text{SSError}\) and state probabilities for the strategies
        of~\cite{Stewart2012}, ordered both by number of wins and overall score.
        Note that \(P(DC)\) is not shown as it corresponds to the transpose of
        \(P(CD)\). Cooperator and Defector are omitted as they do not visit all
        the states.}
    \label{fig:SSError_overall_in_stewart_plotkin}
\end{figure}

Here, the work of~\cite{Stewart2012} is extended by investigating a tournament
with \input{assets/tex/number_of_full_strategies/main.tex}
strategies.

The results of this analysis are shown in
Figure~\ref{fig:SSError_and_probabilities_in_full}. The top ranking strategies
by number of wins seem to be extortionate (but not against all strategies) and
it can be seen that a small sub group of strategies achieve mutual defection.
All the top ranking strategies according to score achieve mutual cooperation and
do not extort each other, however they
\textbf{do} exhibit extortionate behaviour towards a number of the lower ranking
strategies.

\begin{figure}[!htbp]
    \centering
    \includegraphics[width=.8\textwidth]{./assets/img/SSError_and_probabilities_in_full/main.pdf}
    \caption{\(\text{SSError}\) for the strategies for the full tournament. Only
    strategy interactions for which \(p_4=0\) and \(\chi>1\) are displayed.}
    \label{fig:SSError_and_probabilities_in_full}
\end{figure}

\section{Conclusion}\label{sec:conclusion}

This work defines an approach to measure whether or not a player is playing a
strategy that corresponds to an extortionate strategy as defined
in~\cite{Press2012}: a mathematical model for suspicion. Indeed, all
extortionate strategies have been
 classified as lying on a triangular plane.
This rigorous classification fails to be robust to small measurement error, thus
a statistical approach is proposed.
This is done through a linear algebraic approach for approximating the solution
of a linear system. Using this, a large number of pairwise interactions is
simulated and in fact very few strategies are found to act extortionately.

The work of~\cite{Press2012}, whilst showing that a clever approach to taking
advantage of another memory one strategy exists: this is incomplete. Whilst the
elegance of this result is very attractive, just as the simplicity of the
victory of Tit For Tat in Axelrod's original tournaments was, it is incomplete.
Extortionate strategies achieve a high number of wins but they do not
achieve a high score which corresponds to the fitness landscape in an
evolutionary sense. From the large number of interactions a payoff matrix \(S\)
can be measured where \(S_{ij}\) denotes the score (using standard values of
\((R, S, T, P) = (3, 0, 5, 1)\)) of the \(i\)th strategy
against the \(j\)th strategy. Using this, the replicator equation
describes the evolution of the system based on a population density fitness
function:

\begin{equation}\label{eqn:replicator_dynamics}
    \frac{dx}{dt} = x(S-x^TS x)
\end{equation}

Equation (\ref{eqn:replicator_dynamics}) is solved numerically through an
integration technique described in~\cite{Petzold1983} and
Figure~\ref{fig:replicator_dynamics} shows the evolution of the distribution of
the system: the various strategies are ranked by scores. It is clear to see that
only the high ranking strategies survive the evolutionary process (in fact,
only \input{./assets/img/replicator_dynamics/main.tex}
have a final distribution greater than \(10 ^ {-2}\)). This confirms the
findings of~\cite{Moran1707} in which sophisticated strategies resist
evolutionary invasion of shorter memory strategies. Recalling
Figure~\ref{fig:SSError_and_probabilities_in_full} this demonstrates that:

\begin{itemize}
    \item Cooperation emerges through the evolutionary process: the high scoring
        strategies do not exhibit extortionate behaviour towards each other.
    \item Extortionate strategies do not survive the evolutionary process.
\end{itemize}

\begin{figure}[!htbp]
    \centering
    \includegraphics[width=.8\textwidth]{./assets/img/replicator_dynamics/main.pdf}
    \caption{Numerical simulation of the replicator equation
    (\ref{eqn:replicator_dynamics}): strategies are ordered by score, only the strategies with a high score survive the evolutionary process.}
    \label{fig:replicator_dynamics}
\end{figure}

This work can be used to classify plays of the IPD\@: data can be collected from
actual interactions (in lab or in the field). Furthermore, this allows for a
classification method similar to the notion of fingerprinting presented
in~\cite{Ashlock2008}. Trained strategies can potentially be classified as
extortionate or not or it could be possible to even constrain the reinforcement
learning approaches that are becoming prevalent in the literature.
Alternatively, this mathematical approach for recognising extortion could be
used in sophisticated strategies to defend against invasion. Arguably, some of
the strategies considered here exhibit this behaviour, indeed as described
in~\cite{Harper2017}, the top ranking strategies in the full tournament are
obtained using evolutionary reinforcement learning techniques, thus, suspicion
of extortionate behaviour could in fact be an evolutionary trait.

\section*{Acknowledgements}

The following open source software libraries were used in this research:

\begin{itemize}
    \item The Axelrod ~\cite{Knight2016, Knight2018} library (IPD strategies and
        tournaments).
    \item The sympy library~\cite{Meurer2017} (verification of all symbolic
        calculations).
    \item The matplotlib~\cite{Droettboom2018} library (visualisation).
    \item The pandas~\cite{Structures2010}, dask~\cite{Dask2016} and
        NumPy~\cite{Oliphant2015} libraries (data manipulation).
    \item The SciPy~\cite{Jones2001} library (numerical integration of the
        replicator equation).
\end{itemize}

This work was performed using the computational facilities of the Advanced
Research Computing @ Cardiff (ARCCA) Division, Cardiff University.

\printbibliography

\newpage
\section*{Supplementary materials}

\includepdf{assets/pdf/proof_of_form_of_extortionate_strategies/main.pdf}

\newpage

Using the pair wise interactions the transition rates \(p,
q\) can be measured and the steady state probabilities inferred and compared to
the actual probabilities of each state.
This is done numerically by computing the singular eigenvector of the
matrix \(A\) \cite{Stewart2009}:

\[
    A =
    \begin{bmatrix}
        p_1 q_1 & p_1 (1 - q_1) & (1 - p_1) q_1 & (1 -p_1) (1 - q_1) \\
        p_2 q_2 & p_2 (1 - q_2) & (1 - p_2) q_2 & (1 -p_2) (1 - q_2) \\
        p_3 q_3 & p_3 (1 - q_3) & (1 - p_3) q_3 & (1 -p_3) (1 - q_3) \\
        p_4 q_4 & p_4 (1 - q_4) & (1 - p_4) q_4 & (1 -p_4) (1 - q_4) \\
    \end{bmatrix}
\]

Figure~\ref{fig:computed_probabilities_vs_theoretic_probabilities} shows a
regression line fitted to every pairwise interaction with a reported
\(\text{SSError}\) value (pairwise interactions with missing states were
omitted). This serves to validate the approach: a part from some edge cases the
relationship is consistent.

\begin{figure}[!htbp]
    \centering
    \includegraphics[width=.8\textwidth]{./assets/img/computed_probabilities_vs_theoretic_probabilities/main.pdf}
    \caption{The
        relationship between the steady state probabilities inferred from the
        measured transitions and the actual steady state probabilities. A linear
        regression line is included validating the approach.}
    \label{fig:computed_probabilities_vs_theoretic_probabilities}
\end{figure}


\end{document}
 strategies,
was presented with specific consideration given to ZD strategies. This
tournament is reproduced here using the Axelrod-Python
project~\cite{Knight2016}. To obtain a good measure of the corresponding
transition rates for each strategy all matches have been run for
\documentclass[a4paper]{article}

\usepackage{amsmath}
\usepackage{amssymb}
\usepackage[margin=1.5cm,
            includefoot,
            footskip=30pt]{geometry}
\usepackage{layout}
\usepackage{graphicx}
\usepackage{subcaption}

\usepackage{biblatex}
\usepackage{pdfpages}

\bibliography{main.bib}

\title{Suspicion: Recognising and evaluating the effectiveness
       of extortion in the Iterated Prisoner's Dilemma}
\author{Vincent A. Knight \and Nikoleta E. Glynatsi}
\date{\today}



\begin{document}

\maketitle

\begin{abstract}
    The Iterated Prisoner's Dilemma is a model for rational and evolutionary
    interactive behaviour. It has applications both in the study of human social
    behaviour as well as in biology.
    It is used to understand when and how a rational individual might
    accept an immediate cost to their own utility for the direct benefit of
    another.

    Much attention has been given to a class of strategies called
    Zero Determinant strategies. It has been theoretically shown that these
    strategies can ``extort'' any player.

    In this work, an approach to identify if observed strategies are playing in
    an extortionate way is described. Furthermore, experimental analysis of
    a large tournament with \input{assets/tex/number_of_full_strategies/main.tex}
    strategies is considered. In this setting
    the most highly performing strategies do not play in an extortionate way
    against each other but do against lower performing strategies.
    This suggests that whilst the theory of Zero Determinant strategies
    indicates that memory is not of fundamental importance to the evolution of
    cooperative behaviour, this is incomplete.
\end{abstract}

\section{Introduction}\label{sec:introduction}

Agent based game theoretic models have become a stalwart of the underpinning
mathematics of interactive behaviours. One of the major pieces of work
in this area is the pair of original computer tournaments run by Robert
Axelrod~\cite{Axelrod1980, Axelrod1980a}. These tournaments pitted submitted
computer strategies against each other in plays of the Iterated Prisoner's
Dilemma. A common game where agents can choose to pay a slight cost to their
immediate utility in the hope of building a reputation. This has been used in
economic and evolutionary game theory to understand the evolution of cooperative
behaviour.

Recently, a class of strategies was described in~\cite{Press2012} that can
provably extort any given opponent. In~\cite{Hilbe2013, Moran1707} some
questions have already been asked about the true effectiveness of these
strategies in an evolutionary setting. Here another question is asked: is it
possible to recognise this extortionate behaviour? A mathematical procedure for
suspicion is presented: in the same way that the continued actions of an
extortionate individual might raise suspicion.

This work makes use of the Axelrod Python library~\cite{Knight2018, Knight2016}
with a large number of Prisoner Dilemma strategies available to give an
extensive numerical example of the ideas presented.  The approach is presented
in Section~\ref{sec:delta-zd-strategies}.  All of the code and data discussed
in Section~\ref{sec:numerical-experiments} is open sourced, archived and
written according to best scientific principles~\cite{Wilson2014}. The data
archive can be found at~\cite{vincent_knight_2018_1297075}.

\section{Recognising Extortion}\label{sec:delta-zd-strategies}

In~\cite{Press2012}, given a match between 2 memory-one strategies, the concept
of Zero Determinant (ZD) strategies is introduced. The main result of that paper
shows that given two memory one players \(p, q\in\mathbb{R}^4\) a linear
relationship between the players' scores could be forced by one of the players.

Using the notation of~\cite{Press2012}, assuming the utilities for player \(p\)
are given by \(S_x=(R, S, T, P)\) and for player \(q\) by \(S_y=(R, T, S, P)\)
and that the stationary scores of each player is given by \(S_X\) and \(S_Y\)
respectively. The main result of~\cite{Press2012} is that if

\begin{equation}\label{eqn:linear_relationship_for_p}
    \tilde p=\alpha S_x + \beta S_y + \gamma
\end{equation}

or

\begin{equation}\label{eqn:linear_relationship_for_q}
    \tilde q=\alpha S_x + \beta S_y + \gamma
\end{equation}

where \(\tilde p = (1 - p_1, 1 - p_2, p_3, p_4)\) and
\(\tilde q = (1 - q_1, 1 - q_2, q_3, q_4)\) then:

\begin{equation}
    \alpha S_X + \beta S_Y + \gamma = 0
\end{equation}

In~\cite{Press2012} a particular type of ZD strategy is defined: extortionate
strategies. If:

\begin{equation}\label{eqn:constraint_for_extortion}
    \gamma = - P(\alpha + \beta)
\end{equation}

then the player can ensure they get a score \(\chi\) times
larger than the opponent. This extortion coefficient is given by:

\begin{equation}\label{eqn:definition_of_chi}
    \chi=\frac{-\beta}{\alpha}
\end{equation}

Thus, if (\ref{eqn:constraint_for_extortion}) holds and \(\chi >1\) a player is
said to extort their opponent.
Here, the reverse problem is considered: given a
\(p\in\mathbb{R}^4\) how does one identify \(\alpha, \beta\) if they
exist and is the strategy in fact acting in an extortionate way?

These conditions correspond to:

\begin{align}
    \tilde p_1 & = \alpha R + \beta R - P (\alpha + \beta)
            \label{eqn:condition_for_tilde_p1}\\
    \tilde p_2 & = \alpha S + \beta T - P (\alpha + \beta)
            \label{eqn:condition_for_tilde_p2}\\
    \tilde p_3 & = \alpha T + \beta S - P (\alpha + \beta)
            \label{eqn:condition_for_tilde_p3}\\
    \tilde p_4 & = \alpha P + \beta P - P (\alpha + \beta)
            \label{eqn:condition_for_tilde_p4}
\end{align}

Equation (\ref{eqn:condition_for_tilde_p4}) ensures that \(p_4=\tilde p_4=0\).
Equations (\ref{eqn:condition_for_tilde_p1}-\ref{eqn:condition_for_tilde_p3})
can be used to eliminate \(\alpha, \beta\), giving:

\begin{equation}\label{eqn:planar_definition_of_extortion}
    \tilde p_1 = \frac{(R - P)(\tilde p_2 + \tilde p_3)}{S + T - 2P}
\end{equation}

with:

\begin{equation}\label{eqn:definition_of_chi}
    \chi = \frac{\tilde p_2 (P - T) + \tilde p_3 (S - P)}
                {\tilde p_2 (P - S) + \tilde p_3 (T - P)}
\end{equation}

Given a strategy \(p\in\mathbb{R}^{4\times 1}\) equations
(\ref{eqn:condition_for_tilde_p4}), (\ref{eqn:planar_definition_of_extortion}-\ref{eqn:definition_of_chi}) can be used to check if
a strategy is extortionate. The conditions correspond to:

\begin{align}
    p_1 & = \frac{(R-P)(p_2 + p_3) - R + T + S - P}{S + T - 2P}
     \label{eqn:condition_for_p1}\\
    p_4 & = 0 \label{eqn:condition_for_p4}\\
    1 & > p_2 + p_3\label{eqn:condition_for_chi}
\end{align}

The algebraic steps necessary to prove these results are available in the
supporting materials.

All extortionate strategies reside on a triangular (\ref{eqn:condition_for_chi})
plane (\ref{eqn:condition_for_p1}) in 3 dimensions (\ref{eqn:condition_for_p4}).
Using this formulation it can be seen that a necessary (but not sufficient)
condition for an extortionate strategy is that it cooperates on average less
than 50\% of the time when in a state of disagreement with the opponent.

As an example, consider the known extortionate strategy \(p=(8 / 9, 1 / 2, 1 /
3, 0)\) from~\cite{Stewart2012} which is referred to as \texttt{Extort-2}. In
this case, for the standard values of \((R, T, S, P)\) constraint
(\ref{eqn:condition_for_p1}) corresponds to:

\begin{equation}
    p_1 = \frac{2(p_2 + p_3) + 1}{3}
\end{equation}

It is clear that in this case all constraints hold.

This approach could in fact be used to confirm that a given strategy is acting
in an extortionate manner even if it is not a memory one strategy. However, in
practice, if a closed form for \(p\) is not known, then due to measurement
and/or numerical error this would not work.

This problem can be written in the following linear algebraic form where
\(x=(\alpha, \beta)\)
and \(p^*=(\tilde p_1 - 1, tilde_2 - 1, p_3)\):

\begin{equation}\label{eqn:linear_algebraic_equation_for_p}
    Cx= p^*
\end{equation}

\(C\) corresponds to equations
(\ref{eqn:condition_for_tilde_p1}-\ref{eqn:condition_for_tilde_p3}) and is
given by:

\begin{equation}\label{eqn:definition_of_C}
    C =
    \begin{bmatrix}
        R - P & R- P \\
        S - P & T- P \\
        T - P & S- P \\
    \end{bmatrix}
\end{equation}

Note that in general, equation (\ref{eqn:linear_algebraic_equation_for_p}) will
not necessarily have a solution. From the Rouch\'{e}-Capelli theorem if there is
a solution it is unique as \(\text{rank}(C)=2\) which is the dimension of the
variable \(x\). The best fitting \(x\) is found by minimizing:

\begin{equation}\label{eqn:r_squared}
    \text{SSError} = \|C x- p^*\|_2^2 = \sum_{i=1}^{3}\left((C\bar x)_i-p_i^*\right)^2
\end{equation}

Note that \(\text{SSError}\), which is the square of the Frobenius
norm~\cite{Golub2013}, becomes a measure of how close a strategy is to being an
extortionate strategy. Suspicion
of extortion then corresponds to a threshold on \(\text{SSError}\).

By observing interactions (human or otherwise), their memory one representation
can be inferred and this approach can be used to recognise extortionate
behaviour. The notion of comparing theoretic and actual plays of the IPD is not
novel, see for example~\cite{Rand2013}. Immediately it is noted that if the
environment is noisy~\cite{Wu1995} then no strategy can be considered to be
extortionate as \(p_4>0\).

In the next section, this idea will be illustrated by observing the interactions
that take place in a computer based tournament of the IPD\@.

\section{Numerical experiments}\label{sec:numerical-experiments}

In~\cite{Stewart2012} results from a tournament with
\input{./assets/tex/number_of_stewart_plotkin_strategies/main.tex} strategies,
was presented with specific consideration given to ZD strategies. This
tournament is reproduced here using the Axelrod-Python
project~\cite{Knight2016}. To obtain a good measure of the corresponding
transition rates for each strategy all matches have been run for
\input{assets/tex/number_of_turns/main.tex} turns and every match has been
repeated \input{assets/tex/number_of_repetitions/main.tex} times. All of this
interaction data is available at~\cite{vincent_knight_2018_1297075}. A good
match between the inferred Markov chain and the state distribution of the actual
interactions has been verified. Data for this is presented in the supplementary
materials.

Figure~\ref{fig:SSError_overall_in_stewart_plotkin} shows the \(\text{SSError}\)
values for all the strategies in the tournament, as reported
in~\cite{Stewart2012} the extortionate strategy (which has an expected
\(\text{SSError}\) approximately 0) gains a large number of wins.

\begin{figure}[!htbp]
    \centering
    \includegraphics[width=.8\textwidth]{./assets/img/SSError_overall_in_stewart_plotkin/main.pdf}
    \caption{\(\text{SSError}\) and state probabilities for the strategies
        of~\cite{Stewart2012}, ordered both by number of wins and overall score.
        Note that \(P(DC)\) is not shown as it corresponds to the transpose of
        \(P(CD)\). Cooperator and Defector are omitted as they do not visit all
        the states.}
    \label{fig:SSError_overall_in_stewart_plotkin}
\end{figure}

Here, the work of~\cite{Stewart2012} is extended by investigating a tournament
with \input{assets/tex/number_of_full_strategies/main.tex}
strategies.

The results of this analysis are shown in
Figure~\ref{fig:SSError_and_probabilities_in_full}. The top ranking strategies
by number of wins seem to be extortionate (but not against all strategies) and
it can be seen that a small sub group of strategies achieve mutual defection.
All the top ranking strategies according to score achieve mutual cooperation and
do not extort each other, however they
\textbf{do} exhibit extortionate behaviour towards a number of the lower ranking
strategies.

\begin{figure}[!htbp]
    \centering
    \includegraphics[width=.8\textwidth]{./assets/img/SSError_and_probabilities_in_full/main.pdf}
    \caption{\(\text{SSError}\) for the strategies for the full tournament. Only
    strategy interactions for which \(p_4=0\) and \(\chi>1\) are displayed.}
    \label{fig:SSError_and_probabilities_in_full}
\end{figure}

\section{Conclusion}\label{sec:conclusion}

This work defines an approach to measure whether or not a player is playing a
strategy that corresponds to an extortionate strategy as defined
in~\cite{Press2012}: a mathematical model for suspicion. Indeed, all
extortionate strategies have been
 classified as lying on a triangular plane.
This rigorous classification fails to be robust to small measurement error, thus
a statistical approach is proposed.
This is done through a linear algebraic approach for approximating the solution
of a linear system. Using this, a large number of pairwise interactions is
simulated and in fact very few strategies are found to act extortionately.

The work of~\cite{Press2012}, whilst showing that a clever approach to taking
advantage of another memory one strategy exists: this is incomplete. Whilst the
elegance of this result is very attractive, just as the simplicity of the
victory of Tit For Tat in Axelrod's original tournaments was, it is incomplete.
Extortionate strategies achieve a high number of wins but they do not
achieve a high score which corresponds to the fitness landscape in an
evolutionary sense. From the large number of interactions a payoff matrix \(S\)
can be measured where \(S_{ij}\) denotes the score (using standard values of
\((R, S, T, P) = (3, 0, 5, 1)\)) of the \(i\)th strategy
against the \(j\)th strategy. Using this, the replicator equation
describes the evolution of the system based on a population density fitness
function:

\begin{equation}\label{eqn:replicator_dynamics}
    \frac{dx}{dt} = x(S-x^TS x)
\end{equation}

Equation (\ref{eqn:replicator_dynamics}) is solved numerically through an
integration technique described in~\cite{Petzold1983} and
Figure~\ref{fig:replicator_dynamics} shows the evolution of the distribution of
the system: the various strategies are ranked by scores. It is clear to see that
only the high ranking strategies survive the evolutionary process (in fact,
only \input{./assets/img/replicator_dynamics/main.tex}
have a final distribution greater than \(10 ^ {-2}\)). This confirms the
findings of~\cite{Moran1707} in which sophisticated strategies resist
evolutionary invasion of shorter memory strategies. Recalling
Figure~\ref{fig:SSError_and_probabilities_in_full} this demonstrates that:

\begin{itemize}
    \item Cooperation emerges through the evolutionary process: the high scoring
        strategies do not exhibit extortionate behaviour towards each other.
    \item Extortionate strategies do not survive the evolutionary process.
\end{itemize}

\begin{figure}[!htbp]
    \centering
    \includegraphics[width=.8\textwidth]{./assets/img/replicator_dynamics/main.pdf}
    \caption{Numerical simulation of the replicator equation
    (\ref{eqn:replicator_dynamics}): strategies are ordered by score, only the strategies with a high score survive the evolutionary process.}
    \label{fig:replicator_dynamics}
\end{figure}

This work can be used to classify plays of the IPD\@: data can be collected from
actual interactions (in lab or in the field). Furthermore, this allows for a
classification method similar to the notion of fingerprinting presented
in~\cite{Ashlock2008}. Trained strategies can potentially be classified as
extortionate or not or it could be possible to even constrain the reinforcement
learning approaches that are becoming prevalent in the literature.
Alternatively, this mathematical approach for recognising extortion could be
used in sophisticated strategies to defend against invasion. Arguably, some of
the strategies considered here exhibit this behaviour, indeed as described
in~\cite{Harper2017}, the top ranking strategies in the full tournament are
obtained using evolutionary reinforcement learning techniques, thus, suspicion
of extortionate behaviour could in fact be an evolutionary trait.

\section*{Acknowledgements}

The following open source software libraries were used in this research:

\begin{itemize}
    \item The Axelrod ~\cite{Knight2016, Knight2018} library (IPD strategies and
        tournaments).
    \item The sympy library~\cite{Meurer2017} (verification of all symbolic
        calculations).
    \item The matplotlib~\cite{Droettboom2018} library (visualisation).
    \item The pandas~\cite{Structures2010}, dask~\cite{Dask2016} and
        NumPy~\cite{Oliphant2015} libraries (data manipulation).
    \item The SciPy~\cite{Jones2001} library (numerical integration of the
        replicator equation).
\end{itemize}

This work was performed using the computational facilities of the Advanced
Research Computing @ Cardiff (ARCCA) Division, Cardiff University.

\printbibliography

\newpage
\section*{Supplementary materials}

\includepdf{assets/pdf/proof_of_form_of_extortionate_strategies/main.pdf}

\newpage

Using the pair wise interactions the transition rates \(p,
q\) can be measured and the steady state probabilities inferred and compared to
the actual probabilities of each state.
This is done numerically by computing the singular eigenvector of the
matrix \(A\) \cite{Stewart2009}:

\[
    A =
    \begin{bmatrix}
        p_1 q_1 & p_1 (1 - q_1) & (1 - p_1) q_1 & (1 -p_1) (1 - q_1) \\
        p_2 q_2 & p_2 (1 - q_2) & (1 - p_2) q_2 & (1 -p_2) (1 - q_2) \\
        p_3 q_3 & p_3 (1 - q_3) & (1 - p_3) q_3 & (1 -p_3) (1 - q_3) \\
        p_4 q_4 & p_4 (1 - q_4) & (1 - p_4) q_4 & (1 -p_4) (1 - q_4) \\
    \end{bmatrix}
\]

Figure~\ref{fig:computed_probabilities_vs_theoretic_probabilities} shows a
regression line fitted to every pairwise interaction with a reported
\(\text{SSError}\) value (pairwise interactions with missing states were
omitted). This serves to validate the approach: a part from some edge cases the
relationship is consistent.

\begin{figure}[!htbp]
    \centering
    \includegraphics[width=.8\textwidth]{./assets/img/computed_probabilities_vs_theoretic_probabilities/main.pdf}
    \caption{The
        relationship between the steady state probabilities inferred from the
        measured transitions and the actual steady state probabilities. A linear
        regression line is included validating the approach.}
    \label{fig:computed_probabilities_vs_theoretic_probabilities}
\end{figure}


\end{document}
 turns and every match has been
repeated \documentclass[a4paper]{article}

\usepackage{amsmath}
\usepackage{amssymb}
\usepackage[margin=1.5cm,
            includefoot,
            footskip=30pt]{geometry}
\usepackage{layout}
\usepackage{graphicx}
\usepackage{subcaption}

\usepackage{biblatex}
\usepackage{pdfpages}

\bibliography{main.bib}

\title{Suspicion: Recognising and evaluating the effectiveness
       of extortion in the Iterated Prisoner's Dilemma}
\author{Vincent A. Knight \and Nikoleta E. Glynatsi}
\date{\today}



\begin{document}

\maketitle

\begin{abstract}
    The Iterated Prisoner's Dilemma is a model for rational and evolutionary
    interactive behaviour. It has applications both in the study of human social
    behaviour as well as in biology.
    It is used to understand when and how a rational individual might
    accept an immediate cost to their own utility for the direct benefit of
    another.

    Much attention has been given to a class of strategies called
    Zero Determinant strategies. It has been theoretically shown that these
    strategies can ``extort'' any player.

    In this work, an approach to identify if observed strategies are playing in
    an extortionate way is described. Furthermore, experimental analysis of
    a large tournament with \input{assets/tex/number_of_full_strategies/main.tex}
    strategies is considered. In this setting
    the most highly performing strategies do not play in an extortionate way
    against each other but do against lower performing strategies.
    This suggests that whilst the theory of Zero Determinant strategies
    indicates that memory is not of fundamental importance to the evolution of
    cooperative behaviour, this is incomplete.
\end{abstract}

\section{Introduction}\label{sec:introduction}

Agent based game theoretic models have become a stalwart of the underpinning
mathematics of interactive behaviours. One of the major pieces of work
in this area is the pair of original computer tournaments run by Robert
Axelrod~\cite{Axelrod1980, Axelrod1980a}. These tournaments pitted submitted
computer strategies against each other in plays of the Iterated Prisoner's
Dilemma. A common game where agents can choose to pay a slight cost to their
immediate utility in the hope of building a reputation. This has been used in
economic and evolutionary game theory to understand the evolution of cooperative
behaviour.

Recently, a class of strategies was described in~\cite{Press2012} that can
provably extort any given opponent. In~\cite{Hilbe2013, Moran1707} some
questions have already been asked about the true effectiveness of these
strategies in an evolutionary setting. Here another question is asked: is it
possible to recognise this extortionate behaviour? A mathematical procedure for
suspicion is presented: in the same way that the continued actions of an
extortionate individual might raise suspicion.

This work makes use of the Axelrod Python library~\cite{Knight2018, Knight2016}
with a large number of Prisoner Dilemma strategies available to give an
extensive numerical example of the ideas presented.  The approach is presented
in Section~\ref{sec:delta-zd-strategies}.  All of the code and data discussed
in Section~\ref{sec:numerical-experiments} is open sourced, archived and
written according to best scientific principles~\cite{Wilson2014}. The data
archive can be found at~\cite{vincent_knight_2018_1297075}.

\section{Recognising Extortion}\label{sec:delta-zd-strategies}

In~\cite{Press2012}, given a match between 2 memory-one strategies, the concept
of Zero Determinant (ZD) strategies is introduced. The main result of that paper
shows that given two memory one players \(p, q\in\mathbb{R}^4\) a linear
relationship between the players' scores could be forced by one of the players.

Using the notation of~\cite{Press2012}, assuming the utilities for player \(p\)
are given by \(S_x=(R, S, T, P)\) and for player \(q\) by \(S_y=(R, T, S, P)\)
and that the stationary scores of each player is given by \(S_X\) and \(S_Y\)
respectively. The main result of~\cite{Press2012} is that if

\begin{equation}\label{eqn:linear_relationship_for_p}
    \tilde p=\alpha S_x + \beta S_y + \gamma
\end{equation}

or

\begin{equation}\label{eqn:linear_relationship_for_q}
    \tilde q=\alpha S_x + \beta S_y + \gamma
\end{equation}

where \(\tilde p = (1 - p_1, 1 - p_2, p_3, p_4)\) and
\(\tilde q = (1 - q_1, 1 - q_2, q_3, q_4)\) then:

\begin{equation}
    \alpha S_X + \beta S_Y + \gamma = 0
\end{equation}

In~\cite{Press2012} a particular type of ZD strategy is defined: extortionate
strategies. If:

\begin{equation}\label{eqn:constraint_for_extortion}
    \gamma = - P(\alpha + \beta)
\end{equation}

then the player can ensure they get a score \(\chi\) times
larger than the opponent. This extortion coefficient is given by:

\begin{equation}\label{eqn:definition_of_chi}
    \chi=\frac{-\beta}{\alpha}
\end{equation}

Thus, if (\ref{eqn:constraint_for_extortion}) holds and \(\chi >1\) a player is
said to extort their opponent.
Here, the reverse problem is considered: given a
\(p\in\mathbb{R}^4\) how does one identify \(\alpha, \beta\) if they
exist and is the strategy in fact acting in an extortionate way?

These conditions correspond to:

\begin{align}
    \tilde p_1 & = \alpha R + \beta R - P (\alpha + \beta)
            \label{eqn:condition_for_tilde_p1}\\
    \tilde p_2 & = \alpha S + \beta T - P (\alpha + \beta)
            \label{eqn:condition_for_tilde_p2}\\
    \tilde p_3 & = \alpha T + \beta S - P (\alpha + \beta)
            \label{eqn:condition_for_tilde_p3}\\
    \tilde p_4 & = \alpha P + \beta P - P (\alpha + \beta)
            \label{eqn:condition_for_tilde_p4}
\end{align}

Equation (\ref{eqn:condition_for_tilde_p4}) ensures that \(p_4=\tilde p_4=0\).
Equations (\ref{eqn:condition_for_tilde_p1}-\ref{eqn:condition_for_tilde_p3})
can be used to eliminate \(\alpha, \beta\), giving:

\begin{equation}\label{eqn:planar_definition_of_extortion}
    \tilde p_1 = \frac{(R - P)(\tilde p_2 + \tilde p_3)}{S + T - 2P}
\end{equation}

with:

\begin{equation}\label{eqn:definition_of_chi}
    \chi = \frac{\tilde p_2 (P - T) + \tilde p_3 (S - P)}
                {\tilde p_2 (P - S) + \tilde p_3 (T - P)}
\end{equation}

Given a strategy \(p\in\mathbb{R}^{4\times 1}\) equations
(\ref{eqn:condition_for_tilde_p4}), (\ref{eqn:planar_definition_of_extortion}-\ref{eqn:definition_of_chi}) can be used to check if
a strategy is extortionate. The conditions correspond to:

\begin{align}
    p_1 & = \frac{(R-P)(p_2 + p_3) - R + T + S - P}{S + T - 2P}
     \label{eqn:condition_for_p1}\\
    p_4 & = 0 \label{eqn:condition_for_p4}\\
    1 & > p_2 + p_3\label{eqn:condition_for_chi}
\end{align}

The algebraic steps necessary to prove these results are available in the
supporting materials.

All extortionate strategies reside on a triangular (\ref{eqn:condition_for_chi})
plane (\ref{eqn:condition_for_p1}) in 3 dimensions (\ref{eqn:condition_for_p4}).
Using this formulation it can be seen that a necessary (but not sufficient)
condition for an extortionate strategy is that it cooperates on average less
than 50\% of the time when in a state of disagreement with the opponent.

As an example, consider the known extortionate strategy \(p=(8 / 9, 1 / 2, 1 /
3, 0)\) from~\cite{Stewart2012} which is referred to as \texttt{Extort-2}. In
this case, for the standard values of \((R, T, S, P)\) constraint
(\ref{eqn:condition_for_p1}) corresponds to:

\begin{equation}
    p_1 = \frac{2(p_2 + p_3) + 1}{3}
\end{equation}

It is clear that in this case all constraints hold.

This approach could in fact be used to confirm that a given strategy is acting
in an extortionate manner even if it is not a memory one strategy. However, in
practice, if a closed form for \(p\) is not known, then due to measurement
and/or numerical error this would not work.

This problem can be written in the following linear algebraic form where
\(x=(\alpha, \beta)\)
and \(p^*=(\tilde p_1 - 1, tilde_2 - 1, p_3)\):

\begin{equation}\label{eqn:linear_algebraic_equation_for_p}
    Cx= p^*
\end{equation}

\(C\) corresponds to equations
(\ref{eqn:condition_for_tilde_p1}-\ref{eqn:condition_for_tilde_p3}) and is
given by:

\begin{equation}\label{eqn:definition_of_C}
    C =
    \begin{bmatrix}
        R - P & R- P \\
        S - P & T- P \\
        T - P & S- P \\
    \end{bmatrix}
\end{equation}

Note that in general, equation (\ref{eqn:linear_algebraic_equation_for_p}) will
not necessarily have a solution. From the Rouch\'{e}-Capelli theorem if there is
a solution it is unique as \(\text{rank}(C)=2\) which is the dimension of the
variable \(x\). The best fitting \(x\) is found by minimizing:

\begin{equation}\label{eqn:r_squared}
    \text{SSError} = \|C x- p^*\|_2^2 = \sum_{i=1}^{3}\left((C\bar x)_i-p_i^*\right)^2
\end{equation}

Note that \(\text{SSError}\), which is the square of the Frobenius
norm~\cite{Golub2013}, becomes a measure of how close a strategy is to being an
extortionate strategy. Suspicion
of extortion then corresponds to a threshold on \(\text{SSError}\).

By observing interactions (human or otherwise), their memory one representation
can be inferred and this approach can be used to recognise extortionate
behaviour. The notion of comparing theoretic and actual plays of the IPD is not
novel, see for example~\cite{Rand2013}. Immediately it is noted that if the
environment is noisy~\cite{Wu1995} then no strategy can be considered to be
extortionate as \(p_4>0\).

In the next section, this idea will be illustrated by observing the interactions
that take place in a computer based tournament of the IPD\@.

\section{Numerical experiments}\label{sec:numerical-experiments}

In~\cite{Stewart2012} results from a tournament with
\input{./assets/tex/number_of_stewart_plotkin_strategies/main.tex} strategies,
was presented with specific consideration given to ZD strategies. This
tournament is reproduced here using the Axelrod-Python
project~\cite{Knight2016}. To obtain a good measure of the corresponding
transition rates for each strategy all matches have been run for
\input{assets/tex/number_of_turns/main.tex} turns and every match has been
repeated \input{assets/tex/number_of_repetitions/main.tex} times. All of this
interaction data is available at~\cite{vincent_knight_2018_1297075}. A good
match between the inferred Markov chain and the state distribution of the actual
interactions has been verified. Data for this is presented in the supplementary
materials.

Figure~\ref{fig:SSError_overall_in_stewart_plotkin} shows the \(\text{SSError}\)
values for all the strategies in the tournament, as reported
in~\cite{Stewart2012} the extortionate strategy (which has an expected
\(\text{SSError}\) approximately 0) gains a large number of wins.

\begin{figure}[!htbp]
    \centering
    \includegraphics[width=.8\textwidth]{./assets/img/SSError_overall_in_stewart_plotkin/main.pdf}
    \caption{\(\text{SSError}\) and state probabilities for the strategies
        of~\cite{Stewart2012}, ordered both by number of wins and overall score.
        Note that \(P(DC)\) is not shown as it corresponds to the transpose of
        \(P(CD)\). Cooperator and Defector are omitted as they do not visit all
        the states.}
    \label{fig:SSError_overall_in_stewart_plotkin}
\end{figure}

Here, the work of~\cite{Stewart2012} is extended by investigating a tournament
with \input{assets/tex/number_of_full_strategies/main.tex}
strategies.

The results of this analysis are shown in
Figure~\ref{fig:SSError_and_probabilities_in_full}. The top ranking strategies
by number of wins seem to be extortionate (but not against all strategies) and
it can be seen that a small sub group of strategies achieve mutual defection.
All the top ranking strategies according to score achieve mutual cooperation and
do not extort each other, however they
\textbf{do} exhibit extortionate behaviour towards a number of the lower ranking
strategies.

\begin{figure}[!htbp]
    \centering
    \includegraphics[width=.8\textwidth]{./assets/img/SSError_and_probabilities_in_full/main.pdf}
    \caption{\(\text{SSError}\) for the strategies for the full tournament. Only
    strategy interactions for which \(p_4=0\) and \(\chi>1\) are displayed.}
    \label{fig:SSError_and_probabilities_in_full}
\end{figure}

\section{Conclusion}\label{sec:conclusion}

This work defines an approach to measure whether or not a player is playing a
strategy that corresponds to an extortionate strategy as defined
in~\cite{Press2012}: a mathematical model for suspicion. Indeed, all
extortionate strategies have been
 classified as lying on a triangular plane.
This rigorous classification fails to be robust to small measurement error, thus
a statistical approach is proposed.
This is done through a linear algebraic approach for approximating the solution
of a linear system. Using this, a large number of pairwise interactions is
simulated and in fact very few strategies are found to act extortionately.

The work of~\cite{Press2012}, whilst showing that a clever approach to taking
advantage of another memory one strategy exists: this is incomplete. Whilst the
elegance of this result is very attractive, just as the simplicity of the
victory of Tit For Tat in Axelrod's original tournaments was, it is incomplete.
Extortionate strategies achieve a high number of wins but they do not
achieve a high score which corresponds to the fitness landscape in an
evolutionary sense. From the large number of interactions a payoff matrix \(S\)
can be measured where \(S_{ij}\) denotes the score (using standard values of
\((R, S, T, P) = (3, 0, 5, 1)\)) of the \(i\)th strategy
against the \(j\)th strategy. Using this, the replicator equation
describes the evolution of the system based on a population density fitness
function:

\begin{equation}\label{eqn:replicator_dynamics}
    \frac{dx}{dt} = x(S-x^TS x)
\end{equation}

Equation (\ref{eqn:replicator_dynamics}) is solved numerically through an
integration technique described in~\cite{Petzold1983} and
Figure~\ref{fig:replicator_dynamics} shows the evolution of the distribution of
the system: the various strategies are ranked by scores. It is clear to see that
only the high ranking strategies survive the evolutionary process (in fact,
only \input{./assets/img/replicator_dynamics/main.tex}
have a final distribution greater than \(10 ^ {-2}\)). This confirms the
findings of~\cite{Moran1707} in which sophisticated strategies resist
evolutionary invasion of shorter memory strategies. Recalling
Figure~\ref{fig:SSError_and_probabilities_in_full} this demonstrates that:

\begin{itemize}
    \item Cooperation emerges through the evolutionary process: the high scoring
        strategies do not exhibit extortionate behaviour towards each other.
    \item Extortionate strategies do not survive the evolutionary process.
\end{itemize}

\begin{figure}[!htbp]
    \centering
    \includegraphics[width=.8\textwidth]{./assets/img/replicator_dynamics/main.pdf}
    \caption{Numerical simulation of the replicator equation
    (\ref{eqn:replicator_dynamics}): strategies are ordered by score, only the strategies with a high score survive the evolutionary process.}
    \label{fig:replicator_dynamics}
\end{figure}

This work can be used to classify plays of the IPD\@: data can be collected from
actual interactions (in lab or in the field). Furthermore, this allows for a
classification method similar to the notion of fingerprinting presented
in~\cite{Ashlock2008}. Trained strategies can potentially be classified as
extortionate or not or it could be possible to even constrain the reinforcement
learning approaches that are becoming prevalent in the literature.
Alternatively, this mathematical approach for recognising extortion could be
used in sophisticated strategies to defend against invasion. Arguably, some of
the strategies considered here exhibit this behaviour, indeed as described
in~\cite{Harper2017}, the top ranking strategies in the full tournament are
obtained using evolutionary reinforcement learning techniques, thus, suspicion
of extortionate behaviour could in fact be an evolutionary trait.

\section*{Acknowledgements}

The following open source software libraries were used in this research:

\begin{itemize}
    \item The Axelrod ~\cite{Knight2016, Knight2018} library (IPD strategies and
        tournaments).
    \item The sympy library~\cite{Meurer2017} (verification of all symbolic
        calculations).
    \item The matplotlib~\cite{Droettboom2018} library (visualisation).
    \item The pandas~\cite{Structures2010}, dask~\cite{Dask2016} and
        NumPy~\cite{Oliphant2015} libraries (data manipulation).
    \item The SciPy~\cite{Jones2001} library (numerical integration of the
        replicator equation).
\end{itemize}

This work was performed using the computational facilities of the Advanced
Research Computing @ Cardiff (ARCCA) Division, Cardiff University.

\printbibliography

\newpage
\section*{Supplementary materials}

\includepdf{assets/pdf/proof_of_form_of_extortionate_strategies/main.pdf}

\newpage

Using the pair wise interactions the transition rates \(p,
q\) can be measured and the steady state probabilities inferred and compared to
the actual probabilities of each state.
This is done numerically by computing the singular eigenvector of the
matrix \(A\) \cite{Stewart2009}:

\[
    A =
    \begin{bmatrix}
        p_1 q_1 & p_1 (1 - q_1) & (1 - p_1) q_1 & (1 -p_1) (1 - q_1) \\
        p_2 q_2 & p_2 (1 - q_2) & (1 - p_2) q_2 & (1 -p_2) (1 - q_2) \\
        p_3 q_3 & p_3 (1 - q_3) & (1 - p_3) q_3 & (1 -p_3) (1 - q_3) \\
        p_4 q_4 & p_4 (1 - q_4) & (1 - p_4) q_4 & (1 -p_4) (1 - q_4) \\
    \end{bmatrix}
\]

Figure~\ref{fig:computed_probabilities_vs_theoretic_probabilities} shows a
regression line fitted to every pairwise interaction with a reported
\(\text{SSError}\) value (pairwise interactions with missing states were
omitted). This serves to validate the approach: a part from some edge cases the
relationship is consistent.

\begin{figure}[!htbp]
    \centering
    \includegraphics[width=.8\textwidth]{./assets/img/computed_probabilities_vs_theoretic_probabilities/main.pdf}
    \caption{The
        relationship between the steady state probabilities inferred from the
        measured transitions and the actual steady state probabilities. A linear
        regression line is included validating the approach.}
    \label{fig:computed_probabilities_vs_theoretic_probabilities}
\end{figure}


\end{document}
 times. All of this
interaction data is available at~\cite{vincent_knight_2018_1297075}. A good
match between the inferred Markov chain and the state distribution of the actual
interactions has been verified. Data for this is presented in the supplementary
materials.

Figure~\ref{fig:SSError_overall_in_stewart_plotkin} shows the \(\text{SSError}\)
values for all the strategies in the tournament, as reported
in~\cite{Stewart2012} the extortionate strategy (which has an expected
\(\text{SSError}\) approximately 0) gains a large number of wins.

\begin{figure}[!htbp]
    \centering
    \includegraphics[width=.8\textwidth]{./assets/img/SSError_overall_in_stewart_plotkin/main.pdf}
    \caption{\(\text{SSError}\) and state probabilities for the strategies
        of~\cite{Stewart2012}, ordered both by number of wins and overall score.
        Note that \(P(DC)\) is not shown as it corresponds to the transpose of
        \(P(CD)\). Cooperator and Defector are omitted as they do not visit all
        the states.}
    \label{fig:SSError_overall_in_stewart_plotkin}
\end{figure}

Here, the work of~\cite{Stewart2012} is extended by investigating a tournament
with \documentclass[a4paper]{article}

\usepackage{amsmath}
\usepackage{amssymb}
\usepackage[margin=1.5cm,
            includefoot,
            footskip=30pt]{geometry}
\usepackage{layout}
\usepackage{graphicx}
\usepackage{subcaption}

\usepackage{biblatex}
\usepackage{pdfpages}

\bibliography{main.bib}

\title{Suspicion: Recognising and evaluating the effectiveness
       of extortion in the Iterated Prisoner's Dilemma}
\author{Vincent A. Knight \and Nikoleta E. Glynatsi}
\date{\today}



\begin{document}

\maketitle

\begin{abstract}
    The Iterated Prisoner's Dilemma is a model for rational and evolutionary
    interactive behaviour. It has applications both in the study of human social
    behaviour as well as in biology.
    It is used to understand when and how a rational individual might
    accept an immediate cost to their own utility for the direct benefit of
    another.

    Much attention has been given to a class of strategies called
    Zero Determinant strategies. It has been theoretically shown that these
    strategies can ``extort'' any player.

    In this work, an approach to identify if observed strategies are playing in
    an extortionate way is described. Furthermore, experimental analysis of
    a large tournament with \input{assets/tex/number_of_full_strategies/main.tex}
    strategies is considered. In this setting
    the most highly performing strategies do not play in an extortionate way
    against each other but do against lower performing strategies.
    This suggests that whilst the theory of Zero Determinant strategies
    indicates that memory is not of fundamental importance to the evolution of
    cooperative behaviour, this is incomplete.
\end{abstract}

\section{Introduction}\label{sec:introduction}

Agent based game theoretic models have become a stalwart of the underpinning
mathematics of interactive behaviours. One of the major pieces of work
in this area is the pair of original computer tournaments run by Robert
Axelrod~\cite{Axelrod1980, Axelrod1980a}. These tournaments pitted submitted
computer strategies against each other in plays of the Iterated Prisoner's
Dilemma. A common game where agents can choose to pay a slight cost to their
immediate utility in the hope of building a reputation. This has been used in
economic and evolutionary game theory to understand the evolution of cooperative
behaviour.

Recently, a class of strategies was described in~\cite{Press2012} that can
provably extort any given opponent. In~\cite{Hilbe2013, Moran1707} some
questions have already been asked about the true effectiveness of these
strategies in an evolutionary setting. Here another question is asked: is it
possible to recognise this extortionate behaviour? A mathematical procedure for
suspicion is presented: in the same way that the continued actions of an
extortionate individual might raise suspicion.

This work makes use of the Axelrod Python library~\cite{Knight2018, Knight2016}
with a large number of Prisoner Dilemma strategies available to give an
extensive numerical example of the ideas presented.  The approach is presented
in Section~\ref{sec:delta-zd-strategies}.  All of the code and data discussed
in Section~\ref{sec:numerical-experiments} is open sourced, archived and
written according to best scientific principles~\cite{Wilson2014}. The data
archive can be found at~\cite{vincent_knight_2018_1297075}.

\section{Recognising Extortion}\label{sec:delta-zd-strategies}

In~\cite{Press2012}, given a match between 2 memory-one strategies, the concept
of Zero Determinant (ZD) strategies is introduced. The main result of that paper
shows that given two memory one players \(p, q\in\mathbb{R}^4\) a linear
relationship between the players' scores could be forced by one of the players.

Using the notation of~\cite{Press2012}, assuming the utilities for player \(p\)
are given by \(S_x=(R, S, T, P)\) and for player \(q\) by \(S_y=(R, T, S, P)\)
and that the stationary scores of each player is given by \(S_X\) and \(S_Y\)
respectively. The main result of~\cite{Press2012} is that if

\begin{equation}\label{eqn:linear_relationship_for_p}
    \tilde p=\alpha S_x + \beta S_y + \gamma
\end{equation}

or

\begin{equation}\label{eqn:linear_relationship_for_q}
    \tilde q=\alpha S_x + \beta S_y + \gamma
\end{equation}

where \(\tilde p = (1 - p_1, 1 - p_2, p_3, p_4)\) and
\(\tilde q = (1 - q_1, 1 - q_2, q_3, q_4)\) then:

\begin{equation}
    \alpha S_X + \beta S_Y + \gamma = 0
\end{equation}

In~\cite{Press2012} a particular type of ZD strategy is defined: extortionate
strategies. If:

\begin{equation}\label{eqn:constraint_for_extortion}
    \gamma = - P(\alpha + \beta)
\end{equation}

then the player can ensure they get a score \(\chi\) times
larger than the opponent. This extortion coefficient is given by:

\begin{equation}\label{eqn:definition_of_chi}
    \chi=\frac{-\beta}{\alpha}
\end{equation}

Thus, if (\ref{eqn:constraint_for_extortion}) holds and \(\chi >1\) a player is
said to extort their opponent.
Here, the reverse problem is considered: given a
\(p\in\mathbb{R}^4\) how does one identify \(\alpha, \beta\) if they
exist and is the strategy in fact acting in an extortionate way?

These conditions correspond to:

\begin{align}
    \tilde p_1 & = \alpha R + \beta R - P (\alpha + \beta)
            \label{eqn:condition_for_tilde_p1}\\
    \tilde p_2 & = \alpha S + \beta T - P (\alpha + \beta)
            \label{eqn:condition_for_tilde_p2}\\
    \tilde p_3 & = \alpha T + \beta S - P (\alpha + \beta)
            \label{eqn:condition_for_tilde_p3}\\
    \tilde p_4 & = \alpha P + \beta P - P (\alpha + \beta)
            \label{eqn:condition_for_tilde_p4}
\end{align}

Equation (\ref{eqn:condition_for_tilde_p4}) ensures that \(p_4=\tilde p_4=0\).
Equations (\ref{eqn:condition_for_tilde_p1}-\ref{eqn:condition_for_tilde_p3})
can be used to eliminate \(\alpha, \beta\), giving:

\begin{equation}\label{eqn:planar_definition_of_extortion}
    \tilde p_1 = \frac{(R - P)(\tilde p_2 + \tilde p_3)}{S + T - 2P}
\end{equation}

with:

\begin{equation}\label{eqn:definition_of_chi}
    \chi = \frac{\tilde p_2 (P - T) + \tilde p_3 (S - P)}
                {\tilde p_2 (P - S) + \tilde p_3 (T - P)}
\end{equation}

Given a strategy \(p\in\mathbb{R}^{4\times 1}\) equations
(\ref{eqn:condition_for_tilde_p4}), (\ref{eqn:planar_definition_of_extortion}-\ref{eqn:definition_of_chi}) can be used to check if
a strategy is extortionate. The conditions correspond to:

\begin{align}
    p_1 & = \frac{(R-P)(p_2 + p_3) - R + T + S - P}{S + T - 2P}
     \label{eqn:condition_for_p1}\\
    p_4 & = 0 \label{eqn:condition_for_p4}\\
    1 & > p_2 + p_3\label{eqn:condition_for_chi}
\end{align}

The algebraic steps necessary to prove these results are available in the
supporting materials.

All extortionate strategies reside on a triangular (\ref{eqn:condition_for_chi})
plane (\ref{eqn:condition_for_p1}) in 3 dimensions (\ref{eqn:condition_for_p4}).
Using this formulation it can be seen that a necessary (but not sufficient)
condition for an extortionate strategy is that it cooperates on average less
than 50\% of the time when in a state of disagreement with the opponent.

As an example, consider the known extortionate strategy \(p=(8 / 9, 1 / 2, 1 /
3, 0)\) from~\cite{Stewart2012} which is referred to as \texttt{Extort-2}. In
this case, for the standard values of \((R, T, S, P)\) constraint
(\ref{eqn:condition_for_p1}) corresponds to:

\begin{equation}
    p_1 = \frac{2(p_2 + p_3) + 1}{3}
\end{equation}

It is clear that in this case all constraints hold.

This approach could in fact be used to confirm that a given strategy is acting
in an extortionate manner even if it is not a memory one strategy. However, in
practice, if a closed form for \(p\) is not known, then due to measurement
and/or numerical error this would not work.

This problem can be written in the following linear algebraic form where
\(x=(\alpha, \beta)\)
and \(p^*=(\tilde p_1 - 1, tilde_2 - 1, p_3)\):

\begin{equation}\label{eqn:linear_algebraic_equation_for_p}
    Cx= p^*
\end{equation}

\(C\) corresponds to equations
(\ref{eqn:condition_for_tilde_p1}-\ref{eqn:condition_for_tilde_p3}) and is
given by:

\begin{equation}\label{eqn:definition_of_C}
    C =
    \begin{bmatrix}
        R - P & R- P \\
        S - P & T- P \\
        T - P & S- P \\
    \end{bmatrix}
\end{equation}

Note that in general, equation (\ref{eqn:linear_algebraic_equation_for_p}) will
not necessarily have a solution. From the Rouch\'{e}-Capelli theorem if there is
a solution it is unique as \(\text{rank}(C)=2\) which is the dimension of the
variable \(x\). The best fitting \(x\) is found by minimizing:

\begin{equation}\label{eqn:r_squared}
    \text{SSError} = \|C x- p^*\|_2^2 = \sum_{i=1}^{3}\left((C\bar x)_i-p_i^*\right)^2
\end{equation}

Note that \(\text{SSError}\), which is the square of the Frobenius
norm~\cite{Golub2013}, becomes a measure of how close a strategy is to being an
extortionate strategy. Suspicion
of extortion then corresponds to a threshold on \(\text{SSError}\).

By observing interactions (human or otherwise), their memory one representation
can be inferred and this approach can be used to recognise extortionate
behaviour. The notion of comparing theoretic and actual plays of the IPD is not
novel, see for example~\cite{Rand2013}. Immediately it is noted that if the
environment is noisy~\cite{Wu1995} then no strategy can be considered to be
extortionate as \(p_4>0\).

In the next section, this idea will be illustrated by observing the interactions
that take place in a computer based tournament of the IPD\@.

\section{Numerical experiments}\label{sec:numerical-experiments}

In~\cite{Stewart2012} results from a tournament with
\input{./assets/tex/number_of_stewart_plotkin_strategies/main.tex} strategies,
was presented with specific consideration given to ZD strategies. This
tournament is reproduced here using the Axelrod-Python
project~\cite{Knight2016}. To obtain a good measure of the corresponding
transition rates for each strategy all matches have been run for
\input{assets/tex/number_of_turns/main.tex} turns and every match has been
repeated \input{assets/tex/number_of_repetitions/main.tex} times. All of this
interaction data is available at~\cite{vincent_knight_2018_1297075}. A good
match between the inferred Markov chain and the state distribution of the actual
interactions has been verified. Data for this is presented in the supplementary
materials.

Figure~\ref{fig:SSError_overall_in_stewart_plotkin} shows the \(\text{SSError}\)
values for all the strategies in the tournament, as reported
in~\cite{Stewart2012} the extortionate strategy (which has an expected
\(\text{SSError}\) approximately 0) gains a large number of wins.

\begin{figure}[!htbp]
    \centering
    \includegraphics[width=.8\textwidth]{./assets/img/SSError_overall_in_stewart_plotkin/main.pdf}
    \caption{\(\text{SSError}\) and state probabilities for the strategies
        of~\cite{Stewart2012}, ordered both by number of wins and overall score.
        Note that \(P(DC)\) is not shown as it corresponds to the transpose of
        \(P(CD)\). Cooperator and Defector are omitted as they do not visit all
        the states.}
    \label{fig:SSError_overall_in_stewart_plotkin}
\end{figure}

Here, the work of~\cite{Stewart2012} is extended by investigating a tournament
with \input{assets/tex/number_of_full_strategies/main.tex}
strategies.

The results of this analysis are shown in
Figure~\ref{fig:SSError_and_probabilities_in_full}. The top ranking strategies
by number of wins seem to be extortionate (but not against all strategies) and
it can be seen that a small sub group of strategies achieve mutual defection.
All the top ranking strategies according to score achieve mutual cooperation and
do not extort each other, however they
\textbf{do} exhibit extortionate behaviour towards a number of the lower ranking
strategies.

\begin{figure}[!htbp]
    \centering
    \includegraphics[width=.8\textwidth]{./assets/img/SSError_and_probabilities_in_full/main.pdf}
    \caption{\(\text{SSError}\) for the strategies for the full tournament. Only
    strategy interactions for which \(p_4=0\) and \(\chi>1\) are displayed.}
    \label{fig:SSError_and_probabilities_in_full}
\end{figure}

\section{Conclusion}\label{sec:conclusion}

This work defines an approach to measure whether or not a player is playing a
strategy that corresponds to an extortionate strategy as defined
in~\cite{Press2012}: a mathematical model for suspicion. Indeed, all
extortionate strategies have been
 classified as lying on a triangular plane.
This rigorous classification fails to be robust to small measurement error, thus
a statistical approach is proposed.
This is done through a linear algebraic approach for approximating the solution
of a linear system. Using this, a large number of pairwise interactions is
simulated and in fact very few strategies are found to act extortionately.

The work of~\cite{Press2012}, whilst showing that a clever approach to taking
advantage of another memory one strategy exists: this is incomplete. Whilst the
elegance of this result is very attractive, just as the simplicity of the
victory of Tit For Tat in Axelrod's original tournaments was, it is incomplete.
Extortionate strategies achieve a high number of wins but they do not
achieve a high score which corresponds to the fitness landscape in an
evolutionary sense. From the large number of interactions a payoff matrix \(S\)
can be measured where \(S_{ij}\) denotes the score (using standard values of
\((R, S, T, P) = (3, 0, 5, 1)\)) of the \(i\)th strategy
against the \(j\)th strategy. Using this, the replicator equation
describes the evolution of the system based on a population density fitness
function:

\begin{equation}\label{eqn:replicator_dynamics}
    \frac{dx}{dt} = x(S-x^TS x)
\end{equation}

Equation (\ref{eqn:replicator_dynamics}) is solved numerically through an
integration technique described in~\cite{Petzold1983} and
Figure~\ref{fig:replicator_dynamics} shows the evolution of the distribution of
the system: the various strategies are ranked by scores. It is clear to see that
only the high ranking strategies survive the evolutionary process (in fact,
only \input{./assets/img/replicator_dynamics/main.tex}
have a final distribution greater than \(10 ^ {-2}\)). This confirms the
findings of~\cite{Moran1707} in which sophisticated strategies resist
evolutionary invasion of shorter memory strategies. Recalling
Figure~\ref{fig:SSError_and_probabilities_in_full} this demonstrates that:

\begin{itemize}
    \item Cooperation emerges through the evolutionary process: the high scoring
        strategies do not exhibit extortionate behaviour towards each other.
    \item Extortionate strategies do not survive the evolutionary process.
\end{itemize}

\begin{figure}[!htbp]
    \centering
    \includegraphics[width=.8\textwidth]{./assets/img/replicator_dynamics/main.pdf}
    \caption{Numerical simulation of the replicator equation
    (\ref{eqn:replicator_dynamics}): strategies are ordered by score, only the strategies with a high score survive the evolutionary process.}
    \label{fig:replicator_dynamics}
\end{figure}

This work can be used to classify plays of the IPD\@: data can be collected from
actual interactions (in lab or in the field). Furthermore, this allows for a
classification method similar to the notion of fingerprinting presented
in~\cite{Ashlock2008}. Trained strategies can potentially be classified as
extortionate or not or it could be possible to even constrain the reinforcement
learning approaches that are becoming prevalent in the literature.
Alternatively, this mathematical approach for recognising extortion could be
used in sophisticated strategies to defend against invasion. Arguably, some of
the strategies considered here exhibit this behaviour, indeed as described
in~\cite{Harper2017}, the top ranking strategies in the full tournament are
obtained using evolutionary reinforcement learning techniques, thus, suspicion
of extortionate behaviour could in fact be an evolutionary trait.

\section*{Acknowledgements}

The following open source software libraries were used in this research:

\begin{itemize}
    \item The Axelrod ~\cite{Knight2016, Knight2018} library (IPD strategies and
        tournaments).
    \item The sympy library~\cite{Meurer2017} (verification of all symbolic
        calculations).
    \item The matplotlib~\cite{Droettboom2018} library (visualisation).
    \item The pandas~\cite{Structures2010}, dask~\cite{Dask2016} and
        NumPy~\cite{Oliphant2015} libraries (data manipulation).
    \item The SciPy~\cite{Jones2001} library (numerical integration of the
        replicator equation).
\end{itemize}

This work was performed using the computational facilities of the Advanced
Research Computing @ Cardiff (ARCCA) Division, Cardiff University.

\printbibliography

\newpage
\section*{Supplementary materials}

\includepdf{assets/pdf/proof_of_form_of_extortionate_strategies/main.pdf}

\newpage

Using the pair wise interactions the transition rates \(p,
q\) can be measured and the steady state probabilities inferred and compared to
the actual probabilities of each state.
This is done numerically by computing the singular eigenvector of the
matrix \(A\) \cite{Stewart2009}:

\[
    A =
    \begin{bmatrix}
        p_1 q_1 & p_1 (1 - q_1) & (1 - p_1) q_1 & (1 -p_1) (1 - q_1) \\
        p_2 q_2 & p_2 (1 - q_2) & (1 - p_2) q_2 & (1 -p_2) (1 - q_2) \\
        p_3 q_3 & p_3 (1 - q_3) & (1 - p_3) q_3 & (1 -p_3) (1 - q_3) \\
        p_4 q_4 & p_4 (1 - q_4) & (1 - p_4) q_4 & (1 -p_4) (1 - q_4) \\
    \end{bmatrix}
\]

Figure~\ref{fig:computed_probabilities_vs_theoretic_probabilities} shows a
regression line fitted to every pairwise interaction with a reported
\(\text{SSError}\) value (pairwise interactions with missing states were
omitted). This serves to validate the approach: a part from some edge cases the
relationship is consistent.

\begin{figure}[!htbp]
    \centering
    \includegraphics[width=.8\textwidth]{./assets/img/computed_probabilities_vs_theoretic_probabilities/main.pdf}
    \caption{The
        relationship between the steady state probabilities inferred from the
        measured transitions and the actual steady state probabilities. A linear
        regression line is included validating the approach.}
    \label{fig:computed_probabilities_vs_theoretic_probabilities}
\end{figure}


\end{document}

strategies.

The results of this analysis are shown in
Figure~\ref{fig:SSError_and_probabilities_in_full}. The top ranking strategies
by number of wins seem to be extortionate (but not against all strategies) and
it can be seen that a small sub group of strategies achieve mutual defection.
All the top ranking strategies according to score achieve mutual cooperation and
do not extort each other, however they
\textbf{do} exhibit extortionate behaviour towards a number of the lower ranking
strategies.

\begin{figure}[!htbp]
    \centering
    \includegraphics[width=.8\textwidth]{./assets/img/SSError_and_probabilities_in_full/main.pdf}
    \caption{\(\text{SSError}\) for the strategies for the full tournament. Only
    strategy interactions for which \(p_4=0\) and \(\chi>1\) are displayed.}
    \label{fig:SSError_and_probabilities_in_full}
\end{figure}

\section{Conclusion}\label{sec:conclusion}

This work defines an approach to measure whether or not a player is playing a
strategy that corresponds to an extortionate strategy as defined
in~\cite{Press2012}: a mathematical model for suspicion. Indeed, all
extortionate strategies have been
 classified as lying on a triangular plane.
This rigorous classification fails to be robust to small measurement error, thus
a statistical approach is proposed.
This is done through a linear algebraic approach for approximating the solution
of a linear system. Using this, a large number of pairwise interactions is
simulated and in fact very few strategies are found to act extortionately.

The work of~\cite{Press2012}, whilst showing that a clever approach to taking
advantage of another memory one strategy exists: this is incomplete. Whilst the
elegance of this result is very attractive, just as the simplicity of the
victory of Tit For Tat in Axelrod's original tournaments was, it is incomplete.
Extortionate strategies achieve a high number of wins but they do not
achieve a high score which corresponds to the fitness landscape in an
evolutionary sense. From the large number of interactions a payoff matrix \(S\)
can be measured where \(S_{ij}\) denotes the score (using standard values of
\((R, S, T, P) = (3, 0, 5, 1)\)) of the \(i\)th strategy
against the \(j\)th strategy. Using this, the replicator equation
describes the evolution of the system based on a population density fitness
function:

\begin{equation}\label{eqn:replicator_dynamics}
    \frac{dx}{dt} = x(S-x^TS x)
\end{equation}

Equation (\ref{eqn:replicator_dynamics}) is solved numerically through an
integration technique described in~\cite{Petzold1983} and
Figure~\ref{fig:replicator_dynamics} shows the evolution of the distribution of
the system: the various strategies are ranked by scores. It is clear to see that
only the high ranking strategies survive the evolutionary process (in fact,
only \documentclass[a4paper]{article}

\usepackage{amsmath}
\usepackage{amssymb}
\usepackage[margin=1.5cm,
            includefoot,
            footskip=30pt]{geometry}
\usepackage{layout}
\usepackage{graphicx}
\usepackage{subcaption}

\usepackage{biblatex}
\usepackage{pdfpages}

\bibliography{main.bib}

\title{Suspicion: Recognising and evaluating the effectiveness
       of extortion in the Iterated Prisoner's Dilemma}
\author{Vincent A. Knight \and Nikoleta E. Glynatsi}
\date{\today}



\begin{document}

\maketitle

\begin{abstract}
    The Iterated Prisoner's Dilemma is a model for rational and evolutionary
    interactive behaviour. It has applications both in the study of human social
    behaviour as well as in biology.
    It is used to understand when and how a rational individual might
    accept an immediate cost to their own utility for the direct benefit of
    another.

    Much attention has been given to a class of strategies called
    Zero Determinant strategies. It has been theoretically shown that these
    strategies can ``extort'' any player.

    In this work, an approach to identify if observed strategies are playing in
    an extortionate way is described. Furthermore, experimental analysis of
    a large tournament with \input{assets/tex/number_of_full_strategies/main.tex}
    strategies is considered. In this setting
    the most highly performing strategies do not play in an extortionate way
    against each other but do against lower performing strategies.
    This suggests that whilst the theory of Zero Determinant strategies
    indicates that memory is not of fundamental importance to the evolution of
    cooperative behaviour, this is incomplete.
\end{abstract}

\section{Introduction}\label{sec:introduction}

Agent based game theoretic models have become a stalwart of the underpinning
mathematics of interactive behaviours. One of the major pieces of work
in this area is the pair of original computer tournaments run by Robert
Axelrod~\cite{Axelrod1980, Axelrod1980a}. These tournaments pitted submitted
computer strategies against each other in plays of the Iterated Prisoner's
Dilemma. A common game where agents can choose to pay a slight cost to their
immediate utility in the hope of building a reputation. This has been used in
economic and evolutionary game theory to understand the evolution of cooperative
behaviour.

Recently, a class of strategies was described in~\cite{Press2012} that can
provably extort any given opponent. In~\cite{Hilbe2013, Moran1707} some
questions have already been asked about the true effectiveness of these
strategies in an evolutionary setting. Here another question is asked: is it
possible to recognise this extortionate behaviour? A mathematical procedure for
suspicion is presented: in the same way that the continued actions of an
extortionate individual might raise suspicion.

This work makes use of the Axelrod Python library~\cite{Knight2018, Knight2016}
with a large number of Prisoner Dilemma strategies available to give an
extensive numerical example of the ideas presented.  The approach is presented
in Section~\ref{sec:delta-zd-strategies}.  All of the code and data discussed
in Section~\ref{sec:numerical-experiments} is open sourced, archived and
written according to best scientific principles~\cite{Wilson2014}. The data
archive can be found at~\cite{vincent_knight_2018_1297075}.

\section{Recognising Extortion}\label{sec:delta-zd-strategies}

In~\cite{Press2012}, given a match between 2 memory-one strategies, the concept
of Zero Determinant (ZD) strategies is introduced. The main result of that paper
shows that given two memory one players \(p, q\in\mathbb{R}^4\) a linear
relationship between the players' scores could be forced by one of the players.

Using the notation of~\cite{Press2012}, assuming the utilities for player \(p\)
are given by \(S_x=(R, S, T, P)\) and for player \(q\) by \(S_y=(R, T, S, P)\)
and that the stationary scores of each player is given by \(S_X\) and \(S_Y\)
respectively. The main result of~\cite{Press2012} is that if

\begin{equation}\label{eqn:linear_relationship_for_p}
    \tilde p=\alpha S_x + \beta S_y + \gamma
\end{equation}

or

\begin{equation}\label{eqn:linear_relationship_for_q}
    \tilde q=\alpha S_x + \beta S_y + \gamma
\end{equation}

where \(\tilde p = (1 - p_1, 1 - p_2, p_3, p_4)\) and
\(\tilde q = (1 - q_1, 1 - q_2, q_3, q_4)\) then:

\begin{equation}
    \alpha S_X + \beta S_Y + \gamma = 0
\end{equation}

In~\cite{Press2012} a particular type of ZD strategy is defined: extortionate
strategies. If:

\begin{equation}\label{eqn:constraint_for_extortion}
    \gamma = - P(\alpha + \beta)
\end{equation}

then the player can ensure they get a score \(\chi\) times
larger than the opponent. This extortion coefficient is given by:

\begin{equation}\label{eqn:definition_of_chi}
    \chi=\frac{-\beta}{\alpha}
\end{equation}

Thus, if (\ref{eqn:constraint_for_extortion}) holds and \(\chi >1\) a player is
said to extort their opponent.
Here, the reverse problem is considered: given a
\(p\in\mathbb{R}^4\) how does one identify \(\alpha, \beta\) if they
exist and is the strategy in fact acting in an extortionate way?

These conditions correspond to:

\begin{align}
    \tilde p_1 & = \alpha R + \beta R - P (\alpha + \beta)
            \label{eqn:condition_for_tilde_p1}\\
    \tilde p_2 & = \alpha S + \beta T - P (\alpha + \beta)
            \label{eqn:condition_for_tilde_p2}\\
    \tilde p_3 & = \alpha T + \beta S - P (\alpha + \beta)
            \label{eqn:condition_for_tilde_p3}\\
    \tilde p_4 & = \alpha P + \beta P - P (\alpha + \beta)
            \label{eqn:condition_for_tilde_p4}
\end{align}

Equation (\ref{eqn:condition_for_tilde_p4}) ensures that \(p_4=\tilde p_4=0\).
Equations (\ref{eqn:condition_for_tilde_p1}-\ref{eqn:condition_for_tilde_p3})
can be used to eliminate \(\alpha, \beta\), giving:

\begin{equation}\label{eqn:planar_definition_of_extortion}
    \tilde p_1 = \frac{(R - P)(\tilde p_2 + \tilde p_3)}{S + T - 2P}
\end{equation}

with:

\begin{equation}\label{eqn:definition_of_chi}
    \chi = \frac{\tilde p_2 (P - T) + \tilde p_3 (S - P)}
                {\tilde p_2 (P - S) + \tilde p_3 (T - P)}
\end{equation}

Given a strategy \(p\in\mathbb{R}^{4\times 1}\) equations
(\ref{eqn:condition_for_tilde_p4}), (\ref{eqn:planar_definition_of_extortion}-\ref{eqn:definition_of_chi}) can be used to check if
a strategy is extortionate. The conditions correspond to:

\begin{align}
    p_1 & = \frac{(R-P)(p_2 + p_3) - R + T + S - P}{S + T - 2P}
     \label{eqn:condition_for_p1}\\
    p_4 & = 0 \label{eqn:condition_for_p4}\\
    1 & > p_2 + p_3\label{eqn:condition_for_chi}
\end{align}

The algebraic steps necessary to prove these results are available in the
supporting materials.

All extortionate strategies reside on a triangular (\ref{eqn:condition_for_chi})
plane (\ref{eqn:condition_for_p1}) in 3 dimensions (\ref{eqn:condition_for_p4}).
Using this formulation it can be seen that a necessary (but not sufficient)
condition for an extortionate strategy is that it cooperates on average less
than 50\% of the time when in a state of disagreement with the opponent.

As an example, consider the known extortionate strategy \(p=(8 / 9, 1 / 2, 1 /
3, 0)\) from~\cite{Stewart2012} which is referred to as \texttt{Extort-2}. In
this case, for the standard values of \((R, T, S, P)\) constraint
(\ref{eqn:condition_for_p1}) corresponds to:

\begin{equation}
    p_1 = \frac{2(p_2 + p_3) + 1}{3}
\end{equation}

It is clear that in this case all constraints hold.

This approach could in fact be used to confirm that a given strategy is acting
in an extortionate manner even if it is not a memory one strategy. However, in
practice, if a closed form for \(p\) is not known, then due to measurement
and/or numerical error this would not work.

This problem can be written in the following linear algebraic form where
\(x=(\alpha, \beta)\)
and \(p^*=(\tilde p_1 - 1, tilde_2 - 1, p_3)\):

\begin{equation}\label{eqn:linear_algebraic_equation_for_p}
    Cx= p^*
\end{equation}

\(C\) corresponds to equations
(\ref{eqn:condition_for_tilde_p1}-\ref{eqn:condition_for_tilde_p3}) and is
given by:

\begin{equation}\label{eqn:definition_of_C}
    C =
    \begin{bmatrix}
        R - P & R- P \\
        S - P & T- P \\
        T - P & S- P \\
    \end{bmatrix}
\end{equation}

Note that in general, equation (\ref{eqn:linear_algebraic_equation_for_p}) will
not necessarily have a solution. From the Rouch\'{e}-Capelli theorem if there is
a solution it is unique as \(\text{rank}(C)=2\) which is the dimension of the
variable \(x\). The best fitting \(x\) is found by minimizing:

\begin{equation}\label{eqn:r_squared}
    \text{SSError} = \|C x- p^*\|_2^2 = \sum_{i=1}^{3}\left((C\bar x)_i-p_i^*\right)^2
\end{equation}

Note that \(\text{SSError}\), which is the square of the Frobenius
norm~\cite{Golub2013}, becomes a measure of how close a strategy is to being an
extortionate strategy. Suspicion
of extortion then corresponds to a threshold on \(\text{SSError}\).

By observing interactions (human or otherwise), their memory one representation
can be inferred and this approach can be used to recognise extortionate
behaviour. The notion of comparing theoretic and actual plays of the IPD is not
novel, see for example~\cite{Rand2013}. Immediately it is noted that if the
environment is noisy~\cite{Wu1995} then no strategy can be considered to be
extortionate as \(p_4>0\).

In the next section, this idea will be illustrated by observing the interactions
that take place in a computer based tournament of the IPD\@.

\section{Numerical experiments}\label{sec:numerical-experiments}

In~\cite{Stewart2012} results from a tournament with
\input{./assets/tex/number_of_stewart_plotkin_strategies/main.tex} strategies,
was presented with specific consideration given to ZD strategies. This
tournament is reproduced here using the Axelrod-Python
project~\cite{Knight2016}. To obtain a good measure of the corresponding
transition rates for each strategy all matches have been run for
\input{assets/tex/number_of_turns/main.tex} turns and every match has been
repeated \input{assets/tex/number_of_repetitions/main.tex} times. All of this
interaction data is available at~\cite{vincent_knight_2018_1297075}. A good
match between the inferred Markov chain and the state distribution of the actual
interactions has been verified. Data for this is presented in the supplementary
materials.

Figure~\ref{fig:SSError_overall_in_stewart_plotkin} shows the \(\text{SSError}\)
values for all the strategies in the tournament, as reported
in~\cite{Stewart2012} the extortionate strategy (which has an expected
\(\text{SSError}\) approximately 0) gains a large number of wins.

\begin{figure}[!htbp]
    \centering
    \includegraphics[width=.8\textwidth]{./assets/img/SSError_overall_in_stewart_plotkin/main.pdf}
    \caption{\(\text{SSError}\) and state probabilities for the strategies
        of~\cite{Stewart2012}, ordered both by number of wins and overall score.
        Note that \(P(DC)\) is not shown as it corresponds to the transpose of
        \(P(CD)\). Cooperator and Defector are omitted as they do not visit all
        the states.}
    \label{fig:SSError_overall_in_stewart_plotkin}
\end{figure}

Here, the work of~\cite{Stewart2012} is extended by investigating a tournament
with \input{assets/tex/number_of_full_strategies/main.tex}
strategies.

The results of this analysis are shown in
Figure~\ref{fig:SSError_and_probabilities_in_full}. The top ranking strategies
by number of wins seem to be extortionate (but not against all strategies) and
it can be seen that a small sub group of strategies achieve mutual defection.
All the top ranking strategies according to score achieve mutual cooperation and
do not extort each other, however they
\textbf{do} exhibit extortionate behaviour towards a number of the lower ranking
strategies.

\begin{figure}[!htbp]
    \centering
    \includegraphics[width=.8\textwidth]{./assets/img/SSError_and_probabilities_in_full/main.pdf}
    \caption{\(\text{SSError}\) for the strategies for the full tournament. Only
    strategy interactions for which \(p_4=0\) and \(\chi>1\) are displayed.}
    \label{fig:SSError_and_probabilities_in_full}
\end{figure}

\section{Conclusion}\label{sec:conclusion}

This work defines an approach to measure whether or not a player is playing a
strategy that corresponds to an extortionate strategy as defined
in~\cite{Press2012}: a mathematical model for suspicion. Indeed, all
extortionate strategies have been
 classified as lying on a triangular plane.
This rigorous classification fails to be robust to small measurement error, thus
a statistical approach is proposed.
This is done through a linear algebraic approach for approximating the solution
of a linear system. Using this, a large number of pairwise interactions is
simulated and in fact very few strategies are found to act extortionately.

The work of~\cite{Press2012}, whilst showing that a clever approach to taking
advantage of another memory one strategy exists: this is incomplete. Whilst the
elegance of this result is very attractive, just as the simplicity of the
victory of Tit For Tat in Axelrod's original tournaments was, it is incomplete.
Extortionate strategies achieve a high number of wins but they do not
achieve a high score which corresponds to the fitness landscape in an
evolutionary sense. From the large number of interactions a payoff matrix \(S\)
can be measured where \(S_{ij}\) denotes the score (using standard values of
\((R, S, T, P) = (3, 0, 5, 1)\)) of the \(i\)th strategy
against the \(j\)th strategy. Using this, the replicator equation
describes the evolution of the system based on a population density fitness
function:

\begin{equation}\label{eqn:replicator_dynamics}
    \frac{dx}{dt} = x(S-x^TS x)
\end{equation}

Equation (\ref{eqn:replicator_dynamics}) is solved numerically through an
integration technique described in~\cite{Petzold1983} and
Figure~\ref{fig:replicator_dynamics} shows the evolution of the distribution of
the system: the various strategies are ranked by scores. It is clear to see that
only the high ranking strategies survive the evolutionary process (in fact,
only \input{./assets/img/replicator_dynamics/main.tex}
have a final distribution greater than \(10 ^ {-2}\)). This confirms the
findings of~\cite{Moran1707} in which sophisticated strategies resist
evolutionary invasion of shorter memory strategies. Recalling
Figure~\ref{fig:SSError_and_probabilities_in_full} this demonstrates that:

\begin{itemize}
    \item Cooperation emerges through the evolutionary process: the high scoring
        strategies do not exhibit extortionate behaviour towards each other.
    \item Extortionate strategies do not survive the evolutionary process.
\end{itemize}

\begin{figure}[!htbp]
    \centering
    \includegraphics[width=.8\textwidth]{./assets/img/replicator_dynamics/main.pdf}
    \caption{Numerical simulation of the replicator equation
    (\ref{eqn:replicator_dynamics}): strategies are ordered by score, only the strategies with a high score survive the evolutionary process.}
    \label{fig:replicator_dynamics}
\end{figure}

This work can be used to classify plays of the IPD\@: data can be collected from
actual interactions (in lab or in the field). Furthermore, this allows for a
classification method similar to the notion of fingerprinting presented
in~\cite{Ashlock2008}. Trained strategies can potentially be classified as
extortionate or not or it could be possible to even constrain the reinforcement
learning approaches that are becoming prevalent in the literature.
Alternatively, this mathematical approach for recognising extortion could be
used in sophisticated strategies to defend against invasion. Arguably, some of
the strategies considered here exhibit this behaviour, indeed as described
in~\cite{Harper2017}, the top ranking strategies in the full tournament are
obtained using evolutionary reinforcement learning techniques, thus, suspicion
of extortionate behaviour could in fact be an evolutionary trait.

\section*{Acknowledgements}

The following open source software libraries were used in this research:

\begin{itemize}
    \item The Axelrod ~\cite{Knight2016, Knight2018} library (IPD strategies and
        tournaments).
    \item The sympy library~\cite{Meurer2017} (verification of all symbolic
        calculations).
    \item The matplotlib~\cite{Droettboom2018} library (visualisation).
    \item The pandas~\cite{Structures2010}, dask~\cite{Dask2016} and
        NumPy~\cite{Oliphant2015} libraries (data manipulation).
    \item The SciPy~\cite{Jones2001} library (numerical integration of the
        replicator equation).
\end{itemize}

This work was performed using the computational facilities of the Advanced
Research Computing @ Cardiff (ARCCA) Division, Cardiff University.

\printbibliography

\newpage
\section*{Supplementary materials}

\includepdf{assets/pdf/proof_of_form_of_extortionate_strategies/main.pdf}

\newpage

Using the pair wise interactions the transition rates \(p,
q\) can be measured and the steady state probabilities inferred and compared to
the actual probabilities of each state.
This is done numerically by computing the singular eigenvector of the
matrix \(A\) \cite{Stewart2009}:

\[
    A =
    \begin{bmatrix}
        p_1 q_1 & p_1 (1 - q_1) & (1 - p_1) q_1 & (1 -p_1) (1 - q_1) \\
        p_2 q_2 & p_2 (1 - q_2) & (1 - p_2) q_2 & (1 -p_2) (1 - q_2) \\
        p_3 q_3 & p_3 (1 - q_3) & (1 - p_3) q_3 & (1 -p_3) (1 - q_3) \\
        p_4 q_4 & p_4 (1 - q_4) & (1 - p_4) q_4 & (1 -p_4) (1 - q_4) \\
    \end{bmatrix}
\]

Figure~\ref{fig:computed_probabilities_vs_theoretic_probabilities} shows a
regression line fitted to every pairwise interaction with a reported
\(\text{SSError}\) value (pairwise interactions with missing states were
omitted). This serves to validate the approach: a part from some edge cases the
relationship is consistent.

\begin{figure}[!htbp]
    \centering
    \includegraphics[width=.8\textwidth]{./assets/img/computed_probabilities_vs_theoretic_probabilities/main.pdf}
    \caption{The
        relationship between the steady state probabilities inferred from the
        measured transitions and the actual steady state probabilities. A linear
        regression line is included validating the approach.}
    \label{fig:computed_probabilities_vs_theoretic_probabilities}
\end{figure}


\end{document}

have a final distribution greater than \(10 ^ {-2}\)). This confirms the
findings of~\cite{Moran1707} in which sophisticated strategies resist
evolutionary invasion of shorter memory strategies. Recalling
Figure~\ref{fig:SSError_and_probabilities_in_full} this demonstrates that:

\begin{itemize}
    \item Cooperation emerges through the evolutionary process: the high scoring
        strategies do not exhibit extortionate behaviour towards each other.
    \item Extortionate strategies do not survive the evolutionary process.
\end{itemize}

\begin{figure}[!htbp]
    \centering
    \includegraphics[width=.8\textwidth]{./assets/img/replicator_dynamics/main.pdf}
    \caption{Numerical simulation of the replicator equation
    (\ref{eqn:replicator_dynamics}): strategies are ordered by score, only the strategies with a high score survive the evolutionary process.}
    \label{fig:replicator_dynamics}
\end{figure}

This work can be used to classify plays of the IPD\@: data can be collected from
actual interactions (in lab or in the field). Furthermore, this allows for a
classification method similar to the notion of fingerprinting presented
in~\cite{Ashlock2008}. Trained strategies can potentially be classified as
extortionate or not or it could be possible to even constrain the reinforcement
learning approaches that are becoming prevalent in the literature.
Alternatively, this mathematical approach for recognising extortion could be
used in sophisticated strategies to defend against invasion. Arguably, some of
the strategies considered here exhibit this behaviour, indeed as described
in~\cite{Harper2017}, the top ranking strategies in the full tournament are
obtained using evolutionary reinforcement learning techniques, thus, suspicion
of extortionate behaviour could in fact be an evolutionary trait.

\section*{Acknowledgements}

The following open source software libraries were used in this research:

\begin{itemize}
    \item The Axelrod ~\cite{Knight2016, Knight2018} library (IPD strategies and
        tournaments).
    \item The sympy library~\cite{Meurer2017} (verification of all symbolic
        calculations).
    \item The matplotlib~\cite{Droettboom2018} library (visualisation).
    \item The pandas~\cite{Structures2010}, dask~\cite{Dask2016} and
        NumPy~\cite{Oliphant2015} libraries (data manipulation).
    \item The SciPy~\cite{Jones2001} library (numerical integration of the
        replicator equation).
\end{itemize}

This work was performed using the computational facilities of the Advanced
Research Computing @ Cardiff (ARCCA) Division, Cardiff University.

\printbibliography

\newpage
\section*{Supplementary materials}

\includepdf{assets/pdf/proof_of_form_of_extortionate_strategies/main.pdf}

\newpage

Using the pair wise interactions the transition rates \(p,
q\) can be measured and the steady state probabilities inferred and compared to
the actual probabilities of each state.
This is done numerically by computing the singular eigenvector of the
matrix \(A\) \cite{Stewart2009}:

\[
    A =
    \begin{bmatrix}
        p_1 q_1 & p_1 (1 - q_1) & (1 - p_1) q_1 & (1 -p_1) (1 - q_1) \\
        p_2 q_2 & p_2 (1 - q_2) & (1 - p_2) q_2 & (1 -p_2) (1 - q_2) \\
        p_3 q_3 & p_3 (1 - q_3) & (1 - p_3) q_3 & (1 -p_3) (1 - q_3) \\
        p_4 q_4 & p_4 (1 - q_4) & (1 - p_4) q_4 & (1 -p_4) (1 - q_4) \\
    \end{bmatrix}
\]

Figure~\ref{fig:computed_probabilities_vs_theoretic_probabilities} shows a
regression line fitted to every pairwise interaction with a reported
\(\text{SSError}\) value (pairwise interactions with missing states were
omitted). This serves to validate the approach: a part from some edge cases the
relationship is consistent.

\begin{figure}[!htbp]
    \centering
    \includegraphics[width=.8\textwidth]{./assets/img/computed_probabilities_vs_theoretic_probabilities/main.pdf}
    \caption{The
        relationship between the steady state probabilities inferred from the
        measured transitions and the actual steady state probabilities. A linear
        regression line is included validating the approach.}
    \label{fig:computed_probabilities_vs_theoretic_probabilities}
\end{figure}


\end{document}

    strategies is considered. In this setting
    the most highly performing strategies do not play in an extortionate way
    against each other but do against lower performing strategies.
    This suggests that whilst the theory of Zero Determinant strategies
    indicates that memory is not of fundamental importance to the evolution of
    cooperative behaviour, this is incomplete.
\end{abstract}

\section{Introduction}\label{sec:introduction}

Agent based game theoretic models have become a stalwart of the underpinning
mathematics of interactive behaviours. One of the major pieces of work
in this area is the pair of original computer tournaments run by Robert
Axelrod~\cite{Axelrod1980, Axelrod1980a}. These tournaments pitted submitted
computer strategies against each other in plays of the Iterated Prisoner's
Dilemma. A common game where agents can choose to pay a slight cost to their
immediate utility in the hope of building a reputation. This has been used in
economic and evolutionary game theory to understand the evolution of cooperative
behaviour.

Recently, a class of strategies was described in~\cite{Press2012} that can
provably extort any given opponent. In~\cite{Hilbe2013, Moran1707} some
questions have already been asked about the true effectiveness of these
strategies in an evolutionary setting. Here another question is asked: is it
possible to recognise this extortionate behaviour? A mathematical procedure for
suspicion is presented: in the same way that the continued actions of an
extortionate individual might raise suspicion.

This work makes use of the Axelrod Python library~\cite{Knight2018, Knight2016}
with a large number of Prisoner Dilemma strategies available to give an
extensive numerical example of the ideas presented.  The approach is presented
in Section~\ref{sec:delta-zd-strategies}.  All of the code and data discussed
in Section~\ref{sec:numerical-experiments} is open sourced, archived and
written according to best scientific principles~\cite{Wilson2014}. The data
archive can be found at~\cite{vincent_knight_2018_1297075}.

\section{Recognising Extortion}\label{sec:delta-zd-strategies}

In~\cite{Press2012}, given a match between 2 memory-one strategies, the concept
of Zero Determinant (ZD) strategies is introduced. The main result of that paper
shows that given two memory one players \(p, q\in\mathbb{R}^4\) a linear
relationship between the players' scores could be forced by one of the players.

Using the notation of~\cite{Press2012}, assuming the utilities for player \(p\)
are given by \(S_x=(R, S, T, P)\) and for player \(q\) by \(S_y=(R, T, S, P)\)
and that the stationary scores of each player is given by \(S_X\) and \(S_Y\)
respectively. The main result of~\cite{Press2012} is that if

\begin{equation}\label{eqn:linear_relationship_for_p}
    \tilde p=\alpha S_x + \beta S_y + \gamma
\end{equation}

or

\begin{equation}\label{eqn:linear_relationship_for_q}
    \tilde q=\alpha S_x + \beta S_y + \gamma
\end{equation}

where \(\tilde p = (1 - p_1, 1 - p_2, p_3, p_4)\) and
\(\tilde q = (1 - q_1, 1 - q_2, q_3, q_4)\) then:

\begin{equation}
    \alpha S_X + \beta S_Y + \gamma = 0
\end{equation}

In~\cite{Press2012} a particular type of ZD strategy is defined: extortionate
strategies. If:

\begin{equation}\label{eqn:constraint_for_extortion}
    \gamma = - P(\alpha + \beta)
\end{equation}

then the player can ensure they get a score \(\chi\) times
larger than the opponent. This extortion coefficient is given by:

\begin{equation}\label{eqn:definition_of_chi}
    \chi=\frac{-\beta}{\alpha}
\end{equation}

Thus, if (\ref{eqn:constraint_for_extortion}) holds and \(\chi >1\) a player is
said to extort their opponent.
Here, the reverse problem is considered: given a
\(p\in\mathbb{R}^4\) how does one identify \(\alpha, \beta\) if they
exist and is the strategy in fact acting in an extortionate way?

These conditions correspond to:

\begin{align}
    \tilde p_1 & = \alpha R + \beta R - P (\alpha + \beta)
            \label{eqn:condition_for_tilde_p1}\\
    \tilde p_2 & = \alpha S + \beta T - P (\alpha + \beta)
            \label{eqn:condition_for_tilde_p2}\\
    \tilde p_3 & = \alpha T + \beta S - P (\alpha + \beta)
            \label{eqn:condition_for_tilde_p3}\\
    \tilde p_4 & = \alpha P + \beta P - P (\alpha + \beta)
            \label{eqn:condition_for_tilde_p4}
\end{align}

Equation (\ref{eqn:condition_for_tilde_p4}) ensures that \(p_4=\tilde p_4=0\).
Equations (\ref{eqn:condition_for_tilde_p1}-\ref{eqn:condition_for_tilde_p3})
can be used to eliminate \(\alpha, \beta\), giving:

\begin{equation}\label{eqn:planar_definition_of_extortion}
    \tilde p_1 = \frac{(R - P)(\tilde p_2 + \tilde p_3)}{S + T - 2P}
\end{equation}

with:

\begin{equation}\label{eqn:definition_of_chi}
    \chi = \frac{\tilde p_2 (P - T) + \tilde p_3 (S - P)}
                {\tilde p_2 (P - S) + \tilde p_3 (T - P)}
\end{equation}

Given a strategy \(p\in\mathbb{R}^{4\times 1}\) equations
(\ref{eqn:condition_for_tilde_p4}), (\ref{eqn:planar_definition_of_extortion}-\ref{eqn:definition_of_chi}) can be used to check if
a strategy is extortionate. The conditions correspond to:

\begin{align}
    p_1 & = \frac{(R-P)(p_2 + p_3) - R + T + S - P}{S + T - 2P}
     \label{eqn:condition_for_p1}\\
    p_4 & = 0 \label{eqn:condition_for_p4}\\
    1 & > p_2 + p_3\label{eqn:condition_for_chi}
\end{align}

The algebraic steps necessary to prove these results are available in the
supporting materials.

All extortionate strategies reside on a triangular (\ref{eqn:condition_for_chi})
plane (\ref{eqn:condition_for_p1}) in 3 dimensions (\ref{eqn:condition_for_p4}).
Using this formulation it can be seen that a necessary (but not sufficient)
condition for an extortionate strategy is that it cooperates on average less
than 50\% of the time when in a state of disagreement with the opponent.

As an example, consider the known extortionate strategy \(p=(8 / 9, 1 / 2, 1 /
3, 0)\) from~\cite{Stewart2012} which is referred to as \texttt{Extort-2}. In
this case, for the standard values of \((R, T, S, P)\) constraint
(\ref{eqn:condition_for_p1}) corresponds to:

\begin{equation}
    p_1 = \frac{2(p_2 + p_3) + 1}{3}
\end{equation}

It is clear that in this case all constraints hold.

This approach could in fact be used to confirm that a given strategy is acting
in an extortionate manner even if it is not a memory one strategy. However, in
practice, if a closed form for \(p\) is not known, then due to measurement
and/or numerical error this would not work.

This problem can be written in the following linear algebraic form where
\(x=(\alpha, \beta)\)
and \(p^*=(\tilde p_1 - 1, tilde_2 - 1, p_3)\):

\begin{equation}\label{eqn:linear_algebraic_equation_for_p}
    Cx= p^*
\end{equation}

\(C\) corresponds to equations
(\ref{eqn:condition_for_tilde_p1}-\ref{eqn:condition_for_tilde_p3}) and is
given by:

\begin{equation}\label{eqn:definition_of_C}
    C =
    \begin{bmatrix}
        R - P & R- P \\
        S - P & T- P \\
        T - P & S- P \\
    \end{bmatrix}
\end{equation}

Note that in general, equation (\ref{eqn:linear_algebraic_equation_for_p}) will
not necessarily have a solution. From the Rouch\'{e}-Capelli theorem if there is
a solution it is unique as \(\text{rank}(C)=2\) which is the dimension of the
variable \(x\). The best fitting \(x\) is found by minimizing:

\begin{equation}\label{eqn:r_squared}
    \text{SSError} = \|C x- p^*\|_2^2 = \sum_{i=1}^{3}\left((C\bar x)_i-p_i^*\right)^2
\end{equation}

Note that \(\text{SSError}\), which is the square of the Frobenius
norm~\cite{Golub2013}, becomes a measure of how close a strategy is to being an
extortionate strategy. Suspicion
of extortion then corresponds to a threshold on \(\text{SSError}\).

By observing interactions (human or otherwise), their memory one representation
can be inferred and this approach can be used to recognise extortionate
behaviour. The notion of comparing theoretic and actual plays of the IPD is not
novel, see for example~\cite{Rand2013}. Immediately it is noted that if the
environment is noisy~\cite{Wu1995} then no strategy can be considered to be
extortionate as \(p_4>0\).

In the next section, this idea will be illustrated by observing the interactions
that take place in a computer based tournament of the IPD\@.

\section{Numerical experiments}\label{sec:numerical-experiments}

In~\cite{Stewart2012} results from a tournament with
\documentclass[a4paper]{article}

\usepackage{amsmath}
\usepackage{amssymb}
\usepackage[margin=1.5cm,
            includefoot,
            footskip=30pt]{geometry}
\usepackage{layout}
\usepackage{graphicx}
\usepackage{subcaption}

\usepackage{biblatex}
\usepackage{pdfpages}

\bibliography{main.bib}

\title{Suspicion: Recognising and evaluating the effectiveness
       of extortion in the Iterated Prisoner's Dilemma}
\author{Vincent A. Knight \and Nikoleta E. Glynatsi}
\date{\today}



\begin{document}

\maketitle

\begin{abstract}
    The Iterated Prisoner's Dilemma is a model for rational and evolutionary
    interactive behaviour. It has applications both in the study of human social
    behaviour as well as in biology.
    It is used to understand when and how a rational individual might
    accept an immediate cost to their own utility for the direct benefit of
    another.

    Much attention has been given to a class of strategies called
    Zero Determinant strategies. It has been theoretically shown that these
    strategies can ``extort'' any player.

    In this work, an approach to identify if observed strategies are playing in
    an extortionate way is described. Furthermore, experimental analysis of
    a large tournament with \documentclass[a4paper]{article}

\usepackage{amsmath}
\usepackage{amssymb}
\usepackage[margin=1.5cm,
            includefoot,
            footskip=30pt]{geometry}
\usepackage{layout}
\usepackage{graphicx}
\usepackage{subcaption}

\usepackage{biblatex}
\usepackage{pdfpages}

\bibliography{main.bib}

\title{Suspicion: Recognising and evaluating the effectiveness
       of extortion in the Iterated Prisoner's Dilemma}
\author{Vincent A. Knight \and Nikoleta E. Glynatsi}
\date{\today}



\begin{document}

\maketitle

\begin{abstract}
    The Iterated Prisoner's Dilemma is a model for rational and evolutionary
    interactive behaviour. It has applications both in the study of human social
    behaviour as well as in biology.
    It is used to understand when and how a rational individual might
    accept an immediate cost to their own utility for the direct benefit of
    another.

    Much attention has been given to a class of strategies called
    Zero Determinant strategies. It has been theoretically shown that these
    strategies can ``extort'' any player.

    In this work, an approach to identify if observed strategies are playing in
    an extortionate way is described. Furthermore, experimental analysis of
    a large tournament with \input{assets/tex/number_of_full_strategies/main.tex}
    strategies is considered. In this setting
    the most highly performing strategies do not play in an extortionate way
    against each other but do against lower performing strategies.
    This suggests that whilst the theory of Zero Determinant strategies
    indicates that memory is not of fundamental importance to the evolution of
    cooperative behaviour, this is incomplete.
\end{abstract}

\section{Introduction}\label{sec:introduction}

Agent based game theoretic models have become a stalwart of the underpinning
mathematics of interactive behaviours. One of the major pieces of work
in this area is the pair of original computer tournaments run by Robert
Axelrod~\cite{Axelrod1980, Axelrod1980a}. These tournaments pitted submitted
computer strategies against each other in plays of the Iterated Prisoner's
Dilemma. A common game where agents can choose to pay a slight cost to their
immediate utility in the hope of building a reputation. This has been used in
economic and evolutionary game theory to understand the evolution of cooperative
behaviour.

Recently, a class of strategies was described in~\cite{Press2012} that can
provably extort any given opponent. In~\cite{Hilbe2013, Moran1707} some
questions have already been asked about the true effectiveness of these
strategies in an evolutionary setting. Here another question is asked: is it
possible to recognise this extortionate behaviour? A mathematical procedure for
suspicion is presented: in the same way that the continued actions of an
extortionate individual might raise suspicion.

This work makes use of the Axelrod Python library~\cite{Knight2018, Knight2016}
with a large number of Prisoner Dilemma strategies available to give an
extensive numerical example of the ideas presented.  The approach is presented
in Section~\ref{sec:delta-zd-strategies}.  All of the code and data discussed
in Section~\ref{sec:numerical-experiments} is open sourced, archived and
written according to best scientific principles~\cite{Wilson2014}. The data
archive can be found at~\cite{vincent_knight_2018_1297075}.

\section{Recognising Extortion}\label{sec:delta-zd-strategies}

In~\cite{Press2012}, given a match between 2 memory-one strategies, the concept
of Zero Determinant (ZD) strategies is introduced. The main result of that paper
shows that given two memory one players \(p, q\in\mathbb{R}^4\) a linear
relationship between the players' scores could be forced by one of the players.

Using the notation of~\cite{Press2012}, assuming the utilities for player \(p\)
are given by \(S_x=(R, S, T, P)\) and for player \(q\) by \(S_y=(R, T, S, P)\)
and that the stationary scores of each player is given by \(S_X\) and \(S_Y\)
respectively. The main result of~\cite{Press2012} is that if

\begin{equation}\label{eqn:linear_relationship_for_p}
    \tilde p=\alpha S_x + \beta S_y + \gamma
\end{equation}

or

\begin{equation}\label{eqn:linear_relationship_for_q}
    \tilde q=\alpha S_x + \beta S_y + \gamma
\end{equation}

where \(\tilde p = (1 - p_1, 1 - p_2, p_3, p_4)\) and
\(\tilde q = (1 - q_1, 1 - q_2, q_3, q_4)\) then:

\begin{equation}
    \alpha S_X + \beta S_Y + \gamma = 0
\end{equation}

In~\cite{Press2012} a particular type of ZD strategy is defined: extortionate
strategies. If:

\begin{equation}\label{eqn:constraint_for_extortion}
    \gamma = - P(\alpha + \beta)
\end{equation}

then the player can ensure they get a score \(\chi\) times
larger than the opponent. This extortion coefficient is given by:

\begin{equation}\label{eqn:definition_of_chi}
    \chi=\frac{-\beta}{\alpha}
\end{equation}

Thus, if (\ref{eqn:constraint_for_extortion}) holds and \(\chi >1\) a player is
said to extort their opponent.
Here, the reverse problem is considered: given a
\(p\in\mathbb{R}^4\) how does one identify \(\alpha, \beta\) if they
exist and is the strategy in fact acting in an extortionate way?

These conditions correspond to:

\begin{align}
    \tilde p_1 & = \alpha R + \beta R - P (\alpha + \beta)
            \label{eqn:condition_for_tilde_p1}\\
    \tilde p_2 & = \alpha S + \beta T - P (\alpha + \beta)
            \label{eqn:condition_for_tilde_p2}\\
    \tilde p_3 & = \alpha T + \beta S - P (\alpha + \beta)
            \label{eqn:condition_for_tilde_p3}\\
    \tilde p_4 & = \alpha P + \beta P - P (\alpha + \beta)
            \label{eqn:condition_for_tilde_p4}
\end{align}

Equation (\ref{eqn:condition_for_tilde_p4}) ensures that \(p_4=\tilde p_4=0\).
Equations (\ref{eqn:condition_for_tilde_p1}-\ref{eqn:condition_for_tilde_p3})
can be used to eliminate \(\alpha, \beta\), giving:

\begin{equation}\label{eqn:planar_definition_of_extortion}
    \tilde p_1 = \frac{(R - P)(\tilde p_2 + \tilde p_3)}{S + T - 2P}
\end{equation}

with:

\begin{equation}\label{eqn:definition_of_chi}
    \chi = \frac{\tilde p_2 (P - T) + \tilde p_3 (S - P)}
                {\tilde p_2 (P - S) + \tilde p_3 (T - P)}
\end{equation}

Given a strategy \(p\in\mathbb{R}^{4\times 1}\) equations
(\ref{eqn:condition_for_tilde_p4}), (\ref{eqn:planar_definition_of_extortion}-\ref{eqn:definition_of_chi}) can be used to check if
a strategy is extortionate. The conditions correspond to:

\begin{align}
    p_1 & = \frac{(R-P)(p_2 + p_3) - R + T + S - P}{S + T - 2P}
     \label{eqn:condition_for_p1}\\
    p_4 & = 0 \label{eqn:condition_for_p4}\\
    1 & > p_2 + p_3\label{eqn:condition_for_chi}
\end{align}

The algebraic steps necessary to prove these results are available in the
supporting materials.

All extortionate strategies reside on a triangular (\ref{eqn:condition_for_chi})
plane (\ref{eqn:condition_for_p1}) in 3 dimensions (\ref{eqn:condition_for_p4}).
Using this formulation it can be seen that a necessary (but not sufficient)
condition for an extortionate strategy is that it cooperates on average less
than 50\% of the time when in a state of disagreement with the opponent.

As an example, consider the known extortionate strategy \(p=(8 / 9, 1 / 2, 1 /
3, 0)\) from~\cite{Stewart2012} which is referred to as \texttt{Extort-2}. In
this case, for the standard values of \((R, T, S, P)\) constraint
(\ref{eqn:condition_for_p1}) corresponds to:

\begin{equation}
    p_1 = \frac{2(p_2 + p_3) + 1}{3}
\end{equation}

It is clear that in this case all constraints hold.

This approach could in fact be used to confirm that a given strategy is acting
in an extortionate manner even if it is not a memory one strategy. However, in
practice, if a closed form for \(p\) is not known, then due to measurement
and/or numerical error this would not work.

This problem can be written in the following linear algebraic form where
\(x=(\alpha, \beta)\)
and \(p^*=(\tilde p_1 - 1, tilde_2 - 1, p_3)\):

\begin{equation}\label{eqn:linear_algebraic_equation_for_p}
    Cx= p^*
\end{equation}

\(C\) corresponds to equations
(\ref{eqn:condition_for_tilde_p1}-\ref{eqn:condition_for_tilde_p3}) and is
given by:

\begin{equation}\label{eqn:definition_of_C}
    C =
    \begin{bmatrix}
        R - P & R- P \\
        S - P & T- P \\
        T - P & S- P \\
    \end{bmatrix}
\end{equation}

Note that in general, equation (\ref{eqn:linear_algebraic_equation_for_p}) will
not necessarily have a solution. From the Rouch\'{e}-Capelli theorem if there is
a solution it is unique as \(\text{rank}(C)=2\) which is the dimension of the
variable \(x\). The best fitting \(x\) is found by minimizing:

\begin{equation}\label{eqn:r_squared}
    \text{SSError} = \|C x- p^*\|_2^2 = \sum_{i=1}^{3}\left((C\bar x)_i-p_i^*\right)^2
\end{equation}

Note that \(\text{SSError}\), which is the square of the Frobenius
norm~\cite{Golub2013}, becomes a measure of how close a strategy is to being an
extortionate strategy. Suspicion
of extortion then corresponds to a threshold on \(\text{SSError}\).

By observing interactions (human or otherwise), their memory one representation
can be inferred and this approach can be used to recognise extortionate
behaviour. The notion of comparing theoretic and actual plays of the IPD is not
novel, see for example~\cite{Rand2013}. Immediately it is noted that if the
environment is noisy~\cite{Wu1995} then no strategy can be considered to be
extortionate as \(p_4>0\).

In the next section, this idea will be illustrated by observing the interactions
that take place in a computer based tournament of the IPD\@.

\section{Numerical experiments}\label{sec:numerical-experiments}

In~\cite{Stewart2012} results from a tournament with
\input{./assets/tex/number_of_stewart_plotkin_strategies/main.tex} strategies,
was presented with specific consideration given to ZD strategies. This
tournament is reproduced here using the Axelrod-Python
project~\cite{Knight2016}. To obtain a good measure of the corresponding
transition rates for each strategy all matches have been run for
\input{assets/tex/number_of_turns/main.tex} turns and every match has been
repeated \input{assets/tex/number_of_repetitions/main.tex} times. All of this
interaction data is available at~\cite{vincent_knight_2018_1297075}. A good
match between the inferred Markov chain and the state distribution of the actual
interactions has been verified. Data for this is presented in the supplementary
materials.

Figure~\ref{fig:SSError_overall_in_stewart_plotkin} shows the \(\text{SSError}\)
values for all the strategies in the tournament, as reported
in~\cite{Stewart2012} the extortionate strategy (which has an expected
\(\text{SSError}\) approximately 0) gains a large number of wins.

\begin{figure}[!htbp]
    \centering
    \includegraphics[width=.8\textwidth]{./assets/img/SSError_overall_in_stewart_plotkin/main.pdf}
    \caption{\(\text{SSError}\) and state probabilities for the strategies
        of~\cite{Stewart2012}, ordered both by number of wins and overall score.
        Note that \(P(DC)\) is not shown as it corresponds to the transpose of
        \(P(CD)\). Cooperator and Defector are omitted as they do not visit all
        the states.}
    \label{fig:SSError_overall_in_stewart_plotkin}
\end{figure}

Here, the work of~\cite{Stewart2012} is extended by investigating a tournament
with \input{assets/tex/number_of_full_strategies/main.tex}
strategies.

The results of this analysis are shown in
Figure~\ref{fig:SSError_and_probabilities_in_full}. The top ranking strategies
by number of wins seem to be extortionate (but not against all strategies) and
it can be seen that a small sub group of strategies achieve mutual defection.
All the top ranking strategies according to score achieve mutual cooperation and
do not extort each other, however they
\textbf{do} exhibit extortionate behaviour towards a number of the lower ranking
strategies.

\begin{figure}[!htbp]
    \centering
    \includegraphics[width=.8\textwidth]{./assets/img/SSError_and_probabilities_in_full/main.pdf}
    \caption{\(\text{SSError}\) for the strategies for the full tournament. Only
    strategy interactions for which \(p_4=0\) and \(\chi>1\) are displayed.}
    \label{fig:SSError_and_probabilities_in_full}
\end{figure}

\section{Conclusion}\label{sec:conclusion}

This work defines an approach to measure whether or not a player is playing a
strategy that corresponds to an extortionate strategy as defined
in~\cite{Press2012}: a mathematical model for suspicion. Indeed, all
extortionate strategies have been
 classified as lying on a triangular plane.
This rigorous classification fails to be robust to small measurement error, thus
a statistical approach is proposed.
This is done through a linear algebraic approach for approximating the solution
of a linear system. Using this, a large number of pairwise interactions is
simulated and in fact very few strategies are found to act extortionately.

The work of~\cite{Press2012}, whilst showing that a clever approach to taking
advantage of another memory one strategy exists: this is incomplete. Whilst the
elegance of this result is very attractive, just as the simplicity of the
victory of Tit For Tat in Axelrod's original tournaments was, it is incomplete.
Extortionate strategies achieve a high number of wins but they do not
achieve a high score which corresponds to the fitness landscape in an
evolutionary sense. From the large number of interactions a payoff matrix \(S\)
can be measured where \(S_{ij}\) denotes the score (using standard values of
\((R, S, T, P) = (3, 0, 5, 1)\)) of the \(i\)th strategy
against the \(j\)th strategy. Using this, the replicator equation
describes the evolution of the system based on a population density fitness
function:

\begin{equation}\label{eqn:replicator_dynamics}
    \frac{dx}{dt} = x(S-x^TS x)
\end{equation}

Equation (\ref{eqn:replicator_dynamics}) is solved numerically through an
integration technique described in~\cite{Petzold1983} and
Figure~\ref{fig:replicator_dynamics} shows the evolution of the distribution of
the system: the various strategies are ranked by scores. It is clear to see that
only the high ranking strategies survive the evolutionary process (in fact,
only \input{./assets/img/replicator_dynamics/main.tex}
have a final distribution greater than \(10 ^ {-2}\)). This confirms the
findings of~\cite{Moran1707} in which sophisticated strategies resist
evolutionary invasion of shorter memory strategies. Recalling
Figure~\ref{fig:SSError_and_probabilities_in_full} this demonstrates that:

\begin{itemize}
    \item Cooperation emerges through the evolutionary process: the high scoring
        strategies do not exhibit extortionate behaviour towards each other.
    \item Extortionate strategies do not survive the evolutionary process.
\end{itemize}

\begin{figure}[!htbp]
    \centering
    \includegraphics[width=.8\textwidth]{./assets/img/replicator_dynamics/main.pdf}
    \caption{Numerical simulation of the replicator equation
    (\ref{eqn:replicator_dynamics}): strategies are ordered by score, only the strategies with a high score survive the evolutionary process.}
    \label{fig:replicator_dynamics}
\end{figure}

This work can be used to classify plays of the IPD\@: data can be collected from
actual interactions (in lab or in the field). Furthermore, this allows for a
classification method similar to the notion of fingerprinting presented
in~\cite{Ashlock2008}. Trained strategies can potentially be classified as
extortionate or not or it could be possible to even constrain the reinforcement
learning approaches that are becoming prevalent in the literature.
Alternatively, this mathematical approach for recognising extortion could be
used in sophisticated strategies to defend against invasion. Arguably, some of
the strategies considered here exhibit this behaviour, indeed as described
in~\cite{Harper2017}, the top ranking strategies in the full tournament are
obtained using evolutionary reinforcement learning techniques, thus, suspicion
of extortionate behaviour could in fact be an evolutionary trait.

\section*{Acknowledgements}

The following open source software libraries were used in this research:

\begin{itemize}
    \item The Axelrod ~\cite{Knight2016, Knight2018} library (IPD strategies and
        tournaments).
    \item The sympy library~\cite{Meurer2017} (verification of all symbolic
        calculations).
    \item The matplotlib~\cite{Droettboom2018} library (visualisation).
    \item The pandas~\cite{Structures2010}, dask~\cite{Dask2016} and
        NumPy~\cite{Oliphant2015} libraries (data manipulation).
    \item The SciPy~\cite{Jones2001} library (numerical integration of the
        replicator equation).
\end{itemize}

This work was performed using the computational facilities of the Advanced
Research Computing @ Cardiff (ARCCA) Division, Cardiff University.

\printbibliography

\newpage
\section*{Supplementary materials}

\includepdf{assets/pdf/proof_of_form_of_extortionate_strategies/main.pdf}

\newpage

Using the pair wise interactions the transition rates \(p,
q\) can be measured and the steady state probabilities inferred and compared to
the actual probabilities of each state.
This is done numerically by computing the singular eigenvector of the
matrix \(A\) \cite{Stewart2009}:

\[
    A =
    \begin{bmatrix}
        p_1 q_1 & p_1 (1 - q_1) & (1 - p_1) q_1 & (1 -p_1) (1 - q_1) \\
        p_2 q_2 & p_2 (1 - q_2) & (1 - p_2) q_2 & (1 -p_2) (1 - q_2) \\
        p_3 q_3 & p_3 (1 - q_3) & (1 - p_3) q_3 & (1 -p_3) (1 - q_3) \\
        p_4 q_4 & p_4 (1 - q_4) & (1 - p_4) q_4 & (1 -p_4) (1 - q_4) \\
    \end{bmatrix}
\]

Figure~\ref{fig:computed_probabilities_vs_theoretic_probabilities} shows a
regression line fitted to every pairwise interaction with a reported
\(\text{SSError}\) value (pairwise interactions with missing states were
omitted). This serves to validate the approach: a part from some edge cases the
relationship is consistent.

\begin{figure}[!htbp]
    \centering
    \includegraphics[width=.8\textwidth]{./assets/img/computed_probabilities_vs_theoretic_probabilities/main.pdf}
    \caption{The
        relationship between the steady state probabilities inferred from the
        measured transitions and the actual steady state probabilities. A linear
        regression line is included validating the approach.}
    \label{fig:computed_probabilities_vs_theoretic_probabilities}
\end{figure}


\end{document}

    strategies is considered. In this setting
    the most highly performing strategies do not play in an extortionate way
    against each other but do against lower performing strategies.
    This suggests that whilst the theory of Zero Determinant strategies
    indicates that memory is not of fundamental importance to the evolution of
    cooperative behaviour, this is incomplete.
\end{abstract}

\section{Introduction}\label{sec:introduction}

Agent based game theoretic models have become a stalwart of the underpinning
mathematics of interactive behaviours. One of the major pieces of work
in this area is the pair of original computer tournaments run by Robert
Axelrod~\cite{Axelrod1980, Axelrod1980a}. These tournaments pitted submitted
computer strategies against each other in plays of the Iterated Prisoner's
Dilemma. A common game where agents can choose to pay a slight cost to their
immediate utility in the hope of building a reputation. This has been used in
economic and evolutionary game theory to understand the evolution of cooperative
behaviour.

Recently, a class of strategies was described in~\cite{Press2012} that can
provably extort any given opponent. In~\cite{Hilbe2013, Moran1707} some
questions have already been asked about the true effectiveness of these
strategies in an evolutionary setting. Here another question is asked: is it
possible to recognise this extortionate behaviour? A mathematical procedure for
suspicion is presented: in the same way that the continued actions of an
extortionate individual might raise suspicion.

This work makes use of the Axelrod Python library~\cite{Knight2018, Knight2016}
with a large number of Prisoner Dilemma strategies available to give an
extensive numerical example of the ideas presented.  The approach is presented
in Section~\ref{sec:delta-zd-strategies}.  All of the code and data discussed
in Section~\ref{sec:numerical-experiments} is open sourced, archived and
written according to best scientific principles~\cite{Wilson2014}. The data
archive can be found at~\cite{vincent_knight_2018_1297075}.

\section{Recognising Extortion}\label{sec:delta-zd-strategies}

In~\cite{Press2012}, given a match between 2 memory-one strategies, the concept
of Zero Determinant (ZD) strategies is introduced. The main result of that paper
shows that given two memory one players \(p, q\in\mathbb{R}^4\) a linear
relationship between the players' scores could be forced by one of the players.

Using the notation of~\cite{Press2012}, assuming the utilities for player \(p\)
are given by \(S_x=(R, S, T, P)\) and for player \(q\) by \(S_y=(R, T, S, P)\)
and that the stationary scores of each player is given by \(S_X\) and \(S_Y\)
respectively. The main result of~\cite{Press2012} is that if

\begin{equation}\label{eqn:linear_relationship_for_p}
    \tilde p=\alpha S_x + \beta S_y + \gamma
\end{equation}

or

\begin{equation}\label{eqn:linear_relationship_for_q}
    \tilde q=\alpha S_x + \beta S_y + \gamma
\end{equation}

where \(\tilde p = (1 - p_1, 1 - p_2, p_3, p_4)\) and
\(\tilde q = (1 - q_1, 1 - q_2, q_3, q_4)\) then:

\begin{equation}
    \alpha S_X + \beta S_Y + \gamma = 0
\end{equation}

In~\cite{Press2012} a particular type of ZD strategy is defined: extortionate
strategies. If:

\begin{equation}\label{eqn:constraint_for_extortion}
    \gamma = - P(\alpha + \beta)
\end{equation}

then the player can ensure they get a score \(\chi\) times
larger than the opponent. This extortion coefficient is given by:

\begin{equation}\label{eqn:definition_of_chi}
    \chi=\frac{-\beta}{\alpha}
\end{equation}

Thus, if (\ref{eqn:constraint_for_extortion}) holds and \(\chi >1\) a player is
said to extort their opponent.
Here, the reverse problem is considered: given a
\(p\in\mathbb{R}^4\) how does one identify \(\alpha, \beta\) if they
exist and is the strategy in fact acting in an extortionate way?

These conditions correspond to:

\begin{align}
    \tilde p_1 & = \alpha R + \beta R - P (\alpha + \beta)
            \label{eqn:condition_for_tilde_p1}\\
    \tilde p_2 & = \alpha S + \beta T - P (\alpha + \beta)
            \label{eqn:condition_for_tilde_p2}\\
    \tilde p_3 & = \alpha T + \beta S - P (\alpha + \beta)
            \label{eqn:condition_for_tilde_p3}\\
    \tilde p_4 & = \alpha P + \beta P - P (\alpha + \beta)
            \label{eqn:condition_for_tilde_p4}
\end{align}

Equation (\ref{eqn:condition_for_tilde_p4}) ensures that \(p_4=\tilde p_4=0\).
Equations (\ref{eqn:condition_for_tilde_p1}-\ref{eqn:condition_for_tilde_p3})
can be used to eliminate \(\alpha, \beta\), giving:

\begin{equation}\label{eqn:planar_definition_of_extortion}
    \tilde p_1 = \frac{(R - P)(\tilde p_2 + \tilde p_3)}{S + T - 2P}
\end{equation}

with:

\begin{equation}\label{eqn:definition_of_chi}
    \chi = \frac{\tilde p_2 (P - T) + \tilde p_3 (S - P)}
                {\tilde p_2 (P - S) + \tilde p_3 (T - P)}
\end{equation}

Given a strategy \(p\in\mathbb{R}^{4\times 1}\) equations
(\ref{eqn:condition_for_tilde_p4}), (\ref{eqn:planar_definition_of_extortion}-\ref{eqn:definition_of_chi}) can be used to check if
a strategy is extortionate. The conditions correspond to:

\begin{align}
    p_1 & = \frac{(R-P)(p_2 + p_3) - R + T + S - P}{S + T - 2P}
     \label{eqn:condition_for_p1}\\
    p_4 & = 0 \label{eqn:condition_for_p4}\\
    1 & > p_2 + p_3\label{eqn:condition_for_chi}
\end{align}

The algebraic steps necessary to prove these results are available in the
supporting materials.

All extortionate strategies reside on a triangular (\ref{eqn:condition_for_chi})
plane (\ref{eqn:condition_for_p1}) in 3 dimensions (\ref{eqn:condition_for_p4}).
Using this formulation it can be seen that a necessary (but not sufficient)
condition for an extortionate strategy is that it cooperates on average less
than 50\% of the time when in a state of disagreement with the opponent.

As an example, consider the known extortionate strategy \(p=(8 / 9, 1 / 2, 1 /
3, 0)\) from~\cite{Stewart2012} which is referred to as \texttt{Extort-2}. In
this case, for the standard values of \((R, T, S, P)\) constraint
(\ref{eqn:condition_for_p1}) corresponds to:

\begin{equation}
    p_1 = \frac{2(p_2 + p_3) + 1}{3}
\end{equation}

It is clear that in this case all constraints hold.

This approach could in fact be used to confirm that a given strategy is acting
in an extortionate manner even if it is not a memory one strategy. However, in
practice, if a closed form for \(p\) is not known, then due to measurement
and/or numerical error this would not work.

This problem can be written in the following linear algebraic form where
\(x=(\alpha, \beta)\)
and \(p^*=(\tilde p_1 - 1, tilde_2 - 1, p_3)\):

\begin{equation}\label{eqn:linear_algebraic_equation_for_p}
    Cx= p^*
\end{equation}

\(C\) corresponds to equations
(\ref{eqn:condition_for_tilde_p1}-\ref{eqn:condition_for_tilde_p3}) and is
given by:

\begin{equation}\label{eqn:definition_of_C}
    C =
    \begin{bmatrix}
        R - P & R- P \\
        S - P & T- P \\
        T - P & S- P \\
    \end{bmatrix}
\end{equation}

Note that in general, equation (\ref{eqn:linear_algebraic_equation_for_p}) will
not necessarily have a solution. From the Rouch\'{e}-Capelli theorem if there is
a solution it is unique as \(\text{rank}(C)=2\) which is the dimension of the
variable \(x\). The best fitting \(x\) is found by minimizing:

\begin{equation}\label{eqn:r_squared}
    \text{SSError} = \|C x- p^*\|_2^2 = \sum_{i=1}^{3}\left((C\bar x)_i-p_i^*\right)^2
\end{equation}

Note that \(\text{SSError}\), which is the square of the Frobenius
norm~\cite{Golub2013}, becomes a measure of how close a strategy is to being an
extortionate strategy. Suspicion
of extortion then corresponds to a threshold on \(\text{SSError}\).

By observing interactions (human or otherwise), their memory one representation
can be inferred and this approach can be used to recognise extortionate
behaviour. The notion of comparing theoretic and actual plays of the IPD is not
novel, see for example~\cite{Rand2013}. Immediately it is noted that if the
environment is noisy~\cite{Wu1995} then no strategy can be considered to be
extortionate as \(p_4>0\).

In the next section, this idea will be illustrated by observing the interactions
that take place in a computer based tournament of the IPD\@.

\section{Numerical experiments}\label{sec:numerical-experiments}

In~\cite{Stewart2012} results from a tournament with
\documentclass[a4paper]{article}

\usepackage{amsmath}
\usepackage{amssymb}
\usepackage[margin=1.5cm,
            includefoot,
            footskip=30pt]{geometry}
\usepackage{layout}
\usepackage{graphicx}
\usepackage{subcaption}

\usepackage{biblatex}
\usepackage{pdfpages}

\bibliography{main.bib}

\title{Suspicion: Recognising and evaluating the effectiveness
       of extortion in the Iterated Prisoner's Dilemma}
\author{Vincent A. Knight \and Nikoleta E. Glynatsi}
\date{\today}



\begin{document}

\maketitle

\begin{abstract}
    The Iterated Prisoner's Dilemma is a model for rational and evolutionary
    interactive behaviour. It has applications both in the study of human social
    behaviour as well as in biology.
    It is used to understand when and how a rational individual might
    accept an immediate cost to their own utility for the direct benefit of
    another.

    Much attention has been given to a class of strategies called
    Zero Determinant strategies. It has been theoretically shown that these
    strategies can ``extort'' any player.

    In this work, an approach to identify if observed strategies are playing in
    an extortionate way is described. Furthermore, experimental analysis of
    a large tournament with \input{assets/tex/number_of_full_strategies/main.tex}
    strategies is considered. In this setting
    the most highly performing strategies do not play in an extortionate way
    against each other but do against lower performing strategies.
    This suggests that whilst the theory of Zero Determinant strategies
    indicates that memory is not of fundamental importance to the evolution of
    cooperative behaviour, this is incomplete.
\end{abstract}

\section{Introduction}\label{sec:introduction}

Agent based game theoretic models have become a stalwart of the underpinning
mathematics of interactive behaviours. One of the major pieces of work
in this area is the pair of original computer tournaments run by Robert
Axelrod~\cite{Axelrod1980, Axelrod1980a}. These tournaments pitted submitted
computer strategies against each other in plays of the Iterated Prisoner's
Dilemma. A common game where agents can choose to pay a slight cost to their
immediate utility in the hope of building a reputation. This has been used in
economic and evolutionary game theory to understand the evolution of cooperative
behaviour.

Recently, a class of strategies was described in~\cite{Press2012} that can
provably extort any given opponent. In~\cite{Hilbe2013, Moran1707} some
questions have already been asked about the true effectiveness of these
strategies in an evolutionary setting. Here another question is asked: is it
possible to recognise this extortionate behaviour? A mathematical procedure for
suspicion is presented: in the same way that the continued actions of an
extortionate individual might raise suspicion.

This work makes use of the Axelrod Python library~\cite{Knight2018, Knight2016}
with a large number of Prisoner Dilemma strategies available to give an
extensive numerical example of the ideas presented.  The approach is presented
in Section~\ref{sec:delta-zd-strategies}.  All of the code and data discussed
in Section~\ref{sec:numerical-experiments} is open sourced, archived and
written according to best scientific principles~\cite{Wilson2014}. The data
archive can be found at~\cite{vincent_knight_2018_1297075}.

\section{Recognising Extortion}\label{sec:delta-zd-strategies}

In~\cite{Press2012}, given a match between 2 memory-one strategies, the concept
of Zero Determinant (ZD) strategies is introduced. The main result of that paper
shows that given two memory one players \(p, q\in\mathbb{R}^4\) a linear
relationship between the players' scores could be forced by one of the players.

Using the notation of~\cite{Press2012}, assuming the utilities for player \(p\)
are given by \(S_x=(R, S, T, P)\) and for player \(q\) by \(S_y=(R, T, S, P)\)
and that the stationary scores of each player is given by \(S_X\) and \(S_Y\)
respectively. The main result of~\cite{Press2012} is that if

\begin{equation}\label{eqn:linear_relationship_for_p}
    \tilde p=\alpha S_x + \beta S_y + \gamma
\end{equation}

or

\begin{equation}\label{eqn:linear_relationship_for_q}
    \tilde q=\alpha S_x + \beta S_y + \gamma
\end{equation}

where \(\tilde p = (1 - p_1, 1 - p_2, p_3, p_4)\) and
\(\tilde q = (1 - q_1, 1 - q_2, q_3, q_4)\) then:

\begin{equation}
    \alpha S_X + \beta S_Y + \gamma = 0
\end{equation}

In~\cite{Press2012} a particular type of ZD strategy is defined: extortionate
strategies. If:

\begin{equation}\label{eqn:constraint_for_extortion}
    \gamma = - P(\alpha + \beta)
\end{equation}

then the player can ensure they get a score \(\chi\) times
larger than the opponent. This extortion coefficient is given by:

\begin{equation}\label{eqn:definition_of_chi}
    \chi=\frac{-\beta}{\alpha}
\end{equation}

Thus, if (\ref{eqn:constraint_for_extortion}) holds and \(\chi >1\) a player is
said to extort their opponent.
Here, the reverse problem is considered: given a
\(p\in\mathbb{R}^4\) how does one identify \(\alpha, \beta\) if they
exist and is the strategy in fact acting in an extortionate way?

These conditions correspond to:

\begin{align}
    \tilde p_1 & = \alpha R + \beta R - P (\alpha + \beta)
            \label{eqn:condition_for_tilde_p1}\\
    \tilde p_2 & = \alpha S + \beta T - P (\alpha + \beta)
            \label{eqn:condition_for_tilde_p2}\\
    \tilde p_3 & = \alpha T + \beta S - P (\alpha + \beta)
            \label{eqn:condition_for_tilde_p3}\\
    \tilde p_4 & = \alpha P + \beta P - P (\alpha + \beta)
            \label{eqn:condition_for_tilde_p4}
\end{align}

Equation (\ref{eqn:condition_for_tilde_p4}) ensures that \(p_4=\tilde p_4=0\).
Equations (\ref{eqn:condition_for_tilde_p1}-\ref{eqn:condition_for_tilde_p3})
can be used to eliminate \(\alpha, \beta\), giving:

\begin{equation}\label{eqn:planar_definition_of_extortion}
    \tilde p_1 = \frac{(R - P)(\tilde p_2 + \tilde p_3)}{S + T - 2P}
\end{equation}

with:

\begin{equation}\label{eqn:definition_of_chi}
    \chi = \frac{\tilde p_2 (P - T) + \tilde p_3 (S - P)}
                {\tilde p_2 (P - S) + \tilde p_3 (T - P)}
\end{equation}

Given a strategy \(p\in\mathbb{R}^{4\times 1}\) equations
(\ref{eqn:condition_for_tilde_p4}), (\ref{eqn:planar_definition_of_extortion}-\ref{eqn:definition_of_chi}) can be used to check if
a strategy is extortionate. The conditions correspond to:

\begin{align}
    p_1 & = \frac{(R-P)(p_2 + p_3) - R + T + S - P}{S + T - 2P}
     \label{eqn:condition_for_p1}\\
    p_4 & = 0 \label{eqn:condition_for_p4}\\
    1 & > p_2 + p_3\label{eqn:condition_for_chi}
\end{align}

The algebraic steps necessary to prove these results are available in the
supporting materials.

All extortionate strategies reside on a triangular (\ref{eqn:condition_for_chi})
plane (\ref{eqn:condition_for_p1}) in 3 dimensions (\ref{eqn:condition_for_p4}).
Using this formulation it can be seen that a necessary (but not sufficient)
condition for an extortionate strategy is that it cooperates on average less
than 50\% of the time when in a state of disagreement with the opponent.

As an example, consider the known extortionate strategy \(p=(8 / 9, 1 / 2, 1 /
3, 0)\) from~\cite{Stewart2012} which is referred to as \texttt{Extort-2}. In
this case, for the standard values of \((R, T, S, P)\) constraint
(\ref{eqn:condition_for_p1}) corresponds to:

\begin{equation}
    p_1 = \frac{2(p_2 + p_3) + 1}{3}
\end{equation}

It is clear that in this case all constraints hold.

This approach could in fact be used to confirm that a given strategy is acting
in an extortionate manner even if it is not a memory one strategy. However, in
practice, if a closed form for \(p\) is not known, then due to measurement
and/or numerical error this would not work.

This problem can be written in the following linear algebraic form where
\(x=(\alpha, \beta)\)
and \(p^*=(\tilde p_1 - 1, tilde_2 - 1, p_3)\):

\begin{equation}\label{eqn:linear_algebraic_equation_for_p}
    Cx= p^*
\end{equation}

\(C\) corresponds to equations
(\ref{eqn:condition_for_tilde_p1}-\ref{eqn:condition_for_tilde_p3}) and is
given by:

\begin{equation}\label{eqn:definition_of_C}
    C =
    \begin{bmatrix}
        R - P & R- P \\
        S - P & T- P \\
        T - P & S- P \\
    \end{bmatrix}
\end{equation}

Note that in general, equation (\ref{eqn:linear_algebraic_equation_for_p}) will
not necessarily have a solution. From the Rouch\'{e}-Capelli theorem if there is
a solution it is unique as \(\text{rank}(C)=2\) which is the dimension of the
variable \(x\). The best fitting \(x\) is found by minimizing:

\begin{equation}\label{eqn:r_squared}
    \text{SSError} = \|C x- p^*\|_2^2 = \sum_{i=1}^{3}\left((C\bar x)_i-p_i^*\right)^2
\end{equation}

Note that \(\text{SSError}\), which is the square of the Frobenius
norm~\cite{Golub2013}, becomes a measure of how close a strategy is to being an
extortionate strategy. Suspicion
of extortion then corresponds to a threshold on \(\text{SSError}\).

By observing interactions (human or otherwise), their memory one representation
can be inferred and this approach can be used to recognise extortionate
behaviour. The notion of comparing theoretic and actual plays of the IPD is not
novel, see for example~\cite{Rand2013}. Immediately it is noted that if the
environment is noisy~\cite{Wu1995} then no strategy can be considered to be
extortionate as \(p_4>0\).

In the next section, this idea will be illustrated by observing the interactions
that take place in a computer based tournament of the IPD\@.

\section{Numerical experiments}\label{sec:numerical-experiments}

In~\cite{Stewart2012} results from a tournament with
\input{./assets/tex/number_of_stewart_plotkin_strategies/main.tex} strategies,
was presented with specific consideration given to ZD strategies. This
tournament is reproduced here using the Axelrod-Python
project~\cite{Knight2016}. To obtain a good measure of the corresponding
transition rates for each strategy all matches have been run for
\input{assets/tex/number_of_turns/main.tex} turns and every match has been
repeated \input{assets/tex/number_of_repetitions/main.tex} times. All of this
interaction data is available at~\cite{vincent_knight_2018_1297075}. A good
match between the inferred Markov chain and the state distribution of the actual
interactions has been verified. Data for this is presented in the supplementary
materials.

Figure~\ref{fig:SSError_overall_in_stewart_plotkin} shows the \(\text{SSError}\)
values for all the strategies in the tournament, as reported
in~\cite{Stewart2012} the extortionate strategy (which has an expected
\(\text{SSError}\) approximately 0) gains a large number of wins.

\begin{figure}[!htbp]
    \centering
    \includegraphics[width=.8\textwidth]{./assets/img/SSError_overall_in_stewart_plotkin/main.pdf}
    \caption{\(\text{SSError}\) and state probabilities for the strategies
        of~\cite{Stewart2012}, ordered both by number of wins and overall score.
        Note that \(P(DC)\) is not shown as it corresponds to the transpose of
        \(P(CD)\). Cooperator and Defector are omitted as they do not visit all
        the states.}
    \label{fig:SSError_overall_in_stewart_plotkin}
\end{figure}

Here, the work of~\cite{Stewart2012} is extended by investigating a tournament
with \input{assets/tex/number_of_full_strategies/main.tex}
strategies.

The results of this analysis are shown in
Figure~\ref{fig:SSError_and_probabilities_in_full}. The top ranking strategies
by number of wins seem to be extortionate (but not against all strategies) and
it can be seen that a small sub group of strategies achieve mutual defection.
All the top ranking strategies according to score achieve mutual cooperation and
do not extort each other, however they
\textbf{do} exhibit extortionate behaviour towards a number of the lower ranking
strategies.

\begin{figure}[!htbp]
    \centering
    \includegraphics[width=.8\textwidth]{./assets/img/SSError_and_probabilities_in_full/main.pdf}
    \caption{\(\text{SSError}\) for the strategies for the full tournament. Only
    strategy interactions for which \(p_4=0\) and \(\chi>1\) are displayed.}
    \label{fig:SSError_and_probabilities_in_full}
\end{figure}

\section{Conclusion}\label{sec:conclusion}

This work defines an approach to measure whether or not a player is playing a
strategy that corresponds to an extortionate strategy as defined
in~\cite{Press2012}: a mathematical model for suspicion. Indeed, all
extortionate strategies have been
 classified as lying on a triangular plane.
This rigorous classification fails to be robust to small measurement error, thus
a statistical approach is proposed.
This is done through a linear algebraic approach for approximating the solution
of a linear system. Using this, a large number of pairwise interactions is
simulated and in fact very few strategies are found to act extortionately.

The work of~\cite{Press2012}, whilst showing that a clever approach to taking
advantage of another memory one strategy exists: this is incomplete. Whilst the
elegance of this result is very attractive, just as the simplicity of the
victory of Tit For Tat in Axelrod's original tournaments was, it is incomplete.
Extortionate strategies achieve a high number of wins but they do not
achieve a high score which corresponds to the fitness landscape in an
evolutionary sense. From the large number of interactions a payoff matrix \(S\)
can be measured where \(S_{ij}\) denotes the score (using standard values of
\((R, S, T, P) = (3, 0, 5, 1)\)) of the \(i\)th strategy
against the \(j\)th strategy. Using this, the replicator equation
describes the evolution of the system based on a population density fitness
function:

\begin{equation}\label{eqn:replicator_dynamics}
    \frac{dx}{dt} = x(S-x^TS x)
\end{equation}

Equation (\ref{eqn:replicator_dynamics}) is solved numerically through an
integration technique described in~\cite{Petzold1983} and
Figure~\ref{fig:replicator_dynamics} shows the evolution of the distribution of
the system: the various strategies are ranked by scores. It is clear to see that
only the high ranking strategies survive the evolutionary process (in fact,
only \input{./assets/img/replicator_dynamics/main.tex}
have a final distribution greater than \(10 ^ {-2}\)). This confirms the
findings of~\cite{Moran1707} in which sophisticated strategies resist
evolutionary invasion of shorter memory strategies. Recalling
Figure~\ref{fig:SSError_and_probabilities_in_full} this demonstrates that:

\begin{itemize}
    \item Cooperation emerges through the evolutionary process: the high scoring
        strategies do not exhibit extortionate behaviour towards each other.
    \item Extortionate strategies do not survive the evolutionary process.
\end{itemize}

\begin{figure}[!htbp]
    \centering
    \includegraphics[width=.8\textwidth]{./assets/img/replicator_dynamics/main.pdf}
    \caption{Numerical simulation of the replicator equation
    (\ref{eqn:replicator_dynamics}): strategies are ordered by score, only the strategies with a high score survive the evolutionary process.}
    \label{fig:replicator_dynamics}
\end{figure}

This work can be used to classify plays of the IPD\@: data can be collected from
actual interactions (in lab or in the field). Furthermore, this allows for a
classification method similar to the notion of fingerprinting presented
in~\cite{Ashlock2008}. Trained strategies can potentially be classified as
extortionate or not or it could be possible to even constrain the reinforcement
learning approaches that are becoming prevalent in the literature.
Alternatively, this mathematical approach for recognising extortion could be
used in sophisticated strategies to defend against invasion. Arguably, some of
the strategies considered here exhibit this behaviour, indeed as described
in~\cite{Harper2017}, the top ranking strategies in the full tournament are
obtained using evolutionary reinforcement learning techniques, thus, suspicion
of extortionate behaviour could in fact be an evolutionary trait.

\section*{Acknowledgements}

The following open source software libraries were used in this research:

\begin{itemize}
    \item The Axelrod ~\cite{Knight2016, Knight2018} library (IPD strategies and
        tournaments).
    \item The sympy library~\cite{Meurer2017} (verification of all symbolic
        calculations).
    \item The matplotlib~\cite{Droettboom2018} library (visualisation).
    \item The pandas~\cite{Structures2010}, dask~\cite{Dask2016} and
        NumPy~\cite{Oliphant2015} libraries (data manipulation).
    \item The SciPy~\cite{Jones2001} library (numerical integration of the
        replicator equation).
\end{itemize}

This work was performed using the computational facilities of the Advanced
Research Computing @ Cardiff (ARCCA) Division, Cardiff University.

\printbibliography

\newpage
\section*{Supplementary materials}

\includepdf{assets/pdf/proof_of_form_of_extortionate_strategies/main.pdf}

\newpage

Using the pair wise interactions the transition rates \(p,
q\) can be measured and the steady state probabilities inferred and compared to
the actual probabilities of each state.
This is done numerically by computing the singular eigenvector of the
matrix \(A\) \cite{Stewart2009}:

\[
    A =
    \begin{bmatrix}
        p_1 q_1 & p_1 (1 - q_1) & (1 - p_1) q_1 & (1 -p_1) (1 - q_1) \\
        p_2 q_2 & p_2 (1 - q_2) & (1 - p_2) q_2 & (1 -p_2) (1 - q_2) \\
        p_3 q_3 & p_3 (1 - q_3) & (1 - p_3) q_3 & (1 -p_3) (1 - q_3) \\
        p_4 q_4 & p_4 (1 - q_4) & (1 - p_4) q_4 & (1 -p_4) (1 - q_4) \\
    \end{bmatrix}
\]

Figure~\ref{fig:computed_probabilities_vs_theoretic_probabilities} shows a
regression line fitted to every pairwise interaction with a reported
\(\text{SSError}\) value (pairwise interactions with missing states were
omitted). This serves to validate the approach: a part from some edge cases the
relationship is consistent.

\begin{figure}[!htbp]
    \centering
    \includegraphics[width=.8\textwidth]{./assets/img/computed_probabilities_vs_theoretic_probabilities/main.pdf}
    \caption{The
        relationship between the steady state probabilities inferred from the
        measured transitions and the actual steady state probabilities. A linear
        regression line is included validating the approach.}
    \label{fig:computed_probabilities_vs_theoretic_probabilities}
\end{figure}


\end{document}
 strategies,
was presented with specific consideration given to ZD strategies. This
tournament is reproduced here using the Axelrod-Python
project~\cite{Knight2016}. To obtain a good measure of the corresponding
transition rates for each strategy all matches have been run for
\documentclass[a4paper]{article}

\usepackage{amsmath}
\usepackage{amssymb}
\usepackage[margin=1.5cm,
            includefoot,
            footskip=30pt]{geometry}
\usepackage{layout}
\usepackage{graphicx}
\usepackage{subcaption}

\usepackage{biblatex}
\usepackage{pdfpages}

\bibliography{main.bib}

\title{Suspicion: Recognising and evaluating the effectiveness
       of extortion in the Iterated Prisoner's Dilemma}
\author{Vincent A. Knight \and Nikoleta E. Glynatsi}
\date{\today}



\begin{document}

\maketitle

\begin{abstract}
    The Iterated Prisoner's Dilemma is a model for rational and evolutionary
    interactive behaviour. It has applications both in the study of human social
    behaviour as well as in biology.
    It is used to understand when and how a rational individual might
    accept an immediate cost to their own utility for the direct benefit of
    another.

    Much attention has been given to a class of strategies called
    Zero Determinant strategies. It has been theoretically shown that these
    strategies can ``extort'' any player.

    In this work, an approach to identify if observed strategies are playing in
    an extortionate way is described. Furthermore, experimental analysis of
    a large tournament with \input{assets/tex/number_of_full_strategies/main.tex}
    strategies is considered. In this setting
    the most highly performing strategies do not play in an extortionate way
    against each other but do against lower performing strategies.
    This suggests that whilst the theory of Zero Determinant strategies
    indicates that memory is not of fundamental importance to the evolution of
    cooperative behaviour, this is incomplete.
\end{abstract}

\section{Introduction}\label{sec:introduction}

Agent based game theoretic models have become a stalwart of the underpinning
mathematics of interactive behaviours. One of the major pieces of work
in this area is the pair of original computer tournaments run by Robert
Axelrod~\cite{Axelrod1980, Axelrod1980a}. These tournaments pitted submitted
computer strategies against each other in plays of the Iterated Prisoner's
Dilemma. A common game where agents can choose to pay a slight cost to their
immediate utility in the hope of building a reputation. This has been used in
economic and evolutionary game theory to understand the evolution of cooperative
behaviour.

Recently, a class of strategies was described in~\cite{Press2012} that can
provably extort any given opponent. In~\cite{Hilbe2013, Moran1707} some
questions have already been asked about the true effectiveness of these
strategies in an evolutionary setting. Here another question is asked: is it
possible to recognise this extortionate behaviour? A mathematical procedure for
suspicion is presented: in the same way that the continued actions of an
extortionate individual might raise suspicion.

This work makes use of the Axelrod Python library~\cite{Knight2018, Knight2016}
with a large number of Prisoner Dilemma strategies available to give an
extensive numerical example of the ideas presented.  The approach is presented
in Section~\ref{sec:delta-zd-strategies}.  All of the code and data discussed
in Section~\ref{sec:numerical-experiments} is open sourced, archived and
written according to best scientific principles~\cite{Wilson2014}. The data
archive can be found at~\cite{vincent_knight_2018_1297075}.

\section{Recognising Extortion}\label{sec:delta-zd-strategies}

In~\cite{Press2012}, given a match between 2 memory-one strategies, the concept
of Zero Determinant (ZD) strategies is introduced. The main result of that paper
shows that given two memory one players \(p, q\in\mathbb{R}^4\) a linear
relationship between the players' scores could be forced by one of the players.

Using the notation of~\cite{Press2012}, assuming the utilities for player \(p\)
are given by \(S_x=(R, S, T, P)\) and for player \(q\) by \(S_y=(R, T, S, P)\)
and that the stationary scores of each player is given by \(S_X\) and \(S_Y\)
respectively. The main result of~\cite{Press2012} is that if

\begin{equation}\label{eqn:linear_relationship_for_p}
    \tilde p=\alpha S_x + \beta S_y + \gamma
\end{equation}

or

\begin{equation}\label{eqn:linear_relationship_for_q}
    \tilde q=\alpha S_x + \beta S_y + \gamma
\end{equation}

where \(\tilde p = (1 - p_1, 1 - p_2, p_3, p_4)\) and
\(\tilde q = (1 - q_1, 1 - q_2, q_3, q_4)\) then:

\begin{equation}
    \alpha S_X + \beta S_Y + \gamma = 0
\end{equation}

In~\cite{Press2012} a particular type of ZD strategy is defined: extortionate
strategies. If:

\begin{equation}\label{eqn:constraint_for_extortion}
    \gamma = - P(\alpha + \beta)
\end{equation}

then the player can ensure they get a score \(\chi\) times
larger than the opponent. This extortion coefficient is given by:

\begin{equation}\label{eqn:definition_of_chi}
    \chi=\frac{-\beta}{\alpha}
\end{equation}

Thus, if (\ref{eqn:constraint_for_extortion}) holds and \(\chi >1\) a player is
said to extort their opponent.
Here, the reverse problem is considered: given a
\(p\in\mathbb{R}^4\) how does one identify \(\alpha, \beta\) if they
exist and is the strategy in fact acting in an extortionate way?

These conditions correspond to:

\begin{align}
    \tilde p_1 & = \alpha R + \beta R - P (\alpha + \beta)
            \label{eqn:condition_for_tilde_p1}\\
    \tilde p_2 & = \alpha S + \beta T - P (\alpha + \beta)
            \label{eqn:condition_for_tilde_p2}\\
    \tilde p_3 & = \alpha T + \beta S - P (\alpha + \beta)
            \label{eqn:condition_for_tilde_p3}\\
    \tilde p_4 & = \alpha P + \beta P - P (\alpha + \beta)
            \label{eqn:condition_for_tilde_p4}
\end{align}

Equation (\ref{eqn:condition_for_tilde_p4}) ensures that \(p_4=\tilde p_4=0\).
Equations (\ref{eqn:condition_for_tilde_p1}-\ref{eqn:condition_for_tilde_p3})
can be used to eliminate \(\alpha, \beta\), giving:

\begin{equation}\label{eqn:planar_definition_of_extortion}
    \tilde p_1 = \frac{(R - P)(\tilde p_2 + \tilde p_3)}{S + T - 2P}
\end{equation}

with:

\begin{equation}\label{eqn:definition_of_chi}
    \chi = \frac{\tilde p_2 (P - T) + \tilde p_3 (S - P)}
                {\tilde p_2 (P - S) + \tilde p_3 (T - P)}
\end{equation}

Given a strategy \(p\in\mathbb{R}^{4\times 1}\) equations
(\ref{eqn:condition_for_tilde_p4}), (\ref{eqn:planar_definition_of_extortion}-\ref{eqn:definition_of_chi}) can be used to check if
a strategy is extortionate. The conditions correspond to:

\begin{align}
    p_1 & = \frac{(R-P)(p_2 + p_3) - R + T + S - P}{S + T - 2P}
     \label{eqn:condition_for_p1}\\
    p_4 & = 0 \label{eqn:condition_for_p4}\\
    1 & > p_2 + p_3\label{eqn:condition_for_chi}
\end{align}

The algebraic steps necessary to prove these results are available in the
supporting materials.

All extortionate strategies reside on a triangular (\ref{eqn:condition_for_chi})
plane (\ref{eqn:condition_for_p1}) in 3 dimensions (\ref{eqn:condition_for_p4}).
Using this formulation it can be seen that a necessary (but not sufficient)
condition for an extortionate strategy is that it cooperates on average less
than 50\% of the time when in a state of disagreement with the opponent.

As an example, consider the known extortionate strategy \(p=(8 / 9, 1 / 2, 1 /
3, 0)\) from~\cite{Stewart2012} which is referred to as \texttt{Extort-2}. In
this case, for the standard values of \((R, T, S, P)\) constraint
(\ref{eqn:condition_for_p1}) corresponds to:

\begin{equation}
    p_1 = \frac{2(p_2 + p_3) + 1}{3}
\end{equation}

It is clear that in this case all constraints hold.

This approach could in fact be used to confirm that a given strategy is acting
in an extortionate manner even if it is not a memory one strategy. However, in
practice, if a closed form for \(p\) is not known, then due to measurement
and/or numerical error this would not work.

This problem can be written in the following linear algebraic form where
\(x=(\alpha, \beta)\)
and \(p^*=(\tilde p_1 - 1, tilde_2 - 1, p_3)\):

\begin{equation}\label{eqn:linear_algebraic_equation_for_p}
    Cx= p^*
\end{equation}

\(C\) corresponds to equations
(\ref{eqn:condition_for_tilde_p1}-\ref{eqn:condition_for_tilde_p3}) and is
given by:

\begin{equation}\label{eqn:definition_of_C}
    C =
    \begin{bmatrix}
        R - P & R- P \\
        S - P & T- P \\
        T - P & S- P \\
    \end{bmatrix}
\end{equation}

Note that in general, equation (\ref{eqn:linear_algebraic_equation_for_p}) will
not necessarily have a solution. From the Rouch\'{e}-Capelli theorem if there is
a solution it is unique as \(\text{rank}(C)=2\) which is the dimension of the
variable \(x\). The best fitting \(x\) is found by minimizing:

\begin{equation}\label{eqn:r_squared}
    \text{SSError} = \|C x- p^*\|_2^2 = \sum_{i=1}^{3}\left((C\bar x)_i-p_i^*\right)^2
\end{equation}

Note that \(\text{SSError}\), which is the square of the Frobenius
norm~\cite{Golub2013}, becomes a measure of how close a strategy is to being an
extortionate strategy. Suspicion
of extortion then corresponds to a threshold on \(\text{SSError}\).

By observing interactions (human or otherwise), their memory one representation
can be inferred and this approach can be used to recognise extortionate
behaviour. The notion of comparing theoretic and actual plays of the IPD is not
novel, see for example~\cite{Rand2013}. Immediately it is noted that if the
environment is noisy~\cite{Wu1995} then no strategy can be considered to be
extortionate as \(p_4>0\).

In the next section, this idea will be illustrated by observing the interactions
that take place in a computer based tournament of the IPD\@.

\section{Numerical experiments}\label{sec:numerical-experiments}

In~\cite{Stewart2012} results from a tournament with
\input{./assets/tex/number_of_stewart_plotkin_strategies/main.tex} strategies,
was presented with specific consideration given to ZD strategies. This
tournament is reproduced here using the Axelrod-Python
project~\cite{Knight2016}. To obtain a good measure of the corresponding
transition rates for each strategy all matches have been run for
\input{assets/tex/number_of_turns/main.tex} turns and every match has been
repeated \input{assets/tex/number_of_repetitions/main.tex} times. All of this
interaction data is available at~\cite{vincent_knight_2018_1297075}. A good
match between the inferred Markov chain and the state distribution of the actual
interactions has been verified. Data for this is presented in the supplementary
materials.

Figure~\ref{fig:SSError_overall_in_stewart_plotkin} shows the \(\text{SSError}\)
values for all the strategies in the tournament, as reported
in~\cite{Stewart2012} the extortionate strategy (which has an expected
\(\text{SSError}\) approximately 0) gains a large number of wins.

\begin{figure}[!htbp]
    \centering
    \includegraphics[width=.8\textwidth]{./assets/img/SSError_overall_in_stewart_plotkin/main.pdf}
    \caption{\(\text{SSError}\) and state probabilities for the strategies
        of~\cite{Stewart2012}, ordered both by number of wins and overall score.
        Note that \(P(DC)\) is not shown as it corresponds to the transpose of
        \(P(CD)\). Cooperator and Defector are omitted as they do not visit all
        the states.}
    \label{fig:SSError_overall_in_stewart_plotkin}
\end{figure}

Here, the work of~\cite{Stewart2012} is extended by investigating a tournament
with \input{assets/tex/number_of_full_strategies/main.tex}
strategies.

The results of this analysis are shown in
Figure~\ref{fig:SSError_and_probabilities_in_full}. The top ranking strategies
by number of wins seem to be extortionate (but not against all strategies) and
it can be seen that a small sub group of strategies achieve mutual defection.
All the top ranking strategies according to score achieve mutual cooperation and
do not extort each other, however they
\textbf{do} exhibit extortionate behaviour towards a number of the lower ranking
strategies.

\begin{figure}[!htbp]
    \centering
    \includegraphics[width=.8\textwidth]{./assets/img/SSError_and_probabilities_in_full/main.pdf}
    \caption{\(\text{SSError}\) for the strategies for the full tournament. Only
    strategy interactions for which \(p_4=0\) and \(\chi>1\) are displayed.}
    \label{fig:SSError_and_probabilities_in_full}
\end{figure}

\section{Conclusion}\label{sec:conclusion}

This work defines an approach to measure whether or not a player is playing a
strategy that corresponds to an extortionate strategy as defined
in~\cite{Press2012}: a mathematical model for suspicion. Indeed, all
extortionate strategies have been
 classified as lying on a triangular plane.
This rigorous classification fails to be robust to small measurement error, thus
a statistical approach is proposed.
This is done through a linear algebraic approach for approximating the solution
of a linear system. Using this, a large number of pairwise interactions is
simulated and in fact very few strategies are found to act extortionately.

The work of~\cite{Press2012}, whilst showing that a clever approach to taking
advantage of another memory one strategy exists: this is incomplete. Whilst the
elegance of this result is very attractive, just as the simplicity of the
victory of Tit For Tat in Axelrod's original tournaments was, it is incomplete.
Extortionate strategies achieve a high number of wins but they do not
achieve a high score which corresponds to the fitness landscape in an
evolutionary sense. From the large number of interactions a payoff matrix \(S\)
can be measured where \(S_{ij}\) denotes the score (using standard values of
\((R, S, T, P) = (3, 0, 5, 1)\)) of the \(i\)th strategy
against the \(j\)th strategy. Using this, the replicator equation
describes the evolution of the system based on a population density fitness
function:

\begin{equation}\label{eqn:replicator_dynamics}
    \frac{dx}{dt} = x(S-x^TS x)
\end{equation}

Equation (\ref{eqn:replicator_dynamics}) is solved numerically through an
integration technique described in~\cite{Petzold1983} and
Figure~\ref{fig:replicator_dynamics} shows the evolution of the distribution of
the system: the various strategies are ranked by scores. It is clear to see that
only the high ranking strategies survive the evolutionary process (in fact,
only \input{./assets/img/replicator_dynamics/main.tex}
have a final distribution greater than \(10 ^ {-2}\)). This confirms the
findings of~\cite{Moran1707} in which sophisticated strategies resist
evolutionary invasion of shorter memory strategies. Recalling
Figure~\ref{fig:SSError_and_probabilities_in_full} this demonstrates that:

\begin{itemize}
    \item Cooperation emerges through the evolutionary process: the high scoring
        strategies do not exhibit extortionate behaviour towards each other.
    \item Extortionate strategies do not survive the evolutionary process.
\end{itemize}

\begin{figure}[!htbp]
    \centering
    \includegraphics[width=.8\textwidth]{./assets/img/replicator_dynamics/main.pdf}
    \caption{Numerical simulation of the replicator equation
    (\ref{eqn:replicator_dynamics}): strategies are ordered by score, only the strategies with a high score survive the evolutionary process.}
    \label{fig:replicator_dynamics}
\end{figure}

This work can be used to classify plays of the IPD\@: data can be collected from
actual interactions (in lab or in the field). Furthermore, this allows for a
classification method similar to the notion of fingerprinting presented
in~\cite{Ashlock2008}. Trained strategies can potentially be classified as
extortionate or not or it could be possible to even constrain the reinforcement
learning approaches that are becoming prevalent in the literature.
Alternatively, this mathematical approach for recognising extortion could be
used in sophisticated strategies to defend against invasion. Arguably, some of
the strategies considered here exhibit this behaviour, indeed as described
in~\cite{Harper2017}, the top ranking strategies in the full tournament are
obtained using evolutionary reinforcement learning techniques, thus, suspicion
of extortionate behaviour could in fact be an evolutionary trait.

\section*{Acknowledgements}

The following open source software libraries were used in this research:

\begin{itemize}
    \item The Axelrod ~\cite{Knight2016, Knight2018} library (IPD strategies and
        tournaments).
    \item The sympy library~\cite{Meurer2017} (verification of all symbolic
        calculations).
    \item The matplotlib~\cite{Droettboom2018} library (visualisation).
    \item The pandas~\cite{Structures2010}, dask~\cite{Dask2016} and
        NumPy~\cite{Oliphant2015} libraries (data manipulation).
    \item The SciPy~\cite{Jones2001} library (numerical integration of the
        replicator equation).
\end{itemize}

This work was performed using the computational facilities of the Advanced
Research Computing @ Cardiff (ARCCA) Division, Cardiff University.

\printbibliography

\newpage
\section*{Supplementary materials}

\includepdf{assets/pdf/proof_of_form_of_extortionate_strategies/main.pdf}

\newpage

Using the pair wise interactions the transition rates \(p,
q\) can be measured and the steady state probabilities inferred and compared to
the actual probabilities of each state.
This is done numerically by computing the singular eigenvector of the
matrix \(A\) \cite{Stewart2009}:

\[
    A =
    \begin{bmatrix}
        p_1 q_1 & p_1 (1 - q_1) & (1 - p_1) q_1 & (1 -p_1) (1 - q_1) \\
        p_2 q_2 & p_2 (1 - q_2) & (1 - p_2) q_2 & (1 -p_2) (1 - q_2) \\
        p_3 q_3 & p_3 (1 - q_3) & (1 - p_3) q_3 & (1 -p_3) (1 - q_3) \\
        p_4 q_4 & p_4 (1 - q_4) & (1 - p_4) q_4 & (1 -p_4) (1 - q_4) \\
    \end{bmatrix}
\]

Figure~\ref{fig:computed_probabilities_vs_theoretic_probabilities} shows a
regression line fitted to every pairwise interaction with a reported
\(\text{SSError}\) value (pairwise interactions with missing states were
omitted). This serves to validate the approach: a part from some edge cases the
relationship is consistent.

\begin{figure}[!htbp]
    \centering
    \includegraphics[width=.8\textwidth]{./assets/img/computed_probabilities_vs_theoretic_probabilities/main.pdf}
    \caption{The
        relationship between the steady state probabilities inferred from the
        measured transitions and the actual steady state probabilities. A linear
        regression line is included validating the approach.}
    \label{fig:computed_probabilities_vs_theoretic_probabilities}
\end{figure}


\end{document}
 turns and every match has been
repeated \documentclass[a4paper]{article}

\usepackage{amsmath}
\usepackage{amssymb}
\usepackage[margin=1.5cm,
            includefoot,
            footskip=30pt]{geometry}
\usepackage{layout}
\usepackage{graphicx}
\usepackage{subcaption}

\usepackage{biblatex}
\usepackage{pdfpages}

\bibliography{main.bib}

\title{Suspicion: Recognising and evaluating the effectiveness
       of extortion in the Iterated Prisoner's Dilemma}
\author{Vincent A. Knight \and Nikoleta E. Glynatsi}
\date{\today}



\begin{document}

\maketitle

\begin{abstract}
    The Iterated Prisoner's Dilemma is a model for rational and evolutionary
    interactive behaviour. It has applications both in the study of human social
    behaviour as well as in biology.
    It is used to understand when and how a rational individual might
    accept an immediate cost to their own utility for the direct benefit of
    another.

    Much attention has been given to a class of strategies called
    Zero Determinant strategies. It has been theoretically shown that these
    strategies can ``extort'' any player.

    In this work, an approach to identify if observed strategies are playing in
    an extortionate way is described. Furthermore, experimental analysis of
    a large tournament with \input{assets/tex/number_of_full_strategies/main.tex}
    strategies is considered. In this setting
    the most highly performing strategies do not play in an extortionate way
    against each other but do against lower performing strategies.
    This suggests that whilst the theory of Zero Determinant strategies
    indicates that memory is not of fundamental importance to the evolution of
    cooperative behaviour, this is incomplete.
\end{abstract}

\section{Introduction}\label{sec:introduction}

Agent based game theoretic models have become a stalwart of the underpinning
mathematics of interactive behaviours. One of the major pieces of work
in this area is the pair of original computer tournaments run by Robert
Axelrod~\cite{Axelrod1980, Axelrod1980a}. These tournaments pitted submitted
computer strategies against each other in plays of the Iterated Prisoner's
Dilemma. A common game where agents can choose to pay a slight cost to their
immediate utility in the hope of building a reputation. This has been used in
economic and evolutionary game theory to understand the evolution of cooperative
behaviour.

Recently, a class of strategies was described in~\cite{Press2012} that can
provably extort any given opponent. In~\cite{Hilbe2013, Moran1707} some
questions have already been asked about the true effectiveness of these
strategies in an evolutionary setting. Here another question is asked: is it
possible to recognise this extortionate behaviour? A mathematical procedure for
suspicion is presented: in the same way that the continued actions of an
extortionate individual might raise suspicion.

This work makes use of the Axelrod Python library~\cite{Knight2018, Knight2016}
with a large number of Prisoner Dilemma strategies available to give an
extensive numerical example of the ideas presented.  The approach is presented
in Section~\ref{sec:delta-zd-strategies}.  All of the code and data discussed
in Section~\ref{sec:numerical-experiments} is open sourced, archived and
written according to best scientific principles~\cite{Wilson2014}. The data
archive can be found at~\cite{vincent_knight_2018_1297075}.

\section{Recognising Extortion}\label{sec:delta-zd-strategies}

In~\cite{Press2012}, given a match between 2 memory-one strategies, the concept
of Zero Determinant (ZD) strategies is introduced. The main result of that paper
shows that given two memory one players \(p, q\in\mathbb{R}^4\) a linear
relationship between the players' scores could be forced by one of the players.

Using the notation of~\cite{Press2012}, assuming the utilities for player \(p\)
are given by \(S_x=(R, S, T, P)\) and for player \(q\) by \(S_y=(R, T, S, P)\)
and that the stationary scores of each player is given by \(S_X\) and \(S_Y\)
respectively. The main result of~\cite{Press2012} is that if

\begin{equation}\label{eqn:linear_relationship_for_p}
    \tilde p=\alpha S_x + \beta S_y + \gamma
\end{equation}

or

\begin{equation}\label{eqn:linear_relationship_for_q}
    \tilde q=\alpha S_x + \beta S_y + \gamma
\end{equation}

where \(\tilde p = (1 - p_1, 1 - p_2, p_3, p_4)\) and
\(\tilde q = (1 - q_1, 1 - q_2, q_3, q_4)\) then:

\begin{equation}
    \alpha S_X + \beta S_Y + \gamma = 0
\end{equation}

In~\cite{Press2012} a particular type of ZD strategy is defined: extortionate
strategies. If:

\begin{equation}\label{eqn:constraint_for_extortion}
    \gamma = - P(\alpha + \beta)
\end{equation}

then the player can ensure they get a score \(\chi\) times
larger than the opponent. This extortion coefficient is given by:

\begin{equation}\label{eqn:definition_of_chi}
    \chi=\frac{-\beta}{\alpha}
\end{equation}

Thus, if (\ref{eqn:constraint_for_extortion}) holds and \(\chi >1\) a player is
said to extort their opponent.
Here, the reverse problem is considered: given a
\(p\in\mathbb{R}^4\) how does one identify \(\alpha, \beta\) if they
exist and is the strategy in fact acting in an extortionate way?

These conditions correspond to:

\begin{align}
    \tilde p_1 & = \alpha R + \beta R - P (\alpha + \beta)
            \label{eqn:condition_for_tilde_p1}\\
    \tilde p_2 & = \alpha S + \beta T - P (\alpha + \beta)
            \label{eqn:condition_for_tilde_p2}\\
    \tilde p_3 & = \alpha T + \beta S - P (\alpha + \beta)
            \label{eqn:condition_for_tilde_p3}\\
    \tilde p_4 & = \alpha P + \beta P - P (\alpha + \beta)
            \label{eqn:condition_for_tilde_p4}
\end{align}

Equation (\ref{eqn:condition_for_tilde_p4}) ensures that \(p_4=\tilde p_4=0\).
Equations (\ref{eqn:condition_for_tilde_p1}-\ref{eqn:condition_for_tilde_p3})
can be used to eliminate \(\alpha, \beta\), giving:

\begin{equation}\label{eqn:planar_definition_of_extortion}
    \tilde p_1 = \frac{(R - P)(\tilde p_2 + \tilde p_3)}{S + T - 2P}
\end{equation}

with:

\begin{equation}\label{eqn:definition_of_chi}
    \chi = \frac{\tilde p_2 (P - T) + \tilde p_3 (S - P)}
                {\tilde p_2 (P - S) + \tilde p_3 (T - P)}
\end{equation}

Given a strategy \(p\in\mathbb{R}^{4\times 1}\) equations
(\ref{eqn:condition_for_tilde_p4}), (\ref{eqn:planar_definition_of_extortion}-\ref{eqn:definition_of_chi}) can be used to check if
a strategy is extortionate. The conditions correspond to:

\begin{align}
    p_1 & = \frac{(R-P)(p_2 + p_3) - R + T + S - P}{S + T - 2P}
     \label{eqn:condition_for_p1}\\
    p_4 & = 0 \label{eqn:condition_for_p4}\\
    1 & > p_2 + p_3\label{eqn:condition_for_chi}
\end{align}

The algebraic steps necessary to prove these results are available in the
supporting materials.

All extortionate strategies reside on a triangular (\ref{eqn:condition_for_chi})
plane (\ref{eqn:condition_for_p1}) in 3 dimensions (\ref{eqn:condition_for_p4}).
Using this formulation it can be seen that a necessary (but not sufficient)
condition for an extortionate strategy is that it cooperates on average less
than 50\% of the time when in a state of disagreement with the opponent.

As an example, consider the known extortionate strategy \(p=(8 / 9, 1 / 2, 1 /
3, 0)\) from~\cite{Stewart2012} which is referred to as \texttt{Extort-2}. In
this case, for the standard values of \((R, T, S, P)\) constraint
(\ref{eqn:condition_for_p1}) corresponds to:

\begin{equation}
    p_1 = \frac{2(p_2 + p_3) + 1}{3}
\end{equation}

It is clear that in this case all constraints hold.

This approach could in fact be used to confirm that a given strategy is acting
in an extortionate manner even if it is not a memory one strategy. However, in
practice, if a closed form for \(p\) is not known, then due to measurement
and/or numerical error this would not work.

This problem can be written in the following linear algebraic form where
\(x=(\alpha, \beta)\)
and \(p^*=(\tilde p_1 - 1, tilde_2 - 1, p_3)\):

\begin{equation}\label{eqn:linear_algebraic_equation_for_p}
    Cx= p^*
\end{equation}

\(C\) corresponds to equations
(\ref{eqn:condition_for_tilde_p1}-\ref{eqn:condition_for_tilde_p3}) and is
given by:

\begin{equation}\label{eqn:definition_of_C}
    C =
    \begin{bmatrix}
        R - P & R- P \\
        S - P & T- P \\
        T - P & S- P \\
    \end{bmatrix}
\end{equation}

Note that in general, equation (\ref{eqn:linear_algebraic_equation_for_p}) will
not necessarily have a solution. From the Rouch\'{e}-Capelli theorem if there is
a solution it is unique as \(\text{rank}(C)=2\) which is the dimension of the
variable \(x\). The best fitting \(x\) is found by minimizing:

\begin{equation}\label{eqn:r_squared}
    \text{SSError} = \|C x- p^*\|_2^2 = \sum_{i=1}^{3}\left((C\bar x)_i-p_i^*\right)^2
\end{equation}

Note that \(\text{SSError}\), which is the square of the Frobenius
norm~\cite{Golub2013}, becomes a measure of how close a strategy is to being an
extortionate strategy. Suspicion
of extortion then corresponds to a threshold on \(\text{SSError}\).

By observing interactions (human or otherwise), their memory one representation
can be inferred and this approach can be used to recognise extortionate
behaviour. The notion of comparing theoretic and actual plays of the IPD is not
novel, see for example~\cite{Rand2013}. Immediately it is noted that if the
environment is noisy~\cite{Wu1995} then no strategy can be considered to be
extortionate as \(p_4>0\).

In the next section, this idea will be illustrated by observing the interactions
that take place in a computer based tournament of the IPD\@.

\section{Numerical experiments}\label{sec:numerical-experiments}

In~\cite{Stewart2012} results from a tournament with
\input{./assets/tex/number_of_stewart_plotkin_strategies/main.tex} strategies,
was presented with specific consideration given to ZD strategies. This
tournament is reproduced here using the Axelrod-Python
project~\cite{Knight2016}. To obtain a good measure of the corresponding
transition rates for each strategy all matches have been run for
\input{assets/tex/number_of_turns/main.tex} turns and every match has been
repeated \input{assets/tex/number_of_repetitions/main.tex} times. All of this
interaction data is available at~\cite{vincent_knight_2018_1297075}. A good
match between the inferred Markov chain and the state distribution of the actual
interactions has been verified. Data for this is presented in the supplementary
materials.

Figure~\ref{fig:SSError_overall_in_stewart_plotkin} shows the \(\text{SSError}\)
values for all the strategies in the tournament, as reported
in~\cite{Stewart2012} the extortionate strategy (which has an expected
\(\text{SSError}\) approximately 0) gains a large number of wins.

\begin{figure}[!htbp]
    \centering
    \includegraphics[width=.8\textwidth]{./assets/img/SSError_overall_in_stewart_plotkin/main.pdf}
    \caption{\(\text{SSError}\) and state probabilities for the strategies
        of~\cite{Stewart2012}, ordered both by number of wins and overall score.
        Note that \(P(DC)\) is not shown as it corresponds to the transpose of
        \(P(CD)\). Cooperator and Defector are omitted as they do not visit all
        the states.}
    \label{fig:SSError_overall_in_stewart_plotkin}
\end{figure}

Here, the work of~\cite{Stewart2012} is extended by investigating a tournament
with \input{assets/tex/number_of_full_strategies/main.tex}
strategies.

The results of this analysis are shown in
Figure~\ref{fig:SSError_and_probabilities_in_full}. The top ranking strategies
by number of wins seem to be extortionate (but not against all strategies) and
it can be seen that a small sub group of strategies achieve mutual defection.
All the top ranking strategies according to score achieve mutual cooperation and
do not extort each other, however they
\textbf{do} exhibit extortionate behaviour towards a number of the lower ranking
strategies.

\begin{figure}[!htbp]
    \centering
    \includegraphics[width=.8\textwidth]{./assets/img/SSError_and_probabilities_in_full/main.pdf}
    \caption{\(\text{SSError}\) for the strategies for the full tournament. Only
    strategy interactions for which \(p_4=0\) and \(\chi>1\) are displayed.}
    \label{fig:SSError_and_probabilities_in_full}
\end{figure}

\section{Conclusion}\label{sec:conclusion}

This work defines an approach to measure whether or not a player is playing a
strategy that corresponds to an extortionate strategy as defined
in~\cite{Press2012}: a mathematical model for suspicion. Indeed, all
extortionate strategies have been
 classified as lying on a triangular plane.
This rigorous classification fails to be robust to small measurement error, thus
a statistical approach is proposed.
This is done through a linear algebraic approach for approximating the solution
of a linear system. Using this, a large number of pairwise interactions is
simulated and in fact very few strategies are found to act extortionately.

The work of~\cite{Press2012}, whilst showing that a clever approach to taking
advantage of another memory one strategy exists: this is incomplete. Whilst the
elegance of this result is very attractive, just as the simplicity of the
victory of Tit For Tat in Axelrod's original tournaments was, it is incomplete.
Extortionate strategies achieve a high number of wins but they do not
achieve a high score which corresponds to the fitness landscape in an
evolutionary sense. From the large number of interactions a payoff matrix \(S\)
can be measured where \(S_{ij}\) denotes the score (using standard values of
\((R, S, T, P) = (3, 0, 5, 1)\)) of the \(i\)th strategy
against the \(j\)th strategy. Using this, the replicator equation
describes the evolution of the system based on a population density fitness
function:

\begin{equation}\label{eqn:replicator_dynamics}
    \frac{dx}{dt} = x(S-x^TS x)
\end{equation}

Equation (\ref{eqn:replicator_dynamics}) is solved numerically through an
integration technique described in~\cite{Petzold1983} and
Figure~\ref{fig:replicator_dynamics} shows the evolution of the distribution of
the system: the various strategies are ranked by scores. It is clear to see that
only the high ranking strategies survive the evolutionary process (in fact,
only \input{./assets/img/replicator_dynamics/main.tex}
have a final distribution greater than \(10 ^ {-2}\)). This confirms the
findings of~\cite{Moran1707} in which sophisticated strategies resist
evolutionary invasion of shorter memory strategies. Recalling
Figure~\ref{fig:SSError_and_probabilities_in_full} this demonstrates that:

\begin{itemize}
    \item Cooperation emerges through the evolutionary process: the high scoring
        strategies do not exhibit extortionate behaviour towards each other.
    \item Extortionate strategies do not survive the evolutionary process.
\end{itemize}

\begin{figure}[!htbp]
    \centering
    \includegraphics[width=.8\textwidth]{./assets/img/replicator_dynamics/main.pdf}
    \caption{Numerical simulation of the replicator equation
    (\ref{eqn:replicator_dynamics}): strategies are ordered by score, only the strategies with a high score survive the evolutionary process.}
    \label{fig:replicator_dynamics}
\end{figure}

This work can be used to classify plays of the IPD\@: data can be collected from
actual interactions (in lab or in the field). Furthermore, this allows for a
classification method similar to the notion of fingerprinting presented
in~\cite{Ashlock2008}. Trained strategies can potentially be classified as
extortionate or not or it could be possible to even constrain the reinforcement
learning approaches that are becoming prevalent in the literature.
Alternatively, this mathematical approach for recognising extortion could be
used in sophisticated strategies to defend against invasion. Arguably, some of
the strategies considered here exhibit this behaviour, indeed as described
in~\cite{Harper2017}, the top ranking strategies in the full tournament are
obtained using evolutionary reinforcement learning techniques, thus, suspicion
of extortionate behaviour could in fact be an evolutionary trait.

\section*{Acknowledgements}

The following open source software libraries were used in this research:

\begin{itemize}
    \item The Axelrod ~\cite{Knight2016, Knight2018} library (IPD strategies and
        tournaments).
    \item The sympy library~\cite{Meurer2017} (verification of all symbolic
        calculations).
    \item The matplotlib~\cite{Droettboom2018} library (visualisation).
    \item The pandas~\cite{Structures2010}, dask~\cite{Dask2016} and
        NumPy~\cite{Oliphant2015} libraries (data manipulation).
    \item The SciPy~\cite{Jones2001} library (numerical integration of the
        replicator equation).
\end{itemize}

This work was performed using the computational facilities of the Advanced
Research Computing @ Cardiff (ARCCA) Division, Cardiff University.

\printbibliography

\newpage
\section*{Supplementary materials}

\includepdf{assets/pdf/proof_of_form_of_extortionate_strategies/main.pdf}

\newpage

Using the pair wise interactions the transition rates \(p,
q\) can be measured and the steady state probabilities inferred and compared to
the actual probabilities of each state.
This is done numerically by computing the singular eigenvector of the
matrix \(A\) \cite{Stewart2009}:

\[
    A =
    \begin{bmatrix}
        p_1 q_1 & p_1 (1 - q_1) & (1 - p_1) q_1 & (1 -p_1) (1 - q_1) \\
        p_2 q_2 & p_2 (1 - q_2) & (1 - p_2) q_2 & (1 -p_2) (1 - q_2) \\
        p_3 q_3 & p_3 (1 - q_3) & (1 - p_3) q_3 & (1 -p_3) (1 - q_3) \\
        p_4 q_4 & p_4 (1 - q_4) & (1 - p_4) q_4 & (1 -p_4) (1 - q_4) \\
    \end{bmatrix}
\]

Figure~\ref{fig:computed_probabilities_vs_theoretic_probabilities} shows a
regression line fitted to every pairwise interaction with a reported
\(\text{SSError}\) value (pairwise interactions with missing states were
omitted). This serves to validate the approach: a part from some edge cases the
relationship is consistent.

\begin{figure}[!htbp]
    \centering
    \includegraphics[width=.8\textwidth]{./assets/img/computed_probabilities_vs_theoretic_probabilities/main.pdf}
    \caption{The
        relationship between the steady state probabilities inferred from the
        measured transitions and the actual steady state probabilities. A linear
        regression line is included validating the approach.}
    \label{fig:computed_probabilities_vs_theoretic_probabilities}
\end{figure}


\end{document}
 times. All of this
interaction data is available at~\cite{vincent_knight_2018_1297075}. A good
match between the inferred Markov chain and the state distribution of the actual
interactions has been verified. Data for this is presented in the supplementary
materials.

Figure~\ref{fig:SSError_overall_in_stewart_plotkin} shows the \(\text{SSError}\)
values for all the strategies in the tournament, as reported
in~\cite{Stewart2012} the extortionate strategy (which has an expected
\(\text{SSError}\) approximately 0) gains a large number of wins.

\begin{figure}[!htbp]
    \centering
    \includegraphics[width=.8\textwidth]{./assets/img/SSError_overall_in_stewart_plotkin/main.pdf}
    \caption{\(\text{SSError}\) and state probabilities for the strategies
        of~\cite{Stewart2012}, ordered both by number of wins and overall score.
        Note that \(P(DC)\) is not shown as it corresponds to the transpose of
        \(P(CD)\). Cooperator and Defector are omitted as they do not visit all
        the states.}
    \label{fig:SSError_overall_in_stewart_plotkin}
\end{figure}

Here, the work of~\cite{Stewart2012} is extended by investigating a tournament
with \documentclass[a4paper]{article}

\usepackage{amsmath}
\usepackage{amssymb}
\usepackage[margin=1.5cm,
            includefoot,
            footskip=30pt]{geometry}
\usepackage{layout}
\usepackage{graphicx}
\usepackage{subcaption}

\usepackage{biblatex}
\usepackage{pdfpages}

\bibliography{main.bib}

\title{Suspicion: Recognising and evaluating the effectiveness
       of extortion in the Iterated Prisoner's Dilemma}
\author{Vincent A. Knight \and Nikoleta E. Glynatsi}
\date{\today}



\begin{document}

\maketitle

\begin{abstract}
    The Iterated Prisoner's Dilemma is a model for rational and evolutionary
    interactive behaviour. It has applications both in the study of human social
    behaviour as well as in biology.
    It is used to understand when and how a rational individual might
    accept an immediate cost to their own utility for the direct benefit of
    another.

    Much attention has been given to a class of strategies called
    Zero Determinant strategies. It has been theoretically shown that these
    strategies can ``extort'' any player.

    In this work, an approach to identify if observed strategies are playing in
    an extortionate way is described. Furthermore, experimental analysis of
    a large tournament with \input{assets/tex/number_of_full_strategies/main.tex}
    strategies is considered. In this setting
    the most highly performing strategies do not play in an extortionate way
    against each other but do against lower performing strategies.
    This suggests that whilst the theory of Zero Determinant strategies
    indicates that memory is not of fundamental importance to the evolution of
    cooperative behaviour, this is incomplete.
\end{abstract}

\section{Introduction}\label{sec:introduction}

Agent based game theoretic models have become a stalwart of the underpinning
mathematics of interactive behaviours. One of the major pieces of work
in this area is the pair of original computer tournaments run by Robert
Axelrod~\cite{Axelrod1980, Axelrod1980a}. These tournaments pitted submitted
computer strategies against each other in plays of the Iterated Prisoner's
Dilemma. A common game where agents can choose to pay a slight cost to their
immediate utility in the hope of building a reputation. This has been used in
economic and evolutionary game theory to understand the evolution of cooperative
behaviour.

Recently, a class of strategies was described in~\cite{Press2012} that can
provably extort any given opponent. In~\cite{Hilbe2013, Moran1707} some
questions have already been asked about the true effectiveness of these
strategies in an evolutionary setting. Here another question is asked: is it
possible to recognise this extortionate behaviour? A mathematical procedure for
suspicion is presented: in the same way that the continued actions of an
extortionate individual might raise suspicion.

This work makes use of the Axelrod Python library~\cite{Knight2018, Knight2016}
with a large number of Prisoner Dilemma strategies available to give an
extensive numerical example of the ideas presented.  The approach is presented
in Section~\ref{sec:delta-zd-strategies}.  All of the code and data discussed
in Section~\ref{sec:numerical-experiments} is open sourced, archived and
written according to best scientific principles~\cite{Wilson2014}. The data
archive can be found at~\cite{vincent_knight_2018_1297075}.

\section{Recognising Extortion}\label{sec:delta-zd-strategies}

In~\cite{Press2012}, given a match between 2 memory-one strategies, the concept
of Zero Determinant (ZD) strategies is introduced. The main result of that paper
shows that given two memory one players \(p, q\in\mathbb{R}^4\) a linear
relationship between the players' scores could be forced by one of the players.

Using the notation of~\cite{Press2012}, assuming the utilities for player \(p\)
are given by \(S_x=(R, S, T, P)\) and for player \(q\) by \(S_y=(R, T, S, P)\)
and that the stationary scores of each player is given by \(S_X\) and \(S_Y\)
respectively. The main result of~\cite{Press2012} is that if

\begin{equation}\label{eqn:linear_relationship_for_p}
    \tilde p=\alpha S_x + \beta S_y + \gamma
\end{equation}

or

\begin{equation}\label{eqn:linear_relationship_for_q}
    \tilde q=\alpha S_x + \beta S_y + \gamma
\end{equation}

where \(\tilde p = (1 - p_1, 1 - p_2, p_3, p_4)\) and
\(\tilde q = (1 - q_1, 1 - q_2, q_3, q_4)\) then:

\begin{equation}
    \alpha S_X + \beta S_Y + \gamma = 0
\end{equation}

In~\cite{Press2012} a particular type of ZD strategy is defined: extortionate
strategies. If:

\begin{equation}\label{eqn:constraint_for_extortion}
    \gamma = - P(\alpha + \beta)
\end{equation}

then the player can ensure they get a score \(\chi\) times
larger than the opponent. This extortion coefficient is given by:

\begin{equation}\label{eqn:definition_of_chi}
    \chi=\frac{-\beta}{\alpha}
\end{equation}

Thus, if (\ref{eqn:constraint_for_extortion}) holds and \(\chi >1\) a player is
said to extort their opponent.
Here, the reverse problem is considered: given a
\(p\in\mathbb{R}^4\) how does one identify \(\alpha, \beta\) if they
exist and is the strategy in fact acting in an extortionate way?

These conditions correspond to:

\begin{align}
    \tilde p_1 & = \alpha R + \beta R - P (\alpha + \beta)
            \label{eqn:condition_for_tilde_p1}\\
    \tilde p_2 & = \alpha S + \beta T - P (\alpha + \beta)
            \label{eqn:condition_for_tilde_p2}\\
    \tilde p_3 & = \alpha T + \beta S - P (\alpha + \beta)
            \label{eqn:condition_for_tilde_p3}\\
    \tilde p_4 & = \alpha P + \beta P - P (\alpha + \beta)
            \label{eqn:condition_for_tilde_p4}
\end{align}

Equation (\ref{eqn:condition_for_tilde_p4}) ensures that \(p_4=\tilde p_4=0\).
Equations (\ref{eqn:condition_for_tilde_p1}-\ref{eqn:condition_for_tilde_p3})
can be used to eliminate \(\alpha, \beta\), giving:

\begin{equation}\label{eqn:planar_definition_of_extortion}
    \tilde p_1 = \frac{(R - P)(\tilde p_2 + \tilde p_3)}{S + T - 2P}
\end{equation}

with:

\begin{equation}\label{eqn:definition_of_chi}
    \chi = \frac{\tilde p_2 (P - T) + \tilde p_3 (S - P)}
                {\tilde p_2 (P - S) + \tilde p_3 (T - P)}
\end{equation}

Given a strategy \(p\in\mathbb{R}^{4\times 1}\) equations
(\ref{eqn:condition_for_tilde_p4}), (\ref{eqn:planar_definition_of_extortion}-\ref{eqn:definition_of_chi}) can be used to check if
a strategy is extortionate. The conditions correspond to:

\begin{align}
    p_1 & = \frac{(R-P)(p_2 + p_3) - R + T + S - P}{S + T - 2P}
     \label{eqn:condition_for_p1}\\
    p_4 & = 0 \label{eqn:condition_for_p4}\\
    1 & > p_2 + p_3\label{eqn:condition_for_chi}
\end{align}

The algebraic steps necessary to prove these results are available in the
supporting materials.

All extortionate strategies reside on a triangular (\ref{eqn:condition_for_chi})
plane (\ref{eqn:condition_for_p1}) in 3 dimensions (\ref{eqn:condition_for_p4}).
Using this formulation it can be seen that a necessary (but not sufficient)
condition for an extortionate strategy is that it cooperates on average less
than 50\% of the time when in a state of disagreement with the opponent.

As an example, consider the known extortionate strategy \(p=(8 / 9, 1 / 2, 1 /
3, 0)\) from~\cite{Stewart2012} which is referred to as \texttt{Extort-2}. In
this case, for the standard values of \((R, T, S, P)\) constraint
(\ref{eqn:condition_for_p1}) corresponds to:

\begin{equation}
    p_1 = \frac{2(p_2 + p_3) + 1}{3}
\end{equation}

It is clear that in this case all constraints hold.

This approach could in fact be used to confirm that a given strategy is acting
in an extortionate manner even if it is not a memory one strategy. However, in
practice, if a closed form for \(p\) is not known, then due to measurement
and/or numerical error this would not work.

This problem can be written in the following linear algebraic form where
\(x=(\alpha, \beta)\)
and \(p^*=(\tilde p_1 - 1, tilde_2 - 1, p_3)\):

\begin{equation}\label{eqn:linear_algebraic_equation_for_p}
    Cx= p^*
\end{equation}

\(C\) corresponds to equations
(\ref{eqn:condition_for_tilde_p1}-\ref{eqn:condition_for_tilde_p3}) and is
given by:

\begin{equation}\label{eqn:definition_of_C}
    C =
    \begin{bmatrix}
        R - P & R- P \\
        S - P & T- P \\
        T - P & S- P \\
    \end{bmatrix}
\end{equation}

Note that in general, equation (\ref{eqn:linear_algebraic_equation_for_p}) will
not necessarily have a solution. From the Rouch\'{e}-Capelli theorem if there is
a solution it is unique as \(\text{rank}(C)=2\) which is the dimension of the
variable \(x\). The best fitting \(x\) is found by minimizing:

\begin{equation}\label{eqn:r_squared}
    \text{SSError} = \|C x- p^*\|_2^2 = \sum_{i=1}^{3}\left((C\bar x)_i-p_i^*\right)^2
\end{equation}

Note that \(\text{SSError}\), which is the square of the Frobenius
norm~\cite{Golub2013}, becomes a measure of how close a strategy is to being an
extortionate strategy. Suspicion
of extortion then corresponds to a threshold on \(\text{SSError}\).

By observing interactions (human or otherwise), their memory one representation
can be inferred and this approach can be used to recognise extortionate
behaviour. The notion of comparing theoretic and actual plays of the IPD is not
novel, see for example~\cite{Rand2013}. Immediately it is noted that if the
environment is noisy~\cite{Wu1995} then no strategy can be considered to be
extortionate as \(p_4>0\).

In the next section, this idea will be illustrated by observing the interactions
that take place in a computer based tournament of the IPD\@.

\section{Numerical experiments}\label{sec:numerical-experiments}

In~\cite{Stewart2012} results from a tournament with
\input{./assets/tex/number_of_stewart_plotkin_strategies/main.tex} strategies,
was presented with specific consideration given to ZD strategies. This
tournament is reproduced here using the Axelrod-Python
project~\cite{Knight2016}. To obtain a good measure of the corresponding
transition rates for each strategy all matches have been run for
\input{assets/tex/number_of_turns/main.tex} turns and every match has been
repeated \input{assets/tex/number_of_repetitions/main.tex} times. All of this
interaction data is available at~\cite{vincent_knight_2018_1297075}. A good
match between the inferred Markov chain and the state distribution of the actual
interactions has been verified. Data for this is presented in the supplementary
materials.

Figure~\ref{fig:SSError_overall_in_stewart_plotkin} shows the \(\text{SSError}\)
values for all the strategies in the tournament, as reported
in~\cite{Stewart2012} the extortionate strategy (which has an expected
\(\text{SSError}\) approximately 0) gains a large number of wins.

\begin{figure}[!htbp]
    \centering
    \includegraphics[width=.8\textwidth]{./assets/img/SSError_overall_in_stewart_plotkin/main.pdf}
    \caption{\(\text{SSError}\) and state probabilities for the strategies
        of~\cite{Stewart2012}, ordered both by number of wins and overall score.
        Note that \(P(DC)\) is not shown as it corresponds to the transpose of
        \(P(CD)\). Cooperator and Defector are omitted as they do not visit all
        the states.}
    \label{fig:SSError_overall_in_stewart_plotkin}
\end{figure}

Here, the work of~\cite{Stewart2012} is extended by investigating a tournament
with \input{assets/tex/number_of_full_strategies/main.tex}
strategies.

The results of this analysis are shown in
Figure~\ref{fig:SSError_and_probabilities_in_full}. The top ranking strategies
by number of wins seem to be extortionate (but not against all strategies) and
it can be seen that a small sub group of strategies achieve mutual defection.
All the top ranking strategies according to score achieve mutual cooperation and
do not extort each other, however they
\textbf{do} exhibit extortionate behaviour towards a number of the lower ranking
strategies.

\begin{figure}[!htbp]
    \centering
    \includegraphics[width=.8\textwidth]{./assets/img/SSError_and_probabilities_in_full/main.pdf}
    \caption{\(\text{SSError}\) for the strategies for the full tournament. Only
    strategy interactions for which \(p_4=0\) and \(\chi>1\) are displayed.}
    \label{fig:SSError_and_probabilities_in_full}
\end{figure}

\section{Conclusion}\label{sec:conclusion}

This work defines an approach to measure whether or not a player is playing a
strategy that corresponds to an extortionate strategy as defined
in~\cite{Press2012}: a mathematical model for suspicion. Indeed, all
extortionate strategies have been
 classified as lying on a triangular plane.
This rigorous classification fails to be robust to small measurement error, thus
a statistical approach is proposed.
This is done through a linear algebraic approach for approximating the solution
of a linear system. Using this, a large number of pairwise interactions is
simulated and in fact very few strategies are found to act extortionately.

The work of~\cite{Press2012}, whilst showing that a clever approach to taking
advantage of another memory one strategy exists: this is incomplete. Whilst the
elegance of this result is very attractive, just as the simplicity of the
victory of Tit For Tat in Axelrod's original tournaments was, it is incomplete.
Extortionate strategies achieve a high number of wins but they do not
achieve a high score which corresponds to the fitness landscape in an
evolutionary sense. From the large number of interactions a payoff matrix \(S\)
can be measured where \(S_{ij}\) denotes the score (using standard values of
\((R, S, T, P) = (3, 0, 5, 1)\)) of the \(i\)th strategy
against the \(j\)th strategy. Using this, the replicator equation
describes the evolution of the system based on a population density fitness
function:

\begin{equation}\label{eqn:replicator_dynamics}
    \frac{dx}{dt} = x(S-x^TS x)
\end{equation}

Equation (\ref{eqn:replicator_dynamics}) is solved numerically through an
integration technique described in~\cite{Petzold1983} and
Figure~\ref{fig:replicator_dynamics} shows the evolution of the distribution of
the system: the various strategies are ranked by scores. It is clear to see that
only the high ranking strategies survive the evolutionary process (in fact,
only \input{./assets/img/replicator_dynamics/main.tex}
have a final distribution greater than \(10 ^ {-2}\)). This confirms the
findings of~\cite{Moran1707} in which sophisticated strategies resist
evolutionary invasion of shorter memory strategies. Recalling
Figure~\ref{fig:SSError_and_probabilities_in_full} this demonstrates that:

\begin{itemize}
    \item Cooperation emerges through the evolutionary process: the high scoring
        strategies do not exhibit extortionate behaviour towards each other.
    \item Extortionate strategies do not survive the evolutionary process.
\end{itemize}

\begin{figure}[!htbp]
    \centering
    \includegraphics[width=.8\textwidth]{./assets/img/replicator_dynamics/main.pdf}
    \caption{Numerical simulation of the replicator equation
    (\ref{eqn:replicator_dynamics}): strategies are ordered by score, only the strategies with a high score survive the evolutionary process.}
    \label{fig:replicator_dynamics}
\end{figure}

This work can be used to classify plays of the IPD\@: data can be collected from
actual interactions (in lab or in the field). Furthermore, this allows for a
classification method similar to the notion of fingerprinting presented
in~\cite{Ashlock2008}. Trained strategies can potentially be classified as
extortionate or not or it could be possible to even constrain the reinforcement
learning approaches that are becoming prevalent in the literature.
Alternatively, this mathematical approach for recognising extortion could be
used in sophisticated strategies to defend against invasion. Arguably, some of
the strategies considered here exhibit this behaviour, indeed as described
in~\cite{Harper2017}, the top ranking strategies in the full tournament are
obtained using evolutionary reinforcement learning techniques, thus, suspicion
of extortionate behaviour could in fact be an evolutionary trait.

\section*{Acknowledgements}

The following open source software libraries were used in this research:

\begin{itemize}
    \item The Axelrod ~\cite{Knight2016, Knight2018} library (IPD strategies and
        tournaments).
    \item The sympy library~\cite{Meurer2017} (verification of all symbolic
        calculations).
    \item The matplotlib~\cite{Droettboom2018} library (visualisation).
    \item The pandas~\cite{Structures2010}, dask~\cite{Dask2016} and
        NumPy~\cite{Oliphant2015} libraries (data manipulation).
    \item The SciPy~\cite{Jones2001} library (numerical integration of the
        replicator equation).
\end{itemize}

This work was performed using the computational facilities of the Advanced
Research Computing @ Cardiff (ARCCA) Division, Cardiff University.

\printbibliography

\newpage
\section*{Supplementary materials}

\includepdf{assets/pdf/proof_of_form_of_extortionate_strategies/main.pdf}

\newpage

Using the pair wise interactions the transition rates \(p,
q\) can be measured and the steady state probabilities inferred and compared to
the actual probabilities of each state.
This is done numerically by computing the singular eigenvector of the
matrix \(A\) \cite{Stewart2009}:

\[
    A =
    \begin{bmatrix}
        p_1 q_1 & p_1 (1 - q_1) & (1 - p_1) q_1 & (1 -p_1) (1 - q_1) \\
        p_2 q_2 & p_2 (1 - q_2) & (1 - p_2) q_2 & (1 -p_2) (1 - q_2) \\
        p_3 q_3 & p_3 (1 - q_3) & (1 - p_3) q_3 & (1 -p_3) (1 - q_3) \\
        p_4 q_4 & p_4 (1 - q_4) & (1 - p_4) q_4 & (1 -p_4) (1 - q_4) \\
    \end{bmatrix}
\]

Figure~\ref{fig:computed_probabilities_vs_theoretic_probabilities} shows a
regression line fitted to every pairwise interaction with a reported
\(\text{SSError}\) value (pairwise interactions with missing states were
omitted). This serves to validate the approach: a part from some edge cases the
relationship is consistent.

\begin{figure}[!htbp]
    \centering
    \includegraphics[width=.8\textwidth]{./assets/img/computed_probabilities_vs_theoretic_probabilities/main.pdf}
    \caption{The
        relationship between the steady state probabilities inferred from the
        measured transitions and the actual steady state probabilities. A linear
        regression line is included validating the approach.}
    \label{fig:computed_probabilities_vs_theoretic_probabilities}
\end{figure}


\end{document}

strategies.

The results of this analysis are shown in
Figure~\ref{fig:SSError_and_probabilities_in_full}. The top ranking strategies
by number of wins seem to be extortionate (but not against all strategies) and
it can be seen that a small sub group of strategies achieve mutual defection.
All the top ranking strategies according to score achieve mutual cooperation and
do not extort each other, however they
\textbf{do} exhibit extortionate behaviour towards a number of the lower ranking
strategies.

\begin{figure}[!htbp]
    \centering
    \includegraphics[width=.8\textwidth]{./assets/img/SSError_and_probabilities_in_full/main.pdf}
    \caption{\(\text{SSError}\) for the strategies for the full tournament. Only
    strategy interactions for which \(p_4=0\) and \(\chi>1\) are displayed.}
    \label{fig:SSError_and_probabilities_in_full}
\end{figure}

\section{Conclusion}\label{sec:conclusion}

This work defines an approach to measure whether or not a player is playing a
strategy that corresponds to an extortionate strategy as defined
in~\cite{Press2012}: a mathematical model for suspicion. Indeed, all
extortionate strategies have been
 classified as lying on a triangular plane.
This rigorous classification fails to be robust to small measurement error, thus
a statistical approach is proposed.
This is done through a linear algebraic approach for approximating the solution
of a linear system. Using this, a large number of pairwise interactions is
simulated and in fact very few strategies are found to act extortionately.

The work of~\cite{Press2012}, whilst showing that a clever approach to taking
advantage of another memory one strategy exists: this is incomplete. Whilst the
elegance of this result is very attractive, just as the simplicity of the
victory of Tit For Tat in Axelrod's original tournaments was, it is incomplete.
Extortionate strategies achieve a high number of wins but they do not
achieve a high score which corresponds to the fitness landscape in an
evolutionary sense. From the large number of interactions a payoff matrix \(S\)
can be measured where \(S_{ij}\) denotes the score (using standard values of
\((R, S, T, P) = (3, 0, 5, 1)\)) of the \(i\)th strategy
against the \(j\)th strategy. Using this, the replicator equation
describes the evolution of the system based on a population density fitness
function:

\begin{equation}\label{eqn:replicator_dynamics}
    \frac{dx}{dt} = x(S-x^TS x)
\end{equation}

Equation (\ref{eqn:replicator_dynamics}) is solved numerically through an
integration technique described in~\cite{Petzold1983} and
Figure~\ref{fig:replicator_dynamics} shows the evolution of the distribution of
the system: the various strategies are ranked by scores. It is clear to see that
only the high ranking strategies survive the evolutionary process (in fact,
only \documentclass[a4paper]{article}

\usepackage{amsmath}
\usepackage{amssymb}
\usepackage[margin=1.5cm,
            includefoot,
            footskip=30pt]{geometry}
\usepackage{layout}
\usepackage{graphicx}
\usepackage{subcaption}

\usepackage{biblatex}
\usepackage{pdfpages}

\bibliography{main.bib}

\title{Suspicion: Recognising and evaluating the effectiveness
       of extortion in the Iterated Prisoner's Dilemma}
\author{Vincent A. Knight \and Nikoleta E. Glynatsi}
\date{\today}



\begin{document}

\maketitle

\begin{abstract}
    The Iterated Prisoner's Dilemma is a model for rational and evolutionary
    interactive behaviour. It has applications both in the study of human social
    behaviour as well as in biology.
    It is used to understand when and how a rational individual might
    accept an immediate cost to their own utility for the direct benefit of
    another.

    Much attention has been given to a class of strategies called
    Zero Determinant strategies. It has been theoretically shown that these
    strategies can ``extort'' any player.

    In this work, an approach to identify if observed strategies are playing in
    an extortionate way is described. Furthermore, experimental analysis of
    a large tournament with \input{assets/tex/number_of_full_strategies/main.tex}
    strategies is considered. In this setting
    the most highly performing strategies do not play in an extortionate way
    against each other but do against lower performing strategies.
    This suggests that whilst the theory of Zero Determinant strategies
    indicates that memory is not of fundamental importance to the evolution of
    cooperative behaviour, this is incomplete.
\end{abstract}

\section{Introduction}\label{sec:introduction}

Agent based game theoretic models have become a stalwart of the underpinning
mathematics of interactive behaviours. One of the major pieces of work
in this area is the pair of original computer tournaments run by Robert
Axelrod~\cite{Axelrod1980, Axelrod1980a}. These tournaments pitted submitted
computer strategies against each other in plays of the Iterated Prisoner's
Dilemma. A common game where agents can choose to pay a slight cost to their
immediate utility in the hope of building a reputation. This has been used in
economic and evolutionary game theory to understand the evolution of cooperative
behaviour.

Recently, a class of strategies was described in~\cite{Press2012} that can
provably extort any given opponent. In~\cite{Hilbe2013, Moran1707} some
questions have already been asked about the true effectiveness of these
strategies in an evolutionary setting. Here another question is asked: is it
possible to recognise this extortionate behaviour? A mathematical procedure for
suspicion is presented: in the same way that the continued actions of an
extortionate individual might raise suspicion.

This work makes use of the Axelrod Python library~\cite{Knight2018, Knight2016}
with a large number of Prisoner Dilemma strategies available to give an
extensive numerical example of the ideas presented.  The approach is presented
in Section~\ref{sec:delta-zd-strategies}.  All of the code and data discussed
in Section~\ref{sec:numerical-experiments} is open sourced, archived and
written according to best scientific principles~\cite{Wilson2014}. The data
archive can be found at~\cite{vincent_knight_2018_1297075}.

\section{Recognising Extortion}\label{sec:delta-zd-strategies}

In~\cite{Press2012}, given a match between 2 memory-one strategies, the concept
of Zero Determinant (ZD) strategies is introduced. The main result of that paper
shows that given two memory one players \(p, q\in\mathbb{R}^4\) a linear
relationship between the players' scores could be forced by one of the players.

Using the notation of~\cite{Press2012}, assuming the utilities for player \(p\)
are given by \(S_x=(R, S, T, P)\) and for player \(q\) by \(S_y=(R, T, S, P)\)
and that the stationary scores of each player is given by \(S_X\) and \(S_Y\)
respectively. The main result of~\cite{Press2012} is that if

\begin{equation}\label{eqn:linear_relationship_for_p}
    \tilde p=\alpha S_x + \beta S_y + \gamma
\end{equation}

or

\begin{equation}\label{eqn:linear_relationship_for_q}
    \tilde q=\alpha S_x + \beta S_y + \gamma
\end{equation}

where \(\tilde p = (1 - p_1, 1 - p_2, p_3, p_4)\) and
\(\tilde q = (1 - q_1, 1 - q_2, q_3, q_4)\) then:

\begin{equation}
    \alpha S_X + \beta S_Y + \gamma = 0
\end{equation}

In~\cite{Press2012} a particular type of ZD strategy is defined: extortionate
strategies. If:

\begin{equation}\label{eqn:constraint_for_extortion}
    \gamma = - P(\alpha + \beta)
\end{equation}

then the player can ensure they get a score \(\chi\) times
larger than the opponent. This extortion coefficient is given by:

\begin{equation}\label{eqn:definition_of_chi}
    \chi=\frac{-\beta}{\alpha}
\end{equation}

Thus, if (\ref{eqn:constraint_for_extortion}) holds and \(\chi >1\) a player is
said to extort their opponent.
Here, the reverse problem is considered: given a
\(p\in\mathbb{R}^4\) how does one identify \(\alpha, \beta\) if they
exist and is the strategy in fact acting in an extortionate way?

These conditions correspond to:

\begin{align}
    \tilde p_1 & = \alpha R + \beta R - P (\alpha + \beta)
            \label{eqn:condition_for_tilde_p1}\\
    \tilde p_2 & = \alpha S + \beta T - P (\alpha + \beta)
            \label{eqn:condition_for_tilde_p2}\\
    \tilde p_3 & = \alpha T + \beta S - P (\alpha + \beta)
            \label{eqn:condition_for_tilde_p3}\\
    \tilde p_4 & = \alpha P + \beta P - P (\alpha + \beta)
            \label{eqn:condition_for_tilde_p4}
\end{align}

Equation (\ref{eqn:condition_for_tilde_p4}) ensures that \(p_4=\tilde p_4=0\).
Equations (\ref{eqn:condition_for_tilde_p1}-\ref{eqn:condition_for_tilde_p3})
can be used to eliminate \(\alpha, \beta\), giving:

\begin{equation}\label{eqn:planar_definition_of_extortion}
    \tilde p_1 = \frac{(R - P)(\tilde p_2 + \tilde p_3)}{S + T - 2P}
\end{equation}

with:

\begin{equation}\label{eqn:definition_of_chi}
    \chi = \frac{\tilde p_2 (P - T) + \tilde p_3 (S - P)}
                {\tilde p_2 (P - S) + \tilde p_3 (T - P)}
\end{equation}

Given a strategy \(p\in\mathbb{R}^{4\times 1}\) equations
(\ref{eqn:condition_for_tilde_p4}), (\ref{eqn:planar_definition_of_extortion}-\ref{eqn:definition_of_chi}) can be used to check if
a strategy is extortionate. The conditions correspond to:

\begin{align}
    p_1 & = \frac{(R-P)(p_2 + p_3) - R + T + S - P}{S + T - 2P}
     \label{eqn:condition_for_p1}\\
    p_4 & = 0 \label{eqn:condition_for_p4}\\
    1 & > p_2 + p_3\label{eqn:condition_for_chi}
\end{align}

The algebraic steps necessary to prove these results are available in the
supporting materials.

All extortionate strategies reside on a triangular (\ref{eqn:condition_for_chi})
plane (\ref{eqn:condition_for_p1}) in 3 dimensions (\ref{eqn:condition_for_p4}).
Using this formulation it can be seen that a necessary (but not sufficient)
condition for an extortionate strategy is that it cooperates on average less
than 50\% of the time when in a state of disagreement with the opponent.

As an example, consider the known extortionate strategy \(p=(8 / 9, 1 / 2, 1 /
3, 0)\) from~\cite{Stewart2012} which is referred to as \texttt{Extort-2}. In
this case, for the standard values of \((R, T, S, P)\) constraint
(\ref{eqn:condition_for_p1}) corresponds to:

\begin{equation}
    p_1 = \frac{2(p_2 + p_3) + 1}{3}
\end{equation}

It is clear that in this case all constraints hold.

This approach could in fact be used to confirm that a given strategy is acting
in an extortionate manner even if it is not a memory one strategy. However, in
practice, if a closed form for \(p\) is not known, then due to measurement
and/or numerical error this would not work.

This problem can be written in the following linear algebraic form where
\(x=(\alpha, \beta)\)
and \(p^*=(\tilde p_1 - 1, tilde_2 - 1, p_3)\):

\begin{equation}\label{eqn:linear_algebraic_equation_for_p}
    Cx= p^*
\end{equation}

\(C\) corresponds to equations
(\ref{eqn:condition_for_tilde_p1}-\ref{eqn:condition_for_tilde_p3}) and is
given by:

\begin{equation}\label{eqn:definition_of_C}
    C =
    \begin{bmatrix}
        R - P & R- P \\
        S - P & T- P \\
        T - P & S- P \\
    \end{bmatrix}
\end{equation}

Note that in general, equation (\ref{eqn:linear_algebraic_equation_for_p}) will
not necessarily have a solution. From the Rouch\'{e}-Capelli theorem if there is
a solution it is unique as \(\text{rank}(C)=2\) which is the dimension of the
variable \(x\). The best fitting \(x\) is found by minimizing:

\begin{equation}\label{eqn:r_squared}
    \text{SSError} = \|C x- p^*\|_2^2 = \sum_{i=1}^{3}\left((C\bar x)_i-p_i^*\right)^2
\end{equation}

Note that \(\text{SSError}\), which is the square of the Frobenius
norm~\cite{Golub2013}, becomes a measure of how close a strategy is to being an
extortionate strategy. Suspicion
of extortion then corresponds to a threshold on \(\text{SSError}\).

By observing interactions (human or otherwise), their memory one representation
can be inferred and this approach can be used to recognise extortionate
behaviour. The notion of comparing theoretic and actual plays of the IPD is not
novel, see for example~\cite{Rand2013}. Immediately it is noted that if the
environment is noisy~\cite{Wu1995} then no strategy can be considered to be
extortionate as \(p_4>0\).

In the next section, this idea will be illustrated by observing the interactions
that take place in a computer based tournament of the IPD\@.

\section{Numerical experiments}\label{sec:numerical-experiments}

In~\cite{Stewart2012} results from a tournament with
\input{./assets/tex/number_of_stewart_plotkin_strategies/main.tex} strategies,
was presented with specific consideration given to ZD strategies. This
tournament is reproduced here using the Axelrod-Python
project~\cite{Knight2016}. To obtain a good measure of the corresponding
transition rates for each strategy all matches have been run for
\input{assets/tex/number_of_turns/main.tex} turns and every match has been
repeated \input{assets/tex/number_of_repetitions/main.tex} times. All of this
interaction data is available at~\cite{vincent_knight_2018_1297075}. A good
match between the inferred Markov chain and the state distribution of the actual
interactions has been verified. Data for this is presented in the supplementary
materials.

Figure~\ref{fig:SSError_overall_in_stewart_plotkin} shows the \(\text{SSError}\)
values for all the strategies in the tournament, as reported
in~\cite{Stewart2012} the extortionate strategy (which has an expected
\(\text{SSError}\) approximately 0) gains a large number of wins.

\begin{figure}[!htbp]
    \centering
    \includegraphics[width=.8\textwidth]{./assets/img/SSError_overall_in_stewart_plotkin/main.pdf}
    \caption{\(\text{SSError}\) and state probabilities for the strategies
        of~\cite{Stewart2012}, ordered both by number of wins and overall score.
        Note that \(P(DC)\) is not shown as it corresponds to the transpose of
        \(P(CD)\). Cooperator and Defector are omitted as they do not visit all
        the states.}
    \label{fig:SSError_overall_in_stewart_plotkin}
\end{figure}

Here, the work of~\cite{Stewart2012} is extended by investigating a tournament
with \input{assets/tex/number_of_full_strategies/main.tex}
strategies.

The results of this analysis are shown in
Figure~\ref{fig:SSError_and_probabilities_in_full}. The top ranking strategies
by number of wins seem to be extortionate (but not against all strategies) and
it can be seen that a small sub group of strategies achieve mutual defection.
All the top ranking strategies according to score achieve mutual cooperation and
do not extort each other, however they
\textbf{do} exhibit extortionate behaviour towards a number of the lower ranking
strategies.

\begin{figure}[!htbp]
    \centering
    \includegraphics[width=.8\textwidth]{./assets/img/SSError_and_probabilities_in_full/main.pdf}
    \caption{\(\text{SSError}\) for the strategies for the full tournament. Only
    strategy interactions for which \(p_4=0\) and \(\chi>1\) are displayed.}
    \label{fig:SSError_and_probabilities_in_full}
\end{figure}

\section{Conclusion}\label{sec:conclusion}

This work defines an approach to measure whether or not a player is playing a
strategy that corresponds to an extortionate strategy as defined
in~\cite{Press2012}: a mathematical model for suspicion. Indeed, all
extortionate strategies have been
 classified as lying on a triangular plane.
This rigorous classification fails to be robust to small measurement error, thus
a statistical approach is proposed.
This is done through a linear algebraic approach for approximating the solution
of a linear system. Using this, a large number of pairwise interactions is
simulated and in fact very few strategies are found to act extortionately.

The work of~\cite{Press2012}, whilst showing that a clever approach to taking
advantage of another memory one strategy exists: this is incomplete. Whilst the
elegance of this result is very attractive, just as the simplicity of the
victory of Tit For Tat in Axelrod's original tournaments was, it is incomplete.
Extortionate strategies achieve a high number of wins but they do not
achieve a high score which corresponds to the fitness landscape in an
evolutionary sense. From the large number of interactions a payoff matrix \(S\)
can be measured where \(S_{ij}\) denotes the score (using standard values of
\((R, S, T, P) = (3, 0, 5, 1)\)) of the \(i\)th strategy
against the \(j\)th strategy. Using this, the replicator equation
describes the evolution of the system based on a population density fitness
function:

\begin{equation}\label{eqn:replicator_dynamics}
    \frac{dx}{dt} = x(S-x^TS x)
\end{equation}

Equation (\ref{eqn:replicator_dynamics}) is solved numerically through an
integration technique described in~\cite{Petzold1983} and
Figure~\ref{fig:replicator_dynamics} shows the evolution of the distribution of
the system: the various strategies are ranked by scores. It is clear to see that
only the high ranking strategies survive the evolutionary process (in fact,
only \input{./assets/img/replicator_dynamics/main.tex}
have a final distribution greater than \(10 ^ {-2}\)). This confirms the
findings of~\cite{Moran1707} in which sophisticated strategies resist
evolutionary invasion of shorter memory strategies. Recalling
Figure~\ref{fig:SSError_and_probabilities_in_full} this demonstrates that:

\begin{itemize}
    \item Cooperation emerges through the evolutionary process: the high scoring
        strategies do not exhibit extortionate behaviour towards each other.
    \item Extortionate strategies do not survive the evolutionary process.
\end{itemize}

\begin{figure}[!htbp]
    \centering
    \includegraphics[width=.8\textwidth]{./assets/img/replicator_dynamics/main.pdf}
    \caption{Numerical simulation of the replicator equation
    (\ref{eqn:replicator_dynamics}): strategies are ordered by score, only the strategies with a high score survive the evolutionary process.}
    \label{fig:replicator_dynamics}
\end{figure}

This work can be used to classify plays of the IPD\@: data can be collected from
actual interactions (in lab or in the field). Furthermore, this allows for a
classification method similar to the notion of fingerprinting presented
in~\cite{Ashlock2008}. Trained strategies can potentially be classified as
extortionate or not or it could be possible to even constrain the reinforcement
learning approaches that are becoming prevalent in the literature.
Alternatively, this mathematical approach for recognising extortion could be
used in sophisticated strategies to defend against invasion. Arguably, some of
the strategies considered here exhibit this behaviour, indeed as described
in~\cite{Harper2017}, the top ranking strategies in the full tournament are
obtained using evolutionary reinforcement learning techniques, thus, suspicion
of extortionate behaviour could in fact be an evolutionary trait.

\section*{Acknowledgements}

The following open source software libraries were used in this research:

\begin{itemize}
    \item The Axelrod ~\cite{Knight2016, Knight2018} library (IPD strategies and
        tournaments).
    \item The sympy library~\cite{Meurer2017} (verification of all symbolic
        calculations).
    \item The matplotlib~\cite{Droettboom2018} library (visualisation).
    \item The pandas~\cite{Structures2010}, dask~\cite{Dask2016} and
        NumPy~\cite{Oliphant2015} libraries (data manipulation).
    \item The SciPy~\cite{Jones2001} library (numerical integration of the
        replicator equation).
\end{itemize}

This work was performed using the computational facilities of the Advanced
Research Computing @ Cardiff (ARCCA) Division, Cardiff University.

\printbibliography

\newpage
\section*{Supplementary materials}

\includepdf{assets/pdf/proof_of_form_of_extortionate_strategies/main.pdf}

\newpage

Using the pair wise interactions the transition rates \(p,
q\) can be measured and the steady state probabilities inferred and compared to
the actual probabilities of each state.
This is done numerically by computing the singular eigenvector of the
matrix \(A\) \cite{Stewart2009}:

\[
    A =
    \begin{bmatrix}
        p_1 q_1 & p_1 (1 - q_1) & (1 - p_1) q_1 & (1 -p_1) (1 - q_1) \\
        p_2 q_2 & p_2 (1 - q_2) & (1 - p_2) q_2 & (1 -p_2) (1 - q_2) \\
        p_3 q_3 & p_3 (1 - q_3) & (1 - p_3) q_3 & (1 -p_3) (1 - q_3) \\
        p_4 q_4 & p_4 (1 - q_4) & (1 - p_4) q_4 & (1 -p_4) (1 - q_4) \\
    \end{bmatrix}
\]

Figure~\ref{fig:computed_probabilities_vs_theoretic_probabilities} shows a
regression line fitted to every pairwise interaction with a reported
\(\text{SSError}\) value (pairwise interactions with missing states were
omitted). This serves to validate the approach: a part from some edge cases the
relationship is consistent.

\begin{figure}[!htbp]
    \centering
    \includegraphics[width=.8\textwidth]{./assets/img/computed_probabilities_vs_theoretic_probabilities/main.pdf}
    \caption{The
        relationship between the steady state probabilities inferred from the
        measured transitions and the actual steady state probabilities. A linear
        regression line is included validating the approach.}
    \label{fig:computed_probabilities_vs_theoretic_probabilities}
\end{figure}


\end{document}

have a final distribution greater than \(10 ^ {-2}\)). This confirms the
findings of~\cite{Moran1707} in which sophisticated strategies resist
evolutionary invasion of shorter memory strategies. Recalling
Figure~\ref{fig:SSError_and_probabilities_in_full} this demonstrates that:

\begin{itemize}
    \item Cooperation emerges through the evolutionary process: the high scoring
        strategies do not exhibit extortionate behaviour towards each other.
    \item Extortionate strategies do not survive the evolutionary process.
\end{itemize}

\begin{figure}[!htbp]
    \centering
    \includegraphics[width=.8\textwidth]{./assets/img/replicator_dynamics/main.pdf}
    \caption{Numerical simulation of the replicator equation
    (\ref{eqn:replicator_dynamics}): strategies are ordered by score, only the strategies with a high score survive the evolutionary process.}
    \label{fig:replicator_dynamics}
\end{figure}

This work can be used to classify plays of the IPD\@: data can be collected from
actual interactions (in lab or in the field). Furthermore, this allows for a
classification method similar to the notion of fingerprinting presented
in~\cite{Ashlock2008}. Trained strategies can potentially be classified as
extortionate or not or it could be possible to even constrain the reinforcement
learning approaches that are becoming prevalent in the literature.
Alternatively, this mathematical approach for recognising extortion could be
used in sophisticated strategies to defend against invasion. Arguably, some of
the strategies considered here exhibit this behaviour, indeed as described
in~\cite{Harper2017}, the top ranking strategies in the full tournament are
obtained using evolutionary reinforcement learning techniques, thus, suspicion
of extortionate behaviour could in fact be an evolutionary trait.

\section*{Acknowledgements}

The following open source software libraries were used in this research:

\begin{itemize}
    \item The Axelrod ~\cite{Knight2016, Knight2018} library (IPD strategies and
        tournaments).
    \item The sympy library~\cite{Meurer2017} (verification of all symbolic
        calculations).
    \item The matplotlib~\cite{Droettboom2018} library (visualisation).
    \item The pandas~\cite{Structures2010}, dask~\cite{Dask2016} and
        NumPy~\cite{Oliphant2015} libraries (data manipulation).
    \item The SciPy~\cite{Jones2001} library (numerical integration of the
        replicator equation).
\end{itemize}

This work was performed using the computational facilities of the Advanced
Research Computing @ Cardiff (ARCCA) Division, Cardiff University.

\printbibliography

\newpage
\section*{Supplementary materials}

\includepdf{assets/pdf/proof_of_form_of_extortionate_strategies/main.pdf}

\newpage

Using the pair wise interactions the transition rates \(p,
q\) can be measured and the steady state probabilities inferred and compared to
the actual probabilities of each state.
This is done numerically by computing the singular eigenvector of the
matrix \(A\) \cite{Stewart2009}:

\[
    A =
    \begin{bmatrix}
        p_1 q_1 & p_1 (1 - q_1) & (1 - p_1) q_1 & (1 -p_1) (1 - q_1) \\
        p_2 q_2 & p_2 (1 - q_2) & (1 - p_2) q_2 & (1 -p_2) (1 - q_2) \\
        p_3 q_3 & p_3 (1 - q_3) & (1 - p_3) q_3 & (1 -p_3) (1 - q_3) \\
        p_4 q_4 & p_4 (1 - q_4) & (1 - p_4) q_4 & (1 -p_4) (1 - q_4) \\
    \end{bmatrix}
\]

Figure~\ref{fig:computed_probabilities_vs_theoretic_probabilities} shows a
regression line fitted to every pairwise interaction with a reported
\(\text{SSError}\) value (pairwise interactions with missing states were
omitted). This serves to validate the approach: a part from some edge cases the
relationship is consistent.

\begin{figure}[!htbp]
    \centering
    \includegraphics[width=.8\textwidth]{./assets/img/computed_probabilities_vs_theoretic_probabilities/main.pdf}
    \caption{The
        relationship between the steady state probabilities inferred from the
        measured transitions and the actual steady state probabilities. A linear
        regression line is included validating the approach.}
    \label{fig:computed_probabilities_vs_theoretic_probabilities}
\end{figure}


\end{document}
 strategies,
was presented with specific consideration given to ZD strategies. This
tournament is reproduced here using the Axelrod-Python
project~\cite{Knight2016}. To obtain a good measure of the corresponding
transition rates for each strategy all matches have been run for
\documentclass[a4paper]{article}

\usepackage{amsmath}
\usepackage{amssymb}
\usepackage[margin=1.5cm,
            includefoot,
            footskip=30pt]{geometry}
\usepackage{layout}
\usepackage{graphicx}
\usepackage{subcaption}

\usepackage{biblatex}
\usepackage{pdfpages}

\bibliography{main.bib}

\title{Suspicion: Recognising and evaluating the effectiveness
       of extortion in the Iterated Prisoner's Dilemma}
\author{Vincent A. Knight \and Nikoleta E. Glynatsi}
\date{\today}



\begin{document}

\maketitle

\begin{abstract}
    The Iterated Prisoner's Dilemma is a model for rational and evolutionary
    interactive behaviour. It has applications both in the study of human social
    behaviour as well as in biology.
    It is used to understand when and how a rational individual might
    accept an immediate cost to their own utility for the direct benefit of
    another.

    Much attention has been given to a class of strategies called
    Zero Determinant strategies. It has been theoretically shown that these
    strategies can ``extort'' any player.

    In this work, an approach to identify if observed strategies are playing in
    an extortionate way is described. Furthermore, experimental analysis of
    a large tournament with \documentclass[a4paper]{article}

\usepackage{amsmath}
\usepackage{amssymb}
\usepackage[margin=1.5cm,
            includefoot,
            footskip=30pt]{geometry}
\usepackage{layout}
\usepackage{graphicx}
\usepackage{subcaption}

\usepackage{biblatex}
\usepackage{pdfpages}

\bibliography{main.bib}

\title{Suspicion: Recognising and evaluating the effectiveness
       of extortion in the Iterated Prisoner's Dilemma}
\author{Vincent A. Knight \and Nikoleta E. Glynatsi}
\date{\today}



\begin{document}

\maketitle

\begin{abstract}
    The Iterated Prisoner's Dilemma is a model for rational and evolutionary
    interactive behaviour. It has applications both in the study of human social
    behaviour as well as in biology.
    It is used to understand when and how a rational individual might
    accept an immediate cost to their own utility for the direct benefit of
    another.

    Much attention has been given to a class of strategies called
    Zero Determinant strategies. It has been theoretically shown that these
    strategies can ``extort'' any player.

    In this work, an approach to identify if observed strategies are playing in
    an extortionate way is described. Furthermore, experimental analysis of
    a large tournament with \input{assets/tex/number_of_full_strategies/main.tex}
    strategies is considered. In this setting
    the most highly performing strategies do not play in an extortionate way
    against each other but do against lower performing strategies.
    This suggests that whilst the theory of Zero Determinant strategies
    indicates that memory is not of fundamental importance to the evolution of
    cooperative behaviour, this is incomplete.
\end{abstract}

\section{Introduction}\label{sec:introduction}

Agent based game theoretic models have become a stalwart of the underpinning
mathematics of interactive behaviours. One of the major pieces of work
in this area is the pair of original computer tournaments run by Robert
Axelrod~\cite{Axelrod1980, Axelrod1980a}. These tournaments pitted submitted
computer strategies against each other in plays of the Iterated Prisoner's
Dilemma. A common game where agents can choose to pay a slight cost to their
immediate utility in the hope of building a reputation. This has been used in
economic and evolutionary game theory to understand the evolution of cooperative
behaviour.

Recently, a class of strategies was described in~\cite{Press2012} that can
provably extort any given opponent. In~\cite{Hilbe2013, Moran1707} some
questions have already been asked about the true effectiveness of these
strategies in an evolutionary setting. Here another question is asked: is it
possible to recognise this extortionate behaviour? A mathematical procedure for
suspicion is presented: in the same way that the continued actions of an
extortionate individual might raise suspicion.

This work makes use of the Axelrod Python library~\cite{Knight2018, Knight2016}
with a large number of Prisoner Dilemma strategies available to give an
extensive numerical example of the ideas presented.  The approach is presented
in Section~\ref{sec:delta-zd-strategies}.  All of the code and data discussed
in Section~\ref{sec:numerical-experiments} is open sourced, archived and
written according to best scientific principles~\cite{Wilson2014}. The data
archive can be found at~\cite{vincent_knight_2018_1297075}.

\section{Recognising Extortion}\label{sec:delta-zd-strategies}

In~\cite{Press2012}, given a match between 2 memory-one strategies, the concept
of Zero Determinant (ZD) strategies is introduced. The main result of that paper
shows that given two memory one players \(p, q\in\mathbb{R}^4\) a linear
relationship between the players' scores could be forced by one of the players.

Using the notation of~\cite{Press2012}, assuming the utilities for player \(p\)
are given by \(S_x=(R, S, T, P)\) and for player \(q\) by \(S_y=(R, T, S, P)\)
and that the stationary scores of each player is given by \(S_X\) and \(S_Y\)
respectively. The main result of~\cite{Press2012} is that if

\begin{equation}\label{eqn:linear_relationship_for_p}
    \tilde p=\alpha S_x + \beta S_y + \gamma
\end{equation}

or

\begin{equation}\label{eqn:linear_relationship_for_q}
    \tilde q=\alpha S_x + \beta S_y + \gamma
\end{equation}

where \(\tilde p = (1 - p_1, 1 - p_2, p_3, p_4)\) and
\(\tilde q = (1 - q_1, 1 - q_2, q_3, q_4)\) then:

\begin{equation}
    \alpha S_X + \beta S_Y + \gamma = 0
\end{equation}

In~\cite{Press2012} a particular type of ZD strategy is defined: extortionate
strategies. If:

\begin{equation}\label{eqn:constraint_for_extortion}
    \gamma = - P(\alpha + \beta)
\end{equation}

then the player can ensure they get a score \(\chi\) times
larger than the opponent. This extortion coefficient is given by:

\begin{equation}\label{eqn:definition_of_chi}
    \chi=\frac{-\beta}{\alpha}
\end{equation}

Thus, if (\ref{eqn:constraint_for_extortion}) holds and \(\chi >1\) a player is
said to extort their opponent.
Here, the reverse problem is considered: given a
\(p\in\mathbb{R}^4\) how does one identify \(\alpha, \beta\) if they
exist and is the strategy in fact acting in an extortionate way?

These conditions correspond to:

\begin{align}
    \tilde p_1 & = \alpha R + \beta R - P (\alpha + \beta)
            \label{eqn:condition_for_tilde_p1}\\
    \tilde p_2 & = \alpha S + \beta T - P (\alpha + \beta)
            \label{eqn:condition_for_tilde_p2}\\
    \tilde p_3 & = \alpha T + \beta S - P (\alpha + \beta)
            \label{eqn:condition_for_tilde_p3}\\
    \tilde p_4 & = \alpha P + \beta P - P (\alpha + \beta)
            \label{eqn:condition_for_tilde_p4}
\end{align}

Equation (\ref{eqn:condition_for_tilde_p4}) ensures that \(p_4=\tilde p_4=0\).
Equations (\ref{eqn:condition_for_tilde_p1}-\ref{eqn:condition_for_tilde_p3})
can be used to eliminate \(\alpha, \beta\), giving:

\begin{equation}\label{eqn:planar_definition_of_extortion}
    \tilde p_1 = \frac{(R - P)(\tilde p_2 + \tilde p_3)}{S + T - 2P}
\end{equation}

with:

\begin{equation}\label{eqn:definition_of_chi}
    \chi = \frac{\tilde p_2 (P - T) + \tilde p_3 (S - P)}
                {\tilde p_2 (P - S) + \tilde p_3 (T - P)}
\end{equation}

Given a strategy \(p\in\mathbb{R}^{4\times 1}\) equations
(\ref{eqn:condition_for_tilde_p4}), (\ref{eqn:planar_definition_of_extortion}-\ref{eqn:definition_of_chi}) can be used to check if
a strategy is extortionate. The conditions correspond to:

\begin{align}
    p_1 & = \frac{(R-P)(p_2 + p_3) - R + T + S - P}{S + T - 2P}
     \label{eqn:condition_for_p1}\\
    p_4 & = 0 \label{eqn:condition_for_p4}\\
    1 & > p_2 + p_3\label{eqn:condition_for_chi}
\end{align}

The algebraic steps necessary to prove these results are available in the
supporting materials.

All extortionate strategies reside on a triangular (\ref{eqn:condition_for_chi})
plane (\ref{eqn:condition_for_p1}) in 3 dimensions (\ref{eqn:condition_for_p4}).
Using this formulation it can be seen that a necessary (but not sufficient)
condition for an extortionate strategy is that it cooperates on average less
than 50\% of the time when in a state of disagreement with the opponent.

As an example, consider the known extortionate strategy \(p=(8 / 9, 1 / 2, 1 /
3, 0)\) from~\cite{Stewart2012} which is referred to as \texttt{Extort-2}. In
this case, for the standard values of \((R, T, S, P)\) constraint
(\ref{eqn:condition_for_p1}) corresponds to:

\begin{equation}
    p_1 = \frac{2(p_2 + p_3) + 1}{3}
\end{equation}

It is clear that in this case all constraints hold.

This approach could in fact be used to confirm that a given strategy is acting
in an extortionate manner even if it is not a memory one strategy. However, in
practice, if a closed form for \(p\) is not known, then due to measurement
and/or numerical error this would not work.

This problem can be written in the following linear algebraic form where
\(x=(\alpha, \beta)\)
and \(p^*=(\tilde p_1 - 1, tilde_2 - 1, p_3)\):

\begin{equation}\label{eqn:linear_algebraic_equation_for_p}
    Cx= p^*
\end{equation}

\(C\) corresponds to equations
(\ref{eqn:condition_for_tilde_p1}-\ref{eqn:condition_for_tilde_p3}) and is
given by:

\begin{equation}\label{eqn:definition_of_C}
    C =
    \begin{bmatrix}
        R - P & R- P \\
        S - P & T- P \\
        T - P & S- P \\
    \end{bmatrix}
\end{equation}

Note that in general, equation (\ref{eqn:linear_algebraic_equation_for_p}) will
not necessarily have a solution. From the Rouch\'{e}-Capelli theorem if there is
a solution it is unique as \(\text{rank}(C)=2\) which is the dimension of the
variable \(x\). The best fitting \(x\) is found by minimizing:

\begin{equation}\label{eqn:r_squared}
    \text{SSError} = \|C x- p^*\|_2^2 = \sum_{i=1}^{3}\left((C\bar x)_i-p_i^*\right)^2
\end{equation}

Note that \(\text{SSError}\), which is the square of the Frobenius
norm~\cite{Golub2013}, becomes a measure of how close a strategy is to being an
extortionate strategy. Suspicion
of extortion then corresponds to a threshold on \(\text{SSError}\).

By observing interactions (human or otherwise), their memory one representation
can be inferred and this approach can be used to recognise extortionate
behaviour. The notion of comparing theoretic and actual plays of the IPD is not
novel, see for example~\cite{Rand2013}. Immediately it is noted that if the
environment is noisy~\cite{Wu1995} then no strategy can be considered to be
extortionate as \(p_4>0\).

In the next section, this idea will be illustrated by observing the interactions
that take place in a computer based tournament of the IPD\@.

\section{Numerical experiments}\label{sec:numerical-experiments}

In~\cite{Stewart2012} results from a tournament with
\input{./assets/tex/number_of_stewart_plotkin_strategies/main.tex} strategies,
was presented with specific consideration given to ZD strategies. This
tournament is reproduced here using the Axelrod-Python
project~\cite{Knight2016}. To obtain a good measure of the corresponding
transition rates for each strategy all matches have been run for
\input{assets/tex/number_of_turns/main.tex} turns and every match has been
repeated \input{assets/tex/number_of_repetitions/main.tex} times. All of this
interaction data is available at~\cite{vincent_knight_2018_1297075}. A good
match between the inferred Markov chain and the state distribution of the actual
interactions has been verified. Data for this is presented in the supplementary
materials.

Figure~\ref{fig:SSError_overall_in_stewart_plotkin} shows the \(\text{SSError}\)
values for all the strategies in the tournament, as reported
in~\cite{Stewart2012} the extortionate strategy (which has an expected
\(\text{SSError}\) approximately 0) gains a large number of wins.

\begin{figure}[!htbp]
    \centering
    \includegraphics[width=.8\textwidth]{./assets/img/SSError_overall_in_stewart_plotkin/main.pdf}
    \caption{\(\text{SSError}\) and state probabilities for the strategies
        of~\cite{Stewart2012}, ordered both by number of wins and overall score.
        Note that \(P(DC)\) is not shown as it corresponds to the transpose of
        \(P(CD)\). Cooperator and Defector are omitted as they do not visit all
        the states.}
    \label{fig:SSError_overall_in_stewart_plotkin}
\end{figure}

Here, the work of~\cite{Stewart2012} is extended by investigating a tournament
with \input{assets/tex/number_of_full_strategies/main.tex}
strategies.

The results of this analysis are shown in
Figure~\ref{fig:SSError_and_probabilities_in_full}. The top ranking strategies
by number of wins seem to be extortionate (but not against all strategies) and
it can be seen that a small sub group of strategies achieve mutual defection.
All the top ranking strategies according to score achieve mutual cooperation and
do not extort each other, however they
\textbf{do} exhibit extortionate behaviour towards a number of the lower ranking
strategies.

\begin{figure}[!htbp]
    \centering
    \includegraphics[width=.8\textwidth]{./assets/img/SSError_and_probabilities_in_full/main.pdf}
    \caption{\(\text{SSError}\) for the strategies for the full tournament. Only
    strategy interactions for which \(p_4=0\) and \(\chi>1\) are displayed.}
    \label{fig:SSError_and_probabilities_in_full}
\end{figure}

\section{Conclusion}\label{sec:conclusion}

This work defines an approach to measure whether or not a player is playing a
strategy that corresponds to an extortionate strategy as defined
in~\cite{Press2012}: a mathematical model for suspicion. Indeed, all
extortionate strategies have been
 classified as lying on a triangular plane.
This rigorous classification fails to be robust to small measurement error, thus
a statistical approach is proposed.
This is done through a linear algebraic approach for approximating the solution
of a linear system. Using this, a large number of pairwise interactions is
simulated and in fact very few strategies are found to act extortionately.

The work of~\cite{Press2012}, whilst showing that a clever approach to taking
advantage of another memory one strategy exists: this is incomplete. Whilst the
elegance of this result is very attractive, just as the simplicity of the
victory of Tit For Tat in Axelrod's original tournaments was, it is incomplete.
Extortionate strategies achieve a high number of wins but they do not
achieve a high score which corresponds to the fitness landscape in an
evolutionary sense. From the large number of interactions a payoff matrix \(S\)
can be measured where \(S_{ij}\) denotes the score (using standard values of
\((R, S, T, P) = (3, 0, 5, 1)\)) of the \(i\)th strategy
against the \(j\)th strategy. Using this, the replicator equation
describes the evolution of the system based on a population density fitness
function:

\begin{equation}\label{eqn:replicator_dynamics}
    \frac{dx}{dt} = x(S-x^TS x)
\end{equation}

Equation (\ref{eqn:replicator_dynamics}) is solved numerically through an
integration technique described in~\cite{Petzold1983} and
Figure~\ref{fig:replicator_dynamics} shows the evolution of the distribution of
the system: the various strategies are ranked by scores. It is clear to see that
only the high ranking strategies survive the evolutionary process (in fact,
only \input{./assets/img/replicator_dynamics/main.tex}
have a final distribution greater than \(10 ^ {-2}\)). This confirms the
findings of~\cite{Moran1707} in which sophisticated strategies resist
evolutionary invasion of shorter memory strategies. Recalling
Figure~\ref{fig:SSError_and_probabilities_in_full} this demonstrates that:

\begin{itemize}
    \item Cooperation emerges through the evolutionary process: the high scoring
        strategies do not exhibit extortionate behaviour towards each other.
    \item Extortionate strategies do not survive the evolutionary process.
\end{itemize}

\begin{figure}[!htbp]
    \centering
    \includegraphics[width=.8\textwidth]{./assets/img/replicator_dynamics/main.pdf}
    \caption{Numerical simulation of the replicator equation
    (\ref{eqn:replicator_dynamics}): strategies are ordered by score, only the strategies with a high score survive the evolutionary process.}
    \label{fig:replicator_dynamics}
\end{figure}

This work can be used to classify plays of the IPD\@: data can be collected from
actual interactions (in lab or in the field). Furthermore, this allows for a
classification method similar to the notion of fingerprinting presented
in~\cite{Ashlock2008}. Trained strategies can potentially be classified as
extortionate or not or it could be possible to even constrain the reinforcement
learning approaches that are becoming prevalent in the literature.
Alternatively, this mathematical approach for recognising extortion could be
used in sophisticated strategies to defend against invasion. Arguably, some of
the strategies considered here exhibit this behaviour, indeed as described
in~\cite{Harper2017}, the top ranking strategies in the full tournament are
obtained using evolutionary reinforcement learning techniques, thus, suspicion
of extortionate behaviour could in fact be an evolutionary trait.

\section*{Acknowledgements}

The following open source software libraries were used in this research:

\begin{itemize}
    \item The Axelrod ~\cite{Knight2016, Knight2018} library (IPD strategies and
        tournaments).
    \item The sympy library~\cite{Meurer2017} (verification of all symbolic
        calculations).
    \item The matplotlib~\cite{Droettboom2018} library (visualisation).
    \item The pandas~\cite{Structures2010}, dask~\cite{Dask2016} and
        NumPy~\cite{Oliphant2015} libraries (data manipulation).
    \item The SciPy~\cite{Jones2001} library (numerical integration of the
        replicator equation).
\end{itemize}

This work was performed using the computational facilities of the Advanced
Research Computing @ Cardiff (ARCCA) Division, Cardiff University.

\printbibliography

\newpage
\section*{Supplementary materials}

\includepdf{assets/pdf/proof_of_form_of_extortionate_strategies/main.pdf}

\newpage

Using the pair wise interactions the transition rates \(p,
q\) can be measured and the steady state probabilities inferred and compared to
the actual probabilities of each state.
This is done numerically by computing the singular eigenvector of the
matrix \(A\) \cite{Stewart2009}:

\[
    A =
    \begin{bmatrix}
        p_1 q_1 & p_1 (1 - q_1) & (1 - p_1) q_1 & (1 -p_1) (1 - q_1) \\
        p_2 q_2 & p_2 (1 - q_2) & (1 - p_2) q_2 & (1 -p_2) (1 - q_2) \\
        p_3 q_3 & p_3 (1 - q_3) & (1 - p_3) q_3 & (1 -p_3) (1 - q_3) \\
        p_4 q_4 & p_4 (1 - q_4) & (1 - p_4) q_4 & (1 -p_4) (1 - q_4) \\
    \end{bmatrix}
\]

Figure~\ref{fig:computed_probabilities_vs_theoretic_probabilities} shows a
regression line fitted to every pairwise interaction with a reported
\(\text{SSError}\) value (pairwise interactions with missing states were
omitted). This serves to validate the approach: a part from some edge cases the
relationship is consistent.

\begin{figure}[!htbp]
    \centering
    \includegraphics[width=.8\textwidth]{./assets/img/computed_probabilities_vs_theoretic_probabilities/main.pdf}
    \caption{The
        relationship between the steady state probabilities inferred from the
        measured transitions and the actual steady state probabilities. A linear
        regression line is included validating the approach.}
    \label{fig:computed_probabilities_vs_theoretic_probabilities}
\end{figure}


\end{document}

    strategies is considered. In this setting
    the most highly performing strategies do not play in an extortionate way
    against each other but do against lower performing strategies.
    This suggests that whilst the theory of Zero Determinant strategies
    indicates that memory is not of fundamental importance to the evolution of
    cooperative behaviour, this is incomplete.
\end{abstract}

\section{Introduction}\label{sec:introduction}

Agent based game theoretic models have become a stalwart of the underpinning
mathematics of interactive behaviours. One of the major pieces of work
in this area is the pair of original computer tournaments run by Robert
Axelrod~\cite{Axelrod1980, Axelrod1980a}. These tournaments pitted submitted
computer strategies against each other in plays of the Iterated Prisoner's
Dilemma. A common game where agents can choose to pay a slight cost to their
immediate utility in the hope of building a reputation. This has been used in
economic and evolutionary game theory to understand the evolution of cooperative
behaviour.

Recently, a class of strategies was described in~\cite{Press2012} that can
provably extort any given opponent. In~\cite{Hilbe2013, Moran1707} some
questions have already been asked about the true effectiveness of these
strategies in an evolutionary setting. Here another question is asked: is it
possible to recognise this extortionate behaviour? A mathematical procedure for
suspicion is presented: in the same way that the continued actions of an
extortionate individual might raise suspicion.

This work makes use of the Axelrod Python library~\cite{Knight2018, Knight2016}
with a large number of Prisoner Dilemma strategies available to give an
extensive numerical example of the ideas presented.  The approach is presented
in Section~\ref{sec:delta-zd-strategies}.  All of the code and data discussed
in Section~\ref{sec:numerical-experiments} is open sourced, archived and
written according to best scientific principles~\cite{Wilson2014}. The data
archive can be found at~\cite{vincent_knight_2018_1297075}.

\section{Recognising Extortion}\label{sec:delta-zd-strategies}

In~\cite{Press2012}, given a match between 2 memory-one strategies, the concept
of Zero Determinant (ZD) strategies is introduced. The main result of that paper
shows that given two memory one players \(p, q\in\mathbb{R}^4\) a linear
relationship between the players' scores could be forced by one of the players.

Using the notation of~\cite{Press2012}, assuming the utilities for player \(p\)
are given by \(S_x=(R, S, T, P)\) and for player \(q\) by \(S_y=(R, T, S, P)\)
and that the stationary scores of each player is given by \(S_X\) and \(S_Y\)
respectively. The main result of~\cite{Press2012} is that if

\begin{equation}\label{eqn:linear_relationship_for_p}
    \tilde p=\alpha S_x + \beta S_y + \gamma
\end{equation}

or

\begin{equation}\label{eqn:linear_relationship_for_q}
    \tilde q=\alpha S_x + \beta S_y + \gamma
\end{equation}

where \(\tilde p = (1 - p_1, 1 - p_2, p_3, p_4)\) and
\(\tilde q = (1 - q_1, 1 - q_2, q_3, q_4)\) then:

\begin{equation}
    \alpha S_X + \beta S_Y + \gamma = 0
\end{equation}

In~\cite{Press2012} a particular type of ZD strategy is defined: extortionate
strategies. If:

\begin{equation}\label{eqn:constraint_for_extortion}
    \gamma = - P(\alpha + \beta)
\end{equation}

then the player can ensure they get a score \(\chi\) times
larger than the opponent. This extortion coefficient is given by:

\begin{equation}\label{eqn:definition_of_chi}
    \chi=\frac{-\beta}{\alpha}
\end{equation}

Thus, if (\ref{eqn:constraint_for_extortion}) holds and \(\chi >1\) a player is
said to extort their opponent.
Here, the reverse problem is considered: given a
\(p\in\mathbb{R}^4\) how does one identify \(\alpha, \beta\) if they
exist and is the strategy in fact acting in an extortionate way?

These conditions correspond to:

\begin{align}
    \tilde p_1 & = \alpha R + \beta R - P (\alpha + \beta)
            \label{eqn:condition_for_tilde_p1}\\
    \tilde p_2 & = \alpha S + \beta T - P (\alpha + \beta)
            \label{eqn:condition_for_tilde_p2}\\
    \tilde p_3 & = \alpha T + \beta S - P (\alpha + \beta)
            \label{eqn:condition_for_tilde_p3}\\
    \tilde p_4 & = \alpha P + \beta P - P (\alpha + \beta)
            \label{eqn:condition_for_tilde_p4}
\end{align}

Equation (\ref{eqn:condition_for_tilde_p4}) ensures that \(p_4=\tilde p_4=0\).
Equations (\ref{eqn:condition_for_tilde_p1}-\ref{eqn:condition_for_tilde_p3})
can be used to eliminate \(\alpha, \beta\), giving:

\begin{equation}\label{eqn:planar_definition_of_extortion}
    \tilde p_1 = \frac{(R - P)(\tilde p_2 + \tilde p_3)}{S + T - 2P}
\end{equation}

with:

\begin{equation}\label{eqn:definition_of_chi}
    \chi = \frac{\tilde p_2 (P - T) + \tilde p_3 (S - P)}
                {\tilde p_2 (P - S) + \tilde p_3 (T - P)}
\end{equation}

Given a strategy \(p\in\mathbb{R}^{4\times 1}\) equations
(\ref{eqn:condition_for_tilde_p4}), (\ref{eqn:planar_definition_of_extortion}-\ref{eqn:definition_of_chi}) can be used to check if
a strategy is extortionate. The conditions correspond to:

\begin{align}
    p_1 & = \frac{(R-P)(p_2 + p_3) - R + T + S - P}{S + T - 2P}
     \label{eqn:condition_for_p1}\\
    p_4 & = 0 \label{eqn:condition_for_p4}\\
    1 & > p_2 + p_3\label{eqn:condition_for_chi}
\end{align}

The algebraic steps necessary to prove these results are available in the
supporting materials.

All extortionate strategies reside on a triangular (\ref{eqn:condition_for_chi})
plane (\ref{eqn:condition_for_p1}) in 3 dimensions (\ref{eqn:condition_for_p4}).
Using this formulation it can be seen that a necessary (but not sufficient)
condition for an extortionate strategy is that it cooperates on average less
than 50\% of the time when in a state of disagreement with the opponent.

As an example, consider the known extortionate strategy \(p=(8 / 9, 1 / 2, 1 /
3, 0)\) from~\cite{Stewart2012} which is referred to as \texttt{Extort-2}. In
this case, for the standard values of \((R, T, S, P)\) constraint
(\ref{eqn:condition_for_p1}) corresponds to:

\begin{equation}
    p_1 = \frac{2(p_2 + p_3) + 1}{3}
\end{equation}

It is clear that in this case all constraints hold.

This approach could in fact be used to confirm that a given strategy is acting
in an extortionate manner even if it is not a memory one strategy. However, in
practice, if a closed form for \(p\) is not known, then due to measurement
and/or numerical error this would not work.

This problem can be written in the following linear algebraic form where
\(x=(\alpha, \beta)\)
and \(p^*=(\tilde p_1 - 1, tilde_2 - 1, p_3)\):

\begin{equation}\label{eqn:linear_algebraic_equation_for_p}
    Cx= p^*
\end{equation}

\(C\) corresponds to equations
(\ref{eqn:condition_for_tilde_p1}-\ref{eqn:condition_for_tilde_p3}) and is
given by:

\begin{equation}\label{eqn:definition_of_C}
    C =
    \begin{bmatrix}
        R - P & R- P \\
        S - P & T- P \\
        T - P & S- P \\
    \end{bmatrix}
\end{equation}

Note that in general, equation (\ref{eqn:linear_algebraic_equation_for_p}) will
not necessarily have a solution. From the Rouch\'{e}-Capelli theorem if there is
a solution it is unique as \(\text{rank}(C)=2\) which is the dimension of the
variable \(x\). The best fitting \(x\) is found by minimizing:

\begin{equation}\label{eqn:r_squared}
    \text{SSError} = \|C x- p^*\|_2^2 = \sum_{i=1}^{3}\left((C\bar x)_i-p_i^*\right)^2
\end{equation}

Note that \(\text{SSError}\), which is the square of the Frobenius
norm~\cite{Golub2013}, becomes a measure of how close a strategy is to being an
extortionate strategy. Suspicion
of extortion then corresponds to a threshold on \(\text{SSError}\).

By observing interactions (human or otherwise), their memory one representation
can be inferred and this approach can be used to recognise extortionate
behaviour. The notion of comparing theoretic and actual plays of the IPD is not
novel, see for example~\cite{Rand2013}. Immediately it is noted that if the
environment is noisy~\cite{Wu1995} then no strategy can be considered to be
extortionate as \(p_4>0\).

In the next section, this idea will be illustrated by observing the interactions
that take place in a computer based tournament of the IPD\@.

\section{Numerical experiments}\label{sec:numerical-experiments}

In~\cite{Stewart2012} results from a tournament with
\documentclass[a4paper]{article}

\usepackage{amsmath}
\usepackage{amssymb}
\usepackage[margin=1.5cm,
            includefoot,
            footskip=30pt]{geometry}
\usepackage{layout}
\usepackage{graphicx}
\usepackage{subcaption}

\usepackage{biblatex}
\usepackage{pdfpages}

\bibliography{main.bib}

\title{Suspicion: Recognising and evaluating the effectiveness
       of extortion in the Iterated Prisoner's Dilemma}
\author{Vincent A. Knight \and Nikoleta E. Glynatsi}
\date{\today}



\begin{document}

\maketitle

\begin{abstract}
    The Iterated Prisoner's Dilemma is a model for rational and evolutionary
    interactive behaviour. It has applications both in the study of human social
    behaviour as well as in biology.
    It is used to understand when and how a rational individual might
    accept an immediate cost to their own utility for the direct benefit of
    another.

    Much attention has been given to a class of strategies called
    Zero Determinant strategies. It has been theoretically shown that these
    strategies can ``extort'' any player.

    In this work, an approach to identify if observed strategies are playing in
    an extortionate way is described. Furthermore, experimental analysis of
    a large tournament with \input{assets/tex/number_of_full_strategies/main.tex}
    strategies is considered. In this setting
    the most highly performing strategies do not play in an extortionate way
    against each other but do against lower performing strategies.
    This suggests that whilst the theory of Zero Determinant strategies
    indicates that memory is not of fundamental importance to the evolution of
    cooperative behaviour, this is incomplete.
\end{abstract}

\section{Introduction}\label{sec:introduction}

Agent based game theoretic models have become a stalwart of the underpinning
mathematics of interactive behaviours. One of the major pieces of work
in this area is the pair of original computer tournaments run by Robert
Axelrod~\cite{Axelrod1980, Axelrod1980a}. These tournaments pitted submitted
computer strategies against each other in plays of the Iterated Prisoner's
Dilemma. A common game where agents can choose to pay a slight cost to their
immediate utility in the hope of building a reputation. This has been used in
economic and evolutionary game theory to understand the evolution of cooperative
behaviour.

Recently, a class of strategies was described in~\cite{Press2012} that can
provably extort any given opponent. In~\cite{Hilbe2013, Moran1707} some
questions have already been asked about the true effectiveness of these
strategies in an evolutionary setting. Here another question is asked: is it
possible to recognise this extortionate behaviour? A mathematical procedure for
suspicion is presented: in the same way that the continued actions of an
extortionate individual might raise suspicion.

This work makes use of the Axelrod Python library~\cite{Knight2018, Knight2016}
with a large number of Prisoner Dilemma strategies available to give an
extensive numerical example of the ideas presented.  The approach is presented
in Section~\ref{sec:delta-zd-strategies}.  All of the code and data discussed
in Section~\ref{sec:numerical-experiments} is open sourced, archived and
written according to best scientific principles~\cite{Wilson2014}. The data
archive can be found at~\cite{vincent_knight_2018_1297075}.

\section{Recognising Extortion}\label{sec:delta-zd-strategies}

In~\cite{Press2012}, given a match between 2 memory-one strategies, the concept
of Zero Determinant (ZD) strategies is introduced. The main result of that paper
shows that given two memory one players \(p, q\in\mathbb{R}^4\) a linear
relationship between the players' scores could be forced by one of the players.

Using the notation of~\cite{Press2012}, assuming the utilities for player \(p\)
are given by \(S_x=(R, S, T, P)\) and for player \(q\) by \(S_y=(R, T, S, P)\)
and that the stationary scores of each player is given by \(S_X\) and \(S_Y\)
respectively. The main result of~\cite{Press2012} is that if

\begin{equation}\label{eqn:linear_relationship_for_p}
    \tilde p=\alpha S_x + \beta S_y + \gamma
\end{equation}

or

\begin{equation}\label{eqn:linear_relationship_for_q}
    \tilde q=\alpha S_x + \beta S_y + \gamma
\end{equation}

where \(\tilde p = (1 - p_1, 1 - p_2, p_3, p_4)\) and
\(\tilde q = (1 - q_1, 1 - q_2, q_3, q_4)\) then:

\begin{equation}
    \alpha S_X + \beta S_Y + \gamma = 0
\end{equation}

In~\cite{Press2012} a particular type of ZD strategy is defined: extortionate
strategies. If:

\begin{equation}\label{eqn:constraint_for_extortion}
    \gamma = - P(\alpha + \beta)
\end{equation}

then the player can ensure they get a score \(\chi\) times
larger than the opponent. This extortion coefficient is given by:

\begin{equation}\label{eqn:definition_of_chi}
    \chi=\frac{-\beta}{\alpha}
\end{equation}

Thus, if (\ref{eqn:constraint_for_extortion}) holds and \(\chi >1\) a player is
said to extort their opponent.
Here, the reverse problem is considered: given a
\(p\in\mathbb{R}^4\) how does one identify \(\alpha, \beta\) if they
exist and is the strategy in fact acting in an extortionate way?

These conditions correspond to:

\begin{align}
    \tilde p_1 & = \alpha R + \beta R - P (\alpha + \beta)
            \label{eqn:condition_for_tilde_p1}\\
    \tilde p_2 & = \alpha S + \beta T - P (\alpha + \beta)
            \label{eqn:condition_for_tilde_p2}\\
    \tilde p_3 & = \alpha T + \beta S - P (\alpha + \beta)
            \label{eqn:condition_for_tilde_p3}\\
    \tilde p_4 & = \alpha P + \beta P - P (\alpha + \beta)
            \label{eqn:condition_for_tilde_p4}
\end{align}

Equation (\ref{eqn:condition_for_tilde_p4}) ensures that \(p_4=\tilde p_4=0\).
Equations (\ref{eqn:condition_for_tilde_p1}-\ref{eqn:condition_for_tilde_p3})
can be used to eliminate \(\alpha, \beta\), giving:

\begin{equation}\label{eqn:planar_definition_of_extortion}
    \tilde p_1 = \frac{(R - P)(\tilde p_2 + \tilde p_3)}{S + T - 2P}
\end{equation}

with:

\begin{equation}\label{eqn:definition_of_chi}
    \chi = \frac{\tilde p_2 (P - T) + \tilde p_3 (S - P)}
                {\tilde p_2 (P - S) + \tilde p_3 (T - P)}
\end{equation}

Given a strategy \(p\in\mathbb{R}^{4\times 1}\) equations
(\ref{eqn:condition_for_tilde_p4}), (\ref{eqn:planar_definition_of_extortion}-\ref{eqn:definition_of_chi}) can be used to check if
a strategy is extortionate. The conditions correspond to:

\begin{align}
    p_1 & = \frac{(R-P)(p_2 + p_3) - R + T + S - P}{S + T - 2P}
     \label{eqn:condition_for_p1}\\
    p_4 & = 0 \label{eqn:condition_for_p4}\\
    1 & > p_2 + p_3\label{eqn:condition_for_chi}
\end{align}

The algebraic steps necessary to prove these results are available in the
supporting materials.

All extortionate strategies reside on a triangular (\ref{eqn:condition_for_chi})
plane (\ref{eqn:condition_for_p1}) in 3 dimensions (\ref{eqn:condition_for_p4}).
Using this formulation it can be seen that a necessary (but not sufficient)
condition for an extortionate strategy is that it cooperates on average less
than 50\% of the time when in a state of disagreement with the opponent.

As an example, consider the known extortionate strategy \(p=(8 / 9, 1 / 2, 1 /
3, 0)\) from~\cite{Stewart2012} which is referred to as \texttt{Extort-2}. In
this case, for the standard values of \((R, T, S, P)\) constraint
(\ref{eqn:condition_for_p1}) corresponds to:

\begin{equation}
    p_1 = \frac{2(p_2 + p_3) + 1}{3}
\end{equation}

It is clear that in this case all constraints hold.

This approach could in fact be used to confirm that a given strategy is acting
in an extortionate manner even if it is not a memory one strategy. However, in
practice, if a closed form for \(p\) is not known, then due to measurement
and/or numerical error this would not work.

This problem can be written in the following linear algebraic form where
\(x=(\alpha, \beta)\)
and \(p^*=(\tilde p_1 - 1, tilde_2 - 1, p_3)\):

\begin{equation}\label{eqn:linear_algebraic_equation_for_p}
    Cx= p^*
\end{equation}

\(C\) corresponds to equations
(\ref{eqn:condition_for_tilde_p1}-\ref{eqn:condition_for_tilde_p3}) and is
given by:

\begin{equation}\label{eqn:definition_of_C}
    C =
    \begin{bmatrix}
        R - P & R- P \\
        S - P & T- P \\
        T - P & S- P \\
    \end{bmatrix}
\end{equation}

Note that in general, equation (\ref{eqn:linear_algebraic_equation_for_p}) will
not necessarily have a solution. From the Rouch\'{e}-Capelli theorem if there is
a solution it is unique as \(\text{rank}(C)=2\) which is the dimension of the
variable \(x\). The best fitting \(x\) is found by minimizing:

\begin{equation}\label{eqn:r_squared}
    \text{SSError} = \|C x- p^*\|_2^2 = \sum_{i=1}^{3}\left((C\bar x)_i-p_i^*\right)^2
\end{equation}

Note that \(\text{SSError}\), which is the square of the Frobenius
norm~\cite{Golub2013}, becomes a measure of how close a strategy is to being an
extortionate strategy. Suspicion
of extortion then corresponds to a threshold on \(\text{SSError}\).

By observing interactions (human or otherwise), their memory one representation
can be inferred and this approach can be used to recognise extortionate
behaviour. The notion of comparing theoretic and actual plays of the IPD is not
novel, see for example~\cite{Rand2013}. Immediately it is noted that if the
environment is noisy~\cite{Wu1995} then no strategy can be considered to be
extortionate as \(p_4>0\).

In the next section, this idea will be illustrated by observing the interactions
that take place in a computer based tournament of the IPD\@.

\section{Numerical experiments}\label{sec:numerical-experiments}

In~\cite{Stewart2012} results from a tournament with
\input{./assets/tex/number_of_stewart_plotkin_strategies/main.tex} strategies,
was presented with specific consideration given to ZD strategies. This
tournament is reproduced here using the Axelrod-Python
project~\cite{Knight2016}. To obtain a good measure of the corresponding
transition rates for each strategy all matches have been run for
\input{assets/tex/number_of_turns/main.tex} turns and every match has been
repeated \input{assets/tex/number_of_repetitions/main.tex} times. All of this
interaction data is available at~\cite{vincent_knight_2018_1297075}. A good
match between the inferred Markov chain and the state distribution of the actual
interactions has been verified. Data for this is presented in the supplementary
materials.

Figure~\ref{fig:SSError_overall_in_stewart_plotkin} shows the \(\text{SSError}\)
values for all the strategies in the tournament, as reported
in~\cite{Stewart2012} the extortionate strategy (which has an expected
\(\text{SSError}\) approximately 0) gains a large number of wins.

\begin{figure}[!htbp]
    \centering
    \includegraphics[width=.8\textwidth]{./assets/img/SSError_overall_in_stewart_plotkin/main.pdf}
    \caption{\(\text{SSError}\) and state probabilities for the strategies
        of~\cite{Stewart2012}, ordered both by number of wins and overall score.
        Note that \(P(DC)\) is not shown as it corresponds to the transpose of
        \(P(CD)\). Cooperator and Defector are omitted as they do not visit all
        the states.}
    \label{fig:SSError_overall_in_stewart_plotkin}
\end{figure}

Here, the work of~\cite{Stewart2012} is extended by investigating a tournament
with \input{assets/tex/number_of_full_strategies/main.tex}
strategies.

The results of this analysis are shown in
Figure~\ref{fig:SSError_and_probabilities_in_full}. The top ranking strategies
by number of wins seem to be extortionate (but not against all strategies) and
it can be seen that a small sub group of strategies achieve mutual defection.
All the top ranking strategies according to score achieve mutual cooperation and
do not extort each other, however they
\textbf{do} exhibit extortionate behaviour towards a number of the lower ranking
strategies.

\begin{figure}[!htbp]
    \centering
    \includegraphics[width=.8\textwidth]{./assets/img/SSError_and_probabilities_in_full/main.pdf}
    \caption{\(\text{SSError}\) for the strategies for the full tournament. Only
    strategy interactions for which \(p_4=0\) and \(\chi>1\) are displayed.}
    \label{fig:SSError_and_probabilities_in_full}
\end{figure}

\section{Conclusion}\label{sec:conclusion}

This work defines an approach to measure whether or not a player is playing a
strategy that corresponds to an extortionate strategy as defined
in~\cite{Press2012}: a mathematical model for suspicion. Indeed, all
extortionate strategies have been
 classified as lying on a triangular plane.
This rigorous classification fails to be robust to small measurement error, thus
a statistical approach is proposed.
This is done through a linear algebraic approach for approximating the solution
of a linear system. Using this, a large number of pairwise interactions is
simulated and in fact very few strategies are found to act extortionately.

The work of~\cite{Press2012}, whilst showing that a clever approach to taking
advantage of another memory one strategy exists: this is incomplete. Whilst the
elegance of this result is very attractive, just as the simplicity of the
victory of Tit For Tat in Axelrod's original tournaments was, it is incomplete.
Extortionate strategies achieve a high number of wins but they do not
achieve a high score which corresponds to the fitness landscape in an
evolutionary sense. From the large number of interactions a payoff matrix \(S\)
can be measured where \(S_{ij}\) denotes the score (using standard values of
\((R, S, T, P) = (3, 0, 5, 1)\)) of the \(i\)th strategy
against the \(j\)th strategy. Using this, the replicator equation
describes the evolution of the system based on a population density fitness
function:

\begin{equation}\label{eqn:replicator_dynamics}
    \frac{dx}{dt} = x(S-x^TS x)
\end{equation}

Equation (\ref{eqn:replicator_dynamics}) is solved numerically through an
integration technique described in~\cite{Petzold1983} and
Figure~\ref{fig:replicator_dynamics} shows the evolution of the distribution of
the system: the various strategies are ranked by scores. It is clear to see that
only the high ranking strategies survive the evolutionary process (in fact,
only \input{./assets/img/replicator_dynamics/main.tex}
have a final distribution greater than \(10 ^ {-2}\)). This confirms the
findings of~\cite{Moran1707} in which sophisticated strategies resist
evolutionary invasion of shorter memory strategies. Recalling
Figure~\ref{fig:SSError_and_probabilities_in_full} this demonstrates that:

\begin{itemize}
    \item Cooperation emerges through the evolutionary process: the high scoring
        strategies do not exhibit extortionate behaviour towards each other.
    \item Extortionate strategies do not survive the evolutionary process.
\end{itemize}

\begin{figure}[!htbp]
    \centering
    \includegraphics[width=.8\textwidth]{./assets/img/replicator_dynamics/main.pdf}
    \caption{Numerical simulation of the replicator equation
    (\ref{eqn:replicator_dynamics}): strategies are ordered by score, only the strategies with a high score survive the evolutionary process.}
    \label{fig:replicator_dynamics}
\end{figure}

This work can be used to classify plays of the IPD\@: data can be collected from
actual interactions (in lab or in the field). Furthermore, this allows for a
classification method similar to the notion of fingerprinting presented
in~\cite{Ashlock2008}. Trained strategies can potentially be classified as
extortionate or not or it could be possible to even constrain the reinforcement
learning approaches that are becoming prevalent in the literature.
Alternatively, this mathematical approach for recognising extortion could be
used in sophisticated strategies to defend against invasion. Arguably, some of
the strategies considered here exhibit this behaviour, indeed as described
in~\cite{Harper2017}, the top ranking strategies in the full tournament are
obtained using evolutionary reinforcement learning techniques, thus, suspicion
of extortionate behaviour could in fact be an evolutionary trait.

\section*{Acknowledgements}

The following open source software libraries were used in this research:

\begin{itemize}
    \item The Axelrod ~\cite{Knight2016, Knight2018} library (IPD strategies and
        tournaments).
    \item The sympy library~\cite{Meurer2017} (verification of all symbolic
        calculations).
    \item The matplotlib~\cite{Droettboom2018} library (visualisation).
    \item The pandas~\cite{Structures2010}, dask~\cite{Dask2016} and
        NumPy~\cite{Oliphant2015} libraries (data manipulation).
    \item The SciPy~\cite{Jones2001} library (numerical integration of the
        replicator equation).
\end{itemize}

This work was performed using the computational facilities of the Advanced
Research Computing @ Cardiff (ARCCA) Division, Cardiff University.

\printbibliography

\newpage
\section*{Supplementary materials}

\includepdf{assets/pdf/proof_of_form_of_extortionate_strategies/main.pdf}

\newpage

Using the pair wise interactions the transition rates \(p,
q\) can be measured and the steady state probabilities inferred and compared to
the actual probabilities of each state.
This is done numerically by computing the singular eigenvector of the
matrix \(A\) \cite{Stewart2009}:

\[
    A =
    \begin{bmatrix}
        p_1 q_1 & p_1 (1 - q_1) & (1 - p_1) q_1 & (1 -p_1) (1 - q_1) \\
        p_2 q_2 & p_2 (1 - q_2) & (1 - p_2) q_2 & (1 -p_2) (1 - q_2) \\
        p_3 q_3 & p_3 (1 - q_3) & (1 - p_3) q_3 & (1 -p_3) (1 - q_3) \\
        p_4 q_4 & p_4 (1 - q_4) & (1 - p_4) q_4 & (1 -p_4) (1 - q_4) \\
    \end{bmatrix}
\]

Figure~\ref{fig:computed_probabilities_vs_theoretic_probabilities} shows a
regression line fitted to every pairwise interaction with a reported
\(\text{SSError}\) value (pairwise interactions with missing states were
omitted). This serves to validate the approach: a part from some edge cases the
relationship is consistent.

\begin{figure}[!htbp]
    \centering
    \includegraphics[width=.8\textwidth]{./assets/img/computed_probabilities_vs_theoretic_probabilities/main.pdf}
    \caption{The
        relationship between the steady state probabilities inferred from the
        measured transitions and the actual steady state probabilities. A linear
        regression line is included validating the approach.}
    \label{fig:computed_probabilities_vs_theoretic_probabilities}
\end{figure}


\end{document}
 strategies,
was presented with specific consideration given to ZD strategies. This
tournament is reproduced here using the Axelrod-Python
project~\cite{Knight2016}. To obtain a good measure of the corresponding
transition rates for each strategy all matches have been run for
\documentclass[a4paper]{article}

\usepackage{amsmath}
\usepackage{amssymb}
\usepackage[margin=1.5cm,
            includefoot,
            footskip=30pt]{geometry}
\usepackage{layout}
\usepackage{graphicx}
\usepackage{subcaption}

\usepackage{biblatex}
\usepackage{pdfpages}

\bibliography{main.bib}

\title{Suspicion: Recognising and evaluating the effectiveness
       of extortion in the Iterated Prisoner's Dilemma}
\author{Vincent A. Knight \and Nikoleta E. Glynatsi}
\date{\today}



\begin{document}

\maketitle

\begin{abstract}
    The Iterated Prisoner's Dilemma is a model for rational and evolutionary
    interactive behaviour. It has applications both in the study of human social
    behaviour as well as in biology.
    It is used to understand when and how a rational individual might
    accept an immediate cost to their own utility for the direct benefit of
    another.

    Much attention has been given to a class of strategies called
    Zero Determinant strategies. It has been theoretically shown that these
    strategies can ``extort'' any player.

    In this work, an approach to identify if observed strategies are playing in
    an extortionate way is described. Furthermore, experimental analysis of
    a large tournament with \input{assets/tex/number_of_full_strategies/main.tex}
    strategies is considered. In this setting
    the most highly performing strategies do not play in an extortionate way
    against each other but do against lower performing strategies.
    This suggests that whilst the theory of Zero Determinant strategies
    indicates that memory is not of fundamental importance to the evolution of
    cooperative behaviour, this is incomplete.
\end{abstract}

\section{Introduction}\label{sec:introduction}

Agent based game theoretic models have become a stalwart of the underpinning
mathematics of interactive behaviours. One of the major pieces of work
in this area is the pair of original computer tournaments run by Robert
Axelrod~\cite{Axelrod1980, Axelrod1980a}. These tournaments pitted submitted
computer strategies against each other in plays of the Iterated Prisoner's
Dilemma. A common game where agents can choose to pay a slight cost to their
immediate utility in the hope of building a reputation. This has been used in
economic and evolutionary game theory to understand the evolution of cooperative
behaviour.

Recently, a class of strategies was described in~\cite{Press2012} that can
provably extort any given opponent. In~\cite{Hilbe2013, Moran1707} some
questions have already been asked about the true effectiveness of these
strategies in an evolutionary setting. Here another question is asked: is it
possible to recognise this extortionate behaviour? A mathematical procedure for
suspicion is presented: in the same way that the continued actions of an
extortionate individual might raise suspicion.

This work makes use of the Axelrod Python library~\cite{Knight2018, Knight2016}
with a large number of Prisoner Dilemma strategies available to give an
extensive numerical example of the ideas presented.  The approach is presented
in Section~\ref{sec:delta-zd-strategies}.  All of the code and data discussed
in Section~\ref{sec:numerical-experiments} is open sourced, archived and
written according to best scientific principles~\cite{Wilson2014}. The data
archive can be found at~\cite{vincent_knight_2018_1297075}.

\section{Recognising Extortion}\label{sec:delta-zd-strategies}

In~\cite{Press2012}, given a match between 2 memory-one strategies, the concept
of Zero Determinant (ZD) strategies is introduced. The main result of that paper
shows that given two memory one players \(p, q\in\mathbb{R}^4\) a linear
relationship between the players' scores could be forced by one of the players.

Using the notation of~\cite{Press2012}, assuming the utilities for player \(p\)
are given by \(S_x=(R, S, T, P)\) and for player \(q\) by \(S_y=(R, T, S, P)\)
and that the stationary scores of each player is given by \(S_X\) and \(S_Y\)
respectively. The main result of~\cite{Press2012} is that if

\begin{equation}\label{eqn:linear_relationship_for_p}
    \tilde p=\alpha S_x + \beta S_y + \gamma
\end{equation}

or

\begin{equation}\label{eqn:linear_relationship_for_q}
    \tilde q=\alpha S_x + \beta S_y + \gamma
\end{equation}

where \(\tilde p = (1 - p_1, 1 - p_2, p_3, p_4)\) and
\(\tilde q = (1 - q_1, 1 - q_2, q_3, q_4)\) then:

\begin{equation}
    \alpha S_X + \beta S_Y + \gamma = 0
\end{equation}

In~\cite{Press2012} a particular type of ZD strategy is defined: extortionate
strategies. If:

\begin{equation}\label{eqn:constraint_for_extortion}
    \gamma = - P(\alpha + \beta)
\end{equation}

then the player can ensure they get a score \(\chi\) times
larger than the opponent. This extortion coefficient is given by:

\begin{equation}\label{eqn:definition_of_chi}
    \chi=\frac{-\beta}{\alpha}
\end{equation}

Thus, if (\ref{eqn:constraint_for_extortion}) holds and \(\chi >1\) a player is
said to extort their opponent.
Here, the reverse problem is considered: given a
\(p\in\mathbb{R}^4\) how does one identify \(\alpha, \beta\) if they
exist and is the strategy in fact acting in an extortionate way?

These conditions correspond to:

\begin{align}
    \tilde p_1 & = \alpha R + \beta R - P (\alpha + \beta)
            \label{eqn:condition_for_tilde_p1}\\
    \tilde p_2 & = \alpha S + \beta T - P (\alpha + \beta)
            \label{eqn:condition_for_tilde_p2}\\
    \tilde p_3 & = \alpha T + \beta S - P (\alpha + \beta)
            \label{eqn:condition_for_tilde_p3}\\
    \tilde p_4 & = \alpha P + \beta P - P (\alpha + \beta)
            \label{eqn:condition_for_tilde_p4}
\end{align}

Equation (\ref{eqn:condition_for_tilde_p4}) ensures that \(p_4=\tilde p_4=0\).
Equations (\ref{eqn:condition_for_tilde_p1}-\ref{eqn:condition_for_tilde_p3})
can be used to eliminate \(\alpha, \beta\), giving:

\begin{equation}\label{eqn:planar_definition_of_extortion}
    \tilde p_1 = \frac{(R - P)(\tilde p_2 + \tilde p_3)}{S + T - 2P}
\end{equation}

with:

\begin{equation}\label{eqn:definition_of_chi}
    \chi = \frac{\tilde p_2 (P - T) + \tilde p_3 (S - P)}
                {\tilde p_2 (P - S) + \tilde p_3 (T - P)}
\end{equation}

Given a strategy \(p\in\mathbb{R}^{4\times 1}\) equations
(\ref{eqn:condition_for_tilde_p4}), (\ref{eqn:planar_definition_of_extortion}-\ref{eqn:definition_of_chi}) can be used to check if
a strategy is extortionate. The conditions correspond to:

\begin{align}
    p_1 & = \frac{(R-P)(p_2 + p_3) - R + T + S - P}{S + T - 2P}
     \label{eqn:condition_for_p1}\\
    p_4 & = 0 \label{eqn:condition_for_p4}\\
    1 & > p_2 + p_3\label{eqn:condition_for_chi}
\end{align}

The algebraic steps necessary to prove these results are available in the
supporting materials.

All extortionate strategies reside on a triangular (\ref{eqn:condition_for_chi})
plane (\ref{eqn:condition_for_p1}) in 3 dimensions (\ref{eqn:condition_for_p4}).
Using this formulation it can be seen that a necessary (but not sufficient)
condition for an extortionate strategy is that it cooperates on average less
than 50\% of the time when in a state of disagreement with the opponent.

As an example, consider the known extortionate strategy \(p=(8 / 9, 1 / 2, 1 /
3, 0)\) from~\cite{Stewart2012} which is referred to as \texttt{Extort-2}. In
this case, for the standard values of \((R, T, S, P)\) constraint
(\ref{eqn:condition_for_p1}) corresponds to:

\begin{equation}
    p_1 = \frac{2(p_2 + p_3) + 1}{3}
\end{equation}

It is clear that in this case all constraints hold.

This approach could in fact be used to confirm that a given strategy is acting
in an extortionate manner even if it is not a memory one strategy. However, in
practice, if a closed form for \(p\) is not known, then due to measurement
and/or numerical error this would not work.

This problem can be written in the following linear algebraic form where
\(x=(\alpha, \beta)\)
and \(p^*=(\tilde p_1 - 1, tilde_2 - 1, p_3)\):

\begin{equation}\label{eqn:linear_algebraic_equation_for_p}
    Cx= p^*
\end{equation}

\(C\) corresponds to equations
(\ref{eqn:condition_for_tilde_p1}-\ref{eqn:condition_for_tilde_p3}) and is
given by:

\begin{equation}\label{eqn:definition_of_C}
    C =
    \begin{bmatrix}
        R - P & R- P \\
        S - P & T- P \\
        T - P & S- P \\
    \end{bmatrix}
\end{equation}

Note that in general, equation (\ref{eqn:linear_algebraic_equation_for_p}) will
not necessarily have a solution. From the Rouch\'{e}-Capelli theorem if there is
a solution it is unique as \(\text{rank}(C)=2\) which is the dimension of the
variable \(x\). The best fitting \(x\) is found by minimizing:

\begin{equation}\label{eqn:r_squared}
    \text{SSError} = \|C x- p^*\|_2^2 = \sum_{i=1}^{3}\left((C\bar x)_i-p_i^*\right)^2
\end{equation}

Note that \(\text{SSError}\), which is the square of the Frobenius
norm~\cite{Golub2013}, becomes a measure of how close a strategy is to being an
extortionate strategy. Suspicion
of extortion then corresponds to a threshold on \(\text{SSError}\).

By observing interactions (human or otherwise), their memory one representation
can be inferred and this approach can be used to recognise extortionate
behaviour. The notion of comparing theoretic and actual plays of the IPD is not
novel, see for example~\cite{Rand2013}. Immediately it is noted that if the
environment is noisy~\cite{Wu1995} then no strategy can be considered to be
extortionate as \(p_4>0\).

In the next section, this idea will be illustrated by observing the interactions
that take place in a computer based tournament of the IPD\@.

\section{Numerical experiments}\label{sec:numerical-experiments}

In~\cite{Stewart2012} results from a tournament with
\input{./assets/tex/number_of_stewart_plotkin_strategies/main.tex} strategies,
was presented with specific consideration given to ZD strategies. This
tournament is reproduced here using the Axelrod-Python
project~\cite{Knight2016}. To obtain a good measure of the corresponding
transition rates for each strategy all matches have been run for
\input{assets/tex/number_of_turns/main.tex} turns and every match has been
repeated \input{assets/tex/number_of_repetitions/main.tex} times. All of this
interaction data is available at~\cite{vincent_knight_2018_1297075}. A good
match between the inferred Markov chain and the state distribution of the actual
interactions has been verified. Data for this is presented in the supplementary
materials.

Figure~\ref{fig:SSError_overall_in_stewart_plotkin} shows the \(\text{SSError}\)
values for all the strategies in the tournament, as reported
in~\cite{Stewart2012} the extortionate strategy (which has an expected
\(\text{SSError}\) approximately 0) gains a large number of wins.

\begin{figure}[!htbp]
    \centering
    \includegraphics[width=.8\textwidth]{./assets/img/SSError_overall_in_stewart_plotkin/main.pdf}
    \caption{\(\text{SSError}\) and state probabilities for the strategies
        of~\cite{Stewart2012}, ordered both by number of wins and overall score.
        Note that \(P(DC)\) is not shown as it corresponds to the transpose of
        \(P(CD)\). Cooperator and Defector are omitted as they do not visit all
        the states.}
    \label{fig:SSError_overall_in_stewart_plotkin}
\end{figure}

Here, the work of~\cite{Stewart2012} is extended by investigating a tournament
with \input{assets/tex/number_of_full_strategies/main.tex}
strategies.

The results of this analysis are shown in
Figure~\ref{fig:SSError_and_probabilities_in_full}. The top ranking strategies
by number of wins seem to be extortionate (but not against all strategies) and
it can be seen that a small sub group of strategies achieve mutual defection.
All the top ranking strategies according to score achieve mutual cooperation and
do not extort each other, however they
\textbf{do} exhibit extortionate behaviour towards a number of the lower ranking
strategies.

\begin{figure}[!htbp]
    \centering
    \includegraphics[width=.8\textwidth]{./assets/img/SSError_and_probabilities_in_full/main.pdf}
    \caption{\(\text{SSError}\) for the strategies for the full tournament. Only
    strategy interactions for which \(p_4=0\) and \(\chi>1\) are displayed.}
    \label{fig:SSError_and_probabilities_in_full}
\end{figure}

\section{Conclusion}\label{sec:conclusion}

This work defines an approach to measure whether or not a player is playing a
strategy that corresponds to an extortionate strategy as defined
in~\cite{Press2012}: a mathematical model for suspicion. Indeed, all
extortionate strategies have been
 classified as lying on a triangular plane.
This rigorous classification fails to be robust to small measurement error, thus
a statistical approach is proposed.
This is done through a linear algebraic approach for approximating the solution
of a linear system. Using this, a large number of pairwise interactions is
simulated and in fact very few strategies are found to act extortionately.

The work of~\cite{Press2012}, whilst showing that a clever approach to taking
advantage of another memory one strategy exists: this is incomplete. Whilst the
elegance of this result is very attractive, just as the simplicity of the
victory of Tit For Tat in Axelrod's original tournaments was, it is incomplete.
Extortionate strategies achieve a high number of wins but they do not
achieve a high score which corresponds to the fitness landscape in an
evolutionary sense. From the large number of interactions a payoff matrix \(S\)
can be measured where \(S_{ij}\) denotes the score (using standard values of
\((R, S, T, P) = (3, 0, 5, 1)\)) of the \(i\)th strategy
against the \(j\)th strategy. Using this, the replicator equation
describes the evolution of the system based on a population density fitness
function:

\begin{equation}\label{eqn:replicator_dynamics}
    \frac{dx}{dt} = x(S-x^TS x)
\end{equation}

Equation (\ref{eqn:replicator_dynamics}) is solved numerically through an
integration technique described in~\cite{Petzold1983} and
Figure~\ref{fig:replicator_dynamics} shows the evolution of the distribution of
the system: the various strategies are ranked by scores. It is clear to see that
only the high ranking strategies survive the evolutionary process (in fact,
only \input{./assets/img/replicator_dynamics/main.tex}
have a final distribution greater than \(10 ^ {-2}\)). This confirms the
findings of~\cite{Moran1707} in which sophisticated strategies resist
evolutionary invasion of shorter memory strategies. Recalling
Figure~\ref{fig:SSError_and_probabilities_in_full} this demonstrates that:

\begin{itemize}
    \item Cooperation emerges through the evolutionary process: the high scoring
        strategies do not exhibit extortionate behaviour towards each other.
    \item Extortionate strategies do not survive the evolutionary process.
\end{itemize}

\begin{figure}[!htbp]
    \centering
    \includegraphics[width=.8\textwidth]{./assets/img/replicator_dynamics/main.pdf}
    \caption{Numerical simulation of the replicator equation
    (\ref{eqn:replicator_dynamics}): strategies are ordered by score, only the strategies with a high score survive the evolutionary process.}
    \label{fig:replicator_dynamics}
\end{figure}

This work can be used to classify plays of the IPD\@: data can be collected from
actual interactions (in lab or in the field). Furthermore, this allows for a
classification method similar to the notion of fingerprinting presented
in~\cite{Ashlock2008}. Trained strategies can potentially be classified as
extortionate or not or it could be possible to even constrain the reinforcement
learning approaches that are becoming prevalent in the literature.
Alternatively, this mathematical approach for recognising extortion could be
used in sophisticated strategies to defend against invasion. Arguably, some of
the strategies considered here exhibit this behaviour, indeed as described
in~\cite{Harper2017}, the top ranking strategies in the full tournament are
obtained using evolutionary reinforcement learning techniques, thus, suspicion
of extortionate behaviour could in fact be an evolutionary trait.

\section*{Acknowledgements}

The following open source software libraries were used in this research:

\begin{itemize}
    \item The Axelrod ~\cite{Knight2016, Knight2018} library (IPD strategies and
        tournaments).
    \item The sympy library~\cite{Meurer2017} (verification of all symbolic
        calculations).
    \item The matplotlib~\cite{Droettboom2018} library (visualisation).
    \item The pandas~\cite{Structures2010}, dask~\cite{Dask2016} and
        NumPy~\cite{Oliphant2015} libraries (data manipulation).
    \item The SciPy~\cite{Jones2001} library (numerical integration of the
        replicator equation).
\end{itemize}

This work was performed using the computational facilities of the Advanced
Research Computing @ Cardiff (ARCCA) Division, Cardiff University.

\printbibliography

\newpage
\section*{Supplementary materials}

\includepdf{assets/pdf/proof_of_form_of_extortionate_strategies/main.pdf}

\newpage

Using the pair wise interactions the transition rates \(p,
q\) can be measured and the steady state probabilities inferred and compared to
the actual probabilities of each state.
This is done numerically by computing the singular eigenvector of the
matrix \(A\) \cite{Stewart2009}:

\[
    A =
    \begin{bmatrix}
        p_1 q_1 & p_1 (1 - q_1) & (1 - p_1) q_1 & (1 -p_1) (1 - q_1) \\
        p_2 q_2 & p_2 (1 - q_2) & (1 - p_2) q_2 & (1 -p_2) (1 - q_2) \\
        p_3 q_3 & p_3 (1 - q_3) & (1 - p_3) q_3 & (1 -p_3) (1 - q_3) \\
        p_4 q_4 & p_4 (1 - q_4) & (1 - p_4) q_4 & (1 -p_4) (1 - q_4) \\
    \end{bmatrix}
\]

Figure~\ref{fig:computed_probabilities_vs_theoretic_probabilities} shows a
regression line fitted to every pairwise interaction with a reported
\(\text{SSError}\) value (pairwise interactions with missing states were
omitted). This serves to validate the approach: a part from some edge cases the
relationship is consistent.

\begin{figure}[!htbp]
    \centering
    \includegraphics[width=.8\textwidth]{./assets/img/computed_probabilities_vs_theoretic_probabilities/main.pdf}
    \caption{The
        relationship between the steady state probabilities inferred from the
        measured transitions and the actual steady state probabilities. A linear
        regression line is included validating the approach.}
    \label{fig:computed_probabilities_vs_theoretic_probabilities}
\end{figure}


\end{document}
 turns and every match has been
repeated \documentclass[a4paper]{article}

\usepackage{amsmath}
\usepackage{amssymb}
\usepackage[margin=1.5cm,
            includefoot,
            footskip=30pt]{geometry}
\usepackage{layout}
\usepackage{graphicx}
\usepackage{subcaption}

\usepackage{biblatex}
\usepackage{pdfpages}

\bibliography{main.bib}

\title{Suspicion: Recognising and evaluating the effectiveness
       of extortion in the Iterated Prisoner's Dilemma}
\author{Vincent A. Knight \and Nikoleta E. Glynatsi}
\date{\today}



\begin{document}

\maketitle

\begin{abstract}
    The Iterated Prisoner's Dilemma is a model for rational and evolutionary
    interactive behaviour. It has applications both in the study of human social
    behaviour as well as in biology.
    It is used to understand when and how a rational individual might
    accept an immediate cost to their own utility for the direct benefit of
    another.

    Much attention has been given to a class of strategies called
    Zero Determinant strategies. It has been theoretically shown that these
    strategies can ``extort'' any player.

    In this work, an approach to identify if observed strategies are playing in
    an extortionate way is described. Furthermore, experimental analysis of
    a large tournament with \input{assets/tex/number_of_full_strategies/main.tex}
    strategies is considered. In this setting
    the most highly performing strategies do not play in an extortionate way
    against each other but do against lower performing strategies.
    This suggests that whilst the theory of Zero Determinant strategies
    indicates that memory is not of fundamental importance to the evolution of
    cooperative behaviour, this is incomplete.
\end{abstract}

\section{Introduction}\label{sec:introduction}

Agent based game theoretic models have become a stalwart of the underpinning
mathematics of interactive behaviours. One of the major pieces of work
in this area is the pair of original computer tournaments run by Robert
Axelrod~\cite{Axelrod1980, Axelrod1980a}. These tournaments pitted submitted
computer strategies against each other in plays of the Iterated Prisoner's
Dilemma. A common game where agents can choose to pay a slight cost to their
immediate utility in the hope of building a reputation. This has been used in
economic and evolutionary game theory to understand the evolution of cooperative
behaviour.

Recently, a class of strategies was described in~\cite{Press2012} that can
provably extort any given opponent. In~\cite{Hilbe2013, Moran1707} some
questions have already been asked about the true effectiveness of these
strategies in an evolutionary setting. Here another question is asked: is it
possible to recognise this extortionate behaviour? A mathematical procedure for
suspicion is presented: in the same way that the continued actions of an
extortionate individual might raise suspicion.

This work makes use of the Axelrod Python library~\cite{Knight2018, Knight2016}
with a large number of Prisoner Dilemma strategies available to give an
extensive numerical example of the ideas presented.  The approach is presented
in Section~\ref{sec:delta-zd-strategies}.  All of the code and data discussed
in Section~\ref{sec:numerical-experiments} is open sourced, archived and
written according to best scientific principles~\cite{Wilson2014}. The data
archive can be found at~\cite{vincent_knight_2018_1297075}.

\section{Recognising Extortion}\label{sec:delta-zd-strategies}

In~\cite{Press2012}, given a match between 2 memory-one strategies, the concept
of Zero Determinant (ZD) strategies is introduced. The main result of that paper
shows that given two memory one players \(p, q\in\mathbb{R}^4\) a linear
relationship between the players' scores could be forced by one of the players.

Using the notation of~\cite{Press2012}, assuming the utilities for player \(p\)
are given by \(S_x=(R, S, T, P)\) and for player \(q\) by \(S_y=(R, T, S, P)\)
and that the stationary scores of each player is given by \(S_X\) and \(S_Y\)
respectively. The main result of~\cite{Press2012} is that if

\begin{equation}\label{eqn:linear_relationship_for_p}
    \tilde p=\alpha S_x + \beta S_y + \gamma
\end{equation}

or

\begin{equation}\label{eqn:linear_relationship_for_q}
    \tilde q=\alpha S_x + \beta S_y + \gamma
\end{equation}

where \(\tilde p = (1 - p_1, 1 - p_2, p_3, p_4)\) and
\(\tilde q = (1 - q_1, 1 - q_2, q_3, q_4)\) then:

\begin{equation}
    \alpha S_X + \beta S_Y + \gamma = 0
\end{equation}

In~\cite{Press2012} a particular type of ZD strategy is defined: extortionate
strategies. If:

\begin{equation}\label{eqn:constraint_for_extortion}
    \gamma = - P(\alpha + \beta)
\end{equation}

then the player can ensure they get a score \(\chi\) times
larger than the opponent. This extortion coefficient is given by:

\begin{equation}\label{eqn:definition_of_chi}
    \chi=\frac{-\beta}{\alpha}
\end{equation}

Thus, if (\ref{eqn:constraint_for_extortion}) holds and \(\chi >1\) a player is
said to extort their opponent.
Here, the reverse problem is considered: given a
\(p\in\mathbb{R}^4\) how does one identify \(\alpha, \beta\) if they
exist and is the strategy in fact acting in an extortionate way?

These conditions correspond to:

\begin{align}
    \tilde p_1 & = \alpha R + \beta R - P (\alpha + \beta)
            \label{eqn:condition_for_tilde_p1}\\
    \tilde p_2 & = \alpha S + \beta T - P (\alpha + \beta)
            \label{eqn:condition_for_tilde_p2}\\
    \tilde p_3 & = \alpha T + \beta S - P (\alpha + \beta)
            \label{eqn:condition_for_tilde_p3}\\
    \tilde p_4 & = \alpha P + \beta P - P (\alpha + \beta)
            \label{eqn:condition_for_tilde_p4}
\end{align}

Equation (\ref{eqn:condition_for_tilde_p4}) ensures that \(p_4=\tilde p_4=0\).
Equations (\ref{eqn:condition_for_tilde_p1}-\ref{eqn:condition_for_tilde_p3})
can be used to eliminate \(\alpha, \beta\), giving:

\begin{equation}\label{eqn:planar_definition_of_extortion}
    \tilde p_1 = \frac{(R - P)(\tilde p_2 + \tilde p_3)}{S + T - 2P}
\end{equation}

with:

\begin{equation}\label{eqn:definition_of_chi}
    \chi = \frac{\tilde p_2 (P - T) + \tilde p_3 (S - P)}
                {\tilde p_2 (P - S) + \tilde p_3 (T - P)}
\end{equation}

Given a strategy \(p\in\mathbb{R}^{4\times 1}\) equations
(\ref{eqn:condition_for_tilde_p4}), (\ref{eqn:planar_definition_of_extortion}-\ref{eqn:definition_of_chi}) can be used to check if
a strategy is extortionate. The conditions correspond to:

\begin{align}
    p_1 & = \frac{(R-P)(p_2 + p_3) - R + T + S - P}{S + T - 2P}
     \label{eqn:condition_for_p1}\\
    p_4 & = 0 \label{eqn:condition_for_p4}\\
    1 & > p_2 + p_3\label{eqn:condition_for_chi}
\end{align}

The algebraic steps necessary to prove these results are available in the
supporting materials.

All extortionate strategies reside on a triangular (\ref{eqn:condition_for_chi})
plane (\ref{eqn:condition_for_p1}) in 3 dimensions (\ref{eqn:condition_for_p4}).
Using this formulation it can be seen that a necessary (but not sufficient)
condition for an extortionate strategy is that it cooperates on average less
than 50\% of the time when in a state of disagreement with the opponent.

As an example, consider the known extortionate strategy \(p=(8 / 9, 1 / 2, 1 /
3, 0)\) from~\cite{Stewart2012} which is referred to as \texttt{Extort-2}. In
this case, for the standard values of \((R, T, S, P)\) constraint
(\ref{eqn:condition_for_p1}) corresponds to:

\begin{equation}
    p_1 = \frac{2(p_2 + p_3) + 1}{3}
\end{equation}

It is clear that in this case all constraints hold.

This approach could in fact be used to confirm that a given strategy is acting
in an extortionate manner even if it is not a memory one strategy. However, in
practice, if a closed form for \(p\) is not known, then due to measurement
and/or numerical error this would not work.

This problem can be written in the following linear algebraic form where
\(x=(\alpha, \beta)\)
and \(p^*=(\tilde p_1 - 1, tilde_2 - 1, p_3)\):

\begin{equation}\label{eqn:linear_algebraic_equation_for_p}
    Cx= p^*
\end{equation}

\(C\) corresponds to equations
(\ref{eqn:condition_for_tilde_p1}-\ref{eqn:condition_for_tilde_p3}) and is
given by:

\begin{equation}\label{eqn:definition_of_C}
    C =
    \begin{bmatrix}
        R - P & R- P \\
        S - P & T- P \\
        T - P & S- P \\
    \end{bmatrix}
\end{equation}

Note that in general, equation (\ref{eqn:linear_algebraic_equation_for_p}) will
not necessarily have a solution. From the Rouch\'{e}-Capelli theorem if there is
a solution it is unique as \(\text{rank}(C)=2\) which is the dimension of the
variable \(x\). The best fitting \(x\) is found by minimizing:

\begin{equation}\label{eqn:r_squared}
    \text{SSError} = \|C x- p^*\|_2^2 = \sum_{i=1}^{3}\left((C\bar x)_i-p_i^*\right)^2
\end{equation}

Note that \(\text{SSError}\), which is the square of the Frobenius
norm~\cite{Golub2013}, becomes a measure of how close a strategy is to being an
extortionate strategy. Suspicion
of extortion then corresponds to a threshold on \(\text{SSError}\).

By observing interactions (human or otherwise), their memory one representation
can be inferred and this approach can be used to recognise extortionate
behaviour. The notion of comparing theoretic and actual plays of the IPD is not
novel, see for example~\cite{Rand2013}. Immediately it is noted that if the
environment is noisy~\cite{Wu1995} then no strategy can be considered to be
extortionate as \(p_4>0\).

In the next section, this idea will be illustrated by observing the interactions
that take place in a computer based tournament of the IPD\@.

\section{Numerical experiments}\label{sec:numerical-experiments}

In~\cite{Stewart2012} results from a tournament with
\input{./assets/tex/number_of_stewart_plotkin_strategies/main.tex} strategies,
was presented with specific consideration given to ZD strategies. This
tournament is reproduced here using the Axelrod-Python
project~\cite{Knight2016}. To obtain a good measure of the corresponding
transition rates for each strategy all matches have been run for
\input{assets/tex/number_of_turns/main.tex} turns and every match has been
repeated \input{assets/tex/number_of_repetitions/main.tex} times. All of this
interaction data is available at~\cite{vincent_knight_2018_1297075}. A good
match between the inferred Markov chain and the state distribution of the actual
interactions has been verified. Data for this is presented in the supplementary
materials.

Figure~\ref{fig:SSError_overall_in_stewart_plotkin} shows the \(\text{SSError}\)
values for all the strategies in the tournament, as reported
in~\cite{Stewart2012} the extortionate strategy (which has an expected
\(\text{SSError}\) approximately 0) gains a large number of wins.

\begin{figure}[!htbp]
    \centering
    \includegraphics[width=.8\textwidth]{./assets/img/SSError_overall_in_stewart_plotkin/main.pdf}
    \caption{\(\text{SSError}\) and state probabilities for the strategies
        of~\cite{Stewart2012}, ordered both by number of wins and overall score.
        Note that \(P(DC)\) is not shown as it corresponds to the transpose of
        \(P(CD)\). Cooperator and Defector are omitted as they do not visit all
        the states.}
    \label{fig:SSError_overall_in_stewart_plotkin}
\end{figure}

Here, the work of~\cite{Stewart2012} is extended by investigating a tournament
with \input{assets/tex/number_of_full_strategies/main.tex}
strategies.

The results of this analysis are shown in
Figure~\ref{fig:SSError_and_probabilities_in_full}. The top ranking strategies
by number of wins seem to be extortionate (but not against all strategies) and
it can be seen that a small sub group of strategies achieve mutual defection.
All the top ranking strategies according to score achieve mutual cooperation and
do not extort each other, however they
\textbf{do} exhibit extortionate behaviour towards a number of the lower ranking
strategies.

\begin{figure}[!htbp]
    \centering
    \includegraphics[width=.8\textwidth]{./assets/img/SSError_and_probabilities_in_full/main.pdf}
    \caption{\(\text{SSError}\) for the strategies for the full tournament. Only
    strategy interactions for which \(p_4=0\) and \(\chi>1\) are displayed.}
    \label{fig:SSError_and_probabilities_in_full}
\end{figure}

\section{Conclusion}\label{sec:conclusion}

This work defines an approach to measure whether or not a player is playing a
strategy that corresponds to an extortionate strategy as defined
in~\cite{Press2012}: a mathematical model for suspicion. Indeed, all
extortionate strategies have been
 classified as lying on a triangular plane.
This rigorous classification fails to be robust to small measurement error, thus
a statistical approach is proposed.
This is done through a linear algebraic approach for approximating the solution
of a linear system. Using this, a large number of pairwise interactions is
simulated and in fact very few strategies are found to act extortionately.

The work of~\cite{Press2012}, whilst showing that a clever approach to taking
advantage of another memory one strategy exists: this is incomplete. Whilst the
elegance of this result is very attractive, just as the simplicity of the
victory of Tit For Tat in Axelrod's original tournaments was, it is incomplete.
Extortionate strategies achieve a high number of wins but they do not
achieve a high score which corresponds to the fitness landscape in an
evolutionary sense. From the large number of interactions a payoff matrix \(S\)
can be measured where \(S_{ij}\) denotes the score (using standard values of
\((R, S, T, P) = (3, 0, 5, 1)\)) of the \(i\)th strategy
against the \(j\)th strategy. Using this, the replicator equation
describes the evolution of the system based on a population density fitness
function:

\begin{equation}\label{eqn:replicator_dynamics}
    \frac{dx}{dt} = x(S-x^TS x)
\end{equation}

Equation (\ref{eqn:replicator_dynamics}) is solved numerically through an
integration technique described in~\cite{Petzold1983} and
Figure~\ref{fig:replicator_dynamics} shows the evolution of the distribution of
the system: the various strategies are ranked by scores. It is clear to see that
only the high ranking strategies survive the evolutionary process (in fact,
only \input{./assets/img/replicator_dynamics/main.tex}
have a final distribution greater than \(10 ^ {-2}\)). This confirms the
findings of~\cite{Moran1707} in which sophisticated strategies resist
evolutionary invasion of shorter memory strategies. Recalling
Figure~\ref{fig:SSError_and_probabilities_in_full} this demonstrates that:

\begin{itemize}
    \item Cooperation emerges through the evolutionary process: the high scoring
        strategies do not exhibit extortionate behaviour towards each other.
    \item Extortionate strategies do not survive the evolutionary process.
\end{itemize}

\begin{figure}[!htbp]
    \centering
    \includegraphics[width=.8\textwidth]{./assets/img/replicator_dynamics/main.pdf}
    \caption{Numerical simulation of the replicator equation
    (\ref{eqn:replicator_dynamics}): strategies are ordered by score, only the strategies with a high score survive the evolutionary process.}
    \label{fig:replicator_dynamics}
\end{figure}

This work can be used to classify plays of the IPD\@: data can be collected from
actual interactions (in lab or in the field). Furthermore, this allows for a
classification method similar to the notion of fingerprinting presented
in~\cite{Ashlock2008}. Trained strategies can potentially be classified as
extortionate or not or it could be possible to even constrain the reinforcement
learning approaches that are becoming prevalent in the literature.
Alternatively, this mathematical approach for recognising extortion could be
used in sophisticated strategies to defend against invasion. Arguably, some of
the strategies considered here exhibit this behaviour, indeed as described
in~\cite{Harper2017}, the top ranking strategies in the full tournament are
obtained using evolutionary reinforcement learning techniques, thus, suspicion
of extortionate behaviour could in fact be an evolutionary trait.

\section*{Acknowledgements}

The following open source software libraries were used in this research:

\begin{itemize}
    \item The Axelrod ~\cite{Knight2016, Knight2018} library (IPD strategies and
        tournaments).
    \item The sympy library~\cite{Meurer2017} (verification of all symbolic
        calculations).
    \item The matplotlib~\cite{Droettboom2018} library (visualisation).
    \item The pandas~\cite{Structures2010}, dask~\cite{Dask2016} and
        NumPy~\cite{Oliphant2015} libraries (data manipulation).
    \item The SciPy~\cite{Jones2001} library (numerical integration of the
        replicator equation).
\end{itemize}

This work was performed using the computational facilities of the Advanced
Research Computing @ Cardiff (ARCCA) Division, Cardiff University.

\printbibliography

\newpage
\section*{Supplementary materials}

\includepdf{assets/pdf/proof_of_form_of_extortionate_strategies/main.pdf}

\newpage

Using the pair wise interactions the transition rates \(p,
q\) can be measured and the steady state probabilities inferred and compared to
the actual probabilities of each state.
This is done numerically by computing the singular eigenvector of the
matrix \(A\) \cite{Stewart2009}:

\[
    A =
    \begin{bmatrix}
        p_1 q_1 & p_1 (1 - q_1) & (1 - p_1) q_1 & (1 -p_1) (1 - q_1) \\
        p_2 q_2 & p_2 (1 - q_2) & (1 - p_2) q_2 & (1 -p_2) (1 - q_2) \\
        p_3 q_3 & p_3 (1 - q_3) & (1 - p_3) q_3 & (1 -p_3) (1 - q_3) \\
        p_4 q_4 & p_4 (1 - q_4) & (1 - p_4) q_4 & (1 -p_4) (1 - q_4) \\
    \end{bmatrix}
\]

Figure~\ref{fig:computed_probabilities_vs_theoretic_probabilities} shows a
regression line fitted to every pairwise interaction with a reported
\(\text{SSError}\) value (pairwise interactions with missing states were
omitted). This serves to validate the approach: a part from some edge cases the
relationship is consistent.

\begin{figure}[!htbp]
    \centering
    \includegraphics[width=.8\textwidth]{./assets/img/computed_probabilities_vs_theoretic_probabilities/main.pdf}
    \caption{The
        relationship between the steady state probabilities inferred from the
        measured transitions and the actual steady state probabilities. A linear
        regression line is included validating the approach.}
    \label{fig:computed_probabilities_vs_theoretic_probabilities}
\end{figure}


\end{document}
 times. All of this
interaction data is available at~\cite{vincent_knight_2018_1297075}. A good
match between the inferred Markov chain and the state distribution of the actual
interactions has been verified. Data for this is presented in the supplementary
materials.

Figure~\ref{fig:SSError_overall_in_stewart_plotkin} shows the \(\text{SSError}\)
values for all the strategies in the tournament, as reported
in~\cite{Stewart2012} the extortionate strategy (which has an expected
\(\text{SSError}\) approximately 0) gains a large number of wins.

\begin{figure}[!htbp]
    \centering
    \includegraphics[width=.8\textwidth]{./assets/img/SSError_overall_in_stewart_plotkin/main.pdf}
    \caption{\(\text{SSError}\) and state probabilities for the strategies
        of~\cite{Stewart2012}, ordered both by number of wins and overall score.
        Note that \(P(DC)\) is not shown as it corresponds to the transpose of
        \(P(CD)\). Cooperator and Defector are omitted as they do not visit all
        the states.}
    \label{fig:SSError_overall_in_stewart_plotkin}
\end{figure}

Here, the work of~\cite{Stewart2012} is extended by investigating a tournament
with \documentclass[a4paper]{article}

\usepackage{amsmath}
\usepackage{amssymb}
\usepackage[margin=1.5cm,
            includefoot,
            footskip=30pt]{geometry}
\usepackage{layout}
\usepackage{graphicx}
\usepackage{subcaption}

\usepackage{biblatex}
\usepackage{pdfpages}

\bibliography{main.bib}

\title{Suspicion: Recognising and evaluating the effectiveness
       of extortion in the Iterated Prisoner's Dilemma}
\author{Vincent A. Knight \and Nikoleta E. Glynatsi}
\date{\today}



\begin{document}

\maketitle

\begin{abstract}
    The Iterated Prisoner's Dilemma is a model for rational and evolutionary
    interactive behaviour. It has applications both in the study of human social
    behaviour as well as in biology.
    It is used to understand when and how a rational individual might
    accept an immediate cost to their own utility for the direct benefit of
    another.

    Much attention has been given to a class of strategies called
    Zero Determinant strategies. It has been theoretically shown that these
    strategies can ``extort'' any player.

    In this work, an approach to identify if observed strategies are playing in
    an extortionate way is described. Furthermore, experimental analysis of
    a large tournament with \input{assets/tex/number_of_full_strategies/main.tex}
    strategies is considered. In this setting
    the most highly performing strategies do not play in an extortionate way
    against each other but do against lower performing strategies.
    This suggests that whilst the theory of Zero Determinant strategies
    indicates that memory is not of fundamental importance to the evolution of
    cooperative behaviour, this is incomplete.
\end{abstract}

\section{Introduction}\label{sec:introduction}

Agent based game theoretic models have become a stalwart of the underpinning
mathematics of interactive behaviours. One of the major pieces of work
in this area is the pair of original computer tournaments run by Robert
Axelrod~\cite{Axelrod1980, Axelrod1980a}. These tournaments pitted submitted
computer strategies against each other in plays of the Iterated Prisoner's
Dilemma. A common game where agents can choose to pay a slight cost to their
immediate utility in the hope of building a reputation. This has been used in
economic and evolutionary game theory to understand the evolution of cooperative
behaviour.

Recently, a class of strategies was described in~\cite{Press2012} that can
provably extort any given opponent. In~\cite{Hilbe2013, Moran1707} some
questions have already been asked about the true effectiveness of these
strategies in an evolutionary setting. Here another question is asked: is it
possible to recognise this extortionate behaviour? A mathematical procedure for
suspicion is presented: in the same way that the continued actions of an
extortionate individual might raise suspicion.

This work makes use of the Axelrod Python library~\cite{Knight2018, Knight2016}
with a large number of Prisoner Dilemma strategies available to give an
extensive numerical example of the ideas presented.  The approach is presented
in Section~\ref{sec:delta-zd-strategies}.  All of the code and data discussed
in Section~\ref{sec:numerical-experiments} is open sourced, archived and
written according to best scientific principles~\cite{Wilson2014}. The data
archive can be found at~\cite{vincent_knight_2018_1297075}.

\section{Recognising Extortion}\label{sec:delta-zd-strategies}

In~\cite{Press2012}, given a match between 2 memory-one strategies, the concept
of Zero Determinant (ZD) strategies is introduced. The main result of that paper
shows that given two memory one players \(p, q\in\mathbb{R}^4\) a linear
relationship between the players' scores could be forced by one of the players.

Using the notation of~\cite{Press2012}, assuming the utilities for player \(p\)
are given by \(S_x=(R, S, T, P)\) and for player \(q\) by \(S_y=(R, T, S, P)\)
and that the stationary scores of each player is given by \(S_X\) and \(S_Y\)
respectively. The main result of~\cite{Press2012} is that if

\begin{equation}\label{eqn:linear_relationship_for_p}
    \tilde p=\alpha S_x + \beta S_y + \gamma
\end{equation}

or

\begin{equation}\label{eqn:linear_relationship_for_q}
    \tilde q=\alpha S_x + \beta S_y + \gamma
\end{equation}

where \(\tilde p = (1 - p_1, 1 - p_2, p_3, p_4)\) and
\(\tilde q = (1 - q_1, 1 - q_2, q_3, q_4)\) then:

\begin{equation}
    \alpha S_X + \beta S_Y + \gamma = 0
\end{equation}

In~\cite{Press2012} a particular type of ZD strategy is defined: extortionate
strategies. If:

\begin{equation}\label{eqn:constraint_for_extortion}
    \gamma = - P(\alpha + \beta)
\end{equation}

then the player can ensure they get a score \(\chi\) times
larger than the opponent. This extortion coefficient is given by:

\begin{equation}\label{eqn:definition_of_chi}
    \chi=\frac{-\beta}{\alpha}
\end{equation}

Thus, if (\ref{eqn:constraint_for_extortion}) holds and \(\chi >1\) a player is
said to extort their opponent.
Here, the reverse problem is considered: given a
\(p\in\mathbb{R}^4\) how does one identify \(\alpha, \beta\) if they
exist and is the strategy in fact acting in an extortionate way?

These conditions correspond to:

\begin{align}
    \tilde p_1 & = \alpha R + \beta R - P (\alpha + \beta)
            \label{eqn:condition_for_tilde_p1}\\
    \tilde p_2 & = \alpha S + \beta T - P (\alpha + \beta)
            \label{eqn:condition_for_tilde_p2}\\
    \tilde p_3 & = \alpha T + \beta S - P (\alpha + \beta)
            \label{eqn:condition_for_tilde_p3}\\
    \tilde p_4 & = \alpha P + \beta P - P (\alpha + \beta)
            \label{eqn:condition_for_tilde_p4}
\end{align}

Equation (\ref{eqn:condition_for_tilde_p4}) ensures that \(p_4=\tilde p_4=0\).
Equations (\ref{eqn:condition_for_tilde_p1}-\ref{eqn:condition_for_tilde_p3})
can be used to eliminate \(\alpha, \beta\), giving:

\begin{equation}\label{eqn:planar_definition_of_extortion}
    \tilde p_1 = \frac{(R - P)(\tilde p_2 + \tilde p_3)}{S + T - 2P}
\end{equation}

with:

\begin{equation}\label{eqn:definition_of_chi}
    \chi = \frac{\tilde p_2 (P - T) + \tilde p_3 (S - P)}
                {\tilde p_2 (P - S) + \tilde p_3 (T - P)}
\end{equation}

Given a strategy \(p\in\mathbb{R}^{4\times 1}\) equations
(\ref{eqn:condition_for_tilde_p4}), (\ref{eqn:planar_definition_of_extortion}-\ref{eqn:definition_of_chi}) can be used to check if
a strategy is extortionate. The conditions correspond to:

\begin{align}
    p_1 & = \frac{(R-P)(p_2 + p_3) - R + T + S - P}{S + T - 2P}
     \label{eqn:condition_for_p1}\\
    p_4 & = 0 \label{eqn:condition_for_p4}\\
    1 & > p_2 + p_3\label{eqn:condition_for_chi}
\end{align}

The algebraic steps necessary to prove these results are available in the
supporting materials.

All extortionate strategies reside on a triangular (\ref{eqn:condition_for_chi})
plane (\ref{eqn:condition_for_p1}) in 3 dimensions (\ref{eqn:condition_for_p4}).
Using this formulation it can be seen that a necessary (but not sufficient)
condition for an extortionate strategy is that it cooperates on average less
than 50\% of the time when in a state of disagreement with the opponent.

As an example, consider the known extortionate strategy \(p=(8 / 9, 1 / 2, 1 /
3, 0)\) from~\cite{Stewart2012} which is referred to as \texttt{Extort-2}. In
this case, for the standard values of \((R, T, S, P)\) constraint
(\ref{eqn:condition_for_p1}) corresponds to:

\begin{equation}
    p_1 = \frac{2(p_2 + p_3) + 1}{3}
\end{equation}

It is clear that in this case all constraints hold.

This approach could in fact be used to confirm that a given strategy is acting
in an extortionate manner even if it is not a memory one strategy. However, in
practice, if a closed form for \(p\) is not known, then due to measurement
and/or numerical error this would not work.

This problem can be written in the following linear algebraic form where
\(x=(\alpha, \beta)\)
and \(p^*=(\tilde p_1 - 1, tilde_2 - 1, p_3)\):

\begin{equation}\label{eqn:linear_algebraic_equation_for_p}
    Cx= p^*
\end{equation}

\(C\) corresponds to equations
(\ref{eqn:condition_for_tilde_p1}-\ref{eqn:condition_for_tilde_p3}) and is
given by:

\begin{equation}\label{eqn:definition_of_C}
    C =
    \begin{bmatrix}
        R - P & R- P \\
        S - P & T- P \\
        T - P & S- P \\
    \end{bmatrix}
\end{equation}

Note that in general, equation (\ref{eqn:linear_algebraic_equation_for_p}) will
not necessarily have a solution. From the Rouch\'{e}-Capelli theorem if there is
a solution it is unique as \(\text{rank}(C)=2\) which is the dimension of the
variable \(x\). The best fitting \(x\) is found by minimizing:

\begin{equation}\label{eqn:r_squared}
    \text{SSError} = \|C x- p^*\|_2^2 = \sum_{i=1}^{3}\left((C\bar x)_i-p_i^*\right)^2
\end{equation}

Note that \(\text{SSError}\), which is the square of the Frobenius
norm~\cite{Golub2013}, becomes a measure of how close a strategy is to being an
extortionate strategy. Suspicion
of extortion then corresponds to a threshold on \(\text{SSError}\).

By observing interactions (human or otherwise), their memory one representation
can be inferred and this approach can be used to recognise extortionate
behaviour. The notion of comparing theoretic and actual plays of the IPD is not
novel, see for example~\cite{Rand2013}. Immediately it is noted that if the
environment is noisy~\cite{Wu1995} then no strategy can be considered to be
extortionate as \(p_4>0\).

In the next section, this idea will be illustrated by observing the interactions
that take place in a computer based tournament of the IPD\@.

\section{Numerical experiments}\label{sec:numerical-experiments}

In~\cite{Stewart2012} results from a tournament with
\input{./assets/tex/number_of_stewart_plotkin_strategies/main.tex} strategies,
was presented with specific consideration given to ZD strategies. This
tournament is reproduced here using the Axelrod-Python
project~\cite{Knight2016}. To obtain a good measure of the corresponding
transition rates for each strategy all matches have been run for
\input{assets/tex/number_of_turns/main.tex} turns and every match has been
repeated \input{assets/tex/number_of_repetitions/main.tex} times. All of this
interaction data is available at~\cite{vincent_knight_2018_1297075}. A good
match between the inferred Markov chain and the state distribution of the actual
interactions has been verified. Data for this is presented in the supplementary
materials.

Figure~\ref{fig:SSError_overall_in_stewart_plotkin} shows the \(\text{SSError}\)
values for all the strategies in the tournament, as reported
in~\cite{Stewart2012} the extortionate strategy (which has an expected
\(\text{SSError}\) approximately 0) gains a large number of wins.

\begin{figure}[!htbp]
    \centering
    \includegraphics[width=.8\textwidth]{./assets/img/SSError_overall_in_stewart_plotkin/main.pdf}
    \caption{\(\text{SSError}\) and state probabilities for the strategies
        of~\cite{Stewart2012}, ordered both by number of wins and overall score.
        Note that \(P(DC)\) is not shown as it corresponds to the transpose of
        \(P(CD)\). Cooperator and Defector are omitted as they do not visit all
        the states.}
    \label{fig:SSError_overall_in_stewart_plotkin}
\end{figure}

Here, the work of~\cite{Stewart2012} is extended by investigating a tournament
with \input{assets/tex/number_of_full_strategies/main.tex}
strategies.

The results of this analysis are shown in
Figure~\ref{fig:SSError_and_probabilities_in_full}. The top ranking strategies
by number of wins seem to be extortionate (but not against all strategies) and
it can be seen that a small sub group of strategies achieve mutual defection.
All the top ranking strategies according to score achieve mutual cooperation and
do not extort each other, however they
\textbf{do} exhibit extortionate behaviour towards a number of the lower ranking
strategies.

\begin{figure}[!htbp]
    \centering
    \includegraphics[width=.8\textwidth]{./assets/img/SSError_and_probabilities_in_full/main.pdf}
    \caption{\(\text{SSError}\) for the strategies for the full tournament. Only
    strategy interactions for which \(p_4=0\) and \(\chi>1\) are displayed.}
    \label{fig:SSError_and_probabilities_in_full}
\end{figure}

\section{Conclusion}\label{sec:conclusion}

This work defines an approach to measure whether or not a player is playing a
strategy that corresponds to an extortionate strategy as defined
in~\cite{Press2012}: a mathematical model for suspicion. Indeed, all
extortionate strategies have been
 classified as lying on a triangular plane.
This rigorous classification fails to be robust to small measurement error, thus
a statistical approach is proposed.
This is done through a linear algebraic approach for approximating the solution
of a linear system. Using this, a large number of pairwise interactions is
simulated and in fact very few strategies are found to act extortionately.

The work of~\cite{Press2012}, whilst showing that a clever approach to taking
advantage of another memory one strategy exists: this is incomplete. Whilst the
elegance of this result is very attractive, just as the simplicity of the
victory of Tit For Tat in Axelrod's original tournaments was, it is incomplete.
Extortionate strategies achieve a high number of wins but they do not
achieve a high score which corresponds to the fitness landscape in an
evolutionary sense. From the large number of interactions a payoff matrix \(S\)
can be measured where \(S_{ij}\) denotes the score (using standard values of
\((R, S, T, P) = (3, 0, 5, 1)\)) of the \(i\)th strategy
against the \(j\)th strategy. Using this, the replicator equation
describes the evolution of the system based on a population density fitness
function:

\begin{equation}\label{eqn:replicator_dynamics}
    \frac{dx}{dt} = x(S-x^TS x)
\end{equation}

Equation (\ref{eqn:replicator_dynamics}) is solved numerically through an
integration technique described in~\cite{Petzold1983} and
Figure~\ref{fig:replicator_dynamics} shows the evolution of the distribution of
the system: the various strategies are ranked by scores. It is clear to see that
only the high ranking strategies survive the evolutionary process (in fact,
only \input{./assets/img/replicator_dynamics/main.tex}
have a final distribution greater than \(10 ^ {-2}\)). This confirms the
findings of~\cite{Moran1707} in which sophisticated strategies resist
evolutionary invasion of shorter memory strategies. Recalling
Figure~\ref{fig:SSError_and_probabilities_in_full} this demonstrates that:

\begin{itemize}
    \item Cooperation emerges through the evolutionary process: the high scoring
        strategies do not exhibit extortionate behaviour towards each other.
    \item Extortionate strategies do not survive the evolutionary process.
\end{itemize}

\begin{figure}[!htbp]
    \centering
    \includegraphics[width=.8\textwidth]{./assets/img/replicator_dynamics/main.pdf}
    \caption{Numerical simulation of the replicator equation
    (\ref{eqn:replicator_dynamics}): strategies are ordered by score, only the strategies with a high score survive the evolutionary process.}
    \label{fig:replicator_dynamics}
\end{figure}

This work can be used to classify plays of the IPD\@: data can be collected from
actual interactions (in lab or in the field). Furthermore, this allows for a
classification method similar to the notion of fingerprinting presented
in~\cite{Ashlock2008}. Trained strategies can potentially be classified as
extortionate or not or it could be possible to even constrain the reinforcement
learning approaches that are becoming prevalent in the literature.
Alternatively, this mathematical approach for recognising extortion could be
used in sophisticated strategies to defend against invasion. Arguably, some of
the strategies considered here exhibit this behaviour, indeed as described
in~\cite{Harper2017}, the top ranking strategies in the full tournament are
obtained using evolutionary reinforcement learning techniques, thus, suspicion
of extortionate behaviour could in fact be an evolutionary trait.

\section*{Acknowledgements}

The following open source software libraries were used in this research:

\begin{itemize}
    \item The Axelrod ~\cite{Knight2016, Knight2018} library (IPD strategies and
        tournaments).
    \item The sympy library~\cite{Meurer2017} (verification of all symbolic
        calculations).
    \item The matplotlib~\cite{Droettboom2018} library (visualisation).
    \item The pandas~\cite{Structures2010}, dask~\cite{Dask2016} and
        NumPy~\cite{Oliphant2015} libraries (data manipulation).
    \item The SciPy~\cite{Jones2001} library (numerical integration of the
        replicator equation).
\end{itemize}

This work was performed using the computational facilities of the Advanced
Research Computing @ Cardiff (ARCCA) Division, Cardiff University.

\printbibliography

\newpage
\section*{Supplementary materials}

\includepdf{assets/pdf/proof_of_form_of_extortionate_strategies/main.pdf}

\newpage

Using the pair wise interactions the transition rates \(p,
q\) can be measured and the steady state probabilities inferred and compared to
the actual probabilities of each state.
This is done numerically by computing the singular eigenvector of the
matrix \(A\) \cite{Stewart2009}:

\[
    A =
    \begin{bmatrix}
        p_1 q_1 & p_1 (1 - q_1) & (1 - p_1) q_1 & (1 -p_1) (1 - q_1) \\
        p_2 q_2 & p_2 (1 - q_2) & (1 - p_2) q_2 & (1 -p_2) (1 - q_2) \\
        p_3 q_3 & p_3 (1 - q_3) & (1 - p_3) q_3 & (1 -p_3) (1 - q_3) \\
        p_4 q_4 & p_4 (1 - q_4) & (1 - p_4) q_4 & (1 -p_4) (1 - q_4) \\
    \end{bmatrix}
\]

Figure~\ref{fig:computed_probabilities_vs_theoretic_probabilities} shows a
regression line fitted to every pairwise interaction with a reported
\(\text{SSError}\) value (pairwise interactions with missing states were
omitted). This serves to validate the approach: a part from some edge cases the
relationship is consistent.

\begin{figure}[!htbp]
    \centering
    \includegraphics[width=.8\textwidth]{./assets/img/computed_probabilities_vs_theoretic_probabilities/main.pdf}
    \caption{The
        relationship between the steady state probabilities inferred from the
        measured transitions and the actual steady state probabilities. A linear
        regression line is included validating the approach.}
    \label{fig:computed_probabilities_vs_theoretic_probabilities}
\end{figure}


\end{document}

strategies.

The results of this analysis are shown in
Figure~\ref{fig:SSError_and_probabilities_in_full}. The top ranking strategies
by number of wins seem to be extortionate (but not against all strategies) and
it can be seen that a small sub group of strategies achieve mutual defection.
All the top ranking strategies according to score achieve mutual cooperation and
do not extort each other, however they
\textbf{do} exhibit extortionate behaviour towards a number of the lower ranking
strategies.

\begin{figure}[!htbp]
    \centering
    \includegraphics[width=.8\textwidth]{./assets/img/SSError_and_probabilities_in_full/main.pdf}
    \caption{\(\text{SSError}\) for the strategies for the full tournament. Only
    strategy interactions for which \(p_4=0\) and \(\chi>1\) are displayed.}
    \label{fig:SSError_and_probabilities_in_full}
\end{figure}

\section{Conclusion}\label{sec:conclusion}

This work defines an approach to measure whether or not a player is playing a
strategy that corresponds to an extortionate strategy as defined
in~\cite{Press2012}: a mathematical model for suspicion. Indeed, all
extortionate strategies have been
 classified as lying on a triangular plane.
This rigorous classification fails to be robust to small measurement error, thus
a statistical approach is proposed.
This is done through a linear algebraic approach for approximating the solution
of a linear system. Using this, a large number of pairwise interactions is
simulated and in fact very few strategies are found to act extortionately.

The work of~\cite{Press2012}, whilst showing that a clever approach to taking
advantage of another memory one strategy exists: this is incomplete. Whilst the
elegance of this result is very attractive, just as the simplicity of the
victory of Tit For Tat in Axelrod's original tournaments was, it is incomplete.
Extortionate strategies achieve a high number of wins but they do not
achieve a high score which corresponds to the fitness landscape in an
evolutionary sense. From the large number of interactions a payoff matrix \(S\)
can be measured where \(S_{ij}\) denotes the score (using standard values of
\((R, S, T, P) = (3, 0, 5, 1)\)) of the \(i\)th strategy
against the \(j\)th strategy. Using this, the replicator equation
describes the evolution of the system based on a population density fitness
function:

\begin{equation}\label{eqn:replicator_dynamics}
    \frac{dx}{dt} = x(S-x^TS x)
\end{equation}

Equation (\ref{eqn:replicator_dynamics}) is solved numerically through an
integration technique described in~\cite{Petzold1983} and
Figure~\ref{fig:replicator_dynamics} shows the evolution of the distribution of
the system: the various strategies are ranked by scores. It is clear to see that
only the high ranking strategies survive the evolutionary process (in fact,
only \documentclass[a4paper]{article}

\usepackage{amsmath}
\usepackage{amssymb}
\usepackage[margin=1.5cm,
            includefoot,
            footskip=30pt]{geometry}
\usepackage{layout}
\usepackage{graphicx}
\usepackage{subcaption}

\usepackage{biblatex}
\usepackage{pdfpages}

\bibliography{main.bib}

\title{Suspicion: Recognising and evaluating the effectiveness
       of extortion in the Iterated Prisoner's Dilemma}
\author{Vincent A. Knight \and Nikoleta E. Glynatsi}
\date{\today}



\begin{document}

\maketitle

\begin{abstract}
    The Iterated Prisoner's Dilemma is a model for rational and evolutionary
    interactive behaviour. It has applications both in the study of human social
    behaviour as well as in biology.
    It is used to understand when and how a rational individual might
    accept an immediate cost to their own utility for the direct benefit of
    another.

    Much attention has been given to a class of strategies called
    Zero Determinant strategies. It has been theoretically shown that these
    strategies can ``extort'' any player.

    In this work, an approach to identify if observed strategies are playing in
    an extortionate way is described. Furthermore, experimental analysis of
    a large tournament with \input{assets/tex/number_of_full_strategies/main.tex}
    strategies is considered. In this setting
    the most highly performing strategies do not play in an extortionate way
    against each other but do against lower performing strategies.
    This suggests that whilst the theory of Zero Determinant strategies
    indicates that memory is not of fundamental importance to the evolution of
    cooperative behaviour, this is incomplete.
\end{abstract}

\section{Introduction}\label{sec:introduction}

Agent based game theoretic models have become a stalwart of the underpinning
mathematics of interactive behaviours. One of the major pieces of work
in this area is the pair of original computer tournaments run by Robert
Axelrod~\cite{Axelrod1980, Axelrod1980a}. These tournaments pitted submitted
computer strategies against each other in plays of the Iterated Prisoner's
Dilemma. A common game where agents can choose to pay a slight cost to their
immediate utility in the hope of building a reputation. This has been used in
economic and evolutionary game theory to understand the evolution of cooperative
behaviour.

Recently, a class of strategies was described in~\cite{Press2012} that can
provably extort any given opponent. In~\cite{Hilbe2013, Moran1707} some
questions have already been asked about the true effectiveness of these
strategies in an evolutionary setting. Here another question is asked: is it
possible to recognise this extortionate behaviour? A mathematical procedure for
suspicion is presented: in the same way that the continued actions of an
extortionate individual might raise suspicion.

This work makes use of the Axelrod Python library~\cite{Knight2018, Knight2016}
with a large number of Prisoner Dilemma strategies available to give an
extensive numerical example of the ideas presented.  The approach is presented
in Section~\ref{sec:delta-zd-strategies}.  All of the code and data discussed
in Section~\ref{sec:numerical-experiments} is open sourced, archived and
written according to best scientific principles~\cite{Wilson2014}. The data
archive can be found at~\cite{vincent_knight_2018_1297075}.

\section{Recognising Extortion}\label{sec:delta-zd-strategies}

In~\cite{Press2012}, given a match between 2 memory-one strategies, the concept
of Zero Determinant (ZD) strategies is introduced. The main result of that paper
shows that given two memory one players \(p, q\in\mathbb{R}^4\) a linear
relationship between the players' scores could be forced by one of the players.

Using the notation of~\cite{Press2012}, assuming the utilities for player \(p\)
are given by \(S_x=(R, S, T, P)\) and for player \(q\) by \(S_y=(R, T, S, P)\)
and that the stationary scores of each player is given by \(S_X\) and \(S_Y\)
respectively. The main result of~\cite{Press2012} is that if

\begin{equation}\label{eqn:linear_relationship_for_p}
    \tilde p=\alpha S_x + \beta S_y + \gamma
\end{equation}

or

\begin{equation}\label{eqn:linear_relationship_for_q}
    \tilde q=\alpha S_x + \beta S_y + \gamma
\end{equation}

where \(\tilde p = (1 - p_1, 1 - p_2, p_3, p_4)\) and
\(\tilde q = (1 - q_1, 1 - q_2, q_3, q_4)\) then:

\begin{equation}
    \alpha S_X + \beta S_Y + \gamma = 0
\end{equation}

In~\cite{Press2012} a particular type of ZD strategy is defined: extortionate
strategies. If:

\begin{equation}\label{eqn:constraint_for_extortion}
    \gamma = - P(\alpha + \beta)
\end{equation}

then the player can ensure they get a score \(\chi\) times
larger than the opponent. This extortion coefficient is given by:

\begin{equation}\label{eqn:definition_of_chi}
    \chi=\frac{-\beta}{\alpha}
\end{equation}

Thus, if (\ref{eqn:constraint_for_extortion}) holds and \(\chi >1\) a player is
said to extort their opponent.
Here, the reverse problem is considered: given a
\(p\in\mathbb{R}^4\) how does one identify \(\alpha, \beta\) if they
exist and is the strategy in fact acting in an extortionate way?

These conditions correspond to:

\begin{align}
    \tilde p_1 & = \alpha R + \beta R - P (\alpha + \beta)
            \label{eqn:condition_for_tilde_p1}\\
    \tilde p_2 & = \alpha S + \beta T - P (\alpha + \beta)
            \label{eqn:condition_for_tilde_p2}\\
    \tilde p_3 & = \alpha T + \beta S - P (\alpha + \beta)
            \label{eqn:condition_for_tilde_p3}\\
    \tilde p_4 & = \alpha P + \beta P - P (\alpha + \beta)
            \label{eqn:condition_for_tilde_p4}
\end{align}

Equation (\ref{eqn:condition_for_tilde_p4}) ensures that \(p_4=\tilde p_4=0\).
Equations (\ref{eqn:condition_for_tilde_p1}-\ref{eqn:condition_for_tilde_p3})
can be used to eliminate \(\alpha, \beta\), giving:

\begin{equation}\label{eqn:planar_definition_of_extortion}
    \tilde p_1 = \frac{(R - P)(\tilde p_2 + \tilde p_3)}{S + T - 2P}
\end{equation}

with:

\begin{equation}\label{eqn:definition_of_chi}
    \chi = \frac{\tilde p_2 (P - T) + \tilde p_3 (S - P)}
                {\tilde p_2 (P - S) + \tilde p_3 (T - P)}
\end{equation}

Given a strategy \(p\in\mathbb{R}^{4\times 1}\) equations
(\ref{eqn:condition_for_tilde_p4}), (\ref{eqn:planar_definition_of_extortion}-\ref{eqn:definition_of_chi}) can be used to check if
a strategy is extortionate. The conditions correspond to:

\begin{align}
    p_1 & = \frac{(R-P)(p_2 + p_3) - R + T + S - P}{S + T - 2P}
     \label{eqn:condition_for_p1}\\
    p_4 & = 0 \label{eqn:condition_for_p4}\\
    1 & > p_2 + p_3\label{eqn:condition_for_chi}
\end{align}

The algebraic steps necessary to prove these results are available in the
supporting materials.

All extortionate strategies reside on a triangular (\ref{eqn:condition_for_chi})
plane (\ref{eqn:condition_for_p1}) in 3 dimensions (\ref{eqn:condition_for_p4}).
Using this formulation it can be seen that a necessary (but not sufficient)
condition for an extortionate strategy is that it cooperates on average less
than 50\% of the time when in a state of disagreement with the opponent.

As an example, consider the known extortionate strategy \(p=(8 / 9, 1 / 2, 1 /
3, 0)\) from~\cite{Stewart2012} which is referred to as \texttt{Extort-2}. In
this case, for the standard values of \((R, T, S, P)\) constraint
(\ref{eqn:condition_for_p1}) corresponds to:

\begin{equation}
    p_1 = \frac{2(p_2 + p_3) + 1}{3}
\end{equation}

It is clear that in this case all constraints hold.

This approach could in fact be used to confirm that a given strategy is acting
in an extortionate manner even if it is not a memory one strategy. However, in
practice, if a closed form for \(p\) is not known, then due to measurement
and/or numerical error this would not work.

This problem can be written in the following linear algebraic form where
\(x=(\alpha, \beta)\)
and \(p^*=(\tilde p_1 - 1, tilde_2 - 1, p_3)\):

\begin{equation}\label{eqn:linear_algebraic_equation_for_p}
    Cx= p^*
\end{equation}

\(C\) corresponds to equations
(\ref{eqn:condition_for_tilde_p1}-\ref{eqn:condition_for_tilde_p3}) and is
given by:

\begin{equation}\label{eqn:definition_of_C}
    C =
    \begin{bmatrix}
        R - P & R- P \\
        S - P & T- P \\
        T - P & S- P \\
    \end{bmatrix}
\end{equation}

Note that in general, equation (\ref{eqn:linear_algebraic_equation_for_p}) will
not necessarily have a solution. From the Rouch\'{e}-Capelli theorem if there is
a solution it is unique as \(\text{rank}(C)=2\) which is the dimension of the
variable \(x\). The best fitting \(x\) is found by minimizing:

\begin{equation}\label{eqn:r_squared}
    \text{SSError} = \|C x- p^*\|_2^2 = \sum_{i=1}^{3}\left((C\bar x)_i-p_i^*\right)^2
\end{equation}

Note that \(\text{SSError}\), which is the square of the Frobenius
norm~\cite{Golub2013}, becomes a measure of how close a strategy is to being an
extortionate strategy. Suspicion
of extortion then corresponds to a threshold on \(\text{SSError}\).

By observing interactions (human or otherwise), their memory one representation
can be inferred and this approach can be used to recognise extortionate
behaviour. The notion of comparing theoretic and actual plays of the IPD is not
novel, see for example~\cite{Rand2013}. Immediately it is noted that if the
environment is noisy~\cite{Wu1995} then no strategy can be considered to be
extortionate as \(p_4>0\).

In the next section, this idea will be illustrated by observing the interactions
that take place in a computer based tournament of the IPD\@.

\section{Numerical experiments}\label{sec:numerical-experiments}

In~\cite{Stewart2012} results from a tournament with
\input{./assets/tex/number_of_stewart_plotkin_strategies/main.tex} strategies,
was presented with specific consideration given to ZD strategies. This
tournament is reproduced here using the Axelrod-Python
project~\cite{Knight2016}. To obtain a good measure of the corresponding
transition rates for each strategy all matches have been run for
\input{assets/tex/number_of_turns/main.tex} turns and every match has been
repeated \input{assets/tex/number_of_repetitions/main.tex} times. All of this
interaction data is available at~\cite{vincent_knight_2018_1297075}. A good
match between the inferred Markov chain and the state distribution of the actual
interactions has been verified. Data for this is presented in the supplementary
materials.

Figure~\ref{fig:SSError_overall_in_stewart_plotkin} shows the \(\text{SSError}\)
values for all the strategies in the tournament, as reported
in~\cite{Stewart2012} the extortionate strategy (which has an expected
\(\text{SSError}\) approximately 0) gains a large number of wins.

\begin{figure}[!htbp]
    \centering
    \includegraphics[width=.8\textwidth]{./assets/img/SSError_overall_in_stewart_plotkin/main.pdf}
    \caption{\(\text{SSError}\) and state probabilities for the strategies
        of~\cite{Stewart2012}, ordered both by number of wins and overall score.
        Note that \(P(DC)\) is not shown as it corresponds to the transpose of
        \(P(CD)\). Cooperator and Defector are omitted as they do not visit all
        the states.}
    \label{fig:SSError_overall_in_stewart_plotkin}
\end{figure}

Here, the work of~\cite{Stewart2012} is extended by investigating a tournament
with \input{assets/tex/number_of_full_strategies/main.tex}
strategies.

The results of this analysis are shown in
Figure~\ref{fig:SSError_and_probabilities_in_full}. The top ranking strategies
by number of wins seem to be extortionate (but not against all strategies) and
it can be seen that a small sub group of strategies achieve mutual defection.
All the top ranking strategies according to score achieve mutual cooperation and
do not extort each other, however they
\textbf{do} exhibit extortionate behaviour towards a number of the lower ranking
strategies.

\begin{figure}[!htbp]
    \centering
    \includegraphics[width=.8\textwidth]{./assets/img/SSError_and_probabilities_in_full/main.pdf}
    \caption{\(\text{SSError}\) for the strategies for the full tournament. Only
    strategy interactions for which \(p_4=0\) and \(\chi>1\) are displayed.}
    \label{fig:SSError_and_probabilities_in_full}
\end{figure}

\section{Conclusion}\label{sec:conclusion}

This work defines an approach to measure whether or not a player is playing a
strategy that corresponds to an extortionate strategy as defined
in~\cite{Press2012}: a mathematical model for suspicion. Indeed, all
extortionate strategies have been
 classified as lying on a triangular plane.
This rigorous classification fails to be robust to small measurement error, thus
a statistical approach is proposed.
This is done through a linear algebraic approach for approximating the solution
of a linear system. Using this, a large number of pairwise interactions is
simulated and in fact very few strategies are found to act extortionately.

The work of~\cite{Press2012}, whilst showing that a clever approach to taking
advantage of another memory one strategy exists: this is incomplete. Whilst the
elegance of this result is very attractive, just as the simplicity of the
victory of Tit For Tat in Axelrod's original tournaments was, it is incomplete.
Extortionate strategies achieve a high number of wins but they do not
achieve a high score which corresponds to the fitness landscape in an
evolutionary sense. From the large number of interactions a payoff matrix \(S\)
can be measured where \(S_{ij}\) denotes the score (using standard values of
\((R, S, T, P) = (3, 0, 5, 1)\)) of the \(i\)th strategy
against the \(j\)th strategy. Using this, the replicator equation
describes the evolution of the system based on a population density fitness
function:

\begin{equation}\label{eqn:replicator_dynamics}
    \frac{dx}{dt} = x(S-x^TS x)
\end{equation}

Equation (\ref{eqn:replicator_dynamics}) is solved numerically through an
integration technique described in~\cite{Petzold1983} and
Figure~\ref{fig:replicator_dynamics} shows the evolution of the distribution of
the system: the various strategies are ranked by scores. It is clear to see that
only the high ranking strategies survive the evolutionary process (in fact,
only \input{./assets/img/replicator_dynamics/main.tex}
have a final distribution greater than \(10 ^ {-2}\)). This confirms the
findings of~\cite{Moran1707} in which sophisticated strategies resist
evolutionary invasion of shorter memory strategies. Recalling
Figure~\ref{fig:SSError_and_probabilities_in_full} this demonstrates that:

\begin{itemize}
    \item Cooperation emerges through the evolutionary process: the high scoring
        strategies do not exhibit extortionate behaviour towards each other.
    \item Extortionate strategies do not survive the evolutionary process.
\end{itemize}

\begin{figure}[!htbp]
    \centering
    \includegraphics[width=.8\textwidth]{./assets/img/replicator_dynamics/main.pdf}
    \caption{Numerical simulation of the replicator equation
    (\ref{eqn:replicator_dynamics}): strategies are ordered by score, only the strategies with a high score survive the evolutionary process.}
    \label{fig:replicator_dynamics}
\end{figure}

This work can be used to classify plays of the IPD\@: data can be collected from
actual interactions (in lab or in the field). Furthermore, this allows for a
classification method similar to the notion of fingerprinting presented
in~\cite{Ashlock2008}. Trained strategies can potentially be classified as
extortionate or not or it could be possible to even constrain the reinforcement
learning approaches that are becoming prevalent in the literature.
Alternatively, this mathematical approach for recognising extortion could be
used in sophisticated strategies to defend against invasion. Arguably, some of
the strategies considered here exhibit this behaviour, indeed as described
in~\cite{Harper2017}, the top ranking strategies in the full tournament are
obtained using evolutionary reinforcement learning techniques, thus, suspicion
of extortionate behaviour could in fact be an evolutionary trait.

\section*{Acknowledgements}

The following open source software libraries were used in this research:

\begin{itemize}
    \item The Axelrod ~\cite{Knight2016, Knight2018} library (IPD strategies and
        tournaments).
    \item The sympy library~\cite{Meurer2017} (verification of all symbolic
        calculations).
    \item The matplotlib~\cite{Droettboom2018} library (visualisation).
    \item The pandas~\cite{Structures2010}, dask~\cite{Dask2016} and
        NumPy~\cite{Oliphant2015} libraries (data manipulation).
    \item The SciPy~\cite{Jones2001} library (numerical integration of the
        replicator equation).
\end{itemize}

This work was performed using the computational facilities of the Advanced
Research Computing @ Cardiff (ARCCA) Division, Cardiff University.

\printbibliography

\newpage
\section*{Supplementary materials}

\includepdf{assets/pdf/proof_of_form_of_extortionate_strategies/main.pdf}

\newpage

Using the pair wise interactions the transition rates \(p,
q\) can be measured and the steady state probabilities inferred and compared to
the actual probabilities of each state.
This is done numerically by computing the singular eigenvector of the
matrix \(A\) \cite{Stewart2009}:

\[
    A =
    \begin{bmatrix}
        p_1 q_1 & p_1 (1 - q_1) & (1 - p_1) q_1 & (1 -p_1) (1 - q_1) \\
        p_2 q_2 & p_2 (1 - q_2) & (1 - p_2) q_2 & (1 -p_2) (1 - q_2) \\
        p_3 q_3 & p_3 (1 - q_3) & (1 - p_3) q_3 & (1 -p_3) (1 - q_3) \\
        p_4 q_4 & p_4 (1 - q_4) & (1 - p_4) q_4 & (1 -p_4) (1 - q_4) \\
    \end{bmatrix}
\]

Figure~\ref{fig:computed_probabilities_vs_theoretic_probabilities} shows a
regression line fitted to every pairwise interaction with a reported
\(\text{SSError}\) value (pairwise interactions with missing states were
omitted). This serves to validate the approach: a part from some edge cases the
relationship is consistent.

\begin{figure}[!htbp]
    \centering
    \includegraphics[width=.8\textwidth]{./assets/img/computed_probabilities_vs_theoretic_probabilities/main.pdf}
    \caption{The
        relationship between the steady state probabilities inferred from the
        measured transitions and the actual steady state probabilities. A linear
        regression line is included validating the approach.}
    \label{fig:computed_probabilities_vs_theoretic_probabilities}
\end{figure}


\end{document}

have a final distribution greater than \(10 ^ {-2}\)). This confirms the
findings of~\cite{Moran1707} in which sophisticated strategies resist
evolutionary invasion of shorter memory strategies. Recalling
Figure~\ref{fig:SSError_and_probabilities_in_full} this demonstrates that:

\begin{itemize}
    \item Cooperation emerges through the evolutionary process: the high scoring
        strategies do not exhibit extortionate behaviour towards each other.
    \item Extortionate strategies do not survive the evolutionary process.
\end{itemize}

\begin{figure}[!htbp]
    \centering
    \includegraphics[width=.8\textwidth]{./assets/img/replicator_dynamics/main.pdf}
    \caption{Numerical simulation of the replicator equation
    (\ref{eqn:replicator_dynamics}): strategies are ordered by score, only the strategies with a high score survive the evolutionary process.}
    \label{fig:replicator_dynamics}
\end{figure}

This work can be used to classify plays of the IPD\@: data can be collected from
actual interactions (in lab or in the field). Furthermore, this allows for a
classification method similar to the notion of fingerprinting presented
in~\cite{Ashlock2008}. Trained strategies can potentially be classified as
extortionate or not or it could be possible to even constrain the reinforcement
learning approaches that are becoming prevalent in the literature.
Alternatively, this mathematical approach for recognising extortion could be
used in sophisticated strategies to defend against invasion. Arguably, some of
the strategies considered here exhibit this behaviour, indeed as described
in~\cite{Harper2017}, the top ranking strategies in the full tournament are
obtained using evolutionary reinforcement learning techniques, thus, suspicion
of extortionate behaviour could in fact be an evolutionary trait.

\section*{Acknowledgements}

The following open source software libraries were used in this research:

\begin{itemize}
    \item The Axelrod ~\cite{Knight2016, Knight2018} library (IPD strategies and
        tournaments).
    \item The sympy library~\cite{Meurer2017} (verification of all symbolic
        calculations).
    \item The matplotlib~\cite{Droettboom2018} library (visualisation).
    \item The pandas~\cite{Structures2010}, dask~\cite{Dask2016} and
        NumPy~\cite{Oliphant2015} libraries (data manipulation).
    \item The SciPy~\cite{Jones2001} library (numerical integration of the
        replicator equation).
\end{itemize}

This work was performed using the computational facilities of the Advanced
Research Computing @ Cardiff (ARCCA) Division, Cardiff University.

\printbibliography

\newpage
\section*{Supplementary materials}

\includepdf{assets/pdf/proof_of_form_of_extortionate_strategies/main.pdf}

\newpage

Using the pair wise interactions the transition rates \(p,
q\) can be measured and the steady state probabilities inferred and compared to
the actual probabilities of each state.
This is done numerically by computing the singular eigenvector of the
matrix \(A\) \cite{Stewart2009}:

\[
    A =
    \begin{bmatrix}
        p_1 q_1 & p_1 (1 - q_1) & (1 - p_1) q_1 & (1 -p_1) (1 - q_1) \\
        p_2 q_2 & p_2 (1 - q_2) & (1 - p_2) q_2 & (1 -p_2) (1 - q_2) \\
        p_3 q_3 & p_3 (1 - q_3) & (1 - p_3) q_3 & (1 -p_3) (1 - q_3) \\
        p_4 q_4 & p_4 (1 - q_4) & (1 - p_4) q_4 & (1 -p_4) (1 - q_4) \\
    \end{bmatrix}
\]

Figure~\ref{fig:computed_probabilities_vs_theoretic_probabilities} shows a
regression line fitted to every pairwise interaction with a reported
\(\text{SSError}\) value (pairwise interactions with missing states were
omitted). This serves to validate the approach: a part from some edge cases the
relationship is consistent.

\begin{figure}[!htbp]
    \centering
    \includegraphics[width=.8\textwidth]{./assets/img/computed_probabilities_vs_theoretic_probabilities/main.pdf}
    \caption{The
        relationship between the steady state probabilities inferred from the
        measured transitions and the actual steady state probabilities. A linear
        regression line is included validating the approach.}
    \label{fig:computed_probabilities_vs_theoretic_probabilities}
\end{figure}


\end{document}
 turns and every match has been
repeated \documentclass[a4paper]{article}

\usepackage{amsmath}
\usepackage{amssymb}
\usepackage[margin=1.5cm,
            includefoot,
            footskip=30pt]{geometry}
\usepackage{layout}
\usepackage{graphicx}
\usepackage{subcaption}

\usepackage{biblatex}
\usepackage{pdfpages}

\bibliography{main.bib}

\title{Suspicion: Recognising and evaluating the effectiveness
       of extortion in the Iterated Prisoner's Dilemma}
\author{Vincent A. Knight \and Nikoleta E. Glynatsi}
\date{\today}



\begin{document}

\maketitle

\begin{abstract}
    The Iterated Prisoner's Dilemma is a model for rational and evolutionary
    interactive behaviour. It has applications both in the study of human social
    behaviour as well as in biology.
    It is used to understand when and how a rational individual might
    accept an immediate cost to their own utility for the direct benefit of
    another.

    Much attention has been given to a class of strategies called
    Zero Determinant strategies. It has been theoretically shown that these
    strategies can ``extort'' any player.

    In this work, an approach to identify if observed strategies are playing in
    an extortionate way is described. Furthermore, experimental analysis of
    a large tournament with \documentclass[a4paper]{article}

\usepackage{amsmath}
\usepackage{amssymb}
\usepackage[margin=1.5cm,
            includefoot,
            footskip=30pt]{geometry}
\usepackage{layout}
\usepackage{graphicx}
\usepackage{subcaption}

\usepackage{biblatex}
\usepackage{pdfpages}

\bibliography{main.bib}

\title{Suspicion: Recognising and evaluating the effectiveness
       of extortion in the Iterated Prisoner's Dilemma}
\author{Vincent A. Knight \and Nikoleta E. Glynatsi}
\date{\today}



\begin{document}

\maketitle

\begin{abstract}
    The Iterated Prisoner's Dilemma is a model for rational and evolutionary
    interactive behaviour. It has applications both in the study of human social
    behaviour as well as in biology.
    It is used to understand when and how a rational individual might
    accept an immediate cost to their own utility for the direct benefit of
    another.

    Much attention has been given to a class of strategies called
    Zero Determinant strategies. It has been theoretically shown that these
    strategies can ``extort'' any player.

    In this work, an approach to identify if observed strategies are playing in
    an extortionate way is described. Furthermore, experimental analysis of
    a large tournament with \input{assets/tex/number_of_full_strategies/main.tex}
    strategies is considered. In this setting
    the most highly performing strategies do not play in an extortionate way
    against each other but do against lower performing strategies.
    This suggests that whilst the theory of Zero Determinant strategies
    indicates that memory is not of fundamental importance to the evolution of
    cooperative behaviour, this is incomplete.
\end{abstract}

\section{Introduction}\label{sec:introduction}

Agent based game theoretic models have become a stalwart of the underpinning
mathematics of interactive behaviours. One of the major pieces of work
in this area is the pair of original computer tournaments run by Robert
Axelrod~\cite{Axelrod1980, Axelrod1980a}. These tournaments pitted submitted
computer strategies against each other in plays of the Iterated Prisoner's
Dilemma. A common game where agents can choose to pay a slight cost to their
immediate utility in the hope of building a reputation. This has been used in
economic and evolutionary game theory to understand the evolution of cooperative
behaviour.

Recently, a class of strategies was described in~\cite{Press2012} that can
provably extort any given opponent. In~\cite{Hilbe2013, Moran1707} some
questions have already been asked about the true effectiveness of these
strategies in an evolutionary setting. Here another question is asked: is it
possible to recognise this extortionate behaviour? A mathematical procedure for
suspicion is presented: in the same way that the continued actions of an
extortionate individual might raise suspicion.

This work makes use of the Axelrod Python library~\cite{Knight2018, Knight2016}
with a large number of Prisoner Dilemma strategies available to give an
extensive numerical example of the ideas presented.  The approach is presented
in Section~\ref{sec:delta-zd-strategies}.  All of the code and data discussed
in Section~\ref{sec:numerical-experiments} is open sourced, archived and
written according to best scientific principles~\cite{Wilson2014}. The data
archive can be found at~\cite{vincent_knight_2018_1297075}.

\section{Recognising Extortion}\label{sec:delta-zd-strategies}

In~\cite{Press2012}, given a match between 2 memory-one strategies, the concept
of Zero Determinant (ZD) strategies is introduced. The main result of that paper
shows that given two memory one players \(p, q\in\mathbb{R}^4\) a linear
relationship between the players' scores could be forced by one of the players.

Using the notation of~\cite{Press2012}, assuming the utilities for player \(p\)
are given by \(S_x=(R, S, T, P)\) and for player \(q\) by \(S_y=(R, T, S, P)\)
and that the stationary scores of each player is given by \(S_X\) and \(S_Y\)
respectively. The main result of~\cite{Press2012} is that if

\begin{equation}\label{eqn:linear_relationship_for_p}
    \tilde p=\alpha S_x + \beta S_y + \gamma
\end{equation}

or

\begin{equation}\label{eqn:linear_relationship_for_q}
    \tilde q=\alpha S_x + \beta S_y + \gamma
\end{equation}

where \(\tilde p = (1 - p_1, 1 - p_2, p_3, p_4)\) and
\(\tilde q = (1 - q_1, 1 - q_2, q_3, q_4)\) then:

\begin{equation}
    \alpha S_X + \beta S_Y + \gamma = 0
\end{equation}

In~\cite{Press2012} a particular type of ZD strategy is defined: extortionate
strategies. If:

\begin{equation}\label{eqn:constraint_for_extortion}
    \gamma = - P(\alpha + \beta)
\end{equation}

then the player can ensure they get a score \(\chi\) times
larger than the opponent. This extortion coefficient is given by:

\begin{equation}\label{eqn:definition_of_chi}
    \chi=\frac{-\beta}{\alpha}
\end{equation}

Thus, if (\ref{eqn:constraint_for_extortion}) holds and \(\chi >1\) a player is
said to extort their opponent.
Here, the reverse problem is considered: given a
\(p\in\mathbb{R}^4\) how does one identify \(\alpha, \beta\) if they
exist and is the strategy in fact acting in an extortionate way?

These conditions correspond to:

\begin{align}
    \tilde p_1 & = \alpha R + \beta R - P (\alpha + \beta)
            \label{eqn:condition_for_tilde_p1}\\
    \tilde p_2 & = \alpha S + \beta T - P (\alpha + \beta)
            \label{eqn:condition_for_tilde_p2}\\
    \tilde p_3 & = \alpha T + \beta S - P (\alpha + \beta)
            \label{eqn:condition_for_tilde_p3}\\
    \tilde p_4 & = \alpha P + \beta P - P (\alpha + \beta)
            \label{eqn:condition_for_tilde_p4}
\end{align}

Equation (\ref{eqn:condition_for_tilde_p4}) ensures that \(p_4=\tilde p_4=0\).
Equations (\ref{eqn:condition_for_tilde_p1}-\ref{eqn:condition_for_tilde_p3})
can be used to eliminate \(\alpha, \beta\), giving:

\begin{equation}\label{eqn:planar_definition_of_extortion}
    \tilde p_1 = \frac{(R - P)(\tilde p_2 + \tilde p_3)}{S + T - 2P}
\end{equation}

with:

\begin{equation}\label{eqn:definition_of_chi}
    \chi = \frac{\tilde p_2 (P - T) + \tilde p_3 (S - P)}
                {\tilde p_2 (P - S) + \tilde p_3 (T - P)}
\end{equation}

Given a strategy \(p\in\mathbb{R}^{4\times 1}\) equations
(\ref{eqn:condition_for_tilde_p4}), (\ref{eqn:planar_definition_of_extortion}-\ref{eqn:definition_of_chi}) can be used to check if
a strategy is extortionate. The conditions correspond to:

\begin{align}
    p_1 & = \frac{(R-P)(p_2 + p_3) - R + T + S - P}{S + T - 2P}
     \label{eqn:condition_for_p1}\\
    p_4 & = 0 \label{eqn:condition_for_p4}\\
    1 & > p_2 + p_3\label{eqn:condition_for_chi}
\end{align}

The algebraic steps necessary to prove these results are available in the
supporting materials.

All extortionate strategies reside on a triangular (\ref{eqn:condition_for_chi})
plane (\ref{eqn:condition_for_p1}) in 3 dimensions (\ref{eqn:condition_for_p4}).
Using this formulation it can be seen that a necessary (but not sufficient)
condition for an extortionate strategy is that it cooperates on average less
than 50\% of the time when in a state of disagreement with the opponent.

As an example, consider the known extortionate strategy \(p=(8 / 9, 1 / 2, 1 /
3, 0)\) from~\cite{Stewart2012} which is referred to as \texttt{Extort-2}. In
this case, for the standard values of \((R, T, S, P)\) constraint
(\ref{eqn:condition_for_p1}) corresponds to:

\begin{equation}
    p_1 = \frac{2(p_2 + p_3) + 1}{3}
\end{equation}

It is clear that in this case all constraints hold.

This approach could in fact be used to confirm that a given strategy is acting
in an extortionate manner even if it is not a memory one strategy. However, in
practice, if a closed form for \(p\) is not known, then due to measurement
and/or numerical error this would not work.

This problem can be written in the following linear algebraic form where
\(x=(\alpha, \beta)\)
and \(p^*=(\tilde p_1 - 1, tilde_2 - 1, p_3)\):

\begin{equation}\label{eqn:linear_algebraic_equation_for_p}
    Cx= p^*
\end{equation}

\(C\) corresponds to equations
(\ref{eqn:condition_for_tilde_p1}-\ref{eqn:condition_for_tilde_p3}) and is
given by:

\begin{equation}\label{eqn:definition_of_C}
    C =
    \begin{bmatrix}
        R - P & R- P \\
        S - P & T- P \\
        T - P & S- P \\
    \end{bmatrix}
\end{equation}

Note that in general, equation (\ref{eqn:linear_algebraic_equation_for_p}) will
not necessarily have a solution. From the Rouch\'{e}-Capelli theorem if there is
a solution it is unique as \(\text{rank}(C)=2\) which is the dimension of the
variable \(x\). The best fitting \(x\) is found by minimizing:

\begin{equation}\label{eqn:r_squared}
    \text{SSError} = \|C x- p^*\|_2^2 = \sum_{i=1}^{3}\left((C\bar x)_i-p_i^*\right)^2
\end{equation}

Note that \(\text{SSError}\), which is the square of the Frobenius
norm~\cite{Golub2013}, becomes a measure of how close a strategy is to being an
extortionate strategy. Suspicion
of extortion then corresponds to a threshold on \(\text{SSError}\).

By observing interactions (human or otherwise), their memory one representation
can be inferred and this approach can be used to recognise extortionate
behaviour. The notion of comparing theoretic and actual plays of the IPD is not
novel, see for example~\cite{Rand2013}. Immediately it is noted that if the
environment is noisy~\cite{Wu1995} then no strategy can be considered to be
extortionate as \(p_4>0\).

In the next section, this idea will be illustrated by observing the interactions
that take place in a computer based tournament of the IPD\@.

\section{Numerical experiments}\label{sec:numerical-experiments}

In~\cite{Stewart2012} results from a tournament with
\input{./assets/tex/number_of_stewart_plotkin_strategies/main.tex} strategies,
was presented with specific consideration given to ZD strategies. This
tournament is reproduced here using the Axelrod-Python
project~\cite{Knight2016}. To obtain a good measure of the corresponding
transition rates for each strategy all matches have been run for
\input{assets/tex/number_of_turns/main.tex} turns and every match has been
repeated \input{assets/tex/number_of_repetitions/main.tex} times. All of this
interaction data is available at~\cite{vincent_knight_2018_1297075}. A good
match between the inferred Markov chain and the state distribution of the actual
interactions has been verified. Data for this is presented in the supplementary
materials.

Figure~\ref{fig:SSError_overall_in_stewart_plotkin} shows the \(\text{SSError}\)
values for all the strategies in the tournament, as reported
in~\cite{Stewart2012} the extortionate strategy (which has an expected
\(\text{SSError}\) approximately 0) gains a large number of wins.

\begin{figure}[!htbp]
    \centering
    \includegraphics[width=.8\textwidth]{./assets/img/SSError_overall_in_stewart_plotkin/main.pdf}
    \caption{\(\text{SSError}\) and state probabilities for the strategies
        of~\cite{Stewart2012}, ordered both by number of wins and overall score.
        Note that \(P(DC)\) is not shown as it corresponds to the transpose of
        \(P(CD)\). Cooperator and Defector are omitted as they do not visit all
        the states.}
    \label{fig:SSError_overall_in_stewart_plotkin}
\end{figure}

Here, the work of~\cite{Stewart2012} is extended by investigating a tournament
with \input{assets/tex/number_of_full_strategies/main.tex}
strategies.

The results of this analysis are shown in
Figure~\ref{fig:SSError_and_probabilities_in_full}. The top ranking strategies
by number of wins seem to be extortionate (but not against all strategies) and
it can be seen that a small sub group of strategies achieve mutual defection.
All the top ranking strategies according to score achieve mutual cooperation and
do not extort each other, however they
\textbf{do} exhibit extortionate behaviour towards a number of the lower ranking
strategies.

\begin{figure}[!htbp]
    \centering
    \includegraphics[width=.8\textwidth]{./assets/img/SSError_and_probabilities_in_full/main.pdf}
    \caption{\(\text{SSError}\) for the strategies for the full tournament. Only
    strategy interactions for which \(p_4=0\) and \(\chi>1\) are displayed.}
    \label{fig:SSError_and_probabilities_in_full}
\end{figure}

\section{Conclusion}\label{sec:conclusion}

This work defines an approach to measure whether or not a player is playing a
strategy that corresponds to an extortionate strategy as defined
in~\cite{Press2012}: a mathematical model for suspicion. Indeed, all
extortionate strategies have been
 classified as lying on a triangular plane.
This rigorous classification fails to be robust to small measurement error, thus
a statistical approach is proposed.
This is done through a linear algebraic approach for approximating the solution
of a linear system. Using this, a large number of pairwise interactions is
simulated and in fact very few strategies are found to act extortionately.

The work of~\cite{Press2012}, whilst showing that a clever approach to taking
advantage of another memory one strategy exists: this is incomplete. Whilst the
elegance of this result is very attractive, just as the simplicity of the
victory of Tit For Tat in Axelrod's original tournaments was, it is incomplete.
Extortionate strategies achieve a high number of wins but they do not
achieve a high score which corresponds to the fitness landscape in an
evolutionary sense. From the large number of interactions a payoff matrix \(S\)
can be measured where \(S_{ij}\) denotes the score (using standard values of
\((R, S, T, P) = (3, 0, 5, 1)\)) of the \(i\)th strategy
against the \(j\)th strategy. Using this, the replicator equation
describes the evolution of the system based on a population density fitness
function:

\begin{equation}\label{eqn:replicator_dynamics}
    \frac{dx}{dt} = x(S-x^TS x)
\end{equation}

Equation (\ref{eqn:replicator_dynamics}) is solved numerically through an
integration technique described in~\cite{Petzold1983} and
Figure~\ref{fig:replicator_dynamics} shows the evolution of the distribution of
the system: the various strategies are ranked by scores. It is clear to see that
only the high ranking strategies survive the evolutionary process (in fact,
only \input{./assets/img/replicator_dynamics/main.tex}
have a final distribution greater than \(10 ^ {-2}\)). This confirms the
findings of~\cite{Moran1707} in which sophisticated strategies resist
evolutionary invasion of shorter memory strategies. Recalling
Figure~\ref{fig:SSError_and_probabilities_in_full} this demonstrates that:

\begin{itemize}
    \item Cooperation emerges through the evolutionary process: the high scoring
        strategies do not exhibit extortionate behaviour towards each other.
    \item Extortionate strategies do not survive the evolutionary process.
\end{itemize}

\begin{figure}[!htbp]
    \centering
    \includegraphics[width=.8\textwidth]{./assets/img/replicator_dynamics/main.pdf}
    \caption{Numerical simulation of the replicator equation
    (\ref{eqn:replicator_dynamics}): strategies are ordered by score, only the strategies with a high score survive the evolutionary process.}
    \label{fig:replicator_dynamics}
\end{figure}

This work can be used to classify plays of the IPD\@: data can be collected from
actual interactions (in lab or in the field). Furthermore, this allows for a
classification method similar to the notion of fingerprinting presented
in~\cite{Ashlock2008}. Trained strategies can potentially be classified as
extortionate or not or it could be possible to even constrain the reinforcement
learning approaches that are becoming prevalent in the literature.
Alternatively, this mathematical approach for recognising extortion could be
used in sophisticated strategies to defend against invasion. Arguably, some of
the strategies considered here exhibit this behaviour, indeed as described
in~\cite{Harper2017}, the top ranking strategies in the full tournament are
obtained using evolutionary reinforcement learning techniques, thus, suspicion
of extortionate behaviour could in fact be an evolutionary trait.

\section*{Acknowledgements}

The following open source software libraries were used in this research:

\begin{itemize}
    \item The Axelrod ~\cite{Knight2016, Knight2018} library (IPD strategies and
        tournaments).
    \item The sympy library~\cite{Meurer2017} (verification of all symbolic
        calculations).
    \item The matplotlib~\cite{Droettboom2018} library (visualisation).
    \item The pandas~\cite{Structures2010}, dask~\cite{Dask2016} and
        NumPy~\cite{Oliphant2015} libraries (data manipulation).
    \item The SciPy~\cite{Jones2001} library (numerical integration of the
        replicator equation).
\end{itemize}

This work was performed using the computational facilities of the Advanced
Research Computing @ Cardiff (ARCCA) Division, Cardiff University.

\printbibliography

\newpage
\section*{Supplementary materials}

\includepdf{assets/pdf/proof_of_form_of_extortionate_strategies/main.pdf}

\newpage

Using the pair wise interactions the transition rates \(p,
q\) can be measured and the steady state probabilities inferred and compared to
the actual probabilities of each state.
This is done numerically by computing the singular eigenvector of the
matrix \(A\) \cite{Stewart2009}:

\[
    A =
    \begin{bmatrix}
        p_1 q_1 & p_1 (1 - q_1) & (1 - p_1) q_1 & (1 -p_1) (1 - q_1) \\
        p_2 q_2 & p_2 (1 - q_2) & (1 - p_2) q_2 & (1 -p_2) (1 - q_2) \\
        p_3 q_3 & p_3 (1 - q_3) & (1 - p_3) q_3 & (1 -p_3) (1 - q_3) \\
        p_4 q_4 & p_4 (1 - q_4) & (1 - p_4) q_4 & (1 -p_4) (1 - q_4) \\
    \end{bmatrix}
\]

Figure~\ref{fig:computed_probabilities_vs_theoretic_probabilities} shows a
regression line fitted to every pairwise interaction with a reported
\(\text{SSError}\) value (pairwise interactions with missing states were
omitted). This serves to validate the approach: a part from some edge cases the
relationship is consistent.

\begin{figure}[!htbp]
    \centering
    \includegraphics[width=.8\textwidth]{./assets/img/computed_probabilities_vs_theoretic_probabilities/main.pdf}
    \caption{The
        relationship between the steady state probabilities inferred from the
        measured transitions and the actual steady state probabilities. A linear
        regression line is included validating the approach.}
    \label{fig:computed_probabilities_vs_theoretic_probabilities}
\end{figure}


\end{document}

    strategies is considered. In this setting
    the most highly performing strategies do not play in an extortionate way
    against each other but do against lower performing strategies.
    This suggests that whilst the theory of Zero Determinant strategies
    indicates that memory is not of fundamental importance to the evolution of
    cooperative behaviour, this is incomplete.
\end{abstract}

\section{Introduction}\label{sec:introduction}

Agent based game theoretic models have become a stalwart of the underpinning
mathematics of interactive behaviours. One of the major pieces of work
in this area is the pair of original computer tournaments run by Robert
Axelrod~\cite{Axelrod1980, Axelrod1980a}. These tournaments pitted submitted
computer strategies against each other in plays of the Iterated Prisoner's
Dilemma. A common game where agents can choose to pay a slight cost to their
immediate utility in the hope of building a reputation. This has been used in
economic and evolutionary game theory to understand the evolution of cooperative
behaviour.

Recently, a class of strategies was described in~\cite{Press2012} that can
provably extort any given opponent. In~\cite{Hilbe2013, Moran1707} some
questions have already been asked about the true effectiveness of these
strategies in an evolutionary setting. Here another question is asked: is it
possible to recognise this extortionate behaviour? A mathematical procedure for
suspicion is presented: in the same way that the continued actions of an
extortionate individual might raise suspicion.

This work makes use of the Axelrod Python library~\cite{Knight2018, Knight2016}
with a large number of Prisoner Dilemma strategies available to give an
extensive numerical example of the ideas presented.  The approach is presented
in Section~\ref{sec:delta-zd-strategies}.  All of the code and data discussed
in Section~\ref{sec:numerical-experiments} is open sourced, archived and
written according to best scientific principles~\cite{Wilson2014}. The data
archive can be found at~\cite{vincent_knight_2018_1297075}.

\section{Recognising Extortion}\label{sec:delta-zd-strategies}

In~\cite{Press2012}, given a match between 2 memory-one strategies, the concept
of Zero Determinant (ZD) strategies is introduced. The main result of that paper
shows that given two memory one players \(p, q\in\mathbb{R}^4\) a linear
relationship between the players' scores could be forced by one of the players.

Using the notation of~\cite{Press2012}, assuming the utilities for player \(p\)
are given by \(S_x=(R, S, T, P)\) and for player \(q\) by \(S_y=(R, T, S, P)\)
and that the stationary scores of each player is given by \(S_X\) and \(S_Y\)
respectively. The main result of~\cite{Press2012} is that if

\begin{equation}\label{eqn:linear_relationship_for_p}
    \tilde p=\alpha S_x + \beta S_y + \gamma
\end{equation}

or

\begin{equation}\label{eqn:linear_relationship_for_q}
    \tilde q=\alpha S_x + \beta S_y + \gamma
\end{equation}

where \(\tilde p = (1 - p_1, 1 - p_2, p_3, p_4)\) and
\(\tilde q = (1 - q_1, 1 - q_2, q_3, q_4)\) then:

\begin{equation}
    \alpha S_X + \beta S_Y + \gamma = 0
\end{equation}

In~\cite{Press2012} a particular type of ZD strategy is defined: extortionate
strategies. If:

\begin{equation}\label{eqn:constraint_for_extortion}
    \gamma = - P(\alpha + \beta)
\end{equation}

then the player can ensure they get a score \(\chi\) times
larger than the opponent. This extortion coefficient is given by:

\begin{equation}\label{eqn:definition_of_chi}
    \chi=\frac{-\beta}{\alpha}
\end{equation}

Thus, if (\ref{eqn:constraint_for_extortion}) holds and \(\chi >1\) a player is
said to extort their opponent.
Here, the reverse problem is considered: given a
\(p\in\mathbb{R}^4\) how does one identify \(\alpha, \beta\) if they
exist and is the strategy in fact acting in an extortionate way?

These conditions correspond to:

\begin{align}
    \tilde p_1 & = \alpha R + \beta R - P (\alpha + \beta)
            \label{eqn:condition_for_tilde_p1}\\
    \tilde p_2 & = \alpha S + \beta T - P (\alpha + \beta)
            \label{eqn:condition_for_tilde_p2}\\
    \tilde p_3 & = \alpha T + \beta S - P (\alpha + \beta)
            \label{eqn:condition_for_tilde_p3}\\
    \tilde p_4 & = \alpha P + \beta P - P (\alpha + \beta)
            \label{eqn:condition_for_tilde_p4}
\end{align}

Equation (\ref{eqn:condition_for_tilde_p4}) ensures that \(p_4=\tilde p_4=0\).
Equations (\ref{eqn:condition_for_tilde_p1}-\ref{eqn:condition_for_tilde_p3})
can be used to eliminate \(\alpha, \beta\), giving:

\begin{equation}\label{eqn:planar_definition_of_extortion}
    \tilde p_1 = \frac{(R - P)(\tilde p_2 + \tilde p_3)}{S + T - 2P}
\end{equation}

with:

\begin{equation}\label{eqn:definition_of_chi}
    \chi = \frac{\tilde p_2 (P - T) + \tilde p_3 (S - P)}
                {\tilde p_2 (P - S) + \tilde p_3 (T - P)}
\end{equation}

Given a strategy \(p\in\mathbb{R}^{4\times 1}\) equations
(\ref{eqn:condition_for_tilde_p4}), (\ref{eqn:planar_definition_of_extortion}-\ref{eqn:definition_of_chi}) can be used to check if
a strategy is extortionate. The conditions correspond to:

\begin{align}
    p_1 & = \frac{(R-P)(p_2 + p_3) - R + T + S - P}{S + T - 2P}
     \label{eqn:condition_for_p1}\\
    p_4 & = 0 \label{eqn:condition_for_p4}\\
    1 & > p_2 + p_3\label{eqn:condition_for_chi}
\end{align}

The algebraic steps necessary to prove these results are available in the
supporting materials.

All extortionate strategies reside on a triangular (\ref{eqn:condition_for_chi})
plane (\ref{eqn:condition_for_p1}) in 3 dimensions (\ref{eqn:condition_for_p4}).
Using this formulation it can be seen that a necessary (but not sufficient)
condition for an extortionate strategy is that it cooperates on average less
than 50\% of the time when in a state of disagreement with the opponent.

As an example, consider the known extortionate strategy \(p=(8 / 9, 1 / 2, 1 /
3, 0)\) from~\cite{Stewart2012} which is referred to as \texttt{Extort-2}. In
this case, for the standard values of \((R, T, S, P)\) constraint
(\ref{eqn:condition_for_p1}) corresponds to:

\begin{equation}
    p_1 = \frac{2(p_2 + p_3) + 1}{3}
\end{equation}

It is clear that in this case all constraints hold.

This approach could in fact be used to confirm that a given strategy is acting
in an extortionate manner even if it is not a memory one strategy. However, in
practice, if a closed form for \(p\) is not known, then due to measurement
and/or numerical error this would not work.

This problem can be written in the following linear algebraic form where
\(x=(\alpha, \beta)\)
and \(p^*=(\tilde p_1 - 1, tilde_2 - 1, p_3)\):

\begin{equation}\label{eqn:linear_algebraic_equation_for_p}
    Cx= p^*
\end{equation}

\(C\) corresponds to equations
(\ref{eqn:condition_for_tilde_p1}-\ref{eqn:condition_for_tilde_p3}) and is
given by:

\begin{equation}\label{eqn:definition_of_C}
    C =
    \begin{bmatrix}
        R - P & R- P \\
        S - P & T- P \\
        T - P & S- P \\
    \end{bmatrix}
\end{equation}

Note that in general, equation (\ref{eqn:linear_algebraic_equation_for_p}) will
not necessarily have a solution. From the Rouch\'{e}-Capelli theorem if there is
a solution it is unique as \(\text{rank}(C)=2\) which is the dimension of the
variable \(x\). The best fitting \(x\) is found by minimizing:

\begin{equation}\label{eqn:r_squared}
    \text{SSError} = \|C x- p^*\|_2^2 = \sum_{i=1}^{3}\left((C\bar x)_i-p_i^*\right)^2
\end{equation}

Note that \(\text{SSError}\), which is the square of the Frobenius
norm~\cite{Golub2013}, becomes a measure of how close a strategy is to being an
extortionate strategy. Suspicion
of extortion then corresponds to a threshold on \(\text{SSError}\).

By observing interactions (human or otherwise), their memory one representation
can be inferred and this approach can be used to recognise extortionate
behaviour. The notion of comparing theoretic and actual plays of the IPD is not
novel, see for example~\cite{Rand2013}. Immediately it is noted that if the
environment is noisy~\cite{Wu1995} then no strategy can be considered to be
extortionate as \(p_4>0\).

In the next section, this idea will be illustrated by observing the interactions
that take place in a computer based tournament of the IPD\@.

\section{Numerical experiments}\label{sec:numerical-experiments}

In~\cite{Stewart2012} results from a tournament with
\documentclass[a4paper]{article}

\usepackage{amsmath}
\usepackage{amssymb}
\usepackage[margin=1.5cm,
            includefoot,
            footskip=30pt]{geometry}
\usepackage{layout}
\usepackage{graphicx}
\usepackage{subcaption}

\usepackage{biblatex}
\usepackage{pdfpages}

\bibliography{main.bib}

\title{Suspicion: Recognising and evaluating the effectiveness
       of extortion in the Iterated Prisoner's Dilemma}
\author{Vincent A. Knight \and Nikoleta E. Glynatsi}
\date{\today}



\begin{document}

\maketitle

\begin{abstract}
    The Iterated Prisoner's Dilemma is a model for rational and evolutionary
    interactive behaviour. It has applications both in the study of human social
    behaviour as well as in biology.
    It is used to understand when and how a rational individual might
    accept an immediate cost to their own utility for the direct benefit of
    another.

    Much attention has been given to a class of strategies called
    Zero Determinant strategies. It has been theoretically shown that these
    strategies can ``extort'' any player.

    In this work, an approach to identify if observed strategies are playing in
    an extortionate way is described. Furthermore, experimental analysis of
    a large tournament with \input{assets/tex/number_of_full_strategies/main.tex}
    strategies is considered. In this setting
    the most highly performing strategies do not play in an extortionate way
    against each other but do against lower performing strategies.
    This suggests that whilst the theory of Zero Determinant strategies
    indicates that memory is not of fundamental importance to the evolution of
    cooperative behaviour, this is incomplete.
\end{abstract}

\section{Introduction}\label{sec:introduction}

Agent based game theoretic models have become a stalwart of the underpinning
mathematics of interactive behaviours. One of the major pieces of work
in this area is the pair of original computer tournaments run by Robert
Axelrod~\cite{Axelrod1980, Axelrod1980a}. These tournaments pitted submitted
computer strategies against each other in plays of the Iterated Prisoner's
Dilemma. A common game where agents can choose to pay a slight cost to their
immediate utility in the hope of building a reputation. This has been used in
economic and evolutionary game theory to understand the evolution of cooperative
behaviour.

Recently, a class of strategies was described in~\cite{Press2012} that can
provably extort any given opponent. In~\cite{Hilbe2013, Moran1707} some
questions have already been asked about the true effectiveness of these
strategies in an evolutionary setting. Here another question is asked: is it
possible to recognise this extortionate behaviour? A mathematical procedure for
suspicion is presented: in the same way that the continued actions of an
extortionate individual might raise suspicion.

This work makes use of the Axelrod Python library~\cite{Knight2018, Knight2016}
with a large number of Prisoner Dilemma strategies available to give an
extensive numerical example of the ideas presented.  The approach is presented
in Section~\ref{sec:delta-zd-strategies}.  All of the code and data discussed
in Section~\ref{sec:numerical-experiments} is open sourced, archived and
written according to best scientific principles~\cite{Wilson2014}. The data
archive can be found at~\cite{vincent_knight_2018_1297075}.

\section{Recognising Extortion}\label{sec:delta-zd-strategies}

In~\cite{Press2012}, given a match between 2 memory-one strategies, the concept
of Zero Determinant (ZD) strategies is introduced. The main result of that paper
shows that given two memory one players \(p, q\in\mathbb{R}^4\) a linear
relationship between the players' scores could be forced by one of the players.

Using the notation of~\cite{Press2012}, assuming the utilities for player \(p\)
are given by \(S_x=(R, S, T, P)\) and for player \(q\) by \(S_y=(R, T, S, P)\)
and that the stationary scores of each player is given by \(S_X\) and \(S_Y\)
respectively. The main result of~\cite{Press2012} is that if

\begin{equation}\label{eqn:linear_relationship_for_p}
    \tilde p=\alpha S_x + \beta S_y + \gamma
\end{equation}

or

\begin{equation}\label{eqn:linear_relationship_for_q}
    \tilde q=\alpha S_x + \beta S_y + \gamma
\end{equation}

where \(\tilde p = (1 - p_1, 1 - p_2, p_3, p_4)\) and
\(\tilde q = (1 - q_1, 1 - q_2, q_3, q_4)\) then:

\begin{equation}
    \alpha S_X + \beta S_Y + \gamma = 0
\end{equation}

In~\cite{Press2012} a particular type of ZD strategy is defined: extortionate
strategies. If:

\begin{equation}\label{eqn:constraint_for_extortion}
    \gamma = - P(\alpha + \beta)
\end{equation}

then the player can ensure they get a score \(\chi\) times
larger than the opponent. This extortion coefficient is given by:

\begin{equation}\label{eqn:definition_of_chi}
    \chi=\frac{-\beta}{\alpha}
\end{equation}

Thus, if (\ref{eqn:constraint_for_extortion}) holds and \(\chi >1\) a player is
said to extort their opponent.
Here, the reverse problem is considered: given a
\(p\in\mathbb{R}^4\) how does one identify \(\alpha, \beta\) if they
exist and is the strategy in fact acting in an extortionate way?

These conditions correspond to:

\begin{align}
    \tilde p_1 & = \alpha R + \beta R - P (\alpha + \beta)
            \label{eqn:condition_for_tilde_p1}\\
    \tilde p_2 & = \alpha S + \beta T - P (\alpha + \beta)
            \label{eqn:condition_for_tilde_p2}\\
    \tilde p_3 & = \alpha T + \beta S - P (\alpha + \beta)
            \label{eqn:condition_for_tilde_p3}\\
    \tilde p_4 & = \alpha P + \beta P - P (\alpha + \beta)
            \label{eqn:condition_for_tilde_p4}
\end{align}

Equation (\ref{eqn:condition_for_tilde_p4}) ensures that \(p_4=\tilde p_4=0\).
Equations (\ref{eqn:condition_for_tilde_p1}-\ref{eqn:condition_for_tilde_p3})
can be used to eliminate \(\alpha, \beta\), giving:

\begin{equation}\label{eqn:planar_definition_of_extortion}
    \tilde p_1 = \frac{(R - P)(\tilde p_2 + \tilde p_3)}{S + T - 2P}
\end{equation}

with:

\begin{equation}\label{eqn:definition_of_chi}
    \chi = \frac{\tilde p_2 (P - T) + \tilde p_3 (S - P)}
                {\tilde p_2 (P - S) + \tilde p_3 (T - P)}
\end{equation}

Given a strategy \(p\in\mathbb{R}^{4\times 1}\) equations
(\ref{eqn:condition_for_tilde_p4}), (\ref{eqn:planar_definition_of_extortion}-\ref{eqn:definition_of_chi}) can be used to check if
a strategy is extortionate. The conditions correspond to:

\begin{align}
    p_1 & = \frac{(R-P)(p_2 + p_3) - R + T + S - P}{S + T - 2P}
     \label{eqn:condition_for_p1}\\
    p_4 & = 0 \label{eqn:condition_for_p4}\\
    1 & > p_2 + p_3\label{eqn:condition_for_chi}
\end{align}

The algebraic steps necessary to prove these results are available in the
supporting materials.

All extortionate strategies reside on a triangular (\ref{eqn:condition_for_chi})
plane (\ref{eqn:condition_for_p1}) in 3 dimensions (\ref{eqn:condition_for_p4}).
Using this formulation it can be seen that a necessary (but not sufficient)
condition for an extortionate strategy is that it cooperates on average less
than 50\% of the time when in a state of disagreement with the opponent.

As an example, consider the known extortionate strategy \(p=(8 / 9, 1 / 2, 1 /
3, 0)\) from~\cite{Stewart2012} which is referred to as \texttt{Extort-2}. In
this case, for the standard values of \((R, T, S, P)\) constraint
(\ref{eqn:condition_for_p1}) corresponds to:

\begin{equation}
    p_1 = \frac{2(p_2 + p_3) + 1}{3}
\end{equation}

It is clear that in this case all constraints hold.

This approach could in fact be used to confirm that a given strategy is acting
in an extortionate manner even if it is not a memory one strategy. However, in
practice, if a closed form for \(p\) is not known, then due to measurement
and/or numerical error this would not work.

This problem can be written in the following linear algebraic form where
\(x=(\alpha, \beta)\)
and \(p^*=(\tilde p_1 - 1, tilde_2 - 1, p_3)\):

\begin{equation}\label{eqn:linear_algebraic_equation_for_p}
    Cx= p^*
\end{equation}

\(C\) corresponds to equations
(\ref{eqn:condition_for_tilde_p1}-\ref{eqn:condition_for_tilde_p3}) and is
given by:

\begin{equation}\label{eqn:definition_of_C}
    C =
    \begin{bmatrix}
        R - P & R- P \\
        S - P & T- P \\
        T - P & S- P \\
    \end{bmatrix}
\end{equation}

Note that in general, equation (\ref{eqn:linear_algebraic_equation_for_p}) will
not necessarily have a solution. From the Rouch\'{e}-Capelli theorem if there is
a solution it is unique as \(\text{rank}(C)=2\) which is the dimension of the
variable \(x\). The best fitting \(x\) is found by minimizing:

\begin{equation}\label{eqn:r_squared}
    \text{SSError} = \|C x- p^*\|_2^2 = \sum_{i=1}^{3}\left((C\bar x)_i-p_i^*\right)^2
\end{equation}

Note that \(\text{SSError}\), which is the square of the Frobenius
norm~\cite{Golub2013}, becomes a measure of how close a strategy is to being an
extortionate strategy. Suspicion
of extortion then corresponds to a threshold on \(\text{SSError}\).

By observing interactions (human or otherwise), their memory one representation
can be inferred and this approach can be used to recognise extortionate
behaviour. The notion of comparing theoretic and actual plays of the IPD is not
novel, see for example~\cite{Rand2013}. Immediately it is noted that if the
environment is noisy~\cite{Wu1995} then no strategy can be considered to be
extortionate as \(p_4>0\).

In the next section, this idea will be illustrated by observing the interactions
that take place in a computer based tournament of the IPD\@.

\section{Numerical experiments}\label{sec:numerical-experiments}

In~\cite{Stewart2012} results from a tournament with
\input{./assets/tex/number_of_stewart_plotkin_strategies/main.tex} strategies,
was presented with specific consideration given to ZD strategies. This
tournament is reproduced here using the Axelrod-Python
project~\cite{Knight2016}. To obtain a good measure of the corresponding
transition rates for each strategy all matches have been run for
\input{assets/tex/number_of_turns/main.tex} turns and every match has been
repeated \input{assets/tex/number_of_repetitions/main.tex} times. All of this
interaction data is available at~\cite{vincent_knight_2018_1297075}. A good
match between the inferred Markov chain and the state distribution of the actual
interactions has been verified. Data for this is presented in the supplementary
materials.

Figure~\ref{fig:SSError_overall_in_stewart_plotkin} shows the \(\text{SSError}\)
values for all the strategies in the tournament, as reported
in~\cite{Stewart2012} the extortionate strategy (which has an expected
\(\text{SSError}\) approximately 0) gains a large number of wins.

\begin{figure}[!htbp]
    \centering
    \includegraphics[width=.8\textwidth]{./assets/img/SSError_overall_in_stewart_plotkin/main.pdf}
    \caption{\(\text{SSError}\) and state probabilities for the strategies
        of~\cite{Stewart2012}, ordered both by number of wins and overall score.
        Note that \(P(DC)\) is not shown as it corresponds to the transpose of
        \(P(CD)\). Cooperator and Defector are omitted as they do not visit all
        the states.}
    \label{fig:SSError_overall_in_stewart_plotkin}
\end{figure}

Here, the work of~\cite{Stewart2012} is extended by investigating a tournament
with \input{assets/tex/number_of_full_strategies/main.tex}
strategies.

The results of this analysis are shown in
Figure~\ref{fig:SSError_and_probabilities_in_full}. The top ranking strategies
by number of wins seem to be extortionate (but not against all strategies) and
it can be seen that a small sub group of strategies achieve mutual defection.
All the top ranking strategies according to score achieve mutual cooperation and
do not extort each other, however they
\textbf{do} exhibit extortionate behaviour towards a number of the lower ranking
strategies.

\begin{figure}[!htbp]
    \centering
    \includegraphics[width=.8\textwidth]{./assets/img/SSError_and_probabilities_in_full/main.pdf}
    \caption{\(\text{SSError}\) for the strategies for the full tournament. Only
    strategy interactions for which \(p_4=0\) and \(\chi>1\) are displayed.}
    \label{fig:SSError_and_probabilities_in_full}
\end{figure}

\section{Conclusion}\label{sec:conclusion}

This work defines an approach to measure whether or not a player is playing a
strategy that corresponds to an extortionate strategy as defined
in~\cite{Press2012}: a mathematical model for suspicion. Indeed, all
extortionate strategies have been
 classified as lying on a triangular plane.
This rigorous classification fails to be robust to small measurement error, thus
a statistical approach is proposed.
This is done through a linear algebraic approach for approximating the solution
of a linear system. Using this, a large number of pairwise interactions is
simulated and in fact very few strategies are found to act extortionately.

The work of~\cite{Press2012}, whilst showing that a clever approach to taking
advantage of another memory one strategy exists: this is incomplete. Whilst the
elegance of this result is very attractive, just as the simplicity of the
victory of Tit For Tat in Axelrod's original tournaments was, it is incomplete.
Extortionate strategies achieve a high number of wins but they do not
achieve a high score which corresponds to the fitness landscape in an
evolutionary sense. From the large number of interactions a payoff matrix \(S\)
can be measured where \(S_{ij}\) denotes the score (using standard values of
\((R, S, T, P) = (3, 0, 5, 1)\)) of the \(i\)th strategy
against the \(j\)th strategy. Using this, the replicator equation
describes the evolution of the system based on a population density fitness
function:

\begin{equation}\label{eqn:replicator_dynamics}
    \frac{dx}{dt} = x(S-x^TS x)
\end{equation}

Equation (\ref{eqn:replicator_dynamics}) is solved numerically through an
integration technique described in~\cite{Petzold1983} and
Figure~\ref{fig:replicator_dynamics} shows the evolution of the distribution of
the system: the various strategies are ranked by scores. It is clear to see that
only the high ranking strategies survive the evolutionary process (in fact,
only \input{./assets/img/replicator_dynamics/main.tex}
have a final distribution greater than \(10 ^ {-2}\)). This confirms the
findings of~\cite{Moran1707} in which sophisticated strategies resist
evolutionary invasion of shorter memory strategies. Recalling
Figure~\ref{fig:SSError_and_probabilities_in_full} this demonstrates that:

\begin{itemize}
    \item Cooperation emerges through the evolutionary process: the high scoring
        strategies do not exhibit extortionate behaviour towards each other.
    \item Extortionate strategies do not survive the evolutionary process.
\end{itemize}

\begin{figure}[!htbp]
    \centering
    \includegraphics[width=.8\textwidth]{./assets/img/replicator_dynamics/main.pdf}
    \caption{Numerical simulation of the replicator equation
    (\ref{eqn:replicator_dynamics}): strategies are ordered by score, only the strategies with a high score survive the evolutionary process.}
    \label{fig:replicator_dynamics}
\end{figure}

This work can be used to classify plays of the IPD\@: data can be collected from
actual interactions (in lab or in the field). Furthermore, this allows for a
classification method similar to the notion of fingerprinting presented
in~\cite{Ashlock2008}. Trained strategies can potentially be classified as
extortionate or not or it could be possible to even constrain the reinforcement
learning approaches that are becoming prevalent in the literature.
Alternatively, this mathematical approach for recognising extortion could be
used in sophisticated strategies to defend against invasion. Arguably, some of
the strategies considered here exhibit this behaviour, indeed as described
in~\cite{Harper2017}, the top ranking strategies in the full tournament are
obtained using evolutionary reinforcement learning techniques, thus, suspicion
of extortionate behaviour could in fact be an evolutionary trait.

\section*{Acknowledgements}

The following open source software libraries were used in this research:

\begin{itemize}
    \item The Axelrod ~\cite{Knight2016, Knight2018} library (IPD strategies and
        tournaments).
    \item The sympy library~\cite{Meurer2017} (verification of all symbolic
        calculations).
    \item The matplotlib~\cite{Droettboom2018} library (visualisation).
    \item The pandas~\cite{Structures2010}, dask~\cite{Dask2016} and
        NumPy~\cite{Oliphant2015} libraries (data manipulation).
    \item The SciPy~\cite{Jones2001} library (numerical integration of the
        replicator equation).
\end{itemize}

This work was performed using the computational facilities of the Advanced
Research Computing @ Cardiff (ARCCA) Division, Cardiff University.

\printbibliography

\newpage
\section*{Supplementary materials}

\includepdf{assets/pdf/proof_of_form_of_extortionate_strategies/main.pdf}

\newpage

Using the pair wise interactions the transition rates \(p,
q\) can be measured and the steady state probabilities inferred and compared to
the actual probabilities of each state.
This is done numerically by computing the singular eigenvector of the
matrix \(A\) \cite{Stewart2009}:

\[
    A =
    \begin{bmatrix}
        p_1 q_1 & p_1 (1 - q_1) & (1 - p_1) q_1 & (1 -p_1) (1 - q_1) \\
        p_2 q_2 & p_2 (1 - q_2) & (1 - p_2) q_2 & (1 -p_2) (1 - q_2) \\
        p_3 q_3 & p_3 (1 - q_3) & (1 - p_3) q_3 & (1 -p_3) (1 - q_3) \\
        p_4 q_4 & p_4 (1 - q_4) & (1 - p_4) q_4 & (1 -p_4) (1 - q_4) \\
    \end{bmatrix}
\]

Figure~\ref{fig:computed_probabilities_vs_theoretic_probabilities} shows a
regression line fitted to every pairwise interaction with a reported
\(\text{SSError}\) value (pairwise interactions with missing states were
omitted). This serves to validate the approach: a part from some edge cases the
relationship is consistent.

\begin{figure}[!htbp]
    \centering
    \includegraphics[width=.8\textwidth]{./assets/img/computed_probabilities_vs_theoretic_probabilities/main.pdf}
    \caption{The
        relationship between the steady state probabilities inferred from the
        measured transitions and the actual steady state probabilities. A linear
        regression line is included validating the approach.}
    \label{fig:computed_probabilities_vs_theoretic_probabilities}
\end{figure}


\end{document}
 strategies,
was presented with specific consideration given to ZD strategies. This
tournament is reproduced here using the Axelrod-Python
project~\cite{Knight2016}. To obtain a good measure of the corresponding
transition rates for each strategy all matches have been run for
\documentclass[a4paper]{article}

\usepackage{amsmath}
\usepackage{amssymb}
\usepackage[margin=1.5cm,
            includefoot,
            footskip=30pt]{geometry}
\usepackage{layout}
\usepackage{graphicx}
\usepackage{subcaption}

\usepackage{biblatex}
\usepackage{pdfpages}

\bibliography{main.bib}

\title{Suspicion: Recognising and evaluating the effectiveness
       of extortion in the Iterated Prisoner's Dilemma}
\author{Vincent A. Knight \and Nikoleta E. Glynatsi}
\date{\today}



\begin{document}

\maketitle

\begin{abstract}
    The Iterated Prisoner's Dilemma is a model for rational and evolutionary
    interactive behaviour. It has applications both in the study of human social
    behaviour as well as in biology.
    It is used to understand when and how a rational individual might
    accept an immediate cost to their own utility for the direct benefit of
    another.

    Much attention has been given to a class of strategies called
    Zero Determinant strategies. It has been theoretically shown that these
    strategies can ``extort'' any player.

    In this work, an approach to identify if observed strategies are playing in
    an extortionate way is described. Furthermore, experimental analysis of
    a large tournament with \input{assets/tex/number_of_full_strategies/main.tex}
    strategies is considered. In this setting
    the most highly performing strategies do not play in an extortionate way
    against each other but do against lower performing strategies.
    This suggests that whilst the theory of Zero Determinant strategies
    indicates that memory is not of fundamental importance to the evolution of
    cooperative behaviour, this is incomplete.
\end{abstract}

\section{Introduction}\label{sec:introduction}

Agent based game theoretic models have become a stalwart of the underpinning
mathematics of interactive behaviours. One of the major pieces of work
in this area is the pair of original computer tournaments run by Robert
Axelrod~\cite{Axelrod1980, Axelrod1980a}. These tournaments pitted submitted
computer strategies against each other in plays of the Iterated Prisoner's
Dilemma. A common game where agents can choose to pay a slight cost to their
immediate utility in the hope of building a reputation. This has been used in
economic and evolutionary game theory to understand the evolution of cooperative
behaviour.

Recently, a class of strategies was described in~\cite{Press2012} that can
provably extort any given opponent. In~\cite{Hilbe2013, Moran1707} some
questions have already been asked about the true effectiveness of these
strategies in an evolutionary setting. Here another question is asked: is it
possible to recognise this extortionate behaviour? A mathematical procedure for
suspicion is presented: in the same way that the continued actions of an
extortionate individual might raise suspicion.

This work makes use of the Axelrod Python library~\cite{Knight2018, Knight2016}
with a large number of Prisoner Dilemma strategies available to give an
extensive numerical example of the ideas presented.  The approach is presented
in Section~\ref{sec:delta-zd-strategies}.  All of the code and data discussed
in Section~\ref{sec:numerical-experiments} is open sourced, archived and
written according to best scientific principles~\cite{Wilson2014}. The data
archive can be found at~\cite{vincent_knight_2018_1297075}.

\section{Recognising Extortion}\label{sec:delta-zd-strategies}

In~\cite{Press2012}, given a match between 2 memory-one strategies, the concept
of Zero Determinant (ZD) strategies is introduced. The main result of that paper
shows that given two memory one players \(p, q\in\mathbb{R}^4\) a linear
relationship between the players' scores could be forced by one of the players.

Using the notation of~\cite{Press2012}, assuming the utilities for player \(p\)
are given by \(S_x=(R, S, T, P)\) and for player \(q\) by \(S_y=(R, T, S, P)\)
and that the stationary scores of each player is given by \(S_X\) and \(S_Y\)
respectively. The main result of~\cite{Press2012} is that if

\begin{equation}\label{eqn:linear_relationship_for_p}
    \tilde p=\alpha S_x + \beta S_y + \gamma
\end{equation}

or

\begin{equation}\label{eqn:linear_relationship_for_q}
    \tilde q=\alpha S_x + \beta S_y + \gamma
\end{equation}

where \(\tilde p = (1 - p_1, 1 - p_2, p_3, p_4)\) and
\(\tilde q = (1 - q_1, 1 - q_2, q_3, q_4)\) then:

\begin{equation}
    \alpha S_X + \beta S_Y + \gamma = 0
\end{equation}

In~\cite{Press2012} a particular type of ZD strategy is defined: extortionate
strategies. If:

\begin{equation}\label{eqn:constraint_for_extortion}
    \gamma = - P(\alpha + \beta)
\end{equation}

then the player can ensure they get a score \(\chi\) times
larger than the opponent. This extortion coefficient is given by:

\begin{equation}\label{eqn:definition_of_chi}
    \chi=\frac{-\beta}{\alpha}
\end{equation}

Thus, if (\ref{eqn:constraint_for_extortion}) holds and \(\chi >1\) a player is
said to extort their opponent.
Here, the reverse problem is considered: given a
\(p\in\mathbb{R}^4\) how does one identify \(\alpha, \beta\) if they
exist and is the strategy in fact acting in an extortionate way?

These conditions correspond to:

\begin{align}
    \tilde p_1 & = \alpha R + \beta R - P (\alpha + \beta)
            \label{eqn:condition_for_tilde_p1}\\
    \tilde p_2 & = \alpha S + \beta T - P (\alpha + \beta)
            \label{eqn:condition_for_tilde_p2}\\
    \tilde p_3 & = \alpha T + \beta S - P (\alpha + \beta)
            \label{eqn:condition_for_tilde_p3}\\
    \tilde p_4 & = \alpha P + \beta P - P (\alpha + \beta)
            \label{eqn:condition_for_tilde_p4}
\end{align}

Equation (\ref{eqn:condition_for_tilde_p4}) ensures that \(p_4=\tilde p_4=0\).
Equations (\ref{eqn:condition_for_tilde_p1}-\ref{eqn:condition_for_tilde_p3})
can be used to eliminate \(\alpha, \beta\), giving:

\begin{equation}\label{eqn:planar_definition_of_extortion}
    \tilde p_1 = \frac{(R - P)(\tilde p_2 + \tilde p_3)}{S + T - 2P}
\end{equation}

with:

\begin{equation}\label{eqn:definition_of_chi}
    \chi = \frac{\tilde p_2 (P - T) + \tilde p_3 (S - P)}
                {\tilde p_2 (P - S) + \tilde p_3 (T - P)}
\end{equation}

Given a strategy \(p\in\mathbb{R}^{4\times 1}\) equations
(\ref{eqn:condition_for_tilde_p4}), (\ref{eqn:planar_definition_of_extortion}-\ref{eqn:definition_of_chi}) can be used to check if
a strategy is extortionate. The conditions correspond to:

\begin{align}
    p_1 & = \frac{(R-P)(p_2 + p_3) - R + T + S - P}{S + T - 2P}
     \label{eqn:condition_for_p1}\\
    p_4 & = 0 \label{eqn:condition_for_p4}\\
    1 & > p_2 + p_3\label{eqn:condition_for_chi}
\end{align}

The algebraic steps necessary to prove these results are available in the
supporting materials.

All extortionate strategies reside on a triangular (\ref{eqn:condition_for_chi})
plane (\ref{eqn:condition_for_p1}) in 3 dimensions (\ref{eqn:condition_for_p4}).
Using this formulation it can be seen that a necessary (but not sufficient)
condition for an extortionate strategy is that it cooperates on average less
than 50\% of the time when in a state of disagreement with the opponent.

As an example, consider the known extortionate strategy \(p=(8 / 9, 1 / 2, 1 /
3, 0)\) from~\cite{Stewart2012} which is referred to as \texttt{Extort-2}. In
this case, for the standard values of \((R, T, S, P)\) constraint
(\ref{eqn:condition_for_p1}) corresponds to:

\begin{equation}
    p_1 = \frac{2(p_2 + p_3) + 1}{3}
\end{equation}

It is clear that in this case all constraints hold.

This approach could in fact be used to confirm that a given strategy is acting
in an extortionate manner even if it is not a memory one strategy. However, in
practice, if a closed form for \(p\) is not known, then due to measurement
and/or numerical error this would not work.

This problem can be written in the following linear algebraic form where
\(x=(\alpha, \beta)\)
and \(p^*=(\tilde p_1 - 1, tilde_2 - 1, p_3)\):

\begin{equation}\label{eqn:linear_algebraic_equation_for_p}
    Cx= p^*
\end{equation}

\(C\) corresponds to equations
(\ref{eqn:condition_for_tilde_p1}-\ref{eqn:condition_for_tilde_p3}) and is
given by:

\begin{equation}\label{eqn:definition_of_C}
    C =
    \begin{bmatrix}
        R - P & R- P \\
        S - P & T- P \\
        T - P & S- P \\
    \end{bmatrix}
\end{equation}

Note that in general, equation (\ref{eqn:linear_algebraic_equation_for_p}) will
not necessarily have a solution. From the Rouch\'{e}-Capelli theorem if there is
a solution it is unique as \(\text{rank}(C)=2\) which is the dimension of the
variable \(x\). The best fitting \(x\) is found by minimizing:

\begin{equation}\label{eqn:r_squared}
    \text{SSError} = \|C x- p^*\|_2^2 = \sum_{i=1}^{3}\left((C\bar x)_i-p_i^*\right)^2
\end{equation}

Note that \(\text{SSError}\), which is the square of the Frobenius
norm~\cite{Golub2013}, becomes a measure of how close a strategy is to being an
extortionate strategy. Suspicion
of extortion then corresponds to a threshold on \(\text{SSError}\).

By observing interactions (human or otherwise), their memory one representation
can be inferred and this approach can be used to recognise extortionate
behaviour. The notion of comparing theoretic and actual plays of the IPD is not
novel, see for example~\cite{Rand2013}. Immediately it is noted that if the
environment is noisy~\cite{Wu1995} then no strategy can be considered to be
extortionate as \(p_4>0\).

In the next section, this idea will be illustrated by observing the interactions
that take place in a computer based tournament of the IPD\@.

\section{Numerical experiments}\label{sec:numerical-experiments}

In~\cite{Stewart2012} results from a tournament with
\input{./assets/tex/number_of_stewart_plotkin_strategies/main.tex} strategies,
was presented with specific consideration given to ZD strategies. This
tournament is reproduced here using the Axelrod-Python
project~\cite{Knight2016}. To obtain a good measure of the corresponding
transition rates for each strategy all matches have been run for
\input{assets/tex/number_of_turns/main.tex} turns and every match has been
repeated \input{assets/tex/number_of_repetitions/main.tex} times. All of this
interaction data is available at~\cite{vincent_knight_2018_1297075}. A good
match between the inferred Markov chain and the state distribution of the actual
interactions has been verified. Data for this is presented in the supplementary
materials.

Figure~\ref{fig:SSError_overall_in_stewart_plotkin} shows the \(\text{SSError}\)
values for all the strategies in the tournament, as reported
in~\cite{Stewart2012} the extortionate strategy (which has an expected
\(\text{SSError}\) approximately 0) gains a large number of wins.

\begin{figure}[!htbp]
    \centering
    \includegraphics[width=.8\textwidth]{./assets/img/SSError_overall_in_stewart_plotkin/main.pdf}
    \caption{\(\text{SSError}\) and state probabilities for the strategies
        of~\cite{Stewart2012}, ordered both by number of wins and overall score.
        Note that \(P(DC)\) is not shown as it corresponds to the transpose of
        \(P(CD)\). Cooperator and Defector are omitted as they do not visit all
        the states.}
    \label{fig:SSError_overall_in_stewart_plotkin}
\end{figure}

Here, the work of~\cite{Stewart2012} is extended by investigating a tournament
with \input{assets/tex/number_of_full_strategies/main.tex}
strategies.

The results of this analysis are shown in
Figure~\ref{fig:SSError_and_probabilities_in_full}. The top ranking strategies
by number of wins seem to be extortionate (but not against all strategies) and
it can be seen that a small sub group of strategies achieve mutual defection.
All the top ranking strategies according to score achieve mutual cooperation and
do not extort each other, however they
\textbf{do} exhibit extortionate behaviour towards a number of the lower ranking
strategies.

\begin{figure}[!htbp]
    \centering
    \includegraphics[width=.8\textwidth]{./assets/img/SSError_and_probabilities_in_full/main.pdf}
    \caption{\(\text{SSError}\) for the strategies for the full tournament. Only
    strategy interactions for which \(p_4=0\) and \(\chi>1\) are displayed.}
    \label{fig:SSError_and_probabilities_in_full}
\end{figure}

\section{Conclusion}\label{sec:conclusion}

This work defines an approach to measure whether or not a player is playing a
strategy that corresponds to an extortionate strategy as defined
in~\cite{Press2012}: a mathematical model for suspicion. Indeed, all
extortionate strategies have been
 classified as lying on a triangular plane.
This rigorous classification fails to be robust to small measurement error, thus
a statistical approach is proposed.
This is done through a linear algebraic approach for approximating the solution
of a linear system. Using this, a large number of pairwise interactions is
simulated and in fact very few strategies are found to act extortionately.

The work of~\cite{Press2012}, whilst showing that a clever approach to taking
advantage of another memory one strategy exists: this is incomplete. Whilst the
elegance of this result is very attractive, just as the simplicity of the
victory of Tit For Tat in Axelrod's original tournaments was, it is incomplete.
Extortionate strategies achieve a high number of wins but they do not
achieve a high score which corresponds to the fitness landscape in an
evolutionary sense. From the large number of interactions a payoff matrix \(S\)
can be measured where \(S_{ij}\) denotes the score (using standard values of
\((R, S, T, P) = (3, 0, 5, 1)\)) of the \(i\)th strategy
against the \(j\)th strategy. Using this, the replicator equation
describes the evolution of the system based on a population density fitness
function:

\begin{equation}\label{eqn:replicator_dynamics}
    \frac{dx}{dt} = x(S-x^TS x)
\end{equation}

Equation (\ref{eqn:replicator_dynamics}) is solved numerically through an
integration technique described in~\cite{Petzold1983} and
Figure~\ref{fig:replicator_dynamics} shows the evolution of the distribution of
the system: the various strategies are ranked by scores. It is clear to see that
only the high ranking strategies survive the evolutionary process (in fact,
only \input{./assets/img/replicator_dynamics/main.tex}
have a final distribution greater than \(10 ^ {-2}\)). This confirms the
findings of~\cite{Moran1707} in which sophisticated strategies resist
evolutionary invasion of shorter memory strategies. Recalling
Figure~\ref{fig:SSError_and_probabilities_in_full} this demonstrates that:

\begin{itemize}
    \item Cooperation emerges through the evolutionary process: the high scoring
        strategies do not exhibit extortionate behaviour towards each other.
    \item Extortionate strategies do not survive the evolutionary process.
\end{itemize}

\begin{figure}[!htbp]
    \centering
    \includegraphics[width=.8\textwidth]{./assets/img/replicator_dynamics/main.pdf}
    \caption{Numerical simulation of the replicator equation
    (\ref{eqn:replicator_dynamics}): strategies are ordered by score, only the strategies with a high score survive the evolutionary process.}
    \label{fig:replicator_dynamics}
\end{figure}

This work can be used to classify plays of the IPD\@: data can be collected from
actual interactions (in lab or in the field). Furthermore, this allows for a
classification method similar to the notion of fingerprinting presented
in~\cite{Ashlock2008}. Trained strategies can potentially be classified as
extortionate or not or it could be possible to even constrain the reinforcement
learning approaches that are becoming prevalent in the literature.
Alternatively, this mathematical approach for recognising extortion could be
used in sophisticated strategies to defend against invasion. Arguably, some of
the strategies considered here exhibit this behaviour, indeed as described
in~\cite{Harper2017}, the top ranking strategies in the full tournament are
obtained using evolutionary reinforcement learning techniques, thus, suspicion
of extortionate behaviour could in fact be an evolutionary trait.

\section*{Acknowledgements}

The following open source software libraries were used in this research:

\begin{itemize}
    \item The Axelrod ~\cite{Knight2016, Knight2018} library (IPD strategies and
        tournaments).
    \item The sympy library~\cite{Meurer2017} (verification of all symbolic
        calculations).
    \item The matplotlib~\cite{Droettboom2018} library (visualisation).
    \item The pandas~\cite{Structures2010}, dask~\cite{Dask2016} and
        NumPy~\cite{Oliphant2015} libraries (data manipulation).
    \item The SciPy~\cite{Jones2001} library (numerical integration of the
        replicator equation).
\end{itemize}

This work was performed using the computational facilities of the Advanced
Research Computing @ Cardiff (ARCCA) Division, Cardiff University.

\printbibliography

\newpage
\section*{Supplementary materials}

\includepdf{assets/pdf/proof_of_form_of_extortionate_strategies/main.pdf}

\newpage

Using the pair wise interactions the transition rates \(p,
q\) can be measured and the steady state probabilities inferred and compared to
the actual probabilities of each state.
This is done numerically by computing the singular eigenvector of the
matrix \(A\) \cite{Stewart2009}:

\[
    A =
    \begin{bmatrix}
        p_1 q_1 & p_1 (1 - q_1) & (1 - p_1) q_1 & (1 -p_1) (1 - q_1) \\
        p_2 q_2 & p_2 (1 - q_2) & (1 - p_2) q_2 & (1 -p_2) (1 - q_2) \\
        p_3 q_3 & p_3 (1 - q_3) & (1 - p_3) q_3 & (1 -p_3) (1 - q_3) \\
        p_4 q_4 & p_4 (1 - q_4) & (1 - p_4) q_4 & (1 -p_4) (1 - q_4) \\
    \end{bmatrix}
\]

Figure~\ref{fig:computed_probabilities_vs_theoretic_probabilities} shows a
regression line fitted to every pairwise interaction with a reported
\(\text{SSError}\) value (pairwise interactions with missing states were
omitted). This serves to validate the approach: a part from some edge cases the
relationship is consistent.

\begin{figure}[!htbp]
    \centering
    \includegraphics[width=.8\textwidth]{./assets/img/computed_probabilities_vs_theoretic_probabilities/main.pdf}
    \caption{The
        relationship between the steady state probabilities inferred from the
        measured transitions and the actual steady state probabilities. A linear
        regression line is included validating the approach.}
    \label{fig:computed_probabilities_vs_theoretic_probabilities}
\end{figure}


\end{document}
 turns and every match has been
repeated \documentclass[a4paper]{article}

\usepackage{amsmath}
\usepackage{amssymb}
\usepackage[margin=1.5cm,
            includefoot,
            footskip=30pt]{geometry}
\usepackage{layout}
\usepackage{graphicx}
\usepackage{subcaption}

\usepackage{biblatex}
\usepackage{pdfpages}

\bibliography{main.bib}

\title{Suspicion: Recognising and evaluating the effectiveness
       of extortion in the Iterated Prisoner's Dilemma}
\author{Vincent A. Knight \and Nikoleta E. Glynatsi}
\date{\today}



\begin{document}

\maketitle

\begin{abstract}
    The Iterated Prisoner's Dilemma is a model for rational and evolutionary
    interactive behaviour. It has applications both in the study of human social
    behaviour as well as in biology.
    It is used to understand when and how a rational individual might
    accept an immediate cost to their own utility for the direct benefit of
    another.

    Much attention has been given to a class of strategies called
    Zero Determinant strategies. It has been theoretically shown that these
    strategies can ``extort'' any player.

    In this work, an approach to identify if observed strategies are playing in
    an extortionate way is described. Furthermore, experimental analysis of
    a large tournament with \input{assets/tex/number_of_full_strategies/main.tex}
    strategies is considered. In this setting
    the most highly performing strategies do not play in an extortionate way
    against each other but do against lower performing strategies.
    This suggests that whilst the theory of Zero Determinant strategies
    indicates that memory is not of fundamental importance to the evolution of
    cooperative behaviour, this is incomplete.
\end{abstract}

\section{Introduction}\label{sec:introduction}

Agent based game theoretic models have become a stalwart of the underpinning
mathematics of interactive behaviours. One of the major pieces of work
in this area is the pair of original computer tournaments run by Robert
Axelrod~\cite{Axelrod1980, Axelrod1980a}. These tournaments pitted submitted
computer strategies against each other in plays of the Iterated Prisoner's
Dilemma. A common game where agents can choose to pay a slight cost to their
immediate utility in the hope of building a reputation. This has been used in
economic and evolutionary game theory to understand the evolution of cooperative
behaviour.

Recently, a class of strategies was described in~\cite{Press2012} that can
provably extort any given opponent. In~\cite{Hilbe2013, Moran1707} some
questions have already been asked about the true effectiveness of these
strategies in an evolutionary setting. Here another question is asked: is it
possible to recognise this extortionate behaviour? A mathematical procedure for
suspicion is presented: in the same way that the continued actions of an
extortionate individual might raise suspicion.

This work makes use of the Axelrod Python library~\cite{Knight2018, Knight2016}
with a large number of Prisoner Dilemma strategies available to give an
extensive numerical example of the ideas presented.  The approach is presented
in Section~\ref{sec:delta-zd-strategies}.  All of the code and data discussed
in Section~\ref{sec:numerical-experiments} is open sourced, archived and
written according to best scientific principles~\cite{Wilson2014}. The data
archive can be found at~\cite{vincent_knight_2018_1297075}.

\section{Recognising Extortion}\label{sec:delta-zd-strategies}

In~\cite{Press2012}, given a match between 2 memory-one strategies, the concept
of Zero Determinant (ZD) strategies is introduced. The main result of that paper
shows that given two memory one players \(p, q\in\mathbb{R}^4\) a linear
relationship between the players' scores could be forced by one of the players.

Using the notation of~\cite{Press2012}, assuming the utilities for player \(p\)
are given by \(S_x=(R, S, T, P)\) and for player \(q\) by \(S_y=(R, T, S, P)\)
and that the stationary scores of each player is given by \(S_X\) and \(S_Y\)
respectively. The main result of~\cite{Press2012} is that if

\begin{equation}\label{eqn:linear_relationship_for_p}
    \tilde p=\alpha S_x + \beta S_y + \gamma
\end{equation}

or

\begin{equation}\label{eqn:linear_relationship_for_q}
    \tilde q=\alpha S_x + \beta S_y + \gamma
\end{equation}

where \(\tilde p = (1 - p_1, 1 - p_2, p_3, p_4)\) and
\(\tilde q = (1 - q_1, 1 - q_2, q_3, q_4)\) then:

\begin{equation}
    \alpha S_X + \beta S_Y + \gamma = 0
\end{equation}

In~\cite{Press2012} a particular type of ZD strategy is defined: extortionate
strategies. If:

\begin{equation}\label{eqn:constraint_for_extortion}
    \gamma = - P(\alpha + \beta)
\end{equation}

then the player can ensure they get a score \(\chi\) times
larger than the opponent. This extortion coefficient is given by:

\begin{equation}\label{eqn:definition_of_chi}
    \chi=\frac{-\beta}{\alpha}
\end{equation}

Thus, if (\ref{eqn:constraint_for_extortion}) holds and \(\chi >1\) a player is
said to extort their opponent.
Here, the reverse problem is considered: given a
\(p\in\mathbb{R}^4\) how does one identify \(\alpha, \beta\) if they
exist and is the strategy in fact acting in an extortionate way?

These conditions correspond to:

\begin{align}
    \tilde p_1 & = \alpha R + \beta R - P (\alpha + \beta)
            \label{eqn:condition_for_tilde_p1}\\
    \tilde p_2 & = \alpha S + \beta T - P (\alpha + \beta)
            \label{eqn:condition_for_tilde_p2}\\
    \tilde p_3 & = \alpha T + \beta S - P (\alpha + \beta)
            \label{eqn:condition_for_tilde_p3}\\
    \tilde p_4 & = \alpha P + \beta P - P (\alpha + \beta)
            \label{eqn:condition_for_tilde_p4}
\end{align}

Equation (\ref{eqn:condition_for_tilde_p4}) ensures that \(p_4=\tilde p_4=0\).
Equations (\ref{eqn:condition_for_tilde_p1}-\ref{eqn:condition_for_tilde_p3})
can be used to eliminate \(\alpha, \beta\), giving:

\begin{equation}\label{eqn:planar_definition_of_extortion}
    \tilde p_1 = \frac{(R - P)(\tilde p_2 + \tilde p_3)}{S + T - 2P}
\end{equation}

with:

\begin{equation}\label{eqn:definition_of_chi}
    \chi = \frac{\tilde p_2 (P - T) + \tilde p_3 (S - P)}
                {\tilde p_2 (P - S) + \tilde p_3 (T - P)}
\end{equation}

Given a strategy \(p\in\mathbb{R}^{4\times 1}\) equations
(\ref{eqn:condition_for_tilde_p4}), (\ref{eqn:planar_definition_of_extortion}-\ref{eqn:definition_of_chi}) can be used to check if
a strategy is extortionate. The conditions correspond to:

\begin{align}
    p_1 & = \frac{(R-P)(p_2 + p_3) - R + T + S - P}{S + T - 2P}
     \label{eqn:condition_for_p1}\\
    p_4 & = 0 \label{eqn:condition_for_p4}\\
    1 & > p_2 + p_3\label{eqn:condition_for_chi}
\end{align}

The algebraic steps necessary to prove these results are available in the
supporting materials.

All extortionate strategies reside on a triangular (\ref{eqn:condition_for_chi})
plane (\ref{eqn:condition_for_p1}) in 3 dimensions (\ref{eqn:condition_for_p4}).
Using this formulation it can be seen that a necessary (but not sufficient)
condition for an extortionate strategy is that it cooperates on average less
than 50\% of the time when in a state of disagreement with the opponent.

As an example, consider the known extortionate strategy \(p=(8 / 9, 1 / 2, 1 /
3, 0)\) from~\cite{Stewart2012} which is referred to as \texttt{Extort-2}. In
this case, for the standard values of \((R, T, S, P)\) constraint
(\ref{eqn:condition_for_p1}) corresponds to:

\begin{equation}
    p_1 = \frac{2(p_2 + p_3) + 1}{3}
\end{equation}

It is clear that in this case all constraints hold.

This approach could in fact be used to confirm that a given strategy is acting
in an extortionate manner even if it is not a memory one strategy. However, in
practice, if a closed form for \(p\) is not known, then due to measurement
and/or numerical error this would not work.

This problem can be written in the following linear algebraic form where
\(x=(\alpha, \beta)\)
and \(p^*=(\tilde p_1 - 1, tilde_2 - 1, p_3)\):

\begin{equation}\label{eqn:linear_algebraic_equation_for_p}
    Cx= p^*
\end{equation}

\(C\) corresponds to equations
(\ref{eqn:condition_for_tilde_p1}-\ref{eqn:condition_for_tilde_p3}) and is
given by:

\begin{equation}\label{eqn:definition_of_C}
    C =
    \begin{bmatrix}
        R - P & R- P \\
        S - P & T- P \\
        T - P & S- P \\
    \end{bmatrix}
\end{equation}

Note that in general, equation (\ref{eqn:linear_algebraic_equation_for_p}) will
not necessarily have a solution. From the Rouch\'{e}-Capelli theorem if there is
a solution it is unique as \(\text{rank}(C)=2\) which is the dimension of the
variable \(x\). The best fitting \(x\) is found by minimizing:

\begin{equation}\label{eqn:r_squared}
    \text{SSError} = \|C x- p^*\|_2^2 = \sum_{i=1}^{3}\left((C\bar x)_i-p_i^*\right)^2
\end{equation}

Note that \(\text{SSError}\), which is the square of the Frobenius
norm~\cite{Golub2013}, becomes a measure of how close a strategy is to being an
extortionate strategy. Suspicion
of extortion then corresponds to a threshold on \(\text{SSError}\).

By observing interactions (human or otherwise), their memory one representation
can be inferred and this approach can be used to recognise extortionate
behaviour. The notion of comparing theoretic and actual plays of the IPD is not
novel, see for example~\cite{Rand2013}. Immediately it is noted that if the
environment is noisy~\cite{Wu1995} then no strategy can be considered to be
extortionate as \(p_4>0\).

In the next section, this idea will be illustrated by observing the interactions
that take place in a computer based tournament of the IPD\@.

\section{Numerical experiments}\label{sec:numerical-experiments}

In~\cite{Stewart2012} results from a tournament with
\input{./assets/tex/number_of_stewart_plotkin_strategies/main.tex} strategies,
was presented with specific consideration given to ZD strategies. This
tournament is reproduced here using the Axelrod-Python
project~\cite{Knight2016}. To obtain a good measure of the corresponding
transition rates for each strategy all matches have been run for
\input{assets/tex/number_of_turns/main.tex} turns and every match has been
repeated \input{assets/tex/number_of_repetitions/main.tex} times. All of this
interaction data is available at~\cite{vincent_knight_2018_1297075}. A good
match between the inferred Markov chain and the state distribution of the actual
interactions has been verified. Data for this is presented in the supplementary
materials.

Figure~\ref{fig:SSError_overall_in_stewart_plotkin} shows the \(\text{SSError}\)
values for all the strategies in the tournament, as reported
in~\cite{Stewart2012} the extortionate strategy (which has an expected
\(\text{SSError}\) approximately 0) gains a large number of wins.

\begin{figure}[!htbp]
    \centering
    \includegraphics[width=.8\textwidth]{./assets/img/SSError_overall_in_stewart_plotkin/main.pdf}
    \caption{\(\text{SSError}\) and state probabilities for the strategies
        of~\cite{Stewart2012}, ordered both by number of wins and overall score.
        Note that \(P(DC)\) is not shown as it corresponds to the transpose of
        \(P(CD)\). Cooperator and Defector are omitted as they do not visit all
        the states.}
    \label{fig:SSError_overall_in_stewart_plotkin}
\end{figure}

Here, the work of~\cite{Stewart2012} is extended by investigating a tournament
with \input{assets/tex/number_of_full_strategies/main.tex}
strategies.

The results of this analysis are shown in
Figure~\ref{fig:SSError_and_probabilities_in_full}. The top ranking strategies
by number of wins seem to be extortionate (but not against all strategies) and
it can be seen that a small sub group of strategies achieve mutual defection.
All the top ranking strategies according to score achieve mutual cooperation and
do not extort each other, however they
\textbf{do} exhibit extortionate behaviour towards a number of the lower ranking
strategies.

\begin{figure}[!htbp]
    \centering
    \includegraphics[width=.8\textwidth]{./assets/img/SSError_and_probabilities_in_full/main.pdf}
    \caption{\(\text{SSError}\) for the strategies for the full tournament. Only
    strategy interactions for which \(p_4=0\) and \(\chi>1\) are displayed.}
    \label{fig:SSError_and_probabilities_in_full}
\end{figure}

\section{Conclusion}\label{sec:conclusion}

This work defines an approach to measure whether or not a player is playing a
strategy that corresponds to an extortionate strategy as defined
in~\cite{Press2012}: a mathematical model for suspicion. Indeed, all
extortionate strategies have been
 classified as lying on a triangular plane.
This rigorous classification fails to be robust to small measurement error, thus
a statistical approach is proposed.
This is done through a linear algebraic approach for approximating the solution
of a linear system. Using this, a large number of pairwise interactions is
simulated and in fact very few strategies are found to act extortionately.

The work of~\cite{Press2012}, whilst showing that a clever approach to taking
advantage of another memory one strategy exists: this is incomplete. Whilst the
elegance of this result is very attractive, just as the simplicity of the
victory of Tit For Tat in Axelrod's original tournaments was, it is incomplete.
Extortionate strategies achieve a high number of wins but they do not
achieve a high score which corresponds to the fitness landscape in an
evolutionary sense. From the large number of interactions a payoff matrix \(S\)
can be measured where \(S_{ij}\) denotes the score (using standard values of
\((R, S, T, P) = (3, 0, 5, 1)\)) of the \(i\)th strategy
against the \(j\)th strategy. Using this, the replicator equation
describes the evolution of the system based on a population density fitness
function:

\begin{equation}\label{eqn:replicator_dynamics}
    \frac{dx}{dt} = x(S-x^TS x)
\end{equation}

Equation (\ref{eqn:replicator_dynamics}) is solved numerically through an
integration technique described in~\cite{Petzold1983} and
Figure~\ref{fig:replicator_dynamics} shows the evolution of the distribution of
the system: the various strategies are ranked by scores. It is clear to see that
only the high ranking strategies survive the evolutionary process (in fact,
only \input{./assets/img/replicator_dynamics/main.tex}
have a final distribution greater than \(10 ^ {-2}\)). This confirms the
findings of~\cite{Moran1707} in which sophisticated strategies resist
evolutionary invasion of shorter memory strategies. Recalling
Figure~\ref{fig:SSError_and_probabilities_in_full} this demonstrates that:

\begin{itemize}
    \item Cooperation emerges through the evolutionary process: the high scoring
        strategies do not exhibit extortionate behaviour towards each other.
    \item Extortionate strategies do not survive the evolutionary process.
\end{itemize}

\begin{figure}[!htbp]
    \centering
    \includegraphics[width=.8\textwidth]{./assets/img/replicator_dynamics/main.pdf}
    \caption{Numerical simulation of the replicator equation
    (\ref{eqn:replicator_dynamics}): strategies are ordered by score, only the strategies with a high score survive the evolutionary process.}
    \label{fig:replicator_dynamics}
\end{figure}

This work can be used to classify plays of the IPD\@: data can be collected from
actual interactions (in lab or in the field). Furthermore, this allows for a
classification method similar to the notion of fingerprinting presented
in~\cite{Ashlock2008}. Trained strategies can potentially be classified as
extortionate or not or it could be possible to even constrain the reinforcement
learning approaches that are becoming prevalent in the literature.
Alternatively, this mathematical approach for recognising extortion could be
used in sophisticated strategies to defend against invasion. Arguably, some of
the strategies considered here exhibit this behaviour, indeed as described
in~\cite{Harper2017}, the top ranking strategies in the full tournament are
obtained using evolutionary reinforcement learning techniques, thus, suspicion
of extortionate behaviour could in fact be an evolutionary trait.

\section*{Acknowledgements}

The following open source software libraries were used in this research:

\begin{itemize}
    \item The Axelrod ~\cite{Knight2016, Knight2018} library (IPD strategies and
        tournaments).
    \item The sympy library~\cite{Meurer2017} (verification of all symbolic
        calculations).
    \item The matplotlib~\cite{Droettboom2018} library (visualisation).
    \item The pandas~\cite{Structures2010}, dask~\cite{Dask2016} and
        NumPy~\cite{Oliphant2015} libraries (data manipulation).
    \item The SciPy~\cite{Jones2001} library (numerical integration of the
        replicator equation).
\end{itemize}

This work was performed using the computational facilities of the Advanced
Research Computing @ Cardiff (ARCCA) Division, Cardiff University.

\printbibliography

\newpage
\section*{Supplementary materials}

\includepdf{assets/pdf/proof_of_form_of_extortionate_strategies/main.pdf}

\newpage

Using the pair wise interactions the transition rates \(p,
q\) can be measured and the steady state probabilities inferred and compared to
the actual probabilities of each state.
This is done numerically by computing the singular eigenvector of the
matrix \(A\) \cite{Stewart2009}:

\[
    A =
    \begin{bmatrix}
        p_1 q_1 & p_1 (1 - q_1) & (1 - p_1) q_1 & (1 -p_1) (1 - q_1) \\
        p_2 q_2 & p_2 (1 - q_2) & (1 - p_2) q_2 & (1 -p_2) (1 - q_2) \\
        p_3 q_3 & p_3 (1 - q_3) & (1 - p_3) q_3 & (1 -p_3) (1 - q_3) \\
        p_4 q_4 & p_4 (1 - q_4) & (1 - p_4) q_4 & (1 -p_4) (1 - q_4) \\
    \end{bmatrix}
\]

Figure~\ref{fig:computed_probabilities_vs_theoretic_probabilities} shows a
regression line fitted to every pairwise interaction with a reported
\(\text{SSError}\) value (pairwise interactions with missing states were
omitted). This serves to validate the approach: a part from some edge cases the
relationship is consistent.

\begin{figure}[!htbp]
    \centering
    \includegraphics[width=.8\textwidth]{./assets/img/computed_probabilities_vs_theoretic_probabilities/main.pdf}
    \caption{The
        relationship between the steady state probabilities inferred from the
        measured transitions and the actual steady state probabilities. A linear
        regression line is included validating the approach.}
    \label{fig:computed_probabilities_vs_theoretic_probabilities}
\end{figure}


\end{document}
 times. All of this
interaction data is available at~\cite{vincent_knight_2018_1297075}. A good
match between the inferred Markov chain and the state distribution of the actual
interactions has been verified. Data for this is presented in the supplementary
materials.

Figure~\ref{fig:SSError_overall_in_stewart_plotkin} shows the \(\text{SSError}\)
values for all the strategies in the tournament, as reported
in~\cite{Stewart2012} the extortionate strategy (which has an expected
\(\text{SSError}\) approximately 0) gains a large number of wins.

\begin{figure}[!htbp]
    \centering
    \includegraphics[width=.8\textwidth]{./assets/img/SSError_overall_in_stewart_plotkin/main.pdf}
    \caption{\(\text{SSError}\) and state probabilities for the strategies
        of~\cite{Stewart2012}, ordered both by number of wins and overall score.
        Note that \(P(DC)\) is not shown as it corresponds to the transpose of
        \(P(CD)\). Cooperator and Defector are omitted as they do not visit all
        the states.}
    \label{fig:SSError_overall_in_stewart_plotkin}
\end{figure}

Here, the work of~\cite{Stewart2012} is extended by investigating a tournament
with \documentclass[a4paper]{article}

\usepackage{amsmath}
\usepackage{amssymb}
\usepackage[margin=1.5cm,
            includefoot,
            footskip=30pt]{geometry}
\usepackage{layout}
\usepackage{graphicx}
\usepackage{subcaption}

\usepackage{biblatex}
\usepackage{pdfpages}

\bibliography{main.bib}

\title{Suspicion: Recognising and evaluating the effectiveness
       of extortion in the Iterated Prisoner's Dilemma}
\author{Vincent A. Knight \and Nikoleta E. Glynatsi}
\date{\today}



\begin{document}

\maketitle

\begin{abstract}
    The Iterated Prisoner's Dilemma is a model for rational and evolutionary
    interactive behaviour. It has applications both in the study of human social
    behaviour as well as in biology.
    It is used to understand when and how a rational individual might
    accept an immediate cost to their own utility for the direct benefit of
    another.

    Much attention has been given to a class of strategies called
    Zero Determinant strategies. It has been theoretically shown that these
    strategies can ``extort'' any player.

    In this work, an approach to identify if observed strategies are playing in
    an extortionate way is described. Furthermore, experimental analysis of
    a large tournament with \input{assets/tex/number_of_full_strategies/main.tex}
    strategies is considered. In this setting
    the most highly performing strategies do not play in an extortionate way
    against each other but do against lower performing strategies.
    This suggests that whilst the theory of Zero Determinant strategies
    indicates that memory is not of fundamental importance to the evolution of
    cooperative behaviour, this is incomplete.
\end{abstract}

\section{Introduction}\label{sec:introduction}

Agent based game theoretic models have become a stalwart of the underpinning
mathematics of interactive behaviours. One of the major pieces of work
in this area is the pair of original computer tournaments run by Robert
Axelrod~\cite{Axelrod1980, Axelrod1980a}. These tournaments pitted submitted
computer strategies against each other in plays of the Iterated Prisoner's
Dilemma. A common game where agents can choose to pay a slight cost to their
immediate utility in the hope of building a reputation. This has been used in
economic and evolutionary game theory to understand the evolution of cooperative
behaviour.

Recently, a class of strategies was described in~\cite{Press2012} that can
provably extort any given opponent. In~\cite{Hilbe2013, Moran1707} some
questions have already been asked about the true effectiveness of these
strategies in an evolutionary setting. Here another question is asked: is it
possible to recognise this extortionate behaviour? A mathematical procedure for
suspicion is presented: in the same way that the continued actions of an
extortionate individual might raise suspicion.

This work makes use of the Axelrod Python library~\cite{Knight2018, Knight2016}
with a large number of Prisoner Dilemma strategies available to give an
extensive numerical example of the ideas presented.  The approach is presented
in Section~\ref{sec:delta-zd-strategies}.  All of the code and data discussed
in Section~\ref{sec:numerical-experiments} is open sourced, archived and
written according to best scientific principles~\cite{Wilson2014}. The data
archive can be found at~\cite{vincent_knight_2018_1297075}.

\section{Recognising Extortion}\label{sec:delta-zd-strategies}

In~\cite{Press2012}, given a match between 2 memory-one strategies, the concept
of Zero Determinant (ZD) strategies is introduced. The main result of that paper
shows that given two memory one players \(p, q\in\mathbb{R}^4\) a linear
relationship between the players' scores could be forced by one of the players.

Using the notation of~\cite{Press2012}, assuming the utilities for player \(p\)
are given by \(S_x=(R, S, T, P)\) and for player \(q\) by \(S_y=(R, T, S, P)\)
and that the stationary scores of each player is given by \(S_X\) and \(S_Y\)
respectively. The main result of~\cite{Press2012} is that if

\begin{equation}\label{eqn:linear_relationship_for_p}
    \tilde p=\alpha S_x + \beta S_y + \gamma
\end{equation}

or

\begin{equation}\label{eqn:linear_relationship_for_q}
    \tilde q=\alpha S_x + \beta S_y + \gamma
\end{equation}

where \(\tilde p = (1 - p_1, 1 - p_2, p_3, p_4)\) and
\(\tilde q = (1 - q_1, 1 - q_2, q_3, q_4)\) then:

\begin{equation}
    \alpha S_X + \beta S_Y + \gamma = 0
\end{equation}

In~\cite{Press2012} a particular type of ZD strategy is defined: extortionate
strategies. If:

\begin{equation}\label{eqn:constraint_for_extortion}
    \gamma = - P(\alpha + \beta)
\end{equation}

then the player can ensure they get a score \(\chi\) times
larger than the opponent. This extortion coefficient is given by:

\begin{equation}\label{eqn:definition_of_chi}
    \chi=\frac{-\beta}{\alpha}
\end{equation}

Thus, if (\ref{eqn:constraint_for_extortion}) holds and \(\chi >1\) a player is
said to extort their opponent.
Here, the reverse problem is considered: given a
\(p\in\mathbb{R}^4\) how does one identify \(\alpha, \beta\) if they
exist and is the strategy in fact acting in an extortionate way?

These conditions correspond to:

\begin{align}
    \tilde p_1 & = \alpha R + \beta R - P (\alpha + \beta)
            \label{eqn:condition_for_tilde_p1}\\
    \tilde p_2 & = \alpha S + \beta T - P (\alpha + \beta)
            \label{eqn:condition_for_tilde_p2}\\
    \tilde p_3 & = \alpha T + \beta S - P (\alpha + \beta)
            \label{eqn:condition_for_tilde_p3}\\
    \tilde p_4 & = \alpha P + \beta P - P (\alpha + \beta)
            \label{eqn:condition_for_tilde_p4}
\end{align}

Equation (\ref{eqn:condition_for_tilde_p4}) ensures that \(p_4=\tilde p_4=0\).
Equations (\ref{eqn:condition_for_tilde_p1}-\ref{eqn:condition_for_tilde_p3})
can be used to eliminate \(\alpha, \beta\), giving:

\begin{equation}\label{eqn:planar_definition_of_extortion}
    \tilde p_1 = \frac{(R - P)(\tilde p_2 + \tilde p_3)}{S + T - 2P}
\end{equation}

with:

\begin{equation}\label{eqn:definition_of_chi}
    \chi = \frac{\tilde p_2 (P - T) + \tilde p_3 (S - P)}
                {\tilde p_2 (P - S) + \tilde p_3 (T - P)}
\end{equation}

Given a strategy \(p\in\mathbb{R}^{4\times 1}\) equations
(\ref{eqn:condition_for_tilde_p4}), (\ref{eqn:planar_definition_of_extortion}-\ref{eqn:definition_of_chi}) can be used to check if
a strategy is extortionate. The conditions correspond to:

\begin{align}
    p_1 & = \frac{(R-P)(p_2 + p_3) - R + T + S - P}{S + T - 2P}
     \label{eqn:condition_for_p1}\\
    p_4 & = 0 \label{eqn:condition_for_p4}\\
    1 & > p_2 + p_3\label{eqn:condition_for_chi}
\end{align}

The algebraic steps necessary to prove these results are available in the
supporting materials.

All extortionate strategies reside on a triangular (\ref{eqn:condition_for_chi})
plane (\ref{eqn:condition_for_p1}) in 3 dimensions (\ref{eqn:condition_for_p4}).
Using this formulation it can be seen that a necessary (but not sufficient)
condition for an extortionate strategy is that it cooperates on average less
than 50\% of the time when in a state of disagreement with the opponent.

As an example, consider the known extortionate strategy \(p=(8 / 9, 1 / 2, 1 /
3, 0)\) from~\cite{Stewart2012} which is referred to as \texttt{Extort-2}. In
this case, for the standard values of \((R, T, S, P)\) constraint
(\ref{eqn:condition_for_p1}) corresponds to:

\begin{equation}
    p_1 = \frac{2(p_2 + p_3) + 1}{3}
\end{equation}

It is clear that in this case all constraints hold.

This approach could in fact be used to confirm that a given strategy is acting
in an extortionate manner even if it is not a memory one strategy. However, in
practice, if a closed form for \(p\) is not known, then due to measurement
and/or numerical error this would not work.

This problem can be written in the following linear algebraic form where
\(x=(\alpha, \beta)\)
and \(p^*=(\tilde p_1 - 1, tilde_2 - 1, p_3)\):

\begin{equation}\label{eqn:linear_algebraic_equation_for_p}
    Cx= p^*
\end{equation}

\(C\) corresponds to equations
(\ref{eqn:condition_for_tilde_p1}-\ref{eqn:condition_for_tilde_p3}) and is
given by:

\begin{equation}\label{eqn:definition_of_C}
    C =
    \begin{bmatrix}
        R - P & R- P \\
        S - P & T- P \\
        T - P & S- P \\
    \end{bmatrix}
\end{equation}

Note that in general, equation (\ref{eqn:linear_algebraic_equation_for_p}) will
not necessarily have a solution. From the Rouch\'{e}-Capelli theorem if there is
a solution it is unique as \(\text{rank}(C)=2\) which is the dimension of the
variable \(x\). The best fitting \(x\) is found by minimizing:

\begin{equation}\label{eqn:r_squared}
    \text{SSError} = \|C x- p^*\|_2^2 = \sum_{i=1}^{3}\left((C\bar x)_i-p_i^*\right)^2
\end{equation}

Note that \(\text{SSError}\), which is the square of the Frobenius
norm~\cite{Golub2013}, becomes a measure of how close a strategy is to being an
extortionate strategy. Suspicion
of extortion then corresponds to a threshold on \(\text{SSError}\).

By observing interactions (human or otherwise), their memory one representation
can be inferred and this approach can be used to recognise extortionate
behaviour. The notion of comparing theoretic and actual plays of the IPD is not
novel, see for example~\cite{Rand2013}. Immediately it is noted that if the
environment is noisy~\cite{Wu1995} then no strategy can be considered to be
extortionate as \(p_4>0\).

In the next section, this idea will be illustrated by observing the interactions
that take place in a computer based tournament of the IPD\@.

\section{Numerical experiments}\label{sec:numerical-experiments}

In~\cite{Stewart2012} results from a tournament with
\input{./assets/tex/number_of_stewart_plotkin_strategies/main.tex} strategies,
was presented with specific consideration given to ZD strategies. This
tournament is reproduced here using the Axelrod-Python
project~\cite{Knight2016}. To obtain a good measure of the corresponding
transition rates for each strategy all matches have been run for
\input{assets/tex/number_of_turns/main.tex} turns and every match has been
repeated \input{assets/tex/number_of_repetitions/main.tex} times. All of this
interaction data is available at~\cite{vincent_knight_2018_1297075}. A good
match between the inferred Markov chain and the state distribution of the actual
interactions has been verified. Data for this is presented in the supplementary
materials.

Figure~\ref{fig:SSError_overall_in_stewart_plotkin} shows the \(\text{SSError}\)
values for all the strategies in the tournament, as reported
in~\cite{Stewart2012} the extortionate strategy (which has an expected
\(\text{SSError}\) approximately 0) gains a large number of wins.

\begin{figure}[!htbp]
    \centering
    \includegraphics[width=.8\textwidth]{./assets/img/SSError_overall_in_stewart_plotkin/main.pdf}
    \caption{\(\text{SSError}\) and state probabilities for the strategies
        of~\cite{Stewart2012}, ordered both by number of wins and overall score.
        Note that \(P(DC)\) is not shown as it corresponds to the transpose of
        \(P(CD)\). Cooperator and Defector are omitted as they do not visit all
        the states.}
    \label{fig:SSError_overall_in_stewart_plotkin}
\end{figure}

Here, the work of~\cite{Stewart2012} is extended by investigating a tournament
with \input{assets/tex/number_of_full_strategies/main.tex}
strategies.

The results of this analysis are shown in
Figure~\ref{fig:SSError_and_probabilities_in_full}. The top ranking strategies
by number of wins seem to be extortionate (but not against all strategies) and
it can be seen that a small sub group of strategies achieve mutual defection.
All the top ranking strategies according to score achieve mutual cooperation and
do not extort each other, however they
\textbf{do} exhibit extortionate behaviour towards a number of the lower ranking
strategies.

\begin{figure}[!htbp]
    \centering
    \includegraphics[width=.8\textwidth]{./assets/img/SSError_and_probabilities_in_full/main.pdf}
    \caption{\(\text{SSError}\) for the strategies for the full tournament. Only
    strategy interactions for which \(p_4=0\) and \(\chi>1\) are displayed.}
    \label{fig:SSError_and_probabilities_in_full}
\end{figure}

\section{Conclusion}\label{sec:conclusion}

This work defines an approach to measure whether or not a player is playing a
strategy that corresponds to an extortionate strategy as defined
in~\cite{Press2012}: a mathematical model for suspicion. Indeed, all
extortionate strategies have been
 classified as lying on a triangular plane.
This rigorous classification fails to be robust to small measurement error, thus
a statistical approach is proposed.
This is done through a linear algebraic approach for approximating the solution
of a linear system. Using this, a large number of pairwise interactions is
simulated and in fact very few strategies are found to act extortionately.

The work of~\cite{Press2012}, whilst showing that a clever approach to taking
advantage of another memory one strategy exists: this is incomplete. Whilst the
elegance of this result is very attractive, just as the simplicity of the
victory of Tit For Tat in Axelrod's original tournaments was, it is incomplete.
Extortionate strategies achieve a high number of wins but they do not
achieve a high score which corresponds to the fitness landscape in an
evolutionary sense. From the large number of interactions a payoff matrix \(S\)
can be measured where \(S_{ij}\) denotes the score (using standard values of
\((R, S, T, P) = (3, 0, 5, 1)\)) of the \(i\)th strategy
against the \(j\)th strategy. Using this, the replicator equation
describes the evolution of the system based on a population density fitness
function:

\begin{equation}\label{eqn:replicator_dynamics}
    \frac{dx}{dt} = x(S-x^TS x)
\end{equation}

Equation (\ref{eqn:replicator_dynamics}) is solved numerically through an
integration technique described in~\cite{Petzold1983} and
Figure~\ref{fig:replicator_dynamics} shows the evolution of the distribution of
the system: the various strategies are ranked by scores. It is clear to see that
only the high ranking strategies survive the evolutionary process (in fact,
only \input{./assets/img/replicator_dynamics/main.tex}
have a final distribution greater than \(10 ^ {-2}\)). This confirms the
findings of~\cite{Moran1707} in which sophisticated strategies resist
evolutionary invasion of shorter memory strategies. Recalling
Figure~\ref{fig:SSError_and_probabilities_in_full} this demonstrates that:

\begin{itemize}
    \item Cooperation emerges through the evolutionary process: the high scoring
        strategies do not exhibit extortionate behaviour towards each other.
    \item Extortionate strategies do not survive the evolutionary process.
\end{itemize}

\begin{figure}[!htbp]
    \centering
    \includegraphics[width=.8\textwidth]{./assets/img/replicator_dynamics/main.pdf}
    \caption{Numerical simulation of the replicator equation
    (\ref{eqn:replicator_dynamics}): strategies are ordered by score, only the strategies with a high score survive the evolutionary process.}
    \label{fig:replicator_dynamics}
\end{figure}

This work can be used to classify plays of the IPD\@: data can be collected from
actual interactions (in lab or in the field). Furthermore, this allows for a
classification method similar to the notion of fingerprinting presented
in~\cite{Ashlock2008}. Trained strategies can potentially be classified as
extortionate or not or it could be possible to even constrain the reinforcement
learning approaches that are becoming prevalent in the literature.
Alternatively, this mathematical approach for recognising extortion could be
used in sophisticated strategies to defend against invasion. Arguably, some of
the strategies considered here exhibit this behaviour, indeed as described
in~\cite{Harper2017}, the top ranking strategies in the full tournament are
obtained using evolutionary reinforcement learning techniques, thus, suspicion
of extortionate behaviour could in fact be an evolutionary trait.

\section*{Acknowledgements}

The following open source software libraries were used in this research:

\begin{itemize}
    \item The Axelrod ~\cite{Knight2016, Knight2018} library (IPD strategies and
        tournaments).
    \item The sympy library~\cite{Meurer2017} (verification of all symbolic
        calculations).
    \item The matplotlib~\cite{Droettboom2018} library (visualisation).
    \item The pandas~\cite{Structures2010}, dask~\cite{Dask2016} and
        NumPy~\cite{Oliphant2015} libraries (data manipulation).
    \item The SciPy~\cite{Jones2001} library (numerical integration of the
        replicator equation).
\end{itemize}

This work was performed using the computational facilities of the Advanced
Research Computing @ Cardiff (ARCCA) Division, Cardiff University.

\printbibliography

\newpage
\section*{Supplementary materials}

\includepdf{assets/pdf/proof_of_form_of_extortionate_strategies/main.pdf}

\newpage

Using the pair wise interactions the transition rates \(p,
q\) can be measured and the steady state probabilities inferred and compared to
the actual probabilities of each state.
This is done numerically by computing the singular eigenvector of the
matrix \(A\) \cite{Stewart2009}:

\[
    A =
    \begin{bmatrix}
        p_1 q_1 & p_1 (1 - q_1) & (1 - p_1) q_1 & (1 -p_1) (1 - q_1) \\
        p_2 q_2 & p_2 (1 - q_2) & (1 - p_2) q_2 & (1 -p_2) (1 - q_2) \\
        p_3 q_3 & p_3 (1 - q_3) & (1 - p_3) q_3 & (1 -p_3) (1 - q_3) \\
        p_4 q_4 & p_4 (1 - q_4) & (1 - p_4) q_4 & (1 -p_4) (1 - q_4) \\
    \end{bmatrix}
\]

Figure~\ref{fig:computed_probabilities_vs_theoretic_probabilities} shows a
regression line fitted to every pairwise interaction with a reported
\(\text{SSError}\) value (pairwise interactions with missing states were
omitted). This serves to validate the approach: a part from some edge cases the
relationship is consistent.

\begin{figure}[!htbp]
    \centering
    \includegraphics[width=.8\textwidth]{./assets/img/computed_probabilities_vs_theoretic_probabilities/main.pdf}
    \caption{The
        relationship between the steady state probabilities inferred from the
        measured transitions and the actual steady state probabilities. A linear
        regression line is included validating the approach.}
    \label{fig:computed_probabilities_vs_theoretic_probabilities}
\end{figure}


\end{document}

strategies.

The results of this analysis are shown in
Figure~\ref{fig:SSError_and_probabilities_in_full}. The top ranking strategies
by number of wins seem to be extortionate (but not against all strategies) and
it can be seen that a small sub group of strategies achieve mutual defection.
All the top ranking strategies according to score achieve mutual cooperation and
do not extort each other, however they
\textbf{do} exhibit extortionate behaviour towards a number of the lower ranking
strategies.

\begin{figure}[!htbp]
    \centering
    \includegraphics[width=.8\textwidth]{./assets/img/SSError_and_probabilities_in_full/main.pdf}
    \caption{\(\text{SSError}\) for the strategies for the full tournament. Only
    strategy interactions for which \(p_4=0\) and \(\chi>1\) are displayed.}
    \label{fig:SSError_and_probabilities_in_full}
\end{figure}

\section{Conclusion}\label{sec:conclusion}

This work defines an approach to measure whether or not a player is playing a
strategy that corresponds to an extortionate strategy as defined
in~\cite{Press2012}: a mathematical model for suspicion. Indeed, all
extortionate strategies have been
 classified as lying on a triangular plane.
This rigorous classification fails to be robust to small measurement error, thus
a statistical approach is proposed.
This is done through a linear algebraic approach for approximating the solution
of a linear system. Using this, a large number of pairwise interactions is
simulated and in fact very few strategies are found to act extortionately.

The work of~\cite{Press2012}, whilst showing that a clever approach to taking
advantage of another memory one strategy exists: this is incomplete. Whilst the
elegance of this result is very attractive, just as the simplicity of the
victory of Tit For Tat in Axelrod's original tournaments was, it is incomplete.
Extortionate strategies achieve a high number of wins but they do not
achieve a high score which corresponds to the fitness landscape in an
evolutionary sense. From the large number of interactions a payoff matrix \(S\)
can be measured where \(S_{ij}\) denotes the score (using standard values of
\((R, S, T, P) = (3, 0, 5, 1)\)) of the \(i\)th strategy
against the \(j\)th strategy. Using this, the replicator equation
describes the evolution of the system based on a population density fitness
function:

\begin{equation}\label{eqn:replicator_dynamics}
    \frac{dx}{dt} = x(S-x^TS x)
\end{equation}

Equation (\ref{eqn:replicator_dynamics}) is solved numerically through an
integration technique described in~\cite{Petzold1983} and
Figure~\ref{fig:replicator_dynamics} shows the evolution of the distribution of
the system: the various strategies are ranked by scores. It is clear to see that
only the high ranking strategies survive the evolutionary process (in fact,
only \documentclass[a4paper]{article}

\usepackage{amsmath}
\usepackage{amssymb}
\usepackage[margin=1.5cm,
            includefoot,
            footskip=30pt]{geometry}
\usepackage{layout}
\usepackage{graphicx}
\usepackage{subcaption}

\usepackage{biblatex}
\usepackage{pdfpages}

\bibliography{main.bib}

\title{Suspicion: Recognising and evaluating the effectiveness
       of extortion in the Iterated Prisoner's Dilemma}
\author{Vincent A. Knight \and Nikoleta E. Glynatsi}
\date{\today}



\begin{document}

\maketitle

\begin{abstract}
    The Iterated Prisoner's Dilemma is a model for rational and evolutionary
    interactive behaviour. It has applications both in the study of human social
    behaviour as well as in biology.
    It is used to understand when and how a rational individual might
    accept an immediate cost to their own utility for the direct benefit of
    another.

    Much attention has been given to a class of strategies called
    Zero Determinant strategies. It has been theoretically shown that these
    strategies can ``extort'' any player.

    In this work, an approach to identify if observed strategies are playing in
    an extortionate way is described. Furthermore, experimental analysis of
    a large tournament with \input{assets/tex/number_of_full_strategies/main.tex}
    strategies is considered. In this setting
    the most highly performing strategies do not play in an extortionate way
    against each other but do against lower performing strategies.
    This suggests that whilst the theory of Zero Determinant strategies
    indicates that memory is not of fundamental importance to the evolution of
    cooperative behaviour, this is incomplete.
\end{abstract}

\section{Introduction}\label{sec:introduction}

Agent based game theoretic models have become a stalwart of the underpinning
mathematics of interactive behaviours. One of the major pieces of work
in this area is the pair of original computer tournaments run by Robert
Axelrod~\cite{Axelrod1980, Axelrod1980a}. These tournaments pitted submitted
computer strategies against each other in plays of the Iterated Prisoner's
Dilemma. A common game where agents can choose to pay a slight cost to their
immediate utility in the hope of building a reputation. This has been used in
economic and evolutionary game theory to understand the evolution of cooperative
behaviour.

Recently, a class of strategies was described in~\cite{Press2012} that can
provably extort any given opponent. In~\cite{Hilbe2013, Moran1707} some
questions have already been asked about the true effectiveness of these
strategies in an evolutionary setting. Here another question is asked: is it
possible to recognise this extortionate behaviour? A mathematical procedure for
suspicion is presented: in the same way that the continued actions of an
extortionate individual might raise suspicion.

This work makes use of the Axelrod Python library~\cite{Knight2018, Knight2016}
with a large number of Prisoner Dilemma strategies available to give an
extensive numerical example of the ideas presented.  The approach is presented
in Section~\ref{sec:delta-zd-strategies}.  All of the code and data discussed
in Section~\ref{sec:numerical-experiments} is open sourced, archived and
written according to best scientific principles~\cite{Wilson2014}. The data
archive can be found at~\cite{vincent_knight_2018_1297075}.

\section{Recognising Extortion}\label{sec:delta-zd-strategies}

In~\cite{Press2012}, given a match between 2 memory-one strategies, the concept
of Zero Determinant (ZD) strategies is introduced. The main result of that paper
shows that given two memory one players \(p, q\in\mathbb{R}^4\) a linear
relationship between the players' scores could be forced by one of the players.

Using the notation of~\cite{Press2012}, assuming the utilities for player \(p\)
are given by \(S_x=(R, S, T, P)\) and for player \(q\) by \(S_y=(R, T, S, P)\)
and that the stationary scores of each player is given by \(S_X\) and \(S_Y\)
respectively. The main result of~\cite{Press2012} is that if

\begin{equation}\label{eqn:linear_relationship_for_p}
    \tilde p=\alpha S_x + \beta S_y + \gamma
\end{equation}

or

\begin{equation}\label{eqn:linear_relationship_for_q}
    \tilde q=\alpha S_x + \beta S_y + \gamma
\end{equation}

where \(\tilde p = (1 - p_1, 1 - p_2, p_3, p_4)\) and
\(\tilde q = (1 - q_1, 1 - q_2, q_3, q_4)\) then:

\begin{equation}
    \alpha S_X + \beta S_Y + \gamma = 0
\end{equation}

In~\cite{Press2012} a particular type of ZD strategy is defined: extortionate
strategies. If:

\begin{equation}\label{eqn:constraint_for_extortion}
    \gamma = - P(\alpha + \beta)
\end{equation}

then the player can ensure they get a score \(\chi\) times
larger than the opponent. This extortion coefficient is given by:

\begin{equation}\label{eqn:definition_of_chi}
    \chi=\frac{-\beta}{\alpha}
\end{equation}

Thus, if (\ref{eqn:constraint_for_extortion}) holds and \(\chi >1\) a player is
said to extort their opponent.
Here, the reverse problem is considered: given a
\(p\in\mathbb{R}^4\) how does one identify \(\alpha, \beta\) if they
exist and is the strategy in fact acting in an extortionate way?

These conditions correspond to:

\begin{align}
    \tilde p_1 & = \alpha R + \beta R - P (\alpha + \beta)
            \label{eqn:condition_for_tilde_p1}\\
    \tilde p_2 & = \alpha S + \beta T - P (\alpha + \beta)
            \label{eqn:condition_for_tilde_p2}\\
    \tilde p_3 & = \alpha T + \beta S - P (\alpha + \beta)
            \label{eqn:condition_for_tilde_p3}\\
    \tilde p_4 & = \alpha P + \beta P - P (\alpha + \beta)
            \label{eqn:condition_for_tilde_p4}
\end{align}

Equation (\ref{eqn:condition_for_tilde_p4}) ensures that \(p_4=\tilde p_4=0\).
Equations (\ref{eqn:condition_for_tilde_p1}-\ref{eqn:condition_for_tilde_p3})
can be used to eliminate \(\alpha, \beta\), giving:

\begin{equation}\label{eqn:planar_definition_of_extortion}
    \tilde p_1 = \frac{(R - P)(\tilde p_2 + \tilde p_3)}{S + T - 2P}
\end{equation}

with:

\begin{equation}\label{eqn:definition_of_chi}
    \chi = \frac{\tilde p_2 (P - T) + \tilde p_3 (S - P)}
                {\tilde p_2 (P - S) + \tilde p_3 (T - P)}
\end{equation}

Given a strategy \(p\in\mathbb{R}^{4\times 1}\) equations
(\ref{eqn:condition_for_tilde_p4}), (\ref{eqn:planar_definition_of_extortion}-\ref{eqn:definition_of_chi}) can be used to check if
a strategy is extortionate. The conditions correspond to:

\begin{align}
    p_1 & = \frac{(R-P)(p_2 + p_3) - R + T + S - P}{S + T - 2P}
     \label{eqn:condition_for_p1}\\
    p_4 & = 0 \label{eqn:condition_for_p4}\\
    1 & > p_2 + p_3\label{eqn:condition_for_chi}
\end{align}

The algebraic steps necessary to prove these results are available in the
supporting materials.

All extortionate strategies reside on a triangular (\ref{eqn:condition_for_chi})
plane (\ref{eqn:condition_for_p1}) in 3 dimensions (\ref{eqn:condition_for_p4}).
Using this formulation it can be seen that a necessary (but not sufficient)
condition for an extortionate strategy is that it cooperates on average less
than 50\% of the time when in a state of disagreement with the opponent.

As an example, consider the known extortionate strategy \(p=(8 / 9, 1 / 2, 1 /
3, 0)\) from~\cite{Stewart2012} which is referred to as \texttt{Extort-2}. In
this case, for the standard values of \((R, T, S, P)\) constraint
(\ref{eqn:condition_for_p1}) corresponds to:

\begin{equation}
    p_1 = \frac{2(p_2 + p_3) + 1}{3}
\end{equation}

It is clear that in this case all constraints hold.

This approach could in fact be used to confirm that a given strategy is acting
in an extortionate manner even if it is not a memory one strategy. However, in
practice, if a closed form for \(p\) is not known, then due to measurement
and/or numerical error this would not work.

This problem can be written in the following linear algebraic form where
\(x=(\alpha, \beta)\)
and \(p^*=(\tilde p_1 - 1, tilde_2 - 1, p_3)\):

\begin{equation}\label{eqn:linear_algebraic_equation_for_p}
    Cx= p^*
\end{equation}

\(C\) corresponds to equations
(\ref{eqn:condition_for_tilde_p1}-\ref{eqn:condition_for_tilde_p3}) and is
given by:

\begin{equation}\label{eqn:definition_of_C}
    C =
    \begin{bmatrix}
        R - P & R- P \\
        S - P & T- P \\
        T - P & S- P \\
    \end{bmatrix}
\end{equation}

Note that in general, equation (\ref{eqn:linear_algebraic_equation_for_p}) will
not necessarily have a solution. From the Rouch\'{e}-Capelli theorem if there is
a solution it is unique as \(\text{rank}(C)=2\) which is the dimension of the
variable \(x\). The best fitting \(x\) is found by minimizing:

\begin{equation}\label{eqn:r_squared}
    \text{SSError} = \|C x- p^*\|_2^2 = \sum_{i=1}^{3}\left((C\bar x)_i-p_i^*\right)^2
\end{equation}

Note that \(\text{SSError}\), which is the square of the Frobenius
norm~\cite{Golub2013}, becomes a measure of how close a strategy is to being an
extortionate strategy. Suspicion
of extortion then corresponds to a threshold on \(\text{SSError}\).

By observing interactions (human or otherwise), their memory one representation
can be inferred and this approach can be used to recognise extortionate
behaviour. The notion of comparing theoretic and actual plays of the IPD is not
novel, see for example~\cite{Rand2013}. Immediately it is noted that if the
environment is noisy~\cite{Wu1995} then no strategy can be considered to be
extortionate as \(p_4>0\).

In the next section, this idea will be illustrated by observing the interactions
that take place in a computer based tournament of the IPD\@.

\section{Numerical experiments}\label{sec:numerical-experiments}

In~\cite{Stewart2012} results from a tournament with
\input{./assets/tex/number_of_stewart_plotkin_strategies/main.tex} strategies,
was presented with specific consideration given to ZD strategies. This
tournament is reproduced here using the Axelrod-Python
project~\cite{Knight2016}. To obtain a good measure of the corresponding
transition rates for each strategy all matches have been run for
\input{assets/tex/number_of_turns/main.tex} turns and every match has been
repeated \input{assets/tex/number_of_repetitions/main.tex} times. All of this
interaction data is available at~\cite{vincent_knight_2018_1297075}. A good
match between the inferred Markov chain and the state distribution of the actual
interactions has been verified. Data for this is presented in the supplementary
materials.

Figure~\ref{fig:SSError_overall_in_stewart_plotkin} shows the \(\text{SSError}\)
values for all the strategies in the tournament, as reported
in~\cite{Stewart2012} the extortionate strategy (which has an expected
\(\text{SSError}\) approximately 0) gains a large number of wins.

\begin{figure}[!htbp]
    \centering
    \includegraphics[width=.8\textwidth]{./assets/img/SSError_overall_in_stewart_plotkin/main.pdf}
    \caption{\(\text{SSError}\) and state probabilities for the strategies
        of~\cite{Stewart2012}, ordered both by number of wins and overall score.
        Note that \(P(DC)\) is not shown as it corresponds to the transpose of
        \(P(CD)\). Cooperator and Defector are omitted as they do not visit all
        the states.}
    \label{fig:SSError_overall_in_stewart_plotkin}
\end{figure}

Here, the work of~\cite{Stewart2012} is extended by investigating a tournament
with \input{assets/tex/number_of_full_strategies/main.tex}
strategies.

The results of this analysis are shown in
Figure~\ref{fig:SSError_and_probabilities_in_full}. The top ranking strategies
by number of wins seem to be extortionate (but not against all strategies) and
it can be seen that a small sub group of strategies achieve mutual defection.
All the top ranking strategies according to score achieve mutual cooperation and
do not extort each other, however they
\textbf{do} exhibit extortionate behaviour towards a number of the lower ranking
strategies.

\begin{figure}[!htbp]
    \centering
    \includegraphics[width=.8\textwidth]{./assets/img/SSError_and_probabilities_in_full/main.pdf}
    \caption{\(\text{SSError}\) for the strategies for the full tournament. Only
    strategy interactions for which \(p_4=0\) and \(\chi>1\) are displayed.}
    \label{fig:SSError_and_probabilities_in_full}
\end{figure}

\section{Conclusion}\label{sec:conclusion}

This work defines an approach to measure whether or not a player is playing a
strategy that corresponds to an extortionate strategy as defined
in~\cite{Press2012}: a mathematical model for suspicion. Indeed, all
extortionate strategies have been
 classified as lying on a triangular plane.
This rigorous classification fails to be robust to small measurement error, thus
a statistical approach is proposed.
This is done through a linear algebraic approach for approximating the solution
of a linear system. Using this, a large number of pairwise interactions is
simulated and in fact very few strategies are found to act extortionately.

The work of~\cite{Press2012}, whilst showing that a clever approach to taking
advantage of another memory one strategy exists: this is incomplete. Whilst the
elegance of this result is very attractive, just as the simplicity of the
victory of Tit For Tat in Axelrod's original tournaments was, it is incomplete.
Extortionate strategies achieve a high number of wins but they do not
achieve a high score which corresponds to the fitness landscape in an
evolutionary sense. From the large number of interactions a payoff matrix \(S\)
can be measured where \(S_{ij}\) denotes the score (using standard values of
\((R, S, T, P) = (3, 0, 5, 1)\)) of the \(i\)th strategy
against the \(j\)th strategy. Using this, the replicator equation
describes the evolution of the system based on a population density fitness
function:

\begin{equation}\label{eqn:replicator_dynamics}
    \frac{dx}{dt} = x(S-x^TS x)
\end{equation}

Equation (\ref{eqn:replicator_dynamics}) is solved numerically through an
integration technique described in~\cite{Petzold1983} and
Figure~\ref{fig:replicator_dynamics} shows the evolution of the distribution of
the system: the various strategies are ranked by scores. It is clear to see that
only the high ranking strategies survive the evolutionary process (in fact,
only \input{./assets/img/replicator_dynamics/main.tex}
have a final distribution greater than \(10 ^ {-2}\)). This confirms the
findings of~\cite{Moran1707} in which sophisticated strategies resist
evolutionary invasion of shorter memory strategies. Recalling
Figure~\ref{fig:SSError_and_probabilities_in_full} this demonstrates that:

\begin{itemize}
    \item Cooperation emerges through the evolutionary process: the high scoring
        strategies do not exhibit extortionate behaviour towards each other.
    \item Extortionate strategies do not survive the evolutionary process.
\end{itemize}

\begin{figure}[!htbp]
    \centering
    \includegraphics[width=.8\textwidth]{./assets/img/replicator_dynamics/main.pdf}
    \caption{Numerical simulation of the replicator equation
    (\ref{eqn:replicator_dynamics}): strategies are ordered by score, only the strategies with a high score survive the evolutionary process.}
    \label{fig:replicator_dynamics}
\end{figure}

This work can be used to classify plays of the IPD\@: data can be collected from
actual interactions (in lab or in the field). Furthermore, this allows for a
classification method similar to the notion of fingerprinting presented
in~\cite{Ashlock2008}. Trained strategies can potentially be classified as
extortionate or not or it could be possible to even constrain the reinforcement
learning approaches that are becoming prevalent in the literature.
Alternatively, this mathematical approach for recognising extortion could be
used in sophisticated strategies to defend against invasion. Arguably, some of
the strategies considered here exhibit this behaviour, indeed as described
in~\cite{Harper2017}, the top ranking strategies in the full tournament are
obtained using evolutionary reinforcement learning techniques, thus, suspicion
of extortionate behaviour could in fact be an evolutionary trait.

\section*{Acknowledgements}

The following open source software libraries were used in this research:

\begin{itemize}
    \item The Axelrod ~\cite{Knight2016, Knight2018} library (IPD strategies and
        tournaments).
    \item The sympy library~\cite{Meurer2017} (verification of all symbolic
        calculations).
    \item The matplotlib~\cite{Droettboom2018} library (visualisation).
    \item The pandas~\cite{Structures2010}, dask~\cite{Dask2016} and
        NumPy~\cite{Oliphant2015} libraries (data manipulation).
    \item The SciPy~\cite{Jones2001} library (numerical integration of the
        replicator equation).
\end{itemize}

This work was performed using the computational facilities of the Advanced
Research Computing @ Cardiff (ARCCA) Division, Cardiff University.

\printbibliography

\newpage
\section*{Supplementary materials}

\includepdf{assets/pdf/proof_of_form_of_extortionate_strategies/main.pdf}

\newpage

Using the pair wise interactions the transition rates \(p,
q\) can be measured and the steady state probabilities inferred and compared to
the actual probabilities of each state.
This is done numerically by computing the singular eigenvector of the
matrix \(A\) \cite{Stewart2009}:

\[
    A =
    \begin{bmatrix}
        p_1 q_1 & p_1 (1 - q_1) & (1 - p_1) q_1 & (1 -p_1) (1 - q_1) \\
        p_2 q_2 & p_2 (1 - q_2) & (1 - p_2) q_2 & (1 -p_2) (1 - q_2) \\
        p_3 q_3 & p_3 (1 - q_3) & (1 - p_3) q_3 & (1 -p_3) (1 - q_3) \\
        p_4 q_4 & p_4 (1 - q_4) & (1 - p_4) q_4 & (1 -p_4) (1 - q_4) \\
    \end{bmatrix}
\]

Figure~\ref{fig:computed_probabilities_vs_theoretic_probabilities} shows a
regression line fitted to every pairwise interaction with a reported
\(\text{SSError}\) value (pairwise interactions with missing states were
omitted). This serves to validate the approach: a part from some edge cases the
relationship is consistent.

\begin{figure}[!htbp]
    \centering
    \includegraphics[width=.8\textwidth]{./assets/img/computed_probabilities_vs_theoretic_probabilities/main.pdf}
    \caption{The
        relationship between the steady state probabilities inferred from the
        measured transitions and the actual steady state probabilities. A linear
        regression line is included validating the approach.}
    \label{fig:computed_probabilities_vs_theoretic_probabilities}
\end{figure}


\end{document}

have a final distribution greater than \(10 ^ {-2}\)). This confirms the
findings of~\cite{Moran1707} in which sophisticated strategies resist
evolutionary invasion of shorter memory strategies. Recalling
Figure~\ref{fig:SSError_and_probabilities_in_full} this demonstrates that:

\begin{itemize}
    \item Cooperation emerges through the evolutionary process: the high scoring
        strategies do not exhibit extortionate behaviour towards each other.
    \item Extortionate strategies do not survive the evolutionary process.
\end{itemize}

\begin{figure}[!htbp]
    \centering
    \includegraphics[width=.8\textwidth]{./assets/img/replicator_dynamics/main.pdf}
    \caption{Numerical simulation of the replicator equation
    (\ref{eqn:replicator_dynamics}): strategies are ordered by score, only the strategies with a high score survive the evolutionary process.}
    \label{fig:replicator_dynamics}
\end{figure}

This work can be used to classify plays of the IPD\@: data can be collected from
actual interactions (in lab or in the field). Furthermore, this allows for a
classification method similar to the notion of fingerprinting presented
in~\cite{Ashlock2008}. Trained strategies can potentially be classified as
extortionate or not or it could be possible to even constrain the reinforcement
learning approaches that are becoming prevalent in the literature.
Alternatively, this mathematical approach for recognising extortion could be
used in sophisticated strategies to defend against invasion. Arguably, some of
the strategies considered here exhibit this behaviour, indeed as described
in~\cite{Harper2017}, the top ranking strategies in the full tournament are
obtained using evolutionary reinforcement learning techniques, thus, suspicion
of extortionate behaviour could in fact be an evolutionary trait.

\section*{Acknowledgements}

The following open source software libraries were used in this research:

\begin{itemize}
    \item The Axelrod ~\cite{Knight2016, Knight2018} library (IPD strategies and
        tournaments).
    \item The sympy library~\cite{Meurer2017} (verification of all symbolic
        calculations).
    \item The matplotlib~\cite{Droettboom2018} library (visualisation).
    \item The pandas~\cite{Structures2010}, dask~\cite{Dask2016} and
        NumPy~\cite{Oliphant2015} libraries (data manipulation).
    \item The SciPy~\cite{Jones2001} library (numerical integration of the
        replicator equation).
\end{itemize}

This work was performed using the computational facilities of the Advanced
Research Computing @ Cardiff (ARCCA) Division, Cardiff University.

\printbibliography

\newpage
\section*{Supplementary materials}

\includepdf{assets/pdf/proof_of_form_of_extortionate_strategies/main.pdf}

\newpage

Using the pair wise interactions the transition rates \(p,
q\) can be measured and the steady state probabilities inferred and compared to
the actual probabilities of each state.
This is done numerically by computing the singular eigenvector of the
matrix \(A\) \cite{Stewart2009}:

\[
    A =
    \begin{bmatrix}
        p_1 q_1 & p_1 (1 - q_1) & (1 - p_1) q_1 & (1 -p_1) (1 - q_1) \\
        p_2 q_2 & p_2 (1 - q_2) & (1 - p_2) q_2 & (1 -p_2) (1 - q_2) \\
        p_3 q_3 & p_3 (1 - q_3) & (1 - p_3) q_3 & (1 -p_3) (1 - q_3) \\
        p_4 q_4 & p_4 (1 - q_4) & (1 - p_4) q_4 & (1 -p_4) (1 - q_4) \\
    \end{bmatrix}
\]

Figure~\ref{fig:computed_probabilities_vs_theoretic_probabilities} shows a
regression line fitted to every pairwise interaction with a reported
\(\text{SSError}\) value (pairwise interactions with missing states were
omitted). This serves to validate the approach: a part from some edge cases the
relationship is consistent.

\begin{figure}[!htbp]
    \centering
    \includegraphics[width=.8\textwidth]{./assets/img/computed_probabilities_vs_theoretic_probabilities/main.pdf}
    \caption{The
        relationship between the steady state probabilities inferred from the
        measured transitions and the actual steady state probabilities. A linear
        regression line is included validating the approach.}
    \label{fig:computed_probabilities_vs_theoretic_probabilities}
\end{figure}


\end{document}
 times. All of this
interaction data is available at~\cite{vincent_knight_2018_1297075}. A good
match between the inferred Markov chain and the state distribution of the actual
interactions has been verified. Data for this is presented in the supplementary
materials.

Figure~\ref{fig:SSError_overall_in_stewart_plotkin} shows the \(\text{SSError}\)
values for all the strategies in the tournament, as reported
in~\cite{Stewart2012} the extortionate strategy (which has an expected
\(\text{SSError}\) approximately 0) gains a large number of wins.

\begin{figure}[!htbp]
    \centering
    \includegraphics[width=.8\textwidth]{./assets/img/SSError_overall_in_stewart_plotkin/main.pdf}
    \caption{\(\text{SSError}\) and state probabilities for the strategies
        of~\cite{Stewart2012}, ordered both by number of wins and overall score.
        Note that \(P(DC)\) is not shown as it corresponds to the transpose of
        \(P(CD)\). Cooperator and Defector are omitted as they do not visit all
        the states.}
    \label{fig:SSError_overall_in_stewart_plotkin}
\end{figure}

Here, the work of~\cite{Stewart2012} is extended by investigating a tournament
with \documentclass[a4paper]{article}

\usepackage{amsmath}
\usepackage{amssymb}
\usepackage[margin=1.5cm,
            includefoot,
            footskip=30pt]{geometry}
\usepackage{layout}
\usepackage{graphicx}
\usepackage{subcaption}

\usepackage{biblatex}
\usepackage{pdfpages}

\bibliography{main.bib}

\title{Suspicion: Recognising and evaluating the effectiveness
       of extortion in the Iterated Prisoner's Dilemma}
\author{Vincent A. Knight \and Nikoleta E. Glynatsi}
\date{\today}



\begin{document}

\maketitle

\begin{abstract}
    The Iterated Prisoner's Dilemma is a model for rational and evolutionary
    interactive behaviour. It has applications both in the study of human social
    behaviour as well as in biology.
    It is used to understand when and how a rational individual might
    accept an immediate cost to their own utility for the direct benefit of
    another.

    Much attention has been given to a class of strategies called
    Zero Determinant strategies. It has been theoretically shown that these
    strategies can ``extort'' any player.

    In this work, an approach to identify if observed strategies are playing in
    an extortionate way is described. Furthermore, experimental analysis of
    a large tournament with \documentclass[a4paper]{article}

\usepackage{amsmath}
\usepackage{amssymb}
\usepackage[margin=1.5cm,
            includefoot,
            footskip=30pt]{geometry}
\usepackage{layout}
\usepackage{graphicx}
\usepackage{subcaption}

\usepackage{biblatex}
\usepackage{pdfpages}

\bibliography{main.bib}

\title{Suspicion: Recognising and evaluating the effectiveness
       of extortion in the Iterated Prisoner's Dilemma}
\author{Vincent A. Knight \and Nikoleta E. Glynatsi}
\date{\today}



\begin{document}

\maketitle

\begin{abstract}
    The Iterated Prisoner's Dilemma is a model for rational and evolutionary
    interactive behaviour. It has applications both in the study of human social
    behaviour as well as in biology.
    It is used to understand when and how a rational individual might
    accept an immediate cost to their own utility for the direct benefit of
    another.

    Much attention has been given to a class of strategies called
    Zero Determinant strategies. It has been theoretically shown that these
    strategies can ``extort'' any player.

    In this work, an approach to identify if observed strategies are playing in
    an extortionate way is described. Furthermore, experimental analysis of
    a large tournament with \input{assets/tex/number_of_full_strategies/main.tex}
    strategies is considered. In this setting
    the most highly performing strategies do not play in an extortionate way
    against each other but do against lower performing strategies.
    This suggests that whilst the theory of Zero Determinant strategies
    indicates that memory is not of fundamental importance to the evolution of
    cooperative behaviour, this is incomplete.
\end{abstract}

\section{Introduction}\label{sec:introduction}

Agent based game theoretic models have become a stalwart of the underpinning
mathematics of interactive behaviours. One of the major pieces of work
in this area is the pair of original computer tournaments run by Robert
Axelrod~\cite{Axelrod1980, Axelrod1980a}. These tournaments pitted submitted
computer strategies against each other in plays of the Iterated Prisoner's
Dilemma. A common game where agents can choose to pay a slight cost to their
immediate utility in the hope of building a reputation. This has been used in
economic and evolutionary game theory to understand the evolution of cooperative
behaviour.

Recently, a class of strategies was described in~\cite{Press2012} that can
provably extort any given opponent. In~\cite{Hilbe2013, Moran1707} some
questions have already been asked about the true effectiveness of these
strategies in an evolutionary setting. Here another question is asked: is it
possible to recognise this extortionate behaviour? A mathematical procedure for
suspicion is presented: in the same way that the continued actions of an
extortionate individual might raise suspicion.

This work makes use of the Axelrod Python library~\cite{Knight2018, Knight2016}
with a large number of Prisoner Dilemma strategies available to give an
extensive numerical example of the ideas presented.  The approach is presented
in Section~\ref{sec:delta-zd-strategies}.  All of the code and data discussed
in Section~\ref{sec:numerical-experiments} is open sourced, archived and
written according to best scientific principles~\cite{Wilson2014}. The data
archive can be found at~\cite{vincent_knight_2018_1297075}.

\section{Recognising Extortion}\label{sec:delta-zd-strategies}

In~\cite{Press2012}, given a match between 2 memory-one strategies, the concept
of Zero Determinant (ZD) strategies is introduced. The main result of that paper
shows that given two memory one players \(p, q\in\mathbb{R}^4\) a linear
relationship between the players' scores could be forced by one of the players.

Using the notation of~\cite{Press2012}, assuming the utilities for player \(p\)
are given by \(S_x=(R, S, T, P)\) and for player \(q\) by \(S_y=(R, T, S, P)\)
and that the stationary scores of each player is given by \(S_X\) and \(S_Y\)
respectively. The main result of~\cite{Press2012} is that if

\begin{equation}\label{eqn:linear_relationship_for_p}
    \tilde p=\alpha S_x + \beta S_y + \gamma
\end{equation}

or

\begin{equation}\label{eqn:linear_relationship_for_q}
    \tilde q=\alpha S_x + \beta S_y + \gamma
\end{equation}

where \(\tilde p = (1 - p_1, 1 - p_2, p_3, p_4)\) and
\(\tilde q = (1 - q_1, 1 - q_2, q_3, q_4)\) then:

\begin{equation}
    \alpha S_X + \beta S_Y + \gamma = 0
\end{equation}

In~\cite{Press2012} a particular type of ZD strategy is defined: extortionate
strategies. If:

\begin{equation}\label{eqn:constraint_for_extortion}
    \gamma = - P(\alpha + \beta)
\end{equation}

then the player can ensure they get a score \(\chi\) times
larger than the opponent. This extortion coefficient is given by:

\begin{equation}\label{eqn:definition_of_chi}
    \chi=\frac{-\beta}{\alpha}
\end{equation}

Thus, if (\ref{eqn:constraint_for_extortion}) holds and \(\chi >1\) a player is
said to extort their opponent.
Here, the reverse problem is considered: given a
\(p\in\mathbb{R}^4\) how does one identify \(\alpha, \beta\) if they
exist and is the strategy in fact acting in an extortionate way?

These conditions correspond to:

\begin{align}
    \tilde p_1 & = \alpha R + \beta R - P (\alpha + \beta)
            \label{eqn:condition_for_tilde_p1}\\
    \tilde p_2 & = \alpha S + \beta T - P (\alpha + \beta)
            \label{eqn:condition_for_tilde_p2}\\
    \tilde p_3 & = \alpha T + \beta S - P (\alpha + \beta)
            \label{eqn:condition_for_tilde_p3}\\
    \tilde p_4 & = \alpha P + \beta P - P (\alpha + \beta)
            \label{eqn:condition_for_tilde_p4}
\end{align}

Equation (\ref{eqn:condition_for_tilde_p4}) ensures that \(p_4=\tilde p_4=0\).
Equations (\ref{eqn:condition_for_tilde_p1}-\ref{eqn:condition_for_tilde_p3})
can be used to eliminate \(\alpha, \beta\), giving:

\begin{equation}\label{eqn:planar_definition_of_extortion}
    \tilde p_1 = \frac{(R - P)(\tilde p_2 + \tilde p_3)}{S + T - 2P}
\end{equation}

with:

\begin{equation}\label{eqn:definition_of_chi}
    \chi = \frac{\tilde p_2 (P - T) + \tilde p_3 (S - P)}
                {\tilde p_2 (P - S) + \tilde p_3 (T - P)}
\end{equation}

Given a strategy \(p\in\mathbb{R}^{4\times 1}\) equations
(\ref{eqn:condition_for_tilde_p4}), (\ref{eqn:planar_definition_of_extortion}-\ref{eqn:definition_of_chi}) can be used to check if
a strategy is extortionate. The conditions correspond to:

\begin{align}
    p_1 & = \frac{(R-P)(p_2 + p_3) - R + T + S - P}{S + T - 2P}
     \label{eqn:condition_for_p1}\\
    p_4 & = 0 \label{eqn:condition_for_p4}\\
    1 & > p_2 + p_3\label{eqn:condition_for_chi}
\end{align}

The algebraic steps necessary to prove these results are available in the
supporting materials.

All extortionate strategies reside on a triangular (\ref{eqn:condition_for_chi})
plane (\ref{eqn:condition_for_p1}) in 3 dimensions (\ref{eqn:condition_for_p4}).
Using this formulation it can be seen that a necessary (but not sufficient)
condition for an extortionate strategy is that it cooperates on average less
than 50\% of the time when in a state of disagreement with the opponent.

As an example, consider the known extortionate strategy \(p=(8 / 9, 1 / 2, 1 /
3, 0)\) from~\cite{Stewart2012} which is referred to as \texttt{Extort-2}. In
this case, for the standard values of \((R, T, S, P)\) constraint
(\ref{eqn:condition_for_p1}) corresponds to:

\begin{equation}
    p_1 = \frac{2(p_2 + p_3) + 1}{3}
\end{equation}

It is clear that in this case all constraints hold.

This approach could in fact be used to confirm that a given strategy is acting
in an extortionate manner even if it is not a memory one strategy. However, in
practice, if a closed form for \(p\) is not known, then due to measurement
and/or numerical error this would not work.

This problem can be written in the following linear algebraic form where
\(x=(\alpha, \beta)\)
and \(p^*=(\tilde p_1 - 1, tilde_2 - 1, p_3)\):

\begin{equation}\label{eqn:linear_algebraic_equation_for_p}
    Cx= p^*
\end{equation}

\(C\) corresponds to equations
(\ref{eqn:condition_for_tilde_p1}-\ref{eqn:condition_for_tilde_p3}) and is
given by:

\begin{equation}\label{eqn:definition_of_C}
    C =
    \begin{bmatrix}
        R - P & R- P \\
        S - P & T- P \\
        T - P & S- P \\
    \end{bmatrix}
\end{equation}

Note that in general, equation (\ref{eqn:linear_algebraic_equation_for_p}) will
not necessarily have a solution. From the Rouch\'{e}-Capelli theorem if there is
a solution it is unique as \(\text{rank}(C)=2\) which is the dimension of the
variable \(x\). The best fitting \(x\) is found by minimizing:

\begin{equation}\label{eqn:r_squared}
    \text{SSError} = \|C x- p^*\|_2^2 = \sum_{i=1}^{3}\left((C\bar x)_i-p_i^*\right)^2
\end{equation}

Note that \(\text{SSError}\), which is the square of the Frobenius
norm~\cite{Golub2013}, becomes a measure of how close a strategy is to being an
extortionate strategy. Suspicion
of extortion then corresponds to a threshold on \(\text{SSError}\).

By observing interactions (human or otherwise), their memory one representation
can be inferred and this approach can be used to recognise extortionate
behaviour. The notion of comparing theoretic and actual plays of the IPD is not
novel, see for example~\cite{Rand2013}. Immediately it is noted that if the
environment is noisy~\cite{Wu1995} then no strategy can be considered to be
extortionate as \(p_4>0\).

In the next section, this idea will be illustrated by observing the interactions
that take place in a computer based tournament of the IPD\@.

\section{Numerical experiments}\label{sec:numerical-experiments}

In~\cite{Stewart2012} results from a tournament with
\input{./assets/tex/number_of_stewart_plotkin_strategies/main.tex} strategies,
was presented with specific consideration given to ZD strategies. This
tournament is reproduced here using the Axelrod-Python
project~\cite{Knight2016}. To obtain a good measure of the corresponding
transition rates for each strategy all matches have been run for
\input{assets/tex/number_of_turns/main.tex} turns and every match has been
repeated \input{assets/tex/number_of_repetitions/main.tex} times. All of this
interaction data is available at~\cite{vincent_knight_2018_1297075}. A good
match between the inferred Markov chain and the state distribution of the actual
interactions has been verified. Data for this is presented in the supplementary
materials.

Figure~\ref{fig:SSError_overall_in_stewart_plotkin} shows the \(\text{SSError}\)
values for all the strategies in the tournament, as reported
in~\cite{Stewart2012} the extortionate strategy (which has an expected
\(\text{SSError}\) approximately 0) gains a large number of wins.

\begin{figure}[!htbp]
    \centering
    \includegraphics[width=.8\textwidth]{./assets/img/SSError_overall_in_stewart_plotkin/main.pdf}
    \caption{\(\text{SSError}\) and state probabilities for the strategies
        of~\cite{Stewart2012}, ordered both by number of wins and overall score.
        Note that \(P(DC)\) is not shown as it corresponds to the transpose of
        \(P(CD)\). Cooperator and Defector are omitted as they do not visit all
        the states.}
    \label{fig:SSError_overall_in_stewart_plotkin}
\end{figure}

Here, the work of~\cite{Stewart2012} is extended by investigating a tournament
with \input{assets/tex/number_of_full_strategies/main.tex}
strategies.

The results of this analysis are shown in
Figure~\ref{fig:SSError_and_probabilities_in_full}. The top ranking strategies
by number of wins seem to be extortionate (but not against all strategies) and
it can be seen that a small sub group of strategies achieve mutual defection.
All the top ranking strategies according to score achieve mutual cooperation and
do not extort each other, however they
\textbf{do} exhibit extortionate behaviour towards a number of the lower ranking
strategies.

\begin{figure}[!htbp]
    \centering
    \includegraphics[width=.8\textwidth]{./assets/img/SSError_and_probabilities_in_full/main.pdf}
    \caption{\(\text{SSError}\) for the strategies for the full tournament. Only
    strategy interactions for which \(p_4=0\) and \(\chi>1\) are displayed.}
    \label{fig:SSError_and_probabilities_in_full}
\end{figure}

\section{Conclusion}\label{sec:conclusion}

This work defines an approach to measure whether or not a player is playing a
strategy that corresponds to an extortionate strategy as defined
in~\cite{Press2012}: a mathematical model for suspicion. Indeed, all
extortionate strategies have been
 classified as lying on a triangular plane.
This rigorous classification fails to be robust to small measurement error, thus
a statistical approach is proposed.
This is done through a linear algebraic approach for approximating the solution
of a linear system. Using this, a large number of pairwise interactions is
simulated and in fact very few strategies are found to act extortionately.

The work of~\cite{Press2012}, whilst showing that a clever approach to taking
advantage of another memory one strategy exists: this is incomplete. Whilst the
elegance of this result is very attractive, just as the simplicity of the
victory of Tit For Tat in Axelrod's original tournaments was, it is incomplete.
Extortionate strategies achieve a high number of wins but they do not
achieve a high score which corresponds to the fitness landscape in an
evolutionary sense. From the large number of interactions a payoff matrix \(S\)
can be measured where \(S_{ij}\) denotes the score (using standard values of
\((R, S, T, P) = (3, 0, 5, 1)\)) of the \(i\)th strategy
against the \(j\)th strategy. Using this, the replicator equation
describes the evolution of the system based on a population density fitness
function:

\begin{equation}\label{eqn:replicator_dynamics}
    \frac{dx}{dt} = x(S-x^TS x)
\end{equation}

Equation (\ref{eqn:replicator_dynamics}) is solved numerically through an
integration technique described in~\cite{Petzold1983} and
Figure~\ref{fig:replicator_dynamics} shows the evolution of the distribution of
the system: the various strategies are ranked by scores. It is clear to see that
only the high ranking strategies survive the evolutionary process (in fact,
only \input{./assets/img/replicator_dynamics/main.tex}
have a final distribution greater than \(10 ^ {-2}\)). This confirms the
findings of~\cite{Moran1707} in which sophisticated strategies resist
evolutionary invasion of shorter memory strategies. Recalling
Figure~\ref{fig:SSError_and_probabilities_in_full} this demonstrates that:

\begin{itemize}
    \item Cooperation emerges through the evolutionary process: the high scoring
        strategies do not exhibit extortionate behaviour towards each other.
    \item Extortionate strategies do not survive the evolutionary process.
\end{itemize}

\begin{figure}[!htbp]
    \centering
    \includegraphics[width=.8\textwidth]{./assets/img/replicator_dynamics/main.pdf}
    \caption{Numerical simulation of the replicator equation
    (\ref{eqn:replicator_dynamics}): strategies are ordered by score, only the strategies with a high score survive the evolutionary process.}
    \label{fig:replicator_dynamics}
\end{figure}

This work can be used to classify plays of the IPD\@: data can be collected from
actual interactions (in lab or in the field). Furthermore, this allows for a
classification method similar to the notion of fingerprinting presented
in~\cite{Ashlock2008}. Trained strategies can potentially be classified as
extortionate or not or it could be possible to even constrain the reinforcement
learning approaches that are becoming prevalent in the literature.
Alternatively, this mathematical approach for recognising extortion could be
used in sophisticated strategies to defend against invasion. Arguably, some of
the strategies considered here exhibit this behaviour, indeed as described
in~\cite{Harper2017}, the top ranking strategies in the full tournament are
obtained using evolutionary reinforcement learning techniques, thus, suspicion
of extortionate behaviour could in fact be an evolutionary trait.

\section*{Acknowledgements}

The following open source software libraries were used in this research:

\begin{itemize}
    \item The Axelrod ~\cite{Knight2016, Knight2018} library (IPD strategies and
        tournaments).
    \item The sympy library~\cite{Meurer2017} (verification of all symbolic
        calculations).
    \item The matplotlib~\cite{Droettboom2018} library (visualisation).
    \item The pandas~\cite{Structures2010}, dask~\cite{Dask2016} and
        NumPy~\cite{Oliphant2015} libraries (data manipulation).
    \item The SciPy~\cite{Jones2001} library (numerical integration of the
        replicator equation).
\end{itemize}

This work was performed using the computational facilities of the Advanced
Research Computing @ Cardiff (ARCCA) Division, Cardiff University.

\printbibliography

\newpage
\section*{Supplementary materials}

\includepdf{assets/pdf/proof_of_form_of_extortionate_strategies/main.pdf}

\newpage

Using the pair wise interactions the transition rates \(p,
q\) can be measured and the steady state probabilities inferred and compared to
the actual probabilities of each state.
This is done numerically by computing the singular eigenvector of the
matrix \(A\) \cite{Stewart2009}:

\[
    A =
    \begin{bmatrix}
        p_1 q_1 & p_1 (1 - q_1) & (1 - p_1) q_1 & (1 -p_1) (1 - q_1) \\
        p_2 q_2 & p_2 (1 - q_2) & (1 - p_2) q_2 & (1 -p_2) (1 - q_2) \\
        p_3 q_3 & p_3 (1 - q_3) & (1 - p_3) q_3 & (1 -p_3) (1 - q_3) \\
        p_4 q_4 & p_4 (1 - q_4) & (1 - p_4) q_4 & (1 -p_4) (1 - q_4) \\
    \end{bmatrix}
\]

Figure~\ref{fig:computed_probabilities_vs_theoretic_probabilities} shows a
regression line fitted to every pairwise interaction with a reported
\(\text{SSError}\) value (pairwise interactions with missing states were
omitted). This serves to validate the approach: a part from some edge cases the
relationship is consistent.

\begin{figure}[!htbp]
    \centering
    \includegraphics[width=.8\textwidth]{./assets/img/computed_probabilities_vs_theoretic_probabilities/main.pdf}
    \caption{The
        relationship between the steady state probabilities inferred from the
        measured transitions and the actual steady state probabilities. A linear
        regression line is included validating the approach.}
    \label{fig:computed_probabilities_vs_theoretic_probabilities}
\end{figure}


\end{document}

    strategies is considered. In this setting
    the most highly performing strategies do not play in an extortionate way
    against each other but do against lower performing strategies.
    This suggests that whilst the theory of Zero Determinant strategies
    indicates that memory is not of fundamental importance to the evolution of
    cooperative behaviour, this is incomplete.
\end{abstract}

\section{Introduction}\label{sec:introduction}

Agent based game theoretic models have become a stalwart of the underpinning
mathematics of interactive behaviours. One of the major pieces of work
in this area is the pair of original computer tournaments run by Robert
Axelrod~\cite{Axelrod1980, Axelrod1980a}. These tournaments pitted submitted
computer strategies against each other in plays of the Iterated Prisoner's
Dilemma. A common game where agents can choose to pay a slight cost to their
immediate utility in the hope of building a reputation. This has been used in
economic and evolutionary game theory to understand the evolution of cooperative
behaviour.

Recently, a class of strategies was described in~\cite{Press2012} that can
provably extort any given opponent. In~\cite{Hilbe2013, Moran1707} some
questions have already been asked about the true effectiveness of these
strategies in an evolutionary setting. Here another question is asked: is it
possible to recognise this extortionate behaviour? A mathematical procedure for
suspicion is presented: in the same way that the continued actions of an
extortionate individual might raise suspicion.

This work makes use of the Axelrod Python library~\cite{Knight2018, Knight2016}
with a large number of Prisoner Dilemma strategies available to give an
extensive numerical example of the ideas presented.  The approach is presented
in Section~\ref{sec:delta-zd-strategies}.  All of the code and data discussed
in Section~\ref{sec:numerical-experiments} is open sourced, archived and
written according to best scientific principles~\cite{Wilson2014}. The data
archive can be found at~\cite{vincent_knight_2018_1297075}.

\section{Recognising Extortion}\label{sec:delta-zd-strategies}

In~\cite{Press2012}, given a match between 2 memory-one strategies, the concept
of Zero Determinant (ZD) strategies is introduced. The main result of that paper
shows that given two memory one players \(p, q\in\mathbb{R}^4\) a linear
relationship between the players' scores could be forced by one of the players.

Using the notation of~\cite{Press2012}, assuming the utilities for player \(p\)
are given by \(S_x=(R, S, T, P)\) and for player \(q\) by \(S_y=(R, T, S, P)\)
and that the stationary scores of each player is given by \(S_X\) and \(S_Y\)
respectively. The main result of~\cite{Press2012} is that if

\begin{equation}\label{eqn:linear_relationship_for_p}
    \tilde p=\alpha S_x + \beta S_y + \gamma
\end{equation}

or

\begin{equation}\label{eqn:linear_relationship_for_q}
    \tilde q=\alpha S_x + \beta S_y + \gamma
\end{equation}

where \(\tilde p = (1 - p_1, 1 - p_2, p_3, p_4)\) and
\(\tilde q = (1 - q_1, 1 - q_2, q_3, q_4)\) then:

\begin{equation}
    \alpha S_X + \beta S_Y + \gamma = 0
\end{equation}

In~\cite{Press2012} a particular type of ZD strategy is defined: extortionate
strategies. If:

\begin{equation}\label{eqn:constraint_for_extortion}
    \gamma = - P(\alpha + \beta)
\end{equation}

then the player can ensure they get a score \(\chi\) times
larger than the opponent. This extortion coefficient is given by:

\begin{equation}\label{eqn:definition_of_chi}
    \chi=\frac{-\beta}{\alpha}
\end{equation}

Thus, if (\ref{eqn:constraint_for_extortion}) holds and \(\chi >1\) a player is
said to extort their opponent.
Here, the reverse problem is considered: given a
\(p\in\mathbb{R}^4\) how does one identify \(\alpha, \beta\) if they
exist and is the strategy in fact acting in an extortionate way?

These conditions correspond to:

\begin{align}
    \tilde p_1 & = \alpha R + \beta R - P (\alpha + \beta)
            \label{eqn:condition_for_tilde_p1}\\
    \tilde p_2 & = \alpha S + \beta T - P (\alpha + \beta)
            \label{eqn:condition_for_tilde_p2}\\
    \tilde p_3 & = \alpha T + \beta S - P (\alpha + \beta)
            \label{eqn:condition_for_tilde_p3}\\
    \tilde p_4 & = \alpha P + \beta P - P (\alpha + \beta)
            \label{eqn:condition_for_tilde_p4}
\end{align}

Equation (\ref{eqn:condition_for_tilde_p4}) ensures that \(p_4=\tilde p_4=0\).
Equations (\ref{eqn:condition_for_tilde_p1}-\ref{eqn:condition_for_tilde_p3})
can be used to eliminate \(\alpha, \beta\), giving:

\begin{equation}\label{eqn:planar_definition_of_extortion}
    \tilde p_1 = \frac{(R - P)(\tilde p_2 + \tilde p_3)}{S + T - 2P}
\end{equation}

with:

\begin{equation}\label{eqn:definition_of_chi}
    \chi = \frac{\tilde p_2 (P - T) + \tilde p_3 (S - P)}
                {\tilde p_2 (P - S) + \tilde p_3 (T - P)}
\end{equation}

Given a strategy \(p\in\mathbb{R}^{4\times 1}\) equations
(\ref{eqn:condition_for_tilde_p4}), (\ref{eqn:planar_definition_of_extortion}-\ref{eqn:definition_of_chi}) can be used to check if
a strategy is extortionate. The conditions correspond to:

\begin{align}
    p_1 & = \frac{(R-P)(p_2 + p_3) - R + T + S - P}{S + T - 2P}
     \label{eqn:condition_for_p1}\\
    p_4 & = 0 \label{eqn:condition_for_p4}\\
    1 & > p_2 + p_3\label{eqn:condition_for_chi}
\end{align}

The algebraic steps necessary to prove these results are available in the
supporting materials.

All extortionate strategies reside on a triangular (\ref{eqn:condition_for_chi})
plane (\ref{eqn:condition_for_p1}) in 3 dimensions (\ref{eqn:condition_for_p4}).
Using this formulation it can be seen that a necessary (but not sufficient)
condition for an extortionate strategy is that it cooperates on average less
than 50\% of the time when in a state of disagreement with the opponent.

As an example, consider the known extortionate strategy \(p=(8 / 9, 1 / 2, 1 /
3, 0)\) from~\cite{Stewart2012} which is referred to as \texttt{Extort-2}. In
this case, for the standard values of \((R, T, S, P)\) constraint
(\ref{eqn:condition_for_p1}) corresponds to:

\begin{equation}
    p_1 = \frac{2(p_2 + p_3) + 1}{3}
\end{equation}

It is clear that in this case all constraints hold.

This approach could in fact be used to confirm that a given strategy is acting
in an extortionate manner even if it is not a memory one strategy. However, in
practice, if a closed form for \(p\) is not known, then due to measurement
and/or numerical error this would not work.

This problem can be written in the following linear algebraic form where
\(x=(\alpha, \beta)\)
and \(p^*=(\tilde p_1 - 1, tilde_2 - 1, p_3)\):

\begin{equation}\label{eqn:linear_algebraic_equation_for_p}
    Cx= p^*
\end{equation}

\(C\) corresponds to equations
(\ref{eqn:condition_for_tilde_p1}-\ref{eqn:condition_for_tilde_p3}) and is
given by:

\begin{equation}\label{eqn:definition_of_C}
    C =
    \begin{bmatrix}
        R - P & R- P \\
        S - P & T- P \\
        T - P & S- P \\
    \end{bmatrix}
\end{equation}

Note that in general, equation (\ref{eqn:linear_algebraic_equation_for_p}) will
not necessarily have a solution. From the Rouch\'{e}-Capelli theorem if there is
a solution it is unique as \(\text{rank}(C)=2\) which is the dimension of the
variable \(x\). The best fitting \(x\) is found by minimizing:

\begin{equation}\label{eqn:r_squared}
    \text{SSError} = \|C x- p^*\|_2^2 = \sum_{i=1}^{3}\left((C\bar x)_i-p_i^*\right)^2
\end{equation}

Note that \(\text{SSError}\), which is the square of the Frobenius
norm~\cite{Golub2013}, becomes a measure of how close a strategy is to being an
extortionate strategy. Suspicion
of extortion then corresponds to a threshold on \(\text{SSError}\).

By observing interactions (human or otherwise), their memory one representation
can be inferred and this approach can be used to recognise extortionate
behaviour. The notion of comparing theoretic and actual plays of the IPD is not
novel, see for example~\cite{Rand2013}. Immediately it is noted that if the
environment is noisy~\cite{Wu1995} then no strategy can be considered to be
extortionate as \(p_4>0\).

In the next section, this idea will be illustrated by observing the interactions
that take place in a computer based tournament of the IPD\@.

\section{Numerical experiments}\label{sec:numerical-experiments}

In~\cite{Stewart2012} results from a tournament with
\documentclass[a4paper]{article}

\usepackage{amsmath}
\usepackage{amssymb}
\usepackage[margin=1.5cm,
            includefoot,
            footskip=30pt]{geometry}
\usepackage{layout}
\usepackage{graphicx}
\usepackage{subcaption}

\usepackage{biblatex}
\usepackage{pdfpages}

\bibliography{main.bib}

\title{Suspicion: Recognising and evaluating the effectiveness
       of extortion in the Iterated Prisoner's Dilemma}
\author{Vincent A. Knight \and Nikoleta E. Glynatsi}
\date{\today}



\begin{document}

\maketitle

\begin{abstract}
    The Iterated Prisoner's Dilemma is a model for rational and evolutionary
    interactive behaviour. It has applications both in the study of human social
    behaviour as well as in biology.
    It is used to understand when and how a rational individual might
    accept an immediate cost to their own utility for the direct benefit of
    another.

    Much attention has been given to a class of strategies called
    Zero Determinant strategies. It has been theoretically shown that these
    strategies can ``extort'' any player.

    In this work, an approach to identify if observed strategies are playing in
    an extortionate way is described. Furthermore, experimental analysis of
    a large tournament with \input{assets/tex/number_of_full_strategies/main.tex}
    strategies is considered. In this setting
    the most highly performing strategies do not play in an extortionate way
    against each other but do against lower performing strategies.
    This suggests that whilst the theory of Zero Determinant strategies
    indicates that memory is not of fundamental importance to the evolution of
    cooperative behaviour, this is incomplete.
\end{abstract}

\section{Introduction}\label{sec:introduction}

Agent based game theoretic models have become a stalwart of the underpinning
mathematics of interactive behaviours. One of the major pieces of work
in this area is the pair of original computer tournaments run by Robert
Axelrod~\cite{Axelrod1980, Axelrod1980a}. These tournaments pitted submitted
computer strategies against each other in plays of the Iterated Prisoner's
Dilemma. A common game where agents can choose to pay a slight cost to their
immediate utility in the hope of building a reputation. This has been used in
economic and evolutionary game theory to understand the evolution of cooperative
behaviour.

Recently, a class of strategies was described in~\cite{Press2012} that can
provably extort any given opponent. In~\cite{Hilbe2013, Moran1707} some
questions have already been asked about the true effectiveness of these
strategies in an evolutionary setting. Here another question is asked: is it
possible to recognise this extortionate behaviour? A mathematical procedure for
suspicion is presented: in the same way that the continued actions of an
extortionate individual might raise suspicion.

This work makes use of the Axelrod Python library~\cite{Knight2018, Knight2016}
with a large number of Prisoner Dilemma strategies available to give an
extensive numerical example of the ideas presented.  The approach is presented
in Section~\ref{sec:delta-zd-strategies}.  All of the code and data discussed
in Section~\ref{sec:numerical-experiments} is open sourced, archived and
written according to best scientific principles~\cite{Wilson2014}. The data
archive can be found at~\cite{vincent_knight_2018_1297075}.

\section{Recognising Extortion}\label{sec:delta-zd-strategies}

In~\cite{Press2012}, given a match between 2 memory-one strategies, the concept
of Zero Determinant (ZD) strategies is introduced. The main result of that paper
shows that given two memory one players \(p, q\in\mathbb{R}^4\) a linear
relationship between the players' scores could be forced by one of the players.

Using the notation of~\cite{Press2012}, assuming the utilities for player \(p\)
are given by \(S_x=(R, S, T, P)\) and for player \(q\) by \(S_y=(R, T, S, P)\)
and that the stationary scores of each player is given by \(S_X\) and \(S_Y\)
respectively. The main result of~\cite{Press2012} is that if

\begin{equation}\label{eqn:linear_relationship_for_p}
    \tilde p=\alpha S_x + \beta S_y + \gamma
\end{equation}

or

\begin{equation}\label{eqn:linear_relationship_for_q}
    \tilde q=\alpha S_x + \beta S_y + \gamma
\end{equation}

where \(\tilde p = (1 - p_1, 1 - p_2, p_3, p_4)\) and
\(\tilde q = (1 - q_1, 1 - q_2, q_3, q_4)\) then:

\begin{equation}
    \alpha S_X + \beta S_Y + \gamma = 0
\end{equation}

In~\cite{Press2012} a particular type of ZD strategy is defined: extortionate
strategies. If:

\begin{equation}\label{eqn:constraint_for_extortion}
    \gamma = - P(\alpha + \beta)
\end{equation}

then the player can ensure they get a score \(\chi\) times
larger than the opponent. This extortion coefficient is given by:

\begin{equation}\label{eqn:definition_of_chi}
    \chi=\frac{-\beta}{\alpha}
\end{equation}

Thus, if (\ref{eqn:constraint_for_extortion}) holds and \(\chi >1\) a player is
said to extort their opponent.
Here, the reverse problem is considered: given a
\(p\in\mathbb{R}^4\) how does one identify \(\alpha, \beta\) if they
exist and is the strategy in fact acting in an extortionate way?

These conditions correspond to:

\begin{align}
    \tilde p_1 & = \alpha R + \beta R - P (\alpha + \beta)
            \label{eqn:condition_for_tilde_p1}\\
    \tilde p_2 & = \alpha S + \beta T - P (\alpha + \beta)
            \label{eqn:condition_for_tilde_p2}\\
    \tilde p_3 & = \alpha T + \beta S - P (\alpha + \beta)
            \label{eqn:condition_for_tilde_p3}\\
    \tilde p_4 & = \alpha P + \beta P - P (\alpha + \beta)
            \label{eqn:condition_for_tilde_p4}
\end{align}

Equation (\ref{eqn:condition_for_tilde_p4}) ensures that \(p_4=\tilde p_4=0\).
Equations (\ref{eqn:condition_for_tilde_p1}-\ref{eqn:condition_for_tilde_p3})
can be used to eliminate \(\alpha, \beta\), giving:

\begin{equation}\label{eqn:planar_definition_of_extortion}
    \tilde p_1 = \frac{(R - P)(\tilde p_2 + \tilde p_3)}{S + T - 2P}
\end{equation}

with:

\begin{equation}\label{eqn:definition_of_chi}
    \chi = \frac{\tilde p_2 (P - T) + \tilde p_3 (S - P)}
                {\tilde p_2 (P - S) + \tilde p_3 (T - P)}
\end{equation}

Given a strategy \(p\in\mathbb{R}^{4\times 1}\) equations
(\ref{eqn:condition_for_tilde_p4}), (\ref{eqn:planar_definition_of_extortion}-\ref{eqn:definition_of_chi}) can be used to check if
a strategy is extortionate. The conditions correspond to:

\begin{align}
    p_1 & = \frac{(R-P)(p_2 + p_3) - R + T + S - P}{S + T - 2P}
     \label{eqn:condition_for_p1}\\
    p_4 & = 0 \label{eqn:condition_for_p4}\\
    1 & > p_2 + p_3\label{eqn:condition_for_chi}
\end{align}

The algebraic steps necessary to prove these results are available in the
supporting materials.

All extortionate strategies reside on a triangular (\ref{eqn:condition_for_chi})
plane (\ref{eqn:condition_for_p1}) in 3 dimensions (\ref{eqn:condition_for_p4}).
Using this formulation it can be seen that a necessary (but not sufficient)
condition for an extortionate strategy is that it cooperates on average less
than 50\% of the time when in a state of disagreement with the opponent.

As an example, consider the known extortionate strategy \(p=(8 / 9, 1 / 2, 1 /
3, 0)\) from~\cite{Stewart2012} which is referred to as \texttt{Extort-2}. In
this case, for the standard values of \((R, T, S, P)\) constraint
(\ref{eqn:condition_for_p1}) corresponds to:

\begin{equation}
    p_1 = \frac{2(p_2 + p_3) + 1}{3}
\end{equation}

It is clear that in this case all constraints hold.

This approach could in fact be used to confirm that a given strategy is acting
in an extortionate manner even if it is not a memory one strategy. However, in
practice, if a closed form for \(p\) is not known, then due to measurement
and/or numerical error this would not work.

This problem can be written in the following linear algebraic form where
\(x=(\alpha, \beta)\)
and \(p^*=(\tilde p_1 - 1, tilde_2 - 1, p_3)\):

\begin{equation}\label{eqn:linear_algebraic_equation_for_p}
    Cx= p^*
\end{equation}

\(C\) corresponds to equations
(\ref{eqn:condition_for_tilde_p1}-\ref{eqn:condition_for_tilde_p3}) and is
given by:

\begin{equation}\label{eqn:definition_of_C}
    C =
    \begin{bmatrix}
        R - P & R- P \\
        S - P & T- P \\
        T - P & S- P \\
    \end{bmatrix}
\end{equation}

Note that in general, equation (\ref{eqn:linear_algebraic_equation_for_p}) will
not necessarily have a solution. From the Rouch\'{e}-Capelli theorem if there is
a solution it is unique as \(\text{rank}(C)=2\) which is the dimension of the
variable \(x\). The best fitting \(x\) is found by minimizing:

\begin{equation}\label{eqn:r_squared}
    \text{SSError} = \|C x- p^*\|_2^2 = \sum_{i=1}^{3}\left((C\bar x)_i-p_i^*\right)^2
\end{equation}

Note that \(\text{SSError}\), which is the square of the Frobenius
norm~\cite{Golub2013}, becomes a measure of how close a strategy is to being an
extortionate strategy. Suspicion
of extortion then corresponds to a threshold on \(\text{SSError}\).

By observing interactions (human or otherwise), their memory one representation
can be inferred and this approach can be used to recognise extortionate
behaviour. The notion of comparing theoretic and actual plays of the IPD is not
novel, see for example~\cite{Rand2013}. Immediately it is noted that if the
environment is noisy~\cite{Wu1995} then no strategy can be considered to be
extortionate as \(p_4>0\).

In the next section, this idea will be illustrated by observing the interactions
that take place in a computer based tournament of the IPD\@.

\section{Numerical experiments}\label{sec:numerical-experiments}

In~\cite{Stewart2012} results from a tournament with
\input{./assets/tex/number_of_stewart_plotkin_strategies/main.tex} strategies,
was presented with specific consideration given to ZD strategies. This
tournament is reproduced here using the Axelrod-Python
project~\cite{Knight2016}. To obtain a good measure of the corresponding
transition rates for each strategy all matches have been run for
\input{assets/tex/number_of_turns/main.tex} turns and every match has been
repeated \input{assets/tex/number_of_repetitions/main.tex} times. All of this
interaction data is available at~\cite{vincent_knight_2018_1297075}. A good
match between the inferred Markov chain and the state distribution of the actual
interactions has been verified. Data for this is presented in the supplementary
materials.

Figure~\ref{fig:SSError_overall_in_stewart_plotkin} shows the \(\text{SSError}\)
values for all the strategies in the tournament, as reported
in~\cite{Stewart2012} the extortionate strategy (which has an expected
\(\text{SSError}\) approximately 0) gains a large number of wins.

\begin{figure}[!htbp]
    \centering
    \includegraphics[width=.8\textwidth]{./assets/img/SSError_overall_in_stewart_plotkin/main.pdf}
    \caption{\(\text{SSError}\) and state probabilities for the strategies
        of~\cite{Stewart2012}, ordered both by number of wins and overall score.
        Note that \(P(DC)\) is not shown as it corresponds to the transpose of
        \(P(CD)\). Cooperator and Defector are omitted as they do not visit all
        the states.}
    \label{fig:SSError_overall_in_stewart_plotkin}
\end{figure}

Here, the work of~\cite{Stewart2012} is extended by investigating a tournament
with \input{assets/tex/number_of_full_strategies/main.tex}
strategies.

The results of this analysis are shown in
Figure~\ref{fig:SSError_and_probabilities_in_full}. The top ranking strategies
by number of wins seem to be extortionate (but not against all strategies) and
it can be seen that a small sub group of strategies achieve mutual defection.
All the top ranking strategies according to score achieve mutual cooperation and
do not extort each other, however they
\textbf{do} exhibit extortionate behaviour towards a number of the lower ranking
strategies.

\begin{figure}[!htbp]
    \centering
    \includegraphics[width=.8\textwidth]{./assets/img/SSError_and_probabilities_in_full/main.pdf}
    \caption{\(\text{SSError}\) for the strategies for the full tournament. Only
    strategy interactions for which \(p_4=0\) and \(\chi>1\) are displayed.}
    \label{fig:SSError_and_probabilities_in_full}
\end{figure}

\section{Conclusion}\label{sec:conclusion}

This work defines an approach to measure whether or not a player is playing a
strategy that corresponds to an extortionate strategy as defined
in~\cite{Press2012}: a mathematical model for suspicion. Indeed, all
extortionate strategies have been
 classified as lying on a triangular plane.
This rigorous classification fails to be robust to small measurement error, thus
a statistical approach is proposed.
This is done through a linear algebraic approach for approximating the solution
of a linear system. Using this, a large number of pairwise interactions is
simulated and in fact very few strategies are found to act extortionately.

The work of~\cite{Press2012}, whilst showing that a clever approach to taking
advantage of another memory one strategy exists: this is incomplete. Whilst the
elegance of this result is very attractive, just as the simplicity of the
victory of Tit For Tat in Axelrod's original tournaments was, it is incomplete.
Extortionate strategies achieve a high number of wins but they do not
achieve a high score which corresponds to the fitness landscape in an
evolutionary sense. From the large number of interactions a payoff matrix \(S\)
can be measured where \(S_{ij}\) denotes the score (using standard values of
\((R, S, T, P) = (3, 0, 5, 1)\)) of the \(i\)th strategy
against the \(j\)th strategy. Using this, the replicator equation
describes the evolution of the system based on a population density fitness
function:

\begin{equation}\label{eqn:replicator_dynamics}
    \frac{dx}{dt} = x(S-x^TS x)
\end{equation}

Equation (\ref{eqn:replicator_dynamics}) is solved numerically through an
integration technique described in~\cite{Petzold1983} and
Figure~\ref{fig:replicator_dynamics} shows the evolution of the distribution of
the system: the various strategies are ranked by scores. It is clear to see that
only the high ranking strategies survive the evolutionary process (in fact,
only \input{./assets/img/replicator_dynamics/main.tex}
have a final distribution greater than \(10 ^ {-2}\)). This confirms the
findings of~\cite{Moran1707} in which sophisticated strategies resist
evolutionary invasion of shorter memory strategies. Recalling
Figure~\ref{fig:SSError_and_probabilities_in_full} this demonstrates that:

\begin{itemize}
    \item Cooperation emerges through the evolutionary process: the high scoring
        strategies do not exhibit extortionate behaviour towards each other.
    \item Extortionate strategies do not survive the evolutionary process.
\end{itemize}

\begin{figure}[!htbp]
    \centering
    \includegraphics[width=.8\textwidth]{./assets/img/replicator_dynamics/main.pdf}
    \caption{Numerical simulation of the replicator equation
    (\ref{eqn:replicator_dynamics}): strategies are ordered by score, only the strategies with a high score survive the evolutionary process.}
    \label{fig:replicator_dynamics}
\end{figure}

This work can be used to classify plays of the IPD\@: data can be collected from
actual interactions (in lab or in the field). Furthermore, this allows for a
classification method similar to the notion of fingerprinting presented
in~\cite{Ashlock2008}. Trained strategies can potentially be classified as
extortionate or not or it could be possible to even constrain the reinforcement
learning approaches that are becoming prevalent in the literature.
Alternatively, this mathematical approach for recognising extortion could be
used in sophisticated strategies to defend against invasion. Arguably, some of
the strategies considered here exhibit this behaviour, indeed as described
in~\cite{Harper2017}, the top ranking strategies in the full tournament are
obtained using evolutionary reinforcement learning techniques, thus, suspicion
of extortionate behaviour could in fact be an evolutionary trait.

\section*{Acknowledgements}

The following open source software libraries were used in this research:

\begin{itemize}
    \item The Axelrod ~\cite{Knight2016, Knight2018} library (IPD strategies and
        tournaments).
    \item The sympy library~\cite{Meurer2017} (verification of all symbolic
        calculations).
    \item The matplotlib~\cite{Droettboom2018} library (visualisation).
    \item The pandas~\cite{Structures2010}, dask~\cite{Dask2016} and
        NumPy~\cite{Oliphant2015} libraries (data manipulation).
    \item The SciPy~\cite{Jones2001} library (numerical integration of the
        replicator equation).
\end{itemize}

This work was performed using the computational facilities of the Advanced
Research Computing @ Cardiff (ARCCA) Division, Cardiff University.

\printbibliography

\newpage
\section*{Supplementary materials}

\includepdf{assets/pdf/proof_of_form_of_extortionate_strategies/main.pdf}

\newpage

Using the pair wise interactions the transition rates \(p,
q\) can be measured and the steady state probabilities inferred and compared to
the actual probabilities of each state.
This is done numerically by computing the singular eigenvector of the
matrix \(A\) \cite{Stewart2009}:

\[
    A =
    \begin{bmatrix}
        p_1 q_1 & p_1 (1 - q_1) & (1 - p_1) q_1 & (1 -p_1) (1 - q_1) \\
        p_2 q_2 & p_2 (1 - q_2) & (1 - p_2) q_2 & (1 -p_2) (1 - q_2) \\
        p_3 q_3 & p_3 (1 - q_3) & (1 - p_3) q_3 & (1 -p_3) (1 - q_3) \\
        p_4 q_4 & p_4 (1 - q_4) & (1 - p_4) q_4 & (1 -p_4) (1 - q_4) \\
    \end{bmatrix}
\]

Figure~\ref{fig:computed_probabilities_vs_theoretic_probabilities} shows a
regression line fitted to every pairwise interaction with a reported
\(\text{SSError}\) value (pairwise interactions with missing states were
omitted). This serves to validate the approach: a part from some edge cases the
relationship is consistent.

\begin{figure}[!htbp]
    \centering
    \includegraphics[width=.8\textwidth]{./assets/img/computed_probabilities_vs_theoretic_probabilities/main.pdf}
    \caption{The
        relationship between the steady state probabilities inferred from the
        measured transitions and the actual steady state probabilities. A linear
        regression line is included validating the approach.}
    \label{fig:computed_probabilities_vs_theoretic_probabilities}
\end{figure}


\end{document}
 strategies,
was presented with specific consideration given to ZD strategies. This
tournament is reproduced here using the Axelrod-Python
project~\cite{Knight2016}. To obtain a good measure of the corresponding
transition rates for each strategy all matches have been run for
\documentclass[a4paper]{article}

\usepackage{amsmath}
\usepackage{amssymb}
\usepackage[margin=1.5cm,
            includefoot,
            footskip=30pt]{geometry}
\usepackage{layout}
\usepackage{graphicx}
\usepackage{subcaption}

\usepackage{biblatex}
\usepackage{pdfpages}

\bibliography{main.bib}

\title{Suspicion: Recognising and evaluating the effectiveness
       of extortion in the Iterated Prisoner's Dilemma}
\author{Vincent A. Knight \and Nikoleta E. Glynatsi}
\date{\today}



\begin{document}

\maketitle

\begin{abstract}
    The Iterated Prisoner's Dilemma is a model for rational and evolutionary
    interactive behaviour. It has applications both in the study of human social
    behaviour as well as in biology.
    It is used to understand when and how a rational individual might
    accept an immediate cost to their own utility for the direct benefit of
    another.

    Much attention has been given to a class of strategies called
    Zero Determinant strategies. It has been theoretically shown that these
    strategies can ``extort'' any player.

    In this work, an approach to identify if observed strategies are playing in
    an extortionate way is described. Furthermore, experimental analysis of
    a large tournament with \input{assets/tex/number_of_full_strategies/main.tex}
    strategies is considered. In this setting
    the most highly performing strategies do not play in an extortionate way
    against each other but do against lower performing strategies.
    This suggests that whilst the theory of Zero Determinant strategies
    indicates that memory is not of fundamental importance to the evolution of
    cooperative behaviour, this is incomplete.
\end{abstract}

\section{Introduction}\label{sec:introduction}

Agent based game theoretic models have become a stalwart of the underpinning
mathematics of interactive behaviours. One of the major pieces of work
in this area is the pair of original computer tournaments run by Robert
Axelrod~\cite{Axelrod1980, Axelrod1980a}. These tournaments pitted submitted
computer strategies against each other in plays of the Iterated Prisoner's
Dilemma. A common game where agents can choose to pay a slight cost to their
immediate utility in the hope of building a reputation. This has been used in
economic and evolutionary game theory to understand the evolution of cooperative
behaviour.

Recently, a class of strategies was described in~\cite{Press2012} that can
provably extort any given opponent. In~\cite{Hilbe2013, Moran1707} some
questions have already been asked about the true effectiveness of these
strategies in an evolutionary setting. Here another question is asked: is it
possible to recognise this extortionate behaviour? A mathematical procedure for
suspicion is presented: in the same way that the continued actions of an
extortionate individual might raise suspicion.

This work makes use of the Axelrod Python library~\cite{Knight2018, Knight2016}
with a large number of Prisoner Dilemma strategies available to give an
extensive numerical example of the ideas presented.  The approach is presented
in Section~\ref{sec:delta-zd-strategies}.  All of the code and data discussed
in Section~\ref{sec:numerical-experiments} is open sourced, archived and
written according to best scientific principles~\cite{Wilson2014}. The data
archive can be found at~\cite{vincent_knight_2018_1297075}.

\section{Recognising Extortion}\label{sec:delta-zd-strategies}

In~\cite{Press2012}, given a match between 2 memory-one strategies, the concept
of Zero Determinant (ZD) strategies is introduced. The main result of that paper
shows that given two memory one players \(p, q\in\mathbb{R}^4\) a linear
relationship between the players' scores could be forced by one of the players.

Using the notation of~\cite{Press2012}, assuming the utilities for player \(p\)
are given by \(S_x=(R, S, T, P)\) and for player \(q\) by \(S_y=(R, T, S, P)\)
and that the stationary scores of each player is given by \(S_X\) and \(S_Y\)
respectively. The main result of~\cite{Press2012} is that if

\begin{equation}\label{eqn:linear_relationship_for_p}
    \tilde p=\alpha S_x + \beta S_y + \gamma
\end{equation}

or

\begin{equation}\label{eqn:linear_relationship_for_q}
    \tilde q=\alpha S_x + \beta S_y + \gamma
\end{equation}

where \(\tilde p = (1 - p_1, 1 - p_2, p_3, p_4)\) and
\(\tilde q = (1 - q_1, 1 - q_2, q_3, q_4)\) then:

\begin{equation}
    \alpha S_X + \beta S_Y + \gamma = 0
\end{equation}

In~\cite{Press2012} a particular type of ZD strategy is defined: extortionate
strategies. If:

\begin{equation}\label{eqn:constraint_for_extortion}
    \gamma = - P(\alpha + \beta)
\end{equation}

then the player can ensure they get a score \(\chi\) times
larger than the opponent. This extortion coefficient is given by:

\begin{equation}\label{eqn:definition_of_chi}
    \chi=\frac{-\beta}{\alpha}
\end{equation}

Thus, if (\ref{eqn:constraint_for_extortion}) holds and \(\chi >1\) a player is
said to extort their opponent.
Here, the reverse problem is considered: given a
\(p\in\mathbb{R}^4\) how does one identify \(\alpha, \beta\) if they
exist and is the strategy in fact acting in an extortionate way?

These conditions correspond to:

\begin{align}
    \tilde p_1 & = \alpha R + \beta R - P (\alpha + \beta)
            \label{eqn:condition_for_tilde_p1}\\
    \tilde p_2 & = \alpha S + \beta T - P (\alpha + \beta)
            \label{eqn:condition_for_tilde_p2}\\
    \tilde p_3 & = \alpha T + \beta S - P (\alpha + \beta)
            \label{eqn:condition_for_tilde_p3}\\
    \tilde p_4 & = \alpha P + \beta P - P (\alpha + \beta)
            \label{eqn:condition_for_tilde_p4}
\end{align}

Equation (\ref{eqn:condition_for_tilde_p4}) ensures that \(p_4=\tilde p_4=0\).
Equations (\ref{eqn:condition_for_tilde_p1}-\ref{eqn:condition_for_tilde_p3})
can be used to eliminate \(\alpha, \beta\), giving:

\begin{equation}\label{eqn:planar_definition_of_extortion}
    \tilde p_1 = \frac{(R - P)(\tilde p_2 + \tilde p_3)}{S + T - 2P}
\end{equation}

with:

\begin{equation}\label{eqn:definition_of_chi}
    \chi = \frac{\tilde p_2 (P - T) + \tilde p_3 (S - P)}
                {\tilde p_2 (P - S) + \tilde p_3 (T - P)}
\end{equation}

Given a strategy \(p\in\mathbb{R}^{4\times 1}\) equations
(\ref{eqn:condition_for_tilde_p4}), (\ref{eqn:planar_definition_of_extortion}-\ref{eqn:definition_of_chi}) can be used to check if
a strategy is extortionate. The conditions correspond to:

\begin{align}
    p_1 & = \frac{(R-P)(p_2 + p_3) - R + T + S - P}{S + T - 2P}
     \label{eqn:condition_for_p1}\\
    p_4 & = 0 \label{eqn:condition_for_p4}\\
    1 & > p_2 + p_3\label{eqn:condition_for_chi}
\end{align}

The algebraic steps necessary to prove these results are available in the
supporting materials.

All extortionate strategies reside on a triangular (\ref{eqn:condition_for_chi})
plane (\ref{eqn:condition_for_p1}) in 3 dimensions (\ref{eqn:condition_for_p4}).
Using this formulation it can be seen that a necessary (but not sufficient)
condition for an extortionate strategy is that it cooperates on average less
than 50\% of the time when in a state of disagreement with the opponent.

As an example, consider the known extortionate strategy \(p=(8 / 9, 1 / 2, 1 /
3, 0)\) from~\cite{Stewart2012} which is referred to as \texttt{Extort-2}. In
this case, for the standard values of \((R, T, S, P)\) constraint
(\ref{eqn:condition_for_p1}) corresponds to:

\begin{equation}
    p_1 = \frac{2(p_2 + p_3) + 1}{3}
\end{equation}

It is clear that in this case all constraints hold.

This approach could in fact be used to confirm that a given strategy is acting
in an extortionate manner even if it is not a memory one strategy. However, in
practice, if a closed form for \(p\) is not known, then due to measurement
and/or numerical error this would not work.

This problem can be written in the following linear algebraic form where
\(x=(\alpha, \beta)\)
and \(p^*=(\tilde p_1 - 1, tilde_2 - 1, p_3)\):

\begin{equation}\label{eqn:linear_algebraic_equation_for_p}
    Cx= p^*
\end{equation}

\(C\) corresponds to equations
(\ref{eqn:condition_for_tilde_p1}-\ref{eqn:condition_for_tilde_p3}) and is
given by:

\begin{equation}\label{eqn:definition_of_C}
    C =
    \begin{bmatrix}
        R - P & R- P \\
        S - P & T- P \\
        T - P & S- P \\
    \end{bmatrix}
\end{equation}

Note that in general, equation (\ref{eqn:linear_algebraic_equation_for_p}) will
not necessarily have a solution. From the Rouch\'{e}-Capelli theorem if there is
a solution it is unique as \(\text{rank}(C)=2\) which is the dimension of the
variable \(x\). The best fitting \(x\) is found by minimizing:

\begin{equation}\label{eqn:r_squared}
    \text{SSError} = \|C x- p^*\|_2^2 = \sum_{i=1}^{3}\left((C\bar x)_i-p_i^*\right)^2
\end{equation}

Note that \(\text{SSError}\), which is the square of the Frobenius
norm~\cite{Golub2013}, becomes a measure of how close a strategy is to being an
extortionate strategy. Suspicion
of extortion then corresponds to a threshold on \(\text{SSError}\).

By observing interactions (human or otherwise), their memory one representation
can be inferred and this approach can be used to recognise extortionate
behaviour. The notion of comparing theoretic and actual plays of the IPD is not
novel, see for example~\cite{Rand2013}. Immediately it is noted that if the
environment is noisy~\cite{Wu1995} then no strategy can be considered to be
extortionate as \(p_4>0\).

In the next section, this idea will be illustrated by observing the interactions
that take place in a computer based tournament of the IPD\@.

\section{Numerical experiments}\label{sec:numerical-experiments}

In~\cite{Stewart2012} results from a tournament with
\input{./assets/tex/number_of_stewart_plotkin_strategies/main.tex} strategies,
was presented with specific consideration given to ZD strategies. This
tournament is reproduced here using the Axelrod-Python
project~\cite{Knight2016}. To obtain a good measure of the corresponding
transition rates for each strategy all matches have been run for
\input{assets/tex/number_of_turns/main.tex} turns and every match has been
repeated \input{assets/tex/number_of_repetitions/main.tex} times. All of this
interaction data is available at~\cite{vincent_knight_2018_1297075}. A good
match between the inferred Markov chain and the state distribution of the actual
interactions has been verified. Data for this is presented in the supplementary
materials.

Figure~\ref{fig:SSError_overall_in_stewart_plotkin} shows the \(\text{SSError}\)
values for all the strategies in the tournament, as reported
in~\cite{Stewart2012} the extortionate strategy (which has an expected
\(\text{SSError}\) approximately 0) gains a large number of wins.

\begin{figure}[!htbp]
    \centering
    \includegraphics[width=.8\textwidth]{./assets/img/SSError_overall_in_stewart_plotkin/main.pdf}
    \caption{\(\text{SSError}\) and state probabilities for the strategies
        of~\cite{Stewart2012}, ordered both by number of wins and overall score.
        Note that \(P(DC)\) is not shown as it corresponds to the transpose of
        \(P(CD)\). Cooperator and Defector are omitted as they do not visit all
        the states.}
    \label{fig:SSError_overall_in_stewart_plotkin}
\end{figure}

Here, the work of~\cite{Stewart2012} is extended by investigating a tournament
with \input{assets/tex/number_of_full_strategies/main.tex}
strategies.

The results of this analysis are shown in
Figure~\ref{fig:SSError_and_probabilities_in_full}. The top ranking strategies
by number of wins seem to be extortionate (but not against all strategies) and
it can be seen that a small sub group of strategies achieve mutual defection.
All the top ranking strategies according to score achieve mutual cooperation and
do not extort each other, however they
\textbf{do} exhibit extortionate behaviour towards a number of the lower ranking
strategies.

\begin{figure}[!htbp]
    \centering
    \includegraphics[width=.8\textwidth]{./assets/img/SSError_and_probabilities_in_full/main.pdf}
    \caption{\(\text{SSError}\) for the strategies for the full tournament. Only
    strategy interactions for which \(p_4=0\) and \(\chi>1\) are displayed.}
    \label{fig:SSError_and_probabilities_in_full}
\end{figure}

\section{Conclusion}\label{sec:conclusion}

This work defines an approach to measure whether or not a player is playing a
strategy that corresponds to an extortionate strategy as defined
in~\cite{Press2012}: a mathematical model for suspicion. Indeed, all
extortionate strategies have been
 classified as lying on a triangular plane.
This rigorous classification fails to be robust to small measurement error, thus
a statistical approach is proposed.
This is done through a linear algebraic approach for approximating the solution
of a linear system. Using this, a large number of pairwise interactions is
simulated and in fact very few strategies are found to act extortionately.

The work of~\cite{Press2012}, whilst showing that a clever approach to taking
advantage of another memory one strategy exists: this is incomplete. Whilst the
elegance of this result is very attractive, just as the simplicity of the
victory of Tit For Tat in Axelrod's original tournaments was, it is incomplete.
Extortionate strategies achieve a high number of wins but they do not
achieve a high score which corresponds to the fitness landscape in an
evolutionary sense. From the large number of interactions a payoff matrix \(S\)
can be measured where \(S_{ij}\) denotes the score (using standard values of
\((R, S, T, P) = (3, 0, 5, 1)\)) of the \(i\)th strategy
against the \(j\)th strategy. Using this, the replicator equation
describes the evolution of the system based on a population density fitness
function:

\begin{equation}\label{eqn:replicator_dynamics}
    \frac{dx}{dt} = x(S-x^TS x)
\end{equation}

Equation (\ref{eqn:replicator_dynamics}) is solved numerically through an
integration technique described in~\cite{Petzold1983} and
Figure~\ref{fig:replicator_dynamics} shows the evolution of the distribution of
the system: the various strategies are ranked by scores. It is clear to see that
only the high ranking strategies survive the evolutionary process (in fact,
only \input{./assets/img/replicator_dynamics/main.tex}
have a final distribution greater than \(10 ^ {-2}\)). This confirms the
findings of~\cite{Moran1707} in which sophisticated strategies resist
evolutionary invasion of shorter memory strategies. Recalling
Figure~\ref{fig:SSError_and_probabilities_in_full} this demonstrates that:

\begin{itemize}
    \item Cooperation emerges through the evolutionary process: the high scoring
        strategies do not exhibit extortionate behaviour towards each other.
    \item Extortionate strategies do not survive the evolutionary process.
\end{itemize}

\begin{figure}[!htbp]
    \centering
    \includegraphics[width=.8\textwidth]{./assets/img/replicator_dynamics/main.pdf}
    \caption{Numerical simulation of the replicator equation
    (\ref{eqn:replicator_dynamics}): strategies are ordered by score, only the strategies with a high score survive the evolutionary process.}
    \label{fig:replicator_dynamics}
\end{figure}

This work can be used to classify plays of the IPD\@: data can be collected from
actual interactions (in lab or in the field). Furthermore, this allows for a
classification method similar to the notion of fingerprinting presented
in~\cite{Ashlock2008}. Trained strategies can potentially be classified as
extortionate or not or it could be possible to even constrain the reinforcement
learning approaches that are becoming prevalent in the literature.
Alternatively, this mathematical approach for recognising extortion could be
used in sophisticated strategies to defend against invasion. Arguably, some of
the strategies considered here exhibit this behaviour, indeed as described
in~\cite{Harper2017}, the top ranking strategies in the full tournament are
obtained using evolutionary reinforcement learning techniques, thus, suspicion
of extortionate behaviour could in fact be an evolutionary trait.

\section*{Acknowledgements}

The following open source software libraries were used in this research:

\begin{itemize}
    \item The Axelrod ~\cite{Knight2016, Knight2018} library (IPD strategies and
        tournaments).
    \item The sympy library~\cite{Meurer2017} (verification of all symbolic
        calculations).
    \item The matplotlib~\cite{Droettboom2018} library (visualisation).
    \item The pandas~\cite{Structures2010}, dask~\cite{Dask2016} and
        NumPy~\cite{Oliphant2015} libraries (data manipulation).
    \item The SciPy~\cite{Jones2001} library (numerical integration of the
        replicator equation).
\end{itemize}

This work was performed using the computational facilities of the Advanced
Research Computing @ Cardiff (ARCCA) Division, Cardiff University.

\printbibliography

\newpage
\section*{Supplementary materials}

\includepdf{assets/pdf/proof_of_form_of_extortionate_strategies/main.pdf}

\newpage

Using the pair wise interactions the transition rates \(p,
q\) can be measured and the steady state probabilities inferred and compared to
the actual probabilities of each state.
This is done numerically by computing the singular eigenvector of the
matrix \(A\) \cite{Stewart2009}:

\[
    A =
    \begin{bmatrix}
        p_1 q_1 & p_1 (1 - q_1) & (1 - p_1) q_1 & (1 -p_1) (1 - q_1) \\
        p_2 q_2 & p_2 (1 - q_2) & (1 - p_2) q_2 & (1 -p_2) (1 - q_2) \\
        p_3 q_3 & p_3 (1 - q_3) & (1 - p_3) q_3 & (1 -p_3) (1 - q_3) \\
        p_4 q_4 & p_4 (1 - q_4) & (1 - p_4) q_4 & (1 -p_4) (1 - q_4) \\
    \end{bmatrix}
\]

Figure~\ref{fig:computed_probabilities_vs_theoretic_probabilities} shows a
regression line fitted to every pairwise interaction with a reported
\(\text{SSError}\) value (pairwise interactions with missing states were
omitted). This serves to validate the approach: a part from some edge cases the
relationship is consistent.

\begin{figure}[!htbp]
    \centering
    \includegraphics[width=.8\textwidth]{./assets/img/computed_probabilities_vs_theoretic_probabilities/main.pdf}
    \caption{The
        relationship between the steady state probabilities inferred from the
        measured transitions and the actual steady state probabilities. A linear
        regression line is included validating the approach.}
    \label{fig:computed_probabilities_vs_theoretic_probabilities}
\end{figure}


\end{document}
 turns and every match has been
repeated \documentclass[a4paper]{article}

\usepackage{amsmath}
\usepackage{amssymb}
\usepackage[margin=1.5cm,
            includefoot,
            footskip=30pt]{geometry}
\usepackage{layout}
\usepackage{graphicx}
\usepackage{subcaption}

\usepackage{biblatex}
\usepackage{pdfpages}

\bibliography{main.bib}

\title{Suspicion: Recognising and evaluating the effectiveness
       of extortion in the Iterated Prisoner's Dilemma}
\author{Vincent A. Knight \and Nikoleta E. Glynatsi}
\date{\today}



\begin{document}

\maketitle

\begin{abstract}
    The Iterated Prisoner's Dilemma is a model for rational and evolutionary
    interactive behaviour. It has applications both in the study of human social
    behaviour as well as in biology.
    It is used to understand when and how a rational individual might
    accept an immediate cost to their own utility for the direct benefit of
    another.

    Much attention has been given to a class of strategies called
    Zero Determinant strategies. It has been theoretically shown that these
    strategies can ``extort'' any player.

    In this work, an approach to identify if observed strategies are playing in
    an extortionate way is described. Furthermore, experimental analysis of
    a large tournament with \input{assets/tex/number_of_full_strategies/main.tex}
    strategies is considered. In this setting
    the most highly performing strategies do not play in an extortionate way
    against each other but do against lower performing strategies.
    This suggests that whilst the theory of Zero Determinant strategies
    indicates that memory is not of fundamental importance to the evolution of
    cooperative behaviour, this is incomplete.
\end{abstract}

\section{Introduction}\label{sec:introduction}

Agent based game theoretic models have become a stalwart of the underpinning
mathematics of interactive behaviours. One of the major pieces of work
in this area is the pair of original computer tournaments run by Robert
Axelrod~\cite{Axelrod1980, Axelrod1980a}. These tournaments pitted submitted
computer strategies against each other in plays of the Iterated Prisoner's
Dilemma. A common game where agents can choose to pay a slight cost to their
immediate utility in the hope of building a reputation. This has been used in
economic and evolutionary game theory to understand the evolution of cooperative
behaviour.

Recently, a class of strategies was described in~\cite{Press2012} that can
provably extort any given opponent. In~\cite{Hilbe2013, Moran1707} some
questions have already been asked about the true effectiveness of these
strategies in an evolutionary setting. Here another question is asked: is it
possible to recognise this extortionate behaviour? A mathematical procedure for
suspicion is presented: in the same way that the continued actions of an
extortionate individual might raise suspicion.

This work makes use of the Axelrod Python library~\cite{Knight2018, Knight2016}
with a large number of Prisoner Dilemma strategies available to give an
extensive numerical example of the ideas presented.  The approach is presented
in Section~\ref{sec:delta-zd-strategies}.  All of the code and data discussed
in Section~\ref{sec:numerical-experiments} is open sourced, archived and
written according to best scientific principles~\cite{Wilson2014}. The data
archive can be found at~\cite{vincent_knight_2018_1297075}.

\section{Recognising Extortion}\label{sec:delta-zd-strategies}

In~\cite{Press2012}, given a match between 2 memory-one strategies, the concept
of Zero Determinant (ZD) strategies is introduced. The main result of that paper
shows that given two memory one players \(p, q\in\mathbb{R}^4\) a linear
relationship between the players' scores could be forced by one of the players.

Using the notation of~\cite{Press2012}, assuming the utilities for player \(p\)
are given by \(S_x=(R, S, T, P)\) and for player \(q\) by \(S_y=(R, T, S, P)\)
and that the stationary scores of each player is given by \(S_X\) and \(S_Y\)
respectively. The main result of~\cite{Press2012} is that if

\begin{equation}\label{eqn:linear_relationship_for_p}
    \tilde p=\alpha S_x + \beta S_y + \gamma
\end{equation}

or

\begin{equation}\label{eqn:linear_relationship_for_q}
    \tilde q=\alpha S_x + \beta S_y + \gamma
\end{equation}

where \(\tilde p = (1 - p_1, 1 - p_2, p_3, p_4)\) and
\(\tilde q = (1 - q_1, 1 - q_2, q_3, q_4)\) then:

\begin{equation}
    \alpha S_X + \beta S_Y + \gamma = 0
\end{equation}

In~\cite{Press2012} a particular type of ZD strategy is defined: extortionate
strategies. If:

\begin{equation}\label{eqn:constraint_for_extortion}
    \gamma = - P(\alpha + \beta)
\end{equation}

then the player can ensure they get a score \(\chi\) times
larger than the opponent. This extortion coefficient is given by:

\begin{equation}\label{eqn:definition_of_chi}
    \chi=\frac{-\beta}{\alpha}
\end{equation}

Thus, if (\ref{eqn:constraint_for_extortion}) holds and \(\chi >1\) a player is
said to extort their opponent.
Here, the reverse problem is considered: given a
\(p\in\mathbb{R}^4\) how does one identify \(\alpha, \beta\) if they
exist and is the strategy in fact acting in an extortionate way?

These conditions correspond to:

\begin{align}
    \tilde p_1 & = \alpha R + \beta R - P (\alpha + \beta)
            \label{eqn:condition_for_tilde_p1}\\
    \tilde p_2 & = \alpha S + \beta T - P (\alpha + \beta)
            \label{eqn:condition_for_tilde_p2}\\
    \tilde p_3 & = \alpha T + \beta S - P (\alpha + \beta)
            \label{eqn:condition_for_tilde_p3}\\
    \tilde p_4 & = \alpha P + \beta P - P (\alpha + \beta)
            \label{eqn:condition_for_tilde_p4}
\end{align}

Equation (\ref{eqn:condition_for_tilde_p4}) ensures that \(p_4=\tilde p_4=0\).
Equations (\ref{eqn:condition_for_tilde_p1}-\ref{eqn:condition_for_tilde_p3})
can be used to eliminate \(\alpha, \beta\), giving:

\begin{equation}\label{eqn:planar_definition_of_extortion}
    \tilde p_1 = \frac{(R - P)(\tilde p_2 + \tilde p_3)}{S + T - 2P}
\end{equation}

with:

\begin{equation}\label{eqn:definition_of_chi}
    \chi = \frac{\tilde p_2 (P - T) + \tilde p_3 (S - P)}
                {\tilde p_2 (P - S) + \tilde p_3 (T - P)}
\end{equation}

Given a strategy \(p\in\mathbb{R}^{4\times 1}\) equations
(\ref{eqn:condition_for_tilde_p4}), (\ref{eqn:planar_definition_of_extortion}-\ref{eqn:definition_of_chi}) can be used to check if
a strategy is extortionate. The conditions correspond to:

\begin{align}
    p_1 & = \frac{(R-P)(p_2 + p_3) - R + T + S - P}{S + T - 2P}
     \label{eqn:condition_for_p1}\\
    p_4 & = 0 \label{eqn:condition_for_p4}\\
    1 & > p_2 + p_3\label{eqn:condition_for_chi}
\end{align}

The algebraic steps necessary to prove these results are available in the
supporting materials.

All extortionate strategies reside on a triangular (\ref{eqn:condition_for_chi})
plane (\ref{eqn:condition_for_p1}) in 3 dimensions (\ref{eqn:condition_for_p4}).
Using this formulation it can be seen that a necessary (but not sufficient)
condition for an extortionate strategy is that it cooperates on average less
than 50\% of the time when in a state of disagreement with the opponent.

As an example, consider the known extortionate strategy \(p=(8 / 9, 1 / 2, 1 /
3, 0)\) from~\cite{Stewart2012} which is referred to as \texttt{Extort-2}. In
this case, for the standard values of \((R, T, S, P)\) constraint
(\ref{eqn:condition_for_p1}) corresponds to:

\begin{equation}
    p_1 = \frac{2(p_2 + p_3) + 1}{3}
\end{equation}

It is clear that in this case all constraints hold.

This approach could in fact be used to confirm that a given strategy is acting
in an extortionate manner even if it is not a memory one strategy. However, in
practice, if a closed form for \(p\) is not known, then due to measurement
and/or numerical error this would not work.

This problem can be written in the following linear algebraic form where
\(x=(\alpha, \beta)\)
and \(p^*=(\tilde p_1 - 1, tilde_2 - 1, p_3)\):

\begin{equation}\label{eqn:linear_algebraic_equation_for_p}
    Cx= p^*
\end{equation}

\(C\) corresponds to equations
(\ref{eqn:condition_for_tilde_p1}-\ref{eqn:condition_for_tilde_p3}) and is
given by:

\begin{equation}\label{eqn:definition_of_C}
    C =
    \begin{bmatrix}
        R - P & R- P \\
        S - P & T- P \\
        T - P & S- P \\
    \end{bmatrix}
\end{equation}

Note that in general, equation (\ref{eqn:linear_algebraic_equation_for_p}) will
not necessarily have a solution. From the Rouch\'{e}-Capelli theorem if there is
a solution it is unique as \(\text{rank}(C)=2\) which is the dimension of the
variable \(x\). The best fitting \(x\) is found by minimizing:

\begin{equation}\label{eqn:r_squared}
    \text{SSError} = \|C x- p^*\|_2^2 = \sum_{i=1}^{3}\left((C\bar x)_i-p_i^*\right)^2
\end{equation}

Note that \(\text{SSError}\), which is the square of the Frobenius
norm~\cite{Golub2013}, becomes a measure of how close a strategy is to being an
extortionate strategy. Suspicion
of extortion then corresponds to a threshold on \(\text{SSError}\).

By observing interactions (human or otherwise), their memory one representation
can be inferred and this approach can be used to recognise extortionate
behaviour. The notion of comparing theoretic and actual plays of the IPD is not
novel, see for example~\cite{Rand2013}. Immediately it is noted that if the
environment is noisy~\cite{Wu1995} then no strategy can be considered to be
extortionate as \(p_4>0\).

In the next section, this idea will be illustrated by observing the interactions
that take place in a computer based tournament of the IPD\@.

\section{Numerical experiments}\label{sec:numerical-experiments}

In~\cite{Stewart2012} results from a tournament with
\input{./assets/tex/number_of_stewart_plotkin_strategies/main.tex} strategies,
was presented with specific consideration given to ZD strategies. This
tournament is reproduced here using the Axelrod-Python
project~\cite{Knight2016}. To obtain a good measure of the corresponding
transition rates for each strategy all matches have been run for
\input{assets/tex/number_of_turns/main.tex} turns and every match has been
repeated \input{assets/tex/number_of_repetitions/main.tex} times. All of this
interaction data is available at~\cite{vincent_knight_2018_1297075}. A good
match between the inferred Markov chain and the state distribution of the actual
interactions has been verified. Data for this is presented in the supplementary
materials.

Figure~\ref{fig:SSError_overall_in_stewart_plotkin} shows the \(\text{SSError}\)
values for all the strategies in the tournament, as reported
in~\cite{Stewart2012} the extortionate strategy (which has an expected
\(\text{SSError}\) approximately 0) gains a large number of wins.

\begin{figure}[!htbp]
    \centering
    \includegraphics[width=.8\textwidth]{./assets/img/SSError_overall_in_stewart_plotkin/main.pdf}
    \caption{\(\text{SSError}\) and state probabilities for the strategies
        of~\cite{Stewart2012}, ordered both by number of wins and overall score.
        Note that \(P(DC)\) is not shown as it corresponds to the transpose of
        \(P(CD)\). Cooperator and Defector are omitted as they do not visit all
        the states.}
    \label{fig:SSError_overall_in_stewart_plotkin}
\end{figure}

Here, the work of~\cite{Stewart2012} is extended by investigating a tournament
with \input{assets/tex/number_of_full_strategies/main.tex}
strategies.

The results of this analysis are shown in
Figure~\ref{fig:SSError_and_probabilities_in_full}. The top ranking strategies
by number of wins seem to be extortionate (but not against all strategies) and
it can be seen that a small sub group of strategies achieve mutual defection.
All the top ranking strategies according to score achieve mutual cooperation and
do not extort each other, however they
\textbf{do} exhibit extortionate behaviour towards a number of the lower ranking
strategies.

\begin{figure}[!htbp]
    \centering
    \includegraphics[width=.8\textwidth]{./assets/img/SSError_and_probabilities_in_full/main.pdf}
    \caption{\(\text{SSError}\) for the strategies for the full tournament. Only
    strategy interactions for which \(p_4=0\) and \(\chi>1\) are displayed.}
    \label{fig:SSError_and_probabilities_in_full}
\end{figure}

\section{Conclusion}\label{sec:conclusion}

This work defines an approach to measure whether or not a player is playing a
strategy that corresponds to an extortionate strategy as defined
in~\cite{Press2012}: a mathematical model for suspicion. Indeed, all
extortionate strategies have been
 classified as lying on a triangular plane.
This rigorous classification fails to be robust to small measurement error, thus
a statistical approach is proposed.
This is done through a linear algebraic approach for approximating the solution
of a linear system. Using this, a large number of pairwise interactions is
simulated and in fact very few strategies are found to act extortionately.

The work of~\cite{Press2012}, whilst showing that a clever approach to taking
advantage of another memory one strategy exists: this is incomplete. Whilst the
elegance of this result is very attractive, just as the simplicity of the
victory of Tit For Tat in Axelrod's original tournaments was, it is incomplete.
Extortionate strategies achieve a high number of wins but they do not
achieve a high score which corresponds to the fitness landscape in an
evolutionary sense. From the large number of interactions a payoff matrix \(S\)
can be measured where \(S_{ij}\) denotes the score (using standard values of
\((R, S, T, P) = (3, 0, 5, 1)\)) of the \(i\)th strategy
against the \(j\)th strategy. Using this, the replicator equation
describes the evolution of the system based on a population density fitness
function:

\begin{equation}\label{eqn:replicator_dynamics}
    \frac{dx}{dt} = x(S-x^TS x)
\end{equation}

Equation (\ref{eqn:replicator_dynamics}) is solved numerically through an
integration technique described in~\cite{Petzold1983} and
Figure~\ref{fig:replicator_dynamics} shows the evolution of the distribution of
the system: the various strategies are ranked by scores. It is clear to see that
only the high ranking strategies survive the evolutionary process (in fact,
only \input{./assets/img/replicator_dynamics/main.tex}
have a final distribution greater than \(10 ^ {-2}\)). This confirms the
findings of~\cite{Moran1707} in which sophisticated strategies resist
evolutionary invasion of shorter memory strategies. Recalling
Figure~\ref{fig:SSError_and_probabilities_in_full} this demonstrates that:

\begin{itemize}
    \item Cooperation emerges through the evolutionary process: the high scoring
        strategies do not exhibit extortionate behaviour towards each other.
    \item Extortionate strategies do not survive the evolutionary process.
\end{itemize}

\begin{figure}[!htbp]
    \centering
    \includegraphics[width=.8\textwidth]{./assets/img/replicator_dynamics/main.pdf}
    \caption{Numerical simulation of the replicator equation
    (\ref{eqn:replicator_dynamics}): strategies are ordered by score, only the strategies with a high score survive the evolutionary process.}
    \label{fig:replicator_dynamics}
\end{figure}

This work can be used to classify plays of the IPD\@: data can be collected from
actual interactions (in lab or in the field). Furthermore, this allows for a
classification method similar to the notion of fingerprinting presented
in~\cite{Ashlock2008}. Trained strategies can potentially be classified as
extortionate or not or it could be possible to even constrain the reinforcement
learning approaches that are becoming prevalent in the literature.
Alternatively, this mathematical approach for recognising extortion could be
used in sophisticated strategies to defend against invasion. Arguably, some of
the strategies considered here exhibit this behaviour, indeed as described
in~\cite{Harper2017}, the top ranking strategies in the full tournament are
obtained using evolutionary reinforcement learning techniques, thus, suspicion
of extortionate behaviour could in fact be an evolutionary trait.

\section*{Acknowledgements}

The following open source software libraries were used in this research:

\begin{itemize}
    \item The Axelrod ~\cite{Knight2016, Knight2018} library (IPD strategies and
        tournaments).
    \item The sympy library~\cite{Meurer2017} (verification of all symbolic
        calculations).
    \item The matplotlib~\cite{Droettboom2018} library (visualisation).
    \item The pandas~\cite{Structures2010}, dask~\cite{Dask2016} and
        NumPy~\cite{Oliphant2015} libraries (data manipulation).
    \item The SciPy~\cite{Jones2001} library (numerical integration of the
        replicator equation).
\end{itemize}

This work was performed using the computational facilities of the Advanced
Research Computing @ Cardiff (ARCCA) Division, Cardiff University.

\printbibliography

\newpage
\section*{Supplementary materials}

\includepdf{assets/pdf/proof_of_form_of_extortionate_strategies/main.pdf}

\newpage

Using the pair wise interactions the transition rates \(p,
q\) can be measured and the steady state probabilities inferred and compared to
the actual probabilities of each state.
This is done numerically by computing the singular eigenvector of the
matrix \(A\) \cite{Stewart2009}:

\[
    A =
    \begin{bmatrix}
        p_1 q_1 & p_1 (1 - q_1) & (1 - p_1) q_1 & (1 -p_1) (1 - q_1) \\
        p_2 q_2 & p_2 (1 - q_2) & (1 - p_2) q_2 & (1 -p_2) (1 - q_2) \\
        p_3 q_3 & p_3 (1 - q_3) & (1 - p_3) q_3 & (1 -p_3) (1 - q_3) \\
        p_4 q_4 & p_4 (1 - q_4) & (1 - p_4) q_4 & (1 -p_4) (1 - q_4) \\
    \end{bmatrix}
\]

Figure~\ref{fig:computed_probabilities_vs_theoretic_probabilities} shows a
regression line fitted to every pairwise interaction with a reported
\(\text{SSError}\) value (pairwise interactions with missing states were
omitted). This serves to validate the approach: a part from some edge cases the
relationship is consistent.

\begin{figure}[!htbp]
    \centering
    \includegraphics[width=.8\textwidth]{./assets/img/computed_probabilities_vs_theoretic_probabilities/main.pdf}
    \caption{The
        relationship between the steady state probabilities inferred from the
        measured transitions and the actual steady state probabilities. A linear
        regression line is included validating the approach.}
    \label{fig:computed_probabilities_vs_theoretic_probabilities}
\end{figure}


\end{document}
 times. All of this
interaction data is available at~\cite{vincent_knight_2018_1297075}. A good
match between the inferred Markov chain and the state distribution of the actual
interactions has been verified. Data for this is presented in the supplementary
materials.

Figure~\ref{fig:SSError_overall_in_stewart_plotkin} shows the \(\text{SSError}\)
values for all the strategies in the tournament, as reported
in~\cite{Stewart2012} the extortionate strategy (which has an expected
\(\text{SSError}\) approximately 0) gains a large number of wins.

\begin{figure}[!htbp]
    \centering
    \includegraphics[width=.8\textwidth]{./assets/img/SSError_overall_in_stewart_plotkin/main.pdf}
    \caption{\(\text{SSError}\) and state probabilities for the strategies
        of~\cite{Stewart2012}, ordered both by number of wins and overall score.
        Note that \(P(DC)\) is not shown as it corresponds to the transpose of
        \(P(CD)\). Cooperator and Defector are omitted as they do not visit all
        the states.}
    \label{fig:SSError_overall_in_stewart_plotkin}
\end{figure}

Here, the work of~\cite{Stewart2012} is extended by investigating a tournament
with \documentclass[a4paper]{article}

\usepackage{amsmath}
\usepackage{amssymb}
\usepackage[margin=1.5cm,
            includefoot,
            footskip=30pt]{geometry}
\usepackage{layout}
\usepackage{graphicx}
\usepackage{subcaption}

\usepackage{biblatex}
\usepackage{pdfpages}

\bibliography{main.bib}

\title{Suspicion: Recognising and evaluating the effectiveness
       of extortion in the Iterated Prisoner's Dilemma}
\author{Vincent A. Knight \and Nikoleta E. Glynatsi}
\date{\today}



\begin{document}

\maketitle

\begin{abstract}
    The Iterated Prisoner's Dilemma is a model for rational and evolutionary
    interactive behaviour. It has applications both in the study of human social
    behaviour as well as in biology.
    It is used to understand when and how a rational individual might
    accept an immediate cost to their own utility for the direct benefit of
    another.

    Much attention has been given to a class of strategies called
    Zero Determinant strategies. It has been theoretically shown that these
    strategies can ``extort'' any player.

    In this work, an approach to identify if observed strategies are playing in
    an extortionate way is described. Furthermore, experimental analysis of
    a large tournament with \input{assets/tex/number_of_full_strategies/main.tex}
    strategies is considered. In this setting
    the most highly performing strategies do not play in an extortionate way
    against each other but do against lower performing strategies.
    This suggests that whilst the theory of Zero Determinant strategies
    indicates that memory is not of fundamental importance to the evolution of
    cooperative behaviour, this is incomplete.
\end{abstract}

\section{Introduction}\label{sec:introduction}

Agent based game theoretic models have become a stalwart of the underpinning
mathematics of interactive behaviours. One of the major pieces of work
in this area is the pair of original computer tournaments run by Robert
Axelrod~\cite{Axelrod1980, Axelrod1980a}. These tournaments pitted submitted
computer strategies against each other in plays of the Iterated Prisoner's
Dilemma. A common game where agents can choose to pay a slight cost to their
immediate utility in the hope of building a reputation. This has been used in
economic and evolutionary game theory to understand the evolution of cooperative
behaviour.

Recently, a class of strategies was described in~\cite{Press2012} that can
provably extort any given opponent. In~\cite{Hilbe2013, Moran1707} some
questions have already been asked about the true effectiveness of these
strategies in an evolutionary setting. Here another question is asked: is it
possible to recognise this extortionate behaviour? A mathematical procedure for
suspicion is presented: in the same way that the continued actions of an
extortionate individual might raise suspicion.

This work makes use of the Axelrod Python library~\cite{Knight2018, Knight2016}
with a large number of Prisoner Dilemma strategies available to give an
extensive numerical example of the ideas presented.  The approach is presented
in Section~\ref{sec:delta-zd-strategies}.  All of the code and data discussed
in Section~\ref{sec:numerical-experiments} is open sourced, archived and
written according to best scientific principles~\cite{Wilson2014}. The data
archive can be found at~\cite{vincent_knight_2018_1297075}.

\section{Recognising Extortion}\label{sec:delta-zd-strategies}

In~\cite{Press2012}, given a match between 2 memory-one strategies, the concept
of Zero Determinant (ZD) strategies is introduced. The main result of that paper
shows that given two memory one players \(p, q\in\mathbb{R}^4\) a linear
relationship between the players' scores could be forced by one of the players.

Using the notation of~\cite{Press2012}, assuming the utilities for player \(p\)
are given by \(S_x=(R, S, T, P)\) and for player \(q\) by \(S_y=(R, T, S, P)\)
and that the stationary scores of each player is given by \(S_X\) and \(S_Y\)
respectively. The main result of~\cite{Press2012} is that if

\begin{equation}\label{eqn:linear_relationship_for_p}
    \tilde p=\alpha S_x + \beta S_y + \gamma
\end{equation}

or

\begin{equation}\label{eqn:linear_relationship_for_q}
    \tilde q=\alpha S_x + \beta S_y + \gamma
\end{equation}

where \(\tilde p = (1 - p_1, 1 - p_2, p_3, p_4)\) and
\(\tilde q = (1 - q_1, 1 - q_2, q_3, q_4)\) then:

\begin{equation}
    \alpha S_X + \beta S_Y + \gamma = 0
\end{equation}

In~\cite{Press2012} a particular type of ZD strategy is defined: extortionate
strategies. If:

\begin{equation}\label{eqn:constraint_for_extortion}
    \gamma = - P(\alpha + \beta)
\end{equation}

then the player can ensure they get a score \(\chi\) times
larger than the opponent. This extortion coefficient is given by:

\begin{equation}\label{eqn:definition_of_chi}
    \chi=\frac{-\beta}{\alpha}
\end{equation}

Thus, if (\ref{eqn:constraint_for_extortion}) holds and \(\chi >1\) a player is
said to extort their opponent.
Here, the reverse problem is considered: given a
\(p\in\mathbb{R}^4\) how does one identify \(\alpha, \beta\) if they
exist and is the strategy in fact acting in an extortionate way?

These conditions correspond to:

\begin{align}
    \tilde p_1 & = \alpha R + \beta R - P (\alpha + \beta)
            \label{eqn:condition_for_tilde_p1}\\
    \tilde p_2 & = \alpha S + \beta T - P (\alpha + \beta)
            \label{eqn:condition_for_tilde_p2}\\
    \tilde p_3 & = \alpha T + \beta S - P (\alpha + \beta)
            \label{eqn:condition_for_tilde_p3}\\
    \tilde p_4 & = \alpha P + \beta P - P (\alpha + \beta)
            \label{eqn:condition_for_tilde_p4}
\end{align}

Equation (\ref{eqn:condition_for_tilde_p4}) ensures that \(p_4=\tilde p_4=0\).
Equations (\ref{eqn:condition_for_tilde_p1}-\ref{eqn:condition_for_tilde_p3})
can be used to eliminate \(\alpha, \beta\), giving:

\begin{equation}\label{eqn:planar_definition_of_extortion}
    \tilde p_1 = \frac{(R - P)(\tilde p_2 + \tilde p_3)}{S + T - 2P}
\end{equation}

with:

\begin{equation}\label{eqn:definition_of_chi}
    \chi = \frac{\tilde p_2 (P - T) + \tilde p_3 (S - P)}
                {\tilde p_2 (P - S) + \tilde p_3 (T - P)}
\end{equation}

Given a strategy \(p\in\mathbb{R}^{4\times 1}\) equations
(\ref{eqn:condition_for_tilde_p4}), (\ref{eqn:planar_definition_of_extortion}-\ref{eqn:definition_of_chi}) can be used to check if
a strategy is extortionate. The conditions correspond to:

\begin{align}
    p_1 & = \frac{(R-P)(p_2 + p_3) - R + T + S - P}{S + T - 2P}
     \label{eqn:condition_for_p1}\\
    p_4 & = 0 \label{eqn:condition_for_p4}\\
    1 & > p_2 + p_3\label{eqn:condition_for_chi}
\end{align}

The algebraic steps necessary to prove these results are available in the
supporting materials.

All extortionate strategies reside on a triangular (\ref{eqn:condition_for_chi})
plane (\ref{eqn:condition_for_p1}) in 3 dimensions (\ref{eqn:condition_for_p4}).
Using this formulation it can be seen that a necessary (but not sufficient)
condition for an extortionate strategy is that it cooperates on average less
than 50\% of the time when in a state of disagreement with the opponent.

As an example, consider the known extortionate strategy \(p=(8 / 9, 1 / 2, 1 /
3, 0)\) from~\cite{Stewart2012} which is referred to as \texttt{Extort-2}. In
this case, for the standard values of \((R, T, S, P)\) constraint
(\ref{eqn:condition_for_p1}) corresponds to:

\begin{equation}
    p_1 = \frac{2(p_2 + p_3) + 1}{3}
\end{equation}

It is clear that in this case all constraints hold.

This approach could in fact be used to confirm that a given strategy is acting
in an extortionate manner even if it is not a memory one strategy. However, in
practice, if a closed form for \(p\) is not known, then due to measurement
and/or numerical error this would not work.

This problem can be written in the following linear algebraic form where
\(x=(\alpha, \beta)\)
and \(p^*=(\tilde p_1 - 1, tilde_2 - 1, p_3)\):

\begin{equation}\label{eqn:linear_algebraic_equation_for_p}
    Cx= p^*
\end{equation}

\(C\) corresponds to equations
(\ref{eqn:condition_for_tilde_p1}-\ref{eqn:condition_for_tilde_p3}) and is
given by:

\begin{equation}\label{eqn:definition_of_C}
    C =
    \begin{bmatrix}
        R - P & R- P \\
        S - P & T- P \\
        T - P & S- P \\
    \end{bmatrix}
\end{equation}

Note that in general, equation (\ref{eqn:linear_algebraic_equation_for_p}) will
not necessarily have a solution. From the Rouch\'{e}-Capelli theorem if there is
a solution it is unique as \(\text{rank}(C)=2\) which is the dimension of the
variable \(x\). The best fitting \(x\) is found by minimizing:

\begin{equation}\label{eqn:r_squared}
    \text{SSError} = \|C x- p^*\|_2^2 = \sum_{i=1}^{3}\left((C\bar x)_i-p_i^*\right)^2
\end{equation}

Note that \(\text{SSError}\), which is the square of the Frobenius
norm~\cite{Golub2013}, becomes a measure of how close a strategy is to being an
extortionate strategy. Suspicion
of extortion then corresponds to a threshold on \(\text{SSError}\).

By observing interactions (human or otherwise), their memory one representation
can be inferred and this approach can be used to recognise extortionate
behaviour. The notion of comparing theoretic and actual plays of the IPD is not
novel, see for example~\cite{Rand2013}. Immediately it is noted that if the
environment is noisy~\cite{Wu1995} then no strategy can be considered to be
extortionate as \(p_4>0\).

In the next section, this idea will be illustrated by observing the interactions
that take place in a computer based tournament of the IPD\@.

\section{Numerical experiments}\label{sec:numerical-experiments}

In~\cite{Stewart2012} results from a tournament with
\input{./assets/tex/number_of_stewart_plotkin_strategies/main.tex} strategies,
was presented with specific consideration given to ZD strategies. This
tournament is reproduced here using the Axelrod-Python
project~\cite{Knight2016}. To obtain a good measure of the corresponding
transition rates for each strategy all matches have been run for
\input{assets/tex/number_of_turns/main.tex} turns and every match has been
repeated \input{assets/tex/number_of_repetitions/main.tex} times. All of this
interaction data is available at~\cite{vincent_knight_2018_1297075}. A good
match between the inferred Markov chain and the state distribution of the actual
interactions has been verified. Data for this is presented in the supplementary
materials.

Figure~\ref{fig:SSError_overall_in_stewart_plotkin} shows the \(\text{SSError}\)
values for all the strategies in the tournament, as reported
in~\cite{Stewart2012} the extortionate strategy (which has an expected
\(\text{SSError}\) approximately 0) gains a large number of wins.

\begin{figure}[!htbp]
    \centering
    \includegraphics[width=.8\textwidth]{./assets/img/SSError_overall_in_stewart_plotkin/main.pdf}
    \caption{\(\text{SSError}\) and state probabilities for the strategies
        of~\cite{Stewart2012}, ordered both by number of wins and overall score.
        Note that \(P(DC)\) is not shown as it corresponds to the transpose of
        \(P(CD)\). Cooperator and Defector are omitted as they do not visit all
        the states.}
    \label{fig:SSError_overall_in_stewart_plotkin}
\end{figure}

Here, the work of~\cite{Stewart2012} is extended by investigating a tournament
with \input{assets/tex/number_of_full_strategies/main.tex}
strategies.

The results of this analysis are shown in
Figure~\ref{fig:SSError_and_probabilities_in_full}. The top ranking strategies
by number of wins seem to be extortionate (but not against all strategies) and
it can be seen that a small sub group of strategies achieve mutual defection.
All the top ranking strategies according to score achieve mutual cooperation and
do not extort each other, however they
\textbf{do} exhibit extortionate behaviour towards a number of the lower ranking
strategies.

\begin{figure}[!htbp]
    \centering
    \includegraphics[width=.8\textwidth]{./assets/img/SSError_and_probabilities_in_full/main.pdf}
    \caption{\(\text{SSError}\) for the strategies for the full tournament. Only
    strategy interactions for which \(p_4=0\) and \(\chi>1\) are displayed.}
    \label{fig:SSError_and_probabilities_in_full}
\end{figure}

\section{Conclusion}\label{sec:conclusion}

This work defines an approach to measure whether or not a player is playing a
strategy that corresponds to an extortionate strategy as defined
in~\cite{Press2012}: a mathematical model for suspicion. Indeed, all
extortionate strategies have been
 classified as lying on a triangular plane.
This rigorous classification fails to be robust to small measurement error, thus
a statistical approach is proposed.
This is done through a linear algebraic approach for approximating the solution
of a linear system. Using this, a large number of pairwise interactions is
simulated and in fact very few strategies are found to act extortionately.

The work of~\cite{Press2012}, whilst showing that a clever approach to taking
advantage of another memory one strategy exists: this is incomplete. Whilst the
elegance of this result is very attractive, just as the simplicity of the
victory of Tit For Tat in Axelrod's original tournaments was, it is incomplete.
Extortionate strategies achieve a high number of wins but they do not
achieve a high score which corresponds to the fitness landscape in an
evolutionary sense. From the large number of interactions a payoff matrix \(S\)
can be measured where \(S_{ij}\) denotes the score (using standard values of
\((R, S, T, P) = (3, 0, 5, 1)\)) of the \(i\)th strategy
against the \(j\)th strategy. Using this, the replicator equation
describes the evolution of the system based on a population density fitness
function:

\begin{equation}\label{eqn:replicator_dynamics}
    \frac{dx}{dt} = x(S-x^TS x)
\end{equation}

Equation (\ref{eqn:replicator_dynamics}) is solved numerically through an
integration technique described in~\cite{Petzold1983} and
Figure~\ref{fig:replicator_dynamics} shows the evolution of the distribution of
the system: the various strategies are ranked by scores. It is clear to see that
only the high ranking strategies survive the evolutionary process (in fact,
only \input{./assets/img/replicator_dynamics/main.tex}
have a final distribution greater than \(10 ^ {-2}\)). This confirms the
findings of~\cite{Moran1707} in which sophisticated strategies resist
evolutionary invasion of shorter memory strategies. Recalling
Figure~\ref{fig:SSError_and_probabilities_in_full} this demonstrates that:

\begin{itemize}
    \item Cooperation emerges through the evolutionary process: the high scoring
        strategies do not exhibit extortionate behaviour towards each other.
    \item Extortionate strategies do not survive the evolutionary process.
\end{itemize}

\begin{figure}[!htbp]
    \centering
    \includegraphics[width=.8\textwidth]{./assets/img/replicator_dynamics/main.pdf}
    \caption{Numerical simulation of the replicator equation
    (\ref{eqn:replicator_dynamics}): strategies are ordered by score, only the strategies with a high score survive the evolutionary process.}
    \label{fig:replicator_dynamics}
\end{figure}

This work can be used to classify plays of the IPD\@: data can be collected from
actual interactions (in lab or in the field). Furthermore, this allows for a
classification method similar to the notion of fingerprinting presented
in~\cite{Ashlock2008}. Trained strategies can potentially be classified as
extortionate or not or it could be possible to even constrain the reinforcement
learning approaches that are becoming prevalent in the literature.
Alternatively, this mathematical approach for recognising extortion could be
used in sophisticated strategies to defend against invasion. Arguably, some of
the strategies considered here exhibit this behaviour, indeed as described
in~\cite{Harper2017}, the top ranking strategies in the full tournament are
obtained using evolutionary reinforcement learning techniques, thus, suspicion
of extortionate behaviour could in fact be an evolutionary trait.

\section*{Acknowledgements}

The following open source software libraries were used in this research:

\begin{itemize}
    \item The Axelrod ~\cite{Knight2016, Knight2018} library (IPD strategies and
        tournaments).
    \item The sympy library~\cite{Meurer2017} (verification of all symbolic
        calculations).
    \item The matplotlib~\cite{Droettboom2018} library (visualisation).
    \item The pandas~\cite{Structures2010}, dask~\cite{Dask2016} and
        NumPy~\cite{Oliphant2015} libraries (data manipulation).
    \item The SciPy~\cite{Jones2001} library (numerical integration of the
        replicator equation).
\end{itemize}

This work was performed using the computational facilities of the Advanced
Research Computing @ Cardiff (ARCCA) Division, Cardiff University.

\printbibliography

\newpage
\section*{Supplementary materials}

\includepdf{assets/pdf/proof_of_form_of_extortionate_strategies/main.pdf}

\newpage

Using the pair wise interactions the transition rates \(p,
q\) can be measured and the steady state probabilities inferred and compared to
the actual probabilities of each state.
This is done numerically by computing the singular eigenvector of the
matrix \(A\) \cite{Stewart2009}:

\[
    A =
    \begin{bmatrix}
        p_1 q_1 & p_1 (1 - q_1) & (1 - p_1) q_1 & (1 -p_1) (1 - q_1) \\
        p_2 q_2 & p_2 (1 - q_2) & (1 - p_2) q_2 & (1 -p_2) (1 - q_2) \\
        p_3 q_3 & p_3 (1 - q_3) & (1 - p_3) q_3 & (1 -p_3) (1 - q_3) \\
        p_4 q_4 & p_4 (1 - q_4) & (1 - p_4) q_4 & (1 -p_4) (1 - q_4) \\
    \end{bmatrix}
\]

Figure~\ref{fig:computed_probabilities_vs_theoretic_probabilities} shows a
regression line fitted to every pairwise interaction with a reported
\(\text{SSError}\) value (pairwise interactions with missing states were
omitted). This serves to validate the approach: a part from some edge cases the
relationship is consistent.

\begin{figure}[!htbp]
    \centering
    \includegraphics[width=.8\textwidth]{./assets/img/computed_probabilities_vs_theoretic_probabilities/main.pdf}
    \caption{The
        relationship between the steady state probabilities inferred from the
        measured transitions and the actual steady state probabilities. A linear
        regression line is included validating the approach.}
    \label{fig:computed_probabilities_vs_theoretic_probabilities}
\end{figure}


\end{document}

strategies.

The results of this analysis are shown in
Figure~\ref{fig:SSError_and_probabilities_in_full}. The top ranking strategies
by number of wins seem to be extortionate (but not against all strategies) and
it can be seen that a small sub group of strategies achieve mutual defection.
All the top ranking strategies according to score achieve mutual cooperation and
do not extort each other, however they
\textbf{do} exhibit extortionate behaviour towards a number of the lower ranking
strategies.

\begin{figure}[!htbp]
    \centering
    \includegraphics[width=.8\textwidth]{./assets/img/SSError_and_probabilities_in_full/main.pdf}
    \caption{\(\text{SSError}\) for the strategies for the full tournament. Only
    strategy interactions for which \(p_4=0\) and \(\chi>1\) are displayed.}
    \label{fig:SSError_and_probabilities_in_full}
\end{figure}

\section{Conclusion}\label{sec:conclusion}

This work defines an approach to measure whether or not a player is playing a
strategy that corresponds to an extortionate strategy as defined
in~\cite{Press2012}: a mathematical model for suspicion. Indeed, all
extortionate strategies have been
 classified as lying on a triangular plane.
This rigorous classification fails to be robust to small measurement error, thus
a statistical approach is proposed.
This is done through a linear algebraic approach for approximating the solution
of a linear system. Using this, a large number of pairwise interactions is
simulated and in fact very few strategies are found to act extortionately.

The work of~\cite{Press2012}, whilst showing that a clever approach to taking
advantage of another memory one strategy exists: this is incomplete. Whilst the
elegance of this result is very attractive, just as the simplicity of the
victory of Tit For Tat in Axelrod's original tournaments was, it is incomplete.
Extortionate strategies achieve a high number of wins but they do not
achieve a high score which corresponds to the fitness landscape in an
evolutionary sense. From the large number of interactions a payoff matrix \(S\)
can be measured where \(S_{ij}\) denotes the score (using standard values of
\((R, S, T, P) = (3, 0, 5, 1)\)) of the \(i\)th strategy
against the \(j\)th strategy. Using this, the replicator equation
describes the evolution of the system based on a population density fitness
function:

\begin{equation}\label{eqn:replicator_dynamics}
    \frac{dx}{dt} = x(S-x^TS x)
\end{equation}

Equation (\ref{eqn:replicator_dynamics}) is solved numerically through an
integration technique described in~\cite{Petzold1983} and
Figure~\ref{fig:replicator_dynamics} shows the evolution of the distribution of
the system: the various strategies are ranked by scores. It is clear to see that
only the high ranking strategies survive the evolutionary process (in fact,
only \documentclass[a4paper]{article}

\usepackage{amsmath}
\usepackage{amssymb}
\usepackage[margin=1.5cm,
            includefoot,
            footskip=30pt]{geometry}
\usepackage{layout}
\usepackage{graphicx}
\usepackage{subcaption}

\usepackage{biblatex}
\usepackage{pdfpages}

\bibliography{main.bib}

\title{Suspicion: Recognising and evaluating the effectiveness
       of extortion in the Iterated Prisoner's Dilemma}
\author{Vincent A. Knight \and Nikoleta E. Glynatsi}
\date{\today}



\begin{document}

\maketitle

\begin{abstract}
    The Iterated Prisoner's Dilemma is a model for rational and evolutionary
    interactive behaviour. It has applications both in the study of human social
    behaviour as well as in biology.
    It is used to understand when and how a rational individual might
    accept an immediate cost to their own utility for the direct benefit of
    another.

    Much attention has been given to a class of strategies called
    Zero Determinant strategies. It has been theoretically shown that these
    strategies can ``extort'' any player.

    In this work, an approach to identify if observed strategies are playing in
    an extortionate way is described. Furthermore, experimental analysis of
    a large tournament with \input{assets/tex/number_of_full_strategies/main.tex}
    strategies is considered. In this setting
    the most highly performing strategies do not play in an extortionate way
    against each other but do against lower performing strategies.
    This suggests that whilst the theory of Zero Determinant strategies
    indicates that memory is not of fundamental importance to the evolution of
    cooperative behaviour, this is incomplete.
\end{abstract}

\section{Introduction}\label{sec:introduction}

Agent based game theoretic models have become a stalwart of the underpinning
mathematics of interactive behaviours. One of the major pieces of work
in this area is the pair of original computer tournaments run by Robert
Axelrod~\cite{Axelrod1980, Axelrod1980a}. These tournaments pitted submitted
computer strategies against each other in plays of the Iterated Prisoner's
Dilemma. A common game where agents can choose to pay a slight cost to their
immediate utility in the hope of building a reputation. This has been used in
economic and evolutionary game theory to understand the evolution of cooperative
behaviour.

Recently, a class of strategies was described in~\cite{Press2012} that can
provably extort any given opponent. In~\cite{Hilbe2013, Moran1707} some
questions have already been asked about the true effectiveness of these
strategies in an evolutionary setting. Here another question is asked: is it
possible to recognise this extortionate behaviour? A mathematical procedure for
suspicion is presented: in the same way that the continued actions of an
extortionate individual might raise suspicion.

This work makes use of the Axelrod Python library~\cite{Knight2018, Knight2016}
with a large number of Prisoner Dilemma strategies available to give an
extensive numerical example of the ideas presented.  The approach is presented
in Section~\ref{sec:delta-zd-strategies}.  All of the code and data discussed
in Section~\ref{sec:numerical-experiments} is open sourced, archived and
written according to best scientific principles~\cite{Wilson2014}. The data
archive can be found at~\cite{vincent_knight_2018_1297075}.

\section{Recognising Extortion}\label{sec:delta-zd-strategies}

In~\cite{Press2012}, given a match between 2 memory-one strategies, the concept
of Zero Determinant (ZD) strategies is introduced. The main result of that paper
shows that given two memory one players \(p, q\in\mathbb{R}^4\) a linear
relationship between the players' scores could be forced by one of the players.

Using the notation of~\cite{Press2012}, assuming the utilities for player \(p\)
are given by \(S_x=(R, S, T, P)\) and for player \(q\) by \(S_y=(R, T, S, P)\)
and that the stationary scores of each player is given by \(S_X\) and \(S_Y\)
respectively. The main result of~\cite{Press2012} is that if

\begin{equation}\label{eqn:linear_relationship_for_p}
    \tilde p=\alpha S_x + \beta S_y + \gamma
\end{equation}

or

\begin{equation}\label{eqn:linear_relationship_for_q}
    \tilde q=\alpha S_x + \beta S_y + \gamma
\end{equation}

where \(\tilde p = (1 - p_1, 1 - p_2, p_3, p_4)\) and
\(\tilde q = (1 - q_1, 1 - q_2, q_3, q_4)\) then:

\begin{equation}
    \alpha S_X + \beta S_Y + \gamma = 0
\end{equation}

In~\cite{Press2012} a particular type of ZD strategy is defined: extortionate
strategies. If:

\begin{equation}\label{eqn:constraint_for_extortion}
    \gamma = - P(\alpha + \beta)
\end{equation}

then the player can ensure they get a score \(\chi\) times
larger than the opponent. This extortion coefficient is given by:

\begin{equation}\label{eqn:definition_of_chi}
    \chi=\frac{-\beta}{\alpha}
\end{equation}

Thus, if (\ref{eqn:constraint_for_extortion}) holds and \(\chi >1\) a player is
said to extort their opponent.
Here, the reverse problem is considered: given a
\(p\in\mathbb{R}^4\) how does one identify \(\alpha, \beta\) if they
exist and is the strategy in fact acting in an extortionate way?

These conditions correspond to:

\begin{align}
    \tilde p_1 & = \alpha R + \beta R - P (\alpha + \beta)
            \label{eqn:condition_for_tilde_p1}\\
    \tilde p_2 & = \alpha S + \beta T - P (\alpha + \beta)
            \label{eqn:condition_for_tilde_p2}\\
    \tilde p_3 & = \alpha T + \beta S - P (\alpha + \beta)
            \label{eqn:condition_for_tilde_p3}\\
    \tilde p_4 & = \alpha P + \beta P - P (\alpha + \beta)
            \label{eqn:condition_for_tilde_p4}
\end{align}

Equation (\ref{eqn:condition_for_tilde_p4}) ensures that \(p_4=\tilde p_4=0\).
Equations (\ref{eqn:condition_for_tilde_p1}-\ref{eqn:condition_for_tilde_p3})
can be used to eliminate \(\alpha, \beta\), giving:

\begin{equation}\label{eqn:planar_definition_of_extortion}
    \tilde p_1 = \frac{(R - P)(\tilde p_2 + \tilde p_3)}{S + T - 2P}
\end{equation}

with:

\begin{equation}\label{eqn:definition_of_chi}
    \chi = \frac{\tilde p_2 (P - T) + \tilde p_3 (S - P)}
                {\tilde p_2 (P - S) + \tilde p_3 (T - P)}
\end{equation}

Given a strategy \(p\in\mathbb{R}^{4\times 1}\) equations
(\ref{eqn:condition_for_tilde_p4}), (\ref{eqn:planar_definition_of_extortion}-\ref{eqn:definition_of_chi}) can be used to check if
a strategy is extortionate. The conditions correspond to:

\begin{align}
    p_1 & = \frac{(R-P)(p_2 + p_3) - R + T + S - P}{S + T - 2P}
     \label{eqn:condition_for_p1}\\
    p_4 & = 0 \label{eqn:condition_for_p4}\\
    1 & > p_2 + p_3\label{eqn:condition_for_chi}
\end{align}

The algebraic steps necessary to prove these results are available in the
supporting materials.

All extortionate strategies reside on a triangular (\ref{eqn:condition_for_chi})
plane (\ref{eqn:condition_for_p1}) in 3 dimensions (\ref{eqn:condition_for_p4}).
Using this formulation it can be seen that a necessary (but not sufficient)
condition for an extortionate strategy is that it cooperates on average less
than 50\% of the time when in a state of disagreement with the opponent.

As an example, consider the known extortionate strategy \(p=(8 / 9, 1 / 2, 1 /
3, 0)\) from~\cite{Stewart2012} which is referred to as \texttt{Extort-2}. In
this case, for the standard values of \((R, T, S, P)\) constraint
(\ref{eqn:condition_for_p1}) corresponds to:

\begin{equation}
    p_1 = \frac{2(p_2 + p_3) + 1}{3}
\end{equation}

It is clear that in this case all constraints hold.

This approach could in fact be used to confirm that a given strategy is acting
in an extortionate manner even if it is not a memory one strategy. However, in
practice, if a closed form for \(p\) is not known, then due to measurement
and/or numerical error this would not work.

This problem can be written in the following linear algebraic form where
\(x=(\alpha, \beta)\)
and \(p^*=(\tilde p_1 - 1, tilde_2 - 1, p_3)\):

\begin{equation}\label{eqn:linear_algebraic_equation_for_p}
    Cx= p^*
\end{equation}

\(C\) corresponds to equations
(\ref{eqn:condition_for_tilde_p1}-\ref{eqn:condition_for_tilde_p3}) and is
given by:

\begin{equation}\label{eqn:definition_of_C}
    C =
    \begin{bmatrix}
        R - P & R- P \\
        S - P & T- P \\
        T - P & S- P \\
    \end{bmatrix}
\end{equation}

Note that in general, equation (\ref{eqn:linear_algebraic_equation_for_p}) will
not necessarily have a solution. From the Rouch\'{e}-Capelli theorem if there is
a solution it is unique as \(\text{rank}(C)=2\) which is the dimension of the
variable \(x\). The best fitting \(x\) is found by minimizing:

\begin{equation}\label{eqn:r_squared}
    \text{SSError} = \|C x- p^*\|_2^2 = \sum_{i=1}^{3}\left((C\bar x)_i-p_i^*\right)^2
\end{equation}

Note that \(\text{SSError}\), which is the square of the Frobenius
norm~\cite{Golub2013}, becomes a measure of how close a strategy is to being an
extortionate strategy. Suspicion
of extortion then corresponds to a threshold on \(\text{SSError}\).

By observing interactions (human or otherwise), their memory one representation
can be inferred and this approach can be used to recognise extortionate
behaviour. The notion of comparing theoretic and actual plays of the IPD is not
novel, see for example~\cite{Rand2013}. Immediately it is noted that if the
environment is noisy~\cite{Wu1995} then no strategy can be considered to be
extortionate as \(p_4>0\).

In the next section, this idea will be illustrated by observing the interactions
that take place in a computer based tournament of the IPD\@.

\section{Numerical experiments}\label{sec:numerical-experiments}

In~\cite{Stewart2012} results from a tournament with
\input{./assets/tex/number_of_stewart_plotkin_strategies/main.tex} strategies,
was presented with specific consideration given to ZD strategies. This
tournament is reproduced here using the Axelrod-Python
project~\cite{Knight2016}. To obtain a good measure of the corresponding
transition rates for each strategy all matches have been run for
\input{assets/tex/number_of_turns/main.tex} turns and every match has been
repeated \input{assets/tex/number_of_repetitions/main.tex} times. All of this
interaction data is available at~\cite{vincent_knight_2018_1297075}. A good
match between the inferred Markov chain and the state distribution of the actual
interactions has been verified. Data for this is presented in the supplementary
materials.

Figure~\ref{fig:SSError_overall_in_stewart_plotkin} shows the \(\text{SSError}\)
values for all the strategies in the tournament, as reported
in~\cite{Stewart2012} the extortionate strategy (which has an expected
\(\text{SSError}\) approximately 0) gains a large number of wins.

\begin{figure}[!htbp]
    \centering
    \includegraphics[width=.8\textwidth]{./assets/img/SSError_overall_in_stewart_plotkin/main.pdf}
    \caption{\(\text{SSError}\) and state probabilities for the strategies
        of~\cite{Stewart2012}, ordered both by number of wins and overall score.
        Note that \(P(DC)\) is not shown as it corresponds to the transpose of
        \(P(CD)\). Cooperator and Defector are omitted as they do not visit all
        the states.}
    \label{fig:SSError_overall_in_stewart_plotkin}
\end{figure}

Here, the work of~\cite{Stewart2012} is extended by investigating a tournament
with \input{assets/tex/number_of_full_strategies/main.tex}
strategies.

The results of this analysis are shown in
Figure~\ref{fig:SSError_and_probabilities_in_full}. The top ranking strategies
by number of wins seem to be extortionate (but not against all strategies) and
it can be seen that a small sub group of strategies achieve mutual defection.
All the top ranking strategies according to score achieve mutual cooperation and
do not extort each other, however they
\textbf{do} exhibit extortionate behaviour towards a number of the lower ranking
strategies.

\begin{figure}[!htbp]
    \centering
    \includegraphics[width=.8\textwidth]{./assets/img/SSError_and_probabilities_in_full/main.pdf}
    \caption{\(\text{SSError}\) for the strategies for the full tournament. Only
    strategy interactions for which \(p_4=0\) and \(\chi>1\) are displayed.}
    \label{fig:SSError_and_probabilities_in_full}
\end{figure}

\section{Conclusion}\label{sec:conclusion}

This work defines an approach to measure whether or not a player is playing a
strategy that corresponds to an extortionate strategy as defined
in~\cite{Press2012}: a mathematical model for suspicion. Indeed, all
extortionate strategies have been
 classified as lying on a triangular plane.
This rigorous classification fails to be robust to small measurement error, thus
a statistical approach is proposed.
This is done through a linear algebraic approach for approximating the solution
of a linear system. Using this, a large number of pairwise interactions is
simulated and in fact very few strategies are found to act extortionately.

The work of~\cite{Press2012}, whilst showing that a clever approach to taking
advantage of another memory one strategy exists: this is incomplete. Whilst the
elegance of this result is very attractive, just as the simplicity of the
victory of Tit For Tat in Axelrod's original tournaments was, it is incomplete.
Extortionate strategies achieve a high number of wins but they do not
achieve a high score which corresponds to the fitness landscape in an
evolutionary sense. From the large number of interactions a payoff matrix \(S\)
can be measured where \(S_{ij}\) denotes the score (using standard values of
\((R, S, T, P) = (3, 0, 5, 1)\)) of the \(i\)th strategy
against the \(j\)th strategy. Using this, the replicator equation
describes the evolution of the system based on a population density fitness
function:

\begin{equation}\label{eqn:replicator_dynamics}
    \frac{dx}{dt} = x(S-x^TS x)
\end{equation}

Equation (\ref{eqn:replicator_dynamics}) is solved numerically through an
integration technique described in~\cite{Petzold1983} and
Figure~\ref{fig:replicator_dynamics} shows the evolution of the distribution of
the system: the various strategies are ranked by scores. It is clear to see that
only the high ranking strategies survive the evolutionary process (in fact,
only \input{./assets/img/replicator_dynamics/main.tex}
have a final distribution greater than \(10 ^ {-2}\)). This confirms the
findings of~\cite{Moran1707} in which sophisticated strategies resist
evolutionary invasion of shorter memory strategies. Recalling
Figure~\ref{fig:SSError_and_probabilities_in_full} this demonstrates that:

\begin{itemize}
    \item Cooperation emerges through the evolutionary process: the high scoring
        strategies do not exhibit extortionate behaviour towards each other.
    \item Extortionate strategies do not survive the evolutionary process.
\end{itemize}

\begin{figure}[!htbp]
    \centering
    \includegraphics[width=.8\textwidth]{./assets/img/replicator_dynamics/main.pdf}
    \caption{Numerical simulation of the replicator equation
    (\ref{eqn:replicator_dynamics}): strategies are ordered by score, only the strategies with a high score survive the evolutionary process.}
    \label{fig:replicator_dynamics}
\end{figure}

This work can be used to classify plays of the IPD\@: data can be collected from
actual interactions (in lab or in the field). Furthermore, this allows for a
classification method similar to the notion of fingerprinting presented
in~\cite{Ashlock2008}. Trained strategies can potentially be classified as
extortionate or not or it could be possible to even constrain the reinforcement
learning approaches that are becoming prevalent in the literature.
Alternatively, this mathematical approach for recognising extortion could be
used in sophisticated strategies to defend against invasion. Arguably, some of
the strategies considered here exhibit this behaviour, indeed as described
in~\cite{Harper2017}, the top ranking strategies in the full tournament are
obtained using evolutionary reinforcement learning techniques, thus, suspicion
of extortionate behaviour could in fact be an evolutionary trait.

\section*{Acknowledgements}

The following open source software libraries were used in this research:

\begin{itemize}
    \item The Axelrod ~\cite{Knight2016, Knight2018} library (IPD strategies and
        tournaments).
    \item The sympy library~\cite{Meurer2017} (verification of all symbolic
        calculations).
    \item The matplotlib~\cite{Droettboom2018} library (visualisation).
    \item The pandas~\cite{Structures2010}, dask~\cite{Dask2016} and
        NumPy~\cite{Oliphant2015} libraries (data manipulation).
    \item The SciPy~\cite{Jones2001} library (numerical integration of the
        replicator equation).
\end{itemize}

This work was performed using the computational facilities of the Advanced
Research Computing @ Cardiff (ARCCA) Division, Cardiff University.

\printbibliography

\newpage
\section*{Supplementary materials}

\includepdf{assets/pdf/proof_of_form_of_extortionate_strategies/main.pdf}

\newpage

Using the pair wise interactions the transition rates \(p,
q\) can be measured and the steady state probabilities inferred and compared to
the actual probabilities of each state.
This is done numerically by computing the singular eigenvector of the
matrix \(A\) \cite{Stewart2009}:

\[
    A =
    \begin{bmatrix}
        p_1 q_1 & p_1 (1 - q_1) & (1 - p_1) q_1 & (1 -p_1) (1 - q_1) \\
        p_2 q_2 & p_2 (1 - q_2) & (1 - p_2) q_2 & (1 -p_2) (1 - q_2) \\
        p_3 q_3 & p_3 (1 - q_3) & (1 - p_3) q_3 & (1 -p_3) (1 - q_3) \\
        p_4 q_4 & p_4 (1 - q_4) & (1 - p_4) q_4 & (1 -p_4) (1 - q_4) \\
    \end{bmatrix}
\]

Figure~\ref{fig:computed_probabilities_vs_theoretic_probabilities} shows a
regression line fitted to every pairwise interaction with a reported
\(\text{SSError}\) value (pairwise interactions with missing states were
omitted). This serves to validate the approach: a part from some edge cases the
relationship is consistent.

\begin{figure}[!htbp]
    \centering
    \includegraphics[width=.8\textwidth]{./assets/img/computed_probabilities_vs_theoretic_probabilities/main.pdf}
    \caption{The
        relationship between the steady state probabilities inferred from the
        measured transitions and the actual steady state probabilities. A linear
        regression line is included validating the approach.}
    \label{fig:computed_probabilities_vs_theoretic_probabilities}
\end{figure}


\end{document}

have a final distribution greater than \(10 ^ {-2}\)). This confirms the
findings of~\cite{Moran1707} in which sophisticated strategies resist
evolutionary invasion of shorter memory strategies. Recalling
Figure~\ref{fig:SSError_and_probabilities_in_full} this demonstrates that:

\begin{itemize}
    \item Cooperation emerges through the evolutionary process: the high scoring
        strategies do not exhibit extortionate behaviour towards each other.
    \item Extortionate strategies do not survive the evolutionary process.
\end{itemize}

\begin{figure}[!htbp]
    \centering
    \includegraphics[width=.8\textwidth]{./assets/img/replicator_dynamics/main.pdf}
    \caption{Numerical simulation of the replicator equation
    (\ref{eqn:replicator_dynamics}): strategies are ordered by score, only the strategies with a high score survive the evolutionary process.}
    \label{fig:replicator_dynamics}
\end{figure}

This work can be used to classify plays of the IPD\@: data can be collected from
actual interactions (in lab or in the field). Furthermore, this allows for a
classification method similar to the notion of fingerprinting presented
in~\cite{Ashlock2008}. Trained strategies can potentially be classified as
extortionate or not or it could be possible to even constrain the reinforcement
learning approaches that are becoming prevalent in the literature.
Alternatively, this mathematical approach for recognising extortion could be
used in sophisticated strategies to defend against invasion. Arguably, some of
the strategies considered here exhibit this behaviour, indeed as described
in~\cite{Harper2017}, the top ranking strategies in the full tournament are
obtained using evolutionary reinforcement learning techniques, thus, suspicion
of extortionate behaviour could in fact be an evolutionary trait.

\section*{Acknowledgements}

The following open source software libraries were used in this research:

\begin{itemize}
    \item The Axelrod ~\cite{Knight2016, Knight2018} library (IPD strategies and
        tournaments).
    \item The sympy library~\cite{Meurer2017} (verification of all symbolic
        calculations).
    \item The matplotlib~\cite{Droettboom2018} library (visualisation).
    \item The pandas~\cite{Structures2010}, dask~\cite{Dask2016} and
        NumPy~\cite{Oliphant2015} libraries (data manipulation).
    \item The SciPy~\cite{Jones2001} library (numerical integration of the
        replicator equation).
\end{itemize}

This work was performed using the computational facilities of the Advanced
Research Computing @ Cardiff (ARCCA) Division, Cardiff University.

\printbibliography

\newpage
\section*{Supplementary materials}

\includepdf{assets/pdf/proof_of_form_of_extortionate_strategies/main.pdf}

\newpage

Using the pair wise interactions the transition rates \(p,
q\) can be measured and the steady state probabilities inferred and compared to
the actual probabilities of each state.
This is done numerically by computing the singular eigenvector of the
matrix \(A\) \cite{Stewart2009}:

\[
    A =
    \begin{bmatrix}
        p_1 q_1 & p_1 (1 - q_1) & (1 - p_1) q_1 & (1 -p_1) (1 - q_1) \\
        p_2 q_2 & p_2 (1 - q_2) & (1 - p_2) q_2 & (1 -p_2) (1 - q_2) \\
        p_3 q_3 & p_3 (1 - q_3) & (1 - p_3) q_3 & (1 -p_3) (1 - q_3) \\
        p_4 q_4 & p_4 (1 - q_4) & (1 - p_4) q_4 & (1 -p_4) (1 - q_4) \\
    \end{bmatrix}
\]

Figure~\ref{fig:computed_probabilities_vs_theoretic_probabilities} shows a
regression line fitted to every pairwise interaction with a reported
\(\text{SSError}\) value (pairwise interactions with missing states were
omitted). This serves to validate the approach: a part from some edge cases the
relationship is consistent.

\begin{figure}[!htbp]
    \centering
    \includegraphics[width=.8\textwidth]{./assets/img/computed_probabilities_vs_theoretic_probabilities/main.pdf}
    \caption{The
        relationship between the steady state probabilities inferred from the
        measured transitions and the actual steady state probabilities. A linear
        regression line is included validating the approach.}
    \label{fig:computed_probabilities_vs_theoretic_probabilities}
\end{figure}


\end{document}

strategies.

The results of this analysis are shown in
Figure~\ref{fig:SSError_and_probabilities_in_full}. The top ranking strategies
by number of wins seem to be extortionate (but not against all strategies) and
it can be seen that a small sub group of strategies achieve mutual defection.
All the top ranking strategies according to score achieve mutual cooperation and
do not extort each other, however they
\textbf{do} exhibit extortionate behaviour towards a number of the lower ranking
strategies.

\begin{figure}[!htbp]
    \centering
    \includegraphics[width=.8\textwidth]{./assets/img/SSError_and_probabilities_in_full/main.pdf}
    \caption{\(\text{SSError}\) for the strategies for the full tournament. Only
    strategy interactions for which \(p_4=0\) and \(\chi>1\) are displayed.}
    \label{fig:SSError_and_probabilities_in_full}
\end{figure}

\section{Conclusion}\label{sec:conclusion}

This work defines an approach to measure whether or not a player is playing a
strategy that corresponds to an extortionate strategy as defined
in~\cite{Press2012}: a mathematical model for suspicion. Indeed, all
extortionate strategies have been
 classified as lying on a triangular plane.
This rigorous classification fails to be robust to small measurement error, thus
a statistical approach is proposed.
This is done through a linear algebraic approach for approximating the solution
of a linear system. Using this, a large number of pairwise interactions is
simulated and in fact very few strategies are found to act extortionately.

The work of~\cite{Press2012}, whilst showing that a clever approach to taking
advantage of another memory one strategy exists: this is incomplete. Whilst the
elegance of this result is very attractive, just as the simplicity of the
victory of Tit For Tat in Axelrod's original tournaments was, it is incomplete.
Extortionate strategies achieve a high number of wins but they do not
achieve a high score which corresponds to the fitness landscape in an
evolutionary sense. From the large number of interactions a payoff matrix \(S\)
can be measured where \(S_{ij}\) denotes the score (using standard values of
\((R, S, T, P) = (3, 0, 5, 1)\)) of the \(i\)th strategy
against the \(j\)th strategy. Using this, the replicator equation
describes the evolution of the system based on a population density fitness
function:

\begin{equation}\label{eqn:replicator_dynamics}
    \frac{dx}{dt} = x(S-x^TS x)
\end{equation}

Equation (\ref{eqn:replicator_dynamics}) is solved numerically through an
integration technique described in~\cite{Petzold1983} and
Figure~\ref{fig:replicator_dynamics} shows the evolution of the distribution of
the system: the various strategies are ranked by scores. It is clear to see that
only the high ranking strategies survive the evolutionary process (in fact,
only \documentclass[a4paper]{article}

\usepackage{amsmath}
\usepackage{amssymb}
\usepackage[margin=1.5cm,
            includefoot,
            footskip=30pt]{geometry}
\usepackage{layout}
\usepackage{graphicx}
\usepackage{subcaption}

\usepackage{biblatex}
\usepackage{pdfpages}

\bibliography{main.bib}

\title{Suspicion: Recognising and evaluating the effectiveness
       of extortion in the Iterated Prisoner's Dilemma}
\author{Vincent A. Knight \and Nikoleta E. Glynatsi}
\date{\today}



\begin{document}

\maketitle

\begin{abstract}
    The Iterated Prisoner's Dilemma is a model for rational and evolutionary
    interactive behaviour. It has applications both in the study of human social
    behaviour as well as in biology.
    It is used to understand when and how a rational individual might
    accept an immediate cost to their own utility for the direct benefit of
    another.

    Much attention has been given to a class of strategies called
    Zero Determinant strategies. It has been theoretically shown that these
    strategies can ``extort'' any player.

    In this work, an approach to identify if observed strategies are playing in
    an extortionate way is described. Furthermore, experimental analysis of
    a large tournament with \documentclass[a4paper]{article}

\usepackage{amsmath}
\usepackage{amssymb}
\usepackage[margin=1.5cm,
            includefoot,
            footskip=30pt]{geometry}
\usepackage{layout}
\usepackage{graphicx}
\usepackage{subcaption}

\usepackage{biblatex}
\usepackage{pdfpages}

\bibliography{main.bib}

\title{Suspicion: Recognising and evaluating the effectiveness
       of extortion in the Iterated Prisoner's Dilemma}
\author{Vincent A. Knight \and Nikoleta E. Glynatsi}
\date{\today}



\begin{document}

\maketitle

\begin{abstract}
    The Iterated Prisoner's Dilemma is a model for rational and evolutionary
    interactive behaviour. It has applications both in the study of human social
    behaviour as well as in biology.
    It is used to understand when and how a rational individual might
    accept an immediate cost to their own utility for the direct benefit of
    another.

    Much attention has been given to a class of strategies called
    Zero Determinant strategies. It has been theoretically shown that these
    strategies can ``extort'' any player.

    In this work, an approach to identify if observed strategies are playing in
    an extortionate way is described. Furthermore, experimental analysis of
    a large tournament with \input{assets/tex/number_of_full_strategies/main.tex}
    strategies is considered. In this setting
    the most highly performing strategies do not play in an extortionate way
    against each other but do against lower performing strategies.
    This suggests that whilst the theory of Zero Determinant strategies
    indicates that memory is not of fundamental importance to the evolution of
    cooperative behaviour, this is incomplete.
\end{abstract}

\section{Introduction}\label{sec:introduction}

Agent based game theoretic models have become a stalwart of the underpinning
mathematics of interactive behaviours. One of the major pieces of work
in this area is the pair of original computer tournaments run by Robert
Axelrod~\cite{Axelrod1980, Axelrod1980a}. These tournaments pitted submitted
computer strategies against each other in plays of the Iterated Prisoner's
Dilemma. A common game where agents can choose to pay a slight cost to their
immediate utility in the hope of building a reputation. This has been used in
economic and evolutionary game theory to understand the evolution of cooperative
behaviour.

Recently, a class of strategies was described in~\cite{Press2012} that can
provably extort any given opponent. In~\cite{Hilbe2013, Moran1707} some
questions have already been asked about the true effectiveness of these
strategies in an evolutionary setting. Here another question is asked: is it
possible to recognise this extortionate behaviour? A mathematical procedure for
suspicion is presented: in the same way that the continued actions of an
extortionate individual might raise suspicion.

This work makes use of the Axelrod Python library~\cite{Knight2018, Knight2016}
with a large number of Prisoner Dilemma strategies available to give an
extensive numerical example of the ideas presented.  The approach is presented
in Section~\ref{sec:delta-zd-strategies}.  All of the code and data discussed
in Section~\ref{sec:numerical-experiments} is open sourced, archived and
written according to best scientific principles~\cite{Wilson2014}. The data
archive can be found at~\cite{vincent_knight_2018_1297075}.

\section{Recognising Extortion}\label{sec:delta-zd-strategies}

In~\cite{Press2012}, given a match between 2 memory-one strategies, the concept
of Zero Determinant (ZD) strategies is introduced. The main result of that paper
shows that given two memory one players \(p, q\in\mathbb{R}^4\) a linear
relationship between the players' scores could be forced by one of the players.

Using the notation of~\cite{Press2012}, assuming the utilities for player \(p\)
are given by \(S_x=(R, S, T, P)\) and for player \(q\) by \(S_y=(R, T, S, P)\)
and that the stationary scores of each player is given by \(S_X\) and \(S_Y\)
respectively. The main result of~\cite{Press2012} is that if

\begin{equation}\label{eqn:linear_relationship_for_p}
    \tilde p=\alpha S_x + \beta S_y + \gamma
\end{equation}

or

\begin{equation}\label{eqn:linear_relationship_for_q}
    \tilde q=\alpha S_x + \beta S_y + \gamma
\end{equation}

where \(\tilde p = (1 - p_1, 1 - p_2, p_3, p_4)\) and
\(\tilde q = (1 - q_1, 1 - q_2, q_3, q_4)\) then:

\begin{equation}
    \alpha S_X + \beta S_Y + \gamma = 0
\end{equation}

In~\cite{Press2012} a particular type of ZD strategy is defined: extortionate
strategies. If:

\begin{equation}\label{eqn:constraint_for_extortion}
    \gamma = - P(\alpha + \beta)
\end{equation}

then the player can ensure they get a score \(\chi\) times
larger than the opponent. This extortion coefficient is given by:

\begin{equation}\label{eqn:definition_of_chi}
    \chi=\frac{-\beta}{\alpha}
\end{equation}

Thus, if (\ref{eqn:constraint_for_extortion}) holds and \(\chi >1\) a player is
said to extort their opponent.
Here, the reverse problem is considered: given a
\(p\in\mathbb{R}^4\) how does one identify \(\alpha, \beta\) if they
exist and is the strategy in fact acting in an extortionate way?

These conditions correspond to:

\begin{align}
    \tilde p_1 & = \alpha R + \beta R - P (\alpha + \beta)
            \label{eqn:condition_for_tilde_p1}\\
    \tilde p_2 & = \alpha S + \beta T - P (\alpha + \beta)
            \label{eqn:condition_for_tilde_p2}\\
    \tilde p_3 & = \alpha T + \beta S - P (\alpha + \beta)
            \label{eqn:condition_for_tilde_p3}\\
    \tilde p_4 & = \alpha P + \beta P - P (\alpha + \beta)
            \label{eqn:condition_for_tilde_p4}
\end{align}

Equation (\ref{eqn:condition_for_tilde_p4}) ensures that \(p_4=\tilde p_4=0\).
Equations (\ref{eqn:condition_for_tilde_p1}-\ref{eqn:condition_for_tilde_p3})
can be used to eliminate \(\alpha, \beta\), giving:

\begin{equation}\label{eqn:planar_definition_of_extortion}
    \tilde p_1 = \frac{(R - P)(\tilde p_2 + \tilde p_3)}{S + T - 2P}
\end{equation}

with:

\begin{equation}\label{eqn:definition_of_chi}
    \chi = \frac{\tilde p_2 (P - T) + \tilde p_3 (S - P)}
                {\tilde p_2 (P - S) + \tilde p_3 (T - P)}
\end{equation}

Given a strategy \(p\in\mathbb{R}^{4\times 1}\) equations
(\ref{eqn:condition_for_tilde_p4}), (\ref{eqn:planar_definition_of_extortion}-\ref{eqn:definition_of_chi}) can be used to check if
a strategy is extortionate. The conditions correspond to:

\begin{align}
    p_1 & = \frac{(R-P)(p_2 + p_3) - R + T + S - P}{S + T - 2P}
     \label{eqn:condition_for_p1}\\
    p_4 & = 0 \label{eqn:condition_for_p4}\\
    1 & > p_2 + p_3\label{eqn:condition_for_chi}
\end{align}

The algebraic steps necessary to prove these results are available in the
supporting materials.

All extortionate strategies reside on a triangular (\ref{eqn:condition_for_chi})
plane (\ref{eqn:condition_for_p1}) in 3 dimensions (\ref{eqn:condition_for_p4}).
Using this formulation it can be seen that a necessary (but not sufficient)
condition for an extortionate strategy is that it cooperates on average less
than 50\% of the time when in a state of disagreement with the opponent.

As an example, consider the known extortionate strategy \(p=(8 / 9, 1 / 2, 1 /
3, 0)\) from~\cite{Stewart2012} which is referred to as \texttt{Extort-2}. In
this case, for the standard values of \((R, T, S, P)\) constraint
(\ref{eqn:condition_for_p1}) corresponds to:

\begin{equation}
    p_1 = \frac{2(p_2 + p_3) + 1}{3}
\end{equation}

It is clear that in this case all constraints hold.

This approach could in fact be used to confirm that a given strategy is acting
in an extortionate manner even if it is not a memory one strategy. However, in
practice, if a closed form for \(p\) is not known, then due to measurement
and/or numerical error this would not work.

This problem can be written in the following linear algebraic form where
\(x=(\alpha, \beta)\)
and \(p^*=(\tilde p_1 - 1, tilde_2 - 1, p_3)\):

\begin{equation}\label{eqn:linear_algebraic_equation_for_p}
    Cx= p^*
\end{equation}

\(C\) corresponds to equations
(\ref{eqn:condition_for_tilde_p1}-\ref{eqn:condition_for_tilde_p3}) and is
given by:

\begin{equation}\label{eqn:definition_of_C}
    C =
    \begin{bmatrix}
        R - P & R- P \\
        S - P & T- P \\
        T - P & S- P \\
    \end{bmatrix}
\end{equation}

Note that in general, equation (\ref{eqn:linear_algebraic_equation_for_p}) will
not necessarily have a solution. From the Rouch\'{e}-Capelli theorem if there is
a solution it is unique as \(\text{rank}(C)=2\) which is the dimension of the
variable \(x\). The best fitting \(x\) is found by minimizing:

\begin{equation}\label{eqn:r_squared}
    \text{SSError} = \|C x- p^*\|_2^2 = \sum_{i=1}^{3}\left((C\bar x)_i-p_i^*\right)^2
\end{equation}

Note that \(\text{SSError}\), which is the square of the Frobenius
norm~\cite{Golub2013}, becomes a measure of how close a strategy is to being an
extortionate strategy. Suspicion
of extortion then corresponds to a threshold on \(\text{SSError}\).

By observing interactions (human or otherwise), their memory one representation
can be inferred and this approach can be used to recognise extortionate
behaviour. The notion of comparing theoretic and actual plays of the IPD is not
novel, see for example~\cite{Rand2013}. Immediately it is noted that if the
environment is noisy~\cite{Wu1995} then no strategy can be considered to be
extortionate as \(p_4>0\).

In the next section, this idea will be illustrated by observing the interactions
that take place in a computer based tournament of the IPD\@.

\section{Numerical experiments}\label{sec:numerical-experiments}

In~\cite{Stewart2012} results from a tournament with
\input{./assets/tex/number_of_stewart_plotkin_strategies/main.tex} strategies,
was presented with specific consideration given to ZD strategies. This
tournament is reproduced here using the Axelrod-Python
project~\cite{Knight2016}. To obtain a good measure of the corresponding
transition rates for each strategy all matches have been run for
\input{assets/tex/number_of_turns/main.tex} turns and every match has been
repeated \input{assets/tex/number_of_repetitions/main.tex} times. All of this
interaction data is available at~\cite{vincent_knight_2018_1297075}. A good
match between the inferred Markov chain and the state distribution of the actual
interactions has been verified. Data for this is presented in the supplementary
materials.

Figure~\ref{fig:SSError_overall_in_stewart_plotkin} shows the \(\text{SSError}\)
values for all the strategies in the tournament, as reported
in~\cite{Stewart2012} the extortionate strategy (which has an expected
\(\text{SSError}\) approximately 0) gains a large number of wins.

\begin{figure}[!htbp]
    \centering
    \includegraphics[width=.8\textwidth]{./assets/img/SSError_overall_in_stewart_plotkin/main.pdf}
    \caption{\(\text{SSError}\) and state probabilities for the strategies
        of~\cite{Stewart2012}, ordered both by number of wins and overall score.
        Note that \(P(DC)\) is not shown as it corresponds to the transpose of
        \(P(CD)\). Cooperator and Defector are omitted as they do not visit all
        the states.}
    \label{fig:SSError_overall_in_stewart_plotkin}
\end{figure}

Here, the work of~\cite{Stewart2012} is extended by investigating a tournament
with \input{assets/tex/number_of_full_strategies/main.tex}
strategies.

The results of this analysis are shown in
Figure~\ref{fig:SSError_and_probabilities_in_full}. The top ranking strategies
by number of wins seem to be extortionate (but not against all strategies) and
it can be seen that a small sub group of strategies achieve mutual defection.
All the top ranking strategies according to score achieve mutual cooperation and
do not extort each other, however they
\textbf{do} exhibit extortionate behaviour towards a number of the lower ranking
strategies.

\begin{figure}[!htbp]
    \centering
    \includegraphics[width=.8\textwidth]{./assets/img/SSError_and_probabilities_in_full/main.pdf}
    \caption{\(\text{SSError}\) for the strategies for the full tournament. Only
    strategy interactions for which \(p_4=0\) and \(\chi>1\) are displayed.}
    \label{fig:SSError_and_probabilities_in_full}
\end{figure}

\section{Conclusion}\label{sec:conclusion}

This work defines an approach to measure whether or not a player is playing a
strategy that corresponds to an extortionate strategy as defined
in~\cite{Press2012}: a mathematical model for suspicion. Indeed, all
extortionate strategies have been
 classified as lying on a triangular plane.
This rigorous classification fails to be robust to small measurement error, thus
a statistical approach is proposed.
This is done through a linear algebraic approach for approximating the solution
of a linear system. Using this, a large number of pairwise interactions is
simulated and in fact very few strategies are found to act extortionately.

The work of~\cite{Press2012}, whilst showing that a clever approach to taking
advantage of another memory one strategy exists: this is incomplete. Whilst the
elegance of this result is very attractive, just as the simplicity of the
victory of Tit For Tat in Axelrod's original tournaments was, it is incomplete.
Extortionate strategies achieve a high number of wins but they do not
achieve a high score which corresponds to the fitness landscape in an
evolutionary sense. From the large number of interactions a payoff matrix \(S\)
can be measured where \(S_{ij}\) denotes the score (using standard values of
\((R, S, T, P) = (3, 0, 5, 1)\)) of the \(i\)th strategy
against the \(j\)th strategy. Using this, the replicator equation
describes the evolution of the system based on a population density fitness
function:

\begin{equation}\label{eqn:replicator_dynamics}
    \frac{dx}{dt} = x(S-x^TS x)
\end{equation}

Equation (\ref{eqn:replicator_dynamics}) is solved numerically through an
integration technique described in~\cite{Petzold1983} and
Figure~\ref{fig:replicator_dynamics} shows the evolution of the distribution of
the system: the various strategies are ranked by scores. It is clear to see that
only the high ranking strategies survive the evolutionary process (in fact,
only \input{./assets/img/replicator_dynamics/main.tex}
have a final distribution greater than \(10 ^ {-2}\)). This confirms the
findings of~\cite{Moran1707} in which sophisticated strategies resist
evolutionary invasion of shorter memory strategies. Recalling
Figure~\ref{fig:SSError_and_probabilities_in_full} this demonstrates that:

\begin{itemize}
    \item Cooperation emerges through the evolutionary process: the high scoring
        strategies do not exhibit extortionate behaviour towards each other.
    \item Extortionate strategies do not survive the evolutionary process.
\end{itemize}

\begin{figure}[!htbp]
    \centering
    \includegraphics[width=.8\textwidth]{./assets/img/replicator_dynamics/main.pdf}
    \caption{Numerical simulation of the replicator equation
    (\ref{eqn:replicator_dynamics}): strategies are ordered by score, only the strategies with a high score survive the evolutionary process.}
    \label{fig:replicator_dynamics}
\end{figure}

This work can be used to classify plays of the IPD\@: data can be collected from
actual interactions (in lab or in the field). Furthermore, this allows for a
classification method similar to the notion of fingerprinting presented
in~\cite{Ashlock2008}. Trained strategies can potentially be classified as
extortionate or not or it could be possible to even constrain the reinforcement
learning approaches that are becoming prevalent in the literature.
Alternatively, this mathematical approach for recognising extortion could be
used in sophisticated strategies to defend against invasion. Arguably, some of
the strategies considered here exhibit this behaviour, indeed as described
in~\cite{Harper2017}, the top ranking strategies in the full tournament are
obtained using evolutionary reinforcement learning techniques, thus, suspicion
of extortionate behaviour could in fact be an evolutionary trait.

\section*{Acknowledgements}

The following open source software libraries were used in this research:

\begin{itemize}
    \item The Axelrod ~\cite{Knight2016, Knight2018} library (IPD strategies and
        tournaments).
    \item The sympy library~\cite{Meurer2017} (verification of all symbolic
        calculations).
    \item The matplotlib~\cite{Droettboom2018} library (visualisation).
    \item The pandas~\cite{Structures2010}, dask~\cite{Dask2016} and
        NumPy~\cite{Oliphant2015} libraries (data manipulation).
    \item The SciPy~\cite{Jones2001} library (numerical integration of the
        replicator equation).
\end{itemize}

This work was performed using the computational facilities of the Advanced
Research Computing @ Cardiff (ARCCA) Division, Cardiff University.

\printbibliography

\newpage
\section*{Supplementary materials}

\includepdf{assets/pdf/proof_of_form_of_extortionate_strategies/main.pdf}

\newpage

Using the pair wise interactions the transition rates \(p,
q\) can be measured and the steady state probabilities inferred and compared to
the actual probabilities of each state.
This is done numerically by computing the singular eigenvector of the
matrix \(A\) \cite{Stewart2009}:

\[
    A =
    \begin{bmatrix}
        p_1 q_1 & p_1 (1 - q_1) & (1 - p_1) q_1 & (1 -p_1) (1 - q_1) \\
        p_2 q_2 & p_2 (1 - q_2) & (1 - p_2) q_2 & (1 -p_2) (1 - q_2) \\
        p_3 q_3 & p_3 (1 - q_3) & (1 - p_3) q_3 & (1 -p_3) (1 - q_3) \\
        p_4 q_4 & p_4 (1 - q_4) & (1 - p_4) q_4 & (1 -p_4) (1 - q_4) \\
    \end{bmatrix}
\]

Figure~\ref{fig:computed_probabilities_vs_theoretic_probabilities} shows a
regression line fitted to every pairwise interaction with a reported
\(\text{SSError}\) value (pairwise interactions with missing states were
omitted). This serves to validate the approach: a part from some edge cases the
relationship is consistent.

\begin{figure}[!htbp]
    \centering
    \includegraphics[width=.8\textwidth]{./assets/img/computed_probabilities_vs_theoretic_probabilities/main.pdf}
    \caption{The
        relationship between the steady state probabilities inferred from the
        measured transitions and the actual steady state probabilities. A linear
        regression line is included validating the approach.}
    \label{fig:computed_probabilities_vs_theoretic_probabilities}
\end{figure}


\end{document}

    strategies is considered. In this setting
    the most highly performing strategies do not play in an extortionate way
    against each other but do against lower performing strategies.
    This suggests that whilst the theory of Zero Determinant strategies
    indicates that memory is not of fundamental importance to the evolution of
    cooperative behaviour, this is incomplete.
\end{abstract}

\section{Introduction}\label{sec:introduction}

Agent based game theoretic models have become a stalwart of the underpinning
mathematics of interactive behaviours. One of the major pieces of work
in this area is the pair of original computer tournaments run by Robert
Axelrod~\cite{Axelrod1980, Axelrod1980a}. These tournaments pitted submitted
computer strategies against each other in plays of the Iterated Prisoner's
Dilemma. A common game where agents can choose to pay a slight cost to their
immediate utility in the hope of building a reputation. This has been used in
economic and evolutionary game theory to understand the evolution of cooperative
behaviour.

Recently, a class of strategies was described in~\cite{Press2012} that can
provably extort any given opponent. In~\cite{Hilbe2013, Moran1707} some
questions have already been asked about the true effectiveness of these
strategies in an evolutionary setting. Here another question is asked: is it
possible to recognise this extortionate behaviour? A mathematical procedure for
suspicion is presented: in the same way that the continued actions of an
extortionate individual might raise suspicion.

This work makes use of the Axelrod Python library~\cite{Knight2018, Knight2016}
with a large number of Prisoner Dilemma strategies available to give an
extensive numerical example of the ideas presented.  The approach is presented
in Section~\ref{sec:delta-zd-strategies}.  All of the code and data discussed
in Section~\ref{sec:numerical-experiments} is open sourced, archived and
written according to best scientific principles~\cite{Wilson2014}. The data
archive can be found at~\cite{vincent_knight_2018_1297075}.

\section{Recognising Extortion}\label{sec:delta-zd-strategies}

In~\cite{Press2012}, given a match between 2 memory-one strategies, the concept
of Zero Determinant (ZD) strategies is introduced. The main result of that paper
shows that given two memory one players \(p, q\in\mathbb{R}^4\) a linear
relationship between the players' scores could be forced by one of the players.

Using the notation of~\cite{Press2012}, assuming the utilities for player \(p\)
are given by \(S_x=(R, S, T, P)\) and for player \(q\) by \(S_y=(R, T, S, P)\)
and that the stationary scores of each player is given by \(S_X\) and \(S_Y\)
respectively. The main result of~\cite{Press2012} is that if

\begin{equation}\label{eqn:linear_relationship_for_p}
    \tilde p=\alpha S_x + \beta S_y + \gamma
\end{equation}

or

\begin{equation}\label{eqn:linear_relationship_for_q}
    \tilde q=\alpha S_x + \beta S_y + \gamma
\end{equation}

where \(\tilde p = (1 - p_1, 1 - p_2, p_3, p_4)\) and
\(\tilde q = (1 - q_1, 1 - q_2, q_3, q_4)\) then:

\begin{equation}
    \alpha S_X + \beta S_Y + \gamma = 0
\end{equation}

In~\cite{Press2012} a particular type of ZD strategy is defined: extortionate
strategies. If:

\begin{equation}\label{eqn:constraint_for_extortion}
    \gamma = - P(\alpha + \beta)
\end{equation}

then the player can ensure they get a score \(\chi\) times
larger than the opponent. This extortion coefficient is given by:

\begin{equation}\label{eqn:definition_of_chi}
    \chi=\frac{-\beta}{\alpha}
\end{equation}

Thus, if (\ref{eqn:constraint_for_extortion}) holds and \(\chi >1\) a player is
said to extort their opponent.
Here, the reverse problem is considered: given a
\(p\in\mathbb{R}^4\) how does one identify \(\alpha, \beta\) if they
exist and is the strategy in fact acting in an extortionate way?

These conditions correspond to:

\begin{align}
    \tilde p_1 & = \alpha R + \beta R - P (\alpha + \beta)
            \label{eqn:condition_for_tilde_p1}\\
    \tilde p_2 & = \alpha S + \beta T - P (\alpha + \beta)
            \label{eqn:condition_for_tilde_p2}\\
    \tilde p_3 & = \alpha T + \beta S - P (\alpha + \beta)
            \label{eqn:condition_for_tilde_p3}\\
    \tilde p_4 & = \alpha P + \beta P - P (\alpha + \beta)
            \label{eqn:condition_for_tilde_p4}
\end{align}

Equation (\ref{eqn:condition_for_tilde_p4}) ensures that \(p_4=\tilde p_4=0\).
Equations (\ref{eqn:condition_for_tilde_p1}-\ref{eqn:condition_for_tilde_p3})
can be used to eliminate \(\alpha, \beta\), giving:

\begin{equation}\label{eqn:planar_definition_of_extortion}
    \tilde p_1 = \frac{(R - P)(\tilde p_2 + \tilde p_3)}{S + T - 2P}
\end{equation}

with:

\begin{equation}\label{eqn:definition_of_chi}
    \chi = \frac{\tilde p_2 (P - T) + \tilde p_3 (S - P)}
                {\tilde p_2 (P - S) + \tilde p_3 (T - P)}
\end{equation}

Given a strategy \(p\in\mathbb{R}^{4\times 1}\) equations
(\ref{eqn:condition_for_tilde_p4}), (\ref{eqn:planar_definition_of_extortion}-\ref{eqn:definition_of_chi}) can be used to check if
a strategy is extortionate. The conditions correspond to:

\begin{align}
    p_1 & = \frac{(R-P)(p_2 + p_3) - R + T + S - P}{S + T - 2P}
     \label{eqn:condition_for_p1}\\
    p_4 & = 0 \label{eqn:condition_for_p4}\\
    1 & > p_2 + p_3\label{eqn:condition_for_chi}
\end{align}

The algebraic steps necessary to prove these results are available in the
supporting materials.

All extortionate strategies reside on a triangular (\ref{eqn:condition_for_chi})
plane (\ref{eqn:condition_for_p1}) in 3 dimensions (\ref{eqn:condition_for_p4}).
Using this formulation it can be seen that a necessary (but not sufficient)
condition for an extortionate strategy is that it cooperates on average less
than 50\% of the time when in a state of disagreement with the opponent.

As an example, consider the known extortionate strategy \(p=(8 / 9, 1 / 2, 1 /
3, 0)\) from~\cite{Stewart2012} which is referred to as \texttt{Extort-2}. In
this case, for the standard values of \((R, T, S, P)\) constraint
(\ref{eqn:condition_for_p1}) corresponds to:

\begin{equation}
    p_1 = \frac{2(p_2 + p_3) + 1}{3}
\end{equation}

It is clear that in this case all constraints hold.

This approach could in fact be used to confirm that a given strategy is acting
in an extortionate manner even if it is not a memory one strategy. However, in
practice, if a closed form for \(p\) is not known, then due to measurement
and/or numerical error this would not work.

This problem can be written in the following linear algebraic form where
\(x=(\alpha, \beta)\)
and \(p^*=(\tilde p_1 - 1, tilde_2 - 1, p_3)\):

\begin{equation}\label{eqn:linear_algebraic_equation_for_p}
    Cx= p^*
\end{equation}

\(C\) corresponds to equations
(\ref{eqn:condition_for_tilde_p1}-\ref{eqn:condition_for_tilde_p3}) and is
given by:

\begin{equation}\label{eqn:definition_of_C}
    C =
    \begin{bmatrix}
        R - P & R- P \\
        S - P & T- P \\
        T - P & S- P \\
    \end{bmatrix}
\end{equation}

Note that in general, equation (\ref{eqn:linear_algebraic_equation_for_p}) will
not necessarily have a solution. From the Rouch\'{e}-Capelli theorem if there is
a solution it is unique as \(\text{rank}(C)=2\) which is the dimension of the
variable \(x\). The best fitting \(x\) is found by minimizing:

\begin{equation}\label{eqn:r_squared}
    \text{SSError} = \|C x- p^*\|_2^2 = \sum_{i=1}^{3}\left((C\bar x)_i-p_i^*\right)^2
\end{equation}

Note that \(\text{SSError}\), which is the square of the Frobenius
norm~\cite{Golub2013}, becomes a measure of how close a strategy is to being an
extortionate strategy. Suspicion
of extortion then corresponds to a threshold on \(\text{SSError}\).

By observing interactions (human or otherwise), their memory one representation
can be inferred and this approach can be used to recognise extortionate
behaviour. The notion of comparing theoretic and actual plays of the IPD is not
novel, see for example~\cite{Rand2013}. Immediately it is noted that if the
environment is noisy~\cite{Wu1995} then no strategy can be considered to be
extortionate as \(p_4>0\).

In the next section, this idea will be illustrated by observing the interactions
that take place in a computer based tournament of the IPD\@.

\section{Numerical experiments}\label{sec:numerical-experiments}

In~\cite{Stewart2012} results from a tournament with
\documentclass[a4paper]{article}

\usepackage{amsmath}
\usepackage{amssymb}
\usepackage[margin=1.5cm,
            includefoot,
            footskip=30pt]{geometry}
\usepackage{layout}
\usepackage{graphicx}
\usepackage{subcaption}

\usepackage{biblatex}
\usepackage{pdfpages}

\bibliography{main.bib}

\title{Suspicion: Recognising and evaluating the effectiveness
       of extortion in the Iterated Prisoner's Dilemma}
\author{Vincent A. Knight \and Nikoleta E. Glynatsi}
\date{\today}



\begin{document}

\maketitle

\begin{abstract}
    The Iterated Prisoner's Dilemma is a model for rational and evolutionary
    interactive behaviour. It has applications both in the study of human social
    behaviour as well as in biology.
    It is used to understand when and how a rational individual might
    accept an immediate cost to their own utility for the direct benefit of
    another.

    Much attention has been given to a class of strategies called
    Zero Determinant strategies. It has been theoretically shown that these
    strategies can ``extort'' any player.

    In this work, an approach to identify if observed strategies are playing in
    an extortionate way is described. Furthermore, experimental analysis of
    a large tournament with \input{assets/tex/number_of_full_strategies/main.tex}
    strategies is considered. In this setting
    the most highly performing strategies do not play in an extortionate way
    against each other but do against lower performing strategies.
    This suggests that whilst the theory of Zero Determinant strategies
    indicates that memory is not of fundamental importance to the evolution of
    cooperative behaviour, this is incomplete.
\end{abstract}

\section{Introduction}\label{sec:introduction}

Agent based game theoretic models have become a stalwart of the underpinning
mathematics of interactive behaviours. One of the major pieces of work
in this area is the pair of original computer tournaments run by Robert
Axelrod~\cite{Axelrod1980, Axelrod1980a}. These tournaments pitted submitted
computer strategies against each other in plays of the Iterated Prisoner's
Dilemma. A common game where agents can choose to pay a slight cost to their
immediate utility in the hope of building a reputation. This has been used in
economic and evolutionary game theory to understand the evolution of cooperative
behaviour.

Recently, a class of strategies was described in~\cite{Press2012} that can
provably extort any given opponent. In~\cite{Hilbe2013, Moran1707} some
questions have already been asked about the true effectiveness of these
strategies in an evolutionary setting. Here another question is asked: is it
possible to recognise this extortionate behaviour? A mathematical procedure for
suspicion is presented: in the same way that the continued actions of an
extortionate individual might raise suspicion.

This work makes use of the Axelrod Python library~\cite{Knight2018, Knight2016}
with a large number of Prisoner Dilemma strategies available to give an
extensive numerical example of the ideas presented.  The approach is presented
in Section~\ref{sec:delta-zd-strategies}.  All of the code and data discussed
in Section~\ref{sec:numerical-experiments} is open sourced, archived and
written according to best scientific principles~\cite{Wilson2014}. The data
archive can be found at~\cite{vincent_knight_2018_1297075}.

\section{Recognising Extortion}\label{sec:delta-zd-strategies}

In~\cite{Press2012}, given a match between 2 memory-one strategies, the concept
of Zero Determinant (ZD) strategies is introduced. The main result of that paper
shows that given two memory one players \(p, q\in\mathbb{R}^4\) a linear
relationship between the players' scores could be forced by one of the players.

Using the notation of~\cite{Press2012}, assuming the utilities for player \(p\)
are given by \(S_x=(R, S, T, P)\) and for player \(q\) by \(S_y=(R, T, S, P)\)
and that the stationary scores of each player is given by \(S_X\) and \(S_Y\)
respectively. The main result of~\cite{Press2012} is that if

\begin{equation}\label{eqn:linear_relationship_for_p}
    \tilde p=\alpha S_x + \beta S_y + \gamma
\end{equation}

or

\begin{equation}\label{eqn:linear_relationship_for_q}
    \tilde q=\alpha S_x + \beta S_y + \gamma
\end{equation}

where \(\tilde p = (1 - p_1, 1 - p_2, p_3, p_4)\) and
\(\tilde q = (1 - q_1, 1 - q_2, q_3, q_4)\) then:

\begin{equation}
    \alpha S_X + \beta S_Y + \gamma = 0
\end{equation}

In~\cite{Press2012} a particular type of ZD strategy is defined: extortionate
strategies. If:

\begin{equation}\label{eqn:constraint_for_extortion}
    \gamma = - P(\alpha + \beta)
\end{equation}

then the player can ensure they get a score \(\chi\) times
larger than the opponent. This extortion coefficient is given by:

\begin{equation}\label{eqn:definition_of_chi}
    \chi=\frac{-\beta}{\alpha}
\end{equation}

Thus, if (\ref{eqn:constraint_for_extortion}) holds and \(\chi >1\) a player is
said to extort their opponent.
Here, the reverse problem is considered: given a
\(p\in\mathbb{R}^4\) how does one identify \(\alpha, \beta\) if they
exist and is the strategy in fact acting in an extortionate way?

These conditions correspond to:

\begin{align}
    \tilde p_1 & = \alpha R + \beta R - P (\alpha + \beta)
            \label{eqn:condition_for_tilde_p1}\\
    \tilde p_2 & = \alpha S + \beta T - P (\alpha + \beta)
            \label{eqn:condition_for_tilde_p2}\\
    \tilde p_3 & = \alpha T + \beta S - P (\alpha + \beta)
            \label{eqn:condition_for_tilde_p3}\\
    \tilde p_4 & = \alpha P + \beta P - P (\alpha + \beta)
            \label{eqn:condition_for_tilde_p4}
\end{align}

Equation (\ref{eqn:condition_for_tilde_p4}) ensures that \(p_4=\tilde p_4=0\).
Equations (\ref{eqn:condition_for_tilde_p1}-\ref{eqn:condition_for_tilde_p3})
can be used to eliminate \(\alpha, \beta\), giving:

\begin{equation}\label{eqn:planar_definition_of_extortion}
    \tilde p_1 = \frac{(R - P)(\tilde p_2 + \tilde p_3)}{S + T - 2P}
\end{equation}

with:

\begin{equation}\label{eqn:definition_of_chi}
    \chi = \frac{\tilde p_2 (P - T) + \tilde p_3 (S - P)}
                {\tilde p_2 (P - S) + \tilde p_3 (T - P)}
\end{equation}

Given a strategy \(p\in\mathbb{R}^{4\times 1}\) equations
(\ref{eqn:condition_for_tilde_p4}), (\ref{eqn:planar_definition_of_extortion}-\ref{eqn:definition_of_chi}) can be used to check if
a strategy is extortionate. The conditions correspond to:

\begin{align}
    p_1 & = \frac{(R-P)(p_2 + p_3) - R + T + S - P}{S + T - 2P}
     \label{eqn:condition_for_p1}\\
    p_4 & = 0 \label{eqn:condition_for_p4}\\
    1 & > p_2 + p_3\label{eqn:condition_for_chi}
\end{align}

The algebraic steps necessary to prove these results are available in the
supporting materials.

All extortionate strategies reside on a triangular (\ref{eqn:condition_for_chi})
plane (\ref{eqn:condition_for_p1}) in 3 dimensions (\ref{eqn:condition_for_p4}).
Using this formulation it can be seen that a necessary (but not sufficient)
condition for an extortionate strategy is that it cooperates on average less
than 50\% of the time when in a state of disagreement with the opponent.

As an example, consider the known extortionate strategy \(p=(8 / 9, 1 / 2, 1 /
3, 0)\) from~\cite{Stewart2012} which is referred to as \texttt{Extort-2}. In
this case, for the standard values of \((R, T, S, P)\) constraint
(\ref{eqn:condition_for_p1}) corresponds to:

\begin{equation}
    p_1 = \frac{2(p_2 + p_3) + 1}{3}
\end{equation}

It is clear that in this case all constraints hold.

This approach could in fact be used to confirm that a given strategy is acting
in an extortionate manner even if it is not a memory one strategy. However, in
practice, if a closed form for \(p\) is not known, then due to measurement
and/or numerical error this would not work.

This problem can be written in the following linear algebraic form where
\(x=(\alpha, \beta)\)
and \(p^*=(\tilde p_1 - 1, tilde_2 - 1, p_3)\):

\begin{equation}\label{eqn:linear_algebraic_equation_for_p}
    Cx= p^*
\end{equation}

\(C\) corresponds to equations
(\ref{eqn:condition_for_tilde_p1}-\ref{eqn:condition_for_tilde_p3}) and is
given by:

\begin{equation}\label{eqn:definition_of_C}
    C =
    \begin{bmatrix}
        R - P & R- P \\
        S - P & T- P \\
        T - P & S- P \\
    \end{bmatrix}
\end{equation}

Note that in general, equation (\ref{eqn:linear_algebraic_equation_for_p}) will
not necessarily have a solution. From the Rouch\'{e}-Capelli theorem if there is
a solution it is unique as \(\text{rank}(C)=2\) which is the dimension of the
variable \(x\). The best fitting \(x\) is found by minimizing:

\begin{equation}\label{eqn:r_squared}
    \text{SSError} = \|C x- p^*\|_2^2 = \sum_{i=1}^{3}\left((C\bar x)_i-p_i^*\right)^2
\end{equation}

Note that \(\text{SSError}\), which is the square of the Frobenius
norm~\cite{Golub2013}, becomes a measure of how close a strategy is to being an
extortionate strategy. Suspicion
of extortion then corresponds to a threshold on \(\text{SSError}\).

By observing interactions (human or otherwise), their memory one representation
can be inferred and this approach can be used to recognise extortionate
behaviour. The notion of comparing theoretic and actual plays of the IPD is not
novel, see for example~\cite{Rand2013}. Immediately it is noted that if the
environment is noisy~\cite{Wu1995} then no strategy can be considered to be
extortionate as \(p_4>0\).

In the next section, this idea will be illustrated by observing the interactions
that take place in a computer based tournament of the IPD\@.

\section{Numerical experiments}\label{sec:numerical-experiments}

In~\cite{Stewart2012} results from a tournament with
\input{./assets/tex/number_of_stewart_plotkin_strategies/main.tex} strategies,
was presented with specific consideration given to ZD strategies. This
tournament is reproduced here using the Axelrod-Python
project~\cite{Knight2016}. To obtain a good measure of the corresponding
transition rates for each strategy all matches have been run for
\input{assets/tex/number_of_turns/main.tex} turns and every match has been
repeated \input{assets/tex/number_of_repetitions/main.tex} times. All of this
interaction data is available at~\cite{vincent_knight_2018_1297075}. A good
match between the inferred Markov chain and the state distribution of the actual
interactions has been verified. Data for this is presented in the supplementary
materials.

Figure~\ref{fig:SSError_overall_in_stewart_plotkin} shows the \(\text{SSError}\)
values for all the strategies in the tournament, as reported
in~\cite{Stewart2012} the extortionate strategy (which has an expected
\(\text{SSError}\) approximately 0) gains a large number of wins.

\begin{figure}[!htbp]
    \centering
    \includegraphics[width=.8\textwidth]{./assets/img/SSError_overall_in_stewart_plotkin/main.pdf}
    \caption{\(\text{SSError}\) and state probabilities for the strategies
        of~\cite{Stewart2012}, ordered both by number of wins and overall score.
        Note that \(P(DC)\) is not shown as it corresponds to the transpose of
        \(P(CD)\). Cooperator and Defector are omitted as they do not visit all
        the states.}
    \label{fig:SSError_overall_in_stewart_plotkin}
\end{figure}

Here, the work of~\cite{Stewart2012} is extended by investigating a tournament
with \input{assets/tex/number_of_full_strategies/main.tex}
strategies.

The results of this analysis are shown in
Figure~\ref{fig:SSError_and_probabilities_in_full}. The top ranking strategies
by number of wins seem to be extortionate (but not against all strategies) and
it can be seen that a small sub group of strategies achieve mutual defection.
All the top ranking strategies according to score achieve mutual cooperation and
do not extort each other, however they
\textbf{do} exhibit extortionate behaviour towards a number of the lower ranking
strategies.

\begin{figure}[!htbp]
    \centering
    \includegraphics[width=.8\textwidth]{./assets/img/SSError_and_probabilities_in_full/main.pdf}
    \caption{\(\text{SSError}\) for the strategies for the full tournament. Only
    strategy interactions for which \(p_4=0\) and \(\chi>1\) are displayed.}
    \label{fig:SSError_and_probabilities_in_full}
\end{figure}

\section{Conclusion}\label{sec:conclusion}

This work defines an approach to measure whether or not a player is playing a
strategy that corresponds to an extortionate strategy as defined
in~\cite{Press2012}: a mathematical model for suspicion. Indeed, all
extortionate strategies have been
 classified as lying on a triangular plane.
This rigorous classification fails to be robust to small measurement error, thus
a statistical approach is proposed.
This is done through a linear algebraic approach for approximating the solution
of a linear system. Using this, a large number of pairwise interactions is
simulated and in fact very few strategies are found to act extortionately.

The work of~\cite{Press2012}, whilst showing that a clever approach to taking
advantage of another memory one strategy exists: this is incomplete. Whilst the
elegance of this result is very attractive, just as the simplicity of the
victory of Tit For Tat in Axelrod's original tournaments was, it is incomplete.
Extortionate strategies achieve a high number of wins but they do not
achieve a high score which corresponds to the fitness landscape in an
evolutionary sense. From the large number of interactions a payoff matrix \(S\)
can be measured where \(S_{ij}\) denotes the score (using standard values of
\((R, S, T, P) = (3, 0, 5, 1)\)) of the \(i\)th strategy
against the \(j\)th strategy. Using this, the replicator equation
describes the evolution of the system based on a population density fitness
function:

\begin{equation}\label{eqn:replicator_dynamics}
    \frac{dx}{dt} = x(S-x^TS x)
\end{equation}

Equation (\ref{eqn:replicator_dynamics}) is solved numerically through an
integration technique described in~\cite{Petzold1983} and
Figure~\ref{fig:replicator_dynamics} shows the evolution of the distribution of
the system: the various strategies are ranked by scores. It is clear to see that
only the high ranking strategies survive the evolutionary process (in fact,
only \input{./assets/img/replicator_dynamics/main.tex}
have a final distribution greater than \(10 ^ {-2}\)). This confirms the
findings of~\cite{Moran1707} in which sophisticated strategies resist
evolutionary invasion of shorter memory strategies. Recalling
Figure~\ref{fig:SSError_and_probabilities_in_full} this demonstrates that:

\begin{itemize}
    \item Cooperation emerges through the evolutionary process: the high scoring
        strategies do not exhibit extortionate behaviour towards each other.
    \item Extortionate strategies do not survive the evolutionary process.
\end{itemize}

\begin{figure}[!htbp]
    \centering
    \includegraphics[width=.8\textwidth]{./assets/img/replicator_dynamics/main.pdf}
    \caption{Numerical simulation of the replicator equation
    (\ref{eqn:replicator_dynamics}): strategies are ordered by score, only the strategies with a high score survive the evolutionary process.}
    \label{fig:replicator_dynamics}
\end{figure}

This work can be used to classify plays of the IPD\@: data can be collected from
actual interactions (in lab or in the field). Furthermore, this allows for a
classification method similar to the notion of fingerprinting presented
in~\cite{Ashlock2008}. Trained strategies can potentially be classified as
extortionate or not or it could be possible to even constrain the reinforcement
learning approaches that are becoming prevalent in the literature.
Alternatively, this mathematical approach for recognising extortion could be
used in sophisticated strategies to defend against invasion. Arguably, some of
the strategies considered here exhibit this behaviour, indeed as described
in~\cite{Harper2017}, the top ranking strategies in the full tournament are
obtained using evolutionary reinforcement learning techniques, thus, suspicion
of extortionate behaviour could in fact be an evolutionary trait.

\section*{Acknowledgements}

The following open source software libraries were used in this research:

\begin{itemize}
    \item The Axelrod ~\cite{Knight2016, Knight2018} library (IPD strategies and
        tournaments).
    \item The sympy library~\cite{Meurer2017} (verification of all symbolic
        calculations).
    \item The matplotlib~\cite{Droettboom2018} library (visualisation).
    \item The pandas~\cite{Structures2010}, dask~\cite{Dask2016} and
        NumPy~\cite{Oliphant2015} libraries (data manipulation).
    \item The SciPy~\cite{Jones2001} library (numerical integration of the
        replicator equation).
\end{itemize}

This work was performed using the computational facilities of the Advanced
Research Computing @ Cardiff (ARCCA) Division, Cardiff University.

\printbibliography

\newpage
\section*{Supplementary materials}

\includepdf{assets/pdf/proof_of_form_of_extortionate_strategies/main.pdf}

\newpage

Using the pair wise interactions the transition rates \(p,
q\) can be measured and the steady state probabilities inferred and compared to
the actual probabilities of each state.
This is done numerically by computing the singular eigenvector of the
matrix \(A\) \cite{Stewart2009}:

\[
    A =
    \begin{bmatrix}
        p_1 q_1 & p_1 (1 - q_1) & (1 - p_1) q_1 & (1 -p_1) (1 - q_1) \\
        p_2 q_2 & p_2 (1 - q_2) & (1 - p_2) q_2 & (1 -p_2) (1 - q_2) \\
        p_3 q_3 & p_3 (1 - q_3) & (1 - p_3) q_3 & (1 -p_3) (1 - q_3) \\
        p_4 q_4 & p_4 (1 - q_4) & (1 - p_4) q_4 & (1 -p_4) (1 - q_4) \\
    \end{bmatrix}
\]

Figure~\ref{fig:computed_probabilities_vs_theoretic_probabilities} shows a
regression line fitted to every pairwise interaction with a reported
\(\text{SSError}\) value (pairwise interactions with missing states were
omitted). This serves to validate the approach: a part from some edge cases the
relationship is consistent.

\begin{figure}[!htbp]
    \centering
    \includegraphics[width=.8\textwidth]{./assets/img/computed_probabilities_vs_theoretic_probabilities/main.pdf}
    \caption{The
        relationship between the steady state probabilities inferred from the
        measured transitions and the actual steady state probabilities. A linear
        regression line is included validating the approach.}
    \label{fig:computed_probabilities_vs_theoretic_probabilities}
\end{figure}


\end{document}
 strategies,
was presented with specific consideration given to ZD strategies. This
tournament is reproduced here using the Axelrod-Python
project~\cite{Knight2016}. To obtain a good measure of the corresponding
transition rates for each strategy all matches have been run for
\documentclass[a4paper]{article}

\usepackage{amsmath}
\usepackage{amssymb}
\usepackage[margin=1.5cm,
            includefoot,
            footskip=30pt]{geometry}
\usepackage{layout}
\usepackage{graphicx}
\usepackage{subcaption}

\usepackage{biblatex}
\usepackage{pdfpages}

\bibliography{main.bib}

\title{Suspicion: Recognising and evaluating the effectiveness
       of extortion in the Iterated Prisoner's Dilemma}
\author{Vincent A. Knight \and Nikoleta E. Glynatsi}
\date{\today}



\begin{document}

\maketitle

\begin{abstract}
    The Iterated Prisoner's Dilemma is a model for rational and evolutionary
    interactive behaviour. It has applications both in the study of human social
    behaviour as well as in biology.
    It is used to understand when and how a rational individual might
    accept an immediate cost to their own utility for the direct benefit of
    another.

    Much attention has been given to a class of strategies called
    Zero Determinant strategies. It has been theoretically shown that these
    strategies can ``extort'' any player.

    In this work, an approach to identify if observed strategies are playing in
    an extortionate way is described. Furthermore, experimental analysis of
    a large tournament with \input{assets/tex/number_of_full_strategies/main.tex}
    strategies is considered. In this setting
    the most highly performing strategies do not play in an extortionate way
    against each other but do against lower performing strategies.
    This suggests that whilst the theory of Zero Determinant strategies
    indicates that memory is not of fundamental importance to the evolution of
    cooperative behaviour, this is incomplete.
\end{abstract}

\section{Introduction}\label{sec:introduction}

Agent based game theoretic models have become a stalwart of the underpinning
mathematics of interactive behaviours. One of the major pieces of work
in this area is the pair of original computer tournaments run by Robert
Axelrod~\cite{Axelrod1980, Axelrod1980a}. These tournaments pitted submitted
computer strategies against each other in plays of the Iterated Prisoner's
Dilemma. A common game where agents can choose to pay a slight cost to their
immediate utility in the hope of building a reputation. This has been used in
economic and evolutionary game theory to understand the evolution of cooperative
behaviour.

Recently, a class of strategies was described in~\cite{Press2012} that can
provably extort any given opponent. In~\cite{Hilbe2013, Moran1707} some
questions have already been asked about the true effectiveness of these
strategies in an evolutionary setting. Here another question is asked: is it
possible to recognise this extortionate behaviour? A mathematical procedure for
suspicion is presented: in the same way that the continued actions of an
extortionate individual might raise suspicion.

This work makes use of the Axelrod Python library~\cite{Knight2018, Knight2016}
with a large number of Prisoner Dilemma strategies available to give an
extensive numerical example of the ideas presented.  The approach is presented
in Section~\ref{sec:delta-zd-strategies}.  All of the code and data discussed
in Section~\ref{sec:numerical-experiments} is open sourced, archived and
written according to best scientific principles~\cite{Wilson2014}. The data
archive can be found at~\cite{vincent_knight_2018_1297075}.

\section{Recognising Extortion}\label{sec:delta-zd-strategies}

In~\cite{Press2012}, given a match between 2 memory-one strategies, the concept
of Zero Determinant (ZD) strategies is introduced. The main result of that paper
shows that given two memory one players \(p, q\in\mathbb{R}^4\) a linear
relationship between the players' scores could be forced by one of the players.

Using the notation of~\cite{Press2012}, assuming the utilities for player \(p\)
are given by \(S_x=(R, S, T, P)\) and for player \(q\) by \(S_y=(R, T, S, P)\)
and that the stationary scores of each player is given by \(S_X\) and \(S_Y\)
respectively. The main result of~\cite{Press2012} is that if

\begin{equation}\label{eqn:linear_relationship_for_p}
    \tilde p=\alpha S_x + \beta S_y + \gamma
\end{equation}

or

\begin{equation}\label{eqn:linear_relationship_for_q}
    \tilde q=\alpha S_x + \beta S_y + \gamma
\end{equation}

where \(\tilde p = (1 - p_1, 1 - p_2, p_3, p_4)\) and
\(\tilde q = (1 - q_1, 1 - q_2, q_3, q_4)\) then:

\begin{equation}
    \alpha S_X + \beta S_Y + \gamma = 0
\end{equation}

In~\cite{Press2012} a particular type of ZD strategy is defined: extortionate
strategies. If:

\begin{equation}\label{eqn:constraint_for_extortion}
    \gamma = - P(\alpha + \beta)
\end{equation}

then the player can ensure they get a score \(\chi\) times
larger than the opponent. This extortion coefficient is given by:

\begin{equation}\label{eqn:definition_of_chi}
    \chi=\frac{-\beta}{\alpha}
\end{equation}

Thus, if (\ref{eqn:constraint_for_extortion}) holds and \(\chi >1\) a player is
said to extort their opponent.
Here, the reverse problem is considered: given a
\(p\in\mathbb{R}^4\) how does one identify \(\alpha, \beta\) if they
exist and is the strategy in fact acting in an extortionate way?

These conditions correspond to:

\begin{align}
    \tilde p_1 & = \alpha R + \beta R - P (\alpha + \beta)
            \label{eqn:condition_for_tilde_p1}\\
    \tilde p_2 & = \alpha S + \beta T - P (\alpha + \beta)
            \label{eqn:condition_for_tilde_p2}\\
    \tilde p_3 & = \alpha T + \beta S - P (\alpha + \beta)
            \label{eqn:condition_for_tilde_p3}\\
    \tilde p_4 & = \alpha P + \beta P - P (\alpha + \beta)
            \label{eqn:condition_for_tilde_p4}
\end{align}

Equation (\ref{eqn:condition_for_tilde_p4}) ensures that \(p_4=\tilde p_4=0\).
Equations (\ref{eqn:condition_for_tilde_p1}-\ref{eqn:condition_for_tilde_p3})
can be used to eliminate \(\alpha, \beta\), giving:

\begin{equation}\label{eqn:planar_definition_of_extortion}
    \tilde p_1 = \frac{(R - P)(\tilde p_2 + \tilde p_3)}{S + T - 2P}
\end{equation}

with:

\begin{equation}\label{eqn:definition_of_chi}
    \chi = \frac{\tilde p_2 (P - T) + \tilde p_3 (S - P)}
                {\tilde p_2 (P - S) + \tilde p_3 (T - P)}
\end{equation}

Given a strategy \(p\in\mathbb{R}^{4\times 1}\) equations
(\ref{eqn:condition_for_tilde_p4}), (\ref{eqn:planar_definition_of_extortion}-\ref{eqn:definition_of_chi}) can be used to check if
a strategy is extortionate. The conditions correspond to:

\begin{align}
    p_1 & = \frac{(R-P)(p_2 + p_3) - R + T + S - P}{S + T - 2P}
     \label{eqn:condition_for_p1}\\
    p_4 & = 0 \label{eqn:condition_for_p4}\\
    1 & > p_2 + p_3\label{eqn:condition_for_chi}
\end{align}

The algebraic steps necessary to prove these results are available in the
supporting materials.

All extortionate strategies reside on a triangular (\ref{eqn:condition_for_chi})
plane (\ref{eqn:condition_for_p1}) in 3 dimensions (\ref{eqn:condition_for_p4}).
Using this formulation it can be seen that a necessary (but not sufficient)
condition for an extortionate strategy is that it cooperates on average less
than 50\% of the time when in a state of disagreement with the opponent.

As an example, consider the known extortionate strategy \(p=(8 / 9, 1 / 2, 1 /
3, 0)\) from~\cite{Stewart2012} which is referred to as \texttt{Extort-2}. In
this case, for the standard values of \((R, T, S, P)\) constraint
(\ref{eqn:condition_for_p1}) corresponds to:

\begin{equation}
    p_1 = \frac{2(p_2 + p_3) + 1}{3}
\end{equation}

It is clear that in this case all constraints hold.

This approach could in fact be used to confirm that a given strategy is acting
in an extortionate manner even if it is not a memory one strategy. However, in
practice, if a closed form for \(p\) is not known, then due to measurement
and/or numerical error this would not work.

This problem can be written in the following linear algebraic form where
\(x=(\alpha, \beta)\)
and \(p^*=(\tilde p_1 - 1, tilde_2 - 1, p_3)\):

\begin{equation}\label{eqn:linear_algebraic_equation_for_p}
    Cx= p^*
\end{equation}

\(C\) corresponds to equations
(\ref{eqn:condition_for_tilde_p1}-\ref{eqn:condition_for_tilde_p3}) and is
given by:

\begin{equation}\label{eqn:definition_of_C}
    C =
    \begin{bmatrix}
        R - P & R- P \\
        S - P & T- P \\
        T - P & S- P \\
    \end{bmatrix}
\end{equation}

Note that in general, equation (\ref{eqn:linear_algebraic_equation_for_p}) will
not necessarily have a solution. From the Rouch\'{e}-Capelli theorem if there is
a solution it is unique as \(\text{rank}(C)=2\) which is the dimension of the
variable \(x\). The best fitting \(x\) is found by minimizing:

\begin{equation}\label{eqn:r_squared}
    \text{SSError} = \|C x- p^*\|_2^2 = \sum_{i=1}^{3}\left((C\bar x)_i-p_i^*\right)^2
\end{equation}

Note that \(\text{SSError}\), which is the square of the Frobenius
norm~\cite{Golub2013}, becomes a measure of how close a strategy is to being an
extortionate strategy. Suspicion
of extortion then corresponds to a threshold on \(\text{SSError}\).

By observing interactions (human or otherwise), their memory one representation
can be inferred and this approach can be used to recognise extortionate
behaviour. The notion of comparing theoretic and actual plays of the IPD is not
novel, see for example~\cite{Rand2013}. Immediately it is noted that if the
environment is noisy~\cite{Wu1995} then no strategy can be considered to be
extortionate as \(p_4>0\).

In the next section, this idea will be illustrated by observing the interactions
that take place in a computer based tournament of the IPD\@.

\section{Numerical experiments}\label{sec:numerical-experiments}

In~\cite{Stewart2012} results from a tournament with
\input{./assets/tex/number_of_stewart_plotkin_strategies/main.tex} strategies,
was presented with specific consideration given to ZD strategies. This
tournament is reproduced here using the Axelrod-Python
project~\cite{Knight2016}. To obtain a good measure of the corresponding
transition rates for each strategy all matches have been run for
\input{assets/tex/number_of_turns/main.tex} turns and every match has been
repeated \input{assets/tex/number_of_repetitions/main.tex} times. All of this
interaction data is available at~\cite{vincent_knight_2018_1297075}. A good
match between the inferred Markov chain and the state distribution of the actual
interactions has been verified. Data for this is presented in the supplementary
materials.

Figure~\ref{fig:SSError_overall_in_stewart_plotkin} shows the \(\text{SSError}\)
values for all the strategies in the tournament, as reported
in~\cite{Stewart2012} the extortionate strategy (which has an expected
\(\text{SSError}\) approximately 0) gains a large number of wins.

\begin{figure}[!htbp]
    \centering
    \includegraphics[width=.8\textwidth]{./assets/img/SSError_overall_in_stewart_plotkin/main.pdf}
    \caption{\(\text{SSError}\) and state probabilities for the strategies
        of~\cite{Stewart2012}, ordered both by number of wins and overall score.
        Note that \(P(DC)\) is not shown as it corresponds to the transpose of
        \(P(CD)\). Cooperator and Defector are omitted as they do not visit all
        the states.}
    \label{fig:SSError_overall_in_stewart_plotkin}
\end{figure}

Here, the work of~\cite{Stewart2012} is extended by investigating a tournament
with \input{assets/tex/number_of_full_strategies/main.tex}
strategies.

The results of this analysis are shown in
Figure~\ref{fig:SSError_and_probabilities_in_full}. The top ranking strategies
by number of wins seem to be extortionate (but not against all strategies) and
it can be seen that a small sub group of strategies achieve mutual defection.
All the top ranking strategies according to score achieve mutual cooperation and
do not extort each other, however they
\textbf{do} exhibit extortionate behaviour towards a number of the lower ranking
strategies.

\begin{figure}[!htbp]
    \centering
    \includegraphics[width=.8\textwidth]{./assets/img/SSError_and_probabilities_in_full/main.pdf}
    \caption{\(\text{SSError}\) for the strategies for the full tournament. Only
    strategy interactions for which \(p_4=0\) and \(\chi>1\) are displayed.}
    \label{fig:SSError_and_probabilities_in_full}
\end{figure}

\section{Conclusion}\label{sec:conclusion}

This work defines an approach to measure whether or not a player is playing a
strategy that corresponds to an extortionate strategy as defined
in~\cite{Press2012}: a mathematical model for suspicion. Indeed, all
extortionate strategies have been
 classified as lying on a triangular plane.
This rigorous classification fails to be robust to small measurement error, thus
a statistical approach is proposed.
This is done through a linear algebraic approach for approximating the solution
of a linear system. Using this, a large number of pairwise interactions is
simulated and in fact very few strategies are found to act extortionately.

The work of~\cite{Press2012}, whilst showing that a clever approach to taking
advantage of another memory one strategy exists: this is incomplete. Whilst the
elegance of this result is very attractive, just as the simplicity of the
victory of Tit For Tat in Axelrod's original tournaments was, it is incomplete.
Extortionate strategies achieve a high number of wins but they do not
achieve a high score which corresponds to the fitness landscape in an
evolutionary sense. From the large number of interactions a payoff matrix \(S\)
can be measured where \(S_{ij}\) denotes the score (using standard values of
\((R, S, T, P) = (3, 0, 5, 1)\)) of the \(i\)th strategy
against the \(j\)th strategy. Using this, the replicator equation
describes the evolution of the system based on a population density fitness
function:

\begin{equation}\label{eqn:replicator_dynamics}
    \frac{dx}{dt} = x(S-x^TS x)
\end{equation}

Equation (\ref{eqn:replicator_dynamics}) is solved numerically through an
integration technique described in~\cite{Petzold1983} and
Figure~\ref{fig:replicator_dynamics} shows the evolution of the distribution of
the system: the various strategies are ranked by scores. It is clear to see that
only the high ranking strategies survive the evolutionary process (in fact,
only \input{./assets/img/replicator_dynamics/main.tex}
have a final distribution greater than \(10 ^ {-2}\)). This confirms the
findings of~\cite{Moran1707} in which sophisticated strategies resist
evolutionary invasion of shorter memory strategies. Recalling
Figure~\ref{fig:SSError_and_probabilities_in_full} this demonstrates that:

\begin{itemize}
    \item Cooperation emerges through the evolutionary process: the high scoring
        strategies do not exhibit extortionate behaviour towards each other.
    \item Extortionate strategies do not survive the evolutionary process.
\end{itemize}

\begin{figure}[!htbp]
    \centering
    \includegraphics[width=.8\textwidth]{./assets/img/replicator_dynamics/main.pdf}
    \caption{Numerical simulation of the replicator equation
    (\ref{eqn:replicator_dynamics}): strategies are ordered by score, only the strategies with a high score survive the evolutionary process.}
    \label{fig:replicator_dynamics}
\end{figure}

This work can be used to classify plays of the IPD\@: data can be collected from
actual interactions (in lab or in the field). Furthermore, this allows for a
classification method similar to the notion of fingerprinting presented
in~\cite{Ashlock2008}. Trained strategies can potentially be classified as
extortionate or not or it could be possible to even constrain the reinforcement
learning approaches that are becoming prevalent in the literature.
Alternatively, this mathematical approach for recognising extortion could be
used in sophisticated strategies to defend against invasion. Arguably, some of
the strategies considered here exhibit this behaviour, indeed as described
in~\cite{Harper2017}, the top ranking strategies in the full tournament are
obtained using evolutionary reinforcement learning techniques, thus, suspicion
of extortionate behaviour could in fact be an evolutionary trait.

\section*{Acknowledgements}

The following open source software libraries were used in this research:

\begin{itemize}
    \item The Axelrod ~\cite{Knight2016, Knight2018} library (IPD strategies and
        tournaments).
    \item The sympy library~\cite{Meurer2017} (verification of all symbolic
        calculations).
    \item The matplotlib~\cite{Droettboom2018} library (visualisation).
    \item The pandas~\cite{Structures2010}, dask~\cite{Dask2016} and
        NumPy~\cite{Oliphant2015} libraries (data manipulation).
    \item The SciPy~\cite{Jones2001} library (numerical integration of the
        replicator equation).
\end{itemize}

This work was performed using the computational facilities of the Advanced
Research Computing @ Cardiff (ARCCA) Division, Cardiff University.

\printbibliography

\newpage
\section*{Supplementary materials}

\includepdf{assets/pdf/proof_of_form_of_extortionate_strategies/main.pdf}

\newpage

Using the pair wise interactions the transition rates \(p,
q\) can be measured and the steady state probabilities inferred and compared to
the actual probabilities of each state.
This is done numerically by computing the singular eigenvector of the
matrix \(A\) \cite{Stewart2009}:

\[
    A =
    \begin{bmatrix}
        p_1 q_1 & p_1 (1 - q_1) & (1 - p_1) q_1 & (1 -p_1) (1 - q_1) \\
        p_2 q_2 & p_2 (1 - q_2) & (1 - p_2) q_2 & (1 -p_2) (1 - q_2) \\
        p_3 q_3 & p_3 (1 - q_3) & (1 - p_3) q_3 & (1 -p_3) (1 - q_3) \\
        p_4 q_4 & p_4 (1 - q_4) & (1 - p_4) q_4 & (1 -p_4) (1 - q_4) \\
    \end{bmatrix}
\]

Figure~\ref{fig:computed_probabilities_vs_theoretic_probabilities} shows a
regression line fitted to every pairwise interaction with a reported
\(\text{SSError}\) value (pairwise interactions with missing states were
omitted). This serves to validate the approach: a part from some edge cases the
relationship is consistent.

\begin{figure}[!htbp]
    \centering
    \includegraphics[width=.8\textwidth]{./assets/img/computed_probabilities_vs_theoretic_probabilities/main.pdf}
    \caption{The
        relationship between the steady state probabilities inferred from the
        measured transitions and the actual steady state probabilities. A linear
        regression line is included validating the approach.}
    \label{fig:computed_probabilities_vs_theoretic_probabilities}
\end{figure}


\end{document}
 turns and every match has been
repeated \documentclass[a4paper]{article}

\usepackage{amsmath}
\usepackage{amssymb}
\usepackage[margin=1.5cm,
            includefoot,
            footskip=30pt]{geometry}
\usepackage{layout}
\usepackage{graphicx}
\usepackage{subcaption}

\usepackage{biblatex}
\usepackage{pdfpages}

\bibliography{main.bib}

\title{Suspicion: Recognising and evaluating the effectiveness
       of extortion in the Iterated Prisoner's Dilemma}
\author{Vincent A. Knight \and Nikoleta E. Glynatsi}
\date{\today}



\begin{document}

\maketitle

\begin{abstract}
    The Iterated Prisoner's Dilemma is a model for rational and evolutionary
    interactive behaviour. It has applications both in the study of human social
    behaviour as well as in biology.
    It is used to understand when and how a rational individual might
    accept an immediate cost to their own utility for the direct benefit of
    another.

    Much attention has been given to a class of strategies called
    Zero Determinant strategies. It has been theoretically shown that these
    strategies can ``extort'' any player.

    In this work, an approach to identify if observed strategies are playing in
    an extortionate way is described. Furthermore, experimental analysis of
    a large tournament with \input{assets/tex/number_of_full_strategies/main.tex}
    strategies is considered. In this setting
    the most highly performing strategies do not play in an extortionate way
    against each other but do against lower performing strategies.
    This suggests that whilst the theory of Zero Determinant strategies
    indicates that memory is not of fundamental importance to the evolution of
    cooperative behaviour, this is incomplete.
\end{abstract}

\section{Introduction}\label{sec:introduction}

Agent based game theoretic models have become a stalwart of the underpinning
mathematics of interactive behaviours. One of the major pieces of work
in this area is the pair of original computer tournaments run by Robert
Axelrod~\cite{Axelrod1980, Axelrod1980a}. These tournaments pitted submitted
computer strategies against each other in plays of the Iterated Prisoner's
Dilemma. A common game where agents can choose to pay a slight cost to their
immediate utility in the hope of building a reputation. This has been used in
economic and evolutionary game theory to understand the evolution of cooperative
behaviour.

Recently, a class of strategies was described in~\cite{Press2012} that can
provably extort any given opponent. In~\cite{Hilbe2013, Moran1707} some
questions have already been asked about the true effectiveness of these
strategies in an evolutionary setting. Here another question is asked: is it
possible to recognise this extortionate behaviour? A mathematical procedure for
suspicion is presented: in the same way that the continued actions of an
extortionate individual might raise suspicion.

This work makes use of the Axelrod Python library~\cite{Knight2018, Knight2016}
with a large number of Prisoner Dilemma strategies available to give an
extensive numerical example of the ideas presented.  The approach is presented
in Section~\ref{sec:delta-zd-strategies}.  All of the code and data discussed
in Section~\ref{sec:numerical-experiments} is open sourced, archived and
written according to best scientific principles~\cite{Wilson2014}. The data
archive can be found at~\cite{vincent_knight_2018_1297075}.

\section{Recognising Extortion}\label{sec:delta-zd-strategies}

In~\cite{Press2012}, given a match between 2 memory-one strategies, the concept
of Zero Determinant (ZD) strategies is introduced. The main result of that paper
shows that given two memory one players \(p, q\in\mathbb{R}^4\) a linear
relationship between the players' scores could be forced by one of the players.

Using the notation of~\cite{Press2012}, assuming the utilities for player \(p\)
are given by \(S_x=(R, S, T, P)\) and for player \(q\) by \(S_y=(R, T, S, P)\)
and that the stationary scores of each player is given by \(S_X\) and \(S_Y\)
respectively. The main result of~\cite{Press2012} is that if

\begin{equation}\label{eqn:linear_relationship_for_p}
    \tilde p=\alpha S_x + \beta S_y + \gamma
\end{equation}

or

\begin{equation}\label{eqn:linear_relationship_for_q}
    \tilde q=\alpha S_x + \beta S_y + \gamma
\end{equation}

where \(\tilde p = (1 - p_1, 1 - p_2, p_3, p_4)\) and
\(\tilde q = (1 - q_1, 1 - q_2, q_3, q_4)\) then:

\begin{equation}
    \alpha S_X + \beta S_Y + \gamma = 0
\end{equation}

In~\cite{Press2012} a particular type of ZD strategy is defined: extortionate
strategies. If:

\begin{equation}\label{eqn:constraint_for_extortion}
    \gamma = - P(\alpha + \beta)
\end{equation}

then the player can ensure they get a score \(\chi\) times
larger than the opponent. This extortion coefficient is given by:

\begin{equation}\label{eqn:definition_of_chi}
    \chi=\frac{-\beta}{\alpha}
\end{equation}

Thus, if (\ref{eqn:constraint_for_extortion}) holds and \(\chi >1\) a player is
said to extort their opponent.
Here, the reverse problem is considered: given a
\(p\in\mathbb{R}^4\) how does one identify \(\alpha, \beta\) if they
exist and is the strategy in fact acting in an extortionate way?

These conditions correspond to:

\begin{align}
    \tilde p_1 & = \alpha R + \beta R - P (\alpha + \beta)
            \label{eqn:condition_for_tilde_p1}\\
    \tilde p_2 & = \alpha S + \beta T - P (\alpha + \beta)
            \label{eqn:condition_for_tilde_p2}\\
    \tilde p_3 & = \alpha T + \beta S - P (\alpha + \beta)
            \label{eqn:condition_for_tilde_p3}\\
    \tilde p_4 & = \alpha P + \beta P - P (\alpha + \beta)
            \label{eqn:condition_for_tilde_p4}
\end{align}

Equation (\ref{eqn:condition_for_tilde_p4}) ensures that \(p_4=\tilde p_4=0\).
Equations (\ref{eqn:condition_for_tilde_p1}-\ref{eqn:condition_for_tilde_p3})
can be used to eliminate \(\alpha, \beta\), giving:

\begin{equation}\label{eqn:planar_definition_of_extortion}
    \tilde p_1 = \frac{(R - P)(\tilde p_2 + \tilde p_3)}{S + T - 2P}
\end{equation}

with:

\begin{equation}\label{eqn:definition_of_chi}
    \chi = \frac{\tilde p_2 (P - T) + \tilde p_3 (S - P)}
                {\tilde p_2 (P - S) + \tilde p_3 (T - P)}
\end{equation}

Given a strategy \(p\in\mathbb{R}^{4\times 1}\) equations
(\ref{eqn:condition_for_tilde_p4}), (\ref{eqn:planar_definition_of_extortion}-\ref{eqn:definition_of_chi}) can be used to check if
a strategy is extortionate. The conditions correspond to:

\begin{align}
    p_1 & = \frac{(R-P)(p_2 + p_3) - R + T + S - P}{S + T - 2P}
     \label{eqn:condition_for_p1}\\
    p_4 & = 0 \label{eqn:condition_for_p4}\\
    1 & > p_2 + p_3\label{eqn:condition_for_chi}
\end{align}

The algebraic steps necessary to prove these results are available in the
supporting materials.

All extortionate strategies reside on a triangular (\ref{eqn:condition_for_chi})
plane (\ref{eqn:condition_for_p1}) in 3 dimensions (\ref{eqn:condition_for_p4}).
Using this formulation it can be seen that a necessary (but not sufficient)
condition for an extortionate strategy is that it cooperates on average less
than 50\% of the time when in a state of disagreement with the opponent.

As an example, consider the known extortionate strategy \(p=(8 / 9, 1 / 2, 1 /
3, 0)\) from~\cite{Stewart2012} which is referred to as \texttt{Extort-2}. In
this case, for the standard values of \((R, T, S, P)\) constraint
(\ref{eqn:condition_for_p1}) corresponds to:

\begin{equation}
    p_1 = \frac{2(p_2 + p_3) + 1}{3}
\end{equation}

It is clear that in this case all constraints hold.

This approach could in fact be used to confirm that a given strategy is acting
in an extortionate manner even if it is not a memory one strategy. However, in
practice, if a closed form for \(p\) is not known, then due to measurement
and/or numerical error this would not work.

This problem can be written in the following linear algebraic form where
\(x=(\alpha, \beta)\)
and \(p^*=(\tilde p_1 - 1, tilde_2 - 1, p_3)\):

\begin{equation}\label{eqn:linear_algebraic_equation_for_p}
    Cx= p^*
\end{equation}

\(C\) corresponds to equations
(\ref{eqn:condition_for_tilde_p1}-\ref{eqn:condition_for_tilde_p3}) and is
given by:

\begin{equation}\label{eqn:definition_of_C}
    C =
    \begin{bmatrix}
        R - P & R- P \\
        S - P & T- P \\
        T - P & S- P \\
    \end{bmatrix}
\end{equation}

Note that in general, equation (\ref{eqn:linear_algebraic_equation_for_p}) will
not necessarily have a solution. From the Rouch\'{e}-Capelli theorem if there is
a solution it is unique as \(\text{rank}(C)=2\) which is the dimension of the
variable \(x\). The best fitting \(x\) is found by minimizing:

\begin{equation}\label{eqn:r_squared}
    \text{SSError} = \|C x- p^*\|_2^2 = \sum_{i=1}^{3}\left((C\bar x)_i-p_i^*\right)^2
\end{equation}

Note that \(\text{SSError}\), which is the square of the Frobenius
norm~\cite{Golub2013}, becomes a measure of how close a strategy is to being an
extortionate strategy. Suspicion
of extortion then corresponds to a threshold on \(\text{SSError}\).

By observing interactions (human or otherwise), their memory one representation
can be inferred and this approach can be used to recognise extortionate
behaviour. The notion of comparing theoretic and actual plays of the IPD is not
novel, see for example~\cite{Rand2013}. Immediately it is noted that if the
environment is noisy~\cite{Wu1995} then no strategy can be considered to be
extortionate as \(p_4>0\).

In the next section, this idea will be illustrated by observing the interactions
that take place in a computer based tournament of the IPD\@.

\section{Numerical experiments}\label{sec:numerical-experiments}

In~\cite{Stewart2012} results from a tournament with
\input{./assets/tex/number_of_stewart_plotkin_strategies/main.tex} strategies,
was presented with specific consideration given to ZD strategies. This
tournament is reproduced here using the Axelrod-Python
project~\cite{Knight2016}. To obtain a good measure of the corresponding
transition rates for each strategy all matches have been run for
\input{assets/tex/number_of_turns/main.tex} turns and every match has been
repeated \input{assets/tex/number_of_repetitions/main.tex} times. All of this
interaction data is available at~\cite{vincent_knight_2018_1297075}. A good
match between the inferred Markov chain and the state distribution of the actual
interactions has been verified. Data for this is presented in the supplementary
materials.

Figure~\ref{fig:SSError_overall_in_stewart_plotkin} shows the \(\text{SSError}\)
values for all the strategies in the tournament, as reported
in~\cite{Stewart2012} the extortionate strategy (which has an expected
\(\text{SSError}\) approximately 0) gains a large number of wins.

\begin{figure}[!htbp]
    \centering
    \includegraphics[width=.8\textwidth]{./assets/img/SSError_overall_in_stewart_plotkin/main.pdf}
    \caption{\(\text{SSError}\) and state probabilities for the strategies
        of~\cite{Stewart2012}, ordered both by number of wins and overall score.
        Note that \(P(DC)\) is not shown as it corresponds to the transpose of
        \(P(CD)\). Cooperator and Defector are omitted as they do not visit all
        the states.}
    \label{fig:SSError_overall_in_stewart_plotkin}
\end{figure}

Here, the work of~\cite{Stewart2012} is extended by investigating a tournament
with \input{assets/tex/number_of_full_strategies/main.tex}
strategies.

The results of this analysis are shown in
Figure~\ref{fig:SSError_and_probabilities_in_full}. The top ranking strategies
by number of wins seem to be extortionate (but not against all strategies) and
it can be seen that a small sub group of strategies achieve mutual defection.
All the top ranking strategies according to score achieve mutual cooperation and
do not extort each other, however they
\textbf{do} exhibit extortionate behaviour towards a number of the lower ranking
strategies.

\begin{figure}[!htbp]
    \centering
    \includegraphics[width=.8\textwidth]{./assets/img/SSError_and_probabilities_in_full/main.pdf}
    \caption{\(\text{SSError}\) for the strategies for the full tournament. Only
    strategy interactions for which \(p_4=0\) and \(\chi>1\) are displayed.}
    \label{fig:SSError_and_probabilities_in_full}
\end{figure}

\section{Conclusion}\label{sec:conclusion}

This work defines an approach to measure whether or not a player is playing a
strategy that corresponds to an extortionate strategy as defined
in~\cite{Press2012}: a mathematical model for suspicion. Indeed, all
extortionate strategies have been
 classified as lying on a triangular plane.
This rigorous classification fails to be robust to small measurement error, thus
a statistical approach is proposed.
This is done through a linear algebraic approach for approximating the solution
of a linear system. Using this, a large number of pairwise interactions is
simulated and in fact very few strategies are found to act extortionately.

The work of~\cite{Press2012}, whilst showing that a clever approach to taking
advantage of another memory one strategy exists: this is incomplete. Whilst the
elegance of this result is very attractive, just as the simplicity of the
victory of Tit For Tat in Axelrod's original tournaments was, it is incomplete.
Extortionate strategies achieve a high number of wins but they do not
achieve a high score which corresponds to the fitness landscape in an
evolutionary sense. From the large number of interactions a payoff matrix \(S\)
can be measured where \(S_{ij}\) denotes the score (using standard values of
\((R, S, T, P) = (3, 0, 5, 1)\)) of the \(i\)th strategy
against the \(j\)th strategy. Using this, the replicator equation
describes the evolution of the system based on a population density fitness
function:

\begin{equation}\label{eqn:replicator_dynamics}
    \frac{dx}{dt} = x(S-x^TS x)
\end{equation}

Equation (\ref{eqn:replicator_dynamics}) is solved numerically through an
integration technique described in~\cite{Petzold1983} and
Figure~\ref{fig:replicator_dynamics} shows the evolution of the distribution of
the system: the various strategies are ranked by scores. It is clear to see that
only the high ranking strategies survive the evolutionary process (in fact,
only \input{./assets/img/replicator_dynamics/main.tex}
have a final distribution greater than \(10 ^ {-2}\)). This confirms the
findings of~\cite{Moran1707} in which sophisticated strategies resist
evolutionary invasion of shorter memory strategies. Recalling
Figure~\ref{fig:SSError_and_probabilities_in_full} this demonstrates that:

\begin{itemize}
    \item Cooperation emerges through the evolutionary process: the high scoring
        strategies do not exhibit extortionate behaviour towards each other.
    \item Extortionate strategies do not survive the evolutionary process.
\end{itemize}

\begin{figure}[!htbp]
    \centering
    \includegraphics[width=.8\textwidth]{./assets/img/replicator_dynamics/main.pdf}
    \caption{Numerical simulation of the replicator equation
    (\ref{eqn:replicator_dynamics}): strategies are ordered by score, only the strategies with a high score survive the evolutionary process.}
    \label{fig:replicator_dynamics}
\end{figure}

This work can be used to classify plays of the IPD\@: data can be collected from
actual interactions (in lab or in the field). Furthermore, this allows for a
classification method similar to the notion of fingerprinting presented
in~\cite{Ashlock2008}. Trained strategies can potentially be classified as
extortionate or not or it could be possible to even constrain the reinforcement
learning approaches that are becoming prevalent in the literature.
Alternatively, this mathematical approach for recognising extortion could be
used in sophisticated strategies to defend against invasion. Arguably, some of
the strategies considered here exhibit this behaviour, indeed as described
in~\cite{Harper2017}, the top ranking strategies in the full tournament are
obtained using evolutionary reinforcement learning techniques, thus, suspicion
of extortionate behaviour could in fact be an evolutionary trait.

\section*{Acknowledgements}

The following open source software libraries were used in this research:

\begin{itemize}
    \item The Axelrod ~\cite{Knight2016, Knight2018} library (IPD strategies and
        tournaments).
    \item The sympy library~\cite{Meurer2017} (verification of all symbolic
        calculations).
    \item The matplotlib~\cite{Droettboom2018} library (visualisation).
    \item The pandas~\cite{Structures2010}, dask~\cite{Dask2016} and
        NumPy~\cite{Oliphant2015} libraries (data manipulation).
    \item The SciPy~\cite{Jones2001} library (numerical integration of the
        replicator equation).
\end{itemize}

This work was performed using the computational facilities of the Advanced
Research Computing @ Cardiff (ARCCA) Division, Cardiff University.

\printbibliography

\newpage
\section*{Supplementary materials}

\includepdf{assets/pdf/proof_of_form_of_extortionate_strategies/main.pdf}

\newpage

Using the pair wise interactions the transition rates \(p,
q\) can be measured and the steady state probabilities inferred and compared to
the actual probabilities of each state.
This is done numerically by computing the singular eigenvector of the
matrix \(A\) \cite{Stewart2009}:

\[
    A =
    \begin{bmatrix}
        p_1 q_1 & p_1 (1 - q_1) & (1 - p_1) q_1 & (1 -p_1) (1 - q_1) \\
        p_2 q_2 & p_2 (1 - q_2) & (1 - p_2) q_2 & (1 -p_2) (1 - q_2) \\
        p_3 q_3 & p_3 (1 - q_3) & (1 - p_3) q_3 & (1 -p_3) (1 - q_3) \\
        p_4 q_4 & p_4 (1 - q_4) & (1 - p_4) q_4 & (1 -p_4) (1 - q_4) \\
    \end{bmatrix}
\]

Figure~\ref{fig:computed_probabilities_vs_theoretic_probabilities} shows a
regression line fitted to every pairwise interaction with a reported
\(\text{SSError}\) value (pairwise interactions with missing states were
omitted). This serves to validate the approach: a part from some edge cases the
relationship is consistent.

\begin{figure}[!htbp]
    \centering
    \includegraphics[width=.8\textwidth]{./assets/img/computed_probabilities_vs_theoretic_probabilities/main.pdf}
    \caption{The
        relationship between the steady state probabilities inferred from the
        measured transitions and the actual steady state probabilities. A linear
        regression line is included validating the approach.}
    \label{fig:computed_probabilities_vs_theoretic_probabilities}
\end{figure}


\end{document}
 times. All of this
interaction data is available at~\cite{vincent_knight_2018_1297075}. A good
match between the inferred Markov chain and the state distribution of the actual
interactions has been verified. Data for this is presented in the supplementary
materials.

Figure~\ref{fig:SSError_overall_in_stewart_plotkin} shows the \(\text{SSError}\)
values for all the strategies in the tournament, as reported
in~\cite{Stewart2012} the extortionate strategy (which has an expected
\(\text{SSError}\) approximately 0) gains a large number of wins.

\begin{figure}[!htbp]
    \centering
    \includegraphics[width=.8\textwidth]{./assets/img/SSError_overall_in_stewart_plotkin/main.pdf}
    \caption{\(\text{SSError}\) and state probabilities for the strategies
        of~\cite{Stewart2012}, ordered both by number of wins and overall score.
        Note that \(P(DC)\) is not shown as it corresponds to the transpose of
        \(P(CD)\). Cooperator and Defector are omitted as they do not visit all
        the states.}
    \label{fig:SSError_overall_in_stewart_plotkin}
\end{figure}

Here, the work of~\cite{Stewart2012} is extended by investigating a tournament
with \documentclass[a4paper]{article}

\usepackage{amsmath}
\usepackage{amssymb}
\usepackage[margin=1.5cm,
            includefoot,
            footskip=30pt]{geometry}
\usepackage{layout}
\usepackage{graphicx}
\usepackage{subcaption}

\usepackage{biblatex}
\usepackage{pdfpages}

\bibliography{main.bib}

\title{Suspicion: Recognising and evaluating the effectiveness
       of extortion in the Iterated Prisoner's Dilemma}
\author{Vincent A. Knight \and Nikoleta E. Glynatsi}
\date{\today}



\begin{document}

\maketitle

\begin{abstract}
    The Iterated Prisoner's Dilemma is a model for rational and evolutionary
    interactive behaviour. It has applications both in the study of human social
    behaviour as well as in biology.
    It is used to understand when and how a rational individual might
    accept an immediate cost to their own utility for the direct benefit of
    another.

    Much attention has been given to a class of strategies called
    Zero Determinant strategies. It has been theoretically shown that these
    strategies can ``extort'' any player.

    In this work, an approach to identify if observed strategies are playing in
    an extortionate way is described. Furthermore, experimental analysis of
    a large tournament with \input{assets/tex/number_of_full_strategies/main.tex}
    strategies is considered. In this setting
    the most highly performing strategies do not play in an extortionate way
    against each other but do against lower performing strategies.
    This suggests that whilst the theory of Zero Determinant strategies
    indicates that memory is not of fundamental importance to the evolution of
    cooperative behaviour, this is incomplete.
\end{abstract}

\section{Introduction}\label{sec:introduction}

Agent based game theoretic models have become a stalwart of the underpinning
mathematics of interactive behaviours. One of the major pieces of work
in this area is the pair of original computer tournaments run by Robert
Axelrod~\cite{Axelrod1980, Axelrod1980a}. These tournaments pitted submitted
computer strategies against each other in plays of the Iterated Prisoner's
Dilemma. A common game where agents can choose to pay a slight cost to their
immediate utility in the hope of building a reputation. This has been used in
economic and evolutionary game theory to understand the evolution of cooperative
behaviour.

Recently, a class of strategies was described in~\cite{Press2012} that can
provably extort any given opponent. In~\cite{Hilbe2013, Moran1707} some
questions have already been asked about the true effectiveness of these
strategies in an evolutionary setting. Here another question is asked: is it
possible to recognise this extortionate behaviour? A mathematical procedure for
suspicion is presented: in the same way that the continued actions of an
extortionate individual might raise suspicion.

This work makes use of the Axelrod Python library~\cite{Knight2018, Knight2016}
with a large number of Prisoner Dilemma strategies available to give an
extensive numerical example of the ideas presented.  The approach is presented
in Section~\ref{sec:delta-zd-strategies}.  All of the code and data discussed
in Section~\ref{sec:numerical-experiments} is open sourced, archived and
written according to best scientific principles~\cite{Wilson2014}. The data
archive can be found at~\cite{vincent_knight_2018_1297075}.

\section{Recognising Extortion}\label{sec:delta-zd-strategies}

In~\cite{Press2012}, given a match between 2 memory-one strategies, the concept
of Zero Determinant (ZD) strategies is introduced. The main result of that paper
shows that given two memory one players \(p, q\in\mathbb{R}^4\) a linear
relationship between the players' scores could be forced by one of the players.

Using the notation of~\cite{Press2012}, assuming the utilities for player \(p\)
are given by \(S_x=(R, S, T, P)\) and for player \(q\) by \(S_y=(R, T, S, P)\)
and that the stationary scores of each player is given by \(S_X\) and \(S_Y\)
respectively. The main result of~\cite{Press2012} is that if

\begin{equation}\label{eqn:linear_relationship_for_p}
    \tilde p=\alpha S_x + \beta S_y + \gamma
\end{equation}

or

\begin{equation}\label{eqn:linear_relationship_for_q}
    \tilde q=\alpha S_x + \beta S_y + \gamma
\end{equation}

where \(\tilde p = (1 - p_1, 1 - p_2, p_3, p_4)\) and
\(\tilde q = (1 - q_1, 1 - q_2, q_3, q_4)\) then:

\begin{equation}
    \alpha S_X + \beta S_Y + \gamma = 0
\end{equation}

In~\cite{Press2012} a particular type of ZD strategy is defined: extortionate
strategies. If:

\begin{equation}\label{eqn:constraint_for_extortion}
    \gamma = - P(\alpha + \beta)
\end{equation}

then the player can ensure they get a score \(\chi\) times
larger than the opponent. This extortion coefficient is given by:

\begin{equation}\label{eqn:definition_of_chi}
    \chi=\frac{-\beta}{\alpha}
\end{equation}

Thus, if (\ref{eqn:constraint_for_extortion}) holds and \(\chi >1\) a player is
said to extort their opponent.
Here, the reverse problem is considered: given a
\(p\in\mathbb{R}^4\) how does one identify \(\alpha, \beta\) if they
exist and is the strategy in fact acting in an extortionate way?

These conditions correspond to:

\begin{align}
    \tilde p_1 & = \alpha R + \beta R - P (\alpha + \beta)
            \label{eqn:condition_for_tilde_p1}\\
    \tilde p_2 & = \alpha S + \beta T - P (\alpha + \beta)
            \label{eqn:condition_for_tilde_p2}\\
    \tilde p_3 & = \alpha T + \beta S - P (\alpha + \beta)
            \label{eqn:condition_for_tilde_p3}\\
    \tilde p_4 & = \alpha P + \beta P - P (\alpha + \beta)
            \label{eqn:condition_for_tilde_p4}
\end{align}

Equation (\ref{eqn:condition_for_tilde_p4}) ensures that \(p_4=\tilde p_4=0\).
Equations (\ref{eqn:condition_for_tilde_p1}-\ref{eqn:condition_for_tilde_p3})
can be used to eliminate \(\alpha, \beta\), giving:

\begin{equation}\label{eqn:planar_definition_of_extortion}
    \tilde p_1 = \frac{(R - P)(\tilde p_2 + \tilde p_3)}{S + T - 2P}
\end{equation}

with:

\begin{equation}\label{eqn:definition_of_chi}
    \chi = \frac{\tilde p_2 (P - T) + \tilde p_3 (S - P)}
                {\tilde p_2 (P - S) + \tilde p_3 (T - P)}
\end{equation}

Given a strategy \(p\in\mathbb{R}^{4\times 1}\) equations
(\ref{eqn:condition_for_tilde_p4}), (\ref{eqn:planar_definition_of_extortion}-\ref{eqn:definition_of_chi}) can be used to check if
a strategy is extortionate. The conditions correspond to:

\begin{align}
    p_1 & = \frac{(R-P)(p_2 + p_3) - R + T + S - P}{S + T - 2P}
     \label{eqn:condition_for_p1}\\
    p_4 & = 0 \label{eqn:condition_for_p4}\\
    1 & > p_2 + p_3\label{eqn:condition_for_chi}
\end{align}

The algebraic steps necessary to prove these results are available in the
supporting materials.

All extortionate strategies reside on a triangular (\ref{eqn:condition_for_chi})
plane (\ref{eqn:condition_for_p1}) in 3 dimensions (\ref{eqn:condition_for_p4}).
Using this formulation it can be seen that a necessary (but not sufficient)
condition for an extortionate strategy is that it cooperates on average less
than 50\% of the time when in a state of disagreement with the opponent.

As an example, consider the known extortionate strategy \(p=(8 / 9, 1 / 2, 1 /
3, 0)\) from~\cite{Stewart2012} which is referred to as \texttt{Extort-2}. In
this case, for the standard values of \((R, T, S, P)\) constraint
(\ref{eqn:condition_for_p1}) corresponds to:

\begin{equation}
    p_1 = \frac{2(p_2 + p_3) + 1}{3}
\end{equation}

It is clear that in this case all constraints hold.

This approach could in fact be used to confirm that a given strategy is acting
in an extortionate manner even if it is not a memory one strategy. However, in
practice, if a closed form for \(p\) is not known, then due to measurement
and/or numerical error this would not work.

This problem can be written in the following linear algebraic form where
\(x=(\alpha, \beta)\)
and \(p^*=(\tilde p_1 - 1, tilde_2 - 1, p_3)\):

\begin{equation}\label{eqn:linear_algebraic_equation_for_p}
    Cx= p^*
\end{equation}

\(C\) corresponds to equations
(\ref{eqn:condition_for_tilde_p1}-\ref{eqn:condition_for_tilde_p3}) and is
given by:

\begin{equation}\label{eqn:definition_of_C}
    C =
    \begin{bmatrix}
        R - P & R- P \\
        S - P & T- P \\
        T - P & S- P \\
    \end{bmatrix}
\end{equation}

Note that in general, equation (\ref{eqn:linear_algebraic_equation_for_p}) will
not necessarily have a solution. From the Rouch\'{e}-Capelli theorem if there is
a solution it is unique as \(\text{rank}(C)=2\) which is the dimension of the
variable \(x\). The best fitting \(x\) is found by minimizing:

\begin{equation}\label{eqn:r_squared}
    \text{SSError} = \|C x- p^*\|_2^2 = \sum_{i=1}^{3}\left((C\bar x)_i-p_i^*\right)^2
\end{equation}

Note that \(\text{SSError}\), which is the square of the Frobenius
norm~\cite{Golub2013}, becomes a measure of how close a strategy is to being an
extortionate strategy. Suspicion
of extortion then corresponds to a threshold on \(\text{SSError}\).

By observing interactions (human or otherwise), their memory one representation
can be inferred and this approach can be used to recognise extortionate
behaviour. The notion of comparing theoretic and actual plays of the IPD is not
novel, see for example~\cite{Rand2013}. Immediately it is noted that if the
environment is noisy~\cite{Wu1995} then no strategy can be considered to be
extortionate as \(p_4>0\).

In the next section, this idea will be illustrated by observing the interactions
that take place in a computer based tournament of the IPD\@.

\section{Numerical experiments}\label{sec:numerical-experiments}

In~\cite{Stewart2012} results from a tournament with
\input{./assets/tex/number_of_stewart_plotkin_strategies/main.tex} strategies,
was presented with specific consideration given to ZD strategies. This
tournament is reproduced here using the Axelrod-Python
project~\cite{Knight2016}. To obtain a good measure of the corresponding
transition rates for each strategy all matches have been run for
\input{assets/tex/number_of_turns/main.tex} turns and every match has been
repeated \input{assets/tex/number_of_repetitions/main.tex} times. All of this
interaction data is available at~\cite{vincent_knight_2018_1297075}. A good
match between the inferred Markov chain and the state distribution of the actual
interactions has been verified. Data for this is presented in the supplementary
materials.

Figure~\ref{fig:SSError_overall_in_stewart_plotkin} shows the \(\text{SSError}\)
values for all the strategies in the tournament, as reported
in~\cite{Stewart2012} the extortionate strategy (which has an expected
\(\text{SSError}\) approximately 0) gains a large number of wins.

\begin{figure}[!htbp]
    \centering
    \includegraphics[width=.8\textwidth]{./assets/img/SSError_overall_in_stewart_plotkin/main.pdf}
    \caption{\(\text{SSError}\) and state probabilities for the strategies
        of~\cite{Stewart2012}, ordered both by number of wins and overall score.
        Note that \(P(DC)\) is not shown as it corresponds to the transpose of
        \(P(CD)\). Cooperator and Defector are omitted as they do not visit all
        the states.}
    \label{fig:SSError_overall_in_stewart_plotkin}
\end{figure}

Here, the work of~\cite{Stewart2012} is extended by investigating a tournament
with \input{assets/tex/number_of_full_strategies/main.tex}
strategies.

The results of this analysis are shown in
Figure~\ref{fig:SSError_and_probabilities_in_full}. The top ranking strategies
by number of wins seem to be extortionate (but not against all strategies) and
it can be seen that a small sub group of strategies achieve mutual defection.
All the top ranking strategies according to score achieve mutual cooperation and
do not extort each other, however they
\textbf{do} exhibit extortionate behaviour towards a number of the lower ranking
strategies.

\begin{figure}[!htbp]
    \centering
    \includegraphics[width=.8\textwidth]{./assets/img/SSError_and_probabilities_in_full/main.pdf}
    \caption{\(\text{SSError}\) for the strategies for the full tournament. Only
    strategy interactions for which \(p_4=0\) and \(\chi>1\) are displayed.}
    \label{fig:SSError_and_probabilities_in_full}
\end{figure}

\section{Conclusion}\label{sec:conclusion}

This work defines an approach to measure whether or not a player is playing a
strategy that corresponds to an extortionate strategy as defined
in~\cite{Press2012}: a mathematical model for suspicion. Indeed, all
extortionate strategies have been
 classified as lying on a triangular plane.
This rigorous classification fails to be robust to small measurement error, thus
a statistical approach is proposed.
This is done through a linear algebraic approach for approximating the solution
of a linear system. Using this, a large number of pairwise interactions is
simulated and in fact very few strategies are found to act extortionately.

The work of~\cite{Press2012}, whilst showing that a clever approach to taking
advantage of another memory one strategy exists: this is incomplete. Whilst the
elegance of this result is very attractive, just as the simplicity of the
victory of Tit For Tat in Axelrod's original tournaments was, it is incomplete.
Extortionate strategies achieve a high number of wins but they do not
achieve a high score which corresponds to the fitness landscape in an
evolutionary sense. From the large number of interactions a payoff matrix \(S\)
can be measured where \(S_{ij}\) denotes the score (using standard values of
\((R, S, T, P) = (3, 0, 5, 1)\)) of the \(i\)th strategy
against the \(j\)th strategy. Using this, the replicator equation
describes the evolution of the system based on a population density fitness
function:

\begin{equation}\label{eqn:replicator_dynamics}
    \frac{dx}{dt} = x(S-x^TS x)
\end{equation}

Equation (\ref{eqn:replicator_dynamics}) is solved numerically through an
integration technique described in~\cite{Petzold1983} and
Figure~\ref{fig:replicator_dynamics} shows the evolution of the distribution of
the system: the various strategies are ranked by scores. It is clear to see that
only the high ranking strategies survive the evolutionary process (in fact,
only \input{./assets/img/replicator_dynamics/main.tex}
have a final distribution greater than \(10 ^ {-2}\)). This confirms the
findings of~\cite{Moran1707} in which sophisticated strategies resist
evolutionary invasion of shorter memory strategies. Recalling
Figure~\ref{fig:SSError_and_probabilities_in_full} this demonstrates that:

\begin{itemize}
    \item Cooperation emerges through the evolutionary process: the high scoring
        strategies do not exhibit extortionate behaviour towards each other.
    \item Extortionate strategies do not survive the evolutionary process.
\end{itemize}

\begin{figure}[!htbp]
    \centering
    \includegraphics[width=.8\textwidth]{./assets/img/replicator_dynamics/main.pdf}
    \caption{Numerical simulation of the replicator equation
    (\ref{eqn:replicator_dynamics}): strategies are ordered by score, only the strategies with a high score survive the evolutionary process.}
    \label{fig:replicator_dynamics}
\end{figure}

This work can be used to classify plays of the IPD\@: data can be collected from
actual interactions (in lab or in the field). Furthermore, this allows for a
classification method similar to the notion of fingerprinting presented
in~\cite{Ashlock2008}. Trained strategies can potentially be classified as
extortionate or not or it could be possible to even constrain the reinforcement
learning approaches that are becoming prevalent in the literature.
Alternatively, this mathematical approach for recognising extortion could be
used in sophisticated strategies to defend against invasion. Arguably, some of
the strategies considered here exhibit this behaviour, indeed as described
in~\cite{Harper2017}, the top ranking strategies in the full tournament are
obtained using evolutionary reinforcement learning techniques, thus, suspicion
of extortionate behaviour could in fact be an evolutionary trait.

\section*{Acknowledgements}

The following open source software libraries were used in this research:

\begin{itemize}
    \item The Axelrod ~\cite{Knight2016, Knight2018} library (IPD strategies and
        tournaments).
    \item The sympy library~\cite{Meurer2017} (verification of all symbolic
        calculations).
    \item The matplotlib~\cite{Droettboom2018} library (visualisation).
    \item The pandas~\cite{Structures2010}, dask~\cite{Dask2016} and
        NumPy~\cite{Oliphant2015} libraries (data manipulation).
    \item The SciPy~\cite{Jones2001} library (numerical integration of the
        replicator equation).
\end{itemize}

This work was performed using the computational facilities of the Advanced
Research Computing @ Cardiff (ARCCA) Division, Cardiff University.

\printbibliography

\newpage
\section*{Supplementary materials}

\includepdf{assets/pdf/proof_of_form_of_extortionate_strategies/main.pdf}

\newpage

Using the pair wise interactions the transition rates \(p,
q\) can be measured and the steady state probabilities inferred and compared to
the actual probabilities of each state.
This is done numerically by computing the singular eigenvector of the
matrix \(A\) \cite{Stewart2009}:

\[
    A =
    \begin{bmatrix}
        p_1 q_1 & p_1 (1 - q_1) & (1 - p_1) q_1 & (1 -p_1) (1 - q_1) \\
        p_2 q_2 & p_2 (1 - q_2) & (1 - p_2) q_2 & (1 -p_2) (1 - q_2) \\
        p_3 q_3 & p_3 (1 - q_3) & (1 - p_3) q_3 & (1 -p_3) (1 - q_3) \\
        p_4 q_4 & p_4 (1 - q_4) & (1 - p_4) q_4 & (1 -p_4) (1 - q_4) \\
    \end{bmatrix}
\]

Figure~\ref{fig:computed_probabilities_vs_theoretic_probabilities} shows a
regression line fitted to every pairwise interaction with a reported
\(\text{SSError}\) value (pairwise interactions with missing states were
omitted). This serves to validate the approach: a part from some edge cases the
relationship is consistent.

\begin{figure}[!htbp]
    \centering
    \includegraphics[width=.8\textwidth]{./assets/img/computed_probabilities_vs_theoretic_probabilities/main.pdf}
    \caption{The
        relationship between the steady state probabilities inferred from the
        measured transitions and the actual steady state probabilities. A linear
        regression line is included validating the approach.}
    \label{fig:computed_probabilities_vs_theoretic_probabilities}
\end{figure}


\end{document}

strategies.

The results of this analysis are shown in
Figure~\ref{fig:SSError_and_probabilities_in_full}. The top ranking strategies
by number of wins seem to be extortionate (but not against all strategies) and
it can be seen that a small sub group of strategies achieve mutual defection.
All the top ranking strategies according to score achieve mutual cooperation and
do not extort each other, however they
\textbf{do} exhibit extortionate behaviour towards a number of the lower ranking
strategies.

\begin{figure}[!htbp]
    \centering
    \includegraphics[width=.8\textwidth]{./assets/img/SSError_and_probabilities_in_full/main.pdf}
    \caption{\(\text{SSError}\) for the strategies for the full tournament. Only
    strategy interactions for which \(p_4=0\) and \(\chi>1\) are displayed.}
    \label{fig:SSError_and_probabilities_in_full}
\end{figure}

\section{Conclusion}\label{sec:conclusion}

This work defines an approach to measure whether or not a player is playing a
strategy that corresponds to an extortionate strategy as defined
in~\cite{Press2012}: a mathematical model for suspicion. Indeed, all
extortionate strategies have been
 classified as lying on a triangular plane.
This rigorous classification fails to be robust to small measurement error, thus
a statistical approach is proposed.
This is done through a linear algebraic approach for approximating the solution
of a linear system. Using this, a large number of pairwise interactions is
simulated and in fact very few strategies are found to act extortionately.

The work of~\cite{Press2012}, whilst showing that a clever approach to taking
advantage of another memory one strategy exists: this is incomplete. Whilst the
elegance of this result is very attractive, just as the simplicity of the
victory of Tit For Tat in Axelrod's original tournaments was, it is incomplete.
Extortionate strategies achieve a high number of wins but they do not
achieve a high score which corresponds to the fitness landscape in an
evolutionary sense. From the large number of interactions a payoff matrix \(S\)
can be measured where \(S_{ij}\) denotes the score (using standard values of
\((R, S, T, P) = (3, 0, 5, 1)\)) of the \(i\)th strategy
against the \(j\)th strategy. Using this, the replicator equation
describes the evolution of the system based on a population density fitness
function:

\begin{equation}\label{eqn:replicator_dynamics}
    \frac{dx}{dt} = x(S-x^TS x)
\end{equation}

Equation (\ref{eqn:replicator_dynamics}) is solved numerically through an
integration technique described in~\cite{Petzold1983} and
Figure~\ref{fig:replicator_dynamics} shows the evolution of the distribution of
the system: the various strategies are ranked by scores. It is clear to see that
only the high ranking strategies survive the evolutionary process (in fact,
only \documentclass[a4paper]{article}

\usepackage{amsmath}
\usepackage{amssymb}
\usepackage[margin=1.5cm,
            includefoot,
            footskip=30pt]{geometry}
\usepackage{layout}
\usepackage{graphicx}
\usepackage{subcaption}

\usepackage{biblatex}
\usepackage{pdfpages}

\bibliography{main.bib}

\title{Suspicion: Recognising and evaluating the effectiveness
       of extortion in the Iterated Prisoner's Dilemma}
\author{Vincent A. Knight \and Nikoleta E. Glynatsi}
\date{\today}



\begin{document}

\maketitle

\begin{abstract}
    The Iterated Prisoner's Dilemma is a model for rational and evolutionary
    interactive behaviour. It has applications both in the study of human social
    behaviour as well as in biology.
    It is used to understand when and how a rational individual might
    accept an immediate cost to their own utility for the direct benefit of
    another.

    Much attention has been given to a class of strategies called
    Zero Determinant strategies. It has been theoretically shown that these
    strategies can ``extort'' any player.

    In this work, an approach to identify if observed strategies are playing in
    an extortionate way is described. Furthermore, experimental analysis of
    a large tournament with \input{assets/tex/number_of_full_strategies/main.tex}
    strategies is considered. In this setting
    the most highly performing strategies do not play in an extortionate way
    against each other but do against lower performing strategies.
    This suggests that whilst the theory of Zero Determinant strategies
    indicates that memory is not of fundamental importance to the evolution of
    cooperative behaviour, this is incomplete.
\end{abstract}

\section{Introduction}\label{sec:introduction}

Agent based game theoretic models have become a stalwart of the underpinning
mathematics of interactive behaviours. One of the major pieces of work
in this area is the pair of original computer tournaments run by Robert
Axelrod~\cite{Axelrod1980, Axelrod1980a}. These tournaments pitted submitted
computer strategies against each other in plays of the Iterated Prisoner's
Dilemma. A common game where agents can choose to pay a slight cost to their
immediate utility in the hope of building a reputation. This has been used in
economic and evolutionary game theory to understand the evolution of cooperative
behaviour.

Recently, a class of strategies was described in~\cite{Press2012} that can
provably extort any given opponent. In~\cite{Hilbe2013, Moran1707} some
questions have already been asked about the true effectiveness of these
strategies in an evolutionary setting. Here another question is asked: is it
possible to recognise this extortionate behaviour? A mathematical procedure for
suspicion is presented: in the same way that the continued actions of an
extortionate individual might raise suspicion.

This work makes use of the Axelrod Python library~\cite{Knight2018, Knight2016}
with a large number of Prisoner Dilemma strategies available to give an
extensive numerical example of the ideas presented.  The approach is presented
in Section~\ref{sec:delta-zd-strategies}.  All of the code and data discussed
in Section~\ref{sec:numerical-experiments} is open sourced, archived and
written according to best scientific principles~\cite{Wilson2014}. The data
archive can be found at~\cite{vincent_knight_2018_1297075}.

\section{Recognising Extortion}\label{sec:delta-zd-strategies}

In~\cite{Press2012}, given a match between 2 memory-one strategies, the concept
of Zero Determinant (ZD) strategies is introduced. The main result of that paper
shows that given two memory one players \(p, q\in\mathbb{R}^4\) a linear
relationship between the players' scores could be forced by one of the players.

Using the notation of~\cite{Press2012}, assuming the utilities for player \(p\)
are given by \(S_x=(R, S, T, P)\) and for player \(q\) by \(S_y=(R, T, S, P)\)
and that the stationary scores of each player is given by \(S_X\) and \(S_Y\)
respectively. The main result of~\cite{Press2012} is that if

\begin{equation}\label{eqn:linear_relationship_for_p}
    \tilde p=\alpha S_x + \beta S_y + \gamma
\end{equation}

or

\begin{equation}\label{eqn:linear_relationship_for_q}
    \tilde q=\alpha S_x + \beta S_y + \gamma
\end{equation}

where \(\tilde p = (1 - p_1, 1 - p_2, p_3, p_4)\) and
\(\tilde q = (1 - q_1, 1 - q_2, q_3, q_4)\) then:

\begin{equation}
    \alpha S_X + \beta S_Y + \gamma = 0
\end{equation}

In~\cite{Press2012} a particular type of ZD strategy is defined: extortionate
strategies. If:

\begin{equation}\label{eqn:constraint_for_extortion}
    \gamma = - P(\alpha + \beta)
\end{equation}

then the player can ensure they get a score \(\chi\) times
larger than the opponent. This extortion coefficient is given by:

\begin{equation}\label{eqn:definition_of_chi}
    \chi=\frac{-\beta}{\alpha}
\end{equation}

Thus, if (\ref{eqn:constraint_for_extortion}) holds and \(\chi >1\) a player is
said to extort their opponent.
Here, the reverse problem is considered: given a
\(p\in\mathbb{R}^4\) how does one identify \(\alpha, \beta\) if they
exist and is the strategy in fact acting in an extortionate way?

These conditions correspond to:

\begin{align}
    \tilde p_1 & = \alpha R + \beta R - P (\alpha + \beta)
            \label{eqn:condition_for_tilde_p1}\\
    \tilde p_2 & = \alpha S + \beta T - P (\alpha + \beta)
            \label{eqn:condition_for_tilde_p2}\\
    \tilde p_3 & = \alpha T + \beta S - P (\alpha + \beta)
            \label{eqn:condition_for_tilde_p3}\\
    \tilde p_4 & = \alpha P + \beta P - P (\alpha + \beta)
            \label{eqn:condition_for_tilde_p4}
\end{align}

Equation (\ref{eqn:condition_for_tilde_p4}) ensures that \(p_4=\tilde p_4=0\).
Equations (\ref{eqn:condition_for_tilde_p1}-\ref{eqn:condition_for_tilde_p3})
can be used to eliminate \(\alpha, \beta\), giving:

\begin{equation}\label{eqn:planar_definition_of_extortion}
    \tilde p_1 = \frac{(R - P)(\tilde p_2 + \tilde p_3)}{S + T - 2P}
\end{equation}

with:

\begin{equation}\label{eqn:definition_of_chi}
    \chi = \frac{\tilde p_2 (P - T) + \tilde p_3 (S - P)}
                {\tilde p_2 (P - S) + \tilde p_3 (T - P)}
\end{equation}

Given a strategy \(p\in\mathbb{R}^{4\times 1}\) equations
(\ref{eqn:condition_for_tilde_p4}), (\ref{eqn:planar_definition_of_extortion}-\ref{eqn:definition_of_chi}) can be used to check if
a strategy is extortionate. The conditions correspond to:

\begin{align}
    p_1 & = \frac{(R-P)(p_2 + p_3) - R + T + S - P}{S + T - 2P}
     \label{eqn:condition_for_p1}\\
    p_4 & = 0 \label{eqn:condition_for_p4}\\
    1 & > p_2 + p_3\label{eqn:condition_for_chi}
\end{align}

The algebraic steps necessary to prove these results are available in the
supporting materials.

All extortionate strategies reside on a triangular (\ref{eqn:condition_for_chi})
plane (\ref{eqn:condition_for_p1}) in 3 dimensions (\ref{eqn:condition_for_p4}).
Using this formulation it can be seen that a necessary (but not sufficient)
condition for an extortionate strategy is that it cooperates on average less
than 50\% of the time when in a state of disagreement with the opponent.

As an example, consider the known extortionate strategy \(p=(8 / 9, 1 / 2, 1 /
3, 0)\) from~\cite{Stewart2012} which is referred to as \texttt{Extort-2}. In
this case, for the standard values of \((R, T, S, P)\) constraint
(\ref{eqn:condition_for_p1}) corresponds to:

\begin{equation}
    p_1 = \frac{2(p_2 + p_3) + 1}{3}
\end{equation}

It is clear that in this case all constraints hold.

This approach could in fact be used to confirm that a given strategy is acting
in an extortionate manner even if it is not a memory one strategy. However, in
practice, if a closed form for \(p\) is not known, then due to measurement
and/or numerical error this would not work.

This problem can be written in the following linear algebraic form where
\(x=(\alpha, \beta)\)
and \(p^*=(\tilde p_1 - 1, tilde_2 - 1, p_3)\):

\begin{equation}\label{eqn:linear_algebraic_equation_for_p}
    Cx= p^*
\end{equation}

\(C\) corresponds to equations
(\ref{eqn:condition_for_tilde_p1}-\ref{eqn:condition_for_tilde_p3}) and is
given by:

\begin{equation}\label{eqn:definition_of_C}
    C =
    \begin{bmatrix}
        R - P & R- P \\
        S - P & T- P \\
        T - P & S- P \\
    \end{bmatrix}
\end{equation}

Note that in general, equation (\ref{eqn:linear_algebraic_equation_for_p}) will
not necessarily have a solution. From the Rouch\'{e}-Capelli theorem if there is
a solution it is unique as \(\text{rank}(C)=2\) which is the dimension of the
variable \(x\). The best fitting \(x\) is found by minimizing:

\begin{equation}\label{eqn:r_squared}
    \text{SSError} = \|C x- p^*\|_2^2 = \sum_{i=1}^{3}\left((C\bar x)_i-p_i^*\right)^2
\end{equation}

Note that \(\text{SSError}\), which is the square of the Frobenius
norm~\cite{Golub2013}, becomes a measure of how close a strategy is to being an
extortionate strategy. Suspicion
of extortion then corresponds to a threshold on \(\text{SSError}\).

By observing interactions (human or otherwise), their memory one representation
can be inferred and this approach can be used to recognise extortionate
behaviour. The notion of comparing theoretic and actual plays of the IPD is not
novel, see for example~\cite{Rand2013}. Immediately it is noted that if the
environment is noisy~\cite{Wu1995} then no strategy can be considered to be
extortionate as \(p_4>0\).

In the next section, this idea will be illustrated by observing the interactions
that take place in a computer based tournament of the IPD\@.

\section{Numerical experiments}\label{sec:numerical-experiments}

In~\cite{Stewart2012} results from a tournament with
\input{./assets/tex/number_of_stewart_plotkin_strategies/main.tex} strategies,
was presented with specific consideration given to ZD strategies. This
tournament is reproduced here using the Axelrod-Python
project~\cite{Knight2016}. To obtain a good measure of the corresponding
transition rates for each strategy all matches have been run for
\input{assets/tex/number_of_turns/main.tex} turns and every match has been
repeated \input{assets/tex/number_of_repetitions/main.tex} times. All of this
interaction data is available at~\cite{vincent_knight_2018_1297075}. A good
match between the inferred Markov chain and the state distribution of the actual
interactions has been verified. Data for this is presented in the supplementary
materials.

Figure~\ref{fig:SSError_overall_in_stewart_plotkin} shows the \(\text{SSError}\)
values for all the strategies in the tournament, as reported
in~\cite{Stewart2012} the extortionate strategy (which has an expected
\(\text{SSError}\) approximately 0) gains a large number of wins.

\begin{figure}[!htbp]
    \centering
    \includegraphics[width=.8\textwidth]{./assets/img/SSError_overall_in_stewart_plotkin/main.pdf}
    \caption{\(\text{SSError}\) and state probabilities for the strategies
        of~\cite{Stewart2012}, ordered both by number of wins and overall score.
        Note that \(P(DC)\) is not shown as it corresponds to the transpose of
        \(P(CD)\). Cooperator and Defector are omitted as they do not visit all
        the states.}
    \label{fig:SSError_overall_in_stewart_plotkin}
\end{figure}

Here, the work of~\cite{Stewart2012} is extended by investigating a tournament
with \input{assets/tex/number_of_full_strategies/main.tex}
strategies.

The results of this analysis are shown in
Figure~\ref{fig:SSError_and_probabilities_in_full}. The top ranking strategies
by number of wins seem to be extortionate (but not against all strategies) and
it can be seen that a small sub group of strategies achieve mutual defection.
All the top ranking strategies according to score achieve mutual cooperation and
do not extort each other, however they
\textbf{do} exhibit extortionate behaviour towards a number of the lower ranking
strategies.

\begin{figure}[!htbp]
    \centering
    \includegraphics[width=.8\textwidth]{./assets/img/SSError_and_probabilities_in_full/main.pdf}
    \caption{\(\text{SSError}\) for the strategies for the full tournament. Only
    strategy interactions for which \(p_4=0\) and \(\chi>1\) are displayed.}
    \label{fig:SSError_and_probabilities_in_full}
\end{figure}

\section{Conclusion}\label{sec:conclusion}

This work defines an approach to measure whether or not a player is playing a
strategy that corresponds to an extortionate strategy as defined
in~\cite{Press2012}: a mathematical model for suspicion. Indeed, all
extortionate strategies have been
 classified as lying on a triangular plane.
This rigorous classification fails to be robust to small measurement error, thus
a statistical approach is proposed.
This is done through a linear algebraic approach for approximating the solution
of a linear system. Using this, a large number of pairwise interactions is
simulated and in fact very few strategies are found to act extortionately.

The work of~\cite{Press2012}, whilst showing that a clever approach to taking
advantage of another memory one strategy exists: this is incomplete. Whilst the
elegance of this result is very attractive, just as the simplicity of the
victory of Tit For Tat in Axelrod's original tournaments was, it is incomplete.
Extortionate strategies achieve a high number of wins but they do not
achieve a high score which corresponds to the fitness landscape in an
evolutionary sense. From the large number of interactions a payoff matrix \(S\)
can be measured where \(S_{ij}\) denotes the score (using standard values of
\((R, S, T, P) = (3, 0, 5, 1)\)) of the \(i\)th strategy
against the \(j\)th strategy. Using this, the replicator equation
describes the evolution of the system based on a population density fitness
function:

\begin{equation}\label{eqn:replicator_dynamics}
    \frac{dx}{dt} = x(S-x^TS x)
\end{equation}

Equation (\ref{eqn:replicator_dynamics}) is solved numerically through an
integration technique described in~\cite{Petzold1983} and
Figure~\ref{fig:replicator_dynamics} shows the evolution of the distribution of
the system: the various strategies are ranked by scores. It is clear to see that
only the high ranking strategies survive the evolutionary process (in fact,
only \input{./assets/img/replicator_dynamics/main.tex}
have a final distribution greater than \(10 ^ {-2}\)). This confirms the
findings of~\cite{Moran1707} in which sophisticated strategies resist
evolutionary invasion of shorter memory strategies. Recalling
Figure~\ref{fig:SSError_and_probabilities_in_full} this demonstrates that:

\begin{itemize}
    \item Cooperation emerges through the evolutionary process: the high scoring
        strategies do not exhibit extortionate behaviour towards each other.
    \item Extortionate strategies do not survive the evolutionary process.
\end{itemize}

\begin{figure}[!htbp]
    \centering
    \includegraphics[width=.8\textwidth]{./assets/img/replicator_dynamics/main.pdf}
    \caption{Numerical simulation of the replicator equation
    (\ref{eqn:replicator_dynamics}): strategies are ordered by score, only the strategies with a high score survive the evolutionary process.}
    \label{fig:replicator_dynamics}
\end{figure}

This work can be used to classify plays of the IPD\@: data can be collected from
actual interactions (in lab or in the field). Furthermore, this allows for a
classification method similar to the notion of fingerprinting presented
in~\cite{Ashlock2008}. Trained strategies can potentially be classified as
extortionate or not or it could be possible to even constrain the reinforcement
learning approaches that are becoming prevalent in the literature.
Alternatively, this mathematical approach for recognising extortion could be
used in sophisticated strategies to defend against invasion. Arguably, some of
the strategies considered here exhibit this behaviour, indeed as described
in~\cite{Harper2017}, the top ranking strategies in the full tournament are
obtained using evolutionary reinforcement learning techniques, thus, suspicion
of extortionate behaviour could in fact be an evolutionary trait.

\section*{Acknowledgements}

The following open source software libraries were used in this research:

\begin{itemize}
    \item The Axelrod ~\cite{Knight2016, Knight2018} library (IPD strategies and
        tournaments).
    \item The sympy library~\cite{Meurer2017} (verification of all symbolic
        calculations).
    \item The matplotlib~\cite{Droettboom2018} library (visualisation).
    \item The pandas~\cite{Structures2010}, dask~\cite{Dask2016} and
        NumPy~\cite{Oliphant2015} libraries (data manipulation).
    \item The SciPy~\cite{Jones2001} library (numerical integration of the
        replicator equation).
\end{itemize}

This work was performed using the computational facilities of the Advanced
Research Computing @ Cardiff (ARCCA) Division, Cardiff University.

\printbibliography

\newpage
\section*{Supplementary materials}

\includepdf{assets/pdf/proof_of_form_of_extortionate_strategies/main.pdf}

\newpage

Using the pair wise interactions the transition rates \(p,
q\) can be measured and the steady state probabilities inferred and compared to
the actual probabilities of each state.
This is done numerically by computing the singular eigenvector of the
matrix \(A\) \cite{Stewart2009}:

\[
    A =
    \begin{bmatrix}
        p_1 q_1 & p_1 (1 - q_1) & (1 - p_1) q_1 & (1 -p_1) (1 - q_1) \\
        p_2 q_2 & p_2 (1 - q_2) & (1 - p_2) q_2 & (1 -p_2) (1 - q_2) \\
        p_3 q_3 & p_3 (1 - q_3) & (1 - p_3) q_3 & (1 -p_3) (1 - q_3) \\
        p_4 q_4 & p_4 (1 - q_4) & (1 - p_4) q_4 & (1 -p_4) (1 - q_4) \\
    \end{bmatrix}
\]

Figure~\ref{fig:computed_probabilities_vs_theoretic_probabilities} shows a
regression line fitted to every pairwise interaction with a reported
\(\text{SSError}\) value (pairwise interactions with missing states were
omitted). This serves to validate the approach: a part from some edge cases the
relationship is consistent.

\begin{figure}[!htbp]
    \centering
    \includegraphics[width=.8\textwidth]{./assets/img/computed_probabilities_vs_theoretic_probabilities/main.pdf}
    \caption{The
        relationship between the steady state probabilities inferred from the
        measured transitions and the actual steady state probabilities. A linear
        regression line is included validating the approach.}
    \label{fig:computed_probabilities_vs_theoretic_probabilities}
\end{figure}


\end{document}

have a final distribution greater than \(10 ^ {-2}\)). This confirms the
findings of~\cite{Moran1707} in which sophisticated strategies resist
evolutionary invasion of shorter memory strategies. Recalling
Figure~\ref{fig:SSError_and_probabilities_in_full} this demonstrates that:

\begin{itemize}
    \item Cooperation emerges through the evolutionary process: the high scoring
        strategies do not exhibit extortionate behaviour towards each other.
    \item Extortionate strategies do not survive the evolutionary process.
\end{itemize}

\begin{figure}[!htbp]
    \centering
    \includegraphics[width=.8\textwidth]{./assets/img/replicator_dynamics/main.pdf}
    \caption{Numerical simulation of the replicator equation
    (\ref{eqn:replicator_dynamics}): strategies are ordered by score, only the strategies with a high score survive the evolutionary process.}
    \label{fig:replicator_dynamics}
\end{figure}

This work can be used to classify plays of the IPD\@: data can be collected from
actual interactions (in lab or in the field). Furthermore, this allows for a
classification method similar to the notion of fingerprinting presented
in~\cite{Ashlock2008}. Trained strategies can potentially be classified as
extortionate or not or it could be possible to even constrain the reinforcement
learning approaches that are becoming prevalent in the literature.
Alternatively, this mathematical approach for recognising extortion could be
used in sophisticated strategies to defend against invasion. Arguably, some of
the strategies considered here exhibit this behaviour, indeed as described
in~\cite{Harper2017}, the top ranking strategies in the full tournament are
obtained using evolutionary reinforcement learning techniques, thus, suspicion
of extortionate behaviour could in fact be an evolutionary trait.

\section*{Acknowledgements}

The following open source software libraries were used in this research:

\begin{itemize}
    \item The Axelrod ~\cite{Knight2016, Knight2018} library (IPD strategies and
        tournaments).
    \item The sympy library~\cite{Meurer2017} (verification of all symbolic
        calculations).
    \item The matplotlib~\cite{Droettboom2018} library (visualisation).
    \item The pandas~\cite{Structures2010}, dask~\cite{Dask2016} and
        NumPy~\cite{Oliphant2015} libraries (data manipulation).
    \item The SciPy~\cite{Jones2001} library (numerical integration of the
        replicator equation).
\end{itemize}

This work was performed using the computational facilities of the Advanced
Research Computing @ Cardiff (ARCCA) Division, Cardiff University.

\printbibliography

\newpage
\section*{Supplementary materials}

\includepdf{assets/pdf/proof_of_form_of_extortionate_strategies/main.pdf}

\newpage

Using the pair wise interactions the transition rates \(p,
q\) can be measured and the steady state probabilities inferred and compared to
the actual probabilities of each state.
This is done numerically by computing the singular eigenvector of the
matrix \(A\) \cite{Stewart2009}:

\[
    A =
    \begin{bmatrix}
        p_1 q_1 & p_1 (1 - q_1) & (1 - p_1) q_1 & (1 -p_1) (1 - q_1) \\
        p_2 q_2 & p_2 (1 - q_2) & (1 - p_2) q_2 & (1 -p_2) (1 - q_2) \\
        p_3 q_3 & p_3 (1 - q_3) & (1 - p_3) q_3 & (1 -p_3) (1 - q_3) \\
        p_4 q_4 & p_4 (1 - q_4) & (1 - p_4) q_4 & (1 -p_4) (1 - q_4) \\
    \end{bmatrix}
\]

Figure~\ref{fig:computed_probabilities_vs_theoretic_probabilities} shows a
regression line fitted to every pairwise interaction with a reported
\(\text{SSError}\) value (pairwise interactions with missing states were
omitted). This serves to validate the approach: a part from some edge cases the
relationship is consistent.

\begin{figure}[!htbp]
    \centering
    \includegraphics[width=.8\textwidth]{./assets/img/computed_probabilities_vs_theoretic_probabilities/main.pdf}
    \caption{The
        relationship between the steady state probabilities inferred from the
        measured transitions and the actual steady state probabilities. A linear
        regression line is included validating the approach.}
    \label{fig:computed_probabilities_vs_theoretic_probabilities}
\end{figure}


\end{document}

have a final distribution greater than \(10 ^ {-2}\)). This confirms the
findings of~\cite{Moran1707} in which sophisticated strategies resist
evolutionary invasion of shorter memory strategies. Recalling
Figure~\ref{fig:SSError_and_probabilities_in_full} this demonstrates that:

\begin{itemize}
    \item Cooperation emerges through the evolutionary process: the high scoring
        strategies do not exhibit extortionate behaviour towards each other.
    \item Extortionate strategies do not survive the evolutionary process.
\end{itemize}

\begin{figure}[!htbp]
    \centering
    \includegraphics[width=.8\textwidth]{./assets/img/replicator_dynamics/main.pdf}
    \caption{Numerical simulation of the replicator equation
    (\ref{eqn:replicator_dynamics}): strategies are ordered by score, only the strategies with a high score survive the evolutionary process.}
    \label{fig:replicator_dynamics}
\end{figure}

This work can be used to classify plays of the IPD\@: data can be collected from
actual interactions (in lab or in the field). Furthermore, this allows for a
classification method similar to the notion of fingerprinting presented
in~\cite{Ashlock2008}. Trained strategies can potentially be classified as
extortionate or not or it could be possible to even constrain the reinforcement
learning approaches that are becoming prevalent in the literature.
Alternatively, this mathematical approach for recognising extortion could be
used in sophisticated strategies to defend against invasion. Arguably, some of
the strategies considered here exhibit this behaviour, indeed as described
in~\cite{Harper2017}, the top ranking strategies in the full tournament are
obtained using evolutionary reinforcement learning techniques, thus, suspicion
of extortionate behaviour could in fact be an evolutionary trait.

\section*{Acknowledgements}

The following open source software libraries were used in this research:

\begin{itemize}
    \item The Axelrod ~\cite{Knight2016, Knight2018} library (IPD strategies and
        tournaments).
    \item The sympy library~\cite{Meurer2017} (verification of all symbolic
        calculations).
    \item The matplotlib~\cite{Droettboom2018} library (visualisation).
    \item The pandas~\cite{Structures2010}, dask~\cite{Dask2016} and
        NumPy~\cite{Oliphant2015} libraries (data manipulation).
    \item The SciPy~\cite{Jones2001} library (numerical integration of the
        replicator equation).
\end{itemize}

This work was performed using the computational facilities of the Advanced
Research Computing @ Cardiff (ARCCA) Division, Cardiff University.

\printbibliography

\newpage
\section*{Supplementary materials}

\includepdf{assets/pdf/proof_of_form_of_extortionate_strategies/main.pdf}

\newpage

Using the pair wise interactions the transition rates \(p,
q\) can be measured and the steady state probabilities inferred and compared to
the actual probabilities of each state.
This is done numerically by computing the singular eigenvector of the
matrix \(A\) \cite{Stewart2009}:

\[
    A =
    \begin{bmatrix}
        p_1 q_1 & p_1 (1 - q_1) & (1 - p_1) q_1 & (1 -p_1) (1 - q_1) \\
        p_2 q_2 & p_2 (1 - q_2) & (1 - p_2) q_2 & (1 -p_2) (1 - q_2) \\
        p_3 q_3 & p_3 (1 - q_3) & (1 - p_3) q_3 & (1 -p_3) (1 - q_3) \\
        p_4 q_4 & p_4 (1 - q_4) & (1 - p_4) q_4 & (1 -p_4) (1 - q_4) \\
    \end{bmatrix}
\]

Figure~\ref{fig:computed_probabilities_vs_theoretic_probabilities} shows a
regression line fitted to every pairwise interaction with a reported
\(\text{SSError}\) value (pairwise interactions with missing states were
omitted). This serves to validate the approach: a part from some edge cases the
relationship is consistent.

\begin{figure}[!htbp]
    \centering
    \includegraphics[width=.8\textwidth]{./assets/img/computed_probabilities_vs_theoretic_probabilities/main.pdf}
    \caption{The
        relationship between the steady state probabilities inferred from the
        measured transitions and the actual steady state probabilities. A linear
        regression line is included validating the approach.}
    \label{fig:computed_probabilities_vs_theoretic_probabilities}
\end{figure}


\end{document}
times. All of this
interaction data is available at~\cite{vincent_knight_2018_1297075}. Note that
in the interest of open scientific practice,~\cite{vincent_knight_2018_1297075}
also contains interaction data for noisy and probabilistic ending interactions
which are not investigated here.

Figure~\ref{fig:sserror_in_stewart_plotkin} shows the
\(\SSe\) values for all the strategies in the tournament, as
reported in~\cite{Stewart2012} the extortionate strategy Extort-2 gains a large number of
wins. Notice that the mean \(\SSe\) for Extort-2 is approximately zero, while for
the always cooperating strategy Cooperator the \(\SSe\) is far from zero. It is
also clear that ZD-GTFT2 defined as a ZD strategy does not act
extortionately. This is evident by the fact that it does not rank highly according
to wins which is due to its value of \(\chi\) being less than 1.

\begin{figure}[!htbp]
    \centering
    \includegraphics[width=.8\textwidth]{./assets/img/sserror_in_stewart_plotkin/main.pdf}
    \caption{\(\SSe\) and best fitting \(\chi\) for~\cite{Stewart2012},
        ordered both by number of wins and overall score.
        The strategies with a positive skew
        \(\SSe\) and high \(\chi\) win the most matches, although even the known
        extortionate strategy does not act in a perfectly extortionate manner in
        all matches. The strategies with a high score have a negatively skewed
        \(\SSe\).
        }
    \label{fig:sserror_in_stewart_plotkin}
\end{figure}

Next, the results of a much larger tournament are presented.
As a final validation of the proposed methodology here,
Table~\ref{tbl:chi_versus_observed_chi_for_zd} shows the known values of
\(\chi\) versus the measured values for all ZD strategies in the tournament. It
Clearly, the method accurately recovers \(\chi\) from the observed play of
the strategies. Furthermore, the \(\SSe\) value is low for all of these. The
values of \(\SSe\) above 1 indicate that whilst these strategies are designed to
act extortionately they do not do so in all cases. This will be discussed in
more detail in the next section.

\begin{table}[!hbtp]
    \begin{center}
   \begin{tabular}{lrrr}
\toprule
                  Name &  Measured chi &  Theoretic chi &     SSE \\
\midrule
         Firm But Fair &        1.0000 &     1.0000 &  0.4446 \\
                  GTFT &        0.6999 &     0.7000 &  0.1373 \\
                  Joss &        1.2428 &     1.2431 &  0.0006 \\
             Soft Joss &        0.9110 &     0.9112 &  0.0123 \\
 Stochastic Cooperator &        3.0248 &     3.0276 &  0.2158 \\
       Stochastic WSLS &       12.6105 &    12.6000 &  1.0627 \\
   Win-Shift Lose-Stay &        1.8333 &     1.8333 &  1.4706 \\
   Win-Stay Lose-Shift &       16.0000 &    16.0000 &  1.2353 \\
          ZD-Extortion &       10.0067 &    10.0000 &  0.0000 \\
           ZD-Extort-2 &        1.9978 &     2.0000 &  0.0000 \\
            ZD-Extort3 &        3.0022 &     3.0000 &  0.0000 \\
        ZD-Extort-2 v2 &        2.0020 &     2.0000 &  0.0000 \\
           ZD-Extort-4 &        3.9998 &     4.0000 &  0.0000 \\
             ZD-GTFT-2 &        0.8887 &     0.8889 &  0.0662 \\
              ZD-GEN-2 &        0.8892 &     0.8889 &  0.0165 \\
              ZD-SET-2 &        2.4022 &     2.4000 &  0.0661 \\
\bottomrule
\end{tabular}

    \end{center}
    \caption{Validating the approach by comparing the measured values of \(\chi\) and the known values of
    \(\chi\) for all ZD strategies in the larger tournament. The value of
    \(\chi\) is effectively recovered from observed play and the \(\SSe\)
    indicates that not all strategies are able to play as expected all the
time.}
    \label{tbl:chi_versus_observed_chi_for_zd}
\end{table}

\subsection{Numerical experiments}

Next we investigate a tournament with
\documentclass[a4paper]{article}

\usepackage{amsmath}
\usepackage{amssymb}
\usepackage[margin=1.5cm,
            includefoot,
            footskip=30pt]{geometry}
\usepackage{layout}
\usepackage{graphicx}
\usepackage{subcaption}

\usepackage{biblatex}
\usepackage{pdfpages}

\bibliography{main.bib}

\title{Suspicion: Recognising and evaluating the effectiveness
       of extortion in the Iterated Prisoner's Dilemma}
\author{Vincent A. Knight \and Nikoleta E. Glynatsi}
\date{\today}



\begin{document}

\maketitle

\begin{abstract}
    The Iterated Prisoner's Dilemma is a model for rational and evolutionary
    interactive behaviour. It has applications both in the study of human social
    behaviour as well as in biology.
    It is used to understand when and how a rational individual might
    accept an immediate cost to their own utility for the direct benefit of
    another.

    Much attention has been given to a class of strategies called
    Zero Determinant strategies. It has been theoretically shown that these
    strategies can ``extort'' any player.

    In this work, an approach to identify if observed strategies are playing in
    an extortionate way is described. Furthermore, experimental analysis of
    a large tournament with \documentclass[a4paper]{article}

\usepackage{amsmath}
\usepackage{amssymb}
\usepackage[margin=1.5cm,
            includefoot,
            footskip=30pt]{geometry}
\usepackage{layout}
\usepackage{graphicx}
\usepackage{subcaption}

\usepackage{biblatex}
\usepackage{pdfpages}

\bibliography{main.bib}

\title{Suspicion: Recognising and evaluating the effectiveness
       of extortion in the Iterated Prisoner's Dilemma}
\author{Vincent A. Knight \and Nikoleta E. Glynatsi}
\date{\today}



\begin{document}

\maketitle

\begin{abstract}
    The Iterated Prisoner's Dilemma is a model for rational and evolutionary
    interactive behaviour. It has applications both in the study of human social
    behaviour as well as in biology.
    It is used to understand when and how a rational individual might
    accept an immediate cost to their own utility for the direct benefit of
    another.

    Much attention has been given to a class of strategies called
    Zero Determinant strategies. It has been theoretically shown that these
    strategies can ``extort'' any player.

    In this work, an approach to identify if observed strategies are playing in
    an extortionate way is described. Furthermore, experimental analysis of
    a large tournament with \documentclass[a4paper]{article}

\usepackage{amsmath}
\usepackage{amssymb}
\usepackage[margin=1.5cm,
            includefoot,
            footskip=30pt]{geometry}
\usepackage{layout}
\usepackage{graphicx}
\usepackage{subcaption}

\usepackage{biblatex}
\usepackage{pdfpages}

\bibliography{main.bib}

\title{Suspicion: Recognising and evaluating the effectiveness
       of extortion in the Iterated Prisoner's Dilemma}
\author{Vincent A. Knight \and Nikoleta E. Glynatsi}
\date{\today}



\begin{document}

\maketitle

\begin{abstract}
    The Iterated Prisoner's Dilemma is a model for rational and evolutionary
    interactive behaviour. It has applications both in the study of human social
    behaviour as well as in biology.
    It is used to understand when and how a rational individual might
    accept an immediate cost to their own utility for the direct benefit of
    another.

    Much attention has been given to a class of strategies called
    Zero Determinant strategies. It has been theoretically shown that these
    strategies can ``extort'' any player.

    In this work, an approach to identify if observed strategies are playing in
    an extortionate way is described. Furthermore, experimental analysis of
    a large tournament with \input{assets/tex/number_of_full_strategies/main.tex}
    strategies is considered. In this setting
    the most highly performing strategies do not play in an extortionate way
    against each other but do against lower performing strategies.
    This suggests that whilst the theory of Zero Determinant strategies
    indicates that memory is not of fundamental importance to the evolution of
    cooperative behaviour, this is incomplete.
\end{abstract}

\section{Introduction}\label{sec:introduction}

Agent based game theoretic models have become a stalwart of the underpinning
mathematics of interactive behaviours. One of the major pieces of work
in this area is the pair of original computer tournaments run by Robert
Axelrod~\cite{Axelrod1980, Axelrod1980a}. These tournaments pitted submitted
computer strategies against each other in plays of the Iterated Prisoner's
Dilemma. A common game where agents can choose to pay a slight cost to their
immediate utility in the hope of building a reputation. This has been used in
economic and evolutionary game theory to understand the evolution of cooperative
behaviour.

Recently, a class of strategies was described in~\cite{Press2012} that can
provably extort any given opponent. In~\cite{Hilbe2013, Moran1707} some
questions have already been asked about the true effectiveness of these
strategies in an evolutionary setting. Here another question is asked: is it
possible to recognise this extortionate behaviour? A mathematical procedure for
suspicion is presented: in the same way that the continued actions of an
extortionate individual might raise suspicion.

This work makes use of the Axelrod Python library~\cite{Knight2018, Knight2016}
with a large number of Prisoner Dilemma strategies available to give an
extensive numerical example of the ideas presented.  The approach is presented
in Section~\ref{sec:delta-zd-strategies}.  All of the code and data discussed
in Section~\ref{sec:numerical-experiments} is open sourced, archived and
written according to best scientific principles~\cite{Wilson2014}. The data
archive can be found at~\cite{vincent_knight_2018_1297075}.

\section{Recognising Extortion}\label{sec:delta-zd-strategies}

In~\cite{Press2012}, given a match between 2 memory-one strategies, the concept
of Zero Determinant (ZD) strategies is introduced. The main result of that paper
shows that given two memory one players \(p, q\in\mathbb{R}^4\) a linear
relationship between the players' scores could be forced by one of the players.

Using the notation of~\cite{Press2012}, assuming the utilities for player \(p\)
are given by \(S_x=(R, S, T, P)\) and for player \(q\) by \(S_y=(R, T, S, P)\)
and that the stationary scores of each player is given by \(S_X\) and \(S_Y\)
respectively. The main result of~\cite{Press2012} is that if

\begin{equation}\label{eqn:linear_relationship_for_p}
    \tilde p=\alpha S_x + \beta S_y + \gamma
\end{equation}

or

\begin{equation}\label{eqn:linear_relationship_for_q}
    \tilde q=\alpha S_x + \beta S_y + \gamma
\end{equation}

where \(\tilde p = (1 - p_1, 1 - p_2, p_3, p_4)\) and
\(\tilde q = (1 - q_1, 1 - q_2, q_3, q_4)\) then:

\begin{equation}
    \alpha S_X + \beta S_Y + \gamma = 0
\end{equation}

In~\cite{Press2012} a particular type of ZD strategy is defined: extortionate
strategies. If:

\begin{equation}\label{eqn:constraint_for_extortion}
    \gamma = - P(\alpha + \beta)
\end{equation}

then the player can ensure they get a score \(\chi\) times
larger than the opponent. This extortion coefficient is given by:

\begin{equation}\label{eqn:definition_of_chi}
    \chi=\frac{-\beta}{\alpha}
\end{equation}

Thus, if (\ref{eqn:constraint_for_extortion}) holds and \(\chi >1\) a player is
said to extort their opponent.
Here, the reverse problem is considered: given a
\(p\in\mathbb{R}^4\) how does one identify \(\alpha, \beta\) if they
exist and is the strategy in fact acting in an extortionate way?

These conditions correspond to:

\begin{align}
    \tilde p_1 & = \alpha R + \beta R - P (\alpha + \beta)
            \label{eqn:condition_for_tilde_p1}\\
    \tilde p_2 & = \alpha S + \beta T - P (\alpha + \beta)
            \label{eqn:condition_for_tilde_p2}\\
    \tilde p_3 & = \alpha T + \beta S - P (\alpha + \beta)
            \label{eqn:condition_for_tilde_p3}\\
    \tilde p_4 & = \alpha P + \beta P - P (\alpha + \beta)
            \label{eqn:condition_for_tilde_p4}
\end{align}

Equation (\ref{eqn:condition_for_tilde_p4}) ensures that \(p_4=\tilde p_4=0\).
Equations (\ref{eqn:condition_for_tilde_p1}-\ref{eqn:condition_for_tilde_p3})
can be used to eliminate \(\alpha, \beta\), giving:

\begin{equation}\label{eqn:planar_definition_of_extortion}
    \tilde p_1 = \frac{(R - P)(\tilde p_2 + \tilde p_3)}{S + T - 2P}
\end{equation}

with:

\begin{equation}\label{eqn:definition_of_chi}
    \chi = \frac{\tilde p_2 (P - T) + \tilde p_3 (S - P)}
                {\tilde p_2 (P - S) + \tilde p_3 (T - P)}
\end{equation}

Given a strategy \(p\in\mathbb{R}^{4\times 1}\) equations
(\ref{eqn:condition_for_tilde_p4}), (\ref{eqn:planar_definition_of_extortion}-\ref{eqn:definition_of_chi}) can be used to check if
a strategy is extortionate. The conditions correspond to:

\begin{align}
    p_1 & = \frac{(R-P)(p_2 + p_3) - R + T + S - P}{S + T - 2P}
     \label{eqn:condition_for_p1}\\
    p_4 & = 0 \label{eqn:condition_for_p4}\\
    1 & > p_2 + p_3\label{eqn:condition_for_chi}
\end{align}

The algebraic steps necessary to prove these results are available in the
supporting materials.

All extortionate strategies reside on a triangular (\ref{eqn:condition_for_chi})
plane (\ref{eqn:condition_for_p1}) in 3 dimensions (\ref{eqn:condition_for_p4}).
Using this formulation it can be seen that a necessary (but not sufficient)
condition for an extortionate strategy is that it cooperates on average less
than 50\% of the time when in a state of disagreement with the opponent.

As an example, consider the known extortionate strategy \(p=(8 / 9, 1 / 2, 1 /
3, 0)\) from~\cite{Stewart2012} which is referred to as \texttt{Extort-2}. In
this case, for the standard values of \((R, T, S, P)\) constraint
(\ref{eqn:condition_for_p1}) corresponds to:

\begin{equation}
    p_1 = \frac{2(p_2 + p_3) + 1}{3}
\end{equation}

It is clear that in this case all constraints hold.

This approach could in fact be used to confirm that a given strategy is acting
in an extortionate manner even if it is not a memory one strategy. However, in
practice, if a closed form for \(p\) is not known, then due to measurement
and/or numerical error this would not work.

This problem can be written in the following linear algebraic form where
\(x=(\alpha, \beta)\)
and \(p^*=(\tilde p_1 - 1, tilde_2 - 1, p_3)\):

\begin{equation}\label{eqn:linear_algebraic_equation_for_p}
    Cx= p^*
\end{equation}

\(C\) corresponds to equations
(\ref{eqn:condition_for_tilde_p1}-\ref{eqn:condition_for_tilde_p3}) and is
given by:

\begin{equation}\label{eqn:definition_of_C}
    C =
    \begin{bmatrix}
        R - P & R- P \\
        S - P & T- P \\
        T - P & S- P \\
    \end{bmatrix}
\end{equation}

Note that in general, equation (\ref{eqn:linear_algebraic_equation_for_p}) will
not necessarily have a solution. From the Rouch\'{e}-Capelli theorem if there is
a solution it is unique as \(\text{rank}(C)=2\) which is the dimension of the
variable \(x\). The best fitting \(x\) is found by minimizing:

\begin{equation}\label{eqn:r_squared}
    \text{SSError} = \|C x- p^*\|_2^2 = \sum_{i=1}^{3}\left((C\bar x)_i-p_i^*\right)^2
\end{equation}

Note that \(\text{SSError}\), which is the square of the Frobenius
norm~\cite{Golub2013}, becomes a measure of how close a strategy is to being an
extortionate strategy. Suspicion
of extortion then corresponds to a threshold on \(\text{SSError}\).

By observing interactions (human or otherwise), their memory one representation
can be inferred and this approach can be used to recognise extortionate
behaviour. The notion of comparing theoretic and actual plays of the IPD is not
novel, see for example~\cite{Rand2013}. Immediately it is noted that if the
environment is noisy~\cite{Wu1995} then no strategy can be considered to be
extortionate as \(p_4>0\).

In the next section, this idea will be illustrated by observing the interactions
that take place in a computer based tournament of the IPD\@.

\section{Numerical experiments}\label{sec:numerical-experiments}

In~\cite{Stewart2012} results from a tournament with
\input{./assets/tex/number_of_stewart_plotkin_strategies/main.tex} strategies,
was presented with specific consideration given to ZD strategies. This
tournament is reproduced here using the Axelrod-Python
project~\cite{Knight2016}. To obtain a good measure of the corresponding
transition rates for each strategy all matches have been run for
\input{assets/tex/number_of_turns/main.tex} turns and every match has been
repeated \input{assets/tex/number_of_repetitions/main.tex} times. All of this
interaction data is available at~\cite{vincent_knight_2018_1297075}. A good
match between the inferred Markov chain and the state distribution of the actual
interactions has been verified. Data for this is presented in the supplementary
materials.

Figure~\ref{fig:SSError_overall_in_stewart_plotkin} shows the \(\text{SSError}\)
values for all the strategies in the tournament, as reported
in~\cite{Stewart2012} the extortionate strategy (which has an expected
\(\text{SSError}\) approximately 0) gains a large number of wins.

\begin{figure}[!htbp]
    \centering
    \includegraphics[width=.8\textwidth]{./assets/img/SSError_overall_in_stewart_plotkin/main.pdf}
    \caption{\(\text{SSError}\) and state probabilities for the strategies
        of~\cite{Stewart2012}, ordered both by number of wins and overall score.
        Note that \(P(DC)\) is not shown as it corresponds to the transpose of
        \(P(CD)\). Cooperator and Defector are omitted as they do not visit all
        the states.}
    \label{fig:SSError_overall_in_stewart_plotkin}
\end{figure}

Here, the work of~\cite{Stewart2012} is extended by investigating a tournament
with \input{assets/tex/number_of_full_strategies/main.tex}
strategies.

The results of this analysis are shown in
Figure~\ref{fig:SSError_and_probabilities_in_full}. The top ranking strategies
by number of wins seem to be extortionate (but not against all strategies) and
it can be seen that a small sub group of strategies achieve mutual defection.
All the top ranking strategies according to score achieve mutual cooperation and
do not extort each other, however they
\textbf{do} exhibit extortionate behaviour towards a number of the lower ranking
strategies.

\begin{figure}[!htbp]
    \centering
    \includegraphics[width=.8\textwidth]{./assets/img/SSError_and_probabilities_in_full/main.pdf}
    \caption{\(\text{SSError}\) for the strategies for the full tournament. Only
    strategy interactions for which \(p_4=0\) and \(\chi>1\) are displayed.}
    \label{fig:SSError_and_probabilities_in_full}
\end{figure}

\section{Conclusion}\label{sec:conclusion}

This work defines an approach to measure whether or not a player is playing a
strategy that corresponds to an extortionate strategy as defined
in~\cite{Press2012}: a mathematical model for suspicion. Indeed, all
extortionate strategies have been
 classified as lying on a triangular plane.
This rigorous classification fails to be robust to small measurement error, thus
a statistical approach is proposed.
This is done through a linear algebraic approach for approximating the solution
of a linear system. Using this, a large number of pairwise interactions is
simulated and in fact very few strategies are found to act extortionately.

The work of~\cite{Press2012}, whilst showing that a clever approach to taking
advantage of another memory one strategy exists: this is incomplete. Whilst the
elegance of this result is very attractive, just as the simplicity of the
victory of Tit For Tat in Axelrod's original tournaments was, it is incomplete.
Extortionate strategies achieve a high number of wins but they do not
achieve a high score which corresponds to the fitness landscape in an
evolutionary sense. From the large number of interactions a payoff matrix \(S\)
can be measured where \(S_{ij}\) denotes the score (using standard values of
\((R, S, T, P) = (3, 0, 5, 1)\)) of the \(i\)th strategy
against the \(j\)th strategy. Using this, the replicator equation
describes the evolution of the system based on a population density fitness
function:

\begin{equation}\label{eqn:replicator_dynamics}
    \frac{dx}{dt} = x(S-x^TS x)
\end{equation}

Equation (\ref{eqn:replicator_dynamics}) is solved numerically through an
integration technique described in~\cite{Petzold1983} and
Figure~\ref{fig:replicator_dynamics} shows the evolution of the distribution of
the system: the various strategies are ranked by scores. It is clear to see that
only the high ranking strategies survive the evolutionary process (in fact,
only \input{./assets/img/replicator_dynamics/main.tex}
have a final distribution greater than \(10 ^ {-2}\)). This confirms the
findings of~\cite{Moran1707} in which sophisticated strategies resist
evolutionary invasion of shorter memory strategies. Recalling
Figure~\ref{fig:SSError_and_probabilities_in_full} this demonstrates that:

\begin{itemize}
    \item Cooperation emerges through the evolutionary process: the high scoring
        strategies do not exhibit extortionate behaviour towards each other.
    \item Extortionate strategies do not survive the evolutionary process.
\end{itemize}

\begin{figure}[!htbp]
    \centering
    \includegraphics[width=.8\textwidth]{./assets/img/replicator_dynamics/main.pdf}
    \caption{Numerical simulation of the replicator equation
    (\ref{eqn:replicator_dynamics}): strategies are ordered by score, only the strategies with a high score survive the evolutionary process.}
    \label{fig:replicator_dynamics}
\end{figure}

This work can be used to classify plays of the IPD\@: data can be collected from
actual interactions (in lab or in the field). Furthermore, this allows for a
classification method similar to the notion of fingerprinting presented
in~\cite{Ashlock2008}. Trained strategies can potentially be classified as
extortionate or not or it could be possible to even constrain the reinforcement
learning approaches that are becoming prevalent in the literature.
Alternatively, this mathematical approach for recognising extortion could be
used in sophisticated strategies to defend against invasion. Arguably, some of
the strategies considered here exhibit this behaviour, indeed as described
in~\cite{Harper2017}, the top ranking strategies in the full tournament are
obtained using evolutionary reinforcement learning techniques, thus, suspicion
of extortionate behaviour could in fact be an evolutionary trait.

\section*{Acknowledgements}

The following open source software libraries were used in this research:

\begin{itemize}
    \item The Axelrod ~\cite{Knight2016, Knight2018} library (IPD strategies and
        tournaments).
    \item The sympy library~\cite{Meurer2017} (verification of all symbolic
        calculations).
    \item The matplotlib~\cite{Droettboom2018} library (visualisation).
    \item The pandas~\cite{Structures2010}, dask~\cite{Dask2016} and
        NumPy~\cite{Oliphant2015} libraries (data manipulation).
    \item The SciPy~\cite{Jones2001} library (numerical integration of the
        replicator equation).
\end{itemize}

This work was performed using the computational facilities of the Advanced
Research Computing @ Cardiff (ARCCA) Division, Cardiff University.

\printbibliography

\newpage
\section*{Supplementary materials}

\includepdf{assets/pdf/proof_of_form_of_extortionate_strategies/main.pdf}

\newpage

Using the pair wise interactions the transition rates \(p,
q\) can be measured and the steady state probabilities inferred and compared to
the actual probabilities of each state.
This is done numerically by computing the singular eigenvector of the
matrix \(A\) \cite{Stewart2009}:

\[
    A =
    \begin{bmatrix}
        p_1 q_1 & p_1 (1 - q_1) & (1 - p_1) q_1 & (1 -p_1) (1 - q_1) \\
        p_2 q_2 & p_2 (1 - q_2) & (1 - p_2) q_2 & (1 -p_2) (1 - q_2) \\
        p_3 q_3 & p_3 (1 - q_3) & (1 - p_3) q_3 & (1 -p_3) (1 - q_3) \\
        p_4 q_4 & p_4 (1 - q_4) & (1 - p_4) q_4 & (1 -p_4) (1 - q_4) \\
    \end{bmatrix}
\]

Figure~\ref{fig:computed_probabilities_vs_theoretic_probabilities} shows a
regression line fitted to every pairwise interaction with a reported
\(\text{SSError}\) value (pairwise interactions with missing states were
omitted). This serves to validate the approach: a part from some edge cases the
relationship is consistent.

\begin{figure}[!htbp]
    \centering
    \includegraphics[width=.8\textwidth]{./assets/img/computed_probabilities_vs_theoretic_probabilities/main.pdf}
    \caption{The
        relationship between the steady state probabilities inferred from the
        measured transitions and the actual steady state probabilities. A linear
        regression line is included validating the approach.}
    \label{fig:computed_probabilities_vs_theoretic_probabilities}
\end{figure}


\end{document}

    strategies is considered. In this setting
    the most highly performing strategies do not play in an extortionate way
    against each other but do against lower performing strategies.
    This suggests that whilst the theory of Zero Determinant strategies
    indicates that memory is not of fundamental importance to the evolution of
    cooperative behaviour, this is incomplete.
\end{abstract}

\section{Introduction}\label{sec:introduction}

Agent based game theoretic models have become a stalwart of the underpinning
mathematics of interactive behaviours. One of the major pieces of work
in this area is the pair of original computer tournaments run by Robert
Axelrod~\cite{Axelrod1980, Axelrod1980a}. These tournaments pitted submitted
computer strategies against each other in plays of the Iterated Prisoner's
Dilemma. A common game where agents can choose to pay a slight cost to their
immediate utility in the hope of building a reputation. This has been used in
economic and evolutionary game theory to understand the evolution of cooperative
behaviour.

Recently, a class of strategies was described in~\cite{Press2012} that can
provably extort any given opponent. In~\cite{Hilbe2013, Moran1707} some
questions have already been asked about the true effectiveness of these
strategies in an evolutionary setting. Here another question is asked: is it
possible to recognise this extortionate behaviour? A mathematical procedure for
suspicion is presented: in the same way that the continued actions of an
extortionate individual might raise suspicion.

This work makes use of the Axelrod Python library~\cite{Knight2018, Knight2016}
with a large number of Prisoner Dilemma strategies available to give an
extensive numerical example of the ideas presented.  The approach is presented
in Section~\ref{sec:delta-zd-strategies}.  All of the code and data discussed
in Section~\ref{sec:numerical-experiments} is open sourced, archived and
written according to best scientific principles~\cite{Wilson2014}. The data
archive can be found at~\cite{vincent_knight_2018_1297075}.

\section{Recognising Extortion}\label{sec:delta-zd-strategies}

In~\cite{Press2012}, given a match between 2 memory-one strategies, the concept
of Zero Determinant (ZD) strategies is introduced. The main result of that paper
shows that given two memory one players \(p, q\in\mathbb{R}^4\) a linear
relationship between the players' scores could be forced by one of the players.

Using the notation of~\cite{Press2012}, assuming the utilities for player \(p\)
are given by \(S_x=(R, S, T, P)\) and for player \(q\) by \(S_y=(R, T, S, P)\)
and that the stationary scores of each player is given by \(S_X\) and \(S_Y\)
respectively. The main result of~\cite{Press2012} is that if

\begin{equation}\label{eqn:linear_relationship_for_p}
    \tilde p=\alpha S_x + \beta S_y + \gamma
\end{equation}

or

\begin{equation}\label{eqn:linear_relationship_for_q}
    \tilde q=\alpha S_x + \beta S_y + \gamma
\end{equation}

where \(\tilde p = (1 - p_1, 1 - p_2, p_3, p_4)\) and
\(\tilde q = (1 - q_1, 1 - q_2, q_3, q_4)\) then:

\begin{equation}
    \alpha S_X + \beta S_Y + \gamma = 0
\end{equation}

In~\cite{Press2012} a particular type of ZD strategy is defined: extortionate
strategies. If:

\begin{equation}\label{eqn:constraint_for_extortion}
    \gamma = - P(\alpha + \beta)
\end{equation}

then the player can ensure they get a score \(\chi\) times
larger than the opponent. This extortion coefficient is given by:

\begin{equation}\label{eqn:definition_of_chi}
    \chi=\frac{-\beta}{\alpha}
\end{equation}

Thus, if (\ref{eqn:constraint_for_extortion}) holds and \(\chi >1\) a player is
said to extort their opponent.
Here, the reverse problem is considered: given a
\(p\in\mathbb{R}^4\) how does one identify \(\alpha, \beta\) if they
exist and is the strategy in fact acting in an extortionate way?

These conditions correspond to:

\begin{align}
    \tilde p_1 & = \alpha R + \beta R - P (\alpha + \beta)
            \label{eqn:condition_for_tilde_p1}\\
    \tilde p_2 & = \alpha S + \beta T - P (\alpha + \beta)
            \label{eqn:condition_for_tilde_p2}\\
    \tilde p_3 & = \alpha T + \beta S - P (\alpha + \beta)
            \label{eqn:condition_for_tilde_p3}\\
    \tilde p_4 & = \alpha P + \beta P - P (\alpha + \beta)
            \label{eqn:condition_for_tilde_p4}
\end{align}

Equation (\ref{eqn:condition_for_tilde_p4}) ensures that \(p_4=\tilde p_4=0\).
Equations (\ref{eqn:condition_for_tilde_p1}-\ref{eqn:condition_for_tilde_p3})
can be used to eliminate \(\alpha, \beta\), giving:

\begin{equation}\label{eqn:planar_definition_of_extortion}
    \tilde p_1 = \frac{(R - P)(\tilde p_2 + \tilde p_3)}{S + T - 2P}
\end{equation}

with:

\begin{equation}\label{eqn:definition_of_chi}
    \chi = \frac{\tilde p_2 (P - T) + \tilde p_3 (S - P)}
                {\tilde p_2 (P - S) + \tilde p_3 (T - P)}
\end{equation}

Given a strategy \(p\in\mathbb{R}^{4\times 1}\) equations
(\ref{eqn:condition_for_tilde_p4}), (\ref{eqn:planar_definition_of_extortion}-\ref{eqn:definition_of_chi}) can be used to check if
a strategy is extortionate. The conditions correspond to:

\begin{align}
    p_1 & = \frac{(R-P)(p_2 + p_3) - R + T + S - P}{S + T - 2P}
     \label{eqn:condition_for_p1}\\
    p_4 & = 0 \label{eqn:condition_for_p4}\\
    1 & > p_2 + p_3\label{eqn:condition_for_chi}
\end{align}

The algebraic steps necessary to prove these results are available in the
supporting materials.

All extortionate strategies reside on a triangular (\ref{eqn:condition_for_chi})
plane (\ref{eqn:condition_for_p1}) in 3 dimensions (\ref{eqn:condition_for_p4}).
Using this formulation it can be seen that a necessary (but not sufficient)
condition for an extortionate strategy is that it cooperates on average less
than 50\% of the time when in a state of disagreement with the opponent.

As an example, consider the known extortionate strategy \(p=(8 / 9, 1 / 2, 1 /
3, 0)\) from~\cite{Stewart2012} which is referred to as \texttt{Extort-2}. In
this case, for the standard values of \((R, T, S, P)\) constraint
(\ref{eqn:condition_for_p1}) corresponds to:

\begin{equation}
    p_1 = \frac{2(p_2 + p_3) + 1}{3}
\end{equation}

It is clear that in this case all constraints hold.

This approach could in fact be used to confirm that a given strategy is acting
in an extortionate manner even if it is not a memory one strategy. However, in
practice, if a closed form for \(p\) is not known, then due to measurement
and/or numerical error this would not work.

This problem can be written in the following linear algebraic form where
\(x=(\alpha, \beta)\)
and \(p^*=(\tilde p_1 - 1, tilde_2 - 1, p_3)\):

\begin{equation}\label{eqn:linear_algebraic_equation_for_p}
    Cx= p^*
\end{equation}

\(C\) corresponds to equations
(\ref{eqn:condition_for_tilde_p1}-\ref{eqn:condition_for_tilde_p3}) and is
given by:

\begin{equation}\label{eqn:definition_of_C}
    C =
    \begin{bmatrix}
        R - P & R- P \\
        S - P & T- P \\
        T - P & S- P \\
    \end{bmatrix}
\end{equation}

Note that in general, equation (\ref{eqn:linear_algebraic_equation_for_p}) will
not necessarily have a solution. From the Rouch\'{e}-Capelli theorem if there is
a solution it is unique as \(\text{rank}(C)=2\) which is the dimension of the
variable \(x\). The best fitting \(x\) is found by minimizing:

\begin{equation}\label{eqn:r_squared}
    \text{SSError} = \|C x- p^*\|_2^2 = \sum_{i=1}^{3}\left((C\bar x)_i-p_i^*\right)^2
\end{equation}

Note that \(\text{SSError}\), which is the square of the Frobenius
norm~\cite{Golub2013}, becomes a measure of how close a strategy is to being an
extortionate strategy. Suspicion
of extortion then corresponds to a threshold on \(\text{SSError}\).

By observing interactions (human or otherwise), their memory one representation
can be inferred and this approach can be used to recognise extortionate
behaviour. The notion of comparing theoretic and actual plays of the IPD is not
novel, see for example~\cite{Rand2013}. Immediately it is noted that if the
environment is noisy~\cite{Wu1995} then no strategy can be considered to be
extortionate as \(p_4>0\).

In the next section, this idea will be illustrated by observing the interactions
that take place in a computer based tournament of the IPD\@.

\section{Numerical experiments}\label{sec:numerical-experiments}

In~\cite{Stewart2012} results from a tournament with
\documentclass[a4paper]{article}

\usepackage{amsmath}
\usepackage{amssymb}
\usepackage[margin=1.5cm,
            includefoot,
            footskip=30pt]{geometry}
\usepackage{layout}
\usepackage{graphicx}
\usepackage{subcaption}

\usepackage{biblatex}
\usepackage{pdfpages}

\bibliography{main.bib}

\title{Suspicion: Recognising and evaluating the effectiveness
       of extortion in the Iterated Prisoner's Dilemma}
\author{Vincent A. Knight \and Nikoleta E. Glynatsi}
\date{\today}



\begin{document}

\maketitle

\begin{abstract}
    The Iterated Prisoner's Dilemma is a model for rational and evolutionary
    interactive behaviour. It has applications both in the study of human social
    behaviour as well as in biology.
    It is used to understand when and how a rational individual might
    accept an immediate cost to their own utility for the direct benefit of
    another.

    Much attention has been given to a class of strategies called
    Zero Determinant strategies. It has been theoretically shown that these
    strategies can ``extort'' any player.

    In this work, an approach to identify if observed strategies are playing in
    an extortionate way is described. Furthermore, experimental analysis of
    a large tournament with \input{assets/tex/number_of_full_strategies/main.tex}
    strategies is considered. In this setting
    the most highly performing strategies do not play in an extortionate way
    against each other but do against lower performing strategies.
    This suggests that whilst the theory of Zero Determinant strategies
    indicates that memory is not of fundamental importance to the evolution of
    cooperative behaviour, this is incomplete.
\end{abstract}

\section{Introduction}\label{sec:introduction}

Agent based game theoretic models have become a stalwart of the underpinning
mathematics of interactive behaviours. One of the major pieces of work
in this area is the pair of original computer tournaments run by Robert
Axelrod~\cite{Axelrod1980, Axelrod1980a}. These tournaments pitted submitted
computer strategies against each other in plays of the Iterated Prisoner's
Dilemma. A common game where agents can choose to pay a slight cost to their
immediate utility in the hope of building a reputation. This has been used in
economic and evolutionary game theory to understand the evolution of cooperative
behaviour.

Recently, a class of strategies was described in~\cite{Press2012} that can
provably extort any given opponent. In~\cite{Hilbe2013, Moran1707} some
questions have already been asked about the true effectiveness of these
strategies in an evolutionary setting. Here another question is asked: is it
possible to recognise this extortionate behaviour? A mathematical procedure for
suspicion is presented: in the same way that the continued actions of an
extortionate individual might raise suspicion.

This work makes use of the Axelrod Python library~\cite{Knight2018, Knight2016}
with a large number of Prisoner Dilemma strategies available to give an
extensive numerical example of the ideas presented.  The approach is presented
in Section~\ref{sec:delta-zd-strategies}.  All of the code and data discussed
in Section~\ref{sec:numerical-experiments} is open sourced, archived and
written according to best scientific principles~\cite{Wilson2014}. The data
archive can be found at~\cite{vincent_knight_2018_1297075}.

\section{Recognising Extortion}\label{sec:delta-zd-strategies}

In~\cite{Press2012}, given a match between 2 memory-one strategies, the concept
of Zero Determinant (ZD) strategies is introduced. The main result of that paper
shows that given two memory one players \(p, q\in\mathbb{R}^4\) a linear
relationship between the players' scores could be forced by one of the players.

Using the notation of~\cite{Press2012}, assuming the utilities for player \(p\)
are given by \(S_x=(R, S, T, P)\) and for player \(q\) by \(S_y=(R, T, S, P)\)
and that the stationary scores of each player is given by \(S_X\) and \(S_Y\)
respectively. The main result of~\cite{Press2012} is that if

\begin{equation}\label{eqn:linear_relationship_for_p}
    \tilde p=\alpha S_x + \beta S_y + \gamma
\end{equation}

or

\begin{equation}\label{eqn:linear_relationship_for_q}
    \tilde q=\alpha S_x + \beta S_y + \gamma
\end{equation}

where \(\tilde p = (1 - p_1, 1 - p_2, p_3, p_4)\) and
\(\tilde q = (1 - q_1, 1 - q_2, q_3, q_4)\) then:

\begin{equation}
    \alpha S_X + \beta S_Y + \gamma = 0
\end{equation}

In~\cite{Press2012} a particular type of ZD strategy is defined: extortionate
strategies. If:

\begin{equation}\label{eqn:constraint_for_extortion}
    \gamma = - P(\alpha + \beta)
\end{equation}

then the player can ensure they get a score \(\chi\) times
larger than the opponent. This extortion coefficient is given by:

\begin{equation}\label{eqn:definition_of_chi}
    \chi=\frac{-\beta}{\alpha}
\end{equation}

Thus, if (\ref{eqn:constraint_for_extortion}) holds and \(\chi >1\) a player is
said to extort their opponent.
Here, the reverse problem is considered: given a
\(p\in\mathbb{R}^4\) how does one identify \(\alpha, \beta\) if they
exist and is the strategy in fact acting in an extortionate way?

These conditions correspond to:

\begin{align}
    \tilde p_1 & = \alpha R + \beta R - P (\alpha + \beta)
            \label{eqn:condition_for_tilde_p1}\\
    \tilde p_2 & = \alpha S + \beta T - P (\alpha + \beta)
            \label{eqn:condition_for_tilde_p2}\\
    \tilde p_3 & = \alpha T + \beta S - P (\alpha + \beta)
            \label{eqn:condition_for_tilde_p3}\\
    \tilde p_4 & = \alpha P + \beta P - P (\alpha + \beta)
            \label{eqn:condition_for_tilde_p4}
\end{align}

Equation (\ref{eqn:condition_for_tilde_p4}) ensures that \(p_4=\tilde p_4=0\).
Equations (\ref{eqn:condition_for_tilde_p1}-\ref{eqn:condition_for_tilde_p3})
can be used to eliminate \(\alpha, \beta\), giving:

\begin{equation}\label{eqn:planar_definition_of_extortion}
    \tilde p_1 = \frac{(R - P)(\tilde p_2 + \tilde p_3)}{S + T - 2P}
\end{equation}

with:

\begin{equation}\label{eqn:definition_of_chi}
    \chi = \frac{\tilde p_2 (P - T) + \tilde p_3 (S - P)}
                {\tilde p_2 (P - S) + \tilde p_3 (T - P)}
\end{equation}

Given a strategy \(p\in\mathbb{R}^{4\times 1}\) equations
(\ref{eqn:condition_for_tilde_p4}), (\ref{eqn:planar_definition_of_extortion}-\ref{eqn:definition_of_chi}) can be used to check if
a strategy is extortionate. The conditions correspond to:

\begin{align}
    p_1 & = \frac{(R-P)(p_2 + p_3) - R + T + S - P}{S + T - 2P}
     \label{eqn:condition_for_p1}\\
    p_4 & = 0 \label{eqn:condition_for_p4}\\
    1 & > p_2 + p_3\label{eqn:condition_for_chi}
\end{align}

The algebraic steps necessary to prove these results are available in the
supporting materials.

All extortionate strategies reside on a triangular (\ref{eqn:condition_for_chi})
plane (\ref{eqn:condition_for_p1}) in 3 dimensions (\ref{eqn:condition_for_p4}).
Using this formulation it can be seen that a necessary (but not sufficient)
condition for an extortionate strategy is that it cooperates on average less
than 50\% of the time when in a state of disagreement with the opponent.

As an example, consider the known extortionate strategy \(p=(8 / 9, 1 / 2, 1 /
3, 0)\) from~\cite{Stewart2012} which is referred to as \texttt{Extort-2}. In
this case, for the standard values of \((R, T, S, P)\) constraint
(\ref{eqn:condition_for_p1}) corresponds to:

\begin{equation}
    p_1 = \frac{2(p_2 + p_3) + 1}{3}
\end{equation}

It is clear that in this case all constraints hold.

This approach could in fact be used to confirm that a given strategy is acting
in an extortionate manner even if it is not a memory one strategy. However, in
practice, if a closed form for \(p\) is not known, then due to measurement
and/or numerical error this would not work.

This problem can be written in the following linear algebraic form where
\(x=(\alpha, \beta)\)
and \(p^*=(\tilde p_1 - 1, tilde_2 - 1, p_3)\):

\begin{equation}\label{eqn:linear_algebraic_equation_for_p}
    Cx= p^*
\end{equation}

\(C\) corresponds to equations
(\ref{eqn:condition_for_tilde_p1}-\ref{eqn:condition_for_tilde_p3}) and is
given by:

\begin{equation}\label{eqn:definition_of_C}
    C =
    \begin{bmatrix}
        R - P & R- P \\
        S - P & T- P \\
        T - P & S- P \\
    \end{bmatrix}
\end{equation}

Note that in general, equation (\ref{eqn:linear_algebraic_equation_for_p}) will
not necessarily have a solution. From the Rouch\'{e}-Capelli theorem if there is
a solution it is unique as \(\text{rank}(C)=2\) which is the dimension of the
variable \(x\). The best fitting \(x\) is found by minimizing:

\begin{equation}\label{eqn:r_squared}
    \text{SSError} = \|C x- p^*\|_2^2 = \sum_{i=1}^{3}\left((C\bar x)_i-p_i^*\right)^2
\end{equation}

Note that \(\text{SSError}\), which is the square of the Frobenius
norm~\cite{Golub2013}, becomes a measure of how close a strategy is to being an
extortionate strategy. Suspicion
of extortion then corresponds to a threshold on \(\text{SSError}\).

By observing interactions (human or otherwise), their memory one representation
can be inferred and this approach can be used to recognise extortionate
behaviour. The notion of comparing theoretic and actual plays of the IPD is not
novel, see for example~\cite{Rand2013}. Immediately it is noted that if the
environment is noisy~\cite{Wu1995} then no strategy can be considered to be
extortionate as \(p_4>0\).

In the next section, this idea will be illustrated by observing the interactions
that take place in a computer based tournament of the IPD\@.

\section{Numerical experiments}\label{sec:numerical-experiments}

In~\cite{Stewart2012} results from a tournament with
\input{./assets/tex/number_of_stewart_plotkin_strategies/main.tex} strategies,
was presented with specific consideration given to ZD strategies. This
tournament is reproduced here using the Axelrod-Python
project~\cite{Knight2016}. To obtain a good measure of the corresponding
transition rates for each strategy all matches have been run for
\input{assets/tex/number_of_turns/main.tex} turns and every match has been
repeated \input{assets/tex/number_of_repetitions/main.tex} times. All of this
interaction data is available at~\cite{vincent_knight_2018_1297075}. A good
match between the inferred Markov chain and the state distribution of the actual
interactions has been verified. Data for this is presented in the supplementary
materials.

Figure~\ref{fig:SSError_overall_in_stewart_plotkin} shows the \(\text{SSError}\)
values for all the strategies in the tournament, as reported
in~\cite{Stewart2012} the extortionate strategy (which has an expected
\(\text{SSError}\) approximately 0) gains a large number of wins.

\begin{figure}[!htbp]
    \centering
    \includegraphics[width=.8\textwidth]{./assets/img/SSError_overall_in_stewart_plotkin/main.pdf}
    \caption{\(\text{SSError}\) and state probabilities for the strategies
        of~\cite{Stewart2012}, ordered both by number of wins and overall score.
        Note that \(P(DC)\) is not shown as it corresponds to the transpose of
        \(P(CD)\). Cooperator and Defector are omitted as they do not visit all
        the states.}
    \label{fig:SSError_overall_in_stewart_plotkin}
\end{figure}

Here, the work of~\cite{Stewart2012} is extended by investigating a tournament
with \input{assets/tex/number_of_full_strategies/main.tex}
strategies.

The results of this analysis are shown in
Figure~\ref{fig:SSError_and_probabilities_in_full}. The top ranking strategies
by number of wins seem to be extortionate (but not against all strategies) and
it can be seen that a small sub group of strategies achieve mutual defection.
All the top ranking strategies according to score achieve mutual cooperation and
do not extort each other, however they
\textbf{do} exhibit extortionate behaviour towards a number of the lower ranking
strategies.

\begin{figure}[!htbp]
    \centering
    \includegraphics[width=.8\textwidth]{./assets/img/SSError_and_probabilities_in_full/main.pdf}
    \caption{\(\text{SSError}\) for the strategies for the full tournament. Only
    strategy interactions for which \(p_4=0\) and \(\chi>1\) are displayed.}
    \label{fig:SSError_and_probabilities_in_full}
\end{figure}

\section{Conclusion}\label{sec:conclusion}

This work defines an approach to measure whether or not a player is playing a
strategy that corresponds to an extortionate strategy as defined
in~\cite{Press2012}: a mathematical model for suspicion. Indeed, all
extortionate strategies have been
 classified as lying on a triangular plane.
This rigorous classification fails to be robust to small measurement error, thus
a statistical approach is proposed.
This is done through a linear algebraic approach for approximating the solution
of a linear system. Using this, a large number of pairwise interactions is
simulated and in fact very few strategies are found to act extortionately.

The work of~\cite{Press2012}, whilst showing that a clever approach to taking
advantage of another memory one strategy exists: this is incomplete. Whilst the
elegance of this result is very attractive, just as the simplicity of the
victory of Tit For Tat in Axelrod's original tournaments was, it is incomplete.
Extortionate strategies achieve a high number of wins but they do not
achieve a high score which corresponds to the fitness landscape in an
evolutionary sense. From the large number of interactions a payoff matrix \(S\)
can be measured where \(S_{ij}\) denotes the score (using standard values of
\((R, S, T, P) = (3, 0, 5, 1)\)) of the \(i\)th strategy
against the \(j\)th strategy. Using this, the replicator equation
describes the evolution of the system based on a population density fitness
function:

\begin{equation}\label{eqn:replicator_dynamics}
    \frac{dx}{dt} = x(S-x^TS x)
\end{equation}

Equation (\ref{eqn:replicator_dynamics}) is solved numerically through an
integration technique described in~\cite{Petzold1983} and
Figure~\ref{fig:replicator_dynamics} shows the evolution of the distribution of
the system: the various strategies are ranked by scores. It is clear to see that
only the high ranking strategies survive the evolutionary process (in fact,
only \input{./assets/img/replicator_dynamics/main.tex}
have a final distribution greater than \(10 ^ {-2}\)). This confirms the
findings of~\cite{Moran1707} in which sophisticated strategies resist
evolutionary invasion of shorter memory strategies. Recalling
Figure~\ref{fig:SSError_and_probabilities_in_full} this demonstrates that:

\begin{itemize}
    \item Cooperation emerges through the evolutionary process: the high scoring
        strategies do not exhibit extortionate behaviour towards each other.
    \item Extortionate strategies do not survive the evolutionary process.
\end{itemize}

\begin{figure}[!htbp]
    \centering
    \includegraphics[width=.8\textwidth]{./assets/img/replicator_dynamics/main.pdf}
    \caption{Numerical simulation of the replicator equation
    (\ref{eqn:replicator_dynamics}): strategies are ordered by score, only the strategies with a high score survive the evolutionary process.}
    \label{fig:replicator_dynamics}
\end{figure}

This work can be used to classify plays of the IPD\@: data can be collected from
actual interactions (in lab or in the field). Furthermore, this allows for a
classification method similar to the notion of fingerprinting presented
in~\cite{Ashlock2008}. Trained strategies can potentially be classified as
extortionate or not or it could be possible to even constrain the reinforcement
learning approaches that are becoming prevalent in the literature.
Alternatively, this mathematical approach for recognising extortion could be
used in sophisticated strategies to defend against invasion. Arguably, some of
the strategies considered here exhibit this behaviour, indeed as described
in~\cite{Harper2017}, the top ranking strategies in the full tournament are
obtained using evolutionary reinforcement learning techniques, thus, suspicion
of extortionate behaviour could in fact be an evolutionary trait.

\section*{Acknowledgements}

The following open source software libraries were used in this research:

\begin{itemize}
    \item The Axelrod ~\cite{Knight2016, Knight2018} library (IPD strategies and
        tournaments).
    \item The sympy library~\cite{Meurer2017} (verification of all symbolic
        calculations).
    \item The matplotlib~\cite{Droettboom2018} library (visualisation).
    \item The pandas~\cite{Structures2010}, dask~\cite{Dask2016} and
        NumPy~\cite{Oliphant2015} libraries (data manipulation).
    \item The SciPy~\cite{Jones2001} library (numerical integration of the
        replicator equation).
\end{itemize}

This work was performed using the computational facilities of the Advanced
Research Computing @ Cardiff (ARCCA) Division, Cardiff University.

\printbibliography

\newpage
\section*{Supplementary materials}

\includepdf{assets/pdf/proof_of_form_of_extortionate_strategies/main.pdf}

\newpage

Using the pair wise interactions the transition rates \(p,
q\) can be measured and the steady state probabilities inferred and compared to
the actual probabilities of each state.
This is done numerically by computing the singular eigenvector of the
matrix \(A\) \cite{Stewart2009}:

\[
    A =
    \begin{bmatrix}
        p_1 q_1 & p_1 (1 - q_1) & (1 - p_1) q_1 & (1 -p_1) (1 - q_1) \\
        p_2 q_2 & p_2 (1 - q_2) & (1 - p_2) q_2 & (1 -p_2) (1 - q_2) \\
        p_3 q_3 & p_3 (1 - q_3) & (1 - p_3) q_3 & (1 -p_3) (1 - q_3) \\
        p_4 q_4 & p_4 (1 - q_4) & (1 - p_4) q_4 & (1 -p_4) (1 - q_4) \\
    \end{bmatrix}
\]

Figure~\ref{fig:computed_probabilities_vs_theoretic_probabilities} shows a
regression line fitted to every pairwise interaction with a reported
\(\text{SSError}\) value (pairwise interactions with missing states were
omitted). This serves to validate the approach: a part from some edge cases the
relationship is consistent.

\begin{figure}[!htbp]
    \centering
    \includegraphics[width=.8\textwidth]{./assets/img/computed_probabilities_vs_theoretic_probabilities/main.pdf}
    \caption{The
        relationship between the steady state probabilities inferred from the
        measured transitions and the actual steady state probabilities. A linear
        regression line is included validating the approach.}
    \label{fig:computed_probabilities_vs_theoretic_probabilities}
\end{figure}


\end{document}
 strategies,
was presented with specific consideration given to ZD strategies. This
tournament is reproduced here using the Axelrod-Python
project~\cite{Knight2016}. To obtain a good measure of the corresponding
transition rates for each strategy all matches have been run for
\documentclass[a4paper]{article}

\usepackage{amsmath}
\usepackage{amssymb}
\usepackage[margin=1.5cm,
            includefoot,
            footskip=30pt]{geometry}
\usepackage{layout}
\usepackage{graphicx}
\usepackage{subcaption}

\usepackage{biblatex}
\usepackage{pdfpages}

\bibliography{main.bib}

\title{Suspicion: Recognising and evaluating the effectiveness
       of extortion in the Iterated Prisoner's Dilemma}
\author{Vincent A. Knight \and Nikoleta E. Glynatsi}
\date{\today}



\begin{document}

\maketitle

\begin{abstract}
    The Iterated Prisoner's Dilemma is a model for rational and evolutionary
    interactive behaviour. It has applications both in the study of human social
    behaviour as well as in biology.
    It is used to understand when and how a rational individual might
    accept an immediate cost to their own utility for the direct benefit of
    another.

    Much attention has been given to a class of strategies called
    Zero Determinant strategies. It has been theoretically shown that these
    strategies can ``extort'' any player.

    In this work, an approach to identify if observed strategies are playing in
    an extortionate way is described. Furthermore, experimental analysis of
    a large tournament with \input{assets/tex/number_of_full_strategies/main.tex}
    strategies is considered. In this setting
    the most highly performing strategies do not play in an extortionate way
    against each other but do against lower performing strategies.
    This suggests that whilst the theory of Zero Determinant strategies
    indicates that memory is not of fundamental importance to the evolution of
    cooperative behaviour, this is incomplete.
\end{abstract}

\section{Introduction}\label{sec:introduction}

Agent based game theoretic models have become a stalwart of the underpinning
mathematics of interactive behaviours. One of the major pieces of work
in this area is the pair of original computer tournaments run by Robert
Axelrod~\cite{Axelrod1980, Axelrod1980a}. These tournaments pitted submitted
computer strategies against each other in plays of the Iterated Prisoner's
Dilemma. A common game where agents can choose to pay a slight cost to their
immediate utility in the hope of building a reputation. This has been used in
economic and evolutionary game theory to understand the evolution of cooperative
behaviour.

Recently, a class of strategies was described in~\cite{Press2012} that can
provably extort any given opponent. In~\cite{Hilbe2013, Moran1707} some
questions have already been asked about the true effectiveness of these
strategies in an evolutionary setting. Here another question is asked: is it
possible to recognise this extortionate behaviour? A mathematical procedure for
suspicion is presented: in the same way that the continued actions of an
extortionate individual might raise suspicion.

This work makes use of the Axelrod Python library~\cite{Knight2018, Knight2016}
with a large number of Prisoner Dilemma strategies available to give an
extensive numerical example of the ideas presented.  The approach is presented
in Section~\ref{sec:delta-zd-strategies}.  All of the code and data discussed
in Section~\ref{sec:numerical-experiments} is open sourced, archived and
written according to best scientific principles~\cite{Wilson2014}. The data
archive can be found at~\cite{vincent_knight_2018_1297075}.

\section{Recognising Extortion}\label{sec:delta-zd-strategies}

In~\cite{Press2012}, given a match between 2 memory-one strategies, the concept
of Zero Determinant (ZD) strategies is introduced. The main result of that paper
shows that given two memory one players \(p, q\in\mathbb{R}^4\) a linear
relationship between the players' scores could be forced by one of the players.

Using the notation of~\cite{Press2012}, assuming the utilities for player \(p\)
are given by \(S_x=(R, S, T, P)\) and for player \(q\) by \(S_y=(R, T, S, P)\)
and that the stationary scores of each player is given by \(S_X\) and \(S_Y\)
respectively. The main result of~\cite{Press2012} is that if

\begin{equation}\label{eqn:linear_relationship_for_p}
    \tilde p=\alpha S_x + \beta S_y + \gamma
\end{equation}

or

\begin{equation}\label{eqn:linear_relationship_for_q}
    \tilde q=\alpha S_x + \beta S_y + \gamma
\end{equation}

where \(\tilde p = (1 - p_1, 1 - p_2, p_3, p_4)\) and
\(\tilde q = (1 - q_1, 1 - q_2, q_3, q_4)\) then:

\begin{equation}
    \alpha S_X + \beta S_Y + \gamma = 0
\end{equation}

In~\cite{Press2012} a particular type of ZD strategy is defined: extortionate
strategies. If:

\begin{equation}\label{eqn:constraint_for_extortion}
    \gamma = - P(\alpha + \beta)
\end{equation}

then the player can ensure they get a score \(\chi\) times
larger than the opponent. This extortion coefficient is given by:

\begin{equation}\label{eqn:definition_of_chi}
    \chi=\frac{-\beta}{\alpha}
\end{equation}

Thus, if (\ref{eqn:constraint_for_extortion}) holds and \(\chi >1\) a player is
said to extort their opponent.
Here, the reverse problem is considered: given a
\(p\in\mathbb{R}^4\) how does one identify \(\alpha, \beta\) if they
exist and is the strategy in fact acting in an extortionate way?

These conditions correspond to:

\begin{align}
    \tilde p_1 & = \alpha R + \beta R - P (\alpha + \beta)
            \label{eqn:condition_for_tilde_p1}\\
    \tilde p_2 & = \alpha S + \beta T - P (\alpha + \beta)
            \label{eqn:condition_for_tilde_p2}\\
    \tilde p_3 & = \alpha T + \beta S - P (\alpha + \beta)
            \label{eqn:condition_for_tilde_p3}\\
    \tilde p_4 & = \alpha P + \beta P - P (\alpha + \beta)
            \label{eqn:condition_for_tilde_p4}
\end{align}

Equation (\ref{eqn:condition_for_tilde_p4}) ensures that \(p_4=\tilde p_4=0\).
Equations (\ref{eqn:condition_for_tilde_p1}-\ref{eqn:condition_for_tilde_p3})
can be used to eliminate \(\alpha, \beta\), giving:

\begin{equation}\label{eqn:planar_definition_of_extortion}
    \tilde p_1 = \frac{(R - P)(\tilde p_2 + \tilde p_3)}{S + T - 2P}
\end{equation}

with:

\begin{equation}\label{eqn:definition_of_chi}
    \chi = \frac{\tilde p_2 (P - T) + \tilde p_3 (S - P)}
                {\tilde p_2 (P - S) + \tilde p_3 (T - P)}
\end{equation}

Given a strategy \(p\in\mathbb{R}^{4\times 1}\) equations
(\ref{eqn:condition_for_tilde_p4}), (\ref{eqn:planar_definition_of_extortion}-\ref{eqn:definition_of_chi}) can be used to check if
a strategy is extortionate. The conditions correspond to:

\begin{align}
    p_1 & = \frac{(R-P)(p_2 + p_3) - R + T + S - P}{S + T - 2P}
     \label{eqn:condition_for_p1}\\
    p_4 & = 0 \label{eqn:condition_for_p4}\\
    1 & > p_2 + p_3\label{eqn:condition_for_chi}
\end{align}

The algebraic steps necessary to prove these results are available in the
supporting materials.

All extortionate strategies reside on a triangular (\ref{eqn:condition_for_chi})
plane (\ref{eqn:condition_for_p1}) in 3 dimensions (\ref{eqn:condition_for_p4}).
Using this formulation it can be seen that a necessary (but not sufficient)
condition for an extortionate strategy is that it cooperates on average less
than 50\% of the time when in a state of disagreement with the opponent.

As an example, consider the known extortionate strategy \(p=(8 / 9, 1 / 2, 1 /
3, 0)\) from~\cite{Stewart2012} which is referred to as \texttt{Extort-2}. In
this case, for the standard values of \((R, T, S, P)\) constraint
(\ref{eqn:condition_for_p1}) corresponds to:

\begin{equation}
    p_1 = \frac{2(p_2 + p_3) + 1}{3}
\end{equation}

It is clear that in this case all constraints hold.

This approach could in fact be used to confirm that a given strategy is acting
in an extortionate manner even if it is not a memory one strategy. However, in
practice, if a closed form for \(p\) is not known, then due to measurement
and/or numerical error this would not work.

This problem can be written in the following linear algebraic form where
\(x=(\alpha, \beta)\)
and \(p^*=(\tilde p_1 - 1, tilde_2 - 1, p_3)\):

\begin{equation}\label{eqn:linear_algebraic_equation_for_p}
    Cx= p^*
\end{equation}

\(C\) corresponds to equations
(\ref{eqn:condition_for_tilde_p1}-\ref{eqn:condition_for_tilde_p3}) and is
given by:

\begin{equation}\label{eqn:definition_of_C}
    C =
    \begin{bmatrix}
        R - P & R- P \\
        S - P & T- P \\
        T - P & S- P \\
    \end{bmatrix}
\end{equation}

Note that in general, equation (\ref{eqn:linear_algebraic_equation_for_p}) will
not necessarily have a solution. From the Rouch\'{e}-Capelli theorem if there is
a solution it is unique as \(\text{rank}(C)=2\) which is the dimension of the
variable \(x\). The best fitting \(x\) is found by minimizing:

\begin{equation}\label{eqn:r_squared}
    \text{SSError} = \|C x- p^*\|_2^2 = \sum_{i=1}^{3}\left((C\bar x)_i-p_i^*\right)^2
\end{equation}

Note that \(\text{SSError}\), which is the square of the Frobenius
norm~\cite{Golub2013}, becomes a measure of how close a strategy is to being an
extortionate strategy. Suspicion
of extortion then corresponds to a threshold on \(\text{SSError}\).

By observing interactions (human or otherwise), their memory one representation
can be inferred and this approach can be used to recognise extortionate
behaviour. The notion of comparing theoretic and actual plays of the IPD is not
novel, see for example~\cite{Rand2013}. Immediately it is noted that if the
environment is noisy~\cite{Wu1995} then no strategy can be considered to be
extortionate as \(p_4>0\).

In the next section, this idea will be illustrated by observing the interactions
that take place in a computer based tournament of the IPD\@.

\section{Numerical experiments}\label{sec:numerical-experiments}

In~\cite{Stewart2012} results from a tournament with
\input{./assets/tex/number_of_stewart_plotkin_strategies/main.tex} strategies,
was presented with specific consideration given to ZD strategies. This
tournament is reproduced here using the Axelrod-Python
project~\cite{Knight2016}. To obtain a good measure of the corresponding
transition rates for each strategy all matches have been run for
\input{assets/tex/number_of_turns/main.tex} turns and every match has been
repeated \input{assets/tex/number_of_repetitions/main.tex} times. All of this
interaction data is available at~\cite{vincent_knight_2018_1297075}. A good
match between the inferred Markov chain and the state distribution of the actual
interactions has been verified. Data for this is presented in the supplementary
materials.

Figure~\ref{fig:SSError_overall_in_stewart_plotkin} shows the \(\text{SSError}\)
values for all the strategies in the tournament, as reported
in~\cite{Stewart2012} the extortionate strategy (which has an expected
\(\text{SSError}\) approximately 0) gains a large number of wins.

\begin{figure}[!htbp]
    \centering
    \includegraphics[width=.8\textwidth]{./assets/img/SSError_overall_in_stewart_plotkin/main.pdf}
    \caption{\(\text{SSError}\) and state probabilities for the strategies
        of~\cite{Stewart2012}, ordered both by number of wins and overall score.
        Note that \(P(DC)\) is not shown as it corresponds to the transpose of
        \(P(CD)\). Cooperator and Defector are omitted as they do not visit all
        the states.}
    \label{fig:SSError_overall_in_stewart_plotkin}
\end{figure}

Here, the work of~\cite{Stewart2012} is extended by investigating a tournament
with \input{assets/tex/number_of_full_strategies/main.tex}
strategies.

The results of this analysis are shown in
Figure~\ref{fig:SSError_and_probabilities_in_full}. The top ranking strategies
by number of wins seem to be extortionate (but not against all strategies) and
it can be seen that a small sub group of strategies achieve mutual defection.
All the top ranking strategies according to score achieve mutual cooperation and
do not extort each other, however they
\textbf{do} exhibit extortionate behaviour towards a number of the lower ranking
strategies.

\begin{figure}[!htbp]
    \centering
    \includegraphics[width=.8\textwidth]{./assets/img/SSError_and_probabilities_in_full/main.pdf}
    \caption{\(\text{SSError}\) for the strategies for the full tournament. Only
    strategy interactions for which \(p_4=0\) and \(\chi>1\) are displayed.}
    \label{fig:SSError_and_probabilities_in_full}
\end{figure}

\section{Conclusion}\label{sec:conclusion}

This work defines an approach to measure whether or not a player is playing a
strategy that corresponds to an extortionate strategy as defined
in~\cite{Press2012}: a mathematical model for suspicion. Indeed, all
extortionate strategies have been
 classified as lying on a triangular plane.
This rigorous classification fails to be robust to small measurement error, thus
a statistical approach is proposed.
This is done through a linear algebraic approach for approximating the solution
of a linear system. Using this, a large number of pairwise interactions is
simulated and in fact very few strategies are found to act extortionately.

The work of~\cite{Press2012}, whilst showing that a clever approach to taking
advantage of another memory one strategy exists: this is incomplete. Whilst the
elegance of this result is very attractive, just as the simplicity of the
victory of Tit For Tat in Axelrod's original tournaments was, it is incomplete.
Extortionate strategies achieve a high number of wins but they do not
achieve a high score which corresponds to the fitness landscape in an
evolutionary sense. From the large number of interactions a payoff matrix \(S\)
can be measured where \(S_{ij}\) denotes the score (using standard values of
\((R, S, T, P) = (3, 0, 5, 1)\)) of the \(i\)th strategy
against the \(j\)th strategy. Using this, the replicator equation
describes the evolution of the system based on a population density fitness
function:

\begin{equation}\label{eqn:replicator_dynamics}
    \frac{dx}{dt} = x(S-x^TS x)
\end{equation}

Equation (\ref{eqn:replicator_dynamics}) is solved numerically through an
integration technique described in~\cite{Petzold1983} and
Figure~\ref{fig:replicator_dynamics} shows the evolution of the distribution of
the system: the various strategies are ranked by scores. It is clear to see that
only the high ranking strategies survive the evolutionary process (in fact,
only \input{./assets/img/replicator_dynamics/main.tex}
have a final distribution greater than \(10 ^ {-2}\)). This confirms the
findings of~\cite{Moran1707} in which sophisticated strategies resist
evolutionary invasion of shorter memory strategies. Recalling
Figure~\ref{fig:SSError_and_probabilities_in_full} this demonstrates that:

\begin{itemize}
    \item Cooperation emerges through the evolutionary process: the high scoring
        strategies do not exhibit extortionate behaviour towards each other.
    \item Extortionate strategies do not survive the evolutionary process.
\end{itemize}

\begin{figure}[!htbp]
    \centering
    \includegraphics[width=.8\textwidth]{./assets/img/replicator_dynamics/main.pdf}
    \caption{Numerical simulation of the replicator equation
    (\ref{eqn:replicator_dynamics}): strategies are ordered by score, only the strategies with a high score survive the evolutionary process.}
    \label{fig:replicator_dynamics}
\end{figure}

This work can be used to classify plays of the IPD\@: data can be collected from
actual interactions (in lab or in the field). Furthermore, this allows for a
classification method similar to the notion of fingerprinting presented
in~\cite{Ashlock2008}. Trained strategies can potentially be classified as
extortionate or not or it could be possible to even constrain the reinforcement
learning approaches that are becoming prevalent in the literature.
Alternatively, this mathematical approach for recognising extortion could be
used in sophisticated strategies to defend against invasion. Arguably, some of
the strategies considered here exhibit this behaviour, indeed as described
in~\cite{Harper2017}, the top ranking strategies in the full tournament are
obtained using evolutionary reinforcement learning techniques, thus, suspicion
of extortionate behaviour could in fact be an evolutionary trait.

\section*{Acknowledgements}

The following open source software libraries were used in this research:

\begin{itemize}
    \item The Axelrod ~\cite{Knight2016, Knight2018} library (IPD strategies and
        tournaments).
    \item The sympy library~\cite{Meurer2017} (verification of all symbolic
        calculations).
    \item The matplotlib~\cite{Droettboom2018} library (visualisation).
    \item The pandas~\cite{Structures2010}, dask~\cite{Dask2016} and
        NumPy~\cite{Oliphant2015} libraries (data manipulation).
    \item The SciPy~\cite{Jones2001} library (numerical integration of the
        replicator equation).
\end{itemize}

This work was performed using the computational facilities of the Advanced
Research Computing @ Cardiff (ARCCA) Division, Cardiff University.

\printbibliography

\newpage
\section*{Supplementary materials}

\includepdf{assets/pdf/proof_of_form_of_extortionate_strategies/main.pdf}

\newpage

Using the pair wise interactions the transition rates \(p,
q\) can be measured and the steady state probabilities inferred and compared to
the actual probabilities of each state.
This is done numerically by computing the singular eigenvector of the
matrix \(A\) \cite{Stewart2009}:

\[
    A =
    \begin{bmatrix}
        p_1 q_1 & p_1 (1 - q_1) & (1 - p_1) q_1 & (1 -p_1) (1 - q_1) \\
        p_2 q_2 & p_2 (1 - q_2) & (1 - p_2) q_2 & (1 -p_2) (1 - q_2) \\
        p_3 q_3 & p_3 (1 - q_3) & (1 - p_3) q_3 & (1 -p_3) (1 - q_3) \\
        p_4 q_4 & p_4 (1 - q_4) & (1 - p_4) q_4 & (1 -p_4) (1 - q_4) \\
    \end{bmatrix}
\]

Figure~\ref{fig:computed_probabilities_vs_theoretic_probabilities} shows a
regression line fitted to every pairwise interaction with a reported
\(\text{SSError}\) value (pairwise interactions with missing states were
omitted). This serves to validate the approach: a part from some edge cases the
relationship is consistent.

\begin{figure}[!htbp]
    \centering
    \includegraphics[width=.8\textwidth]{./assets/img/computed_probabilities_vs_theoretic_probabilities/main.pdf}
    \caption{The
        relationship between the steady state probabilities inferred from the
        measured transitions and the actual steady state probabilities. A linear
        regression line is included validating the approach.}
    \label{fig:computed_probabilities_vs_theoretic_probabilities}
\end{figure}


\end{document}
 turns and every match has been
repeated \documentclass[a4paper]{article}

\usepackage{amsmath}
\usepackage{amssymb}
\usepackage[margin=1.5cm,
            includefoot,
            footskip=30pt]{geometry}
\usepackage{layout}
\usepackage{graphicx}
\usepackage{subcaption}

\usepackage{biblatex}
\usepackage{pdfpages}

\bibliography{main.bib}

\title{Suspicion: Recognising and evaluating the effectiveness
       of extortion in the Iterated Prisoner's Dilemma}
\author{Vincent A. Knight \and Nikoleta E. Glynatsi}
\date{\today}



\begin{document}

\maketitle

\begin{abstract}
    The Iterated Prisoner's Dilemma is a model for rational and evolutionary
    interactive behaviour. It has applications both in the study of human social
    behaviour as well as in biology.
    It is used to understand when and how a rational individual might
    accept an immediate cost to their own utility for the direct benefit of
    another.

    Much attention has been given to a class of strategies called
    Zero Determinant strategies. It has been theoretically shown that these
    strategies can ``extort'' any player.

    In this work, an approach to identify if observed strategies are playing in
    an extortionate way is described. Furthermore, experimental analysis of
    a large tournament with \input{assets/tex/number_of_full_strategies/main.tex}
    strategies is considered. In this setting
    the most highly performing strategies do not play in an extortionate way
    against each other but do against lower performing strategies.
    This suggests that whilst the theory of Zero Determinant strategies
    indicates that memory is not of fundamental importance to the evolution of
    cooperative behaviour, this is incomplete.
\end{abstract}

\section{Introduction}\label{sec:introduction}

Agent based game theoretic models have become a stalwart of the underpinning
mathematics of interactive behaviours. One of the major pieces of work
in this area is the pair of original computer tournaments run by Robert
Axelrod~\cite{Axelrod1980, Axelrod1980a}. These tournaments pitted submitted
computer strategies against each other in plays of the Iterated Prisoner's
Dilemma. A common game where agents can choose to pay a slight cost to their
immediate utility in the hope of building a reputation. This has been used in
economic and evolutionary game theory to understand the evolution of cooperative
behaviour.

Recently, a class of strategies was described in~\cite{Press2012} that can
provably extort any given opponent. In~\cite{Hilbe2013, Moran1707} some
questions have already been asked about the true effectiveness of these
strategies in an evolutionary setting. Here another question is asked: is it
possible to recognise this extortionate behaviour? A mathematical procedure for
suspicion is presented: in the same way that the continued actions of an
extortionate individual might raise suspicion.

This work makes use of the Axelrod Python library~\cite{Knight2018, Knight2016}
with a large number of Prisoner Dilemma strategies available to give an
extensive numerical example of the ideas presented.  The approach is presented
in Section~\ref{sec:delta-zd-strategies}.  All of the code and data discussed
in Section~\ref{sec:numerical-experiments} is open sourced, archived and
written according to best scientific principles~\cite{Wilson2014}. The data
archive can be found at~\cite{vincent_knight_2018_1297075}.

\section{Recognising Extortion}\label{sec:delta-zd-strategies}

In~\cite{Press2012}, given a match between 2 memory-one strategies, the concept
of Zero Determinant (ZD) strategies is introduced. The main result of that paper
shows that given two memory one players \(p, q\in\mathbb{R}^4\) a linear
relationship between the players' scores could be forced by one of the players.

Using the notation of~\cite{Press2012}, assuming the utilities for player \(p\)
are given by \(S_x=(R, S, T, P)\) and for player \(q\) by \(S_y=(R, T, S, P)\)
and that the stationary scores of each player is given by \(S_X\) and \(S_Y\)
respectively. The main result of~\cite{Press2012} is that if

\begin{equation}\label{eqn:linear_relationship_for_p}
    \tilde p=\alpha S_x + \beta S_y + \gamma
\end{equation}

or

\begin{equation}\label{eqn:linear_relationship_for_q}
    \tilde q=\alpha S_x + \beta S_y + \gamma
\end{equation}

where \(\tilde p = (1 - p_1, 1 - p_2, p_3, p_4)\) and
\(\tilde q = (1 - q_1, 1 - q_2, q_3, q_4)\) then:

\begin{equation}
    \alpha S_X + \beta S_Y + \gamma = 0
\end{equation}

In~\cite{Press2012} a particular type of ZD strategy is defined: extortionate
strategies. If:

\begin{equation}\label{eqn:constraint_for_extortion}
    \gamma = - P(\alpha + \beta)
\end{equation}

then the player can ensure they get a score \(\chi\) times
larger than the opponent. This extortion coefficient is given by:

\begin{equation}\label{eqn:definition_of_chi}
    \chi=\frac{-\beta}{\alpha}
\end{equation}

Thus, if (\ref{eqn:constraint_for_extortion}) holds and \(\chi >1\) a player is
said to extort their opponent.
Here, the reverse problem is considered: given a
\(p\in\mathbb{R}^4\) how does one identify \(\alpha, \beta\) if they
exist and is the strategy in fact acting in an extortionate way?

These conditions correspond to:

\begin{align}
    \tilde p_1 & = \alpha R + \beta R - P (\alpha + \beta)
            \label{eqn:condition_for_tilde_p1}\\
    \tilde p_2 & = \alpha S + \beta T - P (\alpha + \beta)
            \label{eqn:condition_for_tilde_p2}\\
    \tilde p_3 & = \alpha T + \beta S - P (\alpha + \beta)
            \label{eqn:condition_for_tilde_p3}\\
    \tilde p_4 & = \alpha P + \beta P - P (\alpha + \beta)
            \label{eqn:condition_for_tilde_p4}
\end{align}

Equation (\ref{eqn:condition_for_tilde_p4}) ensures that \(p_4=\tilde p_4=0\).
Equations (\ref{eqn:condition_for_tilde_p1}-\ref{eqn:condition_for_tilde_p3})
can be used to eliminate \(\alpha, \beta\), giving:

\begin{equation}\label{eqn:planar_definition_of_extortion}
    \tilde p_1 = \frac{(R - P)(\tilde p_2 + \tilde p_3)}{S + T - 2P}
\end{equation}

with:

\begin{equation}\label{eqn:definition_of_chi}
    \chi = \frac{\tilde p_2 (P - T) + \tilde p_3 (S - P)}
                {\tilde p_2 (P - S) + \tilde p_3 (T - P)}
\end{equation}

Given a strategy \(p\in\mathbb{R}^{4\times 1}\) equations
(\ref{eqn:condition_for_tilde_p4}), (\ref{eqn:planar_definition_of_extortion}-\ref{eqn:definition_of_chi}) can be used to check if
a strategy is extortionate. The conditions correspond to:

\begin{align}
    p_1 & = \frac{(R-P)(p_2 + p_3) - R + T + S - P}{S + T - 2P}
     \label{eqn:condition_for_p1}\\
    p_4 & = 0 \label{eqn:condition_for_p4}\\
    1 & > p_2 + p_3\label{eqn:condition_for_chi}
\end{align}

The algebraic steps necessary to prove these results are available in the
supporting materials.

All extortionate strategies reside on a triangular (\ref{eqn:condition_for_chi})
plane (\ref{eqn:condition_for_p1}) in 3 dimensions (\ref{eqn:condition_for_p4}).
Using this formulation it can be seen that a necessary (but not sufficient)
condition for an extortionate strategy is that it cooperates on average less
than 50\% of the time when in a state of disagreement with the opponent.

As an example, consider the known extortionate strategy \(p=(8 / 9, 1 / 2, 1 /
3, 0)\) from~\cite{Stewart2012} which is referred to as \texttt{Extort-2}. In
this case, for the standard values of \((R, T, S, P)\) constraint
(\ref{eqn:condition_for_p1}) corresponds to:

\begin{equation}
    p_1 = \frac{2(p_2 + p_3) + 1}{3}
\end{equation}

It is clear that in this case all constraints hold.

This approach could in fact be used to confirm that a given strategy is acting
in an extortionate manner even if it is not a memory one strategy. However, in
practice, if a closed form for \(p\) is not known, then due to measurement
and/or numerical error this would not work.

This problem can be written in the following linear algebraic form where
\(x=(\alpha, \beta)\)
and \(p^*=(\tilde p_1 - 1, tilde_2 - 1, p_3)\):

\begin{equation}\label{eqn:linear_algebraic_equation_for_p}
    Cx= p^*
\end{equation}

\(C\) corresponds to equations
(\ref{eqn:condition_for_tilde_p1}-\ref{eqn:condition_for_tilde_p3}) and is
given by:

\begin{equation}\label{eqn:definition_of_C}
    C =
    \begin{bmatrix}
        R - P & R- P \\
        S - P & T- P \\
        T - P & S- P \\
    \end{bmatrix}
\end{equation}

Note that in general, equation (\ref{eqn:linear_algebraic_equation_for_p}) will
not necessarily have a solution. From the Rouch\'{e}-Capelli theorem if there is
a solution it is unique as \(\text{rank}(C)=2\) which is the dimension of the
variable \(x\). The best fitting \(x\) is found by minimizing:

\begin{equation}\label{eqn:r_squared}
    \text{SSError} = \|C x- p^*\|_2^2 = \sum_{i=1}^{3}\left((C\bar x)_i-p_i^*\right)^2
\end{equation}

Note that \(\text{SSError}\), which is the square of the Frobenius
norm~\cite{Golub2013}, becomes a measure of how close a strategy is to being an
extortionate strategy. Suspicion
of extortion then corresponds to a threshold on \(\text{SSError}\).

By observing interactions (human or otherwise), their memory one representation
can be inferred and this approach can be used to recognise extortionate
behaviour. The notion of comparing theoretic and actual plays of the IPD is not
novel, see for example~\cite{Rand2013}. Immediately it is noted that if the
environment is noisy~\cite{Wu1995} then no strategy can be considered to be
extortionate as \(p_4>0\).

In the next section, this idea will be illustrated by observing the interactions
that take place in a computer based tournament of the IPD\@.

\section{Numerical experiments}\label{sec:numerical-experiments}

In~\cite{Stewart2012} results from a tournament with
\input{./assets/tex/number_of_stewart_plotkin_strategies/main.tex} strategies,
was presented with specific consideration given to ZD strategies. This
tournament is reproduced here using the Axelrod-Python
project~\cite{Knight2016}. To obtain a good measure of the corresponding
transition rates for each strategy all matches have been run for
\input{assets/tex/number_of_turns/main.tex} turns and every match has been
repeated \input{assets/tex/number_of_repetitions/main.tex} times. All of this
interaction data is available at~\cite{vincent_knight_2018_1297075}. A good
match between the inferred Markov chain and the state distribution of the actual
interactions has been verified. Data for this is presented in the supplementary
materials.

Figure~\ref{fig:SSError_overall_in_stewart_plotkin} shows the \(\text{SSError}\)
values for all the strategies in the tournament, as reported
in~\cite{Stewart2012} the extortionate strategy (which has an expected
\(\text{SSError}\) approximately 0) gains a large number of wins.

\begin{figure}[!htbp]
    \centering
    \includegraphics[width=.8\textwidth]{./assets/img/SSError_overall_in_stewart_plotkin/main.pdf}
    \caption{\(\text{SSError}\) and state probabilities for the strategies
        of~\cite{Stewart2012}, ordered both by number of wins and overall score.
        Note that \(P(DC)\) is not shown as it corresponds to the transpose of
        \(P(CD)\). Cooperator and Defector are omitted as they do not visit all
        the states.}
    \label{fig:SSError_overall_in_stewart_plotkin}
\end{figure}

Here, the work of~\cite{Stewart2012} is extended by investigating a tournament
with \input{assets/tex/number_of_full_strategies/main.tex}
strategies.

The results of this analysis are shown in
Figure~\ref{fig:SSError_and_probabilities_in_full}. The top ranking strategies
by number of wins seem to be extortionate (but not against all strategies) and
it can be seen that a small sub group of strategies achieve mutual defection.
All the top ranking strategies according to score achieve mutual cooperation and
do not extort each other, however they
\textbf{do} exhibit extortionate behaviour towards a number of the lower ranking
strategies.

\begin{figure}[!htbp]
    \centering
    \includegraphics[width=.8\textwidth]{./assets/img/SSError_and_probabilities_in_full/main.pdf}
    \caption{\(\text{SSError}\) for the strategies for the full tournament. Only
    strategy interactions for which \(p_4=0\) and \(\chi>1\) are displayed.}
    \label{fig:SSError_and_probabilities_in_full}
\end{figure}

\section{Conclusion}\label{sec:conclusion}

This work defines an approach to measure whether or not a player is playing a
strategy that corresponds to an extortionate strategy as defined
in~\cite{Press2012}: a mathematical model for suspicion. Indeed, all
extortionate strategies have been
 classified as lying on a triangular plane.
This rigorous classification fails to be robust to small measurement error, thus
a statistical approach is proposed.
This is done through a linear algebraic approach for approximating the solution
of a linear system. Using this, a large number of pairwise interactions is
simulated and in fact very few strategies are found to act extortionately.

The work of~\cite{Press2012}, whilst showing that a clever approach to taking
advantage of another memory one strategy exists: this is incomplete. Whilst the
elegance of this result is very attractive, just as the simplicity of the
victory of Tit For Tat in Axelrod's original tournaments was, it is incomplete.
Extortionate strategies achieve a high number of wins but they do not
achieve a high score which corresponds to the fitness landscape in an
evolutionary sense. From the large number of interactions a payoff matrix \(S\)
can be measured where \(S_{ij}\) denotes the score (using standard values of
\((R, S, T, P) = (3, 0, 5, 1)\)) of the \(i\)th strategy
against the \(j\)th strategy. Using this, the replicator equation
describes the evolution of the system based on a population density fitness
function:

\begin{equation}\label{eqn:replicator_dynamics}
    \frac{dx}{dt} = x(S-x^TS x)
\end{equation}

Equation (\ref{eqn:replicator_dynamics}) is solved numerically through an
integration technique described in~\cite{Petzold1983} and
Figure~\ref{fig:replicator_dynamics} shows the evolution of the distribution of
the system: the various strategies are ranked by scores. It is clear to see that
only the high ranking strategies survive the evolutionary process (in fact,
only \input{./assets/img/replicator_dynamics/main.tex}
have a final distribution greater than \(10 ^ {-2}\)). This confirms the
findings of~\cite{Moran1707} in which sophisticated strategies resist
evolutionary invasion of shorter memory strategies. Recalling
Figure~\ref{fig:SSError_and_probabilities_in_full} this demonstrates that:

\begin{itemize}
    \item Cooperation emerges through the evolutionary process: the high scoring
        strategies do not exhibit extortionate behaviour towards each other.
    \item Extortionate strategies do not survive the evolutionary process.
\end{itemize}

\begin{figure}[!htbp]
    \centering
    \includegraphics[width=.8\textwidth]{./assets/img/replicator_dynamics/main.pdf}
    \caption{Numerical simulation of the replicator equation
    (\ref{eqn:replicator_dynamics}): strategies are ordered by score, only the strategies with a high score survive the evolutionary process.}
    \label{fig:replicator_dynamics}
\end{figure}

This work can be used to classify plays of the IPD\@: data can be collected from
actual interactions (in lab or in the field). Furthermore, this allows for a
classification method similar to the notion of fingerprinting presented
in~\cite{Ashlock2008}. Trained strategies can potentially be classified as
extortionate or not or it could be possible to even constrain the reinforcement
learning approaches that are becoming prevalent in the literature.
Alternatively, this mathematical approach for recognising extortion could be
used in sophisticated strategies to defend against invasion. Arguably, some of
the strategies considered here exhibit this behaviour, indeed as described
in~\cite{Harper2017}, the top ranking strategies in the full tournament are
obtained using evolutionary reinforcement learning techniques, thus, suspicion
of extortionate behaviour could in fact be an evolutionary trait.

\section*{Acknowledgements}

The following open source software libraries were used in this research:

\begin{itemize}
    \item The Axelrod ~\cite{Knight2016, Knight2018} library (IPD strategies and
        tournaments).
    \item The sympy library~\cite{Meurer2017} (verification of all symbolic
        calculations).
    \item The matplotlib~\cite{Droettboom2018} library (visualisation).
    \item The pandas~\cite{Structures2010}, dask~\cite{Dask2016} and
        NumPy~\cite{Oliphant2015} libraries (data manipulation).
    \item The SciPy~\cite{Jones2001} library (numerical integration of the
        replicator equation).
\end{itemize}

This work was performed using the computational facilities of the Advanced
Research Computing @ Cardiff (ARCCA) Division, Cardiff University.

\printbibliography

\newpage
\section*{Supplementary materials}

\includepdf{assets/pdf/proof_of_form_of_extortionate_strategies/main.pdf}

\newpage

Using the pair wise interactions the transition rates \(p,
q\) can be measured and the steady state probabilities inferred and compared to
the actual probabilities of each state.
This is done numerically by computing the singular eigenvector of the
matrix \(A\) \cite{Stewart2009}:

\[
    A =
    \begin{bmatrix}
        p_1 q_1 & p_1 (1 - q_1) & (1 - p_1) q_1 & (1 -p_1) (1 - q_1) \\
        p_2 q_2 & p_2 (1 - q_2) & (1 - p_2) q_2 & (1 -p_2) (1 - q_2) \\
        p_3 q_3 & p_3 (1 - q_3) & (1 - p_3) q_3 & (1 -p_3) (1 - q_3) \\
        p_4 q_4 & p_4 (1 - q_4) & (1 - p_4) q_4 & (1 -p_4) (1 - q_4) \\
    \end{bmatrix}
\]

Figure~\ref{fig:computed_probabilities_vs_theoretic_probabilities} shows a
regression line fitted to every pairwise interaction with a reported
\(\text{SSError}\) value (pairwise interactions with missing states were
omitted). This serves to validate the approach: a part from some edge cases the
relationship is consistent.

\begin{figure}[!htbp]
    \centering
    \includegraphics[width=.8\textwidth]{./assets/img/computed_probabilities_vs_theoretic_probabilities/main.pdf}
    \caption{The
        relationship between the steady state probabilities inferred from the
        measured transitions and the actual steady state probabilities. A linear
        regression line is included validating the approach.}
    \label{fig:computed_probabilities_vs_theoretic_probabilities}
\end{figure}


\end{document}
 times. All of this
interaction data is available at~\cite{vincent_knight_2018_1297075}. A good
match between the inferred Markov chain and the state distribution of the actual
interactions has been verified. Data for this is presented in the supplementary
materials.

Figure~\ref{fig:SSError_overall_in_stewart_plotkin} shows the \(\text{SSError}\)
values for all the strategies in the tournament, as reported
in~\cite{Stewart2012} the extortionate strategy (which has an expected
\(\text{SSError}\) approximately 0) gains a large number of wins.

\begin{figure}[!htbp]
    \centering
    \includegraphics[width=.8\textwidth]{./assets/img/SSError_overall_in_stewart_plotkin/main.pdf}
    \caption{\(\text{SSError}\) and state probabilities for the strategies
        of~\cite{Stewart2012}, ordered both by number of wins and overall score.
        Note that \(P(DC)\) is not shown as it corresponds to the transpose of
        \(P(CD)\). Cooperator and Defector are omitted as they do not visit all
        the states.}
    \label{fig:SSError_overall_in_stewart_plotkin}
\end{figure}

Here, the work of~\cite{Stewart2012} is extended by investigating a tournament
with \documentclass[a4paper]{article}

\usepackage{amsmath}
\usepackage{amssymb}
\usepackage[margin=1.5cm,
            includefoot,
            footskip=30pt]{geometry}
\usepackage{layout}
\usepackage{graphicx}
\usepackage{subcaption}

\usepackage{biblatex}
\usepackage{pdfpages}

\bibliography{main.bib}

\title{Suspicion: Recognising and evaluating the effectiveness
       of extortion in the Iterated Prisoner's Dilemma}
\author{Vincent A. Knight \and Nikoleta E. Glynatsi}
\date{\today}



\begin{document}

\maketitle

\begin{abstract}
    The Iterated Prisoner's Dilemma is a model for rational and evolutionary
    interactive behaviour. It has applications both in the study of human social
    behaviour as well as in biology.
    It is used to understand when and how a rational individual might
    accept an immediate cost to their own utility for the direct benefit of
    another.

    Much attention has been given to a class of strategies called
    Zero Determinant strategies. It has been theoretically shown that these
    strategies can ``extort'' any player.

    In this work, an approach to identify if observed strategies are playing in
    an extortionate way is described. Furthermore, experimental analysis of
    a large tournament with \input{assets/tex/number_of_full_strategies/main.tex}
    strategies is considered. In this setting
    the most highly performing strategies do not play in an extortionate way
    against each other but do against lower performing strategies.
    This suggests that whilst the theory of Zero Determinant strategies
    indicates that memory is not of fundamental importance to the evolution of
    cooperative behaviour, this is incomplete.
\end{abstract}

\section{Introduction}\label{sec:introduction}

Agent based game theoretic models have become a stalwart of the underpinning
mathematics of interactive behaviours. One of the major pieces of work
in this area is the pair of original computer tournaments run by Robert
Axelrod~\cite{Axelrod1980, Axelrod1980a}. These tournaments pitted submitted
computer strategies against each other in plays of the Iterated Prisoner's
Dilemma. A common game where agents can choose to pay a slight cost to their
immediate utility in the hope of building a reputation. This has been used in
economic and evolutionary game theory to understand the evolution of cooperative
behaviour.

Recently, a class of strategies was described in~\cite{Press2012} that can
provably extort any given opponent. In~\cite{Hilbe2013, Moran1707} some
questions have already been asked about the true effectiveness of these
strategies in an evolutionary setting. Here another question is asked: is it
possible to recognise this extortionate behaviour? A mathematical procedure for
suspicion is presented: in the same way that the continued actions of an
extortionate individual might raise suspicion.

This work makes use of the Axelrod Python library~\cite{Knight2018, Knight2016}
with a large number of Prisoner Dilemma strategies available to give an
extensive numerical example of the ideas presented.  The approach is presented
in Section~\ref{sec:delta-zd-strategies}.  All of the code and data discussed
in Section~\ref{sec:numerical-experiments} is open sourced, archived and
written according to best scientific principles~\cite{Wilson2014}. The data
archive can be found at~\cite{vincent_knight_2018_1297075}.

\section{Recognising Extortion}\label{sec:delta-zd-strategies}

In~\cite{Press2012}, given a match between 2 memory-one strategies, the concept
of Zero Determinant (ZD) strategies is introduced. The main result of that paper
shows that given two memory one players \(p, q\in\mathbb{R}^4\) a linear
relationship between the players' scores could be forced by one of the players.

Using the notation of~\cite{Press2012}, assuming the utilities for player \(p\)
are given by \(S_x=(R, S, T, P)\) and for player \(q\) by \(S_y=(R, T, S, P)\)
and that the stationary scores of each player is given by \(S_X\) and \(S_Y\)
respectively. The main result of~\cite{Press2012} is that if

\begin{equation}\label{eqn:linear_relationship_for_p}
    \tilde p=\alpha S_x + \beta S_y + \gamma
\end{equation}

or

\begin{equation}\label{eqn:linear_relationship_for_q}
    \tilde q=\alpha S_x + \beta S_y + \gamma
\end{equation}

where \(\tilde p = (1 - p_1, 1 - p_2, p_3, p_4)\) and
\(\tilde q = (1 - q_1, 1 - q_2, q_3, q_4)\) then:

\begin{equation}
    \alpha S_X + \beta S_Y + \gamma = 0
\end{equation}

In~\cite{Press2012} a particular type of ZD strategy is defined: extortionate
strategies. If:

\begin{equation}\label{eqn:constraint_for_extortion}
    \gamma = - P(\alpha + \beta)
\end{equation}

then the player can ensure they get a score \(\chi\) times
larger than the opponent. This extortion coefficient is given by:

\begin{equation}\label{eqn:definition_of_chi}
    \chi=\frac{-\beta}{\alpha}
\end{equation}

Thus, if (\ref{eqn:constraint_for_extortion}) holds and \(\chi >1\) a player is
said to extort their opponent.
Here, the reverse problem is considered: given a
\(p\in\mathbb{R}^4\) how does one identify \(\alpha, \beta\) if they
exist and is the strategy in fact acting in an extortionate way?

These conditions correspond to:

\begin{align}
    \tilde p_1 & = \alpha R + \beta R - P (\alpha + \beta)
            \label{eqn:condition_for_tilde_p1}\\
    \tilde p_2 & = \alpha S + \beta T - P (\alpha + \beta)
            \label{eqn:condition_for_tilde_p2}\\
    \tilde p_3 & = \alpha T + \beta S - P (\alpha + \beta)
            \label{eqn:condition_for_tilde_p3}\\
    \tilde p_4 & = \alpha P + \beta P - P (\alpha + \beta)
            \label{eqn:condition_for_tilde_p4}
\end{align}

Equation (\ref{eqn:condition_for_tilde_p4}) ensures that \(p_4=\tilde p_4=0\).
Equations (\ref{eqn:condition_for_tilde_p1}-\ref{eqn:condition_for_tilde_p3})
can be used to eliminate \(\alpha, \beta\), giving:

\begin{equation}\label{eqn:planar_definition_of_extortion}
    \tilde p_1 = \frac{(R - P)(\tilde p_2 + \tilde p_3)}{S + T - 2P}
\end{equation}

with:

\begin{equation}\label{eqn:definition_of_chi}
    \chi = \frac{\tilde p_2 (P - T) + \tilde p_3 (S - P)}
                {\tilde p_2 (P - S) + \tilde p_3 (T - P)}
\end{equation}

Given a strategy \(p\in\mathbb{R}^{4\times 1}\) equations
(\ref{eqn:condition_for_tilde_p4}), (\ref{eqn:planar_definition_of_extortion}-\ref{eqn:definition_of_chi}) can be used to check if
a strategy is extortionate. The conditions correspond to:

\begin{align}
    p_1 & = \frac{(R-P)(p_2 + p_3) - R + T + S - P}{S + T - 2P}
     \label{eqn:condition_for_p1}\\
    p_4 & = 0 \label{eqn:condition_for_p4}\\
    1 & > p_2 + p_3\label{eqn:condition_for_chi}
\end{align}

The algebraic steps necessary to prove these results are available in the
supporting materials.

All extortionate strategies reside on a triangular (\ref{eqn:condition_for_chi})
plane (\ref{eqn:condition_for_p1}) in 3 dimensions (\ref{eqn:condition_for_p4}).
Using this formulation it can be seen that a necessary (but not sufficient)
condition for an extortionate strategy is that it cooperates on average less
than 50\% of the time when in a state of disagreement with the opponent.

As an example, consider the known extortionate strategy \(p=(8 / 9, 1 / 2, 1 /
3, 0)\) from~\cite{Stewart2012} which is referred to as \texttt{Extort-2}. In
this case, for the standard values of \((R, T, S, P)\) constraint
(\ref{eqn:condition_for_p1}) corresponds to:

\begin{equation}
    p_1 = \frac{2(p_2 + p_3) + 1}{3}
\end{equation}

It is clear that in this case all constraints hold.

This approach could in fact be used to confirm that a given strategy is acting
in an extortionate manner even if it is not a memory one strategy. However, in
practice, if a closed form for \(p\) is not known, then due to measurement
and/or numerical error this would not work.

This problem can be written in the following linear algebraic form where
\(x=(\alpha, \beta)\)
and \(p^*=(\tilde p_1 - 1, tilde_2 - 1, p_3)\):

\begin{equation}\label{eqn:linear_algebraic_equation_for_p}
    Cx= p^*
\end{equation}

\(C\) corresponds to equations
(\ref{eqn:condition_for_tilde_p1}-\ref{eqn:condition_for_tilde_p3}) and is
given by:

\begin{equation}\label{eqn:definition_of_C}
    C =
    \begin{bmatrix}
        R - P & R- P \\
        S - P & T- P \\
        T - P & S- P \\
    \end{bmatrix}
\end{equation}

Note that in general, equation (\ref{eqn:linear_algebraic_equation_for_p}) will
not necessarily have a solution. From the Rouch\'{e}-Capelli theorem if there is
a solution it is unique as \(\text{rank}(C)=2\) which is the dimension of the
variable \(x\). The best fitting \(x\) is found by minimizing:

\begin{equation}\label{eqn:r_squared}
    \text{SSError} = \|C x- p^*\|_2^2 = \sum_{i=1}^{3}\left((C\bar x)_i-p_i^*\right)^2
\end{equation}

Note that \(\text{SSError}\), which is the square of the Frobenius
norm~\cite{Golub2013}, becomes a measure of how close a strategy is to being an
extortionate strategy. Suspicion
of extortion then corresponds to a threshold on \(\text{SSError}\).

By observing interactions (human or otherwise), their memory one representation
can be inferred and this approach can be used to recognise extortionate
behaviour. The notion of comparing theoretic and actual plays of the IPD is not
novel, see for example~\cite{Rand2013}. Immediately it is noted that if the
environment is noisy~\cite{Wu1995} then no strategy can be considered to be
extortionate as \(p_4>0\).

In the next section, this idea will be illustrated by observing the interactions
that take place in a computer based tournament of the IPD\@.

\section{Numerical experiments}\label{sec:numerical-experiments}

In~\cite{Stewart2012} results from a tournament with
\input{./assets/tex/number_of_stewart_plotkin_strategies/main.tex} strategies,
was presented with specific consideration given to ZD strategies. This
tournament is reproduced here using the Axelrod-Python
project~\cite{Knight2016}. To obtain a good measure of the corresponding
transition rates for each strategy all matches have been run for
\input{assets/tex/number_of_turns/main.tex} turns and every match has been
repeated \input{assets/tex/number_of_repetitions/main.tex} times. All of this
interaction data is available at~\cite{vincent_knight_2018_1297075}. A good
match between the inferred Markov chain and the state distribution of the actual
interactions has been verified. Data for this is presented in the supplementary
materials.

Figure~\ref{fig:SSError_overall_in_stewart_plotkin} shows the \(\text{SSError}\)
values for all the strategies in the tournament, as reported
in~\cite{Stewart2012} the extortionate strategy (which has an expected
\(\text{SSError}\) approximately 0) gains a large number of wins.

\begin{figure}[!htbp]
    \centering
    \includegraphics[width=.8\textwidth]{./assets/img/SSError_overall_in_stewart_plotkin/main.pdf}
    \caption{\(\text{SSError}\) and state probabilities for the strategies
        of~\cite{Stewart2012}, ordered both by number of wins and overall score.
        Note that \(P(DC)\) is not shown as it corresponds to the transpose of
        \(P(CD)\). Cooperator and Defector are omitted as they do not visit all
        the states.}
    \label{fig:SSError_overall_in_stewart_plotkin}
\end{figure}

Here, the work of~\cite{Stewart2012} is extended by investigating a tournament
with \input{assets/tex/number_of_full_strategies/main.tex}
strategies.

The results of this analysis are shown in
Figure~\ref{fig:SSError_and_probabilities_in_full}. The top ranking strategies
by number of wins seem to be extortionate (but not against all strategies) and
it can be seen that a small sub group of strategies achieve mutual defection.
All the top ranking strategies according to score achieve mutual cooperation and
do not extort each other, however they
\textbf{do} exhibit extortionate behaviour towards a number of the lower ranking
strategies.

\begin{figure}[!htbp]
    \centering
    \includegraphics[width=.8\textwidth]{./assets/img/SSError_and_probabilities_in_full/main.pdf}
    \caption{\(\text{SSError}\) for the strategies for the full tournament. Only
    strategy interactions for which \(p_4=0\) and \(\chi>1\) are displayed.}
    \label{fig:SSError_and_probabilities_in_full}
\end{figure}

\section{Conclusion}\label{sec:conclusion}

This work defines an approach to measure whether or not a player is playing a
strategy that corresponds to an extortionate strategy as defined
in~\cite{Press2012}: a mathematical model for suspicion. Indeed, all
extortionate strategies have been
 classified as lying on a triangular plane.
This rigorous classification fails to be robust to small measurement error, thus
a statistical approach is proposed.
This is done through a linear algebraic approach for approximating the solution
of a linear system. Using this, a large number of pairwise interactions is
simulated and in fact very few strategies are found to act extortionately.

The work of~\cite{Press2012}, whilst showing that a clever approach to taking
advantage of another memory one strategy exists: this is incomplete. Whilst the
elegance of this result is very attractive, just as the simplicity of the
victory of Tit For Tat in Axelrod's original tournaments was, it is incomplete.
Extortionate strategies achieve a high number of wins but they do not
achieve a high score which corresponds to the fitness landscape in an
evolutionary sense. From the large number of interactions a payoff matrix \(S\)
can be measured where \(S_{ij}\) denotes the score (using standard values of
\((R, S, T, P) = (3, 0, 5, 1)\)) of the \(i\)th strategy
against the \(j\)th strategy. Using this, the replicator equation
describes the evolution of the system based on a population density fitness
function:

\begin{equation}\label{eqn:replicator_dynamics}
    \frac{dx}{dt} = x(S-x^TS x)
\end{equation}

Equation (\ref{eqn:replicator_dynamics}) is solved numerically through an
integration technique described in~\cite{Petzold1983} and
Figure~\ref{fig:replicator_dynamics} shows the evolution of the distribution of
the system: the various strategies are ranked by scores. It is clear to see that
only the high ranking strategies survive the evolutionary process (in fact,
only \input{./assets/img/replicator_dynamics/main.tex}
have a final distribution greater than \(10 ^ {-2}\)). This confirms the
findings of~\cite{Moran1707} in which sophisticated strategies resist
evolutionary invasion of shorter memory strategies. Recalling
Figure~\ref{fig:SSError_and_probabilities_in_full} this demonstrates that:

\begin{itemize}
    \item Cooperation emerges through the evolutionary process: the high scoring
        strategies do not exhibit extortionate behaviour towards each other.
    \item Extortionate strategies do not survive the evolutionary process.
\end{itemize}

\begin{figure}[!htbp]
    \centering
    \includegraphics[width=.8\textwidth]{./assets/img/replicator_dynamics/main.pdf}
    \caption{Numerical simulation of the replicator equation
    (\ref{eqn:replicator_dynamics}): strategies are ordered by score, only the strategies with a high score survive the evolutionary process.}
    \label{fig:replicator_dynamics}
\end{figure}

This work can be used to classify plays of the IPD\@: data can be collected from
actual interactions (in lab or in the field). Furthermore, this allows for a
classification method similar to the notion of fingerprinting presented
in~\cite{Ashlock2008}. Trained strategies can potentially be classified as
extortionate or not or it could be possible to even constrain the reinforcement
learning approaches that are becoming prevalent in the literature.
Alternatively, this mathematical approach for recognising extortion could be
used in sophisticated strategies to defend against invasion. Arguably, some of
the strategies considered here exhibit this behaviour, indeed as described
in~\cite{Harper2017}, the top ranking strategies in the full tournament are
obtained using evolutionary reinforcement learning techniques, thus, suspicion
of extortionate behaviour could in fact be an evolutionary trait.

\section*{Acknowledgements}

The following open source software libraries were used in this research:

\begin{itemize}
    \item The Axelrod ~\cite{Knight2016, Knight2018} library (IPD strategies and
        tournaments).
    \item The sympy library~\cite{Meurer2017} (verification of all symbolic
        calculations).
    \item The matplotlib~\cite{Droettboom2018} library (visualisation).
    \item The pandas~\cite{Structures2010}, dask~\cite{Dask2016} and
        NumPy~\cite{Oliphant2015} libraries (data manipulation).
    \item The SciPy~\cite{Jones2001} library (numerical integration of the
        replicator equation).
\end{itemize}

This work was performed using the computational facilities of the Advanced
Research Computing @ Cardiff (ARCCA) Division, Cardiff University.

\printbibliography

\newpage
\section*{Supplementary materials}

\includepdf{assets/pdf/proof_of_form_of_extortionate_strategies/main.pdf}

\newpage

Using the pair wise interactions the transition rates \(p,
q\) can be measured and the steady state probabilities inferred and compared to
the actual probabilities of each state.
This is done numerically by computing the singular eigenvector of the
matrix \(A\) \cite{Stewart2009}:

\[
    A =
    \begin{bmatrix}
        p_1 q_1 & p_1 (1 - q_1) & (1 - p_1) q_1 & (1 -p_1) (1 - q_1) \\
        p_2 q_2 & p_2 (1 - q_2) & (1 - p_2) q_2 & (1 -p_2) (1 - q_2) \\
        p_3 q_3 & p_3 (1 - q_3) & (1 - p_3) q_3 & (1 -p_3) (1 - q_3) \\
        p_4 q_4 & p_4 (1 - q_4) & (1 - p_4) q_4 & (1 -p_4) (1 - q_4) \\
    \end{bmatrix}
\]

Figure~\ref{fig:computed_probabilities_vs_theoretic_probabilities} shows a
regression line fitted to every pairwise interaction with a reported
\(\text{SSError}\) value (pairwise interactions with missing states were
omitted). This serves to validate the approach: a part from some edge cases the
relationship is consistent.

\begin{figure}[!htbp]
    \centering
    \includegraphics[width=.8\textwidth]{./assets/img/computed_probabilities_vs_theoretic_probabilities/main.pdf}
    \caption{The
        relationship between the steady state probabilities inferred from the
        measured transitions and the actual steady state probabilities. A linear
        regression line is included validating the approach.}
    \label{fig:computed_probabilities_vs_theoretic_probabilities}
\end{figure}


\end{document}

strategies.

The results of this analysis are shown in
Figure~\ref{fig:SSError_and_probabilities_in_full}. The top ranking strategies
by number of wins seem to be extortionate (but not against all strategies) and
it can be seen that a small sub group of strategies achieve mutual defection.
All the top ranking strategies according to score achieve mutual cooperation and
do not extort each other, however they
\textbf{do} exhibit extortionate behaviour towards a number of the lower ranking
strategies.

\begin{figure}[!htbp]
    \centering
    \includegraphics[width=.8\textwidth]{./assets/img/SSError_and_probabilities_in_full/main.pdf}
    \caption{\(\text{SSError}\) for the strategies for the full tournament. Only
    strategy interactions for which \(p_4=0\) and \(\chi>1\) are displayed.}
    \label{fig:SSError_and_probabilities_in_full}
\end{figure}

\section{Conclusion}\label{sec:conclusion}

This work defines an approach to measure whether or not a player is playing a
strategy that corresponds to an extortionate strategy as defined
in~\cite{Press2012}: a mathematical model for suspicion. Indeed, all
extortionate strategies have been
 classified as lying on a triangular plane.
This rigorous classification fails to be robust to small measurement error, thus
a statistical approach is proposed.
This is done through a linear algebraic approach for approximating the solution
of a linear system. Using this, a large number of pairwise interactions is
simulated and in fact very few strategies are found to act extortionately.

The work of~\cite{Press2012}, whilst showing that a clever approach to taking
advantage of another memory one strategy exists: this is incomplete. Whilst the
elegance of this result is very attractive, just as the simplicity of the
victory of Tit For Tat in Axelrod's original tournaments was, it is incomplete.
Extortionate strategies achieve a high number of wins but they do not
achieve a high score which corresponds to the fitness landscape in an
evolutionary sense. From the large number of interactions a payoff matrix \(S\)
can be measured where \(S_{ij}\) denotes the score (using standard values of
\((R, S, T, P) = (3, 0, 5, 1)\)) of the \(i\)th strategy
against the \(j\)th strategy. Using this, the replicator equation
describes the evolution of the system based on a population density fitness
function:

\begin{equation}\label{eqn:replicator_dynamics}
    \frac{dx}{dt} = x(S-x^TS x)
\end{equation}

Equation (\ref{eqn:replicator_dynamics}) is solved numerically through an
integration technique described in~\cite{Petzold1983} and
Figure~\ref{fig:replicator_dynamics} shows the evolution of the distribution of
the system: the various strategies are ranked by scores. It is clear to see that
only the high ranking strategies survive the evolutionary process (in fact,
only \documentclass[a4paper]{article}

\usepackage{amsmath}
\usepackage{amssymb}
\usepackage[margin=1.5cm,
            includefoot,
            footskip=30pt]{geometry}
\usepackage{layout}
\usepackage{graphicx}
\usepackage{subcaption}

\usepackage{biblatex}
\usepackage{pdfpages}

\bibliography{main.bib}

\title{Suspicion: Recognising and evaluating the effectiveness
       of extortion in the Iterated Prisoner's Dilemma}
\author{Vincent A. Knight \and Nikoleta E. Glynatsi}
\date{\today}



\begin{document}

\maketitle

\begin{abstract}
    The Iterated Prisoner's Dilemma is a model for rational and evolutionary
    interactive behaviour. It has applications both in the study of human social
    behaviour as well as in biology.
    It is used to understand when and how a rational individual might
    accept an immediate cost to their own utility for the direct benefit of
    another.

    Much attention has been given to a class of strategies called
    Zero Determinant strategies. It has been theoretically shown that these
    strategies can ``extort'' any player.

    In this work, an approach to identify if observed strategies are playing in
    an extortionate way is described. Furthermore, experimental analysis of
    a large tournament with \input{assets/tex/number_of_full_strategies/main.tex}
    strategies is considered. In this setting
    the most highly performing strategies do not play in an extortionate way
    against each other but do against lower performing strategies.
    This suggests that whilst the theory of Zero Determinant strategies
    indicates that memory is not of fundamental importance to the evolution of
    cooperative behaviour, this is incomplete.
\end{abstract}

\section{Introduction}\label{sec:introduction}

Agent based game theoretic models have become a stalwart of the underpinning
mathematics of interactive behaviours. One of the major pieces of work
in this area is the pair of original computer tournaments run by Robert
Axelrod~\cite{Axelrod1980, Axelrod1980a}. These tournaments pitted submitted
computer strategies against each other in plays of the Iterated Prisoner's
Dilemma. A common game where agents can choose to pay a slight cost to their
immediate utility in the hope of building a reputation. This has been used in
economic and evolutionary game theory to understand the evolution of cooperative
behaviour.

Recently, a class of strategies was described in~\cite{Press2012} that can
provably extort any given opponent. In~\cite{Hilbe2013, Moran1707} some
questions have already been asked about the true effectiveness of these
strategies in an evolutionary setting. Here another question is asked: is it
possible to recognise this extortionate behaviour? A mathematical procedure for
suspicion is presented: in the same way that the continued actions of an
extortionate individual might raise suspicion.

This work makes use of the Axelrod Python library~\cite{Knight2018, Knight2016}
with a large number of Prisoner Dilemma strategies available to give an
extensive numerical example of the ideas presented.  The approach is presented
in Section~\ref{sec:delta-zd-strategies}.  All of the code and data discussed
in Section~\ref{sec:numerical-experiments} is open sourced, archived and
written according to best scientific principles~\cite{Wilson2014}. The data
archive can be found at~\cite{vincent_knight_2018_1297075}.

\section{Recognising Extortion}\label{sec:delta-zd-strategies}

In~\cite{Press2012}, given a match between 2 memory-one strategies, the concept
of Zero Determinant (ZD) strategies is introduced. The main result of that paper
shows that given two memory one players \(p, q\in\mathbb{R}^4\) a linear
relationship between the players' scores could be forced by one of the players.

Using the notation of~\cite{Press2012}, assuming the utilities for player \(p\)
are given by \(S_x=(R, S, T, P)\) and for player \(q\) by \(S_y=(R, T, S, P)\)
and that the stationary scores of each player is given by \(S_X\) and \(S_Y\)
respectively. The main result of~\cite{Press2012} is that if

\begin{equation}\label{eqn:linear_relationship_for_p}
    \tilde p=\alpha S_x + \beta S_y + \gamma
\end{equation}

or

\begin{equation}\label{eqn:linear_relationship_for_q}
    \tilde q=\alpha S_x + \beta S_y + \gamma
\end{equation}

where \(\tilde p = (1 - p_1, 1 - p_2, p_3, p_4)\) and
\(\tilde q = (1 - q_1, 1 - q_2, q_3, q_4)\) then:

\begin{equation}
    \alpha S_X + \beta S_Y + \gamma = 0
\end{equation}

In~\cite{Press2012} a particular type of ZD strategy is defined: extortionate
strategies. If:

\begin{equation}\label{eqn:constraint_for_extortion}
    \gamma = - P(\alpha + \beta)
\end{equation}

then the player can ensure they get a score \(\chi\) times
larger than the opponent. This extortion coefficient is given by:

\begin{equation}\label{eqn:definition_of_chi}
    \chi=\frac{-\beta}{\alpha}
\end{equation}

Thus, if (\ref{eqn:constraint_for_extortion}) holds and \(\chi >1\) a player is
said to extort their opponent.
Here, the reverse problem is considered: given a
\(p\in\mathbb{R}^4\) how does one identify \(\alpha, \beta\) if they
exist and is the strategy in fact acting in an extortionate way?

These conditions correspond to:

\begin{align}
    \tilde p_1 & = \alpha R + \beta R - P (\alpha + \beta)
            \label{eqn:condition_for_tilde_p1}\\
    \tilde p_2 & = \alpha S + \beta T - P (\alpha + \beta)
            \label{eqn:condition_for_tilde_p2}\\
    \tilde p_3 & = \alpha T + \beta S - P (\alpha + \beta)
            \label{eqn:condition_for_tilde_p3}\\
    \tilde p_4 & = \alpha P + \beta P - P (\alpha + \beta)
            \label{eqn:condition_for_tilde_p4}
\end{align}

Equation (\ref{eqn:condition_for_tilde_p4}) ensures that \(p_4=\tilde p_4=0\).
Equations (\ref{eqn:condition_for_tilde_p1}-\ref{eqn:condition_for_tilde_p3})
can be used to eliminate \(\alpha, \beta\), giving:

\begin{equation}\label{eqn:planar_definition_of_extortion}
    \tilde p_1 = \frac{(R - P)(\tilde p_2 + \tilde p_3)}{S + T - 2P}
\end{equation}

with:

\begin{equation}\label{eqn:definition_of_chi}
    \chi = \frac{\tilde p_2 (P - T) + \tilde p_3 (S - P)}
                {\tilde p_2 (P - S) + \tilde p_3 (T - P)}
\end{equation}

Given a strategy \(p\in\mathbb{R}^{4\times 1}\) equations
(\ref{eqn:condition_for_tilde_p4}), (\ref{eqn:planar_definition_of_extortion}-\ref{eqn:definition_of_chi}) can be used to check if
a strategy is extortionate. The conditions correspond to:

\begin{align}
    p_1 & = \frac{(R-P)(p_2 + p_3) - R + T + S - P}{S + T - 2P}
     \label{eqn:condition_for_p1}\\
    p_4 & = 0 \label{eqn:condition_for_p4}\\
    1 & > p_2 + p_3\label{eqn:condition_for_chi}
\end{align}

The algebraic steps necessary to prove these results are available in the
supporting materials.

All extortionate strategies reside on a triangular (\ref{eqn:condition_for_chi})
plane (\ref{eqn:condition_for_p1}) in 3 dimensions (\ref{eqn:condition_for_p4}).
Using this formulation it can be seen that a necessary (but not sufficient)
condition for an extortionate strategy is that it cooperates on average less
than 50\% of the time when in a state of disagreement with the opponent.

As an example, consider the known extortionate strategy \(p=(8 / 9, 1 / 2, 1 /
3, 0)\) from~\cite{Stewart2012} which is referred to as \texttt{Extort-2}. In
this case, for the standard values of \((R, T, S, P)\) constraint
(\ref{eqn:condition_for_p1}) corresponds to:

\begin{equation}
    p_1 = \frac{2(p_2 + p_3) + 1}{3}
\end{equation}

It is clear that in this case all constraints hold.

This approach could in fact be used to confirm that a given strategy is acting
in an extortionate manner even if it is not a memory one strategy. However, in
practice, if a closed form for \(p\) is not known, then due to measurement
and/or numerical error this would not work.

This problem can be written in the following linear algebraic form where
\(x=(\alpha, \beta)\)
and \(p^*=(\tilde p_1 - 1, tilde_2 - 1, p_3)\):

\begin{equation}\label{eqn:linear_algebraic_equation_for_p}
    Cx= p^*
\end{equation}

\(C\) corresponds to equations
(\ref{eqn:condition_for_tilde_p1}-\ref{eqn:condition_for_tilde_p3}) and is
given by:

\begin{equation}\label{eqn:definition_of_C}
    C =
    \begin{bmatrix}
        R - P & R- P \\
        S - P & T- P \\
        T - P & S- P \\
    \end{bmatrix}
\end{equation}

Note that in general, equation (\ref{eqn:linear_algebraic_equation_for_p}) will
not necessarily have a solution. From the Rouch\'{e}-Capelli theorem if there is
a solution it is unique as \(\text{rank}(C)=2\) which is the dimension of the
variable \(x\). The best fitting \(x\) is found by minimizing:

\begin{equation}\label{eqn:r_squared}
    \text{SSError} = \|C x- p^*\|_2^2 = \sum_{i=1}^{3}\left((C\bar x)_i-p_i^*\right)^2
\end{equation}

Note that \(\text{SSError}\), which is the square of the Frobenius
norm~\cite{Golub2013}, becomes a measure of how close a strategy is to being an
extortionate strategy. Suspicion
of extortion then corresponds to a threshold on \(\text{SSError}\).

By observing interactions (human or otherwise), their memory one representation
can be inferred and this approach can be used to recognise extortionate
behaviour. The notion of comparing theoretic and actual plays of the IPD is not
novel, see for example~\cite{Rand2013}. Immediately it is noted that if the
environment is noisy~\cite{Wu1995} then no strategy can be considered to be
extortionate as \(p_4>0\).

In the next section, this idea will be illustrated by observing the interactions
that take place in a computer based tournament of the IPD\@.

\section{Numerical experiments}\label{sec:numerical-experiments}

In~\cite{Stewart2012} results from a tournament with
\input{./assets/tex/number_of_stewart_plotkin_strategies/main.tex} strategies,
was presented with specific consideration given to ZD strategies. This
tournament is reproduced here using the Axelrod-Python
project~\cite{Knight2016}. To obtain a good measure of the corresponding
transition rates for each strategy all matches have been run for
\input{assets/tex/number_of_turns/main.tex} turns and every match has been
repeated \input{assets/tex/number_of_repetitions/main.tex} times. All of this
interaction data is available at~\cite{vincent_knight_2018_1297075}. A good
match between the inferred Markov chain and the state distribution of the actual
interactions has been verified. Data for this is presented in the supplementary
materials.

Figure~\ref{fig:SSError_overall_in_stewart_plotkin} shows the \(\text{SSError}\)
values for all the strategies in the tournament, as reported
in~\cite{Stewart2012} the extortionate strategy (which has an expected
\(\text{SSError}\) approximately 0) gains a large number of wins.

\begin{figure}[!htbp]
    \centering
    \includegraphics[width=.8\textwidth]{./assets/img/SSError_overall_in_stewart_plotkin/main.pdf}
    \caption{\(\text{SSError}\) and state probabilities for the strategies
        of~\cite{Stewart2012}, ordered both by number of wins and overall score.
        Note that \(P(DC)\) is not shown as it corresponds to the transpose of
        \(P(CD)\). Cooperator and Defector are omitted as they do not visit all
        the states.}
    \label{fig:SSError_overall_in_stewart_plotkin}
\end{figure}

Here, the work of~\cite{Stewart2012} is extended by investigating a tournament
with \input{assets/tex/number_of_full_strategies/main.tex}
strategies.

The results of this analysis are shown in
Figure~\ref{fig:SSError_and_probabilities_in_full}. The top ranking strategies
by number of wins seem to be extortionate (but not against all strategies) and
it can be seen that a small sub group of strategies achieve mutual defection.
All the top ranking strategies according to score achieve mutual cooperation and
do not extort each other, however they
\textbf{do} exhibit extortionate behaviour towards a number of the lower ranking
strategies.

\begin{figure}[!htbp]
    \centering
    \includegraphics[width=.8\textwidth]{./assets/img/SSError_and_probabilities_in_full/main.pdf}
    \caption{\(\text{SSError}\) for the strategies for the full tournament. Only
    strategy interactions for which \(p_4=0\) and \(\chi>1\) are displayed.}
    \label{fig:SSError_and_probabilities_in_full}
\end{figure}

\section{Conclusion}\label{sec:conclusion}

This work defines an approach to measure whether or not a player is playing a
strategy that corresponds to an extortionate strategy as defined
in~\cite{Press2012}: a mathematical model for suspicion. Indeed, all
extortionate strategies have been
 classified as lying on a triangular plane.
This rigorous classification fails to be robust to small measurement error, thus
a statistical approach is proposed.
This is done through a linear algebraic approach for approximating the solution
of a linear system. Using this, a large number of pairwise interactions is
simulated and in fact very few strategies are found to act extortionately.

The work of~\cite{Press2012}, whilst showing that a clever approach to taking
advantage of another memory one strategy exists: this is incomplete. Whilst the
elegance of this result is very attractive, just as the simplicity of the
victory of Tit For Tat in Axelrod's original tournaments was, it is incomplete.
Extortionate strategies achieve a high number of wins but they do not
achieve a high score which corresponds to the fitness landscape in an
evolutionary sense. From the large number of interactions a payoff matrix \(S\)
can be measured where \(S_{ij}\) denotes the score (using standard values of
\((R, S, T, P) = (3, 0, 5, 1)\)) of the \(i\)th strategy
against the \(j\)th strategy. Using this, the replicator equation
describes the evolution of the system based on a population density fitness
function:

\begin{equation}\label{eqn:replicator_dynamics}
    \frac{dx}{dt} = x(S-x^TS x)
\end{equation}

Equation (\ref{eqn:replicator_dynamics}) is solved numerically through an
integration technique described in~\cite{Petzold1983} and
Figure~\ref{fig:replicator_dynamics} shows the evolution of the distribution of
the system: the various strategies are ranked by scores. It is clear to see that
only the high ranking strategies survive the evolutionary process (in fact,
only \input{./assets/img/replicator_dynamics/main.tex}
have a final distribution greater than \(10 ^ {-2}\)). This confirms the
findings of~\cite{Moran1707} in which sophisticated strategies resist
evolutionary invasion of shorter memory strategies. Recalling
Figure~\ref{fig:SSError_and_probabilities_in_full} this demonstrates that:

\begin{itemize}
    \item Cooperation emerges through the evolutionary process: the high scoring
        strategies do not exhibit extortionate behaviour towards each other.
    \item Extortionate strategies do not survive the evolutionary process.
\end{itemize}

\begin{figure}[!htbp]
    \centering
    \includegraphics[width=.8\textwidth]{./assets/img/replicator_dynamics/main.pdf}
    \caption{Numerical simulation of the replicator equation
    (\ref{eqn:replicator_dynamics}): strategies are ordered by score, only the strategies with a high score survive the evolutionary process.}
    \label{fig:replicator_dynamics}
\end{figure}

This work can be used to classify plays of the IPD\@: data can be collected from
actual interactions (in lab or in the field). Furthermore, this allows for a
classification method similar to the notion of fingerprinting presented
in~\cite{Ashlock2008}. Trained strategies can potentially be classified as
extortionate or not or it could be possible to even constrain the reinforcement
learning approaches that are becoming prevalent in the literature.
Alternatively, this mathematical approach for recognising extortion could be
used in sophisticated strategies to defend against invasion. Arguably, some of
the strategies considered here exhibit this behaviour, indeed as described
in~\cite{Harper2017}, the top ranking strategies in the full tournament are
obtained using evolutionary reinforcement learning techniques, thus, suspicion
of extortionate behaviour could in fact be an evolutionary trait.

\section*{Acknowledgements}

The following open source software libraries were used in this research:

\begin{itemize}
    \item The Axelrod ~\cite{Knight2016, Knight2018} library (IPD strategies and
        tournaments).
    \item The sympy library~\cite{Meurer2017} (verification of all symbolic
        calculations).
    \item The matplotlib~\cite{Droettboom2018} library (visualisation).
    \item The pandas~\cite{Structures2010}, dask~\cite{Dask2016} and
        NumPy~\cite{Oliphant2015} libraries (data manipulation).
    \item The SciPy~\cite{Jones2001} library (numerical integration of the
        replicator equation).
\end{itemize}

This work was performed using the computational facilities of the Advanced
Research Computing @ Cardiff (ARCCA) Division, Cardiff University.

\printbibliography

\newpage
\section*{Supplementary materials}

\includepdf{assets/pdf/proof_of_form_of_extortionate_strategies/main.pdf}

\newpage

Using the pair wise interactions the transition rates \(p,
q\) can be measured and the steady state probabilities inferred and compared to
the actual probabilities of each state.
This is done numerically by computing the singular eigenvector of the
matrix \(A\) \cite{Stewart2009}:

\[
    A =
    \begin{bmatrix}
        p_1 q_1 & p_1 (1 - q_1) & (1 - p_1) q_1 & (1 -p_1) (1 - q_1) \\
        p_2 q_2 & p_2 (1 - q_2) & (1 - p_2) q_2 & (1 -p_2) (1 - q_2) \\
        p_3 q_3 & p_3 (1 - q_3) & (1 - p_3) q_3 & (1 -p_3) (1 - q_3) \\
        p_4 q_4 & p_4 (1 - q_4) & (1 - p_4) q_4 & (1 -p_4) (1 - q_4) \\
    \end{bmatrix}
\]

Figure~\ref{fig:computed_probabilities_vs_theoretic_probabilities} shows a
regression line fitted to every pairwise interaction with a reported
\(\text{SSError}\) value (pairwise interactions with missing states were
omitted). This serves to validate the approach: a part from some edge cases the
relationship is consistent.

\begin{figure}[!htbp]
    \centering
    \includegraphics[width=.8\textwidth]{./assets/img/computed_probabilities_vs_theoretic_probabilities/main.pdf}
    \caption{The
        relationship between the steady state probabilities inferred from the
        measured transitions and the actual steady state probabilities. A linear
        regression line is included validating the approach.}
    \label{fig:computed_probabilities_vs_theoretic_probabilities}
\end{figure}


\end{document}

have a final distribution greater than \(10 ^ {-2}\)). This confirms the
findings of~\cite{Moran1707} in which sophisticated strategies resist
evolutionary invasion of shorter memory strategies. Recalling
Figure~\ref{fig:SSError_and_probabilities_in_full} this demonstrates that:

\begin{itemize}
    \item Cooperation emerges through the evolutionary process: the high scoring
        strategies do not exhibit extortionate behaviour towards each other.
    \item Extortionate strategies do not survive the evolutionary process.
\end{itemize}

\begin{figure}[!htbp]
    \centering
    \includegraphics[width=.8\textwidth]{./assets/img/replicator_dynamics/main.pdf}
    \caption{Numerical simulation of the replicator equation
    (\ref{eqn:replicator_dynamics}): strategies are ordered by score, only the strategies with a high score survive the evolutionary process.}
    \label{fig:replicator_dynamics}
\end{figure}

This work can be used to classify plays of the IPD\@: data can be collected from
actual interactions (in lab or in the field). Furthermore, this allows for a
classification method similar to the notion of fingerprinting presented
in~\cite{Ashlock2008}. Trained strategies can potentially be classified as
extortionate or not or it could be possible to even constrain the reinforcement
learning approaches that are becoming prevalent in the literature.
Alternatively, this mathematical approach for recognising extortion could be
used in sophisticated strategies to defend against invasion. Arguably, some of
the strategies considered here exhibit this behaviour, indeed as described
in~\cite{Harper2017}, the top ranking strategies in the full tournament are
obtained using evolutionary reinforcement learning techniques, thus, suspicion
of extortionate behaviour could in fact be an evolutionary trait.

\section*{Acknowledgements}

The following open source software libraries were used in this research:

\begin{itemize}
    \item The Axelrod ~\cite{Knight2016, Knight2018} library (IPD strategies and
        tournaments).
    \item The sympy library~\cite{Meurer2017} (verification of all symbolic
        calculations).
    \item The matplotlib~\cite{Droettboom2018} library (visualisation).
    \item The pandas~\cite{Structures2010}, dask~\cite{Dask2016} and
        NumPy~\cite{Oliphant2015} libraries (data manipulation).
    \item The SciPy~\cite{Jones2001} library (numerical integration of the
        replicator equation).
\end{itemize}

This work was performed using the computational facilities of the Advanced
Research Computing @ Cardiff (ARCCA) Division, Cardiff University.

\printbibliography

\newpage
\section*{Supplementary materials}

\includepdf{assets/pdf/proof_of_form_of_extortionate_strategies/main.pdf}

\newpage

Using the pair wise interactions the transition rates \(p,
q\) can be measured and the steady state probabilities inferred and compared to
the actual probabilities of each state.
This is done numerically by computing the singular eigenvector of the
matrix \(A\) \cite{Stewart2009}:

\[
    A =
    \begin{bmatrix}
        p_1 q_1 & p_1 (1 - q_1) & (1 - p_1) q_1 & (1 -p_1) (1 - q_1) \\
        p_2 q_2 & p_2 (1 - q_2) & (1 - p_2) q_2 & (1 -p_2) (1 - q_2) \\
        p_3 q_3 & p_3 (1 - q_3) & (1 - p_3) q_3 & (1 -p_3) (1 - q_3) \\
        p_4 q_4 & p_4 (1 - q_4) & (1 - p_4) q_4 & (1 -p_4) (1 - q_4) \\
    \end{bmatrix}
\]

Figure~\ref{fig:computed_probabilities_vs_theoretic_probabilities} shows a
regression line fitted to every pairwise interaction with a reported
\(\text{SSError}\) value (pairwise interactions with missing states were
omitted). This serves to validate the approach: a part from some edge cases the
relationship is consistent.

\begin{figure}[!htbp]
    \centering
    \includegraphics[width=.8\textwidth]{./assets/img/computed_probabilities_vs_theoretic_probabilities/main.pdf}
    \caption{The
        relationship between the steady state probabilities inferred from the
        measured transitions and the actual steady state probabilities. A linear
        regression line is included validating the approach.}
    \label{fig:computed_probabilities_vs_theoretic_probabilities}
\end{figure}


\end{document}

    strategies is considered. In this setting
    the most highly performing strategies do not play in an extortionate way
    against each other but do against lower performing strategies.
    This suggests that whilst the theory of Zero Determinant strategies
    indicates that memory is not of fundamental importance to the evolution of
    cooperative behaviour, this is incomplete.
\end{abstract}

\section{Introduction}\label{sec:introduction}

Agent based game theoretic models have become a stalwart of the underpinning
mathematics of interactive behaviours. One of the major pieces of work
in this area is the pair of original computer tournaments run by Robert
Axelrod~\cite{Axelrod1980, Axelrod1980a}. These tournaments pitted submitted
computer strategies against each other in plays of the Iterated Prisoner's
Dilemma. A common game where agents can choose to pay a slight cost to their
immediate utility in the hope of building a reputation. This has been used in
economic and evolutionary game theory to understand the evolution of cooperative
behaviour.

Recently, a class of strategies was described in~\cite{Press2012} that can
provably extort any given opponent. In~\cite{Hilbe2013, Moran1707} some
questions have already been asked about the true effectiveness of these
strategies in an evolutionary setting. Here another question is asked: is it
possible to recognise this extortionate behaviour? A mathematical procedure for
suspicion is presented: in the same way that the continued actions of an
extortionate individual might raise suspicion.

This work makes use of the Axelrod Python library~\cite{Knight2018, Knight2016}
with a large number of Prisoner Dilemma strategies available to give an
extensive numerical example of the ideas presented.  The approach is presented
in Section~\ref{sec:delta-zd-strategies}.  All of the code and data discussed
in Section~\ref{sec:numerical-experiments} is open sourced, archived and
written according to best scientific principles~\cite{Wilson2014}. The data
archive can be found at~\cite{vincent_knight_2018_1297075}.

\section{Recognising Extortion}\label{sec:delta-zd-strategies}

In~\cite{Press2012}, given a match between 2 memory-one strategies, the concept
of Zero Determinant (ZD) strategies is introduced. The main result of that paper
shows that given two memory one players \(p, q\in\mathbb{R}^4\) a linear
relationship between the players' scores could be forced by one of the players.

Using the notation of~\cite{Press2012}, assuming the utilities for player \(p\)
are given by \(S_x=(R, S, T, P)\) and for player \(q\) by \(S_y=(R, T, S, P)\)
and that the stationary scores of each player is given by \(S_X\) and \(S_Y\)
respectively. The main result of~\cite{Press2012} is that if

\begin{equation}\label{eqn:linear_relationship_for_p}
    \tilde p=\alpha S_x + \beta S_y + \gamma
\end{equation}

or

\begin{equation}\label{eqn:linear_relationship_for_q}
    \tilde q=\alpha S_x + \beta S_y + \gamma
\end{equation}

where \(\tilde p = (1 - p_1, 1 - p_2, p_3, p_4)\) and
\(\tilde q = (1 - q_1, 1 - q_2, q_3, q_4)\) then:

\begin{equation}
    \alpha S_X + \beta S_Y + \gamma = 0
\end{equation}

In~\cite{Press2012} a particular type of ZD strategy is defined: extortionate
strategies. If:

\begin{equation}\label{eqn:constraint_for_extortion}
    \gamma = - P(\alpha + \beta)
\end{equation}

then the player can ensure they get a score \(\chi\) times
larger than the opponent. This extortion coefficient is given by:

\begin{equation}\label{eqn:definition_of_chi}
    \chi=\frac{-\beta}{\alpha}
\end{equation}

Thus, if (\ref{eqn:constraint_for_extortion}) holds and \(\chi >1\) a player is
said to extort their opponent.
Here, the reverse problem is considered: given a
\(p\in\mathbb{R}^4\) how does one identify \(\alpha, \beta\) if they
exist and is the strategy in fact acting in an extortionate way?

These conditions correspond to:

\begin{align}
    \tilde p_1 & = \alpha R + \beta R - P (\alpha + \beta)
            \label{eqn:condition_for_tilde_p1}\\
    \tilde p_2 & = \alpha S + \beta T - P (\alpha + \beta)
            \label{eqn:condition_for_tilde_p2}\\
    \tilde p_3 & = \alpha T + \beta S - P (\alpha + \beta)
            \label{eqn:condition_for_tilde_p3}\\
    \tilde p_4 & = \alpha P + \beta P - P (\alpha + \beta)
            \label{eqn:condition_for_tilde_p4}
\end{align}

Equation (\ref{eqn:condition_for_tilde_p4}) ensures that \(p_4=\tilde p_4=0\).
Equations (\ref{eqn:condition_for_tilde_p1}-\ref{eqn:condition_for_tilde_p3})
can be used to eliminate \(\alpha, \beta\), giving:

\begin{equation}\label{eqn:planar_definition_of_extortion}
    \tilde p_1 = \frac{(R - P)(\tilde p_2 + \tilde p_3)}{S + T - 2P}
\end{equation}

with:

\begin{equation}\label{eqn:definition_of_chi}
    \chi = \frac{\tilde p_2 (P - T) + \tilde p_3 (S - P)}
                {\tilde p_2 (P - S) + \tilde p_3 (T - P)}
\end{equation}

Given a strategy \(p\in\mathbb{R}^{4\times 1}\) equations
(\ref{eqn:condition_for_tilde_p4}), (\ref{eqn:planar_definition_of_extortion}-\ref{eqn:definition_of_chi}) can be used to check if
a strategy is extortionate. The conditions correspond to:

\begin{align}
    p_1 & = \frac{(R-P)(p_2 + p_3) - R + T + S - P}{S + T - 2P}
     \label{eqn:condition_for_p1}\\
    p_4 & = 0 \label{eqn:condition_for_p4}\\
    1 & > p_2 + p_3\label{eqn:condition_for_chi}
\end{align}

The algebraic steps necessary to prove these results are available in the
supporting materials.

All extortionate strategies reside on a triangular (\ref{eqn:condition_for_chi})
plane (\ref{eqn:condition_for_p1}) in 3 dimensions (\ref{eqn:condition_for_p4}).
Using this formulation it can be seen that a necessary (but not sufficient)
condition for an extortionate strategy is that it cooperates on average less
than 50\% of the time when in a state of disagreement with the opponent.

As an example, consider the known extortionate strategy \(p=(8 / 9, 1 / 2, 1 /
3, 0)\) from~\cite{Stewart2012} which is referred to as \texttt{Extort-2}. In
this case, for the standard values of \((R, T, S, P)\) constraint
(\ref{eqn:condition_for_p1}) corresponds to:

\begin{equation}
    p_1 = \frac{2(p_2 + p_3) + 1}{3}
\end{equation}

It is clear that in this case all constraints hold.

This approach could in fact be used to confirm that a given strategy is acting
in an extortionate manner even if it is not a memory one strategy. However, in
practice, if a closed form for \(p\) is not known, then due to measurement
and/or numerical error this would not work.

This problem can be written in the following linear algebraic form where
\(x=(\alpha, \beta)\)
and \(p^*=(\tilde p_1 - 1, tilde_2 - 1, p_3)\):

\begin{equation}\label{eqn:linear_algebraic_equation_for_p}
    Cx= p^*
\end{equation}

\(C\) corresponds to equations
(\ref{eqn:condition_for_tilde_p1}-\ref{eqn:condition_for_tilde_p3}) and is
given by:

\begin{equation}\label{eqn:definition_of_C}
    C =
    \begin{bmatrix}
        R - P & R- P \\
        S - P & T- P \\
        T - P & S- P \\
    \end{bmatrix}
\end{equation}

Note that in general, equation (\ref{eqn:linear_algebraic_equation_for_p}) will
not necessarily have a solution. From the Rouch\'{e}-Capelli theorem if there is
a solution it is unique as \(\text{rank}(C)=2\) which is the dimension of the
variable \(x\). The best fitting \(x\) is found by minimizing:

\begin{equation}\label{eqn:r_squared}
    \text{SSError} = \|C x- p^*\|_2^2 = \sum_{i=1}^{3}\left((C\bar x)_i-p_i^*\right)^2
\end{equation}

Note that \(\text{SSError}\), which is the square of the Frobenius
norm~\cite{Golub2013}, becomes a measure of how close a strategy is to being an
extortionate strategy. Suspicion
of extortion then corresponds to a threshold on \(\text{SSError}\).

By observing interactions (human or otherwise), their memory one representation
can be inferred and this approach can be used to recognise extortionate
behaviour. The notion of comparing theoretic and actual plays of the IPD is not
novel, see for example~\cite{Rand2013}. Immediately it is noted that if the
environment is noisy~\cite{Wu1995} then no strategy can be considered to be
extortionate as \(p_4>0\).

In the next section, this idea will be illustrated by observing the interactions
that take place in a computer based tournament of the IPD\@.

\section{Numerical experiments}\label{sec:numerical-experiments}

In~\cite{Stewart2012} results from a tournament with
\documentclass[a4paper]{article}

\usepackage{amsmath}
\usepackage{amssymb}
\usepackage[margin=1.5cm,
            includefoot,
            footskip=30pt]{geometry}
\usepackage{layout}
\usepackage{graphicx}
\usepackage{subcaption}

\usepackage{biblatex}
\usepackage{pdfpages}

\bibliography{main.bib}

\title{Suspicion: Recognising and evaluating the effectiveness
       of extortion in the Iterated Prisoner's Dilemma}
\author{Vincent A. Knight \and Nikoleta E. Glynatsi}
\date{\today}



\begin{document}

\maketitle

\begin{abstract}
    The Iterated Prisoner's Dilemma is a model for rational and evolutionary
    interactive behaviour. It has applications both in the study of human social
    behaviour as well as in biology.
    It is used to understand when and how a rational individual might
    accept an immediate cost to their own utility for the direct benefit of
    another.

    Much attention has been given to a class of strategies called
    Zero Determinant strategies. It has been theoretically shown that these
    strategies can ``extort'' any player.

    In this work, an approach to identify if observed strategies are playing in
    an extortionate way is described. Furthermore, experimental analysis of
    a large tournament with \documentclass[a4paper]{article}

\usepackage{amsmath}
\usepackage{amssymb}
\usepackage[margin=1.5cm,
            includefoot,
            footskip=30pt]{geometry}
\usepackage{layout}
\usepackage{graphicx}
\usepackage{subcaption}

\usepackage{biblatex}
\usepackage{pdfpages}

\bibliography{main.bib}

\title{Suspicion: Recognising and evaluating the effectiveness
       of extortion in the Iterated Prisoner's Dilemma}
\author{Vincent A. Knight \and Nikoleta E. Glynatsi}
\date{\today}



\begin{document}

\maketitle

\begin{abstract}
    The Iterated Prisoner's Dilemma is a model for rational and evolutionary
    interactive behaviour. It has applications both in the study of human social
    behaviour as well as in biology.
    It is used to understand when and how a rational individual might
    accept an immediate cost to their own utility for the direct benefit of
    another.

    Much attention has been given to a class of strategies called
    Zero Determinant strategies. It has been theoretically shown that these
    strategies can ``extort'' any player.

    In this work, an approach to identify if observed strategies are playing in
    an extortionate way is described. Furthermore, experimental analysis of
    a large tournament with \input{assets/tex/number_of_full_strategies/main.tex}
    strategies is considered. In this setting
    the most highly performing strategies do not play in an extortionate way
    against each other but do against lower performing strategies.
    This suggests that whilst the theory of Zero Determinant strategies
    indicates that memory is not of fundamental importance to the evolution of
    cooperative behaviour, this is incomplete.
\end{abstract}

\section{Introduction}\label{sec:introduction}

Agent based game theoretic models have become a stalwart of the underpinning
mathematics of interactive behaviours. One of the major pieces of work
in this area is the pair of original computer tournaments run by Robert
Axelrod~\cite{Axelrod1980, Axelrod1980a}. These tournaments pitted submitted
computer strategies against each other in plays of the Iterated Prisoner's
Dilemma. A common game where agents can choose to pay a slight cost to their
immediate utility in the hope of building a reputation. This has been used in
economic and evolutionary game theory to understand the evolution of cooperative
behaviour.

Recently, a class of strategies was described in~\cite{Press2012} that can
provably extort any given opponent. In~\cite{Hilbe2013, Moran1707} some
questions have already been asked about the true effectiveness of these
strategies in an evolutionary setting. Here another question is asked: is it
possible to recognise this extortionate behaviour? A mathematical procedure for
suspicion is presented: in the same way that the continued actions of an
extortionate individual might raise suspicion.

This work makes use of the Axelrod Python library~\cite{Knight2018, Knight2016}
with a large number of Prisoner Dilemma strategies available to give an
extensive numerical example of the ideas presented.  The approach is presented
in Section~\ref{sec:delta-zd-strategies}.  All of the code and data discussed
in Section~\ref{sec:numerical-experiments} is open sourced, archived and
written according to best scientific principles~\cite{Wilson2014}. The data
archive can be found at~\cite{vincent_knight_2018_1297075}.

\section{Recognising Extortion}\label{sec:delta-zd-strategies}

In~\cite{Press2012}, given a match between 2 memory-one strategies, the concept
of Zero Determinant (ZD) strategies is introduced. The main result of that paper
shows that given two memory one players \(p, q\in\mathbb{R}^4\) a linear
relationship between the players' scores could be forced by one of the players.

Using the notation of~\cite{Press2012}, assuming the utilities for player \(p\)
are given by \(S_x=(R, S, T, P)\) and for player \(q\) by \(S_y=(R, T, S, P)\)
and that the stationary scores of each player is given by \(S_X\) and \(S_Y\)
respectively. The main result of~\cite{Press2012} is that if

\begin{equation}\label{eqn:linear_relationship_for_p}
    \tilde p=\alpha S_x + \beta S_y + \gamma
\end{equation}

or

\begin{equation}\label{eqn:linear_relationship_for_q}
    \tilde q=\alpha S_x + \beta S_y + \gamma
\end{equation}

where \(\tilde p = (1 - p_1, 1 - p_2, p_3, p_4)\) and
\(\tilde q = (1 - q_1, 1 - q_2, q_3, q_4)\) then:

\begin{equation}
    \alpha S_X + \beta S_Y + \gamma = 0
\end{equation}

In~\cite{Press2012} a particular type of ZD strategy is defined: extortionate
strategies. If:

\begin{equation}\label{eqn:constraint_for_extortion}
    \gamma = - P(\alpha + \beta)
\end{equation}

then the player can ensure they get a score \(\chi\) times
larger than the opponent. This extortion coefficient is given by:

\begin{equation}\label{eqn:definition_of_chi}
    \chi=\frac{-\beta}{\alpha}
\end{equation}

Thus, if (\ref{eqn:constraint_for_extortion}) holds and \(\chi >1\) a player is
said to extort their opponent.
Here, the reverse problem is considered: given a
\(p\in\mathbb{R}^4\) how does one identify \(\alpha, \beta\) if they
exist and is the strategy in fact acting in an extortionate way?

These conditions correspond to:

\begin{align}
    \tilde p_1 & = \alpha R + \beta R - P (\alpha + \beta)
            \label{eqn:condition_for_tilde_p1}\\
    \tilde p_2 & = \alpha S + \beta T - P (\alpha + \beta)
            \label{eqn:condition_for_tilde_p2}\\
    \tilde p_3 & = \alpha T + \beta S - P (\alpha + \beta)
            \label{eqn:condition_for_tilde_p3}\\
    \tilde p_4 & = \alpha P + \beta P - P (\alpha + \beta)
            \label{eqn:condition_for_tilde_p4}
\end{align}

Equation (\ref{eqn:condition_for_tilde_p4}) ensures that \(p_4=\tilde p_4=0\).
Equations (\ref{eqn:condition_for_tilde_p1}-\ref{eqn:condition_for_tilde_p3})
can be used to eliminate \(\alpha, \beta\), giving:

\begin{equation}\label{eqn:planar_definition_of_extortion}
    \tilde p_1 = \frac{(R - P)(\tilde p_2 + \tilde p_3)}{S + T - 2P}
\end{equation}

with:

\begin{equation}\label{eqn:definition_of_chi}
    \chi = \frac{\tilde p_2 (P - T) + \tilde p_3 (S - P)}
                {\tilde p_2 (P - S) + \tilde p_3 (T - P)}
\end{equation}

Given a strategy \(p\in\mathbb{R}^{4\times 1}\) equations
(\ref{eqn:condition_for_tilde_p4}), (\ref{eqn:planar_definition_of_extortion}-\ref{eqn:definition_of_chi}) can be used to check if
a strategy is extortionate. The conditions correspond to:

\begin{align}
    p_1 & = \frac{(R-P)(p_2 + p_3) - R + T + S - P}{S + T - 2P}
     \label{eqn:condition_for_p1}\\
    p_4 & = 0 \label{eqn:condition_for_p4}\\
    1 & > p_2 + p_3\label{eqn:condition_for_chi}
\end{align}

The algebraic steps necessary to prove these results are available in the
supporting materials.

All extortionate strategies reside on a triangular (\ref{eqn:condition_for_chi})
plane (\ref{eqn:condition_for_p1}) in 3 dimensions (\ref{eqn:condition_for_p4}).
Using this formulation it can be seen that a necessary (but not sufficient)
condition for an extortionate strategy is that it cooperates on average less
than 50\% of the time when in a state of disagreement with the opponent.

As an example, consider the known extortionate strategy \(p=(8 / 9, 1 / 2, 1 /
3, 0)\) from~\cite{Stewart2012} which is referred to as \texttt{Extort-2}. In
this case, for the standard values of \((R, T, S, P)\) constraint
(\ref{eqn:condition_for_p1}) corresponds to:

\begin{equation}
    p_1 = \frac{2(p_2 + p_3) + 1}{3}
\end{equation}

It is clear that in this case all constraints hold.

This approach could in fact be used to confirm that a given strategy is acting
in an extortionate manner even if it is not a memory one strategy. However, in
practice, if a closed form for \(p\) is not known, then due to measurement
and/or numerical error this would not work.

This problem can be written in the following linear algebraic form where
\(x=(\alpha, \beta)\)
and \(p^*=(\tilde p_1 - 1, tilde_2 - 1, p_3)\):

\begin{equation}\label{eqn:linear_algebraic_equation_for_p}
    Cx= p^*
\end{equation}

\(C\) corresponds to equations
(\ref{eqn:condition_for_tilde_p1}-\ref{eqn:condition_for_tilde_p3}) and is
given by:

\begin{equation}\label{eqn:definition_of_C}
    C =
    \begin{bmatrix}
        R - P & R- P \\
        S - P & T- P \\
        T - P & S- P \\
    \end{bmatrix}
\end{equation}

Note that in general, equation (\ref{eqn:linear_algebraic_equation_for_p}) will
not necessarily have a solution. From the Rouch\'{e}-Capelli theorem if there is
a solution it is unique as \(\text{rank}(C)=2\) which is the dimension of the
variable \(x\). The best fitting \(x\) is found by minimizing:

\begin{equation}\label{eqn:r_squared}
    \text{SSError} = \|C x- p^*\|_2^2 = \sum_{i=1}^{3}\left((C\bar x)_i-p_i^*\right)^2
\end{equation}

Note that \(\text{SSError}\), which is the square of the Frobenius
norm~\cite{Golub2013}, becomes a measure of how close a strategy is to being an
extortionate strategy. Suspicion
of extortion then corresponds to a threshold on \(\text{SSError}\).

By observing interactions (human or otherwise), their memory one representation
can be inferred and this approach can be used to recognise extortionate
behaviour. The notion of comparing theoretic and actual plays of the IPD is not
novel, see for example~\cite{Rand2013}. Immediately it is noted that if the
environment is noisy~\cite{Wu1995} then no strategy can be considered to be
extortionate as \(p_4>0\).

In the next section, this idea will be illustrated by observing the interactions
that take place in a computer based tournament of the IPD\@.

\section{Numerical experiments}\label{sec:numerical-experiments}

In~\cite{Stewart2012} results from a tournament with
\input{./assets/tex/number_of_stewart_plotkin_strategies/main.tex} strategies,
was presented with specific consideration given to ZD strategies. This
tournament is reproduced here using the Axelrod-Python
project~\cite{Knight2016}. To obtain a good measure of the corresponding
transition rates for each strategy all matches have been run for
\input{assets/tex/number_of_turns/main.tex} turns and every match has been
repeated \input{assets/tex/number_of_repetitions/main.tex} times. All of this
interaction data is available at~\cite{vincent_knight_2018_1297075}. A good
match between the inferred Markov chain and the state distribution of the actual
interactions has been verified. Data for this is presented in the supplementary
materials.

Figure~\ref{fig:SSError_overall_in_stewart_plotkin} shows the \(\text{SSError}\)
values for all the strategies in the tournament, as reported
in~\cite{Stewart2012} the extortionate strategy (which has an expected
\(\text{SSError}\) approximately 0) gains a large number of wins.

\begin{figure}[!htbp]
    \centering
    \includegraphics[width=.8\textwidth]{./assets/img/SSError_overall_in_stewart_plotkin/main.pdf}
    \caption{\(\text{SSError}\) and state probabilities for the strategies
        of~\cite{Stewart2012}, ordered both by number of wins and overall score.
        Note that \(P(DC)\) is not shown as it corresponds to the transpose of
        \(P(CD)\). Cooperator and Defector are omitted as they do not visit all
        the states.}
    \label{fig:SSError_overall_in_stewart_plotkin}
\end{figure}

Here, the work of~\cite{Stewart2012} is extended by investigating a tournament
with \input{assets/tex/number_of_full_strategies/main.tex}
strategies.

The results of this analysis are shown in
Figure~\ref{fig:SSError_and_probabilities_in_full}. The top ranking strategies
by number of wins seem to be extortionate (but not against all strategies) and
it can be seen that a small sub group of strategies achieve mutual defection.
All the top ranking strategies according to score achieve mutual cooperation and
do not extort each other, however they
\textbf{do} exhibit extortionate behaviour towards a number of the lower ranking
strategies.

\begin{figure}[!htbp]
    \centering
    \includegraphics[width=.8\textwidth]{./assets/img/SSError_and_probabilities_in_full/main.pdf}
    \caption{\(\text{SSError}\) for the strategies for the full tournament. Only
    strategy interactions for which \(p_4=0\) and \(\chi>1\) are displayed.}
    \label{fig:SSError_and_probabilities_in_full}
\end{figure}

\section{Conclusion}\label{sec:conclusion}

This work defines an approach to measure whether or not a player is playing a
strategy that corresponds to an extortionate strategy as defined
in~\cite{Press2012}: a mathematical model for suspicion. Indeed, all
extortionate strategies have been
 classified as lying on a triangular plane.
This rigorous classification fails to be robust to small measurement error, thus
a statistical approach is proposed.
This is done through a linear algebraic approach for approximating the solution
of a linear system. Using this, a large number of pairwise interactions is
simulated and in fact very few strategies are found to act extortionately.

The work of~\cite{Press2012}, whilst showing that a clever approach to taking
advantage of another memory one strategy exists: this is incomplete. Whilst the
elegance of this result is very attractive, just as the simplicity of the
victory of Tit For Tat in Axelrod's original tournaments was, it is incomplete.
Extortionate strategies achieve a high number of wins but they do not
achieve a high score which corresponds to the fitness landscape in an
evolutionary sense. From the large number of interactions a payoff matrix \(S\)
can be measured where \(S_{ij}\) denotes the score (using standard values of
\((R, S, T, P) = (3, 0, 5, 1)\)) of the \(i\)th strategy
against the \(j\)th strategy. Using this, the replicator equation
describes the evolution of the system based on a population density fitness
function:

\begin{equation}\label{eqn:replicator_dynamics}
    \frac{dx}{dt} = x(S-x^TS x)
\end{equation}

Equation (\ref{eqn:replicator_dynamics}) is solved numerically through an
integration technique described in~\cite{Petzold1983} and
Figure~\ref{fig:replicator_dynamics} shows the evolution of the distribution of
the system: the various strategies are ranked by scores. It is clear to see that
only the high ranking strategies survive the evolutionary process (in fact,
only \input{./assets/img/replicator_dynamics/main.tex}
have a final distribution greater than \(10 ^ {-2}\)). This confirms the
findings of~\cite{Moran1707} in which sophisticated strategies resist
evolutionary invasion of shorter memory strategies. Recalling
Figure~\ref{fig:SSError_and_probabilities_in_full} this demonstrates that:

\begin{itemize}
    \item Cooperation emerges through the evolutionary process: the high scoring
        strategies do not exhibit extortionate behaviour towards each other.
    \item Extortionate strategies do not survive the evolutionary process.
\end{itemize}

\begin{figure}[!htbp]
    \centering
    \includegraphics[width=.8\textwidth]{./assets/img/replicator_dynamics/main.pdf}
    \caption{Numerical simulation of the replicator equation
    (\ref{eqn:replicator_dynamics}): strategies are ordered by score, only the strategies with a high score survive the evolutionary process.}
    \label{fig:replicator_dynamics}
\end{figure}

This work can be used to classify plays of the IPD\@: data can be collected from
actual interactions (in lab or in the field). Furthermore, this allows for a
classification method similar to the notion of fingerprinting presented
in~\cite{Ashlock2008}. Trained strategies can potentially be classified as
extortionate or not or it could be possible to even constrain the reinforcement
learning approaches that are becoming prevalent in the literature.
Alternatively, this mathematical approach for recognising extortion could be
used in sophisticated strategies to defend against invasion. Arguably, some of
the strategies considered here exhibit this behaviour, indeed as described
in~\cite{Harper2017}, the top ranking strategies in the full tournament are
obtained using evolutionary reinforcement learning techniques, thus, suspicion
of extortionate behaviour could in fact be an evolutionary trait.

\section*{Acknowledgements}

The following open source software libraries were used in this research:

\begin{itemize}
    \item The Axelrod ~\cite{Knight2016, Knight2018} library (IPD strategies and
        tournaments).
    \item The sympy library~\cite{Meurer2017} (verification of all symbolic
        calculations).
    \item The matplotlib~\cite{Droettboom2018} library (visualisation).
    \item The pandas~\cite{Structures2010}, dask~\cite{Dask2016} and
        NumPy~\cite{Oliphant2015} libraries (data manipulation).
    \item The SciPy~\cite{Jones2001} library (numerical integration of the
        replicator equation).
\end{itemize}

This work was performed using the computational facilities of the Advanced
Research Computing @ Cardiff (ARCCA) Division, Cardiff University.

\printbibliography

\newpage
\section*{Supplementary materials}

\includepdf{assets/pdf/proof_of_form_of_extortionate_strategies/main.pdf}

\newpage

Using the pair wise interactions the transition rates \(p,
q\) can be measured and the steady state probabilities inferred and compared to
the actual probabilities of each state.
This is done numerically by computing the singular eigenvector of the
matrix \(A\) \cite{Stewart2009}:

\[
    A =
    \begin{bmatrix}
        p_1 q_1 & p_1 (1 - q_1) & (1 - p_1) q_1 & (1 -p_1) (1 - q_1) \\
        p_2 q_2 & p_2 (1 - q_2) & (1 - p_2) q_2 & (1 -p_2) (1 - q_2) \\
        p_3 q_3 & p_3 (1 - q_3) & (1 - p_3) q_3 & (1 -p_3) (1 - q_3) \\
        p_4 q_4 & p_4 (1 - q_4) & (1 - p_4) q_4 & (1 -p_4) (1 - q_4) \\
    \end{bmatrix}
\]

Figure~\ref{fig:computed_probabilities_vs_theoretic_probabilities} shows a
regression line fitted to every pairwise interaction with a reported
\(\text{SSError}\) value (pairwise interactions with missing states were
omitted). This serves to validate the approach: a part from some edge cases the
relationship is consistent.

\begin{figure}[!htbp]
    \centering
    \includegraphics[width=.8\textwidth]{./assets/img/computed_probabilities_vs_theoretic_probabilities/main.pdf}
    \caption{The
        relationship between the steady state probabilities inferred from the
        measured transitions and the actual steady state probabilities. A linear
        regression line is included validating the approach.}
    \label{fig:computed_probabilities_vs_theoretic_probabilities}
\end{figure}


\end{document}

    strategies is considered. In this setting
    the most highly performing strategies do not play in an extortionate way
    against each other but do against lower performing strategies.
    This suggests that whilst the theory of Zero Determinant strategies
    indicates that memory is not of fundamental importance to the evolution of
    cooperative behaviour, this is incomplete.
\end{abstract}

\section{Introduction}\label{sec:introduction}

Agent based game theoretic models have become a stalwart of the underpinning
mathematics of interactive behaviours. One of the major pieces of work
in this area is the pair of original computer tournaments run by Robert
Axelrod~\cite{Axelrod1980, Axelrod1980a}. These tournaments pitted submitted
computer strategies against each other in plays of the Iterated Prisoner's
Dilemma. A common game where agents can choose to pay a slight cost to their
immediate utility in the hope of building a reputation. This has been used in
economic and evolutionary game theory to understand the evolution of cooperative
behaviour.

Recently, a class of strategies was described in~\cite{Press2012} that can
provably extort any given opponent. In~\cite{Hilbe2013, Moran1707} some
questions have already been asked about the true effectiveness of these
strategies in an evolutionary setting. Here another question is asked: is it
possible to recognise this extortionate behaviour? A mathematical procedure for
suspicion is presented: in the same way that the continued actions of an
extortionate individual might raise suspicion.

This work makes use of the Axelrod Python library~\cite{Knight2018, Knight2016}
with a large number of Prisoner Dilemma strategies available to give an
extensive numerical example of the ideas presented.  The approach is presented
in Section~\ref{sec:delta-zd-strategies}.  All of the code and data discussed
in Section~\ref{sec:numerical-experiments} is open sourced, archived and
written according to best scientific principles~\cite{Wilson2014}. The data
archive can be found at~\cite{vincent_knight_2018_1297075}.

\section{Recognising Extortion}\label{sec:delta-zd-strategies}

In~\cite{Press2012}, given a match between 2 memory-one strategies, the concept
of Zero Determinant (ZD) strategies is introduced. The main result of that paper
shows that given two memory one players \(p, q\in\mathbb{R}^4\) a linear
relationship between the players' scores could be forced by one of the players.

Using the notation of~\cite{Press2012}, assuming the utilities for player \(p\)
are given by \(S_x=(R, S, T, P)\) and for player \(q\) by \(S_y=(R, T, S, P)\)
and that the stationary scores of each player is given by \(S_X\) and \(S_Y\)
respectively. The main result of~\cite{Press2012} is that if

\begin{equation}\label{eqn:linear_relationship_for_p}
    \tilde p=\alpha S_x + \beta S_y + \gamma
\end{equation}

or

\begin{equation}\label{eqn:linear_relationship_for_q}
    \tilde q=\alpha S_x + \beta S_y + \gamma
\end{equation}

where \(\tilde p = (1 - p_1, 1 - p_2, p_3, p_4)\) and
\(\tilde q = (1 - q_1, 1 - q_2, q_3, q_4)\) then:

\begin{equation}
    \alpha S_X + \beta S_Y + \gamma = 0
\end{equation}

In~\cite{Press2012} a particular type of ZD strategy is defined: extortionate
strategies. If:

\begin{equation}\label{eqn:constraint_for_extortion}
    \gamma = - P(\alpha + \beta)
\end{equation}

then the player can ensure they get a score \(\chi\) times
larger than the opponent. This extortion coefficient is given by:

\begin{equation}\label{eqn:definition_of_chi}
    \chi=\frac{-\beta}{\alpha}
\end{equation}

Thus, if (\ref{eqn:constraint_for_extortion}) holds and \(\chi >1\) a player is
said to extort their opponent.
Here, the reverse problem is considered: given a
\(p\in\mathbb{R}^4\) how does one identify \(\alpha, \beta\) if they
exist and is the strategy in fact acting in an extortionate way?

These conditions correspond to:

\begin{align}
    \tilde p_1 & = \alpha R + \beta R - P (\alpha + \beta)
            \label{eqn:condition_for_tilde_p1}\\
    \tilde p_2 & = \alpha S + \beta T - P (\alpha + \beta)
            \label{eqn:condition_for_tilde_p2}\\
    \tilde p_3 & = \alpha T + \beta S - P (\alpha + \beta)
            \label{eqn:condition_for_tilde_p3}\\
    \tilde p_4 & = \alpha P + \beta P - P (\alpha + \beta)
            \label{eqn:condition_for_tilde_p4}
\end{align}

Equation (\ref{eqn:condition_for_tilde_p4}) ensures that \(p_4=\tilde p_4=0\).
Equations (\ref{eqn:condition_for_tilde_p1}-\ref{eqn:condition_for_tilde_p3})
can be used to eliminate \(\alpha, \beta\), giving:

\begin{equation}\label{eqn:planar_definition_of_extortion}
    \tilde p_1 = \frac{(R - P)(\tilde p_2 + \tilde p_3)}{S + T - 2P}
\end{equation}

with:

\begin{equation}\label{eqn:definition_of_chi}
    \chi = \frac{\tilde p_2 (P - T) + \tilde p_3 (S - P)}
                {\tilde p_2 (P - S) + \tilde p_3 (T - P)}
\end{equation}

Given a strategy \(p\in\mathbb{R}^{4\times 1}\) equations
(\ref{eqn:condition_for_tilde_p4}), (\ref{eqn:planar_definition_of_extortion}-\ref{eqn:definition_of_chi}) can be used to check if
a strategy is extortionate. The conditions correspond to:

\begin{align}
    p_1 & = \frac{(R-P)(p_2 + p_3) - R + T + S - P}{S + T - 2P}
     \label{eqn:condition_for_p1}\\
    p_4 & = 0 \label{eqn:condition_for_p4}\\
    1 & > p_2 + p_3\label{eqn:condition_for_chi}
\end{align}

The algebraic steps necessary to prove these results are available in the
supporting materials.

All extortionate strategies reside on a triangular (\ref{eqn:condition_for_chi})
plane (\ref{eqn:condition_for_p1}) in 3 dimensions (\ref{eqn:condition_for_p4}).
Using this formulation it can be seen that a necessary (but not sufficient)
condition for an extortionate strategy is that it cooperates on average less
than 50\% of the time when in a state of disagreement with the opponent.

As an example, consider the known extortionate strategy \(p=(8 / 9, 1 / 2, 1 /
3, 0)\) from~\cite{Stewart2012} which is referred to as \texttt{Extort-2}. In
this case, for the standard values of \((R, T, S, P)\) constraint
(\ref{eqn:condition_for_p1}) corresponds to:

\begin{equation}
    p_1 = \frac{2(p_2 + p_3) + 1}{3}
\end{equation}

It is clear that in this case all constraints hold.

This approach could in fact be used to confirm that a given strategy is acting
in an extortionate manner even if it is not a memory one strategy. However, in
practice, if a closed form for \(p\) is not known, then due to measurement
and/or numerical error this would not work.

This problem can be written in the following linear algebraic form where
\(x=(\alpha, \beta)\)
and \(p^*=(\tilde p_1 - 1, tilde_2 - 1, p_3)\):

\begin{equation}\label{eqn:linear_algebraic_equation_for_p}
    Cx= p^*
\end{equation}

\(C\) corresponds to equations
(\ref{eqn:condition_for_tilde_p1}-\ref{eqn:condition_for_tilde_p3}) and is
given by:

\begin{equation}\label{eqn:definition_of_C}
    C =
    \begin{bmatrix}
        R - P & R- P \\
        S - P & T- P \\
        T - P & S- P \\
    \end{bmatrix}
\end{equation}

Note that in general, equation (\ref{eqn:linear_algebraic_equation_for_p}) will
not necessarily have a solution. From the Rouch\'{e}-Capelli theorem if there is
a solution it is unique as \(\text{rank}(C)=2\) which is the dimension of the
variable \(x\). The best fitting \(x\) is found by minimizing:

\begin{equation}\label{eqn:r_squared}
    \text{SSError} = \|C x- p^*\|_2^2 = \sum_{i=1}^{3}\left((C\bar x)_i-p_i^*\right)^2
\end{equation}

Note that \(\text{SSError}\), which is the square of the Frobenius
norm~\cite{Golub2013}, becomes a measure of how close a strategy is to being an
extortionate strategy. Suspicion
of extortion then corresponds to a threshold on \(\text{SSError}\).

By observing interactions (human or otherwise), their memory one representation
can be inferred and this approach can be used to recognise extortionate
behaviour. The notion of comparing theoretic and actual plays of the IPD is not
novel, see for example~\cite{Rand2013}. Immediately it is noted that if the
environment is noisy~\cite{Wu1995} then no strategy can be considered to be
extortionate as \(p_4>0\).

In the next section, this idea will be illustrated by observing the interactions
that take place in a computer based tournament of the IPD\@.

\section{Numerical experiments}\label{sec:numerical-experiments}

In~\cite{Stewart2012} results from a tournament with
\documentclass[a4paper]{article}

\usepackage{amsmath}
\usepackage{amssymb}
\usepackage[margin=1.5cm,
            includefoot,
            footskip=30pt]{geometry}
\usepackage{layout}
\usepackage{graphicx}
\usepackage{subcaption}

\usepackage{biblatex}
\usepackage{pdfpages}

\bibliography{main.bib}

\title{Suspicion: Recognising and evaluating the effectiveness
       of extortion in the Iterated Prisoner's Dilemma}
\author{Vincent A. Knight \and Nikoleta E. Glynatsi}
\date{\today}



\begin{document}

\maketitle

\begin{abstract}
    The Iterated Prisoner's Dilemma is a model for rational and evolutionary
    interactive behaviour. It has applications both in the study of human social
    behaviour as well as in biology.
    It is used to understand when and how a rational individual might
    accept an immediate cost to their own utility for the direct benefit of
    another.

    Much attention has been given to a class of strategies called
    Zero Determinant strategies. It has been theoretically shown that these
    strategies can ``extort'' any player.

    In this work, an approach to identify if observed strategies are playing in
    an extortionate way is described. Furthermore, experimental analysis of
    a large tournament with \input{assets/tex/number_of_full_strategies/main.tex}
    strategies is considered. In this setting
    the most highly performing strategies do not play in an extortionate way
    against each other but do against lower performing strategies.
    This suggests that whilst the theory of Zero Determinant strategies
    indicates that memory is not of fundamental importance to the evolution of
    cooperative behaviour, this is incomplete.
\end{abstract}

\section{Introduction}\label{sec:introduction}

Agent based game theoretic models have become a stalwart of the underpinning
mathematics of interactive behaviours. One of the major pieces of work
in this area is the pair of original computer tournaments run by Robert
Axelrod~\cite{Axelrod1980, Axelrod1980a}. These tournaments pitted submitted
computer strategies against each other in plays of the Iterated Prisoner's
Dilemma. A common game where agents can choose to pay a slight cost to their
immediate utility in the hope of building a reputation. This has been used in
economic and evolutionary game theory to understand the evolution of cooperative
behaviour.

Recently, a class of strategies was described in~\cite{Press2012} that can
provably extort any given opponent. In~\cite{Hilbe2013, Moran1707} some
questions have already been asked about the true effectiveness of these
strategies in an evolutionary setting. Here another question is asked: is it
possible to recognise this extortionate behaviour? A mathematical procedure for
suspicion is presented: in the same way that the continued actions of an
extortionate individual might raise suspicion.

This work makes use of the Axelrod Python library~\cite{Knight2018, Knight2016}
with a large number of Prisoner Dilemma strategies available to give an
extensive numerical example of the ideas presented.  The approach is presented
in Section~\ref{sec:delta-zd-strategies}.  All of the code and data discussed
in Section~\ref{sec:numerical-experiments} is open sourced, archived and
written according to best scientific principles~\cite{Wilson2014}. The data
archive can be found at~\cite{vincent_knight_2018_1297075}.

\section{Recognising Extortion}\label{sec:delta-zd-strategies}

In~\cite{Press2012}, given a match between 2 memory-one strategies, the concept
of Zero Determinant (ZD) strategies is introduced. The main result of that paper
shows that given two memory one players \(p, q\in\mathbb{R}^4\) a linear
relationship between the players' scores could be forced by one of the players.

Using the notation of~\cite{Press2012}, assuming the utilities for player \(p\)
are given by \(S_x=(R, S, T, P)\) and for player \(q\) by \(S_y=(R, T, S, P)\)
and that the stationary scores of each player is given by \(S_X\) and \(S_Y\)
respectively. The main result of~\cite{Press2012} is that if

\begin{equation}\label{eqn:linear_relationship_for_p}
    \tilde p=\alpha S_x + \beta S_y + \gamma
\end{equation}

or

\begin{equation}\label{eqn:linear_relationship_for_q}
    \tilde q=\alpha S_x + \beta S_y + \gamma
\end{equation}

where \(\tilde p = (1 - p_1, 1 - p_2, p_3, p_4)\) and
\(\tilde q = (1 - q_1, 1 - q_2, q_3, q_4)\) then:

\begin{equation}
    \alpha S_X + \beta S_Y + \gamma = 0
\end{equation}

In~\cite{Press2012} a particular type of ZD strategy is defined: extortionate
strategies. If:

\begin{equation}\label{eqn:constraint_for_extortion}
    \gamma = - P(\alpha + \beta)
\end{equation}

then the player can ensure they get a score \(\chi\) times
larger than the opponent. This extortion coefficient is given by:

\begin{equation}\label{eqn:definition_of_chi}
    \chi=\frac{-\beta}{\alpha}
\end{equation}

Thus, if (\ref{eqn:constraint_for_extortion}) holds and \(\chi >1\) a player is
said to extort their opponent.
Here, the reverse problem is considered: given a
\(p\in\mathbb{R}^4\) how does one identify \(\alpha, \beta\) if they
exist and is the strategy in fact acting in an extortionate way?

These conditions correspond to:

\begin{align}
    \tilde p_1 & = \alpha R + \beta R - P (\alpha + \beta)
            \label{eqn:condition_for_tilde_p1}\\
    \tilde p_2 & = \alpha S + \beta T - P (\alpha + \beta)
            \label{eqn:condition_for_tilde_p2}\\
    \tilde p_3 & = \alpha T + \beta S - P (\alpha + \beta)
            \label{eqn:condition_for_tilde_p3}\\
    \tilde p_4 & = \alpha P + \beta P - P (\alpha + \beta)
            \label{eqn:condition_for_tilde_p4}
\end{align}

Equation (\ref{eqn:condition_for_tilde_p4}) ensures that \(p_4=\tilde p_4=0\).
Equations (\ref{eqn:condition_for_tilde_p1}-\ref{eqn:condition_for_tilde_p3})
can be used to eliminate \(\alpha, \beta\), giving:

\begin{equation}\label{eqn:planar_definition_of_extortion}
    \tilde p_1 = \frac{(R - P)(\tilde p_2 + \tilde p_3)}{S + T - 2P}
\end{equation}

with:

\begin{equation}\label{eqn:definition_of_chi}
    \chi = \frac{\tilde p_2 (P - T) + \tilde p_3 (S - P)}
                {\tilde p_2 (P - S) + \tilde p_3 (T - P)}
\end{equation}

Given a strategy \(p\in\mathbb{R}^{4\times 1}\) equations
(\ref{eqn:condition_for_tilde_p4}), (\ref{eqn:planar_definition_of_extortion}-\ref{eqn:definition_of_chi}) can be used to check if
a strategy is extortionate. The conditions correspond to:

\begin{align}
    p_1 & = \frac{(R-P)(p_2 + p_3) - R + T + S - P}{S + T - 2P}
     \label{eqn:condition_for_p1}\\
    p_4 & = 0 \label{eqn:condition_for_p4}\\
    1 & > p_2 + p_3\label{eqn:condition_for_chi}
\end{align}

The algebraic steps necessary to prove these results are available in the
supporting materials.

All extortionate strategies reside on a triangular (\ref{eqn:condition_for_chi})
plane (\ref{eqn:condition_for_p1}) in 3 dimensions (\ref{eqn:condition_for_p4}).
Using this formulation it can be seen that a necessary (but not sufficient)
condition for an extortionate strategy is that it cooperates on average less
than 50\% of the time when in a state of disagreement with the opponent.

As an example, consider the known extortionate strategy \(p=(8 / 9, 1 / 2, 1 /
3, 0)\) from~\cite{Stewart2012} which is referred to as \texttt{Extort-2}. In
this case, for the standard values of \((R, T, S, P)\) constraint
(\ref{eqn:condition_for_p1}) corresponds to:

\begin{equation}
    p_1 = \frac{2(p_2 + p_3) + 1}{3}
\end{equation}

It is clear that in this case all constraints hold.

This approach could in fact be used to confirm that a given strategy is acting
in an extortionate manner even if it is not a memory one strategy. However, in
practice, if a closed form for \(p\) is not known, then due to measurement
and/or numerical error this would not work.

This problem can be written in the following linear algebraic form where
\(x=(\alpha, \beta)\)
and \(p^*=(\tilde p_1 - 1, tilde_2 - 1, p_3)\):

\begin{equation}\label{eqn:linear_algebraic_equation_for_p}
    Cx= p^*
\end{equation}

\(C\) corresponds to equations
(\ref{eqn:condition_for_tilde_p1}-\ref{eqn:condition_for_tilde_p3}) and is
given by:

\begin{equation}\label{eqn:definition_of_C}
    C =
    \begin{bmatrix}
        R - P & R- P \\
        S - P & T- P \\
        T - P & S- P \\
    \end{bmatrix}
\end{equation}

Note that in general, equation (\ref{eqn:linear_algebraic_equation_for_p}) will
not necessarily have a solution. From the Rouch\'{e}-Capelli theorem if there is
a solution it is unique as \(\text{rank}(C)=2\) which is the dimension of the
variable \(x\). The best fitting \(x\) is found by minimizing:

\begin{equation}\label{eqn:r_squared}
    \text{SSError} = \|C x- p^*\|_2^2 = \sum_{i=1}^{3}\left((C\bar x)_i-p_i^*\right)^2
\end{equation}

Note that \(\text{SSError}\), which is the square of the Frobenius
norm~\cite{Golub2013}, becomes a measure of how close a strategy is to being an
extortionate strategy. Suspicion
of extortion then corresponds to a threshold on \(\text{SSError}\).

By observing interactions (human or otherwise), their memory one representation
can be inferred and this approach can be used to recognise extortionate
behaviour. The notion of comparing theoretic and actual plays of the IPD is not
novel, see for example~\cite{Rand2013}. Immediately it is noted that if the
environment is noisy~\cite{Wu1995} then no strategy can be considered to be
extortionate as \(p_4>0\).

In the next section, this idea will be illustrated by observing the interactions
that take place in a computer based tournament of the IPD\@.

\section{Numerical experiments}\label{sec:numerical-experiments}

In~\cite{Stewart2012} results from a tournament with
\input{./assets/tex/number_of_stewart_plotkin_strategies/main.tex} strategies,
was presented with specific consideration given to ZD strategies. This
tournament is reproduced here using the Axelrod-Python
project~\cite{Knight2016}. To obtain a good measure of the corresponding
transition rates for each strategy all matches have been run for
\input{assets/tex/number_of_turns/main.tex} turns and every match has been
repeated \input{assets/tex/number_of_repetitions/main.tex} times. All of this
interaction data is available at~\cite{vincent_knight_2018_1297075}. A good
match between the inferred Markov chain and the state distribution of the actual
interactions has been verified. Data for this is presented in the supplementary
materials.

Figure~\ref{fig:SSError_overall_in_stewart_plotkin} shows the \(\text{SSError}\)
values for all the strategies in the tournament, as reported
in~\cite{Stewart2012} the extortionate strategy (which has an expected
\(\text{SSError}\) approximately 0) gains a large number of wins.

\begin{figure}[!htbp]
    \centering
    \includegraphics[width=.8\textwidth]{./assets/img/SSError_overall_in_stewart_plotkin/main.pdf}
    \caption{\(\text{SSError}\) and state probabilities for the strategies
        of~\cite{Stewart2012}, ordered both by number of wins and overall score.
        Note that \(P(DC)\) is not shown as it corresponds to the transpose of
        \(P(CD)\). Cooperator and Defector are omitted as they do not visit all
        the states.}
    \label{fig:SSError_overall_in_stewart_plotkin}
\end{figure}

Here, the work of~\cite{Stewart2012} is extended by investigating a tournament
with \input{assets/tex/number_of_full_strategies/main.tex}
strategies.

The results of this analysis are shown in
Figure~\ref{fig:SSError_and_probabilities_in_full}. The top ranking strategies
by number of wins seem to be extortionate (but not against all strategies) and
it can be seen that a small sub group of strategies achieve mutual defection.
All the top ranking strategies according to score achieve mutual cooperation and
do not extort each other, however they
\textbf{do} exhibit extortionate behaviour towards a number of the lower ranking
strategies.

\begin{figure}[!htbp]
    \centering
    \includegraphics[width=.8\textwidth]{./assets/img/SSError_and_probabilities_in_full/main.pdf}
    \caption{\(\text{SSError}\) for the strategies for the full tournament. Only
    strategy interactions for which \(p_4=0\) and \(\chi>1\) are displayed.}
    \label{fig:SSError_and_probabilities_in_full}
\end{figure}

\section{Conclusion}\label{sec:conclusion}

This work defines an approach to measure whether or not a player is playing a
strategy that corresponds to an extortionate strategy as defined
in~\cite{Press2012}: a mathematical model for suspicion. Indeed, all
extortionate strategies have been
 classified as lying on a triangular plane.
This rigorous classification fails to be robust to small measurement error, thus
a statistical approach is proposed.
This is done through a linear algebraic approach for approximating the solution
of a linear system. Using this, a large number of pairwise interactions is
simulated and in fact very few strategies are found to act extortionately.

The work of~\cite{Press2012}, whilst showing that a clever approach to taking
advantage of another memory one strategy exists: this is incomplete. Whilst the
elegance of this result is very attractive, just as the simplicity of the
victory of Tit For Tat in Axelrod's original tournaments was, it is incomplete.
Extortionate strategies achieve a high number of wins but they do not
achieve a high score which corresponds to the fitness landscape in an
evolutionary sense. From the large number of interactions a payoff matrix \(S\)
can be measured where \(S_{ij}\) denotes the score (using standard values of
\((R, S, T, P) = (3, 0, 5, 1)\)) of the \(i\)th strategy
against the \(j\)th strategy. Using this, the replicator equation
describes the evolution of the system based on a population density fitness
function:

\begin{equation}\label{eqn:replicator_dynamics}
    \frac{dx}{dt} = x(S-x^TS x)
\end{equation}

Equation (\ref{eqn:replicator_dynamics}) is solved numerically through an
integration technique described in~\cite{Petzold1983} and
Figure~\ref{fig:replicator_dynamics} shows the evolution of the distribution of
the system: the various strategies are ranked by scores. It is clear to see that
only the high ranking strategies survive the evolutionary process (in fact,
only \input{./assets/img/replicator_dynamics/main.tex}
have a final distribution greater than \(10 ^ {-2}\)). This confirms the
findings of~\cite{Moran1707} in which sophisticated strategies resist
evolutionary invasion of shorter memory strategies. Recalling
Figure~\ref{fig:SSError_and_probabilities_in_full} this demonstrates that:

\begin{itemize}
    \item Cooperation emerges through the evolutionary process: the high scoring
        strategies do not exhibit extortionate behaviour towards each other.
    \item Extortionate strategies do not survive the evolutionary process.
\end{itemize}

\begin{figure}[!htbp]
    \centering
    \includegraphics[width=.8\textwidth]{./assets/img/replicator_dynamics/main.pdf}
    \caption{Numerical simulation of the replicator equation
    (\ref{eqn:replicator_dynamics}): strategies are ordered by score, only the strategies with a high score survive the evolutionary process.}
    \label{fig:replicator_dynamics}
\end{figure}

This work can be used to classify plays of the IPD\@: data can be collected from
actual interactions (in lab or in the field). Furthermore, this allows for a
classification method similar to the notion of fingerprinting presented
in~\cite{Ashlock2008}. Trained strategies can potentially be classified as
extortionate or not or it could be possible to even constrain the reinforcement
learning approaches that are becoming prevalent in the literature.
Alternatively, this mathematical approach for recognising extortion could be
used in sophisticated strategies to defend against invasion. Arguably, some of
the strategies considered here exhibit this behaviour, indeed as described
in~\cite{Harper2017}, the top ranking strategies in the full tournament are
obtained using evolutionary reinforcement learning techniques, thus, suspicion
of extortionate behaviour could in fact be an evolutionary trait.

\section*{Acknowledgements}

The following open source software libraries were used in this research:

\begin{itemize}
    \item The Axelrod ~\cite{Knight2016, Knight2018} library (IPD strategies and
        tournaments).
    \item The sympy library~\cite{Meurer2017} (verification of all symbolic
        calculations).
    \item The matplotlib~\cite{Droettboom2018} library (visualisation).
    \item The pandas~\cite{Structures2010}, dask~\cite{Dask2016} and
        NumPy~\cite{Oliphant2015} libraries (data manipulation).
    \item The SciPy~\cite{Jones2001} library (numerical integration of the
        replicator equation).
\end{itemize}

This work was performed using the computational facilities of the Advanced
Research Computing @ Cardiff (ARCCA) Division, Cardiff University.

\printbibliography

\newpage
\section*{Supplementary materials}

\includepdf{assets/pdf/proof_of_form_of_extortionate_strategies/main.pdf}

\newpage

Using the pair wise interactions the transition rates \(p,
q\) can be measured and the steady state probabilities inferred and compared to
the actual probabilities of each state.
This is done numerically by computing the singular eigenvector of the
matrix \(A\) \cite{Stewart2009}:

\[
    A =
    \begin{bmatrix}
        p_1 q_1 & p_1 (1 - q_1) & (1 - p_1) q_1 & (1 -p_1) (1 - q_1) \\
        p_2 q_2 & p_2 (1 - q_2) & (1 - p_2) q_2 & (1 -p_2) (1 - q_2) \\
        p_3 q_3 & p_3 (1 - q_3) & (1 - p_3) q_3 & (1 -p_3) (1 - q_3) \\
        p_4 q_4 & p_4 (1 - q_4) & (1 - p_4) q_4 & (1 -p_4) (1 - q_4) \\
    \end{bmatrix}
\]

Figure~\ref{fig:computed_probabilities_vs_theoretic_probabilities} shows a
regression line fitted to every pairwise interaction with a reported
\(\text{SSError}\) value (pairwise interactions with missing states were
omitted). This serves to validate the approach: a part from some edge cases the
relationship is consistent.

\begin{figure}[!htbp]
    \centering
    \includegraphics[width=.8\textwidth]{./assets/img/computed_probabilities_vs_theoretic_probabilities/main.pdf}
    \caption{The
        relationship between the steady state probabilities inferred from the
        measured transitions and the actual steady state probabilities. A linear
        regression line is included validating the approach.}
    \label{fig:computed_probabilities_vs_theoretic_probabilities}
\end{figure}


\end{document}
 strategies,
was presented with specific consideration given to ZD strategies. This
tournament is reproduced here using the Axelrod-Python
project~\cite{Knight2016}. To obtain a good measure of the corresponding
transition rates for each strategy all matches have been run for
\documentclass[a4paper]{article}

\usepackage{amsmath}
\usepackage{amssymb}
\usepackage[margin=1.5cm,
            includefoot,
            footskip=30pt]{geometry}
\usepackage{layout}
\usepackage{graphicx}
\usepackage{subcaption}

\usepackage{biblatex}
\usepackage{pdfpages}

\bibliography{main.bib}

\title{Suspicion: Recognising and evaluating the effectiveness
       of extortion in the Iterated Prisoner's Dilemma}
\author{Vincent A. Knight \and Nikoleta E. Glynatsi}
\date{\today}



\begin{document}

\maketitle

\begin{abstract}
    The Iterated Prisoner's Dilemma is a model for rational and evolutionary
    interactive behaviour. It has applications both in the study of human social
    behaviour as well as in biology.
    It is used to understand when and how a rational individual might
    accept an immediate cost to their own utility for the direct benefit of
    another.

    Much attention has been given to a class of strategies called
    Zero Determinant strategies. It has been theoretically shown that these
    strategies can ``extort'' any player.

    In this work, an approach to identify if observed strategies are playing in
    an extortionate way is described. Furthermore, experimental analysis of
    a large tournament with \input{assets/tex/number_of_full_strategies/main.tex}
    strategies is considered. In this setting
    the most highly performing strategies do not play in an extortionate way
    against each other but do against lower performing strategies.
    This suggests that whilst the theory of Zero Determinant strategies
    indicates that memory is not of fundamental importance to the evolution of
    cooperative behaviour, this is incomplete.
\end{abstract}

\section{Introduction}\label{sec:introduction}

Agent based game theoretic models have become a stalwart of the underpinning
mathematics of interactive behaviours. One of the major pieces of work
in this area is the pair of original computer tournaments run by Robert
Axelrod~\cite{Axelrod1980, Axelrod1980a}. These tournaments pitted submitted
computer strategies against each other in plays of the Iterated Prisoner's
Dilemma. A common game where agents can choose to pay a slight cost to their
immediate utility in the hope of building a reputation. This has been used in
economic and evolutionary game theory to understand the evolution of cooperative
behaviour.

Recently, a class of strategies was described in~\cite{Press2012} that can
provably extort any given opponent. In~\cite{Hilbe2013, Moran1707} some
questions have already been asked about the true effectiveness of these
strategies in an evolutionary setting. Here another question is asked: is it
possible to recognise this extortionate behaviour? A mathematical procedure for
suspicion is presented: in the same way that the continued actions of an
extortionate individual might raise suspicion.

This work makes use of the Axelrod Python library~\cite{Knight2018, Knight2016}
with a large number of Prisoner Dilemma strategies available to give an
extensive numerical example of the ideas presented.  The approach is presented
in Section~\ref{sec:delta-zd-strategies}.  All of the code and data discussed
in Section~\ref{sec:numerical-experiments} is open sourced, archived and
written according to best scientific principles~\cite{Wilson2014}. The data
archive can be found at~\cite{vincent_knight_2018_1297075}.

\section{Recognising Extortion}\label{sec:delta-zd-strategies}

In~\cite{Press2012}, given a match between 2 memory-one strategies, the concept
of Zero Determinant (ZD) strategies is introduced. The main result of that paper
shows that given two memory one players \(p, q\in\mathbb{R}^4\) a linear
relationship between the players' scores could be forced by one of the players.

Using the notation of~\cite{Press2012}, assuming the utilities for player \(p\)
are given by \(S_x=(R, S, T, P)\) and for player \(q\) by \(S_y=(R, T, S, P)\)
and that the stationary scores of each player is given by \(S_X\) and \(S_Y\)
respectively. The main result of~\cite{Press2012} is that if

\begin{equation}\label{eqn:linear_relationship_for_p}
    \tilde p=\alpha S_x + \beta S_y + \gamma
\end{equation}

or

\begin{equation}\label{eqn:linear_relationship_for_q}
    \tilde q=\alpha S_x + \beta S_y + \gamma
\end{equation}

where \(\tilde p = (1 - p_1, 1 - p_2, p_3, p_4)\) and
\(\tilde q = (1 - q_1, 1 - q_2, q_3, q_4)\) then:

\begin{equation}
    \alpha S_X + \beta S_Y + \gamma = 0
\end{equation}

In~\cite{Press2012} a particular type of ZD strategy is defined: extortionate
strategies. If:

\begin{equation}\label{eqn:constraint_for_extortion}
    \gamma = - P(\alpha + \beta)
\end{equation}

then the player can ensure they get a score \(\chi\) times
larger than the opponent. This extortion coefficient is given by:

\begin{equation}\label{eqn:definition_of_chi}
    \chi=\frac{-\beta}{\alpha}
\end{equation}

Thus, if (\ref{eqn:constraint_for_extortion}) holds and \(\chi >1\) a player is
said to extort their opponent.
Here, the reverse problem is considered: given a
\(p\in\mathbb{R}^4\) how does one identify \(\alpha, \beta\) if they
exist and is the strategy in fact acting in an extortionate way?

These conditions correspond to:

\begin{align}
    \tilde p_1 & = \alpha R + \beta R - P (\alpha + \beta)
            \label{eqn:condition_for_tilde_p1}\\
    \tilde p_2 & = \alpha S + \beta T - P (\alpha + \beta)
            \label{eqn:condition_for_tilde_p2}\\
    \tilde p_3 & = \alpha T + \beta S - P (\alpha + \beta)
            \label{eqn:condition_for_tilde_p3}\\
    \tilde p_4 & = \alpha P + \beta P - P (\alpha + \beta)
            \label{eqn:condition_for_tilde_p4}
\end{align}

Equation (\ref{eqn:condition_for_tilde_p4}) ensures that \(p_4=\tilde p_4=0\).
Equations (\ref{eqn:condition_for_tilde_p1}-\ref{eqn:condition_for_tilde_p3})
can be used to eliminate \(\alpha, \beta\), giving:

\begin{equation}\label{eqn:planar_definition_of_extortion}
    \tilde p_1 = \frac{(R - P)(\tilde p_2 + \tilde p_3)}{S + T - 2P}
\end{equation}

with:

\begin{equation}\label{eqn:definition_of_chi}
    \chi = \frac{\tilde p_2 (P - T) + \tilde p_3 (S - P)}
                {\tilde p_2 (P - S) + \tilde p_3 (T - P)}
\end{equation}

Given a strategy \(p\in\mathbb{R}^{4\times 1}\) equations
(\ref{eqn:condition_for_tilde_p4}), (\ref{eqn:planar_definition_of_extortion}-\ref{eqn:definition_of_chi}) can be used to check if
a strategy is extortionate. The conditions correspond to:

\begin{align}
    p_1 & = \frac{(R-P)(p_2 + p_3) - R + T + S - P}{S + T - 2P}
     \label{eqn:condition_for_p1}\\
    p_4 & = 0 \label{eqn:condition_for_p4}\\
    1 & > p_2 + p_3\label{eqn:condition_for_chi}
\end{align}

The algebraic steps necessary to prove these results are available in the
supporting materials.

All extortionate strategies reside on a triangular (\ref{eqn:condition_for_chi})
plane (\ref{eqn:condition_for_p1}) in 3 dimensions (\ref{eqn:condition_for_p4}).
Using this formulation it can be seen that a necessary (but not sufficient)
condition for an extortionate strategy is that it cooperates on average less
than 50\% of the time when in a state of disagreement with the opponent.

As an example, consider the known extortionate strategy \(p=(8 / 9, 1 / 2, 1 /
3, 0)\) from~\cite{Stewart2012} which is referred to as \texttt{Extort-2}. In
this case, for the standard values of \((R, T, S, P)\) constraint
(\ref{eqn:condition_for_p1}) corresponds to:

\begin{equation}
    p_1 = \frac{2(p_2 + p_3) + 1}{3}
\end{equation}

It is clear that in this case all constraints hold.

This approach could in fact be used to confirm that a given strategy is acting
in an extortionate manner even if it is not a memory one strategy. However, in
practice, if a closed form for \(p\) is not known, then due to measurement
and/or numerical error this would not work.

This problem can be written in the following linear algebraic form where
\(x=(\alpha, \beta)\)
and \(p^*=(\tilde p_1 - 1, tilde_2 - 1, p_3)\):

\begin{equation}\label{eqn:linear_algebraic_equation_for_p}
    Cx= p^*
\end{equation}

\(C\) corresponds to equations
(\ref{eqn:condition_for_tilde_p1}-\ref{eqn:condition_for_tilde_p3}) and is
given by:

\begin{equation}\label{eqn:definition_of_C}
    C =
    \begin{bmatrix}
        R - P & R- P \\
        S - P & T- P \\
        T - P & S- P \\
    \end{bmatrix}
\end{equation}

Note that in general, equation (\ref{eqn:linear_algebraic_equation_for_p}) will
not necessarily have a solution. From the Rouch\'{e}-Capelli theorem if there is
a solution it is unique as \(\text{rank}(C)=2\) which is the dimension of the
variable \(x\). The best fitting \(x\) is found by minimizing:

\begin{equation}\label{eqn:r_squared}
    \text{SSError} = \|C x- p^*\|_2^2 = \sum_{i=1}^{3}\left((C\bar x)_i-p_i^*\right)^2
\end{equation}

Note that \(\text{SSError}\), which is the square of the Frobenius
norm~\cite{Golub2013}, becomes a measure of how close a strategy is to being an
extortionate strategy. Suspicion
of extortion then corresponds to a threshold on \(\text{SSError}\).

By observing interactions (human or otherwise), their memory one representation
can be inferred and this approach can be used to recognise extortionate
behaviour. The notion of comparing theoretic and actual plays of the IPD is not
novel, see for example~\cite{Rand2013}. Immediately it is noted that if the
environment is noisy~\cite{Wu1995} then no strategy can be considered to be
extortionate as \(p_4>0\).

In the next section, this idea will be illustrated by observing the interactions
that take place in a computer based tournament of the IPD\@.

\section{Numerical experiments}\label{sec:numerical-experiments}

In~\cite{Stewart2012} results from a tournament with
\input{./assets/tex/number_of_stewart_plotkin_strategies/main.tex} strategies,
was presented with specific consideration given to ZD strategies. This
tournament is reproduced here using the Axelrod-Python
project~\cite{Knight2016}. To obtain a good measure of the corresponding
transition rates for each strategy all matches have been run for
\input{assets/tex/number_of_turns/main.tex} turns and every match has been
repeated \input{assets/tex/number_of_repetitions/main.tex} times. All of this
interaction data is available at~\cite{vincent_knight_2018_1297075}. A good
match between the inferred Markov chain and the state distribution of the actual
interactions has been verified. Data for this is presented in the supplementary
materials.

Figure~\ref{fig:SSError_overall_in_stewart_plotkin} shows the \(\text{SSError}\)
values for all the strategies in the tournament, as reported
in~\cite{Stewart2012} the extortionate strategy (which has an expected
\(\text{SSError}\) approximately 0) gains a large number of wins.

\begin{figure}[!htbp]
    \centering
    \includegraphics[width=.8\textwidth]{./assets/img/SSError_overall_in_stewart_plotkin/main.pdf}
    \caption{\(\text{SSError}\) and state probabilities for the strategies
        of~\cite{Stewart2012}, ordered both by number of wins and overall score.
        Note that \(P(DC)\) is not shown as it corresponds to the transpose of
        \(P(CD)\). Cooperator and Defector are omitted as they do not visit all
        the states.}
    \label{fig:SSError_overall_in_stewart_plotkin}
\end{figure}

Here, the work of~\cite{Stewart2012} is extended by investigating a tournament
with \input{assets/tex/number_of_full_strategies/main.tex}
strategies.

The results of this analysis are shown in
Figure~\ref{fig:SSError_and_probabilities_in_full}. The top ranking strategies
by number of wins seem to be extortionate (but not against all strategies) and
it can be seen that a small sub group of strategies achieve mutual defection.
All the top ranking strategies according to score achieve mutual cooperation and
do not extort each other, however they
\textbf{do} exhibit extortionate behaviour towards a number of the lower ranking
strategies.

\begin{figure}[!htbp]
    \centering
    \includegraphics[width=.8\textwidth]{./assets/img/SSError_and_probabilities_in_full/main.pdf}
    \caption{\(\text{SSError}\) for the strategies for the full tournament. Only
    strategy interactions for which \(p_4=0\) and \(\chi>1\) are displayed.}
    \label{fig:SSError_and_probabilities_in_full}
\end{figure}

\section{Conclusion}\label{sec:conclusion}

This work defines an approach to measure whether or not a player is playing a
strategy that corresponds to an extortionate strategy as defined
in~\cite{Press2012}: a mathematical model for suspicion. Indeed, all
extortionate strategies have been
 classified as lying on a triangular plane.
This rigorous classification fails to be robust to small measurement error, thus
a statistical approach is proposed.
This is done through a linear algebraic approach for approximating the solution
of a linear system. Using this, a large number of pairwise interactions is
simulated and in fact very few strategies are found to act extortionately.

The work of~\cite{Press2012}, whilst showing that a clever approach to taking
advantage of another memory one strategy exists: this is incomplete. Whilst the
elegance of this result is very attractive, just as the simplicity of the
victory of Tit For Tat in Axelrod's original tournaments was, it is incomplete.
Extortionate strategies achieve a high number of wins but they do not
achieve a high score which corresponds to the fitness landscape in an
evolutionary sense. From the large number of interactions a payoff matrix \(S\)
can be measured where \(S_{ij}\) denotes the score (using standard values of
\((R, S, T, P) = (3, 0, 5, 1)\)) of the \(i\)th strategy
against the \(j\)th strategy. Using this, the replicator equation
describes the evolution of the system based on a population density fitness
function:

\begin{equation}\label{eqn:replicator_dynamics}
    \frac{dx}{dt} = x(S-x^TS x)
\end{equation}

Equation (\ref{eqn:replicator_dynamics}) is solved numerically through an
integration technique described in~\cite{Petzold1983} and
Figure~\ref{fig:replicator_dynamics} shows the evolution of the distribution of
the system: the various strategies are ranked by scores. It is clear to see that
only the high ranking strategies survive the evolutionary process (in fact,
only \input{./assets/img/replicator_dynamics/main.tex}
have a final distribution greater than \(10 ^ {-2}\)). This confirms the
findings of~\cite{Moran1707} in which sophisticated strategies resist
evolutionary invasion of shorter memory strategies. Recalling
Figure~\ref{fig:SSError_and_probabilities_in_full} this demonstrates that:

\begin{itemize}
    \item Cooperation emerges through the evolutionary process: the high scoring
        strategies do not exhibit extortionate behaviour towards each other.
    \item Extortionate strategies do not survive the evolutionary process.
\end{itemize}

\begin{figure}[!htbp]
    \centering
    \includegraphics[width=.8\textwidth]{./assets/img/replicator_dynamics/main.pdf}
    \caption{Numerical simulation of the replicator equation
    (\ref{eqn:replicator_dynamics}): strategies are ordered by score, only the strategies with a high score survive the evolutionary process.}
    \label{fig:replicator_dynamics}
\end{figure}

This work can be used to classify plays of the IPD\@: data can be collected from
actual interactions (in lab or in the field). Furthermore, this allows for a
classification method similar to the notion of fingerprinting presented
in~\cite{Ashlock2008}. Trained strategies can potentially be classified as
extortionate or not or it could be possible to even constrain the reinforcement
learning approaches that are becoming prevalent in the literature.
Alternatively, this mathematical approach for recognising extortion could be
used in sophisticated strategies to defend against invasion. Arguably, some of
the strategies considered here exhibit this behaviour, indeed as described
in~\cite{Harper2017}, the top ranking strategies in the full tournament are
obtained using evolutionary reinforcement learning techniques, thus, suspicion
of extortionate behaviour could in fact be an evolutionary trait.

\section*{Acknowledgements}

The following open source software libraries were used in this research:

\begin{itemize}
    \item The Axelrod ~\cite{Knight2016, Knight2018} library (IPD strategies and
        tournaments).
    \item The sympy library~\cite{Meurer2017} (verification of all symbolic
        calculations).
    \item The matplotlib~\cite{Droettboom2018} library (visualisation).
    \item The pandas~\cite{Structures2010}, dask~\cite{Dask2016} and
        NumPy~\cite{Oliphant2015} libraries (data manipulation).
    \item The SciPy~\cite{Jones2001} library (numerical integration of the
        replicator equation).
\end{itemize}

This work was performed using the computational facilities of the Advanced
Research Computing @ Cardiff (ARCCA) Division, Cardiff University.

\printbibliography

\newpage
\section*{Supplementary materials}

\includepdf{assets/pdf/proof_of_form_of_extortionate_strategies/main.pdf}

\newpage

Using the pair wise interactions the transition rates \(p,
q\) can be measured and the steady state probabilities inferred and compared to
the actual probabilities of each state.
This is done numerically by computing the singular eigenvector of the
matrix \(A\) \cite{Stewart2009}:

\[
    A =
    \begin{bmatrix}
        p_1 q_1 & p_1 (1 - q_1) & (1 - p_1) q_1 & (1 -p_1) (1 - q_1) \\
        p_2 q_2 & p_2 (1 - q_2) & (1 - p_2) q_2 & (1 -p_2) (1 - q_2) \\
        p_3 q_3 & p_3 (1 - q_3) & (1 - p_3) q_3 & (1 -p_3) (1 - q_3) \\
        p_4 q_4 & p_4 (1 - q_4) & (1 - p_4) q_4 & (1 -p_4) (1 - q_4) \\
    \end{bmatrix}
\]

Figure~\ref{fig:computed_probabilities_vs_theoretic_probabilities} shows a
regression line fitted to every pairwise interaction with a reported
\(\text{SSError}\) value (pairwise interactions with missing states were
omitted). This serves to validate the approach: a part from some edge cases the
relationship is consistent.

\begin{figure}[!htbp]
    \centering
    \includegraphics[width=.8\textwidth]{./assets/img/computed_probabilities_vs_theoretic_probabilities/main.pdf}
    \caption{The
        relationship between the steady state probabilities inferred from the
        measured transitions and the actual steady state probabilities. A linear
        regression line is included validating the approach.}
    \label{fig:computed_probabilities_vs_theoretic_probabilities}
\end{figure}


\end{document}
 turns and every match has been
repeated \documentclass[a4paper]{article}

\usepackage{amsmath}
\usepackage{amssymb}
\usepackage[margin=1.5cm,
            includefoot,
            footskip=30pt]{geometry}
\usepackage{layout}
\usepackage{graphicx}
\usepackage{subcaption}

\usepackage{biblatex}
\usepackage{pdfpages}

\bibliography{main.bib}

\title{Suspicion: Recognising and evaluating the effectiveness
       of extortion in the Iterated Prisoner's Dilemma}
\author{Vincent A. Knight \and Nikoleta E. Glynatsi}
\date{\today}



\begin{document}

\maketitle

\begin{abstract}
    The Iterated Prisoner's Dilemma is a model for rational and evolutionary
    interactive behaviour. It has applications both in the study of human social
    behaviour as well as in biology.
    It is used to understand when and how a rational individual might
    accept an immediate cost to their own utility for the direct benefit of
    another.

    Much attention has been given to a class of strategies called
    Zero Determinant strategies. It has been theoretically shown that these
    strategies can ``extort'' any player.

    In this work, an approach to identify if observed strategies are playing in
    an extortionate way is described. Furthermore, experimental analysis of
    a large tournament with \input{assets/tex/number_of_full_strategies/main.tex}
    strategies is considered. In this setting
    the most highly performing strategies do not play in an extortionate way
    against each other but do against lower performing strategies.
    This suggests that whilst the theory of Zero Determinant strategies
    indicates that memory is not of fundamental importance to the evolution of
    cooperative behaviour, this is incomplete.
\end{abstract}

\section{Introduction}\label{sec:introduction}

Agent based game theoretic models have become a stalwart of the underpinning
mathematics of interactive behaviours. One of the major pieces of work
in this area is the pair of original computer tournaments run by Robert
Axelrod~\cite{Axelrod1980, Axelrod1980a}. These tournaments pitted submitted
computer strategies against each other in plays of the Iterated Prisoner's
Dilemma. A common game where agents can choose to pay a slight cost to their
immediate utility in the hope of building a reputation. This has been used in
economic and evolutionary game theory to understand the evolution of cooperative
behaviour.

Recently, a class of strategies was described in~\cite{Press2012} that can
provably extort any given opponent. In~\cite{Hilbe2013, Moran1707} some
questions have already been asked about the true effectiveness of these
strategies in an evolutionary setting. Here another question is asked: is it
possible to recognise this extortionate behaviour? A mathematical procedure for
suspicion is presented: in the same way that the continued actions of an
extortionate individual might raise suspicion.

This work makes use of the Axelrod Python library~\cite{Knight2018, Knight2016}
with a large number of Prisoner Dilemma strategies available to give an
extensive numerical example of the ideas presented.  The approach is presented
in Section~\ref{sec:delta-zd-strategies}.  All of the code and data discussed
in Section~\ref{sec:numerical-experiments} is open sourced, archived and
written according to best scientific principles~\cite{Wilson2014}. The data
archive can be found at~\cite{vincent_knight_2018_1297075}.

\section{Recognising Extortion}\label{sec:delta-zd-strategies}

In~\cite{Press2012}, given a match between 2 memory-one strategies, the concept
of Zero Determinant (ZD) strategies is introduced. The main result of that paper
shows that given two memory one players \(p, q\in\mathbb{R}^4\) a linear
relationship between the players' scores could be forced by one of the players.

Using the notation of~\cite{Press2012}, assuming the utilities for player \(p\)
are given by \(S_x=(R, S, T, P)\) and for player \(q\) by \(S_y=(R, T, S, P)\)
and that the stationary scores of each player is given by \(S_X\) and \(S_Y\)
respectively. The main result of~\cite{Press2012} is that if

\begin{equation}\label{eqn:linear_relationship_for_p}
    \tilde p=\alpha S_x + \beta S_y + \gamma
\end{equation}

or

\begin{equation}\label{eqn:linear_relationship_for_q}
    \tilde q=\alpha S_x + \beta S_y + \gamma
\end{equation}

where \(\tilde p = (1 - p_1, 1 - p_2, p_3, p_4)\) and
\(\tilde q = (1 - q_1, 1 - q_2, q_3, q_4)\) then:

\begin{equation}
    \alpha S_X + \beta S_Y + \gamma = 0
\end{equation}

In~\cite{Press2012} a particular type of ZD strategy is defined: extortionate
strategies. If:

\begin{equation}\label{eqn:constraint_for_extortion}
    \gamma = - P(\alpha + \beta)
\end{equation}

then the player can ensure they get a score \(\chi\) times
larger than the opponent. This extortion coefficient is given by:

\begin{equation}\label{eqn:definition_of_chi}
    \chi=\frac{-\beta}{\alpha}
\end{equation}

Thus, if (\ref{eqn:constraint_for_extortion}) holds and \(\chi >1\) a player is
said to extort their opponent.
Here, the reverse problem is considered: given a
\(p\in\mathbb{R}^4\) how does one identify \(\alpha, \beta\) if they
exist and is the strategy in fact acting in an extortionate way?

These conditions correspond to:

\begin{align}
    \tilde p_1 & = \alpha R + \beta R - P (\alpha + \beta)
            \label{eqn:condition_for_tilde_p1}\\
    \tilde p_2 & = \alpha S + \beta T - P (\alpha + \beta)
            \label{eqn:condition_for_tilde_p2}\\
    \tilde p_3 & = \alpha T + \beta S - P (\alpha + \beta)
            \label{eqn:condition_for_tilde_p3}\\
    \tilde p_4 & = \alpha P + \beta P - P (\alpha + \beta)
            \label{eqn:condition_for_tilde_p4}
\end{align}

Equation (\ref{eqn:condition_for_tilde_p4}) ensures that \(p_4=\tilde p_4=0\).
Equations (\ref{eqn:condition_for_tilde_p1}-\ref{eqn:condition_for_tilde_p3})
can be used to eliminate \(\alpha, \beta\), giving:

\begin{equation}\label{eqn:planar_definition_of_extortion}
    \tilde p_1 = \frac{(R - P)(\tilde p_2 + \tilde p_3)}{S + T - 2P}
\end{equation}

with:

\begin{equation}\label{eqn:definition_of_chi}
    \chi = \frac{\tilde p_2 (P - T) + \tilde p_3 (S - P)}
                {\tilde p_2 (P - S) + \tilde p_3 (T - P)}
\end{equation}

Given a strategy \(p\in\mathbb{R}^{4\times 1}\) equations
(\ref{eqn:condition_for_tilde_p4}), (\ref{eqn:planar_definition_of_extortion}-\ref{eqn:definition_of_chi}) can be used to check if
a strategy is extortionate. The conditions correspond to:

\begin{align}
    p_1 & = \frac{(R-P)(p_2 + p_3) - R + T + S - P}{S + T - 2P}
     \label{eqn:condition_for_p1}\\
    p_4 & = 0 \label{eqn:condition_for_p4}\\
    1 & > p_2 + p_3\label{eqn:condition_for_chi}
\end{align}

The algebraic steps necessary to prove these results are available in the
supporting materials.

All extortionate strategies reside on a triangular (\ref{eqn:condition_for_chi})
plane (\ref{eqn:condition_for_p1}) in 3 dimensions (\ref{eqn:condition_for_p4}).
Using this formulation it can be seen that a necessary (but not sufficient)
condition for an extortionate strategy is that it cooperates on average less
than 50\% of the time when in a state of disagreement with the opponent.

As an example, consider the known extortionate strategy \(p=(8 / 9, 1 / 2, 1 /
3, 0)\) from~\cite{Stewart2012} which is referred to as \texttt{Extort-2}. In
this case, for the standard values of \((R, T, S, P)\) constraint
(\ref{eqn:condition_for_p1}) corresponds to:

\begin{equation}
    p_1 = \frac{2(p_2 + p_3) + 1}{3}
\end{equation}

It is clear that in this case all constraints hold.

This approach could in fact be used to confirm that a given strategy is acting
in an extortionate manner even if it is not a memory one strategy. However, in
practice, if a closed form for \(p\) is not known, then due to measurement
and/or numerical error this would not work.

This problem can be written in the following linear algebraic form where
\(x=(\alpha, \beta)\)
and \(p^*=(\tilde p_1 - 1, tilde_2 - 1, p_3)\):

\begin{equation}\label{eqn:linear_algebraic_equation_for_p}
    Cx= p^*
\end{equation}

\(C\) corresponds to equations
(\ref{eqn:condition_for_tilde_p1}-\ref{eqn:condition_for_tilde_p3}) and is
given by:

\begin{equation}\label{eqn:definition_of_C}
    C =
    \begin{bmatrix}
        R - P & R- P \\
        S - P & T- P \\
        T - P & S- P \\
    \end{bmatrix}
\end{equation}

Note that in general, equation (\ref{eqn:linear_algebraic_equation_for_p}) will
not necessarily have a solution. From the Rouch\'{e}-Capelli theorem if there is
a solution it is unique as \(\text{rank}(C)=2\) which is the dimension of the
variable \(x\). The best fitting \(x\) is found by minimizing:

\begin{equation}\label{eqn:r_squared}
    \text{SSError} = \|C x- p^*\|_2^2 = \sum_{i=1}^{3}\left((C\bar x)_i-p_i^*\right)^2
\end{equation}

Note that \(\text{SSError}\), which is the square of the Frobenius
norm~\cite{Golub2013}, becomes a measure of how close a strategy is to being an
extortionate strategy. Suspicion
of extortion then corresponds to a threshold on \(\text{SSError}\).

By observing interactions (human or otherwise), their memory one representation
can be inferred and this approach can be used to recognise extortionate
behaviour. The notion of comparing theoretic and actual plays of the IPD is not
novel, see for example~\cite{Rand2013}. Immediately it is noted that if the
environment is noisy~\cite{Wu1995} then no strategy can be considered to be
extortionate as \(p_4>0\).

In the next section, this idea will be illustrated by observing the interactions
that take place in a computer based tournament of the IPD\@.

\section{Numerical experiments}\label{sec:numerical-experiments}

In~\cite{Stewart2012} results from a tournament with
\input{./assets/tex/number_of_stewart_plotkin_strategies/main.tex} strategies,
was presented with specific consideration given to ZD strategies. This
tournament is reproduced here using the Axelrod-Python
project~\cite{Knight2016}. To obtain a good measure of the corresponding
transition rates for each strategy all matches have been run for
\input{assets/tex/number_of_turns/main.tex} turns and every match has been
repeated \input{assets/tex/number_of_repetitions/main.tex} times. All of this
interaction data is available at~\cite{vincent_knight_2018_1297075}. A good
match between the inferred Markov chain and the state distribution of the actual
interactions has been verified. Data for this is presented in the supplementary
materials.

Figure~\ref{fig:SSError_overall_in_stewart_plotkin} shows the \(\text{SSError}\)
values for all the strategies in the tournament, as reported
in~\cite{Stewart2012} the extortionate strategy (which has an expected
\(\text{SSError}\) approximately 0) gains a large number of wins.

\begin{figure}[!htbp]
    \centering
    \includegraphics[width=.8\textwidth]{./assets/img/SSError_overall_in_stewart_plotkin/main.pdf}
    \caption{\(\text{SSError}\) and state probabilities for the strategies
        of~\cite{Stewart2012}, ordered both by number of wins and overall score.
        Note that \(P(DC)\) is not shown as it corresponds to the transpose of
        \(P(CD)\). Cooperator and Defector are omitted as they do not visit all
        the states.}
    \label{fig:SSError_overall_in_stewart_plotkin}
\end{figure}

Here, the work of~\cite{Stewart2012} is extended by investigating a tournament
with \input{assets/tex/number_of_full_strategies/main.tex}
strategies.

The results of this analysis are shown in
Figure~\ref{fig:SSError_and_probabilities_in_full}. The top ranking strategies
by number of wins seem to be extortionate (but not against all strategies) and
it can be seen that a small sub group of strategies achieve mutual defection.
All the top ranking strategies according to score achieve mutual cooperation and
do not extort each other, however they
\textbf{do} exhibit extortionate behaviour towards a number of the lower ranking
strategies.

\begin{figure}[!htbp]
    \centering
    \includegraphics[width=.8\textwidth]{./assets/img/SSError_and_probabilities_in_full/main.pdf}
    \caption{\(\text{SSError}\) for the strategies for the full tournament. Only
    strategy interactions for which \(p_4=0\) and \(\chi>1\) are displayed.}
    \label{fig:SSError_and_probabilities_in_full}
\end{figure}

\section{Conclusion}\label{sec:conclusion}

This work defines an approach to measure whether or not a player is playing a
strategy that corresponds to an extortionate strategy as defined
in~\cite{Press2012}: a mathematical model for suspicion. Indeed, all
extortionate strategies have been
 classified as lying on a triangular plane.
This rigorous classification fails to be robust to small measurement error, thus
a statistical approach is proposed.
This is done through a linear algebraic approach for approximating the solution
of a linear system. Using this, a large number of pairwise interactions is
simulated and in fact very few strategies are found to act extortionately.

The work of~\cite{Press2012}, whilst showing that a clever approach to taking
advantage of another memory one strategy exists: this is incomplete. Whilst the
elegance of this result is very attractive, just as the simplicity of the
victory of Tit For Tat in Axelrod's original tournaments was, it is incomplete.
Extortionate strategies achieve a high number of wins but they do not
achieve a high score which corresponds to the fitness landscape in an
evolutionary sense. From the large number of interactions a payoff matrix \(S\)
can be measured where \(S_{ij}\) denotes the score (using standard values of
\((R, S, T, P) = (3, 0, 5, 1)\)) of the \(i\)th strategy
against the \(j\)th strategy. Using this, the replicator equation
describes the evolution of the system based on a population density fitness
function:

\begin{equation}\label{eqn:replicator_dynamics}
    \frac{dx}{dt} = x(S-x^TS x)
\end{equation}

Equation (\ref{eqn:replicator_dynamics}) is solved numerically through an
integration technique described in~\cite{Petzold1983} and
Figure~\ref{fig:replicator_dynamics} shows the evolution of the distribution of
the system: the various strategies are ranked by scores. It is clear to see that
only the high ranking strategies survive the evolutionary process (in fact,
only \input{./assets/img/replicator_dynamics/main.tex}
have a final distribution greater than \(10 ^ {-2}\)). This confirms the
findings of~\cite{Moran1707} in which sophisticated strategies resist
evolutionary invasion of shorter memory strategies. Recalling
Figure~\ref{fig:SSError_and_probabilities_in_full} this demonstrates that:

\begin{itemize}
    \item Cooperation emerges through the evolutionary process: the high scoring
        strategies do not exhibit extortionate behaviour towards each other.
    \item Extortionate strategies do not survive the evolutionary process.
\end{itemize}

\begin{figure}[!htbp]
    \centering
    \includegraphics[width=.8\textwidth]{./assets/img/replicator_dynamics/main.pdf}
    \caption{Numerical simulation of the replicator equation
    (\ref{eqn:replicator_dynamics}): strategies are ordered by score, only the strategies with a high score survive the evolutionary process.}
    \label{fig:replicator_dynamics}
\end{figure}

This work can be used to classify plays of the IPD\@: data can be collected from
actual interactions (in lab or in the field). Furthermore, this allows for a
classification method similar to the notion of fingerprinting presented
in~\cite{Ashlock2008}. Trained strategies can potentially be classified as
extortionate or not or it could be possible to even constrain the reinforcement
learning approaches that are becoming prevalent in the literature.
Alternatively, this mathematical approach for recognising extortion could be
used in sophisticated strategies to defend against invasion. Arguably, some of
the strategies considered here exhibit this behaviour, indeed as described
in~\cite{Harper2017}, the top ranking strategies in the full tournament are
obtained using evolutionary reinforcement learning techniques, thus, suspicion
of extortionate behaviour could in fact be an evolutionary trait.

\section*{Acknowledgements}

The following open source software libraries were used in this research:

\begin{itemize}
    \item The Axelrod ~\cite{Knight2016, Knight2018} library (IPD strategies and
        tournaments).
    \item The sympy library~\cite{Meurer2017} (verification of all symbolic
        calculations).
    \item The matplotlib~\cite{Droettboom2018} library (visualisation).
    \item The pandas~\cite{Structures2010}, dask~\cite{Dask2016} and
        NumPy~\cite{Oliphant2015} libraries (data manipulation).
    \item The SciPy~\cite{Jones2001} library (numerical integration of the
        replicator equation).
\end{itemize}

This work was performed using the computational facilities of the Advanced
Research Computing @ Cardiff (ARCCA) Division, Cardiff University.

\printbibliography

\newpage
\section*{Supplementary materials}

\includepdf{assets/pdf/proof_of_form_of_extortionate_strategies/main.pdf}

\newpage

Using the pair wise interactions the transition rates \(p,
q\) can be measured and the steady state probabilities inferred and compared to
the actual probabilities of each state.
This is done numerically by computing the singular eigenvector of the
matrix \(A\) \cite{Stewart2009}:

\[
    A =
    \begin{bmatrix}
        p_1 q_1 & p_1 (1 - q_1) & (1 - p_1) q_1 & (1 -p_1) (1 - q_1) \\
        p_2 q_2 & p_2 (1 - q_2) & (1 - p_2) q_2 & (1 -p_2) (1 - q_2) \\
        p_3 q_3 & p_3 (1 - q_3) & (1 - p_3) q_3 & (1 -p_3) (1 - q_3) \\
        p_4 q_4 & p_4 (1 - q_4) & (1 - p_4) q_4 & (1 -p_4) (1 - q_4) \\
    \end{bmatrix}
\]

Figure~\ref{fig:computed_probabilities_vs_theoretic_probabilities} shows a
regression line fitted to every pairwise interaction with a reported
\(\text{SSError}\) value (pairwise interactions with missing states were
omitted). This serves to validate the approach: a part from some edge cases the
relationship is consistent.

\begin{figure}[!htbp]
    \centering
    \includegraphics[width=.8\textwidth]{./assets/img/computed_probabilities_vs_theoretic_probabilities/main.pdf}
    \caption{The
        relationship between the steady state probabilities inferred from the
        measured transitions and the actual steady state probabilities. A linear
        regression line is included validating the approach.}
    \label{fig:computed_probabilities_vs_theoretic_probabilities}
\end{figure}


\end{document}
 times. All of this
interaction data is available at~\cite{vincent_knight_2018_1297075}. A good
match between the inferred Markov chain and the state distribution of the actual
interactions has been verified. Data for this is presented in the supplementary
materials.

Figure~\ref{fig:SSError_overall_in_stewart_plotkin} shows the \(\text{SSError}\)
values for all the strategies in the tournament, as reported
in~\cite{Stewart2012} the extortionate strategy (which has an expected
\(\text{SSError}\) approximately 0) gains a large number of wins.

\begin{figure}[!htbp]
    \centering
    \includegraphics[width=.8\textwidth]{./assets/img/SSError_overall_in_stewart_plotkin/main.pdf}
    \caption{\(\text{SSError}\) and state probabilities for the strategies
        of~\cite{Stewart2012}, ordered both by number of wins and overall score.
        Note that \(P(DC)\) is not shown as it corresponds to the transpose of
        \(P(CD)\). Cooperator and Defector are omitted as they do not visit all
        the states.}
    \label{fig:SSError_overall_in_stewart_plotkin}
\end{figure}

Here, the work of~\cite{Stewart2012} is extended by investigating a tournament
with \documentclass[a4paper]{article}

\usepackage{amsmath}
\usepackage{amssymb}
\usepackage[margin=1.5cm,
            includefoot,
            footskip=30pt]{geometry}
\usepackage{layout}
\usepackage{graphicx}
\usepackage{subcaption}

\usepackage{biblatex}
\usepackage{pdfpages}

\bibliography{main.bib}

\title{Suspicion: Recognising and evaluating the effectiveness
       of extortion in the Iterated Prisoner's Dilemma}
\author{Vincent A. Knight \and Nikoleta E. Glynatsi}
\date{\today}



\begin{document}

\maketitle

\begin{abstract}
    The Iterated Prisoner's Dilemma is a model for rational and evolutionary
    interactive behaviour. It has applications both in the study of human social
    behaviour as well as in biology.
    It is used to understand when and how a rational individual might
    accept an immediate cost to their own utility for the direct benefit of
    another.

    Much attention has been given to a class of strategies called
    Zero Determinant strategies. It has been theoretically shown that these
    strategies can ``extort'' any player.

    In this work, an approach to identify if observed strategies are playing in
    an extortionate way is described. Furthermore, experimental analysis of
    a large tournament with \input{assets/tex/number_of_full_strategies/main.tex}
    strategies is considered. In this setting
    the most highly performing strategies do not play in an extortionate way
    against each other but do against lower performing strategies.
    This suggests that whilst the theory of Zero Determinant strategies
    indicates that memory is not of fundamental importance to the evolution of
    cooperative behaviour, this is incomplete.
\end{abstract}

\section{Introduction}\label{sec:introduction}

Agent based game theoretic models have become a stalwart of the underpinning
mathematics of interactive behaviours. One of the major pieces of work
in this area is the pair of original computer tournaments run by Robert
Axelrod~\cite{Axelrod1980, Axelrod1980a}. These tournaments pitted submitted
computer strategies against each other in plays of the Iterated Prisoner's
Dilemma. A common game where agents can choose to pay a slight cost to their
immediate utility in the hope of building a reputation. This has been used in
economic and evolutionary game theory to understand the evolution of cooperative
behaviour.

Recently, a class of strategies was described in~\cite{Press2012} that can
provably extort any given opponent. In~\cite{Hilbe2013, Moran1707} some
questions have already been asked about the true effectiveness of these
strategies in an evolutionary setting. Here another question is asked: is it
possible to recognise this extortionate behaviour? A mathematical procedure for
suspicion is presented: in the same way that the continued actions of an
extortionate individual might raise suspicion.

This work makes use of the Axelrod Python library~\cite{Knight2018, Knight2016}
with a large number of Prisoner Dilemma strategies available to give an
extensive numerical example of the ideas presented.  The approach is presented
in Section~\ref{sec:delta-zd-strategies}.  All of the code and data discussed
in Section~\ref{sec:numerical-experiments} is open sourced, archived and
written according to best scientific principles~\cite{Wilson2014}. The data
archive can be found at~\cite{vincent_knight_2018_1297075}.

\section{Recognising Extortion}\label{sec:delta-zd-strategies}

In~\cite{Press2012}, given a match between 2 memory-one strategies, the concept
of Zero Determinant (ZD) strategies is introduced. The main result of that paper
shows that given two memory one players \(p, q\in\mathbb{R}^4\) a linear
relationship between the players' scores could be forced by one of the players.

Using the notation of~\cite{Press2012}, assuming the utilities for player \(p\)
are given by \(S_x=(R, S, T, P)\) and for player \(q\) by \(S_y=(R, T, S, P)\)
and that the stationary scores of each player is given by \(S_X\) and \(S_Y\)
respectively. The main result of~\cite{Press2012} is that if

\begin{equation}\label{eqn:linear_relationship_for_p}
    \tilde p=\alpha S_x + \beta S_y + \gamma
\end{equation}

or

\begin{equation}\label{eqn:linear_relationship_for_q}
    \tilde q=\alpha S_x + \beta S_y + \gamma
\end{equation}

where \(\tilde p = (1 - p_1, 1 - p_2, p_3, p_4)\) and
\(\tilde q = (1 - q_1, 1 - q_2, q_3, q_4)\) then:

\begin{equation}
    \alpha S_X + \beta S_Y + \gamma = 0
\end{equation}

In~\cite{Press2012} a particular type of ZD strategy is defined: extortionate
strategies. If:

\begin{equation}\label{eqn:constraint_for_extortion}
    \gamma = - P(\alpha + \beta)
\end{equation}

then the player can ensure they get a score \(\chi\) times
larger than the opponent. This extortion coefficient is given by:

\begin{equation}\label{eqn:definition_of_chi}
    \chi=\frac{-\beta}{\alpha}
\end{equation}

Thus, if (\ref{eqn:constraint_for_extortion}) holds and \(\chi >1\) a player is
said to extort their opponent.
Here, the reverse problem is considered: given a
\(p\in\mathbb{R}^4\) how does one identify \(\alpha, \beta\) if they
exist and is the strategy in fact acting in an extortionate way?

These conditions correspond to:

\begin{align}
    \tilde p_1 & = \alpha R + \beta R - P (\alpha + \beta)
            \label{eqn:condition_for_tilde_p1}\\
    \tilde p_2 & = \alpha S + \beta T - P (\alpha + \beta)
            \label{eqn:condition_for_tilde_p2}\\
    \tilde p_3 & = \alpha T + \beta S - P (\alpha + \beta)
            \label{eqn:condition_for_tilde_p3}\\
    \tilde p_4 & = \alpha P + \beta P - P (\alpha + \beta)
            \label{eqn:condition_for_tilde_p4}
\end{align}

Equation (\ref{eqn:condition_for_tilde_p4}) ensures that \(p_4=\tilde p_4=0\).
Equations (\ref{eqn:condition_for_tilde_p1}-\ref{eqn:condition_for_tilde_p3})
can be used to eliminate \(\alpha, \beta\), giving:

\begin{equation}\label{eqn:planar_definition_of_extortion}
    \tilde p_1 = \frac{(R - P)(\tilde p_2 + \tilde p_3)}{S + T - 2P}
\end{equation}

with:

\begin{equation}\label{eqn:definition_of_chi}
    \chi = \frac{\tilde p_2 (P - T) + \tilde p_3 (S - P)}
                {\tilde p_2 (P - S) + \tilde p_3 (T - P)}
\end{equation}

Given a strategy \(p\in\mathbb{R}^{4\times 1}\) equations
(\ref{eqn:condition_for_tilde_p4}), (\ref{eqn:planar_definition_of_extortion}-\ref{eqn:definition_of_chi}) can be used to check if
a strategy is extortionate. The conditions correspond to:

\begin{align}
    p_1 & = \frac{(R-P)(p_2 + p_3) - R + T + S - P}{S + T - 2P}
     \label{eqn:condition_for_p1}\\
    p_4 & = 0 \label{eqn:condition_for_p4}\\
    1 & > p_2 + p_3\label{eqn:condition_for_chi}
\end{align}

The algebraic steps necessary to prove these results are available in the
supporting materials.

All extortionate strategies reside on a triangular (\ref{eqn:condition_for_chi})
plane (\ref{eqn:condition_for_p1}) in 3 dimensions (\ref{eqn:condition_for_p4}).
Using this formulation it can be seen that a necessary (but not sufficient)
condition for an extortionate strategy is that it cooperates on average less
than 50\% of the time when in a state of disagreement with the opponent.

As an example, consider the known extortionate strategy \(p=(8 / 9, 1 / 2, 1 /
3, 0)\) from~\cite{Stewart2012} which is referred to as \texttt{Extort-2}. In
this case, for the standard values of \((R, T, S, P)\) constraint
(\ref{eqn:condition_for_p1}) corresponds to:

\begin{equation}
    p_1 = \frac{2(p_2 + p_3) + 1}{3}
\end{equation}

It is clear that in this case all constraints hold.

This approach could in fact be used to confirm that a given strategy is acting
in an extortionate manner even if it is not a memory one strategy. However, in
practice, if a closed form for \(p\) is not known, then due to measurement
and/or numerical error this would not work.

This problem can be written in the following linear algebraic form where
\(x=(\alpha, \beta)\)
and \(p^*=(\tilde p_1 - 1, tilde_2 - 1, p_3)\):

\begin{equation}\label{eqn:linear_algebraic_equation_for_p}
    Cx= p^*
\end{equation}

\(C\) corresponds to equations
(\ref{eqn:condition_for_tilde_p1}-\ref{eqn:condition_for_tilde_p3}) and is
given by:

\begin{equation}\label{eqn:definition_of_C}
    C =
    \begin{bmatrix}
        R - P & R- P \\
        S - P & T- P \\
        T - P & S- P \\
    \end{bmatrix}
\end{equation}

Note that in general, equation (\ref{eqn:linear_algebraic_equation_for_p}) will
not necessarily have a solution. From the Rouch\'{e}-Capelli theorem if there is
a solution it is unique as \(\text{rank}(C)=2\) which is the dimension of the
variable \(x\). The best fitting \(x\) is found by minimizing:

\begin{equation}\label{eqn:r_squared}
    \text{SSError} = \|C x- p^*\|_2^2 = \sum_{i=1}^{3}\left((C\bar x)_i-p_i^*\right)^2
\end{equation}

Note that \(\text{SSError}\), which is the square of the Frobenius
norm~\cite{Golub2013}, becomes a measure of how close a strategy is to being an
extortionate strategy. Suspicion
of extortion then corresponds to a threshold on \(\text{SSError}\).

By observing interactions (human or otherwise), their memory one representation
can be inferred and this approach can be used to recognise extortionate
behaviour. The notion of comparing theoretic and actual plays of the IPD is not
novel, see for example~\cite{Rand2013}. Immediately it is noted that if the
environment is noisy~\cite{Wu1995} then no strategy can be considered to be
extortionate as \(p_4>0\).

In the next section, this idea will be illustrated by observing the interactions
that take place in a computer based tournament of the IPD\@.

\section{Numerical experiments}\label{sec:numerical-experiments}

In~\cite{Stewart2012} results from a tournament with
\input{./assets/tex/number_of_stewart_plotkin_strategies/main.tex} strategies,
was presented with specific consideration given to ZD strategies. This
tournament is reproduced here using the Axelrod-Python
project~\cite{Knight2016}. To obtain a good measure of the corresponding
transition rates for each strategy all matches have been run for
\input{assets/tex/number_of_turns/main.tex} turns and every match has been
repeated \input{assets/tex/number_of_repetitions/main.tex} times. All of this
interaction data is available at~\cite{vincent_knight_2018_1297075}. A good
match between the inferred Markov chain and the state distribution of the actual
interactions has been verified. Data for this is presented in the supplementary
materials.

Figure~\ref{fig:SSError_overall_in_stewart_plotkin} shows the \(\text{SSError}\)
values for all the strategies in the tournament, as reported
in~\cite{Stewart2012} the extortionate strategy (which has an expected
\(\text{SSError}\) approximately 0) gains a large number of wins.

\begin{figure}[!htbp]
    \centering
    \includegraphics[width=.8\textwidth]{./assets/img/SSError_overall_in_stewart_plotkin/main.pdf}
    \caption{\(\text{SSError}\) and state probabilities for the strategies
        of~\cite{Stewart2012}, ordered both by number of wins and overall score.
        Note that \(P(DC)\) is not shown as it corresponds to the transpose of
        \(P(CD)\). Cooperator and Defector are omitted as they do not visit all
        the states.}
    \label{fig:SSError_overall_in_stewart_plotkin}
\end{figure}

Here, the work of~\cite{Stewart2012} is extended by investigating a tournament
with \input{assets/tex/number_of_full_strategies/main.tex}
strategies.

The results of this analysis are shown in
Figure~\ref{fig:SSError_and_probabilities_in_full}. The top ranking strategies
by number of wins seem to be extortionate (but not against all strategies) and
it can be seen that a small sub group of strategies achieve mutual defection.
All the top ranking strategies according to score achieve mutual cooperation and
do not extort each other, however they
\textbf{do} exhibit extortionate behaviour towards a number of the lower ranking
strategies.

\begin{figure}[!htbp]
    \centering
    \includegraphics[width=.8\textwidth]{./assets/img/SSError_and_probabilities_in_full/main.pdf}
    \caption{\(\text{SSError}\) for the strategies for the full tournament. Only
    strategy interactions for which \(p_4=0\) and \(\chi>1\) are displayed.}
    \label{fig:SSError_and_probabilities_in_full}
\end{figure}

\section{Conclusion}\label{sec:conclusion}

This work defines an approach to measure whether or not a player is playing a
strategy that corresponds to an extortionate strategy as defined
in~\cite{Press2012}: a mathematical model for suspicion. Indeed, all
extortionate strategies have been
 classified as lying on a triangular plane.
This rigorous classification fails to be robust to small measurement error, thus
a statistical approach is proposed.
This is done through a linear algebraic approach for approximating the solution
of a linear system. Using this, a large number of pairwise interactions is
simulated and in fact very few strategies are found to act extortionately.

The work of~\cite{Press2012}, whilst showing that a clever approach to taking
advantage of another memory one strategy exists: this is incomplete. Whilst the
elegance of this result is very attractive, just as the simplicity of the
victory of Tit For Tat in Axelrod's original tournaments was, it is incomplete.
Extortionate strategies achieve a high number of wins but they do not
achieve a high score which corresponds to the fitness landscape in an
evolutionary sense. From the large number of interactions a payoff matrix \(S\)
can be measured where \(S_{ij}\) denotes the score (using standard values of
\((R, S, T, P) = (3, 0, 5, 1)\)) of the \(i\)th strategy
against the \(j\)th strategy. Using this, the replicator equation
describes the evolution of the system based on a population density fitness
function:

\begin{equation}\label{eqn:replicator_dynamics}
    \frac{dx}{dt} = x(S-x^TS x)
\end{equation}

Equation (\ref{eqn:replicator_dynamics}) is solved numerically through an
integration technique described in~\cite{Petzold1983} and
Figure~\ref{fig:replicator_dynamics} shows the evolution of the distribution of
the system: the various strategies are ranked by scores. It is clear to see that
only the high ranking strategies survive the evolutionary process (in fact,
only \input{./assets/img/replicator_dynamics/main.tex}
have a final distribution greater than \(10 ^ {-2}\)). This confirms the
findings of~\cite{Moran1707} in which sophisticated strategies resist
evolutionary invasion of shorter memory strategies. Recalling
Figure~\ref{fig:SSError_and_probabilities_in_full} this demonstrates that:

\begin{itemize}
    \item Cooperation emerges through the evolutionary process: the high scoring
        strategies do not exhibit extortionate behaviour towards each other.
    \item Extortionate strategies do not survive the evolutionary process.
\end{itemize}

\begin{figure}[!htbp]
    \centering
    \includegraphics[width=.8\textwidth]{./assets/img/replicator_dynamics/main.pdf}
    \caption{Numerical simulation of the replicator equation
    (\ref{eqn:replicator_dynamics}): strategies are ordered by score, only the strategies with a high score survive the evolutionary process.}
    \label{fig:replicator_dynamics}
\end{figure}

This work can be used to classify plays of the IPD\@: data can be collected from
actual interactions (in lab or in the field). Furthermore, this allows for a
classification method similar to the notion of fingerprinting presented
in~\cite{Ashlock2008}. Trained strategies can potentially be classified as
extortionate or not or it could be possible to even constrain the reinforcement
learning approaches that are becoming prevalent in the literature.
Alternatively, this mathematical approach for recognising extortion could be
used in sophisticated strategies to defend against invasion. Arguably, some of
the strategies considered here exhibit this behaviour, indeed as described
in~\cite{Harper2017}, the top ranking strategies in the full tournament are
obtained using evolutionary reinforcement learning techniques, thus, suspicion
of extortionate behaviour could in fact be an evolutionary trait.

\section*{Acknowledgements}

The following open source software libraries were used in this research:

\begin{itemize}
    \item The Axelrod ~\cite{Knight2016, Knight2018} library (IPD strategies and
        tournaments).
    \item The sympy library~\cite{Meurer2017} (verification of all symbolic
        calculations).
    \item The matplotlib~\cite{Droettboom2018} library (visualisation).
    \item The pandas~\cite{Structures2010}, dask~\cite{Dask2016} and
        NumPy~\cite{Oliphant2015} libraries (data manipulation).
    \item The SciPy~\cite{Jones2001} library (numerical integration of the
        replicator equation).
\end{itemize}

This work was performed using the computational facilities of the Advanced
Research Computing @ Cardiff (ARCCA) Division, Cardiff University.

\printbibliography

\newpage
\section*{Supplementary materials}

\includepdf{assets/pdf/proof_of_form_of_extortionate_strategies/main.pdf}

\newpage

Using the pair wise interactions the transition rates \(p,
q\) can be measured and the steady state probabilities inferred and compared to
the actual probabilities of each state.
This is done numerically by computing the singular eigenvector of the
matrix \(A\) \cite{Stewart2009}:

\[
    A =
    \begin{bmatrix}
        p_1 q_1 & p_1 (1 - q_1) & (1 - p_1) q_1 & (1 -p_1) (1 - q_1) \\
        p_2 q_2 & p_2 (1 - q_2) & (1 - p_2) q_2 & (1 -p_2) (1 - q_2) \\
        p_3 q_3 & p_3 (1 - q_3) & (1 - p_3) q_3 & (1 -p_3) (1 - q_3) \\
        p_4 q_4 & p_4 (1 - q_4) & (1 - p_4) q_4 & (1 -p_4) (1 - q_4) \\
    \end{bmatrix}
\]

Figure~\ref{fig:computed_probabilities_vs_theoretic_probabilities} shows a
regression line fitted to every pairwise interaction with a reported
\(\text{SSError}\) value (pairwise interactions with missing states were
omitted). This serves to validate the approach: a part from some edge cases the
relationship is consistent.

\begin{figure}[!htbp]
    \centering
    \includegraphics[width=.8\textwidth]{./assets/img/computed_probabilities_vs_theoretic_probabilities/main.pdf}
    \caption{The
        relationship between the steady state probabilities inferred from the
        measured transitions and the actual steady state probabilities. A linear
        regression line is included validating the approach.}
    \label{fig:computed_probabilities_vs_theoretic_probabilities}
\end{figure}


\end{document}

strategies.

The results of this analysis are shown in
Figure~\ref{fig:SSError_and_probabilities_in_full}. The top ranking strategies
by number of wins seem to be extortionate (but not against all strategies) and
it can be seen that a small sub group of strategies achieve mutual defection.
All the top ranking strategies according to score achieve mutual cooperation and
do not extort each other, however they
\textbf{do} exhibit extortionate behaviour towards a number of the lower ranking
strategies.

\begin{figure}[!htbp]
    \centering
    \includegraphics[width=.8\textwidth]{./assets/img/SSError_and_probabilities_in_full/main.pdf}
    \caption{\(\text{SSError}\) for the strategies for the full tournament. Only
    strategy interactions for which \(p_4=0\) and \(\chi>1\) are displayed.}
    \label{fig:SSError_and_probabilities_in_full}
\end{figure}

\section{Conclusion}\label{sec:conclusion}

This work defines an approach to measure whether or not a player is playing a
strategy that corresponds to an extortionate strategy as defined
in~\cite{Press2012}: a mathematical model for suspicion. Indeed, all
extortionate strategies have been
 classified as lying on a triangular plane.
This rigorous classification fails to be robust to small measurement error, thus
a statistical approach is proposed.
This is done through a linear algebraic approach for approximating the solution
of a linear system. Using this, a large number of pairwise interactions is
simulated and in fact very few strategies are found to act extortionately.

The work of~\cite{Press2012}, whilst showing that a clever approach to taking
advantage of another memory one strategy exists: this is incomplete. Whilst the
elegance of this result is very attractive, just as the simplicity of the
victory of Tit For Tat in Axelrod's original tournaments was, it is incomplete.
Extortionate strategies achieve a high number of wins but they do not
achieve a high score which corresponds to the fitness landscape in an
evolutionary sense. From the large number of interactions a payoff matrix \(S\)
can be measured where \(S_{ij}\) denotes the score (using standard values of
\((R, S, T, P) = (3, 0, 5, 1)\)) of the \(i\)th strategy
against the \(j\)th strategy. Using this, the replicator equation
describes the evolution of the system based on a population density fitness
function:

\begin{equation}\label{eqn:replicator_dynamics}
    \frac{dx}{dt} = x(S-x^TS x)
\end{equation}

Equation (\ref{eqn:replicator_dynamics}) is solved numerically through an
integration technique described in~\cite{Petzold1983} and
Figure~\ref{fig:replicator_dynamics} shows the evolution of the distribution of
the system: the various strategies are ranked by scores. It is clear to see that
only the high ranking strategies survive the evolutionary process (in fact,
only \documentclass[a4paper]{article}

\usepackage{amsmath}
\usepackage{amssymb}
\usepackage[margin=1.5cm,
            includefoot,
            footskip=30pt]{geometry}
\usepackage{layout}
\usepackage{graphicx}
\usepackage{subcaption}

\usepackage{biblatex}
\usepackage{pdfpages}

\bibliography{main.bib}

\title{Suspicion: Recognising and evaluating the effectiveness
       of extortion in the Iterated Prisoner's Dilemma}
\author{Vincent A. Knight \and Nikoleta E. Glynatsi}
\date{\today}



\begin{document}

\maketitle

\begin{abstract}
    The Iterated Prisoner's Dilemma is a model for rational and evolutionary
    interactive behaviour. It has applications both in the study of human social
    behaviour as well as in biology.
    It is used to understand when and how a rational individual might
    accept an immediate cost to their own utility for the direct benefit of
    another.

    Much attention has been given to a class of strategies called
    Zero Determinant strategies. It has been theoretically shown that these
    strategies can ``extort'' any player.

    In this work, an approach to identify if observed strategies are playing in
    an extortionate way is described. Furthermore, experimental analysis of
    a large tournament with \input{assets/tex/number_of_full_strategies/main.tex}
    strategies is considered. In this setting
    the most highly performing strategies do not play in an extortionate way
    against each other but do against lower performing strategies.
    This suggests that whilst the theory of Zero Determinant strategies
    indicates that memory is not of fundamental importance to the evolution of
    cooperative behaviour, this is incomplete.
\end{abstract}

\section{Introduction}\label{sec:introduction}

Agent based game theoretic models have become a stalwart of the underpinning
mathematics of interactive behaviours. One of the major pieces of work
in this area is the pair of original computer tournaments run by Robert
Axelrod~\cite{Axelrod1980, Axelrod1980a}. These tournaments pitted submitted
computer strategies against each other in plays of the Iterated Prisoner's
Dilemma. A common game where agents can choose to pay a slight cost to their
immediate utility in the hope of building a reputation. This has been used in
economic and evolutionary game theory to understand the evolution of cooperative
behaviour.

Recently, a class of strategies was described in~\cite{Press2012} that can
provably extort any given opponent. In~\cite{Hilbe2013, Moran1707} some
questions have already been asked about the true effectiveness of these
strategies in an evolutionary setting. Here another question is asked: is it
possible to recognise this extortionate behaviour? A mathematical procedure for
suspicion is presented: in the same way that the continued actions of an
extortionate individual might raise suspicion.

This work makes use of the Axelrod Python library~\cite{Knight2018, Knight2016}
with a large number of Prisoner Dilemma strategies available to give an
extensive numerical example of the ideas presented.  The approach is presented
in Section~\ref{sec:delta-zd-strategies}.  All of the code and data discussed
in Section~\ref{sec:numerical-experiments} is open sourced, archived and
written according to best scientific principles~\cite{Wilson2014}. The data
archive can be found at~\cite{vincent_knight_2018_1297075}.

\section{Recognising Extortion}\label{sec:delta-zd-strategies}

In~\cite{Press2012}, given a match between 2 memory-one strategies, the concept
of Zero Determinant (ZD) strategies is introduced. The main result of that paper
shows that given two memory one players \(p, q\in\mathbb{R}^4\) a linear
relationship between the players' scores could be forced by one of the players.

Using the notation of~\cite{Press2012}, assuming the utilities for player \(p\)
are given by \(S_x=(R, S, T, P)\) and for player \(q\) by \(S_y=(R, T, S, P)\)
and that the stationary scores of each player is given by \(S_X\) and \(S_Y\)
respectively. The main result of~\cite{Press2012} is that if

\begin{equation}\label{eqn:linear_relationship_for_p}
    \tilde p=\alpha S_x + \beta S_y + \gamma
\end{equation}

or

\begin{equation}\label{eqn:linear_relationship_for_q}
    \tilde q=\alpha S_x + \beta S_y + \gamma
\end{equation}

where \(\tilde p = (1 - p_1, 1 - p_2, p_3, p_4)\) and
\(\tilde q = (1 - q_1, 1 - q_2, q_3, q_4)\) then:

\begin{equation}
    \alpha S_X + \beta S_Y + \gamma = 0
\end{equation}

In~\cite{Press2012} a particular type of ZD strategy is defined: extortionate
strategies. If:

\begin{equation}\label{eqn:constraint_for_extortion}
    \gamma = - P(\alpha + \beta)
\end{equation}

then the player can ensure they get a score \(\chi\) times
larger than the opponent. This extortion coefficient is given by:

\begin{equation}\label{eqn:definition_of_chi}
    \chi=\frac{-\beta}{\alpha}
\end{equation}

Thus, if (\ref{eqn:constraint_for_extortion}) holds and \(\chi >1\) a player is
said to extort their opponent.
Here, the reverse problem is considered: given a
\(p\in\mathbb{R}^4\) how does one identify \(\alpha, \beta\) if they
exist and is the strategy in fact acting in an extortionate way?

These conditions correspond to:

\begin{align}
    \tilde p_1 & = \alpha R + \beta R - P (\alpha + \beta)
            \label{eqn:condition_for_tilde_p1}\\
    \tilde p_2 & = \alpha S + \beta T - P (\alpha + \beta)
            \label{eqn:condition_for_tilde_p2}\\
    \tilde p_3 & = \alpha T + \beta S - P (\alpha + \beta)
            \label{eqn:condition_for_tilde_p3}\\
    \tilde p_4 & = \alpha P + \beta P - P (\alpha + \beta)
            \label{eqn:condition_for_tilde_p4}
\end{align}

Equation (\ref{eqn:condition_for_tilde_p4}) ensures that \(p_4=\tilde p_4=0\).
Equations (\ref{eqn:condition_for_tilde_p1}-\ref{eqn:condition_for_tilde_p3})
can be used to eliminate \(\alpha, \beta\), giving:

\begin{equation}\label{eqn:planar_definition_of_extortion}
    \tilde p_1 = \frac{(R - P)(\tilde p_2 + \tilde p_3)}{S + T - 2P}
\end{equation}

with:

\begin{equation}\label{eqn:definition_of_chi}
    \chi = \frac{\tilde p_2 (P - T) + \tilde p_3 (S - P)}
                {\tilde p_2 (P - S) + \tilde p_3 (T - P)}
\end{equation}

Given a strategy \(p\in\mathbb{R}^{4\times 1}\) equations
(\ref{eqn:condition_for_tilde_p4}), (\ref{eqn:planar_definition_of_extortion}-\ref{eqn:definition_of_chi}) can be used to check if
a strategy is extortionate. The conditions correspond to:

\begin{align}
    p_1 & = \frac{(R-P)(p_2 + p_3) - R + T + S - P}{S + T - 2P}
     \label{eqn:condition_for_p1}\\
    p_4 & = 0 \label{eqn:condition_for_p4}\\
    1 & > p_2 + p_3\label{eqn:condition_for_chi}
\end{align}

The algebraic steps necessary to prove these results are available in the
supporting materials.

All extortionate strategies reside on a triangular (\ref{eqn:condition_for_chi})
plane (\ref{eqn:condition_for_p1}) in 3 dimensions (\ref{eqn:condition_for_p4}).
Using this formulation it can be seen that a necessary (but not sufficient)
condition for an extortionate strategy is that it cooperates on average less
than 50\% of the time when in a state of disagreement with the opponent.

As an example, consider the known extortionate strategy \(p=(8 / 9, 1 / 2, 1 /
3, 0)\) from~\cite{Stewart2012} which is referred to as \texttt{Extort-2}. In
this case, for the standard values of \((R, T, S, P)\) constraint
(\ref{eqn:condition_for_p1}) corresponds to:

\begin{equation}
    p_1 = \frac{2(p_2 + p_3) + 1}{3}
\end{equation}

It is clear that in this case all constraints hold.

This approach could in fact be used to confirm that a given strategy is acting
in an extortionate manner even if it is not a memory one strategy. However, in
practice, if a closed form for \(p\) is not known, then due to measurement
and/or numerical error this would not work.

This problem can be written in the following linear algebraic form where
\(x=(\alpha, \beta)\)
and \(p^*=(\tilde p_1 - 1, tilde_2 - 1, p_3)\):

\begin{equation}\label{eqn:linear_algebraic_equation_for_p}
    Cx= p^*
\end{equation}

\(C\) corresponds to equations
(\ref{eqn:condition_for_tilde_p1}-\ref{eqn:condition_for_tilde_p3}) and is
given by:

\begin{equation}\label{eqn:definition_of_C}
    C =
    \begin{bmatrix}
        R - P & R- P \\
        S - P & T- P \\
        T - P & S- P \\
    \end{bmatrix}
\end{equation}

Note that in general, equation (\ref{eqn:linear_algebraic_equation_for_p}) will
not necessarily have a solution. From the Rouch\'{e}-Capelli theorem if there is
a solution it is unique as \(\text{rank}(C)=2\) which is the dimension of the
variable \(x\). The best fitting \(x\) is found by minimizing:

\begin{equation}\label{eqn:r_squared}
    \text{SSError} = \|C x- p^*\|_2^2 = \sum_{i=1}^{3}\left((C\bar x)_i-p_i^*\right)^2
\end{equation}

Note that \(\text{SSError}\), which is the square of the Frobenius
norm~\cite{Golub2013}, becomes a measure of how close a strategy is to being an
extortionate strategy. Suspicion
of extortion then corresponds to a threshold on \(\text{SSError}\).

By observing interactions (human or otherwise), their memory one representation
can be inferred and this approach can be used to recognise extortionate
behaviour. The notion of comparing theoretic and actual plays of the IPD is not
novel, see for example~\cite{Rand2013}. Immediately it is noted that if the
environment is noisy~\cite{Wu1995} then no strategy can be considered to be
extortionate as \(p_4>0\).

In the next section, this idea will be illustrated by observing the interactions
that take place in a computer based tournament of the IPD\@.

\section{Numerical experiments}\label{sec:numerical-experiments}

In~\cite{Stewart2012} results from a tournament with
\input{./assets/tex/number_of_stewart_plotkin_strategies/main.tex} strategies,
was presented with specific consideration given to ZD strategies. This
tournament is reproduced here using the Axelrod-Python
project~\cite{Knight2016}. To obtain a good measure of the corresponding
transition rates for each strategy all matches have been run for
\input{assets/tex/number_of_turns/main.tex} turns and every match has been
repeated \input{assets/tex/number_of_repetitions/main.tex} times. All of this
interaction data is available at~\cite{vincent_knight_2018_1297075}. A good
match between the inferred Markov chain and the state distribution of the actual
interactions has been verified. Data for this is presented in the supplementary
materials.

Figure~\ref{fig:SSError_overall_in_stewart_plotkin} shows the \(\text{SSError}\)
values for all the strategies in the tournament, as reported
in~\cite{Stewart2012} the extortionate strategy (which has an expected
\(\text{SSError}\) approximately 0) gains a large number of wins.

\begin{figure}[!htbp]
    \centering
    \includegraphics[width=.8\textwidth]{./assets/img/SSError_overall_in_stewart_plotkin/main.pdf}
    \caption{\(\text{SSError}\) and state probabilities for the strategies
        of~\cite{Stewart2012}, ordered both by number of wins and overall score.
        Note that \(P(DC)\) is not shown as it corresponds to the transpose of
        \(P(CD)\). Cooperator and Defector are omitted as they do not visit all
        the states.}
    \label{fig:SSError_overall_in_stewart_plotkin}
\end{figure}

Here, the work of~\cite{Stewart2012} is extended by investigating a tournament
with \input{assets/tex/number_of_full_strategies/main.tex}
strategies.

The results of this analysis are shown in
Figure~\ref{fig:SSError_and_probabilities_in_full}. The top ranking strategies
by number of wins seem to be extortionate (but not against all strategies) and
it can be seen that a small sub group of strategies achieve mutual defection.
All the top ranking strategies according to score achieve mutual cooperation and
do not extort each other, however they
\textbf{do} exhibit extortionate behaviour towards a number of the lower ranking
strategies.

\begin{figure}[!htbp]
    \centering
    \includegraphics[width=.8\textwidth]{./assets/img/SSError_and_probabilities_in_full/main.pdf}
    \caption{\(\text{SSError}\) for the strategies for the full tournament. Only
    strategy interactions for which \(p_4=0\) and \(\chi>1\) are displayed.}
    \label{fig:SSError_and_probabilities_in_full}
\end{figure}

\section{Conclusion}\label{sec:conclusion}

This work defines an approach to measure whether or not a player is playing a
strategy that corresponds to an extortionate strategy as defined
in~\cite{Press2012}: a mathematical model for suspicion. Indeed, all
extortionate strategies have been
 classified as lying on a triangular plane.
This rigorous classification fails to be robust to small measurement error, thus
a statistical approach is proposed.
This is done through a linear algebraic approach for approximating the solution
of a linear system. Using this, a large number of pairwise interactions is
simulated and in fact very few strategies are found to act extortionately.

The work of~\cite{Press2012}, whilst showing that a clever approach to taking
advantage of another memory one strategy exists: this is incomplete. Whilst the
elegance of this result is very attractive, just as the simplicity of the
victory of Tit For Tat in Axelrod's original tournaments was, it is incomplete.
Extortionate strategies achieve a high number of wins but they do not
achieve a high score which corresponds to the fitness landscape in an
evolutionary sense. From the large number of interactions a payoff matrix \(S\)
can be measured where \(S_{ij}\) denotes the score (using standard values of
\((R, S, T, P) = (3, 0, 5, 1)\)) of the \(i\)th strategy
against the \(j\)th strategy. Using this, the replicator equation
describes the evolution of the system based on a population density fitness
function:

\begin{equation}\label{eqn:replicator_dynamics}
    \frac{dx}{dt} = x(S-x^TS x)
\end{equation}

Equation (\ref{eqn:replicator_dynamics}) is solved numerically through an
integration technique described in~\cite{Petzold1983} and
Figure~\ref{fig:replicator_dynamics} shows the evolution of the distribution of
the system: the various strategies are ranked by scores. It is clear to see that
only the high ranking strategies survive the evolutionary process (in fact,
only \input{./assets/img/replicator_dynamics/main.tex}
have a final distribution greater than \(10 ^ {-2}\)). This confirms the
findings of~\cite{Moran1707} in which sophisticated strategies resist
evolutionary invasion of shorter memory strategies. Recalling
Figure~\ref{fig:SSError_and_probabilities_in_full} this demonstrates that:

\begin{itemize}
    \item Cooperation emerges through the evolutionary process: the high scoring
        strategies do not exhibit extortionate behaviour towards each other.
    \item Extortionate strategies do not survive the evolutionary process.
\end{itemize}

\begin{figure}[!htbp]
    \centering
    \includegraphics[width=.8\textwidth]{./assets/img/replicator_dynamics/main.pdf}
    \caption{Numerical simulation of the replicator equation
    (\ref{eqn:replicator_dynamics}): strategies are ordered by score, only the strategies with a high score survive the evolutionary process.}
    \label{fig:replicator_dynamics}
\end{figure}

This work can be used to classify plays of the IPD\@: data can be collected from
actual interactions (in lab or in the field). Furthermore, this allows for a
classification method similar to the notion of fingerprinting presented
in~\cite{Ashlock2008}. Trained strategies can potentially be classified as
extortionate or not or it could be possible to even constrain the reinforcement
learning approaches that are becoming prevalent in the literature.
Alternatively, this mathematical approach for recognising extortion could be
used in sophisticated strategies to defend against invasion. Arguably, some of
the strategies considered here exhibit this behaviour, indeed as described
in~\cite{Harper2017}, the top ranking strategies in the full tournament are
obtained using evolutionary reinforcement learning techniques, thus, suspicion
of extortionate behaviour could in fact be an evolutionary trait.

\section*{Acknowledgements}

The following open source software libraries were used in this research:

\begin{itemize}
    \item The Axelrod ~\cite{Knight2016, Knight2018} library (IPD strategies and
        tournaments).
    \item The sympy library~\cite{Meurer2017} (verification of all symbolic
        calculations).
    \item The matplotlib~\cite{Droettboom2018} library (visualisation).
    \item The pandas~\cite{Structures2010}, dask~\cite{Dask2016} and
        NumPy~\cite{Oliphant2015} libraries (data manipulation).
    \item The SciPy~\cite{Jones2001} library (numerical integration of the
        replicator equation).
\end{itemize}

This work was performed using the computational facilities of the Advanced
Research Computing @ Cardiff (ARCCA) Division, Cardiff University.

\printbibliography

\newpage
\section*{Supplementary materials}

\includepdf{assets/pdf/proof_of_form_of_extortionate_strategies/main.pdf}

\newpage

Using the pair wise interactions the transition rates \(p,
q\) can be measured and the steady state probabilities inferred and compared to
the actual probabilities of each state.
This is done numerically by computing the singular eigenvector of the
matrix \(A\) \cite{Stewart2009}:

\[
    A =
    \begin{bmatrix}
        p_1 q_1 & p_1 (1 - q_1) & (1 - p_1) q_1 & (1 -p_1) (1 - q_1) \\
        p_2 q_2 & p_2 (1 - q_2) & (1 - p_2) q_2 & (1 -p_2) (1 - q_2) \\
        p_3 q_3 & p_3 (1 - q_3) & (1 - p_3) q_3 & (1 -p_3) (1 - q_3) \\
        p_4 q_4 & p_4 (1 - q_4) & (1 - p_4) q_4 & (1 -p_4) (1 - q_4) \\
    \end{bmatrix}
\]

Figure~\ref{fig:computed_probabilities_vs_theoretic_probabilities} shows a
regression line fitted to every pairwise interaction with a reported
\(\text{SSError}\) value (pairwise interactions with missing states were
omitted). This serves to validate the approach: a part from some edge cases the
relationship is consistent.

\begin{figure}[!htbp]
    \centering
    \includegraphics[width=.8\textwidth]{./assets/img/computed_probabilities_vs_theoretic_probabilities/main.pdf}
    \caption{The
        relationship between the steady state probabilities inferred from the
        measured transitions and the actual steady state probabilities. A linear
        regression line is included validating the approach.}
    \label{fig:computed_probabilities_vs_theoretic_probabilities}
\end{figure}


\end{document}

have a final distribution greater than \(10 ^ {-2}\)). This confirms the
findings of~\cite{Moran1707} in which sophisticated strategies resist
evolutionary invasion of shorter memory strategies. Recalling
Figure~\ref{fig:SSError_and_probabilities_in_full} this demonstrates that:

\begin{itemize}
    \item Cooperation emerges through the evolutionary process: the high scoring
        strategies do not exhibit extortionate behaviour towards each other.
    \item Extortionate strategies do not survive the evolutionary process.
\end{itemize}

\begin{figure}[!htbp]
    \centering
    \includegraphics[width=.8\textwidth]{./assets/img/replicator_dynamics/main.pdf}
    \caption{Numerical simulation of the replicator equation
    (\ref{eqn:replicator_dynamics}): strategies are ordered by score, only the strategies with a high score survive the evolutionary process.}
    \label{fig:replicator_dynamics}
\end{figure}

This work can be used to classify plays of the IPD\@: data can be collected from
actual interactions (in lab or in the field). Furthermore, this allows for a
classification method similar to the notion of fingerprinting presented
in~\cite{Ashlock2008}. Trained strategies can potentially be classified as
extortionate or not or it could be possible to even constrain the reinforcement
learning approaches that are becoming prevalent in the literature.
Alternatively, this mathematical approach for recognising extortion could be
used in sophisticated strategies to defend against invasion. Arguably, some of
the strategies considered here exhibit this behaviour, indeed as described
in~\cite{Harper2017}, the top ranking strategies in the full tournament are
obtained using evolutionary reinforcement learning techniques, thus, suspicion
of extortionate behaviour could in fact be an evolutionary trait.

\section*{Acknowledgements}

The following open source software libraries were used in this research:

\begin{itemize}
    \item The Axelrod ~\cite{Knight2016, Knight2018} library (IPD strategies and
        tournaments).
    \item The sympy library~\cite{Meurer2017} (verification of all symbolic
        calculations).
    \item The matplotlib~\cite{Droettboom2018} library (visualisation).
    \item The pandas~\cite{Structures2010}, dask~\cite{Dask2016} and
        NumPy~\cite{Oliphant2015} libraries (data manipulation).
    \item The SciPy~\cite{Jones2001} library (numerical integration of the
        replicator equation).
\end{itemize}

This work was performed using the computational facilities of the Advanced
Research Computing @ Cardiff (ARCCA) Division, Cardiff University.

\printbibliography

\newpage
\section*{Supplementary materials}

\includepdf{assets/pdf/proof_of_form_of_extortionate_strategies/main.pdf}

\newpage

Using the pair wise interactions the transition rates \(p,
q\) can be measured and the steady state probabilities inferred and compared to
the actual probabilities of each state.
This is done numerically by computing the singular eigenvector of the
matrix \(A\) \cite{Stewart2009}:

\[
    A =
    \begin{bmatrix}
        p_1 q_1 & p_1 (1 - q_1) & (1 - p_1) q_1 & (1 -p_1) (1 - q_1) \\
        p_2 q_2 & p_2 (1 - q_2) & (1 - p_2) q_2 & (1 -p_2) (1 - q_2) \\
        p_3 q_3 & p_3 (1 - q_3) & (1 - p_3) q_3 & (1 -p_3) (1 - q_3) \\
        p_4 q_4 & p_4 (1 - q_4) & (1 - p_4) q_4 & (1 -p_4) (1 - q_4) \\
    \end{bmatrix}
\]

Figure~\ref{fig:computed_probabilities_vs_theoretic_probabilities} shows a
regression line fitted to every pairwise interaction with a reported
\(\text{SSError}\) value (pairwise interactions with missing states were
omitted). This serves to validate the approach: a part from some edge cases the
relationship is consistent.

\begin{figure}[!htbp]
    \centering
    \includegraphics[width=.8\textwidth]{./assets/img/computed_probabilities_vs_theoretic_probabilities/main.pdf}
    \caption{The
        relationship between the steady state probabilities inferred from the
        measured transitions and the actual steady state probabilities. A linear
        regression line is included validating the approach.}
    \label{fig:computed_probabilities_vs_theoretic_probabilities}
\end{figure}


\end{document}
 strategies,
was presented with specific consideration given to ZD strategies. This
tournament is reproduced here using the Axelrod-Python
project~\cite{Knight2016}. To obtain a good measure of the corresponding
transition rates for each strategy all matches have been run for
\documentclass[a4paper]{article}

\usepackage{amsmath}
\usepackage{amssymb}
\usepackage[margin=1.5cm,
            includefoot,
            footskip=30pt]{geometry}
\usepackage{layout}
\usepackage{graphicx}
\usepackage{subcaption}

\usepackage{biblatex}
\usepackage{pdfpages}

\bibliography{main.bib}

\title{Suspicion: Recognising and evaluating the effectiveness
       of extortion in the Iterated Prisoner's Dilemma}
\author{Vincent A. Knight \and Nikoleta E. Glynatsi}
\date{\today}



\begin{document}

\maketitle

\begin{abstract}
    The Iterated Prisoner's Dilemma is a model for rational and evolutionary
    interactive behaviour. It has applications both in the study of human social
    behaviour as well as in biology.
    It is used to understand when and how a rational individual might
    accept an immediate cost to their own utility for the direct benefit of
    another.

    Much attention has been given to a class of strategies called
    Zero Determinant strategies. It has been theoretically shown that these
    strategies can ``extort'' any player.

    In this work, an approach to identify if observed strategies are playing in
    an extortionate way is described. Furthermore, experimental analysis of
    a large tournament with \documentclass[a4paper]{article}

\usepackage{amsmath}
\usepackage{amssymb}
\usepackage[margin=1.5cm,
            includefoot,
            footskip=30pt]{geometry}
\usepackage{layout}
\usepackage{graphicx}
\usepackage{subcaption}

\usepackage{biblatex}
\usepackage{pdfpages}

\bibliography{main.bib}

\title{Suspicion: Recognising and evaluating the effectiveness
       of extortion in the Iterated Prisoner's Dilemma}
\author{Vincent A. Knight \and Nikoleta E. Glynatsi}
\date{\today}



\begin{document}

\maketitle

\begin{abstract}
    The Iterated Prisoner's Dilemma is a model for rational and evolutionary
    interactive behaviour. It has applications both in the study of human social
    behaviour as well as in biology.
    It is used to understand when and how a rational individual might
    accept an immediate cost to their own utility for the direct benefit of
    another.

    Much attention has been given to a class of strategies called
    Zero Determinant strategies. It has been theoretically shown that these
    strategies can ``extort'' any player.

    In this work, an approach to identify if observed strategies are playing in
    an extortionate way is described. Furthermore, experimental analysis of
    a large tournament with \input{assets/tex/number_of_full_strategies/main.tex}
    strategies is considered. In this setting
    the most highly performing strategies do not play in an extortionate way
    against each other but do against lower performing strategies.
    This suggests that whilst the theory of Zero Determinant strategies
    indicates that memory is not of fundamental importance to the evolution of
    cooperative behaviour, this is incomplete.
\end{abstract}

\section{Introduction}\label{sec:introduction}

Agent based game theoretic models have become a stalwart of the underpinning
mathematics of interactive behaviours. One of the major pieces of work
in this area is the pair of original computer tournaments run by Robert
Axelrod~\cite{Axelrod1980, Axelrod1980a}. These tournaments pitted submitted
computer strategies against each other in plays of the Iterated Prisoner's
Dilemma. A common game where agents can choose to pay a slight cost to their
immediate utility in the hope of building a reputation. This has been used in
economic and evolutionary game theory to understand the evolution of cooperative
behaviour.

Recently, a class of strategies was described in~\cite{Press2012} that can
provably extort any given opponent. In~\cite{Hilbe2013, Moran1707} some
questions have already been asked about the true effectiveness of these
strategies in an evolutionary setting. Here another question is asked: is it
possible to recognise this extortionate behaviour? A mathematical procedure for
suspicion is presented: in the same way that the continued actions of an
extortionate individual might raise suspicion.

This work makes use of the Axelrod Python library~\cite{Knight2018, Knight2016}
with a large number of Prisoner Dilemma strategies available to give an
extensive numerical example of the ideas presented.  The approach is presented
in Section~\ref{sec:delta-zd-strategies}.  All of the code and data discussed
in Section~\ref{sec:numerical-experiments} is open sourced, archived and
written according to best scientific principles~\cite{Wilson2014}. The data
archive can be found at~\cite{vincent_knight_2018_1297075}.

\section{Recognising Extortion}\label{sec:delta-zd-strategies}

In~\cite{Press2012}, given a match between 2 memory-one strategies, the concept
of Zero Determinant (ZD) strategies is introduced. The main result of that paper
shows that given two memory one players \(p, q\in\mathbb{R}^4\) a linear
relationship between the players' scores could be forced by one of the players.

Using the notation of~\cite{Press2012}, assuming the utilities for player \(p\)
are given by \(S_x=(R, S, T, P)\) and for player \(q\) by \(S_y=(R, T, S, P)\)
and that the stationary scores of each player is given by \(S_X\) and \(S_Y\)
respectively. The main result of~\cite{Press2012} is that if

\begin{equation}\label{eqn:linear_relationship_for_p}
    \tilde p=\alpha S_x + \beta S_y + \gamma
\end{equation}

or

\begin{equation}\label{eqn:linear_relationship_for_q}
    \tilde q=\alpha S_x + \beta S_y + \gamma
\end{equation}

where \(\tilde p = (1 - p_1, 1 - p_2, p_3, p_4)\) and
\(\tilde q = (1 - q_1, 1 - q_2, q_3, q_4)\) then:

\begin{equation}
    \alpha S_X + \beta S_Y + \gamma = 0
\end{equation}

In~\cite{Press2012} a particular type of ZD strategy is defined: extortionate
strategies. If:

\begin{equation}\label{eqn:constraint_for_extortion}
    \gamma = - P(\alpha + \beta)
\end{equation}

then the player can ensure they get a score \(\chi\) times
larger than the opponent. This extortion coefficient is given by:

\begin{equation}\label{eqn:definition_of_chi}
    \chi=\frac{-\beta}{\alpha}
\end{equation}

Thus, if (\ref{eqn:constraint_for_extortion}) holds and \(\chi >1\) a player is
said to extort their opponent.
Here, the reverse problem is considered: given a
\(p\in\mathbb{R}^4\) how does one identify \(\alpha, \beta\) if they
exist and is the strategy in fact acting in an extortionate way?

These conditions correspond to:

\begin{align}
    \tilde p_1 & = \alpha R + \beta R - P (\alpha + \beta)
            \label{eqn:condition_for_tilde_p1}\\
    \tilde p_2 & = \alpha S + \beta T - P (\alpha + \beta)
            \label{eqn:condition_for_tilde_p2}\\
    \tilde p_3 & = \alpha T + \beta S - P (\alpha + \beta)
            \label{eqn:condition_for_tilde_p3}\\
    \tilde p_4 & = \alpha P + \beta P - P (\alpha + \beta)
            \label{eqn:condition_for_tilde_p4}
\end{align}

Equation (\ref{eqn:condition_for_tilde_p4}) ensures that \(p_4=\tilde p_4=0\).
Equations (\ref{eqn:condition_for_tilde_p1}-\ref{eqn:condition_for_tilde_p3})
can be used to eliminate \(\alpha, \beta\), giving:

\begin{equation}\label{eqn:planar_definition_of_extortion}
    \tilde p_1 = \frac{(R - P)(\tilde p_2 + \tilde p_3)}{S + T - 2P}
\end{equation}

with:

\begin{equation}\label{eqn:definition_of_chi}
    \chi = \frac{\tilde p_2 (P - T) + \tilde p_3 (S - P)}
                {\tilde p_2 (P - S) + \tilde p_3 (T - P)}
\end{equation}

Given a strategy \(p\in\mathbb{R}^{4\times 1}\) equations
(\ref{eqn:condition_for_tilde_p4}), (\ref{eqn:planar_definition_of_extortion}-\ref{eqn:definition_of_chi}) can be used to check if
a strategy is extortionate. The conditions correspond to:

\begin{align}
    p_1 & = \frac{(R-P)(p_2 + p_3) - R + T + S - P}{S + T - 2P}
     \label{eqn:condition_for_p1}\\
    p_4 & = 0 \label{eqn:condition_for_p4}\\
    1 & > p_2 + p_3\label{eqn:condition_for_chi}
\end{align}

The algebraic steps necessary to prove these results are available in the
supporting materials.

All extortionate strategies reside on a triangular (\ref{eqn:condition_for_chi})
plane (\ref{eqn:condition_for_p1}) in 3 dimensions (\ref{eqn:condition_for_p4}).
Using this formulation it can be seen that a necessary (but not sufficient)
condition for an extortionate strategy is that it cooperates on average less
than 50\% of the time when in a state of disagreement with the opponent.

As an example, consider the known extortionate strategy \(p=(8 / 9, 1 / 2, 1 /
3, 0)\) from~\cite{Stewart2012} which is referred to as \texttt{Extort-2}. In
this case, for the standard values of \((R, T, S, P)\) constraint
(\ref{eqn:condition_for_p1}) corresponds to:

\begin{equation}
    p_1 = \frac{2(p_2 + p_3) + 1}{3}
\end{equation}

It is clear that in this case all constraints hold.

This approach could in fact be used to confirm that a given strategy is acting
in an extortionate manner even if it is not a memory one strategy. However, in
practice, if a closed form for \(p\) is not known, then due to measurement
and/or numerical error this would not work.

This problem can be written in the following linear algebraic form where
\(x=(\alpha, \beta)\)
and \(p^*=(\tilde p_1 - 1, tilde_2 - 1, p_3)\):

\begin{equation}\label{eqn:linear_algebraic_equation_for_p}
    Cx= p^*
\end{equation}

\(C\) corresponds to equations
(\ref{eqn:condition_for_tilde_p1}-\ref{eqn:condition_for_tilde_p3}) and is
given by:

\begin{equation}\label{eqn:definition_of_C}
    C =
    \begin{bmatrix}
        R - P & R- P \\
        S - P & T- P \\
        T - P & S- P \\
    \end{bmatrix}
\end{equation}

Note that in general, equation (\ref{eqn:linear_algebraic_equation_for_p}) will
not necessarily have a solution. From the Rouch\'{e}-Capelli theorem if there is
a solution it is unique as \(\text{rank}(C)=2\) which is the dimension of the
variable \(x\). The best fitting \(x\) is found by minimizing:

\begin{equation}\label{eqn:r_squared}
    \text{SSError} = \|C x- p^*\|_2^2 = \sum_{i=1}^{3}\left((C\bar x)_i-p_i^*\right)^2
\end{equation}

Note that \(\text{SSError}\), which is the square of the Frobenius
norm~\cite{Golub2013}, becomes a measure of how close a strategy is to being an
extortionate strategy. Suspicion
of extortion then corresponds to a threshold on \(\text{SSError}\).

By observing interactions (human or otherwise), their memory one representation
can be inferred and this approach can be used to recognise extortionate
behaviour. The notion of comparing theoretic and actual plays of the IPD is not
novel, see for example~\cite{Rand2013}. Immediately it is noted that if the
environment is noisy~\cite{Wu1995} then no strategy can be considered to be
extortionate as \(p_4>0\).

In the next section, this idea will be illustrated by observing the interactions
that take place in a computer based tournament of the IPD\@.

\section{Numerical experiments}\label{sec:numerical-experiments}

In~\cite{Stewart2012} results from a tournament with
\input{./assets/tex/number_of_stewart_plotkin_strategies/main.tex} strategies,
was presented with specific consideration given to ZD strategies. This
tournament is reproduced here using the Axelrod-Python
project~\cite{Knight2016}. To obtain a good measure of the corresponding
transition rates for each strategy all matches have been run for
\input{assets/tex/number_of_turns/main.tex} turns and every match has been
repeated \input{assets/tex/number_of_repetitions/main.tex} times. All of this
interaction data is available at~\cite{vincent_knight_2018_1297075}. A good
match between the inferred Markov chain and the state distribution of the actual
interactions has been verified. Data for this is presented in the supplementary
materials.

Figure~\ref{fig:SSError_overall_in_stewart_plotkin} shows the \(\text{SSError}\)
values for all the strategies in the tournament, as reported
in~\cite{Stewart2012} the extortionate strategy (which has an expected
\(\text{SSError}\) approximately 0) gains a large number of wins.

\begin{figure}[!htbp]
    \centering
    \includegraphics[width=.8\textwidth]{./assets/img/SSError_overall_in_stewart_plotkin/main.pdf}
    \caption{\(\text{SSError}\) and state probabilities for the strategies
        of~\cite{Stewart2012}, ordered both by number of wins and overall score.
        Note that \(P(DC)\) is not shown as it corresponds to the transpose of
        \(P(CD)\). Cooperator and Defector are omitted as they do not visit all
        the states.}
    \label{fig:SSError_overall_in_stewart_plotkin}
\end{figure}

Here, the work of~\cite{Stewart2012} is extended by investigating a tournament
with \input{assets/tex/number_of_full_strategies/main.tex}
strategies.

The results of this analysis are shown in
Figure~\ref{fig:SSError_and_probabilities_in_full}. The top ranking strategies
by number of wins seem to be extortionate (but not against all strategies) and
it can be seen that a small sub group of strategies achieve mutual defection.
All the top ranking strategies according to score achieve mutual cooperation and
do not extort each other, however they
\textbf{do} exhibit extortionate behaviour towards a number of the lower ranking
strategies.

\begin{figure}[!htbp]
    \centering
    \includegraphics[width=.8\textwidth]{./assets/img/SSError_and_probabilities_in_full/main.pdf}
    \caption{\(\text{SSError}\) for the strategies for the full tournament. Only
    strategy interactions for which \(p_4=0\) and \(\chi>1\) are displayed.}
    \label{fig:SSError_and_probabilities_in_full}
\end{figure}

\section{Conclusion}\label{sec:conclusion}

This work defines an approach to measure whether or not a player is playing a
strategy that corresponds to an extortionate strategy as defined
in~\cite{Press2012}: a mathematical model for suspicion. Indeed, all
extortionate strategies have been
 classified as lying on a triangular plane.
This rigorous classification fails to be robust to small measurement error, thus
a statistical approach is proposed.
This is done through a linear algebraic approach for approximating the solution
of a linear system. Using this, a large number of pairwise interactions is
simulated and in fact very few strategies are found to act extortionately.

The work of~\cite{Press2012}, whilst showing that a clever approach to taking
advantage of another memory one strategy exists: this is incomplete. Whilst the
elegance of this result is very attractive, just as the simplicity of the
victory of Tit For Tat in Axelrod's original tournaments was, it is incomplete.
Extortionate strategies achieve a high number of wins but they do not
achieve a high score which corresponds to the fitness landscape in an
evolutionary sense. From the large number of interactions a payoff matrix \(S\)
can be measured where \(S_{ij}\) denotes the score (using standard values of
\((R, S, T, P) = (3, 0, 5, 1)\)) of the \(i\)th strategy
against the \(j\)th strategy. Using this, the replicator equation
describes the evolution of the system based on a population density fitness
function:

\begin{equation}\label{eqn:replicator_dynamics}
    \frac{dx}{dt} = x(S-x^TS x)
\end{equation}

Equation (\ref{eqn:replicator_dynamics}) is solved numerically through an
integration technique described in~\cite{Petzold1983} and
Figure~\ref{fig:replicator_dynamics} shows the evolution of the distribution of
the system: the various strategies are ranked by scores. It is clear to see that
only the high ranking strategies survive the evolutionary process (in fact,
only \input{./assets/img/replicator_dynamics/main.tex}
have a final distribution greater than \(10 ^ {-2}\)). This confirms the
findings of~\cite{Moran1707} in which sophisticated strategies resist
evolutionary invasion of shorter memory strategies. Recalling
Figure~\ref{fig:SSError_and_probabilities_in_full} this demonstrates that:

\begin{itemize}
    \item Cooperation emerges through the evolutionary process: the high scoring
        strategies do not exhibit extortionate behaviour towards each other.
    \item Extortionate strategies do not survive the evolutionary process.
\end{itemize}

\begin{figure}[!htbp]
    \centering
    \includegraphics[width=.8\textwidth]{./assets/img/replicator_dynamics/main.pdf}
    \caption{Numerical simulation of the replicator equation
    (\ref{eqn:replicator_dynamics}): strategies are ordered by score, only the strategies with a high score survive the evolutionary process.}
    \label{fig:replicator_dynamics}
\end{figure}

This work can be used to classify plays of the IPD\@: data can be collected from
actual interactions (in lab or in the field). Furthermore, this allows for a
classification method similar to the notion of fingerprinting presented
in~\cite{Ashlock2008}. Trained strategies can potentially be classified as
extortionate or not or it could be possible to even constrain the reinforcement
learning approaches that are becoming prevalent in the literature.
Alternatively, this mathematical approach for recognising extortion could be
used in sophisticated strategies to defend against invasion. Arguably, some of
the strategies considered here exhibit this behaviour, indeed as described
in~\cite{Harper2017}, the top ranking strategies in the full tournament are
obtained using evolutionary reinforcement learning techniques, thus, suspicion
of extortionate behaviour could in fact be an evolutionary trait.

\section*{Acknowledgements}

The following open source software libraries were used in this research:

\begin{itemize}
    \item The Axelrod ~\cite{Knight2016, Knight2018} library (IPD strategies and
        tournaments).
    \item The sympy library~\cite{Meurer2017} (verification of all symbolic
        calculations).
    \item The matplotlib~\cite{Droettboom2018} library (visualisation).
    \item The pandas~\cite{Structures2010}, dask~\cite{Dask2016} and
        NumPy~\cite{Oliphant2015} libraries (data manipulation).
    \item The SciPy~\cite{Jones2001} library (numerical integration of the
        replicator equation).
\end{itemize}

This work was performed using the computational facilities of the Advanced
Research Computing @ Cardiff (ARCCA) Division, Cardiff University.

\printbibliography

\newpage
\section*{Supplementary materials}

\includepdf{assets/pdf/proof_of_form_of_extortionate_strategies/main.pdf}

\newpage

Using the pair wise interactions the transition rates \(p,
q\) can be measured and the steady state probabilities inferred and compared to
the actual probabilities of each state.
This is done numerically by computing the singular eigenvector of the
matrix \(A\) \cite{Stewart2009}:

\[
    A =
    \begin{bmatrix}
        p_1 q_1 & p_1 (1 - q_1) & (1 - p_1) q_1 & (1 -p_1) (1 - q_1) \\
        p_2 q_2 & p_2 (1 - q_2) & (1 - p_2) q_2 & (1 -p_2) (1 - q_2) \\
        p_3 q_3 & p_3 (1 - q_3) & (1 - p_3) q_3 & (1 -p_3) (1 - q_3) \\
        p_4 q_4 & p_4 (1 - q_4) & (1 - p_4) q_4 & (1 -p_4) (1 - q_4) \\
    \end{bmatrix}
\]

Figure~\ref{fig:computed_probabilities_vs_theoretic_probabilities} shows a
regression line fitted to every pairwise interaction with a reported
\(\text{SSError}\) value (pairwise interactions with missing states were
omitted). This serves to validate the approach: a part from some edge cases the
relationship is consistent.

\begin{figure}[!htbp]
    \centering
    \includegraphics[width=.8\textwidth]{./assets/img/computed_probabilities_vs_theoretic_probabilities/main.pdf}
    \caption{The
        relationship between the steady state probabilities inferred from the
        measured transitions and the actual steady state probabilities. A linear
        regression line is included validating the approach.}
    \label{fig:computed_probabilities_vs_theoretic_probabilities}
\end{figure}


\end{document}

    strategies is considered. In this setting
    the most highly performing strategies do not play in an extortionate way
    against each other but do against lower performing strategies.
    This suggests that whilst the theory of Zero Determinant strategies
    indicates that memory is not of fundamental importance to the evolution of
    cooperative behaviour, this is incomplete.
\end{abstract}

\section{Introduction}\label{sec:introduction}

Agent based game theoretic models have become a stalwart of the underpinning
mathematics of interactive behaviours. One of the major pieces of work
in this area is the pair of original computer tournaments run by Robert
Axelrod~\cite{Axelrod1980, Axelrod1980a}. These tournaments pitted submitted
computer strategies against each other in plays of the Iterated Prisoner's
Dilemma. A common game where agents can choose to pay a slight cost to their
immediate utility in the hope of building a reputation. This has been used in
economic and evolutionary game theory to understand the evolution of cooperative
behaviour.

Recently, a class of strategies was described in~\cite{Press2012} that can
provably extort any given opponent. In~\cite{Hilbe2013, Moran1707} some
questions have already been asked about the true effectiveness of these
strategies in an evolutionary setting. Here another question is asked: is it
possible to recognise this extortionate behaviour? A mathematical procedure for
suspicion is presented: in the same way that the continued actions of an
extortionate individual might raise suspicion.

This work makes use of the Axelrod Python library~\cite{Knight2018, Knight2016}
with a large number of Prisoner Dilemma strategies available to give an
extensive numerical example of the ideas presented.  The approach is presented
in Section~\ref{sec:delta-zd-strategies}.  All of the code and data discussed
in Section~\ref{sec:numerical-experiments} is open sourced, archived and
written according to best scientific principles~\cite{Wilson2014}. The data
archive can be found at~\cite{vincent_knight_2018_1297075}.

\section{Recognising Extortion}\label{sec:delta-zd-strategies}

In~\cite{Press2012}, given a match between 2 memory-one strategies, the concept
of Zero Determinant (ZD) strategies is introduced. The main result of that paper
shows that given two memory one players \(p, q\in\mathbb{R}^4\) a linear
relationship between the players' scores could be forced by one of the players.

Using the notation of~\cite{Press2012}, assuming the utilities for player \(p\)
are given by \(S_x=(R, S, T, P)\) and for player \(q\) by \(S_y=(R, T, S, P)\)
and that the stationary scores of each player is given by \(S_X\) and \(S_Y\)
respectively. The main result of~\cite{Press2012} is that if

\begin{equation}\label{eqn:linear_relationship_for_p}
    \tilde p=\alpha S_x + \beta S_y + \gamma
\end{equation}

or

\begin{equation}\label{eqn:linear_relationship_for_q}
    \tilde q=\alpha S_x + \beta S_y + \gamma
\end{equation}

where \(\tilde p = (1 - p_1, 1 - p_2, p_3, p_4)\) and
\(\tilde q = (1 - q_1, 1 - q_2, q_3, q_4)\) then:

\begin{equation}
    \alpha S_X + \beta S_Y + \gamma = 0
\end{equation}

In~\cite{Press2012} a particular type of ZD strategy is defined: extortionate
strategies. If:

\begin{equation}\label{eqn:constraint_for_extortion}
    \gamma = - P(\alpha + \beta)
\end{equation}

then the player can ensure they get a score \(\chi\) times
larger than the opponent. This extortion coefficient is given by:

\begin{equation}\label{eqn:definition_of_chi}
    \chi=\frac{-\beta}{\alpha}
\end{equation}

Thus, if (\ref{eqn:constraint_for_extortion}) holds and \(\chi >1\) a player is
said to extort their opponent.
Here, the reverse problem is considered: given a
\(p\in\mathbb{R}^4\) how does one identify \(\alpha, \beta\) if they
exist and is the strategy in fact acting in an extortionate way?

These conditions correspond to:

\begin{align}
    \tilde p_1 & = \alpha R + \beta R - P (\alpha + \beta)
            \label{eqn:condition_for_tilde_p1}\\
    \tilde p_2 & = \alpha S + \beta T - P (\alpha + \beta)
            \label{eqn:condition_for_tilde_p2}\\
    \tilde p_3 & = \alpha T + \beta S - P (\alpha + \beta)
            \label{eqn:condition_for_tilde_p3}\\
    \tilde p_4 & = \alpha P + \beta P - P (\alpha + \beta)
            \label{eqn:condition_for_tilde_p4}
\end{align}

Equation (\ref{eqn:condition_for_tilde_p4}) ensures that \(p_4=\tilde p_4=0\).
Equations (\ref{eqn:condition_for_tilde_p1}-\ref{eqn:condition_for_tilde_p3})
can be used to eliminate \(\alpha, \beta\), giving:

\begin{equation}\label{eqn:planar_definition_of_extortion}
    \tilde p_1 = \frac{(R - P)(\tilde p_2 + \tilde p_3)}{S + T - 2P}
\end{equation}

with:

\begin{equation}\label{eqn:definition_of_chi}
    \chi = \frac{\tilde p_2 (P - T) + \tilde p_3 (S - P)}
                {\tilde p_2 (P - S) + \tilde p_3 (T - P)}
\end{equation}

Given a strategy \(p\in\mathbb{R}^{4\times 1}\) equations
(\ref{eqn:condition_for_tilde_p4}), (\ref{eqn:planar_definition_of_extortion}-\ref{eqn:definition_of_chi}) can be used to check if
a strategy is extortionate. The conditions correspond to:

\begin{align}
    p_1 & = \frac{(R-P)(p_2 + p_3) - R + T + S - P}{S + T - 2P}
     \label{eqn:condition_for_p1}\\
    p_4 & = 0 \label{eqn:condition_for_p4}\\
    1 & > p_2 + p_3\label{eqn:condition_for_chi}
\end{align}

The algebraic steps necessary to prove these results are available in the
supporting materials.

All extortionate strategies reside on a triangular (\ref{eqn:condition_for_chi})
plane (\ref{eqn:condition_for_p1}) in 3 dimensions (\ref{eqn:condition_for_p4}).
Using this formulation it can be seen that a necessary (but not sufficient)
condition for an extortionate strategy is that it cooperates on average less
than 50\% of the time when in a state of disagreement with the opponent.

As an example, consider the known extortionate strategy \(p=(8 / 9, 1 / 2, 1 /
3, 0)\) from~\cite{Stewart2012} which is referred to as \texttt{Extort-2}. In
this case, for the standard values of \((R, T, S, P)\) constraint
(\ref{eqn:condition_for_p1}) corresponds to:

\begin{equation}
    p_1 = \frac{2(p_2 + p_3) + 1}{3}
\end{equation}

It is clear that in this case all constraints hold.

This approach could in fact be used to confirm that a given strategy is acting
in an extortionate manner even if it is not a memory one strategy. However, in
practice, if a closed form for \(p\) is not known, then due to measurement
and/or numerical error this would not work.

This problem can be written in the following linear algebraic form where
\(x=(\alpha, \beta)\)
and \(p^*=(\tilde p_1 - 1, tilde_2 - 1, p_3)\):

\begin{equation}\label{eqn:linear_algebraic_equation_for_p}
    Cx= p^*
\end{equation}

\(C\) corresponds to equations
(\ref{eqn:condition_for_tilde_p1}-\ref{eqn:condition_for_tilde_p3}) and is
given by:

\begin{equation}\label{eqn:definition_of_C}
    C =
    \begin{bmatrix}
        R - P & R- P \\
        S - P & T- P \\
        T - P & S- P \\
    \end{bmatrix}
\end{equation}

Note that in general, equation (\ref{eqn:linear_algebraic_equation_for_p}) will
not necessarily have a solution. From the Rouch\'{e}-Capelli theorem if there is
a solution it is unique as \(\text{rank}(C)=2\) which is the dimension of the
variable \(x\). The best fitting \(x\) is found by minimizing:

\begin{equation}\label{eqn:r_squared}
    \text{SSError} = \|C x- p^*\|_2^2 = \sum_{i=1}^{3}\left((C\bar x)_i-p_i^*\right)^2
\end{equation}

Note that \(\text{SSError}\), which is the square of the Frobenius
norm~\cite{Golub2013}, becomes a measure of how close a strategy is to being an
extortionate strategy. Suspicion
of extortion then corresponds to a threshold on \(\text{SSError}\).

By observing interactions (human or otherwise), their memory one representation
can be inferred and this approach can be used to recognise extortionate
behaviour. The notion of comparing theoretic and actual plays of the IPD is not
novel, see for example~\cite{Rand2013}. Immediately it is noted that if the
environment is noisy~\cite{Wu1995} then no strategy can be considered to be
extortionate as \(p_4>0\).

In the next section, this idea will be illustrated by observing the interactions
that take place in a computer based tournament of the IPD\@.

\section{Numerical experiments}\label{sec:numerical-experiments}

In~\cite{Stewart2012} results from a tournament with
\documentclass[a4paper]{article}

\usepackage{amsmath}
\usepackage{amssymb}
\usepackage[margin=1.5cm,
            includefoot,
            footskip=30pt]{geometry}
\usepackage{layout}
\usepackage{graphicx}
\usepackage{subcaption}

\usepackage{biblatex}
\usepackage{pdfpages}

\bibliography{main.bib}

\title{Suspicion: Recognising and evaluating the effectiveness
       of extortion in the Iterated Prisoner's Dilemma}
\author{Vincent A. Knight \and Nikoleta E. Glynatsi}
\date{\today}



\begin{document}

\maketitle

\begin{abstract}
    The Iterated Prisoner's Dilemma is a model for rational and evolutionary
    interactive behaviour. It has applications both in the study of human social
    behaviour as well as in biology.
    It is used to understand when and how a rational individual might
    accept an immediate cost to their own utility for the direct benefit of
    another.

    Much attention has been given to a class of strategies called
    Zero Determinant strategies. It has been theoretically shown that these
    strategies can ``extort'' any player.

    In this work, an approach to identify if observed strategies are playing in
    an extortionate way is described. Furthermore, experimental analysis of
    a large tournament with \input{assets/tex/number_of_full_strategies/main.tex}
    strategies is considered. In this setting
    the most highly performing strategies do not play in an extortionate way
    against each other but do against lower performing strategies.
    This suggests that whilst the theory of Zero Determinant strategies
    indicates that memory is not of fundamental importance to the evolution of
    cooperative behaviour, this is incomplete.
\end{abstract}

\section{Introduction}\label{sec:introduction}

Agent based game theoretic models have become a stalwart of the underpinning
mathematics of interactive behaviours. One of the major pieces of work
in this area is the pair of original computer tournaments run by Robert
Axelrod~\cite{Axelrod1980, Axelrod1980a}. These tournaments pitted submitted
computer strategies against each other in plays of the Iterated Prisoner's
Dilemma. A common game where agents can choose to pay a slight cost to their
immediate utility in the hope of building a reputation. This has been used in
economic and evolutionary game theory to understand the evolution of cooperative
behaviour.

Recently, a class of strategies was described in~\cite{Press2012} that can
provably extort any given opponent. In~\cite{Hilbe2013, Moran1707} some
questions have already been asked about the true effectiveness of these
strategies in an evolutionary setting. Here another question is asked: is it
possible to recognise this extortionate behaviour? A mathematical procedure for
suspicion is presented: in the same way that the continued actions of an
extortionate individual might raise suspicion.

This work makes use of the Axelrod Python library~\cite{Knight2018, Knight2016}
with a large number of Prisoner Dilemma strategies available to give an
extensive numerical example of the ideas presented.  The approach is presented
in Section~\ref{sec:delta-zd-strategies}.  All of the code and data discussed
in Section~\ref{sec:numerical-experiments} is open sourced, archived and
written according to best scientific principles~\cite{Wilson2014}. The data
archive can be found at~\cite{vincent_knight_2018_1297075}.

\section{Recognising Extortion}\label{sec:delta-zd-strategies}

In~\cite{Press2012}, given a match between 2 memory-one strategies, the concept
of Zero Determinant (ZD) strategies is introduced. The main result of that paper
shows that given two memory one players \(p, q\in\mathbb{R}^4\) a linear
relationship between the players' scores could be forced by one of the players.

Using the notation of~\cite{Press2012}, assuming the utilities for player \(p\)
are given by \(S_x=(R, S, T, P)\) and for player \(q\) by \(S_y=(R, T, S, P)\)
and that the stationary scores of each player is given by \(S_X\) and \(S_Y\)
respectively. The main result of~\cite{Press2012} is that if

\begin{equation}\label{eqn:linear_relationship_for_p}
    \tilde p=\alpha S_x + \beta S_y + \gamma
\end{equation}

or

\begin{equation}\label{eqn:linear_relationship_for_q}
    \tilde q=\alpha S_x + \beta S_y + \gamma
\end{equation}

where \(\tilde p = (1 - p_1, 1 - p_2, p_3, p_4)\) and
\(\tilde q = (1 - q_1, 1 - q_2, q_3, q_4)\) then:

\begin{equation}
    \alpha S_X + \beta S_Y + \gamma = 0
\end{equation}

In~\cite{Press2012} a particular type of ZD strategy is defined: extortionate
strategies. If:

\begin{equation}\label{eqn:constraint_for_extortion}
    \gamma = - P(\alpha + \beta)
\end{equation}

then the player can ensure they get a score \(\chi\) times
larger than the opponent. This extortion coefficient is given by:

\begin{equation}\label{eqn:definition_of_chi}
    \chi=\frac{-\beta}{\alpha}
\end{equation}

Thus, if (\ref{eqn:constraint_for_extortion}) holds and \(\chi >1\) a player is
said to extort their opponent.
Here, the reverse problem is considered: given a
\(p\in\mathbb{R}^4\) how does one identify \(\alpha, \beta\) if they
exist and is the strategy in fact acting in an extortionate way?

These conditions correspond to:

\begin{align}
    \tilde p_1 & = \alpha R + \beta R - P (\alpha + \beta)
            \label{eqn:condition_for_tilde_p1}\\
    \tilde p_2 & = \alpha S + \beta T - P (\alpha + \beta)
            \label{eqn:condition_for_tilde_p2}\\
    \tilde p_3 & = \alpha T + \beta S - P (\alpha + \beta)
            \label{eqn:condition_for_tilde_p3}\\
    \tilde p_4 & = \alpha P + \beta P - P (\alpha + \beta)
            \label{eqn:condition_for_tilde_p4}
\end{align}

Equation (\ref{eqn:condition_for_tilde_p4}) ensures that \(p_4=\tilde p_4=0\).
Equations (\ref{eqn:condition_for_tilde_p1}-\ref{eqn:condition_for_tilde_p3})
can be used to eliminate \(\alpha, \beta\), giving:

\begin{equation}\label{eqn:planar_definition_of_extortion}
    \tilde p_1 = \frac{(R - P)(\tilde p_2 + \tilde p_3)}{S + T - 2P}
\end{equation}

with:

\begin{equation}\label{eqn:definition_of_chi}
    \chi = \frac{\tilde p_2 (P - T) + \tilde p_3 (S - P)}
                {\tilde p_2 (P - S) + \tilde p_3 (T - P)}
\end{equation}

Given a strategy \(p\in\mathbb{R}^{4\times 1}\) equations
(\ref{eqn:condition_for_tilde_p4}), (\ref{eqn:planar_definition_of_extortion}-\ref{eqn:definition_of_chi}) can be used to check if
a strategy is extortionate. The conditions correspond to:

\begin{align}
    p_1 & = \frac{(R-P)(p_2 + p_3) - R + T + S - P}{S + T - 2P}
     \label{eqn:condition_for_p1}\\
    p_4 & = 0 \label{eqn:condition_for_p4}\\
    1 & > p_2 + p_3\label{eqn:condition_for_chi}
\end{align}

The algebraic steps necessary to prove these results are available in the
supporting materials.

All extortionate strategies reside on a triangular (\ref{eqn:condition_for_chi})
plane (\ref{eqn:condition_for_p1}) in 3 dimensions (\ref{eqn:condition_for_p4}).
Using this formulation it can be seen that a necessary (but not sufficient)
condition for an extortionate strategy is that it cooperates on average less
than 50\% of the time when in a state of disagreement with the opponent.

As an example, consider the known extortionate strategy \(p=(8 / 9, 1 / 2, 1 /
3, 0)\) from~\cite{Stewart2012} which is referred to as \texttt{Extort-2}. In
this case, for the standard values of \((R, T, S, P)\) constraint
(\ref{eqn:condition_for_p1}) corresponds to:

\begin{equation}
    p_1 = \frac{2(p_2 + p_3) + 1}{3}
\end{equation}

It is clear that in this case all constraints hold.

This approach could in fact be used to confirm that a given strategy is acting
in an extortionate manner even if it is not a memory one strategy. However, in
practice, if a closed form for \(p\) is not known, then due to measurement
and/or numerical error this would not work.

This problem can be written in the following linear algebraic form where
\(x=(\alpha, \beta)\)
and \(p^*=(\tilde p_1 - 1, tilde_2 - 1, p_3)\):

\begin{equation}\label{eqn:linear_algebraic_equation_for_p}
    Cx= p^*
\end{equation}

\(C\) corresponds to equations
(\ref{eqn:condition_for_tilde_p1}-\ref{eqn:condition_for_tilde_p3}) and is
given by:

\begin{equation}\label{eqn:definition_of_C}
    C =
    \begin{bmatrix}
        R - P & R- P \\
        S - P & T- P \\
        T - P & S- P \\
    \end{bmatrix}
\end{equation}

Note that in general, equation (\ref{eqn:linear_algebraic_equation_for_p}) will
not necessarily have a solution. From the Rouch\'{e}-Capelli theorem if there is
a solution it is unique as \(\text{rank}(C)=2\) which is the dimension of the
variable \(x\). The best fitting \(x\) is found by minimizing:

\begin{equation}\label{eqn:r_squared}
    \text{SSError} = \|C x- p^*\|_2^2 = \sum_{i=1}^{3}\left((C\bar x)_i-p_i^*\right)^2
\end{equation}

Note that \(\text{SSError}\), which is the square of the Frobenius
norm~\cite{Golub2013}, becomes a measure of how close a strategy is to being an
extortionate strategy. Suspicion
of extortion then corresponds to a threshold on \(\text{SSError}\).

By observing interactions (human or otherwise), their memory one representation
can be inferred and this approach can be used to recognise extortionate
behaviour. The notion of comparing theoretic and actual plays of the IPD is not
novel, see for example~\cite{Rand2013}. Immediately it is noted that if the
environment is noisy~\cite{Wu1995} then no strategy can be considered to be
extortionate as \(p_4>0\).

In the next section, this idea will be illustrated by observing the interactions
that take place in a computer based tournament of the IPD\@.

\section{Numerical experiments}\label{sec:numerical-experiments}

In~\cite{Stewart2012} results from a tournament with
\input{./assets/tex/number_of_stewart_plotkin_strategies/main.tex} strategies,
was presented with specific consideration given to ZD strategies. This
tournament is reproduced here using the Axelrod-Python
project~\cite{Knight2016}. To obtain a good measure of the corresponding
transition rates for each strategy all matches have been run for
\input{assets/tex/number_of_turns/main.tex} turns and every match has been
repeated \input{assets/tex/number_of_repetitions/main.tex} times. All of this
interaction data is available at~\cite{vincent_knight_2018_1297075}. A good
match between the inferred Markov chain and the state distribution of the actual
interactions has been verified. Data for this is presented in the supplementary
materials.

Figure~\ref{fig:SSError_overall_in_stewart_plotkin} shows the \(\text{SSError}\)
values for all the strategies in the tournament, as reported
in~\cite{Stewart2012} the extortionate strategy (which has an expected
\(\text{SSError}\) approximately 0) gains a large number of wins.

\begin{figure}[!htbp]
    \centering
    \includegraphics[width=.8\textwidth]{./assets/img/SSError_overall_in_stewart_plotkin/main.pdf}
    \caption{\(\text{SSError}\) and state probabilities for the strategies
        of~\cite{Stewart2012}, ordered both by number of wins and overall score.
        Note that \(P(DC)\) is not shown as it corresponds to the transpose of
        \(P(CD)\). Cooperator and Defector are omitted as they do not visit all
        the states.}
    \label{fig:SSError_overall_in_stewart_plotkin}
\end{figure}

Here, the work of~\cite{Stewart2012} is extended by investigating a tournament
with \input{assets/tex/number_of_full_strategies/main.tex}
strategies.

The results of this analysis are shown in
Figure~\ref{fig:SSError_and_probabilities_in_full}. The top ranking strategies
by number of wins seem to be extortionate (but not against all strategies) and
it can be seen that a small sub group of strategies achieve mutual defection.
All the top ranking strategies according to score achieve mutual cooperation and
do not extort each other, however they
\textbf{do} exhibit extortionate behaviour towards a number of the lower ranking
strategies.

\begin{figure}[!htbp]
    \centering
    \includegraphics[width=.8\textwidth]{./assets/img/SSError_and_probabilities_in_full/main.pdf}
    \caption{\(\text{SSError}\) for the strategies for the full tournament. Only
    strategy interactions for which \(p_4=0\) and \(\chi>1\) are displayed.}
    \label{fig:SSError_and_probabilities_in_full}
\end{figure}

\section{Conclusion}\label{sec:conclusion}

This work defines an approach to measure whether or not a player is playing a
strategy that corresponds to an extortionate strategy as defined
in~\cite{Press2012}: a mathematical model for suspicion. Indeed, all
extortionate strategies have been
 classified as lying on a triangular plane.
This rigorous classification fails to be robust to small measurement error, thus
a statistical approach is proposed.
This is done through a linear algebraic approach for approximating the solution
of a linear system. Using this, a large number of pairwise interactions is
simulated and in fact very few strategies are found to act extortionately.

The work of~\cite{Press2012}, whilst showing that a clever approach to taking
advantage of another memory one strategy exists: this is incomplete. Whilst the
elegance of this result is very attractive, just as the simplicity of the
victory of Tit For Tat in Axelrod's original tournaments was, it is incomplete.
Extortionate strategies achieve a high number of wins but they do not
achieve a high score which corresponds to the fitness landscape in an
evolutionary sense. From the large number of interactions a payoff matrix \(S\)
can be measured where \(S_{ij}\) denotes the score (using standard values of
\((R, S, T, P) = (3, 0, 5, 1)\)) of the \(i\)th strategy
against the \(j\)th strategy. Using this, the replicator equation
describes the evolution of the system based on a population density fitness
function:

\begin{equation}\label{eqn:replicator_dynamics}
    \frac{dx}{dt} = x(S-x^TS x)
\end{equation}

Equation (\ref{eqn:replicator_dynamics}) is solved numerically through an
integration technique described in~\cite{Petzold1983} and
Figure~\ref{fig:replicator_dynamics} shows the evolution of the distribution of
the system: the various strategies are ranked by scores. It is clear to see that
only the high ranking strategies survive the evolutionary process (in fact,
only \input{./assets/img/replicator_dynamics/main.tex}
have a final distribution greater than \(10 ^ {-2}\)). This confirms the
findings of~\cite{Moran1707} in which sophisticated strategies resist
evolutionary invasion of shorter memory strategies. Recalling
Figure~\ref{fig:SSError_and_probabilities_in_full} this demonstrates that:

\begin{itemize}
    \item Cooperation emerges through the evolutionary process: the high scoring
        strategies do not exhibit extortionate behaviour towards each other.
    \item Extortionate strategies do not survive the evolutionary process.
\end{itemize}

\begin{figure}[!htbp]
    \centering
    \includegraphics[width=.8\textwidth]{./assets/img/replicator_dynamics/main.pdf}
    \caption{Numerical simulation of the replicator equation
    (\ref{eqn:replicator_dynamics}): strategies are ordered by score, only the strategies with a high score survive the evolutionary process.}
    \label{fig:replicator_dynamics}
\end{figure}

This work can be used to classify plays of the IPD\@: data can be collected from
actual interactions (in lab or in the field). Furthermore, this allows for a
classification method similar to the notion of fingerprinting presented
in~\cite{Ashlock2008}. Trained strategies can potentially be classified as
extortionate or not or it could be possible to even constrain the reinforcement
learning approaches that are becoming prevalent in the literature.
Alternatively, this mathematical approach for recognising extortion could be
used in sophisticated strategies to defend against invasion. Arguably, some of
the strategies considered here exhibit this behaviour, indeed as described
in~\cite{Harper2017}, the top ranking strategies in the full tournament are
obtained using evolutionary reinforcement learning techniques, thus, suspicion
of extortionate behaviour could in fact be an evolutionary trait.

\section*{Acknowledgements}

The following open source software libraries were used in this research:

\begin{itemize}
    \item The Axelrod ~\cite{Knight2016, Knight2018} library (IPD strategies and
        tournaments).
    \item The sympy library~\cite{Meurer2017} (verification of all symbolic
        calculations).
    \item The matplotlib~\cite{Droettboom2018} library (visualisation).
    \item The pandas~\cite{Structures2010}, dask~\cite{Dask2016} and
        NumPy~\cite{Oliphant2015} libraries (data manipulation).
    \item The SciPy~\cite{Jones2001} library (numerical integration of the
        replicator equation).
\end{itemize}

This work was performed using the computational facilities of the Advanced
Research Computing @ Cardiff (ARCCA) Division, Cardiff University.

\printbibliography

\newpage
\section*{Supplementary materials}

\includepdf{assets/pdf/proof_of_form_of_extortionate_strategies/main.pdf}

\newpage

Using the pair wise interactions the transition rates \(p,
q\) can be measured and the steady state probabilities inferred and compared to
the actual probabilities of each state.
This is done numerically by computing the singular eigenvector of the
matrix \(A\) \cite{Stewart2009}:

\[
    A =
    \begin{bmatrix}
        p_1 q_1 & p_1 (1 - q_1) & (1 - p_1) q_1 & (1 -p_1) (1 - q_1) \\
        p_2 q_2 & p_2 (1 - q_2) & (1 - p_2) q_2 & (1 -p_2) (1 - q_2) \\
        p_3 q_3 & p_3 (1 - q_3) & (1 - p_3) q_3 & (1 -p_3) (1 - q_3) \\
        p_4 q_4 & p_4 (1 - q_4) & (1 - p_4) q_4 & (1 -p_4) (1 - q_4) \\
    \end{bmatrix}
\]

Figure~\ref{fig:computed_probabilities_vs_theoretic_probabilities} shows a
regression line fitted to every pairwise interaction with a reported
\(\text{SSError}\) value (pairwise interactions with missing states were
omitted). This serves to validate the approach: a part from some edge cases the
relationship is consistent.

\begin{figure}[!htbp]
    \centering
    \includegraphics[width=.8\textwidth]{./assets/img/computed_probabilities_vs_theoretic_probabilities/main.pdf}
    \caption{The
        relationship between the steady state probabilities inferred from the
        measured transitions and the actual steady state probabilities. A linear
        regression line is included validating the approach.}
    \label{fig:computed_probabilities_vs_theoretic_probabilities}
\end{figure}


\end{document}
 strategies,
was presented with specific consideration given to ZD strategies. This
tournament is reproduced here using the Axelrod-Python
project~\cite{Knight2016}. To obtain a good measure of the corresponding
transition rates for each strategy all matches have been run for
\documentclass[a4paper]{article}

\usepackage{amsmath}
\usepackage{amssymb}
\usepackage[margin=1.5cm,
            includefoot,
            footskip=30pt]{geometry}
\usepackage{layout}
\usepackage{graphicx}
\usepackage{subcaption}

\usepackage{biblatex}
\usepackage{pdfpages}

\bibliography{main.bib}

\title{Suspicion: Recognising and evaluating the effectiveness
       of extortion in the Iterated Prisoner's Dilemma}
\author{Vincent A. Knight \and Nikoleta E. Glynatsi}
\date{\today}



\begin{document}

\maketitle

\begin{abstract}
    The Iterated Prisoner's Dilemma is a model for rational and evolutionary
    interactive behaviour. It has applications both in the study of human social
    behaviour as well as in biology.
    It is used to understand when and how a rational individual might
    accept an immediate cost to their own utility for the direct benefit of
    another.

    Much attention has been given to a class of strategies called
    Zero Determinant strategies. It has been theoretically shown that these
    strategies can ``extort'' any player.

    In this work, an approach to identify if observed strategies are playing in
    an extortionate way is described. Furthermore, experimental analysis of
    a large tournament with \input{assets/tex/number_of_full_strategies/main.tex}
    strategies is considered. In this setting
    the most highly performing strategies do not play in an extortionate way
    against each other but do against lower performing strategies.
    This suggests that whilst the theory of Zero Determinant strategies
    indicates that memory is not of fundamental importance to the evolution of
    cooperative behaviour, this is incomplete.
\end{abstract}

\section{Introduction}\label{sec:introduction}

Agent based game theoretic models have become a stalwart of the underpinning
mathematics of interactive behaviours. One of the major pieces of work
in this area is the pair of original computer tournaments run by Robert
Axelrod~\cite{Axelrod1980, Axelrod1980a}. These tournaments pitted submitted
computer strategies against each other in plays of the Iterated Prisoner's
Dilemma. A common game where agents can choose to pay a slight cost to their
immediate utility in the hope of building a reputation. This has been used in
economic and evolutionary game theory to understand the evolution of cooperative
behaviour.

Recently, a class of strategies was described in~\cite{Press2012} that can
provably extort any given opponent. In~\cite{Hilbe2013, Moran1707} some
questions have already been asked about the true effectiveness of these
strategies in an evolutionary setting. Here another question is asked: is it
possible to recognise this extortionate behaviour? A mathematical procedure for
suspicion is presented: in the same way that the continued actions of an
extortionate individual might raise suspicion.

This work makes use of the Axelrod Python library~\cite{Knight2018, Knight2016}
with a large number of Prisoner Dilemma strategies available to give an
extensive numerical example of the ideas presented.  The approach is presented
in Section~\ref{sec:delta-zd-strategies}.  All of the code and data discussed
in Section~\ref{sec:numerical-experiments} is open sourced, archived and
written according to best scientific principles~\cite{Wilson2014}. The data
archive can be found at~\cite{vincent_knight_2018_1297075}.

\section{Recognising Extortion}\label{sec:delta-zd-strategies}

In~\cite{Press2012}, given a match between 2 memory-one strategies, the concept
of Zero Determinant (ZD) strategies is introduced. The main result of that paper
shows that given two memory one players \(p, q\in\mathbb{R}^4\) a linear
relationship between the players' scores could be forced by one of the players.

Using the notation of~\cite{Press2012}, assuming the utilities for player \(p\)
are given by \(S_x=(R, S, T, P)\) and for player \(q\) by \(S_y=(R, T, S, P)\)
and that the stationary scores of each player is given by \(S_X\) and \(S_Y\)
respectively. The main result of~\cite{Press2012} is that if

\begin{equation}\label{eqn:linear_relationship_for_p}
    \tilde p=\alpha S_x + \beta S_y + \gamma
\end{equation}

or

\begin{equation}\label{eqn:linear_relationship_for_q}
    \tilde q=\alpha S_x + \beta S_y + \gamma
\end{equation}

where \(\tilde p = (1 - p_1, 1 - p_2, p_3, p_4)\) and
\(\tilde q = (1 - q_1, 1 - q_2, q_3, q_4)\) then:

\begin{equation}
    \alpha S_X + \beta S_Y + \gamma = 0
\end{equation}

In~\cite{Press2012} a particular type of ZD strategy is defined: extortionate
strategies. If:

\begin{equation}\label{eqn:constraint_for_extortion}
    \gamma = - P(\alpha + \beta)
\end{equation}

then the player can ensure they get a score \(\chi\) times
larger than the opponent. This extortion coefficient is given by:

\begin{equation}\label{eqn:definition_of_chi}
    \chi=\frac{-\beta}{\alpha}
\end{equation}

Thus, if (\ref{eqn:constraint_for_extortion}) holds and \(\chi >1\) a player is
said to extort their opponent.
Here, the reverse problem is considered: given a
\(p\in\mathbb{R}^4\) how does one identify \(\alpha, \beta\) if they
exist and is the strategy in fact acting in an extortionate way?

These conditions correspond to:

\begin{align}
    \tilde p_1 & = \alpha R + \beta R - P (\alpha + \beta)
            \label{eqn:condition_for_tilde_p1}\\
    \tilde p_2 & = \alpha S + \beta T - P (\alpha + \beta)
            \label{eqn:condition_for_tilde_p2}\\
    \tilde p_3 & = \alpha T + \beta S - P (\alpha + \beta)
            \label{eqn:condition_for_tilde_p3}\\
    \tilde p_4 & = \alpha P + \beta P - P (\alpha + \beta)
            \label{eqn:condition_for_tilde_p4}
\end{align}

Equation (\ref{eqn:condition_for_tilde_p4}) ensures that \(p_4=\tilde p_4=0\).
Equations (\ref{eqn:condition_for_tilde_p1}-\ref{eqn:condition_for_tilde_p3})
can be used to eliminate \(\alpha, \beta\), giving:

\begin{equation}\label{eqn:planar_definition_of_extortion}
    \tilde p_1 = \frac{(R - P)(\tilde p_2 + \tilde p_3)}{S + T - 2P}
\end{equation}

with:

\begin{equation}\label{eqn:definition_of_chi}
    \chi = \frac{\tilde p_2 (P - T) + \tilde p_3 (S - P)}
                {\tilde p_2 (P - S) + \tilde p_3 (T - P)}
\end{equation}

Given a strategy \(p\in\mathbb{R}^{4\times 1}\) equations
(\ref{eqn:condition_for_tilde_p4}), (\ref{eqn:planar_definition_of_extortion}-\ref{eqn:definition_of_chi}) can be used to check if
a strategy is extortionate. The conditions correspond to:

\begin{align}
    p_1 & = \frac{(R-P)(p_2 + p_3) - R + T + S - P}{S + T - 2P}
     \label{eqn:condition_for_p1}\\
    p_4 & = 0 \label{eqn:condition_for_p4}\\
    1 & > p_2 + p_3\label{eqn:condition_for_chi}
\end{align}

The algebraic steps necessary to prove these results are available in the
supporting materials.

All extortionate strategies reside on a triangular (\ref{eqn:condition_for_chi})
plane (\ref{eqn:condition_for_p1}) in 3 dimensions (\ref{eqn:condition_for_p4}).
Using this formulation it can be seen that a necessary (but not sufficient)
condition for an extortionate strategy is that it cooperates on average less
than 50\% of the time when in a state of disagreement with the opponent.

As an example, consider the known extortionate strategy \(p=(8 / 9, 1 / 2, 1 /
3, 0)\) from~\cite{Stewart2012} which is referred to as \texttt{Extort-2}. In
this case, for the standard values of \((R, T, S, P)\) constraint
(\ref{eqn:condition_for_p1}) corresponds to:

\begin{equation}
    p_1 = \frac{2(p_2 + p_3) + 1}{3}
\end{equation}

It is clear that in this case all constraints hold.

This approach could in fact be used to confirm that a given strategy is acting
in an extortionate manner even if it is not a memory one strategy. However, in
practice, if a closed form for \(p\) is not known, then due to measurement
and/or numerical error this would not work.

This problem can be written in the following linear algebraic form where
\(x=(\alpha, \beta)\)
and \(p^*=(\tilde p_1 - 1, tilde_2 - 1, p_3)\):

\begin{equation}\label{eqn:linear_algebraic_equation_for_p}
    Cx= p^*
\end{equation}

\(C\) corresponds to equations
(\ref{eqn:condition_for_tilde_p1}-\ref{eqn:condition_for_tilde_p3}) and is
given by:

\begin{equation}\label{eqn:definition_of_C}
    C =
    \begin{bmatrix}
        R - P & R- P \\
        S - P & T- P \\
        T - P & S- P \\
    \end{bmatrix}
\end{equation}

Note that in general, equation (\ref{eqn:linear_algebraic_equation_for_p}) will
not necessarily have a solution. From the Rouch\'{e}-Capelli theorem if there is
a solution it is unique as \(\text{rank}(C)=2\) which is the dimension of the
variable \(x\). The best fitting \(x\) is found by minimizing:

\begin{equation}\label{eqn:r_squared}
    \text{SSError} = \|C x- p^*\|_2^2 = \sum_{i=1}^{3}\left((C\bar x)_i-p_i^*\right)^2
\end{equation}

Note that \(\text{SSError}\), which is the square of the Frobenius
norm~\cite{Golub2013}, becomes a measure of how close a strategy is to being an
extortionate strategy. Suspicion
of extortion then corresponds to a threshold on \(\text{SSError}\).

By observing interactions (human or otherwise), their memory one representation
can be inferred and this approach can be used to recognise extortionate
behaviour. The notion of comparing theoretic and actual plays of the IPD is not
novel, see for example~\cite{Rand2013}. Immediately it is noted that if the
environment is noisy~\cite{Wu1995} then no strategy can be considered to be
extortionate as \(p_4>0\).

In the next section, this idea will be illustrated by observing the interactions
that take place in a computer based tournament of the IPD\@.

\section{Numerical experiments}\label{sec:numerical-experiments}

In~\cite{Stewart2012} results from a tournament with
\input{./assets/tex/number_of_stewart_plotkin_strategies/main.tex} strategies,
was presented with specific consideration given to ZD strategies. This
tournament is reproduced here using the Axelrod-Python
project~\cite{Knight2016}. To obtain a good measure of the corresponding
transition rates for each strategy all matches have been run for
\input{assets/tex/number_of_turns/main.tex} turns and every match has been
repeated \input{assets/tex/number_of_repetitions/main.tex} times. All of this
interaction data is available at~\cite{vincent_knight_2018_1297075}. A good
match between the inferred Markov chain and the state distribution of the actual
interactions has been verified. Data for this is presented in the supplementary
materials.

Figure~\ref{fig:SSError_overall_in_stewart_plotkin} shows the \(\text{SSError}\)
values for all the strategies in the tournament, as reported
in~\cite{Stewart2012} the extortionate strategy (which has an expected
\(\text{SSError}\) approximately 0) gains a large number of wins.

\begin{figure}[!htbp]
    \centering
    \includegraphics[width=.8\textwidth]{./assets/img/SSError_overall_in_stewart_plotkin/main.pdf}
    \caption{\(\text{SSError}\) and state probabilities for the strategies
        of~\cite{Stewart2012}, ordered both by number of wins and overall score.
        Note that \(P(DC)\) is not shown as it corresponds to the transpose of
        \(P(CD)\). Cooperator and Defector are omitted as they do not visit all
        the states.}
    \label{fig:SSError_overall_in_stewart_plotkin}
\end{figure}

Here, the work of~\cite{Stewart2012} is extended by investigating a tournament
with \input{assets/tex/number_of_full_strategies/main.tex}
strategies.

The results of this analysis are shown in
Figure~\ref{fig:SSError_and_probabilities_in_full}. The top ranking strategies
by number of wins seem to be extortionate (but not against all strategies) and
it can be seen that a small sub group of strategies achieve mutual defection.
All the top ranking strategies according to score achieve mutual cooperation and
do not extort each other, however they
\textbf{do} exhibit extortionate behaviour towards a number of the lower ranking
strategies.

\begin{figure}[!htbp]
    \centering
    \includegraphics[width=.8\textwidth]{./assets/img/SSError_and_probabilities_in_full/main.pdf}
    \caption{\(\text{SSError}\) for the strategies for the full tournament. Only
    strategy interactions for which \(p_4=0\) and \(\chi>1\) are displayed.}
    \label{fig:SSError_and_probabilities_in_full}
\end{figure}

\section{Conclusion}\label{sec:conclusion}

This work defines an approach to measure whether or not a player is playing a
strategy that corresponds to an extortionate strategy as defined
in~\cite{Press2012}: a mathematical model for suspicion. Indeed, all
extortionate strategies have been
 classified as lying on a triangular plane.
This rigorous classification fails to be robust to small measurement error, thus
a statistical approach is proposed.
This is done through a linear algebraic approach for approximating the solution
of a linear system. Using this, a large number of pairwise interactions is
simulated and in fact very few strategies are found to act extortionately.

The work of~\cite{Press2012}, whilst showing that a clever approach to taking
advantage of another memory one strategy exists: this is incomplete. Whilst the
elegance of this result is very attractive, just as the simplicity of the
victory of Tit For Tat in Axelrod's original tournaments was, it is incomplete.
Extortionate strategies achieve a high number of wins but they do not
achieve a high score which corresponds to the fitness landscape in an
evolutionary sense. From the large number of interactions a payoff matrix \(S\)
can be measured where \(S_{ij}\) denotes the score (using standard values of
\((R, S, T, P) = (3, 0, 5, 1)\)) of the \(i\)th strategy
against the \(j\)th strategy. Using this, the replicator equation
describes the evolution of the system based on a population density fitness
function:

\begin{equation}\label{eqn:replicator_dynamics}
    \frac{dx}{dt} = x(S-x^TS x)
\end{equation}

Equation (\ref{eqn:replicator_dynamics}) is solved numerically through an
integration technique described in~\cite{Petzold1983} and
Figure~\ref{fig:replicator_dynamics} shows the evolution of the distribution of
the system: the various strategies are ranked by scores. It is clear to see that
only the high ranking strategies survive the evolutionary process (in fact,
only \input{./assets/img/replicator_dynamics/main.tex}
have a final distribution greater than \(10 ^ {-2}\)). This confirms the
findings of~\cite{Moran1707} in which sophisticated strategies resist
evolutionary invasion of shorter memory strategies. Recalling
Figure~\ref{fig:SSError_and_probabilities_in_full} this demonstrates that:

\begin{itemize}
    \item Cooperation emerges through the evolutionary process: the high scoring
        strategies do not exhibit extortionate behaviour towards each other.
    \item Extortionate strategies do not survive the evolutionary process.
\end{itemize}

\begin{figure}[!htbp]
    \centering
    \includegraphics[width=.8\textwidth]{./assets/img/replicator_dynamics/main.pdf}
    \caption{Numerical simulation of the replicator equation
    (\ref{eqn:replicator_dynamics}): strategies are ordered by score, only the strategies with a high score survive the evolutionary process.}
    \label{fig:replicator_dynamics}
\end{figure}

This work can be used to classify plays of the IPD\@: data can be collected from
actual interactions (in lab or in the field). Furthermore, this allows for a
classification method similar to the notion of fingerprinting presented
in~\cite{Ashlock2008}. Trained strategies can potentially be classified as
extortionate or not or it could be possible to even constrain the reinforcement
learning approaches that are becoming prevalent in the literature.
Alternatively, this mathematical approach for recognising extortion could be
used in sophisticated strategies to defend against invasion. Arguably, some of
the strategies considered here exhibit this behaviour, indeed as described
in~\cite{Harper2017}, the top ranking strategies in the full tournament are
obtained using evolutionary reinforcement learning techniques, thus, suspicion
of extortionate behaviour could in fact be an evolutionary trait.

\section*{Acknowledgements}

The following open source software libraries were used in this research:

\begin{itemize}
    \item The Axelrod ~\cite{Knight2016, Knight2018} library (IPD strategies and
        tournaments).
    \item The sympy library~\cite{Meurer2017} (verification of all symbolic
        calculations).
    \item The matplotlib~\cite{Droettboom2018} library (visualisation).
    \item The pandas~\cite{Structures2010}, dask~\cite{Dask2016} and
        NumPy~\cite{Oliphant2015} libraries (data manipulation).
    \item The SciPy~\cite{Jones2001} library (numerical integration of the
        replicator equation).
\end{itemize}

This work was performed using the computational facilities of the Advanced
Research Computing @ Cardiff (ARCCA) Division, Cardiff University.

\printbibliography

\newpage
\section*{Supplementary materials}

\includepdf{assets/pdf/proof_of_form_of_extortionate_strategies/main.pdf}

\newpage

Using the pair wise interactions the transition rates \(p,
q\) can be measured and the steady state probabilities inferred and compared to
the actual probabilities of each state.
This is done numerically by computing the singular eigenvector of the
matrix \(A\) \cite{Stewart2009}:

\[
    A =
    \begin{bmatrix}
        p_1 q_1 & p_1 (1 - q_1) & (1 - p_1) q_1 & (1 -p_1) (1 - q_1) \\
        p_2 q_2 & p_2 (1 - q_2) & (1 - p_2) q_2 & (1 -p_2) (1 - q_2) \\
        p_3 q_3 & p_3 (1 - q_3) & (1 - p_3) q_3 & (1 -p_3) (1 - q_3) \\
        p_4 q_4 & p_4 (1 - q_4) & (1 - p_4) q_4 & (1 -p_4) (1 - q_4) \\
    \end{bmatrix}
\]

Figure~\ref{fig:computed_probabilities_vs_theoretic_probabilities} shows a
regression line fitted to every pairwise interaction with a reported
\(\text{SSError}\) value (pairwise interactions with missing states were
omitted). This serves to validate the approach: a part from some edge cases the
relationship is consistent.

\begin{figure}[!htbp]
    \centering
    \includegraphics[width=.8\textwidth]{./assets/img/computed_probabilities_vs_theoretic_probabilities/main.pdf}
    \caption{The
        relationship between the steady state probabilities inferred from the
        measured transitions and the actual steady state probabilities. A linear
        regression line is included validating the approach.}
    \label{fig:computed_probabilities_vs_theoretic_probabilities}
\end{figure}


\end{document}
 turns and every match has been
repeated \documentclass[a4paper]{article}

\usepackage{amsmath}
\usepackage{amssymb}
\usepackage[margin=1.5cm,
            includefoot,
            footskip=30pt]{geometry}
\usepackage{layout}
\usepackage{graphicx}
\usepackage{subcaption}

\usepackage{biblatex}
\usepackage{pdfpages}

\bibliography{main.bib}

\title{Suspicion: Recognising and evaluating the effectiveness
       of extortion in the Iterated Prisoner's Dilemma}
\author{Vincent A. Knight \and Nikoleta E. Glynatsi}
\date{\today}



\begin{document}

\maketitle

\begin{abstract}
    The Iterated Prisoner's Dilemma is a model for rational and evolutionary
    interactive behaviour. It has applications both in the study of human social
    behaviour as well as in biology.
    It is used to understand when and how a rational individual might
    accept an immediate cost to their own utility for the direct benefit of
    another.

    Much attention has been given to a class of strategies called
    Zero Determinant strategies. It has been theoretically shown that these
    strategies can ``extort'' any player.

    In this work, an approach to identify if observed strategies are playing in
    an extortionate way is described. Furthermore, experimental analysis of
    a large tournament with \input{assets/tex/number_of_full_strategies/main.tex}
    strategies is considered. In this setting
    the most highly performing strategies do not play in an extortionate way
    against each other but do against lower performing strategies.
    This suggests that whilst the theory of Zero Determinant strategies
    indicates that memory is not of fundamental importance to the evolution of
    cooperative behaviour, this is incomplete.
\end{abstract}

\section{Introduction}\label{sec:introduction}

Agent based game theoretic models have become a stalwart of the underpinning
mathematics of interactive behaviours. One of the major pieces of work
in this area is the pair of original computer tournaments run by Robert
Axelrod~\cite{Axelrod1980, Axelrod1980a}. These tournaments pitted submitted
computer strategies against each other in plays of the Iterated Prisoner's
Dilemma. A common game where agents can choose to pay a slight cost to their
immediate utility in the hope of building a reputation. This has been used in
economic and evolutionary game theory to understand the evolution of cooperative
behaviour.

Recently, a class of strategies was described in~\cite{Press2012} that can
provably extort any given opponent. In~\cite{Hilbe2013, Moran1707} some
questions have already been asked about the true effectiveness of these
strategies in an evolutionary setting. Here another question is asked: is it
possible to recognise this extortionate behaviour? A mathematical procedure for
suspicion is presented: in the same way that the continued actions of an
extortionate individual might raise suspicion.

This work makes use of the Axelrod Python library~\cite{Knight2018, Knight2016}
with a large number of Prisoner Dilemma strategies available to give an
extensive numerical example of the ideas presented.  The approach is presented
in Section~\ref{sec:delta-zd-strategies}.  All of the code and data discussed
in Section~\ref{sec:numerical-experiments} is open sourced, archived and
written according to best scientific principles~\cite{Wilson2014}. The data
archive can be found at~\cite{vincent_knight_2018_1297075}.

\section{Recognising Extortion}\label{sec:delta-zd-strategies}

In~\cite{Press2012}, given a match between 2 memory-one strategies, the concept
of Zero Determinant (ZD) strategies is introduced. The main result of that paper
shows that given two memory one players \(p, q\in\mathbb{R}^4\) a linear
relationship between the players' scores could be forced by one of the players.

Using the notation of~\cite{Press2012}, assuming the utilities for player \(p\)
are given by \(S_x=(R, S, T, P)\) and for player \(q\) by \(S_y=(R, T, S, P)\)
and that the stationary scores of each player is given by \(S_X\) and \(S_Y\)
respectively. The main result of~\cite{Press2012} is that if

\begin{equation}\label{eqn:linear_relationship_for_p}
    \tilde p=\alpha S_x + \beta S_y + \gamma
\end{equation}

or

\begin{equation}\label{eqn:linear_relationship_for_q}
    \tilde q=\alpha S_x + \beta S_y + \gamma
\end{equation}

where \(\tilde p = (1 - p_1, 1 - p_2, p_3, p_4)\) and
\(\tilde q = (1 - q_1, 1 - q_2, q_3, q_4)\) then:

\begin{equation}
    \alpha S_X + \beta S_Y + \gamma = 0
\end{equation}

In~\cite{Press2012} a particular type of ZD strategy is defined: extortionate
strategies. If:

\begin{equation}\label{eqn:constraint_for_extortion}
    \gamma = - P(\alpha + \beta)
\end{equation}

then the player can ensure they get a score \(\chi\) times
larger than the opponent. This extortion coefficient is given by:

\begin{equation}\label{eqn:definition_of_chi}
    \chi=\frac{-\beta}{\alpha}
\end{equation}

Thus, if (\ref{eqn:constraint_for_extortion}) holds and \(\chi >1\) a player is
said to extort their opponent.
Here, the reverse problem is considered: given a
\(p\in\mathbb{R}^4\) how does one identify \(\alpha, \beta\) if they
exist and is the strategy in fact acting in an extortionate way?

These conditions correspond to:

\begin{align}
    \tilde p_1 & = \alpha R + \beta R - P (\alpha + \beta)
            \label{eqn:condition_for_tilde_p1}\\
    \tilde p_2 & = \alpha S + \beta T - P (\alpha + \beta)
            \label{eqn:condition_for_tilde_p2}\\
    \tilde p_3 & = \alpha T + \beta S - P (\alpha + \beta)
            \label{eqn:condition_for_tilde_p3}\\
    \tilde p_4 & = \alpha P + \beta P - P (\alpha + \beta)
            \label{eqn:condition_for_tilde_p4}
\end{align}

Equation (\ref{eqn:condition_for_tilde_p4}) ensures that \(p_4=\tilde p_4=0\).
Equations (\ref{eqn:condition_for_tilde_p1}-\ref{eqn:condition_for_tilde_p3})
can be used to eliminate \(\alpha, \beta\), giving:

\begin{equation}\label{eqn:planar_definition_of_extortion}
    \tilde p_1 = \frac{(R - P)(\tilde p_2 + \tilde p_3)}{S + T - 2P}
\end{equation}

with:

\begin{equation}\label{eqn:definition_of_chi}
    \chi = \frac{\tilde p_2 (P - T) + \tilde p_3 (S - P)}
                {\tilde p_2 (P - S) + \tilde p_3 (T - P)}
\end{equation}

Given a strategy \(p\in\mathbb{R}^{4\times 1}\) equations
(\ref{eqn:condition_for_tilde_p4}), (\ref{eqn:planar_definition_of_extortion}-\ref{eqn:definition_of_chi}) can be used to check if
a strategy is extortionate. The conditions correspond to:

\begin{align}
    p_1 & = \frac{(R-P)(p_2 + p_3) - R + T + S - P}{S + T - 2P}
     \label{eqn:condition_for_p1}\\
    p_4 & = 0 \label{eqn:condition_for_p4}\\
    1 & > p_2 + p_3\label{eqn:condition_for_chi}
\end{align}

The algebraic steps necessary to prove these results are available in the
supporting materials.

All extortionate strategies reside on a triangular (\ref{eqn:condition_for_chi})
plane (\ref{eqn:condition_for_p1}) in 3 dimensions (\ref{eqn:condition_for_p4}).
Using this formulation it can be seen that a necessary (but not sufficient)
condition for an extortionate strategy is that it cooperates on average less
than 50\% of the time when in a state of disagreement with the opponent.

As an example, consider the known extortionate strategy \(p=(8 / 9, 1 / 2, 1 /
3, 0)\) from~\cite{Stewart2012} which is referred to as \texttt{Extort-2}. In
this case, for the standard values of \((R, T, S, P)\) constraint
(\ref{eqn:condition_for_p1}) corresponds to:

\begin{equation}
    p_1 = \frac{2(p_2 + p_3) + 1}{3}
\end{equation}

It is clear that in this case all constraints hold.

This approach could in fact be used to confirm that a given strategy is acting
in an extortionate manner even if it is not a memory one strategy. However, in
practice, if a closed form for \(p\) is not known, then due to measurement
and/or numerical error this would not work.

This problem can be written in the following linear algebraic form where
\(x=(\alpha, \beta)\)
and \(p^*=(\tilde p_1 - 1, tilde_2 - 1, p_3)\):

\begin{equation}\label{eqn:linear_algebraic_equation_for_p}
    Cx= p^*
\end{equation}

\(C\) corresponds to equations
(\ref{eqn:condition_for_tilde_p1}-\ref{eqn:condition_for_tilde_p3}) and is
given by:

\begin{equation}\label{eqn:definition_of_C}
    C =
    \begin{bmatrix}
        R - P & R- P \\
        S - P & T- P \\
        T - P & S- P \\
    \end{bmatrix}
\end{equation}

Note that in general, equation (\ref{eqn:linear_algebraic_equation_for_p}) will
not necessarily have a solution. From the Rouch\'{e}-Capelli theorem if there is
a solution it is unique as \(\text{rank}(C)=2\) which is the dimension of the
variable \(x\). The best fitting \(x\) is found by minimizing:

\begin{equation}\label{eqn:r_squared}
    \text{SSError} = \|C x- p^*\|_2^2 = \sum_{i=1}^{3}\left((C\bar x)_i-p_i^*\right)^2
\end{equation}

Note that \(\text{SSError}\), which is the square of the Frobenius
norm~\cite{Golub2013}, becomes a measure of how close a strategy is to being an
extortionate strategy. Suspicion
of extortion then corresponds to a threshold on \(\text{SSError}\).

By observing interactions (human or otherwise), their memory one representation
can be inferred and this approach can be used to recognise extortionate
behaviour. The notion of comparing theoretic and actual plays of the IPD is not
novel, see for example~\cite{Rand2013}. Immediately it is noted that if the
environment is noisy~\cite{Wu1995} then no strategy can be considered to be
extortionate as \(p_4>0\).

In the next section, this idea will be illustrated by observing the interactions
that take place in a computer based tournament of the IPD\@.

\section{Numerical experiments}\label{sec:numerical-experiments}

In~\cite{Stewart2012} results from a tournament with
\input{./assets/tex/number_of_stewart_plotkin_strategies/main.tex} strategies,
was presented with specific consideration given to ZD strategies. This
tournament is reproduced here using the Axelrod-Python
project~\cite{Knight2016}. To obtain a good measure of the corresponding
transition rates for each strategy all matches have been run for
\input{assets/tex/number_of_turns/main.tex} turns and every match has been
repeated \input{assets/tex/number_of_repetitions/main.tex} times. All of this
interaction data is available at~\cite{vincent_knight_2018_1297075}. A good
match between the inferred Markov chain and the state distribution of the actual
interactions has been verified. Data for this is presented in the supplementary
materials.

Figure~\ref{fig:SSError_overall_in_stewart_plotkin} shows the \(\text{SSError}\)
values for all the strategies in the tournament, as reported
in~\cite{Stewart2012} the extortionate strategy (which has an expected
\(\text{SSError}\) approximately 0) gains a large number of wins.

\begin{figure}[!htbp]
    \centering
    \includegraphics[width=.8\textwidth]{./assets/img/SSError_overall_in_stewart_plotkin/main.pdf}
    \caption{\(\text{SSError}\) and state probabilities for the strategies
        of~\cite{Stewart2012}, ordered both by number of wins and overall score.
        Note that \(P(DC)\) is not shown as it corresponds to the transpose of
        \(P(CD)\). Cooperator and Defector are omitted as they do not visit all
        the states.}
    \label{fig:SSError_overall_in_stewart_plotkin}
\end{figure}

Here, the work of~\cite{Stewart2012} is extended by investigating a tournament
with \input{assets/tex/number_of_full_strategies/main.tex}
strategies.

The results of this analysis are shown in
Figure~\ref{fig:SSError_and_probabilities_in_full}. The top ranking strategies
by number of wins seem to be extortionate (but not against all strategies) and
it can be seen that a small sub group of strategies achieve mutual defection.
All the top ranking strategies according to score achieve mutual cooperation and
do not extort each other, however they
\textbf{do} exhibit extortionate behaviour towards a number of the lower ranking
strategies.

\begin{figure}[!htbp]
    \centering
    \includegraphics[width=.8\textwidth]{./assets/img/SSError_and_probabilities_in_full/main.pdf}
    \caption{\(\text{SSError}\) for the strategies for the full tournament. Only
    strategy interactions for which \(p_4=0\) and \(\chi>1\) are displayed.}
    \label{fig:SSError_and_probabilities_in_full}
\end{figure}

\section{Conclusion}\label{sec:conclusion}

This work defines an approach to measure whether or not a player is playing a
strategy that corresponds to an extortionate strategy as defined
in~\cite{Press2012}: a mathematical model for suspicion. Indeed, all
extortionate strategies have been
 classified as lying on a triangular plane.
This rigorous classification fails to be robust to small measurement error, thus
a statistical approach is proposed.
This is done through a linear algebraic approach for approximating the solution
of a linear system. Using this, a large number of pairwise interactions is
simulated and in fact very few strategies are found to act extortionately.

The work of~\cite{Press2012}, whilst showing that a clever approach to taking
advantage of another memory one strategy exists: this is incomplete. Whilst the
elegance of this result is very attractive, just as the simplicity of the
victory of Tit For Tat in Axelrod's original tournaments was, it is incomplete.
Extortionate strategies achieve a high number of wins but they do not
achieve a high score which corresponds to the fitness landscape in an
evolutionary sense. From the large number of interactions a payoff matrix \(S\)
can be measured where \(S_{ij}\) denotes the score (using standard values of
\((R, S, T, P) = (3, 0, 5, 1)\)) of the \(i\)th strategy
against the \(j\)th strategy. Using this, the replicator equation
describes the evolution of the system based on a population density fitness
function:

\begin{equation}\label{eqn:replicator_dynamics}
    \frac{dx}{dt} = x(S-x^TS x)
\end{equation}

Equation (\ref{eqn:replicator_dynamics}) is solved numerically through an
integration technique described in~\cite{Petzold1983} and
Figure~\ref{fig:replicator_dynamics} shows the evolution of the distribution of
the system: the various strategies are ranked by scores. It is clear to see that
only the high ranking strategies survive the evolutionary process (in fact,
only \input{./assets/img/replicator_dynamics/main.tex}
have a final distribution greater than \(10 ^ {-2}\)). This confirms the
findings of~\cite{Moran1707} in which sophisticated strategies resist
evolutionary invasion of shorter memory strategies. Recalling
Figure~\ref{fig:SSError_and_probabilities_in_full} this demonstrates that:

\begin{itemize}
    \item Cooperation emerges through the evolutionary process: the high scoring
        strategies do not exhibit extortionate behaviour towards each other.
    \item Extortionate strategies do not survive the evolutionary process.
\end{itemize}

\begin{figure}[!htbp]
    \centering
    \includegraphics[width=.8\textwidth]{./assets/img/replicator_dynamics/main.pdf}
    \caption{Numerical simulation of the replicator equation
    (\ref{eqn:replicator_dynamics}): strategies are ordered by score, only the strategies with a high score survive the evolutionary process.}
    \label{fig:replicator_dynamics}
\end{figure}

This work can be used to classify plays of the IPD\@: data can be collected from
actual interactions (in lab or in the field). Furthermore, this allows for a
classification method similar to the notion of fingerprinting presented
in~\cite{Ashlock2008}. Trained strategies can potentially be classified as
extortionate or not or it could be possible to even constrain the reinforcement
learning approaches that are becoming prevalent in the literature.
Alternatively, this mathematical approach for recognising extortion could be
used in sophisticated strategies to defend against invasion. Arguably, some of
the strategies considered here exhibit this behaviour, indeed as described
in~\cite{Harper2017}, the top ranking strategies in the full tournament are
obtained using evolutionary reinforcement learning techniques, thus, suspicion
of extortionate behaviour could in fact be an evolutionary trait.

\section*{Acknowledgements}

The following open source software libraries were used in this research:

\begin{itemize}
    \item The Axelrod ~\cite{Knight2016, Knight2018} library (IPD strategies and
        tournaments).
    \item The sympy library~\cite{Meurer2017} (verification of all symbolic
        calculations).
    \item The matplotlib~\cite{Droettboom2018} library (visualisation).
    \item The pandas~\cite{Structures2010}, dask~\cite{Dask2016} and
        NumPy~\cite{Oliphant2015} libraries (data manipulation).
    \item The SciPy~\cite{Jones2001} library (numerical integration of the
        replicator equation).
\end{itemize}

This work was performed using the computational facilities of the Advanced
Research Computing @ Cardiff (ARCCA) Division, Cardiff University.

\printbibliography

\newpage
\section*{Supplementary materials}

\includepdf{assets/pdf/proof_of_form_of_extortionate_strategies/main.pdf}

\newpage

Using the pair wise interactions the transition rates \(p,
q\) can be measured and the steady state probabilities inferred and compared to
the actual probabilities of each state.
This is done numerically by computing the singular eigenvector of the
matrix \(A\) \cite{Stewart2009}:

\[
    A =
    \begin{bmatrix}
        p_1 q_1 & p_1 (1 - q_1) & (1 - p_1) q_1 & (1 -p_1) (1 - q_1) \\
        p_2 q_2 & p_2 (1 - q_2) & (1 - p_2) q_2 & (1 -p_2) (1 - q_2) \\
        p_3 q_3 & p_3 (1 - q_3) & (1 - p_3) q_3 & (1 -p_3) (1 - q_3) \\
        p_4 q_4 & p_4 (1 - q_4) & (1 - p_4) q_4 & (1 -p_4) (1 - q_4) \\
    \end{bmatrix}
\]

Figure~\ref{fig:computed_probabilities_vs_theoretic_probabilities} shows a
regression line fitted to every pairwise interaction with a reported
\(\text{SSError}\) value (pairwise interactions with missing states were
omitted). This serves to validate the approach: a part from some edge cases the
relationship is consistent.

\begin{figure}[!htbp]
    \centering
    \includegraphics[width=.8\textwidth]{./assets/img/computed_probabilities_vs_theoretic_probabilities/main.pdf}
    \caption{The
        relationship between the steady state probabilities inferred from the
        measured transitions and the actual steady state probabilities. A linear
        regression line is included validating the approach.}
    \label{fig:computed_probabilities_vs_theoretic_probabilities}
\end{figure}


\end{document}
 times. All of this
interaction data is available at~\cite{vincent_knight_2018_1297075}. A good
match between the inferred Markov chain and the state distribution of the actual
interactions has been verified. Data for this is presented in the supplementary
materials.

Figure~\ref{fig:SSError_overall_in_stewart_plotkin} shows the \(\text{SSError}\)
values for all the strategies in the tournament, as reported
in~\cite{Stewart2012} the extortionate strategy (which has an expected
\(\text{SSError}\) approximately 0) gains a large number of wins.

\begin{figure}[!htbp]
    \centering
    \includegraphics[width=.8\textwidth]{./assets/img/SSError_overall_in_stewart_plotkin/main.pdf}
    \caption{\(\text{SSError}\) and state probabilities for the strategies
        of~\cite{Stewart2012}, ordered both by number of wins and overall score.
        Note that \(P(DC)\) is not shown as it corresponds to the transpose of
        \(P(CD)\). Cooperator and Defector are omitted as they do not visit all
        the states.}
    \label{fig:SSError_overall_in_stewart_plotkin}
\end{figure}

Here, the work of~\cite{Stewart2012} is extended by investigating a tournament
with \documentclass[a4paper]{article}

\usepackage{amsmath}
\usepackage{amssymb}
\usepackage[margin=1.5cm,
            includefoot,
            footskip=30pt]{geometry}
\usepackage{layout}
\usepackage{graphicx}
\usepackage{subcaption}

\usepackage{biblatex}
\usepackage{pdfpages}

\bibliography{main.bib}

\title{Suspicion: Recognising and evaluating the effectiveness
       of extortion in the Iterated Prisoner's Dilemma}
\author{Vincent A. Knight \and Nikoleta E. Glynatsi}
\date{\today}



\begin{document}

\maketitle

\begin{abstract}
    The Iterated Prisoner's Dilemma is a model for rational and evolutionary
    interactive behaviour. It has applications both in the study of human social
    behaviour as well as in biology.
    It is used to understand when and how a rational individual might
    accept an immediate cost to their own utility for the direct benefit of
    another.

    Much attention has been given to a class of strategies called
    Zero Determinant strategies. It has been theoretically shown that these
    strategies can ``extort'' any player.

    In this work, an approach to identify if observed strategies are playing in
    an extortionate way is described. Furthermore, experimental analysis of
    a large tournament with \input{assets/tex/number_of_full_strategies/main.tex}
    strategies is considered. In this setting
    the most highly performing strategies do not play in an extortionate way
    against each other but do against lower performing strategies.
    This suggests that whilst the theory of Zero Determinant strategies
    indicates that memory is not of fundamental importance to the evolution of
    cooperative behaviour, this is incomplete.
\end{abstract}

\section{Introduction}\label{sec:introduction}

Agent based game theoretic models have become a stalwart of the underpinning
mathematics of interactive behaviours. One of the major pieces of work
in this area is the pair of original computer tournaments run by Robert
Axelrod~\cite{Axelrod1980, Axelrod1980a}. These tournaments pitted submitted
computer strategies against each other in plays of the Iterated Prisoner's
Dilemma. A common game where agents can choose to pay a slight cost to their
immediate utility in the hope of building a reputation. This has been used in
economic and evolutionary game theory to understand the evolution of cooperative
behaviour.

Recently, a class of strategies was described in~\cite{Press2012} that can
provably extort any given opponent. In~\cite{Hilbe2013, Moran1707} some
questions have already been asked about the true effectiveness of these
strategies in an evolutionary setting. Here another question is asked: is it
possible to recognise this extortionate behaviour? A mathematical procedure for
suspicion is presented: in the same way that the continued actions of an
extortionate individual might raise suspicion.

This work makes use of the Axelrod Python library~\cite{Knight2018, Knight2016}
with a large number of Prisoner Dilemma strategies available to give an
extensive numerical example of the ideas presented.  The approach is presented
in Section~\ref{sec:delta-zd-strategies}.  All of the code and data discussed
in Section~\ref{sec:numerical-experiments} is open sourced, archived and
written according to best scientific principles~\cite{Wilson2014}. The data
archive can be found at~\cite{vincent_knight_2018_1297075}.

\section{Recognising Extortion}\label{sec:delta-zd-strategies}

In~\cite{Press2012}, given a match between 2 memory-one strategies, the concept
of Zero Determinant (ZD) strategies is introduced. The main result of that paper
shows that given two memory one players \(p, q\in\mathbb{R}^4\) a linear
relationship between the players' scores could be forced by one of the players.

Using the notation of~\cite{Press2012}, assuming the utilities for player \(p\)
are given by \(S_x=(R, S, T, P)\) and for player \(q\) by \(S_y=(R, T, S, P)\)
and that the stationary scores of each player is given by \(S_X\) and \(S_Y\)
respectively. The main result of~\cite{Press2012} is that if

\begin{equation}\label{eqn:linear_relationship_for_p}
    \tilde p=\alpha S_x + \beta S_y + \gamma
\end{equation}

or

\begin{equation}\label{eqn:linear_relationship_for_q}
    \tilde q=\alpha S_x + \beta S_y + \gamma
\end{equation}

where \(\tilde p = (1 - p_1, 1 - p_2, p_3, p_4)\) and
\(\tilde q = (1 - q_1, 1 - q_2, q_3, q_4)\) then:

\begin{equation}
    \alpha S_X + \beta S_Y + \gamma = 0
\end{equation}

In~\cite{Press2012} a particular type of ZD strategy is defined: extortionate
strategies. If:

\begin{equation}\label{eqn:constraint_for_extortion}
    \gamma = - P(\alpha + \beta)
\end{equation}

then the player can ensure they get a score \(\chi\) times
larger than the opponent. This extortion coefficient is given by:

\begin{equation}\label{eqn:definition_of_chi}
    \chi=\frac{-\beta}{\alpha}
\end{equation}

Thus, if (\ref{eqn:constraint_for_extortion}) holds and \(\chi >1\) a player is
said to extort their opponent.
Here, the reverse problem is considered: given a
\(p\in\mathbb{R}^4\) how does one identify \(\alpha, \beta\) if they
exist and is the strategy in fact acting in an extortionate way?

These conditions correspond to:

\begin{align}
    \tilde p_1 & = \alpha R + \beta R - P (\alpha + \beta)
            \label{eqn:condition_for_tilde_p1}\\
    \tilde p_2 & = \alpha S + \beta T - P (\alpha + \beta)
            \label{eqn:condition_for_tilde_p2}\\
    \tilde p_3 & = \alpha T + \beta S - P (\alpha + \beta)
            \label{eqn:condition_for_tilde_p3}\\
    \tilde p_4 & = \alpha P + \beta P - P (\alpha + \beta)
            \label{eqn:condition_for_tilde_p4}
\end{align}

Equation (\ref{eqn:condition_for_tilde_p4}) ensures that \(p_4=\tilde p_4=0\).
Equations (\ref{eqn:condition_for_tilde_p1}-\ref{eqn:condition_for_tilde_p3})
can be used to eliminate \(\alpha, \beta\), giving:

\begin{equation}\label{eqn:planar_definition_of_extortion}
    \tilde p_1 = \frac{(R - P)(\tilde p_2 + \tilde p_3)}{S + T - 2P}
\end{equation}

with:

\begin{equation}\label{eqn:definition_of_chi}
    \chi = \frac{\tilde p_2 (P - T) + \tilde p_3 (S - P)}
                {\tilde p_2 (P - S) + \tilde p_3 (T - P)}
\end{equation}

Given a strategy \(p\in\mathbb{R}^{4\times 1}\) equations
(\ref{eqn:condition_for_tilde_p4}), (\ref{eqn:planar_definition_of_extortion}-\ref{eqn:definition_of_chi}) can be used to check if
a strategy is extortionate. The conditions correspond to:

\begin{align}
    p_1 & = \frac{(R-P)(p_2 + p_3) - R + T + S - P}{S + T - 2P}
     \label{eqn:condition_for_p1}\\
    p_4 & = 0 \label{eqn:condition_for_p4}\\
    1 & > p_2 + p_3\label{eqn:condition_for_chi}
\end{align}

The algebraic steps necessary to prove these results are available in the
supporting materials.

All extortionate strategies reside on a triangular (\ref{eqn:condition_for_chi})
plane (\ref{eqn:condition_for_p1}) in 3 dimensions (\ref{eqn:condition_for_p4}).
Using this formulation it can be seen that a necessary (but not sufficient)
condition for an extortionate strategy is that it cooperates on average less
than 50\% of the time when in a state of disagreement with the opponent.

As an example, consider the known extortionate strategy \(p=(8 / 9, 1 / 2, 1 /
3, 0)\) from~\cite{Stewart2012} which is referred to as \texttt{Extort-2}. In
this case, for the standard values of \((R, T, S, P)\) constraint
(\ref{eqn:condition_for_p1}) corresponds to:

\begin{equation}
    p_1 = \frac{2(p_2 + p_3) + 1}{3}
\end{equation}

It is clear that in this case all constraints hold.

This approach could in fact be used to confirm that a given strategy is acting
in an extortionate manner even if it is not a memory one strategy. However, in
practice, if a closed form for \(p\) is not known, then due to measurement
and/or numerical error this would not work.

This problem can be written in the following linear algebraic form where
\(x=(\alpha, \beta)\)
and \(p^*=(\tilde p_1 - 1, tilde_2 - 1, p_3)\):

\begin{equation}\label{eqn:linear_algebraic_equation_for_p}
    Cx= p^*
\end{equation}

\(C\) corresponds to equations
(\ref{eqn:condition_for_tilde_p1}-\ref{eqn:condition_for_tilde_p3}) and is
given by:

\begin{equation}\label{eqn:definition_of_C}
    C =
    \begin{bmatrix}
        R - P & R- P \\
        S - P & T- P \\
        T - P & S- P \\
    \end{bmatrix}
\end{equation}

Note that in general, equation (\ref{eqn:linear_algebraic_equation_for_p}) will
not necessarily have a solution. From the Rouch\'{e}-Capelli theorem if there is
a solution it is unique as \(\text{rank}(C)=2\) which is the dimension of the
variable \(x\). The best fitting \(x\) is found by minimizing:

\begin{equation}\label{eqn:r_squared}
    \text{SSError} = \|C x- p^*\|_2^2 = \sum_{i=1}^{3}\left((C\bar x)_i-p_i^*\right)^2
\end{equation}

Note that \(\text{SSError}\), which is the square of the Frobenius
norm~\cite{Golub2013}, becomes a measure of how close a strategy is to being an
extortionate strategy. Suspicion
of extortion then corresponds to a threshold on \(\text{SSError}\).

By observing interactions (human or otherwise), their memory one representation
can be inferred and this approach can be used to recognise extortionate
behaviour. The notion of comparing theoretic and actual plays of the IPD is not
novel, see for example~\cite{Rand2013}. Immediately it is noted that if the
environment is noisy~\cite{Wu1995} then no strategy can be considered to be
extortionate as \(p_4>0\).

In the next section, this idea will be illustrated by observing the interactions
that take place in a computer based tournament of the IPD\@.

\section{Numerical experiments}\label{sec:numerical-experiments}

In~\cite{Stewart2012} results from a tournament with
\input{./assets/tex/number_of_stewart_plotkin_strategies/main.tex} strategies,
was presented with specific consideration given to ZD strategies. This
tournament is reproduced here using the Axelrod-Python
project~\cite{Knight2016}. To obtain a good measure of the corresponding
transition rates for each strategy all matches have been run for
\input{assets/tex/number_of_turns/main.tex} turns and every match has been
repeated \input{assets/tex/number_of_repetitions/main.tex} times. All of this
interaction data is available at~\cite{vincent_knight_2018_1297075}. A good
match between the inferred Markov chain and the state distribution of the actual
interactions has been verified. Data for this is presented in the supplementary
materials.

Figure~\ref{fig:SSError_overall_in_stewart_plotkin} shows the \(\text{SSError}\)
values for all the strategies in the tournament, as reported
in~\cite{Stewart2012} the extortionate strategy (which has an expected
\(\text{SSError}\) approximately 0) gains a large number of wins.

\begin{figure}[!htbp]
    \centering
    \includegraphics[width=.8\textwidth]{./assets/img/SSError_overall_in_stewart_plotkin/main.pdf}
    \caption{\(\text{SSError}\) and state probabilities for the strategies
        of~\cite{Stewart2012}, ordered both by number of wins and overall score.
        Note that \(P(DC)\) is not shown as it corresponds to the transpose of
        \(P(CD)\). Cooperator and Defector are omitted as they do not visit all
        the states.}
    \label{fig:SSError_overall_in_stewart_plotkin}
\end{figure}

Here, the work of~\cite{Stewart2012} is extended by investigating a tournament
with \input{assets/tex/number_of_full_strategies/main.tex}
strategies.

The results of this analysis are shown in
Figure~\ref{fig:SSError_and_probabilities_in_full}. The top ranking strategies
by number of wins seem to be extortionate (but not against all strategies) and
it can be seen that a small sub group of strategies achieve mutual defection.
All the top ranking strategies according to score achieve mutual cooperation and
do not extort each other, however they
\textbf{do} exhibit extortionate behaviour towards a number of the lower ranking
strategies.

\begin{figure}[!htbp]
    \centering
    \includegraphics[width=.8\textwidth]{./assets/img/SSError_and_probabilities_in_full/main.pdf}
    \caption{\(\text{SSError}\) for the strategies for the full tournament. Only
    strategy interactions for which \(p_4=0\) and \(\chi>1\) are displayed.}
    \label{fig:SSError_and_probabilities_in_full}
\end{figure}

\section{Conclusion}\label{sec:conclusion}

This work defines an approach to measure whether or not a player is playing a
strategy that corresponds to an extortionate strategy as defined
in~\cite{Press2012}: a mathematical model for suspicion. Indeed, all
extortionate strategies have been
 classified as lying on a triangular plane.
This rigorous classification fails to be robust to small measurement error, thus
a statistical approach is proposed.
This is done through a linear algebraic approach for approximating the solution
of a linear system. Using this, a large number of pairwise interactions is
simulated and in fact very few strategies are found to act extortionately.

The work of~\cite{Press2012}, whilst showing that a clever approach to taking
advantage of another memory one strategy exists: this is incomplete. Whilst the
elegance of this result is very attractive, just as the simplicity of the
victory of Tit For Tat in Axelrod's original tournaments was, it is incomplete.
Extortionate strategies achieve a high number of wins but they do not
achieve a high score which corresponds to the fitness landscape in an
evolutionary sense. From the large number of interactions a payoff matrix \(S\)
can be measured where \(S_{ij}\) denotes the score (using standard values of
\((R, S, T, P) = (3, 0, 5, 1)\)) of the \(i\)th strategy
against the \(j\)th strategy. Using this, the replicator equation
describes the evolution of the system based on a population density fitness
function:

\begin{equation}\label{eqn:replicator_dynamics}
    \frac{dx}{dt} = x(S-x^TS x)
\end{equation}

Equation (\ref{eqn:replicator_dynamics}) is solved numerically through an
integration technique described in~\cite{Petzold1983} and
Figure~\ref{fig:replicator_dynamics} shows the evolution of the distribution of
the system: the various strategies are ranked by scores. It is clear to see that
only the high ranking strategies survive the evolutionary process (in fact,
only \input{./assets/img/replicator_dynamics/main.tex}
have a final distribution greater than \(10 ^ {-2}\)). This confirms the
findings of~\cite{Moran1707} in which sophisticated strategies resist
evolutionary invasion of shorter memory strategies. Recalling
Figure~\ref{fig:SSError_and_probabilities_in_full} this demonstrates that:

\begin{itemize}
    \item Cooperation emerges through the evolutionary process: the high scoring
        strategies do not exhibit extortionate behaviour towards each other.
    \item Extortionate strategies do not survive the evolutionary process.
\end{itemize}

\begin{figure}[!htbp]
    \centering
    \includegraphics[width=.8\textwidth]{./assets/img/replicator_dynamics/main.pdf}
    \caption{Numerical simulation of the replicator equation
    (\ref{eqn:replicator_dynamics}): strategies are ordered by score, only the strategies with a high score survive the evolutionary process.}
    \label{fig:replicator_dynamics}
\end{figure}

This work can be used to classify plays of the IPD\@: data can be collected from
actual interactions (in lab or in the field). Furthermore, this allows for a
classification method similar to the notion of fingerprinting presented
in~\cite{Ashlock2008}. Trained strategies can potentially be classified as
extortionate or not or it could be possible to even constrain the reinforcement
learning approaches that are becoming prevalent in the literature.
Alternatively, this mathematical approach for recognising extortion could be
used in sophisticated strategies to defend against invasion. Arguably, some of
the strategies considered here exhibit this behaviour, indeed as described
in~\cite{Harper2017}, the top ranking strategies in the full tournament are
obtained using evolutionary reinforcement learning techniques, thus, suspicion
of extortionate behaviour could in fact be an evolutionary trait.

\section*{Acknowledgements}

The following open source software libraries were used in this research:

\begin{itemize}
    \item The Axelrod ~\cite{Knight2016, Knight2018} library (IPD strategies and
        tournaments).
    \item The sympy library~\cite{Meurer2017} (verification of all symbolic
        calculations).
    \item The matplotlib~\cite{Droettboom2018} library (visualisation).
    \item The pandas~\cite{Structures2010}, dask~\cite{Dask2016} and
        NumPy~\cite{Oliphant2015} libraries (data manipulation).
    \item The SciPy~\cite{Jones2001} library (numerical integration of the
        replicator equation).
\end{itemize}

This work was performed using the computational facilities of the Advanced
Research Computing @ Cardiff (ARCCA) Division, Cardiff University.

\printbibliography

\newpage
\section*{Supplementary materials}

\includepdf{assets/pdf/proof_of_form_of_extortionate_strategies/main.pdf}

\newpage

Using the pair wise interactions the transition rates \(p,
q\) can be measured and the steady state probabilities inferred and compared to
the actual probabilities of each state.
This is done numerically by computing the singular eigenvector of the
matrix \(A\) \cite{Stewart2009}:

\[
    A =
    \begin{bmatrix}
        p_1 q_1 & p_1 (1 - q_1) & (1 - p_1) q_1 & (1 -p_1) (1 - q_1) \\
        p_2 q_2 & p_2 (1 - q_2) & (1 - p_2) q_2 & (1 -p_2) (1 - q_2) \\
        p_3 q_3 & p_3 (1 - q_3) & (1 - p_3) q_3 & (1 -p_3) (1 - q_3) \\
        p_4 q_4 & p_4 (1 - q_4) & (1 - p_4) q_4 & (1 -p_4) (1 - q_4) \\
    \end{bmatrix}
\]

Figure~\ref{fig:computed_probabilities_vs_theoretic_probabilities} shows a
regression line fitted to every pairwise interaction with a reported
\(\text{SSError}\) value (pairwise interactions with missing states were
omitted). This serves to validate the approach: a part from some edge cases the
relationship is consistent.

\begin{figure}[!htbp]
    \centering
    \includegraphics[width=.8\textwidth]{./assets/img/computed_probabilities_vs_theoretic_probabilities/main.pdf}
    \caption{The
        relationship between the steady state probabilities inferred from the
        measured transitions and the actual steady state probabilities. A linear
        regression line is included validating the approach.}
    \label{fig:computed_probabilities_vs_theoretic_probabilities}
\end{figure}


\end{document}

strategies.

The results of this analysis are shown in
Figure~\ref{fig:SSError_and_probabilities_in_full}. The top ranking strategies
by number of wins seem to be extortionate (but not against all strategies) and
it can be seen that a small sub group of strategies achieve mutual defection.
All the top ranking strategies according to score achieve mutual cooperation and
do not extort each other, however they
\textbf{do} exhibit extortionate behaviour towards a number of the lower ranking
strategies.

\begin{figure}[!htbp]
    \centering
    \includegraphics[width=.8\textwidth]{./assets/img/SSError_and_probabilities_in_full/main.pdf}
    \caption{\(\text{SSError}\) for the strategies for the full tournament. Only
    strategy interactions for which \(p_4=0\) and \(\chi>1\) are displayed.}
    \label{fig:SSError_and_probabilities_in_full}
\end{figure}

\section{Conclusion}\label{sec:conclusion}

This work defines an approach to measure whether or not a player is playing a
strategy that corresponds to an extortionate strategy as defined
in~\cite{Press2012}: a mathematical model for suspicion. Indeed, all
extortionate strategies have been
 classified as lying on a triangular plane.
This rigorous classification fails to be robust to small measurement error, thus
a statistical approach is proposed.
This is done through a linear algebraic approach for approximating the solution
of a linear system. Using this, a large number of pairwise interactions is
simulated and in fact very few strategies are found to act extortionately.

The work of~\cite{Press2012}, whilst showing that a clever approach to taking
advantage of another memory one strategy exists: this is incomplete. Whilst the
elegance of this result is very attractive, just as the simplicity of the
victory of Tit For Tat in Axelrod's original tournaments was, it is incomplete.
Extortionate strategies achieve a high number of wins but they do not
achieve a high score which corresponds to the fitness landscape in an
evolutionary sense. From the large number of interactions a payoff matrix \(S\)
can be measured where \(S_{ij}\) denotes the score (using standard values of
\((R, S, T, P) = (3, 0, 5, 1)\)) of the \(i\)th strategy
against the \(j\)th strategy. Using this, the replicator equation
describes the evolution of the system based on a population density fitness
function:

\begin{equation}\label{eqn:replicator_dynamics}
    \frac{dx}{dt} = x(S-x^TS x)
\end{equation}

Equation (\ref{eqn:replicator_dynamics}) is solved numerically through an
integration technique described in~\cite{Petzold1983} and
Figure~\ref{fig:replicator_dynamics} shows the evolution of the distribution of
the system: the various strategies are ranked by scores. It is clear to see that
only the high ranking strategies survive the evolutionary process (in fact,
only \documentclass[a4paper]{article}

\usepackage{amsmath}
\usepackage{amssymb}
\usepackage[margin=1.5cm,
            includefoot,
            footskip=30pt]{geometry}
\usepackage{layout}
\usepackage{graphicx}
\usepackage{subcaption}

\usepackage{biblatex}
\usepackage{pdfpages}

\bibliography{main.bib}

\title{Suspicion: Recognising and evaluating the effectiveness
       of extortion in the Iterated Prisoner's Dilemma}
\author{Vincent A. Knight \and Nikoleta E. Glynatsi}
\date{\today}



\begin{document}

\maketitle

\begin{abstract}
    The Iterated Prisoner's Dilemma is a model for rational and evolutionary
    interactive behaviour. It has applications both in the study of human social
    behaviour as well as in biology.
    It is used to understand when and how a rational individual might
    accept an immediate cost to their own utility for the direct benefit of
    another.

    Much attention has been given to a class of strategies called
    Zero Determinant strategies. It has been theoretically shown that these
    strategies can ``extort'' any player.

    In this work, an approach to identify if observed strategies are playing in
    an extortionate way is described. Furthermore, experimental analysis of
    a large tournament with \input{assets/tex/number_of_full_strategies/main.tex}
    strategies is considered. In this setting
    the most highly performing strategies do not play in an extortionate way
    against each other but do against lower performing strategies.
    This suggests that whilst the theory of Zero Determinant strategies
    indicates that memory is not of fundamental importance to the evolution of
    cooperative behaviour, this is incomplete.
\end{abstract}

\section{Introduction}\label{sec:introduction}

Agent based game theoretic models have become a stalwart of the underpinning
mathematics of interactive behaviours. One of the major pieces of work
in this area is the pair of original computer tournaments run by Robert
Axelrod~\cite{Axelrod1980, Axelrod1980a}. These tournaments pitted submitted
computer strategies against each other in plays of the Iterated Prisoner's
Dilemma. A common game where agents can choose to pay a slight cost to their
immediate utility in the hope of building a reputation. This has been used in
economic and evolutionary game theory to understand the evolution of cooperative
behaviour.

Recently, a class of strategies was described in~\cite{Press2012} that can
provably extort any given opponent. In~\cite{Hilbe2013, Moran1707} some
questions have already been asked about the true effectiveness of these
strategies in an evolutionary setting. Here another question is asked: is it
possible to recognise this extortionate behaviour? A mathematical procedure for
suspicion is presented: in the same way that the continued actions of an
extortionate individual might raise suspicion.

This work makes use of the Axelrod Python library~\cite{Knight2018, Knight2016}
with a large number of Prisoner Dilemma strategies available to give an
extensive numerical example of the ideas presented.  The approach is presented
in Section~\ref{sec:delta-zd-strategies}.  All of the code and data discussed
in Section~\ref{sec:numerical-experiments} is open sourced, archived and
written according to best scientific principles~\cite{Wilson2014}. The data
archive can be found at~\cite{vincent_knight_2018_1297075}.

\section{Recognising Extortion}\label{sec:delta-zd-strategies}

In~\cite{Press2012}, given a match between 2 memory-one strategies, the concept
of Zero Determinant (ZD) strategies is introduced. The main result of that paper
shows that given two memory one players \(p, q\in\mathbb{R}^4\) a linear
relationship between the players' scores could be forced by one of the players.

Using the notation of~\cite{Press2012}, assuming the utilities for player \(p\)
are given by \(S_x=(R, S, T, P)\) and for player \(q\) by \(S_y=(R, T, S, P)\)
and that the stationary scores of each player is given by \(S_X\) and \(S_Y\)
respectively. The main result of~\cite{Press2012} is that if

\begin{equation}\label{eqn:linear_relationship_for_p}
    \tilde p=\alpha S_x + \beta S_y + \gamma
\end{equation}

or

\begin{equation}\label{eqn:linear_relationship_for_q}
    \tilde q=\alpha S_x + \beta S_y + \gamma
\end{equation}

where \(\tilde p = (1 - p_1, 1 - p_2, p_3, p_4)\) and
\(\tilde q = (1 - q_1, 1 - q_2, q_3, q_4)\) then:

\begin{equation}
    \alpha S_X + \beta S_Y + \gamma = 0
\end{equation}

In~\cite{Press2012} a particular type of ZD strategy is defined: extortionate
strategies. If:

\begin{equation}\label{eqn:constraint_for_extortion}
    \gamma = - P(\alpha + \beta)
\end{equation}

then the player can ensure they get a score \(\chi\) times
larger than the opponent. This extortion coefficient is given by:

\begin{equation}\label{eqn:definition_of_chi}
    \chi=\frac{-\beta}{\alpha}
\end{equation}

Thus, if (\ref{eqn:constraint_for_extortion}) holds and \(\chi >1\) a player is
said to extort their opponent.
Here, the reverse problem is considered: given a
\(p\in\mathbb{R}^4\) how does one identify \(\alpha, \beta\) if they
exist and is the strategy in fact acting in an extortionate way?

These conditions correspond to:

\begin{align}
    \tilde p_1 & = \alpha R + \beta R - P (\alpha + \beta)
            \label{eqn:condition_for_tilde_p1}\\
    \tilde p_2 & = \alpha S + \beta T - P (\alpha + \beta)
            \label{eqn:condition_for_tilde_p2}\\
    \tilde p_3 & = \alpha T + \beta S - P (\alpha + \beta)
            \label{eqn:condition_for_tilde_p3}\\
    \tilde p_4 & = \alpha P + \beta P - P (\alpha + \beta)
            \label{eqn:condition_for_tilde_p4}
\end{align}

Equation (\ref{eqn:condition_for_tilde_p4}) ensures that \(p_4=\tilde p_4=0\).
Equations (\ref{eqn:condition_for_tilde_p1}-\ref{eqn:condition_for_tilde_p3})
can be used to eliminate \(\alpha, \beta\), giving:

\begin{equation}\label{eqn:planar_definition_of_extortion}
    \tilde p_1 = \frac{(R - P)(\tilde p_2 + \tilde p_3)}{S + T - 2P}
\end{equation}

with:

\begin{equation}\label{eqn:definition_of_chi}
    \chi = \frac{\tilde p_2 (P - T) + \tilde p_3 (S - P)}
                {\tilde p_2 (P - S) + \tilde p_3 (T - P)}
\end{equation}

Given a strategy \(p\in\mathbb{R}^{4\times 1}\) equations
(\ref{eqn:condition_for_tilde_p4}), (\ref{eqn:planar_definition_of_extortion}-\ref{eqn:definition_of_chi}) can be used to check if
a strategy is extortionate. The conditions correspond to:

\begin{align}
    p_1 & = \frac{(R-P)(p_2 + p_3) - R + T + S - P}{S + T - 2P}
     \label{eqn:condition_for_p1}\\
    p_4 & = 0 \label{eqn:condition_for_p4}\\
    1 & > p_2 + p_3\label{eqn:condition_for_chi}
\end{align}

The algebraic steps necessary to prove these results are available in the
supporting materials.

All extortionate strategies reside on a triangular (\ref{eqn:condition_for_chi})
plane (\ref{eqn:condition_for_p1}) in 3 dimensions (\ref{eqn:condition_for_p4}).
Using this formulation it can be seen that a necessary (but not sufficient)
condition for an extortionate strategy is that it cooperates on average less
than 50\% of the time when in a state of disagreement with the opponent.

As an example, consider the known extortionate strategy \(p=(8 / 9, 1 / 2, 1 /
3, 0)\) from~\cite{Stewart2012} which is referred to as \texttt{Extort-2}. In
this case, for the standard values of \((R, T, S, P)\) constraint
(\ref{eqn:condition_for_p1}) corresponds to:

\begin{equation}
    p_1 = \frac{2(p_2 + p_3) + 1}{3}
\end{equation}

It is clear that in this case all constraints hold.

This approach could in fact be used to confirm that a given strategy is acting
in an extortionate manner even if it is not a memory one strategy. However, in
practice, if a closed form for \(p\) is not known, then due to measurement
and/or numerical error this would not work.

This problem can be written in the following linear algebraic form where
\(x=(\alpha, \beta)\)
and \(p^*=(\tilde p_1 - 1, tilde_2 - 1, p_3)\):

\begin{equation}\label{eqn:linear_algebraic_equation_for_p}
    Cx= p^*
\end{equation}

\(C\) corresponds to equations
(\ref{eqn:condition_for_tilde_p1}-\ref{eqn:condition_for_tilde_p3}) and is
given by:

\begin{equation}\label{eqn:definition_of_C}
    C =
    \begin{bmatrix}
        R - P & R- P \\
        S - P & T- P \\
        T - P & S- P \\
    \end{bmatrix}
\end{equation}

Note that in general, equation (\ref{eqn:linear_algebraic_equation_for_p}) will
not necessarily have a solution. From the Rouch\'{e}-Capelli theorem if there is
a solution it is unique as \(\text{rank}(C)=2\) which is the dimension of the
variable \(x\). The best fitting \(x\) is found by minimizing:

\begin{equation}\label{eqn:r_squared}
    \text{SSError} = \|C x- p^*\|_2^2 = \sum_{i=1}^{3}\left((C\bar x)_i-p_i^*\right)^2
\end{equation}

Note that \(\text{SSError}\), which is the square of the Frobenius
norm~\cite{Golub2013}, becomes a measure of how close a strategy is to being an
extortionate strategy. Suspicion
of extortion then corresponds to a threshold on \(\text{SSError}\).

By observing interactions (human or otherwise), their memory one representation
can be inferred and this approach can be used to recognise extortionate
behaviour. The notion of comparing theoretic and actual plays of the IPD is not
novel, see for example~\cite{Rand2013}. Immediately it is noted that if the
environment is noisy~\cite{Wu1995} then no strategy can be considered to be
extortionate as \(p_4>0\).

In the next section, this idea will be illustrated by observing the interactions
that take place in a computer based tournament of the IPD\@.

\section{Numerical experiments}\label{sec:numerical-experiments}

In~\cite{Stewart2012} results from a tournament with
\input{./assets/tex/number_of_stewart_plotkin_strategies/main.tex} strategies,
was presented with specific consideration given to ZD strategies. This
tournament is reproduced here using the Axelrod-Python
project~\cite{Knight2016}. To obtain a good measure of the corresponding
transition rates for each strategy all matches have been run for
\input{assets/tex/number_of_turns/main.tex} turns and every match has been
repeated \input{assets/tex/number_of_repetitions/main.tex} times. All of this
interaction data is available at~\cite{vincent_knight_2018_1297075}. A good
match between the inferred Markov chain and the state distribution of the actual
interactions has been verified. Data for this is presented in the supplementary
materials.

Figure~\ref{fig:SSError_overall_in_stewart_plotkin} shows the \(\text{SSError}\)
values for all the strategies in the tournament, as reported
in~\cite{Stewart2012} the extortionate strategy (which has an expected
\(\text{SSError}\) approximately 0) gains a large number of wins.

\begin{figure}[!htbp]
    \centering
    \includegraphics[width=.8\textwidth]{./assets/img/SSError_overall_in_stewart_plotkin/main.pdf}
    \caption{\(\text{SSError}\) and state probabilities for the strategies
        of~\cite{Stewart2012}, ordered both by number of wins and overall score.
        Note that \(P(DC)\) is not shown as it corresponds to the transpose of
        \(P(CD)\). Cooperator and Defector are omitted as they do not visit all
        the states.}
    \label{fig:SSError_overall_in_stewart_plotkin}
\end{figure}

Here, the work of~\cite{Stewart2012} is extended by investigating a tournament
with \input{assets/tex/number_of_full_strategies/main.tex}
strategies.

The results of this analysis are shown in
Figure~\ref{fig:SSError_and_probabilities_in_full}. The top ranking strategies
by number of wins seem to be extortionate (but not against all strategies) and
it can be seen that a small sub group of strategies achieve mutual defection.
All the top ranking strategies according to score achieve mutual cooperation and
do not extort each other, however they
\textbf{do} exhibit extortionate behaviour towards a number of the lower ranking
strategies.

\begin{figure}[!htbp]
    \centering
    \includegraphics[width=.8\textwidth]{./assets/img/SSError_and_probabilities_in_full/main.pdf}
    \caption{\(\text{SSError}\) for the strategies for the full tournament. Only
    strategy interactions for which \(p_4=0\) and \(\chi>1\) are displayed.}
    \label{fig:SSError_and_probabilities_in_full}
\end{figure}

\section{Conclusion}\label{sec:conclusion}

This work defines an approach to measure whether or not a player is playing a
strategy that corresponds to an extortionate strategy as defined
in~\cite{Press2012}: a mathematical model for suspicion. Indeed, all
extortionate strategies have been
 classified as lying on a triangular plane.
This rigorous classification fails to be robust to small measurement error, thus
a statistical approach is proposed.
This is done through a linear algebraic approach for approximating the solution
of a linear system. Using this, a large number of pairwise interactions is
simulated and in fact very few strategies are found to act extortionately.

The work of~\cite{Press2012}, whilst showing that a clever approach to taking
advantage of another memory one strategy exists: this is incomplete. Whilst the
elegance of this result is very attractive, just as the simplicity of the
victory of Tit For Tat in Axelrod's original tournaments was, it is incomplete.
Extortionate strategies achieve a high number of wins but they do not
achieve a high score which corresponds to the fitness landscape in an
evolutionary sense. From the large number of interactions a payoff matrix \(S\)
can be measured where \(S_{ij}\) denotes the score (using standard values of
\((R, S, T, P) = (3, 0, 5, 1)\)) of the \(i\)th strategy
against the \(j\)th strategy. Using this, the replicator equation
describes the evolution of the system based on a population density fitness
function:

\begin{equation}\label{eqn:replicator_dynamics}
    \frac{dx}{dt} = x(S-x^TS x)
\end{equation}

Equation (\ref{eqn:replicator_dynamics}) is solved numerically through an
integration technique described in~\cite{Petzold1983} and
Figure~\ref{fig:replicator_dynamics} shows the evolution of the distribution of
the system: the various strategies are ranked by scores. It is clear to see that
only the high ranking strategies survive the evolutionary process (in fact,
only \input{./assets/img/replicator_dynamics/main.tex}
have a final distribution greater than \(10 ^ {-2}\)). This confirms the
findings of~\cite{Moran1707} in which sophisticated strategies resist
evolutionary invasion of shorter memory strategies. Recalling
Figure~\ref{fig:SSError_and_probabilities_in_full} this demonstrates that:

\begin{itemize}
    \item Cooperation emerges through the evolutionary process: the high scoring
        strategies do not exhibit extortionate behaviour towards each other.
    \item Extortionate strategies do not survive the evolutionary process.
\end{itemize}

\begin{figure}[!htbp]
    \centering
    \includegraphics[width=.8\textwidth]{./assets/img/replicator_dynamics/main.pdf}
    \caption{Numerical simulation of the replicator equation
    (\ref{eqn:replicator_dynamics}): strategies are ordered by score, only the strategies with a high score survive the evolutionary process.}
    \label{fig:replicator_dynamics}
\end{figure}

This work can be used to classify plays of the IPD\@: data can be collected from
actual interactions (in lab or in the field). Furthermore, this allows for a
classification method similar to the notion of fingerprinting presented
in~\cite{Ashlock2008}. Trained strategies can potentially be classified as
extortionate or not or it could be possible to even constrain the reinforcement
learning approaches that are becoming prevalent in the literature.
Alternatively, this mathematical approach for recognising extortion could be
used in sophisticated strategies to defend against invasion. Arguably, some of
the strategies considered here exhibit this behaviour, indeed as described
in~\cite{Harper2017}, the top ranking strategies in the full tournament are
obtained using evolutionary reinforcement learning techniques, thus, suspicion
of extortionate behaviour could in fact be an evolutionary trait.

\section*{Acknowledgements}

The following open source software libraries were used in this research:

\begin{itemize}
    \item The Axelrod ~\cite{Knight2016, Knight2018} library (IPD strategies and
        tournaments).
    \item The sympy library~\cite{Meurer2017} (verification of all symbolic
        calculations).
    \item The matplotlib~\cite{Droettboom2018} library (visualisation).
    \item The pandas~\cite{Structures2010}, dask~\cite{Dask2016} and
        NumPy~\cite{Oliphant2015} libraries (data manipulation).
    \item The SciPy~\cite{Jones2001} library (numerical integration of the
        replicator equation).
\end{itemize}

This work was performed using the computational facilities of the Advanced
Research Computing @ Cardiff (ARCCA) Division, Cardiff University.

\printbibliography

\newpage
\section*{Supplementary materials}

\includepdf{assets/pdf/proof_of_form_of_extortionate_strategies/main.pdf}

\newpage

Using the pair wise interactions the transition rates \(p,
q\) can be measured and the steady state probabilities inferred and compared to
the actual probabilities of each state.
This is done numerically by computing the singular eigenvector of the
matrix \(A\) \cite{Stewart2009}:

\[
    A =
    \begin{bmatrix}
        p_1 q_1 & p_1 (1 - q_1) & (1 - p_1) q_1 & (1 -p_1) (1 - q_1) \\
        p_2 q_2 & p_2 (1 - q_2) & (1 - p_2) q_2 & (1 -p_2) (1 - q_2) \\
        p_3 q_3 & p_3 (1 - q_3) & (1 - p_3) q_3 & (1 -p_3) (1 - q_3) \\
        p_4 q_4 & p_4 (1 - q_4) & (1 - p_4) q_4 & (1 -p_4) (1 - q_4) \\
    \end{bmatrix}
\]

Figure~\ref{fig:computed_probabilities_vs_theoretic_probabilities} shows a
regression line fitted to every pairwise interaction with a reported
\(\text{SSError}\) value (pairwise interactions with missing states were
omitted). This serves to validate the approach: a part from some edge cases the
relationship is consistent.

\begin{figure}[!htbp]
    \centering
    \includegraphics[width=.8\textwidth]{./assets/img/computed_probabilities_vs_theoretic_probabilities/main.pdf}
    \caption{The
        relationship between the steady state probabilities inferred from the
        measured transitions and the actual steady state probabilities. A linear
        regression line is included validating the approach.}
    \label{fig:computed_probabilities_vs_theoretic_probabilities}
\end{figure}


\end{document}

have a final distribution greater than \(10 ^ {-2}\)). This confirms the
findings of~\cite{Moran1707} in which sophisticated strategies resist
evolutionary invasion of shorter memory strategies. Recalling
Figure~\ref{fig:SSError_and_probabilities_in_full} this demonstrates that:

\begin{itemize}
    \item Cooperation emerges through the evolutionary process: the high scoring
        strategies do not exhibit extortionate behaviour towards each other.
    \item Extortionate strategies do not survive the evolutionary process.
\end{itemize}

\begin{figure}[!htbp]
    \centering
    \includegraphics[width=.8\textwidth]{./assets/img/replicator_dynamics/main.pdf}
    \caption{Numerical simulation of the replicator equation
    (\ref{eqn:replicator_dynamics}): strategies are ordered by score, only the strategies with a high score survive the evolutionary process.}
    \label{fig:replicator_dynamics}
\end{figure}

This work can be used to classify plays of the IPD\@: data can be collected from
actual interactions (in lab or in the field). Furthermore, this allows for a
classification method similar to the notion of fingerprinting presented
in~\cite{Ashlock2008}. Trained strategies can potentially be classified as
extortionate or not or it could be possible to even constrain the reinforcement
learning approaches that are becoming prevalent in the literature.
Alternatively, this mathematical approach for recognising extortion could be
used in sophisticated strategies to defend against invasion. Arguably, some of
the strategies considered here exhibit this behaviour, indeed as described
in~\cite{Harper2017}, the top ranking strategies in the full tournament are
obtained using evolutionary reinforcement learning techniques, thus, suspicion
of extortionate behaviour could in fact be an evolutionary trait.

\section*{Acknowledgements}

The following open source software libraries were used in this research:

\begin{itemize}
    \item The Axelrod ~\cite{Knight2016, Knight2018} library (IPD strategies and
        tournaments).
    \item The sympy library~\cite{Meurer2017} (verification of all symbolic
        calculations).
    \item The matplotlib~\cite{Droettboom2018} library (visualisation).
    \item The pandas~\cite{Structures2010}, dask~\cite{Dask2016} and
        NumPy~\cite{Oliphant2015} libraries (data manipulation).
    \item The SciPy~\cite{Jones2001} library (numerical integration of the
        replicator equation).
\end{itemize}

This work was performed using the computational facilities of the Advanced
Research Computing @ Cardiff (ARCCA) Division, Cardiff University.

\printbibliography

\newpage
\section*{Supplementary materials}

\includepdf{assets/pdf/proof_of_form_of_extortionate_strategies/main.pdf}

\newpage

Using the pair wise interactions the transition rates \(p,
q\) can be measured and the steady state probabilities inferred and compared to
the actual probabilities of each state.
This is done numerically by computing the singular eigenvector of the
matrix \(A\) \cite{Stewart2009}:

\[
    A =
    \begin{bmatrix}
        p_1 q_1 & p_1 (1 - q_1) & (1 - p_1) q_1 & (1 -p_1) (1 - q_1) \\
        p_2 q_2 & p_2 (1 - q_2) & (1 - p_2) q_2 & (1 -p_2) (1 - q_2) \\
        p_3 q_3 & p_3 (1 - q_3) & (1 - p_3) q_3 & (1 -p_3) (1 - q_3) \\
        p_4 q_4 & p_4 (1 - q_4) & (1 - p_4) q_4 & (1 -p_4) (1 - q_4) \\
    \end{bmatrix}
\]

Figure~\ref{fig:computed_probabilities_vs_theoretic_probabilities} shows a
regression line fitted to every pairwise interaction with a reported
\(\text{SSError}\) value (pairwise interactions with missing states were
omitted). This serves to validate the approach: a part from some edge cases the
relationship is consistent.

\begin{figure}[!htbp]
    \centering
    \includegraphics[width=.8\textwidth]{./assets/img/computed_probabilities_vs_theoretic_probabilities/main.pdf}
    \caption{The
        relationship between the steady state probabilities inferred from the
        measured transitions and the actual steady state probabilities. A linear
        regression line is included validating the approach.}
    \label{fig:computed_probabilities_vs_theoretic_probabilities}
\end{figure}


\end{document}
 turns and every match has been
repeated \documentclass[a4paper]{article}

\usepackage{amsmath}
\usepackage{amssymb}
\usepackage[margin=1.5cm,
            includefoot,
            footskip=30pt]{geometry}
\usepackage{layout}
\usepackage{graphicx}
\usepackage{subcaption}

\usepackage{biblatex}
\usepackage{pdfpages}

\bibliography{main.bib}

\title{Suspicion: Recognising and evaluating the effectiveness
       of extortion in the Iterated Prisoner's Dilemma}
\author{Vincent A. Knight \and Nikoleta E. Glynatsi}
\date{\today}



\begin{document}

\maketitle

\begin{abstract}
    The Iterated Prisoner's Dilemma is a model for rational and evolutionary
    interactive behaviour. It has applications both in the study of human social
    behaviour as well as in biology.
    It is used to understand when and how a rational individual might
    accept an immediate cost to their own utility for the direct benefit of
    another.

    Much attention has been given to a class of strategies called
    Zero Determinant strategies. It has been theoretically shown that these
    strategies can ``extort'' any player.

    In this work, an approach to identify if observed strategies are playing in
    an extortionate way is described. Furthermore, experimental analysis of
    a large tournament with \documentclass[a4paper]{article}

\usepackage{amsmath}
\usepackage{amssymb}
\usepackage[margin=1.5cm,
            includefoot,
            footskip=30pt]{geometry}
\usepackage{layout}
\usepackage{graphicx}
\usepackage{subcaption}

\usepackage{biblatex}
\usepackage{pdfpages}

\bibliography{main.bib}

\title{Suspicion: Recognising and evaluating the effectiveness
       of extortion in the Iterated Prisoner's Dilemma}
\author{Vincent A. Knight \and Nikoleta E. Glynatsi}
\date{\today}



\begin{document}

\maketitle

\begin{abstract}
    The Iterated Prisoner's Dilemma is a model for rational and evolutionary
    interactive behaviour. It has applications both in the study of human social
    behaviour as well as in biology.
    It is used to understand when and how a rational individual might
    accept an immediate cost to their own utility for the direct benefit of
    another.

    Much attention has been given to a class of strategies called
    Zero Determinant strategies. It has been theoretically shown that these
    strategies can ``extort'' any player.

    In this work, an approach to identify if observed strategies are playing in
    an extortionate way is described. Furthermore, experimental analysis of
    a large tournament with \input{assets/tex/number_of_full_strategies/main.tex}
    strategies is considered. In this setting
    the most highly performing strategies do not play in an extortionate way
    against each other but do against lower performing strategies.
    This suggests that whilst the theory of Zero Determinant strategies
    indicates that memory is not of fundamental importance to the evolution of
    cooperative behaviour, this is incomplete.
\end{abstract}

\section{Introduction}\label{sec:introduction}

Agent based game theoretic models have become a stalwart of the underpinning
mathematics of interactive behaviours. One of the major pieces of work
in this area is the pair of original computer tournaments run by Robert
Axelrod~\cite{Axelrod1980, Axelrod1980a}. These tournaments pitted submitted
computer strategies against each other in plays of the Iterated Prisoner's
Dilemma. A common game where agents can choose to pay a slight cost to their
immediate utility in the hope of building a reputation. This has been used in
economic and evolutionary game theory to understand the evolution of cooperative
behaviour.

Recently, a class of strategies was described in~\cite{Press2012} that can
provably extort any given opponent. In~\cite{Hilbe2013, Moran1707} some
questions have already been asked about the true effectiveness of these
strategies in an evolutionary setting. Here another question is asked: is it
possible to recognise this extortionate behaviour? A mathematical procedure for
suspicion is presented: in the same way that the continued actions of an
extortionate individual might raise suspicion.

This work makes use of the Axelrod Python library~\cite{Knight2018, Knight2016}
with a large number of Prisoner Dilemma strategies available to give an
extensive numerical example of the ideas presented.  The approach is presented
in Section~\ref{sec:delta-zd-strategies}.  All of the code and data discussed
in Section~\ref{sec:numerical-experiments} is open sourced, archived and
written according to best scientific principles~\cite{Wilson2014}. The data
archive can be found at~\cite{vincent_knight_2018_1297075}.

\section{Recognising Extortion}\label{sec:delta-zd-strategies}

In~\cite{Press2012}, given a match between 2 memory-one strategies, the concept
of Zero Determinant (ZD) strategies is introduced. The main result of that paper
shows that given two memory one players \(p, q\in\mathbb{R}^4\) a linear
relationship between the players' scores could be forced by one of the players.

Using the notation of~\cite{Press2012}, assuming the utilities for player \(p\)
are given by \(S_x=(R, S, T, P)\) and for player \(q\) by \(S_y=(R, T, S, P)\)
and that the stationary scores of each player is given by \(S_X\) and \(S_Y\)
respectively. The main result of~\cite{Press2012} is that if

\begin{equation}\label{eqn:linear_relationship_for_p}
    \tilde p=\alpha S_x + \beta S_y + \gamma
\end{equation}

or

\begin{equation}\label{eqn:linear_relationship_for_q}
    \tilde q=\alpha S_x + \beta S_y + \gamma
\end{equation}

where \(\tilde p = (1 - p_1, 1 - p_2, p_3, p_4)\) and
\(\tilde q = (1 - q_1, 1 - q_2, q_3, q_4)\) then:

\begin{equation}
    \alpha S_X + \beta S_Y + \gamma = 0
\end{equation}

In~\cite{Press2012} a particular type of ZD strategy is defined: extortionate
strategies. If:

\begin{equation}\label{eqn:constraint_for_extortion}
    \gamma = - P(\alpha + \beta)
\end{equation}

then the player can ensure they get a score \(\chi\) times
larger than the opponent. This extortion coefficient is given by:

\begin{equation}\label{eqn:definition_of_chi}
    \chi=\frac{-\beta}{\alpha}
\end{equation}

Thus, if (\ref{eqn:constraint_for_extortion}) holds and \(\chi >1\) a player is
said to extort their opponent.
Here, the reverse problem is considered: given a
\(p\in\mathbb{R}^4\) how does one identify \(\alpha, \beta\) if they
exist and is the strategy in fact acting in an extortionate way?

These conditions correspond to:

\begin{align}
    \tilde p_1 & = \alpha R + \beta R - P (\alpha + \beta)
            \label{eqn:condition_for_tilde_p1}\\
    \tilde p_2 & = \alpha S + \beta T - P (\alpha + \beta)
            \label{eqn:condition_for_tilde_p2}\\
    \tilde p_3 & = \alpha T + \beta S - P (\alpha + \beta)
            \label{eqn:condition_for_tilde_p3}\\
    \tilde p_4 & = \alpha P + \beta P - P (\alpha + \beta)
            \label{eqn:condition_for_tilde_p4}
\end{align}

Equation (\ref{eqn:condition_for_tilde_p4}) ensures that \(p_4=\tilde p_4=0\).
Equations (\ref{eqn:condition_for_tilde_p1}-\ref{eqn:condition_for_tilde_p3})
can be used to eliminate \(\alpha, \beta\), giving:

\begin{equation}\label{eqn:planar_definition_of_extortion}
    \tilde p_1 = \frac{(R - P)(\tilde p_2 + \tilde p_3)}{S + T - 2P}
\end{equation}

with:

\begin{equation}\label{eqn:definition_of_chi}
    \chi = \frac{\tilde p_2 (P - T) + \tilde p_3 (S - P)}
                {\tilde p_2 (P - S) + \tilde p_3 (T - P)}
\end{equation}

Given a strategy \(p\in\mathbb{R}^{4\times 1}\) equations
(\ref{eqn:condition_for_tilde_p4}), (\ref{eqn:planar_definition_of_extortion}-\ref{eqn:definition_of_chi}) can be used to check if
a strategy is extortionate. The conditions correspond to:

\begin{align}
    p_1 & = \frac{(R-P)(p_2 + p_3) - R + T + S - P}{S + T - 2P}
     \label{eqn:condition_for_p1}\\
    p_4 & = 0 \label{eqn:condition_for_p4}\\
    1 & > p_2 + p_3\label{eqn:condition_for_chi}
\end{align}

The algebraic steps necessary to prove these results are available in the
supporting materials.

All extortionate strategies reside on a triangular (\ref{eqn:condition_for_chi})
plane (\ref{eqn:condition_for_p1}) in 3 dimensions (\ref{eqn:condition_for_p4}).
Using this formulation it can be seen that a necessary (but not sufficient)
condition for an extortionate strategy is that it cooperates on average less
than 50\% of the time when in a state of disagreement with the opponent.

As an example, consider the known extortionate strategy \(p=(8 / 9, 1 / 2, 1 /
3, 0)\) from~\cite{Stewart2012} which is referred to as \texttt{Extort-2}. In
this case, for the standard values of \((R, T, S, P)\) constraint
(\ref{eqn:condition_for_p1}) corresponds to:

\begin{equation}
    p_1 = \frac{2(p_2 + p_3) + 1}{3}
\end{equation}

It is clear that in this case all constraints hold.

This approach could in fact be used to confirm that a given strategy is acting
in an extortionate manner even if it is not a memory one strategy. However, in
practice, if a closed form for \(p\) is not known, then due to measurement
and/or numerical error this would not work.

This problem can be written in the following linear algebraic form where
\(x=(\alpha, \beta)\)
and \(p^*=(\tilde p_1 - 1, tilde_2 - 1, p_3)\):

\begin{equation}\label{eqn:linear_algebraic_equation_for_p}
    Cx= p^*
\end{equation}

\(C\) corresponds to equations
(\ref{eqn:condition_for_tilde_p1}-\ref{eqn:condition_for_tilde_p3}) and is
given by:

\begin{equation}\label{eqn:definition_of_C}
    C =
    \begin{bmatrix}
        R - P & R- P \\
        S - P & T- P \\
        T - P & S- P \\
    \end{bmatrix}
\end{equation}

Note that in general, equation (\ref{eqn:linear_algebraic_equation_for_p}) will
not necessarily have a solution. From the Rouch\'{e}-Capelli theorem if there is
a solution it is unique as \(\text{rank}(C)=2\) which is the dimension of the
variable \(x\). The best fitting \(x\) is found by minimizing:

\begin{equation}\label{eqn:r_squared}
    \text{SSError} = \|C x- p^*\|_2^2 = \sum_{i=1}^{3}\left((C\bar x)_i-p_i^*\right)^2
\end{equation}

Note that \(\text{SSError}\), which is the square of the Frobenius
norm~\cite{Golub2013}, becomes a measure of how close a strategy is to being an
extortionate strategy. Suspicion
of extortion then corresponds to a threshold on \(\text{SSError}\).

By observing interactions (human or otherwise), their memory one representation
can be inferred and this approach can be used to recognise extortionate
behaviour. The notion of comparing theoretic and actual plays of the IPD is not
novel, see for example~\cite{Rand2013}. Immediately it is noted that if the
environment is noisy~\cite{Wu1995} then no strategy can be considered to be
extortionate as \(p_4>0\).

In the next section, this idea will be illustrated by observing the interactions
that take place in a computer based tournament of the IPD\@.

\section{Numerical experiments}\label{sec:numerical-experiments}

In~\cite{Stewart2012} results from a tournament with
\input{./assets/tex/number_of_stewart_plotkin_strategies/main.tex} strategies,
was presented with specific consideration given to ZD strategies. This
tournament is reproduced here using the Axelrod-Python
project~\cite{Knight2016}. To obtain a good measure of the corresponding
transition rates for each strategy all matches have been run for
\input{assets/tex/number_of_turns/main.tex} turns and every match has been
repeated \input{assets/tex/number_of_repetitions/main.tex} times. All of this
interaction data is available at~\cite{vincent_knight_2018_1297075}. A good
match between the inferred Markov chain and the state distribution of the actual
interactions has been verified. Data for this is presented in the supplementary
materials.

Figure~\ref{fig:SSError_overall_in_stewart_plotkin} shows the \(\text{SSError}\)
values for all the strategies in the tournament, as reported
in~\cite{Stewart2012} the extortionate strategy (which has an expected
\(\text{SSError}\) approximately 0) gains a large number of wins.

\begin{figure}[!htbp]
    \centering
    \includegraphics[width=.8\textwidth]{./assets/img/SSError_overall_in_stewart_plotkin/main.pdf}
    \caption{\(\text{SSError}\) and state probabilities for the strategies
        of~\cite{Stewart2012}, ordered both by number of wins and overall score.
        Note that \(P(DC)\) is not shown as it corresponds to the transpose of
        \(P(CD)\). Cooperator and Defector are omitted as they do not visit all
        the states.}
    \label{fig:SSError_overall_in_stewart_plotkin}
\end{figure}

Here, the work of~\cite{Stewart2012} is extended by investigating a tournament
with \input{assets/tex/number_of_full_strategies/main.tex}
strategies.

The results of this analysis are shown in
Figure~\ref{fig:SSError_and_probabilities_in_full}. The top ranking strategies
by number of wins seem to be extortionate (but not against all strategies) and
it can be seen that a small sub group of strategies achieve mutual defection.
All the top ranking strategies according to score achieve mutual cooperation and
do not extort each other, however they
\textbf{do} exhibit extortionate behaviour towards a number of the lower ranking
strategies.

\begin{figure}[!htbp]
    \centering
    \includegraphics[width=.8\textwidth]{./assets/img/SSError_and_probabilities_in_full/main.pdf}
    \caption{\(\text{SSError}\) for the strategies for the full tournament. Only
    strategy interactions for which \(p_4=0\) and \(\chi>1\) are displayed.}
    \label{fig:SSError_and_probabilities_in_full}
\end{figure}

\section{Conclusion}\label{sec:conclusion}

This work defines an approach to measure whether or not a player is playing a
strategy that corresponds to an extortionate strategy as defined
in~\cite{Press2012}: a mathematical model for suspicion. Indeed, all
extortionate strategies have been
 classified as lying on a triangular plane.
This rigorous classification fails to be robust to small measurement error, thus
a statistical approach is proposed.
This is done through a linear algebraic approach for approximating the solution
of a linear system. Using this, a large number of pairwise interactions is
simulated and in fact very few strategies are found to act extortionately.

The work of~\cite{Press2012}, whilst showing that a clever approach to taking
advantage of another memory one strategy exists: this is incomplete. Whilst the
elegance of this result is very attractive, just as the simplicity of the
victory of Tit For Tat in Axelrod's original tournaments was, it is incomplete.
Extortionate strategies achieve a high number of wins but they do not
achieve a high score which corresponds to the fitness landscape in an
evolutionary sense. From the large number of interactions a payoff matrix \(S\)
can be measured where \(S_{ij}\) denotes the score (using standard values of
\((R, S, T, P) = (3, 0, 5, 1)\)) of the \(i\)th strategy
against the \(j\)th strategy. Using this, the replicator equation
describes the evolution of the system based on a population density fitness
function:

\begin{equation}\label{eqn:replicator_dynamics}
    \frac{dx}{dt} = x(S-x^TS x)
\end{equation}

Equation (\ref{eqn:replicator_dynamics}) is solved numerically through an
integration technique described in~\cite{Petzold1983} and
Figure~\ref{fig:replicator_dynamics} shows the evolution of the distribution of
the system: the various strategies are ranked by scores. It is clear to see that
only the high ranking strategies survive the evolutionary process (in fact,
only \input{./assets/img/replicator_dynamics/main.tex}
have a final distribution greater than \(10 ^ {-2}\)). This confirms the
findings of~\cite{Moran1707} in which sophisticated strategies resist
evolutionary invasion of shorter memory strategies. Recalling
Figure~\ref{fig:SSError_and_probabilities_in_full} this demonstrates that:

\begin{itemize}
    \item Cooperation emerges through the evolutionary process: the high scoring
        strategies do not exhibit extortionate behaviour towards each other.
    \item Extortionate strategies do not survive the evolutionary process.
\end{itemize}

\begin{figure}[!htbp]
    \centering
    \includegraphics[width=.8\textwidth]{./assets/img/replicator_dynamics/main.pdf}
    \caption{Numerical simulation of the replicator equation
    (\ref{eqn:replicator_dynamics}): strategies are ordered by score, only the strategies with a high score survive the evolutionary process.}
    \label{fig:replicator_dynamics}
\end{figure}

This work can be used to classify plays of the IPD\@: data can be collected from
actual interactions (in lab or in the field). Furthermore, this allows for a
classification method similar to the notion of fingerprinting presented
in~\cite{Ashlock2008}. Trained strategies can potentially be classified as
extortionate or not or it could be possible to even constrain the reinforcement
learning approaches that are becoming prevalent in the literature.
Alternatively, this mathematical approach for recognising extortion could be
used in sophisticated strategies to defend against invasion. Arguably, some of
the strategies considered here exhibit this behaviour, indeed as described
in~\cite{Harper2017}, the top ranking strategies in the full tournament are
obtained using evolutionary reinforcement learning techniques, thus, suspicion
of extortionate behaviour could in fact be an evolutionary trait.

\section*{Acknowledgements}

The following open source software libraries were used in this research:

\begin{itemize}
    \item The Axelrod ~\cite{Knight2016, Knight2018} library (IPD strategies and
        tournaments).
    \item The sympy library~\cite{Meurer2017} (verification of all symbolic
        calculations).
    \item The matplotlib~\cite{Droettboom2018} library (visualisation).
    \item The pandas~\cite{Structures2010}, dask~\cite{Dask2016} and
        NumPy~\cite{Oliphant2015} libraries (data manipulation).
    \item The SciPy~\cite{Jones2001} library (numerical integration of the
        replicator equation).
\end{itemize}

This work was performed using the computational facilities of the Advanced
Research Computing @ Cardiff (ARCCA) Division, Cardiff University.

\printbibliography

\newpage
\section*{Supplementary materials}

\includepdf{assets/pdf/proof_of_form_of_extortionate_strategies/main.pdf}

\newpage

Using the pair wise interactions the transition rates \(p,
q\) can be measured and the steady state probabilities inferred and compared to
the actual probabilities of each state.
This is done numerically by computing the singular eigenvector of the
matrix \(A\) \cite{Stewart2009}:

\[
    A =
    \begin{bmatrix}
        p_1 q_1 & p_1 (1 - q_1) & (1 - p_1) q_1 & (1 -p_1) (1 - q_1) \\
        p_2 q_2 & p_2 (1 - q_2) & (1 - p_2) q_2 & (1 -p_2) (1 - q_2) \\
        p_3 q_3 & p_3 (1 - q_3) & (1 - p_3) q_3 & (1 -p_3) (1 - q_3) \\
        p_4 q_4 & p_4 (1 - q_4) & (1 - p_4) q_4 & (1 -p_4) (1 - q_4) \\
    \end{bmatrix}
\]

Figure~\ref{fig:computed_probabilities_vs_theoretic_probabilities} shows a
regression line fitted to every pairwise interaction with a reported
\(\text{SSError}\) value (pairwise interactions with missing states were
omitted). This serves to validate the approach: a part from some edge cases the
relationship is consistent.

\begin{figure}[!htbp]
    \centering
    \includegraphics[width=.8\textwidth]{./assets/img/computed_probabilities_vs_theoretic_probabilities/main.pdf}
    \caption{The
        relationship between the steady state probabilities inferred from the
        measured transitions and the actual steady state probabilities. A linear
        regression line is included validating the approach.}
    \label{fig:computed_probabilities_vs_theoretic_probabilities}
\end{figure}


\end{document}

    strategies is considered. In this setting
    the most highly performing strategies do not play in an extortionate way
    against each other but do against lower performing strategies.
    This suggests that whilst the theory of Zero Determinant strategies
    indicates that memory is not of fundamental importance to the evolution of
    cooperative behaviour, this is incomplete.
\end{abstract}

\section{Introduction}\label{sec:introduction}

Agent based game theoretic models have become a stalwart of the underpinning
mathematics of interactive behaviours. One of the major pieces of work
in this area is the pair of original computer tournaments run by Robert
Axelrod~\cite{Axelrod1980, Axelrod1980a}. These tournaments pitted submitted
computer strategies against each other in plays of the Iterated Prisoner's
Dilemma. A common game where agents can choose to pay a slight cost to their
immediate utility in the hope of building a reputation. This has been used in
economic and evolutionary game theory to understand the evolution of cooperative
behaviour.

Recently, a class of strategies was described in~\cite{Press2012} that can
provably extort any given opponent. In~\cite{Hilbe2013, Moran1707} some
questions have already been asked about the true effectiveness of these
strategies in an evolutionary setting. Here another question is asked: is it
possible to recognise this extortionate behaviour? A mathematical procedure for
suspicion is presented: in the same way that the continued actions of an
extortionate individual might raise suspicion.

This work makes use of the Axelrod Python library~\cite{Knight2018, Knight2016}
with a large number of Prisoner Dilemma strategies available to give an
extensive numerical example of the ideas presented.  The approach is presented
in Section~\ref{sec:delta-zd-strategies}.  All of the code and data discussed
in Section~\ref{sec:numerical-experiments} is open sourced, archived and
written according to best scientific principles~\cite{Wilson2014}. The data
archive can be found at~\cite{vincent_knight_2018_1297075}.

\section{Recognising Extortion}\label{sec:delta-zd-strategies}

In~\cite{Press2012}, given a match between 2 memory-one strategies, the concept
of Zero Determinant (ZD) strategies is introduced. The main result of that paper
shows that given two memory one players \(p, q\in\mathbb{R}^4\) a linear
relationship between the players' scores could be forced by one of the players.

Using the notation of~\cite{Press2012}, assuming the utilities for player \(p\)
are given by \(S_x=(R, S, T, P)\) and for player \(q\) by \(S_y=(R, T, S, P)\)
and that the stationary scores of each player is given by \(S_X\) and \(S_Y\)
respectively. The main result of~\cite{Press2012} is that if

\begin{equation}\label{eqn:linear_relationship_for_p}
    \tilde p=\alpha S_x + \beta S_y + \gamma
\end{equation}

or

\begin{equation}\label{eqn:linear_relationship_for_q}
    \tilde q=\alpha S_x + \beta S_y + \gamma
\end{equation}

where \(\tilde p = (1 - p_1, 1 - p_2, p_3, p_4)\) and
\(\tilde q = (1 - q_1, 1 - q_2, q_3, q_4)\) then:

\begin{equation}
    \alpha S_X + \beta S_Y + \gamma = 0
\end{equation}

In~\cite{Press2012} a particular type of ZD strategy is defined: extortionate
strategies. If:

\begin{equation}\label{eqn:constraint_for_extortion}
    \gamma = - P(\alpha + \beta)
\end{equation}

then the player can ensure they get a score \(\chi\) times
larger than the opponent. This extortion coefficient is given by:

\begin{equation}\label{eqn:definition_of_chi}
    \chi=\frac{-\beta}{\alpha}
\end{equation}

Thus, if (\ref{eqn:constraint_for_extortion}) holds and \(\chi >1\) a player is
said to extort their opponent.
Here, the reverse problem is considered: given a
\(p\in\mathbb{R}^4\) how does one identify \(\alpha, \beta\) if they
exist and is the strategy in fact acting in an extortionate way?

These conditions correspond to:

\begin{align}
    \tilde p_1 & = \alpha R + \beta R - P (\alpha + \beta)
            \label{eqn:condition_for_tilde_p1}\\
    \tilde p_2 & = \alpha S + \beta T - P (\alpha + \beta)
            \label{eqn:condition_for_tilde_p2}\\
    \tilde p_3 & = \alpha T + \beta S - P (\alpha + \beta)
            \label{eqn:condition_for_tilde_p3}\\
    \tilde p_4 & = \alpha P + \beta P - P (\alpha + \beta)
            \label{eqn:condition_for_tilde_p4}
\end{align}

Equation (\ref{eqn:condition_for_tilde_p4}) ensures that \(p_4=\tilde p_4=0\).
Equations (\ref{eqn:condition_for_tilde_p1}-\ref{eqn:condition_for_tilde_p3})
can be used to eliminate \(\alpha, \beta\), giving:

\begin{equation}\label{eqn:planar_definition_of_extortion}
    \tilde p_1 = \frac{(R - P)(\tilde p_2 + \tilde p_3)}{S + T - 2P}
\end{equation}

with:

\begin{equation}\label{eqn:definition_of_chi}
    \chi = \frac{\tilde p_2 (P - T) + \tilde p_3 (S - P)}
                {\tilde p_2 (P - S) + \tilde p_3 (T - P)}
\end{equation}

Given a strategy \(p\in\mathbb{R}^{4\times 1}\) equations
(\ref{eqn:condition_for_tilde_p4}), (\ref{eqn:planar_definition_of_extortion}-\ref{eqn:definition_of_chi}) can be used to check if
a strategy is extortionate. The conditions correspond to:

\begin{align}
    p_1 & = \frac{(R-P)(p_2 + p_3) - R + T + S - P}{S + T - 2P}
     \label{eqn:condition_for_p1}\\
    p_4 & = 0 \label{eqn:condition_for_p4}\\
    1 & > p_2 + p_3\label{eqn:condition_for_chi}
\end{align}

The algebraic steps necessary to prove these results are available in the
supporting materials.

All extortionate strategies reside on a triangular (\ref{eqn:condition_for_chi})
plane (\ref{eqn:condition_for_p1}) in 3 dimensions (\ref{eqn:condition_for_p4}).
Using this formulation it can be seen that a necessary (but not sufficient)
condition for an extortionate strategy is that it cooperates on average less
than 50\% of the time when in a state of disagreement with the opponent.

As an example, consider the known extortionate strategy \(p=(8 / 9, 1 / 2, 1 /
3, 0)\) from~\cite{Stewart2012} which is referred to as \texttt{Extort-2}. In
this case, for the standard values of \((R, T, S, P)\) constraint
(\ref{eqn:condition_for_p1}) corresponds to:

\begin{equation}
    p_1 = \frac{2(p_2 + p_3) + 1}{3}
\end{equation}

It is clear that in this case all constraints hold.

This approach could in fact be used to confirm that a given strategy is acting
in an extortionate manner even if it is not a memory one strategy. However, in
practice, if a closed form for \(p\) is not known, then due to measurement
and/or numerical error this would not work.

This problem can be written in the following linear algebraic form where
\(x=(\alpha, \beta)\)
and \(p^*=(\tilde p_1 - 1, tilde_2 - 1, p_3)\):

\begin{equation}\label{eqn:linear_algebraic_equation_for_p}
    Cx= p^*
\end{equation}

\(C\) corresponds to equations
(\ref{eqn:condition_for_tilde_p1}-\ref{eqn:condition_for_tilde_p3}) and is
given by:

\begin{equation}\label{eqn:definition_of_C}
    C =
    \begin{bmatrix}
        R - P & R- P \\
        S - P & T- P \\
        T - P & S- P \\
    \end{bmatrix}
\end{equation}

Note that in general, equation (\ref{eqn:linear_algebraic_equation_for_p}) will
not necessarily have a solution. From the Rouch\'{e}-Capelli theorem if there is
a solution it is unique as \(\text{rank}(C)=2\) which is the dimension of the
variable \(x\). The best fitting \(x\) is found by minimizing:

\begin{equation}\label{eqn:r_squared}
    \text{SSError} = \|C x- p^*\|_2^2 = \sum_{i=1}^{3}\left((C\bar x)_i-p_i^*\right)^2
\end{equation}

Note that \(\text{SSError}\), which is the square of the Frobenius
norm~\cite{Golub2013}, becomes a measure of how close a strategy is to being an
extortionate strategy. Suspicion
of extortion then corresponds to a threshold on \(\text{SSError}\).

By observing interactions (human or otherwise), their memory one representation
can be inferred and this approach can be used to recognise extortionate
behaviour. The notion of comparing theoretic and actual plays of the IPD is not
novel, see for example~\cite{Rand2013}. Immediately it is noted that if the
environment is noisy~\cite{Wu1995} then no strategy can be considered to be
extortionate as \(p_4>0\).

In the next section, this idea will be illustrated by observing the interactions
that take place in a computer based tournament of the IPD\@.

\section{Numerical experiments}\label{sec:numerical-experiments}

In~\cite{Stewart2012} results from a tournament with
\documentclass[a4paper]{article}

\usepackage{amsmath}
\usepackage{amssymb}
\usepackage[margin=1.5cm,
            includefoot,
            footskip=30pt]{geometry}
\usepackage{layout}
\usepackage{graphicx}
\usepackage{subcaption}

\usepackage{biblatex}
\usepackage{pdfpages}

\bibliography{main.bib}

\title{Suspicion: Recognising and evaluating the effectiveness
       of extortion in the Iterated Prisoner's Dilemma}
\author{Vincent A. Knight \and Nikoleta E. Glynatsi}
\date{\today}



\begin{document}

\maketitle

\begin{abstract}
    The Iterated Prisoner's Dilemma is a model for rational and evolutionary
    interactive behaviour. It has applications both in the study of human social
    behaviour as well as in biology.
    It is used to understand when and how a rational individual might
    accept an immediate cost to their own utility for the direct benefit of
    another.

    Much attention has been given to a class of strategies called
    Zero Determinant strategies. It has been theoretically shown that these
    strategies can ``extort'' any player.

    In this work, an approach to identify if observed strategies are playing in
    an extortionate way is described. Furthermore, experimental analysis of
    a large tournament with \input{assets/tex/number_of_full_strategies/main.tex}
    strategies is considered. In this setting
    the most highly performing strategies do not play in an extortionate way
    against each other but do against lower performing strategies.
    This suggests that whilst the theory of Zero Determinant strategies
    indicates that memory is not of fundamental importance to the evolution of
    cooperative behaviour, this is incomplete.
\end{abstract}

\section{Introduction}\label{sec:introduction}

Agent based game theoretic models have become a stalwart of the underpinning
mathematics of interactive behaviours. One of the major pieces of work
in this area is the pair of original computer tournaments run by Robert
Axelrod~\cite{Axelrod1980, Axelrod1980a}. These tournaments pitted submitted
computer strategies against each other in plays of the Iterated Prisoner's
Dilemma. A common game where agents can choose to pay a slight cost to their
immediate utility in the hope of building a reputation. This has been used in
economic and evolutionary game theory to understand the evolution of cooperative
behaviour.

Recently, a class of strategies was described in~\cite{Press2012} that can
provably extort any given opponent. In~\cite{Hilbe2013, Moran1707} some
questions have already been asked about the true effectiveness of these
strategies in an evolutionary setting. Here another question is asked: is it
possible to recognise this extortionate behaviour? A mathematical procedure for
suspicion is presented: in the same way that the continued actions of an
extortionate individual might raise suspicion.

This work makes use of the Axelrod Python library~\cite{Knight2018, Knight2016}
with a large number of Prisoner Dilemma strategies available to give an
extensive numerical example of the ideas presented.  The approach is presented
in Section~\ref{sec:delta-zd-strategies}.  All of the code and data discussed
in Section~\ref{sec:numerical-experiments} is open sourced, archived and
written according to best scientific principles~\cite{Wilson2014}. The data
archive can be found at~\cite{vincent_knight_2018_1297075}.

\section{Recognising Extortion}\label{sec:delta-zd-strategies}

In~\cite{Press2012}, given a match between 2 memory-one strategies, the concept
of Zero Determinant (ZD) strategies is introduced. The main result of that paper
shows that given two memory one players \(p, q\in\mathbb{R}^4\) a linear
relationship between the players' scores could be forced by one of the players.

Using the notation of~\cite{Press2012}, assuming the utilities for player \(p\)
are given by \(S_x=(R, S, T, P)\) and for player \(q\) by \(S_y=(R, T, S, P)\)
and that the stationary scores of each player is given by \(S_X\) and \(S_Y\)
respectively. The main result of~\cite{Press2012} is that if

\begin{equation}\label{eqn:linear_relationship_for_p}
    \tilde p=\alpha S_x + \beta S_y + \gamma
\end{equation}

or

\begin{equation}\label{eqn:linear_relationship_for_q}
    \tilde q=\alpha S_x + \beta S_y + \gamma
\end{equation}

where \(\tilde p = (1 - p_1, 1 - p_2, p_3, p_4)\) and
\(\tilde q = (1 - q_1, 1 - q_2, q_3, q_4)\) then:

\begin{equation}
    \alpha S_X + \beta S_Y + \gamma = 0
\end{equation}

In~\cite{Press2012} a particular type of ZD strategy is defined: extortionate
strategies. If:

\begin{equation}\label{eqn:constraint_for_extortion}
    \gamma = - P(\alpha + \beta)
\end{equation}

then the player can ensure they get a score \(\chi\) times
larger than the opponent. This extortion coefficient is given by:

\begin{equation}\label{eqn:definition_of_chi}
    \chi=\frac{-\beta}{\alpha}
\end{equation}

Thus, if (\ref{eqn:constraint_for_extortion}) holds and \(\chi >1\) a player is
said to extort their opponent.
Here, the reverse problem is considered: given a
\(p\in\mathbb{R}^4\) how does one identify \(\alpha, \beta\) if they
exist and is the strategy in fact acting in an extortionate way?

These conditions correspond to:

\begin{align}
    \tilde p_1 & = \alpha R + \beta R - P (\alpha + \beta)
            \label{eqn:condition_for_tilde_p1}\\
    \tilde p_2 & = \alpha S + \beta T - P (\alpha + \beta)
            \label{eqn:condition_for_tilde_p2}\\
    \tilde p_3 & = \alpha T + \beta S - P (\alpha + \beta)
            \label{eqn:condition_for_tilde_p3}\\
    \tilde p_4 & = \alpha P + \beta P - P (\alpha + \beta)
            \label{eqn:condition_for_tilde_p4}
\end{align}

Equation (\ref{eqn:condition_for_tilde_p4}) ensures that \(p_4=\tilde p_4=0\).
Equations (\ref{eqn:condition_for_tilde_p1}-\ref{eqn:condition_for_tilde_p3})
can be used to eliminate \(\alpha, \beta\), giving:

\begin{equation}\label{eqn:planar_definition_of_extortion}
    \tilde p_1 = \frac{(R - P)(\tilde p_2 + \tilde p_3)}{S + T - 2P}
\end{equation}

with:

\begin{equation}\label{eqn:definition_of_chi}
    \chi = \frac{\tilde p_2 (P - T) + \tilde p_3 (S - P)}
                {\tilde p_2 (P - S) + \tilde p_3 (T - P)}
\end{equation}

Given a strategy \(p\in\mathbb{R}^{4\times 1}\) equations
(\ref{eqn:condition_for_tilde_p4}), (\ref{eqn:planar_definition_of_extortion}-\ref{eqn:definition_of_chi}) can be used to check if
a strategy is extortionate. The conditions correspond to:

\begin{align}
    p_1 & = \frac{(R-P)(p_2 + p_3) - R + T + S - P}{S + T - 2P}
     \label{eqn:condition_for_p1}\\
    p_4 & = 0 \label{eqn:condition_for_p4}\\
    1 & > p_2 + p_3\label{eqn:condition_for_chi}
\end{align}

The algebraic steps necessary to prove these results are available in the
supporting materials.

All extortionate strategies reside on a triangular (\ref{eqn:condition_for_chi})
plane (\ref{eqn:condition_for_p1}) in 3 dimensions (\ref{eqn:condition_for_p4}).
Using this formulation it can be seen that a necessary (but not sufficient)
condition for an extortionate strategy is that it cooperates on average less
than 50\% of the time when in a state of disagreement with the opponent.

As an example, consider the known extortionate strategy \(p=(8 / 9, 1 / 2, 1 /
3, 0)\) from~\cite{Stewart2012} which is referred to as \texttt{Extort-2}. In
this case, for the standard values of \((R, T, S, P)\) constraint
(\ref{eqn:condition_for_p1}) corresponds to:

\begin{equation}
    p_1 = \frac{2(p_2 + p_3) + 1}{3}
\end{equation}

It is clear that in this case all constraints hold.

This approach could in fact be used to confirm that a given strategy is acting
in an extortionate manner even if it is not a memory one strategy. However, in
practice, if a closed form for \(p\) is not known, then due to measurement
and/or numerical error this would not work.

This problem can be written in the following linear algebraic form where
\(x=(\alpha, \beta)\)
and \(p^*=(\tilde p_1 - 1, tilde_2 - 1, p_3)\):

\begin{equation}\label{eqn:linear_algebraic_equation_for_p}
    Cx= p^*
\end{equation}

\(C\) corresponds to equations
(\ref{eqn:condition_for_tilde_p1}-\ref{eqn:condition_for_tilde_p3}) and is
given by:

\begin{equation}\label{eqn:definition_of_C}
    C =
    \begin{bmatrix}
        R - P & R- P \\
        S - P & T- P \\
        T - P & S- P \\
    \end{bmatrix}
\end{equation}

Note that in general, equation (\ref{eqn:linear_algebraic_equation_for_p}) will
not necessarily have a solution. From the Rouch\'{e}-Capelli theorem if there is
a solution it is unique as \(\text{rank}(C)=2\) which is the dimension of the
variable \(x\). The best fitting \(x\) is found by minimizing:

\begin{equation}\label{eqn:r_squared}
    \text{SSError} = \|C x- p^*\|_2^2 = \sum_{i=1}^{3}\left((C\bar x)_i-p_i^*\right)^2
\end{equation}

Note that \(\text{SSError}\), which is the square of the Frobenius
norm~\cite{Golub2013}, becomes a measure of how close a strategy is to being an
extortionate strategy. Suspicion
of extortion then corresponds to a threshold on \(\text{SSError}\).

By observing interactions (human or otherwise), their memory one representation
can be inferred and this approach can be used to recognise extortionate
behaviour. The notion of comparing theoretic and actual plays of the IPD is not
novel, see for example~\cite{Rand2013}. Immediately it is noted that if the
environment is noisy~\cite{Wu1995} then no strategy can be considered to be
extortionate as \(p_4>0\).

In the next section, this idea will be illustrated by observing the interactions
that take place in a computer based tournament of the IPD\@.

\section{Numerical experiments}\label{sec:numerical-experiments}

In~\cite{Stewart2012} results from a tournament with
\input{./assets/tex/number_of_stewart_plotkin_strategies/main.tex} strategies,
was presented with specific consideration given to ZD strategies. This
tournament is reproduced here using the Axelrod-Python
project~\cite{Knight2016}. To obtain a good measure of the corresponding
transition rates for each strategy all matches have been run for
\input{assets/tex/number_of_turns/main.tex} turns and every match has been
repeated \input{assets/tex/number_of_repetitions/main.tex} times. All of this
interaction data is available at~\cite{vincent_knight_2018_1297075}. A good
match between the inferred Markov chain and the state distribution of the actual
interactions has been verified. Data for this is presented in the supplementary
materials.

Figure~\ref{fig:SSError_overall_in_stewart_plotkin} shows the \(\text{SSError}\)
values for all the strategies in the tournament, as reported
in~\cite{Stewart2012} the extortionate strategy (which has an expected
\(\text{SSError}\) approximately 0) gains a large number of wins.

\begin{figure}[!htbp]
    \centering
    \includegraphics[width=.8\textwidth]{./assets/img/SSError_overall_in_stewart_plotkin/main.pdf}
    \caption{\(\text{SSError}\) and state probabilities for the strategies
        of~\cite{Stewart2012}, ordered both by number of wins and overall score.
        Note that \(P(DC)\) is not shown as it corresponds to the transpose of
        \(P(CD)\). Cooperator and Defector are omitted as they do not visit all
        the states.}
    \label{fig:SSError_overall_in_stewart_plotkin}
\end{figure}

Here, the work of~\cite{Stewart2012} is extended by investigating a tournament
with \input{assets/tex/number_of_full_strategies/main.tex}
strategies.

The results of this analysis are shown in
Figure~\ref{fig:SSError_and_probabilities_in_full}. The top ranking strategies
by number of wins seem to be extortionate (but not against all strategies) and
it can be seen that a small sub group of strategies achieve mutual defection.
All the top ranking strategies according to score achieve mutual cooperation and
do not extort each other, however they
\textbf{do} exhibit extortionate behaviour towards a number of the lower ranking
strategies.

\begin{figure}[!htbp]
    \centering
    \includegraphics[width=.8\textwidth]{./assets/img/SSError_and_probabilities_in_full/main.pdf}
    \caption{\(\text{SSError}\) for the strategies for the full tournament. Only
    strategy interactions for which \(p_4=0\) and \(\chi>1\) are displayed.}
    \label{fig:SSError_and_probabilities_in_full}
\end{figure}

\section{Conclusion}\label{sec:conclusion}

This work defines an approach to measure whether or not a player is playing a
strategy that corresponds to an extortionate strategy as defined
in~\cite{Press2012}: a mathematical model for suspicion. Indeed, all
extortionate strategies have been
 classified as lying on a triangular plane.
This rigorous classification fails to be robust to small measurement error, thus
a statistical approach is proposed.
This is done through a linear algebraic approach for approximating the solution
of a linear system. Using this, a large number of pairwise interactions is
simulated and in fact very few strategies are found to act extortionately.

The work of~\cite{Press2012}, whilst showing that a clever approach to taking
advantage of another memory one strategy exists: this is incomplete. Whilst the
elegance of this result is very attractive, just as the simplicity of the
victory of Tit For Tat in Axelrod's original tournaments was, it is incomplete.
Extortionate strategies achieve a high number of wins but they do not
achieve a high score which corresponds to the fitness landscape in an
evolutionary sense. From the large number of interactions a payoff matrix \(S\)
can be measured where \(S_{ij}\) denotes the score (using standard values of
\((R, S, T, P) = (3, 0, 5, 1)\)) of the \(i\)th strategy
against the \(j\)th strategy. Using this, the replicator equation
describes the evolution of the system based on a population density fitness
function:

\begin{equation}\label{eqn:replicator_dynamics}
    \frac{dx}{dt} = x(S-x^TS x)
\end{equation}

Equation (\ref{eqn:replicator_dynamics}) is solved numerically through an
integration technique described in~\cite{Petzold1983} and
Figure~\ref{fig:replicator_dynamics} shows the evolution of the distribution of
the system: the various strategies are ranked by scores. It is clear to see that
only the high ranking strategies survive the evolutionary process (in fact,
only \input{./assets/img/replicator_dynamics/main.tex}
have a final distribution greater than \(10 ^ {-2}\)). This confirms the
findings of~\cite{Moran1707} in which sophisticated strategies resist
evolutionary invasion of shorter memory strategies. Recalling
Figure~\ref{fig:SSError_and_probabilities_in_full} this demonstrates that:

\begin{itemize}
    \item Cooperation emerges through the evolutionary process: the high scoring
        strategies do not exhibit extortionate behaviour towards each other.
    \item Extortionate strategies do not survive the evolutionary process.
\end{itemize}

\begin{figure}[!htbp]
    \centering
    \includegraphics[width=.8\textwidth]{./assets/img/replicator_dynamics/main.pdf}
    \caption{Numerical simulation of the replicator equation
    (\ref{eqn:replicator_dynamics}): strategies are ordered by score, only the strategies with a high score survive the evolutionary process.}
    \label{fig:replicator_dynamics}
\end{figure}

This work can be used to classify plays of the IPD\@: data can be collected from
actual interactions (in lab or in the field). Furthermore, this allows for a
classification method similar to the notion of fingerprinting presented
in~\cite{Ashlock2008}. Trained strategies can potentially be classified as
extortionate or not or it could be possible to even constrain the reinforcement
learning approaches that are becoming prevalent in the literature.
Alternatively, this mathematical approach for recognising extortion could be
used in sophisticated strategies to defend against invasion. Arguably, some of
the strategies considered here exhibit this behaviour, indeed as described
in~\cite{Harper2017}, the top ranking strategies in the full tournament are
obtained using evolutionary reinforcement learning techniques, thus, suspicion
of extortionate behaviour could in fact be an evolutionary trait.

\section*{Acknowledgements}

The following open source software libraries were used in this research:

\begin{itemize}
    \item The Axelrod ~\cite{Knight2016, Knight2018} library (IPD strategies and
        tournaments).
    \item The sympy library~\cite{Meurer2017} (verification of all symbolic
        calculations).
    \item The matplotlib~\cite{Droettboom2018} library (visualisation).
    \item The pandas~\cite{Structures2010}, dask~\cite{Dask2016} and
        NumPy~\cite{Oliphant2015} libraries (data manipulation).
    \item The SciPy~\cite{Jones2001} library (numerical integration of the
        replicator equation).
\end{itemize}

This work was performed using the computational facilities of the Advanced
Research Computing @ Cardiff (ARCCA) Division, Cardiff University.

\printbibliography

\newpage
\section*{Supplementary materials}

\includepdf{assets/pdf/proof_of_form_of_extortionate_strategies/main.pdf}

\newpage

Using the pair wise interactions the transition rates \(p,
q\) can be measured and the steady state probabilities inferred and compared to
the actual probabilities of each state.
This is done numerically by computing the singular eigenvector of the
matrix \(A\) \cite{Stewart2009}:

\[
    A =
    \begin{bmatrix}
        p_1 q_1 & p_1 (1 - q_1) & (1 - p_1) q_1 & (1 -p_1) (1 - q_1) \\
        p_2 q_2 & p_2 (1 - q_2) & (1 - p_2) q_2 & (1 -p_2) (1 - q_2) \\
        p_3 q_3 & p_3 (1 - q_3) & (1 - p_3) q_3 & (1 -p_3) (1 - q_3) \\
        p_4 q_4 & p_4 (1 - q_4) & (1 - p_4) q_4 & (1 -p_4) (1 - q_4) \\
    \end{bmatrix}
\]

Figure~\ref{fig:computed_probabilities_vs_theoretic_probabilities} shows a
regression line fitted to every pairwise interaction with a reported
\(\text{SSError}\) value (pairwise interactions with missing states were
omitted). This serves to validate the approach: a part from some edge cases the
relationship is consistent.

\begin{figure}[!htbp]
    \centering
    \includegraphics[width=.8\textwidth]{./assets/img/computed_probabilities_vs_theoretic_probabilities/main.pdf}
    \caption{The
        relationship between the steady state probabilities inferred from the
        measured transitions and the actual steady state probabilities. A linear
        regression line is included validating the approach.}
    \label{fig:computed_probabilities_vs_theoretic_probabilities}
\end{figure}


\end{document}
 strategies,
was presented with specific consideration given to ZD strategies. This
tournament is reproduced here using the Axelrod-Python
project~\cite{Knight2016}. To obtain a good measure of the corresponding
transition rates for each strategy all matches have been run for
\documentclass[a4paper]{article}

\usepackage{amsmath}
\usepackage{amssymb}
\usepackage[margin=1.5cm,
            includefoot,
            footskip=30pt]{geometry}
\usepackage{layout}
\usepackage{graphicx}
\usepackage{subcaption}

\usepackage{biblatex}
\usepackage{pdfpages}

\bibliography{main.bib}

\title{Suspicion: Recognising and evaluating the effectiveness
       of extortion in the Iterated Prisoner's Dilemma}
\author{Vincent A. Knight \and Nikoleta E. Glynatsi}
\date{\today}



\begin{document}

\maketitle

\begin{abstract}
    The Iterated Prisoner's Dilemma is a model for rational and evolutionary
    interactive behaviour. It has applications both in the study of human social
    behaviour as well as in biology.
    It is used to understand when and how a rational individual might
    accept an immediate cost to their own utility for the direct benefit of
    another.

    Much attention has been given to a class of strategies called
    Zero Determinant strategies. It has been theoretically shown that these
    strategies can ``extort'' any player.

    In this work, an approach to identify if observed strategies are playing in
    an extortionate way is described. Furthermore, experimental analysis of
    a large tournament with \input{assets/tex/number_of_full_strategies/main.tex}
    strategies is considered. In this setting
    the most highly performing strategies do not play in an extortionate way
    against each other but do against lower performing strategies.
    This suggests that whilst the theory of Zero Determinant strategies
    indicates that memory is not of fundamental importance to the evolution of
    cooperative behaviour, this is incomplete.
\end{abstract}

\section{Introduction}\label{sec:introduction}

Agent based game theoretic models have become a stalwart of the underpinning
mathematics of interactive behaviours. One of the major pieces of work
in this area is the pair of original computer tournaments run by Robert
Axelrod~\cite{Axelrod1980, Axelrod1980a}. These tournaments pitted submitted
computer strategies against each other in plays of the Iterated Prisoner's
Dilemma. A common game where agents can choose to pay a slight cost to their
immediate utility in the hope of building a reputation. This has been used in
economic and evolutionary game theory to understand the evolution of cooperative
behaviour.

Recently, a class of strategies was described in~\cite{Press2012} that can
provably extort any given opponent. In~\cite{Hilbe2013, Moran1707} some
questions have already been asked about the true effectiveness of these
strategies in an evolutionary setting. Here another question is asked: is it
possible to recognise this extortionate behaviour? A mathematical procedure for
suspicion is presented: in the same way that the continued actions of an
extortionate individual might raise suspicion.

This work makes use of the Axelrod Python library~\cite{Knight2018, Knight2016}
with a large number of Prisoner Dilemma strategies available to give an
extensive numerical example of the ideas presented.  The approach is presented
in Section~\ref{sec:delta-zd-strategies}.  All of the code and data discussed
in Section~\ref{sec:numerical-experiments} is open sourced, archived and
written according to best scientific principles~\cite{Wilson2014}. The data
archive can be found at~\cite{vincent_knight_2018_1297075}.

\section{Recognising Extortion}\label{sec:delta-zd-strategies}

In~\cite{Press2012}, given a match between 2 memory-one strategies, the concept
of Zero Determinant (ZD) strategies is introduced. The main result of that paper
shows that given two memory one players \(p, q\in\mathbb{R}^4\) a linear
relationship between the players' scores could be forced by one of the players.

Using the notation of~\cite{Press2012}, assuming the utilities for player \(p\)
are given by \(S_x=(R, S, T, P)\) and for player \(q\) by \(S_y=(R, T, S, P)\)
and that the stationary scores of each player is given by \(S_X\) and \(S_Y\)
respectively. The main result of~\cite{Press2012} is that if

\begin{equation}\label{eqn:linear_relationship_for_p}
    \tilde p=\alpha S_x + \beta S_y + \gamma
\end{equation}

or

\begin{equation}\label{eqn:linear_relationship_for_q}
    \tilde q=\alpha S_x + \beta S_y + \gamma
\end{equation}

where \(\tilde p = (1 - p_1, 1 - p_2, p_3, p_4)\) and
\(\tilde q = (1 - q_1, 1 - q_2, q_3, q_4)\) then:

\begin{equation}
    \alpha S_X + \beta S_Y + \gamma = 0
\end{equation}

In~\cite{Press2012} a particular type of ZD strategy is defined: extortionate
strategies. If:

\begin{equation}\label{eqn:constraint_for_extortion}
    \gamma = - P(\alpha + \beta)
\end{equation}

then the player can ensure they get a score \(\chi\) times
larger than the opponent. This extortion coefficient is given by:

\begin{equation}\label{eqn:definition_of_chi}
    \chi=\frac{-\beta}{\alpha}
\end{equation}

Thus, if (\ref{eqn:constraint_for_extortion}) holds and \(\chi >1\) a player is
said to extort their opponent.
Here, the reverse problem is considered: given a
\(p\in\mathbb{R}^4\) how does one identify \(\alpha, \beta\) if they
exist and is the strategy in fact acting in an extortionate way?

These conditions correspond to:

\begin{align}
    \tilde p_1 & = \alpha R + \beta R - P (\alpha + \beta)
            \label{eqn:condition_for_tilde_p1}\\
    \tilde p_2 & = \alpha S + \beta T - P (\alpha + \beta)
            \label{eqn:condition_for_tilde_p2}\\
    \tilde p_3 & = \alpha T + \beta S - P (\alpha + \beta)
            \label{eqn:condition_for_tilde_p3}\\
    \tilde p_4 & = \alpha P + \beta P - P (\alpha + \beta)
            \label{eqn:condition_for_tilde_p4}
\end{align}

Equation (\ref{eqn:condition_for_tilde_p4}) ensures that \(p_4=\tilde p_4=0\).
Equations (\ref{eqn:condition_for_tilde_p1}-\ref{eqn:condition_for_tilde_p3})
can be used to eliminate \(\alpha, \beta\), giving:

\begin{equation}\label{eqn:planar_definition_of_extortion}
    \tilde p_1 = \frac{(R - P)(\tilde p_2 + \tilde p_3)}{S + T - 2P}
\end{equation}

with:

\begin{equation}\label{eqn:definition_of_chi}
    \chi = \frac{\tilde p_2 (P - T) + \tilde p_3 (S - P)}
                {\tilde p_2 (P - S) + \tilde p_3 (T - P)}
\end{equation}

Given a strategy \(p\in\mathbb{R}^{4\times 1}\) equations
(\ref{eqn:condition_for_tilde_p4}), (\ref{eqn:planar_definition_of_extortion}-\ref{eqn:definition_of_chi}) can be used to check if
a strategy is extortionate. The conditions correspond to:

\begin{align}
    p_1 & = \frac{(R-P)(p_2 + p_3) - R + T + S - P}{S + T - 2P}
     \label{eqn:condition_for_p1}\\
    p_4 & = 0 \label{eqn:condition_for_p4}\\
    1 & > p_2 + p_3\label{eqn:condition_for_chi}
\end{align}

The algebraic steps necessary to prove these results are available in the
supporting materials.

All extortionate strategies reside on a triangular (\ref{eqn:condition_for_chi})
plane (\ref{eqn:condition_for_p1}) in 3 dimensions (\ref{eqn:condition_for_p4}).
Using this formulation it can be seen that a necessary (but not sufficient)
condition for an extortionate strategy is that it cooperates on average less
than 50\% of the time when in a state of disagreement with the opponent.

As an example, consider the known extortionate strategy \(p=(8 / 9, 1 / 2, 1 /
3, 0)\) from~\cite{Stewart2012} which is referred to as \texttt{Extort-2}. In
this case, for the standard values of \((R, T, S, P)\) constraint
(\ref{eqn:condition_for_p1}) corresponds to:

\begin{equation}
    p_1 = \frac{2(p_2 + p_3) + 1}{3}
\end{equation}

It is clear that in this case all constraints hold.

This approach could in fact be used to confirm that a given strategy is acting
in an extortionate manner even if it is not a memory one strategy. However, in
practice, if a closed form for \(p\) is not known, then due to measurement
and/or numerical error this would not work.

This problem can be written in the following linear algebraic form where
\(x=(\alpha, \beta)\)
and \(p^*=(\tilde p_1 - 1, tilde_2 - 1, p_3)\):

\begin{equation}\label{eqn:linear_algebraic_equation_for_p}
    Cx= p^*
\end{equation}

\(C\) corresponds to equations
(\ref{eqn:condition_for_tilde_p1}-\ref{eqn:condition_for_tilde_p3}) and is
given by:

\begin{equation}\label{eqn:definition_of_C}
    C =
    \begin{bmatrix}
        R - P & R- P \\
        S - P & T- P \\
        T - P & S- P \\
    \end{bmatrix}
\end{equation}

Note that in general, equation (\ref{eqn:linear_algebraic_equation_for_p}) will
not necessarily have a solution. From the Rouch\'{e}-Capelli theorem if there is
a solution it is unique as \(\text{rank}(C)=2\) which is the dimension of the
variable \(x\). The best fitting \(x\) is found by minimizing:

\begin{equation}\label{eqn:r_squared}
    \text{SSError} = \|C x- p^*\|_2^2 = \sum_{i=1}^{3}\left((C\bar x)_i-p_i^*\right)^2
\end{equation}

Note that \(\text{SSError}\), which is the square of the Frobenius
norm~\cite{Golub2013}, becomes a measure of how close a strategy is to being an
extortionate strategy. Suspicion
of extortion then corresponds to a threshold on \(\text{SSError}\).

By observing interactions (human or otherwise), their memory one representation
can be inferred and this approach can be used to recognise extortionate
behaviour. The notion of comparing theoretic and actual plays of the IPD is not
novel, see for example~\cite{Rand2013}. Immediately it is noted that if the
environment is noisy~\cite{Wu1995} then no strategy can be considered to be
extortionate as \(p_4>0\).

In the next section, this idea will be illustrated by observing the interactions
that take place in a computer based tournament of the IPD\@.

\section{Numerical experiments}\label{sec:numerical-experiments}

In~\cite{Stewart2012} results from a tournament with
\input{./assets/tex/number_of_stewart_plotkin_strategies/main.tex} strategies,
was presented with specific consideration given to ZD strategies. This
tournament is reproduced here using the Axelrod-Python
project~\cite{Knight2016}. To obtain a good measure of the corresponding
transition rates for each strategy all matches have been run for
\input{assets/tex/number_of_turns/main.tex} turns and every match has been
repeated \input{assets/tex/number_of_repetitions/main.tex} times. All of this
interaction data is available at~\cite{vincent_knight_2018_1297075}. A good
match between the inferred Markov chain and the state distribution of the actual
interactions has been verified. Data for this is presented in the supplementary
materials.

Figure~\ref{fig:SSError_overall_in_stewart_plotkin} shows the \(\text{SSError}\)
values for all the strategies in the tournament, as reported
in~\cite{Stewart2012} the extortionate strategy (which has an expected
\(\text{SSError}\) approximately 0) gains a large number of wins.

\begin{figure}[!htbp]
    \centering
    \includegraphics[width=.8\textwidth]{./assets/img/SSError_overall_in_stewart_plotkin/main.pdf}
    \caption{\(\text{SSError}\) and state probabilities for the strategies
        of~\cite{Stewart2012}, ordered both by number of wins and overall score.
        Note that \(P(DC)\) is not shown as it corresponds to the transpose of
        \(P(CD)\). Cooperator and Defector are omitted as they do not visit all
        the states.}
    \label{fig:SSError_overall_in_stewart_plotkin}
\end{figure}

Here, the work of~\cite{Stewart2012} is extended by investigating a tournament
with \input{assets/tex/number_of_full_strategies/main.tex}
strategies.

The results of this analysis are shown in
Figure~\ref{fig:SSError_and_probabilities_in_full}. The top ranking strategies
by number of wins seem to be extortionate (but not against all strategies) and
it can be seen that a small sub group of strategies achieve mutual defection.
All the top ranking strategies according to score achieve mutual cooperation and
do not extort each other, however they
\textbf{do} exhibit extortionate behaviour towards a number of the lower ranking
strategies.

\begin{figure}[!htbp]
    \centering
    \includegraphics[width=.8\textwidth]{./assets/img/SSError_and_probabilities_in_full/main.pdf}
    \caption{\(\text{SSError}\) for the strategies for the full tournament. Only
    strategy interactions for which \(p_4=0\) and \(\chi>1\) are displayed.}
    \label{fig:SSError_and_probabilities_in_full}
\end{figure}

\section{Conclusion}\label{sec:conclusion}

This work defines an approach to measure whether or not a player is playing a
strategy that corresponds to an extortionate strategy as defined
in~\cite{Press2012}: a mathematical model for suspicion. Indeed, all
extortionate strategies have been
 classified as lying on a triangular plane.
This rigorous classification fails to be robust to small measurement error, thus
a statistical approach is proposed.
This is done through a linear algebraic approach for approximating the solution
of a linear system. Using this, a large number of pairwise interactions is
simulated and in fact very few strategies are found to act extortionately.

The work of~\cite{Press2012}, whilst showing that a clever approach to taking
advantage of another memory one strategy exists: this is incomplete. Whilst the
elegance of this result is very attractive, just as the simplicity of the
victory of Tit For Tat in Axelrod's original tournaments was, it is incomplete.
Extortionate strategies achieve a high number of wins but they do not
achieve a high score which corresponds to the fitness landscape in an
evolutionary sense. From the large number of interactions a payoff matrix \(S\)
can be measured where \(S_{ij}\) denotes the score (using standard values of
\((R, S, T, P) = (3, 0, 5, 1)\)) of the \(i\)th strategy
against the \(j\)th strategy. Using this, the replicator equation
describes the evolution of the system based on a population density fitness
function:

\begin{equation}\label{eqn:replicator_dynamics}
    \frac{dx}{dt} = x(S-x^TS x)
\end{equation}

Equation (\ref{eqn:replicator_dynamics}) is solved numerically through an
integration technique described in~\cite{Petzold1983} and
Figure~\ref{fig:replicator_dynamics} shows the evolution of the distribution of
the system: the various strategies are ranked by scores. It is clear to see that
only the high ranking strategies survive the evolutionary process (in fact,
only \input{./assets/img/replicator_dynamics/main.tex}
have a final distribution greater than \(10 ^ {-2}\)). This confirms the
findings of~\cite{Moran1707} in which sophisticated strategies resist
evolutionary invasion of shorter memory strategies. Recalling
Figure~\ref{fig:SSError_and_probabilities_in_full} this demonstrates that:

\begin{itemize}
    \item Cooperation emerges through the evolutionary process: the high scoring
        strategies do not exhibit extortionate behaviour towards each other.
    \item Extortionate strategies do not survive the evolutionary process.
\end{itemize}

\begin{figure}[!htbp]
    \centering
    \includegraphics[width=.8\textwidth]{./assets/img/replicator_dynamics/main.pdf}
    \caption{Numerical simulation of the replicator equation
    (\ref{eqn:replicator_dynamics}): strategies are ordered by score, only the strategies with a high score survive the evolutionary process.}
    \label{fig:replicator_dynamics}
\end{figure}

This work can be used to classify plays of the IPD\@: data can be collected from
actual interactions (in lab or in the field). Furthermore, this allows for a
classification method similar to the notion of fingerprinting presented
in~\cite{Ashlock2008}. Trained strategies can potentially be classified as
extortionate or not or it could be possible to even constrain the reinforcement
learning approaches that are becoming prevalent in the literature.
Alternatively, this mathematical approach for recognising extortion could be
used in sophisticated strategies to defend against invasion. Arguably, some of
the strategies considered here exhibit this behaviour, indeed as described
in~\cite{Harper2017}, the top ranking strategies in the full tournament are
obtained using evolutionary reinforcement learning techniques, thus, suspicion
of extortionate behaviour could in fact be an evolutionary trait.

\section*{Acknowledgements}

The following open source software libraries were used in this research:

\begin{itemize}
    \item The Axelrod ~\cite{Knight2016, Knight2018} library (IPD strategies and
        tournaments).
    \item The sympy library~\cite{Meurer2017} (verification of all symbolic
        calculations).
    \item The matplotlib~\cite{Droettboom2018} library (visualisation).
    \item The pandas~\cite{Structures2010}, dask~\cite{Dask2016} and
        NumPy~\cite{Oliphant2015} libraries (data manipulation).
    \item The SciPy~\cite{Jones2001} library (numerical integration of the
        replicator equation).
\end{itemize}

This work was performed using the computational facilities of the Advanced
Research Computing @ Cardiff (ARCCA) Division, Cardiff University.

\printbibliography

\newpage
\section*{Supplementary materials}

\includepdf{assets/pdf/proof_of_form_of_extortionate_strategies/main.pdf}

\newpage

Using the pair wise interactions the transition rates \(p,
q\) can be measured and the steady state probabilities inferred and compared to
the actual probabilities of each state.
This is done numerically by computing the singular eigenvector of the
matrix \(A\) \cite{Stewart2009}:

\[
    A =
    \begin{bmatrix}
        p_1 q_1 & p_1 (1 - q_1) & (1 - p_1) q_1 & (1 -p_1) (1 - q_1) \\
        p_2 q_2 & p_2 (1 - q_2) & (1 - p_2) q_2 & (1 -p_2) (1 - q_2) \\
        p_3 q_3 & p_3 (1 - q_3) & (1 - p_3) q_3 & (1 -p_3) (1 - q_3) \\
        p_4 q_4 & p_4 (1 - q_4) & (1 - p_4) q_4 & (1 -p_4) (1 - q_4) \\
    \end{bmatrix}
\]

Figure~\ref{fig:computed_probabilities_vs_theoretic_probabilities} shows a
regression line fitted to every pairwise interaction with a reported
\(\text{SSError}\) value (pairwise interactions with missing states were
omitted). This serves to validate the approach: a part from some edge cases the
relationship is consistent.

\begin{figure}[!htbp]
    \centering
    \includegraphics[width=.8\textwidth]{./assets/img/computed_probabilities_vs_theoretic_probabilities/main.pdf}
    \caption{The
        relationship between the steady state probabilities inferred from the
        measured transitions and the actual steady state probabilities. A linear
        regression line is included validating the approach.}
    \label{fig:computed_probabilities_vs_theoretic_probabilities}
\end{figure}


\end{document}
 turns and every match has been
repeated \documentclass[a4paper]{article}

\usepackage{amsmath}
\usepackage{amssymb}
\usepackage[margin=1.5cm,
            includefoot,
            footskip=30pt]{geometry}
\usepackage{layout}
\usepackage{graphicx}
\usepackage{subcaption}

\usepackage{biblatex}
\usepackage{pdfpages}

\bibliography{main.bib}

\title{Suspicion: Recognising and evaluating the effectiveness
       of extortion in the Iterated Prisoner's Dilemma}
\author{Vincent A. Knight \and Nikoleta E. Glynatsi}
\date{\today}



\begin{document}

\maketitle

\begin{abstract}
    The Iterated Prisoner's Dilemma is a model for rational and evolutionary
    interactive behaviour. It has applications both in the study of human social
    behaviour as well as in biology.
    It is used to understand when and how a rational individual might
    accept an immediate cost to their own utility for the direct benefit of
    another.

    Much attention has been given to a class of strategies called
    Zero Determinant strategies. It has been theoretically shown that these
    strategies can ``extort'' any player.

    In this work, an approach to identify if observed strategies are playing in
    an extortionate way is described. Furthermore, experimental analysis of
    a large tournament with \input{assets/tex/number_of_full_strategies/main.tex}
    strategies is considered. In this setting
    the most highly performing strategies do not play in an extortionate way
    against each other but do against lower performing strategies.
    This suggests that whilst the theory of Zero Determinant strategies
    indicates that memory is not of fundamental importance to the evolution of
    cooperative behaviour, this is incomplete.
\end{abstract}

\section{Introduction}\label{sec:introduction}

Agent based game theoretic models have become a stalwart of the underpinning
mathematics of interactive behaviours. One of the major pieces of work
in this area is the pair of original computer tournaments run by Robert
Axelrod~\cite{Axelrod1980, Axelrod1980a}. These tournaments pitted submitted
computer strategies against each other in plays of the Iterated Prisoner's
Dilemma. A common game where agents can choose to pay a slight cost to their
immediate utility in the hope of building a reputation. This has been used in
economic and evolutionary game theory to understand the evolution of cooperative
behaviour.

Recently, a class of strategies was described in~\cite{Press2012} that can
provably extort any given opponent. In~\cite{Hilbe2013, Moran1707} some
questions have already been asked about the true effectiveness of these
strategies in an evolutionary setting. Here another question is asked: is it
possible to recognise this extortionate behaviour? A mathematical procedure for
suspicion is presented: in the same way that the continued actions of an
extortionate individual might raise suspicion.

This work makes use of the Axelrod Python library~\cite{Knight2018, Knight2016}
with a large number of Prisoner Dilemma strategies available to give an
extensive numerical example of the ideas presented.  The approach is presented
in Section~\ref{sec:delta-zd-strategies}.  All of the code and data discussed
in Section~\ref{sec:numerical-experiments} is open sourced, archived and
written according to best scientific principles~\cite{Wilson2014}. The data
archive can be found at~\cite{vincent_knight_2018_1297075}.

\section{Recognising Extortion}\label{sec:delta-zd-strategies}

In~\cite{Press2012}, given a match between 2 memory-one strategies, the concept
of Zero Determinant (ZD) strategies is introduced. The main result of that paper
shows that given two memory one players \(p, q\in\mathbb{R}^4\) a linear
relationship between the players' scores could be forced by one of the players.

Using the notation of~\cite{Press2012}, assuming the utilities for player \(p\)
are given by \(S_x=(R, S, T, P)\) and for player \(q\) by \(S_y=(R, T, S, P)\)
and that the stationary scores of each player is given by \(S_X\) and \(S_Y\)
respectively. The main result of~\cite{Press2012} is that if

\begin{equation}\label{eqn:linear_relationship_for_p}
    \tilde p=\alpha S_x + \beta S_y + \gamma
\end{equation}

or

\begin{equation}\label{eqn:linear_relationship_for_q}
    \tilde q=\alpha S_x + \beta S_y + \gamma
\end{equation}

where \(\tilde p = (1 - p_1, 1 - p_2, p_3, p_4)\) and
\(\tilde q = (1 - q_1, 1 - q_2, q_3, q_4)\) then:

\begin{equation}
    \alpha S_X + \beta S_Y + \gamma = 0
\end{equation}

In~\cite{Press2012} a particular type of ZD strategy is defined: extortionate
strategies. If:

\begin{equation}\label{eqn:constraint_for_extortion}
    \gamma = - P(\alpha + \beta)
\end{equation}

then the player can ensure they get a score \(\chi\) times
larger than the opponent. This extortion coefficient is given by:

\begin{equation}\label{eqn:definition_of_chi}
    \chi=\frac{-\beta}{\alpha}
\end{equation}

Thus, if (\ref{eqn:constraint_for_extortion}) holds and \(\chi >1\) a player is
said to extort their opponent.
Here, the reverse problem is considered: given a
\(p\in\mathbb{R}^4\) how does one identify \(\alpha, \beta\) if they
exist and is the strategy in fact acting in an extortionate way?

These conditions correspond to:

\begin{align}
    \tilde p_1 & = \alpha R + \beta R - P (\alpha + \beta)
            \label{eqn:condition_for_tilde_p1}\\
    \tilde p_2 & = \alpha S + \beta T - P (\alpha + \beta)
            \label{eqn:condition_for_tilde_p2}\\
    \tilde p_3 & = \alpha T + \beta S - P (\alpha + \beta)
            \label{eqn:condition_for_tilde_p3}\\
    \tilde p_4 & = \alpha P + \beta P - P (\alpha + \beta)
            \label{eqn:condition_for_tilde_p4}
\end{align}

Equation (\ref{eqn:condition_for_tilde_p4}) ensures that \(p_4=\tilde p_4=0\).
Equations (\ref{eqn:condition_for_tilde_p1}-\ref{eqn:condition_for_tilde_p3})
can be used to eliminate \(\alpha, \beta\), giving:

\begin{equation}\label{eqn:planar_definition_of_extortion}
    \tilde p_1 = \frac{(R - P)(\tilde p_2 + \tilde p_3)}{S + T - 2P}
\end{equation}

with:

\begin{equation}\label{eqn:definition_of_chi}
    \chi = \frac{\tilde p_2 (P - T) + \tilde p_3 (S - P)}
                {\tilde p_2 (P - S) + \tilde p_3 (T - P)}
\end{equation}

Given a strategy \(p\in\mathbb{R}^{4\times 1}\) equations
(\ref{eqn:condition_for_tilde_p4}), (\ref{eqn:planar_definition_of_extortion}-\ref{eqn:definition_of_chi}) can be used to check if
a strategy is extortionate. The conditions correspond to:

\begin{align}
    p_1 & = \frac{(R-P)(p_2 + p_3) - R + T + S - P}{S + T - 2P}
     \label{eqn:condition_for_p1}\\
    p_4 & = 0 \label{eqn:condition_for_p4}\\
    1 & > p_2 + p_3\label{eqn:condition_for_chi}
\end{align}

The algebraic steps necessary to prove these results are available in the
supporting materials.

All extortionate strategies reside on a triangular (\ref{eqn:condition_for_chi})
plane (\ref{eqn:condition_for_p1}) in 3 dimensions (\ref{eqn:condition_for_p4}).
Using this formulation it can be seen that a necessary (but not sufficient)
condition for an extortionate strategy is that it cooperates on average less
than 50\% of the time when in a state of disagreement with the opponent.

As an example, consider the known extortionate strategy \(p=(8 / 9, 1 / 2, 1 /
3, 0)\) from~\cite{Stewart2012} which is referred to as \texttt{Extort-2}. In
this case, for the standard values of \((R, T, S, P)\) constraint
(\ref{eqn:condition_for_p1}) corresponds to:

\begin{equation}
    p_1 = \frac{2(p_2 + p_3) + 1}{3}
\end{equation}

It is clear that in this case all constraints hold.

This approach could in fact be used to confirm that a given strategy is acting
in an extortionate manner even if it is not a memory one strategy. However, in
practice, if a closed form for \(p\) is not known, then due to measurement
and/or numerical error this would not work.

This problem can be written in the following linear algebraic form where
\(x=(\alpha, \beta)\)
and \(p^*=(\tilde p_1 - 1, tilde_2 - 1, p_3)\):

\begin{equation}\label{eqn:linear_algebraic_equation_for_p}
    Cx= p^*
\end{equation}

\(C\) corresponds to equations
(\ref{eqn:condition_for_tilde_p1}-\ref{eqn:condition_for_tilde_p3}) and is
given by:

\begin{equation}\label{eqn:definition_of_C}
    C =
    \begin{bmatrix}
        R - P & R- P \\
        S - P & T- P \\
        T - P & S- P \\
    \end{bmatrix}
\end{equation}

Note that in general, equation (\ref{eqn:linear_algebraic_equation_for_p}) will
not necessarily have a solution. From the Rouch\'{e}-Capelli theorem if there is
a solution it is unique as \(\text{rank}(C)=2\) which is the dimension of the
variable \(x\). The best fitting \(x\) is found by minimizing:

\begin{equation}\label{eqn:r_squared}
    \text{SSError} = \|C x- p^*\|_2^2 = \sum_{i=1}^{3}\left((C\bar x)_i-p_i^*\right)^2
\end{equation}

Note that \(\text{SSError}\), which is the square of the Frobenius
norm~\cite{Golub2013}, becomes a measure of how close a strategy is to being an
extortionate strategy. Suspicion
of extortion then corresponds to a threshold on \(\text{SSError}\).

By observing interactions (human or otherwise), their memory one representation
can be inferred and this approach can be used to recognise extortionate
behaviour. The notion of comparing theoretic and actual plays of the IPD is not
novel, see for example~\cite{Rand2013}. Immediately it is noted that if the
environment is noisy~\cite{Wu1995} then no strategy can be considered to be
extortionate as \(p_4>0\).

In the next section, this idea will be illustrated by observing the interactions
that take place in a computer based tournament of the IPD\@.

\section{Numerical experiments}\label{sec:numerical-experiments}

In~\cite{Stewart2012} results from a tournament with
\input{./assets/tex/number_of_stewart_plotkin_strategies/main.tex} strategies,
was presented with specific consideration given to ZD strategies. This
tournament is reproduced here using the Axelrod-Python
project~\cite{Knight2016}. To obtain a good measure of the corresponding
transition rates for each strategy all matches have been run for
\input{assets/tex/number_of_turns/main.tex} turns and every match has been
repeated \input{assets/tex/number_of_repetitions/main.tex} times. All of this
interaction data is available at~\cite{vincent_knight_2018_1297075}. A good
match between the inferred Markov chain and the state distribution of the actual
interactions has been verified. Data for this is presented in the supplementary
materials.

Figure~\ref{fig:SSError_overall_in_stewart_plotkin} shows the \(\text{SSError}\)
values for all the strategies in the tournament, as reported
in~\cite{Stewart2012} the extortionate strategy (which has an expected
\(\text{SSError}\) approximately 0) gains a large number of wins.

\begin{figure}[!htbp]
    \centering
    \includegraphics[width=.8\textwidth]{./assets/img/SSError_overall_in_stewart_plotkin/main.pdf}
    \caption{\(\text{SSError}\) and state probabilities for the strategies
        of~\cite{Stewart2012}, ordered both by number of wins and overall score.
        Note that \(P(DC)\) is not shown as it corresponds to the transpose of
        \(P(CD)\). Cooperator and Defector are omitted as they do not visit all
        the states.}
    \label{fig:SSError_overall_in_stewart_plotkin}
\end{figure}

Here, the work of~\cite{Stewart2012} is extended by investigating a tournament
with \input{assets/tex/number_of_full_strategies/main.tex}
strategies.

The results of this analysis are shown in
Figure~\ref{fig:SSError_and_probabilities_in_full}. The top ranking strategies
by number of wins seem to be extortionate (but not against all strategies) and
it can be seen that a small sub group of strategies achieve mutual defection.
All the top ranking strategies according to score achieve mutual cooperation and
do not extort each other, however they
\textbf{do} exhibit extortionate behaviour towards a number of the lower ranking
strategies.

\begin{figure}[!htbp]
    \centering
    \includegraphics[width=.8\textwidth]{./assets/img/SSError_and_probabilities_in_full/main.pdf}
    \caption{\(\text{SSError}\) for the strategies for the full tournament. Only
    strategy interactions for which \(p_4=0\) and \(\chi>1\) are displayed.}
    \label{fig:SSError_and_probabilities_in_full}
\end{figure}

\section{Conclusion}\label{sec:conclusion}

This work defines an approach to measure whether or not a player is playing a
strategy that corresponds to an extortionate strategy as defined
in~\cite{Press2012}: a mathematical model for suspicion. Indeed, all
extortionate strategies have been
 classified as lying on a triangular plane.
This rigorous classification fails to be robust to small measurement error, thus
a statistical approach is proposed.
This is done through a linear algebraic approach for approximating the solution
of a linear system. Using this, a large number of pairwise interactions is
simulated and in fact very few strategies are found to act extortionately.

The work of~\cite{Press2012}, whilst showing that a clever approach to taking
advantage of another memory one strategy exists: this is incomplete. Whilst the
elegance of this result is very attractive, just as the simplicity of the
victory of Tit For Tat in Axelrod's original tournaments was, it is incomplete.
Extortionate strategies achieve a high number of wins but they do not
achieve a high score which corresponds to the fitness landscape in an
evolutionary sense. From the large number of interactions a payoff matrix \(S\)
can be measured where \(S_{ij}\) denotes the score (using standard values of
\((R, S, T, P) = (3, 0, 5, 1)\)) of the \(i\)th strategy
against the \(j\)th strategy. Using this, the replicator equation
describes the evolution of the system based on a population density fitness
function:

\begin{equation}\label{eqn:replicator_dynamics}
    \frac{dx}{dt} = x(S-x^TS x)
\end{equation}

Equation (\ref{eqn:replicator_dynamics}) is solved numerically through an
integration technique described in~\cite{Petzold1983} and
Figure~\ref{fig:replicator_dynamics} shows the evolution of the distribution of
the system: the various strategies are ranked by scores. It is clear to see that
only the high ranking strategies survive the evolutionary process (in fact,
only \input{./assets/img/replicator_dynamics/main.tex}
have a final distribution greater than \(10 ^ {-2}\)). This confirms the
findings of~\cite{Moran1707} in which sophisticated strategies resist
evolutionary invasion of shorter memory strategies. Recalling
Figure~\ref{fig:SSError_and_probabilities_in_full} this demonstrates that:

\begin{itemize}
    \item Cooperation emerges through the evolutionary process: the high scoring
        strategies do not exhibit extortionate behaviour towards each other.
    \item Extortionate strategies do not survive the evolutionary process.
\end{itemize}

\begin{figure}[!htbp]
    \centering
    \includegraphics[width=.8\textwidth]{./assets/img/replicator_dynamics/main.pdf}
    \caption{Numerical simulation of the replicator equation
    (\ref{eqn:replicator_dynamics}): strategies are ordered by score, only the strategies with a high score survive the evolutionary process.}
    \label{fig:replicator_dynamics}
\end{figure}

This work can be used to classify plays of the IPD\@: data can be collected from
actual interactions (in lab or in the field). Furthermore, this allows for a
classification method similar to the notion of fingerprinting presented
in~\cite{Ashlock2008}. Trained strategies can potentially be classified as
extortionate or not or it could be possible to even constrain the reinforcement
learning approaches that are becoming prevalent in the literature.
Alternatively, this mathematical approach for recognising extortion could be
used in sophisticated strategies to defend against invasion. Arguably, some of
the strategies considered here exhibit this behaviour, indeed as described
in~\cite{Harper2017}, the top ranking strategies in the full tournament are
obtained using evolutionary reinforcement learning techniques, thus, suspicion
of extortionate behaviour could in fact be an evolutionary trait.

\section*{Acknowledgements}

The following open source software libraries were used in this research:

\begin{itemize}
    \item The Axelrod ~\cite{Knight2016, Knight2018} library (IPD strategies and
        tournaments).
    \item The sympy library~\cite{Meurer2017} (verification of all symbolic
        calculations).
    \item The matplotlib~\cite{Droettboom2018} library (visualisation).
    \item The pandas~\cite{Structures2010}, dask~\cite{Dask2016} and
        NumPy~\cite{Oliphant2015} libraries (data manipulation).
    \item The SciPy~\cite{Jones2001} library (numerical integration of the
        replicator equation).
\end{itemize}

This work was performed using the computational facilities of the Advanced
Research Computing @ Cardiff (ARCCA) Division, Cardiff University.

\printbibliography

\newpage
\section*{Supplementary materials}

\includepdf{assets/pdf/proof_of_form_of_extortionate_strategies/main.pdf}

\newpage

Using the pair wise interactions the transition rates \(p,
q\) can be measured and the steady state probabilities inferred and compared to
the actual probabilities of each state.
This is done numerically by computing the singular eigenvector of the
matrix \(A\) \cite{Stewart2009}:

\[
    A =
    \begin{bmatrix}
        p_1 q_1 & p_1 (1 - q_1) & (1 - p_1) q_1 & (1 -p_1) (1 - q_1) \\
        p_2 q_2 & p_2 (1 - q_2) & (1 - p_2) q_2 & (1 -p_2) (1 - q_2) \\
        p_3 q_3 & p_3 (1 - q_3) & (1 - p_3) q_3 & (1 -p_3) (1 - q_3) \\
        p_4 q_4 & p_4 (1 - q_4) & (1 - p_4) q_4 & (1 -p_4) (1 - q_4) \\
    \end{bmatrix}
\]

Figure~\ref{fig:computed_probabilities_vs_theoretic_probabilities} shows a
regression line fitted to every pairwise interaction with a reported
\(\text{SSError}\) value (pairwise interactions with missing states were
omitted). This serves to validate the approach: a part from some edge cases the
relationship is consistent.

\begin{figure}[!htbp]
    \centering
    \includegraphics[width=.8\textwidth]{./assets/img/computed_probabilities_vs_theoretic_probabilities/main.pdf}
    \caption{The
        relationship between the steady state probabilities inferred from the
        measured transitions and the actual steady state probabilities. A linear
        regression line is included validating the approach.}
    \label{fig:computed_probabilities_vs_theoretic_probabilities}
\end{figure}


\end{document}
 times. All of this
interaction data is available at~\cite{vincent_knight_2018_1297075}. A good
match between the inferred Markov chain and the state distribution of the actual
interactions has been verified. Data for this is presented in the supplementary
materials.

Figure~\ref{fig:SSError_overall_in_stewart_plotkin} shows the \(\text{SSError}\)
values for all the strategies in the tournament, as reported
in~\cite{Stewart2012} the extortionate strategy (which has an expected
\(\text{SSError}\) approximately 0) gains a large number of wins.

\begin{figure}[!htbp]
    \centering
    \includegraphics[width=.8\textwidth]{./assets/img/SSError_overall_in_stewart_plotkin/main.pdf}
    \caption{\(\text{SSError}\) and state probabilities for the strategies
        of~\cite{Stewart2012}, ordered both by number of wins and overall score.
        Note that \(P(DC)\) is not shown as it corresponds to the transpose of
        \(P(CD)\). Cooperator and Defector are omitted as they do not visit all
        the states.}
    \label{fig:SSError_overall_in_stewart_plotkin}
\end{figure}

Here, the work of~\cite{Stewart2012} is extended by investigating a tournament
with \documentclass[a4paper]{article}

\usepackage{amsmath}
\usepackage{amssymb}
\usepackage[margin=1.5cm,
            includefoot,
            footskip=30pt]{geometry}
\usepackage{layout}
\usepackage{graphicx}
\usepackage{subcaption}

\usepackage{biblatex}
\usepackage{pdfpages}

\bibliography{main.bib}

\title{Suspicion: Recognising and evaluating the effectiveness
       of extortion in the Iterated Prisoner's Dilemma}
\author{Vincent A. Knight \and Nikoleta E. Glynatsi}
\date{\today}



\begin{document}

\maketitle

\begin{abstract}
    The Iterated Prisoner's Dilemma is a model for rational and evolutionary
    interactive behaviour. It has applications both in the study of human social
    behaviour as well as in biology.
    It is used to understand when and how a rational individual might
    accept an immediate cost to their own utility for the direct benefit of
    another.

    Much attention has been given to a class of strategies called
    Zero Determinant strategies. It has been theoretically shown that these
    strategies can ``extort'' any player.

    In this work, an approach to identify if observed strategies are playing in
    an extortionate way is described. Furthermore, experimental analysis of
    a large tournament with \input{assets/tex/number_of_full_strategies/main.tex}
    strategies is considered. In this setting
    the most highly performing strategies do not play in an extortionate way
    against each other but do against lower performing strategies.
    This suggests that whilst the theory of Zero Determinant strategies
    indicates that memory is not of fundamental importance to the evolution of
    cooperative behaviour, this is incomplete.
\end{abstract}

\section{Introduction}\label{sec:introduction}

Agent based game theoretic models have become a stalwart of the underpinning
mathematics of interactive behaviours. One of the major pieces of work
in this area is the pair of original computer tournaments run by Robert
Axelrod~\cite{Axelrod1980, Axelrod1980a}. These tournaments pitted submitted
computer strategies against each other in plays of the Iterated Prisoner's
Dilemma. A common game where agents can choose to pay a slight cost to their
immediate utility in the hope of building a reputation. This has been used in
economic and evolutionary game theory to understand the evolution of cooperative
behaviour.

Recently, a class of strategies was described in~\cite{Press2012} that can
provably extort any given opponent. In~\cite{Hilbe2013, Moran1707} some
questions have already been asked about the true effectiveness of these
strategies in an evolutionary setting. Here another question is asked: is it
possible to recognise this extortionate behaviour? A mathematical procedure for
suspicion is presented: in the same way that the continued actions of an
extortionate individual might raise suspicion.

This work makes use of the Axelrod Python library~\cite{Knight2018, Knight2016}
with a large number of Prisoner Dilemma strategies available to give an
extensive numerical example of the ideas presented.  The approach is presented
in Section~\ref{sec:delta-zd-strategies}.  All of the code and data discussed
in Section~\ref{sec:numerical-experiments} is open sourced, archived and
written according to best scientific principles~\cite{Wilson2014}. The data
archive can be found at~\cite{vincent_knight_2018_1297075}.

\section{Recognising Extortion}\label{sec:delta-zd-strategies}

In~\cite{Press2012}, given a match between 2 memory-one strategies, the concept
of Zero Determinant (ZD) strategies is introduced. The main result of that paper
shows that given two memory one players \(p, q\in\mathbb{R}^4\) a linear
relationship between the players' scores could be forced by one of the players.

Using the notation of~\cite{Press2012}, assuming the utilities for player \(p\)
are given by \(S_x=(R, S, T, P)\) and for player \(q\) by \(S_y=(R, T, S, P)\)
and that the stationary scores of each player is given by \(S_X\) and \(S_Y\)
respectively. The main result of~\cite{Press2012} is that if

\begin{equation}\label{eqn:linear_relationship_for_p}
    \tilde p=\alpha S_x + \beta S_y + \gamma
\end{equation}

or

\begin{equation}\label{eqn:linear_relationship_for_q}
    \tilde q=\alpha S_x + \beta S_y + \gamma
\end{equation}

where \(\tilde p = (1 - p_1, 1 - p_2, p_3, p_4)\) and
\(\tilde q = (1 - q_1, 1 - q_2, q_3, q_4)\) then:

\begin{equation}
    \alpha S_X + \beta S_Y + \gamma = 0
\end{equation}

In~\cite{Press2012} a particular type of ZD strategy is defined: extortionate
strategies. If:

\begin{equation}\label{eqn:constraint_for_extortion}
    \gamma = - P(\alpha + \beta)
\end{equation}

then the player can ensure they get a score \(\chi\) times
larger than the opponent. This extortion coefficient is given by:

\begin{equation}\label{eqn:definition_of_chi}
    \chi=\frac{-\beta}{\alpha}
\end{equation}

Thus, if (\ref{eqn:constraint_for_extortion}) holds and \(\chi >1\) a player is
said to extort their opponent.
Here, the reverse problem is considered: given a
\(p\in\mathbb{R}^4\) how does one identify \(\alpha, \beta\) if they
exist and is the strategy in fact acting in an extortionate way?

These conditions correspond to:

\begin{align}
    \tilde p_1 & = \alpha R + \beta R - P (\alpha + \beta)
            \label{eqn:condition_for_tilde_p1}\\
    \tilde p_2 & = \alpha S + \beta T - P (\alpha + \beta)
            \label{eqn:condition_for_tilde_p2}\\
    \tilde p_3 & = \alpha T + \beta S - P (\alpha + \beta)
            \label{eqn:condition_for_tilde_p3}\\
    \tilde p_4 & = \alpha P + \beta P - P (\alpha + \beta)
            \label{eqn:condition_for_tilde_p4}
\end{align}

Equation (\ref{eqn:condition_for_tilde_p4}) ensures that \(p_4=\tilde p_4=0\).
Equations (\ref{eqn:condition_for_tilde_p1}-\ref{eqn:condition_for_tilde_p3})
can be used to eliminate \(\alpha, \beta\), giving:

\begin{equation}\label{eqn:planar_definition_of_extortion}
    \tilde p_1 = \frac{(R - P)(\tilde p_2 + \tilde p_3)}{S + T - 2P}
\end{equation}

with:

\begin{equation}\label{eqn:definition_of_chi}
    \chi = \frac{\tilde p_2 (P - T) + \tilde p_3 (S - P)}
                {\tilde p_2 (P - S) + \tilde p_3 (T - P)}
\end{equation}

Given a strategy \(p\in\mathbb{R}^{4\times 1}\) equations
(\ref{eqn:condition_for_tilde_p4}), (\ref{eqn:planar_definition_of_extortion}-\ref{eqn:definition_of_chi}) can be used to check if
a strategy is extortionate. The conditions correspond to:

\begin{align}
    p_1 & = \frac{(R-P)(p_2 + p_3) - R + T + S - P}{S + T - 2P}
     \label{eqn:condition_for_p1}\\
    p_4 & = 0 \label{eqn:condition_for_p4}\\
    1 & > p_2 + p_3\label{eqn:condition_for_chi}
\end{align}

The algebraic steps necessary to prove these results are available in the
supporting materials.

All extortionate strategies reside on a triangular (\ref{eqn:condition_for_chi})
plane (\ref{eqn:condition_for_p1}) in 3 dimensions (\ref{eqn:condition_for_p4}).
Using this formulation it can be seen that a necessary (but not sufficient)
condition for an extortionate strategy is that it cooperates on average less
than 50\% of the time when in a state of disagreement with the opponent.

As an example, consider the known extortionate strategy \(p=(8 / 9, 1 / 2, 1 /
3, 0)\) from~\cite{Stewart2012} which is referred to as \texttt{Extort-2}. In
this case, for the standard values of \((R, T, S, P)\) constraint
(\ref{eqn:condition_for_p1}) corresponds to:

\begin{equation}
    p_1 = \frac{2(p_2 + p_3) + 1}{3}
\end{equation}

It is clear that in this case all constraints hold.

This approach could in fact be used to confirm that a given strategy is acting
in an extortionate manner even if it is not a memory one strategy. However, in
practice, if a closed form for \(p\) is not known, then due to measurement
and/or numerical error this would not work.

This problem can be written in the following linear algebraic form where
\(x=(\alpha, \beta)\)
and \(p^*=(\tilde p_1 - 1, tilde_2 - 1, p_3)\):

\begin{equation}\label{eqn:linear_algebraic_equation_for_p}
    Cx= p^*
\end{equation}

\(C\) corresponds to equations
(\ref{eqn:condition_for_tilde_p1}-\ref{eqn:condition_for_tilde_p3}) and is
given by:

\begin{equation}\label{eqn:definition_of_C}
    C =
    \begin{bmatrix}
        R - P & R- P \\
        S - P & T- P \\
        T - P & S- P \\
    \end{bmatrix}
\end{equation}

Note that in general, equation (\ref{eqn:linear_algebraic_equation_for_p}) will
not necessarily have a solution. From the Rouch\'{e}-Capelli theorem if there is
a solution it is unique as \(\text{rank}(C)=2\) which is the dimension of the
variable \(x\). The best fitting \(x\) is found by minimizing:

\begin{equation}\label{eqn:r_squared}
    \text{SSError} = \|C x- p^*\|_2^2 = \sum_{i=1}^{3}\left((C\bar x)_i-p_i^*\right)^2
\end{equation}

Note that \(\text{SSError}\), which is the square of the Frobenius
norm~\cite{Golub2013}, becomes a measure of how close a strategy is to being an
extortionate strategy. Suspicion
of extortion then corresponds to a threshold on \(\text{SSError}\).

By observing interactions (human or otherwise), their memory one representation
can be inferred and this approach can be used to recognise extortionate
behaviour. The notion of comparing theoretic and actual plays of the IPD is not
novel, see for example~\cite{Rand2013}. Immediately it is noted that if the
environment is noisy~\cite{Wu1995} then no strategy can be considered to be
extortionate as \(p_4>0\).

In the next section, this idea will be illustrated by observing the interactions
that take place in a computer based tournament of the IPD\@.

\section{Numerical experiments}\label{sec:numerical-experiments}

In~\cite{Stewart2012} results from a tournament with
\input{./assets/tex/number_of_stewart_plotkin_strategies/main.tex} strategies,
was presented with specific consideration given to ZD strategies. This
tournament is reproduced here using the Axelrod-Python
project~\cite{Knight2016}. To obtain a good measure of the corresponding
transition rates for each strategy all matches have been run for
\input{assets/tex/number_of_turns/main.tex} turns and every match has been
repeated \input{assets/tex/number_of_repetitions/main.tex} times. All of this
interaction data is available at~\cite{vincent_knight_2018_1297075}. A good
match between the inferred Markov chain and the state distribution of the actual
interactions has been verified. Data for this is presented in the supplementary
materials.

Figure~\ref{fig:SSError_overall_in_stewart_plotkin} shows the \(\text{SSError}\)
values for all the strategies in the tournament, as reported
in~\cite{Stewart2012} the extortionate strategy (which has an expected
\(\text{SSError}\) approximately 0) gains a large number of wins.

\begin{figure}[!htbp]
    \centering
    \includegraphics[width=.8\textwidth]{./assets/img/SSError_overall_in_stewart_plotkin/main.pdf}
    \caption{\(\text{SSError}\) and state probabilities for the strategies
        of~\cite{Stewart2012}, ordered both by number of wins and overall score.
        Note that \(P(DC)\) is not shown as it corresponds to the transpose of
        \(P(CD)\). Cooperator and Defector are omitted as they do not visit all
        the states.}
    \label{fig:SSError_overall_in_stewart_plotkin}
\end{figure}

Here, the work of~\cite{Stewart2012} is extended by investigating a tournament
with \input{assets/tex/number_of_full_strategies/main.tex}
strategies.

The results of this analysis are shown in
Figure~\ref{fig:SSError_and_probabilities_in_full}. The top ranking strategies
by number of wins seem to be extortionate (but not against all strategies) and
it can be seen that a small sub group of strategies achieve mutual defection.
All the top ranking strategies according to score achieve mutual cooperation and
do not extort each other, however they
\textbf{do} exhibit extortionate behaviour towards a number of the lower ranking
strategies.

\begin{figure}[!htbp]
    \centering
    \includegraphics[width=.8\textwidth]{./assets/img/SSError_and_probabilities_in_full/main.pdf}
    \caption{\(\text{SSError}\) for the strategies for the full tournament. Only
    strategy interactions for which \(p_4=0\) and \(\chi>1\) are displayed.}
    \label{fig:SSError_and_probabilities_in_full}
\end{figure}

\section{Conclusion}\label{sec:conclusion}

This work defines an approach to measure whether or not a player is playing a
strategy that corresponds to an extortionate strategy as defined
in~\cite{Press2012}: a mathematical model for suspicion. Indeed, all
extortionate strategies have been
 classified as lying on a triangular plane.
This rigorous classification fails to be robust to small measurement error, thus
a statistical approach is proposed.
This is done through a linear algebraic approach for approximating the solution
of a linear system. Using this, a large number of pairwise interactions is
simulated and in fact very few strategies are found to act extortionately.

The work of~\cite{Press2012}, whilst showing that a clever approach to taking
advantage of another memory one strategy exists: this is incomplete. Whilst the
elegance of this result is very attractive, just as the simplicity of the
victory of Tit For Tat in Axelrod's original tournaments was, it is incomplete.
Extortionate strategies achieve a high number of wins but they do not
achieve a high score which corresponds to the fitness landscape in an
evolutionary sense. From the large number of interactions a payoff matrix \(S\)
can be measured where \(S_{ij}\) denotes the score (using standard values of
\((R, S, T, P) = (3, 0, 5, 1)\)) of the \(i\)th strategy
against the \(j\)th strategy. Using this, the replicator equation
describes the evolution of the system based on a population density fitness
function:

\begin{equation}\label{eqn:replicator_dynamics}
    \frac{dx}{dt} = x(S-x^TS x)
\end{equation}

Equation (\ref{eqn:replicator_dynamics}) is solved numerically through an
integration technique described in~\cite{Petzold1983} and
Figure~\ref{fig:replicator_dynamics} shows the evolution of the distribution of
the system: the various strategies are ranked by scores. It is clear to see that
only the high ranking strategies survive the evolutionary process (in fact,
only \input{./assets/img/replicator_dynamics/main.tex}
have a final distribution greater than \(10 ^ {-2}\)). This confirms the
findings of~\cite{Moran1707} in which sophisticated strategies resist
evolutionary invasion of shorter memory strategies. Recalling
Figure~\ref{fig:SSError_and_probabilities_in_full} this demonstrates that:

\begin{itemize}
    \item Cooperation emerges through the evolutionary process: the high scoring
        strategies do not exhibit extortionate behaviour towards each other.
    \item Extortionate strategies do not survive the evolutionary process.
\end{itemize}

\begin{figure}[!htbp]
    \centering
    \includegraphics[width=.8\textwidth]{./assets/img/replicator_dynamics/main.pdf}
    \caption{Numerical simulation of the replicator equation
    (\ref{eqn:replicator_dynamics}): strategies are ordered by score, only the strategies with a high score survive the evolutionary process.}
    \label{fig:replicator_dynamics}
\end{figure}

This work can be used to classify plays of the IPD\@: data can be collected from
actual interactions (in lab or in the field). Furthermore, this allows for a
classification method similar to the notion of fingerprinting presented
in~\cite{Ashlock2008}. Trained strategies can potentially be classified as
extortionate or not or it could be possible to even constrain the reinforcement
learning approaches that are becoming prevalent in the literature.
Alternatively, this mathematical approach for recognising extortion could be
used in sophisticated strategies to defend against invasion. Arguably, some of
the strategies considered here exhibit this behaviour, indeed as described
in~\cite{Harper2017}, the top ranking strategies in the full tournament are
obtained using evolutionary reinforcement learning techniques, thus, suspicion
of extortionate behaviour could in fact be an evolutionary trait.

\section*{Acknowledgements}

The following open source software libraries were used in this research:

\begin{itemize}
    \item The Axelrod ~\cite{Knight2016, Knight2018} library (IPD strategies and
        tournaments).
    \item The sympy library~\cite{Meurer2017} (verification of all symbolic
        calculations).
    \item The matplotlib~\cite{Droettboom2018} library (visualisation).
    \item The pandas~\cite{Structures2010}, dask~\cite{Dask2016} and
        NumPy~\cite{Oliphant2015} libraries (data manipulation).
    \item The SciPy~\cite{Jones2001} library (numerical integration of the
        replicator equation).
\end{itemize}

This work was performed using the computational facilities of the Advanced
Research Computing @ Cardiff (ARCCA) Division, Cardiff University.

\printbibliography

\newpage
\section*{Supplementary materials}

\includepdf{assets/pdf/proof_of_form_of_extortionate_strategies/main.pdf}

\newpage

Using the pair wise interactions the transition rates \(p,
q\) can be measured and the steady state probabilities inferred and compared to
the actual probabilities of each state.
This is done numerically by computing the singular eigenvector of the
matrix \(A\) \cite{Stewart2009}:

\[
    A =
    \begin{bmatrix}
        p_1 q_1 & p_1 (1 - q_1) & (1 - p_1) q_1 & (1 -p_1) (1 - q_1) \\
        p_2 q_2 & p_2 (1 - q_2) & (1 - p_2) q_2 & (1 -p_2) (1 - q_2) \\
        p_3 q_3 & p_3 (1 - q_3) & (1 - p_3) q_3 & (1 -p_3) (1 - q_3) \\
        p_4 q_4 & p_4 (1 - q_4) & (1 - p_4) q_4 & (1 -p_4) (1 - q_4) \\
    \end{bmatrix}
\]

Figure~\ref{fig:computed_probabilities_vs_theoretic_probabilities} shows a
regression line fitted to every pairwise interaction with a reported
\(\text{SSError}\) value (pairwise interactions with missing states were
omitted). This serves to validate the approach: a part from some edge cases the
relationship is consistent.

\begin{figure}[!htbp]
    \centering
    \includegraphics[width=.8\textwidth]{./assets/img/computed_probabilities_vs_theoretic_probabilities/main.pdf}
    \caption{The
        relationship between the steady state probabilities inferred from the
        measured transitions and the actual steady state probabilities. A linear
        regression line is included validating the approach.}
    \label{fig:computed_probabilities_vs_theoretic_probabilities}
\end{figure}


\end{document}

strategies.

The results of this analysis are shown in
Figure~\ref{fig:SSError_and_probabilities_in_full}. The top ranking strategies
by number of wins seem to be extortionate (but not against all strategies) and
it can be seen that a small sub group of strategies achieve mutual defection.
All the top ranking strategies according to score achieve mutual cooperation and
do not extort each other, however they
\textbf{do} exhibit extortionate behaviour towards a number of the lower ranking
strategies.

\begin{figure}[!htbp]
    \centering
    \includegraphics[width=.8\textwidth]{./assets/img/SSError_and_probabilities_in_full/main.pdf}
    \caption{\(\text{SSError}\) for the strategies for the full tournament. Only
    strategy interactions for which \(p_4=0\) and \(\chi>1\) are displayed.}
    \label{fig:SSError_and_probabilities_in_full}
\end{figure}

\section{Conclusion}\label{sec:conclusion}

This work defines an approach to measure whether or not a player is playing a
strategy that corresponds to an extortionate strategy as defined
in~\cite{Press2012}: a mathematical model for suspicion. Indeed, all
extortionate strategies have been
 classified as lying on a triangular plane.
This rigorous classification fails to be robust to small measurement error, thus
a statistical approach is proposed.
This is done through a linear algebraic approach for approximating the solution
of a linear system. Using this, a large number of pairwise interactions is
simulated and in fact very few strategies are found to act extortionately.

The work of~\cite{Press2012}, whilst showing that a clever approach to taking
advantage of another memory one strategy exists: this is incomplete. Whilst the
elegance of this result is very attractive, just as the simplicity of the
victory of Tit For Tat in Axelrod's original tournaments was, it is incomplete.
Extortionate strategies achieve a high number of wins but they do not
achieve a high score which corresponds to the fitness landscape in an
evolutionary sense. From the large number of interactions a payoff matrix \(S\)
can be measured where \(S_{ij}\) denotes the score (using standard values of
\((R, S, T, P) = (3, 0, 5, 1)\)) of the \(i\)th strategy
against the \(j\)th strategy. Using this, the replicator equation
describes the evolution of the system based on a population density fitness
function:

\begin{equation}\label{eqn:replicator_dynamics}
    \frac{dx}{dt} = x(S-x^TS x)
\end{equation}

Equation (\ref{eqn:replicator_dynamics}) is solved numerically through an
integration technique described in~\cite{Petzold1983} and
Figure~\ref{fig:replicator_dynamics} shows the evolution of the distribution of
the system: the various strategies are ranked by scores. It is clear to see that
only the high ranking strategies survive the evolutionary process (in fact,
only \documentclass[a4paper]{article}

\usepackage{amsmath}
\usepackage{amssymb}
\usepackage[margin=1.5cm,
            includefoot,
            footskip=30pt]{geometry}
\usepackage{layout}
\usepackage{graphicx}
\usepackage{subcaption}

\usepackage{biblatex}
\usepackage{pdfpages}

\bibliography{main.bib}

\title{Suspicion: Recognising and evaluating the effectiveness
       of extortion in the Iterated Prisoner's Dilemma}
\author{Vincent A. Knight \and Nikoleta E. Glynatsi}
\date{\today}



\begin{document}

\maketitle

\begin{abstract}
    The Iterated Prisoner's Dilemma is a model for rational and evolutionary
    interactive behaviour. It has applications both in the study of human social
    behaviour as well as in biology.
    It is used to understand when and how a rational individual might
    accept an immediate cost to their own utility for the direct benefit of
    another.

    Much attention has been given to a class of strategies called
    Zero Determinant strategies. It has been theoretically shown that these
    strategies can ``extort'' any player.

    In this work, an approach to identify if observed strategies are playing in
    an extortionate way is described. Furthermore, experimental analysis of
    a large tournament with \input{assets/tex/number_of_full_strategies/main.tex}
    strategies is considered. In this setting
    the most highly performing strategies do not play in an extortionate way
    against each other but do against lower performing strategies.
    This suggests that whilst the theory of Zero Determinant strategies
    indicates that memory is not of fundamental importance to the evolution of
    cooperative behaviour, this is incomplete.
\end{abstract}

\section{Introduction}\label{sec:introduction}

Agent based game theoretic models have become a stalwart of the underpinning
mathematics of interactive behaviours. One of the major pieces of work
in this area is the pair of original computer tournaments run by Robert
Axelrod~\cite{Axelrod1980, Axelrod1980a}. These tournaments pitted submitted
computer strategies against each other in plays of the Iterated Prisoner's
Dilemma. A common game where agents can choose to pay a slight cost to their
immediate utility in the hope of building a reputation. This has been used in
economic and evolutionary game theory to understand the evolution of cooperative
behaviour.

Recently, a class of strategies was described in~\cite{Press2012} that can
provably extort any given opponent. In~\cite{Hilbe2013, Moran1707} some
questions have already been asked about the true effectiveness of these
strategies in an evolutionary setting. Here another question is asked: is it
possible to recognise this extortionate behaviour? A mathematical procedure for
suspicion is presented: in the same way that the continued actions of an
extortionate individual might raise suspicion.

This work makes use of the Axelrod Python library~\cite{Knight2018, Knight2016}
with a large number of Prisoner Dilemma strategies available to give an
extensive numerical example of the ideas presented.  The approach is presented
in Section~\ref{sec:delta-zd-strategies}.  All of the code and data discussed
in Section~\ref{sec:numerical-experiments} is open sourced, archived and
written according to best scientific principles~\cite{Wilson2014}. The data
archive can be found at~\cite{vincent_knight_2018_1297075}.

\section{Recognising Extortion}\label{sec:delta-zd-strategies}

In~\cite{Press2012}, given a match between 2 memory-one strategies, the concept
of Zero Determinant (ZD) strategies is introduced. The main result of that paper
shows that given two memory one players \(p, q\in\mathbb{R}^4\) a linear
relationship between the players' scores could be forced by one of the players.

Using the notation of~\cite{Press2012}, assuming the utilities for player \(p\)
are given by \(S_x=(R, S, T, P)\) and for player \(q\) by \(S_y=(R, T, S, P)\)
and that the stationary scores of each player is given by \(S_X\) and \(S_Y\)
respectively. The main result of~\cite{Press2012} is that if

\begin{equation}\label{eqn:linear_relationship_for_p}
    \tilde p=\alpha S_x + \beta S_y + \gamma
\end{equation}

or

\begin{equation}\label{eqn:linear_relationship_for_q}
    \tilde q=\alpha S_x + \beta S_y + \gamma
\end{equation}

where \(\tilde p = (1 - p_1, 1 - p_2, p_3, p_4)\) and
\(\tilde q = (1 - q_1, 1 - q_2, q_3, q_4)\) then:

\begin{equation}
    \alpha S_X + \beta S_Y + \gamma = 0
\end{equation}

In~\cite{Press2012} a particular type of ZD strategy is defined: extortionate
strategies. If:

\begin{equation}\label{eqn:constraint_for_extortion}
    \gamma = - P(\alpha + \beta)
\end{equation}

then the player can ensure they get a score \(\chi\) times
larger than the opponent. This extortion coefficient is given by:

\begin{equation}\label{eqn:definition_of_chi}
    \chi=\frac{-\beta}{\alpha}
\end{equation}

Thus, if (\ref{eqn:constraint_for_extortion}) holds and \(\chi >1\) a player is
said to extort their opponent.
Here, the reverse problem is considered: given a
\(p\in\mathbb{R}^4\) how does one identify \(\alpha, \beta\) if they
exist and is the strategy in fact acting in an extortionate way?

These conditions correspond to:

\begin{align}
    \tilde p_1 & = \alpha R + \beta R - P (\alpha + \beta)
            \label{eqn:condition_for_tilde_p1}\\
    \tilde p_2 & = \alpha S + \beta T - P (\alpha + \beta)
            \label{eqn:condition_for_tilde_p2}\\
    \tilde p_3 & = \alpha T + \beta S - P (\alpha + \beta)
            \label{eqn:condition_for_tilde_p3}\\
    \tilde p_4 & = \alpha P + \beta P - P (\alpha + \beta)
            \label{eqn:condition_for_tilde_p4}
\end{align}

Equation (\ref{eqn:condition_for_tilde_p4}) ensures that \(p_4=\tilde p_4=0\).
Equations (\ref{eqn:condition_for_tilde_p1}-\ref{eqn:condition_for_tilde_p3})
can be used to eliminate \(\alpha, \beta\), giving:

\begin{equation}\label{eqn:planar_definition_of_extortion}
    \tilde p_1 = \frac{(R - P)(\tilde p_2 + \tilde p_3)}{S + T - 2P}
\end{equation}

with:

\begin{equation}\label{eqn:definition_of_chi}
    \chi = \frac{\tilde p_2 (P - T) + \tilde p_3 (S - P)}
                {\tilde p_2 (P - S) + \tilde p_3 (T - P)}
\end{equation}

Given a strategy \(p\in\mathbb{R}^{4\times 1}\) equations
(\ref{eqn:condition_for_tilde_p4}), (\ref{eqn:planar_definition_of_extortion}-\ref{eqn:definition_of_chi}) can be used to check if
a strategy is extortionate. The conditions correspond to:

\begin{align}
    p_1 & = \frac{(R-P)(p_2 + p_3) - R + T + S - P}{S + T - 2P}
     \label{eqn:condition_for_p1}\\
    p_4 & = 0 \label{eqn:condition_for_p4}\\
    1 & > p_2 + p_3\label{eqn:condition_for_chi}
\end{align}

The algebraic steps necessary to prove these results are available in the
supporting materials.

All extortionate strategies reside on a triangular (\ref{eqn:condition_for_chi})
plane (\ref{eqn:condition_for_p1}) in 3 dimensions (\ref{eqn:condition_for_p4}).
Using this formulation it can be seen that a necessary (but not sufficient)
condition for an extortionate strategy is that it cooperates on average less
than 50\% of the time when in a state of disagreement with the opponent.

As an example, consider the known extortionate strategy \(p=(8 / 9, 1 / 2, 1 /
3, 0)\) from~\cite{Stewart2012} which is referred to as \texttt{Extort-2}. In
this case, for the standard values of \((R, T, S, P)\) constraint
(\ref{eqn:condition_for_p1}) corresponds to:

\begin{equation}
    p_1 = \frac{2(p_2 + p_3) + 1}{3}
\end{equation}

It is clear that in this case all constraints hold.

This approach could in fact be used to confirm that a given strategy is acting
in an extortionate manner even if it is not a memory one strategy. However, in
practice, if a closed form for \(p\) is not known, then due to measurement
and/or numerical error this would not work.

This problem can be written in the following linear algebraic form where
\(x=(\alpha, \beta)\)
and \(p^*=(\tilde p_1 - 1, tilde_2 - 1, p_3)\):

\begin{equation}\label{eqn:linear_algebraic_equation_for_p}
    Cx= p^*
\end{equation}

\(C\) corresponds to equations
(\ref{eqn:condition_for_tilde_p1}-\ref{eqn:condition_for_tilde_p3}) and is
given by:

\begin{equation}\label{eqn:definition_of_C}
    C =
    \begin{bmatrix}
        R - P & R- P \\
        S - P & T- P \\
        T - P & S- P \\
    \end{bmatrix}
\end{equation}

Note that in general, equation (\ref{eqn:linear_algebraic_equation_for_p}) will
not necessarily have a solution. From the Rouch\'{e}-Capelli theorem if there is
a solution it is unique as \(\text{rank}(C)=2\) which is the dimension of the
variable \(x\). The best fitting \(x\) is found by minimizing:

\begin{equation}\label{eqn:r_squared}
    \text{SSError} = \|C x- p^*\|_2^2 = \sum_{i=1}^{3}\left((C\bar x)_i-p_i^*\right)^2
\end{equation}

Note that \(\text{SSError}\), which is the square of the Frobenius
norm~\cite{Golub2013}, becomes a measure of how close a strategy is to being an
extortionate strategy. Suspicion
of extortion then corresponds to a threshold on \(\text{SSError}\).

By observing interactions (human or otherwise), their memory one representation
can be inferred and this approach can be used to recognise extortionate
behaviour. The notion of comparing theoretic and actual plays of the IPD is not
novel, see for example~\cite{Rand2013}. Immediately it is noted that if the
environment is noisy~\cite{Wu1995} then no strategy can be considered to be
extortionate as \(p_4>0\).

In the next section, this idea will be illustrated by observing the interactions
that take place in a computer based tournament of the IPD\@.

\section{Numerical experiments}\label{sec:numerical-experiments}

In~\cite{Stewart2012} results from a tournament with
\input{./assets/tex/number_of_stewart_plotkin_strategies/main.tex} strategies,
was presented with specific consideration given to ZD strategies. This
tournament is reproduced here using the Axelrod-Python
project~\cite{Knight2016}. To obtain a good measure of the corresponding
transition rates for each strategy all matches have been run for
\input{assets/tex/number_of_turns/main.tex} turns and every match has been
repeated \input{assets/tex/number_of_repetitions/main.tex} times. All of this
interaction data is available at~\cite{vincent_knight_2018_1297075}. A good
match between the inferred Markov chain and the state distribution of the actual
interactions has been verified. Data for this is presented in the supplementary
materials.

Figure~\ref{fig:SSError_overall_in_stewart_plotkin} shows the \(\text{SSError}\)
values for all the strategies in the tournament, as reported
in~\cite{Stewart2012} the extortionate strategy (which has an expected
\(\text{SSError}\) approximately 0) gains a large number of wins.

\begin{figure}[!htbp]
    \centering
    \includegraphics[width=.8\textwidth]{./assets/img/SSError_overall_in_stewart_plotkin/main.pdf}
    \caption{\(\text{SSError}\) and state probabilities for the strategies
        of~\cite{Stewart2012}, ordered both by number of wins and overall score.
        Note that \(P(DC)\) is not shown as it corresponds to the transpose of
        \(P(CD)\). Cooperator and Defector are omitted as they do not visit all
        the states.}
    \label{fig:SSError_overall_in_stewart_plotkin}
\end{figure}

Here, the work of~\cite{Stewart2012} is extended by investigating a tournament
with \input{assets/tex/number_of_full_strategies/main.tex}
strategies.

The results of this analysis are shown in
Figure~\ref{fig:SSError_and_probabilities_in_full}. The top ranking strategies
by number of wins seem to be extortionate (but not against all strategies) and
it can be seen that a small sub group of strategies achieve mutual defection.
All the top ranking strategies according to score achieve mutual cooperation and
do not extort each other, however they
\textbf{do} exhibit extortionate behaviour towards a number of the lower ranking
strategies.

\begin{figure}[!htbp]
    \centering
    \includegraphics[width=.8\textwidth]{./assets/img/SSError_and_probabilities_in_full/main.pdf}
    \caption{\(\text{SSError}\) for the strategies for the full tournament. Only
    strategy interactions for which \(p_4=0\) and \(\chi>1\) are displayed.}
    \label{fig:SSError_and_probabilities_in_full}
\end{figure}

\section{Conclusion}\label{sec:conclusion}

This work defines an approach to measure whether or not a player is playing a
strategy that corresponds to an extortionate strategy as defined
in~\cite{Press2012}: a mathematical model for suspicion. Indeed, all
extortionate strategies have been
 classified as lying on a triangular plane.
This rigorous classification fails to be robust to small measurement error, thus
a statistical approach is proposed.
This is done through a linear algebraic approach for approximating the solution
of a linear system. Using this, a large number of pairwise interactions is
simulated and in fact very few strategies are found to act extortionately.

The work of~\cite{Press2012}, whilst showing that a clever approach to taking
advantage of another memory one strategy exists: this is incomplete. Whilst the
elegance of this result is very attractive, just as the simplicity of the
victory of Tit For Tat in Axelrod's original tournaments was, it is incomplete.
Extortionate strategies achieve a high number of wins but they do not
achieve a high score which corresponds to the fitness landscape in an
evolutionary sense. From the large number of interactions a payoff matrix \(S\)
can be measured where \(S_{ij}\) denotes the score (using standard values of
\((R, S, T, P) = (3, 0, 5, 1)\)) of the \(i\)th strategy
against the \(j\)th strategy. Using this, the replicator equation
describes the evolution of the system based on a population density fitness
function:

\begin{equation}\label{eqn:replicator_dynamics}
    \frac{dx}{dt} = x(S-x^TS x)
\end{equation}

Equation (\ref{eqn:replicator_dynamics}) is solved numerically through an
integration technique described in~\cite{Petzold1983} and
Figure~\ref{fig:replicator_dynamics} shows the evolution of the distribution of
the system: the various strategies are ranked by scores. It is clear to see that
only the high ranking strategies survive the evolutionary process (in fact,
only \input{./assets/img/replicator_dynamics/main.tex}
have a final distribution greater than \(10 ^ {-2}\)). This confirms the
findings of~\cite{Moran1707} in which sophisticated strategies resist
evolutionary invasion of shorter memory strategies. Recalling
Figure~\ref{fig:SSError_and_probabilities_in_full} this demonstrates that:

\begin{itemize}
    \item Cooperation emerges through the evolutionary process: the high scoring
        strategies do not exhibit extortionate behaviour towards each other.
    \item Extortionate strategies do not survive the evolutionary process.
\end{itemize}

\begin{figure}[!htbp]
    \centering
    \includegraphics[width=.8\textwidth]{./assets/img/replicator_dynamics/main.pdf}
    \caption{Numerical simulation of the replicator equation
    (\ref{eqn:replicator_dynamics}): strategies are ordered by score, only the strategies with a high score survive the evolutionary process.}
    \label{fig:replicator_dynamics}
\end{figure}

This work can be used to classify plays of the IPD\@: data can be collected from
actual interactions (in lab or in the field). Furthermore, this allows for a
classification method similar to the notion of fingerprinting presented
in~\cite{Ashlock2008}. Trained strategies can potentially be classified as
extortionate or not or it could be possible to even constrain the reinforcement
learning approaches that are becoming prevalent in the literature.
Alternatively, this mathematical approach for recognising extortion could be
used in sophisticated strategies to defend against invasion. Arguably, some of
the strategies considered here exhibit this behaviour, indeed as described
in~\cite{Harper2017}, the top ranking strategies in the full tournament are
obtained using evolutionary reinforcement learning techniques, thus, suspicion
of extortionate behaviour could in fact be an evolutionary trait.

\section*{Acknowledgements}

The following open source software libraries were used in this research:

\begin{itemize}
    \item The Axelrod ~\cite{Knight2016, Knight2018} library (IPD strategies and
        tournaments).
    \item The sympy library~\cite{Meurer2017} (verification of all symbolic
        calculations).
    \item The matplotlib~\cite{Droettboom2018} library (visualisation).
    \item The pandas~\cite{Structures2010}, dask~\cite{Dask2016} and
        NumPy~\cite{Oliphant2015} libraries (data manipulation).
    \item The SciPy~\cite{Jones2001} library (numerical integration of the
        replicator equation).
\end{itemize}

This work was performed using the computational facilities of the Advanced
Research Computing @ Cardiff (ARCCA) Division, Cardiff University.

\printbibliography

\newpage
\section*{Supplementary materials}

\includepdf{assets/pdf/proof_of_form_of_extortionate_strategies/main.pdf}

\newpage

Using the pair wise interactions the transition rates \(p,
q\) can be measured and the steady state probabilities inferred and compared to
the actual probabilities of each state.
This is done numerically by computing the singular eigenvector of the
matrix \(A\) \cite{Stewart2009}:

\[
    A =
    \begin{bmatrix}
        p_1 q_1 & p_1 (1 - q_1) & (1 - p_1) q_1 & (1 -p_1) (1 - q_1) \\
        p_2 q_2 & p_2 (1 - q_2) & (1 - p_2) q_2 & (1 -p_2) (1 - q_2) \\
        p_3 q_3 & p_3 (1 - q_3) & (1 - p_3) q_3 & (1 -p_3) (1 - q_3) \\
        p_4 q_4 & p_4 (1 - q_4) & (1 - p_4) q_4 & (1 -p_4) (1 - q_4) \\
    \end{bmatrix}
\]

Figure~\ref{fig:computed_probabilities_vs_theoretic_probabilities} shows a
regression line fitted to every pairwise interaction with a reported
\(\text{SSError}\) value (pairwise interactions with missing states were
omitted). This serves to validate the approach: a part from some edge cases the
relationship is consistent.

\begin{figure}[!htbp]
    \centering
    \includegraphics[width=.8\textwidth]{./assets/img/computed_probabilities_vs_theoretic_probabilities/main.pdf}
    \caption{The
        relationship between the steady state probabilities inferred from the
        measured transitions and the actual steady state probabilities. A linear
        regression line is included validating the approach.}
    \label{fig:computed_probabilities_vs_theoretic_probabilities}
\end{figure}


\end{document}

have a final distribution greater than \(10 ^ {-2}\)). This confirms the
findings of~\cite{Moran1707} in which sophisticated strategies resist
evolutionary invasion of shorter memory strategies. Recalling
Figure~\ref{fig:SSError_and_probabilities_in_full} this demonstrates that:

\begin{itemize}
    \item Cooperation emerges through the evolutionary process: the high scoring
        strategies do not exhibit extortionate behaviour towards each other.
    \item Extortionate strategies do not survive the evolutionary process.
\end{itemize}

\begin{figure}[!htbp]
    \centering
    \includegraphics[width=.8\textwidth]{./assets/img/replicator_dynamics/main.pdf}
    \caption{Numerical simulation of the replicator equation
    (\ref{eqn:replicator_dynamics}): strategies are ordered by score, only the strategies with a high score survive the evolutionary process.}
    \label{fig:replicator_dynamics}
\end{figure}

This work can be used to classify plays of the IPD\@: data can be collected from
actual interactions (in lab or in the field). Furthermore, this allows for a
classification method similar to the notion of fingerprinting presented
in~\cite{Ashlock2008}. Trained strategies can potentially be classified as
extortionate or not or it could be possible to even constrain the reinforcement
learning approaches that are becoming prevalent in the literature.
Alternatively, this mathematical approach for recognising extortion could be
used in sophisticated strategies to defend against invasion. Arguably, some of
the strategies considered here exhibit this behaviour, indeed as described
in~\cite{Harper2017}, the top ranking strategies in the full tournament are
obtained using evolutionary reinforcement learning techniques, thus, suspicion
of extortionate behaviour could in fact be an evolutionary trait.

\section*{Acknowledgements}

The following open source software libraries were used in this research:

\begin{itemize}
    \item The Axelrod ~\cite{Knight2016, Knight2018} library (IPD strategies and
        tournaments).
    \item The sympy library~\cite{Meurer2017} (verification of all symbolic
        calculations).
    \item The matplotlib~\cite{Droettboom2018} library (visualisation).
    \item The pandas~\cite{Structures2010}, dask~\cite{Dask2016} and
        NumPy~\cite{Oliphant2015} libraries (data manipulation).
    \item The SciPy~\cite{Jones2001} library (numerical integration of the
        replicator equation).
\end{itemize}

This work was performed using the computational facilities of the Advanced
Research Computing @ Cardiff (ARCCA) Division, Cardiff University.

\printbibliography

\newpage
\section*{Supplementary materials}

\includepdf{assets/pdf/proof_of_form_of_extortionate_strategies/main.pdf}

\newpage

Using the pair wise interactions the transition rates \(p,
q\) can be measured and the steady state probabilities inferred and compared to
the actual probabilities of each state.
This is done numerically by computing the singular eigenvector of the
matrix \(A\) \cite{Stewart2009}:

\[
    A =
    \begin{bmatrix}
        p_1 q_1 & p_1 (1 - q_1) & (1 - p_1) q_1 & (1 -p_1) (1 - q_1) \\
        p_2 q_2 & p_2 (1 - q_2) & (1 - p_2) q_2 & (1 -p_2) (1 - q_2) \\
        p_3 q_3 & p_3 (1 - q_3) & (1 - p_3) q_3 & (1 -p_3) (1 - q_3) \\
        p_4 q_4 & p_4 (1 - q_4) & (1 - p_4) q_4 & (1 -p_4) (1 - q_4) \\
    \end{bmatrix}
\]

Figure~\ref{fig:computed_probabilities_vs_theoretic_probabilities} shows a
regression line fitted to every pairwise interaction with a reported
\(\text{SSError}\) value (pairwise interactions with missing states were
omitted). This serves to validate the approach: a part from some edge cases the
relationship is consistent.

\begin{figure}[!htbp]
    \centering
    \includegraphics[width=.8\textwidth]{./assets/img/computed_probabilities_vs_theoretic_probabilities/main.pdf}
    \caption{The
        relationship between the steady state probabilities inferred from the
        measured transitions and the actual steady state probabilities. A linear
        regression line is included validating the approach.}
    \label{fig:computed_probabilities_vs_theoretic_probabilities}
\end{figure}


\end{document}
 times. All of this
interaction data is available at~\cite{vincent_knight_2018_1297075}. A good
match between the inferred Markov chain and the state distribution of the actual
interactions has been verified. Data for this is presented in the supplementary
materials.

Figure~\ref{fig:SSError_overall_in_stewart_plotkin} shows the \(\text{SSError}\)
values for all the strategies in the tournament, as reported
in~\cite{Stewart2012} the extortionate strategy (which has an expected
\(\text{SSError}\) approximately 0) gains a large number of wins.

\begin{figure}[!htbp]
    \centering
    \includegraphics[width=.8\textwidth]{./assets/img/SSError_overall_in_stewart_plotkin/main.pdf}
    \caption{\(\text{SSError}\) and state probabilities for the strategies
        of~\cite{Stewart2012}, ordered both by number of wins and overall score.
        Note that \(P(DC)\) is not shown as it corresponds to the transpose of
        \(P(CD)\). Cooperator and Defector are omitted as they do not visit all
        the states.}
    \label{fig:SSError_overall_in_stewart_plotkin}
\end{figure}

Here, the work of~\cite{Stewart2012} is extended by investigating a tournament
with \documentclass[a4paper]{article}

\usepackage{amsmath}
\usepackage{amssymb}
\usepackage[margin=1.5cm,
            includefoot,
            footskip=30pt]{geometry}
\usepackage{layout}
\usepackage{graphicx}
\usepackage{subcaption}

\usepackage{biblatex}
\usepackage{pdfpages}

\bibliography{main.bib}

\title{Suspicion: Recognising and evaluating the effectiveness
       of extortion in the Iterated Prisoner's Dilemma}
\author{Vincent A. Knight \and Nikoleta E. Glynatsi}
\date{\today}



\begin{document}

\maketitle

\begin{abstract}
    The Iterated Prisoner's Dilemma is a model for rational and evolutionary
    interactive behaviour. It has applications both in the study of human social
    behaviour as well as in biology.
    It is used to understand when and how a rational individual might
    accept an immediate cost to their own utility for the direct benefit of
    another.

    Much attention has been given to a class of strategies called
    Zero Determinant strategies. It has been theoretically shown that these
    strategies can ``extort'' any player.

    In this work, an approach to identify if observed strategies are playing in
    an extortionate way is described. Furthermore, experimental analysis of
    a large tournament with \documentclass[a4paper]{article}

\usepackage{amsmath}
\usepackage{amssymb}
\usepackage[margin=1.5cm,
            includefoot,
            footskip=30pt]{geometry}
\usepackage{layout}
\usepackage{graphicx}
\usepackage{subcaption}

\usepackage{biblatex}
\usepackage{pdfpages}

\bibliography{main.bib}

\title{Suspicion: Recognising and evaluating the effectiveness
       of extortion in the Iterated Prisoner's Dilemma}
\author{Vincent A. Knight \and Nikoleta E. Glynatsi}
\date{\today}



\begin{document}

\maketitle

\begin{abstract}
    The Iterated Prisoner's Dilemma is a model for rational and evolutionary
    interactive behaviour. It has applications both in the study of human social
    behaviour as well as in biology.
    It is used to understand when and how a rational individual might
    accept an immediate cost to their own utility for the direct benefit of
    another.

    Much attention has been given to a class of strategies called
    Zero Determinant strategies. It has been theoretically shown that these
    strategies can ``extort'' any player.

    In this work, an approach to identify if observed strategies are playing in
    an extortionate way is described. Furthermore, experimental analysis of
    a large tournament with \input{assets/tex/number_of_full_strategies/main.tex}
    strategies is considered. In this setting
    the most highly performing strategies do not play in an extortionate way
    against each other but do against lower performing strategies.
    This suggests that whilst the theory of Zero Determinant strategies
    indicates that memory is not of fundamental importance to the evolution of
    cooperative behaviour, this is incomplete.
\end{abstract}

\section{Introduction}\label{sec:introduction}

Agent based game theoretic models have become a stalwart of the underpinning
mathematics of interactive behaviours. One of the major pieces of work
in this area is the pair of original computer tournaments run by Robert
Axelrod~\cite{Axelrod1980, Axelrod1980a}. These tournaments pitted submitted
computer strategies against each other in plays of the Iterated Prisoner's
Dilemma. A common game where agents can choose to pay a slight cost to their
immediate utility in the hope of building a reputation. This has been used in
economic and evolutionary game theory to understand the evolution of cooperative
behaviour.

Recently, a class of strategies was described in~\cite{Press2012} that can
provably extort any given opponent. In~\cite{Hilbe2013, Moran1707} some
questions have already been asked about the true effectiveness of these
strategies in an evolutionary setting. Here another question is asked: is it
possible to recognise this extortionate behaviour? A mathematical procedure for
suspicion is presented: in the same way that the continued actions of an
extortionate individual might raise suspicion.

This work makes use of the Axelrod Python library~\cite{Knight2018, Knight2016}
with a large number of Prisoner Dilemma strategies available to give an
extensive numerical example of the ideas presented.  The approach is presented
in Section~\ref{sec:delta-zd-strategies}.  All of the code and data discussed
in Section~\ref{sec:numerical-experiments} is open sourced, archived and
written according to best scientific principles~\cite{Wilson2014}. The data
archive can be found at~\cite{vincent_knight_2018_1297075}.

\section{Recognising Extortion}\label{sec:delta-zd-strategies}

In~\cite{Press2012}, given a match between 2 memory-one strategies, the concept
of Zero Determinant (ZD) strategies is introduced. The main result of that paper
shows that given two memory one players \(p, q\in\mathbb{R}^4\) a linear
relationship between the players' scores could be forced by one of the players.

Using the notation of~\cite{Press2012}, assuming the utilities for player \(p\)
are given by \(S_x=(R, S, T, P)\) and for player \(q\) by \(S_y=(R, T, S, P)\)
and that the stationary scores of each player is given by \(S_X\) and \(S_Y\)
respectively. The main result of~\cite{Press2012} is that if

\begin{equation}\label{eqn:linear_relationship_for_p}
    \tilde p=\alpha S_x + \beta S_y + \gamma
\end{equation}

or

\begin{equation}\label{eqn:linear_relationship_for_q}
    \tilde q=\alpha S_x + \beta S_y + \gamma
\end{equation}

where \(\tilde p = (1 - p_1, 1 - p_2, p_3, p_4)\) and
\(\tilde q = (1 - q_1, 1 - q_2, q_3, q_4)\) then:

\begin{equation}
    \alpha S_X + \beta S_Y + \gamma = 0
\end{equation}

In~\cite{Press2012} a particular type of ZD strategy is defined: extortionate
strategies. If:

\begin{equation}\label{eqn:constraint_for_extortion}
    \gamma = - P(\alpha + \beta)
\end{equation}

then the player can ensure they get a score \(\chi\) times
larger than the opponent. This extortion coefficient is given by:

\begin{equation}\label{eqn:definition_of_chi}
    \chi=\frac{-\beta}{\alpha}
\end{equation}

Thus, if (\ref{eqn:constraint_for_extortion}) holds and \(\chi >1\) a player is
said to extort their opponent.
Here, the reverse problem is considered: given a
\(p\in\mathbb{R}^4\) how does one identify \(\alpha, \beta\) if they
exist and is the strategy in fact acting in an extortionate way?

These conditions correspond to:

\begin{align}
    \tilde p_1 & = \alpha R + \beta R - P (\alpha + \beta)
            \label{eqn:condition_for_tilde_p1}\\
    \tilde p_2 & = \alpha S + \beta T - P (\alpha + \beta)
            \label{eqn:condition_for_tilde_p2}\\
    \tilde p_3 & = \alpha T + \beta S - P (\alpha + \beta)
            \label{eqn:condition_for_tilde_p3}\\
    \tilde p_4 & = \alpha P + \beta P - P (\alpha + \beta)
            \label{eqn:condition_for_tilde_p4}
\end{align}

Equation (\ref{eqn:condition_for_tilde_p4}) ensures that \(p_4=\tilde p_4=0\).
Equations (\ref{eqn:condition_for_tilde_p1}-\ref{eqn:condition_for_tilde_p3})
can be used to eliminate \(\alpha, \beta\), giving:

\begin{equation}\label{eqn:planar_definition_of_extortion}
    \tilde p_1 = \frac{(R - P)(\tilde p_2 + \tilde p_3)}{S + T - 2P}
\end{equation}

with:

\begin{equation}\label{eqn:definition_of_chi}
    \chi = \frac{\tilde p_2 (P - T) + \tilde p_3 (S - P)}
                {\tilde p_2 (P - S) + \tilde p_3 (T - P)}
\end{equation}

Given a strategy \(p\in\mathbb{R}^{4\times 1}\) equations
(\ref{eqn:condition_for_tilde_p4}), (\ref{eqn:planar_definition_of_extortion}-\ref{eqn:definition_of_chi}) can be used to check if
a strategy is extortionate. The conditions correspond to:

\begin{align}
    p_1 & = \frac{(R-P)(p_2 + p_3) - R + T + S - P}{S + T - 2P}
     \label{eqn:condition_for_p1}\\
    p_4 & = 0 \label{eqn:condition_for_p4}\\
    1 & > p_2 + p_3\label{eqn:condition_for_chi}
\end{align}

The algebraic steps necessary to prove these results are available in the
supporting materials.

All extortionate strategies reside on a triangular (\ref{eqn:condition_for_chi})
plane (\ref{eqn:condition_for_p1}) in 3 dimensions (\ref{eqn:condition_for_p4}).
Using this formulation it can be seen that a necessary (but not sufficient)
condition for an extortionate strategy is that it cooperates on average less
than 50\% of the time when in a state of disagreement with the opponent.

As an example, consider the known extortionate strategy \(p=(8 / 9, 1 / 2, 1 /
3, 0)\) from~\cite{Stewart2012} which is referred to as \texttt{Extort-2}. In
this case, for the standard values of \((R, T, S, P)\) constraint
(\ref{eqn:condition_for_p1}) corresponds to:

\begin{equation}
    p_1 = \frac{2(p_2 + p_3) + 1}{3}
\end{equation}

It is clear that in this case all constraints hold.

This approach could in fact be used to confirm that a given strategy is acting
in an extortionate manner even if it is not a memory one strategy. However, in
practice, if a closed form for \(p\) is not known, then due to measurement
and/or numerical error this would not work.

This problem can be written in the following linear algebraic form where
\(x=(\alpha, \beta)\)
and \(p^*=(\tilde p_1 - 1, tilde_2 - 1, p_3)\):

\begin{equation}\label{eqn:linear_algebraic_equation_for_p}
    Cx= p^*
\end{equation}

\(C\) corresponds to equations
(\ref{eqn:condition_for_tilde_p1}-\ref{eqn:condition_for_tilde_p3}) and is
given by:

\begin{equation}\label{eqn:definition_of_C}
    C =
    \begin{bmatrix}
        R - P & R- P \\
        S - P & T- P \\
        T - P & S- P \\
    \end{bmatrix}
\end{equation}

Note that in general, equation (\ref{eqn:linear_algebraic_equation_for_p}) will
not necessarily have a solution. From the Rouch\'{e}-Capelli theorem if there is
a solution it is unique as \(\text{rank}(C)=2\) which is the dimension of the
variable \(x\). The best fitting \(x\) is found by minimizing:

\begin{equation}\label{eqn:r_squared}
    \text{SSError} = \|C x- p^*\|_2^2 = \sum_{i=1}^{3}\left((C\bar x)_i-p_i^*\right)^2
\end{equation}

Note that \(\text{SSError}\), which is the square of the Frobenius
norm~\cite{Golub2013}, becomes a measure of how close a strategy is to being an
extortionate strategy. Suspicion
of extortion then corresponds to a threshold on \(\text{SSError}\).

By observing interactions (human or otherwise), their memory one representation
can be inferred and this approach can be used to recognise extortionate
behaviour. The notion of comparing theoretic and actual plays of the IPD is not
novel, see for example~\cite{Rand2013}. Immediately it is noted that if the
environment is noisy~\cite{Wu1995} then no strategy can be considered to be
extortionate as \(p_4>0\).

In the next section, this idea will be illustrated by observing the interactions
that take place in a computer based tournament of the IPD\@.

\section{Numerical experiments}\label{sec:numerical-experiments}

In~\cite{Stewart2012} results from a tournament with
\input{./assets/tex/number_of_stewart_plotkin_strategies/main.tex} strategies,
was presented with specific consideration given to ZD strategies. This
tournament is reproduced here using the Axelrod-Python
project~\cite{Knight2016}. To obtain a good measure of the corresponding
transition rates for each strategy all matches have been run for
\input{assets/tex/number_of_turns/main.tex} turns and every match has been
repeated \input{assets/tex/number_of_repetitions/main.tex} times. All of this
interaction data is available at~\cite{vincent_knight_2018_1297075}. A good
match between the inferred Markov chain and the state distribution of the actual
interactions has been verified. Data for this is presented in the supplementary
materials.

Figure~\ref{fig:SSError_overall_in_stewart_plotkin} shows the \(\text{SSError}\)
values for all the strategies in the tournament, as reported
in~\cite{Stewart2012} the extortionate strategy (which has an expected
\(\text{SSError}\) approximately 0) gains a large number of wins.

\begin{figure}[!htbp]
    \centering
    \includegraphics[width=.8\textwidth]{./assets/img/SSError_overall_in_stewart_plotkin/main.pdf}
    \caption{\(\text{SSError}\) and state probabilities for the strategies
        of~\cite{Stewart2012}, ordered both by number of wins and overall score.
        Note that \(P(DC)\) is not shown as it corresponds to the transpose of
        \(P(CD)\). Cooperator and Defector are omitted as they do not visit all
        the states.}
    \label{fig:SSError_overall_in_stewart_plotkin}
\end{figure}

Here, the work of~\cite{Stewart2012} is extended by investigating a tournament
with \input{assets/tex/number_of_full_strategies/main.tex}
strategies.

The results of this analysis are shown in
Figure~\ref{fig:SSError_and_probabilities_in_full}. The top ranking strategies
by number of wins seem to be extortionate (but not against all strategies) and
it can be seen that a small sub group of strategies achieve mutual defection.
All the top ranking strategies according to score achieve mutual cooperation and
do not extort each other, however they
\textbf{do} exhibit extortionate behaviour towards a number of the lower ranking
strategies.

\begin{figure}[!htbp]
    \centering
    \includegraphics[width=.8\textwidth]{./assets/img/SSError_and_probabilities_in_full/main.pdf}
    \caption{\(\text{SSError}\) for the strategies for the full tournament. Only
    strategy interactions for which \(p_4=0\) and \(\chi>1\) are displayed.}
    \label{fig:SSError_and_probabilities_in_full}
\end{figure}

\section{Conclusion}\label{sec:conclusion}

This work defines an approach to measure whether or not a player is playing a
strategy that corresponds to an extortionate strategy as defined
in~\cite{Press2012}: a mathematical model for suspicion. Indeed, all
extortionate strategies have been
 classified as lying on a triangular plane.
This rigorous classification fails to be robust to small measurement error, thus
a statistical approach is proposed.
This is done through a linear algebraic approach for approximating the solution
of a linear system. Using this, a large number of pairwise interactions is
simulated and in fact very few strategies are found to act extortionately.

The work of~\cite{Press2012}, whilst showing that a clever approach to taking
advantage of another memory one strategy exists: this is incomplete. Whilst the
elegance of this result is very attractive, just as the simplicity of the
victory of Tit For Tat in Axelrod's original tournaments was, it is incomplete.
Extortionate strategies achieve a high number of wins but they do not
achieve a high score which corresponds to the fitness landscape in an
evolutionary sense. From the large number of interactions a payoff matrix \(S\)
can be measured where \(S_{ij}\) denotes the score (using standard values of
\((R, S, T, P) = (3, 0, 5, 1)\)) of the \(i\)th strategy
against the \(j\)th strategy. Using this, the replicator equation
describes the evolution of the system based on a population density fitness
function:

\begin{equation}\label{eqn:replicator_dynamics}
    \frac{dx}{dt} = x(S-x^TS x)
\end{equation}

Equation (\ref{eqn:replicator_dynamics}) is solved numerically through an
integration technique described in~\cite{Petzold1983} and
Figure~\ref{fig:replicator_dynamics} shows the evolution of the distribution of
the system: the various strategies are ranked by scores. It is clear to see that
only the high ranking strategies survive the evolutionary process (in fact,
only \input{./assets/img/replicator_dynamics/main.tex}
have a final distribution greater than \(10 ^ {-2}\)). This confirms the
findings of~\cite{Moran1707} in which sophisticated strategies resist
evolutionary invasion of shorter memory strategies. Recalling
Figure~\ref{fig:SSError_and_probabilities_in_full} this demonstrates that:

\begin{itemize}
    \item Cooperation emerges through the evolutionary process: the high scoring
        strategies do not exhibit extortionate behaviour towards each other.
    \item Extortionate strategies do not survive the evolutionary process.
\end{itemize}

\begin{figure}[!htbp]
    \centering
    \includegraphics[width=.8\textwidth]{./assets/img/replicator_dynamics/main.pdf}
    \caption{Numerical simulation of the replicator equation
    (\ref{eqn:replicator_dynamics}): strategies are ordered by score, only the strategies with a high score survive the evolutionary process.}
    \label{fig:replicator_dynamics}
\end{figure}

This work can be used to classify plays of the IPD\@: data can be collected from
actual interactions (in lab or in the field). Furthermore, this allows for a
classification method similar to the notion of fingerprinting presented
in~\cite{Ashlock2008}. Trained strategies can potentially be classified as
extortionate or not or it could be possible to even constrain the reinforcement
learning approaches that are becoming prevalent in the literature.
Alternatively, this mathematical approach for recognising extortion could be
used in sophisticated strategies to defend against invasion. Arguably, some of
the strategies considered here exhibit this behaviour, indeed as described
in~\cite{Harper2017}, the top ranking strategies in the full tournament are
obtained using evolutionary reinforcement learning techniques, thus, suspicion
of extortionate behaviour could in fact be an evolutionary trait.

\section*{Acknowledgements}

The following open source software libraries were used in this research:

\begin{itemize}
    \item The Axelrod ~\cite{Knight2016, Knight2018} library (IPD strategies and
        tournaments).
    \item The sympy library~\cite{Meurer2017} (verification of all symbolic
        calculations).
    \item The matplotlib~\cite{Droettboom2018} library (visualisation).
    \item The pandas~\cite{Structures2010}, dask~\cite{Dask2016} and
        NumPy~\cite{Oliphant2015} libraries (data manipulation).
    \item The SciPy~\cite{Jones2001} library (numerical integration of the
        replicator equation).
\end{itemize}

This work was performed using the computational facilities of the Advanced
Research Computing @ Cardiff (ARCCA) Division, Cardiff University.

\printbibliography

\newpage
\section*{Supplementary materials}

\includepdf{assets/pdf/proof_of_form_of_extortionate_strategies/main.pdf}

\newpage

Using the pair wise interactions the transition rates \(p,
q\) can be measured and the steady state probabilities inferred and compared to
the actual probabilities of each state.
This is done numerically by computing the singular eigenvector of the
matrix \(A\) \cite{Stewart2009}:

\[
    A =
    \begin{bmatrix}
        p_1 q_1 & p_1 (1 - q_1) & (1 - p_1) q_1 & (1 -p_1) (1 - q_1) \\
        p_2 q_2 & p_2 (1 - q_2) & (1 - p_2) q_2 & (1 -p_2) (1 - q_2) \\
        p_3 q_3 & p_3 (1 - q_3) & (1 - p_3) q_3 & (1 -p_3) (1 - q_3) \\
        p_4 q_4 & p_4 (1 - q_4) & (1 - p_4) q_4 & (1 -p_4) (1 - q_4) \\
    \end{bmatrix}
\]

Figure~\ref{fig:computed_probabilities_vs_theoretic_probabilities} shows a
regression line fitted to every pairwise interaction with a reported
\(\text{SSError}\) value (pairwise interactions with missing states were
omitted). This serves to validate the approach: a part from some edge cases the
relationship is consistent.

\begin{figure}[!htbp]
    \centering
    \includegraphics[width=.8\textwidth]{./assets/img/computed_probabilities_vs_theoretic_probabilities/main.pdf}
    \caption{The
        relationship between the steady state probabilities inferred from the
        measured transitions and the actual steady state probabilities. A linear
        regression line is included validating the approach.}
    \label{fig:computed_probabilities_vs_theoretic_probabilities}
\end{figure}


\end{document}

    strategies is considered. In this setting
    the most highly performing strategies do not play in an extortionate way
    against each other but do against lower performing strategies.
    This suggests that whilst the theory of Zero Determinant strategies
    indicates that memory is not of fundamental importance to the evolution of
    cooperative behaviour, this is incomplete.
\end{abstract}

\section{Introduction}\label{sec:introduction}

Agent based game theoretic models have become a stalwart of the underpinning
mathematics of interactive behaviours. One of the major pieces of work
in this area is the pair of original computer tournaments run by Robert
Axelrod~\cite{Axelrod1980, Axelrod1980a}. These tournaments pitted submitted
computer strategies against each other in plays of the Iterated Prisoner's
Dilemma. A common game where agents can choose to pay a slight cost to their
immediate utility in the hope of building a reputation. This has been used in
economic and evolutionary game theory to understand the evolution of cooperative
behaviour.

Recently, a class of strategies was described in~\cite{Press2012} that can
provably extort any given opponent. In~\cite{Hilbe2013, Moran1707} some
questions have already been asked about the true effectiveness of these
strategies in an evolutionary setting. Here another question is asked: is it
possible to recognise this extortionate behaviour? A mathematical procedure for
suspicion is presented: in the same way that the continued actions of an
extortionate individual might raise suspicion.

This work makes use of the Axelrod Python library~\cite{Knight2018, Knight2016}
with a large number of Prisoner Dilemma strategies available to give an
extensive numerical example of the ideas presented.  The approach is presented
in Section~\ref{sec:delta-zd-strategies}.  All of the code and data discussed
in Section~\ref{sec:numerical-experiments} is open sourced, archived and
written according to best scientific principles~\cite{Wilson2014}. The data
archive can be found at~\cite{vincent_knight_2018_1297075}.

\section{Recognising Extortion}\label{sec:delta-zd-strategies}

In~\cite{Press2012}, given a match between 2 memory-one strategies, the concept
of Zero Determinant (ZD) strategies is introduced. The main result of that paper
shows that given two memory one players \(p, q\in\mathbb{R}^4\) a linear
relationship between the players' scores could be forced by one of the players.

Using the notation of~\cite{Press2012}, assuming the utilities for player \(p\)
are given by \(S_x=(R, S, T, P)\) and for player \(q\) by \(S_y=(R, T, S, P)\)
and that the stationary scores of each player is given by \(S_X\) and \(S_Y\)
respectively. The main result of~\cite{Press2012} is that if

\begin{equation}\label{eqn:linear_relationship_for_p}
    \tilde p=\alpha S_x + \beta S_y + \gamma
\end{equation}

or

\begin{equation}\label{eqn:linear_relationship_for_q}
    \tilde q=\alpha S_x + \beta S_y + \gamma
\end{equation}

where \(\tilde p = (1 - p_1, 1 - p_2, p_3, p_4)\) and
\(\tilde q = (1 - q_1, 1 - q_2, q_3, q_4)\) then:

\begin{equation}
    \alpha S_X + \beta S_Y + \gamma = 0
\end{equation}

In~\cite{Press2012} a particular type of ZD strategy is defined: extortionate
strategies. If:

\begin{equation}\label{eqn:constraint_for_extortion}
    \gamma = - P(\alpha + \beta)
\end{equation}

then the player can ensure they get a score \(\chi\) times
larger than the opponent. This extortion coefficient is given by:

\begin{equation}\label{eqn:definition_of_chi}
    \chi=\frac{-\beta}{\alpha}
\end{equation}

Thus, if (\ref{eqn:constraint_for_extortion}) holds and \(\chi >1\) a player is
said to extort their opponent.
Here, the reverse problem is considered: given a
\(p\in\mathbb{R}^4\) how does one identify \(\alpha, \beta\) if they
exist and is the strategy in fact acting in an extortionate way?

These conditions correspond to:

\begin{align}
    \tilde p_1 & = \alpha R + \beta R - P (\alpha + \beta)
            \label{eqn:condition_for_tilde_p1}\\
    \tilde p_2 & = \alpha S + \beta T - P (\alpha + \beta)
            \label{eqn:condition_for_tilde_p2}\\
    \tilde p_3 & = \alpha T + \beta S - P (\alpha + \beta)
            \label{eqn:condition_for_tilde_p3}\\
    \tilde p_4 & = \alpha P + \beta P - P (\alpha + \beta)
            \label{eqn:condition_for_tilde_p4}
\end{align}

Equation (\ref{eqn:condition_for_tilde_p4}) ensures that \(p_4=\tilde p_4=0\).
Equations (\ref{eqn:condition_for_tilde_p1}-\ref{eqn:condition_for_tilde_p3})
can be used to eliminate \(\alpha, \beta\), giving:

\begin{equation}\label{eqn:planar_definition_of_extortion}
    \tilde p_1 = \frac{(R - P)(\tilde p_2 + \tilde p_3)}{S + T - 2P}
\end{equation}

with:

\begin{equation}\label{eqn:definition_of_chi}
    \chi = \frac{\tilde p_2 (P - T) + \tilde p_3 (S - P)}
                {\tilde p_2 (P - S) + \tilde p_3 (T - P)}
\end{equation}

Given a strategy \(p\in\mathbb{R}^{4\times 1}\) equations
(\ref{eqn:condition_for_tilde_p4}), (\ref{eqn:planar_definition_of_extortion}-\ref{eqn:definition_of_chi}) can be used to check if
a strategy is extortionate. The conditions correspond to:

\begin{align}
    p_1 & = \frac{(R-P)(p_2 + p_3) - R + T + S - P}{S + T - 2P}
     \label{eqn:condition_for_p1}\\
    p_4 & = 0 \label{eqn:condition_for_p4}\\
    1 & > p_2 + p_3\label{eqn:condition_for_chi}
\end{align}

The algebraic steps necessary to prove these results are available in the
supporting materials.

All extortionate strategies reside on a triangular (\ref{eqn:condition_for_chi})
plane (\ref{eqn:condition_for_p1}) in 3 dimensions (\ref{eqn:condition_for_p4}).
Using this formulation it can be seen that a necessary (but not sufficient)
condition for an extortionate strategy is that it cooperates on average less
than 50\% of the time when in a state of disagreement with the opponent.

As an example, consider the known extortionate strategy \(p=(8 / 9, 1 / 2, 1 /
3, 0)\) from~\cite{Stewart2012} which is referred to as \texttt{Extort-2}. In
this case, for the standard values of \((R, T, S, P)\) constraint
(\ref{eqn:condition_for_p1}) corresponds to:

\begin{equation}
    p_1 = \frac{2(p_2 + p_3) + 1}{3}
\end{equation}

It is clear that in this case all constraints hold.

This approach could in fact be used to confirm that a given strategy is acting
in an extortionate manner even if it is not a memory one strategy. However, in
practice, if a closed form for \(p\) is not known, then due to measurement
and/or numerical error this would not work.

This problem can be written in the following linear algebraic form where
\(x=(\alpha, \beta)\)
and \(p^*=(\tilde p_1 - 1, tilde_2 - 1, p_3)\):

\begin{equation}\label{eqn:linear_algebraic_equation_for_p}
    Cx= p^*
\end{equation}

\(C\) corresponds to equations
(\ref{eqn:condition_for_tilde_p1}-\ref{eqn:condition_for_tilde_p3}) and is
given by:

\begin{equation}\label{eqn:definition_of_C}
    C =
    \begin{bmatrix}
        R - P & R- P \\
        S - P & T- P \\
        T - P & S- P \\
    \end{bmatrix}
\end{equation}

Note that in general, equation (\ref{eqn:linear_algebraic_equation_for_p}) will
not necessarily have a solution. From the Rouch\'{e}-Capelli theorem if there is
a solution it is unique as \(\text{rank}(C)=2\) which is the dimension of the
variable \(x\). The best fitting \(x\) is found by minimizing:

\begin{equation}\label{eqn:r_squared}
    \text{SSError} = \|C x- p^*\|_2^2 = \sum_{i=1}^{3}\left((C\bar x)_i-p_i^*\right)^2
\end{equation}

Note that \(\text{SSError}\), which is the square of the Frobenius
norm~\cite{Golub2013}, becomes a measure of how close a strategy is to being an
extortionate strategy. Suspicion
of extortion then corresponds to a threshold on \(\text{SSError}\).

By observing interactions (human or otherwise), their memory one representation
can be inferred and this approach can be used to recognise extortionate
behaviour. The notion of comparing theoretic and actual plays of the IPD is not
novel, see for example~\cite{Rand2013}. Immediately it is noted that if the
environment is noisy~\cite{Wu1995} then no strategy can be considered to be
extortionate as \(p_4>0\).

In the next section, this idea will be illustrated by observing the interactions
that take place in a computer based tournament of the IPD\@.

\section{Numerical experiments}\label{sec:numerical-experiments}

In~\cite{Stewart2012} results from a tournament with
\documentclass[a4paper]{article}

\usepackage{amsmath}
\usepackage{amssymb}
\usepackage[margin=1.5cm,
            includefoot,
            footskip=30pt]{geometry}
\usepackage{layout}
\usepackage{graphicx}
\usepackage{subcaption}

\usepackage{biblatex}
\usepackage{pdfpages}

\bibliography{main.bib}

\title{Suspicion: Recognising and evaluating the effectiveness
       of extortion in the Iterated Prisoner's Dilemma}
\author{Vincent A. Knight \and Nikoleta E. Glynatsi}
\date{\today}



\begin{document}

\maketitle

\begin{abstract}
    The Iterated Prisoner's Dilemma is a model for rational and evolutionary
    interactive behaviour. It has applications both in the study of human social
    behaviour as well as in biology.
    It is used to understand when and how a rational individual might
    accept an immediate cost to their own utility for the direct benefit of
    another.

    Much attention has been given to a class of strategies called
    Zero Determinant strategies. It has been theoretically shown that these
    strategies can ``extort'' any player.

    In this work, an approach to identify if observed strategies are playing in
    an extortionate way is described. Furthermore, experimental analysis of
    a large tournament with \input{assets/tex/number_of_full_strategies/main.tex}
    strategies is considered. In this setting
    the most highly performing strategies do not play in an extortionate way
    against each other but do against lower performing strategies.
    This suggests that whilst the theory of Zero Determinant strategies
    indicates that memory is not of fundamental importance to the evolution of
    cooperative behaviour, this is incomplete.
\end{abstract}

\section{Introduction}\label{sec:introduction}

Agent based game theoretic models have become a stalwart of the underpinning
mathematics of interactive behaviours. One of the major pieces of work
in this area is the pair of original computer tournaments run by Robert
Axelrod~\cite{Axelrod1980, Axelrod1980a}. These tournaments pitted submitted
computer strategies against each other in plays of the Iterated Prisoner's
Dilemma. A common game where agents can choose to pay a slight cost to their
immediate utility in the hope of building a reputation. This has been used in
economic and evolutionary game theory to understand the evolution of cooperative
behaviour.

Recently, a class of strategies was described in~\cite{Press2012} that can
provably extort any given opponent. In~\cite{Hilbe2013, Moran1707} some
questions have already been asked about the true effectiveness of these
strategies in an evolutionary setting. Here another question is asked: is it
possible to recognise this extortionate behaviour? A mathematical procedure for
suspicion is presented: in the same way that the continued actions of an
extortionate individual might raise suspicion.

This work makes use of the Axelrod Python library~\cite{Knight2018, Knight2016}
with a large number of Prisoner Dilemma strategies available to give an
extensive numerical example of the ideas presented.  The approach is presented
in Section~\ref{sec:delta-zd-strategies}.  All of the code and data discussed
in Section~\ref{sec:numerical-experiments} is open sourced, archived and
written according to best scientific principles~\cite{Wilson2014}. The data
archive can be found at~\cite{vincent_knight_2018_1297075}.

\section{Recognising Extortion}\label{sec:delta-zd-strategies}

In~\cite{Press2012}, given a match between 2 memory-one strategies, the concept
of Zero Determinant (ZD) strategies is introduced. The main result of that paper
shows that given two memory one players \(p, q\in\mathbb{R}^4\) a linear
relationship between the players' scores could be forced by one of the players.

Using the notation of~\cite{Press2012}, assuming the utilities for player \(p\)
are given by \(S_x=(R, S, T, P)\) and for player \(q\) by \(S_y=(R, T, S, P)\)
and that the stationary scores of each player is given by \(S_X\) and \(S_Y\)
respectively. The main result of~\cite{Press2012} is that if

\begin{equation}\label{eqn:linear_relationship_for_p}
    \tilde p=\alpha S_x + \beta S_y + \gamma
\end{equation}

or

\begin{equation}\label{eqn:linear_relationship_for_q}
    \tilde q=\alpha S_x + \beta S_y + \gamma
\end{equation}

where \(\tilde p = (1 - p_1, 1 - p_2, p_3, p_4)\) and
\(\tilde q = (1 - q_1, 1 - q_2, q_3, q_4)\) then:

\begin{equation}
    \alpha S_X + \beta S_Y + \gamma = 0
\end{equation}

In~\cite{Press2012} a particular type of ZD strategy is defined: extortionate
strategies. If:

\begin{equation}\label{eqn:constraint_for_extortion}
    \gamma = - P(\alpha + \beta)
\end{equation}

then the player can ensure they get a score \(\chi\) times
larger than the opponent. This extortion coefficient is given by:

\begin{equation}\label{eqn:definition_of_chi}
    \chi=\frac{-\beta}{\alpha}
\end{equation}

Thus, if (\ref{eqn:constraint_for_extortion}) holds and \(\chi >1\) a player is
said to extort their opponent.
Here, the reverse problem is considered: given a
\(p\in\mathbb{R}^4\) how does one identify \(\alpha, \beta\) if they
exist and is the strategy in fact acting in an extortionate way?

These conditions correspond to:

\begin{align}
    \tilde p_1 & = \alpha R + \beta R - P (\alpha + \beta)
            \label{eqn:condition_for_tilde_p1}\\
    \tilde p_2 & = \alpha S + \beta T - P (\alpha + \beta)
            \label{eqn:condition_for_tilde_p2}\\
    \tilde p_3 & = \alpha T + \beta S - P (\alpha + \beta)
            \label{eqn:condition_for_tilde_p3}\\
    \tilde p_4 & = \alpha P + \beta P - P (\alpha + \beta)
            \label{eqn:condition_for_tilde_p4}
\end{align}

Equation (\ref{eqn:condition_for_tilde_p4}) ensures that \(p_4=\tilde p_4=0\).
Equations (\ref{eqn:condition_for_tilde_p1}-\ref{eqn:condition_for_tilde_p3})
can be used to eliminate \(\alpha, \beta\), giving:

\begin{equation}\label{eqn:planar_definition_of_extortion}
    \tilde p_1 = \frac{(R - P)(\tilde p_2 + \tilde p_3)}{S + T - 2P}
\end{equation}

with:

\begin{equation}\label{eqn:definition_of_chi}
    \chi = \frac{\tilde p_2 (P - T) + \tilde p_3 (S - P)}
                {\tilde p_2 (P - S) + \tilde p_3 (T - P)}
\end{equation}

Given a strategy \(p\in\mathbb{R}^{4\times 1}\) equations
(\ref{eqn:condition_for_tilde_p4}), (\ref{eqn:planar_definition_of_extortion}-\ref{eqn:definition_of_chi}) can be used to check if
a strategy is extortionate. The conditions correspond to:

\begin{align}
    p_1 & = \frac{(R-P)(p_2 + p_3) - R + T + S - P}{S + T - 2P}
     \label{eqn:condition_for_p1}\\
    p_4 & = 0 \label{eqn:condition_for_p4}\\
    1 & > p_2 + p_3\label{eqn:condition_for_chi}
\end{align}

The algebraic steps necessary to prove these results are available in the
supporting materials.

All extortionate strategies reside on a triangular (\ref{eqn:condition_for_chi})
plane (\ref{eqn:condition_for_p1}) in 3 dimensions (\ref{eqn:condition_for_p4}).
Using this formulation it can be seen that a necessary (but not sufficient)
condition for an extortionate strategy is that it cooperates on average less
than 50\% of the time when in a state of disagreement with the opponent.

As an example, consider the known extortionate strategy \(p=(8 / 9, 1 / 2, 1 /
3, 0)\) from~\cite{Stewart2012} which is referred to as \texttt{Extort-2}. In
this case, for the standard values of \((R, T, S, P)\) constraint
(\ref{eqn:condition_for_p1}) corresponds to:

\begin{equation}
    p_1 = \frac{2(p_2 + p_3) + 1}{3}
\end{equation}

It is clear that in this case all constraints hold.

This approach could in fact be used to confirm that a given strategy is acting
in an extortionate manner even if it is not a memory one strategy. However, in
practice, if a closed form for \(p\) is not known, then due to measurement
and/or numerical error this would not work.

This problem can be written in the following linear algebraic form where
\(x=(\alpha, \beta)\)
and \(p^*=(\tilde p_1 - 1, tilde_2 - 1, p_3)\):

\begin{equation}\label{eqn:linear_algebraic_equation_for_p}
    Cx= p^*
\end{equation}

\(C\) corresponds to equations
(\ref{eqn:condition_for_tilde_p1}-\ref{eqn:condition_for_tilde_p3}) and is
given by:

\begin{equation}\label{eqn:definition_of_C}
    C =
    \begin{bmatrix}
        R - P & R- P \\
        S - P & T- P \\
        T - P & S- P \\
    \end{bmatrix}
\end{equation}

Note that in general, equation (\ref{eqn:linear_algebraic_equation_for_p}) will
not necessarily have a solution. From the Rouch\'{e}-Capelli theorem if there is
a solution it is unique as \(\text{rank}(C)=2\) which is the dimension of the
variable \(x\). The best fitting \(x\) is found by minimizing:

\begin{equation}\label{eqn:r_squared}
    \text{SSError} = \|C x- p^*\|_2^2 = \sum_{i=1}^{3}\left((C\bar x)_i-p_i^*\right)^2
\end{equation}

Note that \(\text{SSError}\), which is the square of the Frobenius
norm~\cite{Golub2013}, becomes a measure of how close a strategy is to being an
extortionate strategy. Suspicion
of extortion then corresponds to a threshold on \(\text{SSError}\).

By observing interactions (human or otherwise), their memory one representation
can be inferred and this approach can be used to recognise extortionate
behaviour. The notion of comparing theoretic and actual plays of the IPD is not
novel, see for example~\cite{Rand2013}. Immediately it is noted that if the
environment is noisy~\cite{Wu1995} then no strategy can be considered to be
extortionate as \(p_4>0\).

In the next section, this idea will be illustrated by observing the interactions
that take place in a computer based tournament of the IPD\@.

\section{Numerical experiments}\label{sec:numerical-experiments}

In~\cite{Stewart2012} results from a tournament with
\input{./assets/tex/number_of_stewart_plotkin_strategies/main.tex} strategies,
was presented with specific consideration given to ZD strategies. This
tournament is reproduced here using the Axelrod-Python
project~\cite{Knight2016}. To obtain a good measure of the corresponding
transition rates for each strategy all matches have been run for
\input{assets/tex/number_of_turns/main.tex} turns and every match has been
repeated \input{assets/tex/number_of_repetitions/main.tex} times. All of this
interaction data is available at~\cite{vincent_knight_2018_1297075}. A good
match between the inferred Markov chain and the state distribution of the actual
interactions has been verified. Data for this is presented in the supplementary
materials.

Figure~\ref{fig:SSError_overall_in_stewart_plotkin} shows the \(\text{SSError}\)
values for all the strategies in the tournament, as reported
in~\cite{Stewart2012} the extortionate strategy (which has an expected
\(\text{SSError}\) approximately 0) gains a large number of wins.

\begin{figure}[!htbp]
    \centering
    \includegraphics[width=.8\textwidth]{./assets/img/SSError_overall_in_stewart_plotkin/main.pdf}
    \caption{\(\text{SSError}\) and state probabilities for the strategies
        of~\cite{Stewart2012}, ordered both by number of wins and overall score.
        Note that \(P(DC)\) is not shown as it corresponds to the transpose of
        \(P(CD)\). Cooperator and Defector are omitted as they do not visit all
        the states.}
    \label{fig:SSError_overall_in_stewart_plotkin}
\end{figure}

Here, the work of~\cite{Stewart2012} is extended by investigating a tournament
with \input{assets/tex/number_of_full_strategies/main.tex}
strategies.

The results of this analysis are shown in
Figure~\ref{fig:SSError_and_probabilities_in_full}. The top ranking strategies
by number of wins seem to be extortionate (but not against all strategies) and
it can be seen that a small sub group of strategies achieve mutual defection.
All the top ranking strategies according to score achieve mutual cooperation and
do not extort each other, however they
\textbf{do} exhibit extortionate behaviour towards a number of the lower ranking
strategies.

\begin{figure}[!htbp]
    \centering
    \includegraphics[width=.8\textwidth]{./assets/img/SSError_and_probabilities_in_full/main.pdf}
    \caption{\(\text{SSError}\) for the strategies for the full tournament. Only
    strategy interactions for which \(p_4=0\) and \(\chi>1\) are displayed.}
    \label{fig:SSError_and_probabilities_in_full}
\end{figure}

\section{Conclusion}\label{sec:conclusion}

This work defines an approach to measure whether or not a player is playing a
strategy that corresponds to an extortionate strategy as defined
in~\cite{Press2012}: a mathematical model for suspicion. Indeed, all
extortionate strategies have been
 classified as lying on a triangular plane.
This rigorous classification fails to be robust to small measurement error, thus
a statistical approach is proposed.
This is done through a linear algebraic approach for approximating the solution
of a linear system. Using this, a large number of pairwise interactions is
simulated and in fact very few strategies are found to act extortionately.

The work of~\cite{Press2012}, whilst showing that a clever approach to taking
advantage of another memory one strategy exists: this is incomplete. Whilst the
elegance of this result is very attractive, just as the simplicity of the
victory of Tit For Tat in Axelrod's original tournaments was, it is incomplete.
Extortionate strategies achieve a high number of wins but they do not
achieve a high score which corresponds to the fitness landscape in an
evolutionary sense. From the large number of interactions a payoff matrix \(S\)
can be measured where \(S_{ij}\) denotes the score (using standard values of
\((R, S, T, P) = (3, 0, 5, 1)\)) of the \(i\)th strategy
against the \(j\)th strategy. Using this, the replicator equation
describes the evolution of the system based on a population density fitness
function:

\begin{equation}\label{eqn:replicator_dynamics}
    \frac{dx}{dt} = x(S-x^TS x)
\end{equation}

Equation (\ref{eqn:replicator_dynamics}) is solved numerically through an
integration technique described in~\cite{Petzold1983} and
Figure~\ref{fig:replicator_dynamics} shows the evolution of the distribution of
the system: the various strategies are ranked by scores. It is clear to see that
only the high ranking strategies survive the evolutionary process (in fact,
only \input{./assets/img/replicator_dynamics/main.tex}
have a final distribution greater than \(10 ^ {-2}\)). This confirms the
findings of~\cite{Moran1707} in which sophisticated strategies resist
evolutionary invasion of shorter memory strategies. Recalling
Figure~\ref{fig:SSError_and_probabilities_in_full} this demonstrates that:

\begin{itemize}
    \item Cooperation emerges through the evolutionary process: the high scoring
        strategies do not exhibit extortionate behaviour towards each other.
    \item Extortionate strategies do not survive the evolutionary process.
\end{itemize}

\begin{figure}[!htbp]
    \centering
    \includegraphics[width=.8\textwidth]{./assets/img/replicator_dynamics/main.pdf}
    \caption{Numerical simulation of the replicator equation
    (\ref{eqn:replicator_dynamics}): strategies are ordered by score, only the strategies with a high score survive the evolutionary process.}
    \label{fig:replicator_dynamics}
\end{figure}

This work can be used to classify plays of the IPD\@: data can be collected from
actual interactions (in lab or in the field). Furthermore, this allows for a
classification method similar to the notion of fingerprinting presented
in~\cite{Ashlock2008}. Trained strategies can potentially be classified as
extortionate or not or it could be possible to even constrain the reinforcement
learning approaches that are becoming prevalent in the literature.
Alternatively, this mathematical approach for recognising extortion could be
used in sophisticated strategies to defend against invasion. Arguably, some of
the strategies considered here exhibit this behaviour, indeed as described
in~\cite{Harper2017}, the top ranking strategies in the full tournament are
obtained using evolutionary reinforcement learning techniques, thus, suspicion
of extortionate behaviour could in fact be an evolutionary trait.

\section*{Acknowledgements}

The following open source software libraries were used in this research:

\begin{itemize}
    \item The Axelrod ~\cite{Knight2016, Knight2018} library (IPD strategies and
        tournaments).
    \item The sympy library~\cite{Meurer2017} (verification of all symbolic
        calculations).
    \item The matplotlib~\cite{Droettboom2018} library (visualisation).
    \item The pandas~\cite{Structures2010}, dask~\cite{Dask2016} and
        NumPy~\cite{Oliphant2015} libraries (data manipulation).
    \item The SciPy~\cite{Jones2001} library (numerical integration of the
        replicator equation).
\end{itemize}

This work was performed using the computational facilities of the Advanced
Research Computing @ Cardiff (ARCCA) Division, Cardiff University.

\printbibliography

\newpage
\section*{Supplementary materials}

\includepdf{assets/pdf/proof_of_form_of_extortionate_strategies/main.pdf}

\newpage

Using the pair wise interactions the transition rates \(p,
q\) can be measured and the steady state probabilities inferred and compared to
the actual probabilities of each state.
This is done numerically by computing the singular eigenvector of the
matrix \(A\) \cite{Stewart2009}:

\[
    A =
    \begin{bmatrix}
        p_1 q_1 & p_1 (1 - q_1) & (1 - p_1) q_1 & (1 -p_1) (1 - q_1) \\
        p_2 q_2 & p_2 (1 - q_2) & (1 - p_2) q_2 & (1 -p_2) (1 - q_2) \\
        p_3 q_3 & p_3 (1 - q_3) & (1 - p_3) q_3 & (1 -p_3) (1 - q_3) \\
        p_4 q_4 & p_4 (1 - q_4) & (1 - p_4) q_4 & (1 -p_4) (1 - q_4) \\
    \end{bmatrix}
\]

Figure~\ref{fig:computed_probabilities_vs_theoretic_probabilities} shows a
regression line fitted to every pairwise interaction with a reported
\(\text{SSError}\) value (pairwise interactions with missing states were
omitted). This serves to validate the approach: a part from some edge cases the
relationship is consistent.

\begin{figure}[!htbp]
    \centering
    \includegraphics[width=.8\textwidth]{./assets/img/computed_probabilities_vs_theoretic_probabilities/main.pdf}
    \caption{The
        relationship between the steady state probabilities inferred from the
        measured transitions and the actual steady state probabilities. A linear
        regression line is included validating the approach.}
    \label{fig:computed_probabilities_vs_theoretic_probabilities}
\end{figure}


\end{document}
 strategies,
was presented with specific consideration given to ZD strategies. This
tournament is reproduced here using the Axelrod-Python
project~\cite{Knight2016}. To obtain a good measure of the corresponding
transition rates for each strategy all matches have been run for
\documentclass[a4paper]{article}

\usepackage{amsmath}
\usepackage{amssymb}
\usepackage[margin=1.5cm,
            includefoot,
            footskip=30pt]{geometry}
\usepackage{layout}
\usepackage{graphicx}
\usepackage{subcaption}

\usepackage{biblatex}
\usepackage{pdfpages}

\bibliography{main.bib}

\title{Suspicion: Recognising and evaluating the effectiveness
       of extortion in the Iterated Prisoner's Dilemma}
\author{Vincent A. Knight \and Nikoleta E. Glynatsi}
\date{\today}



\begin{document}

\maketitle

\begin{abstract}
    The Iterated Prisoner's Dilemma is a model for rational and evolutionary
    interactive behaviour. It has applications both in the study of human social
    behaviour as well as in biology.
    It is used to understand when and how a rational individual might
    accept an immediate cost to their own utility for the direct benefit of
    another.

    Much attention has been given to a class of strategies called
    Zero Determinant strategies. It has been theoretically shown that these
    strategies can ``extort'' any player.

    In this work, an approach to identify if observed strategies are playing in
    an extortionate way is described. Furthermore, experimental analysis of
    a large tournament with \input{assets/tex/number_of_full_strategies/main.tex}
    strategies is considered. In this setting
    the most highly performing strategies do not play in an extortionate way
    against each other but do against lower performing strategies.
    This suggests that whilst the theory of Zero Determinant strategies
    indicates that memory is not of fundamental importance to the evolution of
    cooperative behaviour, this is incomplete.
\end{abstract}

\section{Introduction}\label{sec:introduction}

Agent based game theoretic models have become a stalwart of the underpinning
mathematics of interactive behaviours. One of the major pieces of work
in this area is the pair of original computer tournaments run by Robert
Axelrod~\cite{Axelrod1980, Axelrod1980a}. These tournaments pitted submitted
computer strategies against each other in plays of the Iterated Prisoner's
Dilemma. A common game where agents can choose to pay a slight cost to their
immediate utility in the hope of building a reputation. This has been used in
economic and evolutionary game theory to understand the evolution of cooperative
behaviour.

Recently, a class of strategies was described in~\cite{Press2012} that can
provably extort any given opponent. In~\cite{Hilbe2013, Moran1707} some
questions have already been asked about the true effectiveness of these
strategies in an evolutionary setting. Here another question is asked: is it
possible to recognise this extortionate behaviour? A mathematical procedure for
suspicion is presented: in the same way that the continued actions of an
extortionate individual might raise suspicion.

This work makes use of the Axelrod Python library~\cite{Knight2018, Knight2016}
with a large number of Prisoner Dilemma strategies available to give an
extensive numerical example of the ideas presented.  The approach is presented
in Section~\ref{sec:delta-zd-strategies}.  All of the code and data discussed
in Section~\ref{sec:numerical-experiments} is open sourced, archived and
written according to best scientific principles~\cite{Wilson2014}. The data
archive can be found at~\cite{vincent_knight_2018_1297075}.

\section{Recognising Extortion}\label{sec:delta-zd-strategies}

In~\cite{Press2012}, given a match between 2 memory-one strategies, the concept
of Zero Determinant (ZD) strategies is introduced. The main result of that paper
shows that given two memory one players \(p, q\in\mathbb{R}^4\) a linear
relationship between the players' scores could be forced by one of the players.

Using the notation of~\cite{Press2012}, assuming the utilities for player \(p\)
are given by \(S_x=(R, S, T, P)\) and for player \(q\) by \(S_y=(R, T, S, P)\)
and that the stationary scores of each player is given by \(S_X\) and \(S_Y\)
respectively. The main result of~\cite{Press2012} is that if

\begin{equation}\label{eqn:linear_relationship_for_p}
    \tilde p=\alpha S_x + \beta S_y + \gamma
\end{equation}

or

\begin{equation}\label{eqn:linear_relationship_for_q}
    \tilde q=\alpha S_x + \beta S_y + \gamma
\end{equation}

where \(\tilde p = (1 - p_1, 1 - p_2, p_3, p_4)\) and
\(\tilde q = (1 - q_1, 1 - q_2, q_3, q_4)\) then:

\begin{equation}
    \alpha S_X + \beta S_Y + \gamma = 0
\end{equation}

In~\cite{Press2012} a particular type of ZD strategy is defined: extortionate
strategies. If:

\begin{equation}\label{eqn:constraint_for_extortion}
    \gamma = - P(\alpha + \beta)
\end{equation}

then the player can ensure they get a score \(\chi\) times
larger than the opponent. This extortion coefficient is given by:

\begin{equation}\label{eqn:definition_of_chi}
    \chi=\frac{-\beta}{\alpha}
\end{equation}

Thus, if (\ref{eqn:constraint_for_extortion}) holds and \(\chi >1\) a player is
said to extort their opponent.
Here, the reverse problem is considered: given a
\(p\in\mathbb{R}^4\) how does one identify \(\alpha, \beta\) if they
exist and is the strategy in fact acting in an extortionate way?

These conditions correspond to:

\begin{align}
    \tilde p_1 & = \alpha R + \beta R - P (\alpha + \beta)
            \label{eqn:condition_for_tilde_p1}\\
    \tilde p_2 & = \alpha S + \beta T - P (\alpha + \beta)
            \label{eqn:condition_for_tilde_p2}\\
    \tilde p_3 & = \alpha T + \beta S - P (\alpha + \beta)
            \label{eqn:condition_for_tilde_p3}\\
    \tilde p_4 & = \alpha P + \beta P - P (\alpha + \beta)
            \label{eqn:condition_for_tilde_p4}
\end{align}

Equation (\ref{eqn:condition_for_tilde_p4}) ensures that \(p_4=\tilde p_4=0\).
Equations (\ref{eqn:condition_for_tilde_p1}-\ref{eqn:condition_for_tilde_p3})
can be used to eliminate \(\alpha, \beta\), giving:

\begin{equation}\label{eqn:planar_definition_of_extortion}
    \tilde p_1 = \frac{(R - P)(\tilde p_2 + \tilde p_3)}{S + T - 2P}
\end{equation}

with:

\begin{equation}\label{eqn:definition_of_chi}
    \chi = \frac{\tilde p_2 (P - T) + \tilde p_3 (S - P)}
                {\tilde p_2 (P - S) + \tilde p_3 (T - P)}
\end{equation}

Given a strategy \(p\in\mathbb{R}^{4\times 1}\) equations
(\ref{eqn:condition_for_tilde_p4}), (\ref{eqn:planar_definition_of_extortion}-\ref{eqn:definition_of_chi}) can be used to check if
a strategy is extortionate. The conditions correspond to:

\begin{align}
    p_1 & = \frac{(R-P)(p_2 + p_3) - R + T + S - P}{S + T - 2P}
     \label{eqn:condition_for_p1}\\
    p_4 & = 0 \label{eqn:condition_for_p4}\\
    1 & > p_2 + p_3\label{eqn:condition_for_chi}
\end{align}

The algebraic steps necessary to prove these results are available in the
supporting materials.

All extortionate strategies reside on a triangular (\ref{eqn:condition_for_chi})
plane (\ref{eqn:condition_for_p1}) in 3 dimensions (\ref{eqn:condition_for_p4}).
Using this formulation it can be seen that a necessary (but not sufficient)
condition for an extortionate strategy is that it cooperates on average less
than 50\% of the time when in a state of disagreement with the opponent.

As an example, consider the known extortionate strategy \(p=(8 / 9, 1 / 2, 1 /
3, 0)\) from~\cite{Stewart2012} which is referred to as \texttt{Extort-2}. In
this case, for the standard values of \((R, T, S, P)\) constraint
(\ref{eqn:condition_for_p1}) corresponds to:

\begin{equation}
    p_1 = \frac{2(p_2 + p_3) + 1}{3}
\end{equation}

It is clear that in this case all constraints hold.

This approach could in fact be used to confirm that a given strategy is acting
in an extortionate manner even if it is not a memory one strategy. However, in
practice, if a closed form for \(p\) is not known, then due to measurement
and/or numerical error this would not work.

This problem can be written in the following linear algebraic form where
\(x=(\alpha, \beta)\)
and \(p^*=(\tilde p_1 - 1, tilde_2 - 1, p_3)\):

\begin{equation}\label{eqn:linear_algebraic_equation_for_p}
    Cx= p^*
\end{equation}

\(C\) corresponds to equations
(\ref{eqn:condition_for_tilde_p1}-\ref{eqn:condition_for_tilde_p3}) and is
given by:

\begin{equation}\label{eqn:definition_of_C}
    C =
    \begin{bmatrix}
        R - P & R- P \\
        S - P & T- P \\
        T - P & S- P \\
    \end{bmatrix}
\end{equation}

Note that in general, equation (\ref{eqn:linear_algebraic_equation_for_p}) will
not necessarily have a solution. From the Rouch\'{e}-Capelli theorem if there is
a solution it is unique as \(\text{rank}(C)=2\) which is the dimension of the
variable \(x\). The best fitting \(x\) is found by minimizing:

\begin{equation}\label{eqn:r_squared}
    \text{SSError} = \|C x- p^*\|_2^2 = \sum_{i=1}^{3}\left((C\bar x)_i-p_i^*\right)^2
\end{equation}

Note that \(\text{SSError}\), which is the square of the Frobenius
norm~\cite{Golub2013}, becomes a measure of how close a strategy is to being an
extortionate strategy. Suspicion
of extortion then corresponds to a threshold on \(\text{SSError}\).

By observing interactions (human or otherwise), their memory one representation
can be inferred and this approach can be used to recognise extortionate
behaviour. The notion of comparing theoretic and actual plays of the IPD is not
novel, see for example~\cite{Rand2013}. Immediately it is noted that if the
environment is noisy~\cite{Wu1995} then no strategy can be considered to be
extortionate as \(p_4>0\).

In the next section, this idea will be illustrated by observing the interactions
that take place in a computer based tournament of the IPD\@.

\section{Numerical experiments}\label{sec:numerical-experiments}

In~\cite{Stewart2012} results from a tournament with
\input{./assets/tex/number_of_stewart_plotkin_strategies/main.tex} strategies,
was presented with specific consideration given to ZD strategies. This
tournament is reproduced here using the Axelrod-Python
project~\cite{Knight2016}. To obtain a good measure of the corresponding
transition rates for each strategy all matches have been run for
\input{assets/tex/number_of_turns/main.tex} turns and every match has been
repeated \input{assets/tex/number_of_repetitions/main.tex} times. All of this
interaction data is available at~\cite{vincent_knight_2018_1297075}. A good
match between the inferred Markov chain and the state distribution of the actual
interactions has been verified. Data for this is presented in the supplementary
materials.

Figure~\ref{fig:SSError_overall_in_stewart_plotkin} shows the \(\text{SSError}\)
values for all the strategies in the tournament, as reported
in~\cite{Stewart2012} the extortionate strategy (which has an expected
\(\text{SSError}\) approximately 0) gains a large number of wins.

\begin{figure}[!htbp]
    \centering
    \includegraphics[width=.8\textwidth]{./assets/img/SSError_overall_in_stewart_plotkin/main.pdf}
    \caption{\(\text{SSError}\) and state probabilities for the strategies
        of~\cite{Stewart2012}, ordered both by number of wins and overall score.
        Note that \(P(DC)\) is not shown as it corresponds to the transpose of
        \(P(CD)\). Cooperator and Defector are omitted as they do not visit all
        the states.}
    \label{fig:SSError_overall_in_stewart_plotkin}
\end{figure}

Here, the work of~\cite{Stewart2012} is extended by investigating a tournament
with \input{assets/tex/number_of_full_strategies/main.tex}
strategies.

The results of this analysis are shown in
Figure~\ref{fig:SSError_and_probabilities_in_full}. The top ranking strategies
by number of wins seem to be extortionate (but not against all strategies) and
it can be seen that a small sub group of strategies achieve mutual defection.
All the top ranking strategies according to score achieve mutual cooperation and
do not extort each other, however they
\textbf{do} exhibit extortionate behaviour towards a number of the lower ranking
strategies.

\begin{figure}[!htbp]
    \centering
    \includegraphics[width=.8\textwidth]{./assets/img/SSError_and_probabilities_in_full/main.pdf}
    \caption{\(\text{SSError}\) for the strategies for the full tournament. Only
    strategy interactions for which \(p_4=0\) and \(\chi>1\) are displayed.}
    \label{fig:SSError_and_probabilities_in_full}
\end{figure}

\section{Conclusion}\label{sec:conclusion}

This work defines an approach to measure whether or not a player is playing a
strategy that corresponds to an extortionate strategy as defined
in~\cite{Press2012}: a mathematical model for suspicion. Indeed, all
extortionate strategies have been
 classified as lying on a triangular plane.
This rigorous classification fails to be robust to small measurement error, thus
a statistical approach is proposed.
This is done through a linear algebraic approach for approximating the solution
of a linear system. Using this, a large number of pairwise interactions is
simulated and in fact very few strategies are found to act extortionately.

The work of~\cite{Press2012}, whilst showing that a clever approach to taking
advantage of another memory one strategy exists: this is incomplete. Whilst the
elegance of this result is very attractive, just as the simplicity of the
victory of Tit For Tat in Axelrod's original tournaments was, it is incomplete.
Extortionate strategies achieve a high number of wins but they do not
achieve a high score which corresponds to the fitness landscape in an
evolutionary sense. From the large number of interactions a payoff matrix \(S\)
can be measured where \(S_{ij}\) denotes the score (using standard values of
\((R, S, T, P) = (3, 0, 5, 1)\)) of the \(i\)th strategy
against the \(j\)th strategy. Using this, the replicator equation
describes the evolution of the system based on a population density fitness
function:

\begin{equation}\label{eqn:replicator_dynamics}
    \frac{dx}{dt} = x(S-x^TS x)
\end{equation}

Equation (\ref{eqn:replicator_dynamics}) is solved numerically through an
integration technique described in~\cite{Petzold1983} and
Figure~\ref{fig:replicator_dynamics} shows the evolution of the distribution of
the system: the various strategies are ranked by scores. It is clear to see that
only the high ranking strategies survive the evolutionary process (in fact,
only \input{./assets/img/replicator_dynamics/main.tex}
have a final distribution greater than \(10 ^ {-2}\)). This confirms the
findings of~\cite{Moran1707} in which sophisticated strategies resist
evolutionary invasion of shorter memory strategies. Recalling
Figure~\ref{fig:SSError_and_probabilities_in_full} this demonstrates that:

\begin{itemize}
    \item Cooperation emerges through the evolutionary process: the high scoring
        strategies do not exhibit extortionate behaviour towards each other.
    \item Extortionate strategies do not survive the evolutionary process.
\end{itemize}

\begin{figure}[!htbp]
    \centering
    \includegraphics[width=.8\textwidth]{./assets/img/replicator_dynamics/main.pdf}
    \caption{Numerical simulation of the replicator equation
    (\ref{eqn:replicator_dynamics}): strategies are ordered by score, only the strategies with a high score survive the evolutionary process.}
    \label{fig:replicator_dynamics}
\end{figure}

This work can be used to classify plays of the IPD\@: data can be collected from
actual interactions (in lab or in the field). Furthermore, this allows for a
classification method similar to the notion of fingerprinting presented
in~\cite{Ashlock2008}. Trained strategies can potentially be classified as
extortionate or not or it could be possible to even constrain the reinforcement
learning approaches that are becoming prevalent in the literature.
Alternatively, this mathematical approach for recognising extortion could be
used in sophisticated strategies to defend against invasion. Arguably, some of
the strategies considered here exhibit this behaviour, indeed as described
in~\cite{Harper2017}, the top ranking strategies in the full tournament are
obtained using evolutionary reinforcement learning techniques, thus, suspicion
of extortionate behaviour could in fact be an evolutionary trait.

\section*{Acknowledgements}

The following open source software libraries were used in this research:

\begin{itemize}
    \item The Axelrod ~\cite{Knight2016, Knight2018} library (IPD strategies and
        tournaments).
    \item The sympy library~\cite{Meurer2017} (verification of all symbolic
        calculations).
    \item The matplotlib~\cite{Droettboom2018} library (visualisation).
    \item The pandas~\cite{Structures2010}, dask~\cite{Dask2016} and
        NumPy~\cite{Oliphant2015} libraries (data manipulation).
    \item The SciPy~\cite{Jones2001} library (numerical integration of the
        replicator equation).
\end{itemize}

This work was performed using the computational facilities of the Advanced
Research Computing @ Cardiff (ARCCA) Division, Cardiff University.

\printbibliography

\newpage
\section*{Supplementary materials}

\includepdf{assets/pdf/proof_of_form_of_extortionate_strategies/main.pdf}

\newpage

Using the pair wise interactions the transition rates \(p,
q\) can be measured and the steady state probabilities inferred and compared to
the actual probabilities of each state.
This is done numerically by computing the singular eigenvector of the
matrix \(A\) \cite{Stewart2009}:

\[
    A =
    \begin{bmatrix}
        p_1 q_1 & p_1 (1 - q_1) & (1 - p_1) q_1 & (1 -p_1) (1 - q_1) \\
        p_2 q_2 & p_2 (1 - q_2) & (1 - p_2) q_2 & (1 -p_2) (1 - q_2) \\
        p_3 q_3 & p_3 (1 - q_3) & (1 - p_3) q_3 & (1 -p_3) (1 - q_3) \\
        p_4 q_4 & p_4 (1 - q_4) & (1 - p_4) q_4 & (1 -p_4) (1 - q_4) \\
    \end{bmatrix}
\]

Figure~\ref{fig:computed_probabilities_vs_theoretic_probabilities} shows a
regression line fitted to every pairwise interaction with a reported
\(\text{SSError}\) value (pairwise interactions with missing states were
omitted). This serves to validate the approach: a part from some edge cases the
relationship is consistent.

\begin{figure}[!htbp]
    \centering
    \includegraphics[width=.8\textwidth]{./assets/img/computed_probabilities_vs_theoretic_probabilities/main.pdf}
    \caption{The
        relationship between the steady state probabilities inferred from the
        measured transitions and the actual steady state probabilities. A linear
        regression line is included validating the approach.}
    \label{fig:computed_probabilities_vs_theoretic_probabilities}
\end{figure}


\end{document}
 turns and every match has been
repeated \documentclass[a4paper]{article}

\usepackage{amsmath}
\usepackage{amssymb}
\usepackage[margin=1.5cm,
            includefoot,
            footskip=30pt]{geometry}
\usepackage{layout}
\usepackage{graphicx}
\usepackage{subcaption}

\usepackage{biblatex}
\usepackage{pdfpages}

\bibliography{main.bib}

\title{Suspicion: Recognising and evaluating the effectiveness
       of extortion in the Iterated Prisoner's Dilemma}
\author{Vincent A. Knight \and Nikoleta E. Glynatsi}
\date{\today}



\begin{document}

\maketitle

\begin{abstract}
    The Iterated Prisoner's Dilemma is a model for rational and evolutionary
    interactive behaviour. It has applications both in the study of human social
    behaviour as well as in biology.
    It is used to understand when and how a rational individual might
    accept an immediate cost to their own utility for the direct benefit of
    another.

    Much attention has been given to a class of strategies called
    Zero Determinant strategies. It has been theoretically shown that these
    strategies can ``extort'' any player.

    In this work, an approach to identify if observed strategies are playing in
    an extortionate way is described. Furthermore, experimental analysis of
    a large tournament with \input{assets/tex/number_of_full_strategies/main.tex}
    strategies is considered. In this setting
    the most highly performing strategies do not play in an extortionate way
    against each other but do against lower performing strategies.
    This suggests that whilst the theory of Zero Determinant strategies
    indicates that memory is not of fundamental importance to the evolution of
    cooperative behaviour, this is incomplete.
\end{abstract}

\section{Introduction}\label{sec:introduction}

Agent based game theoretic models have become a stalwart of the underpinning
mathematics of interactive behaviours. One of the major pieces of work
in this area is the pair of original computer tournaments run by Robert
Axelrod~\cite{Axelrod1980, Axelrod1980a}. These tournaments pitted submitted
computer strategies against each other in plays of the Iterated Prisoner's
Dilemma. A common game where agents can choose to pay a slight cost to their
immediate utility in the hope of building a reputation. This has been used in
economic and evolutionary game theory to understand the evolution of cooperative
behaviour.

Recently, a class of strategies was described in~\cite{Press2012} that can
provably extort any given opponent. In~\cite{Hilbe2013, Moran1707} some
questions have already been asked about the true effectiveness of these
strategies in an evolutionary setting. Here another question is asked: is it
possible to recognise this extortionate behaviour? A mathematical procedure for
suspicion is presented: in the same way that the continued actions of an
extortionate individual might raise suspicion.

This work makes use of the Axelrod Python library~\cite{Knight2018, Knight2016}
with a large number of Prisoner Dilemma strategies available to give an
extensive numerical example of the ideas presented.  The approach is presented
in Section~\ref{sec:delta-zd-strategies}.  All of the code and data discussed
in Section~\ref{sec:numerical-experiments} is open sourced, archived and
written according to best scientific principles~\cite{Wilson2014}. The data
archive can be found at~\cite{vincent_knight_2018_1297075}.

\section{Recognising Extortion}\label{sec:delta-zd-strategies}

In~\cite{Press2012}, given a match between 2 memory-one strategies, the concept
of Zero Determinant (ZD) strategies is introduced. The main result of that paper
shows that given two memory one players \(p, q\in\mathbb{R}^4\) a linear
relationship between the players' scores could be forced by one of the players.

Using the notation of~\cite{Press2012}, assuming the utilities for player \(p\)
are given by \(S_x=(R, S, T, P)\) and for player \(q\) by \(S_y=(R, T, S, P)\)
and that the stationary scores of each player is given by \(S_X\) and \(S_Y\)
respectively. The main result of~\cite{Press2012} is that if

\begin{equation}\label{eqn:linear_relationship_for_p}
    \tilde p=\alpha S_x + \beta S_y + \gamma
\end{equation}

or

\begin{equation}\label{eqn:linear_relationship_for_q}
    \tilde q=\alpha S_x + \beta S_y + \gamma
\end{equation}

where \(\tilde p = (1 - p_1, 1 - p_2, p_3, p_4)\) and
\(\tilde q = (1 - q_1, 1 - q_2, q_3, q_4)\) then:

\begin{equation}
    \alpha S_X + \beta S_Y + \gamma = 0
\end{equation}

In~\cite{Press2012} a particular type of ZD strategy is defined: extortionate
strategies. If:

\begin{equation}\label{eqn:constraint_for_extortion}
    \gamma = - P(\alpha + \beta)
\end{equation}

then the player can ensure they get a score \(\chi\) times
larger than the opponent. This extortion coefficient is given by:

\begin{equation}\label{eqn:definition_of_chi}
    \chi=\frac{-\beta}{\alpha}
\end{equation}

Thus, if (\ref{eqn:constraint_for_extortion}) holds and \(\chi >1\) a player is
said to extort their opponent.
Here, the reverse problem is considered: given a
\(p\in\mathbb{R}^4\) how does one identify \(\alpha, \beta\) if they
exist and is the strategy in fact acting in an extortionate way?

These conditions correspond to:

\begin{align}
    \tilde p_1 & = \alpha R + \beta R - P (\alpha + \beta)
            \label{eqn:condition_for_tilde_p1}\\
    \tilde p_2 & = \alpha S + \beta T - P (\alpha + \beta)
            \label{eqn:condition_for_tilde_p2}\\
    \tilde p_3 & = \alpha T + \beta S - P (\alpha + \beta)
            \label{eqn:condition_for_tilde_p3}\\
    \tilde p_4 & = \alpha P + \beta P - P (\alpha + \beta)
            \label{eqn:condition_for_tilde_p4}
\end{align}

Equation (\ref{eqn:condition_for_tilde_p4}) ensures that \(p_4=\tilde p_4=0\).
Equations (\ref{eqn:condition_for_tilde_p1}-\ref{eqn:condition_for_tilde_p3})
can be used to eliminate \(\alpha, \beta\), giving:

\begin{equation}\label{eqn:planar_definition_of_extortion}
    \tilde p_1 = \frac{(R - P)(\tilde p_2 + \tilde p_3)}{S + T - 2P}
\end{equation}

with:

\begin{equation}\label{eqn:definition_of_chi}
    \chi = \frac{\tilde p_2 (P - T) + \tilde p_3 (S - P)}
                {\tilde p_2 (P - S) + \tilde p_3 (T - P)}
\end{equation}

Given a strategy \(p\in\mathbb{R}^{4\times 1}\) equations
(\ref{eqn:condition_for_tilde_p4}), (\ref{eqn:planar_definition_of_extortion}-\ref{eqn:definition_of_chi}) can be used to check if
a strategy is extortionate. The conditions correspond to:

\begin{align}
    p_1 & = \frac{(R-P)(p_2 + p_3) - R + T + S - P}{S + T - 2P}
     \label{eqn:condition_for_p1}\\
    p_4 & = 0 \label{eqn:condition_for_p4}\\
    1 & > p_2 + p_3\label{eqn:condition_for_chi}
\end{align}

The algebraic steps necessary to prove these results are available in the
supporting materials.

All extortionate strategies reside on a triangular (\ref{eqn:condition_for_chi})
plane (\ref{eqn:condition_for_p1}) in 3 dimensions (\ref{eqn:condition_for_p4}).
Using this formulation it can be seen that a necessary (but not sufficient)
condition for an extortionate strategy is that it cooperates on average less
than 50\% of the time when in a state of disagreement with the opponent.

As an example, consider the known extortionate strategy \(p=(8 / 9, 1 / 2, 1 /
3, 0)\) from~\cite{Stewart2012} which is referred to as \texttt{Extort-2}. In
this case, for the standard values of \((R, T, S, P)\) constraint
(\ref{eqn:condition_for_p1}) corresponds to:

\begin{equation}
    p_1 = \frac{2(p_2 + p_3) + 1}{3}
\end{equation}

It is clear that in this case all constraints hold.

This approach could in fact be used to confirm that a given strategy is acting
in an extortionate manner even if it is not a memory one strategy. However, in
practice, if a closed form for \(p\) is not known, then due to measurement
and/or numerical error this would not work.

This problem can be written in the following linear algebraic form where
\(x=(\alpha, \beta)\)
and \(p^*=(\tilde p_1 - 1, tilde_2 - 1, p_3)\):

\begin{equation}\label{eqn:linear_algebraic_equation_for_p}
    Cx= p^*
\end{equation}

\(C\) corresponds to equations
(\ref{eqn:condition_for_tilde_p1}-\ref{eqn:condition_for_tilde_p3}) and is
given by:

\begin{equation}\label{eqn:definition_of_C}
    C =
    \begin{bmatrix}
        R - P & R- P \\
        S - P & T- P \\
        T - P & S- P \\
    \end{bmatrix}
\end{equation}

Note that in general, equation (\ref{eqn:linear_algebraic_equation_for_p}) will
not necessarily have a solution. From the Rouch\'{e}-Capelli theorem if there is
a solution it is unique as \(\text{rank}(C)=2\) which is the dimension of the
variable \(x\). The best fitting \(x\) is found by minimizing:

\begin{equation}\label{eqn:r_squared}
    \text{SSError} = \|C x- p^*\|_2^2 = \sum_{i=1}^{3}\left((C\bar x)_i-p_i^*\right)^2
\end{equation}

Note that \(\text{SSError}\), which is the square of the Frobenius
norm~\cite{Golub2013}, becomes a measure of how close a strategy is to being an
extortionate strategy. Suspicion
of extortion then corresponds to a threshold on \(\text{SSError}\).

By observing interactions (human or otherwise), their memory one representation
can be inferred and this approach can be used to recognise extortionate
behaviour. The notion of comparing theoretic and actual plays of the IPD is not
novel, see for example~\cite{Rand2013}. Immediately it is noted that if the
environment is noisy~\cite{Wu1995} then no strategy can be considered to be
extortionate as \(p_4>0\).

In the next section, this idea will be illustrated by observing the interactions
that take place in a computer based tournament of the IPD\@.

\section{Numerical experiments}\label{sec:numerical-experiments}

In~\cite{Stewart2012} results from a tournament with
\input{./assets/tex/number_of_stewart_plotkin_strategies/main.tex} strategies,
was presented with specific consideration given to ZD strategies. This
tournament is reproduced here using the Axelrod-Python
project~\cite{Knight2016}. To obtain a good measure of the corresponding
transition rates for each strategy all matches have been run for
\input{assets/tex/number_of_turns/main.tex} turns and every match has been
repeated \input{assets/tex/number_of_repetitions/main.tex} times. All of this
interaction data is available at~\cite{vincent_knight_2018_1297075}. A good
match between the inferred Markov chain and the state distribution of the actual
interactions has been verified. Data for this is presented in the supplementary
materials.

Figure~\ref{fig:SSError_overall_in_stewart_plotkin} shows the \(\text{SSError}\)
values for all the strategies in the tournament, as reported
in~\cite{Stewart2012} the extortionate strategy (which has an expected
\(\text{SSError}\) approximately 0) gains a large number of wins.

\begin{figure}[!htbp]
    \centering
    \includegraphics[width=.8\textwidth]{./assets/img/SSError_overall_in_stewart_plotkin/main.pdf}
    \caption{\(\text{SSError}\) and state probabilities for the strategies
        of~\cite{Stewart2012}, ordered both by number of wins and overall score.
        Note that \(P(DC)\) is not shown as it corresponds to the transpose of
        \(P(CD)\). Cooperator and Defector are omitted as they do not visit all
        the states.}
    \label{fig:SSError_overall_in_stewart_plotkin}
\end{figure}

Here, the work of~\cite{Stewart2012} is extended by investigating a tournament
with \input{assets/tex/number_of_full_strategies/main.tex}
strategies.

The results of this analysis are shown in
Figure~\ref{fig:SSError_and_probabilities_in_full}. The top ranking strategies
by number of wins seem to be extortionate (but not against all strategies) and
it can be seen that a small sub group of strategies achieve mutual defection.
All the top ranking strategies according to score achieve mutual cooperation and
do not extort each other, however they
\textbf{do} exhibit extortionate behaviour towards a number of the lower ranking
strategies.

\begin{figure}[!htbp]
    \centering
    \includegraphics[width=.8\textwidth]{./assets/img/SSError_and_probabilities_in_full/main.pdf}
    \caption{\(\text{SSError}\) for the strategies for the full tournament. Only
    strategy interactions for which \(p_4=0\) and \(\chi>1\) are displayed.}
    \label{fig:SSError_and_probabilities_in_full}
\end{figure}

\section{Conclusion}\label{sec:conclusion}

This work defines an approach to measure whether or not a player is playing a
strategy that corresponds to an extortionate strategy as defined
in~\cite{Press2012}: a mathematical model for suspicion. Indeed, all
extortionate strategies have been
 classified as lying on a triangular plane.
This rigorous classification fails to be robust to small measurement error, thus
a statistical approach is proposed.
This is done through a linear algebraic approach for approximating the solution
of a linear system. Using this, a large number of pairwise interactions is
simulated and in fact very few strategies are found to act extortionately.

The work of~\cite{Press2012}, whilst showing that a clever approach to taking
advantage of another memory one strategy exists: this is incomplete. Whilst the
elegance of this result is very attractive, just as the simplicity of the
victory of Tit For Tat in Axelrod's original tournaments was, it is incomplete.
Extortionate strategies achieve a high number of wins but they do not
achieve a high score which corresponds to the fitness landscape in an
evolutionary sense. From the large number of interactions a payoff matrix \(S\)
can be measured where \(S_{ij}\) denotes the score (using standard values of
\((R, S, T, P) = (3, 0, 5, 1)\)) of the \(i\)th strategy
against the \(j\)th strategy. Using this, the replicator equation
describes the evolution of the system based on a population density fitness
function:

\begin{equation}\label{eqn:replicator_dynamics}
    \frac{dx}{dt} = x(S-x^TS x)
\end{equation}

Equation (\ref{eqn:replicator_dynamics}) is solved numerically through an
integration technique described in~\cite{Petzold1983} and
Figure~\ref{fig:replicator_dynamics} shows the evolution of the distribution of
the system: the various strategies are ranked by scores. It is clear to see that
only the high ranking strategies survive the evolutionary process (in fact,
only \input{./assets/img/replicator_dynamics/main.tex}
have a final distribution greater than \(10 ^ {-2}\)). This confirms the
findings of~\cite{Moran1707} in which sophisticated strategies resist
evolutionary invasion of shorter memory strategies. Recalling
Figure~\ref{fig:SSError_and_probabilities_in_full} this demonstrates that:

\begin{itemize}
    \item Cooperation emerges through the evolutionary process: the high scoring
        strategies do not exhibit extortionate behaviour towards each other.
    \item Extortionate strategies do not survive the evolutionary process.
\end{itemize}

\begin{figure}[!htbp]
    \centering
    \includegraphics[width=.8\textwidth]{./assets/img/replicator_dynamics/main.pdf}
    \caption{Numerical simulation of the replicator equation
    (\ref{eqn:replicator_dynamics}): strategies are ordered by score, only the strategies with a high score survive the evolutionary process.}
    \label{fig:replicator_dynamics}
\end{figure}

This work can be used to classify plays of the IPD\@: data can be collected from
actual interactions (in lab or in the field). Furthermore, this allows for a
classification method similar to the notion of fingerprinting presented
in~\cite{Ashlock2008}. Trained strategies can potentially be classified as
extortionate or not or it could be possible to even constrain the reinforcement
learning approaches that are becoming prevalent in the literature.
Alternatively, this mathematical approach for recognising extortion could be
used in sophisticated strategies to defend against invasion. Arguably, some of
the strategies considered here exhibit this behaviour, indeed as described
in~\cite{Harper2017}, the top ranking strategies in the full tournament are
obtained using evolutionary reinforcement learning techniques, thus, suspicion
of extortionate behaviour could in fact be an evolutionary trait.

\section*{Acknowledgements}

The following open source software libraries were used in this research:

\begin{itemize}
    \item The Axelrod ~\cite{Knight2016, Knight2018} library (IPD strategies and
        tournaments).
    \item The sympy library~\cite{Meurer2017} (verification of all symbolic
        calculations).
    \item The matplotlib~\cite{Droettboom2018} library (visualisation).
    \item The pandas~\cite{Structures2010}, dask~\cite{Dask2016} and
        NumPy~\cite{Oliphant2015} libraries (data manipulation).
    \item The SciPy~\cite{Jones2001} library (numerical integration of the
        replicator equation).
\end{itemize}

This work was performed using the computational facilities of the Advanced
Research Computing @ Cardiff (ARCCA) Division, Cardiff University.

\printbibliography

\newpage
\section*{Supplementary materials}

\includepdf{assets/pdf/proof_of_form_of_extortionate_strategies/main.pdf}

\newpage

Using the pair wise interactions the transition rates \(p,
q\) can be measured and the steady state probabilities inferred and compared to
the actual probabilities of each state.
This is done numerically by computing the singular eigenvector of the
matrix \(A\) \cite{Stewart2009}:

\[
    A =
    \begin{bmatrix}
        p_1 q_1 & p_1 (1 - q_1) & (1 - p_1) q_1 & (1 -p_1) (1 - q_1) \\
        p_2 q_2 & p_2 (1 - q_2) & (1 - p_2) q_2 & (1 -p_2) (1 - q_2) \\
        p_3 q_3 & p_3 (1 - q_3) & (1 - p_3) q_3 & (1 -p_3) (1 - q_3) \\
        p_4 q_4 & p_4 (1 - q_4) & (1 - p_4) q_4 & (1 -p_4) (1 - q_4) \\
    \end{bmatrix}
\]

Figure~\ref{fig:computed_probabilities_vs_theoretic_probabilities} shows a
regression line fitted to every pairwise interaction with a reported
\(\text{SSError}\) value (pairwise interactions with missing states were
omitted). This serves to validate the approach: a part from some edge cases the
relationship is consistent.

\begin{figure}[!htbp]
    \centering
    \includegraphics[width=.8\textwidth]{./assets/img/computed_probabilities_vs_theoretic_probabilities/main.pdf}
    \caption{The
        relationship between the steady state probabilities inferred from the
        measured transitions and the actual steady state probabilities. A linear
        regression line is included validating the approach.}
    \label{fig:computed_probabilities_vs_theoretic_probabilities}
\end{figure}


\end{document}
 times. All of this
interaction data is available at~\cite{vincent_knight_2018_1297075}. A good
match between the inferred Markov chain and the state distribution of the actual
interactions has been verified. Data for this is presented in the supplementary
materials.

Figure~\ref{fig:SSError_overall_in_stewart_plotkin} shows the \(\text{SSError}\)
values for all the strategies in the tournament, as reported
in~\cite{Stewart2012} the extortionate strategy (which has an expected
\(\text{SSError}\) approximately 0) gains a large number of wins.

\begin{figure}[!htbp]
    \centering
    \includegraphics[width=.8\textwidth]{./assets/img/SSError_overall_in_stewart_plotkin/main.pdf}
    \caption{\(\text{SSError}\) and state probabilities for the strategies
        of~\cite{Stewart2012}, ordered both by number of wins and overall score.
        Note that \(P(DC)\) is not shown as it corresponds to the transpose of
        \(P(CD)\). Cooperator and Defector are omitted as they do not visit all
        the states.}
    \label{fig:SSError_overall_in_stewart_plotkin}
\end{figure}

Here, the work of~\cite{Stewart2012} is extended by investigating a tournament
with \documentclass[a4paper]{article}

\usepackage{amsmath}
\usepackage{amssymb}
\usepackage[margin=1.5cm,
            includefoot,
            footskip=30pt]{geometry}
\usepackage{layout}
\usepackage{graphicx}
\usepackage{subcaption}

\usepackage{biblatex}
\usepackage{pdfpages}

\bibliography{main.bib}

\title{Suspicion: Recognising and evaluating the effectiveness
       of extortion in the Iterated Prisoner's Dilemma}
\author{Vincent A. Knight \and Nikoleta E. Glynatsi}
\date{\today}



\begin{document}

\maketitle

\begin{abstract}
    The Iterated Prisoner's Dilemma is a model for rational and evolutionary
    interactive behaviour. It has applications both in the study of human social
    behaviour as well as in biology.
    It is used to understand when and how a rational individual might
    accept an immediate cost to their own utility for the direct benefit of
    another.

    Much attention has been given to a class of strategies called
    Zero Determinant strategies. It has been theoretically shown that these
    strategies can ``extort'' any player.

    In this work, an approach to identify if observed strategies are playing in
    an extortionate way is described. Furthermore, experimental analysis of
    a large tournament with \input{assets/tex/number_of_full_strategies/main.tex}
    strategies is considered. In this setting
    the most highly performing strategies do not play in an extortionate way
    against each other but do against lower performing strategies.
    This suggests that whilst the theory of Zero Determinant strategies
    indicates that memory is not of fundamental importance to the evolution of
    cooperative behaviour, this is incomplete.
\end{abstract}

\section{Introduction}\label{sec:introduction}

Agent based game theoretic models have become a stalwart of the underpinning
mathematics of interactive behaviours. One of the major pieces of work
in this area is the pair of original computer tournaments run by Robert
Axelrod~\cite{Axelrod1980, Axelrod1980a}. These tournaments pitted submitted
computer strategies against each other in plays of the Iterated Prisoner's
Dilemma. A common game where agents can choose to pay a slight cost to their
immediate utility in the hope of building a reputation. This has been used in
economic and evolutionary game theory to understand the evolution of cooperative
behaviour.

Recently, a class of strategies was described in~\cite{Press2012} that can
provably extort any given opponent. In~\cite{Hilbe2013, Moran1707} some
questions have already been asked about the true effectiveness of these
strategies in an evolutionary setting. Here another question is asked: is it
possible to recognise this extortionate behaviour? A mathematical procedure for
suspicion is presented: in the same way that the continued actions of an
extortionate individual might raise suspicion.

This work makes use of the Axelrod Python library~\cite{Knight2018, Knight2016}
with a large number of Prisoner Dilemma strategies available to give an
extensive numerical example of the ideas presented.  The approach is presented
in Section~\ref{sec:delta-zd-strategies}.  All of the code and data discussed
in Section~\ref{sec:numerical-experiments} is open sourced, archived and
written according to best scientific principles~\cite{Wilson2014}. The data
archive can be found at~\cite{vincent_knight_2018_1297075}.

\section{Recognising Extortion}\label{sec:delta-zd-strategies}

In~\cite{Press2012}, given a match between 2 memory-one strategies, the concept
of Zero Determinant (ZD) strategies is introduced. The main result of that paper
shows that given two memory one players \(p, q\in\mathbb{R}^4\) a linear
relationship between the players' scores could be forced by one of the players.

Using the notation of~\cite{Press2012}, assuming the utilities for player \(p\)
are given by \(S_x=(R, S, T, P)\) and for player \(q\) by \(S_y=(R, T, S, P)\)
and that the stationary scores of each player is given by \(S_X\) and \(S_Y\)
respectively. The main result of~\cite{Press2012} is that if

\begin{equation}\label{eqn:linear_relationship_for_p}
    \tilde p=\alpha S_x + \beta S_y + \gamma
\end{equation}

or

\begin{equation}\label{eqn:linear_relationship_for_q}
    \tilde q=\alpha S_x + \beta S_y + \gamma
\end{equation}

where \(\tilde p = (1 - p_1, 1 - p_2, p_3, p_4)\) and
\(\tilde q = (1 - q_1, 1 - q_2, q_3, q_4)\) then:

\begin{equation}
    \alpha S_X + \beta S_Y + \gamma = 0
\end{equation}

In~\cite{Press2012} a particular type of ZD strategy is defined: extortionate
strategies. If:

\begin{equation}\label{eqn:constraint_for_extortion}
    \gamma = - P(\alpha + \beta)
\end{equation}

then the player can ensure they get a score \(\chi\) times
larger than the opponent. This extortion coefficient is given by:

\begin{equation}\label{eqn:definition_of_chi}
    \chi=\frac{-\beta}{\alpha}
\end{equation}

Thus, if (\ref{eqn:constraint_for_extortion}) holds and \(\chi >1\) a player is
said to extort their opponent.
Here, the reverse problem is considered: given a
\(p\in\mathbb{R}^4\) how does one identify \(\alpha, \beta\) if they
exist and is the strategy in fact acting in an extortionate way?

These conditions correspond to:

\begin{align}
    \tilde p_1 & = \alpha R + \beta R - P (\alpha + \beta)
            \label{eqn:condition_for_tilde_p1}\\
    \tilde p_2 & = \alpha S + \beta T - P (\alpha + \beta)
            \label{eqn:condition_for_tilde_p2}\\
    \tilde p_3 & = \alpha T + \beta S - P (\alpha + \beta)
            \label{eqn:condition_for_tilde_p3}\\
    \tilde p_4 & = \alpha P + \beta P - P (\alpha + \beta)
            \label{eqn:condition_for_tilde_p4}
\end{align}

Equation (\ref{eqn:condition_for_tilde_p4}) ensures that \(p_4=\tilde p_4=0\).
Equations (\ref{eqn:condition_for_tilde_p1}-\ref{eqn:condition_for_tilde_p3})
can be used to eliminate \(\alpha, \beta\), giving:

\begin{equation}\label{eqn:planar_definition_of_extortion}
    \tilde p_1 = \frac{(R - P)(\tilde p_2 + \tilde p_3)}{S + T - 2P}
\end{equation}

with:

\begin{equation}\label{eqn:definition_of_chi}
    \chi = \frac{\tilde p_2 (P - T) + \tilde p_3 (S - P)}
                {\tilde p_2 (P - S) + \tilde p_3 (T - P)}
\end{equation}

Given a strategy \(p\in\mathbb{R}^{4\times 1}\) equations
(\ref{eqn:condition_for_tilde_p4}), (\ref{eqn:planar_definition_of_extortion}-\ref{eqn:definition_of_chi}) can be used to check if
a strategy is extortionate. The conditions correspond to:

\begin{align}
    p_1 & = \frac{(R-P)(p_2 + p_3) - R + T + S - P}{S + T - 2P}
     \label{eqn:condition_for_p1}\\
    p_4 & = 0 \label{eqn:condition_for_p4}\\
    1 & > p_2 + p_3\label{eqn:condition_for_chi}
\end{align}

The algebraic steps necessary to prove these results are available in the
supporting materials.

All extortionate strategies reside on a triangular (\ref{eqn:condition_for_chi})
plane (\ref{eqn:condition_for_p1}) in 3 dimensions (\ref{eqn:condition_for_p4}).
Using this formulation it can be seen that a necessary (but not sufficient)
condition for an extortionate strategy is that it cooperates on average less
than 50\% of the time when in a state of disagreement with the opponent.

As an example, consider the known extortionate strategy \(p=(8 / 9, 1 / 2, 1 /
3, 0)\) from~\cite{Stewart2012} which is referred to as \texttt{Extort-2}. In
this case, for the standard values of \((R, T, S, P)\) constraint
(\ref{eqn:condition_for_p1}) corresponds to:

\begin{equation}
    p_1 = \frac{2(p_2 + p_3) + 1}{3}
\end{equation}

It is clear that in this case all constraints hold.

This approach could in fact be used to confirm that a given strategy is acting
in an extortionate manner even if it is not a memory one strategy. However, in
practice, if a closed form for \(p\) is not known, then due to measurement
and/or numerical error this would not work.

This problem can be written in the following linear algebraic form where
\(x=(\alpha, \beta)\)
and \(p^*=(\tilde p_1 - 1, tilde_2 - 1, p_3)\):

\begin{equation}\label{eqn:linear_algebraic_equation_for_p}
    Cx= p^*
\end{equation}

\(C\) corresponds to equations
(\ref{eqn:condition_for_tilde_p1}-\ref{eqn:condition_for_tilde_p3}) and is
given by:

\begin{equation}\label{eqn:definition_of_C}
    C =
    \begin{bmatrix}
        R - P & R- P \\
        S - P & T- P \\
        T - P & S- P \\
    \end{bmatrix}
\end{equation}

Note that in general, equation (\ref{eqn:linear_algebraic_equation_for_p}) will
not necessarily have a solution. From the Rouch\'{e}-Capelli theorem if there is
a solution it is unique as \(\text{rank}(C)=2\) which is the dimension of the
variable \(x\). The best fitting \(x\) is found by minimizing:

\begin{equation}\label{eqn:r_squared}
    \text{SSError} = \|C x- p^*\|_2^2 = \sum_{i=1}^{3}\left((C\bar x)_i-p_i^*\right)^2
\end{equation}

Note that \(\text{SSError}\), which is the square of the Frobenius
norm~\cite{Golub2013}, becomes a measure of how close a strategy is to being an
extortionate strategy. Suspicion
of extortion then corresponds to a threshold on \(\text{SSError}\).

By observing interactions (human or otherwise), their memory one representation
can be inferred and this approach can be used to recognise extortionate
behaviour. The notion of comparing theoretic and actual plays of the IPD is not
novel, see for example~\cite{Rand2013}. Immediately it is noted that if the
environment is noisy~\cite{Wu1995} then no strategy can be considered to be
extortionate as \(p_4>0\).

In the next section, this idea will be illustrated by observing the interactions
that take place in a computer based tournament of the IPD\@.

\section{Numerical experiments}\label{sec:numerical-experiments}

In~\cite{Stewart2012} results from a tournament with
\input{./assets/tex/number_of_stewart_plotkin_strategies/main.tex} strategies,
was presented with specific consideration given to ZD strategies. This
tournament is reproduced here using the Axelrod-Python
project~\cite{Knight2016}. To obtain a good measure of the corresponding
transition rates for each strategy all matches have been run for
\input{assets/tex/number_of_turns/main.tex} turns and every match has been
repeated \input{assets/tex/number_of_repetitions/main.tex} times. All of this
interaction data is available at~\cite{vincent_knight_2018_1297075}. A good
match between the inferred Markov chain and the state distribution of the actual
interactions has been verified. Data for this is presented in the supplementary
materials.

Figure~\ref{fig:SSError_overall_in_stewart_plotkin} shows the \(\text{SSError}\)
values for all the strategies in the tournament, as reported
in~\cite{Stewart2012} the extortionate strategy (which has an expected
\(\text{SSError}\) approximately 0) gains a large number of wins.

\begin{figure}[!htbp]
    \centering
    \includegraphics[width=.8\textwidth]{./assets/img/SSError_overall_in_stewart_plotkin/main.pdf}
    \caption{\(\text{SSError}\) and state probabilities for the strategies
        of~\cite{Stewart2012}, ordered both by number of wins and overall score.
        Note that \(P(DC)\) is not shown as it corresponds to the transpose of
        \(P(CD)\). Cooperator and Defector are omitted as they do not visit all
        the states.}
    \label{fig:SSError_overall_in_stewart_plotkin}
\end{figure}

Here, the work of~\cite{Stewart2012} is extended by investigating a tournament
with \input{assets/tex/number_of_full_strategies/main.tex}
strategies.

The results of this analysis are shown in
Figure~\ref{fig:SSError_and_probabilities_in_full}. The top ranking strategies
by number of wins seem to be extortionate (but not against all strategies) and
it can be seen that a small sub group of strategies achieve mutual defection.
All the top ranking strategies according to score achieve mutual cooperation and
do not extort each other, however they
\textbf{do} exhibit extortionate behaviour towards a number of the lower ranking
strategies.

\begin{figure}[!htbp]
    \centering
    \includegraphics[width=.8\textwidth]{./assets/img/SSError_and_probabilities_in_full/main.pdf}
    \caption{\(\text{SSError}\) for the strategies for the full tournament. Only
    strategy interactions for which \(p_4=0\) and \(\chi>1\) are displayed.}
    \label{fig:SSError_and_probabilities_in_full}
\end{figure}

\section{Conclusion}\label{sec:conclusion}

This work defines an approach to measure whether or not a player is playing a
strategy that corresponds to an extortionate strategy as defined
in~\cite{Press2012}: a mathematical model for suspicion. Indeed, all
extortionate strategies have been
 classified as lying on a triangular plane.
This rigorous classification fails to be robust to small measurement error, thus
a statistical approach is proposed.
This is done through a linear algebraic approach for approximating the solution
of a linear system. Using this, a large number of pairwise interactions is
simulated and in fact very few strategies are found to act extortionately.

The work of~\cite{Press2012}, whilst showing that a clever approach to taking
advantage of another memory one strategy exists: this is incomplete. Whilst the
elegance of this result is very attractive, just as the simplicity of the
victory of Tit For Tat in Axelrod's original tournaments was, it is incomplete.
Extortionate strategies achieve a high number of wins but they do not
achieve a high score which corresponds to the fitness landscape in an
evolutionary sense. From the large number of interactions a payoff matrix \(S\)
can be measured where \(S_{ij}\) denotes the score (using standard values of
\((R, S, T, P) = (3, 0, 5, 1)\)) of the \(i\)th strategy
against the \(j\)th strategy. Using this, the replicator equation
describes the evolution of the system based on a population density fitness
function:

\begin{equation}\label{eqn:replicator_dynamics}
    \frac{dx}{dt} = x(S-x^TS x)
\end{equation}

Equation (\ref{eqn:replicator_dynamics}) is solved numerically through an
integration technique described in~\cite{Petzold1983} and
Figure~\ref{fig:replicator_dynamics} shows the evolution of the distribution of
the system: the various strategies are ranked by scores. It is clear to see that
only the high ranking strategies survive the evolutionary process (in fact,
only \input{./assets/img/replicator_dynamics/main.tex}
have a final distribution greater than \(10 ^ {-2}\)). This confirms the
findings of~\cite{Moran1707} in which sophisticated strategies resist
evolutionary invasion of shorter memory strategies. Recalling
Figure~\ref{fig:SSError_and_probabilities_in_full} this demonstrates that:

\begin{itemize}
    \item Cooperation emerges through the evolutionary process: the high scoring
        strategies do not exhibit extortionate behaviour towards each other.
    \item Extortionate strategies do not survive the evolutionary process.
\end{itemize}

\begin{figure}[!htbp]
    \centering
    \includegraphics[width=.8\textwidth]{./assets/img/replicator_dynamics/main.pdf}
    \caption{Numerical simulation of the replicator equation
    (\ref{eqn:replicator_dynamics}): strategies are ordered by score, only the strategies with a high score survive the evolutionary process.}
    \label{fig:replicator_dynamics}
\end{figure}

This work can be used to classify plays of the IPD\@: data can be collected from
actual interactions (in lab or in the field). Furthermore, this allows for a
classification method similar to the notion of fingerprinting presented
in~\cite{Ashlock2008}. Trained strategies can potentially be classified as
extortionate or not or it could be possible to even constrain the reinforcement
learning approaches that are becoming prevalent in the literature.
Alternatively, this mathematical approach for recognising extortion could be
used in sophisticated strategies to defend against invasion. Arguably, some of
the strategies considered here exhibit this behaviour, indeed as described
in~\cite{Harper2017}, the top ranking strategies in the full tournament are
obtained using evolutionary reinforcement learning techniques, thus, suspicion
of extortionate behaviour could in fact be an evolutionary trait.

\section*{Acknowledgements}

The following open source software libraries were used in this research:

\begin{itemize}
    \item The Axelrod ~\cite{Knight2016, Knight2018} library (IPD strategies and
        tournaments).
    \item The sympy library~\cite{Meurer2017} (verification of all symbolic
        calculations).
    \item The matplotlib~\cite{Droettboom2018} library (visualisation).
    \item The pandas~\cite{Structures2010}, dask~\cite{Dask2016} and
        NumPy~\cite{Oliphant2015} libraries (data manipulation).
    \item The SciPy~\cite{Jones2001} library (numerical integration of the
        replicator equation).
\end{itemize}

This work was performed using the computational facilities of the Advanced
Research Computing @ Cardiff (ARCCA) Division, Cardiff University.

\printbibliography

\newpage
\section*{Supplementary materials}

\includepdf{assets/pdf/proof_of_form_of_extortionate_strategies/main.pdf}

\newpage

Using the pair wise interactions the transition rates \(p,
q\) can be measured and the steady state probabilities inferred and compared to
the actual probabilities of each state.
This is done numerically by computing the singular eigenvector of the
matrix \(A\) \cite{Stewart2009}:

\[
    A =
    \begin{bmatrix}
        p_1 q_1 & p_1 (1 - q_1) & (1 - p_1) q_1 & (1 -p_1) (1 - q_1) \\
        p_2 q_2 & p_2 (1 - q_2) & (1 - p_2) q_2 & (1 -p_2) (1 - q_2) \\
        p_3 q_3 & p_3 (1 - q_3) & (1 - p_3) q_3 & (1 -p_3) (1 - q_3) \\
        p_4 q_4 & p_4 (1 - q_4) & (1 - p_4) q_4 & (1 -p_4) (1 - q_4) \\
    \end{bmatrix}
\]

Figure~\ref{fig:computed_probabilities_vs_theoretic_probabilities} shows a
regression line fitted to every pairwise interaction with a reported
\(\text{SSError}\) value (pairwise interactions with missing states were
omitted). This serves to validate the approach: a part from some edge cases the
relationship is consistent.

\begin{figure}[!htbp]
    \centering
    \includegraphics[width=.8\textwidth]{./assets/img/computed_probabilities_vs_theoretic_probabilities/main.pdf}
    \caption{The
        relationship between the steady state probabilities inferred from the
        measured transitions and the actual steady state probabilities. A linear
        regression line is included validating the approach.}
    \label{fig:computed_probabilities_vs_theoretic_probabilities}
\end{figure}


\end{document}

strategies.

The results of this analysis are shown in
Figure~\ref{fig:SSError_and_probabilities_in_full}. The top ranking strategies
by number of wins seem to be extortionate (but not against all strategies) and
it can be seen that a small sub group of strategies achieve mutual defection.
All the top ranking strategies according to score achieve mutual cooperation and
do not extort each other, however they
\textbf{do} exhibit extortionate behaviour towards a number of the lower ranking
strategies.

\begin{figure}[!htbp]
    \centering
    \includegraphics[width=.8\textwidth]{./assets/img/SSError_and_probabilities_in_full/main.pdf}
    \caption{\(\text{SSError}\) for the strategies for the full tournament. Only
    strategy interactions for which \(p_4=0\) and \(\chi>1\) are displayed.}
    \label{fig:SSError_and_probabilities_in_full}
\end{figure}

\section{Conclusion}\label{sec:conclusion}

This work defines an approach to measure whether or not a player is playing a
strategy that corresponds to an extortionate strategy as defined
in~\cite{Press2012}: a mathematical model for suspicion. Indeed, all
extortionate strategies have been
 classified as lying on a triangular plane.
This rigorous classification fails to be robust to small measurement error, thus
a statistical approach is proposed.
This is done through a linear algebraic approach for approximating the solution
of a linear system. Using this, a large number of pairwise interactions is
simulated and in fact very few strategies are found to act extortionately.

The work of~\cite{Press2012}, whilst showing that a clever approach to taking
advantage of another memory one strategy exists: this is incomplete. Whilst the
elegance of this result is very attractive, just as the simplicity of the
victory of Tit For Tat in Axelrod's original tournaments was, it is incomplete.
Extortionate strategies achieve a high number of wins but they do not
achieve a high score which corresponds to the fitness landscape in an
evolutionary sense. From the large number of interactions a payoff matrix \(S\)
can be measured where \(S_{ij}\) denotes the score (using standard values of
\((R, S, T, P) = (3, 0, 5, 1)\)) of the \(i\)th strategy
against the \(j\)th strategy. Using this, the replicator equation
describes the evolution of the system based on a population density fitness
function:

\begin{equation}\label{eqn:replicator_dynamics}
    \frac{dx}{dt} = x(S-x^TS x)
\end{equation}

Equation (\ref{eqn:replicator_dynamics}) is solved numerically through an
integration technique described in~\cite{Petzold1983} and
Figure~\ref{fig:replicator_dynamics} shows the evolution of the distribution of
the system: the various strategies are ranked by scores. It is clear to see that
only the high ranking strategies survive the evolutionary process (in fact,
only \documentclass[a4paper]{article}

\usepackage{amsmath}
\usepackage{amssymb}
\usepackage[margin=1.5cm,
            includefoot,
            footskip=30pt]{geometry}
\usepackage{layout}
\usepackage{graphicx}
\usepackage{subcaption}

\usepackage{biblatex}
\usepackage{pdfpages}

\bibliography{main.bib}

\title{Suspicion: Recognising and evaluating the effectiveness
       of extortion in the Iterated Prisoner's Dilemma}
\author{Vincent A. Knight \and Nikoleta E. Glynatsi}
\date{\today}



\begin{document}

\maketitle

\begin{abstract}
    The Iterated Prisoner's Dilemma is a model for rational and evolutionary
    interactive behaviour. It has applications both in the study of human social
    behaviour as well as in biology.
    It is used to understand when and how a rational individual might
    accept an immediate cost to their own utility for the direct benefit of
    another.

    Much attention has been given to a class of strategies called
    Zero Determinant strategies. It has been theoretically shown that these
    strategies can ``extort'' any player.

    In this work, an approach to identify if observed strategies are playing in
    an extortionate way is described. Furthermore, experimental analysis of
    a large tournament with \input{assets/tex/number_of_full_strategies/main.tex}
    strategies is considered. In this setting
    the most highly performing strategies do not play in an extortionate way
    against each other but do against lower performing strategies.
    This suggests that whilst the theory of Zero Determinant strategies
    indicates that memory is not of fundamental importance to the evolution of
    cooperative behaviour, this is incomplete.
\end{abstract}

\section{Introduction}\label{sec:introduction}

Agent based game theoretic models have become a stalwart of the underpinning
mathematics of interactive behaviours. One of the major pieces of work
in this area is the pair of original computer tournaments run by Robert
Axelrod~\cite{Axelrod1980, Axelrod1980a}. These tournaments pitted submitted
computer strategies against each other in plays of the Iterated Prisoner's
Dilemma. A common game where agents can choose to pay a slight cost to their
immediate utility in the hope of building a reputation. This has been used in
economic and evolutionary game theory to understand the evolution of cooperative
behaviour.

Recently, a class of strategies was described in~\cite{Press2012} that can
provably extort any given opponent. In~\cite{Hilbe2013, Moran1707} some
questions have already been asked about the true effectiveness of these
strategies in an evolutionary setting. Here another question is asked: is it
possible to recognise this extortionate behaviour? A mathematical procedure for
suspicion is presented: in the same way that the continued actions of an
extortionate individual might raise suspicion.

This work makes use of the Axelrod Python library~\cite{Knight2018, Knight2016}
with a large number of Prisoner Dilemma strategies available to give an
extensive numerical example of the ideas presented.  The approach is presented
in Section~\ref{sec:delta-zd-strategies}.  All of the code and data discussed
in Section~\ref{sec:numerical-experiments} is open sourced, archived and
written according to best scientific principles~\cite{Wilson2014}. The data
archive can be found at~\cite{vincent_knight_2018_1297075}.

\section{Recognising Extortion}\label{sec:delta-zd-strategies}

In~\cite{Press2012}, given a match between 2 memory-one strategies, the concept
of Zero Determinant (ZD) strategies is introduced. The main result of that paper
shows that given two memory one players \(p, q\in\mathbb{R}^4\) a linear
relationship between the players' scores could be forced by one of the players.

Using the notation of~\cite{Press2012}, assuming the utilities for player \(p\)
are given by \(S_x=(R, S, T, P)\) and for player \(q\) by \(S_y=(R, T, S, P)\)
and that the stationary scores of each player is given by \(S_X\) and \(S_Y\)
respectively. The main result of~\cite{Press2012} is that if

\begin{equation}\label{eqn:linear_relationship_for_p}
    \tilde p=\alpha S_x + \beta S_y + \gamma
\end{equation}

or

\begin{equation}\label{eqn:linear_relationship_for_q}
    \tilde q=\alpha S_x + \beta S_y + \gamma
\end{equation}

where \(\tilde p = (1 - p_1, 1 - p_2, p_3, p_4)\) and
\(\tilde q = (1 - q_1, 1 - q_2, q_3, q_4)\) then:

\begin{equation}
    \alpha S_X + \beta S_Y + \gamma = 0
\end{equation}

In~\cite{Press2012} a particular type of ZD strategy is defined: extortionate
strategies. If:

\begin{equation}\label{eqn:constraint_for_extortion}
    \gamma = - P(\alpha + \beta)
\end{equation}

then the player can ensure they get a score \(\chi\) times
larger than the opponent. This extortion coefficient is given by:

\begin{equation}\label{eqn:definition_of_chi}
    \chi=\frac{-\beta}{\alpha}
\end{equation}

Thus, if (\ref{eqn:constraint_for_extortion}) holds and \(\chi >1\) a player is
said to extort their opponent.
Here, the reverse problem is considered: given a
\(p\in\mathbb{R}^4\) how does one identify \(\alpha, \beta\) if they
exist and is the strategy in fact acting in an extortionate way?

These conditions correspond to:

\begin{align}
    \tilde p_1 & = \alpha R + \beta R - P (\alpha + \beta)
            \label{eqn:condition_for_tilde_p1}\\
    \tilde p_2 & = \alpha S + \beta T - P (\alpha + \beta)
            \label{eqn:condition_for_tilde_p2}\\
    \tilde p_3 & = \alpha T + \beta S - P (\alpha + \beta)
            \label{eqn:condition_for_tilde_p3}\\
    \tilde p_4 & = \alpha P + \beta P - P (\alpha + \beta)
            \label{eqn:condition_for_tilde_p4}
\end{align}

Equation (\ref{eqn:condition_for_tilde_p4}) ensures that \(p_4=\tilde p_4=0\).
Equations (\ref{eqn:condition_for_tilde_p1}-\ref{eqn:condition_for_tilde_p3})
can be used to eliminate \(\alpha, \beta\), giving:

\begin{equation}\label{eqn:planar_definition_of_extortion}
    \tilde p_1 = \frac{(R - P)(\tilde p_2 + \tilde p_3)}{S + T - 2P}
\end{equation}

with:

\begin{equation}\label{eqn:definition_of_chi}
    \chi = \frac{\tilde p_2 (P - T) + \tilde p_3 (S - P)}
                {\tilde p_2 (P - S) + \tilde p_3 (T - P)}
\end{equation}

Given a strategy \(p\in\mathbb{R}^{4\times 1}\) equations
(\ref{eqn:condition_for_tilde_p4}), (\ref{eqn:planar_definition_of_extortion}-\ref{eqn:definition_of_chi}) can be used to check if
a strategy is extortionate. The conditions correspond to:

\begin{align}
    p_1 & = \frac{(R-P)(p_2 + p_3) - R + T + S - P}{S + T - 2P}
     \label{eqn:condition_for_p1}\\
    p_4 & = 0 \label{eqn:condition_for_p4}\\
    1 & > p_2 + p_3\label{eqn:condition_for_chi}
\end{align}

The algebraic steps necessary to prove these results are available in the
supporting materials.

All extortionate strategies reside on a triangular (\ref{eqn:condition_for_chi})
plane (\ref{eqn:condition_for_p1}) in 3 dimensions (\ref{eqn:condition_for_p4}).
Using this formulation it can be seen that a necessary (but not sufficient)
condition for an extortionate strategy is that it cooperates on average less
than 50\% of the time when in a state of disagreement with the opponent.

As an example, consider the known extortionate strategy \(p=(8 / 9, 1 / 2, 1 /
3, 0)\) from~\cite{Stewart2012} which is referred to as \texttt{Extort-2}. In
this case, for the standard values of \((R, T, S, P)\) constraint
(\ref{eqn:condition_for_p1}) corresponds to:

\begin{equation}
    p_1 = \frac{2(p_2 + p_3) + 1}{3}
\end{equation}

It is clear that in this case all constraints hold.

This approach could in fact be used to confirm that a given strategy is acting
in an extortionate manner even if it is not a memory one strategy. However, in
practice, if a closed form for \(p\) is not known, then due to measurement
and/or numerical error this would not work.

This problem can be written in the following linear algebraic form where
\(x=(\alpha, \beta)\)
and \(p^*=(\tilde p_1 - 1, tilde_2 - 1, p_3)\):

\begin{equation}\label{eqn:linear_algebraic_equation_for_p}
    Cx= p^*
\end{equation}

\(C\) corresponds to equations
(\ref{eqn:condition_for_tilde_p1}-\ref{eqn:condition_for_tilde_p3}) and is
given by:

\begin{equation}\label{eqn:definition_of_C}
    C =
    \begin{bmatrix}
        R - P & R- P \\
        S - P & T- P \\
        T - P & S- P \\
    \end{bmatrix}
\end{equation}

Note that in general, equation (\ref{eqn:linear_algebraic_equation_for_p}) will
not necessarily have a solution. From the Rouch\'{e}-Capelli theorem if there is
a solution it is unique as \(\text{rank}(C)=2\) which is the dimension of the
variable \(x\). The best fitting \(x\) is found by minimizing:

\begin{equation}\label{eqn:r_squared}
    \text{SSError} = \|C x- p^*\|_2^2 = \sum_{i=1}^{3}\left((C\bar x)_i-p_i^*\right)^2
\end{equation}

Note that \(\text{SSError}\), which is the square of the Frobenius
norm~\cite{Golub2013}, becomes a measure of how close a strategy is to being an
extortionate strategy. Suspicion
of extortion then corresponds to a threshold on \(\text{SSError}\).

By observing interactions (human or otherwise), their memory one representation
can be inferred and this approach can be used to recognise extortionate
behaviour. The notion of comparing theoretic and actual plays of the IPD is not
novel, see for example~\cite{Rand2013}. Immediately it is noted that if the
environment is noisy~\cite{Wu1995} then no strategy can be considered to be
extortionate as \(p_4>0\).

In the next section, this idea will be illustrated by observing the interactions
that take place in a computer based tournament of the IPD\@.

\section{Numerical experiments}\label{sec:numerical-experiments}

In~\cite{Stewart2012} results from a tournament with
\input{./assets/tex/number_of_stewart_plotkin_strategies/main.tex} strategies,
was presented with specific consideration given to ZD strategies. This
tournament is reproduced here using the Axelrod-Python
project~\cite{Knight2016}. To obtain a good measure of the corresponding
transition rates for each strategy all matches have been run for
\input{assets/tex/number_of_turns/main.tex} turns and every match has been
repeated \input{assets/tex/number_of_repetitions/main.tex} times. All of this
interaction data is available at~\cite{vincent_knight_2018_1297075}. A good
match between the inferred Markov chain and the state distribution of the actual
interactions has been verified. Data for this is presented in the supplementary
materials.

Figure~\ref{fig:SSError_overall_in_stewart_plotkin} shows the \(\text{SSError}\)
values for all the strategies in the tournament, as reported
in~\cite{Stewart2012} the extortionate strategy (which has an expected
\(\text{SSError}\) approximately 0) gains a large number of wins.

\begin{figure}[!htbp]
    \centering
    \includegraphics[width=.8\textwidth]{./assets/img/SSError_overall_in_stewart_plotkin/main.pdf}
    \caption{\(\text{SSError}\) and state probabilities for the strategies
        of~\cite{Stewart2012}, ordered both by number of wins and overall score.
        Note that \(P(DC)\) is not shown as it corresponds to the transpose of
        \(P(CD)\). Cooperator and Defector are omitted as they do not visit all
        the states.}
    \label{fig:SSError_overall_in_stewart_plotkin}
\end{figure}

Here, the work of~\cite{Stewart2012} is extended by investigating a tournament
with \input{assets/tex/number_of_full_strategies/main.tex}
strategies.

The results of this analysis are shown in
Figure~\ref{fig:SSError_and_probabilities_in_full}. The top ranking strategies
by number of wins seem to be extortionate (but not against all strategies) and
it can be seen that a small sub group of strategies achieve mutual defection.
All the top ranking strategies according to score achieve mutual cooperation and
do not extort each other, however they
\textbf{do} exhibit extortionate behaviour towards a number of the lower ranking
strategies.

\begin{figure}[!htbp]
    \centering
    \includegraphics[width=.8\textwidth]{./assets/img/SSError_and_probabilities_in_full/main.pdf}
    \caption{\(\text{SSError}\) for the strategies for the full tournament. Only
    strategy interactions for which \(p_4=0\) and \(\chi>1\) are displayed.}
    \label{fig:SSError_and_probabilities_in_full}
\end{figure}

\section{Conclusion}\label{sec:conclusion}

This work defines an approach to measure whether or not a player is playing a
strategy that corresponds to an extortionate strategy as defined
in~\cite{Press2012}: a mathematical model for suspicion. Indeed, all
extortionate strategies have been
 classified as lying on a triangular plane.
This rigorous classification fails to be robust to small measurement error, thus
a statistical approach is proposed.
This is done through a linear algebraic approach for approximating the solution
of a linear system. Using this, a large number of pairwise interactions is
simulated and in fact very few strategies are found to act extortionately.

The work of~\cite{Press2012}, whilst showing that a clever approach to taking
advantage of another memory one strategy exists: this is incomplete. Whilst the
elegance of this result is very attractive, just as the simplicity of the
victory of Tit For Tat in Axelrod's original tournaments was, it is incomplete.
Extortionate strategies achieve a high number of wins but they do not
achieve a high score which corresponds to the fitness landscape in an
evolutionary sense. From the large number of interactions a payoff matrix \(S\)
can be measured where \(S_{ij}\) denotes the score (using standard values of
\((R, S, T, P) = (3, 0, 5, 1)\)) of the \(i\)th strategy
against the \(j\)th strategy. Using this, the replicator equation
describes the evolution of the system based on a population density fitness
function:

\begin{equation}\label{eqn:replicator_dynamics}
    \frac{dx}{dt} = x(S-x^TS x)
\end{equation}

Equation (\ref{eqn:replicator_dynamics}) is solved numerically through an
integration technique described in~\cite{Petzold1983} and
Figure~\ref{fig:replicator_dynamics} shows the evolution of the distribution of
the system: the various strategies are ranked by scores. It is clear to see that
only the high ranking strategies survive the evolutionary process (in fact,
only \input{./assets/img/replicator_dynamics/main.tex}
have a final distribution greater than \(10 ^ {-2}\)). This confirms the
findings of~\cite{Moran1707} in which sophisticated strategies resist
evolutionary invasion of shorter memory strategies. Recalling
Figure~\ref{fig:SSError_and_probabilities_in_full} this demonstrates that:

\begin{itemize}
    \item Cooperation emerges through the evolutionary process: the high scoring
        strategies do not exhibit extortionate behaviour towards each other.
    \item Extortionate strategies do not survive the evolutionary process.
\end{itemize}

\begin{figure}[!htbp]
    \centering
    \includegraphics[width=.8\textwidth]{./assets/img/replicator_dynamics/main.pdf}
    \caption{Numerical simulation of the replicator equation
    (\ref{eqn:replicator_dynamics}): strategies are ordered by score, only the strategies with a high score survive the evolutionary process.}
    \label{fig:replicator_dynamics}
\end{figure}

This work can be used to classify plays of the IPD\@: data can be collected from
actual interactions (in lab or in the field). Furthermore, this allows for a
classification method similar to the notion of fingerprinting presented
in~\cite{Ashlock2008}. Trained strategies can potentially be classified as
extortionate or not or it could be possible to even constrain the reinforcement
learning approaches that are becoming prevalent in the literature.
Alternatively, this mathematical approach for recognising extortion could be
used in sophisticated strategies to defend against invasion. Arguably, some of
the strategies considered here exhibit this behaviour, indeed as described
in~\cite{Harper2017}, the top ranking strategies in the full tournament are
obtained using evolutionary reinforcement learning techniques, thus, suspicion
of extortionate behaviour could in fact be an evolutionary trait.

\section*{Acknowledgements}

The following open source software libraries were used in this research:

\begin{itemize}
    \item The Axelrod ~\cite{Knight2016, Knight2018} library (IPD strategies and
        tournaments).
    \item The sympy library~\cite{Meurer2017} (verification of all symbolic
        calculations).
    \item The matplotlib~\cite{Droettboom2018} library (visualisation).
    \item The pandas~\cite{Structures2010}, dask~\cite{Dask2016} and
        NumPy~\cite{Oliphant2015} libraries (data manipulation).
    \item The SciPy~\cite{Jones2001} library (numerical integration of the
        replicator equation).
\end{itemize}

This work was performed using the computational facilities of the Advanced
Research Computing @ Cardiff (ARCCA) Division, Cardiff University.

\printbibliography

\newpage
\section*{Supplementary materials}

\includepdf{assets/pdf/proof_of_form_of_extortionate_strategies/main.pdf}

\newpage

Using the pair wise interactions the transition rates \(p,
q\) can be measured and the steady state probabilities inferred and compared to
the actual probabilities of each state.
This is done numerically by computing the singular eigenvector of the
matrix \(A\) \cite{Stewart2009}:

\[
    A =
    \begin{bmatrix}
        p_1 q_1 & p_1 (1 - q_1) & (1 - p_1) q_1 & (1 -p_1) (1 - q_1) \\
        p_2 q_2 & p_2 (1 - q_2) & (1 - p_2) q_2 & (1 -p_2) (1 - q_2) \\
        p_3 q_3 & p_3 (1 - q_3) & (1 - p_3) q_3 & (1 -p_3) (1 - q_3) \\
        p_4 q_4 & p_4 (1 - q_4) & (1 - p_4) q_4 & (1 -p_4) (1 - q_4) \\
    \end{bmatrix}
\]

Figure~\ref{fig:computed_probabilities_vs_theoretic_probabilities} shows a
regression line fitted to every pairwise interaction with a reported
\(\text{SSError}\) value (pairwise interactions with missing states were
omitted). This serves to validate the approach: a part from some edge cases the
relationship is consistent.

\begin{figure}[!htbp]
    \centering
    \includegraphics[width=.8\textwidth]{./assets/img/computed_probabilities_vs_theoretic_probabilities/main.pdf}
    \caption{The
        relationship between the steady state probabilities inferred from the
        measured transitions and the actual steady state probabilities. A linear
        regression line is included validating the approach.}
    \label{fig:computed_probabilities_vs_theoretic_probabilities}
\end{figure}


\end{document}

have a final distribution greater than \(10 ^ {-2}\)). This confirms the
findings of~\cite{Moran1707} in which sophisticated strategies resist
evolutionary invasion of shorter memory strategies. Recalling
Figure~\ref{fig:SSError_and_probabilities_in_full} this demonstrates that:

\begin{itemize}
    \item Cooperation emerges through the evolutionary process: the high scoring
        strategies do not exhibit extortionate behaviour towards each other.
    \item Extortionate strategies do not survive the evolutionary process.
\end{itemize}

\begin{figure}[!htbp]
    \centering
    \includegraphics[width=.8\textwidth]{./assets/img/replicator_dynamics/main.pdf}
    \caption{Numerical simulation of the replicator equation
    (\ref{eqn:replicator_dynamics}): strategies are ordered by score, only the strategies with a high score survive the evolutionary process.}
    \label{fig:replicator_dynamics}
\end{figure}

This work can be used to classify plays of the IPD\@: data can be collected from
actual interactions (in lab or in the field). Furthermore, this allows for a
classification method similar to the notion of fingerprinting presented
in~\cite{Ashlock2008}. Trained strategies can potentially be classified as
extortionate or not or it could be possible to even constrain the reinforcement
learning approaches that are becoming prevalent in the literature.
Alternatively, this mathematical approach for recognising extortion could be
used in sophisticated strategies to defend against invasion. Arguably, some of
the strategies considered here exhibit this behaviour, indeed as described
in~\cite{Harper2017}, the top ranking strategies in the full tournament are
obtained using evolutionary reinforcement learning techniques, thus, suspicion
of extortionate behaviour could in fact be an evolutionary trait.

\section*{Acknowledgements}

The following open source software libraries were used in this research:

\begin{itemize}
    \item The Axelrod ~\cite{Knight2016, Knight2018} library (IPD strategies and
        tournaments).
    \item The sympy library~\cite{Meurer2017} (verification of all symbolic
        calculations).
    \item The matplotlib~\cite{Droettboom2018} library (visualisation).
    \item The pandas~\cite{Structures2010}, dask~\cite{Dask2016} and
        NumPy~\cite{Oliphant2015} libraries (data manipulation).
    \item The SciPy~\cite{Jones2001} library (numerical integration of the
        replicator equation).
\end{itemize}

This work was performed using the computational facilities of the Advanced
Research Computing @ Cardiff (ARCCA) Division, Cardiff University.

\printbibliography

\newpage
\section*{Supplementary materials}

\includepdf{assets/pdf/proof_of_form_of_extortionate_strategies/main.pdf}

\newpage

Using the pair wise interactions the transition rates \(p,
q\) can be measured and the steady state probabilities inferred and compared to
the actual probabilities of each state.
This is done numerically by computing the singular eigenvector of the
matrix \(A\) \cite{Stewart2009}:

\[
    A =
    \begin{bmatrix}
        p_1 q_1 & p_1 (1 - q_1) & (1 - p_1) q_1 & (1 -p_1) (1 - q_1) \\
        p_2 q_2 & p_2 (1 - q_2) & (1 - p_2) q_2 & (1 -p_2) (1 - q_2) \\
        p_3 q_3 & p_3 (1 - q_3) & (1 - p_3) q_3 & (1 -p_3) (1 - q_3) \\
        p_4 q_4 & p_4 (1 - q_4) & (1 - p_4) q_4 & (1 -p_4) (1 - q_4) \\
    \end{bmatrix}
\]

Figure~\ref{fig:computed_probabilities_vs_theoretic_probabilities} shows a
regression line fitted to every pairwise interaction with a reported
\(\text{SSError}\) value (pairwise interactions with missing states were
omitted). This serves to validate the approach: a part from some edge cases the
relationship is consistent.

\begin{figure}[!htbp]
    \centering
    \includegraphics[width=.8\textwidth]{./assets/img/computed_probabilities_vs_theoretic_probabilities/main.pdf}
    \caption{The
        relationship between the steady state probabilities inferred from the
        measured transitions and the actual steady state probabilities. A linear
        regression line is included validating the approach.}
    \label{fig:computed_probabilities_vs_theoretic_probabilities}
\end{figure}


\end{document}

strategies.

The results of this analysis are shown in
Figure~\ref{fig:SSError_and_probabilities_in_full}. The top ranking strategies
by number of wins seem to be extortionate (but not against all strategies) and
it can be seen that a small sub group of strategies achieve mutual defection.
All the top ranking strategies according to score achieve mutual cooperation and
do not extort each other, however they
\textbf{do} exhibit extortionate behaviour towards a number of the lower ranking
strategies.

\begin{figure}[!htbp]
    \centering
    \includegraphics[width=.8\textwidth]{./assets/img/SSError_and_probabilities_in_full/main.pdf}
    \caption{\(\text{SSError}\) for the strategies for the full tournament. Only
    strategy interactions for which \(p_4=0\) and \(\chi>1\) are displayed.}
    \label{fig:SSError_and_probabilities_in_full}
\end{figure}

\section{Conclusion}\label{sec:conclusion}

This work defines an approach to measure whether or not a player is playing a
strategy that corresponds to an extortionate strategy as defined
in~\cite{Press2012}: a mathematical model for suspicion. Indeed, all
extortionate strategies have been
 classified as lying on a triangular plane.
This rigorous classification fails to be robust to small measurement error, thus
a statistical approach is proposed.
This is done through a linear algebraic approach for approximating the solution
of a linear system. Using this, a large number of pairwise interactions is
simulated and in fact very few strategies are found to act extortionately.

The work of~\cite{Press2012}, whilst showing that a clever approach to taking
advantage of another memory one strategy exists: this is incomplete. Whilst the
elegance of this result is very attractive, just as the simplicity of the
victory of Tit For Tat in Axelrod's original tournaments was, it is incomplete.
Extortionate strategies achieve a high number of wins but they do not
achieve a high score which corresponds to the fitness landscape in an
evolutionary sense. From the large number of interactions a payoff matrix \(S\)
can be measured where \(S_{ij}\) denotes the score (using standard values of
\((R, S, T, P) = (3, 0, 5, 1)\)) of the \(i\)th strategy
against the \(j\)th strategy. Using this, the replicator equation
describes the evolution of the system based on a population density fitness
function:

\begin{equation}\label{eqn:replicator_dynamics}
    \frac{dx}{dt} = x(S-x^TS x)
\end{equation}

Equation (\ref{eqn:replicator_dynamics}) is solved numerically through an
integration technique described in~\cite{Petzold1983} and
Figure~\ref{fig:replicator_dynamics} shows the evolution of the distribution of
the system: the various strategies are ranked by scores. It is clear to see that
only the high ranking strategies survive the evolutionary process (in fact,
only \documentclass[a4paper]{article}

\usepackage{amsmath}
\usepackage{amssymb}
\usepackage[margin=1.5cm,
            includefoot,
            footskip=30pt]{geometry}
\usepackage{layout}
\usepackage{graphicx}
\usepackage{subcaption}

\usepackage{biblatex}
\usepackage{pdfpages}

\bibliography{main.bib}

\title{Suspicion: Recognising and evaluating the effectiveness
       of extortion in the Iterated Prisoner's Dilemma}
\author{Vincent A. Knight \and Nikoleta E. Glynatsi}
\date{\today}



\begin{document}

\maketitle

\begin{abstract}
    The Iterated Prisoner's Dilemma is a model for rational and evolutionary
    interactive behaviour. It has applications both in the study of human social
    behaviour as well as in biology.
    It is used to understand when and how a rational individual might
    accept an immediate cost to their own utility for the direct benefit of
    another.

    Much attention has been given to a class of strategies called
    Zero Determinant strategies. It has been theoretically shown that these
    strategies can ``extort'' any player.

    In this work, an approach to identify if observed strategies are playing in
    an extortionate way is described. Furthermore, experimental analysis of
    a large tournament with \documentclass[a4paper]{article}

\usepackage{amsmath}
\usepackage{amssymb}
\usepackage[margin=1.5cm,
            includefoot,
            footskip=30pt]{geometry}
\usepackage{layout}
\usepackage{graphicx}
\usepackage{subcaption}

\usepackage{biblatex}
\usepackage{pdfpages}

\bibliography{main.bib}

\title{Suspicion: Recognising and evaluating the effectiveness
       of extortion in the Iterated Prisoner's Dilemma}
\author{Vincent A. Knight \and Nikoleta E. Glynatsi}
\date{\today}



\begin{document}

\maketitle

\begin{abstract}
    The Iterated Prisoner's Dilemma is a model for rational and evolutionary
    interactive behaviour. It has applications both in the study of human social
    behaviour as well as in biology.
    It is used to understand when and how a rational individual might
    accept an immediate cost to their own utility for the direct benefit of
    another.

    Much attention has been given to a class of strategies called
    Zero Determinant strategies. It has been theoretically shown that these
    strategies can ``extort'' any player.

    In this work, an approach to identify if observed strategies are playing in
    an extortionate way is described. Furthermore, experimental analysis of
    a large tournament with \input{assets/tex/number_of_full_strategies/main.tex}
    strategies is considered. In this setting
    the most highly performing strategies do not play in an extortionate way
    against each other but do against lower performing strategies.
    This suggests that whilst the theory of Zero Determinant strategies
    indicates that memory is not of fundamental importance to the evolution of
    cooperative behaviour, this is incomplete.
\end{abstract}

\section{Introduction}\label{sec:introduction}

Agent based game theoretic models have become a stalwart of the underpinning
mathematics of interactive behaviours. One of the major pieces of work
in this area is the pair of original computer tournaments run by Robert
Axelrod~\cite{Axelrod1980, Axelrod1980a}. These tournaments pitted submitted
computer strategies against each other in plays of the Iterated Prisoner's
Dilemma. A common game where agents can choose to pay a slight cost to their
immediate utility in the hope of building a reputation. This has been used in
economic and evolutionary game theory to understand the evolution of cooperative
behaviour.

Recently, a class of strategies was described in~\cite{Press2012} that can
provably extort any given opponent. In~\cite{Hilbe2013, Moran1707} some
questions have already been asked about the true effectiveness of these
strategies in an evolutionary setting. Here another question is asked: is it
possible to recognise this extortionate behaviour? A mathematical procedure for
suspicion is presented: in the same way that the continued actions of an
extortionate individual might raise suspicion.

This work makes use of the Axelrod Python library~\cite{Knight2018, Knight2016}
with a large number of Prisoner Dilemma strategies available to give an
extensive numerical example of the ideas presented.  The approach is presented
in Section~\ref{sec:delta-zd-strategies}.  All of the code and data discussed
in Section~\ref{sec:numerical-experiments} is open sourced, archived and
written according to best scientific principles~\cite{Wilson2014}. The data
archive can be found at~\cite{vincent_knight_2018_1297075}.

\section{Recognising Extortion}\label{sec:delta-zd-strategies}

In~\cite{Press2012}, given a match between 2 memory-one strategies, the concept
of Zero Determinant (ZD) strategies is introduced. The main result of that paper
shows that given two memory one players \(p, q\in\mathbb{R}^4\) a linear
relationship between the players' scores could be forced by one of the players.

Using the notation of~\cite{Press2012}, assuming the utilities for player \(p\)
are given by \(S_x=(R, S, T, P)\) and for player \(q\) by \(S_y=(R, T, S, P)\)
and that the stationary scores of each player is given by \(S_X\) and \(S_Y\)
respectively. The main result of~\cite{Press2012} is that if

\begin{equation}\label{eqn:linear_relationship_for_p}
    \tilde p=\alpha S_x + \beta S_y + \gamma
\end{equation}

or

\begin{equation}\label{eqn:linear_relationship_for_q}
    \tilde q=\alpha S_x + \beta S_y + \gamma
\end{equation}

where \(\tilde p = (1 - p_1, 1 - p_2, p_3, p_4)\) and
\(\tilde q = (1 - q_1, 1 - q_2, q_3, q_4)\) then:

\begin{equation}
    \alpha S_X + \beta S_Y + \gamma = 0
\end{equation}

In~\cite{Press2012} a particular type of ZD strategy is defined: extortionate
strategies. If:

\begin{equation}\label{eqn:constraint_for_extortion}
    \gamma = - P(\alpha + \beta)
\end{equation}

then the player can ensure they get a score \(\chi\) times
larger than the opponent. This extortion coefficient is given by:

\begin{equation}\label{eqn:definition_of_chi}
    \chi=\frac{-\beta}{\alpha}
\end{equation}

Thus, if (\ref{eqn:constraint_for_extortion}) holds and \(\chi >1\) a player is
said to extort their opponent.
Here, the reverse problem is considered: given a
\(p\in\mathbb{R}^4\) how does one identify \(\alpha, \beta\) if they
exist and is the strategy in fact acting in an extortionate way?

These conditions correspond to:

\begin{align}
    \tilde p_1 & = \alpha R + \beta R - P (\alpha + \beta)
            \label{eqn:condition_for_tilde_p1}\\
    \tilde p_2 & = \alpha S + \beta T - P (\alpha + \beta)
            \label{eqn:condition_for_tilde_p2}\\
    \tilde p_3 & = \alpha T + \beta S - P (\alpha + \beta)
            \label{eqn:condition_for_tilde_p3}\\
    \tilde p_4 & = \alpha P + \beta P - P (\alpha + \beta)
            \label{eqn:condition_for_tilde_p4}
\end{align}

Equation (\ref{eqn:condition_for_tilde_p4}) ensures that \(p_4=\tilde p_4=0\).
Equations (\ref{eqn:condition_for_tilde_p1}-\ref{eqn:condition_for_tilde_p3})
can be used to eliminate \(\alpha, \beta\), giving:

\begin{equation}\label{eqn:planar_definition_of_extortion}
    \tilde p_1 = \frac{(R - P)(\tilde p_2 + \tilde p_3)}{S + T - 2P}
\end{equation}

with:

\begin{equation}\label{eqn:definition_of_chi}
    \chi = \frac{\tilde p_2 (P - T) + \tilde p_3 (S - P)}
                {\tilde p_2 (P - S) + \tilde p_3 (T - P)}
\end{equation}

Given a strategy \(p\in\mathbb{R}^{4\times 1}\) equations
(\ref{eqn:condition_for_tilde_p4}), (\ref{eqn:planar_definition_of_extortion}-\ref{eqn:definition_of_chi}) can be used to check if
a strategy is extortionate. The conditions correspond to:

\begin{align}
    p_1 & = \frac{(R-P)(p_2 + p_3) - R + T + S - P}{S + T - 2P}
     \label{eqn:condition_for_p1}\\
    p_4 & = 0 \label{eqn:condition_for_p4}\\
    1 & > p_2 + p_3\label{eqn:condition_for_chi}
\end{align}

The algebraic steps necessary to prove these results are available in the
supporting materials.

All extortionate strategies reside on a triangular (\ref{eqn:condition_for_chi})
plane (\ref{eqn:condition_for_p1}) in 3 dimensions (\ref{eqn:condition_for_p4}).
Using this formulation it can be seen that a necessary (but not sufficient)
condition for an extortionate strategy is that it cooperates on average less
than 50\% of the time when in a state of disagreement with the opponent.

As an example, consider the known extortionate strategy \(p=(8 / 9, 1 / 2, 1 /
3, 0)\) from~\cite{Stewart2012} which is referred to as \texttt{Extort-2}. In
this case, for the standard values of \((R, T, S, P)\) constraint
(\ref{eqn:condition_for_p1}) corresponds to:

\begin{equation}
    p_1 = \frac{2(p_2 + p_3) + 1}{3}
\end{equation}

It is clear that in this case all constraints hold.

This approach could in fact be used to confirm that a given strategy is acting
in an extortionate manner even if it is not a memory one strategy. However, in
practice, if a closed form for \(p\) is not known, then due to measurement
and/or numerical error this would not work.

This problem can be written in the following linear algebraic form where
\(x=(\alpha, \beta)\)
and \(p^*=(\tilde p_1 - 1, tilde_2 - 1, p_3)\):

\begin{equation}\label{eqn:linear_algebraic_equation_for_p}
    Cx= p^*
\end{equation}

\(C\) corresponds to equations
(\ref{eqn:condition_for_tilde_p1}-\ref{eqn:condition_for_tilde_p3}) and is
given by:

\begin{equation}\label{eqn:definition_of_C}
    C =
    \begin{bmatrix}
        R - P & R- P \\
        S - P & T- P \\
        T - P & S- P \\
    \end{bmatrix}
\end{equation}

Note that in general, equation (\ref{eqn:linear_algebraic_equation_for_p}) will
not necessarily have a solution. From the Rouch\'{e}-Capelli theorem if there is
a solution it is unique as \(\text{rank}(C)=2\) which is the dimension of the
variable \(x\). The best fitting \(x\) is found by minimizing:

\begin{equation}\label{eqn:r_squared}
    \text{SSError} = \|C x- p^*\|_2^2 = \sum_{i=1}^{3}\left((C\bar x)_i-p_i^*\right)^2
\end{equation}

Note that \(\text{SSError}\), which is the square of the Frobenius
norm~\cite{Golub2013}, becomes a measure of how close a strategy is to being an
extortionate strategy. Suspicion
of extortion then corresponds to a threshold on \(\text{SSError}\).

By observing interactions (human or otherwise), their memory one representation
can be inferred and this approach can be used to recognise extortionate
behaviour. The notion of comparing theoretic and actual plays of the IPD is not
novel, see for example~\cite{Rand2013}. Immediately it is noted that if the
environment is noisy~\cite{Wu1995} then no strategy can be considered to be
extortionate as \(p_4>0\).

In the next section, this idea will be illustrated by observing the interactions
that take place in a computer based tournament of the IPD\@.

\section{Numerical experiments}\label{sec:numerical-experiments}

In~\cite{Stewart2012} results from a tournament with
\input{./assets/tex/number_of_stewart_plotkin_strategies/main.tex} strategies,
was presented with specific consideration given to ZD strategies. This
tournament is reproduced here using the Axelrod-Python
project~\cite{Knight2016}. To obtain a good measure of the corresponding
transition rates for each strategy all matches have been run for
\input{assets/tex/number_of_turns/main.tex} turns and every match has been
repeated \input{assets/tex/number_of_repetitions/main.tex} times. All of this
interaction data is available at~\cite{vincent_knight_2018_1297075}. A good
match between the inferred Markov chain and the state distribution of the actual
interactions has been verified. Data for this is presented in the supplementary
materials.

Figure~\ref{fig:SSError_overall_in_stewart_plotkin} shows the \(\text{SSError}\)
values for all the strategies in the tournament, as reported
in~\cite{Stewart2012} the extortionate strategy (which has an expected
\(\text{SSError}\) approximately 0) gains a large number of wins.

\begin{figure}[!htbp]
    \centering
    \includegraphics[width=.8\textwidth]{./assets/img/SSError_overall_in_stewart_plotkin/main.pdf}
    \caption{\(\text{SSError}\) and state probabilities for the strategies
        of~\cite{Stewart2012}, ordered both by number of wins and overall score.
        Note that \(P(DC)\) is not shown as it corresponds to the transpose of
        \(P(CD)\). Cooperator and Defector are omitted as they do not visit all
        the states.}
    \label{fig:SSError_overall_in_stewart_plotkin}
\end{figure}

Here, the work of~\cite{Stewart2012} is extended by investigating a tournament
with \input{assets/tex/number_of_full_strategies/main.tex}
strategies.

The results of this analysis are shown in
Figure~\ref{fig:SSError_and_probabilities_in_full}. The top ranking strategies
by number of wins seem to be extortionate (but not against all strategies) and
it can be seen that a small sub group of strategies achieve mutual defection.
All the top ranking strategies according to score achieve mutual cooperation and
do not extort each other, however they
\textbf{do} exhibit extortionate behaviour towards a number of the lower ranking
strategies.

\begin{figure}[!htbp]
    \centering
    \includegraphics[width=.8\textwidth]{./assets/img/SSError_and_probabilities_in_full/main.pdf}
    \caption{\(\text{SSError}\) for the strategies for the full tournament. Only
    strategy interactions for which \(p_4=0\) and \(\chi>1\) are displayed.}
    \label{fig:SSError_and_probabilities_in_full}
\end{figure}

\section{Conclusion}\label{sec:conclusion}

This work defines an approach to measure whether or not a player is playing a
strategy that corresponds to an extortionate strategy as defined
in~\cite{Press2012}: a mathematical model for suspicion. Indeed, all
extortionate strategies have been
 classified as lying on a triangular plane.
This rigorous classification fails to be robust to small measurement error, thus
a statistical approach is proposed.
This is done through a linear algebraic approach for approximating the solution
of a linear system. Using this, a large number of pairwise interactions is
simulated and in fact very few strategies are found to act extortionately.

The work of~\cite{Press2012}, whilst showing that a clever approach to taking
advantage of another memory one strategy exists: this is incomplete. Whilst the
elegance of this result is very attractive, just as the simplicity of the
victory of Tit For Tat in Axelrod's original tournaments was, it is incomplete.
Extortionate strategies achieve a high number of wins but they do not
achieve a high score which corresponds to the fitness landscape in an
evolutionary sense. From the large number of interactions a payoff matrix \(S\)
can be measured where \(S_{ij}\) denotes the score (using standard values of
\((R, S, T, P) = (3, 0, 5, 1)\)) of the \(i\)th strategy
against the \(j\)th strategy. Using this, the replicator equation
describes the evolution of the system based on a population density fitness
function:

\begin{equation}\label{eqn:replicator_dynamics}
    \frac{dx}{dt} = x(S-x^TS x)
\end{equation}

Equation (\ref{eqn:replicator_dynamics}) is solved numerically through an
integration technique described in~\cite{Petzold1983} and
Figure~\ref{fig:replicator_dynamics} shows the evolution of the distribution of
the system: the various strategies are ranked by scores. It is clear to see that
only the high ranking strategies survive the evolutionary process (in fact,
only \input{./assets/img/replicator_dynamics/main.tex}
have a final distribution greater than \(10 ^ {-2}\)). This confirms the
findings of~\cite{Moran1707} in which sophisticated strategies resist
evolutionary invasion of shorter memory strategies. Recalling
Figure~\ref{fig:SSError_and_probabilities_in_full} this demonstrates that:

\begin{itemize}
    \item Cooperation emerges through the evolutionary process: the high scoring
        strategies do not exhibit extortionate behaviour towards each other.
    \item Extortionate strategies do not survive the evolutionary process.
\end{itemize}

\begin{figure}[!htbp]
    \centering
    \includegraphics[width=.8\textwidth]{./assets/img/replicator_dynamics/main.pdf}
    \caption{Numerical simulation of the replicator equation
    (\ref{eqn:replicator_dynamics}): strategies are ordered by score, only the strategies with a high score survive the evolutionary process.}
    \label{fig:replicator_dynamics}
\end{figure}

This work can be used to classify plays of the IPD\@: data can be collected from
actual interactions (in lab or in the field). Furthermore, this allows for a
classification method similar to the notion of fingerprinting presented
in~\cite{Ashlock2008}. Trained strategies can potentially be classified as
extortionate or not or it could be possible to even constrain the reinforcement
learning approaches that are becoming prevalent in the literature.
Alternatively, this mathematical approach for recognising extortion could be
used in sophisticated strategies to defend against invasion. Arguably, some of
the strategies considered here exhibit this behaviour, indeed as described
in~\cite{Harper2017}, the top ranking strategies in the full tournament are
obtained using evolutionary reinforcement learning techniques, thus, suspicion
of extortionate behaviour could in fact be an evolutionary trait.

\section*{Acknowledgements}

The following open source software libraries were used in this research:

\begin{itemize}
    \item The Axelrod ~\cite{Knight2016, Knight2018} library (IPD strategies and
        tournaments).
    \item The sympy library~\cite{Meurer2017} (verification of all symbolic
        calculations).
    \item The matplotlib~\cite{Droettboom2018} library (visualisation).
    \item The pandas~\cite{Structures2010}, dask~\cite{Dask2016} and
        NumPy~\cite{Oliphant2015} libraries (data manipulation).
    \item The SciPy~\cite{Jones2001} library (numerical integration of the
        replicator equation).
\end{itemize}

This work was performed using the computational facilities of the Advanced
Research Computing @ Cardiff (ARCCA) Division, Cardiff University.

\printbibliography

\newpage
\section*{Supplementary materials}

\includepdf{assets/pdf/proof_of_form_of_extortionate_strategies/main.pdf}

\newpage

Using the pair wise interactions the transition rates \(p,
q\) can be measured and the steady state probabilities inferred and compared to
the actual probabilities of each state.
This is done numerically by computing the singular eigenvector of the
matrix \(A\) \cite{Stewart2009}:

\[
    A =
    \begin{bmatrix}
        p_1 q_1 & p_1 (1 - q_1) & (1 - p_1) q_1 & (1 -p_1) (1 - q_1) \\
        p_2 q_2 & p_2 (1 - q_2) & (1 - p_2) q_2 & (1 -p_2) (1 - q_2) \\
        p_3 q_3 & p_3 (1 - q_3) & (1 - p_3) q_3 & (1 -p_3) (1 - q_3) \\
        p_4 q_4 & p_4 (1 - q_4) & (1 - p_4) q_4 & (1 -p_4) (1 - q_4) \\
    \end{bmatrix}
\]

Figure~\ref{fig:computed_probabilities_vs_theoretic_probabilities} shows a
regression line fitted to every pairwise interaction with a reported
\(\text{SSError}\) value (pairwise interactions with missing states were
omitted). This serves to validate the approach: a part from some edge cases the
relationship is consistent.

\begin{figure}[!htbp]
    \centering
    \includegraphics[width=.8\textwidth]{./assets/img/computed_probabilities_vs_theoretic_probabilities/main.pdf}
    \caption{The
        relationship between the steady state probabilities inferred from the
        measured transitions and the actual steady state probabilities. A linear
        regression line is included validating the approach.}
    \label{fig:computed_probabilities_vs_theoretic_probabilities}
\end{figure}


\end{document}

    strategies is considered. In this setting
    the most highly performing strategies do not play in an extortionate way
    against each other but do against lower performing strategies.
    This suggests that whilst the theory of Zero Determinant strategies
    indicates that memory is not of fundamental importance to the evolution of
    cooperative behaviour, this is incomplete.
\end{abstract}

\section{Introduction}\label{sec:introduction}

Agent based game theoretic models have become a stalwart of the underpinning
mathematics of interactive behaviours. One of the major pieces of work
in this area is the pair of original computer tournaments run by Robert
Axelrod~\cite{Axelrod1980, Axelrod1980a}. These tournaments pitted submitted
computer strategies against each other in plays of the Iterated Prisoner's
Dilemma. A common game where agents can choose to pay a slight cost to their
immediate utility in the hope of building a reputation. This has been used in
economic and evolutionary game theory to understand the evolution of cooperative
behaviour.

Recently, a class of strategies was described in~\cite{Press2012} that can
provably extort any given opponent. In~\cite{Hilbe2013, Moran1707} some
questions have already been asked about the true effectiveness of these
strategies in an evolutionary setting. Here another question is asked: is it
possible to recognise this extortionate behaviour? A mathematical procedure for
suspicion is presented: in the same way that the continued actions of an
extortionate individual might raise suspicion.

This work makes use of the Axelrod Python library~\cite{Knight2018, Knight2016}
with a large number of Prisoner Dilemma strategies available to give an
extensive numerical example of the ideas presented.  The approach is presented
in Section~\ref{sec:delta-zd-strategies}.  All of the code and data discussed
in Section~\ref{sec:numerical-experiments} is open sourced, archived and
written according to best scientific principles~\cite{Wilson2014}. The data
archive can be found at~\cite{vincent_knight_2018_1297075}.

\section{Recognising Extortion}\label{sec:delta-zd-strategies}

In~\cite{Press2012}, given a match between 2 memory-one strategies, the concept
of Zero Determinant (ZD) strategies is introduced. The main result of that paper
shows that given two memory one players \(p, q\in\mathbb{R}^4\) a linear
relationship between the players' scores could be forced by one of the players.

Using the notation of~\cite{Press2012}, assuming the utilities for player \(p\)
are given by \(S_x=(R, S, T, P)\) and for player \(q\) by \(S_y=(R, T, S, P)\)
and that the stationary scores of each player is given by \(S_X\) and \(S_Y\)
respectively. The main result of~\cite{Press2012} is that if

\begin{equation}\label{eqn:linear_relationship_for_p}
    \tilde p=\alpha S_x + \beta S_y + \gamma
\end{equation}

or

\begin{equation}\label{eqn:linear_relationship_for_q}
    \tilde q=\alpha S_x + \beta S_y + \gamma
\end{equation}

where \(\tilde p = (1 - p_1, 1 - p_2, p_3, p_4)\) and
\(\tilde q = (1 - q_1, 1 - q_2, q_3, q_4)\) then:

\begin{equation}
    \alpha S_X + \beta S_Y + \gamma = 0
\end{equation}

In~\cite{Press2012} a particular type of ZD strategy is defined: extortionate
strategies. If:

\begin{equation}\label{eqn:constraint_for_extortion}
    \gamma = - P(\alpha + \beta)
\end{equation}

then the player can ensure they get a score \(\chi\) times
larger than the opponent. This extortion coefficient is given by:

\begin{equation}\label{eqn:definition_of_chi}
    \chi=\frac{-\beta}{\alpha}
\end{equation}

Thus, if (\ref{eqn:constraint_for_extortion}) holds and \(\chi >1\) a player is
said to extort their opponent.
Here, the reverse problem is considered: given a
\(p\in\mathbb{R}^4\) how does one identify \(\alpha, \beta\) if they
exist and is the strategy in fact acting in an extortionate way?

These conditions correspond to:

\begin{align}
    \tilde p_1 & = \alpha R + \beta R - P (\alpha + \beta)
            \label{eqn:condition_for_tilde_p1}\\
    \tilde p_2 & = \alpha S + \beta T - P (\alpha + \beta)
            \label{eqn:condition_for_tilde_p2}\\
    \tilde p_3 & = \alpha T + \beta S - P (\alpha + \beta)
            \label{eqn:condition_for_tilde_p3}\\
    \tilde p_4 & = \alpha P + \beta P - P (\alpha + \beta)
            \label{eqn:condition_for_tilde_p4}
\end{align}

Equation (\ref{eqn:condition_for_tilde_p4}) ensures that \(p_4=\tilde p_4=0\).
Equations (\ref{eqn:condition_for_tilde_p1}-\ref{eqn:condition_for_tilde_p3})
can be used to eliminate \(\alpha, \beta\), giving:

\begin{equation}\label{eqn:planar_definition_of_extortion}
    \tilde p_1 = \frac{(R - P)(\tilde p_2 + \tilde p_3)}{S + T - 2P}
\end{equation}

with:

\begin{equation}\label{eqn:definition_of_chi}
    \chi = \frac{\tilde p_2 (P - T) + \tilde p_3 (S - P)}
                {\tilde p_2 (P - S) + \tilde p_3 (T - P)}
\end{equation}

Given a strategy \(p\in\mathbb{R}^{4\times 1}\) equations
(\ref{eqn:condition_for_tilde_p4}), (\ref{eqn:planar_definition_of_extortion}-\ref{eqn:definition_of_chi}) can be used to check if
a strategy is extortionate. The conditions correspond to:

\begin{align}
    p_1 & = \frac{(R-P)(p_2 + p_3) - R + T + S - P}{S + T - 2P}
     \label{eqn:condition_for_p1}\\
    p_4 & = 0 \label{eqn:condition_for_p4}\\
    1 & > p_2 + p_3\label{eqn:condition_for_chi}
\end{align}

The algebraic steps necessary to prove these results are available in the
supporting materials.

All extortionate strategies reside on a triangular (\ref{eqn:condition_for_chi})
plane (\ref{eqn:condition_for_p1}) in 3 dimensions (\ref{eqn:condition_for_p4}).
Using this formulation it can be seen that a necessary (but not sufficient)
condition for an extortionate strategy is that it cooperates on average less
than 50\% of the time when in a state of disagreement with the opponent.

As an example, consider the known extortionate strategy \(p=(8 / 9, 1 / 2, 1 /
3, 0)\) from~\cite{Stewart2012} which is referred to as \texttt{Extort-2}. In
this case, for the standard values of \((R, T, S, P)\) constraint
(\ref{eqn:condition_for_p1}) corresponds to:

\begin{equation}
    p_1 = \frac{2(p_2 + p_3) + 1}{3}
\end{equation}

It is clear that in this case all constraints hold.

This approach could in fact be used to confirm that a given strategy is acting
in an extortionate manner even if it is not a memory one strategy. However, in
practice, if a closed form for \(p\) is not known, then due to measurement
and/or numerical error this would not work.

This problem can be written in the following linear algebraic form where
\(x=(\alpha, \beta)\)
and \(p^*=(\tilde p_1 - 1, tilde_2 - 1, p_3)\):

\begin{equation}\label{eqn:linear_algebraic_equation_for_p}
    Cx= p^*
\end{equation}

\(C\) corresponds to equations
(\ref{eqn:condition_for_tilde_p1}-\ref{eqn:condition_for_tilde_p3}) and is
given by:

\begin{equation}\label{eqn:definition_of_C}
    C =
    \begin{bmatrix}
        R - P & R- P \\
        S - P & T- P \\
        T - P & S- P \\
    \end{bmatrix}
\end{equation}

Note that in general, equation (\ref{eqn:linear_algebraic_equation_for_p}) will
not necessarily have a solution. From the Rouch\'{e}-Capelli theorem if there is
a solution it is unique as \(\text{rank}(C)=2\) which is the dimension of the
variable \(x\). The best fitting \(x\) is found by minimizing:

\begin{equation}\label{eqn:r_squared}
    \text{SSError} = \|C x- p^*\|_2^2 = \sum_{i=1}^{3}\left((C\bar x)_i-p_i^*\right)^2
\end{equation}

Note that \(\text{SSError}\), which is the square of the Frobenius
norm~\cite{Golub2013}, becomes a measure of how close a strategy is to being an
extortionate strategy. Suspicion
of extortion then corresponds to a threshold on \(\text{SSError}\).

By observing interactions (human or otherwise), their memory one representation
can be inferred and this approach can be used to recognise extortionate
behaviour. The notion of comparing theoretic and actual plays of the IPD is not
novel, see for example~\cite{Rand2013}. Immediately it is noted that if the
environment is noisy~\cite{Wu1995} then no strategy can be considered to be
extortionate as \(p_4>0\).

In the next section, this idea will be illustrated by observing the interactions
that take place in a computer based tournament of the IPD\@.

\section{Numerical experiments}\label{sec:numerical-experiments}

In~\cite{Stewart2012} results from a tournament with
\documentclass[a4paper]{article}

\usepackage{amsmath}
\usepackage{amssymb}
\usepackage[margin=1.5cm,
            includefoot,
            footskip=30pt]{geometry}
\usepackage{layout}
\usepackage{graphicx}
\usepackage{subcaption}

\usepackage{biblatex}
\usepackage{pdfpages}

\bibliography{main.bib}

\title{Suspicion: Recognising and evaluating the effectiveness
       of extortion in the Iterated Prisoner's Dilemma}
\author{Vincent A. Knight \and Nikoleta E. Glynatsi}
\date{\today}



\begin{document}

\maketitle

\begin{abstract}
    The Iterated Prisoner's Dilemma is a model for rational and evolutionary
    interactive behaviour. It has applications both in the study of human social
    behaviour as well as in biology.
    It is used to understand when and how a rational individual might
    accept an immediate cost to their own utility for the direct benefit of
    another.

    Much attention has been given to a class of strategies called
    Zero Determinant strategies. It has been theoretically shown that these
    strategies can ``extort'' any player.

    In this work, an approach to identify if observed strategies are playing in
    an extortionate way is described. Furthermore, experimental analysis of
    a large tournament with \input{assets/tex/number_of_full_strategies/main.tex}
    strategies is considered. In this setting
    the most highly performing strategies do not play in an extortionate way
    against each other but do against lower performing strategies.
    This suggests that whilst the theory of Zero Determinant strategies
    indicates that memory is not of fundamental importance to the evolution of
    cooperative behaviour, this is incomplete.
\end{abstract}

\section{Introduction}\label{sec:introduction}

Agent based game theoretic models have become a stalwart of the underpinning
mathematics of interactive behaviours. One of the major pieces of work
in this area is the pair of original computer tournaments run by Robert
Axelrod~\cite{Axelrod1980, Axelrod1980a}. These tournaments pitted submitted
computer strategies against each other in plays of the Iterated Prisoner's
Dilemma. A common game where agents can choose to pay a slight cost to their
immediate utility in the hope of building a reputation. This has been used in
economic and evolutionary game theory to understand the evolution of cooperative
behaviour.

Recently, a class of strategies was described in~\cite{Press2012} that can
provably extort any given opponent. In~\cite{Hilbe2013, Moran1707} some
questions have already been asked about the true effectiveness of these
strategies in an evolutionary setting. Here another question is asked: is it
possible to recognise this extortionate behaviour? A mathematical procedure for
suspicion is presented: in the same way that the continued actions of an
extortionate individual might raise suspicion.

This work makes use of the Axelrod Python library~\cite{Knight2018, Knight2016}
with a large number of Prisoner Dilemma strategies available to give an
extensive numerical example of the ideas presented.  The approach is presented
in Section~\ref{sec:delta-zd-strategies}.  All of the code and data discussed
in Section~\ref{sec:numerical-experiments} is open sourced, archived and
written according to best scientific principles~\cite{Wilson2014}. The data
archive can be found at~\cite{vincent_knight_2018_1297075}.

\section{Recognising Extortion}\label{sec:delta-zd-strategies}

In~\cite{Press2012}, given a match between 2 memory-one strategies, the concept
of Zero Determinant (ZD) strategies is introduced. The main result of that paper
shows that given two memory one players \(p, q\in\mathbb{R}^4\) a linear
relationship between the players' scores could be forced by one of the players.

Using the notation of~\cite{Press2012}, assuming the utilities for player \(p\)
are given by \(S_x=(R, S, T, P)\) and for player \(q\) by \(S_y=(R, T, S, P)\)
and that the stationary scores of each player is given by \(S_X\) and \(S_Y\)
respectively. The main result of~\cite{Press2012} is that if

\begin{equation}\label{eqn:linear_relationship_for_p}
    \tilde p=\alpha S_x + \beta S_y + \gamma
\end{equation}

or

\begin{equation}\label{eqn:linear_relationship_for_q}
    \tilde q=\alpha S_x + \beta S_y + \gamma
\end{equation}

where \(\tilde p = (1 - p_1, 1 - p_2, p_3, p_4)\) and
\(\tilde q = (1 - q_1, 1 - q_2, q_3, q_4)\) then:

\begin{equation}
    \alpha S_X + \beta S_Y + \gamma = 0
\end{equation}

In~\cite{Press2012} a particular type of ZD strategy is defined: extortionate
strategies. If:

\begin{equation}\label{eqn:constraint_for_extortion}
    \gamma = - P(\alpha + \beta)
\end{equation}

then the player can ensure they get a score \(\chi\) times
larger than the opponent. This extortion coefficient is given by:

\begin{equation}\label{eqn:definition_of_chi}
    \chi=\frac{-\beta}{\alpha}
\end{equation}

Thus, if (\ref{eqn:constraint_for_extortion}) holds and \(\chi >1\) a player is
said to extort their opponent.
Here, the reverse problem is considered: given a
\(p\in\mathbb{R}^4\) how does one identify \(\alpha, \beta\) if they
exist and is the strategy in fact acting in an extortionate way?

These conditions correspond to:

\begin{align}
    \tilde p_1 & = \alpha R + \beta R - P (\alpha + \beta)
            \label{eqn:condition_for_tilde_p1}\\
    \tilde p_2 & = \alpha S + \beta T - P (\alpha + \beta)
            \label{eqn:condition_for_tilde_p2}\\
    \tilde p_3 & = \alpha T + \beta S - P (\alpha + \beta)
            \label{eqn:condition_for_tilde_p3}\\
    \tilde p_4 & = \alpha P + \beta P - P (\alpha + \beta)
            \label{eqn:condition_for_tilde_p4}
\end{align}

Equation (\ref{eqn:condition_for_tilde_p4}) ensures that \(p_4=\tilde p_4=0\).
Equations (\ref{eqn:condition_for_tilde_p1}-\ref{eqn:condition_for_tilde_p3})
can be used to eliminate \(\alpha, \beta\), giving:

\begin{equation}\label{eqn:planar_definition_of_extortion}
    \tilde p_1 = \frac{(R - P)(\tilde p_2 + \tilde p_3)}{S + T - 2P}
\end{equation}

with:

\begin{equation}\label{eqn:definition_of_chi}
    \chi = \frac{\tilde p_2 (P - T) + \tilde p_3 (S - P)}
                {\tilde p_2 (P - S) + \tilde p_3 (T - P)}
\end{equation}

Given a strategy \(p\in\mathbb{R}^{4\times 1}\) equations
(\ref{eqn:condition_for_tilde_p4}), (\ref{eqn:planar_definition_of_extortion}-\ref{eqn:definition_of_chi}) can be used to check if
a strategy is extortionate. The conditions correspond to:

\begin{align}
    p_1 & = \frac{(R-P)(p_2 + p_3) - R + T + S - P}{S + T - 2P}
     \label{eqn:condition_for_p1}\\
    p_4 & = 0 \label{eqn:condition_for_p4}\\
    1 & > p_2 + p_3\label{eqn:condition_for_chi}
\end{align}

The algebraic steps necessary to prove these results are available in the
supporting materials.

All extortionate strategies reside on a triangular (\ref{eqn:condition_for_chi})
plane (\ref{eqn:condition_for_p1}) in 3 dimensions (\ref{eqn:condition_for_p4}).
Using this formulation it can be seen that a necessary (but not sufficient)
condition for an extortionate strategy is that it cooperates on average less
than 50\% of the time when in a state of disagreement with the opponent.

As an example, consider the known extortionate strategy \(p=(8 / 9, 1 / 2, 1 /
3, 0)\) from~\cite{Stewart2012} which is referred to as \texttt{Extort-2}. In
this case, for the standard values of \((R, T, S, P)\) constraint
(\ref{eqn:condition_for_p1}) corresponds to:

\begin{equation}
    p_1 = \frac{2(p_2 + p_3) + 1}{3}
\end{equation}

It is clear that in this case all constraints hold.

This approach could in fact be used to confirm that a given strategy is acting
in an extortionate manner even if it is not a memory one strategy. However, in
practice, if a closed form for \(p\) is not known, then due to measurement
and/or numerical error this would not work.

This problem can be written in the following linear algebraic form where
\(x=(\alpha, \beta)\)
and \(p^*=(\tilde p_1 - 1, tilde_2 - 1, p_3)\):

\begin{equation}\label{eqn:linear_algebraic_equation_for_p}
    Cx= p^*
\end{equation}

\(C\) corresponds to equations
(\ref{eqn:condition_for_tilde_p1}-\ref{eqn:condition_for_tilde_p3}) and is
given by:

\begin{equation}\label{eqn:definition_of_C}
    C =
    \begin{bmatrix}
        R - P & R- P \\
        S - P & T- P \\
        T - P & S- P \\
    \end{bmatrix}
\end{equation}

Note that in general, equation (\ref{eqn:linear_algebraic_equation_for_p}) will
not necessarily have a solution. From the Rouch\'{e}-Capelli theorem if there is
a solution it is unique as \(\text{rank}(C)=2\) which is the dimension of the
variable \(x\). The best fitting \(x\) is found by minimizing:

\begin{equation}\label{eqn:r_squared}
    \text{SSError} = \|C x- p^*\|_2^2 = \sum_{i=1}^{3}\left((C\bar x)_i-p_i^*\right)^2
\end{equation}

Note that \(\text{SSError}\), which is the square of the Frobenius
norm~\cite{Golub2013}, becomes a measure of how close a strategy is to being an
extortionate strategy. Suspicion
of extortion then corresponds to a threshold on \(\text{SSError}\).

By observing interactions (human or otherwise), their memory one representation
can be inferred and this approach can be used to recognise extortionate
behaviour. The notion of comparing theoretic and actual plays of the IPD is not
novel, see for example~\cite{Rand2013}. Immediately it is noted that if the
environment is noisy~\cite{Wu1995} then no strategy can be considered to be
extortionate as \(p_4>0\).

In the next section, this idea will be illustrated by observing the interactions
that take place in a computer based tournament of the IPD\@.

\section{Numerical experiments}\label{sec:numerical-experiments}

In~\cite{Stewart2012} results from a tournament with
\input{./assets/tex/number_of_stewart_plotkin_strategies/main.tex} strategies,
was presented with specific consideration given to ZD strategies. This
tournament is reproduced here using the Axelrod-Python
project~\cite{Knight2016}. To obtain a good measure of the corresponding
transition rates for each strategy all matches have been run for
\input{assets/tex/number_of_turns/main.tex} turns and every match has been
repeated \input{assets/tex/number_of_repetitions/main.tex} times. All of this
interaction data is available at~\cite{vincent_knight_2018_1297075}. A good
match between the inferred Markov chain and the state distribution of the actual
interactions has been verified. Data for this is presented in the supplementary
materials.

Figure~\ref{fig:SSError_overall_in_stewart_plotkin} shows the \(\text{SSError}\)
values for all the strategies in the tournament, as reported
in~\cite{Stewart2012} the extortionate strategy (which has an expected
\(\text{SSError}\) approximately 0) gains a large number of wins.

\begin{figure}[!htbp]
    \centering
    \includegraphics[width=.8\textwidth]{./assets/img/SSError_overall_in_stewart_plotkin/main.pdf}
    \caption{\(\text{SSError}\) and state probabilities for the strategies
        of~\cite{Stewart2012}, ordered both by number of wins and overall score.
        Note that \(P(DC)\) is not shown as it corresponds to the transpose of
        \(P(CD)\). Cooperator and Defector are omitted as they do not visit all
        the states.}
    \label{fig:SSError_overall_in_stewart_plotkin}
\end{figure}

Here, the work of~\cite{Stewart2012} is extended by investigating a tournament
with \input{assets/tex/number_of_full_strategies/main.tex}
strategies.

The results of this analysis are shown in
Figure~\ref{fig:SSError_and_probabilities_in_full}. The top ranking strategies
by number of wins seem to be extortionate (but not against all strategies) and
it can be seen that a small sub group of strategies achieve mutual defection.
All the top ranking strategies according to score achieve mutual cooperation and
do not extort each other, however they
\textbf{do} exhibit extortionate behaviour towards a number of the lower ranking
strategies.

\begin{figure}[!htbp]
    \centering
    \includegraphics[width=.8\textwidth]{./assets/img/SSError_and_probabilities_in_full/main.pdf}
    \caption{\(\text{SSError}\) for the strategies for the full tournament. Only
    strategy interactions for which \(p_4=0\) and \(\chi>1\) are displayed.}
    \label{fig:SSError_and_probabilities_in_full}
\end{figure}

\section{Conclusion}\label{sec:conclusion}

This work defines an approach to measure whether or not a player is playing a
strategy that corresponds to an extortionate strategy as defined
in~\cite{Press2012}: a mathematical model for suspicion. Indeed, all
extortionate strategies have been
 classified as lying on a triangular plane.
This rigorous classification fails to be robust to small measurement error, thus
a statistical approach is proposed.
This is done through a linear algebraic approach for approximating the solution
of a linear system. Using this, a large number of pairwise interactions is
simulated and in fact very few strategies are found to act extortionately.

The work of~\cite{Press2012}, whilst showing that a clever approach to taking
advantage of another memory one strategy exists: this is incomplete. Whilst the
elegance of this result is very attractive, just as the simplicity of the
victory of Tit For Tat in Axelrod's original tournaments was, it is incomplete.
Extortionate strategies achieve a high number of wins but they do not
achieve a high score which corresponds to the fitness landscape in an
evolutionary sense. From the large number of interactions a payoff matrix \(S\)
can be measured where \(S_{ij}\) denotes the score (using standard values of
\((R, S, T, P) = (3, 0, 5, 1)\)) of the \(i\)th strategy
against the \(j\)th strategy. Using this, the replicator equation
describes the evolution of the system based on a population density fitness
function:

\begin{equation}\label{eqn:replicator_dynamics}
    \frac{dx}{dt} = x(S-x^TS x)
\end{equation}

Equation (\ref{eqn:replicator_dynamics}) is solved numerically through an
integration technique described in~\cite{Petzold1983} and
Figure~\ref{fig:replicator_dynamics} shows the evolution of the distribution of
the system: the various strategies are ranked by scores. It is clear to see that
only the high ranking strategies survive the evolutionary process (in fact,
only \input{./assets/img/replicator_dynamics/main.tex}
have a final distribution greater than \(10 ^ {-2}\)). This confirms the
findings of~\cite{Moran1707} in which sophisticated strategies resist
evolutionary invasion of shorter memory strategies. Recalling
Figure~\ref{fig:SSError_and_probabilities_in_full} this demonstrates that:

\begin{itemize}
    \item Cooperation emerges through the evolutionary process: the high scoring
        strategies do not exhibit extortionate behaviour towards each other.
    \item Extortionate strategies do not survive the evolutionary process.
\end{itemize}

\begin{figure}[!htbp]
    \centering
    \includegraphics[width=.8\textwidth]{./assets/img/replicator_dynamics/main.pdf}
    \caption{Numerical simulation of the replicator equation
    (\ref{eqn:replicator_dynamics}): strategies are ordered by score, only the strategies with a high score survive the evolutionary process.}
    \label{fig:replicator_dynamics}
\end{figure}

This work can be used to classify plays of the IPD\@: data can be collected from
actual interactions (in lab or in the field). Furthermore, this allows for a
classification method similar to the notion of fingerprinting presented
in~\cite{Ashlock2008}. Trained strategies can potentially be classified as
extortionate or not or it could be possible to even constrain the reinforcement
learning approaches that are becoming prevalent in the literature.
Alternatively, this mathematical approach for recognising extortion could be
used in sophisticated strategies to defend against invasion. Arguably, some of
the strategies considered here exhibit this behaviour, indeed as described
in~\cite{Harper2017}, the top ranking strategies in the full tournament are
obtained using evolutionary reinforcement learning techniques, thus, suspicion
of extortionate behaviour could in fact be an evolutionary trait.

\section*{Acknowledgements}

The following open source software libraries were used in this research:

\begin{itemize}
    \item The Axelrod ~\cite{Knight2016, Knight2018} library (IPD strategies and
        tournaments).
    \item The sympy library~\cite{Meurer2017} (verification of all symbolic
        calculations).
    \item The matplotlib~\cite{Droettboom2018} library (visualisation).
    \item The pandas~\cite{Structures2010}, dask~\cite{Dask2016} and
        NumPy~\cite{Oliphant2015} libraries (data manipulation).
    \item The SciPy~\cite{Jones2001} library (numerical integration of the
        replicator equation).
\end{itemize}

This work was performed using the computational facilities of the Advanced
Research Computing @ Cardiff (ARCCA) Division, Cardiff University.

\printbibliography

\newpage
\section*{Supplementary materials}

\includepdf{assets/pdf/proof_of_form_of_extortionate_strategies/main.pdf}

\newpage

Using the pair wise interactions the transition rates \(p,
q\) can be measured and the steady state probabilities inferred and compared to
the actual probabilities of each state.
This is done numerically by computing the singular eigenvector of the
matrix \(A\) \cite{Stewart2009}:

\[
    A =
    \begin{bmatrix}
        p_1 q_1 & p_1 (1 - q_1) & (1 - p_1) q_1 & (1 -p_1) (1 - q_1) \\
        p_2 q_2 & p_2 (1 - q_2) & (1 - p_2) q_2 & (1 -p_2) (1 - q_2) \\
        p_3 q_3 & p_3 (1 - q_3) & (1 - p_3) q_3 & (1 -p_3) (1 - q_3) \\
        p_4 q_4 & p_4 (1 - q_4) & (1 - p_4) q_4 & (1 -p_4) (1 - q_4) \\
    \end{bmatrix}
\]

Figure~\ref{fig:computed_probabilities_vs_theoretic_probabilities} shows a
regression line fitted to every pairwise interaction with a reported
\(\text{SSError}\) value (pairwise interactions with missing states were
omitted). This serves to validate the approach: a part from some edge cases the
relationship is consistent.

\begin{figure}[!htbp]
    \centering
    \includegraphics[width=.8\textwidth]{./assets/img/computed_probabilities_vs_theoretic_probabilities/main.pdf}
    \caption{The
        relationship between the steady state probabilities inferred from the
        measured transitions and the actual steady state probabilities. A linear
        regression line is included validating the approach.}
    \label{fig:computed_probabilities_vs_theoretic_probabilities}
\end{figure}


\end{document}
 strategies,
was presented with specific consideration given to ZD strategies. This
tournament is reproduced here using the Axelrod-Python
project~\cite{Knight2016}. To obtain a good measure of the corresponding
transition rates for each strategy all matches have been run for
\documentclass[a4paper]{article}

\usepackage{amsmath}
\usepackage{amssymb}
\usepackage[margin=1.5cm,
            includefoot,
            footskip=30pt]{geometry}
\usepackage{layout}
\usepackage{graphicx}
\usepackage{subcaption}

\usepackage{biblatex}
\usepackage{pdfpages}

\bibliography{main.bib}

\title{Suspicion: Recognising and evaluating the effectiveness
       of extortion in the Iterated Prisoner's Dilemma}
\author{Vincent A. Knight \and Nikoleta E. Glynatsi}
\date{\today}



\begin{document}

\maketitle

\begin{abstract}
    The Iterated Prisoner's Dilemma is a model for rational and evolutionary
    interactive behaviour. It has applications both in the study of human social
    behaviour as well as in biology.
    It is used to understand when and how a rational individual might
    accept an immediate cost to their own utility for the direct benefit of
    another.

    Much attention has been given to a class of strategies called
    Zero Determinant strategies. It has been theoretically shown that these
    strategies can ``extort'' any player.

    In this work, an approach to identify if observed strategies are playing in
    an extortionate way is described. Furthermore, experimental analysis of
    a large tournament with \input{assets/tex/number_of_full_strategies/main.tex}
    strategies is considered. In this setting
    the most highly performing strategies do not play in an extortionate way
    against each other but do against lower performing strategies.
    This suggests that whilst the theory of Zero Determinant strategies
    indicates that memory is not of fundamental importance to the evolution of
    cooperative behaviour, this is incomplete.
\end{abstract}

\section{Introduction}\label{sec:introduction}

Agent based game theoretic models have become a stalwart of the underpinning
mathematics of interactive behaviours. One of the major pieces of work
in this area is the pair of original computer tournaments run by Robert
Axelrod~\cite{Axelrod1980, Axelrod1980a}. These tournaments pitted submitted
computer strategies against each other in plays of the Iterated Prisoner's
Dilemma. A common game where agents can choose to pay a slight cost to their
immediate utility in the hope of building a reputation. This has been used in
economic and evolutionary game theory to understand the evolution of cooperative
behaviour.

Recently, a class of strategies was described in~\cite{Press2012} that can
provably extort any given opponent. In~\cite{Hilbe2013, Moran1707} some
questions have already been asked about the true effectiveness of these
strategies in an evolutionary setting. Here another question is asked: is it
possible to recognise this extortionate behaviour? A mathematical procedure for
suspicion is presented: in the same way that the continued actions of an
extortionate individual might raise suspicion.

This work makes use of the Axelrod Python library~\cite{Knight2018, Knight2016}
with a large number of Prisoner Dilemma strategies available to give an
extensive numerical example of the ideas presented.  The approach is presented
in Section~\ref{sec:delta-zd-strategies}.  All of the code and data discussed
in Section~\ref{sec:numerical-experiments} is open sourced, archived and
written according to best scientific principles~\cite{Wilson2014}. The data
archive can be found at~\cite{vincent_knight_2018_1297075}.

\section{Recognising Extortion}\label{sec:delta-zd-strategies}

In~\cite{Press2012}, given a match between 2 memory-one strategies, the concept
of Zero Determinant (ZD) strategies is introduced. The main result of that paper
shows that given two memory one players \(p, q\in\mathbb{R}^4\) a linear
relationship between the players' scores could be forced by one of the players.

Using the notation of~\cite{Press2012}, assuming the utilities for player \(p\)
are given by \(S_x=(R, S, T, P)\) and for player \(q\) by \(S_y=(R, T, S, P)\)
and that the stationary scores of each player is given by \(S_X\) and \(S_Y\)
respectively. The main result of~\cite{Press2012} is that if

\begin{equation}\label{eqn:linear_relationship_for_p}
    \tilde p=\alpha S_x + \beta S_y + \gamma
\end{equation}

or

\begin{equation}\label{eqn:linear_relationship_for_q}
    \tilde q=\alpha S_x + \beta S_y + \gamma
\end{equation}

where \(\tilde p = (1 - p_1, 1 - p_2, p_3, p_4)\) and
\(\tilde q = (1 - q_1, 1 - q_2, q_3, q_4)\) then:

\begin{equation}
    \alpha S_X + \beta S_Y + \gamma = 0
\end{equation}

In~\cite{Press2012} a particular type of ZD strategy is defined: extortionate
strategies. If:

\begin{equation}\label{eqn:constraint_for_extortion}
    \gamma = - P(\alpha + \beta)
\end{equation}

then the player can ensure they get a score \(\chi\) times
larger than the opponent. This extortion coefficient is given by:

\begin{equation}\label{eqn:definition_of_chi}
    \chi=\frac{-\beta}{\alpha}
\end{equation}

Thus, if (\ref{eqn:constraint_for_extortion}) holds and \(\chi >1\) a player is
said to extort their opponent.
Here, the reverse problem is considered: given a
\(p\in\mathbb{R}^4\) how does one identify \(\alpha, \beta\) if they
exist and is the strategy in fact acting in an extortionate way?

These conditions correspond to:

\begin{align}
    \tilde p_1 & = \alpha R + \beta R - P (\alpha + \beta)
            \label{eqn:condition_for_tilde_p1}\\
    \tilde p_2 & = \alpha S + \beta T - P (\alpha + \beta)
            \label{eqn:condition_for_tilde_p2}\\
    \tilde p_3 & = \alpha T + \beta S - P (\alpha + \beta)
            \label{eqn:condition_for_tilde_p3}\\
    \tilde p_4 & = \alpha P + \beta P - P (\alpha + \beta)
            \label{eqn:condition_for_tilde_p4}
\end{align}

Equation (\ref{eqn:condition_for_tilde_p4}) ensures that \(p_4=\tilde p_4=0\).
Equations (\ref{eqn:condition_for_tilde_p1}-\ref{eqn:condition_for_tilde_p3})
can be used to eliminate \(\alpha, \beta\), giving:

\begin{equation}\label{eqn:planar_definition_of_extortion}
    \tilde p_1 = \frac{(R - P)(\tilde p_2 + \tilde p_3)}{S + T - 2P}
\end{equation}

with:

\begin{equation}\label{eqn:definition_of_chi}
    \chi = \frac{\tilde p_2 (P - T) + \tilde p_3 (S - P)}
                {\tilde p_2 (P - S) + \tilde p_3 (T - P)}
\end{equation}

Given a strategy \(p\in\mathbb{R}^{4\times 1}\) equations
(\ref{eqn:condition_for_tilde_p4}), (\ref{eqn:planar_definition_of_extortion}-\ref{eqn:definition_of_chi}) can be used to check if
a strategy is extortionate. The conditions correspond to:

\begin{align}
    p_1 & = \frac{(R-P)(p_2 + p_3) - R + T + S - P}{S + T - 2P}
     \label{eqn:condition_for_p1}\\
    p_4 & = 0 \label{eqn:condition_for_p4}\\
    1 & > p_2 + p_3\label{eqn:condition_for_chi}
\end{align}

The algebraic steps necessary to prove these results are available in the
supporting materials.

All extortionate strategies reside on a triangular (\ref{eqn:condition_for_chi})
plane (\ref{eqn:condition_for_p1}) in 3 dimensions (\ref{eqn:condition_for_p4}).
Using this formulation it can be seen that a necessary (but not sufficient)
condition for an extortionate strategy is that it cooperates on average less
than 50\% of the time when in a state of disagreement with the opponent.

As an example, consider the known extortionate strategy \(p=(8 / 9, 1 / 2, 1 /
3, 0)\) from~\cite{Stewart2012} which is referred to as \texttt{Extort-2}. In
this case, for the standard values of \((R, T, S, P)\) constraint
(\ref{eqn:condition_for_p1}) corresponds to:

\begin{equation}
    p_1 = \frac{2(p_2 + p_3) + 1}{3}
\end{equation}

It is clear that in this case all constraints hold.

This approach could in fact be used to confirm that a given strategy is acting
in an extortionate manner even if it is not a memory one strategy. However, in
practice, if a closed form for \(p\) is not known, then due to measurement
and/or numerical error this would not work.

This problem can be written in the following linear algebraic form where
\(x=(\alpha, \beta)\)
and \(p^*=(\tilde p_1 - 1, tilde_2 - 1, p_3)\):

\begin{equation}\label{eqn:linear_algebraic_equation_for_p}
    Cx= p^*
\end{equation}

\(C\) corresponds to equations
(\ref{eqn:condition_for_tilde_p1}-\ref{eqn:condition_for_tilde_p3}) and is
given by:

\begin{equation}\label{eqn:definition_of_C}
    C =
    \begin{bmatrix}
        R - P & R- P \\
        S - P & T- P \\
        T - P & S- P \\
    \end{bmatrix}
\end{equation}

Note that in general, equation (\ref{eqn:linear_algebraic_equation_for_p}) will
not necessarily have a solution. From the Rouch\'{e}-Capelli theorem if there is
a solution it is unique as \(\text{rank}(C)=2\) which is the dimension of the
variable \(x\). The best fitting \(x\) is found by minimizing:

\begin{equation}\label{eqn:r_squared}
    \text{SSError} = \|C x- p^*\|_2^2 = \sum_{i=1}^{3}\left((C\bar x)_i-p_i^*\right)^2
\end{equation}

Note that \(\text{SSError}\), which is the square of the Frobenius
norm~\cite{Golub2013}, becomes a measure of how close a strategy is to being an
extortionate strategy. Suspicion
of extortion then corresponds to a threshold on \(\text{SSError}\).

By observing interactions (human or otherwise), their memory one representation
can be inferred and this approach can be used to recognise extortionate
behaviour. The notion of comparing theoretic and actual plays of the IPD is not
novel, see for example~\cite{Rand2013}. Immediately it is noted that if the
environment is noisy~\cite{Wu1995} then no strategy can be considered to be
extortionate as \(p_4>0\).

In the next section, this idea will be illustrated by observing the interactions
that take place in a computer based tournament of the IPD\@.

\section{Numerical experiments}\label{sec:numerical-experiments}

In~\cite{Stewart2012} results from a tournament with
\input{./assets/tex/number_of_stewart_plotkin_strategies/main.tex} strategies,
was presented with specific consideration given to ZD strategies. This
tournament is reproduced here using the Axelrod-Python
project~\cite{Knight2016}. To obtain a good measure of the corresponding
transition rates for each strategy all matches have been run for
\input{assets/tex/number_of_turns/main.tex} turns and every match has been
repeated \input{assets/tex/number_of_repetitions/main.tex} times. All of this
interaction data is available at~\cite{vincent_knight_2018_1297075}. A good
match between the inferred Markov chain and the state distribution of the actual
interactions has been verified. Data for this is presented in the supplementary
materials.

Figure~\ref{fig:SSError_overall_in_stewart_plotkin} shows the \(\text{SSError}\)
values for all the strategies in the tournament, as reported
in~\cite{Stewart2012} the extortionate strategy (which has an expected
\(\text{SSError}\) approximately 0) gains a large number of wins.

\begin{figure}[!htbp]
    \centering
    \includegraphics[width=.8\textwidth]{./assets/img/SSError_overall_in_stewart_plotkin/main.pdf}
    \caption{\(\text{SSError}\) and state probabilities for the strategies
        of~\cite{Stewart2012}, ordered both by number of wins and overall score.
        Note that \(P(DC)\) is not shown as it corresponds to the transpose of
        \(P(CD)\). Cooperator and Defector are omitted as they do not visit all
        the states.}
    \label{fig:SSError_overall_in_stewart_plotkin}
\end{figure}

Here, the work of~\cite{Stewart2012} is extended by investigating a tournament
with \input{assets/tex/number_of_full_strategies/main.tex}
strategies.

The results of this analysis are shown in
Figure~\ref{fig:SSError_and_probabilities_in_full}. The top ranking strategies
by number of wins seem to be extortionate (but not against all strategies) and
it can be seen that a small sub group of strategies achieve mutual defection.
All the top ranking strategies according to score achieve mutual cooperation and
do not extort each other, however they
\textbf{do} exhibit extortionate behaviour towards a number of the lower ranking
strategies.

\begin{figure}[!htbp]
    \centering
    \includegraphics[width=.8\textwidth]{./assets/img/SSError_and_probabilities_in_full/main.pdf}
    \caption{\(\text{SSError}\) for the strategies for the full tournament. Only
    strategy interactions for which \(p_4=0\) and \(\chi>1\) are displayed.}
    \label{fig:SSError_and_probabilities_in_full}
\end{figure}

\section{Conclusion}\label{sec:conclusion}

This work defines an approach to measure whether or not a player is playing a
strategy that corresponds to an extortionate strategy as defined
in~\cite{Press2012}: a mathematical model for suspicion. Indeed, all
extortionate strategies have been
 classified as lying on a triangular plane.
This rigorous classification fails to be robust to small measurement error, thus
a statistical approach is proposed.
This is done through a linear algebraic approach for approximating the solution
of a linear system. Using this, a large number of pairwise interactions is
simulated and in fact very few strategies are found to act extortionately.

The work of~\cite{Press2012}, whilst showing that a clever approach to taking
advantage of another memory one strategy exists: this is incomplete. Whilst the
elegance of this result is very attractive, just as the simplicity of the
victory of Tit For Tat in Axelrod's original tournaments was, it is incomplete.
Extortionate strategies achieve a high number of wins but they do not
achieve a high score which corresponds to the fitness landscape in an
evolutionary sense. From the large number of interactions a payoff matrix \(S\)
can be measured where \(S_{ij}\) denotes the score (using standard values of
\((R, S, T, P) = (3, 0, 5, 1)\)) of the \(i\)th strategy
against the \(j\)th strategy. Using this, the replicator equation
describes the evolution of the system based on a population density fitness
function:

\begin{equation}\label{eqn:replicator_dynamics}
    \frac{dx}{dt} = x(S-x^TS x)
\end{equation}

Equation (\ref{eqn:replicator_dynamics}) is solved numerically through an
integration technique described in~\cite{Petzold1983} and
Figure~\ref{fig:replicator_dynamics} shows the evolution of the distribution of
the system: the various strategies are ranked by scores. It is clear to see that
only the high ranking strategies survive the evolutionary process (in fact,
only \input{./assets/img/replicator_dynamics/main.tex}
have a final distribution greater than \(10 ^ {-2}\)). This confirms the
findings of~\cite{Moran1707} in which sophisticated strategies resist
evolutionary invasion of shorter memory strategies. Recalling
Figure~\ref{fig:SSError_and_probabilities_in_full} this demonstrates that:

\begin{itemize}
    \item Cooperation emerges through the evolutionary process: the high scoring
        strategies do not exhibit extortionate behaviour towards each other.
    \item Extortionate strategies do not survive the evolutionary process.
\end{itemize}

\begin{figure}[!htbp]
    \centering
    \includegraphics[width=.8\textwidth]{./assets/img/replicator_dynamics/main.pdf}
    \caption{Numerical simulation of the replicator equation
    (\ref{eqn:replicator_dynamics}): strategies are ordered by score, only the strategies with a high score survive the evolutionary process.}
    \label{fig:replicator_dynamics}
\end{figure}

This work can be used to classify plays of the IPD\@: data can be collected from
actual interactions (in lab or in the field). Furthermore, this allows for a
classification method similar to the notion of fingerprinting presented
in~\cite{Ashlock2008}. Trained strategies can potentially be classified as
extortionate or not or it could be possible to even constrain the reinforcement
learning approaches that are becoming prevalent in the literature.
Alternatively, this mathematical approach for recognising extortion could be
used in sophisticated strategies to defend against invasion. Arguably, some of
the strategies considered here exhibit this behaviour, indeed as described
in~\cite{Harper2017}, the top ranking strategies in the full tournament are
obtained using evolutionary reinforcement learning techniques, thus, suspicion
of extortionate behaviour could in fact be an evolutionary trait.

\section*{Acknowledgements}

The following open source software libraries were used in this research:

\begin{itemize}
    \item The Axelrod ~\cite{Knight2016, Knight2018} library (IPD strategies and
        tournaments).
    \item The sympy library~\cite{Meurer2017} (verification of all symbolic
        calculations).
    \item The matplotlib~\cite{Droettboom2018} library (visualisation).
    \item The pandas~\cite{Structures2010}, dask~\cite{Dask2016} and
        NumPy~\cite{Oliphant2015} libraries (data manipulation).
    \item The SciPy~\cite{Jones2001} library (numerical integration of the
        replicator equation).
\end{itemize}

This work was performed using the computational facilities of the Advanced
Research Computing @ Cardiff (ARCCA) Division, Cardiff University.

\printbibliography

\newpage
\section*{Supplementary materials}

\includepdf{assets/pdf/proof_of_form_of_extortionate_strategies/main.pdf}

\newpage

Using the pair wise interactions the transition rates \(p,
q\) can be measured and the steady state probabilities inferred and compared to
the actual probabilities of each state.
This is done numerically by computing the singular eigenvector of the
matrix \(A\) \cite{Stewart2009}:

\[
    A =
    \begin{bmatrix}
        p_1 q_1 & p_1 (1 - q_1) & (1 - p_1) q_1 & (1 -p_1) (1 - q_1) \\
        p_2 q_2 & p_2 (1 - q_2) & (1 - p_2) q_2 & (1 -p_2) (1 - q_2) \\
        p_3 q_3 & p_3 (1 - q_3) & (1 - p_3) q_3 & (1 -p_3) (1 - q_3) \\
        p_4 q_4 & p_4 (1 - q_4) & (1 - p_4) q_4 & (1 -p_4) (1 - q_4) \\
    \end{bmatrix}
\]

Figure~\ref{fig:computed_probabilities_vs_theoretic_probabilities} shows a
regression line fitted to every pairwise interaction with a reported
\(\text{SSError}\) value (pairwise interactions with missing states were
omitted). This serves to validate the approach: a part from some edge cases the
relationship is consistent.

\begin{figure}[!htbp]
    \centering
    \includegraphics[width=.8\textwidth]{./assets/img/computed_probabilities_vs_theoretic_probabilities/main.pdf}
    \caption{The
        relationship between the steady state probabilities inferred from the
        measured transitions and the actual steady state probabilities. A linear
        regression line is included validating the approach.}
    \label{fig:computed_probabilities_vs_theoretic_probabilities}
\end{figure}


\end{document}
 turns and every match has been
repeated \documentclass[a4paper]{article}

\usepackage{amsmath}
\usepackage{amssymb}
\usepackage[margin=1.5cm,
            includefoot,
            footskip=30pt]{geometry}
\usepackage{layout}
\usepackage{graphicx}
\usepackage{subcaption}

\usepackage{biblatex}
\usepackage{pdfpages}

\bibliography{main.bib}

\title{Suspicion: Recognising and evaluating the effectiveness
       of extortion in the Iterated Prisoner's Dilemma}
\author{Vincent A. Knight \and Nikoleta E. Glynatsi}
\date{\today}



\begin{document}

\maketitle

\begin{abstract}
    The Iterated Prisoner's Dilemma is a model for rational and evolutionary
    interactive behaviour. It has applications both in the study of human social
    behaviour as well as in biology.
    It is used to understand when and how a rational individual might
    accept an immediate cost to their own utility for the direct benefit of
    another.

    Much attention has been given to a class of strategies called
    Zero Determinant strategies. It has been theoretically shown that these
    strategies can ``extort'' any player.

    In this work, an approach to identify if observed strategies are playing in
    an extortionate way is described. Furthermore, experimental analysis of
    a large tournament with \input{assets/tex/number_of_full_strategies/main.tex}
    strategies is considered. In this setting
    the most highly performing strategies do not play in an extortionate way
    against each other but do against lower performing strategies.
    This suggests that whilst the theory of Zero Determinant strategies
    indicates that memory is not of fundamental importance to the evolution of
    cooperative behaviour, this is incomplete.
\end{abstract}

\section{Introduction}\label{sec:introduction}

Agent based game theoretic models have become a stalwart of the underpinning
mathematics of interactive behaviours. One of the major pieces of work
in this area is the pair of original computer tournaments run by Robert
Axelrod~\cite{Axelrod1980, Axelrod1980a}. These tournaments pitted submitted
computer strategies against each other in plays of the Iterated Prisoner's
Dilemma. A common game where agents can choose to pay a slight cost to their
immediate utility in the hope of building a reputation. This has been used in
economic and evolutionary game theory to understand the evolution of cooperative
behaviour.

Recently, a class of strategies was described in~\cite{Press2012} that can
provably extort any given opponent. In~\cite{Hilbe2013, Moran1707} some
questions have already been asked about the true effectiveness of these
strategies in an evolutionary setting. Here another question is asked: is it
possible to recognise this extortionate behaviour? A mathematical procedure for
suspicion is presented: in the same way that the continued actions of an
extortionate individual might raise suspicion.

This work makes use of the Axelrod Python library~\cite{Knight2018, Knight2016}
with a large number of Prisoner Dilemma strategies available to give an
extensive numerical example of the ideas presented.  The approach is presented
in Section~\ref{sec:delta-zd-strategies}.  All of the code and data discussed
in Section~\ref{sec:numerical-experiments} is open sourced, archived and
written according to best scientific principles~\cite{Wilson2014}. The data
archive can be found at~\cite{vincent_knight_2018_1297075}.

\section{Recognising Extortion}\label{sec:delta-zd-strategies}

In~\cite{Press2012}, given a match between 2 memory-one strategies, the concept
of Zero Determinant (ZD) strategies is introduced. The main result of that paper
shows that given two memory one players \(p, q\in\mathbb{R}^4\) a linear
relationship between the players' scores could be forced by one of the players.

Using the notation of~\cite{Press2012}, assuming the utilities for player \(p\)
are given by \(S_x=(R, S, T, P)\) and for player \(q\) by \(S_y=(R, T, S, P)\)
and that the stationary scores of each player is given by \(S_X\) and \(S_Y\)
respectively. The main result of~\cite{Press2012} is that if

\begin{equation}\label{eqn:linear_relationship_for_p}
    \tilde p=\alpha S_x + \beta S_y + \gamma
\end{equation}

or

\begin{equation}\label{eqn:linear_relationship_for_q}
    \tilde q=\alpha S_x + \beta S_y + \gamma
\end{equation}

where \(\tilde p = (1 - p_1, 1 - p_2, p_3, p_4)\) and
\(\tilde q = (1 - q_1, 1 - q_2, q_3, q_4)\) then:

\begin{equation}
    \alpha S_X + \beta S_Y + \gamma = 0
\end{equation}

In~\cite{Press2012} a particular type of ZD strategy is defined: extortionate
strategies. If:

\begin{equation}\label{eqn:constraint_for_extortion}
    \gamma = - P(\alpha + \beta)
\end{equation}

then the player can ensure they get a score \(\chi\) times
larger than the opponent. This extortion coefficient is given by:

\begin{equation}\label{eqn:definition_of_chi}
    \chi=\frac{-\beta}{\alpha}
\end{equation}

Thus, if (\ref{eqn:constraint_for_extortion}) holds and \(\chi >1\) a player is
said to extort their opponent.
Here, the reverse problem is considered: given a
\(p\in\mathbb{R}^4\) how does one identify \(\alpha, \beta\) if they
exist and is the strategy in fact acting in an extortionate way?

These conditions correspond to:

\begin{align}
    \tilde p_1 & = \alpha R + \beta R - P (\alpha + \beta)
            \label{eqn:condition_for_tilde_p1}\\
    \tilde p_2 & = \alpha S + \beta T - P (\alpha + \beta)
            \label{eqn:condition_for_tilde_p2}\\
    \tilde p_3 & = \alpha T + \beta S - P (\alpha + \beta)
            \label{eqn:condition_for_tilde_p3}\\
    \tilde p_4 & = \alpha P + \beta P - P (\alpha + \beta)
            \label{eqn:condition_for_tilde_p4}
\end{align}

Equation (\ref{eqn:condition_for_tilde_p4}) ensures that \(p_4=\tilde p_4=0\).
Equations (\ref{eqn:condition_for_tilde_p1}-\ref{eqn:condition_for_tilde_p3})
can be used to eliminate \(\alpha, \beta\), giving:

\begin{equation}\label{eqn:planar_definition_of_extortion}
    \tilde p_1 = \frac{(R - P)(\tilde p_2 + \tilde p_3)}{S + T - 2P}
\end{equation}

with:

\begin{equation}\label{eqn:definition_of_chi}
    \chi = \frac{\tilde p_2 (P - T) + \tilde p_3 (S - P)}
                {\tilde p_2 (P - S) + \tilde p_3 (T - P)}
\end{equation}

Given a strategy \(p\in\mathbb{R}^{4\times 1}\) equations
(\ref{eqn:condition_for_tilde_p4}), (\ref{eqn:planar_definition_of_extortion}-\ref{eqn:definition_of_chi}) can be used to check if
a strategy is extortionate. The conditions correspond to:

\begin{align}
    p_1 & = \frac{(R-P)(p_2 + p_3) - R + T + S - P}{S + T - 2P}
     \label{eqn:condition_for_p1}\\
    p_4 & = 0 \label{eqn:condition_for_p4}\\
    1 & > p_2 + p_3\label{eqn:condition_for_chi}
\end{align}

The algebraic steps necessary to prove these results are available in the
supporting materials.

All extortionate strategies reside on a triangular (\ref{eqn:condition_for_chi})
plane (\ref{eqn:condition_for_p1}) in 3 dimensions (\ref{eqn:condition_for_p4}).
Using this formulation it can be seen that a necessary (but not sufficient)
condition for an extortionate strategy is that it cooperates on average less
than 50\% of the time when in a state of disagreement with the opponent.

As an example, consider the known extortionate strategy \(p=(8 / 9, 1 / 2, 1 /
3, 0)\) from~\cite{Stewart2012} which is referred to as \texttt{Extort-2}. In
this case, for the standard values of \((R, T, S, P)\) constraint
(\ref{eqn:condition_for_p1}) corresponds to:

\begin{equation}
    p_1 = \frac{2(p_2 + p_3) + 1}{3}
\end{equation}

It is clear that in this case all constraints hold.

This approach could in fact be used to confirm that a given strategy is acting
in an extortionate manner even if it is not a memory one strategy. However, in
practice, if a closed form for \(p\) is not known, then due to measurement
and/or numerical error this would not work.

This problem can be written in the following linear algebraic form where
\(x=(\alpha, \beta)\)
and \(p^*=(\tilde p_1 - 1, tilde_2 - 1, p_3)\):

\begin{equation}\label{eqn:linear_algebraic_equation_for_p}
    Cx= p^*
\end{equation}

\(C\) corresponds to equations
(\ref{eqn:condition_for_tilde_p1}-\ref{eqn:condition_for_tilde_p3}) and is
given by:

\begin{equation}\label{eqn:definition_of_C}
    C =
    \begin{bmatrix}
        R - P & R- P \\
        S - P & T- P \\
        T - P & S- P \\
    \end{bmatrix}
\end{equation}

Note that in general, equation (\ref{eqn:linear_algebraic_equation_for_p}) will
not necessarily have a solution. From the Rouch\'{e}-Capelli theorem if there is
a solution it is unique as \(\text{rank}(C)=2\) which is the dimension of the
variable \(x\). The best fitting \(x\) is found by minimizing:

\begin{equation}\label{eqn:r_squared}
    \text{SSError} = \|C x- p^*\|_2^2 = \sum_{i=1}^{3}\left((C\bar x)_i-p_i^*\right)^2
\end{equation}

Note that \(\text{SSError}\), which is the square of the Frobenius
norm~\cite{Golub2013}, becomes a measure of how close a strategy is to being an
extortionate strategy. Suspicion
of extortion then corresponds to a threshold on \(\text{SSError}\).

By observing interactions (human or otherwise), their memory one representation
can be inferred and this approach can be used to recognise extortionate
behaviour. The notion of comparing theoretic and actual plays of the IPD is not
novel, see for example~\cite{Rand2013}. Immediately it is noted that if the
environment is noisy~\cite{Wu1995} then no strategy can be considered to be
extortionate as \(p_4>0\).

In the next section, this idea will be illustrated by observing the interactions
that take place in a computer based tournament of the IPD\@.

\section{Numerical experiments}\label{sec:numerical-experiments}

In~\cite{Stewart2012} results from a tournament with
\input{./assets/tex/number_of_stewart_plotkin_strategies/main.tex} strategies,
was presented with specific consideration given to ZD strategies. This
tournament is reproduced here using the Axelrod-Python
project~\cite{Knight2016}. To obtain a good measure of the corresponding
transition rates for each strategy all matches have been run for
\input{assets/tex/number_of_turns/main.tex} turns and every match has been
repeated \input{assets/tex/number_of_repetitions/main.tex} times. All of this
interaction data is available at~\cite{vincent_knight_2018_1297075}. A good
match between the inferred Markov chain and the state distribution of the actual
interactions has been verified. Data for this is presented in the supplementary
materials.

Figure~\ref{fig:SSError_overall_in_stewart_plotkin} shows the \(\text{SSError}\)
values for all the strategies in the tournament, as reported
in~\cite{Stewart2012} the extortionate strategy (which has an expected
\(\text{SSError}\) approximately 0) gains a large number of wins.

\begin{figure}[!htbp]
    \centering
    \includegraphics[width=.8\textwidth]{./assets/img/SSError_overall_in_stewart_plotkin/main.pdf}
    \caption{\(\text{SSError}\) and state probabilities for the strategies
        of~\cite{Stewart2012}, ordered both by number of wins and overall score.
        Note that \(P(DC)\) is not shown as it corresponds to the transpose of
        \(P(CD)\). Cooperator and Defector are omitted as they do not visit all
        the states.}
    \label{fig:SSError_overall_in_stewart_plotkin}
\end{figure}

Here, the work of~\cite{Stewart2012} is extended by investigating a tournament
with \input{assets/tex/number_of_full_strategies/main.tex}
strategies.

The results of this analysis are shown in
Figure~\ref{fig:SSError_and_probabilities_in_full}. The top ranking strategies
by number of wins seem to be extortionate (but not against all strategies) and
it can be seen that a small sub group of strategies achieve mutual defection.
All the top ranking strategies according to score achieve mutual cooperation and
do not extort each other, however they
\textbf{do} exhibit extortionate behaviour towards a number of the lower ranking
strategies.

\begin{figure}[!htbp]
    \centering
    \includegraphics[width=.8\textwidth]{./assets/img/SSError_and_probabilities_in_full/main.pdf}
    \caption{\(\text{SSError}\) for the strategies for the full tournament. Only
    strategy interactions for which \(p_4=0\) and \(\chi>1\) are displayed.}
    \label{fig:SSError_and_probabilities_in_full}
\end{figure}

\section{Conclusion}\label{sec:conclusion}

This work defines an approach to measure whether or not a player is playing a
strategy that corresponds to an extortionate strategy as defined
in~\cite{Press2012}: a mathematical model for suspicion. Indeed, all
extortionate strategies have been
 classified as lying on a triangular plane.
This rigorous classification fails to be robust to small measurement error, thus
a statistical approach is proposed.
This is done through a linear algebraic approach for approximating the solution
of a linear system. Using this, a large number of pairwise interactions is
simulated and in fact very few strategies are found to act extortionately.

The work of~\cite{Press2012}, whilst showing that a clever approach to taking
advantage of another memory one strategy exists: this is incomplete. Whilst the
elegance of this result is very attractive, just as the simplicity of the
victory of Tit For Tat in Axelrod's original tournaments was, it is incomplete.
Extortionate strategies achieve a high number of wins but they do not
achieve a high score which corresponds to the fitness landscape in an
evolutionary sense. From the large number of interactions a payoff matrix \(S\)
can be measured where \(S_{ij}\) denotes the score (using standard values of
\((R, S, T, P) = (3, 0, 5, 1)\)) of the \(i\)th strategy
against the \(j\)th strategy. Using this, the replicator equation
describes the evolution of the system based on a population density fitness
function:

\begin{equation}\label{eqn:replicator_dynamics}
    \frac{dx}{dt} = x(S-x^TS x)
\end{equation}

Equation (\ref{eqn:replicator_dynamics}) is solved numerically through an
integration technique described in~\cite{Petzold1983} and
Figure~\ref{fig:replicator_dynamics} shows the evolution of the distribution of
the system: the various strategies are ranked by scores. It is clear to see that
only the high ranking strategies survive the evolutionary process (in fact,
only \input{./assets/img/replicator_dynamics/main.tex}
have a final distribution greater than \(10 ^ {-2}\)). This confirms the
findings of~\cite{Moran1707} in which sophisticated strategies resist
evolutionary invasion of shorter memory strategies. Recalling
Figure~\ref{fig:SSError_and_probabilities_in_full} this demonstrates that:

\begin{itemize}
    \item Cooperation emerges through the evolutionary process: the high scoring
        strategies do not exhibit extortionate behaviour towards each other.
    \item Extortionate strategies do not survive the evolutionary process.
\end{itemize}

\begin{figure}[!htbp]
    \centering
    \includegraphics[width=.8\textwidth]{./assets/img/replicator_dynamics/main.pdf}
    \caption{Numerical simulation of the replicator equation
    (\ref{eqn:replicator_dynamics}): strategies are ordered by score, only the strategies with a high score survive the evolutionary process.}
    \label{fig:replicator_dynamics}
\end{figure}

This work can be used to classify plays of the IPD\@: data can be collected from
actual interactions (in lab or in the field). Furthermore, this allows for a
classification method similar to the notion of fingerprinting presented
in~\cite{Ashlock2008}. Trained strategies can potentially be classified as
extortionate or not or it could be possible to even constrain the reinforcement
learning approaches that are becoming prevalent in the literature.
Alternatively, this mathematical approach for recognising extortion could be
used in sophisticated strategies to defend against invasion. Arguably, some of
the strategies considered here exhibit this behaviour, indeed as described
in~\cite{Harper2017}, the top ranking strategies in the full tournament are
obtained using evolutionary reinforcement learning techniques, thus, suspicion
of extortionate behaviour could in fact be an evolutionary trait.

\section*{Acknowledgements}

The following open source software libraries were used in this research:

\begin{itemize}
    \item The Axelrod ~\cite{Knight2016, Knight2018} library (IPD strategies and
        tournaments).
    \item The sympy library~\cite{Meurer2017} (verification of all symbolic
        calculations).
    \item The matplotlib~\cite{Droettboom2018} library (visualisation).
    \item The pandas~\cite{Structures2010}, dask~\cite{Dask2016} and
        NumPy~\cite{Oliphant2015} libraries (data manipulation).
    \item The SciPy~\cite{Jones2001} library (numerical integration of the
        replicator equation).
\end{itemize}

This work was performed using the computational facilities of the Advanced
Research Computing @ Cardiff (ARCCA) Division, Cardiff University.

\printbibliography

\newpage
\section*{Supplementary materials}

\includepdf{assets/pdf/proof_of_form_of_extortionate_strategies/main.pdf}

\newpage

Using the pair wise interactions the transition rates \(p,
q\) can be measured and the steady state probabilities inferred and compared to
the actual probabilities of each state.
This is done numerically by computing the singular eigenvector of the
matrix \(A\) \cite{Stewart2009}:

\[
    A =
    \begin{bmatrix}
        p_1 q_1 & p_1 (1 - q_1) & (1 - p_1) q_1 & (1 -p_1) (1 - q_1) \\
        p_2 q_2 & p_2 (1 - q_2) & (1 - p_2) q_2 & (1 -p_2) (1 - q_2) \\
        p_3 q_3 & p_3 (1 - q_3) & (1 - p_3) q_3 & (1 -p_3) (1 - q_3) \\
        p_4 q_4 & p_4 (1 - q_4) & (1 - p_4) q_4 & (1 -p_4) (1 - q_4) \\
    \end{bmatrix}
\]

Figure~\ref{fig:computed_probabilities_vs_theoretic_probabilities} shows a
regression line fitted to every pairwise interaction with a reported
\(\text{SSError}\) value (pairwise interactions with missing states were
omitted). This serves to validate the approach: a part from some edge cases the
relationship is consistent.

\begin{figure}[!htbp]
    \centering
    \includegraphics[width=.8\textwidth]{./assets/img/computed_probabilities_vs_theoretic_probabilities/main.pdf}
    \caption{The
        relationship between the steady state probabilities inferred from the
        measured transitions and the actual steady state probabilities. A linear
        regression line is included validating the approach.}
    \label{fig:computed_probabilities_vs_theoretic_probabilities}
\end{figure}


\end{document}
 times. All of this
interaction data is available at~\cite{vincent_knight_2018_1297075}. A good
match between the inferred Markov chain and the state distribution of the actual
interactions has been verified. Data for this is presented in the supplementary
materials.

Figure~\ref{fig:SSError_overall_in_stewart_plotkin} shows the \(\text{SSError}\)
values for all the strategies in the tournament, as reported
in~\cite{Stewart2012} the extortionate strategy (which has an expected
\(\text{SSError}\) approximately 0) gains a large number of wins.

\begin{figure}[!htbp]
    \centering
    \includegraphics[width=.8\textwidth]{./assets/img/SSError_overall_in_stewart_plotkin/main.pdf}
    \caption{\(\text{SSError}\) and state probabilities for the strategies
        of~\cite{Stewart2012}, ordered both by number of wins and overall score.
        Note that \(P(DC)\) is not shown as it corresponds to the transpose of
        \(P(CD)\). Cooperator and Defector are omitted as they do not visit all
        the states.}
    \label{fig:SSError_overall_in_stewart_plotkin}
\end{figure}

Here, the work of~\cite{Stewart2012} is extended by investigating a tournament
with \documentclass[a4paper]{article}

\usepackage{amsmath}
\usepackage{amssymb}
\usepackage[margin=1.5cm,
            includefoot,
            footskip=30pt]{geometry}
\usepackage{layout}
\usepackage{graphicx}
\usepackage{subcaption}

\usepackage{biblatex}
\usepackage{pdfpages}

\bibliography{main.bib}

\title{Suspicion: Recognising and evaluating the effectiveness
       of extortion in the Iterated Prisoner's Dilemma}
\author{Vincent A. Knight \and Nikoleta E. Glynatsi}
\date{\today}



\begin{document}

\maketitle

\begin{abstract}
    The Iterated Prisoner's Dilemma is a model for rational and evolutionary
    interactive behaviour. It has applications both in the study of human social
    behaviour as well as in biology.
    It is used to understand when and how a rational individual might
    accept an immediate cost to their own utility for the direct benefit of
    another.

    Much attention has been given to a class of strategies called
    Zero Determinant strategies. It has been theoretically shown that these
    strategies can ``extort'' any player.

    In this work, an approach to identify if observed strategies are playing in
    an extortionate way is described. Furthermore, experimental analysis of
    a large tournament with \input{assets/tex/number_of_full_strategies/main.tex}
    strategies is considered. In this setting
    the most highly performing strategies do not play in an extortionate way
    against each other but do against lower performing strategies.
    This suggests that whilst the theory of Zero Determinant strategies
    indicates that memory is not of fundamental importance to the evolution of
    cooperative behaviour, this is incomplete.
\end{abstract}

\section{Introduction}\label{sec:introduction}

Agent based game theoretic models have become a stalwart of the underpinning
mathematics of interactive behaviours. One of the major pieces of work
in this area is the pair of original computer tournaments run by Robert
Axelrod~\cite{Axelrod1980, Axelrod1980a}. These tournaments pitted submitted
computer strategies against each other in plays of the Iterated Prisoner's
Dilemma. A common game where agents can choose to pay a slight cost to their
immediate utility in the hope of building a reputation. This has been used in
economic and evolutionary game theory to understand the evolution of cooperative
behaviour.

Recently, a class of strategies was described in~\cite{Press2012} that can
provably extort any given opponent. In~\cite{Hilbe2013, Moran1707} some
questions have already been asked about the true effectiveness of these
strategies in an evolutionary setting. Here another question is asked: is it
possible to recognise this extortionate behaviour? A mathematical procedure for
suspicion is presented: in the same way that the continued actions of an
extortionate individual might raise suspicion.

This work makes use of the Axelrod Python library~\cite{Knight2018, Knight2016}
with a large number of Prisoner Dilemma strategies available to give an
extensive numerical example of the ideas presented.  The approach is presented
in Section~\ref{sec:delta-zd-strategies}.  All of the code and data discussed
in Section~\ref{sec:numerical-experiments} is open sourced, archived and
written according to best scientific principles~\cite{Wilson2014}. The data
archive can be found at~\cite{vincent_knight_2018_1297075}.

\section{Recognising Extortion}\label{sec:delta-zd-strategies}

In~\cite{Press2012}, given a match between 2 memory-one strategies, the concept
of Zero Determinant (ZD) strategies is introduced. The main result of that paper
shows that given two memory one players \(p, q\in\mathbb{R}^4\) a linear
relationship between the players' scores could be forced by one of the players.

Using the notation of~\cite{Press2012}, assuming the utilities for player \(p\)
are given by \(S_x=(R, S, T, P)\) and for player \(q\) by \(S_y=(R, T, S, P)\)
and that the stationary scores of each player is given by \(S_X\) and \(S_Y\)
respectively. The main result of~\cite{Press2012} is that if

\begin{equation}\label{eqn:linear_relationship_for_p}
    \tilde p=\alpha S_x + \beta S_y + \gamma
\end{equation}

or

\begin{equation}\label{eqn:linear_relationship_for_q}
    \tilde q=\alpha S_x + \beta S_y + \gamma
\end{equation}

where \(\tilde p = (1 - p_1, 1 - p_2, p_3, p_4)\) and
\(\tilde q = (1 - q_1, 1 - q_2, q_3, q_4)\) then:

\begin{equation}
    \alpha S_X + \beta S_Y + \gamma = 0
\end{equation}

In~\cite{Press2012} a particular type of ZD strategy is defined: extortionate
strategies. If:

\begin{equation}\label{eqn:constraint_for_extortion}
    \gamma = - P(\alpha + \beta)
\end{equation}

then the player can ensure they get a score \(\chi\) times
larger than the opponent. This extortion coefficient is given by:

\begin{equation}\label{eqn:definition_of_chi}
    \chi=\frac{-\beta}{\alpha}
\end{equation}

Thus, if (\ref{eqn:constraint_for_extortion}) holds and \(\chi >1\) a player is
said to extort their opponent.
Here, the reverse problem is considered: given a
\(p\in\mathbb{R}^4\) how does one identify \(\alpha, \beta\) if they
exist and is the strategy in fact acting in an extortionate way?

These conditions correspond to:

\begin{align}
    \tilde p_1 & = \alpha R + \beta R - P (\alpha + \beta)
            \label{eqn:condition_for_tilde_p1}\\
    \tilde p_2 & = \alpha S + \beta T - P (\alpha + \beta)
            \label{eqn:condition_for_tilde_p2}\\
    \tilde p_3 & = \alpha T + \beta S - P (\alpha + \beta)
            \label{eqn:condition_for_tilde_p3}\\
    \tilde p_4 & = \alpha P + \beta P - P (\alpha + \beta)
            \label{eqn:condition_for_tilde_p4}
\end{align}

Equation (\ref{eqn:condition_for_tilde_p4}) ensures that \(p_4=\tilde p_4=0\).
Equations (\ref{eqn:condition_for_tilde_p1}-\ref{eqn:condition_for_tilde_p3})
can be used to eliminate \(\alpha, \beta\), giving:

\begin{equation}\label{eqn:planar_definition_of_extortion}
    \tilde p_1 = \frac{(R - P)(\tilde p_2 + \tilde p_3)}{S + T - 2P}
\end{equation}

with:

\begin{equation}\label{eqn:definition_of_chi}
    \chi = \frac{\tilde p_2 (P - T) + \tilde p_3 (S - P)}
                {\tilde p_2 (P - S) + \tilde p_3 (T - P)}
\end{equation}

Given a strategy \(p\in\mathbb{R}^{4\times 1}\) equations
(\ref{eqn:condition_for_tilde_p4}), (\ref{eqn:planar_definition_of_extortion}-\ref{eqn:definition_of_chi}) can be used to check if
a strategy is extortionate. The conditions correspond to:

\begin{align}
    p_1 & = \frac{(R-P)(p_2 + p_3) - R + T + S - P}{S + T - 2P}
     \label{eqn:condition_for_p1}\\
    p_4 & = 0 \label{eqn:condition_for_p4}\\
    1 & > p_2 + p_3\label{eqn:condition_for_chi}
\end{align}

The algebraic steps necessary to prove these results are available in the
supporting materials.

All extortionate strategies reside on a triangular (\ref{eqn:condition_for_chi})
plane (\ref{eqn:condition_for_p1}) in 3 dimensions (\ref{eqn:condition_for_p4}).
Using this formulation it can be seen that a necessary (but not sufficient)
condition for an extortionate strategy is that it cooperates on average less
than 50\% of the time when in a state of disagreement with the opponent.

As an example, consider the known extortionate strategy \(p=(8 / 9, 1 / 2, 1 /
3, 0)\) from~\cite{Stewart2012} which is referred to as \texttt{Extort-2}. In
this case, for the standard values of \((R, T, S, P)\) constraint
(\ref{eqn:condition_for_p1}) corresponds to:

\begin{equation}
    p_1 = \frac{2(p_2 + p_3) + 1}{3}
\end{equation}

It is clear that in this case all constraints hold.

This approach could in fact be used to confirm that a given strategy is acting
in an extortionate manner even if it is not a memory one strategy. However, in
practice, if a closed form for \(p\) is not known, then due to measurement
and/or numerical error this would not work.

This problem can be written in the following linear algebraic form where
\(x=(\alpha, \beta)\)
and \(p^*=(\tilde p_1 - 1, tilde_2 - 1, p_3)\):

\begin{equation}\label{eqn:linear_algebraic_equation_for_p}
    Cx= p^*
\end{equation}

\(C\) corresponds to equations
(\ref{eqn:condition_for_tilde_p1}-\ref{eqn:condition_for_tilde_p3}) and is
given by:

\begin{equation}\label{eqn:definition_of_C}
    C =
    \begin{bmatrix}
        R - P & R- P \\
        S - P & T- P \\
        T - P & S- P \\
    \end{bmatrix}
\end{equation}

Note that in general, equation (\ref{eqn:linear_algebraic_equation_for_p}) will
not necessarily have a solution. From the Rouch\'{e}-Capelli theorem if there is
a solution it is unique as \(\text{rank}(C)=2\) which is the dimension of the
variable \(x\). The best fitting \(x\) is found by minimizing:

\begin{equation}\label{eqn:r_squared}
    \text{SSError} = \|C x- p^*\|_2^2 = \sum_{i=1}^{3}\left((C\bar x)_i-p_i^*\right)^2
\end{equation}

Note that \(\text{SSError}\), which is the square of the Frobenius
norm~\cite{Golub2013}, becomes a measure of how close a strategy is to being an
extortionate strategy. Suspicion
of extortion then corresponds to a threshold on \(\text{SSError}\).

By observing interactions (human or otherwise), their memory one representation
can be inferred and this approach can be used to recognise extortionate
behaviour. The notion of comparing theoretic and actual plays of the IPD is not
novel, see for example~\cite{Rand2013}. Immediately it is noted that if the
environment is noisy~\cite{Wu1995} then no strategy can be considered to be
extortionate as \(p_4>0\).

In the next section, this idea will be illustrated by observing the interactions
that take place in a computer based tournament of the IPD\@.

\section{Numerical experiments}\label{sec:numerical-experiments}

In~\cite{Stewart2012} results from a tournament with
\input{./assets/tex/number_of_stewart_plotkin_strategies/main.tex} strategies,
was presented with specific consideration given to ZD strategies. This
tournament is reproduced here using the Axelrod-Python
project~\cite{Knight2016}. To obtain a good measure of the corresponding
transition rates for each strategy all matches have been run for
\input{assets/tex/number_of_turns/main.tex} turns and every match has been
repeated \input{assets/tex/number_of_repetitions/main.tex} times. All of this
interaction data is available at~\cite{vincent_knight_2018_1297075}. A good
match between the inferred Markov chain and the state distribution of the actual
interactions has been verified. Data for this is presented in the supplementary
materials.

Figure~\ref{fig:SSError_overall_in_stewart_plotkin} shows the \(\text{SSError}\)
values for all the strategies in the tournament, as reported
in~\cite{Stewart2012} the extortionate strategy (which has an expected
\(\text{SSError}\) approximately 0) gains a large number of wins.

\begin{figure}[!htbp]
    \centering
    \includegraphics[width=.8\textwidth]{./assets/img/SSError_overall_in_stewart_plotkin/main.pdf}
    \caption{\(\text{SSError}\) and state probabilities for the strategies
        of~\cite{Stewart2012}, ordered both by number of wins and overall score.
        Note that \(P(DC)\) is not shown as it corresponds to the transpose of
        \(P(CD)\). Cooperator and Defector are omitted as they do not visit all
        the states.}
    \label{fig:SSError_overall_in_stewart_plotkin}
\end{figure}

Here, the work of~\cite{Stewart2012} is extended by investigating a tournament
with \input{assets/tex/number_of_full_strategies/main.tex}
strategies.

The results of this analysis are shown in
Figure~\ref{fig:SSError_and_probabilities_in_full}. The top ranking strategies
by number of wins seem to be extortionate (but not against all strategies) and
it can be seen that a small sub group of strategies achieve mutual defection.
All the top ranking strategies according to score achieve mutual cooperation and
do not extort each other, however they
\textbf{do} exhibit extortionate behaviour towards a number of the lower ranking
strategies.

\begin{figure}[!htbp]
    \centering
    \includegraphics[width=.8\textwidth]{./assets/img/SSError_and_probabilities_in_full/main.pdf}
    \caption{\(\text{SSError}\) for the strategies for the full tournament. Only
    strategy interactions for which \(p_4=0\) and \(\chi>1\) are displayed.}
    \label{fig:SSError_and_probabilities_in_full}
\end{figure}

\section{Conclusion}\label{sec:conclusion}

This work defines an approach to measure whether or not a player is playing a
strategy that corresponds to an extortionate strategy as defined
in~\cite{Press2012}: a mathematical model for suspicion. Indeed, all
extortionate strategies have been
 classified as lying on a triangular plane.
This rigorous classification fails to be robust to small measurement error, thus
a statistical approach is proposed.
This is done through a linear algebraic approach for approximating the solution
of a linear system. Using this, a large number of pairwise interactions is
simulated and in fact very few strategies are found to act extortionately.

The work of~\cite{Press2012}, whilst showing that a clever approach to taking
advantage of another memory one strategy exists: this is incomplete. Whilst the
elegance of this result is very attractive, just as the simplicity of the
victory of Tit For Tat in Axelrod's original tournaments was, it is incomplete.
Extortionate strategies achieve a high number of wins but they do not
achieve a high score which corresponds to the fitness landscape in an
evolutionary sense. From the large number of interactions a payoff matrix \(S\)
can be measured where \(S_{ij}\) denotes the score (using standard values of
\((R, S, T, P) = (3, 0, 5, 1)\)) of the \(i\)th strategy
against the \(j\)th strategy. Using this, the replicator equation
describes the evolution of the system based on a population density fitness
function:

\begin{equation}\label{eqn:replicator_dynamics}
    \frac{dx}{dt} = x(S-x^TS x)
\end{equation}

Equation (\ref{eqn:replicator_dynamics}) is solved numerically through an
integration technique described in~\cite{Petzold1983} and
Figure~\ref{fig:replicator_dynamics} shows the evolution of the distribution of
the system: the various strategies are ranked by scores. It is clear to see that
only the high ranking strategies survive the evolutionary process (in fact,
only \input{./assets/img/replicator_dynamics/main.tex}
have a final distribution greater than \(10 ^ {-2}\)). This confirms the
findings of~\cite{Moran1707} in which sophisticated strategies resist
evolutionary invasion of shorter memory strategies. Recalling
Figure~\ref{fig:SSError_and_probabilities_in_full} this demonstrates that:

\begin{itemize}
    \item Cooperation emerges through the evolutionary process: the high scoring
        strategies do not exhibit extortionate behaviour towards each other.
    \item Extortionate strategies do not survive the evolutionary process.
\end{itemize}

\begin{figure}[!htbp]
    \centering
    \includegraphics[width=.8\textwidth]{./assets/img/replicator_dynamics/main.pdf}
    \caption{Numerical simulation of the replicator equation
    (\ref{eqn:replicator_dynamics}): strategies are ordered by score, only the strategies with a high score survive the evolutionary process.}
    \label{fig:replicator_dynamics}
\end{figure}

This work can be used to classify plays of the IPD\@: data can be collected from
actual interactions (in lab or in the field). Furthermore, this allows for a
classification method similar to the notion of fingerprinting presented
in~\cite{Ashlock2008}. Trained strategies can potentially be classified as
extortionate or not or it could be possible to even constrain the reinforcement
learning approaches that are becoming prevalent in the literature.
Alternatively, this mathematical approach for recognising extortion could be
used in sophisticated strategies to defend against invasion. Arguably, some of
the strategies considered here exhibit this behaviour, indeed as described
in~\cite{Harper2017}, the top ranking strategies in the full tournament are
obtained using evolutionary reinforcement learning techniques, thus, suspicion
of extortionate behaviour could in fact be an evolutionary trait.

\section*{Acknowledgements}

The following open source software libraries were used in this research:

\begin{itemize}
    \item The Axelrod ~\cite{Knight2016, Knight2018} library (IPD strategies and
        tournaments).
    \item The sympy library~\cite{Meurer2017} (verification of all symbolic
        calculations).
    \item The matplotlib~\cite{Droettboom2018} library (visualisation).
    \item The pandas~\cite{Structures2010}, dask~\cite{Dask2016} and
        NumPy~\cite{Oliphant2015} libraries (data manipulation).
    \item The SciPy~\cite{Jones2001} library (numerical integration of the
        replicator equation).
\end{itemize}

This work was performed using the computational facilities of the Advanced
Research Computing @ Cardiff (ARCCA) Division, Cardiff University.

\printbibliography

\newpage
\section*{Supplementary materials}

\includepdf{assets/pdf/proof_of_form_of_extortionate_strategies/main.pdf}

\newpage

Using the pair wise interactions the transition rates \(p,
q\) can be measured and the steady state probabilities inferred and compared to
the actual probabilities of each state.
This is done numerically by computing the singular eigenvector of the
matrix \(A\) \cite{Stewart2009}:

\[
    A =
    \begin{bmatrix}
        p_1 q_1 & p_1 (1 - q_1) & (1 - p_1) q_1 & (1 -p_1) (1 - q_1) \\
        p_2 q_2 & p_2 (1 - q_2) & (1 - p_2) q_2 & (1 -p_2) (1 - q_2) \\
        p_3 q_3 & p_3 (1 - q_3) & (1 - p_3) q_3 & (1 -p_3) (1 - q_3) \\
        p_4 q_4 & p_4 (1 - q_4) & (1 - p_4) q_4 & (1 -p_4) (1 - q_4) \\
    \end{bmatrix}
\]

Figure~\ref{fig:computed_probabilities_vs_theoretic_probabilities} shows a
regression line fitted to every pairwise interaction with a reported
\(\text{SSError}\) value (pairwise interactions with missing states were
omitted). This serves to validate the approach: a part from some edge cases the
relationship is consistent.

\begin{figure}[!htbp]
    \centering
    \includegraphics[width=.8\textwidth]{./assets/img/computed_probabilities_vs_theoretic_probabilities/main.pdf}
    \caption{The
        relationship between the steady state probabilities inferred from the
        measured transitions and the actual steady state probabilities. A linear
        regression line is included validating the approach.}
    \label{fig:computed_probabilities_vs_theoretic_probabilities}
\end{figure}


\end{document}

strategies.

The results of this analysis are shown in
Figure~\ref{fig:SSError_and_probabilities_in_full}. The top ranking strategies
by number of wins seem to be extortionate (but not against all strategies) and
it can be seen that a small sub group of strategies achieve mutual defection.
All the top ranking strategies according to score achieve mutual cooperation and
do not extort each other, however they
\textbf{do} exhibit extortionate behaviour towards a number of the lower ranking
strategies.

\begin{figure}[!htbp]
    \centering
    \includegraphics[width=.8\textwidth]{./assets/img/SSError_and_probabilities_in_full/main.pdf}
    \caption{\(\text{SSError}\) for the strategies for the full tournament. Only
    strategy interactions for which \(p_4=0\) and \(\chi>1\) are displayed.}
    \label{fig:SSError_and_probabilities_in_full}
\end{figure}

\section{Conclusion}\label{sec:conclusion}

This work defines an approach to measure whether or not a player is playing a
strategy that corresponds to an extortionate strategy as defined
in~\cite{Press2012}: a mathematical model for suspicion. Indeed, all
extortionate strategies have been
 classified as lying on a triangular plane.
This rigorous classification fails to be robust to small measurement error, thus
a statistical approach is proposed.
This is done through a linear algebraic approach for approximating the solution
of a linear system. Using this, a large number of pairwise interactions is
simulated and in fact very few strategies are found to act extortionately.

The work of~\cite{Press2012}, whilst showing that a clever approach to taking
advantage of another memory one strategy exists: this is incomplete. Whilst the
elegance of this result is very attractive, just as the simplicity of the
victory of Tit For Tat in Axelrod's original tournaments was, it is incomplete.
Extortionate strategies achieve a high number of wins but they do not
achieve a high score which corresponds to the fitness landscape in an
evolutionary sense. From the large number of interactions a payoff matrix \(S\)
can be measured where \(S_{ij}\) denotes the score (using standard values of
\((R, S, T, P) = (3, 0, 5, 1)\)) of the \(i\)th strategy
against the \(j\)th strategy. Using this, the replicator equation
describes the evolution of the system based on a population density fitness
function:

\begin{equation}\label{eqn:replicator_dynamics}
    \frac{dx}{dt} = x(S-x^TS x)
\end{equation}

Equation (\ref{eqn:replicator_dynamics}) is solved numerically through an
integration technique described in~\cite{Petzold1983} and
Figure~\ref{fig:replicator_dynamics} shows the evolution of the distribution of
the system: the various strategies are ranked by scores. It is clear to see that
only the high ranking strategies survive the evolutionary process (in fact,
only \documentclass[a4paper]{article}

\usepackage{amsmath}
\usepackage{amssymb}
\usepackage[margin=1.5cm,
            includefoot,
            footskip=30pt]{geometry}
\usepackage{layout}
\usepackage{graphicx}
\usepackage{subcaption}

\usepackage{biblatex}
\usepackage{pdfpages}

\bibliography{main.bib}

\title{Suspicion: Recognising and evaluating the effectiveness
       of extortion in the Iterated Prisoner's Dilemma}
\author{Vincent A. Knight \and Nikoleta E. Glynatsi}
\date{\today}



\begin{document}

\maketitle

\begin{abstract}
    The Iterated Prisoner's Dilemma is a model for rational and evolutionary
    interactive behaviour. It has applications both in the study of human social
    behaviour as well as in biology.
    It is used to understand when and how a rational individual might
    accept an immediate cost to their own utility for the direct benefit of
    another.

    Much attention has been given to a class of strategies called
    Zero Determinant strategies. It has been theoretically shown that these
    strategies can ``extort'' any player.

    In this work, an approach to identify if observed strategies are playing in
    an extortionate way is described. Furthermore, experimental analysis of
    a large tournament with \input{assets/tex/number_of_full_strategies/main.tex}
    strategies is considered. In this setting
    the most highly performing strategies do not play in an extortionate way
    against each other but do against lower performing strategies.
    This suggests that whilst the theory of Zero Determinant strategies
    indicates that memory is not of fundamental importance to the evolution of
    cooperative behaviour, this is incomplete.
\end{abstract}

\section{Introduction}\label{sec:introduction}

Agent based game theoretic models have become a stalwart of the underpinning
mathematics of interactive behaviours. One of the major pieces of work
in this area is the pair of original computer tournaments run by Robert
Axelrod~\cite{Axelrod1980, Axelrod1980a}. These tournaments pitted submitted
computer strategies against each other in plays of the Iterated Prisoner's
Dilemma. A common game where agents can choose to pay a slight cost to their
immediate utility in the hope of building a reputation. This has been used in
economic and evolutionary game theory to understand the evolution of cooperative
behaviour.

Recently, a class of strategies was described in~\cite{Press2012} that can
provably extort any given opponent. In~\cite{Hilbe2013, Moran1707} some
questions have already been asked about the true effectiveness of these
strategies in an evolutionary setting. Here another question is asked: is it
possible to recognise this extortionate behaviour? A mathematical procedure for
suspicion is presented: in the same way that the continued actions of an
extortionate individual might raise suspicion.

This work makes use of the Axelrod Python library~\cite{Knight2018, Knight2016}
with a large number of Prisoner Dilemma strategies available to give an
extensive numerical example of the ideas presented.  The approach is presented
in Section~\ref{sec:delta-zd-strategies}.  All of the code and data discussed
in Section~\ref{sec:numerical-experiments} is open sourced, archived and
written according to best scientific principles~\cite{Wilson2014}. The data
archive can be found at~\cite{vincent_knight_2018_1297075}.

\section{Recognising Extortion}\label{sec:delta-zd-strategies}

In~\cite{Press2012}, given a match between 2 memory-one strategies, the concept
of Zero Determinant (ZD) strategies is introduced. The main result of that paper
shows that given two memory one players \(p, q\in\mathbb{R}^4\) a linear
relationship between the players' scores could be forced by one of the players.

Using the notation of~\cite{Press2012}, assuming the utilities for player \(p\)
are given by \(S_x=(R, S, T, P)\) and for player \(q\) by \(S_y=(R, T, S, P)\)
and that the stationary scores of each player is given by \(S_X\) and \(S_Y\)
respectively. The main result of~\cite{Press2012} is that if

\begin{equation}\label{eqn:linear_relationship_for_p}
    \tilde p=\alpha S_x + \beta S_y + \gamma
\end{equation}

or

\begin{equation}\label{eqn:linear_relationship_for_q}
    \tilde q=\alpha S_x + \beta S_y + \gamma
\end{equation}

where \(\tilde p = (1 - p_1, 1 - p_2, p_3, p_4)\) and
\(\tilde q = (1 - q_1, 1 - q_2, q_3, q_4)\) then:

\begin{equation}
    \alpha S_X + \beta S_Y + \gamma = 0
\end{equation}

In~\cite{Press2012} a particular type of ZD strategy is defined: extortionate
strategies. If:

\begin{equation}\label{eqn:constraint_for_extortion}
    \gamma = - P(\alpha + \beta)
\end{equation}

then the player can ensure they get a score \(\chi\) times
larger than the opponent. This extortion coefficient is given by:

\begin{equation}\label{eqn:definition_of_chi}
    \chi=\frac{-\beta}{\alpha}
\end{equation}

Thus, if (\ref{eqn:constraint_for_extortion}) holds and \(\chi >1\) a player is
said to extort their opponent.
Here, the reverse problem is considered: given a
\(p\in\mathbb{R}^4\) how does one identify \(\alpha, \beta\) if they
exist and is the strategy in fact acting in an extortionate way?

These conditions correspond to:

\begin{align}
    \tilde p_1 & = \alpha R + \beta R - P (\alpha + \beta)
            \label{eqn:condition_for_tilde_p1}\\
    \tilde p_2 & = \alpha S + \beta T - P (\alpha + \beta)
            \label{eqn:condition_for_tilde_p2}\\
    \tilde p_3 & = \alpha T + \beta S - P (\alpha + \beta)
            \label{eqn:condition_for_tilde_p3}\\
    \tilde p_4 & = \alpha P + \beta P - P (\alpha + \beta)
            \label{eqn:condition_for_tilde_p4}
\end{align}

Equation (\ref{eqn:condition_for_tilde_p4}) ensures that \(p_4=\tilde p_4=0\).
Equations (\ref{eqn:condition_for_tilde_p1}-\ref{eqn:condition_for_tilde_p3})
can be used to eliminate \(\alpha, \beta\), giving:

\begin{equation}\label{eqn:planar_definition_of_extortion}
    \tilde p_1 = \frac{(R - P)(\tilde p_2 + \tilde p_3)}{S + T - 2P}
\end{equation}

with:

\begin{equation}\label{eqn:definition_of_chi}
    \chi = \frac{\tilde p_2 (P - T) + \tilde p_3 (S - P)}
                {\tilde p_2 (P - S) + \tilde p_3 (T - P)}
\end{equation}

Given a strategy \(p\in\mathbb{R}^{4\times 1}\) equations
(\ref{eqn:condition_for_tilde_p4}), (\ref{eqn:planar_definition_of_extortion}-\ref{eqn:definition_of_chi}) can be used to check if
a strategy is extortionate. The conditions correspond to:

\begin{align}
    p_1 & = \frac{(R-P)(p_2 + p_3) - R + T + S - P}{S + T - 2P}
     \label{eqn:condition_for_p1}\\
    p_4 & = 0 \label{eqn:condition_for_p4}\\
    1 & > p_2 + p_3\label{eqn:condition_for_chi}
\end{align}

The algebraic steps necessary to prove these results are available in the
supporting materials.

All extortionate strategies reside on a triangular (\ref{eqn:condition_for_chi})
plane (\ref{eqn:condition_for_p1}) in 3 dimensions (\ref{eqn:condition_for_p4}).
Using this formulation it can be seen that a necessary (but not sufficient)
condition for an extortionate strategy is that it cooperates on average less
than 50\% of the time when in a state of disagreement with the opponent.

As an example, consider the known extortionate strategy \(p=(8 / 9, 1 / 2, 1 /
3, 0)\) from~\cite{Stewart2012} which is referred to as \texttt{Extort-2}. In
this case, for the standard values of \((R, T, S, P)\) constraint
(\ref{eqn:condition_for_p1}) corresponds to:

\begin{equation}
    p_1 = \frac{2(p_2 + p_3) + 1}{3}
\end{equation}

It is clear that in this case all constraints hold.

This approach could in fact be used to confirm that a given strategy is acting
in an extortionate manner even if it is not a memory one strategy. However, in
practice, if a closed form for \(p\) is not known, then due to measurement
and/or numerical error this would not work.

This problem can be written in the following linear algebraic form where
\(x=(\alpha, \beta)\)
and \(p^*=(\tilde p_1 - 1, tilde_2 - 1, p_3)\):

\begin{equation}\label{eqn:linear_algebraic_equation_for_p}
    Cx= p^*
\end{equation}

\(C\) corresponds to equations
(\ref{eqn:condition_for_tilde_p1}-\ref{eqn:condition_for_tilde_p3}) and is
given by:

\begin{equation}\label{eqn:definition_of_C}
    C =
    \begin{bmatrix}
        R - P & R- P \\
        S - P & T- P \\
        T - P & S- P \\
    \end{bmatrix}
\end{equation}

Note that in general, equation (\ref{eqn:linear_algebraic_equation_for_p}) will
not necessarily have a solution. From the Rouch\'{e}-Capelli theorem if there is
a solution it is unique as \(\text{rank}(C)=2\) which is the dimension of the
variable \(x\). The best fitting \(x\) is found by minimizing:

\begin{equation}\label{eqn:r_squared}
    \text{SSError} = \|C x- p^*\|_2^2 = \sum_{i=1}^{3}\left((C\bar x)_i-p_i^*\right)^2
\end{equation}

Note that \(\text{SSError}\), which is the square of the Frobenius
norm~\cite{Golub2013}, becomes a measure of how close a strategy is to being an
extortionate strategy. Suspicion
of extortion then corresponds to a threshold on \(\text{SSError}\).

By observing interactions (human or otherwise), their memory one representation
can be inferred and this approach can be used to recognise extortionate
behaviour. The notion of comparing theoretic and actual plays of the IPD is not
novel, see for example~\cite{Rand2013}. Immediately it is noted that if the
environment is noisy~\cite{Wu1995} then no strategy can be considered to be
extortionate as \(p_4>0\).

In the next section, this idea will be illustrated by observing the interactions
that take place in a computer based tournament of the IPD\@.

\section{Numerical experiments}\label{sec:numerical-experiments}

In~\cite{Stewart2012} results from a tournament with
\input{./assets/tex/number_of_stewart_plotkin_strategies/main.tex} strategies,
was presented with specific consideration given to ZD strategies. This
tournament is reproduced here using the Axelrod-Python
project~\cite{Knight2016}. To obtain a good measure of the corresponding
transition rates for each strategy all matches have been run for
\input{assets/tex/number_of_turns/main.tex} turns and every match has been
repeated \input{assets/tex/number_of_repetitions/main.tex} times. All of this
interaction data is available at~\cite{vincent_knight_2018_1297075}. A good
match between the inferred Markov chain and the state distribution of the actual
interactions has been verified. Data for this is presented in the supplementary
materials.

Figure~\ref{fig:SSError_overall_in_stewart_plotkin} shows the \(\text{SSError}\)
values for all the strategies in the tournament, as reported
in~\cite{Stewart2012} the extortionate strategy (which has an expected
\(\text{SSError}\) approximately 0) gains a large number of wins.

\begin{figure}[!htbp]
    \centering
    \includegraphics[width=.8\textwidth]{./assets/img/SSError_overall_in_stewart_plotkin/main.pdf}
    \caption{\(\text{SSError}\) and state probabilities for the strategies
        of~\cite{Stewart2012}, ordered both by number of wins and overall score.
        Note that \(P(DC)\) is not shown as it corresponds to the transpose of
        \(P(CD)\). Cooperator and Defector are omitted as they do not visit all
        the states.}
    \label{fig:SSError_overall_in_stewart_plotkin}
\end{figure}

Here, the work of~\cite{Stewart2012} is extended by investigating a tournament
with \input{assets/tex/number_of_full_strategies/main.tex}
strategies.

The results of this analysis are shown in
Figure~\ref{fig:SSError_and_probabilities_in_full}. The top ranking strategies
by number of wins seem to be extortionate (but not against all strategies) and
it can be seen that a small sub group of strategies achieve mutual defection.
All the top ranking strategies according to score achieve mutual cooperation and
do not extort each other, however they
\textbf{do} exhibit extortionate behaviour towards a number of the lower ranking
strategies.

\begin{figure}[!htbp]
    \centering
    \includegraphics[width=.8\textwidth]{./assets/img/SSError_and_probabilities_in_full/main.pdf}
    \caption{\(\text{SSError}\) for the strategies for the full tournament. Only
    strategy interactions for which \(p_4=0\) and \(\chi>1\) are displayed.}
    \label{fig:SSError_and_probabilities_in_full}
\end{figure}

\section{Conclusion}\label{sec:conclusion}

This work defines an approach to measure whether or not a player is playing a
strategy that corresponds to an extortionate strategy as defined
in~\cite{Press2012}: a mathematical model for suspicion. Indeed, all
extortionate strategies have been
 classified as lying on a triangular plane.
This rigorous classification fails to be robust to small measurement error, thus
a statistical approach is proposed.
This is done through a linear algebraic approach for approximating the solution
of a linear system. Using this, a large number of pairwise interactions is
simulated and in fact very few strategies are found to act extortionately.

The work of~\cite{Press2012}, whilst showing that a clever approach to taking
advantage of another memory one strategy exists: this is incomplete. Whilst the
elegance of this result is very attractive, just as the simplicity of the
victory of Tit For Tat in Axelrod's original tournaments was, it is incomplete.
Extortionate strategies achieve a high number of wins but they do not
achieve a high score which corresponds to the fitness landscape in an
evolutionary sense. From the large number of interactions a payoff matrix \(S\)
can be measured where \(S_{ij}\) denotes the score (using standard values of
\((R, S, T, P) = (3, 0, 5, 1)\)) of the \(i\)th strategy
against the \(j\)th strategy. Using this, the replicator equation
describes the evolution of the system based on a population density fitness
function:

\begin{equation}\label{eqn:replicator_dynamics}
    \frac{dx}{dt} = x(S-x^TS x)
\end{equation}

Equation (\ref{eqn:replicator_dynamics}) is solved numerically through an
integration technique described in~\cite{Petzold1983} and
Figure~\ref{fig:replicator_dynamics} shows the evolution of the distribution of
the system: the various strategies are ranked by scores. It is clear to see that
only the high ranking strategies survive the evolutionary process (in fact,
only \input{./assets/img/replicator_dynamics/main.tex}
have a final distribution greater than \(10 ^ {-2}\)). This confirms the
findings of~\cite{Moran1707} in which sophisticated strategies resist
evolutionary invasion of shorter memory strategies. Recalling
Figure~\ref{fig:SSError_and_probabilities_in_full} this demonstrates that:

\begin{itemize}
    \item Cooperation emerges through the evolutionary process: the high scoring
        strategies do not exhibit extortionate behaviour towards each other.
    \item Extortionate strategies do not survive the evolutionary process.
\end{itemize}

\begin{figure}[!htbp]
    \centering
    \includegraphics[width=.8\textwidth]{./assets/img/replicator_dynamics/main.pdf}
    \caption{Numerical simulation of the replicator equation
    (\ref{eqn:replicator_dynamics}): strategies are ordered by score, only the strategies with a high score survive the evolutionary process.}
    \label{fig:replicator_dynamics}
\end{figure}

This work can be used to classify plays of the IPD\@: data can be collected from
actual interactions (in lab or in the field). Furthermore, this allows for a
classification method similar to the notion of fingerprinting presented
in~\cite{Ashlock2008}. Trained strategies can potentially be classified as
extortionate or not or it could be possible to even constrain the reinforcement
learning approaches that are becoming prevalent in the literature.
Alternatively, this mathematical approach for recognising extortion could be
used in sophisticated strategies to defend against invasion. Arguably, some of
the strategies considered here exhibit this behaviour, indeed as described
in~\cite{Harper2017}, the top ranking strategies in the full tournament are
obtained using evolutionary reinforcement learning techniques, thus, suspicion
of extortionate behaviour could in fact be an evolutionary trait.

\section*{Acknowledgements}

The following open source software libraries were used in this research:

\begin{itemize}
    \item The Axelrod ~\cite{Knight2016, Knight2018} library (IPD strategies and
        tournaments).
    \item The sympy library~\cite{Meurer2017} (verification of all symbolic
        calculations).
    \item The matplotlib~\cite{Droettboom2018} library (visualisation).
    \item The pandas~\cite{Structures2010}, dask~\cite{Dask2016} and
        NumPy~\cite{Oliphant2015} libraries (data manipulation).
    \item The SciPy~\cite{Jones2001} library (numerical integration of the
        replicator equation).
\end{itemize}

This work was performed using the computational facilities of the Advanced
Research Computing @ Cardiff (ARCCA) Division, Cardiff University.

\printbibliography

\newpage
\section*{Supplementary materials}

\includepdf{assets/pdf/proof_of_form_of_extortionate_strategies/main.pdf}

\newpage

Using the pair wise interactions the transition rates \(p,
q\) can be measured and the steady state probabilities inferred and compared to
the actual probabilities of each state.
This is done numerically by computing the singular eigenvector of the
matrix \(A\) \cite{Stewart2009}:

\[
    A =
    \begin{bmatrix}
        p_1 q_1 & p_1 (1 - q_1) & (1 - p_1) q_1 & (1 -p_1) (1 - q_1) \\
        p_2 q_2 & p_2 (1 - q_2) & (1 - p_2) q_2 & (1 -p_2) (1 - q_2) \\
        p_3 q_3 & p_3 (1 - q_3) & (1 - p_3) q_3 & (1 -p_3) (1 - q_3) \\
        p_4 q_4 & p_4 (1 - q_4) & (1 - p_4) q_4 & (1 -p_4) (1 - q_4) \\
    \end{bmatrix}
\]

Figure~\ref{fig:computed_probabilities_vs_theoretic_probabilities} shows a
regression line fitted to every pairwise interaction with a reported
\(\text{SSError}\) value (pairwise interactions with missing states were
omitted). This serves to validate the approach: a part from some edge cases the
relationship is consistent.

\begin{figure}[!htbp]
    \centering
    \includegraphics[width=.8\textwidth]{./assets/img/computed_probabilities_vs_theoretic_probabilities/main.pdf}
    \caption{The
        relationship between the steady state probabilities inferred from the
        measured transitions and the actual steady state probabilities. A linear
        regression line is included validating the approach.}
    \label{fig:computed_probabilities_vs_theoretic_probabilities}
\end{figure}


\end{document}

have a final distribution greater than \(10 ^ {-2}\)). This confirms the
findings of~\cite{Moran1707} in which sophisticated strategies resist
evolutionary invasion of shorter memory strategies. Recalling
Figure~\ref{fig:SSError_and_probabilities_in_full} this demonstrates that:

\begin{itemize}
    \item Cooperation emerges through the evolutionary process: the high scoring
        strategies do not exhibit extortionate behaviour towards each other.
    \item Extortionate strategies do not survive the evolutionary process.
\end{itemize}

\begin{figure}[!htbp]
    \centering
    \includegraphics[width=.8\textwidth]{./assets/img/replicator_dynamics/main.pdf}
    \caption{Numerical simulation of the replicator equation
    (\ref{eqn:replicator_dynamics}): strategies are ordered by score, only the strategies with a high score survive the evolutionary process.}
    \label{fig:replicator_dynamics}
\end{figure}

This work can be used to classify plays of the IPD\@: data can be collected from
actual interactions (in lab or in the field). Furthermore, this allows for a
classification method similar to the notion of fingerprinting presented
in~\cite{Ashlock2008}. Trained strategies can potentially be classified as
extortionate or not or it could be possible to even constrain the reinforcement
learning approaches that are becoming prevalent in the literature.
Alternatively, this mathematical approach for recognising extortion could be
used in sophisticated strategies to defend against invasion. Arguably, some of
the strategies considered here exhibit this behaviour, indeed as described
in~\cite{Harper2017}, the top ranking strategies in the full tournament are
obtained using evolutionary reinforcement learning techniques, thus, suspicion
of extortionate behaviour could in fact be an evolutionary trait.

\section*{Acknowledgements}

The following open source software libraries were used in this research:

\begin{itemize}
    \item The Axelrod ~\cite{Knight2016, Knight2018} library (IPD strategies and
        tournaments).
    \item The sympy library~\cite{Meurer2017} (verification of all symbolic
        calculations).
    \item The matplotlib~\cite{Droettboom2018} library (visualisation).
    \item The pandas~\cite{Structures2010}, dask~\cite{Dask2016} and
        NumPy~\cite{Oliphant2015} libraries (data manipulation).
    \item The SciPy~\cite{Jones2001} library (numerical integration of the
        replicator equation).
\end{itemize}

This work was performed using the computational facilities of the Advanced
Research Computing @ Cardiff (ARCCA) Division, Cardiff University.

\printbibliography

\newpage
\section*{Supplementary materials}

\includepdf{assets/pdf/proof_of_form_of_extortionate_strategies/main.pdf}

\newpage

Using the pair wise interactions the transition rates \(p,
q\) can be measured and the steady state probabilities inferred and compared to
the actual probabilities of each state.
This is done numerically by computing the singular eigenvector of the
matrix \(A\) \cite{Stewart2009}:

\[
    A =
    \begin{bmatrix}
        p_1 q_1 & p_1 (1 - q_1) & (1 - p_1) q_1 & (1 -p_1) (1 - q_1) \\
        p_2 q_2 & p_2 (1 - q_2) & (1 - p_2) q_2 & (1 -p_2) (1 - q_2) \\
        p_3 q_3 & p_3 (1 - q_3) & (1 - p_3) q_3 & (1 -p_3) (1 - q_3) \\
        p_4 q_4 & p_4 (1 - q_4) & (1 - p_4) q_4 & (1 -p_4) (1 - q_4) \\
    \end{bmatrix}
\]

Figure~\ref{fig:computed_probabilities_vs_theoretic_probabilities} shows a
regression line fitted to every pairwise interaction with a reported
\(\text{SSError}\) value (pairwise interactions with missing states were
omitted). This serves to validate the approach: a part from some edge cases the
relationship is consistent.

\begin{figure}[!htbp]
    \centering
    \includegraphics[width=.8\textwidth]{./assets/img/computed_probabilities_vs_theoretic_probabilities/main.pdf}
    \caption{The
        relationship between the steady state probabilities inferred from the
        measured transitions and the actual steady state probabilities. A linear
        regression line is included validating the approach.}
    \label{fig:computed_probabilities_vs_theoretic_probabilities}
\end{figure}


\end{document}

have a final distribution greater than \(10 ^ {-2}\)). This confirms the
findings of~\cite{Moran1707} in which sophisticated strategies resist
evolutionary invasion of shorter memory strategies. Recalling
Figure~\ref{fig:SSError_and_probabilities_in_full} this demonstrates that:

\begin{itemize}
    \item Cooperation emerges through the evolutionary process: the high scoring
        strategies do not exhibit extortionate behaviour towards each other.
    \item Extortionate strategies do not survive the evolutionary process.
\end{itemize}

\begin{figure}[!htbp]
    \centering
    \includegraphics[width=.8\textwidth]{./assets/img/replicator_dynamics/main.pdf}
    \caption{Numerical simulation of the replicator equation
    (\ref{eqn:replicator_dynamics}): strategies are ordered by score, only the strategies with a high score survive the evolutionary process.}
    \label{fig:replicator_dynamics}
\end{figure}

This work can be used to classify plays of the IPD\@: data can be collected from
actual interactions (in lab or in the field). Furthermore, this allows for a
classification method similar to the notion of fingerprinting presented
in~\cite{Ashlock2008}. Trained strategies can potentially be classified as
extortionate or not or it could be possible to even constrain the reinforcement
learning approaches that are becoming prevalent in the literature.
Alternatively, this mathematical approach for recognising extortion could be
used in sophisticated strategies to defend against invasion. Arguably, some of
the strategies considered here exhibit this behaviour, indeed as described
in~\cite{Harper2017}, the top ranking strategies in the full tournament are
obtained using evolutionary reinforcement learning techniques, thus, suspicion
of extortionate behaviour could in fact be an evolutionary trait.

\section*{Acknowledgements}

The following open source software libraries were used in this research:

\begin{itemize}
    \item The Axelrod ~\cite{Knight2016, Knight2018} library (IPD strategies and
        tournaments).
    \item The sympy library~\cite{Meurer2017} (verification of all symbolic
        calculations).
    \item The matplotlib~\cite{Droettboom2018} library (visualisation).
    \item The pandas~\cite{Structures2010}, dask~\cite{Dask2016} and
        NumPy~\cite{Oliphant2015} libraries (data manipulation).
    \item The SciPy~\cite{Jones2001} library (numerical integration of the
        replicator equation).
\end{itemize}

This work was performed using the computational facilities of the Advanced
Research Computing @ Cardiff (ARCCA) Division, Cardiff University.

\printbibliography

\newpage
\section*{Supplementary materials}

\includepdf{assets/pdf/proof_of_form_of_extortionate_strategies/main.pdf}

\newpage

Using the pair wise interactions the transition rates \(p,
q\) can be measured and the steady state probabilities inferred and compared to
the actual probabilities of each state.
This is done numerically by computing the singular eigenvector of the
matrix \(A\) \cite{Stewart2009}:

\[
    A =
    \begin{bmatrix}
        p_1 q_1 & p_1 (1 - q_1) & (1 - p_1) q_1 & (1 -p_1) (1 - q_1) \\
        p_2 q_2 & p_2 (1 - q_2) & (1 - p_2) q_2 & (1 -p_2) (1 - q_2) \\
        p_3 q_3 & p_3 (1 - q_3) & (1 - p_3) q_3 & (1 -p_3) (1 - q_3) \\
        p_4 q_4 & p_4 (1 - q_4) & (1 - p_4) q_4 & (1 -p_4) (1 - q_4) \\
    \end{bmatrix}
\]

Figure~\ref{fig:computed_probabilities_vs_theoretic_probabilities} shows a
regression line fitted to every pairwise interaction with a reported
\(\text{SSError}\) value (pairwise interactions with missing states were
omitted). This serves to validate the approach: a part from some edge cases the
relationship is consistent.

\begin{figure}[!htbp]
    \centering
    \includegraphics[width=.8\textwidth]{./assets/img/computed_probabilities_vs_theoretic_probabilities/main.pdf}
    \caption{The
        relationship between the steady state probabilities inferred from the
        measured transitions and the actual steady state probabilities. A linear
        regression line is included validating the approach.}
    \label{fig:computed_probabilities_vs_theoretic_probabilities}
\end{figure}


\end{document}
strategies. The results of
this analysis are shown in Figure~\ref{fig:sse_chi_probabilities_in_full}. The
top ranking strategies by number of wins act in an extortionate way (but not
against all opponents) and it can be seen that a small subgroup of strategies
achieve mutual defection.  All the top ranking strategies according to score
do not extort each other, however they
\textbf{do} exhibit extortionate behaviour towards a number of the lower ranking
strategies.

\begin{figure}[!htbp]
    \centering
    \includegraphics[width=.95\textwidth]{./assets/img/sse_chi_probabilities_in_full/main.pdf}
    \caption{\(\SSe\) and \(P(DD)\) and state probabilities for the strategies for
        the full tournament. The strategies with high number of wins
        have a low \(\SSe\) however are often locked in mutual defection as
        evidenced by a high \(P(DD)\). The strategies with a high score
        have a high \(\SSe\) against the other high scoring strategies
        indicating that fixed linear relationship is being enforced. However
        against the low scoring strategies they have a lower \(\SSe\) and
        against the very lowest scoring strategies a high \(P(DD)\).}
    \label{fig:sse_chi_probabilities_in_full}
\end{figure}

Note that while a strategy may attempt to act extortionately, not all opponents
can be effectively extorted. For example, a strategy that always defects never
receives a lower score than its opponent. As defined by \cite{Press2012}, an
extortionate ZD strategy will mutually defect with such an opponent which
corresponds to the high values of \(P(DD)\) seen in
Figure~\ref{fig:sse_chi_probabilities_in_full} the top left quadrant.

A detailed look at selected strategies is given in
Table~\ref{tbl:overall_summary_results}. The high scoring strategies presented
have a negatively skewed \(\SSe\) whilst the ZD strategies have a low score but
high probability of winning and higher probability of mutual defection.
The skew of \(\SSe\) of all strategies is shown in
Figure~\ref{fig:sserror_in_std} and supports the
same conclusion. This evidences an idea proposed
in~\cite{adami2013evolutionary}: sophisticated strategies are able to recognise
their opponent and defend themselves against extortion.  The high ranking
strategies were in fact trained to maximise score~\cite{Harper2017} which seems
to have created strategies able to extort weaker strategies whilst cooperating
with stronger ones. Indeed unconditional extortion is self defeating.

\begin{table}[!hbtp]
    \begin{center}
    \small
    \begin{tabular}{rlrrrrrrrrrrr}
\toprule
 Rank &                   Name &  Score per turn &  $P($Win$)$ &  $P(CC)$ &  Overall $\kappa$ &  Min $\kappa$ &  5\% CI $\kappa$ &  Median $\kappa$ &  Mean $\kappa$ &  Std $\kappa$ &  95\% CI $\kappa$ &  Max $\kappa$ \\
\midrule
    1 &   EvolvedLookerUp2\_2\_2 &           2.944 &       0.230 &    0.673 &             0.230 &         0.015 &           0.059 &            0.972 &          0.844 &         0.441 &            1.234 &         2.468 \\
    2 &          Evolved HMM 5 &           2.944 &       0.205 &    0.718 &             0.102 &         0.030 &           0.073 &            0.116 &          0.121 &         0.040 &            0.211 &         0.235 \\
    3 &      PSO Gambler 2\_2\_2 &           2.913 &       0.204 &    0.624 &             0.200 &         0.009 &           0.033 &            0.246 &          0.334 &         0.338 &            1.114 &         1.655 \\
    4 &       PSO Gambler Mem1 &           2.908 &       0.211 &    0.715 &             0.068 &         0.054 &           0.064 &            0.068 &          0.069 &         0.006 &            0.081 &         0.096 \\
    5 &      PSO Gambler 1\_1\_1 &           2.906 &       0.221 &    0.696 &             0.138 &         0.044 &           0.050 &            0.196 &          0.141 &         0.069 &            0.202 &         0.249 \\
    7 &          Evolved ANN 5 &           2.893 &       0.225 &    0.682 &             0.001 &         0.000 &           0.000 &            0.013 &          0.024 &         0.038 &            0.075 &         0.235 \\
   31 &              ZD-GTFT-2 &           2.721 &       0.000 &    0.806 &             0.066 &         0.020 &           0.043 &            0.066 &          0.065 &         0.012 &            0.078 &         0.101 \\
   45 &               ZD-GEN-2 &           2.689 &       0.016 &    0.801 &             0.016 &         0.003 &           0.014 &            0.017 &          0.018 &         0.008 &            0.026 &         0.065 \\
   47 &              Eatherley &           2.682 &       0.000 &    0.828 &             0.084 &         0.000 &           0.000 &            0.179 &          0.291 &         0.343 &            1.000 &         1.209 \\
   69 &            Tit For Tat &           2.638 &       0.000 &    0.723 &             0.000 &         0.000 &           0.000 &            0.000 &          0.000 &         0.000 &            0.000 &         0.000 \\
   75 &                 Grumpy &           2.630 &       0.075 &    0.793 &             0.000 &         0.000 &           0.000 &            0.000 &          0.000 &         0.001 &            0.002 &         0.002 \\
   88 &    Win-Stay Lose-Shift &           2.616 &       0.099 &    0.649 &             1.235 &         1.235 &           1.235 &            1.235 &          1.235 &         0.000 &            1.235 &         1.235 \\
  103 &  Eventual Cycle Hunter &           2.565 &       0.067 &    0.770 &             0.000 &         0.000 &           0.001 &            0.012 &          0.019 &         0.042 &            0.044 &         0.250 \\
  107 &  Tricky Level Punisher &           2.537 &       0.062 &    0.828 &             0.000 &         0.000 &           0.000 &            0.000 &          0.029 &         0.064 &            0.223 &         0.236 \\
  127 &               Adaptive &           2.272 &       0.500 &    0.363 &             0.000 &         0.000 &           0.000 &            0.059 &          0.052 &         0.056 &            0.129 &         0.261 \\
  179 &             Alternator &           1.945 &       0.392 &    0.157 &             1.529 &         1.529 &           1.529 &            1.529 &          1.529 &         0.000 &            1.529 &         1.529 \\
  188 &               Hopeless &           1.908 &       0.352 &    0.261 &             2.471 &         2.471 &           2.471 &            2.471 &          2.471 &         0.000 &            2.471 &         2.471 \\
  189 &       Anti Tit For Tat &           1.906 &       0.334 &    0.144 &             1.529 &         1.529 &           1.529 &            1.529 &          1.529 &         0.000 &            1.529 &         1.529 \\
  194 &         Gradual Killer &           1.892 &       0.354 &    0.439 &             0.000 &         0.000 &           0.000 &            0.059 &          0.094 &         0.213 &            0.250 &         1.000 \\
  196 &             Aggravater &           1.879 &       0.930 &    0.087 &             0.235 &         0.105 &           0.105 &            0.235 &          0.215 &         0.038 &            0.235 &         0.235 \\
  200 &            ZD-Extort-2 &           1.821 &       0.851 &    0.179 &             0.000 &         0.000 &           0.000 &            0.000 &          0.001 &         0.003 &            0.004 &         0.026 \\
  201 &            ZD-Extort-4 &           1.820 &       0.865 &    0.106 &             0.000 &         0.000 &           0.000 &            0.000 &          0.001 &         0.002 &            0.003 &         0.011 \\
  202 &             ZD-Extort3 &           1.810 &       0.862 &    0.133 &             0.000 &         0.000 &           0.000 &            0.000 &          0.002 &         0.005 &            0.007 &         0.060 \\
  204 &              Handshake &           1.806 &       0.870 &    0.046 &             0.000 &         0.000 &           0.000 &            0.000 &          0.071 &         0.216 &            0.404 &         1.000 \\
\bottomrule
\end{tabular}

    \end{center}
    \caption{Summary of results for a selected list of strategies. Similarly to
        Figure~\ref{fig:sserror_in_stewart_plotkin}, the high scoring strategies
        have a negatively skewed \(\SSe\). The strategies with a
        large number of wins have a low \(\SSe\) and positively skewed
        \(\SSe\). Note that a value of \(\chi=0.063\) and \(\SSe=1.235\)
        corresponds to a vector \(p=(1,1,1,1)\) which highlights that the high
        scoring strategies, adapt and in fact cooperate often.}
    \label{tbl:overall_summary_results}
\end{table}

\begin{figure}[!htbp]
    \centering
    \includegraphics[width=\textwidth]{./assets/img/sserror_in_std/main.pdf}
    \caption{\(\SSe\) for all strategies considered over all opponents.
        A similar conclusion to that of
        Figure~\ref{fig:sserror_in_stewart_plotkin} can be made: the strategies
        that score highly have a negatively skewed \(\SSe\) highlighting their
        ability to adapt to their opponent. The auxiliary materials include a
        version of this graphic with strategy names.}
        \label{fig:sserror_in_std}
\end{figure}

\section{Evolutionary dynamics}\label{sec:evolutionary-dynamics}

\subsection{Replicator Dynamics}

From the large number of interactions a payoff matrix \(S\)
can be measured where \(S_{ij}\) denotes the score (using standard values of
\((R, S, T, P) = (3, 0, 5, 1)\)) of the \(i\)th strategy against the \(j\)th
strategy. This defines a fitness landscape for which the replicator equation
describes the evolution of a population of strategies:

\begin{equation}\label{eqn:replicator_dynamics}
     \frac{d x_i}{dt} = x_i ((Sx)_i - x^T S x)
\end{equation}

Equation (\ref{eqn:replicator_dynamics}) is solved numerically through an
integration technique described in~\cite{Petzold1983} until a stationary vector
\(x=s\) is found.
Figure~\ref{fig:replicator_dynamics} shows the stationary probabilities for each
strategy ranked by score.
It is clear to see that
only the high ranking strategies survive the evolutionary process (in fact,
only \documentclass[a4paper]{article}

\usepackage{amsmath}
\usepackage{amssymb}
\usepackage[margin=1.5cm,
            includefoot,
            footskip=30pt]{geometry}
\usepackage{layout}
\usepackage{graphicx}
\usepackage{subcaption}

\usepackage{biblatex}
\usepackage{pdfpages}

\bibliography{main.bib}

\title{Suspicion: Recognising and evaluating the effectiveness
       of extortion in the Iterated Prisoner's Dilemma}
\author{Vincent A. Knight \and Nikoleta E. Glynatsi}
\date{\today}



\begin{document}

\maketitle

\begin{abstract}
    The Iterated Prisoner's Dilemma is a model for rational and evolutionary
    interactive behaviour. It has applications both in the study of human social
    behaviour as well as in biology.
    It is used to understand when and how a rational individual might
    accept an immediate cost to their own utility for the direct benefit of
    another.

    Much attention has been given to a class of strategies called
    Zero Determinant strategies. It has been theoretically shown that these
    strategies can ``extort'' any player.

    In this work, an approach to identify if observed strategies are playing in
    an extortionate way is described. Furthermore, experimental analysis of
    a large tournament with \documentclass[a4paper]{article}

\usepackage{amsmath}
\usepackage{amssymb}
\usepackage[margin=1.5cm,
            includefoot,
            footskip=30pt]{geometry}
\usepackage{layout}
\usepackage{graphicx}
\usepackage{subcaption}

\usepackage{biblatex}
\usepackage{pdfpages}

\bibliography{main.bib}

\title{Suspicion: Recognising and evaluating the effectiveness
       of extortion in the Iterated Prisoner's Dilemma}
\author{Vincent A. Knight \and Nikoleta E. Glynatsi}
\date{\today}



\begin{document}

\maketitle

\begin{abstract}
    The Iterated Prisoner's Dilemma is a model for rational and evolutionary
    interactive behaviour. It has applications both in the study of human social
    behaviour as well as in biology.
    It is used to understand when and how a rational individual might
    accept an immediate cost to their own utility for the direct benefit of
    another.

    Much attention has been given to a class of strategies called
    Zero Determinant strategies. It has been theoretically shown that these
    strategies can ``extort'' any player.

    In this work, an approach to identify if observed strategies are playing in
    an extortionate way is described. Furthermore, experimental analysis of
    a large tournament with \documentclass[a4paper]{article}

\usepackage{amsmath}
\usepackage{amssymb}
\usepackage[margin=1.5cm,
            includefoot,
            footskip=30pt]{geometry}
\usepackage{layout}
\usepackage{graphicx}
\usepackage{subcaption}

\usepackage{biblatex}
\usepackage{pdfpages}

\bibliography{main.bib}

\title{Suspicion: Recognising and evaluating the effectiveness
       of extortion in the Iterated Prisoner's Dilemma}
\author{Vincent A. Knight \and Nikoleta E. Glynatsi}
\date{\today}



\begin{document}

\maketitle

\begin{abstract}
    The Iterated Prisoner's Dilemma is a model for rational and evolutionary
    interactive behaviour. It has applications both in the study of human social
    behaviour as well as in biology.
    It is used to understand when and how a rational individual might
    accept an immediate cost to their own utility for the direct benefit of
    another.

    Much attention has been given to a class of strategies called
    Zero Determinant strategies. It has been theoretically shown that these
    strategies can ``extort'' any player.

    In this work, an approach to identify if observed strategies are playing in
    an extortionate way is described. Furthermore, experimental analysis of
    a large tournament with \input{assets/tex/number_of_full_strategies/main.tex}
    strategies is considered. In this setting
    the most highly performing strategies do not play in an extortionate way
    against each other but do against lower performing strategies.
    This suggests that whilst the theory of Zero Determinant strategies
    indicates that memory is not of fundamental importance to the evolution of
    cooperative behaviour, this is incomplete.
\end{abstract}

\section{Introduction}\label{sec:introduction}

Agent based game theoretic models have become a stalwart of the underpinning
mathematics of interactive behaviours. One of the major pieces of work
in this area is the pair of original computer tournaments run by Robert
Axelrod~\cite{Axelrod1980, Axelrod1980a}. These tournaments pitted submitted
computer strategies against each other in plays of the Iterated Prisoner's
Dilemma. A common game where agents can choose to pay a slight cost to their
immediate utility in the hope of building a reputation. This has been used in
economic and evolutionary game theory to understand the evolution of cooperative
behaviour.

Recently, a class of strategies was described in~\cite{Press2012} that can
provably extort any given opponent. In~\cite{Hilbe2013, Moran1707} some
questions have already been asked about the true effectiveness of these
strategies in an evolutionary setting. Here another question is asked: is it
possible to recognise this extortionate behaviour? A mathematical procedure for
suspicion is presented: in the same way that the continued actions of an
extortionate individual might raise suspicion.

This work makes use of the Axelrod Python library~\cite{Knight2018, Knight2016}
with a large number of Prisoner Dilemma strategies available to give an
extensive numerical example of the ideas presented.  The approach is presented
in Section~\ref{sec:delta-zd-strategies}.  All of the code and data discussed
in Section~\ref{sec:numerical-experiments} is open sourced, archived and
written according to best scientific principles~\cite{Wilson2014}. The data
archive can be found at~\cite{vincent_knight_2018_1297075}.

\section{Recognising Extortion}\label{sec:delta-zd-strategies}

In~\cite{Press2012}, given a match between 2 memory-one strategies, the concept
of Zero Determinant (ZD) strategies is introduced. The main result of that paper
shows that given two memory one players \(p, q\in\mathbb{R}^4\) a linear
relationship between the players' scores could be forced by one of the players.

Using the notation of~\cite{Press2012}, assuming the utilities for player \(p\)
are given by \(S_x=(R, S, T, P)\) and for player \(q\) by \(S_y=(R, T, S, P)\)
and that the stationary scores of each player is given by \(S_X\) and \(S_Y\)
respectively. The main result of~\cite{Press2012} is that if

\begin{equation}\label{eqn:linear_relationship_for_p}
    \tilde p=\alpha S_x + \beta S_y + \gamma
\end{equation}

or

\begin{equation}\label{eqn:linear_relationship_for_q}
    \tilde q=\alpha S_x + \beta S_y + \gamma
\end{equation}

where \(\tilde p = (1 - p_1, 1 - p_2, p_3, p_4)\) and
\(\tilde q = (1 - q_1, 1 - q_2, q_3, q_4)\) then:

\begin{equation}
    \alpha S_X + \beta S_Y + \gamma = 0
\end{equation}

In~\cite{Press2012} a particular type of ZD strategy is defined: extortionate
strategies. If:

\begin{equation}\label{eqn:constraint_for_extortion}
    \gamma = - P(\alpha + \beta)
\end{equation}

then the player can ensure they get a score \(\chi\) times
larger than the opponent. This extortion coefficient is given by:

\begin{equation}\label{eqn:definition_of_chi}
    \chi=\frac{-\beta}{\alpha}
\end{equation}

Thus, if (\ref{eqn:constraint_for_extortion}) holds and \(\chi >1\) a player is
said to extort their opponent.
Here, the reverse problem is considered: given a
\(p\in\mathbb{R}^4\) how does one identify \(\alpha, \beta\) if they
exist and is the strategy in fact acting in an extortionate way?

These conditions correspond to:

\begin{align}
    \tilde p_1 & = \alpha R + \beta R - P (\alpha + \beta)
            \label{eqn:condition_for_tilde_p1}\\
    \tilde p_2 & = \alpha S + \beta T - P (\alpha + \beta)
            \label{eqn:condition_for_tilde_p2}\\
    \tilde p_3 & = \alpha T + \beta S - P (\alpha + \beta)
            \label{eqn:condition_for_tilde_p3}\\
    \tilde p_4 & = \alpha P + \beta P - P (\alpha + \beta)
            \label{eqn:condition_for_tilde_p4}
\end{align}

Equation (\ref{eqn:condition_for_tilde_p4}) ensures that \(p_4=\tilde p_4=0\).
Equations (\ref{eqn:condition_for_tilde_p1}-\ref{eqn:condition_for_tilde_p3})
can be used to eliminate \(\alpha, \beta\), giving:

\begin{equation}\label{eqn:planar_definition_of_extortion}
    \tilde p_1 = \frac{(R - P)(\tilde p_2 + \tilde p_3)}{S + T - 2P}
\end{equation}

with:

\begin{equation}\label{eqn:definition_of_chi}
    \chi = \frac{\tilde p_2 (P - T) + \tilde p_3 (S - P)}
                {\tilde p_2 (P - S) + \tilde p_3 (T - P)}
\end{equation}

Given a strategy \(p\in\mathbb{R}^{4\times 1}\) equations
(\ref{eqn:condition_for_tilde_p4}), (\ref{eqn:planar_definition_of_extortion}-\ref{eqn:definition_of_chi}) can be used to check if
a strategy is extortionate. The conditions correspond to:

\begin{align}
    p_1 & = \frac{(R-P)(p_2 + p_3) - R + T + S - P}{S + T - 2P}
     \label{eqn:condition_for_p1}\\
    p_4 & = 0 \label{eqn:condition_for_p4}\\
    1 & > p_2 + p_3\label{eqn:condition_for_chi}
\end{align}

The algebraic steps necessary to prove these results are available in the
supporting materials.

All extortionate strategies reside on a triangular (\ref{eqn:condition_for_chi})
plane (\ref{eqn:condition_for_p1}) in 3 dimensions (\ref{eqn:condition_for_p4}).
Using this formulation it can be seen that a necessary (but not sufficient)
condition for an extortionate strategy is that it cooperates on average less
than 50\% of the time when in a state of disagreement with the opponent.

As an example, consider the known extortionate strategy \(p=(8 / 9, 1 / 2, 1 /
3, 0)\) from~\cite{Stewart2012} which is referred to as \texttt{Extort-2}. In
this case, for the standard values of \((R, T, S, P)\) constraint
(\ref{eqn:condition_for_p1}) corresponds to:

\begin{equation}
    p_1 = \frac{2(p_2 + p_3) + 1}{3}
\end{equation}

It is clear that in this case all constraints hold.

This approach could in fact be used to confirm that a given strategy is acting
in an extortionate manner even if it is not a memory one strategy. However, in
practice, if a closed form for \(p\) is not known, then due to measurement
and/or numerical error this would not work.

This problem can be written in the following linear algebraic form where
\(x=(\alpha, \beta)\)
and \(p^*=(\tilde p_1 - 1, tilde_2 - 1, p_3)\):

\begin{equation}\label{eqn:linear_algebraic_equation_for_p}
    Cx= p^*
\end{equation}

\(C\) corresponds to equations
(\ref{eqn:condition_for_tilde_p1}-\ref{eqn:condition_for_tilde_p3}) and is
given by:

\begin{equation}\label{eqn:definition_of_C}
    C =
    \begin{bmatrix}
        R - P & R- P \\
        S - P & T- P \\
        T - P & S- P \\
    \end{bmatrix}
\end{equation}

Note that in general, equation (\ref{eqn:linear_algebraic_equation_for_p}) will
not necessarily have a solution. From the Rouch\'{e}-Capelli theorem if there is
a solution it is unique as \(\text{rank}(C)=2\) which is the dimension of the
variable \(x\). The best fitting \(x\) is found by minimizing:

\begin{equation}\label{eqn:r_squared}
    \text{SSError} = \|C x- p^*\|_2^2 = \sum_{i=1}^{3}\left((C\bar x)_i-p_i^*\right)^2
\end{equation}

Note that \(\text{SSError}\), which is the square of the Frobenius
norm~\cite{Golub2013}, becomes a measure of how close a strategy is to being an
extortionate strategy. Suspicion
of extortion then corresponds to a threshold on \(\text{SSError}\).

By observing interactions (human or otherwise), their memory one representation
can be inferred and this approach can be used to recognise extortionate
behaviour. The notion of comparing theoretic and actual plays of the IPD is not
novel, see for example~\cite{Rand2013}. Immediately it is noted that if the
environment is noisy~\cite{Wu1995} then no strategy can be considered to be
extortionate as \(p_4>0\).

In the next section, this idea will be illustrated by observing the interactions
that take place in a computer based tournament of the IPD\@.

\section{Numerical experiments}\label{sec:numerical-experiments}

In~\cite{Stewart2012} results from a tournament with
\input{./assets/tex/number_of_stewart_plotkin_strategies/main.tex} strategies,
was presented with specific consideration given to ZD strategies. This
tournament is reproduced here using the Axelrod-Python
project~\cite{Knight2016}. To obtain a good measure of the corresponding
transition rates for each strategy all matches have been run for
\input{assets/tex/number_of_turns/main.tex} turns and every match has been
repeated \input{assets/tex/number_of_repetitions/main.tex} times. All of this
interaction data is available at~\cite{vincent_knight_2018_1297075}. A good
match between the inferred Markov chain and the state distribution of the actual
interactions has been verified. Data for this is presented in the supplementary
materials.

Figure~\ref{fig:SSError_overall_in_stewart_plotkin} shows the \(\text{SSError}\)
values for all the strategies in the tournament, as reported
in~\cite{Stewart2012} the extortionate strategy (which has an expected
\(\text{SSError}\) approximately 0) gains a large number of wins.

\begin{figure}[!htbp]
    \centering
    \includegraphics[width=.8\textwidth]{./assets/img/SSError_overall_in_stewart_plotkin/main.pdf}
    \caption{\(\text{SSError}\) and state probabilities for the strategies
        of~\cite{Stewart2012}, ordered both by number of wins and overall score.
        Note that \(P(DC)\) is not shown as it corresponds to the transpose of
        \(P(CD)\). Cooperator and Defector are omitted as they do not visit all
        the states.}
    \label{fig:SSError_overall_in_stewart_plotkin}
\end{figure}

Here, the work of~\cite{Stewart2012} is extended by investigating a tournament
with \input{assets/tex/number_of_full_strategies/main.tex}
strategies.

The results of this analysis are shown in
Figure~\ref{fig:SSError_and_probabilities_in_full}. The top ranking strategies
by number of wins seem to be extortionate (but not against all strategies) and
it can be seen that a small sub group of strategies achieve mutual defection.
All the top ranking strategies according to score achieve mutual cooperation and
do not extort each other, however they
\textbf{do} exhibit extortionate behaviour towards a number of the lower ranking
strategies.

\begin{figure}[!htbp]
    \centering
    \includegraphics[width=.8\textwidth]{./assets/img/SSError_and_probabilities_in_full/main.pdf}
    \caption{\(\text{SSError}\) for the strategies for the full tournament. Only
    strategy interactions for which \(p_4=0\) and \(\chi>1\) are displayed.}
    \label{fig:SSError_and_probabilities_in_full}
\end{figure}

\section{Conclusion}\label{sec:conclusion}

This work defines an approach to measure whether or not a player is playing a
strategy that corresponds to an extortionate strategy as defined
in~\cite{Press2012}: a mathematical model for suspicion. Indeed, all
extortionate strategies have been
 classified as lying on a triangular plane.
This rigorous classification fails to be robust to small measurement error, thus
a statistical approach is proposed.
This is done through a linear algebraic approach for approximating the solution
of a linear system. Using this, a large number of pairwise interactions is
simulated and in fact very few strategies are found to act extortionately.

The work of~\cite{Press2012}, whilst showing that a clever approach to taking
advantage of another memory one strategy exists: this is incomplete. Whilst the
elegance of this result is very attractive, just as the simplicity of the
victory of Tit For Tat in Axelrod's original tournaments was, it is incomplete.
Extortionate strategies achieve a high number of wins but they do not
achieve a high score which corresponds to the fitness landscape in an
evolutionary sense. From the large number of interactions a payoff matrix \(S\)
can be measured where \(S_{ij}\) denotes the score (using standard values of
\((R, S, T, P) = (3, 0, 5, 1)\)) of the \(i\)th strategy
against the \(j\)th strategy. Using this, the replicator equation
describes the evolution of the system based on a population density fitness
function:

\begin{equation}\label{eqn:replicator_dynamics}
    \frac{dx}{dt} = x(S-x^TS x)
\end{equation}

Equation (\ref{eqn:replicator_dynamics}) is solved numerically through an
integration technique described in~\cite{Petzold1983} and
Figure~\ref{fig:replicator_dynamics} shows the evolution of the distribution of
the system: the various strategies are ranked by scores. It is clear to see that
only the high ranking strategies survive the evolutionary process (in fact,
only \input{./assets/img/replicator_dynamics/main.tex}
have a final distribution greater than \(10 ^ {-2}\)). This confirms the
findings of~\cite{Moran1707} in which sophisticated strategies resist
evolutionary invasion of shorter memory strategies. Recalling
Figure~\ref{fig:SSError_and_probabilities_in_full} this demonstrates that:

\begin{itemize}
    \item Cooperation emerges through the evolutionary process: the high scoring
        strategies do not exhibit extortionate behaviour towards each other.
    \item Extortionate strategies do not survive the evolutionary process.
\end{itemize}

\begin{figure}[!htbp]
    \centering
    \includegraphics[width=.8\textwidth]{./assets/img/replicator_dynamics/main.pdf}
    \caption{Numerical simulation of the replicator equation
    (\ref{eqn:replicator_dynamics}): strategies are ordered by score, only the strategies with a high score survive the evolutionary process.}
    \label{fig:replicator_dynamics}
\end{figure}

This work can be used to classify plays of the IPD\@: data can be collected from
actual interactions (in lab or in the field). Furthermore, this allows for a
classification method similar to the notion of fingerprinting presented
in~\cite{Ashlock2008}. Trained strategies can potentially be classified as
extortionate or not or it could be possible to even constrain the reinforcement
learning approaches that are becoming prevalent in the literature.
Alternatively, this mathematical approach for recognising extortion could be
used in sophisticated strategies to defend against invasion. Arguably, some of
the strategies considered here exhibit this behaviour, indeed as described
in~\cite{Harper2017}, the top ranking strategies in the full tournament are
obtained using evolutionary reinforcement learning techniques, thus, suspicion
of extortionate behaviour could in fact be an evolutionary trait.

\section*{Acknowledgements}

The following open source software libraries were used in this research:

\begin{itemize}
    \item The Axelrod ~\cite{Knight2016, Knight2018} library (IPD strategies and
        tournaments).
    \item The sympy library~\cite{Meurer2017} (verification of all symbolic
        calculations).
    \item The matplotlib~\cite{Droettboom2018} library (visualisation).
    \item The pandas~\cite{Structures2010}, dask~\cite{Dask2016} and
        NumPy~\cite{Oliphant2015} libraries (data manipulation).
    \item The SciPy~\cite{Jones2001} library (numerical integration of the
        replicator equation).
\end{itemize}

This work was performed using the computational facilities of the Advanced
Research Computing @ Cardiff (ARCCA) Division, Cardiff University.

\printbibliography

\newpage
\section*{Supplementary materials}

\includepdf{assets/pdf/proof_of_form_of_extortionate_strategies/main.pdf}

\newpage

Using the pair wise interactions the transition rates \(p,
q\) can be measured and the steady state probabilities inferred and compared to
the actual probabilities of each state.
This is done numerically by computing the singular eigenvector of the
matrix \(A\) \cite{Stewart2009}:

\[
    A =
    \begin{bmatrix}
        p_1 q_1 & p_1 (1 - q_1) & (1 - p_1) q_1 & (1 -p_1) (1 - q_1) \\
        p_2 q_2 & p_2 (1 - q_2) & (1 - p_2) q_2 & (1 -p_2) (1 - q_2) \\
        p_3 q_3 & p_3 (1 - q_3) & (1 - p_3) q_3 & (1 -p_3) (1 - q_3) \\
        p_4 q_4 & p_4 (1 - q_4) & (1 - p_4) q_4 & (1 -p_4) (1 - q_4) \\
    \end{bmatrix}
\]

Figure~\ref{fig:computed_probabilities_vs_theoretic_probabilities} shows a
regression line fitted to every pairwise interaction with a reported
\(\text{SSError}\) value (pairwise interactions with missing states were
omitted). This serves to validate the approach: a part from some edge cases the
relationship is consistent.

\begin{figure}[!htbp]
    \centering
    \includegraphics[width=.8\textwidth]{./assets/img/computed_probabilities_vs_theoretic_probabilities/main.pdf}
    \caption{The
        relationship between the steady state probabilities inferred from the
        measured transitions and the actual steady state probabilities. A linear
        regression line is included validating the approach.}
    \label{fig:computed_probabilities_vs_theoretic_probabilities}
\end{figure}


\end{document}

    strategies is considered. In this setting
    the most highly performing strategies do not play in an extortionate way
    against each other but do against lower performing strategies.
    This suggests that whilst the theory of Zero Determinant strategies
    indicates that memory is not of fundamental importance to the evolution of
    cooperative behaviour, this is incomplete.
\end{abstract}

\section{Introduction}\label{sec:introduction}

Agent based game theoretic models have become a stalwart of the underpinning
mathematics of interactive behaviours. One of the major pieces of work
in this area is the pair of original computer tournaments run by Robert
Axelrod~\cite{Axelrod1980, Axelrod1980a}. These tournaments pitted submitted
computer strategies against each other in plays of the Iterated Prisoner's
Dilemma. A common game where agents can choose to pay a slight cost to their
immediate utility in the hope of building a reputation. This has been used in
economic and evolutionary game theory to understand the evolution of cooperative
behaviour.

Recently, a class of strategies was described in~\cite{Press2012} that can
provably extort any given opponent. In~\cite{Hilbe2013, Moran1707} some
questions have already been asked about the true effectiveness of these
strategies in an evolutionary setting. Here another question is asked: is it
possible to recognise this extortionate behaviour? A mathematical procedure for
suspicion is presented: in the same way that the continued actions of an
extortionate individual might raise suspicion.

This work makes use of the Axelrod Python library~\cite{Knight2018, Knight2016}
with a large number of Prisoner Dilemma strategies available to give an
extensive numerical example of the ideas presented.  The approach is presented
in Section~\ref{sec:delta-zd-strategies}.  All of the code and data discussed
in Section~\ref{sec:numerical-experiments} is open sourced, archived and
written according to best scientific principles~\cite{Wilson2014}. The data
archive can be found at~\cite{vincent_knight_2018_1297075}.

\section{Recognising Extortion}\label{sec:delta-zd-strategies}

In~\cite{Press2012}, given a match between 2 memory-one strategies, the concept
of Zero Determinant (ZD) strategies is introduced. The main result of that paper
shows that given two memory one players \(p, q\in\mathbb{R}^4\) a linear
relationship between the players' scores could be forced by one of the players.

Using the notation of~\cite{Press2012}, assuming the utilities for player \(p\)
are given by \(S_x=(R, S, T, P)\) and for player \(q\) by \(S_y=(R, T, S, P)\)
and that the stationary scores of each player is given by \(S_X\) and \(S_Y\)
respectively. The main result of~\cite{Press2012} is that if

\begin{equation}\label{eqn:linear_relationship_for_p}
    \tilde p=\alpha S_x + \beta S_y + \gamma
\end{equation}

or

\begin{equation}\label{eqn:linear_relationship_for_q}
    \tilde q=\alpha S_x + \beta S_y + \gamma
\end{equation}

where \(\tilde p = (1 - p_1, 1 - p_2, p_3, p_4)\) and
\(\tilde q = (1 - q_1, 1 - q_2, q_3, q_4)\) then:

\begin{equation}
    \alpha S_X + \beta S_Y + \gamma = 0
\end{equation}

In~\cite{Press2012} a particular type of ZD strategy is defined: extortionate
strategies. If:

\begin{equation}\label{eqn:constraint_for_extortion}
    \gamma = - P(\alpha + \beta)
\end{equation}

then the player can ensure they get a score \(\chi\) times
larger than the opponent. This extortion coefficient is given by:

\begin{equation}\label{eqn:definition_of_chi}
    \chi=\frac{-\beta}{\alpha}
\end{equation}

Thus, if (\ref{eqn:constraint_for_extortion}) holds and \(\chi >1\) a player is
said to extort their opponent.
Here, the reverse problem is considered: given a
\(p\in\mathbb{R}^4\) how does one identify \(\alpha, \beta\) if they
exist and is the strategy in fact acting in an extortionate way?

These conditions correspond to:

\begin{align}
    \tilde p_1 & = \alpha R + \beta R - P (\alpha + \beta)
            \label{eqn:condition_for_tilde_p1}\\
    \tilde p_2 & = \alpha S + \beta T - P (\alpha + \beta)
            \label{eqn:condition_for_tilde_p2}\\
    \tilde p_3 & = \alpha T + \beta S - P (\alpha + \beta)
            \label{eqn:condition_for_tilde_p3}\\
    \tilde p_4 & = \alpha P + \beta P - P (\alpha + \beta)
            \label{eqn:condition_for_tilde_p4}
\end{align}

Equation (\ref{eqn:condition_for_tilde_p4}) ensures that \(p_4=\tilde p_4=0\).
Equations (\ref{eqn:condition_for_tilde_p1}-\ref{eqn:condition_for_tilde_p3})
can be used to eliminate \(\alpha, \beta\), giving:

\begin{equation}\label{eqn:planar_definition_of_extortion}
    \tilde p_1 = \frac{(R - P)(\tilde p_2 + \tilde p_3)}{S + T - 2P}
\end{equation}

with:

\begin{equation}\label{eqn:definition_of_chi}
    \chi = \frac{\tilde p_2 (P - T) + \tilde p_3 (S - P)}
                {\tilde p_2 (P - S) + \tilde p_3 (T - P)}
\end{equation}

Given a strategy \(p\in\mathbb{R}^{4\times 1}\) equations
(\ref{eqn:condition_for_tilde_p4}), (\ref{eqn:planar_definition_of_extortion}-\ref{eqn:definition_of_chi}) can be used to check if
a strategy is extortionate. The conditions correspond to:

\begin{align}
    p_1 & = \frac{(R-P)(p_2 + p_3) - R + T + S - P}{S + T - 2P}
     \label{eqn:condition_for_p1}\\
    p_4 & = 0 \label{eqn:condition_for_p4}\\
    1 & > p_2 + p_3\label{eqn:condition_for_chi}
\end{align}

The algebraic steps necessary to prove these results are available in the
supporting materials.

All extortionate strategies reside on a triangular (\ref{eqn:condition_for_chi})
plane (\ref{eqn:condition_for_p1}) in 3 dimensions (\ref{eqn:condition_for_p4}).
Using this formulation it can be seen that a necessary (but not sufficient)
condition for an extortionate strategy is that it cooperates on average less
than 50\% of the time when in a state of disagreement with the opponent.

As an example, consider the known extortionate strategy \(p=(8 / 9, 1 / 2, 1 /
3, 0)\) from~\cite{Stewart2012} which is referred to as \texttt{Extort-2}. In
this case, for the standard values of \((R, T, S, P)\) constraint
(\ref{eqn:condition_for_p1}) corresponds to:

\begin{equation}
    p_1 = \frac{2(p_2 + p_3) + 1}{3}
\end{equation}

It is clear that in this case all constraints hold.

This approach could in fact be used to confirm that a given strategy is acting
in an extortionate manner even if it is not a memory one strategy. However, in
practice, if a closed form for \(p\) is not known, then due to measurement
and/or numerical error this would not work.

This problem can be written in the following linear algebraic form where
\(x=(\alpha, \beta)\)
and \(p^*=(\tilde p_1 - 1, tilde_2 - 1, p_3)\):

\begin{equation}\label{eqn:linear_algebraic_equation_for_p}
    Cx= p^*
\end{equation}

\(C\) corresponds to equations
(\ref{eqn:condition_for_tilde_p1}-\ref{eqn:condition_for_tilde_p3}) and is
given by:

\begin{equation}\label{eqn:definition_of_C}
    C =
    \begin{bmatrix}
        R - P & R- P \\
        S - P & T- P \\
        T - P & S- P \\
    \end{bmatrix}
\end{equation}

Note that in general, equation (\ref{eqn:linear_algebraic_equation_for_p}) will
not necessarily have a solution. From the Rouch\'{e}-Capelli theorem if there is
a solution it is unique as \(\text{rank}(C)=2\) which is the dimension of the
variable \(x\). The best fitting \(x\) is found by minimizing:

\begin{equation}\label{eqn:r_squared}
    \text{SSError} = \|C x- p^*\|_2^2 = \sum_{i=1}^{3}\left((C\bar x)_i-p_i^*\right)^2
\end{equation}

Note that \(\text{SSError}\), which is the square of the Frobenius
norm~\cite{Golub2013}, becomes a measure of how close a strategy is to being an
extortionate strategy. Suspicion
of extortion then corresponds to a threshold on \(\text{SSError}\).

By observing interactions (human or otherwise), their memory one representation
can be inferred and this approach can be used to recognise extortionate
behaviour. The notion of comparing theoretic and actual plays of the IPD is not
novel, see for example~\cite{Rand2013}. Immediately it is noted that if the
environment is noisy~\cite{Wu1995} then no strategy can be considered to be
extortionate as \(p_4>0\).

In the next section, this idea will be illustrated by observing the interactions
that take place in a computer based tournament of the IPD\@.

\section{Numerical experiments}\label{sec:numerical-experiments}

In~\cite{Stewart2012} results from a tournament with
\documentclass[a4paper]{article}

\usepackage{amsmath}
\usepackage{amssymb}
\usepackage[margin=1.5cm,
            includefoot,
            footskip=30pt]{geometry}
\usepackage{layout}
\usepackage{graphicx}
\usepackage{subcaption}

\usepackage{biblatex}
\usepackage{pdfpages}

\bibliography{main.bib}

\title{Suspicion: Recognising and evaluating the effectiveness
       of extortion in the Iterated Prisoner's Dilemma}
\author{Vincent A. Knight \and Nikoleta E. Glynatsi}
\date{\today}



\begin{document}

\maketitle

\begin{abstract}
    The Iterated Prisoner's Dilemma is a model for rational and evolutionary
    interactive behaviour. It has applications both in the study of human social
    behaviour as well as in biology.
    It is used to understand when and how a rational individual might
    accept an immediate cost to their own utility for the direct benefit of
    another.

    Much attention has been given to a class of strategies called
    Zero Determinant strategies. It has been theoretically shown that these
    strategies can ``extort'' any player.

    In this work, an approach to identify if observed strategies are playing in
    an extortionate way is described. Furthermore, experimental analysis of
    a large tournament with \input{assets/tex/number_of_full_strategies/main.tex}
    strategies is considered. In this setting
    the most highly performing strategies do not play in an extortionate way
    against each other but do against lower performing strategies.
    This suggests that whilst the theory of Zero Determinant strategies
    indicates that memory is not of fundamental importance to the evolution of
    cooperative behaviour, this is incomplete.
\end{abstract}

\section{Introduction}\label{sec:introduction}

Agent based game theoretic models have become a stalwart of the underpinning
mathematics of interactive behaviours. One of the major pieces of work
in this area is the pair of original computer tournaments run by Robert
Axelrod~\cite{Axelrod1980, Axelrod1980a}. These tournaments pitted submitted
computer strategies against each other in plays of the Iterated Prisoner's
Dilemma. A common game where agents can choose to pay a slight cost to their
immediate utility in the hope of building a reputation. This has been used in
economic and evolutionary game theory to understand the evolution of cooperative
behaviour.

Recently, a class of strategies was described in~\cite{Press2012} that can
provably extort any given opponent. In~\cite{Hilbe2013, Moran1707} some
questions have already been asked about the true effectiveness of these
strategies in an evolutionary setting. Here another question is asked: is it
possible to recognise this extortionate behaviour? A mathematical procedure for
suspicion is presented: in the same way that the continued actions of an
extortionate individual might raise suspicion.

This work makes use of the Axelrod Python library~\cite{Knight2018, Knight2016}
with a large number of Prisoner Dilemma strategies available to give an
extensive numerical example of the ideas presented.  The approach is presented
in Section~\ref{sec:delta-zd-strategies}.  All of the code and data discussed
in Section~\ref{sec:numerical-experiments} is open sourced, archived and
written according to best scientific principles~\cite{Wilson2014}. The data
archive can be found at~\cite{vincent_knight_2018_1297075}.

\section{Recognising Extortion}\label{sec:delta-zd-strategies}

In~\cite{Press2012}, given a match between 2 memory-one strategies, the concept
of Zero Determinant (ZD) strategies is introduced. The main result of that paper
shows that given two memory one players \(p, q\in\mathbb{R}^4\) a linear
relationship between the players' scores could be forced by one of the players.

Using the notation of~\cite{Press2012}, assuming the utilities for player \(p\)
are given by \(S_x=(R, S, T, P)\) and for player \(q\) by \(S_y=(R, T, S, P)\)
and that the stationary scores of each player is given by \(S_X\) and \(S_Y\)
respectively. The main result of~\cite{Press2012} is that if

\begin{equation}\label{eqn:linear_relationship_for_p}
    \tilde p=\alpha S_x + \beta S_y + \gamma
\end{equation}

or

\begin{equation}\label{eqn:linear_relationship_for_q}
    \tilde q=\alpha S_x + \beta S_y + \gamma
\end{equation}

where \(\tilde p = (1 - p_1, 1 - p_2, p_3, p_4)\) and
\(\tilde q = (1 - q_1, 1 - q_2, q_3, q_4)\) then:

\begin{equation}
    \alpha S_X + \beta S_Y + \gamma = 0
\end{equation}

In~\cite{Press2012} a particular type of ZD strategy is defined: extortionate
strategies. If:

\begin{equation}\label{eqn:constraint_for_extortion}
    \gamma = - P(\alpha + \beta)
\end{equation}

then the player can ensure they get a score \(\chi\) times
larger than the opponent. This extortion coefficient is given by:

\begin{equation}\label{eqn:definition_of_chi}
    \chi=\frac{-\beta}{\alpha}
\end{equation}

Thus, if (\ref{eqn:constraint_for_extortion}) holds and \(\chi >1\) a player is
said to extort their opponent.
Here, the reverse problem is considered: given a
\(p\in\mathbb{R}^4\) how does one identify \(\alpha, \beta\) if they
exist and is the strategy in fact acting in an extortionate way?

These conditions correspond to:

\begin{align}
    \tilde p_1 & = \alpha R + \beta R - P (\alpha + \beta)
            \label{eqn:condition_for_tilde_p1}\\
    \tilde p_2 & = \alpha S + \beta T - P (\alpha + \beta)
            \label{eqn:condition_for_tilde_p2}\\
    \tilde p_3 & = \alpha T + \beta S - P (\alpha + \beta)
            \label{eqn:condition_for_tilde_p3}\\
    \tilde p_4 & = \alpha P + \beta P - P (\alpha + \beta)
            \label{eqn:condition_for_tilde_p4}
\end{align}

Equation (\ref{eqn:condition_for_tilde_p4}) ensures that \(p_4=\tilde p_4=0\).
Equations (\ref{eqn:condition_for_tilde_p1}-\ref{eqn:condition_for_tilde_p3})
can be used to eliminate \(\alpha, \beta\), giving:

\begin{equation}\label{eqn:planar_definition_of_extortion}
    \tilde p_1 = \frac{(R - P)(\tilde p_2 + \tilde p_3)}{S + T - 2P}
\end{equation}

with:

\begin{equation}\label{eqn:definition_of_chi}
    \chi = \frac{\tilde p_2 (P - T) + \tilde p_3 (S - P)}
                {\tilde p_2 (P - S) + \tilde p_3 (T - P)}
\end{equation}

Given a strategy \(p\in\mathbb{R}^{4\times 1}\) equations
(\ref{eqn:condition_for_tilde_p4}), (\ref{eqn:planar_definition_of_extortion}-\ref{eqn:definition_of_chi}) can be used to check if
a strategy is extortionate. The conditions correspond to:

\begin{align}
    p_1 & = \frac{(R-P)(p_2 + p_3) - R + T + S - P}{S + T - 2P}
     \label{eqn:condition_for_p1}\\
    p_4 & = 0 \label{eqn:condition_for_p4}\\
    1 & > p_2 + p_3\label{eqn:condition_for_chi}
\end{align}

The algebraic steps necessary to prove these results are available in the
supporting materials.

All extortionate strategies reside on a triangular (\ref{eqn:condition_for_chi})
plane (\ref{eqn:condition_for_p1}) in 3 dimensions (\ref{eqn:condition_for_p4}).
Using this formulation it can be seen that a necessary (but not sufficient)
condition for an extortionate strategy is that it cooperates on average less
than 50\% of the time when in a state of disagreement with the opponent.

As an example, consider the known extortionate strategy \(p=(8 / 9, 1 / 2, 1 /
3, 0)\) from~\cite{Stewart2012} which is referred to as \texttt{Extort-2}. In
this case, for the standard values of \((R, T, S, P)\) constraint
(\ref{eqn:condition_for_p1}) corresponds to:

\begin{equation}
    p_1 = \frac{2(p_2 + p_3) + 1}{3}
\end{equation}

It is clear that in this case all constraints hold.

This approach could in fact be used to confirm that a given strategy is acting
in an extortionate manner even if it is not a memory one strategy. However, in
practice, if a closed form for \(p\) is not known, then due to measurement
and/or numerical error this would not work.

This problem can be written in the following linear algebraic form where
\(x=(\alpha, \beta)\)
and \(p^*=(\tilde p_1 - 1, tilde_2 - 1, p_3)\):

\begin{equation}\label{eqn:linear_algebraic_equation_for_p}
    Cx= p^*
\end{equation}

\(C\) corresponds to equations
(\ref{eqn:condition_for_tilde_p1}-\ref{eqn:condition_for_tilde_p3}) and is
given by:

\begin{equation}\label{eqn:definition_of_C}
    C =
    \begin{bmatrix}
        R - P & R- P \\
        S - P & T- P \\
        T - P & S- P \\
    \end{bmatrix}
\end{equation}

Note that in general, equation (\ref{eqn:linear_algebraic_equation_for_p}) will
not necessarily have a solution. From the Rouch\'{e}-Capelli theorem if there is
a solution it is unique as \(\text{rank}(C)=2\) which is the dimension of the
variable \(x\). The best fitting \(x\) is found by minimizing:

\begin{equation}\label{eqn:r_squared}
    \text{SSError} = \|C x- p^*\|_2^2 = \sum_{i=1}^{3}\left((C\bar x)_i-p_i^*\right)^2
\end{equation}

Note that \(\text{SSError}\), which is the square of the Frobenius
norm~\cite{Golub2013}, becomes a measure of how close a strategy is to being an
extortionate strategy. Suspicion
of extortion then corresponds to a threshold on \(\text{SSError}\).

By observing interactions (human or otherwise), their memory one representation
can be inferred and this approach can be used to recognise extortionate
behaviour. The notion of comparing theoretic and actual plays of the IPD is not
novel, see for example~\cite{Rand2013}. Immediately it is noted that if the
environment is noisy~\cite{Wu1995} then no strategy can be considered to be
extortionate as \(p_4>0\).

In the next section, this idea will be illustrated by observing the interactions
that take place in a computer based tournament of the IPD\@.

\section{Numerical experiments}\label{sec:numerical-experiments}

In~\cite{Stewart2012} results from a tournament with
\input{./assets/tex/number_of_stewart_plotkin_strategies/main.tex} strategies,
was presented with specific consideration given to ZD strategies. This
tournament is reproduced here using the Axelrod-Python
project~\cite{Knight2016}. To obtain a good measure of the corresponding
transition rates for each strategy all matches have been run for
\input{assets/tex/number_of_turns/main.tex} turns and every match has been
repeated \input{assets/tex/number_of_repetitions/main.tex} times. All of this
interaction data is available at~\cite{vincent_knight_2018_1297075}. A good
match between the inferred Markov chain and the state distribution of the actual
interactions has been verified. Data for this is presented in the supplementary
materials.

Figure~\ref{fig:SSError_overall_in_stewart_plotkin} shows the \(\text{SSError}\)
values for all the strategies in the tournament, as reported
in~\cite{Stewart2012} the extortionate strategy (which has an expected
\(\text{SSError}\) approximately 0) gains a large number of wins.

\begin{figure}[!htbp]
    \centering
    \includegraphics[width=.8\textwidth]{./assets/img/SSError_overall_in_stewart_plotkin/main.pdf}
    \caption{\(\text{SSError}\) and state probabilities for the strategies
        of~\cite{Stewart2012}, ordered both by number of wins and overall score.
        Note that \(P(DC)\) is not shown as it corresponds to the transpose of
        \(P(CD)\). Cooperator and Defector are omitted as they do not visit all
        the states.}
    \label{fig:SSError_overall_in_stewart_plotkin}
\end{figure}

Here, the work of~\cite{Stewart2012} is extended by investigating a tournament
with \input{assets/tex/number_of_full_strategies/main.tex}
strategies.

The results of this analysis are shown in
Figure~\ref{fig:SSError_and_probabilities_in_full}. The top ranking strategies
by number of wins seem to be extortionate (but not against all strategies) and
it can be seen that a small sub group of strategies achieve mutual defection.
All the top ranking strategies according to score achieve mutual cooperation and
do not extort each other, however they
\textbf{do} exhibit extortionate behaviour towards a number of the lower ranking
strategies.

\begin{figure}[!htbp]
    \centering
    \includegraphics[width=.8\textwidth]{./assets/img/SSError_and_probabilities_in_full/main.pdf}
    \caption{\(\text{SSError}\) for the strategies for the full tournament. Only
    strategy interactions for which \(p_4=0\) and \(\chi>1\) are displayed.}
    \label{fig:SSError_and_probabilities_in_full}
\end{figure}

\section{Conclusion}\label{sec:conclusion}

This work defines an approach to measure whether or not a player is playing a
strategy that corresponds to an extortionate strategy as defined
in~\cite{Press2012}: a mathematical model for suspicion. Indeed, all
extortionate strategies have been
 classified as lying on a triangular plane.
This rigorous classification fails to be robust to small measurement error, thus
a statistical approach is proposed.
This is done through a linear algebraic approach for approximating the solution
of a linear system. Using this, a large number of pairwise interactions is
simulated and in fact very few strategies are found to act extortionately.

The work of~\cite{Press2012}, whilst showing that a clever approach to taking
advantage of another memory one strategy exists: this is incomplete. Whilst the
elegance of this result is very attractive, just as the simplicity of the
victory of Tit For Tat in Axelrod's original tournaments was, it is incomplete.
Extortionate strategies achieve a high number of wins but they do not
achieve a high score which corresponds to the fitness landscape in an
evolutionary sense. From the large number of interactions a payoff matrix \(S\)
can be measured where \(S_{ij}\) denotes the score (using standard values of
\((R, S, T, P) = (3, 0, 5, 1)\)) of the \(i\)th strategy
against the \(j\)th strategy. Using this, the replicator equation
describes the evolution of the system based on a population density fitness
function:

\begin{equation}\label{eqn:replicator_dynamics}
    \frac{dx}{dt} = x(S-x^TS x)
\end{equation}

Equation (\ref{eqn:replicator_dynamics}) is solved numerically through an
integration technique described in~\cite{Petzold1983} and
Figure~\ref{fig:replicator_dynamics} shows the evolution of the distribution of
the system: the various strategies are ranked by scores. It is clear to see that
only the high ranking strategies survive the evolutionary process (in fact,
only \input{./assets/img/replicator_dynamics/main.tex}
have a final distribution greater than \(10 ^ {-2}\)). This confirms the
findings of~\cite{Moran1707} in which sophisticated strategies resist
evolutionary invasion of shorter memory strategies. Recalling
Figure~\ref{fig:SSError_and_probabilities_in_full} this demonstrates that:

\begin{itemize}
    \item Cooperation emerges through the evolutionary process: the high scoring
        strategies do not exhibit extortionate behaviour towards each other.
    \item Extortionate strategies do not survive the evolutionary process.
\end{itemize}

\begin{figure}[!htbp]
    \centering
    \includegraphics[width=.8\textwidth]{./assets/img/replicator_dynamics/main.pdf}
    \caption{Numerical simulation of the replicator equation
    (\ref{eqn:replicator_dynamics}): strategies are ordered by score, only the strategies with a high score survive the evolutionary process.}
    \label{fig:replicator_dynamics}
\end{figure}

This work can be used to classify plays of the IPD\@: data can be collected from
actual interactions (in lab or in the field). Furthermore, this allows for a
classification method similar to the notion of fingerprinting presented
in~\cite{Ashlock2008}. Trained strategies can potentially be classified as
extortionate or not or it could be possible to even constrain the reinforcement
learning approaches that are becoming prevalent in the literature.
Alternatively, this mathematical approach for recognising extortion could be
used in sophisticated strategies to defend against invasion. Arguably, some of
the strategies considered here exhibit this behaviour, indeed as described
in~\cite{Harper2017}, the top ranking strategies in the full tournament are
obtained using evolutionary reinforcement learning techniques, thus, suspicion
of extortionate behaviour could in fact be an evolutionary trait.

\section*{Acknowledgements}

The following open source software libraries were used in this research:

\begin{itemize}
    \item The Axelrod ~\cite{Knight2016, Knight2018} library (IPD strategies and
        tournaments).
    \item The sympy library~\cite{Meurer2017} (verification of all symbolic
        calculations).
    \item The matplotlib~\cite{Droettboom2018} library (visualisation).
    \item The pandas~\cite{Structures2010}, dask~\cite{Dask2016} and
        NumPy~\cite{Oliphant2015} libraries (data manipulation).
    \item The SciPy~\cite{Jones2001} library (numerical integration of the
        replicator equation).
\end{itemize}

This work was performed using the computational facilities of the Advanced
Research Computing @ Cardiff (ARCCA) Division, Cardiff University.

\printbibliography

\newpage
\section*{Supplementary materials}

\includepdf{assets/pdf/proof_of_form_of_extortionate_strategies/main.pdf}

\newpage

Using the pair wise interactions the transition rates \(p,
q\) can be measured and the steady state probabilities inferred and compared to
the actual probabilities of each state.
This is done numerically by computing the singular eigenvector of the
matrix \(A\) \cite{Stewart2009}:

\[
    A =
    \begin{bmatrix}
        p_1 q_1 & p_1 (1 - q_1) & (1 - p_1) q_1 & (1 -p_1) (1 - q_1) \\
        p_2 q_2 & p_2 (1 - q_2) & (1 - p_2) q_2 & (1 -p_2) (1 - q_2) \\
        p_3 q_3 & p_3 (1 - q_3) & (1 - p_3) q_3 & (1 -p_3) (1 - q_3) \\
        p_4 q_4 & p_4 (1 - q_4) & (1 - p_4) q_4 & (1 -p_4) (1 - q_4) \\
    \end{bmatrix}
\]

Figure~\ref{fig:computed_probabilities_vs_theoretic_probabilities} shows a
regression line fitted to every pairwise interaction with a reported
\(\text{SSError}\) value (pairwise interactions with missing states were
omitted). This serves to validate the approach: a part from some edge cases the
relationship is consistent.

\begin{figure}[!htbp]
    \centering
    \includegraphics[width=.8\textwidth]{./assets/img/computed_probabilities_vs_theoretic_probabilities/main.pdf}
    \caption{The
        relationship between the steady state probabilities inferred from the
        measured transitions and the actual steady state probabilities. A linear
        regression line is included validating the approach.}
    \label{fig:computed_probabilities_vs_theoretic_probabilities}
\end{figure}


\end{document}
 strategies,
was presented with specific consideration given to ZD strategies. This
tournament is reproduced here using the Axelrod-Python
project~\cite{Knight2016}. To obtain a good measure of the corresponding
transition rates for each strategy all matches have been run for
\documentclass[a4paper]{article}

\usepackage{amsmath}
\usepackage{amssymb}
\usepackage[margin=1.5cm,
            includefoot,
            footskip=30pt]{geometry}
\usepackage{layout}
\usepackage{graphicx}
\usepackage{subcaption}

\usepackage{biblatex}
\usepackage{pdfpages}

\bibliography{main.bib}

\title{Suspicion: Recognising and evaluating the effectiveness
       of extortion in the Iterated Prisoner's Dilemma}
\author{Vincent A. Knight \and Nikoleta E. Glynatsi}
\date{\today}



\begin{document}

\maketitle

\begin{abstract}
    The Iterated Prisoner's Dilemma is a model for rational and evolutionary
    interactive behaviour. It has applications both in the study of human social
    behaviour as well as in biology.
    It is used to understand when and how a rational individual might
    accept an immediate cost to their own utility for the direct benefit of
    another.

    Much attention has been given to a class of strategies called
    Zero Determinant strategies. It has been theoretically shown that these
    strategies can ``extort'' any player.

    In this work, an approach to identify if observed strategies are playing in
    an extortionate way is described. Furthermore, experimental analysis of
    a large tournament with \input{assets/tex/number_of_full_strategies/main.tex}
    strategies is considered. In this setting
    the most highly performing strategies do not play in an extortionate way
    against each other but do against lower performing strategies.
    This suggests that whilst the theory of Zero Determinant strategies
    indicates that memory is not of fundamental importance to the evolution of
    cooperative behaviour, this is incomplete.
\end{abstract}

\section{Introduction}\label{sec:introduction}

Agent based game theoretic models have become a stalwart of the underpinning
mathematics of interactive behaviours. One of the major pieces of work
in this area is the pair of original computer tournaments run by Robert
Axelrod~\cite{Axelrod1980, Axelrod1980a}. These tournaments pitted submitted
computer strategies against each other in plays of the Iterated Prisoner's
Dilemma. A common game where agents can choose to pay a slight cost to their
immediate utility in the hope of building a reputation. This has been used in
economic and evolutionary game theory to understand the evolution of cooperative
behaviour.

Recently, a class of strategies was described in~\cite{Press2012} that can
provably extort any given opponent. In~\cite{Hilbe2013, Moran1707} some
questions have already been asked about the true effectiveness of these
strategies in an evolutionary setting. Here another question is asked: is it
possible to recognise this extortionate behaviour? A mathematical procedure for
suspicion is presented: in the same way that the continued actions of an
extortionate individual might raise suspicion.

This work makes use of the Axelrod Python library~\cite{Knight2018, Knight2016}
with a large number of Prisoner Dilemma strategies available to give an
extensive numerical example of the ideas presented.  The approach is presented
in Section~\ref{sec:delta-zd-strategies}.  All of the code and data discussed
in Section~\ref{sec:numerical-experiments} is open sourced, archived and
written according to best scientific principles~\cite{Wilson2014}. The data
archive can be found at~\cite{vincent_knight_2018_1297075}.

\section{Recognising Extortion}\label{sec:delta-zd-strategies}

In~\cite{Press2012}, given a match between 2 memory-one strategies, the concept
of Zero Determinant (ZD) strategies is introduced. The main result of that paper
shows that given two memory one players \(p, q\in\mathbb{R}^4\) a linear
relationship between the players' scores could be forced by one of the players.

Using the notation of~\cite{Press2012}, assuming the utilities for player \(p\)
are given by \(S_x=(R, S, T, P)\) and for player \(q\) by \(S_y=(R, T, S, P)\)
and that the stationary scores of each player is given by \(S_X\) and \(S_Y\)
respectively. The main result of~\cite{Press2012} is that if

\begin{equation}\label{eqn:linear_relationship_for_p}
    \tilde p=\alpha S_x + \beta S_y + \gamma
\end{equation}

or

\begin{equation}\label{eqn:linear_relationship_for_q}
    \tilde q=\alpha S_x + \beta S_y + \gamma
\end{equation}

where \(\tilde p = (1 - p_1, 1 - p_2, p_3, p_4)\) and
\(\tilde q = (1 - q_1, 1 - q_2, q_3, q_4)\) then:

\begin{equation}
    \alpha S_X + \beta S_Y + \gamma = 0
\end{equation}

In~\cite{Press2012} a particular type of ZD strategy is defined: extortionate
strategies. If:

\begin{equation}\label{eqn:constraint_for_extortion}
    \gamma = - P(\alpha + \beta)
\end{equation}

then the player can ensure they get a score \(\chi\) times
larger than the opponent. This extortion coefficient is given by:

\begin{equation}\label{eqn:definition_of_chi}
    \chi=\frac{-\beta}{\alpha}
\end{equation}

Thus, if (\ref{eqn:constraint_for_extortion}) holds and \(\chi >1\) a player is
said to extort their opponent.
Here, the reverse problem is considered: given a
\(p\in\mathbb{R}^4\) how does one identify \(\alpha, \beta\) if they
exist and is the strategy in fact acting in an extortionate way?

These conditions correspond to:

\begin{align}
    \tilde p_1 & = \alpha R + \beta R - P (\alpha + \beta)
            \label{eqn:condition_for_tilde_p1}\\
    \tilde p_2 & = \alpha S + \beta T - P (\alpha + \beta)
            \label{eqn:condition_for_tilde_p2}\\
    \tilde p_3 & = \alpha T + \beta S - P (\alpha + \beta)
            \label{eqn:condition_for_tilde_p3}\\
    \tilde p_4 & = \alpha P + \beta P - P (\alpha + \beta)
            \label{eqn:condition_for_tilde_p4}
\end{align}

Equation (\ref{eqn:condition_for_tilde_p4}) ensures that \(p_4=\tilde p_4=0\).
Equations (\ref{eqn:condition_for_tilde_p1}-\ref{eqn:condition_for_tilde_p3})
can be used to eliminate \(\alpha, \beta\), giving:

\begin{equation}\label{eqn:planar_definition_of_extortion}
    \tilde p_1 = \frac{(R - P)(\tilde p_2 + \tilde p_3)}{S + T - 2P}
\end{equation}

with:

\begin{equation}\label{eqn:definition_of_chi}
    \chi = \frac{\tilde p_2 (P - T) + \tilde p_3 (S - P)}
                {\tilde p_2 (P - S) + \tilde p_3 (T - P)}
\end{equation}

Given a strategy \(p\in\mathbb{R}^{4\times 1}\) equations
(\ref{eqn:condition_for_tilde_p4}), (\ref{eqn:planar_definition_of_extortion}-\ref{eqn:definition_of_chi}) can be used to check if
a strategy is extortionate. The conditions correspond to:

\begin{align}
    p_1 & = \frac{(R-P)(p_2 + p_3) - R + T + S - P}{S + T - 2P}
     \label{eqn:condition_for_p1}\\
    p_4 & = 0 \label{eqn:condition_for_p4}\\
    1 & > p_2 + p_3\label{eqn:condition_for_chi}
\end{align}

The algebraic steps necessary to prove these results are available in the
supporting materials.

All extortionate strategies reside on a triangular (\ref{eqn:condition_for_chi})
plane (\ref{eqn:condition_for_p1}) in 3 dimensions (\ref{eqn:condition_for_p4}).
Using this formulation it can be seen that a necessary (but not sufficient)
condition for an extortionate strategy is that it cooperates on average less
than 50\% of the time when in a state of disagreement with the opponent.

As an example, consider the known extortionate strategy \(p=(8 / 9, 1 / 2, 1 /
3, 0)\) from~\cite{Stewart2012} which is referred to as \texttt{Extort-2}. In
this case, for the standard values of \((R, T, S, P)\) constraint
(\ref{eqn:condition_for_p1}) corresponds to:

\begin{equation}
    p_1 = \frac{2(p_2 + p_3) + 1}{3}
\end{equation}

It is clear that in this case all constraints hold.

This approach could in fact be used to confirm that a given strategy is acting
in an extortionate manner even if it is not a memory one strategy. However, in
practice, if a closed form for \(p\) is not known, then due to measurement
and/or numerical error this would not work.

This problem can be written in the following linear algebraic form where
\(x=(\alpha, \beta)\)
and \(p^*=(\tilde p_1 - 1, tilde_2 - 1, p_3)\):

\begin{equation}\label{eqn:linear_algebraic_equation_for_p}
    Cx= p^*
\end{equation}

\(C\) corresponds to equations
(\ref{eqn:condition_for_tilde_p1}-\ref{eqn:condition_for_tilde_p3}) and is
given by:

\begin{equation}\label{eqn:definition_of_C}
    C =
    \begin{bmatrix}
        R - P & R- P \\
        S - P & T- P \\
        T - P & S- P \\
    \end{bmatrix}
\end{equation}

Note that in general, equation (\ref{eqn:linear_algebraic_equation_for_p}) will
not necessarily have a solution. From the Rouch\'{e}-Capelli theorem if there is
a solution it is unique as \(\text{rank}(C)=2\) which is the dimension of the
variable \(x\). The best fitting \(x\) is found by minimizing:

\begin{equation}\label{eqn:r_squared}
    \text{SSError} = \|C x- p^*\|_2^2 = \sum_{i=1}^{3}\left((C\bar x)_i-p_i^*\right)^2
\end{equation}

Note that \(\text{SSError}\), which is the square of the Frobenius
norm~\cite{Golub2013}, becomes a measure of how close a strategy is to being an
extortionate strategy. Suspicion
of extortion then corresponds to a threshold on \(\text{SSError}\).

By observing interactions (human or otherwise), their memory one representation
can be inferred and this approach can be used to recognise extortionate
behaviour. The notion of comparing theoretic and actual plays of the IPD is not
novel, see for example~\cite{Rand2013}. Immediately it is noted that if the
environment is noisy~\cite{Wu1995} then no strategy can be considered to be
extortionate as \(p_4>0\).

In the next section, this idea will be illustrated by observing the interactions
that take place in a computer based tournament of the IPD\@.

\section{Numerical experiments}\label{sec:numerical-experiments}

In~\cite{Stewart2012} results from a tournament with
\input{./assets/tex/number_of_stewart_plotkin_strategies/main.tex} strategies,
was presented with specific consideration given to ZD strategies. This
tournament is reproduced here using the Axelrod-Python
project~\cite{Knight2016}. To obtain a good measure of the corresponding
transition rates for each strategy all matches have been run for
\input{assets/tex/number_of_turns/main.tex} turns and every match has been
repeated \input{assets/tex/number_of_repetitions/main.tex} times. All of this
interaction data is available at~\cite{vincent_knight_2018_1297075}. A good
match between the inferred Markov chain and the state distribution of the actual
interactions has been verified. Data for this is presented in the supplementary
materials.

Figure~\ref{fig:SSError_overall_in_stewart_plotkin} shows the \(\text{SSError}\)
values for all the strategies in the tournament, as reported
in~\cite{Stewart2012} the extortionate strategy (which has an expected
\(\text{SSError}\) approximately 0) gains a large number of wins.

\begin{figure}[!htbp]
    \centering
    \includegraphics[width=.8\textwidth]{./assets/img/SSError_overall_in_stewart_plotkin/main.pdf}
    \caption{\(\text{SSError}\) and state probabilities for the strategies
        of~\cite{Stewart2012}, ordered both by number of wins and overall score.
        Note that \(P(DC)\) is not shown as it corresponds to the transpose of
        \(P(CD)\). Cooperator and Defector are omitted as they do not visit all
        the states.}
    \label{fig:SSError_overall_in_stewart_plotkin}
\end{figure}

Here, the work of~\cite{Stewart2012} is extended by investigating a tournament
with \input{assets/tex/number_of_full_strategies/main.tex}
strategies.

The results of this analysis are shown in
Figure~\ref{fig:SSError_and_probabilities_in_full}. The top ranking strategies
by number of wins seem to be extortionate (but not against all strategies) and
it can be seen that a small sub group of strategies achieve mutual defection.
All the top ranking strategies according to score achieve mutual cooperation and
do not extort each other, however they
\textbf{do} exhibit extortionate behaviour towards a number of the lower ranking
strategies.

\begin{figure}[!htbp]
    \centering
    \includegraphics[width=.8\textwidth]{./assets/img/SSError_and_probabilities_in_full/main.pdf}
    \caption{\(\text{SSError}\) for the strategies for the full tournament. Only
    strategy interactions for which \(p_4=0\) and \(\chi>1\) are displayed.}
    \label{fig:SSError_and_probabilities_in_full}
\end{figure}

\section{Conclusion}\label{sec:conclusion}

This work defines an approach to measure whether or not a player is playing a
strategy that corresponds to an extortionate strategy as defined
in~\cite{Press2012}: a mathematical model for suspicion. Indeed, all
extortionate strategies have been
 classified as lying on a triangular plane.
This rigorous classification fails to be robust to small measurement error, thus
a statistical approach is proposed.
This is done through a linear algebraic approach for approximating the solution
of a linear system. Using this, a large number of pairwise interactions is
simulated and in fact very few strategies are found to act extortionately.

The work of~\cite{Press2012}, whilst showing that a clever approach to taking
advantage of another memory one strategy exists: this is incomplete. Whilst the
elegance of this result is very attractive, just as the simplicity of the
victory of Tit For Tat in Axelrod's original tournaments was, it is incomplete.
Extortionate strategies achieve a high number of wins but they do not
achieve a high score which corresponds to the fitness landscape in an
evolutionary sense. From the large number of interactions a payoff matrix \(S\)
can be measured where \(S_{ij}\) denotes the score (using standard values of
\((R, S, T, P) = (3, 0, 5, 1)\)) of the \(i\)th strategy
against the \(j\)th strategy. Using this, the replicator equation
describes the evolution of the system based on a population density fitness
function:

\begin{equation}\label{eqn:replicator_dynamics}
    \frac{dx}{dt} = x(S-x^TS x)
\end{equation}

Equation (\ref{eqn:replicator_dynamics}) is solved numerically through an
integration technique described in~\cite{Petzold1983} and
Figure~\ref{fig:replicator_dynamics} shows the evolution of the distribution of
the system: the various strategies are ranked by scores. It is clear to see that
only the high ranking strategies survive the evolutionary process (in fact,
only \input{./assets/img/replicator_dynamics/main.tex}
have a final distribution greater than \(10 ^ {-2}\)). This confirms the
findings of~\cite{Moran1707} in which sophisticated strategies resist
evolutionary invasion of shorter memory strategies. Recalling
Figure~\ref{fig:SSError_and_probabilities_in_full} this demonstrates that:

\begin{itemize}
    \item Cooperation emerges through the evolutionary process: the high scoring
        strategies do not exhibit extortionate behaviour towards each other.
    \item Extortionate strategies do not survive the evolutionary process.
\end{itemize}

\begin{figure}[!htbp]
    \centering
    \includegraphics[width=.8\textwidth]{./assets/img/replicator_dynamics/main.pdf}
    \caption{Numerical simulation of the replicator equation
    (\ref{eqn:replicator_dynamics}): strategies are ordered by score, only the strategies with a high score survive the evolutionary process.}
    \label{fig:replicator_dynamics}
\end{figure}

This work can be used to classify plays of the IPD\@: data can be collected from
actual interactions (in lab or in the field). Furthermore, this allows for a
classification method similar to the notion of fingerprinting presented
in~\cite{Ashlock2008}. Trained strategies can potentially be classified as
extortionate or not or it could be possible to even constrain the reinforcement
learning approaches that are becoming prevalent in the literature.
Alternatively, this mathematical approach for recognising extortion could be
used in sophisticated strategies to defend against invasion. Arguably, some of
the strategies considered here exhibit this behaviour, indeed as described
in~\cite{Harper2017}, the top ranking strategies in the full tournament are
obtained using evolutionary reinforcement learning techniques, thus, suspicion
of extortionate behaviour could in fact be an evolutionary trait.

\section*{Acknowledgements}

The following open source software libraries were used in this research:

\begin{itemize}
    \item The Axelrod ~\cite{Knight2016, Knight2018} library (IPD strategies and
        tournaments).
    \item The sympy library~\cite{Meurer2017} (verification of all symbolic
        calculations).
    \item The matplotlib~\cite{Droettboom2018} library (visualisation).
    \item The pandas~\cite{Structures2010}, dask~\cite{Dask2016} and
        NumPy~\cite{Oliphant2015} libraries (data manipulation).
    \item The SciPy~\cite{Jones2001} library (numerical integration of the
        replicator equation).
\end{itemize}

This work was performed using the computational facilities of the Advanced
Research Computing @ Cardiff (ARCCA) Division, Cardiff University.

\printbibliography

\newpage
\section*{Supplementary materials}

\includepdf{assets/pdf/proof_of_form_of_extortionate_strategies/main.pdf}

\newpage

Using the pair wise interactions the transition rates \(p,
q\) can be measured and the steady state probabilities inferred and compared to
the actual probabilities of each state.
This is done numerically by computing the singular eigenvector of the
matrix \(A\) \cite{Stewart2009}:

\[
    A =
    \begin{bmatrix}
        p_1 q_1 & p_1 (1 - q_1) & (1 - p_1) q_1 & (1 -p_1) (1 - q_1) \\
        p_2 q_2 & p_2 (1 - q_2) & (1 - p_2) q_2 & (1 -p_2) (1 - q_2) \\
        p_3 q_3 & p_3 (1 - q_3) & (1 - p_3) q_3 & (1 -p_3) (1 - q_3) \\
        p_4 q_4 & p_4 (1 - q_4) & (1 - p_4) q_4 & (1 -p_4) (1 - q_4) \\
    \end{bmatrix}
\]

Figure~\ref{fig:computed_probabilities_vs_theoretic_probabilities} shows a
regression line fitted to every pairwise interaction with a reported
\(\text{SSError}\) value (pairwise interactions with missing states were
omitted). This serves to validate the approach: a part from some edge cases the
relationship is consistent.

\begin{figure}[!htbp]
    \centering
    \includegraphics[width=.8\textwidth]{./assets/img/computed_probabilities_vs_theoretic_probabilities/main.pdf}
    \caption{The
        relationship between the steady state probabilities inferred from the
        measured transitions and the actual steady state probabilities. A linear
        regression line is included validating the approach.}
    \label{fig:computed_probabilities_vs_theoretic_probabilities}
\end{figure}


\end{document}
 turns and every match has been
repeated \documentclass[a4paper]{article}

\usepackage{amsmath}
\usepackage{amssymb}
\usepackage[margin=1.5cm,
            includefoot,
            footskip=30pt]{geometry}
\usepackage{layout}
\usepackage{graphicx}
\usepackage{subcaption}

\usepackage{biblatex}
\usepackage{pdfpages}

\bibliography{main.bib}

\title{Suspicion: Recognising and evaluating the effectiveness
       of extortion in the Iterated Prisoner's Dilemma}
\author{Vincent A. Knight \and Nikoleta E. Glynatsi}
\date{\today}



\begin{document}

\maketitle

\begin{abstract}
    The Iterated Prisoner's Dilemma is a model for rational and evolutionary
    interactive behaviour. It has applications both in the study of human social
    behaviour as well as in biology.
    It is used to understand when and how a rational individual might
    accept an immediate cost to their own utility for the direct benefit of
    another.

    Much attention has been given to a class of strategies called
    Zero Determinant strategies. It has been theoretically shown that these
    strategies can ``extort'' any player.

    In this work, an approach to identify if observed strategies are playing in
    an extortionate way is described. Furthermore, experimental analysis of
    a large tournament with \input{assets/tex/number_of_full_strategies/main.tex}
    strategies is considered. In this setting
    the most highly performing strategies do not play in an extortionate way
    against each other but do against lower performing strategies.
    This suggests that whilst the theory of Zero Determinant strategies
    indicates that memory is not of fundamental importance to the evolution of
    cooperative behaviour, this is incomplete.
\end{abstract}

\section{Introduction}\label{sec:introduction}

Agent based game theoretic models have become a stalwart of the underpinning
mathematics of interactive behaviours. One of the major pieces of work
in this area is the pair of original computer tournaments run by Robert
Axelrod~\cite{Axelrod1980, Axelrod1980a}. These tournaments pitted submitted
computer strategies against each other in plays of the Iterated Prisoner's
Dilemma. A common game where agents can choose to pay a slight cost to their
immediate utility in the hope of building a reputation. This has been used in
economic and evolutionary game theory to understand the evolution of cooperative
behaviour.

Recently, a class of strategies was described in~\cite{Press2012} that can
provably extort any given opponent. In~\cite{Hilbe2013, Moran1707} some
questions have already been asked about the true effectiveness of these
strategies in an evolutionary setting. Here another question is asked: is it
possible to recognise this extortionate behaviour? A mathematical procedure for
suspicion is presented: in the same way that the continued actions of an
extortionate individual might raise suspicion.

This work makes use of the Axelrod Python library~\cite{Knight2018, Knight2016}
with a large number of Prisoner Dilemma strategies available to give an
extensive numerical example of the ideas presented.  The approach is presented
in Section~\ref{sec:delta-zd-strategies}.  All of the code and data discussed
in Section~\ref{sec:numerical-experiments} is open sourced, archived and
written according to best scientific principles~\cite{Wilson2014}. The data
archive can be found at~\cite{vincent_knight_2018_1297075}.

\section{Recognising Extortion}\label{sec:delta-zd-strategies}

In~\cite{Press2012}, given a match between 2 memory-one strategies, the concept
of Zero Determinant (ZD) strategies is introduced. The main result of that paper
shows that given two memory one players \(p, q\in\mathbb{R}^4\) a linear
relationship between the players' scores could be forced by one of the players.

Using the notation of~\cite{Press2012}, assuming the utilities for player \(p\)
are given by \(S_x=(R, S, T, P)\) and for player \(q\) by \(S_y=(R, T, S, P)\)
and that the stationary scores of each player is given by \(S_X\) and \(S_Y\)
respectively. The main result of~\cite{Press2012} is that if

\begin{equation}\label{eqn:linear_relationship_for_p}
    \tilde p=\alpha S_x + \beta S_y + \gamma
\end{equation}

or

\begin{equation}\label{eqn:linear_relationship_for_q}
    \tilde q=\alpha S_x + \beta S_y + \gamma
\end{equation}

where \(\tilde p = (1 - p_1, 1 - p_2, p_3, p_4)\) and
\(\tilde q = (1 - q_1, 1 - q_2, q_3, q_4)\) then:

\begin{equation}
    \alpha S_X + \beta S_Y + \gamma = 0
\end{equation}

In~\cite{Press2012} a particular type of ZD strategy is defined: extortionate
strategies. If:

\begin{equation}\label{eqn:constraint_for_extortion}
    \gamma = - P(\alpha + \beta)
\end{equation}

then the player can ensure they get a score \(\chi\) times
larger than the opponent. This extortion coefficient is given by:

\begin{equation}\label{eqn:definition_of_chi}
    \chi=\frac{-\beta}{\alpha}
\end{equation}

Thus, if (\ref{eqn:constraint_for_extortion}) holds and \(\chi >1\) a player is
said to extort their opponent.
Here, the reverse problem is considered: given a
\(p\in\mathbb{R}^4\) how does one identify \(\alpha, \beta\) if they
exist and is the strategy in fact acting in an extortionate way?

These conditions correspond to:

\begin{align}
    \tilde p_1 & = \alpha R + \beta R - P (\alpha + \beta)
            \label{eqn:condition_for_tilde_p1}\\
    \tilde p_2 & = \alpha S + \beta T - P (\alpha + \beta)
            \label{eqn:condition_for_tilde_p2}\\
    \tilde p_3 & = \alpha T + \beta S - P (\alpha + \beta)
            \label{eqn:condition_for_tilde_p3}\\
    \tilde p_4 & = \alpha P + \beta P - P (\alpha + \beta)
            \label{eqn:condition_for_tilde_p4}
\end{align}

Equation (\ref{eqn:condition_for_tilde_p4}) ensures that \(p_4=\tilde p_4=0\).
Equations (\ref{eqn:condition_for_tilde_p1}-\ref{eqn:condition_for_tilde_p3})
can be used to eliminate \(\alpha, \beta\), giving:

\begin{equation}\label{eqn:planar_definition_of_extortion}
    \tilde p_1 = \frac{(R - P)(\tilde p_2 + \tilde p_3)}{S + T - 2P}
\end{equation}

with:

\begin{equation}\label{eqn:definition_of_chi}
    \chi = \frac{\tilde p_2 (P - T) + \tilde p_3 (S - P)}
                {\tilde p_2 (P - S) + \tilde p_3 (T - P)}
\end{equation}

Given a strategy \(p\in\mathbb{R}^{4\times 1}\) equations
(\ref{eqn:condition_for_tilde_p4}), (\ref{eqn:planar_definition_of_extortion}-\ref{eqn:definition_of_chi}) can be used to check if
a strategy is extortionate. The conditions correspond to:

\begin{align}
    p_1 & = \frac{(R-P)(p_2 + p_3) - R + T + S - P}{S + T - 2P}
     \label{eqn:condition_for_p1}\\
    p_4 & = 0 \label{eqn:condition_for_p4}\\
    1 & > p_2 + p_3\label{eqn:condition_for_chi}
\end{align}

The algebraic steps necessary to prove these results are available in the
supporting materials.

All extortionate strategies reside on a triangular (\ref{eqn:condition_for_chi})
plane (\ref{eqn:condition_for_p1}) in 3 dimensions (\ref{eqn:condition_for_p4}).
Using this formulation it can be seen that a necessary (but not sufficient)
condition for an extortionate strategy is that it cooperates on average less
than 50\% of the time when in a state of disagreement with the opponent.

As an example, consider the known extortionate strategy \(p=(8 / 9, 1 / 2, 1 /
3, 0)\) from~\cite{Stewart2012} which is referred to as \texttt{Extort-2}. In
this case, for the standard values of \((R, T, S, P)\) constraint
(\ref{eqn:condition_for_p1}) corresponds to:

\begin{equation}
    p_1 = \frac{2(p_2 + p_3) + 1}{3}
\end{equation}

It is clear that in this case all constraints hold.

This approach could in fact be used to confirm that a given strategy is acting
in an extortionate manner even if it is not a memory one strategy. However, in
practice, if a closed form for \(p\) is not known, then due to measurement
and/or numerical error this would not work.

This problem can be written in the following linear algebraic form where
\(x=(\alpha, \beta)\)
and \(p^*=(\tilde p_1 - 1, tilde_2 - 1, p_3)\):

\begin{equation}\label{eqn:linear_algebraic_equation_for_p}
    Cx= p^*
\end{equation}

\(C\) corresponds to equations
(\ref{eqn:condition_for_tilde_p1}-\ref{eqn:condition_for_tilde_p3}) and is
given by:

\begin{equation}\label{eqn:definition_of_C}
    C =
    \begin{bmatrix}
        R - P & R- P \\
        S - P & T- P \\
        T - P & S- P \\
    \end{bmatrix}
\end{equation}

Note that in general, equation (\ref{eqn:linear_algebraic_equation_for_p}) will
not necessarily have a solution. From the Rouch\'{e}-Capelli theorem if there is
a solution it is unique as \(\text{rank}(C)=2\) which is the dimension of the
variable \(x\). The best fitting \(x\) is found by minimizing:

\begin{equation}\label{eqn:r_squared}
    \text{SSError} = \|C x- p^*\|_2^2 = \sum_{i=1}^{3}\left((C\bar x)_i-p_i^*\right)^2
\end{equation}

Note that \(\text{SSError}\), which is the square of the Frobenius
norm~\cite{Golub2013}, becomes a measure of how close a strategy is to being an
extortionate strategy. Suspicion
of extortion then corresponds to a threshold on \(\text{SSError}\).

By observing interactions (human or otherwise), their memory one representation
can be inferred and this approach can be used to recognise extortionate
behaviour. The notion of comparing theoretic and actual plays of the IPD is not
novel, see for example~\cite{Rand2013}. Immediately it is noted that if the
environment is noisy~\cite{Wu1995} then no strategy can be considered to be
extortionate as \(p_4>0\).

In the next section, this idea will be illustrated by observing the interactions
that take place in a computer based tournament of the IPD\@.

\section{Numerical experiments}\label{sec:numerical-experiments}

In~\cite{Stewart2012} results from a tournament with
\input{./assets/tex/number_of_stewart_plotkin_strategies/main.tex} strategies,
was presented with specific consideration given to ZD strategies. This
tournament is reproduced here using the Axelrod-Python
project~\cite{Knight2016}. To obtain a good measure of the corresponding
transition rates for each strategy all matches have been run for
\input{assets/tex/number_of_turns/main.tex} turns and every match has been
repeated \input{assets/tex/number_of_repetitions/main.tex} times. All of this
interaction data is available at~\cite{vincent_knight_2018_1297075}. A good
match between the inferred Markov chain and the state distribution of the actual
interactions has been verified. Data for this is presented in the supplementary
materials.

Figure~\ref{fig:SSError_overall_in_stewart_plotkin} shows the \(\text{SSError}\)
values for all the strategies in the tournament, as reported
in~\cite{Stewart2012} the extortionate strategy (which has an expected
\(\text{SSError}\) approximately 0) gains a large number of wins.

\begin{figure}[!htbp]
    \centering
    \includegraphics[width=.8\textwidth]{./assets/img/SSError_overall_in_stewart_plotkin/main.pdf}
    \caption{\(\text{SSError}\) and state probabilities for the strategies
        of~\cite{Stewart2012}, ordered both by number of wins and overall score.
        Note that \(P(DC)\) is not shown as it corresponds to the transpose of
        \(P(CD)\). Cooperator and Defector are omitted as they do not visit all
        the states.}
    \label{fig:SSError_overall_in_stewart_plotkin}
\end{figure}

Here, the work of~\cite{Stewart2012} is extended by investigating a tournament
with \input{assets/tex/number_of_full_strategies/main.tex}
strategies.

The results of this analysis are shown in
Figure~\ref{fig:SSError_and_probabilities_in_full}. The top ranking strategies
by number of wins seem to be extortionate (but not against all strategies) and
it can be seen that a small sub group of strategies achieve mutual defection.
All the top ranking strategies according to score achieve mutual cooperation and
do not extort each other, however they
\textbf{do} exhibit extortionate behaviour towards a number of the lower ranking
strategies.

\begin{figure}[!htbp]
    \centering
    \includegraphics[width=.8\textwidth]{./assets/img/SSError_and_probabilities_in_full/main.pdf}
    \caption{\(\text{SSError}\) for the strategies for the full tournament. Only
    strategy interactions for which \(p_4=0\) and \(\chi>1\) are displayed.}
    \label{fig:SSError_and_probabilities_in_full}
\end{figure}

\section{Conclusion}\label{sec:conclusion}

This work defines an approach to measure whether or not a player is playing a
strategy that corresponds to an extortionate strategy as defined
in~\cite{Press2012}: a mathematical model for suspicion. Indeed, all
extortionate strategies have been
 classified as lying on a triangular plane.
This rigorous classification fails to be robust to small measurement error, thus
a statistical approach is proposed.
This is done through a linear algebraic approach for approximating the solution
of a linear system. Using this, a large number of pairwise interactions is
simulated and in fact very few strategies are found to act extortionately.

The work of~\cite{Press2012}, whilst showing that a clever approach to taking
advantage of another memory one strategy exists: this is incomplete. Whilst the
elegance of this result is very attractive, just as the simplicity of the
victory of Tit For Tat in Axelrod's original tournaments was, it is incomplete.
Extortionate strategies achieve a high number of wins but they do not
achieve a high score which corresponds to the fitness landscape in an
evolutionary sense. From the large number of interactions a payoff matrix \(S\)
can be measured where \(S_{ij}\) denotes the score (using standard values of
\((R, S, T, P) = (3, 0, 5, 1)\)) of the \(i\)th strategy
against the \(j\)th strategy. Using this, the replicator equation
describes the evolution of the system based on a population density fitness
function:

\begin{equation}\label{eqn:replicator_dynamics}
    \frac{dx}{dt} = x(S-x^TS x)
\end{equation}

Equation (\ref{eqn:replicator_dynamics}) is solved numerically through an
integration technique described in~\cite{Petzold1983} and
Figure~\ref{fig:replicator_dynamics} shows the evolution of the distribution of
the system: the various strategies are ranked by scores. It is clear to see that
only the high ranking strategies survive the evolutionary process (in fact,
only \input{./assets/img/replicator_dynamics/main.tex}
have a final distribution greater than \(10 ^ {-2}\)). This confirms the
findings of~\cite{Moran1707} in which sophisticated strategies resist
evolutionary invasion of shorter memory strategies. Recalling
Figure~\ref{fig:SSError_and_probabilities_in_full} this demonstrates that:

\begin{itemize}
    \item Cooperation emerges through the evolutionary process: the high scoring
        strategies do not exhibit extortionate behaviour towards each other.
    \item Extortionate strategies do not survive the evolutionary process.
\end{itemize}

\begin{figure}[!htbp]
    \centering
    \includegraphics[width=.8\textwidth]{./assets/img/replicator_dynamics/main.pdf}
    \caption{Numerical simulation of the replicator equation
    (\ref{eqn:replicator_dynamics}): strategies are ordered by score, only the strategies with a high score survive the evolutionary process.}
    \label{fig:replicator_dynamics}
\end{figure}

This work can be used to classify plays of the IPD\@: data can be collected from
actual interactions (in lab or in the field). Furthermore, this allows for a
classification method similar to the notion of fingerprinting presented
in~\cite{Ashlock2008}. Trained strategies can potentially be classified as
extortionate or not or it could be possible to even constrain the reinforcement
learning approaches that are becoming prevalent in the literature.
Alternatively, this mathematical approach for recognising extortion could be
used in sophisticated strategies to defend against invasion. Arguably, some of
the strategies considered here exhibit this behaviour, indeed as described
in~\cite{Harper2017}, the top ranking strategies in the full tournament are
obtained using evolutionary reinforcement learning techniques, thus, suspicion
of extortionate behaviour could in fact be an evolutionary trait.

\section*{Acknowledgements}

The following open source software libraries were used in this research:

\begin{itemize}
    \item The Axelrod ~\cite{Knight2016, Knight2018} library (IPD strategies and
        tournaments).
    \item The sympy library~\cite{Meurer2017} (verification of all symbolic
        calculations).
    \item The matplotlib~\cite{Droettboom2018} library (visualisation).
    \item The pandas~\cite{Structures2010}, dask~\cite{Dask2016} and
        NumPy~\cite{Oliphant2015} libraries (data manipulation).
    \item The SciPy~\cite{Jones2001} library (numerical integration of the
        replicator equation).
\end{itemize}

This work was performed using the computational facilities of the Advanced
Research Computing @ Cardiff (ARCCA) Division, Cardiff University.

\printbibliography

\newpage
\section*{Supplementary materials}

\includepdf{assets/pdf/proof_of_form_of_extortionate_strategies/main.pdf}

\newpage

Using the pair wise interactions the transition rates \(p,
q\) can be measured and the steady state probabilities inferred and compared to
the actual probabilities of each state.
This is done numerically by computing the singular eigenvector of the
matrix \(A\) \cite{Stewart2009}:

\[
    A =
    \begin{bmatrix}
        p_1 q_1 & p_1 (1 - q_1) & (1 - p_1) q_1 & (1 -p_1) (1 - q_1) \\
        p_2 q_2 & p_2 (1 - q_2) & (1 - p_2) q_2 & (1 -p_2) (1 - q_2) \\
        p_3 q_3 & p_3 (1 - q_3) & (1 - p_3) q_3 & (1 -p_3) (1 - q_3) \\
        p_4 q_4 & p_4 (1 - q_4) & (1 - p_4) q_4 & (1 -p_4) (1 - q_4) \\
    \end{bmatrix}
\]

Figure~\ref{fig:computed_probabilities_vs_theoretic_probabilities} shows a
regression line fitted to every pairwise interaction with a reported
\(\text{SSError}\) value (pairwise interactions with missing states were
omitted). This serves to validate the approach: a part from some edge cases the
relationship is consistent.

\begin{figure}[!htbp]
    \centering
    \includegraphics[width=.8\textwidth]{./assets/img/computed_probabilities_vs_theoretic_probabilities/main.pdf}
    \caption{The
        relationship between the steady state probabilities inferred from the
        measured transitions and the actual steady state probabilities. A linear
        regression line is included validating the approach.}
    \label{fig:computed_probabilities_vs_theoretic_probabilities}
\end{figure}


\end{document}
 times. All of this
interaction data is available at~\cite{vincent_knight_2018_1297075}. A good
match between the inferred Markov chain and the state distribution of the actual
interactions has been verified. Data for this is presented in the supplementary
materials.

Figure~\ref{fig:SSError_overall_in_stewart_plotkin} shows the \(\text{SSError}\)
values for all the strategies in the tournament, as reported
in~\cite{Stewart2012} the extortionate strategy (which has an expected
\(\text{SSError}\) approximately 0) gains a large number of wins.

\begin{figure}[!htbp]
    \centering
    \includegraphics[width=.8\textwidth]{./assets/img/SSError_overall_in_stewart_plotkin/main.pdf}
    \caption{\(\text{SSError}\) and state probabilities for the strategies
        of~\cite{Stewart2012}, ordered both by number of wins and overall score.
        Note that \(P(DC)\) is not shown as it corresponds to the transpose of
        \(P(CD)\). Cooperator and Defector are omitted as they do not visit all
        the states.}
    \label{fig:SSError_overall_in_stewart_plotkin}
\end{figure}

Here, the work of~\cite{Stewart2012} is extended by investigating a tournament
with \documentclass[a4paper]{article}

\usepackage{amsmath}
\usepackage{amssymb}
\usepackage[margin=1.5cm,
            includefoot,
            footskip=30pt]{geometry}
\usepackage{layout}
\usepackage{graphicx}
\usepackage{subcaption}

\usepackage{biblatex}
\usepackage{pdfpages}

\bibliography{main.bib}

\title{Suspicion: Recognising and evaluating the effectiveness
       of extortion in the Iterated Prisoner's Dilemma}
\author{Vincent A. Knight \and Nikoleta E. Glynatsi}
\date{\today}



\begin{document}

\maketitle

\begin{abstract}
    The Iterated Prisoner's Dilemma is a model for rational and evolutionary
    interactive behaviour. It has applications both in the study of human social
    behaviour as well as in biology.
    It is used to understand when and how a rational individual might
    accept an immediate cost to their own utility for the direct benefit of
    another.

    Much attention has been given to a class of strategies called
    Zero Determinant strategies. It has been theoretically shown that these
    strategies can ``extort'' any player.

    In this work, an approach to identify if observed strategies are playing in
    an extortionate way is described. Furthermore, experimental analysis of
    a large tournament with \input{assets/tex/number_of_full_strategies/main.tex}
    strategies is considered. In this setting
    the most highly performing strategies do not play in an extortionate way
    against each other but do against lower performing strategies.
    This suggests that whilst the theory of Zero Determinant strategies
    indicates that memory is not of fundamental importance to the evolution of
    cooperative behaviour, this is incomplete.
\end{abstract}

\section{Introduction}\label{sec:introduction}

Agent based game theoretic models have become a stalwart of the underpinning
mathematics of interactive behaviours. One of the major pieces of work
in this area is the pair of original computer tournaments run by Robert
Axelrod~\cite{Axelrod1980, Axelrod1980a}. These tournaments pitted submitted
computer strategies against each other in plays of the Iterated Prisoner's
Dilemma. A common game where agents can choose to pay a slight cost to their
immediate utility in the hope of building a reputation. This has been used in
economic and evolutionary game theory to understand the evolution of cooperative
behaviour.

Recently, a class of strategies was described in~\cite{Press2012} that can
provably extort any given opponent. In~\cite{Hilbe2013, Moran1707} some
questions have already been asked about the true effectiveness of these
strategies in an evolutionary setting. Here another question is asked: is it
possible to recognise this extortionate behaviour? A mathematical procedure for
suspicion is presented: in the same way that the continued actions of an
extortionate individual might raise suspicion.

This work makes use of the Axelrod Python library~\cite{Knight2018, Knight2016}
with a large number of Prisoner Dilemma strategies available to give an
extensive numerical example of the ideas presented.  The approach is presented
in Section~\ref{sec:delta-zd-strategies}.  All of the code and data discussed
in Section~\ref{sec:numerical-experiments} is open sourced, archived and
written according to best scientific principles~\cite{Wilson2014}. The data
archive can be found at~\cite{vincent_knight_2018_1297075}.

\section{Recognising Extortion}\label{sec:delta-zd-strategies}

In~\cite{Press2012}, given a match between 2 memory-one strategies, the concept
of Zero Determinant (ZD) strategies is introduced. The main result of that paper
shows that given two memory one players \(p, q\in\mathbb{R}^4\) a linear
relationship between the players' scores could be forced by one of the players.

Using the notation of~\cite{Press2012}, assuming the utilities for player \(p\)
are given by \(S_x=(R, S, T, P)\) and for player \(q\) by \(S_y=(R, T, S, P)\)
and that the stationary scores of each player is given by \(S_X\) and \(S_Y\)
respectively. The main result of~\cite{Press2012} is that if

\begin{equation}\label{eqn:linear_relationship_for_p}
    \tilde p=\alpha S_x + \beta S_y + \gamma
\end{equation}

or

\begin{equation}\label{eqn:linear_relationship_for_q}
    \tilde q=\alpha S_x + \beta S_y + \gamma
\end{equation}

where \(\tilde p = (1 - p_1, 1 - p_2, p_3, p_4)\) and
\(\tilde q = (1 - q_1, 1 - q_2, q_3, q_4)\) then:

\begin{equation}
    \alpha S_X + \beta S_Y + \gamma = 0
\end{equation}

In~\cite{Press2012} a particular type of ZD strategy is defined: extortionate
strategies. If:

\begin{equation}\label{eqn:constraint_for_extortion}
    \gamma = - P(\alpha + \beta)
\end{equation}

then the player can ensure they get a score \(\chi\) times
larger than the opponent. This extortion coefficient is given by:

\begin{equation}\label{eqn:definition_of_chi}
    \chi=\frac{-\beta}{\alpha}
\end{equation}

Thus, if (\ref{eqn:constraint_for_extortion}) holds and \(\chi >1\) a player is
said to extort their opponent.
Here, the reverse problem is considered: given a
\(p\in\mathbb{R}^4\) how does one identify \(\alpha, \beta\) if they
exist and is the strategy in fact acting in an extortionate way?

These conditions correspond to:

\begin{align}
    \tilde p_1 & = \alpha R + \beta R - P (\alpha + \beta)
            \label{eqn:condition_for_tilde_p1}\\
    \tilde p_2 & = \alpha S + \beta T - P (\alpha + \beta)
            \label{eqn:condition_for_tilde_p2}\\
    \tilde p_3 & = \alpha T + \beta S - P (\alpha + \beta)
            \label{eqn:condition_for_tilde_p3}\\
    \tilde p_4 & = \alpha P + \beta P - P (\alpha + \beta)
            \label{eqn:condition_for_tilde_p4}
\end{align}

Equation (\ref{eqn:condition_for_tilde_p4}) ensures that \(p_4=\tilde p_4=0\).
Equations (\ref{eqn:condition_for_tilde_p1}-\ref{eqn:condition_for_tilde_p3})
can be used to eliminate \(\alpha, \beta\), giving:

\begin{equation}\label{eqn:planar_definition_of_extortion}
    \tilde p_1 = \frac{(R - P)(\tilde p_2 + \tilde p_3)}{S + T - 2P}
\end{equation}

with:

\begin{equation}\label{eqn:definition_of_chi}
    \chi = \frac{\tilde p_2 (P - T) + \tilde p_3 (S - P)}
                {\tilde p_2 (P - S) + \tilde p_3 (T - P)}
\end{equation}

Given a strategy \(p\in\mathbb{R}^{4\times 1}\) equations
(\ref{eqn:condition_for_tilde_p4}), (\ref{eqn:planar_definition_of_extortion}-\ref{eqn:definition_of_chi}) can be used to check if
a strategy is extortionate. The conditions correspond to:

\begin{align}
    p_1 & = \frac{(R-P)(p_2 + p_3) - R + T + S - P}{S + T - 2P}
     \label{eqn:condition_for_p1}\\
    p_4 & = 0 \label{eqn:condition_for_p4}\\
    1 & > p_2 + p_3\label{eqn:condition_for_chi}
\end{align}

The algebraic steps necessary to prove these results are available in the
supporting materials.

All extortionate strategies reside on a triangular (\ref{eqn:condition_for_chi})
plane (\ref{eqn:condition_for_p1}) in 3 dimensions (\ref{eqn:condition_for_p4}).
Using this formulation it can be seen that a necessary (but not sufficient)
condition for an extortionate strategy is that it cooperates on average less
than 50\% of the time when in a state of disagreement with the opponent.

As an example, consider the known extortionate strategy \(p=(8 / 9, 1 / 2, 1 /
3, 0)\) from~\cite{Stewart2012} which is referred to as \texttt{Extort-2}. In
this case, for the standard values of \((R, T, S, P)\) constraint
(\ref{eqn:condition_for_p1}) corresponds to:

\begin{equation}
    p_1 = \frac{2(p_2 + p_3) + 1}{3}
\end{equation}

It is clear that in this case all constraints hold.

This approach could in fact be used to confirm that a given strategy is acting
in an extortionate manner even if it is not a memory one strategy. However, in
practice, if a closed form for \(p\) is not known, then due to measurement
and/or numerical error this would not work.

This problem can be written in the following linear algebraic form where
\(x=(\alpha, \beta)\)
and \(p^*=(\tilde p_1 - 1, tilde_2 - 1, p_3)\):

\begin{equation}\label{eqn:linear_algebraic_equation_for_p}
    Cx= p^*
\end{equation}

\(C\) corresponds to equations
(\ref{eqn:condition_for_tilde_p1}-\ref{eqn:condition_for_tilde_p3}) and is
given by:

\begin{equation}\label{eqn:definition_of_C}
    C =
    \begin{bmatrix}
        R - P & R- P \\
        S - P & T- P \\
        T - P & S- P \\
    \end{bmatrix}
\end{equation}

Note that in general, equation (\ref{eqn:linear_algebraic_equation_for_p}) will
not necessarily have a solution. From the Rouch\'{e}-Capelli theorem if there is
a solution it is unique as \(\text{rank}(C)=2\) which is the dimension of the
variable \(x\). The best fitting \(x\) is found by minimizing:

\begin{equation}\label{eqn:r_squared}
    \text{SSError} = \|C x- p^*\|_2^2 = \sum_{i=1}^{3}\left((C\bar x)_i-p_i^*\right)^2
\end{equation}

Note that \(\text{SSError}\), which is the square of the Frobenius
norm~\cite{Golub2013}, becomes a measure of how close a strategy is to being an
extortionate strategy. Suspicion
of extortion then corresponds to a threshold on \(\text{SSError}\).

By observing interactions (human or otherwise), their memory one representation
can be inferred and this approach can be used to recognise extortionate
behaviour. The notion of comparing theoretic and actual plays of the IPD is not
novel, see for example~\cite{Rand2013}. Immediately it is noted that if the
environment is noisy~\cite{Wu1995} then no strategy can be considered to be
extortionate as \(p_4>0\).

In the next section, this idea will be illustrated by observing the interactions
that take place in a computer based tournament of the IPD\@.

\section{Numerical experiments}\label{sec:numerical-experiments}

In~\cite{Stewart2012} results from a tournament with
\input{./assets/tex/number_of_stewart_plotkin_strategies/main.tex} strategies,
was presented with specific consideration given to ZD strategies. This
tournament is reproduced here using the Axelrod-Python
project~\cite{Knight2016}. To obtain a good measure of the corresponding
transition rates for each strategy all matches have been run for
\input{assets/tex/number_of_turns/main.tex} turns and every match has been
repeated \input{assets/tex/number_of_repetitions/main.tex} times. All of this
interaction data is available at~\cite{vincent_knight_2018_1297075}. A good
match between the inferred Markov chain and the state distribution of the actual
interactions has been verified. Data for this is presented in the supplementary
materials.

Figure~\ref{fig:SSError_overall_in_stewart_plotkin} shows the \(\text{SSError}\)
values for all the strategies in the tournament, as reported
in~\cite{Stewart2012} the extortionate strategy (which has an expected
\(\text{SSError}\) approximately 0) gains a large number of wins.

\begin{figure}[!htbp]
    \centering
    \includegraphics[width=.8\textwidth]{./assets/img/SSError_overall_in_stewart_plotkin/main.pdf}
    \caption{\(\text{SSError}\) and state probabilities for the strategies
        of~\cite{Stewart2012}, ordered both by number of wins and overall score.
        Note that \(P(DC)\) is not shown as it corresponds to the transpose of
        \(P(CD)\). Cooperator and Defector are omitted as they do not visit all
        the states.}
    \label{fig:SSError_overall_in_stewart_plotkin}
\end{figure}

Here, the work of~\cite{Stewart2012} is extended by investigating a tournament
with \input{assets/tex/number_of_full_strategies/main.tex}
strategies.

The results of this analysis are shown in
Figure~\ref{fig:SSError_and_probabilities_in_full}. The top ranking strategies
by number of wins seem to be extortionate (but not against all strategies) and
it can be seen that a small sub group of strategies achieve mutual defection.
All the top ranking strategies according to score achieve mutual cooperation and
do not extort each other, however they
\textbf{do} exhibit extortionate behaviour towards a number of the lower ranking
strategies.

\begin{figure}[!htbp]
    \centering
    \includegraphics[width=.8\textwidth]{./assets/img/SSError_and_probabilities_in_full/main.pdf}
    \caption{\(\text{SSError}\) for the strategies for the full tournament. Only
    strategy interactions for which \(p_4=0\) and \(\chi>1\) are displayed.}
    \label{fig:SSError_and_probabilities_in_full}
\end{figure}

\section{Conclusion}\label{sec:conclusion}

This work defines an approach to measure whether or not a player is playing a
strategy that corresponds to an extortionate strategy as defined
in~\cite{Press2012}: a mathematical model for suspicion. Indeed, all
extortionate strategies have been
 classified as lying on a triangular plane.
This rigorous classification fails to be robust to small measurement error, thus
a statistical approach is proposed.
This is done through a linear algebraic approach for approximating the solution
of a linear system. Using this, a large number of pairwise interactions is
simulated and in fact very few strategies are found to act extortionately.

The work of~\cite{Press2012}, whilst showing that a clever approach to taking
advantage of another memory one strategy exists: this is incomplete. Whilst the
elegance of this result is very attractive, just as the simplicity of the
victory of Tit For Tat in Axelrod's original tournaments was, it is incomplete.
Extortionate strategies achieve a high number of wins but they do not
achieve a high score which corresponds to the fitness landscape in an
evolutionary sense. From the large number of interactions a payoff matrix \(S\)
can be measured where \(S_{ij}\) denotes the score (using standard values of
\((R, S, T, P) = (3, 0, 5, 1)\)) of the \(i\)th strategy
against the \(j\)th strategy. Using this, the replicator equation
describes the evolution of the system based on a population density fitness
function:

\begin{equation}\label{eqn:replicator_dynamics}
    \frac{dx}{dt} = x(S-x^TS x)
\end{equation}

Equation (\ref{eqn:replicator_dynamics}) is solved numerically through an
integration technique described in~\cite{Petzold1983} and
Figure~\ref{fig:replicator_dynamics} shows the evolution of the distribution of
the system: the various strategies are ranked by scores. It is clear to see that
only the high ranking strategies survive the evolutionary process (in fact,
only \input{./assets/img/replicator_dynamics/main.tex}
have a final distribution greater than \(10 ^ {-2}\)). This confirms the
findings of~\cite{Moran1707} in which sophisticated strategies resist
evolutionary invasion of shorter memory strategies. Recalling
Figure~\ref{fig:SSError_and_probabilities_in_full} this demonstrates that:

\begin{itemize}
    \item Cooperation emerges through the evolutionary process: the high scoring
        strategies do not exhibit extortionate behaviour towards each other.
    \item Extortionate strategies do not survive the evolutionary process.
\end{itemize}

\begin{figure}[!htbp]
    \centering
    \includegraphics[width=.8\textwidth]{./assets/img/replicator_dynamics/main.pdf}
    \caption{Numerical simulation of the replicator equation
    (\ref{eqn:replicator_dynamics}): strategies are ordered by score, only the strategies with a high score survive the evolutionary process.}
    \label{fig:replicator_dynamics}
\end{figure}

This work can be used to classify plays of the IPD\@: data can be collected from
actual interactions (in lab or in the field). Furthermore, this allows for a
classification method similar to the notion of fingerprinting presented
in~\cite{Ashlock2008}. Trained strategies can potentially be classified as
extortionate or not or it could be possible to even constrain the reinforcement
learning approaches that are becoming prevalent in the literature.
Alternatively, this mathematical approach for recognising extortion could be
used in sophisticated strategies to defend against invasion. Arguably, some of
the strategies considered here exhibit this behaviour, indeed as described
in~\cite{Harper2017}, the top ranking strategies in the full tournament are
obtained using evolutionary reinforcement learning techniques, thus, suspicion
of extortionate behaviour could in fact be an evolutionary trait.

\section*{Acknowledgements}

The following open source software libraries were used in this research:

\begin{itemize}
    \item The Axelrod ~\cite{Knight2016, Knight2018} library (IPD strategies and
        tournaments).
    \item The sympy library~\cite{Meurer2017} (verification of all symbolic
        calculations).
    \item The matplotlib~\cite{Droettboom2018} library (visualisation).
    \item The pandas~\cite{Structures2010}, dask~\cite{Dask2016} and
        NumPy~\cite{Oliphant2015} libraries (data manipulation).
    \item The SciPy~\cite{Jones2001} library (numerical integration of the
        replicator equation).
\end{itemize}

This work was performed using the computational facilities of the Advanced
Research Computing @ Cardiff (ARCCA) Division, Cardiff University.

\printbibliography

\newpage
\section*{Supplementary materials}

\includepdf{assets/pdf/proof_of_form_of_extortionate_strategies/main.pdf}

\newpage

Using the pair wise interactions the transition rates \(p,
q\) can be measured and the steady state probabilities inferred and compared to
the actual probabilities of each state.
This is done numerically by computing the singular eigenvector of the
matrix \(A\) \cite{Stewart2009}:

\[
    A =
    \begin{bmatrix}
        p_1 q_1 & p_1 (1 - q_1) & (1 - p_1) q_1 & (1 -p_1) (1 - q_1) \\
        p_2 q_2 & p_2 (1 - q_2) & (1 - p_2) q_2 & (1 -p_2) (1 - q_2) \\
        p_3 q_3 & p_3 (1 - q_3) & (1 - p_3) q_3 & (1 -p_3) (1 - q_3) \\
        p_4 q_4 & p_4 (1 - q_4) & (1 - p_4) q_4 & (1 -p_4) (1 - q_4) \\
    \end{bmatrix}
\]

Figure~\ref{fig:computed_probabilities_vs_theoretic_probabilities} shows a
regression line fitted to every pairwise interaction with a reported
\(\text{SSError}\) value (pairwise interactions with missing states were
omitted). This serves to validate the approach: a part from some edge cases the
relationship is consistent.

\begin{figure}[!htbp]
    \centering
    \includegraphics[width=.8\textwidth]{./assets/img/computed_probabilities_vs_theoretic_probabilities/main.pdf}
    \caption{The
        relationship between the steady state probabilities inferred from the
        measured transitions and the actual steady state probabilities. A linear
        regression line is included validating the approach.}
    \label{fig:computed_probabilities_vs_theoretic_probabilities}
\end{figure}


\end{document}

strategies.

The results of this analysis are shown in
Figure~\ref{fig:SSError_and_probabilities_in_full}. The top ranking strategies
by number of wins seem to be extortionate (but not against all strategies) and
it can be seen that a small sub group of strategies achieve mutual defection.
All the top ranking strategies according to score achieve mutual cooperation and
do not extort each other, however they
\textbf{do} exhibit extortionate behaviour towards a number of the lower ranking
strategies.

\begin{figure}[!htbp]
    \centering
    \includegraphics[width=.8\textwidth]{./assets/img/SSError_and_probabilities_in_full/main.pdf}
    \caption{\(\text{SSError}\) for the strategies for the full tournament. Only
    strategy interactions for which \(p_4=0\) and \(\chi>1\) are displayed.}
    \label{fig:SSError_and_probabilities_in_full}
\end{figure}

\section{Conclusion}\label{sec:conclusion}

This work defines an approach to measure whether or not a player is playing a
strategy that corresponds to an extortionate strategy as defined
in~\cite{Press2012}: a mathematical model for suspicion. Indeed, all
extortionate strategies have been
 classified as lying on a triangular plane.
This rigorous classification fails to be robust to small measurement error, thus
a statistical approach is proposed.
This is done through a linear algebraic approach for approximating the solution
of a linear system. Using this, a large number of pairwise interactions is
simulated and in fact very few strategies are found to act extortionately.

The work of~\cite{Press2012}, whilst showing that a clever approach to taking
advantage of another memory one strategy exists: this is incomplete. Whilst the
elegance of this result is very attractive, just as the simplicity of the
victory of Tit For Tat in Axelrod's original tournaments was, it is incomplete.
Extortionate strategies achieve a high number of wins but they do not
achieve a high score which corresponds to the fitness landscape in an
evolutionary sense. From the large number of interactions a payoff matrix \(S\)
can be measured where \(S_{ij}\) denotes the score (using standard values of
\((R, S, T, P) = (3, 0, 5, 1)\)) of the \(i\)th strategy
against the \(j\)th strategy. Using this, the replicator equation
describes the evolution of the system based on a population density fitness
function:

\begin{equation}\label{eqn:replicator_dynamics}
    \frac{dx}{dt} = x(S-x^TS x)
\end{equation}

Equation (\ref{eqn:replicator_dynamics}) is solved numerically through an
integration technique described in~\cite{Petzold1983} and
Figure~\ref{fig:replicator_dynamics} shows the evolution of the distribution of
the system: the various strategies are ranked by scores. It is clear to see that
only the high ranking strategies survive the evolutionary process (in fact,
only \documentclass[a4paper]{article}

\usepackage{amsmath}
\usepackage{amssymb}
\usepackage[margin=1.5cm,
            includefoot,
            footskip=30pt]{geometry}
\usepackage{layout}
\usepackage{graphicx}
\usepackage{subcaption}

\usepackage{biblatex}
\usepackage{pdfpages}

\bibliography{main.bib}

\title{Suspicion: Recognising and evaluating the effectiveness
       of extortion in the Iterated Prisoner's Dilemma}
\author{Vincent A. Knight \and Nikoleta E. Glynatsi}
\date{\today}



\begin{document}

\maketitle

\begin{abstract}
    The Iterated Prisoner's Dilemma is a model for rational and evolutionary
    interactive behaviour. It has applications both in the study of human social
    behaviour as well as in biology.
    It is used to understand when and how a rational individual might
    accept an immediate cost to their own utility for the direct benefit of
    another.

    Much attention has been given to a class of strategies called
    Zero Determinant strategies. It has been theoretically shown that these
    strategies can ``extort'' any player.

    In this work, an approach to identify if observed strategies are playing in
    an extortionate way is described. Furthermore, experimental analysis of
    a large tournament with \input{assets/tex/number_of_full_strategies/main.tex}
    strategies is considered. In this setting
    the most highly performing strategies do not play in an extortionate way
    against each other but do against lower performing strategies.
    This suggests that whilst the theory of Zero Determinant strategies
    indicates that memory is not of fundamental importance to the evolution of
    cooperative behaviour, this is incomplete.
\end{abstract}

\section{Introduction}\label{sec:introduction}

Agent based game theoretic models have become a stalwart of the underpinning
mathematics of interactive behaviours. One of the major pieces of work
in this area is the pair of original computer tournaments run by Robert
Axelrod~\cite{Axelrod1980, Axelrod1980a}. These tournaments pitted submitted
computer strategies against each other in plays of the Iterated Prisoner's
Dilemma. A common game where agents can choose to pay a slight cost to their
immediate utility in the hope of building a reputation. This has been used in
economic and evolutionary game theory to understand the evolution of cooperative
behaviour.

Recently, a class of strategies was described in~\cite{Press2012} that can
provably extort any given opponent. In~\cite{Hilbe2013, Moran1707} some
questions have already been asked about the true effectiveness of these
strategies in an evolutionary setting. Here another question is asked: is it
possible to recognise this extortionate behaviour? A mathematical procedure for
suspicion is presented: in the same way that the continued actions of an
extortionate individual might raise suspicion.

This work makes use of the Axelrod Python library~\cite{Knight2018, Knight2016}
with a large number of Prisoner Dilemma strategies available to give an
extensive numerical example of the ideas presented.  The approach is presented
in Section~\ref{sec:delta-zd-strategies}.  All of the code and data discussed
in Section~\ref{sec:numerical-experiments} is open sourced, archived and
written according to best scientific principles~\cite{Wilson2014}. The data
archive can be found at~\cite{vincent_knight_2018_1297075}.

\section{Recognising Extortion}\label{sec:delta-zd-strategies}

In~\cite{Press2012}, given a match between 2 memory-one strategies, the concept
of Zero Determinant (ZD) strategies is introduced. The main result of that paper
shows that given two memory one players \(p, q\in\mathbb{R}^4\) a linear
relationship between the players' scores could be forced by one of the players.

Using the notation of~\cite{Press2012}, assuming the utilities for player \(p\)
are given by \(S_x=(R, S, T, P)\) and for player \(q\) by \(S_y=(R, T, S, P)\)
and that the stationary scores of each player is given by \(S_X\) and \(S_Y\)
respectively. The main result of~\cite{Press2012} is that if

\begin{equation}\label{eqn:linear_relationship_for_p}
    \tilde p=\alpha S_x + \beta S_y + \gamma
\end{equation}

or

\begin{equation}\label{eqn:linear_relationship_for_q}
    \tilde q=\alpha S_x + \beta S_y + \gamma
\end{equation}

where \(\tilde p = (1 - p_1, 1 - p_2, p_3, p_4)\) and
\(\tilde q = (1 - q_1, 1 - q_2, q_3, q_4)\) then:

\begin{equation}
    \alpha S_X + \beta S_Y + \gamma = 0
\end{equation}

In~\cite{Press2012} a particular type of ZD strategy is defined: extortionate
strategies. If:

\begin{equation}\label{eqn:constraint_for_extortion}
    \gamma = - P(\alpha + \beta)
\end{equation}

then the player can ensure they get a score \(\chi\) times
larger than the opponent. This extortion coefficient is given by:

\begin{equation}\label{eqn:definition_of_chi}
    \chi=\frac{-\beta}{\alpha}
\end{equation}

Thus, if (\ref{eqn:constraint_for_extortion}) holds and \(\chi >1\) a player is
said to extort their opponent.
Here, the reverse problem is considered: given a
\(p\in\mathbb{R}^4\) how does one identify \(\alpha, \beta\) if they
exist and is the strategy in fact acting in an extortionate way?

These conditions correspond to:

\begin{align}
    \tilde p_1 & = \alpha R + \beta R - P (\alpha + \beta)
            \label{eqn:condition_for_tilde_p1}\\
    \tilde p_2 & = \alpha S + \beta T - P (\alpha + \beta)
            \label{eqn:condition_for_tilde_p2}\\
    \tilde p_3 & = \alpha T + \beta S - P (\alpha + \beta)
            \label{eqn:condition_for_tilde_p3}\\
    \tilde p_4 & = \alpha P + \beta P - P (\alpha + \beta)
            \label{eqn:condition_for_tilde_p4}
\end{align}

Equation (\ref{eqn:condition_for_tilde_p4}) ensures that \(p_4=\tilde p_4=0\).
Equations (\ref{eqn:condition_for_tilde_p1}-\ref{eqn:condition_for_tilde_p3})
can be used to eliminate \(\alpha, \beta\), giving:

\begin{equation}\label{eqn:planar_definition_of_extortion}
    \tilde p_1 = \frac{(R - P)(\tilde p_2 + \tilde p_3)}{S + T - 2P}
\end{equation}

with:

\begin{equation}\label{eqn:definition_of_chi}
    \chi = \frac{\tilde p_2 (P - T) + \tilde p_3 (S - P)}
                {\tilde p_2 (P - S) + \tilde p_3 (T - P)}
\end{equation}

Given a strategy \(p\in\mathbb{R}^{4\times 1}\) equations
(\ref{eqn:condition_for_tilde_p4}), (\ref{eqn:planar_definition_of_extortion}-\ref{eqn:definition_of_chi}) can be used to check if
a strategy is extortionate. The conditions correspond to:

\begin{align}
    p_1 & = \frac{(R-P)(p_2 + p_3) - R + T + S - P}{S + T - 2P}
     \label{eqn:condition_for_p1}\\
    p_4 & = 0 \label{eqn:condition_for_p4}\\
    1 & > p_2 + p_3\label{eqn:condition_for_chi}
\end{align}

The algebraic steps necessary to prove these results are available in the
supporting materials.

All extortionate strategies reside on a triangular (\ref{eqn:condition_for_chi})
plane (\ref{eqn:condition_for_p1}) in 3 dimensions (\ref{eqn:condition_for_p4}).
Using this formulation it can be seen that a necessary (but not sufficient)
condition for an extortionate strategy is that it cooperates on average less
than 50\% of the time when in a state of disagreement with the opponent.

As an example, consider the known extortionate strategy \(p=(8 / 9, 1 / 2, 1 /
3, 0)\) from~\cite{Stewart2012} which is referred to as \texttt{Extort-2}. In
this case, for the standard values of \((R, T, S, P)\) constraint
(\ref{eqn:condition_for_p1}) corresponds to:

\begin{equation}
    p_1 = \frac{2(p_2 + p_3) + 1}{3}
\end{equation}

It is clear that in this case all constraints hold.

This approach could in fact be used to confirm that a given strategy is acting
in an extortionate manner even if it is not a memory one strategy. However, in
practice, if a closed form for \(p\) is not known, then due to measurement
and/or numerical error this would not work.

This problem can be written in the following linear algebraic form where
\(x=(\alpha, \beta)\)
and \(p^*=(\tilde p_1 - 1, tilde_2 - 1, p_3)\):

\begin{equation}\label{eqn:linear_algebraic_equation_for_p}
    Cx= p^*
\end{equation}

\(C\) corresponds to equations
(\ref{eqn:condition_for_tilde_p1}-\ref{eqn:condition_for_tilde_p3}) and is
given by:

\begin{equation}\label{eqn:definition_of_C}
    C =
    \begin{bmatrix}
        R - P & R- P \\
        S - P & T- P \\
        T - P & S- P \\
    \end{bmatrix}
\end{equation}

Note that in general, equation (\ref{eqn:linear_algebraic_equation_for_p}) will
not necessarily have a solution. From the Rouch\'{e}-Capelli theorem if there is
a solution it is unique as \(\text{rank}(C)=2\) which is the dimension of the
variable \(x\). The best fitting \(x\) is found by minimizing:

\begin{equation}\label{eqn:r_squared}
    \text{SSError} = \|C x- p^*\|_2^2 = \sum_{i=1}^{3}\left((C\bar x)_i-p_i^*\right)^2
\end{equation}

Note that \(\text{SSError}\), which is the square of the Frobenius
norm~\cite{Golub2013}, becomes a measure of how close a strategy is to being an
extortionate strategy. Suspicion
of extortion then corresponds to a threshold on \(\text{SSError}\).

By observing interactions (human or otherwise), their memory one representation
can be inferred and this approach can be used to recognise extortionate
behaviour. The notion of comparing theoretic and actual plays of the IPD is not
novel, see for example~\cite{Rand2013}. Immediately it is noted that if the
environment is noisy~\cite{Wu1995} then no strategy can be considered to be
extortionate as \(p_4>0\).

In the next section, this idea will be illustrated by observing the interactions
that take place in a computer based tournament of the IPD\@.

\section{Numerical experiments}\label{sec:numerical-experiments}

In~\cite{Stewart2012} results from a tournament with
\input{./assets/tex/number_of_stewart_plotkin_strategies/main.tex} strategies,
was presented with specific consideration given to ZD strategies. This
tournament is reproduced here using the Axelrod-Python
project~\cite{Knight2016}. To obtain a good measure of the corresponding
transition rates for each strategy all matches have been run for
\input{assets/tex/number_of_turns/main.tex} turns and every match has been
repeated \input{assets/tex/number_of_repetitions/main.tex} times. All of this
interaction data is available at~\cite{vincent_knight_2018_1297075}. A good
match between the inferred Markov chain and the state distribution of the actual
interactions has been verified. Data for this is presented in the supplementary
materials.

Figure~\ref{fig:SSError_overall_in_stewart_plotkin} shows the \(\text{SSError}\)
values for all the strategies in the tournament, as reported
in~\cite{Stewart2012} the extortionate strategy (which has an expected
\(\text{SSError}\) approximately 0) gains a large number of wins.

\begin{figure}[!htbp]
    \centering
    \includegraphics[width=.8\textwidth]{./assets/img/SSError_overall_in_stewart_plotkin/main.pdf}
    \caption{\(\text{SSError}\) and state probabilities for the strategies
        of~\cite{Stewart2012}, ordered both by number of wins and overall score.
        Note that \(P(DC)\) is not shown as it corresponds to the transpose of
        \(P(CD)\). Cooperator and Defector are omitted as they do not visit all
        the states.}
    \label{fig:SSError_overall_in_stewart_plotkin}
\end{figure}

Here, the work of~\cite{Stewart2012} is extended by investigating a tournament
with \input{assets/tex/number_of_full_strategies/main.tex}
strategies.

The results of this analysis are shown in
Figure~\ref{fig:SSError_and_probabilities_in_full}. The top ranking strategies
by number of wins seem to be extortionate (but not against all strategies) and
it can be seen that a small sub group of strategies achieve mutual defection.
All the top ranking strategies according to score achieve mutual cooperation and
do not extort each other, however they
\textbf{do} exhibit extortionate behaviour towards a number of the lower ranking
strategies.

\begin{figure}[!htbp]
    \centering
    \includegraphics[width=.8\textwidth]{./assets/img/SSError_and_probabilities_in_full/main.pdf}
    \caption{\(\text{SSError}\) for the strategies for the full tournament. Only
    strategy interactions for which \(p_4=0\) and \(\chi>1\) are displayed.}
    \label{fig:SSError_and_probabilities_in_full}
\end{figure}

\section{Conclusion}\label{sec:conclusion}

This work defines an approach to measure whether or not a player is playing a
strategy that corresponds to an extortionate strategy as defined
in~\cite{Press2012}: a mathematical model for suspicion. Indeed, all
extortionate strategies have been
 classified as lying on a triangular plane.
This rigorous classification fails to be robust to small measurement error, thus
a statistical approach is proposed.
This is done through a linear algebraic approach for approximating the solution
of a linear system. Using this, a large number of pairwise interactions is
simulated and in fact very few strategies are found to act extortionately.

The work of~\cite{Press2012}, whilst showing that a clever approach to taking
advantage of another memory one strategy exists: this is incomplete. Whilst the
elegance of this result is very attractive, just as the simplicity of the
victory of Tit For Tat in Axelrod's original tournaments was, it is incomplete.
Extortionate strategies achieve a high number of wins but they do not
achieve a high score which corresponds to the fitness landscape in an
evolutionary sense. From the large number of interactions a payoff matrix \(S\)
can be measured where \(S_{ij}\) denotes the score (using standard values of
\((R, S, T, P) = (3, 0, 5, 1)\)) of the \(i\)th strategy
against the \(j\)th strategy. Using this, the replicator equation
describes the evolution of the system based on a population density fitness
function:

\begin{equation}\label{eqn:replicator_dynamics}
    \frac{dx}{dt} = x(S-x^TS x)
\end{equation}

Equation (\ref{eqn:replicator_dynamics}) is solved numerically through an
integration technique described in~\cite{Petzold1983} and
Figure~\ref{fig:replicator_dynamics} shows the evolution of the distribution of
the system: the various strategies are ranked by scores. It is clear to see that
only the high ranking strategies survive the evolutionary process (in fact,
only \input{./assets/img/replicator_dynamics/main.tex}
have a final distribution greater than \(10 ^ {-2}\)). This confirms the
findings of~\cite{Moran1707} in which sophisticated strategies resist
evolutionary invasion of shorter memory strategies. Recalling
Figure~\ref{fig:SSError_and_probabilities_in_full} this demonstrates that:

\begin{itemize}
    \item Cooperation emerges through the evolutionary process: the high scoring
        strategies do not exhibit extortionate behaviour towards each other.
    \item Extortionate strategies do not survive the evolutionary process.
\end{itemize}

\begin{figure}[!htbp]
    \centering
    \includegraphics[width=.8\textwidth]{./assets/img/replicator_dynamics/main.pdf}
    \caption{Numerical simulation of the replicator equation
    (\ref{eqn:replicator_dynamics}): strategies are ordered by score, only the strategies with a high score survive the evolutionary process.}
    \label{fig:replicator_dynamics}
\end{figure}

This work can be used to classify plays of the IPD\@: data can be collected from
actual interactions (in lab or in the field). Furthermore, this allows for a
classification method similar to the notion of fingerprinting presented
in~\cite{Ashlock2008}. Trained strategies can potentially be classified as
extortionate or not or it could be possible to even constrain the reinforcement
learning approaches that are becoming prevalent in the literature.
Alternatively, this mathematical approach for recognising extortion could be
used in sophisticated strategies to defend against invasion. Arguably, some of
the strategies considered here exhibit this behaviour, indeed as described
in~\cite{Harper2017}, the top ranking strategies in the full tournament are
obtained using evolutionary reinforcement learning techniques, thus, suspicion
of extortionate behaviour could in fact be an evolutionary trait.

\section*{Acknowledgements}

The following open source software libraries were used in this research:

\begin{itemize}
    \item The Axelrod ~\cite{Knight2016, Knight2018} library (IPD strategies and
        tournaments).
    \item The sympy library~\cite{Meurer2017} (verification of all symbolic
        calculations).
    \item The matplotlib~\cite{Droettboom2018} library (visualisation).
    \item The pandas~\cite{Structures2010}, dask~\cite{Dask2016} and
        NumPy~\cite{Oliphant2015} libraries (data manipulation).
    \item The SciPy~\cite{Jones2001} library (numerical integration of the
        replicator equation).
\end{itemize}

This work was performed using the computational facilities of the Advanced
Research Computing @ Cardiff (ARCCA) Division, Cardiff University.

\printbibliography

\newpage
\section*{Supplementary materials}

\includepdf{assets/pdf/proof_of_form_of_extortionate_strategies/main.pdf}

\newpage

Using the pair wise interactions the transition rates \(p,
q\) can be measured and the steady state probabilities inferred and compared to
the actual probabilities of each state.
This is done numerically by computing the singular eigenvector of the
matrix \(A\) \cite{Stewart2009}:

\[
    A =
    \begin{bmatrix}
        p_1 q_1 & p_1 (1 - q_1) & (1 - p_1) q_1 & (1 -p_1) (1 - q_1) \\
        p_2 q_2 & p_2 (1 - q_2) & (1 - p_2) q_2 & (1 -p_2) (1 - q_2) \\
        p_3 q_3 & p_3 (1 - q_3) & (1 - p_3) q_3 & (1 -p_3) (1 - q_3) \\
        p_4 q_4 & p_4 (1 - q_4) & (1 - p_4) q_4 & (1 -p_4) (1 - q_4) \\
    \end{bmatrix}
\]

Figure~\ref{fig:computed_probabilities_vs_theoretic_probabilities} shows a
regression line fitted to every pairwise interaction with a reported
\(\text{SSError}\) value (pairwise interactions with missing states were
omitted). This serves to validate the approach: a part from some edge cases the
relationship is consistent.

\begin{figure}[!htbp]
    \centering
    \includegraphics[width=.8\textwidth]{./assets/img/computed_probabilities_vs_theoretic_probabilities/main.pdf}
    \caption{The
        relationship between the steady state probabilities inferred from the
        measured transitions and the actual steady state probabilities. A linear
        regression line is included validating the approach.}
    \label{fig:computed_probabilities_vs_theoretic_probabilities}
\end{figure}


\end{document}

have a final distribution greater than \(10 ^ {-2}\)). This confirms the
findings of~\cite{Moran1707} in which sophisticated strategies resist
evolutionary invasion of shorter memory strategies. Recalling
Figure~\ref{fig:SSError_and_probabilities_in_full} this demonstrates that:

\begin{itemize}
    \item Cooperation emerges through the evolutionary process: the high scoring
        strategies do not exhibit extortionate behaviour towards each other.
    \item Extortionate strategies do not survive the evolutionary process.
\end{itemize}

\begin{figure}[!htbp]
    \centering
    \includegraphics[width=.8\textwidth]{./assets/img/replicator_dynamics/main.pdf}
    \caption{Numerical simulation of the replicator equation
    (\ref{eqn:replicator_dynamics}): strategies are ordered by score, only the strategies with a high score survive the evolutionary process.}
    \label{fig:replicator_dynamics}
\end{figure}

This work can be used to classify plays of the IPD\@: data can be collected from
actual interactions (in lab or in the field). Furthermore, this allows for a
classification method similar to the notion of fingerprinting presented
in~\cite{Ashlock2008}. Trained strategies can potentially be classified as
extortionate or not or it could be possible to even constrain the reinforcement
learning approaches that are becoming prevalent in the literature.
Alternatively, this mathematical approach for recognising extortion could be
used in sophisticated strategies to defend against invasion. Arguably, some of
the strategies considered here exhibit this behaviour, indeed as described
in~\cite{Harper2017}, the top ranking strategies in the full tournament are
obtained using evolutionary reinforcement learning techniques, thus, suspicion
of extortionate behaviour could in fact be an evolutionary trait.

\section*{Acknowledgements}

The following open source software libraries were used in this research:

\begin{itemize}
    \item The Axelrod ~\cite{Knight2016, Knight2018} library (IPD strategies and
        tournaments).
    \item The sympy library~\cite{Meurer2017} (verification of all symbolic
        calculations).
    \item The matplotlib~\cite{Droettboom2018} library (visualisation).
    \item The pandas~\cite{Structures2010}, dask~\cite{Dask2016} and
        NumPy~\cite{Oliphant2015} libraries (data manipulation).
    \item The SciPy~\cite{Jones2001} library (numerical integration of the
        replicator equation).
\end{itemize}

This work was performed using the computational facilities of the Advanced
Research Computing @ Cardiff (ARCCA) Division, Cardiff University.

\printbibliography

\newpage
\section*{Supplementary materials}

\includepdf{assets/pdf/proof_of_form_of_extortionate_strategies/main.pdf}

\newpage

Using the pair wise interactions the transition rates \(p,
q\) can be measured and the steady state probabilities inferred and compared to
the actual probabilities of each state.
This is done numerically by computing the singular eigenvector of the
matrix \(A\) \cite{Stewart2009}:

\[
    A =
    \begin{bmatrix}
        p_1 q_1 & p_1 (1 - q_1) & (1 - p_1) q_1 & (1 -p_1) (1 - q_1) \\
        p_2 q_2 & p_2 (1 - q_2) & (1 - p_2) q_2 & (1 -p_2) (1 - q_2) \\
        p_3 q_3 & p_3 (1 - q_3) & (1 - p_3) q_3 & (1 -p_3) (1 - q_3) \\
        p_4 q_4 & p_4 (1 - q_4) & (1 - p_4) q_4 & (1 -p_4) (1 - q_4) \\
    \end{bmatrix}
\]

Figure~\ref{fig:computed_probabilities_vs_theoretic_probabilities} shows a
regression line fitted to every pairwise interaction with a reported
\(\text{SSError}\) value (pairwise interactions with missing states were
omitted). This serves to validate the approach: a part from some edge cases the
relationship is consistent.

\begin{figure}[!htbp]
    \centering
    \includegraphics[width=.8\textwidth]{./assets/img/computed_probabilities_vs_theoretic_probabilities/main.pdf}
    \caption{The
        relationship between the steady state probabilities inferred from the
        measured transitions and the actual steady state probabilities. A linear
        regression line is included validating the approach.}
    \label{fig:computed_probabilities_vs_theoretic_probabilities}
\end{figure}


\end{document}

    strategies is considered. In this setting
    the most highly performing strategies do not play in an extortionate way
    against each other but do against lower performing strategies.
    This suggests that whilst the theory of Zero Determinant strategies
    indicates that memory is not of fundamental importance to the evolution of
    cooperative behaviour, this is incomplete.
\end{abstract}

\section{Introduction}\label{sec:introduction}

Agent based game theoretic models have become a stalwart of the underpinning
mathematics of interactive behaviours. One of the major pieces of work
in this area is the pair of original computer tournaments run by Robert
Axelrod~\cite{Axelrod1980, Axelrod1980a}. These tournaments pitted submitted
computer strategies against each other in plays of the Iterated Prisoner's
Dilemma. A common game where agents can choose to pay a slight cost to their
immediate utility in the hope of building a reputation. This has been used in
economic and evolutionary game theory to understand the evolution of cooperative
behaviour.

Recently, a class of strategies was described in~\cite{Press2012} that can
provably extort any given opponent. In~\cite{Hilbe2013, Moran1707} some
questions have already been asked about the true effectiveness of these
strategies in an evolutionary setting. Here another question is asked: is it
possible to recognise this extortionate behaviour? A mathematical procedure for
suspicion is presented: in the same way that the continued actions of an
extortionate individual might raise suspicion.

This work makes use of the Axelrod Python library~\cite{Knight2018, Knight2016}
with a large number of Prisoner Dilemma strategies available to give an
extensive numerical example of the ideas presented.  The approach is presented
in Section~\ref{sec:delta-zd-strategies}.  All of the code and data discussed
in Section~\ref{sec:numerical-experiments} is open sourced, archived and
written according to best scientific principles~\cite{Wilson2014}. The data
archive can be found at~\cite{vincent_knight_2018_1297075}.

\section{Recognising Extortion}\label{sec:delta-zd-strategies}

In~\cite{Press2012}, given a match between 2 memory-one strategies, the concept
of Zero Determinant (ZD) strategies is introduced. The main result of that paper
shows that given two memory one players \(p, q\in\mathbb{R}^4\) a linear
relationship between the players' scores could be forced by one of the players.

Using the notation of~\cite{Press2012}, assuming the utilities for player \(p\)
are given by \(S_x=(R, S, T, P)\) and for player \(q\) by \(S_y=(R, T, S, P)\)
and that the stationary scores of each player is given by \(S_X\) and \(S_Y\)
respectively. The main result of~\cite{Press2012} is that if

\begin{equation}\label{eqn:linear_relationship_for_p}
    \tilde p=\alpha S_x + \beta S_y + \gamma
\end{equation}

or

\begin{equation}\label{eqn:linear_relationship_for_q}
    \tilde q=\alpha S_x + \beta S_y + \gamma
\end{equation}

where \(\tilde p = (1 - p_1, 1 - p_2, p_3, p_4)\) and
\(\tilde q = (1 - q_1, 1 - q_2, q_3, q_4)\) then:

\begin{equation}
    \alpha S_X + \beta S_Y + \gamma = 0
\end{equation}

In~\cite{Press2012} a particular type of ZD strategy is defined: extortionate
strategies. If:

\begin{equation}\label{eqn:constraint_for_extortion}
    \gamma = - P(\alpha + \beta)
\end{equation}

then the player can ensure they get a score \(\chi\) times
larger than the opponent. This extortion coefficient is given by:

\begin{equation}\label{eqn:definition_of_chi}
    \chi=\frac{-\beta}{\alpha}
\end{equation}

Thus, if (\ref{eqn:constraint_for_extortion}) holds and \(\chi >1\) a player is
said to extort their opponent.
Here, the reverse problem is considered: given a
\(p\in\mathbb{R}^4\) how does one identify \(\alpha, \beta\) if they
exist and is the strategy in fact acting in an extortionate way?

These conditions correspond to:

\begin{align}
    \tilde p_1 & = \alpha R + \beta R - P (\alpha + \beta)
            \label{eqn:condition_for_tilde_p1}\\
    \tilde p_2 & = \alpha S + \beta T - P (\alpha + \beta)
            \label{eqn:condition_for_tilde_p2}\\
    \tilde p_3 & = \alpha T + \beta S - P (\alpha + \beta)
            \label{eqn:condition_for_tilde_p3}\\
    \tilde p_4 & = \alpha P + \beta P - P (\alpha + \beta)
            \label{eqn:condition_for_tilde_p4}
\end{align}

Equation (\ref{eqn:condition_for_tilde_p4}) ensures that \(p_4=\tilde p_4=0\).
Equations (\ref{eqn:condition_for_tilde_p1}-\ref{eqn:condition_for_tilde_p3})
can be used to eliminate \(\alpha, \beta\), giving:

\begin{equation}\label{eqn:planar_definition_of_extortion}
    \tilde p_1 = \frac{(R - P)(\tilde p_2 + \tilde p_3)}{S + T - 2P}
\end{equation}

with:

\begin{equation}\label{eqn:definition_of_chi}
    \chi = \frac{\tilde p_2 (P - T) + \tilde p_3 (S - P)}
                {\tilde p_2 (P - S) + \tilde p_3 (T - P)}
\end{equation}

Given a strategy \(p\in\mathbb{R}^{4\times 1}\) equations
(\ref{eqn:condition_for_tilde_p4}), (\ref{eqn:planar_definition_of_extortion}-\ref{eqn:definition_of_chi}) can be used to check if
a strategy is extortionate. The conditions correspond to:

\begin{align}
    p_1 & = \frac{(R-P)(p_2 + p_3) - R + T + S - P}{S + T - 2P}
     \label{eqn:condition_for_p1}\\
    p_4 & = 0 \label{eqn:condition_for_p4}\\
    1 & > p_2 + p_3\label{eqn:condition_for_chi}
\end{align}

The algebraic steps necessary to prove these results are available in the
supporting materials.

All extortionate strategies reside on a triangular (\ref{eqn:condition_for_chi})
plane (\ref{eqn:condition_for_p1}) in 3 dimensions (\ref{eqn:condition_for_p4}).
Using this formulation it can be seen that a necessary (but not sufficient)
condition for an extortionate strategy is that it cooperates on average less
than 50\% of the time when in a state of disagreement with the opponent.

As an example, consider the known extortionate strategy \(p=(8 / 9, 1 / 2, 1 /
3, 0)\) from~\cite{Stewart2012} which is referred to as \texttt{Extort-2}. In
this case, for the standard values of \((R, T, S, P)\) constraint
(\ref{eqn:condition_for_p1}) corresponds to:

\begin{equation}
    p_1 = \frac{2(p_2 + p_3) + 1}{3}
\end{equation}

It is clear that in this case all constraints hold.

This approach could in fact be used to confirm that a given strategy is acting
in an extortionate manner even if it is not a memory one strategy. However, in
practice, if a closed form for \(p\) is not known, then due to measurement
and/or numerical error this would not work.

This problem can be written in the following linear algebraic form where
\(x=(\alpha, \beta)\)
and \(p^*=(\tilde p_1 - 1, tilde_2 - 1, p_3)\):

\begin{equation}\label{eqn:linear_algebraic_equation_for_p}
    Cx= p^*
\end{equation}

\(C\) corresponds to equations
(\ref{eqn:condition_for_tilde_p1}-\ref{eqn:condition_for_tilde_p3}) and is
given by:

\begin{equation}\label{eqn:definition_of_C}
    C =
    \begin{bmatrix}
        R - P & R- P \\
        S - P & T- P \\
        T - P & S- P \\
    \end{bmatrix}
\end{equation}

Note that in general, equation (\ref{eqn:linear_algebraic_equation_for_p}) will
not necessarily have a solution. From the Rouch\'{e}-Capelli theorem if there is
a solution it is unique as \(\text{rank}(C)=2\) which is the dimension of the
variable \(x\). The best fitting \(x\) is found by minimizing:

\begin{equation}\label{eqn:r_squared}
    \text{SSError} = \|C x- p^*\|_2^2 = \sum_{i=1}^{3}\left((C\bar x)_i-p_i^*\right)^2
\end{equation}

Note that \(\text{SSError}\), which is the square of the Frobenius
norm~\cite{Golub2013}, becomes a measure of how close a strategy is to being an
extortionate strategy. Suspicion
of extortion then corresponds to a threshold on \(\text{SSError}\).

By observing interactions (human or otherwise), their memory one representation
can be inferred and this approach can be used to recognise extortionate
behaviour. The notion of comparing theoretic and actual plays of the IPD is not
novel, see for example~\cite{Rand2013}. Immediately it is noted that if the
environment is noisy~\cite{Wu1995} then no strategy can be considered to be
extortionate as \(p_4>0\).

In the next section, this idea will be illustrated by observing the interactions
that take place in a computer based tournament of the IPD\@.

\section{Numerical experiments}\label{sec:numerical-experiments}

In~\cite{Stewart2012} results from a tournament with
\documentclass[a4paper]{article}

\usepackage{amsmath}
\usepackage{amssymb}
\usepackage[margin=1.5cm,
            includefoot,
            footskip=30pt]{geometry}
\usepackage{layout}
\usepackage{graphicx}
\usepackage{subcaption}

\usepackage{biblatex}
\usepackage{pdfpages}

\bibliography{main.bib}

\title{Suspicion: Recognising and evaluating the effectiveness
       of extortion in the Iterated Prisoner's Dilemma}
\author{Vincent A. Knight \and Nikoleta E. Glynatsi}
\date{\today}



\begin{document}

\maketitle

\begin{abstract}
    The Iterated Prisoner's Dilemma is a model for rational and evolutionary
    interactive behaviour. It has applications both in the study of human social
    behaviour as well as in biology.
    It is used to understand when and how a rational individual might
    accept an immediate cost to their own utility for the direct benefit of
    another.

    Much attention has been given to a class of strategies called
    Zero Determinant strategies. It has been theoretically shown that these
    strategies can ``extort'' any player.

    In this work, an approach to identify if observed strategies are playing in
    an extortionate way is described. Furthermore, experimental analysis of
    a large tournament with \documentclass[a4paper]{article}

\usepackage{amsmath}
\usepackage{amssymb}
\usepackage[margin=1.5cm,
            includefoot,
            footskip=30pt]{geometry}
\usepackage{layout}
\usepackage{graphicx}
\usepackage{subcaption}

\usepackage{biblatex}
\usepackage{pdfpages}

\bibliography{main.bib}

\title{Suspicion: Recognising and evaluating the effectiveness
       of extortion in the Iterated Prisoner's Dilemma}
\author{Vincent A. Knight \and Nikoleta E. Glynatsi}
\date{\today}



\begin{document}

\maketitle

\begin{abstract}
    The Iterated Prisoner's Dilemma is a model for rational and evolutionary
    interactive behaviour. It has applications both in the study of human social
    behaviour as well as in biology.
    It is used to understand when and how a rational individual might
    accept an immediate cost to their own utility for the direct benefit of
    another.

    Much attention has been given to a class of strategies called
    Zero Determinant strategies. It has been theoretically shown that these
    strategies can ``extort'' any player.

    In this work, an approach to identify if observed strategies are playing in
    an extortionate way is described. Furthermore, experimental analysis of
    a large tournament with \input{assets/tex/number_of_full_strategies/main.tex}
    strategies is considered. In this setting
    the most highly performing strategies do not play in an extortionate way
    against each other but do against lower performing strategies.
    This suggests that whilst the theory of Zero Determinant strategies
    indicates that memory is not of fundamental importance to the evolution of
    cooperative behaviour, this is incomplete.
\end{abstract}

\section{Introduction}\label{sec:introduction}

Agent based game theoretic models have become a stalwart of the underpinning
mathematics of interactive behaviours. One of the major pieces of work
in this area is the pair of original computer tournaments run by Robert
Axelrod~\cite{Axelrod1980, Axelrod1980a}. These tournaments pitted submitted
computer strategies against each other in plays of the Iterated Prisoner's
Dilemma. A common game where agents can choose to pay a slight cost to their
immediate utility in the hope of building a reputation. This has been used in
economic and evolutionary game theory to understand the evolution of cooperative
behaviour.

Recently, a class of strategies was described in~\cite{Press2012} that can
provably extort any given opponent. In~\cite{Hilbe2013, Moran1707} some
questions have already been asked about the true effectiveness of these
strategies in an evolutionary setting. Here another question is asked: is it
possible to recognise this extortionate behaviour? A mathematical procedure for
suspicion is presented: in the same way that the continued actions of an
extortionate individual might raise suspicion.

This work makes use of the Axelrod Python library~\cite{Knight2018, Knight2016}
with a large number of Prisoner Dilemma strategies available to give an
extensive numerical example of the ideas presented.  The approach is presented
in Section~\ref{sec:delta-zd-strategies}.  All of the code and data discussed
in Section~\ref{sec:numerical-experiments} is open sourced, archived and
written according to best scientific principles~\cite{Wilson2014}. The data
archive can be found at~\cite{vincent_knight_2018_1297075}.

\section{Recognising Extortion}\label{sec:delta-zd-strategies}

In~\cite{Press2012}, given a match between 2 memory-one strategies, the concept
of Zero Determinant (ZD) strategies is introduced. The main result of that paper
shows that given two memory one players \(p, q\in\mathbb{R}^4\) a linear
relationship between the players' scores could be forced by one of the players.

Using the notation of~\cite{Press2012}, assuming the utilities for player \(p\)
are given by \(S_x=(R, S, T, P)\) and for player \(q\) by \(S_y=(R, T, S, P)\)
and that the stationary scores of each player is given by \(S_X\) and \(S_Y\)
respectively. The main result of~\cite{Press2012} is that if

\begin{equation}\label{eqn:linear_relationship_for_p}
    \tilde p=\alpha S_x + \beta S_y + \gamma
\end{equation}

or

\begin{equation}\label{eqn:linear_relationship_for_q}
    \tilde q=\alpha S_x + \beta S_y + \gamma
\end{equation}

where \(\tilde p = (1 - p_1, 1 - p_2, p_3, p_4)\) and
\(\tilde q = (1 - q_1, 1 - q_2, q_3, q_4)\) then:

\begin{equation}
    \alpha S_X + \beta S_Y + \gamma = 0
\end{equation}

In~\cite{Press2012} a particular type of ZD strategy is defined: extortionate
strategies. If:

\begin{equation}\label{eqn:constraint_for_extortion}
    \gamma = - P(\alpha + \beta)
\end{equation}

then the player can ensure they get a score \(\chi\) times
larger than the opponent. This extortion coefficient is given by:

\begin{equation}\label{eqn:definition_of_chi}
    \chi=\frac{-\beta}{\alpha}
\end{equation}

Thus, if (\ref{eqn:constraint_for_extortion}) holds and \(\chi >1\) a player is
said to extort their opponent.
Here, the reverse problem is considered: given a
\(p\in\mathbb{R}^4\) how does one identify \(\alpha, \beta\) if they
exist and is the strategy in fact acting in an extortionate way?

These conditions correspond to:

\begin{align}
    \tilde p_1 & = \alpha R + \beta R - P (\alpha + \beta)
            \label{eqn:condition_for_tilde_p1}\\
    \tilde p_2 & = \alpha S + \beta T - P (\alpha + \beta)
            \label{eqn:condition_for_tilde_p2}\\
    \tilde p_3 & = \alpha T + \beta S - P (\alpha + \beta)
            \label{eqn:condition_for_tilde_p3}\\
    \tilde p_4 & = \alpha P + \beta P - P (\alpha + \beta)
            \label{eqn:condition_for_tilde_p4}
\end{align}

Equation (\ref{eqn:condition_for_tilde_p4}) ensures that \(p_4=\tilde p_4=0\).
Equations (\ref{eqn:condition_for_tilde_p1}-\ref{eqn:condition_for_tilde_p3})
can be used to eliminate \(\alpha, \beta\), giving:

\begin{equation}\label{eqn:planar_definition_of_extortion}
    \tilde p_1 = \frac{(R - P)(\tilde p_2 + \tilde p_3)}{S + T - 2P}
\end{equation}

with:

\begin{equation}\label{eqn:definition_of_chi}
    \chi = \frac{\tilde p_2 (P - T) + \tilde p_3 (S - P)}
                {\tilde p_2 (P - S) + \tilde p_3 (T - P)}
\end{equation}

Given a strategy \(p\in\mathbb{R}^{4\times 1}\) equations
(\ref{eqn:condition_for_tilde_p4}), (\ref{eqn:planar_definition_of_extortion}-\ref{eqn:definition_of_chi}) can be used to check if
a strategy is extortionate. The conditions correspond to:

\begin{align}
    p_1 & = \frac{(R-P)(p_2 + p_3) - R + T + S - P}{S + T - 2P}
     \label{eqn:condition_for_p1}\\
    p_4 & = 0 \label{eqn:condition_for_p4}\\
    1 & > p_2 + p_3\label{eqn:condition_for_chi}
\end{align}

The algebraic steps necessary to prove these results are available in the
supporting materials.

All extortionate strategies reside on a triangular (\ref{eqn:condition_for_chi})
plane (\ref{eqn:condition_for_p1}) in 3 dimensions (\ref{eqn:condition_for_p4}).
Using this formulation it can be seen that a necessary (but not sufficient)
condition for an extortionate strategy is that it cooperates on average less
than 50\% of the time when in a state of disagreement with the opponent.

As an example, consider the known extortionate strategy \(p=(8 / 9, 1 / 2, 1 /
3, 0)\) from~\cite{Stewart2012} which is referred to as \texttt{Extort-2}. In
this case, for the standard values of \((R, T, S, P)\) constraint
(\ref{eqn:condition_for_p1}) corresponds to:

\begin{equation}
    p_1 = \frac{2(p_2 + p_3) + 1}{3}
\end{equation}

It is clear that in this case all constraints hold.

This approach could in fact be used to confirm that a given strategy is acting
in an extortionate manner even if it is not a memory one strategy. However, in
practice, if a closed form for \(p\) is not known, then due to measurement
and/or numerical error this would not work.

This problem can be written in the following linear algebraic form where
\(x=(\alpha, \beta)\)
and \(p^*=(\tilde p_1 - 1, tilde_2 - 1, p_3)\):

\begin{equation}\label{eqn:linear_algebraic_equation_for_p}
    Cx= p^*
\end{equation}

\(C\) corresponds to equations
(\ref{eqn:condition_for_tilde_p1}-\ref{eqn:condition_for_tilde_p3}) and is
given by:

\begin{equation}\label{eqn:definition_of_C}
    C =
    \begin{bmatrix}
        R - P & R- P \\
        S - P & T- P \\
        T - P & S- P \\
    \end{bmatrix}
\end{equation}

Note that in general, equation (\ref{eqn:linear_algebraic_equation_for_p}) will
not necessarily have a solution. From the Rouch\'{e}-Capelli theorem if there is
a solution it is unique as \(\text{rank}(C)=2\) which is the dimension of the
variable \(x\). The best fitting \(x\) is found by minimizing:

\begin{equation}\label{eqn:r_squared}
    \text{SSError} = \|C x- p^*\|_2^2 = \sum_{i=1}^{3}\left((C\bar x)_i-p_i^*\right)^2
\end{equation}

Note that \(\text{SSError}\), which is the square of the Frobenius
norm~\cite{Golub2013}, becomes a measure of how close a strategy is to being an
extortionate strategy. Suspicion
of extortion then corresponds to a threshold on \(\text{SSError}\).

By observing interactions (human or otherwise), their memory one representation
can be inferred and this approach can be used to recognise extortionate
behaviour. The notion of comparing theoretic and actual plays of the IPD is not
novel, see for example~\cite{Rand2013}. Immediately it is noted that if the
environment is noisy~\cite{Wu1995} then no strategy can be considered to be
extortionate as \(p_4>0\).

In the next section, this idea will be illustrated by observing the interactions
that take place in a computer based tournament of the IPD\@.

\section{Numerical experiments}\label{sec:numerical-experiments}

In~\cite{Stewart2012} results from a tournament with
\input{./assets/tex/number_of_stewart_plotkin_strategies/main.tex} strategies,
was presented with specific consideration given to ZD strategies. This
tournament is reproduced here using the Axelrod-Python
project~\cite{Knight2016}. To obtain a good measure of the corresponding
transition rates for each strategy all matches have been run for
\input{assets/tex/number_of_turns/main.tex} turns and every match has been
repeated \input{assets/tex/number_of_repetitions/main.tex} times. All of this
interaction data is available at~\cite{vincent_knight_2018_1297075}. A good
match between the inferred Markov chain and the state distribution of the actual
interactions has been verified. Data for this is presented in the supplementary
materials.

Figure~\ref{fig:SSError_overall_in_stewart_plotkin} shows the \(\text{SSError}\)
values for all the strategies in the tournament, as reported
in~\cite{Stewart2012} the extortionate strategy (which has an expected
\(\text{SSError}\) approximately 0) gains a large number of wins.

\begin{figure}[!htbp]
    \centering
    \includegraphics[width=.8\textwidth]{./assets/img/SSError_overall_in_stewart_plotkin/main.pdf}
    \caption{\(\text{SSError}\) and state probabilities for the strategies
        of~\cite{Stewart2012}, ordered both by number of wins and overall score.
        Note that \(P(DC)\) is not shown as it corresponds to the transpose of
        \(P(CD)\). Cooperator and Defector are omitted as they do not visit all
        the states.}
    \label{fig:SSError_overall_in_stewart_plotkin}
\end{figure}

Here, the work of~\cite{Stewart2012} is extended by investigating a tournament
with \input{assets/tex/number_of_full_strategies/main.tex}
strategies.

The results of this analysis are shown in
Figure~\ref{fig:SSError_and_probabilities_in_full}. The top ranking strategies
by number of wins seem to be extortionate (but not against all strategies) and
it can be seen that a small sub group of strategies achieve mutual defection.
All the top ranking strategies according to score achieve mutual cooperation and
do not extort each other, however they
\textbf{do} exhibit extortionate behaviour towards a number of the lower ranking
strategies.

\begin{figure}[!htbp]
    \centering
    \includegraphics[width=.8\textwidth]{./assets/img/SSError_and_probabilities_in_full/main.pdf}
    \caption{\(\text{SSError}\) for the strategies for the full tournament. Only
    strategy interactions for which \(p_4=0\) and \(\chi>1\) are displayed.}
    \label{fig:SSError_and_probabilities_in_full}
\end{figure}

\section{Conclusion}\label{sec:conclusion}

This work defines an approach to measure whether or not a player is playing a
strategy that corresponds to an extortionate strategy as defined
in~\cite{Press2012}: a mathematical model for suspicion. Indeed, all
extortionate strategies have been
 classified as lying on a triangular plane.
This rigorous classification fails to be robust to small measurement error, thus
a statistical approach is proposed.
This is done through a linear algebraic approach for approximating the solution
of a linear system. Using this, a large number of pairwise interactions is
simulated and in fact very few strategies are found to act extortionately.

The work of~\cite{Press2012}, whilst showing that a clever approach to taking
advantage of another memory one strategy exists: this is incomplete. Whilst the
elegance of this result is very attractive, just as the simplicity of the
victory of Tit For Tat in Axelrod's original tournaments was, it is incomplete.
Extortionate strategies achieve a high number of wins but they do not
achieve a high score which corresponds to the fitness landscape in an
evolutionary sense. From the large number of interactions a payoff matrix \(S\)
can be measured where \(S_{ij}\) denotes the score (using standard values of
\((R, S, T, P) = (3, 0, 5, 1)\)) of the \(i\)th strategy
against the \(j\)th strategy. Using this, the replicator equation
describes the evolution of the system based on a population density fitness
function:

\begin{equation}\label{eqn:replicator_dynamics}
    \frac{dx}{dt} = x(S-x^TS x)
\end{equation}

Equation (\ref{eqn:replicator_dynamics}) is solved numerically through an
integration technique described in~\cite{Petzold1983} and
Figure~\ref{fig:replicator_dynamics} shows the evolution of the distribution of
the system: the various strategies are ranked by scores. It is clear to see that
only the high ranking strategies survive the evolutionary process (in fact,
only \input{./assets/img/replicator_dynamics/main.tex}
have a final distribution greater than \(10 ^ {-2}\)). This confirms the
findings of~\cite{Moran1707} in which sophisticated strategies resist
evolutionary invasion of shorter memory strategies. Recalling
Figure~\ref{fig:SSError_and_probabilities_in_full} this demonstrates that:

\begin{itemize}
    \item Cooperation emerges through the evolutionary process: the high scoring
        strategies do not exhibit extortionate behaviour towards each other.
    \item Extortionate strategies do not survive the evolutionary process.
\end{itemize}

\begin{figure}[!htbp]
    \centering
    \includegraphics[width=.8\textwidth]{./assets/img/replicator_dynamics/main.pdf}
    \caption{Numerical simulation of the replicator equation
    (\ref{eqn:replicator_dynamics}): strategies are ordered by score, only the strategies with a high score survive the evolutionary process.}
    \label{fig:replicator_dynamics}
\end{figure}

This work can be used to classify plays of the IPD\@: data can be collected from
actual interactions (in lab or in the field). Furthermore, this allows for a
classification method similar to the notion of fingerprinting presented
in~\cite{Ashlock2008}. Trained strategies can potentially be classified as
extortionate or not or it could be possible to even constrain the reinforcement
learning approaches that are becoming prevalent in the literature.
Alternatively, this mathematical approach for recognising extortion could be
used in sophisticated strategies to defend against invasion. Arguably, some of
the strategies considered here exhibit this behaviour, indeed as described
in~\cite{Harper2017}, the top ranking strategies in the full tournament are
obtained using evolutionary reinforcement learning techniques, thus, suspicion
of extortionate behaviour could in fact be an evolutionary trait.

\section*{Acknowledgements}

The following open source software libraries were used in this research:

\begin{itemize}
    \item The Axelrod ~\cite{Knight2016, Knight2018} library (IPD strategies and
        tournaments).
    \item The sympy library~\cite{Meurer2017} (verification of all symbolic
        calculations).
    \item The matplotlib~\cite{Droettboom2018} library (visualisation).
    \item The pandas~\cite{Structures2010}, dask~\cite{Dask2016} and
        NumPy~\cite{Oliphant2015} libraries (data manipulation).
    \item The SciPy~\cite{Jones2001} library (numerical integration of the
        replicator equation).
\end{itemize}

This work was performed using the computational facilities of the Advanced
Research Computing @ Cardiff (ARCCA) Division, Cardiff University.

\printbibliography

\newpage
\section*{Supplementary materials}

\includepdf{assets/pdf/proof_of_form_of_extortionate_strategies/main.pdf}

\newpage

Using the pair wise interactions the transition rates \(p,
q\) can be measured and the steady state probabilities inferred and compared to
the actual probabilities of each state.
This is done numerically by computing the singular eigenvector of the
matrix \(A\) \cite{Stewart2009}:

\[
    A =
    \begin{bmatrix}
        p_1 q_1 & p_1 (1 - q_1) & (1 - p_1) q_1 & (1 -p_1) (1 - q_1) \\
        p_2 q_2 & p_2 (1 - q_2) & (1 - p_2) q_2 & (1 -p_2) (1 - q_2) \\
        p_3 q_3 & p_3 (1 - q_3) & (1 - p_3) q_3 & (1 -p_3) (1 - q_3) \\
        p_4 q_4 & p_4 (1 - q_4) & (1 - p_4) q_4 & (1 -p_4) (1 - q_4) \\
    \end{bmatrix}
\]

Figure~\ref{fig:computed_probabilities_vs_theoretic_probabilities} shows a
regression line fitted to every pairwise interaction with a reported
\(\text{SSError}\) value (pairwise interactions with missing states were
omitted). This serves to validate the approach: a part from some edge cases the
relationship is consistent.

\begin{figure}[!htbp]
    \centering
    \includegraphics[width=.8\textwidth]{./assets/img/computed_probabilities_vs_theoretic_probabilities/main.pdf}
    \caption{The
        relationship between the steady state probabilities inferred from the
        measured transitions and the actual steady state probabilities. A linear
        regression line is included validating the approach.}
    \label{fig:computed_probabilities_vs_theoretic_probabilities}
\end{figure}


\end{document}

    strategies is considered. In this setting
    the most highly performing strategies do not play in an extortionate way
    against each other but do against lower performing strategies.
    This suggests that whilst the theory of Zero Determinant strategies
    indicates that memory is not of fundamental importance to the evolution of
    cooperative behaviour, this is incomplete.
\end{abstract}

\section{Introduction}\label{sec:introduction}

Agent based game theoretic models have become a stalwart of the underpinning
mathematics of interactive behaviours. One of the major pieces of work
in this area is the pair of original computer tournaments run by Robert
Axelrod~\cite{Axelrod1980, Axelrod1980a}. These tournaments pitted submitted
computer strategies against each other in plays of the Iterated Prisoner's
Dilemma. A common game where agents can choose to pay a slight cost to their
immediate utility in the hope of building a reputation. This has been used in
economic and evolutionary game theory to understand the evolution of cooperative
behaviour.

Recently, a class of strategies was described in~\cite{Press2012} that can
provably extort any given opponent. In~\cite{Hilbe2013, Moran1707} some
questions have already been asked about the true effectiveness of these
strategies in an evolutionary setting. Here another question is asked: is it
possible to recognise this extortionate behaviour? A mathematical procedure for
suspicion is presented: in the same way that the continued actions of an
extortionate individual might raise suspicion.

This work makes use of the Axelrod Python library~\cite{Knight2018, Knight2016}
with a large number of Prisoner Dilemma strategies available to give an
extensive numerical example of the ideas presented.  The approach is presented
in Section~\ref{sec:delta-zd-strategies}.  All of the code and data discussed
in Section~\ref{sec:numerical-experiments} is open sourced, archived and
written according to best scientific principles~\cite{Wilson2014}. The data
archive can be found at~\cite{vincent_knight_2018_1297075}.

\section{Recognising Extortion}\label{sec:delta-zd-strategies}

In~\cite{Press2012}, given a match between 2 memory-one strategies, the concept
of Zero Determinant (ZD) strategies is introduced. The main result of that paper
shows that given two memory one players \(p, q\in\mathbb{R}^4\) a linear
relationship between the players' scores could be forced by one of the players.

Using the notation of~\cite{Press2012}, assuming the utilities for player \(p\)
are given by \(S_x=(R, S, T, P)\) and for player \(q\) by \(S_y=(R, T, S, P)\)
and that the stationary scores of each player is given by \(S_X\) and \(S_Y\)
respectively. The main result of~\cite{Press2012} is that if

\begin{equation}\label{eqn:linear_relationship_for_p}
    \tilde p=\alpha S_x + \beta S_y + \gamma
\end{equation}

or

\begin{equation}\label{eqn:linear_relationship_for_q}
    \tilde q=\alpha S_x + \beta S_y + \gamma
\end{equation}

where \(\tilde p = (1 - p_1, 1 - p_2, p_3, p_4)\) and
\(\tilde q = (1 - q_1, 1 - q_2, q_3, q_4)\) then:

\begin{equation}
    \alpha S_X + \beta S_Y + \gamma = 0
\end{equation}

In~\cite{Press2012} a particular type of ZD strategy is defined: extortionate
strategies. If:

\begin{equation}\label{eqn:constraint_for_extortion}
    \gamma = - P(\alpha + \beta)
\end{equation}

then the player can ensure they get a score \(\chi\) times
larger than the opponent. This extortion coefficient is given by:

\begin{equation}\label{eqn:definition_of_chi}
    \chi=\frac{-\beta}{\alpha}
\end{equation}

Thus, if (\ref{eqn:constraint_for_extortion}) holds and \(\chi >1\) a player is
said to extort their opponent.
Here, the reverse problem is considered: given a
\(p\in\mathbb{R}^4\) how does one identify \(\alpha, \beta\) if they
exist and is the strategy in fact acting in an extortionate way?

These conditions correspond to:

\begin{align}
    \tilde p_1 & = \alpha R + \beta R - P (\alpha + \beta)
            \label{eqn:condition_for_tilde_p1}\\
    \tilde p_2 & = \alpha S + \beta T - P (\alpha + \beta)
            \label{eqn:condition_for_tilde_p2}\\
    \tilde p_3 & = \alpha T + \beta S - P (\alpha + \beta)
            \label{eqn:condition_for_tilde_p3}\\
    \tilde p_4 & = \alpha P + \beta P - P (\alpha + \beta)
            \label{eqn:condition_for_tilde_p4}
\end{align}

Equation (\ref{eqn:condition_for_tilde_p4}) ensures that \(p_4=\tilde p_4=0\).
Equations (\ref{eqn:condition_for_tilde_p1}-\ref{eqn:condition_for_tilde_p3})
can be used to eliminate \(\alpha, \beta\), giving:

\begin{equation}\label{eqn:planar_definition_of_extortion}
    \tilde p_1 = \frac{(R - P)(\tilde p_2 + \tilde p_3)}{S + T - 2P}
\end{equation}

with:

\begin{equation}\label{eqn:definition_of_chi}
    \chi = \frac{\tilde p_2 (P - T) + \tilde p_3 (S - P)}
                {\tilde p_2 (P - S) + \tilde p_3 (T - P)}
\end{equation}

Given a strategy \(p\in\mathbb{R}^{4\times 1}\) equations
(\ref{eqn:condition_for_tilde_p4}), (\ref{eqn:planar_definition_of_extortion}-\ref{eqn:definition_of_chi}) can be used to check if
a strategy is extortionate. The conditions correspond to:

\begin{align}
    p_1 & = \frac{(R-P)(p_2 + p_3) - R + T + S - P}{S + T - 2P}
     \label{eqn:condition_for_p1}\\
    p_4 & = 0 \label{eqn:condition_for_p4}\\
    1 & > p_2 + p_3\label{eqn:condition_for_chi}
\end{align}

The algebraic steps necessary to prove these results are available in the
supporting materials.

All extortionate strategies reside on a triangular (\ref{eqn:condition_for_chi})
plane (\ref{eqn:condition_for_p1}) in 3 dimensions (\ref{eqn:condition_for_p4}).
Using this formulation it can be seen that a necessary (but not sufficient)
condition for an extortionate strategy is that it cooperates on average less
than 50\% of the time when in a state of disagreement with the opponent.

As an example, consider the known extortionate strategy \(p=(8 / 9, 1 / 2, 1 /
3, 0)\) from~\cite{Stewart2012} which is referred to as \texttt{Extort-2}. In
this case, for the standard values of \((R, T, S, P)\) constraint
(\ref{eqn:condition_for_p1}) corresponds to:

\begin{equation}
    p_1 = \frac{2(p_2 + p_3) + 1}{3}
\end{equation}

It is clear that in this case all constraints hold.

This approach could in fact be used to confirm that a given strategy is acting
in an extortionate manner even if it is not a memory one strategy. However, in
practice, if a closed form for \(p\) is not known, then due to measurement
and/or numerical error this would not work.

This problem can be written in the following linear algebraic form where
\(x=(\alpha, \beta)\)
and \(p^*=(\tilde p_1 - 1, tilde_2 - 1, p_3)\):

\begin{equation}\label{eqn:linear_algebraic_equation_for_p}
    Cx= p^*
\end{equation}

\(C\) corresponds to equations
(\ref{eqn:condition_for_tilde_p1}-\ref{eqn:condition_for_tilde_p3}) and is
given by:

\begin{equation}\label{eqn:definition_of_C}
    C =
    \begin{bmatrix}
        R - P & R- P \\
        S - P & T- P \\
        T - P & S- P \\
    \end{bmatrix}
\end{equation}

Note that in general, equation (\ref{eqn:linear_algebraic_equation_for_p}) will
not necessarily have a solution. From the Rouch\'{e}-Capelli theorem if there is
a solution it is unique as \(\text{rank}(C)=2\) which is the dimension of the
variable \(x\). The best fitting \(x\) is found by minimizing:

\begin{equation}\label{eqn:r_squared}
    \text{SSError} = \|C x- p^*\|_2^2 = \sum_{i=1}^{3}\left((C\bar x)_i-p_i^*\right)^2
\end{equation}

Note that \(\text{SSError}\), which is the square of the Frobenius
norm~\cite{Golub2013}, becomes a measure of how close a strategy is to being an
extortionate strategy. Suspicion
of extortion then corresponds to a threshold on \(\text{SSError}\).

By observing interactions (human or otherwise), their memory one representation
can be inferred and this approach can be used to recognise extortionate
behaviour. The notion of comparing theoretic and actual plays of the IPD is not
novel, see for example~\cite{Rand2013}. Immediately it is noted that if the
environment is noisy~\cite{Wu1995} then no strategy can be considered to be
extortionate as \(p_4>0\).

In the next section, this idea will be illustrated by observing the interactions
that take place in a computer based tournament of the IPD\@.

\section{Numerical experiments}\label{sec:numerical-experiments}

In~\cite{Stewart2012} results from a tournament with
\documentclass[a4paper]{article}

\usepackage{amsmath}
\usepackage{amssymb}
\usepackage[margin=1.5cm,
            includefoot,
            footskip=30pt]{geometry}
\usepackage{layout}
\usepackage{graphicx}
\usepackage{subcaption}

\usepackage{biblatex}
\usepackage{pdfpages}

\bibliography{main.bib}

\title{Suspicion: Recognising and evaluating the effectiveness
       of extortion in the Iterated Prisoner's Dilemma}
\author{Vincent A. Knight \and Nikoleta E. Glynatsi}
\date{\today}



\begin{document}

\maketitle

\begin{abstract}
    The Iterated Prisoner's Dilemma is a model for rational and evolutionary
    interactive behaviour. It has applications both in the study of human social
    behaviour as well as in biology.
    It is used to understand when and how a rational individual might
    accept an immediate cost to their own utility for the direct benefit of
    another.

    Much attention has been given to a class of strategies called
    Zero Determinant strategies. It has been theoretically shown that these
    strategies can ``extort'' any player.

    In this work, an approach to identify if observed strategies are playing in
    an extortionate way is described. Furthermore, experimental analysis of
    a large tournament with \input{assets/tex/number_of_full_strategies/main.tex}
    strategies is considered. In this setting
    the most highly performing strategies do not play in an extortionate way
    against each other but do against lower performing strategies.
    This suggests that whilst the theory of Zero Determinant strategies
    indicates that memory is not of fundamental importance to the evolution of
    cooperative behaviour, this is incomplete.
\end{abstract}

\section{Introduction}\label{sec:introduction}

Agent based game theoretic models have become a stalwart of the underpinning
mathematics of interactive behaviours. One of the major pieces of work
in this area is the pair of original computer tournaments run by Robert
Axelrod~\cite{Axelrod1980, Axelrod1980a}. These tournaments pitted submitted
computer strategies against each other in plays of the Iterated Prisoner's
Dilemma. A common game where agents can choose to pay a slight cost to their
immediate utility in the hope of building a reputation. This has been used in
economic and evolutionary game theory to understand the evolution of cooperative
behaviour.

Recently, a class of strategies was described in~\cite{Press2012} that can
provably extort any given opponent. In~\cite{Hilbe2013, Moran1707} some
questions have already been asked about the true effectiveness of these
strategies in an evolutionary setting. Here another question is asked: is it
possible to recognise this extortionate behaviour? A mathematical procedure for
suspicion is presented: in the same way that the continued actions of an
extortionate individual might raise suspicion.

This work makes use of the Axelrod Python library~\cite{Knight2018, Knight2016}
with a large number of Prisoner Dilemma strategies available to give an
extensive numerical example of the ideas presented.  The approach is presented
in Section~\ref{sec:delta-zd-strategies}.  All of the code and data discussed
in Section~\ref{sec:numerical-experiments} is open sourced, archived and
written according to best scientific principles~\cite{Wilson2014}. The data
archive can be found at~\cite{vincent_knight_2018_1297075}.

\section{Recognising Extortion}\label{sec:delta-zd-strategies}

In~\cite{Press2012}, given a match between 2 memory-one strategies, the concept
of Zero Determinant (ZD) strategies is introduced. The main result of that paper
shows that given two memory one players \(p, q\in\mathbb{R}^4\) a linear
relationship between the players' scores could be forced by one of the players.

Using the notation of~\cite{Press2012}, assuming the utilities for player \(p\)
are given by \(S_x=(R, S, T, P)\) and for player \(q\) by \(S_y=(R, T, S, P)\)
and that the stationary scores of each player is given by \(S_X\) and \(S_Y\)
respectively. The main result of~\cite{Press2012} is that if

\begin{equation}\label{eqn:linear_relationship_for_p}
    \tilde p=\alpha S_x + \beta S_y + \gamma
\end{equation}

or

\begin{equation}\label{eqn:linear_relationship_for_q}
    \tilde q=\alpha S_x + \beta S_y + \gamma
\end{equation}

where \(\tilde p = (1 - p_1, 1 - p_2, p_3, p_4)\) and
\(\tilde q = (1 - q_1, 1 - q_2, q_3, q_4)\) then:

\begin{equation}
    \alpha S_X + \beta S_Y + \gamma = 0
\end{equation}

In~\cite{Press2012} a particular type of ZD strategy is defined: extortionate
strategies. If:

\begin{equation}\label{eqn:constraint_for_extortion}
    \gamma = - P(\alpha + \beta)
\end{equation}

then the player can ensure they get a score \(\chi\) times
larger than the opponent. This extortion coefficient is given by:

\begin{equation}\label{eqn:definition_of_chi}
    \chi=\frac{-\beta}{\alpha}
\end{equation}

Thus, if (\ref{eqn:constraint_for_extortion}) holds and \(\chi >1\) a player is
said to extort their opponent.
Here, the reverse problem is considered: given a
\(p\in\mathbb{R}^4\) how does one identify \(\alpha, \beta\) if they
exist and is the strategy in fact acting in an extortionate way?

These conditions correspond to:

\begin{align}
    \tilde p_1 & = \alpha R + \beta R - P (\alpha + \beta)
            \label{eqn:condition_for_tilde_p1}\\
    \tilde p_2 & = \alpha S + \beta T - P (\alpha + \beta)
            \label{eqn:condition_for_tilde_p2}\\
    \tilde p_3 & = \alpha T + \beta S - P (\alpha + \beta)
            \label{eqn:condition_for_tilde_p3}\\
    \tilde p_4 & = \alpha P + \beta P - P (\alpha + \beta)
            \label{eqn:condition_for_tilde_p4}
\end{align}

Equation (\ref{eqn:condition_for_tilde_p4}) ensures that \(p_4=\tilde p_4=0\).
Equations (\ref{eqn:condition_for_tilde_p1}-\ref{eqn:condition_for_tilde_p3})
can be used to eliminate \(\alpha, \beta\), giving:

\begin{equation}\label{eqn:planar_definition_of_extortion}
    \tilde p_1 = \frac{(R - P)(\tilde p_2 + \tilde p_3)}{S + T - 2P}
\end{equation}

with:

\begin{equation}\label{eqn:definition_of_chi}
    \chi = \frac{\tilde p_2 (P - T) + \tilde p_3 (S - P)}
                {\tilde p_2 (P - S) + \tilde p_3 (T - P)}
\end{equation}

Given a strategy \(p\in\mathbb{R}^{4\times 1}\) equations
(\ref{eqn:condition_for_tilde_p4}), (\ref{eqn:planar_definition_of_extortion}-\ref{eqn:definition_of_chi}) can be used to check if
a strategy is extortionate. The conditions correspond to:

\begin{align}
    p_1 & = \frac{(R-P)(p_2 + p_3) - R + T + S - P}{S + T - 2P}
     \label{eqn:condition_for_p1}\\
    p_4 & = 0 \label{eqn:condition_for_p4}\\
    1 & > p_2 + p_3\label{eqn:condition_for_chi}
\end{align}

The algebraic steps necessary to prove these results are available in the
supporting materials.

All extortionate strategies reside on a triangular (\ref{eqn:condition_for_chi})
plane (\ref{eqn:condition_for_p1}) in 3 dimensions (\ref{eqn:condition_for_p4}).
Using this formulation it can be seen that a necessary (but not sufficient)
condition for an extortionate strategy is that it cooperates on average less
than 50\% of the time when in a state of disagreement with the opponent.

As an example, consider the known extortionate strategy \(p=(8 / 9, 1 / 2, 1 /
3, 0)\) from~\cite{Stewart2012} which is referred to as \texttt{Extort-2}. In
this case, for the standard values of \((R, T, S, P)\) constraint
(\ref{eqn:condition_for_p1}) corresponds to:

\begin{equation}
    p_1 = \frac{2(p_2 + p_3) + 1}{3}
\end{equation}

It is clear that in this case all constraints hold.

This approach could in fact be used to confirm that a given strategy is acting
in an extortionate manner even if it is not a memory one strategy. However, in
practice, if a closed form for \(p\) is not known, then due to measurement
and/or numerical error this would not work.

This problem can be written in the following linear algebraic form where
\(x=(\alpha, \beta)\)
and \(p^*=(\tilde p_1 - 1, tilde_2 - 1, p_3)\):

\begin{equation}\label{eqn:linear_algebraic_equation_for_p}
    Cx= p^*
\end{equation}

\(C\) corresponds to equations
(\ref{eqn:condition_for_tilde_p1}-\ref{eqn:condition_for_tilde_p3}) and is
given by:

\begin{equation}\label{eqn:definition_of_C}
    C =
    \begin{bmatrix}
        R - P & R- P \\
        S - P & T- P \\
        T - P & S- P \\
    \end{bmatrix}
\end{equation}

Note that in general, equation (\ref{eqn:linear_algebraic_equation_for_p}) will
not necessarily have a solution. From the Rouch\'{e}-Capelli theorem if there is
a solution it is unique as \(\text{rank}(C)=2\) which is the dimension of the
variable \(x\). The best fitting \(x\) is found by minimizing:

\begin{equation}\label{eqn:r_squared}
    \text{SSError} = \|C x- p^*\|_2^2 = \sum_{i=1}^{3}\left((C\bar x)_i-p_i^*\right)^2
\end{equation}

Note that \(\text{SSError}\), which is the square of the Frobenius
norm~\cite{Golub2013}, becomes a measure of how close a strategy is to being an
extortionate strategy. Suspicion
of extortion then corresponds to a threshold on \(\text{SSError}\).

By observing interactions (human or otherwise), their memory one representation
can be inferred and this approach can be used to recognise extortionate
behaviour. The notion of comparing theoretic and actual plays of the IPD is not
novel, see for example~\cite{Rand2013}. Immediately it is noted that if the
environment is noisy~\cite{Wu1995} then no strategy can be considered to be
extortionate as \(p_4>0\).

In the next section, this idea will be illustrated by observing the interactions
that take place in a computer based tournament of the IPD\@.

\section{Numerical experiments}\label{sec:numerical-experiments}

In~\cite{Stewart2012} results from a tournament with
\input{./assets/tex/number_of_stewart_plotkin_strategies/main.tex} strategies,
was presented with specific consideration given to ZD strategies. This
tournament is reproduced here using the Axelrod-Python
project~\cite{Knight2016}. To obtain a good measure of the corresponding
transition rates for each strategy all matches have been run for
\input{assets/tex/number_of_turns/main.tex} turns and every match has been
repeated \input{assets/tex/number_of_repetitions/main.tex} times. All of this
interaction data is available at~\cite{vincent_knight_2018_1297075}. A good
match between the inferred Markov chain and the state distribution of the actual
interactions has been verified. Data for this is presented in the supplementary
materials.

Figure~\ref{fig:SSError_overall_in_stewart_plotkin} shows the \(\text{SSError}\)
values for all the strategies in the tournament, as reported
in~\cite{Stewart2012} the extortionate strategy (which has an expected
\(\text{SSError}\) approximately 0) gains a large number of wins.

\begin{figure}[!htbp]
    \centering
    \includegraphics[width=.8\textwidth]{./assets/img/SSError_overall_in_stewart_plotkin/main.pdf}
    \caption{\(\text{SSError}\) and state probabilities for the strategies
        of~\cite{Stewart2012}, ordered both by number of wins and overall score.
        Note that \(P(DC)\) is not shown as it corresponds to the transpose of
        \(P(CD)\). Cooperator and Defector are omitted as they do not visit all
        the states.}
    \label{fig:SSError_overall_in_stewart_plotkin}
\end{figure}

Here, the work of~\cite{Stewart2012} is extended by investigating a tournament
with \input{assets/tex/number_of_full_strategies/main.tex}
strategies.

The results of this analysis are shown in
Figure~\ref{fig:SSError_and_probabilities_in_full}. The top ranking strategies
by number of wins seem to be extortionate (but not against all strategies) and
it can be seen that a small sub group of strategies achieve mutual defection.
All the top ranking strategies according to score achieve mutual cooperation and
do not extort each other, however they
\textbf{do} exhibit extortionate behaviour towards a number of the lower ranking
strategies.

\begin{figure}[!htbp]
    \centering
    \includegraphics[width=.8\textwidth]{./assets/img/SSError_and_probabilities_in_full/main.pdf}
    \caption{\(\text{SSError}\) for the strategies for the full tournament. Only
    strategy interactions for which \(p_4=0\) and \(\chi>1\) are displayed.}
    \label{fig:SSError_and_probabilities_in_full}
\end{figure}

\section{Conclusion}\label{sec:conclusion}

This work defines an approach to measure whether or not a player is playing a
strategy that corresponds to an extortionate strategy as defined
in~\cite{Press2012}: a mathematical model for suspicion. Indeed, all
extortionate strategies have been
 classified as lying on a triangular plane.
This rigorous classification fails to be robust to small measurement error, thus
a statistical approach is proposed.
This is done through a linear algebraic approach for approximating the solution
of a linear system. Using this, a large number of pairwise interactions is
simulated and in fact very few strategies are found to act extortionately.

The work of~\cite{Press2012}, whilst showing that a clever approach to taking
advantage of another memory one strategy exists: this is incomplete. Whilst the
elegance of this result is very attractive, just as the simplicity of the
victory of Tit For Tat in Axelrod's original tournaments was, it is incomplete.
Extortionate strategies achieve a high number of wins but they do not
achieve a high score which corresponds to the fitness landscape in an
evolutionary sense. From the large number of interactions a payoff matrix \(S\)
can be measured where \(S_{ij}\) denotes the score (using standard values of
\((R, S, T, P) = (3, 0, 5, 1)\)) of the \(i\)th strategy
against the \(j\)th strategy. Using this, the replicator equation
describes the evolution of the system based on a population density fitness
function:

\begin{equation}\label{eqn:replicator_dynamics}
    \frac{dx}{dt} = x(S-x^TS x)
\end{equation}

Equation (\ref{eqn:replicator_dynamics}) is solved numerically through an
integration technique described in~\cite{Petzold1983} and
Figure~\ref{fig:replicator_dynamics} shows the evolution of the distribution of
the system: the various strategies are ranked by scores. It is clear to see that
only the high ranking strategies survive the evolutionary process (in fact,
only \input{./assets/img/replicator_dynamics/main.tex}
have a final distribution greater than \(10 ^ {-2}\)). This confirms the
findings of~\cite{Moran1707} in which sophisticated strategies resist
evolutionary invasion of shorter memory strategies. Recalling
Figure~\ref{fig:SSError_and_probabilities_in_full} this demonstrates that:

\begin{itemize}
    \item Cooperation emerges through the evolutionary process: the high scoring
        strategies do not exhibit extortionate behaviour towards each other.
    \item Extortionate strategies do not survive the evolutionary process.
\end{itemize}

\begin{figure}[!htbp]
    \centering
    \includegraphics[width=.8\textwidth]{./assets/img/replicator_dynamics/main.pdf}
    \caption{Numerical simulation of the replicator equation
    (\ref{eqn:replicator_dynamics}): strategies are ordered by score, only the strategies with a high score survive the evolutionary process.}
    \label{fig:replicator_dynamics}
\end{figure}

This work can be used to classify plays of the IPD\@: data can be collected from
actual interactions (in lab or in the field). Furthermore, this allows for a
classification method similar to the notion of fingerprinting presented
in~\cite{Ashlock2008}. Trained strategies can potentially be classified as
extortionate or not or it could be possible to even constrain the reinforcement
learning approaches that are becoming prevalent in the literature.
Alternatively, this mathematical approach for recognising extortion could be
used in sophisticated strategies to defend against invasion. Arguably, some of
the strategies considered here exhibit this behaviour, indeed as described
in~\cite{Harper2017}, the top ranking strategies in the full tournament are
obtained using evolutionary reinforcement learning techniques, thus, suspicion
of extortionate behaviour could in fact be an evolutionary trait.

\section*{Acknowledgements}

The following open source software libraries were used in this research:

\begin{itemize}
    \item The Axelrod ~\cite{Knight2016, Knight2018} library (IPD strategies and
        tournaments).
    \item The sympy library~\cite{Meurer2017} (verification of all symbolic
        calculations).
    \item The matplotlib~\cite{Droettboom2018} library (visualisation).
    \item The pandas~\cite{Structures2010}, dask~\cite{Dask2016} and
        NumPy~\cite{Oliphant2015} libraries (data manipulation).
    \item The SciPy~\cite{Jones2001} library (numerical integration of the
        replicator equation).
\end{itemize}

This work was performed using the computational facilities of the Advanced
Research Computing @ Cardiff (ARCCA) Division, Cardiff University.

\printbibliography

\newpage
\section*{Supplementary materials}

\includepdf{assets/pdf/proof_of_form_of_extortionate_strategies/main.pdf}

\newpage

Using the pair wise interactions the transition rates \(p,
q\) can be measured and the steady state probabilities inferred and compared to
the actual probabilities of each state.
This is done numerically by computing the singular eigenvector of the
matrix \(A\) \cite{Stewart2009}:

\[
    A =
    \begin{bmatrix}
        p_1 q_1 & p_1 (1 - q_1) & (1 - p_1) q_1 & (1 -p_1) (1 - q_1) \\
        p_2 q_2 & p_2 (1 - q_2) & (1 - p_2) q_2 & (1 -p_2) (1 - q_2) \\
        p_3 q_3 & p_3 (1 - q_3) & (1 - p_3) q_3 & (1 -p_3) (1 - q_3) \\
        p_4 q_4 & p_4 (1 - q_4) & (1 - p_4) q_4 & (1 -p_4) (1 - q_4) \\
    \end{bmatrix}
\]

Figure~\ref{fig:computed_probabilities_vs_theoretic_probabilities} shows a
regression line fitted to every pairwise interaction with a reported
\(\text{SSError}\) value (pairwise interactions with missing states were
omitted). This serves to validate the approach: a part from some edge cases the
relationship is consistent.

\begin{figure}[!htbp]
    \centering
    \includegraphics[width=.8\textwidth]{./assets/img/computed_probabilities_vs_theoretic_probabilities/main.pdf}
    \caption{The
        relationship between the steady state probabilities inferred from the
        measured transitions and the actual steady state probabilities. A linear
        regression line is included validating the approach.}
    \label{fig:computed_probabilities_vs_theoretic_probabilities}
\end{figure}


\end{document}
 strategies,
was presented with specific consideration given to ZD strategies. This
tournament is reproduced here using the Axelrod-Python
project~\cite{Knight2016}. To obtain a good measure of the corresponding
transition rates for each strategy all matches have been run for
\documentclass[a4paper]{article}

\usepackage{amsmath}
\usepackage{amssymb}
\usepackage[margin=1.5cm,
            includefoot,
            footskip=30pt]{geometry}
\usepackage{layout}
\usepackage{graphicx}
\usepackage{subcaption}

\usepackage{biblatex}
\usepackage{pdfpages}

\bibliography{main.bib}

\title{Suspicion: Recognising and evaluating the effectiveness
       of extortion in the Iterated Prisoner's Dilemma}
\author{Vincent A. Knight \and Nikoleta E. Glynatsi}
\date{\today}



\begin{document}

\maketitle

\begin{abstract}
    The Iterated Prisoner's Dilemma is a model for rational and evolutionary
    interactive behaviour. It has applications both in the study of human social
    behaviour as well as in biology.
    It is used to understand when and how a rational individual might
    accept an immediate cost to their own utility for the direct benefit of
    another.

    Much attention has been given to a class of strategies called
    Zero Determinant strategies. It has been theoretically shown that these
    strategies can ``extort'' any player.

    In this work, an approach to identify if observed strategies are playing in
    an extortionate way is described. Furthermore, experimental analysis of
    a large tournament with \input{assets/tex/number_of_full_strategies/main.tex}
    strategies is considered. In this setting
    the most highly performing strategies do not play in an extortionate way
    against each other but do against lower performing strategies.
    This suggests that whilst the theory of Zero Determinant strategies
    indicates that memory is not of fundamental importance to the evolution of
    cooperative behaviour, this is incomplete.
\end{abstract}

\section{Introduction}\label{sec:introduction}

Agent based game theoretic models have become a stalwart of the underpinning
mathematics of interactive behaviours. One of the major pieces of work
in this area is the pair of original computer tournaments run by Robert
Axelrod~\cite{Axelrod1980, Axelrod1980a}. These tournaments pitted submitted
computer strategies against each other in plays of the Iterated Prisoner's
Dilemma. A common game where agents can choose to pay a slight cost to their
immediate utility in the hope of building a reputation. This has been used in
economic and evolutionary game theory to understand the evolution of cooperative
behaviour.

Recently, a class of strategies was described in~\cite{Press2012} that can
provably extort any given opponent. In~\cite{Hilbe2013, Moran1707} some
questions have already been asked about the true effectiveness of these
strategies in an evolutionary setting. Here another question is asked: is it
possible to recognise this extortionate behaviour? A mathematical procedure for
suspicion is presented: in the same way that the continued actions of an
extortionate individual might raise suspicion.

This work makes use of the Axelrod Python library~\cite{Knight2018, Knight2016}
with a large number of Prisoner Dilemma strategies available to give an
extensive numerical example of the ideas presented.  The approach is presented
in Section~\ref{sec:delta-zd-strategies}.  All of the code and data discussed
in Section~\ref{sec:numerical-experiments} is open sourced, archived and
written according to best scientific principles~\cite{Wilson2014}. The data
archive can be found at~\cite{vincent_knight_2018_1297075}.

\section{Recognising Extortion}\label{sec:delta-zd-strategies}

In~\cite{Press2012}, given a match between 2 memory-one strategies, the concept
of Zero Determinant (ZD) strategies is introduced. The main result of that paper
shows that given two memory one players \(p, q\in\mathbb{R}^4\) a linear
relationship between the players' scores could be forced by one of the players.

Using the notation of~\cite{Press2012}, assuming the utilities for player \(p\)
are given by \(S_x=(R, S, T, P)\) and for player \(q\) by \(S_y=(R, T, S, P)\)
and that the stationary scores of each player is given by \(S_X\) and \(S_Y\)
respectively. The main result of~\cite{Press2012} is that if

\begin{equation}\label{eqn:linear_relationship_for_p}
    \tilde p=\alpha S_x + \beta S_y + \gamma
\end{equation}

or

\begin{equation}\label{eqn:linear_relationship_for_q}
    \tilde q=\alpha S_x + \beta S_y + \gamma
\end{equation}

where \(\tilde p = (1 - p_1, 1 - p_2, p_3, p_4)\) and
\(\tilde q = (1 - q_1, 1 - q_2, q_3, q_4)\) then:

\begin{equation}
    \alpha S_X + \beta S_Y + \gamma = 0
\end{equation}

In~\cite{Press2012} a particular type of ZD strategy is defined: extortionate
strategies. If:

\begin{equation}\label{eqn:constraint_for_extortion}
    \gamma = - P(\alpha + \beta)
\end{equation}

then the player can ensure they get a score \(\chi\) times
larger than the opponent. This extortion coefficient is given by:

\begin{equation}\label{eqn:definition_of_chi}
    \chi=\frac{-\beta}{\alpha}
\end{equation}

Thus, if (\ref{eqn:constraint_for_extortion}) holds and \(\chi >1\) a player is
said to extort their opponent.
Here, the reverse problem is considered: given a
\(p\in\mathbb{R}^4\) how does one identify \(\alpha, \beta\) if they
exist and is the strategy in fact acting in an extortionate way?

These conditions correspond to:

\begin{align}
    \tilde p_1 & = \alpha R + \beta R - P (\alpha + \beta)
            \label{eqn:condition_for_tilde_p1}\\
    \tilde p_2 & = \alpha S + \beta T - P (\alpha + \beta)
            \label{eqn:condition_for_tilde_p2}\\
    \tilde p_3 & = \alpha T + \beta S - P (\alpha + \beta)
            \label{eqn:condition_for_tilde_p3}\\
    \tilde p_4 & = \alpha P + \beta P - P (\alpha + \beta)
            \label{eqn:condition_for_tilde_p4}
\end{align}

Equation (\ref{eqn:condition_for_tilde_p4}) ensures that \(p_4=\tilde p_4=0\).
Equations (\ref{eqn:condition_for_tilde_p1}-\ref{eqn:condition_for_tilde_p3})
can be used to eliminate \(\alpha, \beta\), giving:

\begin{equation}\label{eqn:planar_definition_of_extortion}
    \tilde p_1 = \frac{(R - P)(\tilde p_2 + \tilde p_3)}{S + T - 2P}
\end{equation}

with:

\begin{equation}\label{eqn:definition_of_chi}
    \chi = \frac{\tilde p_2 (P - T) + \tilde p_3 (S - P)}
                {\tilde p_2 (P - S) + \tilde p_3 (T - P)}
\end{equation}

Given a strategy \(p\in\mathbb{R}^{4\times 1}\) equations
(\ref{eqn:condition_for_tilde_p4}), (\ref{eqn:planar_definition_of_extortion}-\ref{eqn:definition_of_chi}) can be used to check if
a strategy is extortionate. The conditions correspond to:

\begin{align}
    p_1 & = \frac{(R-P)(p_2 + p_3) - R + T + S - P}{S + T - 2P}
     \label{eqn:condition_for_p1}\\
    p_4 & = 0 \label{eqn:condition_for_p4}\\
    1 & > p_2 + p_3\label{eqn:condition_for_chi}
\end{align}

The algebraic steps necessary to prove these results are available in the
supporting materials.

All extortionate strategies reside on a triangular (\ref{eqn:condition_for_chi})
plane (\ref{eqn:condition_for_p1}) in 3 dimensions (\ref{eqn:condition_for_p4}).
Using this formulation it can be seen that a necessary (but not sufficient)
condition for an extortionate strategy is that it cooperates on average less
than 50\% of the time when in a state of disagreement with the opponent.

As an example, consider the known extortionate strategy \(p=(8 / 9, 1 / 2, 1 /
3, 0)\) from~\cite{Stewart2012} which is referred to as \texttt{Extort-2}. In
this case, for the standard values of \((R, T, S, P)\) constraint
(\ref{eqn:condition_for_p1}) corresponds to:

\begin{equation}
    p_1 = \frac{2(p_2 + p_3) + 1}{3}
\end{equation}

It is clear that in this case all constraints hold.

This approach could in fact be used to confirm that a given strategy is acting
in an extortionate manner even if it is not a memory one strategy. However, in
practice, if a closed form for \(p\) is not known, then due to measurement
and/or numerical error this would not work.

This problem can be written in the following linear algebraic form where
\(x=(\alpha, \beta)\)
and \(p^*=(\tilde p_1 - 1, tilde_2 - 1, p_3)\):

\begin{equation}\label{eqn:linear_algebraic_equation_for_p}
    Cx= p^*
\end{equation}

\(C\) corresponds to equations
(\ref{eqn:condition_for_tilde_p1}-\ref{eqn:condition_for_tilde_p3}) and is
given by:

\begin{equation}\label{eqn:definition_of_C}
    C =
    \begin{bmatrix}
        R - P & R- P \\
        S - P & T- P \\
        T - P & S- P \\
    \end{bmatrix}
\end{equation}

Note that in general, equation (\ref{eqn:linear_algebraic_equation_for_p}) will
not necessarily have a solution. From the Rouch\'{e}-Capelli theorem if there is
a solution it is unique as \(\text{rank}(C)=2\) which is the dimension of the
variable \(x\). The best fitting \(x\) is found by minimizing:

\begin{equation}\label{eqn:r_squared}
    \text{SSError} = \|C x- p^*\|_2^2 = \sum_{i=1}^{3}\left((C\bar x)_i-p_i^*\right)^2
\end{equation}

Note that \(\text{SSError}\), which is the square of the Frobenius
norm~\cite{Golub2013}, becomes a measure of how close a strategy is to being an
extortionate strategy. Suspicion
of extortion then corresponds to a threshold on \(\text{SSError}\).

By observing interactions (human or otherwise), their memory one representation
can be inferred and this approach can be used to recognise extortionate
behaviour. The notion of comparing theoretic and actual plays of the IPD is not
novel, see for example~\cite{Rand2013}. Immediately it is noted that if the
environment is noisy~\cite{Wu1995} then no strategy can be considered to be
extortionate as \(p_4>0\).

In the next section, this idea will be illustrated by observing the interactions
that take place in a computer based tournament of the IPD\@.

\section{Numerical experiments}\label{sec:numerical-experiments}

In~\cite{Stewart2012} results from a tournament with
\input{./assets/tex/number_of_stewart_plotkin_strategies/main.tex} strategies,
was presented with specific consideration given to ZD strategies. This
tournament is reproduced here using the Axelrod-Python
project~\cite{Knight2016}. To obtain a good measure of the corresponding
transition rates for each strategy all matches have been run for
\input{assets/tex/number_of_turns/main.tex} turns and every match has been
repeated \input{assets/tex/number_of_repetitions/main.tex} times. All of this
interaction data is available at~\cite{vincent_knight_2018_1297075}. A good
match between the inferred Markov chain and the state distribution of the actual
interactions has been verified. Data for this is presented in the supplementary
materials.

Figure~\ref{fig:SSError_overall_in_stewart_plotkin} shows the \(\text{SSError}\)
values for all the strategies in the tournament, as reported
in~\cite{Stewart2012} the extortionate strategy (which has an expected
\(\text{SSError}\) approximately 0) gains a large number of wins.

\begin{figure}[!htbp]
    \centering
    \includegraphics[width=.8\textwidth]{./assets/img/SSError_overall_in_stewart_plotkin/main.pdf}
    \caption{\(\text{SSError}\) and state probabilities for the strategies
        of~\cite{Stewart2012}, ordered both by number of wins and overall score.
        Note that \(P(DC)\) is not shown as it corresponds to the transpose of
        \(P(CD)\). Cooperator and Defector are omitted as they do not visit all
        the states.}
    \label{fig:SSError_overall_in_stewart_plotkin}
\end{figure}

Here, the work of~\cite{Stewart2012} is extended by investigating a tournament
with \input{assets/tex/number_of_full_strategies/main.tex}
strategies.

The results of this analysis are shown in
Figure~\ref{fig:SSError_and_probabilities_in_full}. The top ranking strategies
by number of wins seem to be extortionate (but not against all strategies) and
it can be seen that a small sub group of strategies achieve mutual defection.
All the top ranking strategies according to score achieve mutual cooperation and
do not extort each other, however they
\textbf{do} exhibit extortionate behaviour towards a number of the lower ranking
strategies.

\begin{figure}[!htbp]
    \centering
    \includegraphics[width=.8\textwidth]{./assets/img/SSError_and_probabilities_in_full/main.pdf}
    \caption{\(\text{SSError}\) for the strategies for the full tournament. Only
    strategy interactions for which \(p_4=0\) and \(\chi>1\) are displayed.}
    \label{fig:SSError_and_probabilities_in_full}
\end{figure}

\section{Conclusion}\label{sec:conclusion}

This work defines an approach to measure whether or not a player is playing a
strategy that corresponds to an extortionate strategy as defined
in~\cite{Press2012}: a mathematical model for suspicion. Indeed, all
extortionate strategies have been
 classified as lying on a triangular plane.
This rigorous classification fails to be robust to small measurement error, thus
a statistical approach is proposed.
This is done through a linear algebraic approach for approximating the solution
of a linear system. Using this, a large number of pairwise interactions is
simulated and in fact very few strategies are found to act extortionately.

The work of~\cite{Press2012}, whilst showing that a clever approach to taking
advantage of another memory one strategy exists: this is incomplete. Whilst the
elegance of this result is very attractive, just as the simplicity of the
victory of Tit For Tat in Axelrod's original tournaments was, it is incomplete.
Extortionate strategies achieve a high number of wins but they do not
achieve a high score which corresponds to the fitness landscape in an
evolutionary sense. From the large number of interactions a payoff matrix \(S\)
can be measured where \(S_{ij}\) denotes the score (using standard values of
\((R, S, T, P) = (3, 0, 5, 1)\)) of the \(i\)th strategy
against the \(j\)th strategy. Using this, the replicator equation
describes the evolution of the system based on a population density fitness
function:

\begin{equation}\label{eqn:replicator_dynamics}
    \frac{dx}{dt} = x(S-x^TS x)
\end{equation}

Equation (\ref{eqn:replicator_dynamics}) is solved numerically through an
integration technique described in~\cite{Petzold1983} and
Figure~\ref{fig:replicator_dynamics} shows the evolution of the distribution of
the system: the various strategies are ranked by scores. It is clear to see that
only the high ranking strategies survive the evolutionary process (in fact,
only \input{./assets/img/replicator_dynamics/main.tex}
have a final distribution greater than \(10 ^ {-2}\)). This confirms the
findings of~\cite{Moran1707} in which sophisticated strategies resist
evolutionary invasion of shorter memory strategies. Recalling
Figure~\ref{fig:SSError_and_probabilities_in_full} this demonstrates that:

\begin{itemize}
    \item Cooperation emerges through the evolutionary process: the high scoring
        strategies do not exhibit extortionate behaviour towards each other.
    \item Extortionate strategies do not survive the evolutionary process.
\end{itemize}

\begin{figure}[!htbp]
    \centering
    \includegraphics[width=.8\textwidth]{./assets/img/replicator_dynamics/main.pdf}
    \caption{Numerical simulation of the replicator equation
    (\ref{eqn:replicator_dynamics}): strategies are ordered by score, only the strategies with a high score survive the evolutionary process.}
    \label{fig:replicator_dynamics}
\end{figure}

This work can be used to classify plays of the IPD\@: data can be collected from
actual interactions (in lab or in the field). Furthermore, this allows for a
classification method similar to the notion of fingerprinting presented
in~\cite{Ashlock2008}. Trained strategies can potentially be classified as
extortionate or not or it could be possible to even constrain the reinforcement
learning approaches that are becoming prevalent in the literature.
Alternatively, this mathematical approach for recognising extortion could be
used in sophisticated strategies to defend against invasion. Arguably, some of
the strategies considered here exhibit this behaviour, indeed as described
in~\cite{Harper2017}, the top ranking strategies in the full tournament are
obtained using evolutionary reinforcement learning techniques, thus, suspicion
of extortionate behaviour could in fact be an evolutionary trait.

\section*{Acknowledgements}

The following open source software libraries were used in this research:

\begin{itemize}
    \item The Axelrod ~\cite{Knight2016, Knight2018} library (IPD strategies and
        tournaments).
    \item The sympy library~\cite{Meurer2017} (verification of all symbolic
        calculations).
    \item The matplotlib~\cite{Droettboom2018} library (visualisation).
    \item The pandas~\cite{Structures2010}, dask~\cite{Dask2016} and
        NumPy~\cite{Oliphant2015} libraries (data manipulation).
    \item The SciPy~\cite{Jones2001} library (numerical integration of the
        replicator equation).
\end{itemize}

This work was performed using the computational facilities of the Advanced
Research Computing @ Cardiff (ARCCA) Division, Cardiff University.

\printbibliography

\newpage
\section*{Supplementary materials}

\includepdf{assets/pdf/proof_of_form_of_extortionate_strategies/main.pdf}

\newpage

Using the pair wise interactions the transition rates \(p,
q\) can be measured and the steady state probabilities inferred and compared to
the actual probabilities of each state.
This is done numerically by computing the singular eigenvector of the
matrix \(A\) \cite{Stewart2009}:

\[
    A =
    \begin{bmatrix}
        p_1 q_1 & p_1 (1 - q_1) & (1 - p_1) q_1 & (1 -p_1) (1 - q_1) \\
        p_2 q_2 & p_2 (1 - q_2) & (1 - p_2) q_2 & (1 -p_2) (1 - q_2) \\
        p_3 q_3 & p_3 (1 - q_3) & (1 - p_3) q_3 & (1 -p_3) (1 - q_3) \\
        p_4 q_4 & p_4 (1 - q_4) & (1 - p_4) q_4 & (1 -p_4) (1 - q_4) \\
    \end{bmatrix}
\]

Figure~\ref{fig:computed_probabilities_vs_theoretic_probabilities} shows a
regression line fitted to every pairwise interaction with a reported
\(\text{SSError}\) value (pairwise interactions with missing states were
omitted). This serves to validate the approach: a part from some edge cases the
relationship is consistent.

\begin{figure}[!htbp]
    \centering
    \includegraphics[width=.8\textwidth]{./assets/img/computed_probabilities_vs_theoretic_probabilities/main.pdf}
    \caption{The
        relationship between the steady state probabilities inferred from the
        measured transitions and the actual steady state probabilities. A linear
        regression line is included validating the approach.}
    \label{fig:computed_probabilities_vs_theoretic_probabilities}
\end{figure}


\end{document}
 turns and every match has been
repeated \documentclass[a4paper]{article}

\usepackage{amsmath}
\usepackage{amssymb}
\usepackage[margin=1.5cm,
            includefoot,
            footskip=30pt]{geometry}
\usepackage{layout}
\usepackage{graphicx}
\usepackage{subcaption}

\usepackage{biblatex}
\usepackage{pdfpages}

\bibliography{main.bib}

\title{Suspicion: Recognising and evaluating the effectiveness
       of extortion in the Iterated Prisoner's Dilemma}
\author{Vincent A. Knight \and Nikoleta E. Glynatsi}
\date{\today}



\begin{document}

\maketitle

\begin{abstract}
    The Iterated Prisoner's Dilemma is a model for rational and evolutionary
    interactive behaviour. It has applications both in the study of human social
    behaviour as well as in biology.
    It is used to understand when and how a rational individual might
    accept an immediate cost to their own utility for the direct benefit of
    another.

    Much attention has been given to a class of strategies called
    Zero Determinant strategies. It has been theoretically shown that these
    strategies can ``extort'' any player.

    In this work, an approach to identify if observed strategies are playing in
    an extortionate way is described. Furthermore, experimental analysis of
    a large tournament with \input{assets/tex/number_of_full_strategies/main.tex}
    strategies is considered. In this setting
    the most highly performing strategies do not play in an extortionate way
    against each other but do against lower performing strategies.
    This suggests that whilst the theory of Zero Determinant strategies
    indicates that memory is not of fundamental importance to the evolution of
    cooperative behaviour, this is incomplete.
\end{abstract}

\section{Introduction}\label{sec:introduction}

Agent based game theoretic models have become a stalwart of the underpinning
mathematics of interactive behaviours. One of the major pieces of work
in this area is the pair of original computer tournaments run by Robert
Axelrod~\cite{Axelrod1980, Axelrod1980a}. These tournaments pitted submitted
computer strategies against each other in plays of the Iterated Prisoner's
Dilemma. A common game where agents can choose to pay a slight cost to their
immediate utility in the hope of building a reputation. This has been used in
economic and evolutionary game theory to understand the evolution of cooperative
behaviour.

Recently, a class of strategies was described in~\cite{Press2012} that can
provably extort any given opponent. In~\cite{Hilbe2013, Moran1707} some
questions have already been asked about the true effectiveness of these
strategies in an evolutionary setting. Here another question is asked: is it
possible to recognise this extortionate behaviour? A mathematical procedure for
suspicion is presented: in the same way that the continued actions of an
extortionate individual might raise suspicion.

This work makes use of the Axelrod Python library~\cite{Knight2018, Knight2016}
with a large number of Prisoner Dilemma strategies available to give an
extensive numerical example of the ideas presented.  The approach is presented
in Section~\ref{sec:delta-zd-strategies}.  All of the code and data discussed
in Section~\ref{sec:numerical-experiments} is open sourced, archived and
written according to best scientific principles~\cite{Wilson2014}. The data
archive can be found at~\cite{vincent_knight_2018_1297075}.

\section{Recognising Extortion}\label{sec:delta-zd-strategies}

In~\cite{Press2012}, given a match between 2 memory-one strategies, the concept
of Zero Determinant (ZD) strategies is introduced. The main result of that paper
shows that given two memory one players \(p, q\in\mathbb{R}^4\) a linear
relationship between the players' scores could be forced by one of the players.

Using the notation of~\cite{Press2012}, assuming the utilities for player \(p\)
are given by \(S_x=(R, S, T, P)\) and for player \(q\) by \(S_y=(R, T, S, P)\)
and that the stationary scores of each player is given by \(S_X\) and \(S_Y\)
respectively. The main result of~\cite{Press2012} is that if

\begin{equation}\label{eqn:linear_relationship_for_p}
    \tilde p=\alpha S_x + \beta S_y + \gamma
\end{equation}

or

\begin{equation}\label{eqn:linear_relationship_for_q}
    \tilde q=\alpha S_x + \beta S_y + \gamma
\end{equation}

where \(\tilde p = (1 - p_1, 1 - p_2, p_3, p_4)\) and
\(\tilde q = (1 - q_1, 1 - q_2, q_3, q_4)\) then:

\begin{equation}
    \alpha S_X + \beta S_Y + \gamma = 0
\end{equation}

In~\cite{Press2012} a particular type of ZD strategy is defined: extortionate
strategies. If:

\begin{equation}\label{eqn:constraint_for_extortion}
    \gamma = - P(\alpha + \beta)
\end{equation}

then the player can ensure they get a score \(\chi\) times
larger than the opponent. This extortion coefficient is given by:

\begin{equation}\label{eqn:definition_of_chi}
    \chi=\frac{-\beta}{\alpha}
\end{equation}

Thus, if (\ref{eqn:constraint_for_extortion}) holds and \(\chi >1\) a player is
said to extort their opponent.
Here, the reverse problem is considered: given a
\(p\in\mathbb{R}^4\) how does one identify \(\alpha, \beta\) if they
exist and is the strategy in fact acting in an extortionate way?

These conditions correspond to:

\begin{align}
    \tilde p_1 & = \alpha R + \beta R - P (\alpha + \beta)
            \label{eqn:condition_for_tilde_p1}\\
    \tilde p_2 & = \alpha S + \beta T - P (\alpha + \beta)
            \label{eqn:condition_for_tilde_p2}\\
    \tilde p_3 & = \alpha T + \beta S - P (\alpha + \beta)
            \label{eqn:condition_for_tilde_p3}\\
    \tilde p_4 & = \alpha P + \beta P - P (\alpha + \beta)
            \label{eqn:condition_for_tilde_p4}
\end{align}

Equation (\ref{eqn:condition_for_tilde_p4}) ensures that \(p_4=\tilde p_4=0\).
Equations (\ref{eqn:condition_for_tilde_p1}-\ref{eqn:condition_for_tilde_p3})
can be used to eliminate \(\alpha, \beta\), giving:

\begin{equation}\label{eqn:planar_definition_of_extortion}
    \tilde p_1 = \frac{(R - P)(\tilde p_2 + \tilde p_3)}{S + T - 2P}
\end{equation}

with:

\begin{equation}\label{eqn:definition_of_chi}
    \chi = \frac{\tilde p_2 (P - T) + \tilde p_3 (S - P)}
                {\tilde p_2 (P - S) + \tilde p_3 (T - P)}
\end{equation}

Given a strategy \(p\in\mathbb{R}^{4\times 1}\) equations
(\ref{eqn:condition_for_tilde_p4}), (\ref{eqn:planar_definition_of_extortion}-\ref{eqn:definition_of_chi}) can be used to check if
a strategy is extortionate. The conditions correspond to:

\begin{align}
    p_1 & = \frac{(R-P)(p_2 + p_3) - R + T + S - P}{S + T - 2P}
     \label{eqn:condition_for_p1}\\
    p_4 & = 0 \label{eqn:condition_for_p4}\\
    1 & > p_2 + p_3\label{eqn:condition_for_chi}
\end{align}

The algebraic steps necessary to prove these results are available in the
supporting materials.

All extortionate strategies reside on a triangular (\ref{eqn:condition_for_chi})
plane (\ref{eqn:condition_for_p1}) in 3 dimensions (\ref{eqn:condition_for_p4}).
Using this formulation it can be seen that a necessary (but not sufficient)
condition for an extortionate strategy is that it cooperates on average less
than 50\% of the time when in a state of disagreement with the opponent.

As an example, consider the known extortionate strategy \(p=(8 / 9, 1 / 2, 1 /
3, 0)\) from~\cite{Stewart2012} which is referred to as \texttt{Extort-2}. In
this case, for the standard values of \((R, T, S, P)\) constraint
(\ref{eqn:condition_for_p1}) corresponds to:

\begin{equation}
    p_1 = \frac{2(p_2 + p_3) + 1}{3}
\end{equation}

It is clear that in this case all constraints hold.

This approach could in fact be used to confirm that a given strategy is acting
in an extortionate manner even if it is not a memory one strategy. However, in
practice, if a closed form for \(p\) is not known, then due to measurement
and/or numerical error this would not work.

This problem can be written in the following linear algebraic form where
\(x=(\alpha, \beta)\)
and \(p^*=(\tilde p_1 - 1, tilde_2 - 1, p_3)\):

\begin{equation}\label{eqn:linear_algebraic_equation_for_p}
    Cx= p^*
\end{equation}

\(C\) corresponds to equations
(\ref{eqn:condition_for_tilde_p1}-\ref{eqn:condition_for_tilde_p3}) and is
given by:

\begin{equation}\label{eqn:definition_of_C}
    C =
    \begin{bmatrix}
        R - P & R- P \\
        S - P & T- P \\
        T - P & S- P \\
    \end{bmatrix}
\end{equation}

Note that in general, equation (\ref{eqn:linear_algebraic_equation_for_p}) will
not necessarily have a solution. From the Rouch\'{e}-Capelli theorem if there is
a solution it is unique as \(\text{rank}(C)=2\) which is the dimension of the
variable \(x\). The best fitting \(x\) is found by minimizing:

\begin{equation}\label{eqn:r_squared}
    \text{SSError} = \|C x- p^*\|_2^2 = \sum_{i=1}^{3}\left((C\bar x)_i-p_i^*\right)^2
\end{equation}

Note that \(\text{SSError}\), which is the square of the Frobenius
norm~\cite{Golub2013}, becomes a measure of how close a strategy is to being an
extortionate strategy. Suspicion
of extortion then corresponds to a threshold on \(\text{SSError}\).

By observing interactions (human or otherwise), their memory one representation
can be inferred and this approach can be used to recognise extortionate
behaviour. The notion of comparing theoretic and actual plays of the IPD is not
novel, see for example~\cite{Rand2013}. Immediately it is noted that if the
environment is noisy~\cite{Wu1995} then no strategy can be considered to be
extortionate as \(p_4>0\).

In the next section, this idea will be illustrated by observing the interactions
that take place in a computer based tournament of the IPD\@.

\section{Numerical experiments}\label{sec:numerical-experiments}

In~\cite{Stewart2012} results from a tournament with
\input{./assets/tex/number_of_stewart_plotkin_strategies/main.tex} strategies,
was presented with specific consideration given to ZD strategies. This
tournament is reproduced here using the Axelrod-Python
project~\cite{Knight2016}. To obtain a good measure of the corresponding
transition rates for each strategy all matches have been run for
\input{assets/tex/number_of_turns/main.tex} turns and every match has been
repeated \input{assets/tex/number_of_repetitions/main.tex} times. All of this
interaction data is available at~\cite{vincent_knight_2018_1297075}. A good
match between the inferred Markov chain and the state distribution of the actual
interactions has been verified. Data for this is presented in the supplementary
materials.

Figure~\ref{fig:SSError_overall_in_stewart_plotkin} shows the \(\text{SSError}\)
values for all the strategies in the tournament, as reported
in~\cite{Stewart2012} the extortionate strategy (which has an expected
\(\text{SSError}\) approximately 0) gains a large number of wins.

\begin{figure}[!htbp]
    \centering
    \includegraphics[width=.8\textwidth]{./assets/img/SSError_overall_in_stewart_plotkin/main.pdf}
    \caption{\(\text{SSError}\) and state probabilities for the strategies
        of~\cite{Stewart2012}, ordered both by number of wins and overall score.
        Note that \(P(DC)\) is not shown as it corresponds to the transpose of
        \(P(CD)\). Cooperator and Defector are omitted as they do not visit all
        the states.}
    \label{fig:SSError_overall_in_stewart_plotkin}
\end{figure}

Here, the work of~\cite{Stewart2012} is extended by investigating a tournament
with \input{assets/tex/number_of_full_strategies/main.tex}
strategies.

The results of this analysis are shown in
Figure~\ref{fig:SSError_and_probabilities_in_full}. The top ranking strategies
by number of wins seem to be extortionate (but not against all strategies) and
it can be seen that a small sub group of strategies achieve mutual defection.
All the top ranking strategies according to score achieve mutual cooperation and
do not extort each other, however they
\textbf{do} exhibit extortionate behaviour towards a number of the lower ranking
strategies.

\begin{figure}[!htbp]
    \centering
    \includegraphics[width=.8\textwidth]{./assets/img/SSError_and_probabilities_in_full/main.pdf}
    \caption{\(\text{SSError}\) for the strategies for the full tournament. Only
    strategy interactions for which \(p_4=0\) and \(\chi>1\) are displayed.}
    \label{fig:SSError_and_probabilities_in_full}
\end{figure}

\section{Conclusion}\label{sec:conclusion}

This work defines an approach to measure whether or not a player is playing a
strategy that corresponds to an extortionate strategy as defined
in~\cite{Press2012}: a mathematical model for suspicion. Indeed, all
extortionate strategies have been
 classified as lying on a triangular plane.
This rigorous classification fails to be robust to small measurement error, thus
a statistical approach is proposed.
This is done through a linear algebraic approach for approximating the solution
of a linear system. Using this, a large number of pairwise interactions is
simulated and in fact very few strategies are found to act extortionately.

The work of~\cite{Press2012}, whilst showing that a clever approach to taking
advantage of another memory one strategy exists: this is incomplete. Whilst the
elegance of this result is very attractive, just as the simplicity of the
victory of Tit For Tat in Axelrod's original tournaments was, it is incomplete.
Extortionate strategies achieve a high number of wins but they do not
achieve a high score which corresponds to the fitness landscape in an
evolutionary sense. From the large number of interactions a payoff matrix \(S\)
can be measured where \(S_{ij}\) denotes the score (using standard values of
\((R, S, T, P) = (3, 0, 5, 1)\)) of the \(i\)th strategy
against the \(j\)th strategy. Using this, the replicator equation
describes the evolution of the system based on a population density fitness
function:

\begin{equation}\label{eqn:replicator_dynamics}
    \frac{dx}{dt} = x(S-x^TS x)
\end{equation}

Equation (\ref{eqn:replicator_dynamics}) is solved numerically through an
integration technique described in~\cite{Petzold1983} and
Figure~\ref{fig:replicator_dynamics} shows the evolution of the distribution of
the system: the various strategies are ranked by scores. It is clear to see that
only the high ranking strategies survive the evolutionary process (in fact,
only \input{./assets/img/replicator_dynamics/main.tex}
have a final distribution greater than \(10 ^ {-2}\)). This confirms the
findings of~\cite{Moran1707} in which sophisticated strategies resist
evolutionary invasion of shorter memory strategies. Recalling
Figure~\ref{fig:SSError_and_probabilities_in_full} this demonstrates that:

\begin{itemize}
    \item Cooperation emerges through the evolutionary process: the high scoring
        strategies do not exhibit extortionate behaviour towards each other.
    \item Extortionate strategies do not survive the evolutionary process.
\end{itemize}

\begin{figure}[!htbp]
    \centering
    \includegraphics[width=.8\textwidth]{./assets/img/replicator_dynamics/main.pdf}
    \caption{Numerical simulation of the replicator equation
    (\ref{eqn:replicator_dynamics}): strategies are ordered by score, only the strategies with a high score survive the evolutionary process.}
    \label{fig:replicator_dynamics}
\end{figure}

This work can be used to classify plays of the IPD\@: data can be collected from
actual interactions (in lab or in the field). Furthermore, this allows for a
classification method similar to the notion of fingerprinting presented
in~\cite{Ashlock2008}. Trained strategies can potentially be classified as
extortionate or not or it could be possible to even constrain the reinforcement
learning approaches that are becoming prevalent in the literature.
Alternatively, this mathematical approach for recognising extortion could be
used in sophisticated strategies to defend against invasion. Arguably, some of
the strategies considered here exhibit this behaviour, indeed as described
in~\cite{Harper2017}, the top ranking strategies in the full tournament are
obtained using evolutionary reinforcement learning techniques, thus, suspicion
of extortionate behaviour could in fact be an evolutionary trait.

\section*{Acknowledgements}

The following open source software libraries were used in this research:

\begin{itemize}
    \item The Axelrod ~\cite{Knight2016, Knight2018} library (IPD strategies and
        tournaments).
    \item The sympy library~\cite{Meurer2017} (verification of all symbolic
        calculations).
    \item The matplotlib~\cite{Droettboom2018} library (visualisation).
    \item The pandas~\cite{Structures2010}, dask~\cite{Dask2016} and
        NumPy~\cite{Oliphant2015} libraries (data manipulation).
    \item The SciPy~\cite{Jones2001} library (numerical integration of the
        replicator equation).
\end{itemize}

This work was performed using the computational facilities of the Advanced
Research Computing @ Cardiff (ARCCA) Division, Cardiff University.

\printbibliography

\newpage
\section*{Supplementary materials}

\includepdf{assets/pdf/proof_of_form_of_extortionate_strategies/main.pdf}

\newpage

Using the pair wise interactions the transition rates \(p,
q\) can be measured and the steady state probabilities inferred and compared to
the actual probabilities of each state.
This is done numerically by computing the singular eigenvector of the
matrix \(A\) \cite{Stewart2009}:

\[
    A =
    \begin{bmatrix}
        p_1 q_1 & p_1 (1 - q_1) & (1 - p_1) q_1 & (1 -p_1) (1 - q_1) \\
        p_2 q_2 & p_2 (1 - q_2) & (1 - p_2) q_2 & (1 -p_2) (1 - q_2) \\
        p_3 q_3 & p_3 (1 - q_3) & (1 - p_3) q_3 & (1 -p_3) (1 - q_3) \\
        p_4 q_4 & p_4 (1 - q_4) & (1 - p_4) q_4 & (1 -p_4) (1 - q_4) \\
    \end{bmatrix}
\]

Figure~\ref{fig:computed_probabilities_vs_theoretic_probabilities} shows a
regression line fitted to every pairwise interaction with a reported
\(\text{SSError}\) value (pairwise interactions with missing states were
omitted). This serves to validate the approach: a part from some edge cases the
relationship is consistent.

\begin{figure}[!htbp]
    \centering
    \includegraphics[width=.8\textwidth]{./assets/img/computed_probabilities_vs_theoretic_probabilities/main.pdf}
    \caption{The
        relationship between the steady state probabilities inferred from the
        measured transitions and the actual steady state probabilities. A linear
        regression line is included validating the approach.}
    \label{fig:computed_probabilities_vs_theoretic_probabilities}
\end{figure}


\end{document}
 times. All of this
interaction data is available at~\cite{vincent_knight_2018_1297075}. A good
match between the inferred Markov chain and the state distribution of the actual
interactions has been verified. Data for this is presented in the supplementary
materials.

Figure~\ref{fig:SSError_overall_in_stewart_plotkin} shows the \(\text{SSError}\)
values for all the strategies in the tournament, as reported
in~\cite{Stewart2012} the extortionate strategy (which has an expected
\(\text{SSError}\) approximately 0) gains a large number of wins.

\begin{figure}[!htbp]
    \centering
    \includegraphics[width=.8\textwidth]{./assets/img/SSError_overall_in_stewart_plotkin/main.pdf}
    \caption{\(\text{SSError}\) and state probabilities for the strategies
        of~\cite{Stewart2012}, ordered both by number of wins and overall score.
        Note that \(P(DC)\) is not shown as it corresponds to the transpose of
        \(P(CD)\). Cooperator and Defector are omitted as they do not visit all
        the states.}
    \label{fig:SSError_overall_in_stewart_plotkin}
\end{figure}

Here, the work of~\cite{Stewart2012} is extended by investigating a tournament
with \documentclass[a4paper]{article}

\usepackage{amsmath}
\usepackage{amssymb}
\usepackage[margin=1.5cm,
            includefoot,
            footskip=30pt]{geometry}
\usepackage{layout}
\usepackage{graphicx}
\usepackage{subcaption}

\usepackage{biblatex}
\usepackage{pdfpages}

\bibliography{main.bib}

\title{Suspicion: Recognising and evaluating the effectiveness
       of extortion in the Iterated Prisoner's Dilemma}
\author{Vincent A. Knight \and Nikoleta E. Glynatsi}
\date{\today}



\begin{document}

\maketitle

\begin{abstract}
    The Iterated Prisoner's Dilemma is a model for rational and evolutionary
    interactive behaviour. It has applications both in the study of human social
    behaviour as well as in biology.
    It is used to understand when and how a rational individual might
    accept an immediate cost to their own utility for the direct benefit of
    another.

    Much attention has been given to a class of strategies called
    Zero Determinant strategies. It has been theoretically shown that these
    strategies can ``extort'' any player.

    In this work, an approach to identify if observed strategies are playing in
    an extortionate way is described. Furthermore, experimental analysis of
    a large tournament with \input{assets/tex/number_of_full_strategies/main.tex}
    strategies is considered. In this setting
    the most highly performing strategies do not play in an extortionate way
    against each other but do against lower performing strategies.
    This suggests that whilst the theory of Zero Determinant strategies
    indicates that memory is not of fundamental importance to the evolution of
    cooperative behaviour, this is incomplete.
\end{abstract}

\section{Introduction}\label{sec:introduction}

Agent based game theoretic models have become a stalwart of the underpinning
mathematics of interactive behaviours. One of the major pieces of work
in this area is the pair of original computer tournaments run by Robert
Axelrod~\cite{Axelrod1980, Axelrod1980a}. These tournaments pitted submitted
computer strategies against each other in plays of the Iterated Prisoner's
Dilemma. A common game where agents can choose to pay a slight cost to their
immediate utility in the hope of building a reputation. This has been used in
economic and evolutionary game theory to understand the evolution of cooperative
behaviour.

Recently, a class of strategies was described in~\cite{Press2012} that can
provably extort any given opponent. In~\cite{Hilbe2013, Moran1707} some
questions have already been asked about the true effectiveness of these
strategies in an evolutionary setting. Here another question is asked: is it
possible to recognise this extortionate behaviour? A mathematical procedure for
suspicion is presented: in the same way that the continued actions of an
extortionate individual might raise suspicion.

This work makes use of the Axelrod Python library~\cite{Knight2018, Knight2016}
with a large number of Prisoner Dilemma strategies available to give an
extensive numerical example of the ideas presented.  The approach is presented
in Section~\ref{sec:delta-zd-strategies}.  All of the code and data discussed
in Section~\ref{sec:numerical-experiments} is open sourced, archived and
written according to best scientific principles~\cite{Wilson2014}. The data
archive can be found at~\cite{vincent_knight_2018_1297075}.

\section{Recognising Extortion}\label{sec:delta-zd-strategies}

In~\cite{Press2012}, given a match between 2 memory-one strategies, the concept
of Zero Determinant (ZD) strategies is introduced. The main result of that paper
shows that given two memory one players \(p, q\in\mathbb{R}^4\) a linear
relationship between the players' scores could be forced by one of the players.

Using the notation of~\cite{Press2012}, assuming the utilities for player \(p\)
are given by \(S_x=(R, S, T, P)\) and for player \(q\) by \(S_y=(R, T, S, P)\)
and that the stationary scores of each player is given by \(S_X\) and \(S_Y\)
respectively. The main result of~\cite{Press2012} is that if

\begin{equation}\label{eqn:linear_relationship_for_p}
    \tilde p=\alpha S_x + \beta S_y + \gamma
\end{equation}

or

\begin{equation}\label{eqn:linear_relationship_for_q}
    \tilde q=\alpha S_x + \beta S_y + \gamma
\end{equation}

where \(\tilde p = (1 - p_1, 1 - p_2, p_3, p_4)\) and
\(\tilde q = (1 - q_1, 1 - q_2, q_3, q_4)\) then:

\begin{equation}
    \alpha S_X + \beta S_Y + \gamma = 0
\end{equation}

In~\cite{Press2012} a particular type of ZD strategy is defined: extortionate
strategies. If:

\begin{equation}\label{eqn:constraint_for_extortion}
    \gamma = - P(\alpha + \beta)
\end{equation}

then the player can ensure they get a score \(\chi\) times
larger than the opponent. This extortion coefficient is given by:

\begin{equation}\label{eqn:definition_of_chi}
    \chi=\frac{-\beta}{\alpha}
\end{equation}

Thus, if (\ref{eqn:constraint_for_extortion}) holds and \(\chi >1\) a player is
said to extort their opponent.
Here, the reverse problem is considered: given a
\(p\in\mathbb{R}^4\) how does one identify \(\alpha, \beta\) if they
exist and is the strategy in fact acting in an extortionate way?

These conditions correspond to:

\begin{align}
    \tilde p_1 & = \alpha R + \beta R - P (\alpha + \beta)
            \label{eqn:condition_for_tilde_p1}\\
    \tilde p_2 & = \alpha S + \beta T - P (\alpha + \beta)
            \label{eqn:condition_for_tilde_p2}\\
    \tilde p_3 & = \alpha T + \beta S - P (\alpha + \beta)
            \label{eqn:condition_for_tilde_p3}\\
    \tilde p_4 & = \alpha P + \beta P - P (\alpha + \beta)
            \label{eqn:condition_for_tilde_p4}
\end{align}

Equation (\ref{eqn:condition_for_tilde_p4}) ensures that \(p_4=\tilde p_4=0\).
Equations (\ref{eqn:condition_for_tilde_p1}-\ref{eqn:condition_for_tilde_p3})
can be used to eliminate \(\alpha, \beta\), giving:

\begin{equation}\label{eqn:planar_definition_of_extortion}
    \tilde p_1 = \frac{(R - P)(\tilde p_2 + \tilde p_3)}{S + T - 2P}
\end{equation}

with:

\begin{equation}\label{eqn:definition_of_chi}
    \chi = \frac{\tilde p_2 (P - T) + \tilde p_3 (S - P)}
                {\tilde p_2 (P - S) + \tilde p_3 (T - P)}
\end{equation}

Given a strategy \(p\in\mathbb{R}^{4\times 1}\) equations
(\ref{eqn:condition_for_tilde_p4}), (\ref{eqn:planar_definition_of_extortion}-\ref{eqn:definition_of_chi}) can be used to check if
a strategy is extortionate. The conditions correspond to:

\begin{align}
    p_1 & = \frac{(R-P)(p_2 + p_3) - R + T + S - P}{S + T - 2P}
     \label{eqn:condition_for_p1}\\
    p_4 & = 0 \label{eqn:condition_for_p4}\\
    1 & > p_2 + p_3\label{eqn:condition_for_chi}
\end{align}

The algebraic steps necessary to prove these results are available in the
supporting materials.

All extortionate strategies reside on a triangular (\ref{eqn:condition_for_chi})
plane (\ref{eqn:condition_for_p1}) in 3 dimensions (\ref{eqn:condition_for_p4}).
Using this formulation it can be seen that a necessary (but not sufficient)
condition for an extortionate strategy is that it cooperates on average less
than 50\% of the time when in a state of disagreement with the opponent.

As an example, consider the known extortionate strategy \(p=(8 / 9, 1 / 2, 1 /
3, 0)\) from~\cite{Stewart2012} which is referred to as \texttt{Extort-2}. In
this case, for the standard values of \((R, T, S, P)\) constraint
(\ref{eqn:condition_for_p1}) corresponds to:

\begin{equation}
    p_1 = \frac{2(p_2 + p_3) + 1}{3}
\end{equation}

It is clear that in this case all constraints hold.

This approach could in fact be used to confirm that a given strategy is acting
in an extortionate manner even if it is not a memory one strategy. However, in
practice, if a closed form for \(p\) is not known, then due to measurement
and/or numerical error this would not work.

This problem can be written in the following linear algebraic form where
\(x=(\alpha, \beta)\)
and \(p^*=(\tilde p_1 - 1, tilde_2 - 1, p_3)\):

\begin{equation}\label{eqn:linear_algebraic_equation_for_p}
    Cx= p^*
\end{equation}

\(C\) corresponds to equations
(\ref{eqn:condition_for_tilde_p1}-\ref{eqn:condition_for_tilde_p3}) and is
given by:

\begin{equation}\label{eqn:definition_of_C}
    C =
    \begin{bmatrix}
        R - P & R- P \\
        S - P & T- P \\
        T - P & S- P \\
    \end{bmatrix}
\end{equation}

Note that in general, equation (\ref{eqn:linear_algebraic_equation_for_p}) will
not necessarily have a solution. From the Rouch\'{e}-Capelli theorem if there is
a solution it is unique as \(\text{rank}(C)=2\) which is the dimension of the
variable \(x\). The best fitting \(x\) is found by minimizing:

\begin{equation}\label{eqn:r_squared}
    \text{SSError} = \|C x- p^*\|_2^2 = \sum_{i=1}^{3}\left((C\bar x)_i-p_i^*\right)^2
\end{equation}

Note that \(\text{SSError}\), which is the square of the Frobenius
norm~\cite{Golub2013}, becomes a measure of how close a strategy is to being an
extortionate strategy. Suspicion
of extortion then corresponds to a threshold on \(\text{SSError}\).

By observing interactions (human or otherwise), their memory one representation
can be inferred and this approach can be used to recognise extortionate
behaviour. The notion of comparing theoretic and actual plays of the IPD is not
novel, see for example~\cite{Rand2013}. Immediately it is noted that if the
environment is noisy~\cite{Wu1995} then no strategy can be considered to be
extortionate as \(p_4>0\).

In the next section, this idea will be illustrated by observing the interactions
that take place in a computer based tournament of the IPD\@.

\section{Numerical experiments}\label{sec:numerical-experiments}

In~\cite{Stewart2012} results from a tournament with
\input{./assets/tex/number_of_stewart_plotkin_strategies/main.tex} strategies,
was presented with specific consideration given to ZD strategies. This
tournament is reproduced here using the Axelrod-Python
project~\cite{Knight2016}. To obtain a good measure of the corresponding
transition rates for each strategy all matches have been run for
\input{assets/tex/number_of_turns/main.tex} turns and every match has been
repeated \input{assets/tex/number_of_repetitions/main.tex} times. All of this
interaction data is available at~\cite{vincent_knight_2018_1297075}. A good
match between the inferred Markov chain and the state distribution of the actual
interactions has been verified. Data for this is presented in the supplementary
materials.

Figure~\ref{fig:SSError_overall_in_stewart_plotkin} shows the \(\text{SSError}\)
values for all the strategies in the tournament, as reported
in~\cite{Stewart2012} the extortionate strategy (which has an expected
\(\text{SSError}\) approximately 0) gains a large number of wins.

\begin{figure}[!htbp]
    \centering
    \includegraphics[width=.8\textwidth]{./assets/img/SSError_overall_in_stewart_plotkin/main.pdf}
    \caption{\(\text{SSError}\) and state probabilities for the strategies
        of~\cite{Stewart2012}, ordered both by number of wins and overall score.
        Note that \(P(DC)\) is not shown as it corresponds to the transpose of
        \(P(CD)\). Cooperator and Defector are omitted as they do not visit all
        the states.}
    \label{fig:SSError_overall_in_stewart_plotkin}
\end{figure}

Here, the work of~\cite{Stewart2012} is extended by investigating a tournament
with \input{assets/tex/number_of_full_strategies/main.tex}
strategies.

The results of this analysis are shown in
Figure~\ref{fig:SSError_and_probabilities_in_full}. The top ranking strategies
by number of wins seem to be extortionate (but not against all strategies) and
it can be seen that a small sub group of strategies achieve mutual defection.
All the top ranking strategies according to score achieve mutual cooperation and
do not extort each other, however they
\textbf{do} exhibit extortionate behaviour towards a number of the lower ranking
strategies.

\begin{figure}[!htbp]
    \centering
    \includegraphics[width=.8\textwidth]{./assets/img/SSError_and_probabilities_in_full/main.pdf}
    \caption{\(\text{SSError}\) for the strategies for the full tournament. Only
    strategy interactions for which \(p_4=0\) and \(\chi>1\) are displayed.}
    \label{fig:SSError_and_probabilities_in_full}
\end{figure}

\section{Conclusion}\label{sec:conclusion}

This work defines an approach to measure whether or not a player is playing a
strategy that corresponds to an extortionate strategy as defined
in~\cite{Press2012}: a mathematical model for suspicion. Indeed, all
extortionate strategies have been
 classified as lying on a triangular plane.
This rigorous classification fails to be robust to small measurement error, thus
a statistical approach is proposed.
This is done through a linear algebraic approach for approximating the solution
of a linear system. Using this, a large number of pairwise interactions is
simulated and in fact very few strategies are found to act extortionately.

The work of~\cite{Press2012}, whilst showing that a clever approach to taking
advantage of another memory one strategy exists: this is incomplete. Whilst the
elegance of this result is very attractive, just as the simplicity of the
victory of Tit For Tat in Axelrod's original tournaments was, it is incomplete.
Extortionate strategies achieve a high number of wins but they do not
achieve a high score which corresponds to the fitness landscape in an
evolutionary sense. From the large number of interactions a payoff matrix \(S\)
can be measured where \(S_{ij}\) denotes the score (using standard values of
\((R, S, T, P) = (3, 0, 5, 1)\)) of the \(i\)th strategy
against the \(j\)th strategy. Using this, the replicator equation
describes the evolution of the system based on a population density fitness
function:

\begin{equation}\label{eqn:replicator_dynamics}
    \frac{dx}{dt} = x(S-x^TS x)
\end{equation}

Equation (\ref{eqn:replicator_dynamics}) is solved numerically through an
integration technique described in~\cite{Petzold1983} and
Figure~\ref{fig:replicator_dynamics} shows the evolution of the distribution of
the system: the various strategies are ranked by scores. It is clear to see that
only the high ranking strategies survive the evolutionary process (in fact,
only \input{./assets/img/replicator_dynamics/main.tex}
have a final distribution greater than \(10 ^ {-2}\)). This confirms the
findings of~\cite{Moran1707} in which sophisticated strategies resist
evolutionary invasion of shorter memory strategies. Recalling
Figure~\ref{fig:SSError_and_probabilities_in_full} this demonstrates that:

\begin{itemize}
    \item Cooperation emerges through the evolutionary process: the high scoring
        strategies do not exhibit extortionate behaviour towards each other.
    \item Extortionate strategies do not survive the evolutionary process.
\end{itemize}

\begin{figure}[!htbp]
    \centering
    \includegraphics[width=.8\textwidth]{./assets/img/replicator_dynamics/main.pdf}
    \caption{Numerical simulation of the replicator equation
    (\ref{eqn:replicator_dynamics}): strategies are ordered by score, only the strategies with a high score survive the evolutionary process.}
    \label{fig:replicator_dynamics}
\end{figure}

This work can be used to classify plays of the IPD\@: data can be collected from
actual interactions (in lab or in the field). Furthermore, this allows for a
classification method similar to the notion of fingerprinting presented
in~\cite{Ashlock2008}. Trained strategies can potentially be classified as
extortionate or not or it could be possible to even constrain the reinforcement
learning approaches that are becoming prevalent in the literature.
Alternatively, this mathematical approach for recognising extortion could be
used in sophisticated strategies to defend against invasion. Arguably, some of
the strategies considered here exhibit this behaviour, indeed as described
in~\cite{Harper2017}, the top ranking strategies in the full tournament are
obtained using evolutionary reinforcement learning techniques, thus, suspicion
of extortionate behaviour could in fact be an evolutionary trait.

\section*{Acknowledgements}

The following open source software libraries were used in this research:

\begin{itemize}
    \item The Axelrod ~\cite{Knight2016, Knight2018} library (IPD strategies and
        tournaments).
    \item The sympy library~\cite{Meurer2017} (verification of all symbolic
        calculations).
    \item The matplotlib~\cite{Droettboom2018} library (visualisation).
    \item The pandas~\cite{Structures2010}, dask~\cite{Dask2016} and
        NumPy~\cite{Oliphant2015} libraries (data manipulation).
    \item The SciPy~\cite{Jones2001} library (numerical integration of the
        replicator equation).
\end{itemize}

This work was performed using the computational facilities of the Advanced
Research Computing @ Cardiff (ARCCA) Division, Cardiff University.

\printbibliography

\newpage
\section*{Supplementary materials}

\includepdf{assets/pdf/proof_of_form_of_extortionate_strategies/main.pdf}

\newpage

Using the pair wise interactions the transition rates \(p,
q\) can be measured and the steady state probabilities inferred and compared to
the actual probabilities of each state.
This is done numerically by computing the singular eigenvector of the
matrix \(A\) \cite{Stewart2009}:

\[
    A =
    \begin{bmatrix}
        p_1 q_1 & p_1 (1 - q_1) & (1 - p_1) q_1 & (1 -p_1) (1 - q_1) \\
        p_2 q_2 & p_2 (1 - q_2) & (1 - p_2) q_2 & (1 -p_2) (1 - q_2) \\
        p_3 q_3 & p_3 (1 - q_3) & (1 - p_3) q_3 & (1 -p_3) (1 - q_3) \\
        p_4 q_4 & p_4 (1 - q_4) & (1 - p_4) q_4 & (1 -p_4) (1 - q_4) \\
    \end{bmatrix}
\]

Figure~\ref{fig:computed_probabilities_vs_theoretic_probabilities} shows a
regression line fitted to every pairwise interaction with a reported
\(\text{SSError}\) value (pairwise interactions with missing states were
omitted). This serves to validate the approach: a part from some edge cases the
relationship is consistent.

\begin{figure}[!htbp]
    \centering
    \includegraphics[width=.8\textwidth]{./assets/img/computed_probabilities_vs_theoretic_probabilities/main.pdf}
    \caption{The
        relationship between the steady state probabilities inferred from the
        measured transitions and the actual steady state probabilities. A linear
        regression line is included validating the approach.}
    \label{fig:computed_probabilities_vs_theoretic_probabilities}
\end{figure}


\end{document}

strategies.

The results of this analysis are shown in
Figure~\ref{fig:SSError_and_probabilities_in_full}. The top ranking strategies
by number of wins seem to be extortionate (but not against all strategies) and
it can be seen that a small sub group of strategies achieve mutual defection.
All the top ranking strategies according to score achieve mutual cooperation and
do not extort each other, however they
\textbf{do} exhibit extortionate behaviour towards a number of the lower ranking
strategies.

\begin{figure}[!htbp]
    \centering
    \includegraphics[width=.8\textwidth]{./assets/img/SSError_and_probabilities_in_full/main.pdf}
    \caption{\(\text{SSError}\) for the strategies for the full tournament. Only
    strategy interactions for which \(p_4=0\) and \(\chi>1\) are displayed.}
    \label{fig:SSError_and_probabilities_in_full}
\end{figure}

\section{Conclusion}\label{sec:conclusion}

This work defines an approach to measure whether or not a player is playing a
strategy that corresponds to an extortionate strategy as defined
in~\cite{Press2012}: a mathematical model for suspicion. Indeed, all
extortionate strategies have been
 classified as lying on a triangular plane.
This rigorous classification fails to be robust to small measurement error, thus
a statistical approach is proposed.
This is done through a linear algebraic approach for approximating the solution
of a linear system. Using this, a large number of pairwise interactions is
simulated and in fact very few strategies are found to act extortionately.

The work of~\cite{Press2012}, whilst showing that a clever approach to taking
advantage of another memory one strategy exists: this is incomplete. Whilst the
elegance of this result is very attractive, just as the simplicity of the
victory of Tit For Tat in Axelrod's original tournaments was, it is incomplete.
Extortionate strategies achieve a high number of wins but they do not
achieve a high score which corresponds to the fitness landscape in an
evolutionary sense. From the large number of interactions a payoff matrix \(S\)
can be measured where \(S_{ij}\) denotes the score (using standard values of
\((R, S, T, P) = (3, 0, 5, 1)\)) of the \(i\)th strategy
against the \(j\)th strategy. Using this, the replicator equation
describes the evolution of the system based on a population density fitness
function:

\begin{equation}\label{eqn:replicator_dynamics}
    \frac{dx}{dt} = x(S-x^TS x)
\end{equation}

Equation (\ref{eqn:replicator_dynamics}) is solved numerically through an
integration technique described in~\cite{Petzold1983} and
Figure~\ref{fig:replicator_dynamics} shows the evolution of the distribution of
the system: the various strategies are ranked by scores. It is clear to see that
only the high ranking strategies survive the evolutionary process (in fact,
only \documentclass[a4paper]{article}

\usepackage{amsmath}
\usepackage{amssymb}
\usepackage[margin=1.5cm,
            includefoot,
            footskip=30pt]{geometry}
\usepackage{layout}
\usepackage{graphicx}
\usepackage{subcaption}

\usepackage{biblatex}
\usepackage{pdfpages}

\bibliography{main.bib}

\title{Suspicion: Recognising and evaluating the effectiveness
       of extortion in the Iterated Prisoner's Dilemma}
\author{Vincent A. Knight \and Nikoleta E. Glynatsi}
\date{\today}



\begin{document}

\maketitle

\begin{abstract}
    The Iterated Prisoner's Dilemma is a model for rational and evolutionary
    interactive behaviour. It has applications both in the study of human social
    behaviour as well as in biology.
    It is used to understand when and how a rational individual might
    accept an immediate cost to their own utility for the direct benefit of
    another.

    Much attention has been given to a class of strategies called
    Zero Determinant strategies. It has been theoretically shown that these
    strategies can ``extort'' any player.

    In this work, an approach to identify if observed strategies are playing in
    an extortionate way is described. Furthermore, experimental analysis of
    a large tournament with \input{assets/tex/number_of_full_strategies/main.tex}
    strategies is considered. In this setting
    the most highly performing strategies do not play in an extortionate way
    against each other but do against lower performing strategies.
    This suggests that whilst the theory of Zero Determinant strategies
    indicates that memory is not of fundamental importance to the evolution of
    cooperative behaviour, this is incomplete.
\end{abstract}

\section{Introduction}\label{sec:introduction}

Agent based game theoretic models have become a stalwart of the underpinning
mathematics of interactive behaviours. One of the major pieces of work
in this area is the pair of original computer tournaments run by Robert
Axelrod~\cite{Axelrod1980, Axelrod1980a}. These tournaments pitted submitted
computer strategies against each other in plays of the Iterated Prisoner's
Dilemma. A common game where agents can choose to pay a slight cost to their
immediate utility in the hope of building a reputation. This has been used in
economic and evolutionary game theory to understand the evolution of cooperative
behaviour.

Recently, a class of strategies was described in~\cite{Press2012} that can
provably extort any given opponent. In~\cite{Hilbe2013, Moran1707} some
questions have already been asked about the true effectiveness of these
strategies in an evolutionary setting. Here another question is asked: is it
possible to recognise this extortionate behaviour? A mathematical procedure for
suspicion is presented: in the same way that the continued actions of an
extortionate individual might raise suspicion.

This work makes use of the Axelrod Python library~\cite{Knight2018, Knight2016}
with a large number of Prisoner Dilemma strategies available to give an
extensive numerical example of the ideas presented.  The approach is presented
in Section~\ref{sec:delta-zd-strategies}.  All of the code and data discussed
in Section~\ref{sec:numerical-experiments} is open sourced, archived and
written according to best scientific principles~\cite{Wilson2014}. The data
archive can be found at~\cite{vincent_knight_2018_1297075}.

\section{Recognising Extortion}\label{sec:delta-zd-strategies}

In~\cite{Press2012}, given a match between 2 memory-one strategies, the concept
of Zero Determinant (ZD) strategies is introduced. The main result of that paper
shows that given two memory one players \(p, q\in\mathbb{R}^4\) a linear
relationship between the players' scores could be forced by one of the players.

Using the notation of~\cite{Press2012}, assuming the utilities for player \(p\)
are given by \(S_x=(R, S, T, P)\) and for player \(q\) by \(S_y=(R, T, S, P)\)
and that the stationary scores of each player is given by \(S_X\) and \(S_Y\)
respectively. The main result of~\cite{Press2012} is that if

\begin{equation}\label{eqn:linear_relationship_for_p}
    \tilde p=\alpha S_x + \beta S_y + \gamma
\end{equation}

or

\begin{equation}\label{eqn:linear_relationship_for_q}
    \tilde q=\alpha S_x + \beta S_y + \gamma
\end{equation}

where \(\tilde p = (1 - p_1, 1 - p_2, p_3, p_4)\) and
\(\tilde q = (1 - q_1, 1 - q_2, q_3, q_4)\) then:

\begin{equation}
    \alpha S_X + \beta S_Y + \gamma = 0
\end{equation}

In~\cite{Press2012} a particular type of ZD strategy is defined: extortionate
strategies. If:

\begin{equation}\label{eqn:constraint_for_extortion}
    \gamma = - P(\alpha + \beta)
\end{equation}

then the player can ensure they get a score \(\chi\) times
larger than the opponent. This extortion coefficient is given by:

\begin{equation}\label{eqn:definition_of_chi}
    \chi=\frac{-\beta}{\alpha}
\end{equation}

Thus, if (\ref{eqn:constraint_for_extortion}) holds and \(\chi >1\) a player is
said to extort their opponent.
Here, the reverse problem is considered: given a
\(p\in\mathbb{R}^4\) how does one identify \(\alpha, \beta\) if they
exist and is the strategy in fact acting in an extortionate way?

These conditions correspond to:

\begin{align}
    \tilde p_1 & = \alpha R + \beta R - P (\alpha + \beta)
            \label{eqn:condition_for_tilde_p1}\\
    \tilde p_2 & = \alpha S + \beta T - P (\alpha + \beta)
            \label{eqn:condition_for_tilde_p2}\\
    \tilde p_3 & = \alpha T + \beta S - P (\alpha + \beta)
            \label{eqn:condition_for_tilde_p3}\\
    \tilde p_4 & = \alpha P + \beta P - P (\alpha + \beta)
            \label{eqn:condition_for_tilde_p4}
\end{align}

Equation (\ref{eqn:condition_for_tilde_p4}) ensures that \(p_4=\tilde p_4=0\).
Equations (\ref{eqn:condition_for_tilde_p1}-\ref{eqn:condition_for_tilde_p3})
can be used to eliminate \(\alpha, \beta\), giving:

\begin{equation}\label{eqn:planar_definition_of_extortion}
    \tilde p_1 = \frac{(R - P)(\tilde p_2 + \tilde p_3)}{S + T - 2P}
\end{equation}

with:

\begin{equation}\label{eqn:definition_of_chi}
    \chi = \frac{\tilde p_2 (P - T) + \tilde p_3 (S - P)}
                {\tilde p_2 (P - S) + \tilde p_3 (T - P)}
\end{equation}

Given a strategy \(p\in\mathbb{R}^{4\times 1}\) equations
(\ref{eqn:condition_for_tilde_p4}), (\ref{eqn:planar_definition_of_extortion}-\ref{eqn:definition_of_chi}) can be used to check if
a strategy is extortionate. The conditions correspond to:

\begin{align}
    p_1 & = \frac{(R-P)(p_2 + p_3) - R + T + S - P}{S + T - 2P}
     \label{eqn:condition_for_p1}\\
    p_4 & = 0 \label{eqn:condition_for_p4}\\
    1 & > p_2 + p_3\label{eqn:condition_for_chi}
\end{align}

The algebraic steps necessary to prove these results are available in the
supporting materials.

All extortionate strategies reside on a triangular (\ref{eqn:condition_for_chi})
plane (\ref{eqn:condition_for_p1}) in 3 dimensions (\ref{eqn:condition_for_p4}).
Using this formulation it can be seen that a necessary (but not sufficient)
condition for an extortionate strategy is that it cooperates on average less
than 50\% of the time when in a state of disagreement with the opponent.

As an example, consider the known extortionate strategy \(p=(8 / 9, 1 / 2, 1 /
3, 0)\) from~\cite{Stewart2012} which is referred to as \texttt{Extort-2}. In
this case, for the standard values of \((R, T, S, P)\) constraint
(\ref{eqn:condition_for_p1}) corresponds to:

\begin{equation}
    p_1 = \frac{2(p_2 + p_3) + 1}{3}
\end{equation}

It is clear that in this case all constraints hold.

This approach could in fact be used to confirm that a given strategy is acting
in an extortionate manner even if it is not a memory one strategy. However, in
practice, if a closed form for \(p\) is not known, then due to measurement
and/or numerical error this would not work.

This problem can be written in the following linear algebraic form where
\(x=(\alpha, \beta)\)
and \(p^*=(\tilde p_1 - 1, tilde_2 - 1, p_3)\):

\begin{equation}\label{eqn:linear_algebraic_equation_for_p}
    Cx= p^*
\end{equation}

\(C\) corresponds to equations
(\ref{eqn:condition_for_tilde_p1}-\ref{eqn:condition_for_tilde_p3}) and is
given by:

\begin{equation}\label{eqn:definition_of_C}
    C =
    \begin{bmatrix}
        R - P & R- P \\
        S - P & T- P \\
        T - P & S- P \\
    \end{bmatrix}
\end{equation}

Note that in general, equation (\ref{eqn:linear_algebraic_equation_for_p}) will
not necessarily have a solution. From the Rouch\'{e}-Capelli theorem if there is
a solution it is unique as \(\text{rank}(C)=2\) which is the dimension of the
variable \(x\). The best fitting \(x\) is found by minimizing:

\begin{equation}\label{eqn:r_squared}
    \text{SSError} = \|C x- p^*\|_2^2 = \sum_{i=1}^{3}\left((C\bar x)_i-p_i^*\right)^2
\end{equation}

Note that \(\text{SSError}\), which is the square of the Frobenius
norm~\cite{Golub2013}, becomes a measure of how close a strategy is to being an
extortionate strategy. Suspicion
of extortion then corresponds to a threshold on \(\text{SSError}\).

By observing interactions (human or otherwise), their memory one representation
can be inferred and this approach can be used to recognise extortionate
behaviour. The notion of comparing theoretic and actual plays of the IPD is not
novel, see for example~\cite{Rand2013}. Immediately it is noted that if the
environment is noisy~\cite{Wu1995} then no strategy can be considered to be
extortionate as \(p_4>0\).

In the next section, this idea will be illustrated by observing the interactions
that take place in a computer based tournament of the IPD\@.

\section{Numerical experiments}\label{sec:numerical-experiments}

In~\cite{Stewart2012} results from a tournament with
\input{./assets/tex/number_of_stewart_plotkin_strategies/main.tex} strategies,
was presented with specific consideration given to ZD strategies. This
tournament is reproduced here using the Axelrod-Python
project~\cite{Knight2016}. To obtain a good measure of the corresponding
transition rates for each strategy all matches have been run for
\input{assets/tex/number_of_turns/main.tex} turns and every match has been
repeated \input{assets/tex/number_of_repetitions/main.tex} times. All of this
interaction data is available at~\cite{vincent_knight_2018_1297075}. A good
match between the inferred Markov chain and the state distribution of the actual
interactions has been verified. Data for this is presented in the supplementary
materials.

Figure~\ref{fig:SSError_overall_in_stewart_plotkin} shows the \(\text{SSError}\)
values for all the strategies in the tournament, as reported
in~\cite{Stewart2012} the extortionate strategy (which has an expected
\(\text{SSError}\) approximately 0) gains a large number of wins.

\begin{figure}[!htbp]
    \centering
    \includegraphics[width=.8\textwidth]{./assets/img/SSError_overall_in_stewart_plotkin/main.pdf}
    \caption{\(\text{SSError}\) and state probabilities for the strategies
        of~\cite{Stewart2012}, ordered both by number of wins and overall score.
        Note that \(P(DC)\) is not shown as it corresponds to the transpose of
        \(P(CD)\). Cooperator and Defector are omitted as they do not visit all
        the states.}
    \label{fig:SSError_overall_in_stewart_plotkin}
\end{figure}

Here, the work of~\cite{Stewart2012} is extended by investigating a tournament
with \input{assets/tex/number_of_full_strategies/main.tex}
strategies.

The results of this analysis are shown in
Figure~\ref{fig:SSError_and_probabilities_in_full}. The top ranking strategies
by number of wins seem to be extortionate (but not against all strategies) and
it can be seen that a small sub group of strategies achieve mutual defection.
All the top ranking strategies according to score achieve mutual cooperation and
do not extort each other, however they
\textbf{do} exhibit extortionate behaviour towards a number of the lower ranking
strategies.

\begin{figure}[!htbp]
    \centering
    \includegraphics[width=.8\textwidth]{./assets/img/SSError_and_probabilities_in_full/main.pdf}
    \caption{\(\text{SSError}\) for the strategies for the full tournament. Only
    strategy interactions for which \(p_4=0\) and \(\chi>1\) are displayed.}
    \label{fig:SSError_and_probabilities_in_full}
\end{figure}

\section{Conclusion}\label{sec:conclusion}

This work defines an approach to measure whether or not a player is playing a
strategy that corresponds to an extortionate strategy as defined
in~\cite{Press2012}: a mathematical model for suspicion. Indeed, all
extortionate strategies have been
 classified as lying on a triangular plane.
This rigorous classification fails to be robust to small measurement error, thus
a statistical approach is proposed.
This is done through a linear algebraic approach for approximating the solution
of a linear system. Using this, a large number of pairwise interactions is
simulated and in fact very few strategies are found to act extortionately.

The work of~\cite{Press2012}, whilst showing that a clever approach to taking
advantage of another memory one strategy exists: this is incomplete. Whilst the
elegance of this result is very attractive, just as the simplicity of the
victory of Tit For Tat in Axelrod's original tournaments was, it is incomplete.
Extortionate strategies achieve a high number of wins but they do not
achieve a high score which corresponds to the fitness landscape in an
evolutionary sense. From the large number of interactions a payoff matrix \(S\)
can be measured where \(S_{ij}\) denotes the score (using standard values of
\((R, S, T, P) = (3, 0, 5, 1)\)) of the \(i\)th strategy
against the \(j\)th strategy. Using this, the replicator equation
describes the evolution of the system based on a population density fitness
function:

\begin{equation}\label{eqn:replicator_dynamics}
    \frac{dx}{dt} = x(S-x^TS x)
\end{equation}

Equation (\ref{eqn:replicator_dynamics}) is solved numerically through an
integration technique described in~\cite{Petzold1983} and
Figure~\ref{fig:replicator_dynamics} shows the evolution of the distribution of
the system: the various strategies are ranked by scores. It is clear to see that
only the high ranking strategies survive the evolutionary process (in fact,
only \input{./assets/img/replicator_dynamics/main.tex}
have a final distribution greater than \(10 ^ {-2}\)). This confirms the
findings of~\cite{Moran1707} in which sophisticated strategies resist
evolutionary invasion of shorter memory strategies. Recalling
Figure~\ref{fig:SSError_and_probabilities_in_full} this demonstrates that:

\begin{itemize}
    \item Cooperation emerges through the evolutionary process: the high scoring
        strategies do not exhibit extortionate behaviour towards each other.
    \item Extortionate strategies do not survive the evolutionary process.
\end{itemize}

\begin{figure}[!htbp]
    \centering
    \includegraphics[width=.8\textwidth]{./assets/img/replicator_dynamics/main.pdf}
    \caption{Numerical simulation of the replicator equation
    (\ref{eqn:replicator_dynamics}): strategies are ordered by score, only the strategies with a high score survive the evolutionary process.}
    \label{fig:replicator_dynamics}
\end{figure}

This work can be used to classify plays of the IPD\@: data can be collected from
actual interactions (in lab or in the field). Furthermore, this allows for a
classification method similar to the notion of fingerprinting presented
in~\cite{Ashlock2008}. Trained strategies can potentially be classified as
extortionate or not or it could be possible to even constrain the reinforcement
learning approaches that are becoming prevalent in the literature.
Alternatively, this mathematical approach for recognising extortion could be
used in sophisticated strategies to defend against invasion. Arguably, some of
the strategies considered here exhibit this behaviour, indeed as described
in~\cite{Harper2017}, the top ranking strategies in the full tournament are
obtained using evolutionary reinforcement learning techniques, thus, suspicion
of extortionate behaviour could in fact be an evolutionary trait.

\section*{Acknowledgements}

The following open source software libraries were used in this research:

\begin{itemize}
    \item The Axelrod ~\cite{Knight2016, Knight2018} library (IPD strategies and
        tournaments).
    \item The sympy library~\cite{Meurer2017} (verification of all symbolic
        calculations).
    \item The matplotlib~\cite{Droettboom2018} library (visualisation).
    \item The pandas~\cite{Structures2010}, dask~\cite{Dask2016} and
        NumPy~\cite{Oliphant2015} libraries (data manipulation).
    \item The SciPy~\cite{Jones2001} library (numerical integration of the
        replicator equation).
\end{itemize}

This work was performed using the computational facilities of the Advanced
Research Computing @ Cardiff (ARCCA) Division, Cardiff University.

\printbibliography

\newpage
\section*{Supplementary materials}

\includepdf{assets/pdf/proof_of_form_of_extortionate_strategies/main.pdf}

\newpage

Using the pair wise interactions the transition rates \(p,
q\) can be measured and the steady state probabilities inferred and compared to
the actual probabilities of each state.
This is done numerically by computing the singular eigenvector of the
matrix \(A\) \cite{Stewart2009}:

\[
    A =
    \begin{bmatrix}
        p_1 q_1 & p_1 (1 - q_1) & (1 - p_1) q_1 & (1 -p_1) (1 - q_1) \\
        p_2 q_2 & p_2 (1 - q_2) & (1 - p_2) q_2 & (1 -p_2) (1 - q_2) \\
        p_3 q_3 & p_3 (1 - q_3) & (1 - p_3) q_3 & (1 -p_3) (1 - q_3) \\
        p_4 q_4 & p_4 (1 - q_4) & (1 - p_4) q_4 & (1 -p_4) (1 - q_4) \\
    \end{bmatrix}
\]

Figure~\ref{fig:computed_probabilities_vs_theoretic_probabilities} shows a
regression line fitted to every pairwise interaction with a reported
\(\text{SSError}\) value (pairwise interactions with missing states were
omitted). This serves to validate the approach: a part from some edge cases the
relationship is consistent.

\begin{figure}[!htbp]
    \centering
    \includegraphics[width=.8\textwidth]{./assets/img/computed_probabilities_vs_theoretic_probabilities/main.pdf}
    \caption{The
        relationship between the steady state probabilities inferred from the
        measured transitions and the actual steady state probabilities. A linear
        regression line is included validating the approach.}
    \label{fig:computed_probabilities_vs_theoretic_probabilities}
\end{figure}


\end{document}

have a final distribution greater than \(10 ^ {-2}\)). This confirms the
findings of~\cite{Moran1707} in which sophisticated strategies resist
evolutionary invasion of shorter memory strategies. Recalling
Figure~\ref{fig:SSError_and_probabilities_in_full} this demonstrates that:

\begin{itemize}
    \item Cooperation emerges through the evolutionary process: the high scoring
        strategies do not exhibit extortionate behaviour towards each other.
    \item Extortionate strategies do not survive the evolutionary process.
\end{itemize}

\begin{figure}[!htbp]
    \centering
    \includegraphics[width=.8\textwidth]{./assets/img/replicator_dynamics/main.pdf}
    \caption{Numerical simulation of the replicator equation
    (\ref{eqn:replicator_dynamics}): strategies are ordered by score, only the strategies with a high score survive the evolutionary process.}
    \label{fig:replicator_dynamics}
\end{figure}

This work can be used to classify plays of the IPD\@: data can be collected from
actual interactions (in lab or in the field). Furthermore, this allows for a
classification method similar to the notion of fingerprinting presented
in~\cite{Ashlock2008}. Trained strategies can potentially be classified as
extortionate or not or it could be possible to even constrain the reinforcement
learning approaches that are becoming prevalent in the literature.
Alternatively, this mathematical approach for recognising extortion could be
used in sophisticated strategies to defend against invasion. Arguably, some of
the strategies considered here exhibit this behaviour, indeed as described
in~\cite{Harper2017}, the top ranking strategies in the full tournament are
obtained using evolutionary reinforcement learning techniques, thus, suspicion
of extortionate behaviour could in fact be an evolutionary trait.

\section*{Acknowledgements}

The following open source software libraries were used in this research:

\begin{itemize}
    \item The Axelrod ~\cite{Knight2016, Knight2018} library (IPD strategies and
        tournaments).
    \item The sympy library~\cite{Meurer2017} (verification of all symbolic
        calculations).
    \item The matplotlib~\cite{Droettboom2018} library (visualisation).
    \item The pandas~\cite{Structures2010}, dask~\cite{Dask2016} and
        NumPy~\cite{Oliphant2015} libraries (data manipulation).
    \item The SciPy~\cite{Jones2001} library (numerical integration of the
        replicator equation).
\end{itemize}

This work was performed using the computational facilities of the Advanced
Research Computing @ Cardiff (ARCCA) Division, Cardiff University.

\printbibliography

\newpage
\section*{Supplementary materials}

\includepdf{assets/pdf/proof_of_form_of_extortionate_strategies/main.pdf}

\newpage

Using the pair wise interactions the transition rates \(p,
q\) can be measured and the steady state probabilities inferred and compared to
the actual probabilities of each state.
This is done numerically by computing the singular eigenvector of the
matrix \(A\) \cite{Stewart2009}:

\[
    A =
    \begin{bmatrix}
        p_1 q_1 & p_1 (1 - q_1) & (1 - p_1) q_1 & (1 -p_1) (1 - q_1) \\
        p_2 q_2 & p_2 (1 - q_2) & (1 - p_2) q_2 & (1 -p_2) (1 - q_2) \\
        p_3 q_3 & p_3 (1 - q_3) & (1 - p_3) q_3 & (1 -p_3) (1 - q_3) \\
        p_4 q_4 & p_4 (1 - q_4) & (1 - p_4) q_4 & (1 -p_4) (1 - q_4) \\
    \end{bmatrix}
\]

Figure~\ref{fig:computed_probabilities_vs_theoretic_probabilities} shows a
regression line fitted to every pairwise interaction with a reported
\(\text{SSError}\) value (pairwise interactions with missing states were
omitted). This serves to validate the approach: a part from some edge cases the
relationship is consistent.

\begin{figure}[!htbp]
    \centering
    \includegraphics[width=.8\textwidth]{./assets/img/computed_probabilities_vs_theoretic_probabilities/main.pdf}
    \caption{The
        relationship between the steady state probabilities inferred from the
        measured transitions and the actual steady state probabilities. A linear
        regression line is included validating the approach.}
    \label{fig:computed_probabilities_vs_theoretic_probabilities}
\end{figure}


\end{document}
 strategies,
was presented with specific consideration given to ZD strategies. This
tournament is reproduced here using the Axelrod-Python
project~\cite{Knight2016}. To obtain a good measure of the corresponding
transition rates for each strategy all matches have been run for
\documentclass[a4paper]{article}

\usepackage{amsmath}
\usepackage{amssymb}
\usepackage[margin=1.5cm,
            includefoot,
            footskip=30pt]{geometry}
\usepackage{layout}
\usepackage{graphicx}
\usepackage{subcaption}

\usepackage{biblatex}
\usepackage{pdfpages}

\bibliography{main.bib}

\title{Suspicion: Recognising and evaluating the effectiveness
       of extortion in the Iterated Prisoner's Dilemma}
\author{Vincent A. Knight \and Nikoleta E. Glynatsi}
\date{\today}



\begin{document}

\maketitle

\begin{abstract}
    The Iterated Prisoner's Dilemma is a model for rational and evolutionary
    interactive behaviour. It has applications both in the study of human social
    behaviour as well as in biology.
    It is used to understand when and how a rational individual might
    accept an immediate cost to their own utility for the direct benefit of
    another.

    Much attention has been given to a class of strategies called
    Zero Determinant strategies. It has been theoretically shown that these
    strategies can ``extort'' any player.

    In this work, an approach to identify if observed strategies are playing in
    an extortionate way is described. Furthermore, experimental analysis of
    a large tournament with \documentclass[a4paper]{article}

\usepackage{amsmath}
\usepackage{amssymb}
\usepackage[margin=1.5cm,
            includefoot,
            footskip=30pt]{geometry}
\usepackage{layout}
\usepackage{graphicx}
\usepackage{subcaption}

\usepackage{biblatex}
\usepackage{pdfpages}

\bibliography{main.bib}

\title{Suspicion: Recognising and evaluating the effectiveness
       of extortion in the Iterated Prisoner's Dilemma}
\author{Vincent A. Knight \and Nikoleta E. Glynatsi}
\date{\today}



\begin{document}

\maketitle

\begin{abstract}
    The Iterated Prisoner's Dilemma is a model for rational and evolutionary
    interactive behaviour. It has applications both in the study of human social
    behaviour as well as in biology.
    It is used to understand when and how a rational individual might
    accept an immediate cost to their own utility for the direct benefit of
    another.

    Much attention has been given to a class of strategies called
    Zero Determinant strategies. It has been theoretically shown that these
    strategies can ``extort'' any player.

    In this work, an approach to identify if observed strategies are playing in
    an extortionate way is described. Furthermore, experimental analysis of
    a large tournament with \input{assets/tex/number_of_full_strategies/main.tex}
    strategies is considered. In this setting
    the most highly performing strategies do not play in an extortionate way
    against each other but do against lower performing strategies.
    This suggests that whilst the theory of Zero Determinant strategies
    indicates that memory is not of fundamental importance to the evolution of
    cooperative behaviour, this is incomplete.
\end{abstract}

\section{Introduction}\label{sec:introduction}

Agent based game theoretic models have become a stalwart of the underpinning
mathematics of interactive behaviours. One of the major pieces of work
in this area is the pair of original computer tournaments run by Robert
Axelrod~\cite{Axelrod1980, Axelrod1980a}. These tournaments pitted submitted
computer strategies against each other in plays of the Iterated Prisoner's
Dilemma. A common game where agents can choose to pay a slight cost to their
immediate utility in the hope of building a reputation. This has been used in
economic and evolutionary game theory to understand the evolution of cooperative
behaviour.

Recently, a class of strategies was described in~\cite{Press2012} that can
provably extort any given opponent. In~\cite{Hilbe2013, Moran1707} some
questions have already been asked about the true effectiveness of these
strategies in an evolutionary setting. Here another question is asked: is it
possible to recognise this extortionate behaviour? A mathematical procedure for
suspicion is presented: in the same way that the continued actions of an
extortionate individual might raise suspicion.

This work makes use of the Axelrod Python library~\cite{Knight2018, Knight2016}
with a large number of Prisoner Dilemma strategies available to give an
extensive numerical example of the ideas presented.  The approach is presented
in Section~\ref{sec:delta-zd-strategies}.  All of the code and data discussed
in Section~\ref{sec:numerical-experiments} is open sourced, archived and
written according to best scientific principles~\cite{Wilson2014}. The data
archive can be found at~\cite{vincent_knight_2018_1297075}.

\section{Recognising Extortion}\label{sec:delta-zd-strategies}

In~\cite{Press2012}, given a match between 2 memory-one strategies, the concept
of Zero Determinant (ZD) strategies is introduced. The main result of that paper
shows that given two memory one players \(p, q\in\mathbb{R}^4\) a linear
relationship between the players' scores could be forced by one of the players.

Using the notation of~\cite{Press2012}, assuming the utilities for player \(p\)
are given by \(S_x=(R, S, T, P)\) and for player \(q\) by \(S_y=(R, T, S, P)\)
and that the stationary scores of each player is given by \(S_X\) and \(S_Y\)
respectively. The main result of~\cite{Press2012} is that if

\begin{equation}\label{eqn:linear_relationship_for_p}
    \tilde p=\alpha S_x + \beta S_y + \gamma
\end{equation}

or

\begin{equation}\label{eqn:linear_relationship_for_q}
    \tilde q=\alpha S_x + \beta S_y + \gamma
\end{equation}

where \(\tilde p = (1 - p_1, 1 - p_2, p_3, p_4)\) and
\(\tilde q = (1 - q_1, 1 - q_2, q_3, q_4)\) then:

\begin{equation}
    \alpha S_X + \beta S_Y + \gamma = 0
\end{equation}

In~\cite{Press2012} a particular type of ZD strategy is defined: extortionate
strategies. If:

\begin{equation}\label{eqn:constraint_for_extortion}
    \gamma = - P(\alpha + \beta)
\end{equation}

then the player can ensure they get a score \(\chi\) times
larger than the opponent. This extortion coefficient is given by:

\begin{equation}\label{eqn:definition_of_chi}
    \chi=\frac{-\beta}{\alpha}
\end{equation}

Thus, if (\ref{eqn:constraint_for_extortion}) holds and \(\chi >1\) a player is
said to extort their opponent.
Here, the reverse problem is considered: given a
\(p\in\mathbb{R}^4\) how does one identify \(\alpha, \beta\) if they
exist and is the strategy in fact acting in an extortionate way?

These conditions correspond to:

\begin{align}
    \tilde p_1 & = \alpha R + \beta R - P (\alpha + \beta)
            \label{eqn:condition_for_tilde_p1}\\
    \tilde p_2 & = \alpha S + \beta T - P (\alpha + \beta)
            \label{eqn:condition_for_tilde_p2}\\
    \tilde p_3 & = \alpha T + \beta S - P (\alpha + \beta)
            \label{eqn:condition_for_tilde_p3}\\
    \tilde p_4 & = \alpha P + \beta P - P (\alpha + \beta)
            \label{eqn:condition_for_tilde_p4}
\end{align}

Equation (\ref{eqn:condition_for_tilde_p4}) ensures that \(p_4=\tilde p_4=0\).
Equations (\ref{eqn:condition_for_tilde_p1}-\ref{eqn:condition_for_tilde_p3})
can be used to eliminate \(\alpha, \beta\), giving:

\begin{equation}\label{eqn:planar_definition_of_extortion}
    \tilde p_1 = \frac{(R - P)(\tilde p_2 + \tilde p_3)}{S + T - 2P}
\end{equation}

with:

\begin{equation}\label{eqn:definition_of_chi}
    \chi = \frac{\tilde p_2 (P - T) + \tilde p_3 (S - P)}
                {\tilde p_2 (P - S) + \tilde p_3 (T - P)}
\end{equation}

Given a strategy \(p\in\mathbb{R}^{4\times 1}\) equations
(\ref{eqn:condition_for_tilde_p4}), (\ref{eqn:planar_definition_of_extortion}-\ref{eqn:definition_of_chi}) can be used to check if
a strategy is extortionate. The conditions correspond to:

\begin{align}
    p_1 & = \frac{(R-P)(p_2 + p_3) - R + T + S - P}{S + T - 2P}
     \label{eqn:condition_for_p1}\\
    p_4 & = 0 \label{eqn:condition_for_p4}\\
    1 & > p_2 + p_3\label{eqn:condition_for_chi}
\end{align}

The algebraic steps necessary to prove these results are available in the
supporting materials.

All extortionate strategies reside on a triangular (\ref{eqn:condition_for_chi})
plane (\ref{eqn:condition_for_p1}) in 3 dimensions (\ref{eqn:condition_for_p4}).
Using this formulation it can be seen that a necessary (but not sufficient)
condition for an extortionate strategy is that it cooperates on average less
than 50\% of the time when in a state of disagreement with the opponent.

As an example, consider the known extortionate strategy \(p=(8 / 9, 1 / 2, 1 /
3, 0)\) from~\cite{Stewart2012} which is referred to as \texttt{Extort-2}. In
this case, for the standard values of \((R, T, S, P)\) constraint
(\ref{eqn:condition_for_p1}) corresponds to:

\begin{equation}
    p_1 = \frac{2(p_2 + p_3) + 1}{3}
\end{equation}

It is clear that in this case all constraints hold.

This approach could in fact be used to confirm that a given strategy is acting
in an extortionate manner even if it is not a memory one strategy. However, in
practice, if a closed form for \(p\) is not known, then due to measurement
and/or numerical error this would not work.

This problem can be written in the following linear algebraic form where
\(x=(\alpha, \beta)\)
and \(p^*=(\tilde p_1 - 1, tilde_2 - 1, p_3)\):

\begin{equation}\label{eqn:linear_algebraic_equation_for_p}
    Cx= p^*
\end{equation}

\(C\) corresponds to equations
(\ref{eqn:condition_for_tilde_p1}-\ref{eqn:condition_for_tilde_p3}) and is
given by:

\begin{equation}\label{eqn:definition_of_C}
    C =
    \begin{bmatrix}
        R - P & R- P \\
        S - P & T- P \\
        T - P & S- P \\
    \end{bmatrix}
\end{equation}

Note that in general, equation (\ref{eqn:linear_algebraic_equation_for_p}) will
not necessarily have a solution. From the Rouch\'{e}-Capelli theorem if there is
a solution it is unique as \(\text{rank}(C)=2\) which is the dimension of the
variable \(x\). The best fitting \(x\) is found by minimizing:

\begin{equation}\label{eqn:r_squared}
    \text{SSError} = \|C x- p^*\|_2^2 = \sum_{i=1}^{3}\left((C\bar x)_i-p_i^*\right)^2
\end{equation}

Note that \(\text{SSError}\), which is the square of the Frobenius
norm~\cite{Golub2013}, becomes a measure of how close a strategy is to being an
extortionate strategy. Suspicion
of extortion then corresponds to a threshold on \(\text{SSError}\).

By observing interactions (human or otherwise), their memory one representation
can be inferred and this approach can be used to recognise extortionate
behaviour. The notion of comparing theoretic and actual plays of the IPD is not
novel, see for example~\cite{Rand2013}. Immediately it is noted that if the
environment is noisy~\cite{Wu1995} then no strategy can be considered to be
extortionate as \(p_4>0\).

In the next section, this idea will be illustrated by observing the interactions
that take place in a computer based tournament of the IPD\@.

\section{Numerical experiments}\label{sec:numerical-experiments}

In~\cite{Stewart2012} results from a tournament with
\input{./assets/tex/number_of_stewart_plotkin_strategies/main.tex} strategies,
was presented with specific consideration given to ZD strategies. This
tournament is reproduced here using the Axelrod-Python
project~\cite{Knight2016}. To obtain a good measure of the corresponding
transition rates for each strategy all matches have been run for
\input{assets/tex/number_of_turns/main.tex} turns and every match has been
repeated \input{assets/tex/number_of_repetitions/main.tex} times. All of this
interaction data is available at~\cite{vincent_knight_2018_1297075}. A good
match between the inferred Markov chain and the state distribution of the actual
interactions has been verified. Data for this is presented in the supplementary
materials.

Figure~\ref{fig:SSError_overall_in_stewart_plotkin} shows the \(\text{SSError}\)
values for all the strategies in the tournament, as reported
in~\cite{Stewart2012} the extortionate strategy (which has an expected
\(\text{SSError}\) approximately 0) gains a large number of wins.

\begin{figure}[!htbp]
    \centering
    \includegraphics[width=.8\textwidth]{./assets/img/SSError_overall_in_stewart_plotkin/main.pdf}
    \caption{\(\text{SSError}\) and state probabilities for the strategies
        of~\cite{Stewart2012}, ordered both by number of wins and overall score.
        Note that \(P(DC)\) is not shown as it corresponds to the transpose of
        \(P(CD)\). Cooperator and Defector are omitted as they do not visit all
        the states.}
    \label{fig:SSError_overall_in_stewart_plotkin}
\end{figure}

Here, the work of~\cite{Stewart2012} is extended by investigating a tournament
with \input{assets/tex/number_of_full_strategies/main.tex}
strategies.

The results of this analysis are shown in
Figure~\ref{fig:SSError_and_probabilities_in_full}. The top ranking strategies
by number of wins seem to be extortionate (but not against all strategies) and
it can be seen that a small sub group of strategies achieve mutual defection.
All the top ranking strategies according to score achieve mutual cooperation and
do not extort each other, however they
\textbf{do} exhibit extortionate behaviour towards a number of the lower ranking
strategies.

\begin{figure}[!htbp]
    \centering
    \includegraphics[width=.8\textwidth]{./assets/img/SSError_and_probabilities_in_full/main.pdf}
    \caption{\(\text{SSError}\) for the strategies for the full tournament. Only
    strategy interactions for which \(p_4=0\) and \(\chi>1\) are displayed.}
    \label{fig:SSError_and_probabilities_in_full}
\end{figure}

\section{Conclusion}\label{sec:conclusion}

This work defines an approach to measure whether or not a player is playing a
strategy that corresponds to an extortionate strategy as defined
in~\cite{Press2012}: a mathematical model for suspicion. Indeed, all
extortionate strategies have been
 classified as lying on a triangular plane.
This rigorous classification fails to be robust to small measurement error, thus
a statistical approach is proposed.
This is done through a linear algebraic approach for approximating the solution
of a linear system. Using this, a large number of pairwise interactions is
simulated and in fact very few strategies are found to act extortionately.

The work of~\cite{Press2012}, whilst showing that a clever approach to taking
advantage of another memory one strategy exists: this is incomplete. Whilst the
elegance of this result is very attractive, just as the simplicity of the
victory of Tit For Tat in Axelrod's original tournaments was, it is incomplete.
Extortionate strategies achieve a high number of wins but they do not
achieve a high score which corresponds to the fitness landscape in an
evolutionary sense. From the large number of interactions a payoff matrix \(S\)
can be measured where \(S_{ij}\) denotes the score (using standard values of
\((R, S, T, P) = (3, 0, 5, 1)\)) of the \(i\)th strategy
against the \(j\)th strategy. Using this, the replicator equation
describes the evolution of the system based on a population density fitness
function:

\begin{equation}\label{eqn:replicator_dynamics}
    \frac{dx}{dt} = x(S-x^TS x)
\end{equation}

Equation (\ref{eqn:replicator_dynamics}) is solved numerically through an
integration technique described in~\cite{Petzold1983} and
Figure~\ref{fig:replicator_dynamics} shows the evolution of the distribution of
the system: the various strategies are ranked by scores. It is clear to see that
only the high ranking strategies survive the evolutionary process (in fact,
only \input{./assets/img/replicator_dynamics/main.tex}
have a final distribution greater than \(10 ^ {-2}\)). This confirms the
findings of~\cite{Moran1707} in which sophisticated strategies resist
evolutionary invasion of shorter memory strategies. Recalling
Figure~\ref{fig:SSError_and_probabilities_in_full} this demonstrates that:

\begin{itemize}
    \item Cooperation emerges through the evolutionary process: the high scoring
        strategies do not exhibit extortionate behaviour towards each other.
    \item Extortionate strategies do not survive the evolutionary process.
\end{itemize}

\begin{figure}[!htbp]
    \centering
    \includegraphics[width=.8\textwidth]{./assets/img/replicator_dynamics/main.pdf}
    \caption{Numerical simulation of the replicator equation
    (\ref{eqn:replicator_dynamics}): strategies are ordered by score, only the strategies with a high score survive the evolutionary process.}
    \label{fig:replicator_dynamics}
\end{figure}

This work can be used to classify plays of the IPD\@: data can be collected from
actual interactions (in lab or in the field). Furthermore, this allows for a
classification method similar to the notion of fingerprinting presented
in~\cite{Ashlock2008}. Trained strategies can potentially be classified as
extortionate or not or it could be possible to even constrain the reinforcement
learning approaches that are becoming prevalent in the literature.
Alternatively, this mathematical approach for recognising extortion could be
used in sophisticated strategies to defend against invasion. Arguably, some of
the strategies considered here exhibit this behaviour, indeed as described
in~\cite{Harper2017}, the top ranking strategies in the full tournament are
obtained using evolutionary reinforcement learning techniques, thus, suspicion
of extortionate behaviour could in fact be an evolutionary trait.

\section*{Acknowledgements}

The following open source software libraries were used in this research:

\begin{itemize}
    \item The Axelrod ~\cite{Knight2016, Knight2018} library (IPD strategies and
        tournaments).
    \item The sympy library~\cite{Meurer2017} (verification of all symbolic
        calculations).
    \item The matplotlib~\cite{Droettboom2018} library (visualisation).
    \item The pandas~\cite{Structures2010}, dask~\cite{Dask2016} and
        NumPy~\cite{Oliphant2015} libraries (data manipulation).
    \item The SciPy~\cite{Jones2001} library (numerical integration of the
        replicator equation).
\end{itemize}

This work was performed using the computational facilities of the Advanced
Research Computing @ Cardiff (ARCCA) Division, Cardiff University.

\printbibliography

\newpage
\section*{Supplementary materials}

\includepdf{assets/pdf/proof_of_form_of_extortionate_strategies/main.pdf}

\newpage

Using the pair wise interactions the transition rates \(p,
q\) can be measured and the steady state probabilities inferred and compared to
the actual probabilities of each state.
This is done numerically by computing the singular eigenvector of the
matrix \(A\) \cite{Stewart2009}:

\[
    A =
    \begin{bmatrix}
        p_1 q_1 & p_1 (1 - q_1) & (1 - p_1) q_1 & (1 -p_1) (1 - q_1) \\
        p_2 q_2 & p_2 (1 - q_2) & (1 - p_2) q_2 & (1 -p_2) (1 - q_2) \\
        p_3 q_3 & p_3 (1 - q_3) & (1 - p_3) q_3 & (1 -p_3) (1 - q_3) \\
        p_4 q_4 & p_4 (1 - q_4) & (1 - p_4) q_4 & (1 -p_4) (1 - q_4) \\
    \end{bmatrix}
\]

Figure~\ref{fig:computed_probabilities_vs_theoretic_probabilities} shows a
regression line fitted to every pairwise interaction with a reported
\(\text{SSError}\) value (pairwise interactions with missing states were
omitted). This serves to validate the approach: a part from some edge cases the
relationship is consistent.

\begin{figure}[!htbp]
    \centering
    \includegraphics[width=.8\textwidth]{./assets/img/computed_probabilities_vs_theoretic_probabilities/main.pdf}
    \caption{The
        relationship between the steady state probabilities inferred from the
        measured transitions and the actual steady state probabilities. A linear
        regression line is included validating the approach.}
    \label{fig:computed_probabilities_vs_theoretic_probabilities}
\end{figure}


\end{document}

    strategies is considered. In this setting
    the most highly performing strategies do not play in an extortionate way
    against each other but do against lower performing strategies.
    This suggests that whilst the theory of Zero Determinant strategies
    indicates that memory is not of fundamental importance to the evolution of
    cooperative behaviour, this is incomplete.
\end{abstract}

\section{Introduction}\label{sec:introduction}

Agent based game theoretic models have become a stalwart of the underpinning
mathematics of interactive behaviours. One of the major pieces of work
in this area is the pair of original computer tournaments run by Robert
Axelrod~\cite{Axelrod1980, Axelrod1980a}. These tournaments pitted submitted
computer strategies against each other in plays of the Iterated Prisoner's
Dilemma. A common game where agents can choose to pay a slight cost to their
immediate utility in the hope of building a reputation. This has been used in
economic and evolutionary game theory to understand the evolution of cooperative
behaviour.

Recently, a class of strategies was described in~\cite{Press2012} that can
provably extort any given opponent. In~\cite{Hilbe2013, Moran1707} some
questions have already been asked about the true effectiveness of these
strategies in an evolutionary setting. Here another question is asked: is it
possible to recognise this extortionate behaviour? A mathematical procedure for
suspicion is presented: in the same way that the continued actions of an
extortionate individual might raise suspicion.

This work makes use of the Axelrod Python library~\cite{Knight2018, Knight2016}
with a large number of Prisoner Dilemma strategies available to give an
extensive numerical example of the ideas presented.  The approach is presented
in Section~\ref{sec:delta-zd-strategies}.  All of the code and data discussed
in Section~\ref{sec:numerical-experiments} is open sourced, archived and
written according to best scientific principles~\cite{Wilson2014}. The data
archive can be found at~\cite{vincent_knight_2018_1297075}.

\section{Recognising Extortion}\label{sec:delta-zd-strategies}

In~\cite{Press2012}, given a match between 2 memory-one strategies, the concept
of Zero Determinant (ZD) strategies is introduced. The main result of that paper
shows that given two memory one players \(p, q\in\mathbb{R}^4\) a linear
relationship between the players' scores could be forced by one of the players.

Using the notation of~\cite{Press2012}, assuming the utilities for player \(p\)
are given by \(S_x=(R, S, T, P)\) and for player \(q\) by \(S_y=(R, T, S, P)\)
and that the stationary scores of each player is given by \(S_X\) and \(S_Y\)
respectively. The main result of~\cite{Press2012} is that if

\begin{equation}\label{eqn:linear_relationship_for_p}
    \tilde p=\alpha S_x + \beta S_y + \gamma
\end{equation}

or

\begin{equation}\label{eqn:linear_relationship_for_q}
    \tilde q=\alpha S_x + \beta S_y + \gamma
\end{equation}

where \(\tilde p = (1 - p_1, 1 - p_2, p_3, p_4)\) and
\(\tilde q = (1 - q_1, 1 - q_2, q_3, q_4)\) then:

\begin{equation}
    \alpha S_X + \beta S_Y + \gamma = 0
\end{equation}

In~\cite{Press2012} a particular type of ZD strategy is defined: extortionate
strategies. If:

\begin{equation}\label{eqn:constraint_for_extortion}
    \gamma = - P(\alpha + \beta)
\end{equation}

then the player can ensure they get a score \(\chi\) times
larger than the opponent. This extortion coefficient is given by:

\begin{equation}\label{eqn:definition_of_chi}
    \chi=\frac{-\beta}{\alpha}
\end{equation}

Thus, if (\ref{eqn:constraint_for_extortion}) holds and \(\chi >1\) a player is
said to extort their opponent.
Here, the reverse problem is considered: given a
\(p\in\mathbb{R}^4\) how does one identify \(\alpha, \beta\) if they
exist and is the strategy in fact acting in an extortionate way?

These conditions correspond to:

\begin{align}
    \tilde p_1 & = \alpha R + \beta R - P (\alpha + \beta)
            \label{eqn:condition_for_tilde_p1}\\
    \tilde p_2 & = \alpha S + \beta T - P (\alpha + \beta)
            \label{eqn:condition_for_tilde_p2}\\
    \tilde p_3 & = \alpha T + \beta S - P (\alpha + \beta)
            \label{eqn:condition_for_tilde_p3}\\
    \tilde p_4 & = \alpha P + \beta P - P (\alpha + \beta)
            \label{eqn:condition_for_tilde_p4}
\end{align}

Equation (\ref{eqn:condition_for_tilde_p4}) ensures that \(p_4=\tilde p_4=0\).
Equations (\ref{eqn:condition_for_tilde_p1}-\ref{eqn:condition_for_tilde_p3})
can be used to eliminate \(\alpha, \beta\), giving:

\begin{equation}\label{eqn:planar_definition_of_extortion}
    \tilde p_1 = \frac{(R - P)(\tilde p_2 + \tilde p_3)}{S + T - 2P}
\end{equation}

with:

\begin{equation}\label{eqn:definition_of_chi}
    \chi = \frac{\tilde p_2 (P - T) + \tilde p_3 (S - P)}
                {\tilde p_2 (P - S) + \tilde p_3 (T - P)}
\end{equation}

Given a strategy \(p\in\mathbb{R}^{4\times 1}\) equations
(\ref{eqn:condition_for_tilde_p4}), (\ref{eqn:planar_definition_of_extortion}-\ref{eqn:definition_of_chi}) can be used to check if
a strategy is extortionate. The conditions correspond to:

\begin{align}
    p_1 & = \frac{(R-P)(p_2 + p_3) - R + T + S - P}{S + T - 2P}
     \label{eqn:condition_for_p1}\\
    p_4 & = 0 \label{eqn:condition_for_p4}\\
    1 & > p_2 + p_3\label{eqn:condition_for_chi}
\end{align}

The algebraic steps necessary to prove these results are available in the
supporting materials.

All extortionate strategies reside on a triangular (\ref{eqn:condition_for_chi})
plane (\ref{eqn:condition_for_p1}) in 3 dimensions (\ref{eqn:condition_for_p4}).
Using this formulation it can be seen that a necessary (but not sufficient)
condition for an extortionate strategy is that it cooperates on average less
than 50\% of the time when in a state of disagreement with the opponent.

As an example, consider the known extortionate strategy \(p=(8 / 9, 1 / 2, 1 /
3, 0)\) from~\cite{Stewart2012} which is referred to as \texttt{Extort-2}. In
this case, for the standard values of \((R, T, S, P)\) constraint
(\ref{eqn:condition_for_p1}) corresponds to:

\begin{equation}
    p_1 = \frac{2(p_2 + p_3) + 1}{3}
\end{equation}

It is clear that in this case all constraints hold.

This approach could in fact be used to confirm that a given strategy is acting
in an extortionate manner even if it is not a memory one strategy. However, in
practice, if a closed form for \(p\) is not known, then due to measurement
and/or numerical error this would not work.

This problem can be written in the following linear algebraic form where
\(x=(\alpha, \beta)\)
and \(p^*=(\tilde p_1 - 1, tilde_2 - 1, p_3)\):

\begin{equation}\label{eqn:linear_algebraic_equation_for_p}
    Cx= p^*
\end{equation}

\(C\) corresponds to equations
(\ref{eqn:condition_for_tilde_p1}-\ref{eqn:condition_for_tilde_p3}) and is
given by:

\begin{equation}\label{eqn:definition_of_C}
    C =
    \begin{bmatrix}
        R - P & R- P \\
        S - P & T- P \\
        T - P & S- P \\
    \end{bmatrix}
\end{equation}

Note that in general, equation (\ref{eqn:linear_algebraic_equation_for_p}) will
not necessarily have a solution. From the Rouch\'{e}-Capelli theorem if there is
a solution it is unique as \(\text{rank}(C)=2\) which is the dimension of the
variable \(x\). The best fitting \(x\) is found by minimizing:

\begin{equation}\label{eqn:r_squared}
    \text{SSError} = \|C x- p^*\|_2^2 = \sum_{i=1}^{3}\left((C\bar x)_i-p_i^*\right)^2
\end{equation}

Note that \(\text{SSError}\), which is the square of the Frobenius
norm~\cite{Golub2013}, becomes a measure of how close a strategy is to being an
extortionate strategy. Suspicion
of extortion then corresponds to a threshold on \(\text{SSError}\).

By observing interactions (human or otherwise), their memory one representation
can be inferred and this approach can be used to recognise extortionate
behaviour. The notion of comparing theoretic and actual plays of the IPD is not
novel, see for example~\cite{Rand2013}. Immediately it is noted that if the
environment is noisy~\cite{Wu1995} then no strategy can be considered to be
extortionate as \(p_4>0\).

In the next section, this idea will be illustrated by observing the interactions
that take place in a computer based tournament of the IPD\@.

\section{Numerical experiments}\label{sec:numerical-experiments}

In~\cite{Stewart2012} results from a tournament with
\documentclass[a4paper]{article}

\usepackage{amsmath}
\usepackage{amssymb}
\usepackage[margin=1.5cm,
            includefoot,
            footskip=30pt]{geometry}
\usepackage{layout}
\usepackage{graphicx}
\usepackage{subcaption}

\usepackage{biblatex}
\usepackage{pdfpages}

\bibliography{main.bib}

\title{Suspicion: Recognising and evaluating the effectiveness
       of extortion in the Iterated Prisoner's Dilemma}
\author{Vincent A. Knight \and Nikoleta E. Glynatsi}
\date{\today}



\begin{document}

\maketitle

\begin{abstract}
    The Iterated Prisoner's Dilemma is a model for rational and evolutionary
    interactive behaviour. It has applications both in the study of human social
    behaviour as well as in biology.
    It is used to understand when and how a rational individual might
    accept an immediate cost to their own utility for the direct benefit of
    another.

    Much attention has been given to a class of strategies called
    Zero Determinant strategies. It has been theoretically shown that these
    strategies can ``extort'' any player.

    In this work, an approach to identify if observed strategies are playing in
    an extortionate way is described. Furthermore, experimental analysis of
    a large tournament with \input{assets/tex/number_of_full_strategies/main.tex}
    strategies is considered. In this setting
    the most highly performing strategies do not play in an extortionate way
    against each other but do against lower performing strategies.
    This suggests that whilst the theory of Zero Determinant strategies
    indicates that memory is not of fundamental importance to the evolution of
    cooperative behaviour, this is incomplete.
\end{abstract}

\section{Introduction}\label{sec:introduction}

Agent based game theoretic models have become a stalwart of the underpinning
mathematics of interactive behaviours. One of the major pieces of work
in this area is the pair of original computer tournaments run by Robert
Axelrod~\cite{Axelrod1980, Axelrod1980a}. These tournaments pitted submitted
computer strategies against each other in plays of the Iterated Prisoner's
Dilemma. A common game where agents can choose to pay a slight cost to their
immediate utility in the hope of building a reputation. This has been used in
economic and evolutionary game theory to understand the evolution of cooperative
behaviour.

Recently, a class of strategies was described in~\cite{Press2012} that can
provably extort any given opponent. In~\cite{Hilbe2013, Moran1707} some
questions have already been asked about the true effectiveness of these
strategies in an evolutionary setting. Here another question is asked: is it
possible to recognise this extortionate behaviour? A mathematical procedure for
suspicion is presented: in the same way that the continued actions of an
extortionate individual might raise suspicion.

This work makes use of the Axelrod Python library~\cite{Knight2018, Knight2016}
with a large number of Prisoner Dilemma strategies available to give an
extensive numerical example of the ideas presented.  The approach is presented
in Section~\ref{sec:delta-zd-strategies}.  All of the code and data discussed
in Section~\ref{sec:numerical-experiments} is open sourced, archived and
written according to best scientific principles~\cite{Wilson2014}. The data
archive can be found at~\cite{vincent_knight_2018_1297075}.

\section{Recognising Extortion}\label{sec:delta-zd-strategies}

In~\cite{Press2012}, given a match between 2 memory-one strategies, the concept
of Zero Determinant (ZD) strategies is introduced. The main result of that paper
shows that given two memory one players \(p, q\in\mathbb{R}^4\) a linear
relationship between the players' scores could be forced by one of the players.

Using the notation of~\cite{Press2012}, assuming the utilities for player \(p\)
are given by \(S_x=(R, S, T, P)\) and for player \(q\) by \(S_y=(R, T, S, P)\)
and that the stationary scores of each player is given by \(S_X\) and \(S_Y\)
respectively. The main result of~\cite{Press2012} is that if

\begin{equation}\label{eqn:linear_relationship_for_p}
    \tilde p=\alpha S_x + \beta S_y + \gamma
\end{equation}

or

\begin{equation}\label{eqn:linear_relationship_for_q}
    \tilde q=\alpha S_x + \beta S_y + \gamma
\end{equation}

where \(\tilde p = (1 - p_1, 1 - p_2, p_3, p_4)\) and
\(\tilde q = (1 - q_1, 1 - q_2, q_3, q_4)\) then:

\begin{equation}
    \alpha S_X + \beta S_Y + \gamma = 0
\end{equation}

In~\cite{Press2012} a particular type of ZD strategy is defined: extortionate
strategies. If:

\begin{equation}\label{eqn:constraint_for_extortion}
    \gamma = - P(\alpha + \beta)
\end{equation}

then the player can ensure they get a score \(\chi\) times
larger than the opponent. This extortion coefficient is given by:

\begin{equation}\label{eqn:definition_of_chi}
    \chi=\frac{-\beta}{\alpha}
\end{equation}

Thus, if (\ref{eqn:constraint_for_extortion}) holds and \(\chi >1\) a player is
said to extort their opponent.
Here, the reverse problem is considered: given a
\(p\in\mathbb{R}^4\) how does one identify \(\alpha, \beta\) if they
exist and is the strategy in fact acting in an extortionate way?

These conditions correspond to:

\begin{align}
    \tilde p_1 & = \alpha R + \beta R - P (\alpha + \beta)
            \label{eqn:condition_for_tilde_p1}\\
    \tilde p_2 & = \alpha S + \beta T - P (\alpha + \beta)
            \label{eqn:condition_for_tilde_p2}\\
    \tilde p_3 & = \alpha T + \beta S - P (\alpha + \beta)
            \label{eqn:condition_for_tilde_p3}\\
    \tilde p_4 & = \alpha P + \beta P - P (\alpha + \beta)
            \label{eqn:condition_for_tilde_p4}
\end{align}

Equation (\ref{eqn:condition_for_tilde_p4}) ensures that \(p_4=\tilde p_4=0\).
Equations (\ref{eqn:condition_for_tilde_p1}-\ref{eqn:condition_for_tilde_p3})
can be used to eliminate \(\alpha, \beta\), giving:

\begin{equation}\label{eqn:planar_definition_of_extortion}
    \tilde p_1 = \frac{(R - P)(\tilde p_2 + \tilde p_3)}{S + T - 2P}
\end{equation}

with:

\begin{equation}\label{eqn:definition_of_chi}
    \chi = \frac{\tilde p_2 (P - T) + \tilde p_3 (S - P)}
                {\tilde p_2 (P - S) + \tilde p_3 (T - P)}
\end{equation}

Given a strategy \(p\in\mathbb{R}^{4\times 1}\) equations
(\ref{eqn:condition_for_tilde_p4}), (\ref{eqn:planar_definition_of_extortion}-\ref{eqn:definition_of_chi}) can be used to check if
a strategy is extortionate. The conditions correspond to:

\begin{align}
    p_1 & = \frac{(R-P)(p_2 + p_3) - R + T + S - P}{S + T - 2P}
     \label{eqn:condition_for_p1}\\
    p_4 & = 0 \label{eqn:condition_for_p4}\\
    1 & > p_2 + p_3\label{eqn:condition_for_chi}
\end{align}

The algebraic steps necessary to prove these results are available in the
supporting materials.

All extortionate strategies reside on a triangular (\ref{eqn:condition_for_chi})
plane (\ref{eqn:condition_for_p1}) in 3 dimensions (\ref{eqn:condition_for_p4}).
Using this formulation it can be seen that a necessary (but not sufficient)
condition for an extortionate strategy is that it cooperates on average less
than 50\% of the time when in a state of disagreement with the opponent.

As an example, consider the known extortionate strategy \(p=(8 / 9, 1 / 2, 1 /
3, 0)\) from~\cite{Stewart2012} which is referred to as \texttt{Extort-2}. In
this case, for the standard values of \((R, T, S, P)\) constraint
(\ref{eqn:condition_for_p1}) corresponds to:

\begin{equation}
    p_1 = \frac{2(p_2 + p_3) + 1}{3}
\end{equation}

It is clear that in this case all constraints hold.

This approach could in fact be used to confirm that a given strategy is acting
in an extortionate manner even if it is not a memory one strategy. However, in
practice, if a closed form for \(p\) is not known, then due to measurement
and/or numerical error this would not work.

This problem can be written in the following linear algebraic form where
\(x=(\alpha, \beta)\)
and \(p^*=(\tilde p_1 - 1, tilde_2 - 1, p_3)\):

\begin{equation}\label{eqn:linear_algebraic_equation_for_p}
    Cx= p^*
\end{equation}

\(C\) corresponds to equations
(\ref{eqn:condition_for_tilde_p1}-\ref{eqn:condition_for_tilde_p3}) and is
given by:

\begin{equation}\label{eqn:definition_of_C}
    C =
    \begin{bmatrix}
        R - P & R- P \\
        S - P & T- P \\
        T - P & S- P \\
    \end{bmatrix}
\end{equation}

Note that in general, equation (\ref{eqn:linear_algebraic_equation_for_p}) will
not necessarily have a solution. From the Rouch\'{e}-Capelli theorem if there is
a solution it is unique as \(\text{rank}(C)=2\) which is the dimension of the
variable \(x\). The best fitting \(x\) is found by minimizing:

\begin{equation}\label{eqn:r_squared}
    \text{SSError} = \|C x- p^*\|_2^2 = \sum_{i=1}^{3}\left((C\bar x)_i-p_i^*\right)^2
\end{equation}

Note that \(\text{SSError}\), which is the square of the Frobenius
norm~\cite{Golub2013}, becomes a measure of how close a strategy is to being an
extortionate strategy. Suspicion
of extortion then corresponds to a threshold on \(\text{SSError}\).

By observing interactions (human or otherwise), their memory one representation
can be inferred and this approach can be used to recognise extortionate
behaviour. The notion of comparing theoretic and actual plays of the IPD is not
novel, see for example~\cite{Rand2013}. Immediately it is noted that if the
environment is noisy~\cite{Wu1995} then no strategy can be considered to be
extortionate as \(p_4>0\).

In the next section, this idea will be illustrated by observing the interactions
that take place in a computer based tournament of the IPD\@.

\section{Numerical experiments}\label{sec:numerical-experiments}

In~\cite{Stewart2012} results from a tournament with
\input{./assets/tex/number_of_stewart_plotkin_strategies/main.tex} strategies,
was presented with specific consideration given to ZD strategies. This
tournament is reproduced here using the Axelrod-Python
project~\cite{Knight2016}. To obtain a good measure of the corresponding
transition rates for each strategy all matches have been run for
\input{assets/tex/number_of_turns/main.tex} turns and every match has been
repeated \input{assets/tex/number_of_repetitions/main.tex} times. All of this
interaction data is available at~\cite{vincent_knight_2018_1297075}. A good
match between the inferred Markov chain and the state distribution of the actual
interactions has been verified. Data for this is presented in the supplementary
materials.

Figure~\ref{fig:SSError_overall_in_stewart_plotkin} shows the \(\text{SSError}\)
values for all the strategies in the tournament, as reported
in~\cite{Stewart2012} the extortionate strategy (which has an expected
\(\text{SSError}\) approximately 0) gains a large number of wins.

\begin{figure}[!htbp]
    \centering
    \includegraphics[width=.8\textwidth]{./assets/img/SSError_overall_in_stewart_plotkin/main.pdf}
    \caption{\(\text{SSError}\) and state probabilities for the strategies
        of~\cite{Stewart2012}, ordered both by number of wins and overall score.
        Note that \(P(DC)\) is not shown as it corresponds to the transpose of
        \(P(CD)\). Cooperator and Defector are omitted as they do not visit all
        the states.}
    \label{fig:SSError_overall_in_stewart_plotkin}
\end{figure}

Here, the work of~\cite{Stewart2012} is extended by investigating a tournament
with \input{assets/tex/number_of_full_strategies/main.tex}
strategies.

The results of this analysis are shown in
Figure~\ref{fig:SSError_and_probabilities_in_full}. The top ranking strategies
by number of wins seem to be extortionate (but not against all strategies) and
it can be seen that a small sub group of strategies achieve mutual defection.
All the top ranking strategies according to score achieve mutual cooperation and
do not extort each other, however they
\textbf{do} exhibit extortionate behaviour towards a number of the lower ranking
strategies.

\begin{figure}[!htbp]
    \centering
    \includegraphics[width=.8\textwidth]{./assets/img/SSError_and_probabilities_in_full/main.pdf}
    \caption{\(\text{SSError}\) for the strategies for the full tournament. Only
    strategy interactions for which \(p_4=0\) and \(\chi>1\) are displayed.}
    \label{fig:SSError_and_probabilities_in_full}
\end{figure}

\section{Conclusion}\label{sec:conclusion}

This work defines an approach to measure whether or not a player is playing a
strategy that corresponds to an extortionate strategy as defined
in~\cite{Press2012}: a mathematical model for suspicion. Indeed, all
extortionate strategies have been
 classified as lying on a triangular plane.
This rigorous classification fails to be robust to small measurement error, thus
a statistical approach is proposed.
This is done through a linear algebraic approach for approximating the solution
of a linear system. Using this, a large number of pairwise interactions is
simulated and in fact very few strategies are found to act extortionately.

The work of~\cite{Press2012}, whilst showing that a clever approach to taking
advantage of another memory one strategy exists: this is incomplete. Whilst the
elegance of this result is very attractive, just as the simplicity of the
victory of Tit For Tat in Axelrod's original tournaments was, it is incomplete.
Extortionate strategies achieve a high number of wins but they do not
achieve a high score which corresponds to the fitness landscape in an
evolutionary sense. From the large number of interactions a payoff matrix \(S\)
can be measured where \(S_{ij}\) denotes the score (using standard values of
\((R, S, T, P) = (3, 0, 5, 1)\)) of the \(i\)th strategy
against the \(j\)th strategy. Using this, the replicator equation
describes the evolution of the system based on a population density fitness
function:

\begin{equation}\label{eqn:replicator_dynamics}
    \frac{dx}{dt} = x(S-x^TS x)
\end{equation}

Equation (\ref{eqn:replicator_dynamics}) is solved numerically through an
integration technique described in~\cite{Petzold1983} and
Figure~\ref{fig:replicator_dynamics} shows the evolution of the distribution of
the system: the various strategies are ranked by scores. It is clear to see that
only the high ranking strategies survive the evolutionary process (in fact,
only \input{./assets/img/replicator_dynamics/main.tex}
have a final distribution greater than \(10 ^ {-2}\)). This confirms the
findings of~\cite{Moran1707} in which sophisticated strategies resist
evolutionary invasion of shorter memory strategies. Recalling
Figure~\ref{fig:SSError_and_probabilities_in_full} this demonstrates that:

\begin{itemize}
    \item Cooperation emerges through the evolutionary process: the high scoring
        strategies do not exhibit extortionate behaviour towards each other.
    \item Extortionate strategies do not survive the evolutionary process.
\end{itemize}

\begin{figure}[!htbp]
    \centering
    \includegraphics[width=.8\textwidth]{./assets/img/replicator_dynamics/main.pdf}
    \caption{Numerical simulation of the replicator equation
    (\ref{eqn:replicator_dynamics}): strategies are ordered by score, only the strategies with a high score survive the evolutionary process.}
    \label{fig:replicator_dynamics}
\end{figure}

This work can be used to classify plays of the IPD\@: data can be collected from
actual interactions (in lab or in the field). Furthermore, this allows for a
classification method similar to the notion of fingerprinting presented
in~\cite{Ashlock2008}. Trained strategies can potentially be classified as
extortionate or not or it could be possible to even constrain the reinforcement
learning approaches that are becoming prevalent in the literature.
Alternatively, this mathematical approach for recognising extortion could be
used in sophisticated strategies to defend against invasion. Arguably, some of
the strategies considered here exhibit this behaviour, indeed as described
in~\cite{Harper2017}, the top ranking strategies in the full tournament are
obtained using evolutionary reinforcement learning techniques, thus, suspicion
of extortionate behaviour could in fact be an evolutionary trait.

\section*{Acknowledgements}

The following open source software libraries were used in this research:

\begin{itemize}
    \item The Axelrod ~\cite{Knight2016, Knight2018} library (IPD strategies and
        tournaments).
    \item The sympy library~\cite{Meurer2017} (verification of all symbolic
        calculations).
    \item The matplotlib~\cite{Droettboom2018} library (visualisation).
    \item The pandas~\cite{Structures2010}, dask~\cite{Dask2016} and
        NumPy~\cite{Oliphant2015} libraries (data manipulation).
    \item The SciPy~\cite{Jones2001} library (numerical integration of the
        replicator equation).
\end{itemize}

This work was performed using the computational facilities of the Advanced
Research Computing @ Cardiff (ARCCA) Division, Cardiff University.

\printbibliography

\newpage
\section*{Supplementary materials}

\includepdf{assets/pdf/proof_of_form_of_extortionate_strategies/main.pdf}

\newpage

Using the pair wise interactions the transition rates \(p,
q\) can be measured and the steady state probabilities inferred and compared to
the actual probabilities of each state.
This is done numerically by computing the singular eigenvector of the
matrix \(A\) \cite{Stewart2009}:

\[
    A =
    \begin{bmatrix}
        p_1 q_1 & p_1 (1 - q_1) & (1 - p_1) q_1 & (1 -p_1) (1 - q_1) \\
        p_2 q_2 & p_2 (1 - q_2) & (1 - p_2) q_2 & (1 -p_2) (1 - q_2) \\
        p_3 q_3 & p_3 (1 - q_3) & (1 - p_3) q_3 & (1 -p_3) (1 - q_3) \\
        p_4 q_4 & p_4 (1 - q_4) & (1 - p_4) q_4 & (1 -p_4) (1 - q_4) \\
    \end{bmatrix}
\]

Figure~\ref{fig:computed_probabilities_vs_theoretic_probabilities} shows a
regression line fitted to every pairwise interaction with a reported
\(\text{SSError}\) value (pairwise interactions with missing states were
omitted). This serves to validate the approach: a part from some edge cases the
relationship is consistent.

\begin{figure}[!htbp]
    \centering
    \includegraphics[width=.8\textwidth]{./assets/img/computed_probabilities_vs_theoretic_probabilities/main.pdf}
    \caption{The
        relationship between the steady state probabilities inferred from the
        measured transitions and the actual steady state probabilities. A linear
        regression line is included validating the approach.}
    \label{fig:computed_probabilities_vs_theoretic_probabilities}
\end{figure}


\end{document}
 strategies,
was presented with specific consideration given to ZD strategies. This
tournament is reproduced here using the Axelrod-Python
project~\cite{Knight2016}. To obtain a good measure of the corresponding
transition rates for each strategy all matches have been run for
\documentclass[a4paper]{article}

\usepackage{amsmath}
\usepackage{amssymb}
\usepackage[margin=1.5cm,
            includefoot,
            footskip=30pt]{geometry}
\usepackage{layout}
\usepackage{graphicx}
\usepackage{subcaption}

\usepackage{biblatex}
\usepackage{pdfpages}

\bibliography{main.bib}

\title{Suspicion: Recognising and evaluating the effectiveness
       of extortion in the Iterated Prisoner's Dilemma}
\author{Vincent A. Knight \and Nikoleta E. Glynatsi}
\date{\today}



\begin{document}

\maketitle

\begin{abstract}
    The Iterated Prisoner's Dilemma is a model for rational and evolutionary
    interactive behaviour. It has applications both in the study of human social
    behaviour as well as in biology.
    It is used to understand when and how a rational individual might
    accept an immediate cost to their own utility for the direct benefit of
    another.

    Much attention has been given to a class of strategies called
    Zero Determinant strategies. It has been theoretically shown that these
    strategies can ``extort'' any player.

    In this work, an approach to identify if observed strategies are playing in
    an extortionate way is described. Furthermore, experimental analysis of
    a large tournament with \input{assets/tex/number_of_full_strategies/main.tex}
    strategies is considered. In this setting
    the most highly performing strategies do not play in an extortionate way
    against each other but do against lower performing strategies.
    This suggests that whilst the theory of Zero Determinant strategies
    indicates that memory is not of fundamental importance to the evolution of
    cooperative behaviour, this is incomplete.
\end{abstract}

\section{Introduction}\label{sec:introduction}

Agent based game theoretic models have become a stalwart of the underpinning
mathematics of interactive behaviours. One of the major pieces of work
in this area is the pair of original computer tournaments run by Robert
Axelrod~\cite{Axelrod1980, Axelrod1980a}. These tournaments pitted submitted
computer strategies against each other in plays of the Iterated Prisoner's
Dilemma. A common game where agents can choose to pay a slight cost to their
immediate utility in the hope of building a reputation. This has been used in
economic and evolutionary game theory to understand the evolution of cooperative
behaviour.

Recently, a class of strategies was described in~\cite{Press2012} that can
provably extort any given opponent. In~\cite{Hilbe2013, Moran1707} some
questions have already been asked about the true effectiveness of these
strategies in an evolutionary setting. Here another question is asked: is it
possible to recognise this extortionate behaviour? A mathematical procedure for
suspicion is presented: in the same way that the continued actions of an
extortionate individual might raise suspicion.

This work makes use of the Axelrod Python library~\cite{Knight2018, Knight2016}
with a large number of Prisoner Dilemma strategies available to give an
extensive numerical example of the ideas presented.  The approach is presented
in Section~\ref{sec:delta-zd-strategies}.  All of the code and data discussed
in Section~\ref{sec:numerical-experiments} is open sourced, archived and
written according to best scientific principles~\cite{Wilson2014}. The data
archive can be found at~\cite{vincent_knight_2018_1297075}.

\section{Recognising Extortion}\label{sec:delta-zd-strategies}

In~\cite{Press2012}, given a match between 2 memory-one strategies, the concept
of Zero Determinant (ZD) strategies is introduced. The main result of that paper
shows that given two memory one players \(p, q\in\mathbb{R}^4\) a linear
relationship between the players' scores could be forced by one of the players.

Using the notation of~\cite{Press2012}, assuming the utilities for player \(p\)
are given by \(S_x=(R, S, T, P)\) and for player \(q\) by \(S_y=(R, T, S, P)\)
and that the stationary scores of each player is given by \(S_X\) and \(S_Y\)
respectively. The main result of~\cite{Press2012} is that if

\begin{equation}\label{eqn:linear_relationship_for_p}
    \tilde p=\alpha S_x + \beta S_y + \gamma
\end{equation}

or

\begin{equation}\label{eqn:linear_relationship_for_q}
    \tilde q=\alpha S_x + \beta S_y + \gamma
\end{equation}

where \(\tilde p = (1 - p_1, 1 - p_2, p_3, p_4)\) and
\(\tilde q = (1 - q_1, 1 - q_2, q_3, q_4)\) then:

\begin{equation}
    \alpha S_X + \beta S_Y + \gamma = 0
\end{equation}

In~\cite{Press2012} a particular type of ZD strategy is defined: extortionate
strategies. If:

\begin{equation}\label{eqn:constraint_for_extortion}
    \gamma = - P(\alpha + \beta)
\end{equation}

then the player can ensure they get a score \(\chi\) times
larger than the opponent. This extortion coefficient is given by:

\begin{equation}\label{eqn:definition_of_chi}
    \chi=\frac{-\beta}{\alpha}
\end{equation}

Thus, if (\ref{eqn:constraint_for_extortion}) holds and \(\chi >1\) a player is
said to extort their opponent.
Here, the reverse problem is considered: given a
\(p\in\mathbb{R}^4\) how does one identify \(\alpha, \beta\) if they
exist and is the strategy in fact acting in an extortionate way?

These conditions correspond to:

\begin{align}
    \tilde p_1 & = \alpha R + \beta R - P (\alpha + \beta)
            \label{eqn:condition_for_tilde_p1}\\
    \tilde p_2 & = \alpha S + \beta T - P (\alpha + \beta)
            \label{eqn:condition_for_tilde_p2}\\
    \tilde p_3 & = \alpha T + \beta S - P (\alpha + \beta)
            \label{eqn:condition_for_tilde_p3}\\
    \tilde p_4 & = \alpha P + \beta P - P (\alpha + \beta)
            \label{eqn:condition_for_tilde_p4}
\end{align}

Equation (\ref{eqn:condition_for_tilde_p4}) ensures that \(p_4=\tilde p_4=0\).
Equations (\ref{eqn:condition_for_tilde_p1}-\ref{eqn:condition_for_tilde_p3})
can be used to eliminate \(\alpha, \beta\), giving:

\begin{equation}\label{eqn:planar_definition_of_extortion}
    \tilde p_1 = \frac{(R - P)(\tilde p_2 + \tilde p_3)}{S + T - 2P}
\end{equation}

with:

\begin{equation}\label{eqn:definition_of_chi}
    \chi = \frac{\tilde p_2 (P - T) + \tilde p_3 (S - P)}
                {\tilde p_2 (P - S) + \tilde p_3 (T - P)}
\end{equation}

Given a strategy \(p\in\mathbb{R}^{4\times 1}\) equations
(\ref{eqn:condition_for_tilde_p4}), (\ref{eqn:planar_definition_of_extortion}-\ref{eqn:definition_of_chi}) can be used to check if
a strategy is extortionate. The conditions correspond to:

\begin{align}
    p_1 & = \frac{(R-P)(p_2 + p_3) - R + T + S - P}{S + T - 2P}
     \label{eqn:condition_for_p1}\\
    p_4 & = 0 \label{eqn:condition_for_p4}\\
    1 & > p_2 + p_3\label{eqn:condition_for_chi}
\end{align}

The algebraic steps necessary to prove these results are available in the
supporting materials.

All extortionate strategies reside on a triangular (\ref{eqn:condition_for_chi})
plane (\ref{eqn:condition_for_p1}) in 3 dimensions (\ref{eqn:condition_for_p4}).
Using this formulation it can be seen that a necessary (but not sufficient)
condition for an extortionate strategy is that it cooperates on average less
than 50\% of the time when in a state of disagreement with the opponent.

As an example, consider the known extortionate strategy \(p=(8 / 9, 1 / 2, 1 /
3, 0)\) from~\cite{Stewart2012} which is referred to as \texttt{Extort-2}. In
this case, for the standard values of \((R, T, S, P)\) constraint
(\ref{eqn:condition_for_p1}) corresponds to:

\begin{equation}
    p_1 = \frac{2(p_2 + p_3) + 1}{3}
\end{equation}

It is clear that in this case all constraints hold.

This approach could in fact be used to confirm that a given strategy is acting
in an extortionate manner even if it is not a memory one strategy. However, in
practice, if a closed form for \(p\) is not known, then due to measurement
and/or numerical error this would not work.

This problem can be written in the following linear algebraic form where
\(x=(\alpha, \beta)\)
and \(p^*=(\tilde p_1 - 1, tilde_2 - 1, p_3)\):

\begin{equation}\label{eqn:linear_algebraic_equation_for_p}
    Cx= p^*
\end{equation}

\(C\) corresponds to equations
(\ref{eqn:condition_for_tilde_p1}-\ref{eqn:condition_for_tilde_p3}) and is
given by:

\begin{equation}\label{eqn:definition_of_C}
    C =
    \begin{bmatrix}
        R - P & R- P \\
        S - P & T- P \\
        T - P & S- P \\
    \end{bmatrix}
\end{equation}

Note that in general, equation (\ref{eqn:linear_algebraic_equation_for_p}) will
not necessarily have a solution. From the Rouch\'{e}-Capelli theorem if there is
a solution it is unique as \(\text{rank}(C)=2\) which is the dimension of the
variable \(x\). The best fitting \(x\) is found by minimizing:

\begin{equation}\label{eqn:r_squared}
    \text{SSError} = \|C x- p^*\|_2^2 = \sum_{i=1}^{3}\left((C\bar x)_i-p_i^*\right)^2
\end{equation}

Note that \(\text{SSError}\), which is the square of the Frobenius
norm~\cite{Golub2013}, becomes a measure of how close a strategy is to being an
extortionate strategy. Suspicion
of extortion then corresponds to a threshold on \(\text{SSError}\).

By observing interactions (human or otherwise), their memory one representation
can be inferred and this approach can be used to recognise extortionate
behaviour. The notion of comparing theoretic and actual plays of the IPD is not
novel, see for example~\cite{Rand2013}. Immediately it is noted that if the
environment is noisy~\cite{Wu1995} then no strategy can be considered to be
extortionate as \(p_4>0\).

In the next section, this idea will be illustrated by observing the interactions
that take place in a computer based tournament of the IPD\@.

\section{Numerical experiments}\label{sec:numerical-experiments}

In~\cite{Stewart2012} results from a tournament with
\input{./assets/tex/number_of_stewart_plotkin_strategies/main.tex} strategies,
was presented with specific consideration given to ZD strategies. This
tournament is reproduced here using the Axelrod-Python
project~\cite{Knight2016}. To obtain a good measure of the corresponding
transition rates for each strategy all matches have been run for
\input{assets/tex/number_of_turns/main.tex} turns and every match has been
repeated \input{assets/tex/number_of_repetitions/main.tex} times. All of this
interaction data is available at~\cite{vincent_knight_2018_1297075}. A good
match between the inferred Markov chain and the state distribution of the actual
interactions has been verified. Data for this is presented in the supplementary
materials.

Figure~\ref{fig:SSError_overall_in_stewart_plotkin} shows the \(\text{SSError}\)
values for all the strategies in the tournament, as reported
in~\cite{Stewart2012} the extortionate strategy (which has an expected
\(\text{SSError}\) approximately 0) gains a large number of wins.

\begin{figure}[!htbp]
    \centering
    \includegraphics[width=.8\textwidth]{./assets/img/SSError_overall_in_stewart_plotkin/main.pdf}
    \caption{\(\text{SSError}\) and state probabilities for the strategies
        of~\cite{Stewart2012}, ordered both by number of wins and overall score.
        Note that \(P(DC)\) is not shown as it corresponds to the transpose of
        \(P(CD)\). Cooperator and Defector are omitted as they do not visit all
        the states.}
    \label{fig:SSError_overall_in_stewart_plotkin}
\end{figure}

Here, the work of~\cite{Stewart2012} is extended by investigating a tournament
with \input{assets/tex/number_of_full_strategies/main.tex}
strategies.

The results of this analysis are shown in
Figure~\ref{fig:SSError_and_probabilities_in_full}. The top ranking strategies
by number of wins seem to be extortionate (but not against all strategies) and
it can be seen that a small sub group of strategies achieve mutual defection.
All the top ranking strategies according to score achieve mutual cooperation and
do not extort each other, however they
\textbf{do} exhibit extortionate behaviour towards a number of the lower ranking
strategies.

\begin{figure}[!htbp]
    \centering
    \includegraphics[width=.8\textwidth]{./assets/img/SSError_and_probabilities_in_full/main.pdf}
    \caption{\(\text{SSError}\) for the strategies for the full tournament. Only
    strategy interactions for which \(p_4=0\) and \(\chi>1\) are displayed.}
    \label{fig:SSError_and_probabilities_in_full}
\end{figure}

\section{Conclusion}\label{sec:conclusion}

This work defines an approach to measure whether or not a player is playing a
strategy that corresponds to an extortionate strategy as defined
in~\cite{Press2012}: a mathematical model for suspicion. Indeed, all
extortionate strategies have been
 classified as lying on a triangular plane.
This rigorous classification fails to be robust to small measurement error, thus
a statistical approach is proposed.
This is done through a linear algebraic approach for approximating the solution
of a linear system. Using this, a large number of pairwise interactions is
simulated and in fact very few strategies are found to act extortionately.

The work of~\cite{Press2012}, whilst showing that a clever approach to taking
advantage of another memory one strategy exists: this is incomplete. Whilst the
elegance of this result is very attractive, just as the simplicity of the
victory of Tit For Tat in Axelrod's original tournaments was, it is incomplete.
Extortionate strategies achieve a high number of wins but they do not
achieve a high score which corresponds to the fitness landscape in an
evolutionary sense. From the large number of interactions a payoff matrix \(S\)
can be measured where \(S_{ij}\) denotes the score (using standard values of
\((R, S, T, P) = (3, 0, 5, 1)\)) of the \(i\)th strategy
against the \(j\)th strategy. Using this, the replicator equation
describes the evolution of the system based on a population density fitness
function:

\begin{equation}\label{eqn:replicator_dynamics}
    \frac{dx}{dt} = x(S-x^TS x)
\end{equation}

Equation (\ref{eqn:replicator_dynamics}) is solved numerically through an
integration technique described in~\cite{Petzold1983} and
Figure~\ref{fig:replicator_dynamics} shows the evolution of the distribution of
the system: the various strategies are ranked by scores. It is clear to see that
only the high ranking strategies survive the evolutionary process (in fact,
only \input{./assets/img/replicator_dynamics/main.tex}
have a final distribution greater than \(10 ^ {-2}\)). This confirms the
findings of~\cite{Moran1707} in which sophisticated strategies resist
evolutionary invasion of shorter memory strategies. Recalling
Figure~\ref{fig:SSError_and_probabilities_in_full} this demonstrates that:

\begin{itemize}
    \item Cooperation emerges through the evolutionary process: the high scoring
        strategies do not exhibit extortionate behaviour towards each other.
    \item Extortionate strategies do not survive the evolutionary process.
\end{itemize}

\begin{figure}[!htbp]
    \centering
    \includegraphics[width=.8\textwidth]{./assets/img/replicator_dynamics/main.pdf}
    \caption{Numerical simulation of the replicator equation
    (\ref{eqn:replicator_dynamics}): strategies are ordered by score, only the strategies with a high score survive the evolutionary process.}
    \label{fig:replicator_dynamics}
\end{figure}

This work can be used to classify plays of the IPD\@: data can be collected from
actual interactions (in lab or in the field). Furthermore, this allows for a
classification method similar to the notion of fingerprinting presented
in~\cite{Ashlock2008}. Trained strategies can potentially be classified as
extortionate or not or it could be possible to even constrain the reinforcement
learning approaches that are becoming prevalent in the literature.
Alternatively, this mathematical approach for recognising extortion could be
used in sophisticated strategies to defend against invasion. Arguably, some of
the strategies considered here exhibit this behaviour, indeed as described
in~\cite{Harper2017}, the top ranking strategies in the full tournament are
obtained using evolutionary reinforcement learning techniques, thus, suspicion
of extortionate behaviour could in fact be an evolutionary trait.

\section*{Acknowledgements}

The following open source software libraries were used in this research:

\begin{itemize}
    \item The Axelrod ~\cite{Knight2016, Knight2018} library (IPD strategies and
        tournaments).
    \item The sympy library~\cite{Meurer2017} (verification of all symbolic
        calculations).
    \item The matplotlib~\cite{Droettboom2018} library (visualisation).
    \item The pandas~\cite{Structures2010}, dask~\cite{Dask2016} and
        NumPy~\cite{Oliphant2015} libraries (data manipulation).
    \item The SciPy~\cite{Jones2001} library (numerical integration of the
        replicator equation).
\end{itemize}

This work was performed using the computational facilities of the Advanced
Research Computing @ Cardiff (ARCCA) Division, Cardiff University.

\printbibliography

\newpage
\section*{Supplementary materials}

\includepdf{assets/pdf/proof_of_form_of_extortionate_strategies/main.pdf}

\newpage

Using the pair wise interactions the transition rates \(p,
q\) can be measured and the steady state probabilities inferred and compared to
the actual probabilities of each state.
This is done numerically by computing the singular eigenvector of the
matrix \(A\) \cite{Stewart2009}:

\[
    A =
    \begin{bmatrix}
        p_1 q_1 & p_1 (1 - q_1) & (1 - p_1) q_1 & (1 -p_1) (1 - q_1) \\
        p_2 q_2 & p_2 (1 - q_2) & (1 - p_2) q_2 & (1 -p_2) (1 - q_2) \\
        p_3 q_3 & p_3 (1 - q_3) & (1 - p_3) q_3 & (1 -p_3) (1 - q_3) \\
        p_4 q_4 & p_4 (1 - q_4) & (1 - p_4) q_4 & (1 -p_4) (1 - q_4) \\
    \end{bmatrix}
\]

Figure~\ref{fig:computed_probabilities_vs_theoretic_probabilities} shows a
regression line fitted to every pairwise interaction with a reported
\(\text{SSError}\) value (pairwise interactions with missing states were
omitted). This serves to validate the approach: a part from some edge cases the
relationship is consistent.

\begin{figure}[!htbp]
    \centering
    \includegraphics[width=.8\textwidth]{./assets/img/computed_probabilities_vs_theoretic_probabilities/main.pdf}
    \caption{The
        relationship between the steady state probabilities inferred from the
        measured transitions and the actual steady state probabilities. A linear
        regression line is included validating the approach.}
    \label{fig:computed_probabilities_vs_theoretic_probabilities}
\end{figure}


\end{document}
 turns and every match has been
repeated \documentclass[a4paper]{article}

\usepackage{amsmath}
\usepackage{amssymb}
\usepackage[margin=1.5cm,
            includefoot,
            footskip=30pt]{geometry}
\usepackage{layout}
\usepackage{graphicx}
\usepackage{subcaption}

\usepackage{biblatex}
\usepackage{pdfpages}

\bibliography{main.bib}

\title{Suspicion: Recognising and evaluating the effectiveness
       of extortion in the Iterated Prisoner's Dilemma}
\author{Vincent A. Knight \and Nikoleta E. Glynatsi}
\date{\today}



\begin{document}

\maketitle

\begin{abstract}
    The Iterated Prisoner's Dilemma is a model for rational and evolutionary
    interactive behaviour. It has applications both in the study of human social
    behaviour as well as in biology.
    It is used to understand when and how a rational individual might
    accept an immediate cost to their own utility for the direct benefit of
    another.

    Much attention has been given to a class of strategies called
    Zero Determinant strategies. It has been theoretically shown that these
    strategies can ``extort'' any player.

    In this work, an approach to identify if observed strategies are playing in
    an extortionate way is described. Furthermore, experimental analysis of
    a large tournament with \input{assets/tex/number_of_full_strategies/main.tex}
    strategies is considered. In this setting
    the most highly performing strategies do not play in an extortionate way
    against each other but do against lower performing strategies.
    This suggests that whilst the theory of Zero Determinant strategies
    indicates that memory is not of fundamental importance to the evolution of
    cooperative behaviour, this is incomplete.
\end{abstract}

\section{Introduction}\label{sec:introduction}

Agent based game theoretic models have become a stalwart of the underpinning
mathematics of interactive behaviours. One of the major pieces of work
in this area is the pair of original computer tournaments run by Robert
Axelrod~\cite{Axelrod1980, Axelrod1980a}. These tournaments pitted submitted
computer strategies against each other in plays of the Iterated Prisoner's
Dilemma. A common game where agents can choose to pay a slight cost to their
immediate utility in the hope of building a reputation. This has been used in
economic and evolutionary game theory to understand the evolution of cooperative
behaviour.

Recently, a class of strategies was described in~\cite{Press2012} that can
provably extort any given opponent. In~\cite{Hilbe2013, Moran1707} some
questions have already been asked about the true effectiveness of these
strategies in an evolutionary setting. Here another question is asked: is it
possible to recognise this extortionate behaviour? A mathematical procedure for
suspicion is presented: in the same way that the continued actions of an
extortionate individual might raise suspicion.

This work makes use of the Axelrod Python library~\cite{Knight2018, Knight2016}
with a large number of Prisoner Dilemma strategies available to give an
extensive numerical example of the ideas presented.  The approach is presented
in Section~\ref{sec:delta-zd-strategies}.  All of the code and data discussed
in Section~\ref{sec:numerical-experiments} is open sourced, archived and
written according to best scientific principles~\cite{Wilson2014}. The data
archive can be found at~\cite{vincent_knight_2018_1297075}.

\section{Recognising Extortion}\label{sec:delta-zd-strategies}

In~\cite{Press2012}, given a match between 2 memory-one strategies, the concept
of Zero Determinant (ZD) strategies is introduced. The main result of that paper
shows that given two memory one players \(p, q\in\mathbb{R}^4\) a linear
relationship between the players' scores could be forced by one of the players.

Using the notation of~\cite{Press2012}, assuming the utilities for player \(p\)
are given by \(S_x=(R, S, T, P)\) and for player \(q\) by \(S_y=(R, T, S, P)\)
and that the stationary scores of each player is given by \(S_X\) and \(S_Y\)
respectively. The main result of~\cite{Press2012} is that if

\begin{equation}\label{eqn:linear_relationship_for_p}
    \tilde p=\alpha S_x + \beta S_y + \gamma
\end{equation}

or

\begin{equation}\label{eqn:linear_relationship_for_q}
    \tilde q=\alpha S_x + \beta S_y + \gamma
\end{equation}

where \(\tilde p = (1 - p_1, 1 - p_2, p_3, p_4)\) and
\(\tilde q = (1 - q_1, 1 - q_2, q_3, q_4)\) then:

\begin{equation}
    \alpha S_X + \beta S_Y + \gamma = 0
\end{equation}

In~\cite{Press2012} a particular type of ZD strategy is defined: extortionate
strategies. If:

\begin{equation}\label{eqn:constraint_for_extortion}
    \gamma = - P(\alpha + \beta)
\end{equation}

then the player can ensure they get a score \(\chi\) times
larger than the opponent. This extortion coefficient is given by:

\begin{equation}\label{eqn:definition_of_chi}
    \chi=\frac{-\beta}{\alpha}
\end{equation}

Thus, if (\ref{eqn:constraint_for_extortion}) holds and \(\chi >1\) a player is
said to extort their opponent.
Here, the reverse problem is considered: given a
\(p\in\mathbb{R}^4\) how does one identify \(\alpha, \beta\) if they
exist and is the strategy in fact acting in an extortionate way?

These conditions correspond to:

\begin{align}
    \tilde p_1 & = \alpha R + \beta R - P (\alpha + \beta)
            \label{eqn:condition_for_tilde_p1}\\
    \tilde p_2 & = \alpha S + \beta T - P (\alpha + \beta)
            \label{eqn:condition_for_tilde_p2}\\
    \tilde p_3 & = \alpha T + \beta S - P (\alpha + \beta)
            \label{eqn:condition_for_tilde_p3}\\
    \tilde p_4 & = \alpha P + \beta P - P (\alpha + \beta)
            \label{eqn:condition_for_tilde_p4}
\end{align}

Equation (\ref{eqn:condition_for_tilde_p4}) ensures that \(p_4=\tilde p_4=0\).
Equations (\ref{eqn:condition_for_tilde_p1}-\ref{eqn:condition_for_tilde_p3})
can be used to eliminate \(\alpha, \beta\), giving:

\begin{equation}\label{eqn:planar_definition_of_extortion}
    \tilde p_1 = \frac{(R - P)(\tilde p_2 + \tilde p_3)}{S + T - 2P}
\end{equation}

with:

\begin{equation}\label{eqn:definition_of_chi}
    \chi = \frac{\tilde p_2 (P - T) + \tilde p_3 (S - P)}
                {\tilde p_2 (P - S) + \tilde p_3 (T - P)}
\end{equation}

Given a strategy \(p\in\mathbb{R}^{4\times 1}\) equations
(\ref{eqn:condition_for_tilde_p4}), (\ref{eqn:planar_definition_of_extortion}-\ref{eqn:definition_of_chi}) can be used to check if
a strategy is extortionate. The conditions correspond to:

\begin{align}
    p_1 & = \frac{(R-P)(p_2 + p_3) - R + T + S - P}{S + T - 2P}
     \label{eqn:condition_for_p1}\\
    p_4 & = 0 \label{eqn:condition_for_p4}\\
    1 & > p_2 + p_3\label{eqn:condition_for_chi}
\end{align}

The algebraic steps necessary to prove these results are available in the
supporting materials.

All extortionate strategies reside on a triangular (\ref{eqn:condition_for_chi})
plane (\ref{eqn:condition_for_p1}) in 3 dimensions (\ref{eqn:condition_for_p4}).
Using this formulation it can be seen that a necessary (but not sufficient)
condition for an extortionate strategy is that it cooperates on average less
than 50\% of the time when in a state of disagreement with the opponent.

As an example, consider the known extortionate strategy \(p=(8 / 9, 1 / 2, 1 /
3, 0)\) from~\cite{Stewart2012} which is referred to as \texttt{Extort-2}. In
this case, for the standard values of \((R, T, S, P)\) constraint
(\ref{eqn:condition_for_p1}) corresponds to:

\begin{equation}
    p_1 = \frac{2(p_2 + p_3) + 1}{3}
\end{equation}

It is clear that in this case all constraints hold.

This approach could in fact be used to confirm that a given strategy is acting
in an extortionate manner even if it is not a memory one strategy. However, in
practice, if a closed form for \(p\) is not known, then due to measurement
and/or numerical error this would not work.

This problem can be written in the following linear algebraic form where
\(x=(\alpha, \beta)\)
and \(p^*=(\tilde p_1 - 1, tilde_2 - 1, p_3)\):

\begin{equation}\label{eqn:linear_algebraic_equation_for_p}
    Cx= p^*
\end{equation}

\(C\) corresponds to equations
(\ref{eqn:condition_for_tilde_p1}-\ref{eqn:condition_for_tilde_p3}) and is
given by:

\begin{equation}\label{eqn:definition_of_C}
    C =
    \begin{bmatrix}
        R - P & R- P \\
        S - P & T- P \\
        T - P & S- P \\
    \end{bmatrix}
\end{equation}

Note that in general, equation (\ref{eqn:linear_algebraic_equation_for_p}) will
not necessarily have a solution. From the Rouch\'{e}-Capelli theorem if there is
a solution it is unique as \(\text{rank}(C)=2\) which is the dimension of the
variable \(x\). The best fitting \(x\) is found by minimizing:

\begin{equation}\label{eqn:r_squared}
    \text{SSError} = \|C x- p^*\|_2^2 = \sum_{i=1}^{3}\left((C\bar x)_i-p_i^*\right)^2
\end{equation}

Note that \(\text{SSError}\), which is the square of the Frobenius
norm~\cite{Golub2013}, becomes a measure of how close a strategy is to being an
extortionate strategy. Suspicion
of extortion then corresponds to a threshold on \(\text{SSError}\).

By observing interactions (human or otherwise), their memory one representation
can be inferred and this approach can be used to recognise extortionate
behaviour. The notion of comparing theoretic and actual plays of the IPD is not
novel, see for example~\cite{Rand2013}. Immediately it is noted that if the
environment is noisy~\cite{Wu1995} then no strategy can be considered to be
extortionate as \(p_4>0\).

In the next section, this idea will be illustrated by observing the interactions
that take place in a computer based tournament of the IPD\@.

\section{Numerical experiments}\label{sec:numerical-experiments}

In~\cite{Stewart2012} results from a tournament with
\input{./assets/tex/number_of_stewart_plotkin_strategies/main.tex} strategies,
was presented with specific consideration given to ZD strategies. This
tournament is reproduced here using the Axelrod-Python
project~\cite{Knight2016}. To obtain a good measure of the corresponding
transition rates for each strategy all matches have been run for
\input{assets/tex/number_of_turns/main.tex} turns and every match has been
repeated \input{assets/tex/number_of_repetitions/main.tex} times. All of this
interaction data is available at~\cite{vincent_knight_2018_1297075}. A good
match between the inferred Markov chain and the state distribution of the actual
interactions has been verified. Data for this is presented in the supplementary
materials.

Figure~\ref{fig:SSError_overall_in_stewart_plotkin} shows the \(\text{SSError}\)
values for all the strategies in the tournament, as reported
in~\cite{Stewart2012} the extortionate strategy (which has an expected
\(\text{SSError}\) approximately 0) gains a large number of wins.

\begin{figure}[!htbp]
    \centering
    \includegraphics[width=.8\textwidth]{./assets/img/SSError_overall_in_stewart_plotkin/main.pdf}
    \caption{\(\text{SSError}\) and state probabilities for the strategies
        of~\cite{Stewart2012}, ordered both by number of wins and overall score.
        Note that \(P(DC)\) is not shown as it corresponds to the transpose of
        \(P(CD)\). Cooperator and Defector are omitted as they do not visit all
        the states.}
    \label{fig:SSError_overall_in_stewart_plotkin}
\end{figure}

Here, the work of~\cite{Stewart2012} is extended by investigating a tournament
with \input{assets/tex/number_of_full_strategies/main.tex}
strategies.

The results of this analysis are shown in
Figure~\ref{fig:SSError_and_probabilities_in_full}. The top ranking strategies
by number of wins seem to be extortionate (but not against all strategies) and
it can be seen that a small sub group of strategies achieve mutual defection.
All the top ranking strategies according to score achieve mutual cooperation and
do not extort each other, however they
\textbf{do} exhibit extortionate behaviour towards a number of the lower ranking
strategies.

\begin{figure}[!htbp]
    \centering
    \includegraphics[width=.8\textwidth]{./assets/img/SSError_and_probabilities_in_full/main.pdf}
    \caption{\(\text{SSError}\) for the strategies for the full tournament. Only
    strategy interactions for which \(p_4=0\) and \(\chi>1\) are displayed.}
    \label{fig:SSError_and_probabilities_in_full}
\end{figure}

\section{Conclusion}\label{sec:conclusion}

This work defines an approach to measure whether or not a player is playing a
strategy that corresponds to an extortionate strategy as defined
in~\cite{Press2012}: a mathematical model for suspicion. Indeed, all
extortionate strategies have been
 classified as lying on a triangular plane.
This rigorous classification fails to be robust to small measurement error, thus
a statistical approach is proposed.
This is done through a linear algebraic approach for approximating the solution
of a linear system. Using this, a large number of pairwise interactions is
simulated and in fact very few strategies are found to act extortionately.

The work of~\cite{Press2012}, whilst showing that a clever approach to taking
advantage of another memory one strategy exists: this is incomplete. Whilst the
elegance of this result is very attractive, just as the simplicity of the
victory of Tit For Tat in Axelrod's original tournaments was, it is incomplete.
Extortionate strategies achieve a high number of wins but they do not
achieve a high score which corresponds to the fitness landscape in an
evolutionary sense. From the large number of interactions a payoff matrix \(S\)
can be measured where \(S_{ij}\) denotes the score (using standard values of
\((R, S, T, P) = (3, 0, 5, 1)\)) of the \(i\)th strategy
against the \(j\)th strategy. Using this, the replicator equation
describes the evolution of the system based on a population density fitness
function:

\begin{equation}\label{eqn:replicator_dynamics}
    \frac{dx}{dt} = x(S-x^TS x)
\end{equation}

Equation (\ref{eqn:replicator_dynamics}) is solved numerically through an
integration technique described in~\cite{Petzold1983} and
Figure~\ref{fig:replicator_dynamics} shows the evolution of the distribution of
the system: the various strategies are ranked by scores. It is clear to see that
only the high ranking strategies survive the evolutionary process (in fact,
only \input{./assets/img/replicator_dynamics/main.tex}
have a final distribution greater than \(10 ^ {-2}\)). This confirms the
findings of~\cite{Moran1707} in which sophisticated strategies resist
evolutionary invasion of shorter memory strategies. Recalling
Figure~\ref{fig:SSError_and_probabilities_in_full} this demonstrates that:

\begin{itemize}
    \item Cooperation emerges through the evolutionary process: the high scoring
        strategies do not exhibit extortionate behaviour towards each other.
    \item Extortionate strategies do not survive the evolutionary process.
\end{itemize}

\begin{figure}[!htbp]
    \centering
    \includegraphics[width=.8\textwidth]{./assets/img/replicator_dynamics/main.pdf}
    \caption{Numerical simulation of the replicator equation
    (\ref{eqn:replicator_dynamics}): strategies are ordered by score, only the strategies with a high score survive the evolutionary process.}
    \label{fig:replicator_dynamics}
\end{figure}

This work can be used to classify plays of the IPD\@: data can be collected from
actual interactions (in lab or in the field). Furthermore, this allows for a
classification method similar to the notion of fingerprinting presented
in~\cite{Ashlock2008}. Trained strategies can potentially be classified as
extortionate or not or it could be possible to even constrain the reinforcement
learning approaches that are becoming prevalent in the literature.
Alternatively, this mathematical approach for recognising extortion could be
used in sophisticated strategies to defend against invasion. Arguably, some of
the strategies considered here exhibit this behaviour, indeed as described
in~\cite{Harper2017}, the top ranking strategies in the full tournament are
obtained using evolutionary reinforcement learning techniques, thus, suspicion
of extortionate behaviour could in fact be an evolutionary trait.

\section*{Acknowledgements}

The following open source software libraries were used in this research:

\begin{itemize}
    \item The Axelrod ~\cite{Knight2016, Knight2018} library (IPD strategies and
        tournaments).
    \item The sympy library~\cite{Meurer2017} (verification of all symbolic
        calculations).
    \item The matplotlib~\cite{Droettboom2018} library (visualisation).
    \item The pandas~\cite{Structures2010}, dask~\cite{Dask2016} and
        NumPy~\cite{Oliphant2015} libraries (data manipulation).
    \item The SciPy~\cite{Jones2001} library (numerical integration of the
        replicator equation).
\end{itemize}

This work was performed using the computational facilities of the Advanced
Research Computing @ Cardiff (ARCCA) Division, Cardiff University.

\printbibliography

\newpage
\section*{Supplementary materials}

\includepdf{assets/pdf/proof_of_form_of_extortionate_strategies/main.pdf}

\newpage

Using the pair wise interactions the transition rates \(p,
q\) can be measured and the steady state probabilities inferred and compared to
the actual probabilities of each state.
This is done numerically by computing the singular eigenvector of the
matrix \(A\) \cite{Stewart2009}:

\[
    A =
    \begin{bmatrix}
        p_1 q_1 & p_1 (1 - q_1) & (1 - p_1) q_1 & (1 -p_1) (1 - q_1) \\
        p_2 q_2 & p_2 (1 - q_2) & (1 - p_2) q_2 & (1 -p_2) (1 - q_2) \\
        p_3 q_3 & p_3 (1 - q_3) & (1 - p_3) q_3 & (1 -p_3) (1 - q_3) \\
        p_4 q_4 & p_4 (1 - q_4) & (1 - p_4) q_4 & (1 -p_4) (1 - q_4) \\
    \end{bmatrix}
\]

Figure~\ref{fig:computed_probabilities_vs_theoretic_probabilities} shows a
regression line fitted to every pairwise interaction with a reported
\(\text{SSError}\) value (pairwise interactions with missing states were
omitted). This serves to validate the approach: a part from some edge cases the
relationship is consistent.

\begin{figure}[!htbp]
    \centering
    \includegraphics[width=.8\textwidth]{./assets/img/computed_probabilities_vs_theoretic_probabilities/main.pdf}
    \caption{The
        relationship between the steady state probabilities inferred from the
        measured transitions and the actual steady state probabilities. A linear
        regression line is included validating the approach.}
    \label{fig:computed_probabilities_vs_theoretic_probabilities}
\end{figure}


\end{document}
 times. All of this
interaction data is available at~\cite{vincent_knight_2018_1297075}. A good
match between the inferred Markov chain and the state distribution of the actual
interactions has been verified. Data for this is presented in the supplementary
materials.

Figure~\ref{fig:SSError_overall_in_stewart_plotkin} shows the \(\text{SSError}\)
values for all the strategies in the tournament, as reported
in~\cite{Stewart2012} the extortionate strategy (which has an expected
\(\text{SSError}\) approximately 0) gains a large number of wins.

\begin{figure}[!htbp]
    \centering
    \includegraphics[width=.8\textwidth]{./assets/img/SSError_overall_in_stewart_plotkin/main.pdf}
    \caption{\(\text{SSError}\) and state probabilities for the strategies
        of~\cite{Stewart2012}, ordered both by number of wins and overall score.
        Note that \(P(DC)\) is not shown as it corresponds to the transpose of
        \(P(CD)\). Cooperator and Defector are omitted as they do not visit all
        the states.}
    \label{fig:SSError_overall_in_stewart_plotkin}
\end{figure}

Here, the work of~\cite{Stewart2012} is extended by investigating a tournament
with \documentclass[a4paper]{article}

\usepackage{amsmath}
\usepackage{amssymb}
\usepackage[margin=1.5cm,
            includefoot,
            footskip=30pt]{geometry}
\usepackage{layout}
\usepackage{graphicx}
\usepackage{subcaption}

\usepackage{biblatex}
\usepackage{pdfpages}

\bibliography{main.bib}

\title{Suspicion: Recognising and evaluating the effectiveness
       of extortion in the Iterated Prisoner's Dilemma}
\author{Vincent A. Knight \and Nikoleta E. Glynatsi}
\date{\today}



\begin{document}

\maketitle

\begin{abstract}
    The Iterated Prisoner's Dilemma is a model for rational and evolutionary
    interactive behaviour. It has applications both in the study of human social
    behaviour as well as in biology.
    It is used to understand when and how a rational individual might
    accept an immediate cost to their own utility for the direct benefit of
    another.

    Much attention has been given to a class of strategies called
    Zero Determinant strategies. It has been theoretically shown that these
    strategies can ``extort'' any player.

    In this work, an approach to identify if observed strategies are playing in
    an extortionate way is described. Furthermore, experimental analysis of
    a large tournament with \input{assets/tex/number_of_full_strategies/main.tex}
    strategies is considered. In this setting
    the most highly performing strategies do not play in an extortionate way
    against each other but do against lower performing strategies.
    This suggests that whilst the theory of Zero Determinant strategies
    indicates that memory is not of fundamental importance to the evolution of
    cooperative behaviour, this is incomplete.
\end{abstract}

\section{Introduction}\label{sec:introduction}

Agent based game theoretic models have become a stalwart of the underpinning
mathematics of interactive behaviours. One of the major pieces of work
in this area is the pair of original computer tournaments run by Robert
Axelrod~\cite{Axelrod1980, Axelrod1980a}. These tournaments pitted submitted
computer strategies against each other in plays of the Iterated Prisoner's
Dilemma. A common game where agents can choose to pay a slight cost to their
immediate utility in the hope of building a reputation. This has been used in
economic and evolutionary game theory to understand the evolution of cooperative
behaviour.

Recently, a class of strategies was described in~\cite{Press2012} that can
provably extort any given opponent. In~\cite{Hilbe2013, Moran1707} some
questions have already been asked about the true effectiveness of these
strategies in an evolutionary setting. Here another question is asked: is it
possible to recognise this extortionate behaviour? A mathematical procedure for
suspicion is presented: in the same way that the continued actions of an
extortionate individual might raise suspicion.

This work makes use of the Axelrod Python library~\cite{Knight2018, Knight2016}
with a large number of Prisoner Dilemma strategies available to give an
extensive numerical example of the ideas presented.  The approach is presented
in Section~\ref{sec:delta-zd-strategies}.  All of the code and data discussed
in Section~\ref{sec:numerical-experiments} is open sourced, archived and
written according to best scientific principles~\cite{Wilson2014}. The data
archive can be found at~\cite{vincent_knight_2018_1297075}.

\section{Recognising Extortion}\label{sec:delta-zd-strategies}

In~\cite{Press2012}, given a match between 2 memory-one strategies, the concept
of Zero Determinant (ZD) strategies is introduced. The main result of that paper
shows that given two memory one players \(p, q\in\mathbb{R}^4\) a linear
relationship between the players' scores could be forced by one of the players.

Using the notation of~\cite{Press2012}, assuming the utilities for player \(p\)
are given by \(S_x=(R, S, T, P)\) and for player \(q\) by \(S_y=(R, T, S, P)\)
and that the stationary scores of each player is given by \(S_X\) and \(S_Y\)
respectively. The main result of~\cite{Press2012} is that if

\begin{equation}\label{eqn:linear_relationship_for_p}
    \tilde p=\alpha S_x + \beta S_y + \gamma
\end{equation}

or

\begin{equation}\label{eqn:linear_relationship_for_q}
    \tilde q=\alpha S_x + \beta S_y + \gamma
\end{equation}

where \(\tilde p = (1 - p_1, 1 - p_2, p_3, p_4)\) and
\(\tilde q = (1 - q_1, 1 - q_2, q_3, q_4)\) then:

\begin{equation}
    \alpha S_X + \beta S_Y + \gamma = 0
\end{equation}

In~\cite{Press2012} a particular type of ZD strategy is defined: extortionate
strategies. If:

\begin{equation}\label{eqn:constraint_for_extortion}
    \gamma = - P(\alpha + \beta)
\end{equation}

then the player can ensure they get a score \(\chi\) times
larger than the opponent. This extortion coefficient is given by:

\begin{equation}\label{eqn:definition_of_chi}
    \chi=\frac{-\beta}{\alpha}
\end{equation}

Thus, if (\ref{eqn:constraint_for_extortion}) holds and \(\chi >1\) a player is
said to extort their opponent.
Here, the reverse problem is considered: given a
\(p\in\mathbb{R}^4\) how does one identify \(\alpha, \beta\) if they
exist and is the strategy in fact acting in an extortionate way?

These conditions correspond to:

\begin{align}
    \tilde p_1 & = \alpha R + \beta R - P (\alpha + \beta)
            \label{eqn:condition_for_tilde_p1}\\
    \tilde p_2 & = \alpha S + \beta T - P (\alpha + \beta)
            \label{eqn:condition_for_tilde_p2}\\
    \tilde p_3 & = \alpha T + \beta S - P (\alpha + \beta)
            \label{eqn:condition_for_tilde_p3}\\
    \tilde p_4 & = \alpha P + \beta P - P (\alpha + \beta)
            \label{eqn:condition_for_tilde_p4}
\end{align}

Equation (\ref{eqn:condition_for_tilde_p4}) ensures that \(p_4=\tilde p_4=0\).
Equations (\ref{eqn:condition_for_tilde_p1}-\ref{eqn:condition_for_tilde_p3})
can be used to eliminate \(\alpha, \beta\), giving:

\begin{equation}\label{eqn:planar_definition_of_extortion}
    \tilde p_1 = \frac{(R - P)(\tilde p_2 + \tilde p_3)}{S + T - 2P}
\end{equation}

with:

\begin{equation}\label{eqn:definition_of_chi}
    \chi = \frac{\tilde p_2 (P - T) + \tilde p_3 (S - P)}
                {\tilde p_2 (P - S) + \tilde p_3 (T - P)}
\end{equation}

Given a strategy \(p\in\mathbb{R}^{4\times 1}\) equations
(\ref{eqn:condition_for_tilde_p4}), (\ref{eqn:planar_definition_of_extortion}-\ref{eqn:definition_of_chi}) can be used to check if
a strategy is extortionate. The conditions correspond to:

\begin{align}
    p_1 & = \frac{(R-P)(p_2 + p_3) - R + T + S - P}{S + T - 2P}
     \label{eqn:condition_for_p1}\\
    p_4 & = 0 \label{eqn:condition_for_p4}\\
    1 & > p_2 + p_3\label{eqn:condition_for_chi}
\end{align}

The algebraic steps necessary to prove these results are available in the
supporting materials.

All extortionate strategies reside on a triangular (\ref{eqn:condition_for_chi})
plane (\ref{eqn:condition_for_p1}) in 3 dimensions (\ref{eqn:condition_for_p4}).
Using this formulation it can be seen that a necessary (but not sufficient)
condition for an extortionate strategy is that it cooperates on average less
than 50\% of the time when in a state of disagreement with the opponent.

As an example, consider the known extortionate strategy \(p=(8 / 9, 1 / 2, 1 /
3, 0)\) from~\cite{Stewart2012} which is referred to as \texttt{Extort-2}. In
this case, for the standard values of \((R, T, S, P)\) constraint
(\ref{eqn:condition_for_p1}) corresponds to:

\begin{equation}
    p_1 = \frac{2(p_2 + p_3) + 1}{3}
\end{equation}

It is clear that in this case all constraints hold.

This approach could in fact be used to confirm that a given strategy is acting
in an extortionate manner even if it is not a memory one strategy. However, in
practice, if a closed form for \(p\) is not known, then due to measurement
and/or numerical error this would not work.

This problem can be written in the following linear algebraic form where
\(x=(\alpha, \beta)\)
and \(p^*=(\tilde p_1 - 1, tilde_2 - 1, p_3)\):

\begin{equation}\label{eqn:linear_algebraic_equation_for_p}
    Cx= p^*
\end{equation}

\(C\) corresponds to equations
(\ref{eqn:condition_for_tilde_p1}-\ref{eqn:condition_for_tilde_p3}) and is
given by:

\begin{equation}\label{eqn:definition_of_C}
    C =
    \begin{bmatrix}
        R - P & R- P \\
        S - P & T- P \\
        T - P & S- P \\
    \end{bmatrix}
\end{equation}

Note that in general, equation (\ref{eqn:linear_algebraic_equation_for_p}) will
not necessarily have a solution. From the Rouch\'{e}-Capelli theorem if there is
a solution it is unique as \(\text{rank}(C)=2\) which is the dimension of the
variable \(x\). The best fitting \(x\) is found by minimizing:

\begin{equation}\label{eqn:r_squared}
    \text{SSError} = \|C x- p^*\|_2^2 = \sum_{i=1}^{3}\left((C\bar x)_i-p_i^*\right)^2
\end{equation}

Note that \(\text{SSError}\), which is the square of the Frobenius
norm~\cite{Golub2013}, becomes a measure of how close a strategy is to being an
extortionate strategy. Suspicion
of extortion then corresponds to a threshold on \(\text{SSError}\).

By observing interactions (human or otherwise), their memory one representation
can be inferred and this approach can be used to recognise extortionate
behaviour. The notion of comparing theoretic and actual plays of the IPD is not
novel, see for example~\cite{Rand2013}. Immediately it is noted that if the
environment is noisy~\cite{Wu1995} then no strategy can be considered to be
extortionate as \(p_4>0\).

In the next section, this idea will be illustrated by observing the interactions
that take place in a computer based tournament of the IPD\@.

\section{Numerical experiments}\label{sec:numerical-experiments}

In~\cite{Stewart2012} results from a tournament with
\input{./assets/tex/number_of_stewart_plotkin_strategies/main.tex} strategies,
was presented with specific consideration given to ZD strategies. This
tournament is reproduced here using the Axelrod-Python
project~\cite{Knight2016}. To obtain a good measure of the corresponding
transition rates for each strategy all matches have been run for
\input{assets/tex/number_of_turns/main.tex} turns and every match has been
repeated \input{assets/tex/number_of_repetitions/main.tex} times. All of this
interaction data is available at~\cite{vincent_knight_2018_1297075}. A good
match between the inferred Markov chain and the state distribution of the actual
interactions has been verified. Data for this is presented in the supplementary
materials.

Figure~\ref{fig:SSError_overall_in_stewart_plotkin} shows the \(\text{SSError}\)
values for all the strategies in the tournament, as reported
in~\cite{Stewart2012} the extortionate strategy (which has an expected
\(\text{SSError}\) approximately 0) gains a large number of wins.

\begin{figure}[!htbp]
    \centering
    \includegraphics[width=.8\textwidth]{./assets/img/SSError_overall_in_stewart_plotkin/main.pdf}
    \caption{\(\text{SSError}\) and state probabilities for the strategies
        of~\cite{Stewart2012}, ordered both by number of wins and overall score.
        Note that \(P(DC)\) is not shown as it corresponds to the transpose of
        \(P(CD)\). Cooperator and Defector are omitted as they do not visit all
        the states.}
    \label{fig:SSError_overall_in_stewart_plotkin}
\end{figure}

Here, the work of~\cite{Stewart2012} is extended by investigating a tournament
with \input{assets/tex/number_of_full_strategies/main.tex}
strategies.

The results of this analysis are shown in
Figure~\ref{fig:SSError_and_probabilities_in_full}. The top ranking strategies
by number of wins seem to be extortionate (but not against all strategies) and
it can be seen that a small sub group of strategies achieve mutual defection.
All the top ranking strategies according to score achieve mutual cooperation and
do not extort each other, however they
\textbf{do} exhibit extortionate behaviour towards a number of the lower ranking
strategies.

\begin{figure}[!htbp]
    \centering
    \includegraphics[width=.8\textwidth]{./assets/img/SSError_and_probabilities_in_full/main.pdf}
    \caption{\(\text{SSError}\) for the strategies for the full tournament. Only
    strategy interactions for which \(p_4=0\) and \(\chi>1\) are displayed.}
    \label{fig:SSError_and_probabilities_in_full}
\end{figure}

\section{Conclusion}\label{sec:conclusion}

This work defines an approach to measure whether or not a player is playing a
strategy that corresponds to an extortionate strategy as defined
in~\cite{Press2012}: a mathematical model for suspicion. Indeed, all
extortionate strategies have been
 classified as lying on a triangular plane.
This rigorous classification fails to be robust to small measurement error, thus
a statistical approach is proposed.
This is done through a linear algebraic approach for approximating the solution
of a linear system. Using this, a large number of pairwise interactions is
simulated and in fact very few strategies are found to act extortionately.

The work of~\cite{Press2012}, whilst showing that a clever approach to taking
advantage of another memory one strategy exists: this is incomplete. Whilst the
elegance of this result is very attractive, just as the simplicity of the
victory of Tit For Tat in Axelrod's original tournaments was, it is incomplete.
Extortionate strategies achieve a high number of wins but they do not
achieve a high score which corresponds to the fitness landscape in an
evolutionary sense. From the large number of interactions a payoff matrix \(S\)
can be measured where \(S_{ij}\) denotes the score (using standard values of
\((R, S, T, P) = (3, 0, 5, 1)\)) of the \(i\)th strategy
against the \(j\)th strategy. Using this, the replicator equation
describes the evolution of the system based on a population density fitness
function:

\begin{equation}\label{eqn:replicator_dynamics}
    \frac{dx}{dt} = x(S-x^TS x)
\end{equation}

Equation (\ref{eqn:replicator_dynamics}) is solved numerically through an
integration technique described in~\cite{Petzold1983} and
Figure~\ref{fig:replicator_dynamics} shows the evolution of the distribution of
the system: the various strategies are ranked by scores. It is clear to see that
only the high ranking strategies survive the evolutionary process (in fact,
only \input{./assets/img/replicator_dynamics/main.tex}
have a final distribution greater than \(10 ^ {-2}\)). This confirms the
findings of~\cite{Moran1707} in which sophisticated strategies resist
evolutionary invasion of shorter memory strategies. Recalling
Figure~\ref{fig:SSError_and_probabilities_in_full} this demonstrates that:

\begin{itemize}
    \item Cooperation emerges through the evolutionary process: the high scoring
        strategies do not exhibit extortionate behaviour towards each other.
    \item Extortionate strategies do not survive the evolutionary process.
\end{itemize}

\begin{figure}[!htbp]
    \centering
    \includegraphics[width=.8\textwidth]{./assets/img/replicator_dynamics/main.pdf}
    \caption{Numerical simulation of the replicator equation
    (\ref{eqn:replicator_dynamics}): strategies are ordered by score, only the strategies with a high score survive the evolutionary process.}
    \label{fig:replicator_dynamics}
\end{figure}

This work can be used to classify plays of the IPD\@: data can be collected from
actual interactions (in lab or in the field). Furthermore, this allows for a
classification method similar to the notion of fingerprinting presented
in~\cite{Ashlock2008}. Trained strategies can potentially be classified as
extortionate or not or it could be possible to even constrain the reinforcement
learning approaches that are becoming prevalent in the literature.
Alternatively, this mathematical approach for recognising extortion could be
used in sophisticated strategies to defend against invasion. Arguably, some of
the strategies considered here exhibit this behaviour, indeed as described
in~\cite{Harper2017}, the top ranking strategies in the full tournament are
obtained using evolutionary reinforcement learning techniques, thus, suspicion
of extortionate behaviour could in fact be an evolutionary trait.

\section*{Acknowledgements}

The following open source software libraries were used in this research:

\begin{itemize}
    \item The Axelrod ~\cite{Knight2016, Knight2018} library (IPD strategies and
        tournaments).
    \item The sympy library~\cite{Meurer2017} (verification of all symbolic
        calculations).
    \item The matplotlib~\cite{Droettboom2018} library (visualisation).
    \item The pandas~\cite{Structures2010}, dask~\cite{Dask2016} and
        NumPy~\cite{Oliphant2015} libraries (data manipulation).
    \item The SciPy~\cite{Jones2001} library (numerical integration of the
        replicator equation).
\end{itemize}

This work was performed using the computational facilities of the Advanced
Research Computing @ Cardiff (ARCCA) Division, Cardiff University.

\printbibliography

\newpage
\section*{Supplementary materials}

\includepdf{assets/pdf/proof_of_form_of_extortionate_strategies/main.pdf}

\newpage

Using the pair wise interactions the transition rates \(p,
q\) can be measured and the steady state probabilities inferred and compared to
the actual probabilities of each state.
This is done numerically by computing the singular eigenvector of the
matrix \(A\) \cite{Stewart2009}:

\[
    A =
    \begin{bmatrix}
        p_1 q_1 & p_1 (1 - q_1) & (1 - p_1) q_1 & (1 -p_1) (1 - q_1) \\
        p_2 q_2 & p_2 (1 - q_2) & (1 - p_2) q_2 & (1 -p_2) (1 - q_2) \\
        p_3 q_3 & p_3 (1 - q_3) & (1 - p_3) q_3 & (1 -p_3) (1 - q_3) \\
        p_4 q_4 & p_4 (1 - q_4) & (1 - p_4) q_4 & (1 -p_4) (1 - q_4) \\
    \end{bmatrix}
\]

Figure~\ref{fig:computed_probabilities_vs_theoretic_probabilities} shows a
regression line fitted to every pairwise interaction with a reported
\(\text{SSError}\) value (pairwise interactions with missing states were
omitted). This serves to validate the approach: a part from some edge cases the
relationship is consistent.

\begin{figure}[!htbp]
    \centering
    \includegraphics[width=.8\textwidth]{./assets/img/computed_probabilities_vs_theoretic_probabilities/main.pdf}
    \caption{The
        relationship between the steady state probabilities inferred from the
        measured transitions and the actual steady state probabilities. A linear
        regression line is included validating the approach.}
    \label{fig:computed_probabilities_vs_theoretic_probabilities}
\end{figure}


\end{document}

strategies.

The results of this analysis are shown in
Figure~\ref{fig:SSError_and_probabilities_in_full}. The top ranking strategies
by number of wins seem to be extortionate (but not against all strategies) and
it can be seen that a small sub group of strategies achieve mutual defection.
All the top ranking strategies according to score achieve mutual cooperation and
do not extort each other, however they
\textbf{do} exhibit extortionate behaviour towards a number of the lower ranking
strategies.

\begin{figure}[!htbp]
    \centering
    \includegraphics[width=.8\textwidth]{./assets/img/SSError_and_probabilities_in_full/main.pdf}
    \caption{\(\text{SSError}\) for the strategies for the full tournament. Only
    strategy interactions for which \(p_4=0\) and \(\chi>1\) are displayed.}
    \label{fig:SSError_and_probabilities_in_full}
\end{figure}

\section{Conclusion}\label{sec:conclusion}

This work defines an approach to measure whether or not a player is playing a
strategy that corresponds to an extortionate strategy as defined
in~\cite{Press2012}: a mathematical model for suspicion. Indeed, all
extortionate strategies have been
 classified as lying on a triangular plane.
This rigorous classification fails to be robust to small measurement error, thus
a statistical approach is proposed.
This is done through a linear algebraic approach for approximating the solution
of a linear system. Using this, a large number of pairwise interactions is
simulated and in fact very few strategies are found to act extortionately.

The work of~\cite{Press2012}, whilst showing that a clever approach to taking
advantage of another memory one strategy exists: this is incomplete. Whilst the
elegance of this result is very attractive, just as the simplicity of the
victory of Tit For Tat in Axelrod's original tournaments was, it is incomplete.
Extortionate strategies achieve a high number of wins but they do not
achieve a high score which corresponds to the fitness landscape in an
evolutionary sense. From the large number of interactions a payoff matrix \(S\)
can be measured where \(S_{ij}\) denotes the score (using standard values of
\((R, S, T, P) = (3, 0, 5, 1)\)) of the \(i\)th strategy
against the \(j\)th strategy. Using this, the replicator equation
describes the evolution of the system based on a population density fitness
function:

\begin{equation}\label{eqn:replicator_dynamics}
    \frac{dx}{dt} = x(S-x^TS x)
\end{equation}

Equation (\ref{eqn:replicator_dynamics}) is solved numerically through an
integration technique described in~\cite{Petzold1983} and
Figure~\ref{fig:replicator_dynamics} shows the evolution of the distribution of
the system: the various strategies are ranked by scores. It is clear to see that
only the high ranking strategies survive the evolutionary process (in fact,
only \documentclass[a4paper]{article}

\usepackage{amsmath}
\usepackage{amssymb}
\usepackage[margin=1.5cm,
            includefoot,
            footskip=30pt]{geometry}
\usepackage{layout}
\usepackage{graphicx}
\usepackage{subcaption}

\usepackage{biblatex}
\usepackage{pdfpages}

\bibliography{main.bib}

\title{Suspicion: Recognising and evaluating the effectiveness
       of extortion in the Iterated Prisoner's Dilemma}
\author{Vincent A. Knight \and Nikoleta E. Glynatsi}
\date{\today}



\begin{document}

\maketitle

\begin{abstract}
    The Iterated Prisoner's Dilemma is a model for rational and evolutionary
    interactive behaviour. It has applications both in the study of human social
    behaviour as well as in biology.
    It is used to understand when and how a rational individual might
    accept an immediate cost to their own utility for the direct benefit of
    another.

    Much attention has been given to a class of strategies called
    Zero Determinant strategies. It has been theoretically shown that these
    strategies can ``extort'' any player.

    In this work, an approach to identify if observed strategies are playing in
    an extortionate way is described. Furthermore, experimental analysis of
    a large tournament with \input{assets/tex/number_of_full_strategies/main.tex}
    strategies is considered. In this setting
    the most highly performing strategies do not play in an extortionate way
    against each other but do against lower performing strategies.
    This suggests that whilst the theory of Zero Determinant strategies
    indicates that memory is not of fundamental importance to the evolution of
    cooperative behaviour, this is incomplete.
\end{abstract}

\section{Introduction}\label{sec:introduction}

Agent based game theoretic models have become a stalwart of the underpinning
mathematics of interactive behaviours. One of the major pieces of work
in this area is the pair of original computer tournaments run by Robert
Axelrod~\cite{Axelrod1980, Axelrod1980a}. These tournaments pitted submitted
computer strategies against each other in plays of the Iterated Prisoner's
Dilemma. A common game where agents can choose to pay a slight cost to their
immediate utility in the hope of building a reputation. This has been used in
economic and evolutionary game theory to understand the evolution of cooperative
behaviour.

Recently, a class of strategies was described in~\cite{Press2012} that can
provably extort any given opponent. In~\cite{Hilbe2013, Moran1707} some
questions have already been asked about the true effectiveness of these
strategies in an evolutionary setting. Here another question is asked: is it
possible to recognise this extortionate behaviour? A mathematical procedure for
suspicion is presented: in the same way that the continued actions of an
extortionate individual might raise suspicion.

This work makes use of the Axelrod Python library~\cite{Knight2018, Knight2016}
with a large number of Prisoner Dilemma strategies available to give an
extensive numerical example of the ideas presented.  The approach is presented
in Section~\ref{sec:delta-zd-strategies}.  All of the code and data discussed
in Section~\ref{sec:numerical-experiments} is open sourced, archived and
written according to best scientific principles~\cite{Wilson2014}. The data
archive can be found at~\cite{vincent_knight_2018_1297075}.

\section{Recognising Extortion}\label{sec:delta-zd-strategies}

In~\cite{Press2012}, given a match between 2 memory-one strategies, the concept
of Zero Determinant (ZD) strategies is introduced. The main result of that paper
shows that given two memory one players \(p, q\in\mathbb{R}^4\) a linear
relationship between the players' scores could be forced by one of the players.

Using the notation of~\cite{Press2012}, assuming the utilities for player \(p\)
are given by \(S_x=(R, S, T, P)\) and for player \(q\) by \(S_y=(R, T, S, P)\)
and that the stationary scores of each player is given by \(S_X\) and \(S_Y\)
respectively. The main result of~\cite{Press2012} is that if

\begin{equation}\label{eqn:linear_relationship_for_p}
    \tilde p=\alpha S_x + \beta S_y + \gamma
\end{equation}

or

\begin{equation}\label{eqn:linear_relationship_for_q}
    \tilde q=\alpha S_x + \beta S_y + \gamma
\end{equation}

where \(\tilde p = (1 - p_1, 1 - p_2, p_3, p_4)\) and
\(\tilde q = (1 - q_1, 1 - q_2, q_3, q_4)\) then:

\begin{equation}
    \alpha S_X + \beta S_Y + \gamma = 0
\end{equation}

In~\cite{Press2012} a particular type of ZD strategy is defined: extortionate
strategies. If:

\begin{equation}\label{eqn:constraint_for_extortion}
    \gamma = - P(\alpha + \beta)
\end{equation}

then the player can ensure they get a score \(\chi\) times
larger than the opponent. This extortion coefficient is given by:

\begin{equation}\label{eqn:definition_of_chi}
    \chi=\frac{-\beta}{\alpha}
\end{equation}

Thus, if (\ref{eqn:constraint_for_extortion}) holds and \(\chi >1\) a player is
said to extort their opponent.
Here, the reverse problem is considered: given a
\(p\in\mathbb{R}^4\) how does one identify \(\alpha, \beta\) if they
exist and is the strategy in fact acting in an extortionate way?

These conditions correspond to:

\begin{align}
    \tilde p_1 & = \alpha R + \beta R - P (\alpha + \beta)
            \label{eqn:condition_for_tilde_p1}\\
    \tilde p_2 & = \alpha S + \beta T - P (\alpha + \beta)
            \label{eqn:condition_for_tilde_p2}\\
    \tilde p_3 & = \alpha T + \beta S - P (\alpha + \beta)
            \label{eqn:condition_for_tilde_p3}\\
    \tilde p_4 & = \alpha P + \beta P - P (\alpha + \beta)
            \label{eqn:condition_for_tilde_p4}
\end{align}

Equation (\ref{eqn:condition_for_tilde_p4}) ensures that \(p_4=\tilde p_4=0\).
Equations (\ref{eqn:condition_for_tilde_p1}-\ref{eqn:condition_for_tilde_p3})
can be used to eliminate \(\alpha, \beta\), giving:

\begin{equation}\label{eqn:planar_definition_of_extortion}
    \tilde p_1 = \frac{(R - P)(\tilde p_2 + \tilde p_3)}{S + T - 2P}
\end{equation}

with:

\begin{equation}\label{eqn:definition_of_chi}
    \chi = \frac{\tilde p_2 (P - T) + \tilde p_3 (S - P)}
                {\tilde p_2 (P - S) + \tilde p_3 (T - P)}
\end{equation}

Given a strategy \(p\in\mathbb{R}^{4\times 1}\) equations
(\ref{eqn:condition_for_tilde_p4}), (\ref{eqn:planar_definition_of_extortion}-\ref{eqn:definition_of_chi}) can be used to check if
a strategy is extortionate. The conditions correspond to:

\begin{align}
    p_1 & = \frac{(R-P)(p_2 + p_3) - R + T + S - P}{S + T - 2P}
     \label{eqn:condition_for_p1}\\
    p_4 & = 0 \label{eqn:condition_for_p4}\\
    1 & > p_2 + p_3\label{eqn:condition_for_chi}
\end{align}

The algebraic steps necessary to prove these results are available in the
supporting materials.

All extortionate strategies reside on a triangular (\ref{eqn:condition_for_chi})
plane (\ref{eqn:condition_for_p1}) in 3 dimensions (\ref{eqn:condition_for_p4}).
Using this formulation it can be seen that a necessary (but not sufficient)
condition for an extortionate strategy is that it cooperates on average less
than 50\% of the time when in a state of disagreement with the opponent.

As an example, consider the known extortionate strategy \(p=(8 / 9, 1 / 2, 1 /
3, 0)\) from~\cite{Stewart2012} which is referred to as \texttt{Extort-2}. In
this case, for the standard values of \((R, T, S, P)\) constraint
(\ref{eqn:condition_for_p1}) corresponds to:

\begin{equation}
    p_1 = \frac{2(p_2 + p_3) + 1}{3}
\end{equation}

It is clear that in this case all constraints hold.

This approach could in fact be used to confirm that a given strategy is acting
in an extortionate manner even if it is not a memory one strategy. However, in
practice, if a closed form for \(p\) is not known, then due to measurement
and/or numerical error this would not work.

This problem can be written in the following linear algebraic form where
\(x=(\alpha, \beta)\)
and \(p^*=(\tilde p_1 - 1, tilde_2 - 1, p_3)\):

\begin{equation}\label{eqn:linear_algebraic_equation_for_p}
    Cx= p^*
\end{equation}

\(C\) corresponds to equations
(\ref{eqn:condition_for_tilde_p1}-\ref{eqn:condition_for_tilde_p3}) and is
given by:

\begin{equation}\label{eqn:definition_of_C}
    C =
    \begin{bmatrix}
        R - P & R- P \\
        S - P & T- P \\
        T - P & S- P \\
    \end{bmatrix}
\end{equation}

Note that in general, equation (\ref{eqn:linear_algebraic_equation_for_p}) will
not necessarily have a solution. From the Rouch\'{e}-Capelli theorem if there is
a solution it is unique as \(\text{rank}(C)=2\) which is the dimension of the
variable \(x\). The best fitting \(x\) is found by minimizing:

\begin{equation}\label{eqn:r_squared}
    \text{SSError} = \|C x- p^*\|_2^2 = \sum_{i=1}^{3}\left((C\bar x)_i-p_i^*\right)^2
\end{equation}

Note that \(\text{SSError}\), which is the square of the Frobenius
norm~\cite{Golub2013}, becomes a measure of how close a strategy is to being an
extortionate strategy. Suspicion
of extortion then corresponds to a threshold on \(\text{SSError}\).

By observing interactions (human or otherwise), their memory one representation
can be inferred and this approach can be used to recognise extortionate
behaviour. The notion of comparing theoretic and actual plays of the IPD is not
novel, see for example~\cite{Rand2013}. Immediately it is noted that if the
environment is noisy~\cite{Wu1995} then no strategy can be considered to be
extortionate as \(p_4>0\).

In the next section, this idea will be illustrated by observing the interactions
that take place in a computer based tournament of the IPD\@.

\section{Numerical experiments}\label{sec:numerical-experiments}

In~\cite{Stewart2012} results from a tournament with
\input{./assets/tex/number_of_stewart_plotkin_strategies/main.tex} strategies,
was presented with specific consideration given to ZD strategies. This
tournament is reproduced here using the Axelrod-Python
project~\cite{Knight2016}. To obtain a good measure of the corresponding
transition rates for each strategy all matches have been run for
\input{assets/tex/number_of_turns/main.tex} turns and every match has been
repeated \input{assets/tex/number_of_repetitions/main.tex} times. All of this
interaction data is available at~\cite{vincent_knight_2018_1297075}. A good
match between the inferred Markov chain and the state distribution of the actual
interactions has been verified. Data for this is presented in the supplementary
materials.

Figure~\ref{fig:SSError_overall_in_stewart_plotkin} shows the \(\text{SSError}\)
values for all the strategies in the tournament, as reported
in~\cite{Stewart2012} the extortionate strategy (which has an expected
\(\text{SSError}\) approximately 0) gains a large number of wins.

\begin{figure}[!htbp]
    \centering
    \includegraphics[width=.8\textwidth]{./assets/img/SSError_overall_in_stewart_plotkin/main.pdf}
    \caption{\(\text{SSError}\) and state probabilities for the strategies
        of~\cite{Stewart2012}, ordered both by number of wins and overall score.
        Note that \(P(DC)\) is not shown as it corresponds to the transpose of
        \(P(CD)\). Cooperator and Defector are omitted as they do not visit all
        the states.}
    \label{fig:SSError_overall_in_stewart_plotkin}
\end{figure}

Here, the work of~\cite{Stewart2012} is extended by investigating a tournament
with \input{assets/tex/number_of_full_strategies/main.tex}
strategies.

The results of this analysis are shown in
Figure~\ref{fig:SSError_and_probabilities_in_full}. The top ranking strategies
by number of wins seem to be extortionate (but not against all strategies) and
it can be seen that a small sub group of strategies achieve mutual defection.
All the top ranking strategies according to score achieve mutual cooperation and
do not extort each other, however they
\textbf{do} exhibit extortionate behaviour towards a number of the lower ranking
strategies.

\begin{figure}[!htbp]
    \centering
    \includegraphics[width=.8\textwidth]{./assets/img/SSError_and_probabilities_in_full/main.pdf}
    \caption{\(\text{SSError}\) for the strategies for the full tournament. Only
    strategy interactions for which \(p_4=0\) and \(\chi>1\) are displayed.}
    \label{fig:SSError_and_probabilities_in_full}
\end{figure}

\section{Conclusion}\label{sec:conclusion}

This work defines an approach to measure whether or not a player is playing a
strategy that corresponds to an extortionate strategy as defined
in~\cite{Press2012}: a mathematical model for suspicion. Indeed, all
extortionate strategies have been
 classified as lying on a triangular plane.
This rigorous classification fails to be robust to small measurement error, thus
a statistical approach is proposed.
This is done through a linear algebraic approach for approximating the solution
of a linear system. Using this, a large number of pairwise interactions is
simulated and in fact very few strategies are found to act extortionately.

The work of~\cite{Press2012}, whilst showing that a clever approach to taking
advantage of another memory one strategy exists: this is incomplete. Whilst the
elegance of this result is very attractive, just as the simplicity of the
victory of Tit For Tat in Axelrod's original tournaments was, it is incomplete.
Extortionate strategies achieve a high number of wins but they do not
achieve a high score which corresponds to the fitness landscape in an
evolutionary sense. From the large number of interactions a payoff matrix \(S\)
can be measured where \(S_{ij}\) denotes the score (using standard values of
\((R, S, T, P) = (3, 0, 5, 1)\)) of the \(i\)th strategy
against the \(j\)th strategy. Using this, the replicator equation
describes the evolution of the system based on a population density fitness
function:

\begin{equation}\label{eqn:replicator_dynamics}
    \frac{dx}{dt} = x(S-x^TS x)
\end{equation}

Equation (\ref{eqn:replicator_dynamics}) is solved numerically through an
integration technique described in~\cite{Petzold1983} and
Figure~\ref{fig:replicator_dynamics} shows the evolution of the distribution of
the system: the various strategies are ranked by scores. It is clear to see that
only the high ranking strategies survive the evolutionary process (in fact,
only \input{./assets/img/replicator_dynamics/main.tex}
have a final distribution greater than \(10 ^ {-2}\)). This confirms the
findings of~\cite{Moran1707} in which sophisticated strategies resist
evolutionary invasion of shorter memory strategies. Recalling
Figure~\ref{fig:SSError_and_probabilities_in_full} this demonstrates that:

\begin{itemize}
    \item Cooperation emerges through the evolutionary process: the high scoring
        strategies do not exhibit extortionate behaviour towards each other.
    \item Extortionate strategies do not survive the evolutionary process.
\end{itemize}

\begin{figure}[!htbp]
    \centering
    \includegraphics[width=.8\textwidth]{./assets/img/replicator_dynamics/main.pdf}
    \caption{Numerical simulation of the replicator equation
    (\ref{eqn:replicator_dynamics}): strategies are ordered by score, only the strategies with a high score survive the evolutionary process.}
    \label{fig:replicator_dynamics}
\end{figure}

This work can be used to classify plays of the IPD\@: data can be collected from
actual interactions (in lab or in the field). Furthermore, this allows for a
classification method similar to the notion of fingerprinting presented
in~\cite{Ashlock2008}. Trained strategies can potentially be classified as
extortionate or not or it could be possible to even constrain the reinforcement
learning approaches that are becoming prevalent in the literature.
Alternatively, this mathematical approach for recognising extortion could be
used in sophisticated strategies to defend against invasion. Arguably, some of
the strategies considered here exhibit this behaviour, indeed as described
in~\cite{Harper2017}, the top ranking strategies in the full tournament are
obtained using evolutionary reinforcement learning techniques, thus, suspicion
of extortionate behaviour could in fact be an evolutionary trait.

\section*{Acknowledgements}

The following open source software libraries were used in this research:

\begin{itemize}
    \item The Axelrod ~\cite{Knight2016, Knight2018} library (IPD strategies and
        tournaments).
    \item The sympy library~\cite{Meurer2017} (verification of all symbolic
        calculations).
    \item The matplotlib~\cite{Droettboom2018} library (visualisation).
    \item The pandas~\cite{Structures2010}, dask~\cite{Dask2016} and
        NumPy~\cite{Oliphant2015} libraries (data manipulation).
    \item The SciPy~\cite{Jones2001} library (numerical integration of the
        replicator equation).
\end{itemize}

This work was performed using the computational facilities of the Advanced
Research Computing @ Cardiff (ARCCA) Division, Cardiff University.

\printbibliography

\newpage
\section*{Supplementary materials}

\includepdf{assets/pdf/proof_of_form_of_extortionate_strategies/main.pdf}

\newpage

Using the pair wise interactions the transition rates \(p,
q\) can be measured and the steady state probabilities inferred and compared to
the actual probabilities of each state.
This is done numerically by computing the singular eigenvector of the
matrix \(A\) \cite{Stewart2009}:

\[
    A =
    \begin{bmatrix}
        p_1 q_1 & p_1 (1 - q_1) & (1 - p_1) q_1 & (1 -p_1) (1 - q_1) \\
        p_2 q_2 & p_2 (1 - q_2) & (1 - p_2) q_2 & (1 -p_2) (1 - q_2) \\
        p_3 q_3 & p_3 (1 - q_3) & (1 - p_3) q_3 & (1 -p_3) (1 - q_3) \\
        p_4 q_4 & p_4 (1 - q_4) & (1 - p_4) q_4 & (1 -p_4) (1 - q_4) \\
    \end{bmatrix}
\]

Figure~\ref{fig:computed_probabilities_vs_theoretic_probabilities} shows a
regression line fitted to every pairwise interaction with a reported
\(\text{SSError}\) value (pairwise interactions with missing states were
omitted). This serves to validate the approach: a part from some edge cases the
relationship is consistent.

\begin{figure}[!htbp]
    \centering
    \includegraphics[width=.8\textwidth]{./assets/img/computed_probabilities_vs_theoretic_probabilities/main.pdf}
    \caption{The
        relationship between the steady state probabilities inferred from the
        measured transitions and the actual steady state probabilities. A linear
        regression line is included validating the approach.}
    \label{fig:computed_probabilities_vs_theoretic_probabilities}
\end{figure}


\end{document}

have a final distribution greater than \(10 ^ {-2}\)). This confirms the
findings of~\cite{Moran1707} in which sophisticated strategies resist
evolutionary invasion of shorter memory strategies. Recalling
Figure~\ref{fig:SSError_and_probabilities_in_full} this demonstrates that:

\begin{itemize}
    \item Cooperation emerges through the evolutionary process: the high scoring
        strategies do not exhibit extortionate behaviour towards each other.
    \item Extortionate strategies do not survive the evolutionary process.
\end{itemize}

\begin{figure}[!htbp]
    \centering
    \includegraphics[width=.8\textwidth]{./assets/img/replicator_dynamics/main.pdf}
    \caption{Numerical simulation of the replicator equation
    (\ref{eqn:replicator_dynamics}): strategies are ordered by score, only the strategies with a high score survive the evolutionary process.}
    \label{fig:replicator_dynamics}
\end{figure}

This work can be used to classify plays of the IPD\@: data can be collected from
actual interactions (in lab or in the field). Furthermore, this allows for a
classification method similar to the notion of fingerprinting presented
in~\cite{Ashlock2008}. Trained strategies can potentially be classified as
extortionate or not or it could be possible to even constrain the reinforcement
learning approaches that are becoming prevalent in the literature.
Alternatively, this mathematical approach for recognising extortion could be
used in sophisticated strategies to defend against invasion. Arguably, some of
the strategies considered here exhibit this behaviour, indeed as described
in~\cite{Harper2017}, the top ranking strategies in the full tournament are
obtained using evolutionary reinforcement learning techniques, thus, suspicion
of extortionate behaviour could in fact be an evolutionary trait.

\section*{Acknowledgements}

The following open source software libraries were used in this research:

\begin{itemize}
    \item The Axelrod ~\cite{Knight2016, Knight2018} library (IPD strategies and
        tournaments).
    \item The sympy library~\cite{Meurer2017} (verification of all symbolic
        calculations).
    \item The matplotlib~\cite{Droettboom2018} library (visualisation).
    \item The pandas~\cite{Structures2010}, dask~\cite{Dask2016} and
        NumPy~\cite{Oliphant2015} libraries (data manipulation).
    \item The SciPy~\cite{Jones2001} library (numerical integration of the
        replicator equation).
\end{itemize}

This work was performed using the computational facilities of the Advanced
Research Computing @ Cardiff (ARCCA) Division, Cardiff University.

\printbibliography

\newpage
\section*{Supplementary materials}

\includepdf{assets/pdf/proof_of_form_of_extortionate_strategies/main.pdf}

\newpage

Using the pair wise interactions the transition rates \(p,
q\) can be measured and the steady state probabilities inferred and compared to
the actual probabilities of each state.
This is done numerically by computing the singular eigenvector of the
matrix \(A\) \cite{Stewart2009}:

\[
    A =
    \begin{bmatrix}
        p_1 q_1 & p_1 (1 - q_1) & (1 - p_1) q_1 & (1 -p_1) (1 - q_1) \\
        p_2 q_2 & p_2 (1 - q_2) & (1 - p_2) q_2 & (1 -p_2) (1 - q_2) \\
        p_3 q_3 & p_3 (1 - q_3) & (1 - p_3) q_3 & (1 -p_3) (1 - q_3) \\
        p_4 q_4 & p_4 (1 - q_4) & (1 - p_4) q_4 & (1 -p_4) (1 - q_4) \\
    \end{bmatrix}
\]

Figure~\ref{fig:computed_probabilities_vs_theoretic_probabilities} shows a
regression line fitted to every pairwise interaction with a reported
\(\text{SSError}\) value (pairwise interactions with missing states were
omitted). This serves to validate the approach: a part from some edge cases the
relationship is consistent.

\begin{figure}[!htbp]
    \centering
    \includegraphics[width=.8\textwidth]{./assets/img/computed_probabilities_vs_theoretic_probabilities/main.pdf}
    \caption{The
        relationship between the steady state probabilities inferred from the
        measured transitions and the actual steady state probabilities. A linear
        regression line is included validating the approach.}
    \label{fig:computed_probabilities_vs_theoretic_probabilities}
\end{figure}


\end{document}
 turns and every match has been
repeated \documentclass[a4paper]{article}

\usepackage{amsmath}
\usepackage{amssymb}
\usepackage[margin=1.5cm,
            includefoot,
            footskip=30pt]{geometry}
\usepackage{layout}
\usepackage{graphicx}
\usepackage{subcaption}

\usepackage{biblatex}
\usepackage{pdfpages}

\bibliography{main.bib}

\title{Suspicion: Recognising and evaluating the effectiveness
       of extortion in the Iterated Prisoner's Dilemma}
\author{Vincent A. Knight \and Nikoleta E. Glynatsi}
\date{\today}



\begin{document}

\maketitle

\begin{abstract}
    The Iterated Prisoner's Dilemma is a model for rational and evolutionary
    interactive behaviour. It has applications both in the study of human social
    behaviour as well as in biology.
    It is used to understand when and how a rational individual might
    accept an immediate cost to their own utility for the direct benefit of
    another.

    Much attention has been given to a class of strategies called
    Zero Determinant strategies. It has been theoretically shown that these
    strategies can ``extort'' any player.

    In this work, an approach to identify if observed strategies are playing in
    an extortionate way is described. Furthermore, experimental analysis of
    a large tournament with \documentclass[a4paper]{article}

\usepackage{amsmath}
\usepackage{amssymb}
\usepackage[margin=1.5cm,
            includefoot,
            footskip=30pt]{geometry}
\usepackage{layout}
\usepackage{graphicx}
\usepackage{subcaption}

\usepackage{biblatex}
\usepackage{pdfpages}

\bibliography{main.bib}

\title{Suspicion: Recognising and evaluating the effectiveness
       of extortion in the Iterated Prisoner's Dilemma}
\author{Vincent A. Knight \and Nikoleta E. Glynatsi}
\date{\today}



\begin{document}

\maketitle

\begin{abstract}
    The Iterated Prisoner's Dilemma is a model for rational and evolutionary
    interactive behaviour. It has applications both in the study of human social
    behaviour as well as in biology.
    It is used to understand when and how a rational individual might
    accept an immediate cost to their own utility for the direct benefit of
    another.

    Much attention has been given to a class of strategies called
    Zero Determinant strategies. It has been theoretically shown that these
    strategies can ``extort'' any player.

    In this work, an approach to identify if observed strategies are playing in
    an extortionate way is described. Furthermore, experimental analysis of
    a large tournament with \input{assets/tex/number_of_full_strategies/main.tex}
    strategies is considered. In this setting
    the most highly performing strategies do not play in an extortionate way
    against each other but do against lower performing strategies.
    This suggests that whilst the theory of Zero Determinant strategies
    indicates that memory is not of fundamental importance to the evolution of
    cooperative behaviour, this is incomplete.
\end{abstract}

\section{Introduction}\label{sec:introduction}

Agent based game theoretic models have become a stalwart of the underpinning
mathematics of interactive behaviours. One of the major pieces of work
in this area is the pair of original computer tournaments run by Robert
Axelrod~\cite{Axelrod1980, Axelrod1980a}. These tournaments pitted submitted
computer strategies against each other in plays of the Iterated Prisoner's
Dilemma. A common game where agents can choose to pay a slight cost to their
immediate utility in the hope of building a reputation. This has been used in
economic and evolutionary game theory to understand the evolution of cooperative
behaviour.

Recently, a class of strategies was described in~\cite{Press2012} that can
provably extort any given opponent. In~\cite{Hilbe2013, Moran1707} some
questions have already been asked about the true effectiveness of these
strategies in an evolutionary setting. Here another question is asked: is it
possible to recognise this extortionate behaviour? A mathematical procedure for
suspicion is presented: in the same way that the continued actions of an
extortionate individual might raise suspicion.

This work makes use of the Axelrod Python library~\cite{Knight2018, Knight2016}
with a large number of Prisoner Dilemma strategies available to give an
extensive numerical example of the ideas presented.  The approach is presented
in Section~\ref{sec:delta-zd-strategies}.  All of the code and data discussed
in Section~\ref{sec:numerical-experiments} is open sourced, archived and
written according to best scientific principles~\cite{Wilson2014}. The data
archive can be found at~\cite{vincent_knight_2018_1297075}.

\section{Recognising Extortion}\label{sec:delta-zd-strategies}

In~\cite{Press2012}, given a match between 2 memory-one strategies, the concept
of Zero Determinant (ZD) strategies is introduced. The main result of that paper
shows that given two memory one players \(p, q\in\mathbb{R}^4\) a linear
relationship between the players' scores could be forced by one of the players.

Using the notation of~\cite{Press2012}, assuming the utilities for player \(p\)
are given by \(S_x=(R, S, T, P)\) and for player \(q\) by \(S_y=(R, T, S, P)\)
and that the stationary scores of each player is given by \(S_X\) and \(S_Y\)
respectively. The main result of~\cite{Press2012} is that if

\begin{equation}\label{eqn:linear_relationship_for_p}
    \tilde p=\alpha S_x + \beta S_y + \gamma
\end{equation}

or

\begin{equation}\label{eqn:linear_relationship_for_q}
    \tilde q=\alpha S_x + \beta S_y + \gamma
\end{equation}

where \(\tilde p = (1 - p_1, 1 - p_2, p_3, p_4)\) and
\(\tilde q = (1 - q_1, 1 - q_2, q_3, q_4)\) then:

\begin{equation}
    \alpha S_X + \beta S_Y + \gamma = 0
\end{equation}

In~\cite{Press2012} a particular type of ZD strategy is defined: extortionate
strategies. If:

\begin{equation}\label{eqn:constraint_for_extortion}
    \gamma = - P(\alpha + \beta)
\end{equation}

then the player can ensure they get a score \(\chi\) times
larger than the opponent. This extortion coefficient is given by:

\begin{equation}\label{eqn:definition_of_chi}
    \chi=\frac{-\beta}{\alpha}
\end{equation}

Thus, if (\ref{eqn:constraint_for_extortion}) holds and \(\chi >1\) a player is
said to extort their opponent.
Here, the reverse problem is considered: given a
\(p\in\mathbb{R}^4\) how does one identify \(\alpha, \beta\) if they
exist and is the strategy in fact acting in an extortionate way?

These conditions correspond to:

\begin{align}
    \tilde p_1 & = \alpha R + \beta R - P (\alpha + \beta)
            \label{eqn:condition_for_tilde_p1}\\
    \tilde p_2 & = \alpha S + \beta T - P (\alpha + \beta)
            \label{eqn:condition_for_tilde_p2}\\
    \tilde p_3 & = \alpha T + \beta S - P (\alpha + \beta)
            \label{eqn:condition_for_tilde_p3}\\
    \tilde p_4 & = \alpha P + \beta P - P (\alpha + \beta)
            \label{eqn:condition_for_tilde_p4}
\end{align}

Equation (\ref{eqn:condition_for_tilde_p4}) ensures that \(p_4=\tilde p_4=0\).
Equations (\ref{eqn:condition_for_tilde_p1}-\ref{eqn:condition_for_tilde_p3})
can be used to eliminate \(\alpha, \beta\), giving:

\begin{equation}\label{eqn:planar_definition_of_extortion}
    \tilde p_1 = \frac{(R - P)(\tilde p_2 + \tilde p_3)}{S + T - 2P}
\end{equation}

with:

\begin{equation}\label{eqn:definition_of_chi}
    \chi = \frac{\tilde p_2 (P - T) + \tilde p_3 (S - P)}
                {\tilde p_2 (P - S) + \tilde p_3 (T - P)}
\end{equation}

Given a strategy \(p\in\mathbb{R}^{4\times 1}\) equations
(\ref{eqn:condition_for_tilde_p4}), (\ref{eqn:planar_definition_of_extortion}-\ref{eqn:definition_of_chi}) can be used to check if
a strategy is extortionate. The conditions correspond to:

\begin{align}
    p_1 & = \frac{(R-P)(p_2 + p_3) - R + T + S - P}{S + T - 2P}
     \label{eqn:condition_for_p1}\\
    p_4 & = 0 \label{eqn:condition_for_p4}\\
    1 & > p_2 + p_3\label{eqn:condition_for_chi}
\end{align}

The algebraic steps necessary to prove these results are available in the
supporting materials.

All extortionate strategies reside on a triangular (\ref{eqn:condition_for_chi})
plane (\ref{eqn:condition_for_p1}) in 3 dimensions (\ref{eqn:condition_for_p4}).
Using this formulation it can be seen that a necessary (but not sufficient)
condition for an extortionate strategy is that it cooperates on average less
than 50\% of the time when in a state of disagreement with the opponent.

As an example, consider the known extortionate strategy \(p=(8 / 9, 1 / 2, 1 /
3, 0)\) from~\cite{Stewart2012} which is referred to as \texttt{Extort-2}. In
this case, for the standard values of \((R, T, S, P)\) constraint
(\ref{eqn:condition_for_p1}) corresponds to:

\begin{equation}
    p_1 = \frac{2(p_2 + p_3) + 1}{3}
\end{equation}

It is clear that in this case all constraints hold.

This approach could in fact be used to confirm that a given strategy is acting
in an extortionate manner even if it is not a memory one strategy. However, in
practice, if a closed form for \(p\) is not known, then due to measurement
and/or numerical error this would not work.

This problem can be written in the following linear algebraic form where
\(x=(\alpha, \beta)\)
and \(p^*=(\tilde p_1 - 1, tilde_2 - 1, p_3)\):

\begin{equation}\label{eqn:linear_algebraic_equation_for_p}
    Cx= p^*
\end{equation}

\(C\) corresponds to equations
(\ref{eqn:condition_for_tilde_p1}-\ref{eqn:condition_for_tilde_p3}) and is
given by:

\begin{equation}\label{eqn:definition_of_C}
    C =
    \begin{bmatrix}
        R - P & R- P \\
        S - P & T- P \\
        T - P & S- P \\
    \end{bmatrix}
\end{equation}

Note that in general, equation (\ref{eqn:linear_algebraic_equation_for_p}) will
not necessarily have a solution. From the Rouch\'{e}-Capelli theorem if there is
a solution it is unique as \(\text{rank}(C)=2\) which is the dimension of the
variable \(x\). The best fitting \(x\) is found by minimizing:

\begin{equation}\label{eqn:r_squared}
    \text{SSError} = \|C x- p^*\|_2^2 = \sum_{i=1}^{3}\left((C\bar x)_i-p_i^*\right)^2
\end{equation}

Note that \(\text{SSError}\), which is the square of the Frobenius
norm~\cite{Golub2013}, becomes a measure of how close a strategy is to being an
extortionate strategy. Suspicion
of extortion then corresponds to a threshold on \(\text{SSError}\).

By observing interactions (human or otherwise), their memory one representation
can be inferred and this approach can be used to recognise extortionate
behaviour. The notion of comparing theoretic and actual plays of the IPD is not
novel, see for example~\cite{Rand2013}. Immediately it is noted that if the
environment is noisy~\cite{Wu1995} then no strategy can be considered to be
extortionate as \(p_4>0\).

In the next section, this idea will be illustrated by observing the interactions
that take place in a computer based tournament of the IPD\@.

\section{Numerical experiments}\label{sec:numerical-experiments}

In~\cite{Stewart2012} results from a tournament with
\input{./assets/tex/number_of_stewart_plotkin_strategies/main.tex} strategies,
was presented with specific consideration given to ZD strategies. This
tournament is reproduced here using the Axelrod-Python
project~\cite{Knight2016}. To obtain a good measure of the corresponding
transition rates for each strategy all matches have been run for
\input{assets/tex/number_of_turns/main.tex} turns and every match has been
repeated \input{assets/tex/number_of_repetitions/main.tex} times. All of this
interaction data is available at~\cite{vincent_knight_2018_1297075}. A good
match between the inferred Markov chain and the state distribution of the actual
interactions has been verified. Data for this is presented in the supplementary
materials.

Figure~\ref{fig:SSError_overall_in_stewart_plotkin} shows the \(\text{SSError}\)
values for all the strategies in the tournament, as reported
in~\cite{Stewart2012} the extortionate strategy (which has an expected
\(\text{SSError}\) approximately 0) gains a large number of wins.

\begin{figure}[!htbp]
    \centering
    \includegraphics[width=.8\textwidth]{./assets/img/SSError_overall_in_stewart_plotkin/main.pdf}
    \caption{\(\text{SSError}\) and state probabilities for the strategies
        of~\cite{Stewart2012}, ordered both by number of wins and overall score.
        Note that \(P(DC)\) is not shown as it corresponds to the transpose of
        \(P(CD)\). Cooperator and Defector are omitted as they do not visit all
        the states.}
    \label{fig:SSError_overall_in_stewart_plotkin}
\end{figure}

Here, the work of~\cite{Stewart2012} is extended by investigating a tournament
with \input{assets/tex/number_of_full_strategies/main.tex}
strategies.

The results of this analysis are shown in
Figure~\ref{fig:SSError_and_probabilities_in_full}. The top ranking strategies
by number of wins seem to be extortionate (but not against all strategies) and
it can be seen that a small sub group of strategies achieve mutual defection.
All the top ranking strategies according to score achieve mutual cooperation and
do not extort each other, however they
\textbf{do} exhibit extortionate behaviour towards a number of the lower ranking
strategies.

\begin{figure}[!htbp]
    \centering
    \includegraphics[width=.8\textwidth]{./assets/img/SSError_and_probabilities_in_full/main.pdf}
    \caption{\(\text{SSError}\) for the strategies for the full tournament. Only
    strategy interactions for which \(p_4=0\) and \(\chi>1\) are displayed.}
    \label{fig:SSError_and_probabilities_in_full}
\end{figure}

\section{Conclusion}\label{sec:conclusion}

This work defines an approach to measure whether or not a player is playing a
strategy that corresponds to an extortionate strategy as defined
in~\cite{Press2012}: a mathematical model for suspicion. Indeed, all
extortionate strategies have been
 classified as lying on a triangular plane.
This rigorous classification fails to be robust to small measurement error, thus
a statistical approach is proposed.
This is done through a linear algebraic approach for approximating the solution
of a linear system. Using this, a large number of pairwise interactions is
simulated and in fact very few strategies are found to act extortionately.

The work of~\cite{Press2012}, whilst showing that a clever approach to taking
advantage of another memory one strategy exists: this is incomplete. Whilst the
elegance of this result is very attractive, just as the simplicity of the
victory of Tit For Tat in Axelrod's original tournaments was, it is incomplete.
Extortionate strategies achieve a high number of wins but they do not
achieve a high score which corresponds to the fitness landscape in an
evolutionary sense. From the large number of interactions a payoff matrix \(S\)
can be measured where \(S_{ij}\) denotes the score (using standard values of
\((R, S, T, P) = (3, 0, 5, 1)\)) of the \(i\)th strategy
against the \(j\)th strategy. Using this, the replicator equation
describes the evolution of the system based on a population density fitness
function:

\begin{equation}\label{eqn:replicator_dynamics}
    \frac{dx}{dt} = x(S-x^TS x)
\end{equation}

Equation (\ref{eqn:replicator_dynamics}) is solved numerically through an
integration technique described in~\cite{Petzold1983} and
Figure~\ref{fig:replicator_dynamics} shows the evolution of the distribution of
the system: the various strategies are ranked by scores. It is clear to see that
only the high ranking strategies survive the evolutionary process (in fact,
only \input{./assets/img/replicator_dynamics/main.tex}
have a final distribution greater than \(10 ^ {-2}\)). This confirms the
findings of~\cite{Moran1707} in which sophisticated strategies resist
evolutionary invasion of shorter memory strategies. Recalling
Figure~\ref{fig:SSError_and_probabilities_in_full} this demonstrates that:

\begin{itemize}
    \item Cooperation emerges through the evolutionary process: the high scoring
        strategies do not exhibit extortionate behaviour towards each other.
    \item Extortionate strategies do not survive the evolutionary process.
\end{itemize}

\begin{figure}[!htbp]
    \centering
    \includegraphics[width=.8\textwidth]{./assets/img/replicator_dynamics/main.pdf}
    \caption{Numerical simulation of the replicator equation
    (\ref{eqn:replicator_dynamics}): strategies are ordered by score, only the strategies with a high score survive the evolutionary process.}
    \label{fig:replicator_dynamics}
\end{figure}

This work can be used to classify plays of the IPD\@: data can be collected from
actual interactions (in lab or in the field). Furthermore, this allows for a
classification method similar to the notion of fingerprinting presented
in~\cite{Ashlock2008}. Trained strategies can potentially be classified as
extortionate or not or it could be possible to even constrain the reinforcement
learning approaches that are becoming prevalent in the literature.
Alternatively, this mathematical approach for recognising extortion could be
used in sophisticated strategies to defend against invasion. Arguably, some of
the strategies considered here exhibit this behaviour, indeed as described
in~\cite{Harper2017}, the top ranking strategies in the full tournament are
obtained using evolutionary reinforcement learning techniques, thus, suspicion
of extortionate behaviour could in fact be an evolutionary trait.

\section*{Acknowledgements}

The following open source software libraries were used in this research:

\begin{itemize}
    \item The Axelrod ~\cite{Knight2016, Knight2018} library (IPD strategies and
        tournaments).
    \item The sympy library~\cite{Meurer2017} (verification of all symbolic
        calculations).
    \item The matplotlib~\cite{Droettboom2018} library (visualisation).
    \item The pandas~\cite{Structures2010}, dask~\cite{Dask2016} and
        NumPy~\cite{Oliphant2015} libraries (data manipulation).
    \item The SciPy~\cite{Jones2001} library (numerical integration of the
        replicator equation).
\end{itemize}

This work was performed using the computational facilities of the Advanced
Research Computing @ Cardiff (ARCCA) Division, Cardiff University.

\printbibliography

\newpage
\section*{Supplementary materials}

\includepdf{assets/pdf/proof_of_form_of_extortionate_strategies/main.pdf}

\newpage

Using the pair wise interactions the transition rates \(p,
q\) can be measured and the steady state probabilities inferred and compared to
the actual probabilities of each state.
This is done numerically by computing the singular eigenvector of the
matrix \(A\) \cite{Stewart2009}:

\[
    A =
    \begin{bmatrix}
        p_1 q_1 & p_1 (1 - q_1) & (1 - p_1) q_1 & (1 -p_1) (1 - q_1) \\
        p_2 q_2 & p_2 (1 - q_2) & (1 - p_2) q_2 & (1 -p_2) (1 - q_2) \\
        p_3 q_3 & p_3 (1 - q_3) & (1 - p_3) q_3 & (1 -p_3) (1 - q_3) \\
        p_4 q_4 & p_4 (1 - q_4) & (1 - p_4) q_4 & (1 -p_4) (1 - q_4) \\
    \end{bmatrix}
\]

Figure~\ref{fig:computed_probabilities_vs_theoretic_probabilities} shows a
regression line fitted to every pairwise interaction with a reported
\(\text{SSError}\) value (pairwise interactions with missing states were
omitted). This serves to validate the approach: a part from some edge cases the
relationship is consistent.

\begin{figure}[!htbp]
    \centering
    \includegraphics[width=.8\textwidth]{./assets/img/computed_probabilities_vs_theoretic_probabilities/main.pdf}
    \caption{The
        relationship between the steady state probabilities inferred from the
        measured transitions and the actual steady state probabilities. A linear
        regression line is included validating the approach.}
    \label{fig:computed_probabilities_vs_theoretic_probabilities}
\end{figure}


\end{document}

    strategies is considered. In this setting
    the most highly performing strategies do not play in an extortionate way
    against each other but do against lower performing strategies.
    This suggests that whilst the theory of Zero Determinant strategies
    indicates that memory is not of fundamental importance to the evolution of
    cooperative behaviour, this is incomplete.
\end{abstract}

\section{Introduction}\label{sec:introduction}

Agent based game theoretic models have become a stalwart of the underpinning
mathematics of interactive behaviours. One of the major pieces of work
in this area is the pair of original computer tournaments run by Robert
Axelrod~\cite{Axelrod1980, Axelrod1980a}. These tournaments pitted submitted
computer strategies against each other in plays of the Iterated Prisoner's
Dilemma. A common game where agents can choose to pay a slight cost to their
immediate utility in the hope of building a reputation. This has been used in
economic and evolutionary game theory to understand the evolution of cooperative
behaviour.

Recently, a class of strategies was described in~\cite{Press2012} that can
provably extort any given opponent. In~\cite{Hilbe2013, Moran1707} some
questions have already been asked about the true effectiveness of these
strategies in an evolutionary setting. Here another question is asked: is it
possible to recognise this extortionate behaviour? A mathematical procedure for
suspicion is presented: in the same way that the continued actions of an
extortionate individual might raise suspicion.

This work makes use of the Axelrod Python library~\cite{Knight2018, Knight2016}
with a large number of Prisoner Dilemma strategies available to give an
extensive numerical example of the ideas presented.  The approach is presented
in Section~\ref{sec:delta-zd-strategies}.  All of the code and data discussed
in Section~\ref{sec:numerical-experiments} is open sourced, archived and
written according to best scientific principles~\cite{Wilson2014}. The data
archive can be found at~\cite{vincent_knight_2018_1297075}.

\section{Recognising Extortion}\label{sec:delta-zd-strategies}

In~\cite{Press2012}, given a match between 2 memory-one strategies, the concept
of Zero Determinant (ZD) strategies is introduced. The main result of that paper
shows that given two memory one players \(p, q\in\mathbb{R}^4\) a linear
relationship between the players' scores could be forced by one of the players.

Using the notation of~\cite{Press2012}, assuming the utilities for player \(p\)
are given by \(S_x=(R, S, T, P)\) and for player \(q\) by \(S_y=(R, T, S, P)\)
and that the stationary scores of each player is given by \(S_X\) and \(S_Y\)
respectively. The main result of~\cite{Press2012} is that if

\begin{equation}\label{eqn:linear_relationship_for_p}
    \tilde p=\alpha S_x + \beta S_y + \gamma
\end{equation}

or

\begin{equation}\label{eqn:linear_relationship_for_q}
    \tilde q=\alpha S_x + \beta S_y + \gamma
\end{equation}

where \(\tilde p = (1 - p_1, 1 - p_2, p_3, p_4)\) and
\(\tilde q = (1 - q_1, 1 - q_2, q_3, q_4)\) then:

\begin{equation}
    \alpha S_X + \beta S_Y + \gamma = 0
\end{equation}

In~\cite{Press2012} a particular type of ZD strategy is defined: extortionate
strategies. If:

\begin{equation}\label{eqn:constraint_for_extortion}
    \gamma = - P(\alpha + \beta)
\end{equation}

then the player can ensure they get a score \(\chi\) times
larger than the opponent. This extortion coefficient is given by:

\begin{equation}\label{eqn:definition_of_chi}
    \chi=\frac{-\beta}{\alpha}
\end{equation}

Thus, if (\ref{eqn:constraint_for_extortion}) holds and \(\chi >1\) a player is
said to extort their opponent.
Here, the reverse problem is considered: given a
\(p\in\mathbb{R}^4\) how does one identify \(\alpha, \beta\) if they
exist and is the strategy in fact acting in an extortionate way?

These conditions correspond to:

\begin{align}
    \tilde p_1 & = \alpha R + \beta R - P (\alpha + \beta)
            \label{eqn:condition_for_tilde_p1}\\
    \tilde p_2 & = \alpha S + \beta T - P (\alpha + \beta)
            \label{eqn:condition_for_tilde_p2}\\
    \tilde p_3 & = \alpha T + \beta S - P (\alpha + \beta)
            \label{eqn:condition_for_tilde_p3}\\
    \tilde p_4 & = \alpha P + \beta P - P (\alpha + \beta)
            \label{eqn:condition_for_tilde_p4}
\end{align}

Equation (\ref{eqn:condition_for_tilde_p4}) ensures that \(p_4=\tilde p_4=0\).
Equations (\ref{eqn:condition_for_tilde_p1}-\ref{eqn:condition_for_tilde_p3})
can be used to eliminate \(\alpha, \beta\), giving:

\begin{equation}\label{eqn:planar_definition_of_extortion}
    \tilde p_1 = \frac{(R - P)(\tilde p_2 + \tilde p_3)}{S + T - 2P}
\end{equation}

with:

\begin{equation}\label{eqn:definition_of_chi}
    \chi = \frac{\tilde p_2 (P - T) + \tilde p_3 (S - P)}
                {\tilde p_2 (P - S) + \tilde p_3 (T - P)}
\end{equation}

Given a strategy \(p\in\mathbb{R}^{4\times 1}\) equations
(\ref{eqn:condition_for_tilde_p4}), (\ref{eqn:planar_definition_of_extortion}-\ref{eqn:definition_of_chi}) can be used to check if
a strategy is extortionate. The conditions correspond to:

\begin{align}
    p_1 & = \frac{(R-P)(p_2 + p_3) - R + T + S - P}{S + T - 2P}
     \label{eqn:condition_for_p1}\\
    p_4 & = 0 \label{eqn:condition_for_p4}\\
    1 & > p_2 + p_3\label{eqn:condition_for_chi}
\end{align}

The algebraic steps necessary to prove these results are available in the
supporting materials.

All extortionate strategies reside on a triangular (\ref{eqn:condition_for_chi})
plane (\ref{eqn:condition_for_p1}) in 3 dimensions (\ref{eqn:condition_for_p4}).
Using this formulation it can be seen that a necessary (but not sufficient)
condition for an extortionate strategy is that it cooperates on average less
than 50\% of the time when in a state of disagreement with the opponent.

As an example, consider the known extortionate strategy \(p=(8 / 9, 1 / 2, 1 /
3, 0)\) from~\cite{Stewart2012} which is referred to as \texttt{Extort-2}. In
this case, for the standard values of \((R, T, S, P)\) constraint
(\ref{eqn:condition_for_p1}) corresponds to:

\begin{equation}
    p_1 = \frac{2(p_2 + p_3) + 1}{3}
\end{equation}

It is clear that in this case all constraints hold.

This approach could in fact be used to confirm that a given strategy is acting
in an extortionate manner even if it is not a memory one strategy. However, in
practice, if a closed form for \(p\) is not known, then due to measurement
and/or numerical error this would not work.

This problem can be written in the following linear algebraic form where
\(x=(\alpha, \beta)\)
and \(p^*=(\tilde p_1 - 1, tilde_2 - 1, p_3)\):

\begin{equation}\label{eqn:linear_algebraic_equation_for_p}
    Cx= p^*
\end{equation}

\(C\) corresponds to equations
(\ref{eqn:condition_for_tilde_p1}-\ref{eqn:condition_for_tilde_p3}) and is
given by:

\begin{equation}\label{eqn:definition_of_C}
    C =
    \begin{bmatrix}
        R - P & R- P \\
        S - P & T- P \\
        T - P & S- P \\
    \end{bmatrix}
\end{equation}

Note that in general, equation (\ref{eqn:linear_algebraic_equation_for_p}) will
not necessarily have a solution. From the Rouch\'{e}-Capelli theorem if there is
a solution it is unique as \(\text{rank}(C)=2\) which is the dimension of the
variable \(x\). The best fitting \(x\) is found by minimizing:

\begin{equation}\label{eqn:r_squared}
    \text{SSError} = \|C x- p^*\|_2^2 = \sum_{i=1}^{3}\left((C\bar x)_i-p_i^*\right)^2
\end{equation}

Note that \(\text{SSError}\), which is the square of the Frobenius
norm~\cite{Golub2013}, becomes a measure of how close a strategy is to being an
extortionate strategy. Suspicion
of extortion then corresponds to a threshold on \(\text{SSError}\).

By observing interactions (human or otherwise), their memory one representation
can be inferred and this approach can be used to recognise extortionate
behaviour. The notion of comparing theoretic and actual plays of the IPD is not
novel, see for example~\cite{Rand2013}. Immediately it is noted that if the
environment is noisy~\cite{Wu1995} then no strategy can be considered to be
extortionate as \(p_4>0\).

In the next section, this idea will be illustrated by observing the interactions
that take place in a computer based tournament of the IPD\@.

\section{Numerical experiments}\label{sec:numerical-experiments}

In~\cite{Stewart2012} results from a tournament with
\documentclass[a4paper]{article}

\usepackage{amsmath}
\usepackage{amssymb}
\usepackage[margin=1.5cm,
            includefoot,
            footskip=30pt]{geometry}
\usepackage{layout}
\usepackage{graphicx}
\usepackage{subcaption}

\usepackage{biblatex}
\usepackage{pdfpages}

\bibliography{main.bib}

\title{Suspicion: Recognising and evaluating the effectiveness
       of extortion in the Iterated Prisoner's Dilemma}
\author{Vincent A. Knight \and Nikoleta E. Glynatsi}
\date{\today}



\begin{document}

\maketitle

\begin{abstract}
    The Iterated Prisoner's Dilemma is a model for rational and evolutionary
    interactive behaviour. It has applications both in the study of human social
    behaviour as well as in biology.
    It is used to understand when and how a rational individual might
    accept an immediate cost to their own utility for the direct benefit of
    another.

    Much attention has been given to a class of strategies called
    Zero Determinant strategies. It has been theoretically shown that these
    strategies can ``extort'' any player.

    In this work, an approach to identify if observed strategies are playing in
    an extortionate way is described. Furthermore, experimental analysis of
    a large tournament with \input{assets/tex/number_of_full_strategies/main.tex}
    strategies is considered. In this setting
    the most highly performing strategies do not play in an extortionate way
    against each other but do against lower performing strategies.
    This suggests that whilst the theory of Zero Determinant strategies
    indicates that memory is not of fundamental importance to the evolution of
    cooperative behaviour, this is incomplete.
\end{abstract}

\section{Introduction}\label{sec:introduction}

Agent based game theoretic models have become a stalwart of the underpinning
mathematics of interactive behaviours. One of the major pieces of work
in this area is the pair of original computer tournaments run by Robert
Axelrod~\cite{Axelrod1980, Axelrod1980a}. These tournaments pitted submitted
computer strategies against each other in plays of the Iterated Prisoner's
Dilemma. A common game where agents can choose to pay a slight cost to their
immediate utility in the hope of building a reputation. This has been used in
economic and evolutionary game theory to understand the evolution of cooperative
behaviour.

Recently, a class of strategies was described in~\cite{Press2012} that can
provably extort any given opponent. In~\cite{Hilbe2013, Moran1707} some
questions have already been asked about the true effectiveness of these
strategies in an evolutionary setting. Here another question is asked: is it
possible to recognise this extortionate behaviour? A mathematical procedure for
suspicion is presented: in the same way that the continued actions of an
extortionate individual might raise suspicion.

This work makes use of the Axelrod Python library~\cite{Knight2018, Knight2016}
with a large number of Prisoner Dilemma strategies available to give an
extensive numerical example of the ideas presented.  The approach is presented
in Section~\ref{sec:delta-zd-strategies}.  All of the code and data discussed
in Section~\ref{sec:numerical-experiments} is open sourced, archived and
written according to best scientific principles~\cite{Wilson2014}. The data
archive can be found at~\cite{vincent_knight_2018_1297075}.

\section{Recognising Extortion}\label{sec:delta-zd-strategies}

In~\cite{Press2012}, given a match between 2 memory-one strategies, the concept
of Zero Determinant (ZD) strategies is introduced. The main result of that paper
shows that given two memory one players \(p, q\in\mathbb{R}^4\) a linear
relationship between the players' scores could be forced by one of the players.

Using the notation of~\cite{Press2012}, assuming the utilities for player \(p\)
are given by \(S_x=(R, S, T, P)\) and for player \(q\) by \(S_y=(R, T, S, P)\)
and that the stationary scores of each player is given by \(S_X\) and \(S_Y\)
respectively. The main result of~\cite{Press2012} is that if

\begin{equation}\label{eqn:linear_relationship_for_p}
    \tilde p=\alpha S_x + \beta S_y + \gamma
\end{equation}

or

\begin{equation}\label{eqn:linear_relationship_for_q}
    \tilde q=\alpha S_x + \beta S_y + \gamma
\end{equation}

where \(\tilde p = (1 - p_1, 1 - p_2, p_3, p_4)\) and
\(\tilde q = (1 - q_1, 1 - q_2, q_3, q_4)\) then:

\begin{equation}
    \alpha S_X + \beta S_Y + \gamma = 0
\end{equation}

In~\cite{Press2012} a particular type of ZD strategy is defined: extortionate
strategies. If:

\begin{equation}\label{eqn:constraint_for_extortion}
    \gamma = - P(\alpha + \beta)
\end{equation}

then the player can ensure they get a score \(\chi\) times
larger than the opponent. This extortion coefficient is given by:

\begin{equation}\label{eqn:definition_of_chi}
    \chi=\frac{-\beta}{\alpha}
\end{equation}

Thus, if (\ref{eqn:constraint_for_extortion}) holds and \(\chi >1\) a player is
said to extort their opponent.
Here, the reverse problem is considered: given a
\(p\in\mathbb{R}^4\) how does one identify \(\alpha, \beta\) if they
exist and is the strategy in fact acting in an extortionate way?

These conditions correspond to:

\begin{align}
    \tilde p_1 & = \alpha R + \beta R - P (\alpha + \beta)
            \label{eqn:condition_for_tilde_p1}\\
    \tilde p_2 & = \alpha S + \beta T - P (\alpha + \beta)
            \label{eqn:condition_for_tilde_p2}\\
    \tilde p_3 & = \alpha T + \beta S - P (\alpha + \beta)
            \label{eqn:condition_for_tilde_p3}\\
    \tilde p_4 & = \alpha P + \beta P - P (\alpha + \beta)
            \label{eqn:condition_for_tilde_p4}
\end{align}

Equation (\ref{eqn:condition_for_tilde_p4}) ensures that \(p_4=\tilde p_4=0\).
Equations (\ref{eqn:condition_for_tilde_p1}-\ref{eqn:condition_for_tilde_p3})
can be used to eliminate \(\alpha, \beta\), giving:

\begin{equation}\label{eqn:planar_definition_of_extortion}
    \tilde p_1 = \frac{(R - P)(\tilde p_2 + \tilde p_3)}{S + T - 2P}
\end{equation}

with:

\begin{equation}\label{eqn:definition_of_chi}
    \chi = \frac{\tilde p_2 (P - T) + \tilde p_3 (S - P)}
                {\tilde p_2 (P - S) + \tilde p_3 (T - P)}
\end{equation}

Given a strategy \(p\in\mathbb{R}^{4\times 1}\) equations
(\ref{eqn:condition_for_tilde_p4}), (\ref{eqn:planar_definition_of_extortion}-\ref{eqn:definition_of_chi}) can be used to check if
a strategy is extortionate. The conditions correspond to:

\begin{align}
    p_1 & = \frac{(R-P)(p_2 + p_3) - R + T + S - P}{S + T - 2P}
     \label{eqn:condition_for_p1}\\
    p_4 & = 0 \label{eqn:condition_for_p4}\\
    1 & > p_2 + p_3\label{eqn:condition_for_chi}
\end{align}

The algebraic steps necessary to prove these results are available in the
supporting materials.

All extortionate strategies reside on a triangular (\ref{eqn:condition_for_chi})
plane (\ref{eqn:condition_for_p1}) in 3 dimensions (\ref{eqn:condition_for_p4}).
Using this formulation it can be seen that a necessary (but not sufficient)
condition for an extortionate strategy is that it cooperates on average less
than 50\% of the time when in a state of disagreement with the opponent.

As an example, consider the known extortionate strategy \(p=(8 / 9, 1 / 2, 1 /
3, 0)\) from~\cite{Stewart2012} which is referred to as \texttt{Extort-2}. In
this case, for the standard values of \((R, T, S, P)\) constraint
(\ref{eqn:condition_for_p1}) corresponds to:

\begin{equation}
    p_1 = \frac{2(p_2 + p_3) + 1}{3}
\end{equation}

It is clear that in this case all constraints hold.

This approach could in fact be used to confirm that a given strategy is acting
in an extortionate manner even if it is not a memory one strategy. However, in
practice, if a closed form for \(p\) is not known, then due to measurement
and/or numerical error this would not work.

This problem can be written in the following linear algebraic form where
\(x=(\alpha, \beta)\)
and \(p^*=(\tilde p_1 - 1, tilde_2 - 1, p_3)\):

\begin{equation}\label{eqn:linear_algebraic_equation_for_p}
    Cx= p^*
\end{equation}

\(C\) corresponds to equations
(\ref{eqn:condition_for_tilde_p1}-\ref{eqn:condition_for_tilde_p3}) and is
given by:

\begin{equation}\label{eqn:definition_of_C}
    C =
    \begin{bmatrix}
        R - P & R- P \\
        S - P & T- P \\
        T - P & S- P \\
    \end{bmatrix}
\end{equation}

Note that in general, equation (\ref{eqn:linear_algebraic_equation_for_p}) will
not necessarily have a solution. From the Rouch\'{e}-Capelli theorem if there is
a solution it is unique as \(\text{rank}(C)=2\) which is the dimension of the
variable \(x\). The best fitting \(x\) is found by minimizing:

\begin{equation}\label{eqn:r_squared}
    \text{SSError} = \|C x- p^*\|_2^2 = \sum_{i=1}^{3}\left((C\bar x)_i-p_i^*\right)^2
\end{equation}

Note that \(\text{SSError}\), which is the square of the Frobenius
norm~\cite{Golub2013}, becomes a measure of how close a strategy is to being an
extortionate strategy. Suspicion
of extortion then corresponds to a threshold on \(\text{SSError}\).

By observing interactions (human or otherwise), their memory one representation
can be inferred and this approach can be used to recognise extortionate
behaviour. The notion of comparing theoretic and actual plays of the IPD is not
novel, see for example~\cite{Rand2013}. Immediately it is noted that if the
environment is noisy~\cite{Wu1995} then no strategy can be considered to be
extortionate as \(p_4>0\).

In the next section, this idea will be illustrated by observing the interactions
that take place in a computer based tournament of the IPD\@.

\section{Numerical experiments}\label{sec:numerical-experiments}

In~\cite{Stewart2012} results from a tournament with
\input{./assets/tex/number_of_stewart_plotkin_strategies/main.tex} strategies,
was presented with specific consideration given to ZD strategies. This
tournament is reproduced here using the Axelrod-Python
project~\cite{Knight2016}. To obtain a good measure of the corresponding
transition rates for each strategy all matches have been run for
\input{assets/tex/number_of_turns/main.tex} turns and every match has been
repeated \input{assets/tex/number_of_repetitions/main.tex} times. All of this
interaction data is available at~\cite{vincent_knight_2018_1297075}. A good
match between the inferred Markov chain and the state distribution of the actual
interactions has been verified. Data for this is presented in the supplementary
materials.

Figure~\ref{fig:SSError_overall_in_stewart_plotkin} shows the \(\text{SSError}\)
values for all the strategies in the tournament, as reported
in~\cite{Stewart2012} the extortionate strategy (which has an expected
\(\text{SSError}\) approximately 0) gains a large number of wins.

\begin{figure}[!htbp]
    \centering
    \includegraphics[width=.8\textwidth]{./assets/img/SSError_overall_in_stewart_plotkin/main.pdf}
    \caption{\(\text{SSError}\) and state probabilities for the strategies
        of~\cite{Stewart2012}, ordered both by number of wins and overall score.
        Note that \(P(DC)\) is not shown as it corresponds to the transpose of
        \(P(CD)\). Cooperator and Defector are omitted as they do not visit all
        the states.}
    \label{fig:SSError_overall_in_stewart_plotkin}
\end{figure}

Here, the work of~\cite{Stewart2012} is extended by investigating a tournament
with \input{assets/tex/number_of_full_strategies/main.tex}
strategies.

The results of this analysis are shown in
Figure~\ref{fig:SSError_and_probabilities_in_full}. The top ranking strategies
by number of wins seem to be extortionate (but not against all strategies) and
it can be seen that a small sub group of strategies achieve mutual defection.
All the top ranking strategies according to score achieve mutual cooperation and
do not extort each other, however they
\textbf{do} exhibit extortionate behaviour towards a number of the lower ranking
strategies.

\begin{figure}[!htbp]
    \centering
    \includegraphics[width=.8\textwidth]{./assets/img/SSError_and_probabilities_in_full/main.pdf}
    \caption{\(\text{SSError}\) for the strategies for the full tournament. Only
    strategy interactions for which \(p_4=0\) and \(\chi>1\) are displayed.}
    \label{fig:SSError_and_probabilities_in_full}
\end{figure}

\section{Conclusion}\label{sec:conclusion}

This work defines an approach to measure whether or not a player is playing a
strategy that corresponds to an extortionate strategy as defined
in~\cite{Press2012}: a mathematical model for suspicion. Indeed, all
extortionate strategies have been
 classified as lying on a triangular plane.
This rigorous classification fails to be robust to small measurement error, thus
a statistical approach is proposed.
This is done through a linear algebraic approach for approximating the solution
of a linear system. Using this, a large number of pairwise interactions is
simulated and in fact very few strategies are found to act extortionately.

The work of~\cite{Press2012}, whilst showing that a clever approach to taking
advantage of another memory one strategy exists: this is incomplete. Whilst the
elegance of this result is very attractive, just as the simplicity of the
victory of Tit For Tat in Axelrod's original tournaments was, it is incomplete.
Extortionate strategies achieve a high number of wins but they do not
achieve a high score which corresponds to the fitness landscape in an
evolutionary sense. From the large number of interactions a payoff matrix \(S\)
can be measured where \(S_{ij}\) denotes the score (using standard values of
\((R, S, T, P) = (3, 0, 5, 1)\)) of the \(i\)th strategy
against the \(j\)th strategy. Using this, the replicator equation
describes the evolution of the system based on a population density fitness
function:

\begin{equation}\label{eqn:replicator_dynamics}
    \frac{dx}{dt} = x(S-x^TS x)
\end{equation}

Equation (\ref{eqn:replicator_dynamics}) is solved numerically through an
integration technique described in~\cite{Petzold1983} and
Figure~\ref{fig:replicator_dynamics} shows the evolution of the distribution of
the system: the various strategies are ranked by scores. It is clear to see that
only the high ranking strategies survive the evolutionary process (in fact,
only \input{./assets/img/replicator_dynamics/main.tex}
have a final distribution greater than \(10 ^ {-2}\)). This confirms the
findings of~\cite{Moran1707} in which sophisticated strategies resist
evolutionary invasion of shorter memory strategies. Recalling
Figure~\ref{fig:SSError_and_probabilities_in_full} this demonstrates that:

\begin{itemize}
    \item Cooperation emerges through the evolutionary process: the high scoring
        strategies do not exhibit extortionate behaviour towards each other.
    \item Extortionate strategies do not survive the evolutionary process.
\end{itemize}

\begin{figure}[!htbp]
    \centering
    \includegraphics[width=.8\textwidth]{./assets/img/replicator_dynamics/main.pdf}
    \caption{Numerical simulation of the replicator equation
    (\ref{eqn:replicator_dynamics}): strategies are ordered by score, only the strategies with a high score survive the evolutionary process.}
    \label{fig:replicator_dynamics}
\end{figure}

This work can be used to classify plays of the IPD\@: data can be collected from
actual interactions (in lab or in the field). Furthermore, this allows for a
classification method similar to the notion of fingerprinting presented
in~\cite{Ashlock2008}. Trained strategies can potentially be classified as
extortionate or not or it could be possible to even constrain the reinforcement
learning approaches that are becoming prevalent in the literature.
Alternatively, this mathematical approach for recognising extortion could be
used in sophisticated strategies to defend against invasion. Arguably, some of
the strategies considered here exhibit this behaviour, indeed as described
in~\cite{Harper2017}, the top ranking strategies in the full tournament are
obtained using evolutionary reinforcement learning techniques, thus, suspicion
of extortionate behaviour could in fact be an evolutionary trait.

\section*{Acknowledgements}

The following open source software libraries were used in this research:

\begin{itemize}
    \item The Axelrod ~\cite{Knight2016, Knight2018} library (IPD strategies and
        tournaments).
    \item The sympy library~\cite{Meurer2017} (verification of all symbolic
        calculations).
    \item The matplotlib~\cite{Droettboom2018} library (visualisation).
    \item The pandas~\cite{Structures2010}, dask~\cite{Dask2016} and
        NumPy~\cite{Oliphant2015} libraries (data manipulation).
    \item The SciPy~\cite{Jones2001} library (numerical integration of the
        replicator equation).
\end{itemize}

This work was performed using the computational facilities of the Advanced
Research Computing @ Cardiff (ARCCA) Division, Cardiff University.

\printbibliography

\newpage
\section*{Supplementary materials}

\includepdf{assets/pdf/proof_of_form_of_extortionate_strategies/main.pdf}

\newpage

Using the pair wise interactions the transition rates \(p,
q\) can be measured and the steady state probabilities inferred and compared to
the actual probabilities of each state.
This is done numerically by computing the singular eigenvector of the
matrix \(A\) \cite{Stewart2009}:

\[
    A =
    \begin{bmatrix}
        p_1 q_1 & p_1 (1 - q_1) & (1 - p_1) q_1 & (1 -p_1) (1 - q_1) \\
        p_2 q_2 & p_2 (1 - q_2) & (1 - p_2) q_2 & (1 -p_2) (1 - q_2) \\
        p_3 q_3 & p_3 (1 - q_3) & (1 - p_3) q_3 & (1 -p_3) (1 - q_3) \\
        p_4 q_4 & p_4 (1 - q_4) & (1 - p_4) q_4 & (1 -p_4) (1 - q_4) \\
    \end{bmatrix}
\]

Figure~\ref{fig:computed_probabilities_vs_theoretic_probabilities} shows a
regression line fitted to every pairwise interaction with a reported
\(\text{SSError}\) value (pairwise interactions with missing states were
omitted). This serves to validate the approach: a part from some edge cases the
relationship is consistent.

\begin{figure}[!htbp]
    \centering
    \includegraphics[width=.8\textwidth]{./assets/img/computed_probabilities_vs_theoretic_probabilities/main.pdf}
    \caption{The
        relationship between the steady state probabilities inferred from the
        measured transitions and the actual steady state probabilities. A linear
        regression line is included validating the approach.}
    \label{fig:computed_probabilities_vs_theoretic_probabilities}
\end{figure}


\end{document}
 strategies,
was presented with specific consideration given to ZD strategies. This
tournament is reproduced here using the Axelrod-Python
project~\cite{Knight2016}. To obtain a good measure of the corresponding
transition rates for each strategy all matches have been run for
\documentclass[a4paper]{article}

\usepackage{amsmath}
\usepackage{amssymb}
\usepackage[margin=1.5cm,
            includefoot,
            footskip=30pt]{geometry}
\usepackage{layout}
\usepackage{graphicx}
\usepackage{subcaption}

\usepackage{biblatex}
\usepackage{pdfpages}

\bibliography{main.bib}

\title{Suspicion: Recognising and evaluating the effectiveness
       of extortion in the Iterated Prisoner's Dilemma}
\author{Vincent A. Knight \and Nikoleta E. Glynatsi}
\date{\today}



\begin{document}

\maketitle

\begin{abstract}
    The Iterated Prisoner's Dilemma is a model for rational and evolutionary
    interactive behaviour. It has applications both in the study of human social
    behaviour as well as in biology.
    It is used to understand when and how a rational individual might
    accept an immediate cost to their own utility for the direct benefit of
    another.

    Much attention has been given to a class of strategies called
    Zero Determinant strategies. It has been theoretically shown that these
    strategies can ``extort'' any player.

    In this work, an approach to identify if observed strategies are playing in
    an extortionate way is described. Furthermore, experimental analysis of
    a large tournament with \input{assets/tex/number_of_full_strategies/main.tex}
    strategies is considered. In this setting
    the most highly performing strategies do not play in an extortionate way
    against each other but do against lower performing strategies.
    This suggests that whilst the theory of Zero Determinant strategies
    indicates that memory is not of fundamental importance to the evolution of
    cooperative behaviour, this is incomplete.
\end{abstract}

\section{Introduction}\label{sec:introduction}

Agent based game theoretic models have become a stalwart of the underpinning
mathematics of interactive behaviours. One of the major pieces of work
in this area is the pair of original computer tournaments run by Robert
Axelrod~\cite{Axelrod1980, Axelrod1980a}. These tournaments pitted submitted
computer strategies against each other in plays of the Iterated Prisoner's
Dilemma. A common game where agents can choose to pay a slight cost to their
immediate utility in the hope of building a reputation. This has been used in
economic and evolutionary game theory to understand the evolution of cooperative
behaviour.

Recently, a class of strategies was described in~\cite{Press2012} that can
provably extort any given opponent. In~\cite{Hilbe2013, Moran1707} some
questions have already been asked about the true effectiveness of these
strategies in an evolutionary setting. Here another question is asked: is it
possible to recognise this extortionate behaviour? A mathematical procedure for
suspicion is presented: in the same way that the continued actions of an
extortionate individual might raise suspicion.

This work makes use of the Axelrod Python library~\cite{Knight2018, Knight2016}
with a large number of Prisoner Dilemma strategies available to give an
extensive numerical example of the ideas presented.  The approach is presented
in Section~\ref{sec:delta-zd-strategies}.  All of the code and data discussed
in Section~\ref{sec:numerical-experiments} is open sourced, archived and
written according to best scientific principles~\cite{Wilson2014}. The data
archive can be found at~\cite{vincent_knight_2018_1297075}.

\section{Recognising Extortion}\label{sec:delta-zd-strategies}

In~\cite{Press2012}, given a match between 2 memory-one strategies, the concept
of Zero Determinant (ZD) strategies is introduced. The main result of that paper
shows that given two memory one players \(p, q\in\mathbb{R}^4\) a linear
relationship between the players' scores could be forced by one of the players.

Using the notation of~\cite{Press2012}, assuming the utilities for player \(p\)
are given by \(S_x=(R, S, T, P)\) and for player \(q\) by \(S_y=(R, T, S, P)\)
and that the stationary scores of each player is given by \(S_X\) and \(S_Y\)
respectively. The main result of~\cite{Press2012} is that if

\begin{equation}\label{eqn:linear_relationship_for_p}
    \tilde p=\alpha S_x + \beta S_y + \gamma
\end{equation}

or

\begin{equation}\label{eqn:linear_relationship_for_q}
    \tilde q=\alpha S_x + \beta S_y + \gamma
\end{equation}

where \(\tilde p = (1 - p_1, 1 - p_2, p_3, p_4)\) and
\(\tilde q = (1 - q_1, 1 - q_2, q_3, q_4)\) then:

\begin{equation}
    \alpha S_X + \beta S_Y + \gamma = 0
\end{equation}

In~\cite{Press2012} a particular type of ZD strategy is defined: extortionate
strategies. If:

\begin{equation}\label{eqn:constraint_for_extortion}
    \gamma = - P(\alpha + \beta)
\end{equation}

then the player can ensure they get a score \(\chi\) times
larger than the opponent. This extortion coefficient is given by:

\begin{equation}\label{eqn:definition_of_chi}
    \chi=\frac{-\beta}{\alpha}
\end{equation}

Thus, if (\ref{eqn:constraint_for_extortion}) holds and \(\chi >1\) a player is
said to extort their opponent.
Here, the reverse problem is considered: given a
\(p\in\mathbb{R}^4\) how does one identify \(\alpha, \beta\) if they
exist and is the strategy in fact acting in an extortionate way?

These conditions correspond to:

\begin{align}
    \tilde p_1 & = \alpha R + \beta R - P (\alpha + \beta)
            \label{eqn:condition_for_tilde_p1}\\
    \tilde p_2 & = \alpha S + \beta T - P (\alpha + \beta)
            \label{eqn:condition_for_tilde_p2}\\
    \tilde p_3 & = \alpha T + \beta S - P (\alpha + \beta)
            \label{eqn:condition_for_tilde_p3}\\
    \tilde p_4 & = \alpha P + \beta P - P (\alpha + \beta)
            \label{eqn:condition_for_tilde_p4}
\end{align}

Equation (\ref{eqn:condition_for_tilde_p4}) ensures that \(p_4=\tilde p_4=0\).
Equations (\ref{eqn:condition_for_tilde_p1}-\ref{eqn:condition_for_tilde_p3})
can be used to eliminate \(\alpha, \beta\), giving:

\begin{equation}\label{eqn:planar_definition_of_extortion}
    \tilde p_1 = \frac{(R - P)(\tilde p_2 + \tilde p_3)}{S + T - 2P}
\end{equation}

with:

\begin{equation}\label{eqn:definition_of_chi}
    \chi = \frac{\tilde p_2 (P - T) + \tilde p_3 (S - P)}
                {\tilde p_2 (P - S) + \tilde p_3 (T - P)}
\end{equation}

Given a strategy \(p\in\mathbb{R}^{4\times 1}\) equations
(\ref{eqn:condition_for_tilde_p4}), (\ref{eqn:planar_definition_of_extortion}-\ref{eqn:definition_of_chi}) can be used to check if
a strategy is extortionate. The conditions correspond to:

\begin{align}
    p_1 & = \frac{(R-P)(p_2 + p_3) - R + T + S - P}{S + T - 2P}
     \label{eqn:condition_for_p1}\\
    p_4 & = 0 \label{eqn:condition_for_p4}\\
    1 & > p_2 + p_3\label{eqn:condition_for_chi}
\end{align}

The algebraic steps necessary to prove these results are available in the
supporting materials.

All extortionate strategies reside on a triangular (\ref{eqn:condition_for_chi})
plane (\ref{eqn:condition_for_p1}) in 3 dimensions (\ref{eqn:condition_for_p4}).
Using this formulation it can be seen that a necessary (but not sufficient)
condition for an extortionate strategy is that it cooperates on average less
than 50\% of the time when in a state of disagreement with the opponent.

As an example, consider the known extortionate strategy \(p=(8 / 9, 1 / 2, 1 /
3, 0)\) from~\cite{Stewart2012} which is referred to as \texttt{Extort-2}. In
this case, for the standard values of \((R, T, S, P)\) constraint
(\ref{eqn:condition_for_p1}) corresponds to:

\begin{equation}
    p_1 = \frac{2(p_2 + p_3) + 1}{3}
\end{equation}

It is clear that in this case all constraints hold.

This approach could in fact be used to confirm that a given strategy is acting
in an extortionate manner even if it is not a memory one strategy. However, in
practice, if a closed form for \(p\) is not known, then due to measurement
and/or numerical error this would not work.

This problem can be written in the following linear algebraic form where
\(x=(\alpha, \beta)\)
and \(p^*=(\tilde p_1 - 1, tilde_2 - 1, p_3)\):

\begin{equation}\label{eqn:linear_algebraic_equation_for_p}
    Cx= p^*
\end{equation}

\(C\) corresponds to equations
(\ref{eqn:condition_for_tilde_p1}-\ref{eqn:condition_for_tilde_p3}) and is
given by:

\begin{equation}\label{eqn:definition_of_C}
    C =
    \begin{bmatrix}
        R - P & R- P \\
        S - P & T- P \\
        T - P & S- P \\
    \end{bmatrix}
\end{equation}

Note that in general, equation (\ref{eqn:linear_algebraic_equation_for_p}) will
not necessarily have a solution. From the Rouch\'{e}-Capelli theorem if there is
a solution it is unique as \(\text{rank}(C)=2\) which is the dimension of the
variable \(x\). The best fitting \(x\) is found by minimizing:

\begin{equation}\label{eqn:r_squared}
    \text{SSError} = \|C x- p^*\|_2^2 = \sum_{i=1}^{3}\left((C\bar x)_i-p_i^*\right)^2
\end{equation}

Note that \(\text{SSError}\), which is the square of the Frobenius
norm~\cite{Golub2013}, becomes a measure of how close a strategy is to being an
extortionate strategy. Suspicion
of extortion then corresponds to a threshold on \(\text{SSError}\).

By observing interactions (human or otherwise), their memory one representation
can be inferred and this approach can be used to recognise extortionate
behaviour. The notion of comparing theoretic and actual plays of the IPD is not
novel, see for example~\cite{Rand2013}. Immediately it is noted that if the
environment is noisy~\cite{Wu1995} then no strategy can be considered to be
extortionate as \(p_4>0\).

In the next section, this idea will be illustrated by observing the interactions
that take place in a computer based tournament of the IPD\@.

\section{Numerical experiments}\label{sec:numerical-experiments}

In~\cite{Stewart2012} results from a tournament with
\input{./assets/tex/number_of_stewart_plotkin_strategies/main.tex} strategies,
was presented with specific consideration given to ZD strategies. This
tournament is reproduced here using the Axelrod-Python
project~\cite{Knight2016}. To obtain a good measure of the corresponding
transition rates for each strategy all matches have been run for
\input{assets/tex/number_of_turns/main.tex} turns and every match has been
repeated \input{assets/tex/number_of_repetitions/main.tex} times. All of this
interaction data is available at~\cite{vincent_knight_2018_1297075}. A good
match between the inferred Markov chain and the state distribution of the actual
interactions has been verified. Data for this is presented in the supplementary
materials.

Figure~\ref{fig:SSError_overall_in_stewart_plotkin} shows the \(\text{SSError}\)
values for all the strategies in the tournament, as reported
in~\cite{Stewart2012} the extortionate strategy (which has an expected
\(\text{SSError}\) approximately 0) gains a large number of wins.

\begin{figure}[!htbp]
    \centering
    \includegraphics[width=.8\textwidth]{./assets/img/SSError_overall_in_stewart_plotkin/main.pdf}
    \caption{\(\text{SSError}\) and state probabilities for the strategies
        of~\cite{Stewart2012}, ordered both by number of wins and overall score.
        Note that \(P(DC)\) is not shown as it corresponds to the transpose of
        \(P(CD)\). Cooperator and Defector are omitted as they do not visit all
        the states.}
    \label{fig:SSError_overall_in_stewart_plotkin}
\end{figure}

Here, the work of~\cite{Stewart2012} is extended by investigating a tournament
with \input{assets/tex/number_of_full_strategies/main.tex}
strategies.

The results of this analysis are shown in
Figure~\ref{fig:SSError_and_probabilities_in_full}. The top ranking strategies
by number of wins seem to be extortionate (but not against all strategies) and
it can be seen that a small sub group of strategies achieve mutual defection.
All the top ranking strategies according to score achieve mutual cooperation and
do not extort each other, however they
\textbf{do} exhibit extortionate behaviour towards a number of the lower ranking
strategies.

\begin{figure}[!htbp]
    \centering
    \includegraphics[width=.8\textwidth]{./assets/img/SSError_and_probabilities_in_full/main.pdf}
    \caption{\(\text{SSError}\) for the strategies for the full tournament. Only
    strategy interactions for which \(p_4=0\) and \(\chi>1\) are displayed.}
    \label{fig:SSError_and_probabilities_in_full}
\end{figure}

\section{Conclusion}\label{sec:conclusion}

This work defines an approach to measure whether or not a player is playing a
strategy that corresponds to an extortionate strategy as defined
in~\cite{Press2012}: a mathematical model for suspicion. Indeed, all
extortionate strategies have been
 classified as lying on a triangular plane.
This rigorous classification fails to be robust to small measurement error, thus
a statistical approach is proposed.
This is done through a linear algebraic approach for approximating the solution
of a linear system. Using this, a large number of pairwise interactions is
simulated and in fact very few strategies are found to act extortionately.

The work of~\cite{Press2012}, whilst showing that a clever approach to taking
advantage of another memory one strategy exists: this is incomplete. Whilst the
elegance of this result is very attractive, just as the simplicity of the
victory of Tit For Tat in Axelrod's original tournaments was, it is incomplete.
Extortionate strategies achieve a high number of wins but they do not
achieve a high score which corresponds to the fitness landscape in an
evolutionary sense. From the large number of interactions a payoff matrix \(S\)
can be measured where \(S_{ij}\) denotes the score (using standard values of
\((R, S, T, P) = (3, 0, 5, 1)\)) of the \(i\)th strategy
against the \(j\)th strategy. Using this, the replicator equation
describes the evolution of the system based on a population density fitness
function:

\begin{equation}\label{eqn:replicator_dynamics}
    \frac{dx}{dt} = x(S-x^TS x)
\end{equation}

Equation (\ref{eqn:replicator_dynamics}) is solved numerically through an
integration technique described in~\cite{Petzold1983} and
Figure~\ref{fig:replicator_dynamics} shows the evolution of the distribution of
the system: the various strategies are ranked by scores. It is clear to see that
only the high ranking strategies survive the evolutionary process (in fact,
only \input{./assets/img/replicator_dynamics/main.tex}
have a final distribution greater than \(10 ^ {-2}\)). This confirms the
findings of~\cite{Moran1707} in which sophisticated strategies resist
evolutionary invasion of shorter memory strategies. Recalling
Figure~\ref{fig:SSError_and_probabilities_in_full} this demonstrates that:

\begin{itemize}
    \item Cooperation emerges through the evolutionary process: the high scoring
        strategies do not exhibit extortionate behaviour towards each other.
    \item Extortionate strategies do not survive the evolutionary process.
\end{itemize}

\begin{figure}[!htbp]
    \centering
    \includegraphics[width=.8\textwidth]{./assets/img/replicator_dynamics/main.pdf}
    \caption{Numerical simulation of the replicator equation
    (\ref{eqn:replicator_dynamics}): strategies are ordered by score, only the strategies with a high score survive the evolutionary process.}
    \label{fig:replicator_dynamics}
\end{figure}

This work can be used to classify plays of the IPD\@: data can be collected from
actual interactions (in lab or in the field). Furthermore, this allows for a
classification method similar to the notion of fingerprinting presented
in~\cite{Ashlock2008}. Trained strategies can potentially be classified as
extortionate or not or it could be possible to even constrain the reinforcement
learning approaches that are becoming prevalent in the literature.
Alternatively, this mathematical approach for recognising extortion could be
used in sophisticated strategies to defend against invasion. Arguably, some of
the strategies considered here exhibit this behaviour, indeed as described
in~\cite{Harper2017}, the top ranking strategies in the full tournament are
obtained using evolutionary reinforcement learning techniques, thus, suspicion
of extortionate behaviour could in fact be an evolutionary trait.

\section*{Acknowledgements}

The following open source software libraries were used in this research:

\begin{itemize}
    \item The Axelrod ~\cite{Knight2016, Knight2018} library (IPD strategies and
        tournaments).
    \item The sympy library~\cite{Meurer2017} (verification of all symbolic
        calculations).
    \item The matplotlib~\cite{Droettboom2018} library (visualisation).
    \item The pandas~\cite{Structures2010}, dask~\cite{Dask2016} and
        NumPy~\cite{Oliphant2015} libraries (data manipulation).
    \item The SciPy~\cite{Jones2001} library (numerical integration of the
        replicator equation).
\end{itemize}

This work was performed using the computational facilities of the Advanced
Research Computing @ Cardiff (ARCCA) Division, Cardiff University.

\printbibliography

\newpage
\section*{Supplementary materials}

\includepdf{assets/pdf/proof_of_form_of_extortionate_strategies/main.pdf}

\newpage

Using the pair wise interactions the transition rates \(p,
q\) can be measured and the steady state probabilities inferred and compared to
the actual probabilities of each state.
This is done numerically by computing the singular eigenvector of the
matrix \(A\) \cite{Stewart2009}:

\[
    A =
    \begin{bmatrix}
        p_1 q_1 & p_1 (1 - q_1) & (1 - p_1) q_1 & (1 -p_1) (1 - q_1) \\
        p_2 q_2 & p_2 (1 - q_2) & (1 - p_2) q_2 & (1 -p_2) (1 - q_2) \\
        p_3 q_3 & p_3 (1 - q_3) & (1 - p_3) q_3 & (1 -p_3) (1 - q_3) \\
        p_4 q_4 & p_4 (1 - q_4) & (1 - p_4) q_4 & (1 -p_4) (1 - q_4) \\
    \end{bmatrix}
\]

Figure~\ref{fig:computed_probabilities_vs_theoretic_probabilities} shows a
regression line fitted to every pairwise interaction with a reported
\(\text{SSError}\) value (pairwise interactions with missing states were
omitted). This serves to validate the approach: a part from some edge cases the
relationship is consistent.

\begin{figure}[!htbp]
    \centering
    \includegraphics[width=.8\textwidth]{./assets/img/computed_probabilities_vs_theoretic_probabilities/main.pdf}
    \caption{The
        relationship between the steady state probabilities inferred from the
        measured transitions and the actual steady state probabilities. A linear
        regression line is included validating the approach.}
    \label{fig:computed_probabilities_vs_theoretic_probabilities}
\end{figure}


\end{document}
 turns and every match has been
repeated \documentclass[a4paper]{article}

\usepackage{amsmath}
\usepackage{amssymb}
\usepackage[margin=1.5cm,
            includefoot,
            footskip=30pt]{geometry}
\usepackage{layout}
\usepackage{graphicx}
\usepackage{subcaption}

\usepackage{biblatex}
\usepackage{pdfpages}

\bibliography{main.bib}

\title{Suspicion: Recognising and evaluating the effectiveness
       of extortion in the Iterated Prisoner's Dilemma}
\author{Vincent A. Knight \and Nikoleta E. Glynatsi}
\date{\today}



\begin{document}

\maketitle

\begin{abstract}
    The Iterated Prisoner's Dilemma is a model for rational and evolutionary
    interactive behaviour. It has applications both in the study of human social
    behaviour as well as in biology.
    It is used to understand when and how a rational individual might
    accept an immediate cost to their own utility for the direct benefit of
    another.

    Much attention has been given to a class of strategies called
    Zero Determinant strategies. It has been theoretically shown that these
    strategies can ``extort'' any player.

    In this work, an approach to identify if observed strategies are playing in
    an extortionate way is described. Furthermore, experimental analysis of
    a large tournament with \input{assets/tex/number_of_full_strategies/main.tex}
    strategies is considered. In this setting
    the most highly performing strategies do not play in an extortionate way
    against each other but do against lower performing strategies.
    This suggests that whilst the theory of Zero Determinant strategies
    indicates that memory is not of fundamental importance to the evolution of
    cooperative behaviour, this is incomplete.
\end{abstract}

\section{Introduction}\label{sec:introduction}

Agent based game theoretic models have become a stalwart of the underpinning
mathematics of interactive behaviours. One of the major pieces of work
in this area is the pair of original computer tournaments run by Robert
Axelrod~\cite{Axelrod1980, Axelrod1980a}. These tournaments pitted submitted
computer strategies against each other in plays of the Iterated Prisoner's
Dilemma. A common game where agents can choose to pay a slight cost to their
immediate utility in the hope of building a reputation. This has been used in
economic and evolutionary game theory to understand the evolution of cooperative
behaviour.

Recently, a class of strategies was described in~\cite{Press2012} that can
provably extort any given opponent. In~\cite{Hilbe2013, Moran1707} some
questions have already been asked about the true effectiveness of these
strategies in an evolutionary setting. Here another question is asked: is it
possible to recognise this extortionate behaviour? A mathematical procedure for
suspicion is presented: in the same way that the continued actions of an
extortionate individual might raise suspicion.

This work makes use of the Axelrod Python library~\cite{Knight2018, Knight2016}
with a large number of Prisoner Dilemma strategies available to give an
extensive numerical example of the ideas presented.  The approach is presented
in Section~\ref{sec:delta-zd-strategies}.  All of the code and data discussed
in Section~\ref{sec:numerical-experiments} is open sourced, archived and
written according to best scientific principles~\cite{Wilson2014}. The data
archive can be found at~\cite{vincent_knight_2018_1297075}.

\section{Recognising Extortion}\label{sec:delta-zd-strategies}

In~\cite{Press2012}, given a match between 2 memory-one strategies, the concept
of Zero Determinant (ZD) strategies is introduced. The main result of that paper
shows that given two memory one players \(p, q\in\mathbb{R}^4\) a linear
relationship between the players' scores could be forced by one of the players.

Using the notation of~\cite{Press2012}, assuming the utilities for player \(p\)
are given by \(S_x=(R, S, T, P)\) and for player \(q\) by \(S_y=(R, T, S, P)\)
and that the stationary scores of each player is given by \(S_X\) and \(S_Y\)
respectively. The main result of~\cite{Press2012} is that if

\begin{equation}\label{eqn:linear_relationship_for_p}
    \tilde p=\alpha S_x + \beta S_y + \gamma
\end{equation}

or

\begin{equation}\label{eqn:linear_relationship_for_q}
    \tilde q=\alpha S_x + \beta S_y + \gamma
\end{equation}

where \(\tilde p = (1 - p_1, 1 - p_2, p_3, p_4)\) and
\(\tilde q = (1 - q_1, 1 - q_2, q_3, q_4)\) then:

\begin{equation}
    \alpha S_X + \beta S_Y + \gamma = 0
\end{equation}

In~\cite{Press2012} a particular type of ZD strategy is defined: extortionate
strategies. If:

\begin{equation}\label{eqn:constraint_for_extortion}
    \gamma = - P(\alpha + \beta)
\end{equation}

then the player can ensure they get a score \(\chi\) times
larger than the opponent. This extortion coefficient is given by:

\begin{equation}\label{eqn:definition_of_chi}
    \chi=\frac{-\beta}{\alpha}
\end{equation}

Thus, if (\ref{eqn:constraint_for_extortion}) holds and \(\chi >1\) a player is
said to extort their opponent.
Here, the reverse problem is considered: given a
\(p\in\mathbb{R}^4\) how does one identify \(\alpha, \beta\) if they
exist and is the strategy in fact acting in an extortionate way?

These conditions correspond to:

\begin{align}
    \tilde p_1 & = \alpha R + \beta R - P (\alpha + \beta)
            \label{eqn:condition_for_tilde_p1}\\
    \tilde p_2 & = \alpha S + \beta T - P (\alpha + \beta)
            \label{eqn:condition_for_tilde_p2}\\
    \tilde p_3 & = \alpha T + \beta S - P (\alpha + \beta)
            \label{eqn:condition_for_tilde_p3}\\
    \tilde p_4 & = \alpha P + \beta P - P (\alpha + \beta)
            \label{eqn:condition_for_tilde_p4}
\end{align}

Equation (\ref{eqn:condition_for_tilde_p4}) ensures that \(p_4=\tilde p_4=0\).
Equations (\ref{eqn:condition_for_tilde_p1}-\ref{eqn:condition_for_tilde_p3})
can be used to eliminate \(\alpha, \beta\), giving:

\begin{equation}\label{eqn:planar_definition_of_extortion}
    \tilde p_1 = \frac{(R - P)(\tilde p_2 + \tilde p_3)}{S + T - 2P}
\end{equation}

with:

\begin{equation}\label{eqn:definition_of_chi}
    \chi = \frac{\tilde p_2 (P - T) + \tilde p_3 (S - P)}
                {\tilde p_2 (P - S) + \tilde p_3 (T - P)}
\end{equation}

Given a strategy \(p\in\mathbb{R}^{4\times 1}\) equations
(\ref{eqn:condition_for_tilde_p4}), (\ref{eqn:planar_definition_of_extortion}-\ref{eqn:definition_of_chi}) can be used to check if
a strategy is extortionate. The conditions correspond to:

\begin{align}
    p_1 & = \frac{(R-P)(p_2 + p_3) - R + T + S - P}{S + T - 2P}
     \label{eqn:condition_for_p1}\\
    p_4 & = 0 \label{eqn:condition_for_p4}\\
    1 & > p_2 + p_3\label{eqn:condition_for_chi}
\end{align}

The algebraic steps necessary to prove these results are available in the
supporting materials.

All extortionate strategies reside on a triangular (\ref{eqn:condition_for_chi})
plane (\ref{eqn:condition_for_p1}) in 3 dimensions (\ref{eqn:condition_for_p4}).
Using this formulation it can be seen that a necessary (but not sufficient)
condition for an extortionate strategy is that it cooperates on average less
than 50\% of the time when in a state of disagreement with the opponent.

As an example, consider the known extortionate strategy \(p=(8 / 9, 1 / 2, 1 /
3, 0)\) from~\cite{Stewart2012} which is referred to as \texttt{Extort-2}. In
this case, for the standard values of \((R, T, S, P)\) constraint
(\ref{eqn:condition_for_p1}) corresponds to:

\begin{equation}
    p_1 = \frac{2(p_2 + p_3) + 1}{3}
\end{equation}

It is clear that in this case all constraints hold.

This approach could in fact be used to confirm that a given strategy is acting
in an extortionate manner even if it is not a memory one strategy. However, in
practice, if a closed form for \(p\) is not known, then due to measurement
and/or numerical error this would not work.

This problem can be written in the following linear algebraic form where
\(x=(\alpha, \beta)\)
and \(p^*=(\tilde p_1 - 1, tilde_2 - 1, p_3)\):

\begin{equation}\label{eqn:linear_algebraic_equation_for_p}
    Cx= p^*
\end{equation}

\(C\) corresponds to equations
(\ref{eqn:condition_for_tilde_p1}-\ref{eqn:condition_for_tilde_p3}) and is
given by:

\begin{equation}\label{eqn:definition_of_C}
    C =
    \begin{bmatrix}
        R - P & R- P \\
        S - P & T- P \\
        T - P & S- P \\
    \end{bmatrix}
\end{equation}

Note that in general, equation (\ref{eqn:linear_algebraic_equation_for_p}) will
not necessarily have a solution. From the Rouch\'{e}-Capelli theorem if there is
a solution it is unique as \(\text{rank}(C)=2\) which is the dimension of the
variable \(x\). The best fitting \(x\) is found by minimizing:

\begin{equation}\label{eqn:r_squared}
    \text{SSError} = \|C x- p^*\|_2^2 = \sum_{i=1}^{3}\left((C\bar x)_i-p_i^*\right)^2
\end{equation}

Note that \(\text{SSError}\), which is the square of the Frobenius
norm~\cite{Golub2013}, becomes a measure of how close a strategy is to being an
extortionate strategy. Suspicion
of extortion then corresponds to a threshold on \(\text{SSError}\).

By observing interactions (human or otherwise), their memory one representation
can be inferred and this approach can be used to recognise extortionate
behaviour. The notion of comparing theoretic and actual plays of the IPD is not
novel, see for example~\cite{Rand2013}. Immediately it is noted that if the
environment is noisy~\cite{Wu1995} then no strategy can be considered to be
extortionate as \(p_4>0\).

In the next section, this idea will be illustrated by observing the interactions
that take place in a computer based tournament of the IPD\@.

\section{Numerical experiments}\label{sec:numerical-experiments}

In~\cite{Stewart2012} results from a tournament with
\input{./assets/tex/number_of_stewart_plotkin_strategies/main.tex} strategies,
was presented with specific consideration given to ZD strategies. This
tournament is reproduced here using the Axelrod-Python
project~\cite{Knight2016}. To obtain a good measure of the corresponding
transition rates for each strategy all matches have been run for
\input{assets/tex/number_of_turns/main.tex} turns and every match has been
repeated \input{assets/tex/number_of_repetitions/main.tex} times. All of this
interaction data is available at~\cite{vincent_knight_2018_1297075}. A good
match between the inferred Markov chain and the state distribution of the actual
interactions has been verified. Data for this is presented in the supplementary
materials.

Figure~\ref{fig:SSError_overall_in_stewart_plotkin} shows the \(\text{SSError}\)
values for all the strategies in the tournament, as reported
in~\cite{Stewart2012} the extortionate strategy (which has an expected
\(\text{SSError}\) approximately 0) gains a large number of wins.

\begin{figure}[!htbp]
    \centering
    \includegraphics[width=.8\textwidth]{./assets/img/SSError_overall_in_stewart_plotkin/main.pdf}
    \caption{\(\text{SSError}\) and state probabilities for the strategies
        of~\cite{Stewart2012}, ordered both by number of wins and overall score.
        Note that \(P(DC)\) is not shown as it corresponds to the transpose of
        \(P(CD)\). Cooperator and Defector are omitted as they do not visit all
        the states.}
    \label{fig:SSError_overall_in_stewart_plotkin}
\end{figure}

Here, the work of~\cite{Stewart2012} is extended by investigating a tournament
with \input{assets/tex/number_of_full_strategies/main.tex}
strategies.

The results of this analysis are shown in
Figure~\ref{fig:SSError_and_probabilities_in_full}. The top ranking strategies
by number of wins seem to be extortionate (but not against all strategies) and
it can be seen that a small sub group of strategies achieve mutual defection.
All the top ranking strategies according to score achieve mutual cooperation and
do not extort each other, however they
\textbf{do} exhibit extortionate behaviour towards a number of the lower ranking
strategies.

\begin{figure}[!htbp]
    \centering
    \includegraphics[width=.8\textwidth]{./assets/img/SSError_and_probabilities_in_full/main.pdf}
    \caption{\(\text{SSError}\) for the strategies for the full tournament. Only
    strategy interactions for which \(p_4=0\) and \(\chi>1\) are displayed.}
    \label{fig:SSError_and_probabilities_in_full}
\end{figure}

\section{Conclusion}\label{sec:conclusion}

This work defines an approach to measure whether or not a player is playing a
strategy that corresponds to an extortionate strategy as defined
in~\cite{Press2012}: a mathematical model for suspicion. Indeed, all
extortionate strategies have been
 classified as lying on a triangular plane.
This rigorous classification fails to be robust to small measurement error, thus
a statistical approach is proposed.
This is done through a linear algebraic approach for approximating the solution
of a linear system. Using this, a large number of pairwise interactions is
simulated and in fact very few strategies are found to act extortionately.

The work of~\cite{Press2012}, whilst showing that a clever approach to taking
advantage of another memory one strategy exists: this is incomplete. Whilst the
elegance of this result is very attractive, just as the simplicity of the
victory of Tit For Tat in Axelrod's original tournaments was, it is incomplete.
Extortionate strategies achieve a high number of wins but they do not
achieve a high score which corresponds to the fitness landscape in an
evolutionary sense. From the large number of interactions a payoff matrix \(S\)
can be measured where \(S_{ij}\) denotes the score (using standard values of
\((R, S, T, P) = (3, 0, 5, 1)\)) of the \(i\)th strategy
against the \(j\)th strategy. Using this, the replicator equation
describes the evolution of the system based on a population density fitness
function:

\begin{equation}\label{eqn:replicator_dynamics}
    \frac{dx}{dt} = x(S-x^TS x)
\end{equation}

Equation (\ref{eqn:replicator_dynamics}) is solved numerically through an
integration technique described in~\cite{Petzold1983} and
Figure~\ref{fig:replicator_dynamics} shows the evolution of the distribution of
the system: the various strategies are ranked by scores. It is clear to see that
only the high ranking strategies survive the evolutionary process (in fact,
only \input{./assets/img/replicator_dynamics/main.tex}
have a final distribution greater than \(10 ^ {-2}\)). This confirms the
findings of~\cite{Moran1707} in which sophisticated strategies resist
evolutionary invasion of shorter memory strategies. Recalling
Figure~\ref{fig:SSError_and_probabilities_in_full} this demonstrates that:

\begin{itemize}
    \item Cooperation emerges through the evolutionary process: the high scoring
        strategies do not exhibit extortionate behaviour towards each other.
    \item Extortionate strategies do not survive the evolutionary process.
\end{itemize}

\begin{figure}[!htbp]
    \centering
    \includegraphics[width=.8\textwidth]{./assets/img/replicator_dynamics/main.pdf}
    \caption{Numerical simulation of the replicator equation
    (\ref{eqn:replicator_dynamics}): strategies are ordered by score, only the strategies with a high score survive the evolutionary process.}
    \label{fig:replicator_dynamics}
\end{figure}

This work can be used to classify plays of the IPD\@: data can be collected from
actual interactions (in lab or in the field). Furthermore, this allows for a
classification method similar to the notion of fingerprinting presented
in~\cite{Ashlock2008}. Trained strategies can potentially be classified as
extortionate or not or it could be possible to even constrain the reinforcement
learning approaches that are becoming prevalent in the literature.
Alternatively, this mathematical approach for recognising extortion could be
used in sophisticated strategies to defend against invasion. Arguably, some of
the strategies considered here exhibit this behaviour, indeed as described
in~\cite{Harper2017}, the top ranking strategies in the full tournament are
obtained using evolutionary reinforcement learning techniques, thus, suspicion
of extortionate behaviour could in fact be an evolutionary trait.

\section*{Acknowledgements}

The following open source software libraries were used in this research:

\begin{itemize}
    \item The Axelrod ~\cite{Knight2016, Knight2018} library (IPD strategies and
        tournaments).
    \item The sympy library~\cite{Meurer2017} (verification of all symbolic
        calculations).
    \item The matplotlib~\cite{Droettboom2018} library (visualisation).
    \item The pandas~\cite{Structures2010}, dask~\cite{Dask2016} and
        NumPy~\cite{Oliphant2015} libraries (data manipulation).
    \item The SciPy~\cite{Jones2001} library (numerical integration of the
        replicator equation).
\end{itemize}

This work was performed using the computational facilities of the Advanced
Research Computing @ Cardiff (ARCCA) Division, Cardiff University.

\printbibliography

\newpage
\section*{Supplementary materials}

\includepdf{assets/pdf/proof_of_form_of_extortionate_strategies/main.pdf}

\newpage

Using the pair wise interactions the transition rates \(p,
q\) can be measured and the steady state probabilities inferred and compared to
the actual probabilities of each state.
This is done numerically by computing the singular eigenvector of the
matrix \(A\) \cite{Stewart2009}:

\[
    A =
    \begin{bmatrix}
        p_1 q_1 & p_1 (1 - q_1) & (1 - p_1) q_1 & (1 -p_1) (1 - q_1) \\
        p_2 q_2 & p_2 (1 - q_2) & (1 - p_2) q_2 & (1 -p_2) (1 - q_2) \\
        p_3 q_3 & p_3 (1 - q_3) & (1 - p_3) q_3 & (1 -p_3) (1 - q_3) \\
        p_4 q_4 & p_4 (1 - q_4) & (1 - p_4) q_4 & (1 -p_4) (1 - q_4) \\
    \end{bmatrix}
\]

Figure~\ref{fig:computed_probabilities_vs_theoretic_probabilities} shows a
regression line fitted to every pairwise interaction with a reported
\(\text{SSError}\) value (pairwise interactions with missing states were
omitted). This serves to validate the approach: a part from some edge cases the
relationship is consistent.

\begin{figure}[!htbp]
    \centering
    \includegraphics[width=.8\textwidth]{./assets/img/computed_probabilities_vs_theoretic_probabilities/main.pdf}
    \caption{The
        relationship between the steady state probabilities inferred from the
        measured transitions and the actual steady state probabilities. A linear
        regression line is included validating the approach.}
    \label{fig:computed_probabilities_vs_theoretic_probabilities}
\end{figure}


\end{document}
 times. All of this
interaction data is available at~\cite{vincent_knight_2018_1297075}. A good
match between the inferred Markov chain and the state distribution of the actual
interactions has been verified. Data for this is presented in the supplementary
materials.

Figure~\ref{fig:SSError_overall_in_stewart_plotkin} shows the \(\text{SSError}\)
values for all the strategies in the tournament, as reported
in~\cite{Stewart2012} the extortionate strategy (which has an expected
\(\text{SSError}\) approximately 0) gains a large number of wins.

\begin{figure}[!htbp]
    \centering
    \includegraphics[width=.8\textwidth]{./assets/img/SSError_overall_in_stewart_plotkin/main.pdf}
    \caption{\(\text{SSError}\) and state probabilities for the strategies
        of~\cite{Stewart2012}, ordered both by number of wins and overall score.
        Note that \(P(DC)\) is not shown as it corresponds to the transpose of
        \(P(CD)\). Cooperator and Defector are omitted as they do not visit all
        the states.}
    \label{fig:SSError_overall_in_stewart_plotkin}
\end{figure}

Here, the work of~\cite{Stewart2012} is extended by investigating a tournament
with \documentclass[a4paper]{article}

\usepackage{amsmath}
\usepackage{amssymb}
\usepackage[margin=1.5cm,
            includefoot,
            footskip=30pt]{geometry}
\usepackage{layout}
\usepackage{graphicx}
\usepackage{subcaption}

\usepackage{biblatex}
\usepackage{pdfpages}

\bibliography{main.bib}

\title{Suspicion: Recognising and evaluating the effectiveness
       of extortion in the Iterated Prisoner's Dilemma}
\author{Vincent A. Knight \and Nikoleta E. Glynatsi}
\date{\today}



\begin{document}

\maketitle

\begin{abstract}
    The Iterated Prisoner's Dilemma is a model for rational and evolutionary
    interactive behaviour. It has applications both in the study of human social
    behaviour as well as in biology.
    It is used to understand when and how a rational individual might
    accept an immediate cost to their own utility for the direct benefit of
    another.

    Much attention has been given to a class of strategies called
    Zero Determinant strategies. It has been theoretically shown that these
    strategies can ``extort'' any player.

    In this work, an approach to identify if observed strategies are playing in
    an extortionate way is described. Furthermore, experimental analysis of
    a large tournament with \input{assets/tex/number_of_full_strategies/main.tex}
    strategies is considered. In this setting
    the most highly performing strategies do not play in an extortionate way
    against each other but do against lower performing strategies.
    This suggests that whilst the theory of Zero Determinant strategies
    indicates that memory is not of fundamental importance to the evolution of
    cooperative behaviour, this is incomplete.
\end{abstract}

\section{Introduction}\label{sec:introduction}

Agent based game theoretic models have become a stalwart of the underpinning
mathematics of interactive behaviours. One of the major pieces of work
in this area is the pair of original computer tournaments run by Robert
Axelrod~\cite{Axelrod1980, Axelrod1980a}. These tournaments pitted submitted
computer strategies against each other in plays of the Iterated Prisoner's
Dilemma. A common game where agents can choose to pay a slight cost to their
immediate utility in the hope of building a reputation. This has been used in
economic and evolutionary game theory to understand the evolution of cooperative
behaviour.

Recently, a class of strategies was described in~\cite{Press2012} that can
provably extort any given opponent. In~\cite{Hilbe2013, Moran1707} some
questions have already been asked about the true effectiveness of these
strategies in an evolutionary setting. Here another question is asked: is it
possible to recognise this extortionate behaviour? A mathematical procedure for
suspicion is presented: in the same way that the continued actions of an
extortionate individual might raise suspicion.

This work makes use of the Axelrod Python library~\cite{Knight2018, Knight2016}
with a large number of Prisoner Dilemma strategies available to give an
extensive numerical example of the ideas presented.  The approach is presented
in Section~\ref{sec:delta-zd-strategies}.  All of the code and data discussed
in Section~\ref{sec:numerical-experiments} is open sourced, archived and
written according to best scientific principles~\cite{Wilson2014}. The data
archive can be found at~\cite{vincent_knight_2018_1297075}.

\section{Recognising Extortion}\label{sec:delta-zd-strategies}

In~\cite{Press2012}, given a match between 2 memory-one strategies, the concept
of Zero Determinant (ZD) strategies is introduced. The main result of that paper
shows that given two memory one players \(p, q\in\mathbb{R}^4\) a linear
relationship between the players' scores could be forced by one of the players.

Using the notation of~\cite{Press2012}, assuming the utilities for player \(p\)
are given by \(S_x=(R, S, T, P)\) and for player \(q\) by \(S_y=(R, T, S, P)\)
and that the stationary scores of each player is given by \(S_X\) and \(S_Y\)
respectively. The main result of~\cite{Press2012} is that if

\begin{equation}\label{eqn:linear_relationship_for_p}
    \tilde p=\alpha S_x + \beta S_y + \gamma
\end{equation}

or

\begin{equation}\label{eqn:linear_relationship_for_q}
    \tilde q=\alpha S_x + \beta S_y + \gamma
\end{equation}

where \(\tilde p = (1 - p_1, 1 - p_2, p_3, p_4)\) and
\(\tilde q = (1 - q_1, 1 - q_2, q_3, q_4)\) then:

\begin{equation}
    \alpha S_X + \beta S_Y + \gamma = 0
\end{equation}

In~\cite{Press2012} a particular type of ZD strategy is defined: extortionate
strategies. If:

\begin{equation}\label{eqn:constraint_for_extortion}
    \gamma = - P(\alpha + \beta)
\end{equation}

then the player can ensure they get a score \(\chi\) times
larger than the opponent. This extortion coefficient is given by:

\begin{equation}\label{eqn:definition_of_chi}
    \chi=\frac{-\beta}{\alpha}
\end{equation}

Thus, if (\ref{eqn:constraint_for_extortion}) holds and \(\chi >1\) a player is
said to extort their opponent.
Here, the reverse problem is considered: given a
\(p\in\mathbb{R}^4\) how does one identify \(\alpha, \beta\) if they
exist and is the strategy in fact acting in an extortionate way?

These conditions correspond to:

\begin{align}
    \tilde p_1 & = \alpha R + \beta R - P (\alpha + \beta)
            \label{eqn:condition_for_tilde_p1}\\
    \tilde p_2 & = \alpha S + \beta T - P (\alpha + \beta)
            \label{eqn:condition_for_tilde_p2}\\
    \tilde p_3 & = \alpha T + \beta S - P (\alpha + \beta)
            \label{eqn:condition_for_tilde_p3}\\
    \tilde p_4 & = \alpha P + \beta P - P (\alpha + \beta)
            \label{eqn:condition_for_tilde_p4}
\end{align}

Equation (\ref{eqn:condition_for_tilde_p4}) ensures that \(p_4=\tilde p_4=0\).
Equations (\ref{eqn:condition_for_tilde_p1}-\ref{eqn:condition_for_tilde_p3})
can be used to eliminate \(\alpha, \beta\), giving:

\begin{equation}\label{eqn:planar_definition_of_extortion}
    \tilde p_1 = \frac{(R - P)(\tilde p_2 + \tilde p_3)}{S + T - 2P}
\end{equation}

with:

\begin{equation}\label{eqn:definition_of_chi}
    \chi = \frac{\tilde p_2 (P - T) + \tilde p_3 (S - P)}
                {\tilde p_2 (P - S) + \tilde p_3 (T - P)}
\end{equation}

Given a strategy \(p\in\mathbb{R}^{4\times 1}\) equations
(\ref{eqn:condition_for_tilde_p4}), (\ref{eqn:planar_definition_of_extortion}-\ref{eqn:definition_of_chi}) can be used to check if
a strategy is extortionate. The conditions correspond to:

\begin{align}
    p_1 & = \frac{(R-P)(p_2 + p_3) - R + T + S - P}{S + T - 2P}
     \label{eqn:condition_for_p1}\\
    p_4 & = 0 \label{eqn:condition_for_p4}\\
    1 & > p_2 + p_3\label{eqn:condition_for_chi}
\end{align}

The algebraic steps necessary to prove these results are available in the
supporting materials.

All extortionate strategies reside on a triangular (\ref{eqn:condition_for_chi})
plane (\ref{eqn:condition_for_p1}) in 3 dimensions (\ref{eqn:condition_for_p4}).
Using this formulation it can be seen that a necessary (but not sufficient)
condition for an extortionate strategy is that it cooperates on average less
than 50\% of the time when in a state of disagreement with the opponent.

As an example, consider the known extortionate strategy \(p=(8 / 9, 1 / 2, 1 /
3, 0)\) from~\cite{Stewart2012} which is referred to as \texttt{Extort-2}. In
this case, for the standard values of \((R, T, S, P)\) constraint
(\ref{eqn:condition_for_p1}) corresponds to:

\begin{equation}
    p_1 = \frac{2(p_2 + p_3) + 1}{3}
\end{equation}

It is clear that in this case all constraints hold.

This approach could in fact be used to confirm that a given strategy is acting
in an extortionate manner even if it is not a memory one strategy. However, in
practice, if a closed form for \(p\) is not known, then due to measurement
and/or numerical error this would not work.

This problem can be written in the following linear algebraic form where
\(x=(\alpha, \beta)\)
and \(p^*=(\tilde p_1 - 1, tilde_2 - 1, p_3)\):

\begin{equation}\label{eqn:linear_algebraic_equation_for_p}
    Cx= p^*
\end{equation}

\(C\) corresponds to equations
(\ref{eqn:condition_for_tilde_p1}-\ref{eqn:condition_for_tilde_p3}) and is
given by:

\begin{equation}\label{eqn:definition_of_C}
    C =
    \begin{bmatrix}
        R - P & R- P \\
        S - P & T- P \\
        T - P & S- P \\
    \end{bmatrix}
\end{equation}

Note that in general, equation (\ref{eqn:linear_algebraic_equation_for_p}) will
not necessarily have a solution. From the Rouch\'{e}-Capelli theorem if there is
a solution it is unique as \(\text{rank}(C)=2\) which is the dimension of the
variable \(x\). The best fitting \(x\) is found by minimizing:

\begin{equation}\label{eqn:r_squared}
    \text{SSError} = \|C x- p^*\|_2^2 = \sum_{i=1}^{3}\left((C\bar x)_i-p_i^*\right)^2
\end{equation}

Note that \(\text{SSError}\), which is the square of the Frobenius
norm~\cite{Golub2013}, becomes a measure of how close a strategy is to being an
extortionate strategy. Suspicion
of extortion then corresponds to a threshold on \(\text{SSError}\).

By observing interactions (human or otherwise), their memory one representation
can be inferred and this approach can be used to recognise extortionate
behaviour. The notion of comparing theoretic and actual plays of the IPD is not
novel, see for example~\cite{Rand2013}. Immediately it is noted that if the
environment is noisy~\cite{Wu1995} then no strategy can be considered to be
extortionate as \(p_4>0\).

In the next section, this idea will be illustrated by observing the interactions
that take place in a computer based tournament of the IPD\@.

\section{Numerical experiments}\label{sec:numerical-experiments}

In~\cite{Stewart2012} results from a tournament with
\input{./assets/tex/number_of_stewart_plotkin_strategies/main.tex} strategies,
was presented with specific consideration given to ZD strategies. This
tournament is reproduced here using the Axelrod-Python
project~\cite{Knight2016}. To obtain a good measure of the corresponding
transition rates for each strategy all matches have been run for
\input{assets/tex/number_of_turns/main.tex} turns and every match has been
repeated \input{assets/tex/number_of_repetitions/main.tex} times. All of this
interaction data is available at~\cite{vincent_knight_2018_1297075}. A good
match between the inferred Markov chain and the state distribution of the actual
interactions has been verified. Data for this is presented in the supplementary
materials.

Figure~\ref{fig:SSError_overall_in_stewart_plotkin} shows the \(\text{SSError}\)
values for all the strategies in the tournament, as reported
in~\cite{Stewart2012} the extortionate strategy (which has an expected
\(\text{SSError}\) approximately 0) gains a large number of wins.

\begin{figure}[!htbp]
    \centering
    \includegraphics[width=.8\textwidth]{./assets/img/SSError_overall_in_stewart_plotkin/main.pdf}
    \caption{\(\text{SSError}\) and state probabilities for the strategies
        of~\cite{Stewart2012}, ordered both by number of wins and overall score.
        Note that \(P(DC)\) is not shown as it corresponds to the transpose of
        \(P(CD)\). Cooperator and Defector are omitted as they do not visit all
        the states.}
    \label{fig:SSError_overall_in_stewart_plotkin}
\end{figure}

Here, the work of~\cite{Stewart2012} is extended by investigating a tournament
with \input{assets/tex/number_of_full_strategies/main.tex}
strategies.

The results of this analysis are shown in
Figure~\ref{fig:SSError_and_probabilities_in_full}. The top ranking strategies
by number of wins seem to be extortionate (but not against all strategies) and
it can be seen that a small sub group of strategies achieve mutual defection.
All the top ranking strategies according to score achieve mutual cooperation and
do not extort each other, however they
\textbf{do} exhibit extortionate behaviour towards a number of the lower ranking
strategies.

\begin{figure}[!htbp]
    \centering
    \includegraphics[width=.8\textwidth]{./assets/img/SSError_and_probabilities_in_full/main.pdf}
    \caption{\(\text{SSError}\) for the strategies for the full tournament. Only
    strategy interactions for which \(p_4=0\) and \(\chi>1\) are displayed.}
    \label{fig:SSError_and_probabilities_in_full}
\end{figure}

\section{Conclusion}\label{sec:conclusion}

This work defines an approach to measure whether or not a player is playing a
strategy that corresponds to an extortionate strategy as defined
in~\cite{Press2012}: a mathematical model for suspicion. Indeed, all
extortionate strategies have been
 classified as lying on a triangular plane.
This rigorous classification fails to be robust to small measurement error, thus
a statistical approach is proposed.
This is done through a linear algebraic approach for approximating the solution
of a linear system. Using this, a large number of pairwise interactions is
simulated and in fact very few strategies are found to act extortionately.

The work of~\cite{Press2012}, whilst showing that a clever approach to taking
advantage of another memory one strategy exists: this is incomplete. Whilst the
elegance of this result is very attractive, just as the simplicity of the
victory of Tit For Tat in Axelrod's original tournaments was, it is incomplete.
Extortionate strategies achieve a high number of wins but they do not
achieve a high score which corresponds to the fitness landscape in an
evolutionary sense. From the large number of interactions a payoff matrix \(S\)
can be measured where \(S_{ij}\) denotes the score (using standard values of
\((R, S, T, P) = (3, 0, 5, 1)\)) of the \(i\)th strategy
against the \(j\)th strategy. Using this, the replicator equation
describes the evolution of the system based on a population density fitness
function:

\begin{equation}\label{eqn:replicator_dynamics}
    \frac{dx}{dt} = x(S-x^TS x)
\end{equation}

Equation (\ref{eqn:replicator_dynamics}) is solved numerically through an
integration technique described in~\cite{Petzold1983} and
Figure~\ref{fig:replicator_dynamics} shows the evolution of the distribution of
the system: the various strategies are ranked by scores. It is clear to see that
only the high ranking strategies survive the evolutionary process (in fact,
only \input{./assets/img/replicator_dynamics/main.tex}
have a final distribution greater than \(10 ^ {-2}\)). This confirms the
findings of~\cite{Moran1707} in which sophisticated strategies resist
evolutionary invasion of shorter memory strategies. Recalling
Figure~\ref{fig:SSError_and_probabilities_in_full} this demonstrates that:

\begin{itemize}
    \item Cooperation emerges through the evolutionary process: the high scoring
        strategies do not exhibit extortionate behaviour towards each other.
    \item Extortionate strategies do not survive the evolutionary process.
\end{itemize}

\begin{figure}[!htbp]
    \centering
    \includegraphics[width=.8\textwidth]{./assets/img/replicator_dynamics/main.pdf}
    \caption{Numerical simulation of the replicator equation
    (\ref{eqn:replicator_dynamics}): strategies are ordered by score, only the strategies with a high score survive the evolutionary process.}
    \label{fig:replicator_dynamics}
\end{figure}

This work can be used to classify plays of the IPD\@: data can be collected from
actual interactions (in lab or in the field). Furthermore, this allows for a
classification method similar to the notion of fingerprinting presented
in~\cite{Ashlock2008}. Trained strategies can potentially be classified as
extortionate or not or it could be possible to even constrain the reinforcement
learning approaches that are becoming prevalent in the literature.
Alternatively, this mathematical approach for recognising extortion could be
used in sophisticated strategies to defend against invasion. Arguably, some of
the strategies considered here exhibit this behaviour, indeed as described
in~\cite{Harper2017}, the top ranking strategies in the full tournament are
obtained using evolutionary reinforcement learning techniques, thus, suspicion
of extortionate behaviour could in fact be an evolutionary trait.

\section*{Acknowledgements}

The following open source software libraries were used in this research:

\begin{itemize}
    \item The Axelrod ~\cite{Knight2016, Knight2018} library (IPD strategies and
        tournaments).
    \item The sympy library~\cite{Meurer2017} (verification of all symbolic
        calculations).
    \item The matplotlib~\cite{Droettboom2018} library (visualisation).
    \item The pandas~\cite{Structures2010}, dask~\cite{Dask2016} and
        NumPy~\cite{Oliphant2015} libraries (data manipulation).
    \item The SciPy~\cite{Jones2001} library (numerical integration of the
        replicator equation).
\end{itemize}

This work was performed using the computational facilities of the Advanced
Research Computing @ Cardiff (ARCCA) Division, Cardiff University.

\printbibliography

\newpage
\section*{Supplementary materials}

\includepdf{assets/pdf/proof_of_form_of_extortionate_strategies/main.pdf}

\newpage

Using the pair wise interactions the transition rates \(p,
q\) can be measured and the steady state probabilities inferred and compared to
the actual probabilities of each state.
This is done numerically by computing the singular eigenvector of the
matrix \(A\) \cite{Stewart2009}:

\[
    A =
    \begin{bmatrix}
        p_1 q_1 & p_1 (1 - q_1) & (1 - p_1) q_1 & (1 -p_1) (1 - q_1) \\
        p_2 q_2 & p_2 (1 - q_2) & (1 - p_2) q_2 & (1 -p_2) (1 - q_2) \\
        p_3 q_3 & p_3 (1 - q_3) & (1 - p_3) q_3 & (1 -p_3) (1 - q_3) \\
        p_4 q_4 & p_4 (1 - q_4) & (1 - p_4) q_4 & (1 -p_4) (1 - q_4) \\
    \end{bmatrix}
\]

Figure~\ref{fig:computed_probabilities_vs_theoretic_probabilities} shows a
regression line fitted to every pairwise interaction with a reported
\(\text{SSError}\) value (pairwise interactions with missing states were
omitted). This serves to validate the approach: a part from some edge cases the
relationship is consistent.

\begin{figure}[!htbp]
    \centering
    \includegraphics[width=.8\textwidth]{./assets/img/computed_probabilities_vs_theoretic_probabilities/main.pdf}
    \caption{The
        relationship between the steady state probabilities inferred from the
        measured transitions and the actual steady state probabilities. A linear
        regression line is included validating the approach.}
    \label{fig:computed_probabilities_vs_theoretic_probabilities}
\end{figure}


\end{document}

strategies.

The results of this analysis are shown in
Figure~\ref{fig:SSError_and_probabilities_in_full}. The top ranking strategies
by number of wins seem to be extortionate (but not against all strategies) and
it can be seen that a small sub group of strategies achieve mutual defection.
All the top ranking strategies according to score achieve mutual cooperation and
do not extort each other, however they
\textbf{do} exhibit extortionate behaviour towards a number of the lower ranking
strategies.

\begin{figure}[!htbp]
    \centering
    \includegraphics[width=.8\textwidth]{./assets/img/SSError_and_probabilities_in_full/main.pdf}
    \caption{\(\text{SSError}\) for the strategies for the full tournament. Only
    strategy interactions for which \(p_4=0\) and \(\chi>1\) are displayed.}
    \label{fig:SSError_and_probabilities_in_full}
\end{figure}

\section{Conclusion}\label{sec:conclusion}

This work defines an approach to measure whether or not a player is playing a
strategy that corresponds to an extortionate strategy as defined
in~\cite{Press2012}: a mathematical model for suspicion. Indeed, all
extortionate strategies have been
 classified as lying on a triangular plane.
This rigorous classification fails to be robust to small measurement error, thus
a statistical approach is proposed.
This is done through a linear algebraic approach for approximating the solution
of a linear system. Using this, a large number of pairwise interactions is
simulated and in fact very few strategies are found to act extortionately.

The work of~\cite{Press2012}, whilst showing that a clever approach to taking
advantage of another memory one strategy exists: this is incomplete. Whilst the
elegance of this result is very attractive, just as the simplicity of the
victory of Tit For Tat in Axelrod's original tournaments was, it is incomplete.
Extortionate strategies achieve a high number of wins but they do not
achieve a high score which corresponds to the fitness landscape in an
evolutionary sense. From the large number of interactions a payoff matrix \(S\)
can be measured where \(S_{ij}\) denotes the score (using standard values of
\((R, S, T, P) = (3, 0, 5, 1)\)) of the \(i\)th strategy
against the \(j\)th strategy. Using this, the replicator equation
describes the evolution of the system based on a population density fitness
function:

\begin{equation}\label{eqn:replicator_dynamics}
    \frac{dx}{dt} = x(S-x^TS x)
\end{equation}

Equation (\ref{eqn:replicator_dynamics}) is solved numerically through an
integration technique described in~\cite{Petzold1983} and
Figure~\ref{fig:replicator_dynamics} shows the evolution of the distribution of
the system: the various strategies are ranked by scores. It is clear to see that
only the high ranking strategies survive the evolutionary process (in fact,
only \documentclass[a4paper]{article}

\usepackage{amsmath}
\usepackage{amssymb}
\usepackage[margin=1.5cm,
            includefoot,
            footskip=30pt]{geometry}
\usepackage{layout}
\usepackage{graphicx}
\usepackage{subcaption}

\usepackage{biblatex}
\usepackage{pdfpages}

\bibliography{main.bib}

\title{Suspicion: Recognising and evaluating the effectiveness
       of extortion in the Iterated Prisoner's Dilemma}
\author{Vincent A. Knight \and Nikoleta E. Glynatsi}
\date{\today}



\begin{document}

\maketitle

\begin{abstract}
    The Iterated Prisoner's Dilemma is a model for rational and evolutionary
    interactive behaviour. It has applications both in the study of human social
    behaviour as well as in biology.
    It is used to understand when and how a rational individual might
    accept an immediate cost to their own utility for the direct benefit of
    another.

    Much attention has been given to a class of strategies called
    Zero Determinant strategies. It has been theoretically shown that these
    strategies can ``extort'' any player.

    In this work, an approach to identify if observed strategies are playing in
    an extortionate way is described. Furthermore, experimental analysis of
    a large tournament with \input{assets/tex/number_of_full_strategies/main.tex}
    strategies is considered. In this setting
    the most highly performing strategies do not play in an extortionate way
    against each other but do against lower performing strategies.
    This suggests that whilst the theory of Zero Determinant strategies
    indicates that memory is not of fundamental importance to the evolution of
    cooperative behaviour, this is incomplete.
\end{abstract}

\section{Introduction}\label{sec:introduction}

Agent based game theoretic models have become a stalwart of the underpinning
mathematics of interactive behaviours. One of the major pieces of work
in this area is the pair of original computer tournaments run by Robert
Axelrod~\cite{Axelrod1980, Axelrod1980a}. These tournaments pitted submitted
computer strategies against each other in plays of the Iterated Prisoner's
Dilemma. A common game where agents can choose to pay a slight cost to their
immediate utility in the hope of building a reputation. This has been used in
economic and evolutionary game theory to understand the evolution of cooperative
behaviour.

Recently, a class of strategies was described in~\cite{Press2012} that can
provably extort any given opponent. In~\cite{Hilbe2013, Moran1707} some
questions have already been asked about the true effectiveness of these
strategies in an evolutionary setting. Here another question is asked: is it
possible to recognise this extortionate behaviour? A mathematical procedure for
suspicion is presented: in the same way that the continued actions of an
extortionate individual might raise suspicion.

This work makes use of the Axelrod Python library~\cite{Knight2018, Knight2016}
with a large number of Prisoner Dilemma strategies available to give an
extensive numerical example of the ideas presented.  The approach is presented
in Section~\ref{sec:delta-zd-strategies}.  All of the code and data discussed
in Section~\ref{sec:numerical-experiments} is open sourced, archived and
written according to best scientific principles~\cite{Wilson2014}. The data
archive can be found at~\cite{vincent_knight_2018_1297075}.

\section{Recognising Extortion}\label{sec:delta-zd-strategies}

In~\cite{Press2012}, given a match between 2 memory-one strategies, the concept
of Zero Determinant (ZD) strategies is introduced. The main result of that paper
shows that given two memory one players \(p, q\in\mathbb{R}^4\) a linear
relationship between the players' scores could be forced by one of the players.

Using the notation of~\cite{Press2012}, assuming the utilities for player \(p\)
are given by \(S_x=(R, S, T, P)\) and for player \(q\) by \(S_y=(R, T, S, P)\)
and that the stationary scores of each player is given by \(S_X\) and \(S_Y\)
respectively. The main result of~\cite{Press2012} is that if

\begin{equation}\label{eqn:linear_relationship_for_p}
    \tilde p=\alpha S_x + \beta S_y + \gamma
\end{equation}

or

\begin{equation}\label{eqn:linear_relationship_for_q}
    \tilde q=\alpha S_x + \beta S_y + \gamma
\end{equation}

where \(\tilde p = (1 - p_1, 1 - p_2, p_3, p_4)\) and
\(\tilde q = (1 - q_1, 1 - q_2, q_3, q_4)\) then:

\begin{equation}
    \alpha S_X + \beta S_Y + \gamma = 0
\end{equation}

In~\cite{Press2012} a particular type of ZD strategy is defined: extortionate
strategies. If:

\begin{equation}\label{eqn:constraint_for_extortion}
    \gamma = - P(\alpha + \beta)
\end{equation}

then the player can ensure they get a score \(\chi\) times
larger than the opponent. This extortion coefficient is given by:

\begin{equation}\label{eqn:definition_of_chi}
    \chi=\frac{-\beta}{\alpha}
\end{equation}

Thus, if (\ref{eqn:constraint_for_extortion}) holds and \(\chi >1\) a player is
said to extort their opponent.
Here, the reverse problem is considered: given a
\(p\in\mathbb{R}^4\) how does one identify \(\alpha, \beta\) if they
exist and is the strategy in fact acting in an extortionate way?

These conditions correspond to:

\begin{align}
    \tilde p_1 & = \alpha R + \beta R - P (\alpha + \beta)
            \label{eqn:condition_for_tilde_p1}\\
    \tilde p_2 & = \alpha S + \beta T - P (\alpha + \beta)
            \label{eqn:condition_for_tilde_p2}\\
    \tilde p_3 & = \alpha T + \beta S - P (\alpha + \beta)
            \label{eqn:condition_for_tilde_p3}\\
    \tilde p_4 & = \alpha P + \beta P - P (\alpha + \beta)
            \label{eqn:condition_for_tilde_p4}
\end{align}

Equation (\ref{eqn:condition_for_tilde_p4}) ensures that \(p_4=\tilde p_4=0\).
Equations (\ref{eqn:condition_for_tilde_p1}-\ref{eqn:condition_for_tilde_p3})
can be used to eliminate \(\alpha, \beta\), giving:

\begin{equation}\label{eqn:planar_definition_of_extortion}
    \tilde p_1 = \frac{(R - P)(\tilde p_2 + \tilde p_3)}{S + T - 2P}
\end{equation}

with:

\begin{equation}\label{eqn:definition_of_chi}
    \chi = \frac{\tilde p_2 (P - T) + \tilde p_3 (S - P)}
                {\tilde p_2 (P - S) + \tilde p_3 (T - P)}
\end{equation}

Given a strategy \(p\in\mathbb{R}^{4\times 1}\) equations
(\ref{eqn:condition_for_tilde_p4}), (\ref{eqn:planar_definition_of_extortion}-\ref{eqn:definition_of_chi}) can be used to check if
a strategy is extortionate. The conditions correspond to:

\begin{align}
    p_1 & = \frac{(R-P)(p_2 + p_3) - R + T + S - P}{S + T - 2P}
     \label{eqn:condition_for_p1}\\
    p_4 & = 0 \label{eqn:condition_for_p4}\\
    1 & > p_2 + p_3\label{eqn:condition_for_chi}
\end{align}

The algebraic steps necessary to prove these results are available in the
supporting materials.

All extortionate strategies reside on a triangular (\ref{eqn:condition_for_chi})
plane (\ref{eqn:condition_for_p1}) in 3 dimensions (\ref{eqn:condition_for_p4}).
Using this formulation it can be seen that a necessary (but not sufficient)
condition for an extortionate strategy is that it cooperates on average less
than 50\% of the time when in a state of disagreement with the opponent.

As an example, consider the known extortionate strategy \(p=(8 / 9, 1 / 2, 1 /
3, 0)\) from~\cite{Stewart2012} which is referred to as \texttt{Extort-2}. In
this case, for the standard values of \((R, T, S, P)\) constraint
(\ref{eqn:condition_for_p1}) corresponds to:

\begin{equation}
    p_1 = \frac{2(p_2 + p_3) + 1}{3}
\end{equation}

It is clear that in this case all constraints hold.

This approach could in fact be used to confirm that a given strategy is acting
in an extortionate manner even if it is not a memory one strategy. However, in
practice, if a closed form for \(p\) is not known, then due to measurement
and/or numerical error this would not work.

This problem can be written in the following linear algebraic form where
\(x=(\alpha, \beta)\)
and \(p^*=(\tilde p_1 - 1, tilde_2 - 1, p_3)\):

\begin{equation}\label{eqn:linear_algebraic_equation_for_p}
    Cx= p^*
\end{equation}

\(C\) corresponds to equations
(\ref{eqn:condition_for_tilde_p1}-\ref{eqn:condition_for_tilde_p3}) and is
given by:

\begin{equation}\label{eqn:definition_of_C}
    C =
    \begin{bmatrix}
        R - P & R- P \\
        S - P & T- P \\
        T - P & S- P \\
    \end{bmatrix}
\end{equation}

Note that in general, equation (\ref{eqn:linear_algebraic_equation_for_p}) will
not necessarily have a solution. From the Rouch\'{e}-Capelli theorem if there is
a solution it is unique as \(\text{rank}(C)=2\) which is the dimension of the
variable \(x\). The best fitting \(x\) is found by minimizing:

\begin{equation}\label{eqn:r_squared}
    \text{SSError} = \|C x- p^*\|_2^2 = \sum_{i=1}^{3}\left((C\bar x)_i-p_i^*\right)^2
\end{equation}

Note that \(\text{SSError}\), which is the square of the Frobenius
norm~\cite{Golub2013}, becomes a measure of how close a strategy is to being an
extortionate strategy. Suspicion
of extortion then corresponds to a threshold on \(\text{SSError}\).

By observing interactions (human or otherwise), their memory one representation
can be inferred and this approach can be used to recognise extortionate
behaviour. The notion of comparing theoretic and actual plays of the IPD is not
novel, see for example~\cite{Rand2013}. Immediately it is noted that if the
environment is noisy~\cite{Wu1995} then no strategy can be considered to be
extortionate as \(p_4>0\).

In the next section, this idea will be illustrated by observing the interactions
that take place in a computer based tournament of the IPD\@.

\section{Numerical experiments}\label{sec:numerical-experiments}

In~\cite{Stewart2012} results from a tournament with
\input{./assets/tex/number_of_stewart_plotkin_strategies/main.tex} strategies,
was presented with specific consideration given to ZD strategies. This
tournament is reproduced here using the Axelrod-Python
project~\cite{Knight2016}. To obtain a good measure of the corresponding
transition rates for each strategy all matches have been run for
\input{assets/tex/number_of_turns/main.tex} turns and every match has been
repeated \input{assets/tex/number_of_repetitions/main.tex} times. All of this
interaction data is available at~\cite{vincent_knight_2018_1297075}. A good
match between the inferred Markov chain and the state distribution of the actual
interactions has been verified. Data for this is presented in the supplementary
materials.

Figure~\ref{fig:SSError_overall_in_stewart_plotkin} shows the \(\text{SSError}\)
values for all the strategies in the tournament, as reported
in~\cite{Stewart2012} the extortionate strategy (which has an expected
\(\text{SSError}\) approximately 0) gains a large number of wins.

\begin{figure}[!htbp]
    \centering
    \includegraphics[width=.8\textwidth]{./assets/img/SSError_overall_in_stewart_plotkin/main.pdf}
    \caption{\(\text{SSError}\) and state probabilities for the strategies
        of~\cite{Stewart2012}, ordered both by number of wins and overall score.
        Note that \(P(DC)\) is not shown as it corresponds to the transpose of
        \(P(CD)\). Cooperator and Defector are omitted as they do not visit all
        the states.}
    \label{fig:SSError_overall_in_stewart_plotkin}
\end{figure}

Here, the work of~\cite{Stewart2012} is extended by investigating a tournament
with \input{assets/tex/number_of_full_strategies/main.tex}
strategies.

The results of this analysis are shown in
Figure~\ref{fig:SSError_and_probabilities_in_full}. The top ranking strategies
by number of wins seem to be extortionate (but not against all strategies) and
it can be seen that a small sub group of strategies achieve mutual defection.
All the top ranking strategies according to score achieve mutual cooperation and
do not extort each other, however they
\textbf{do} exhibit extortionate behaviour towards a number of the lower ranking
strategies.

\begin{figure}[!htbp]
    \centering
    \includegraphics[width=.8\textwidth]{./assets/img/SSError_and_probabilities_in_full/main.pdf}
    \caption{\(\text{SSError}\) for the strategies for the full tournament. Only
    strategy interactions for which \(p_4=0\) and \(\chi>1\) are displayed.}
    \label{fig:SSError_and_probabilities_in_full}
\end{figure}

\section{Conclusion}\label{sec:conclusion}

This work defines an approach to measure whether or not a player is playing a
strategy that corresponds to an extortionate strategy as defined
in~\cite{Press2012}: a mathematical model for suspicion. Indeed, all
extortionate strategies have been
 classified as lying on a triangular plane.
This rigorous classification fails to be robust to small measurement error, thus
a statistical approach is proposed.
This is done through a linear algebraic approach for approximating the solution
of a linear system. Using this, a large number of pairwise interactions is
simulated and in fact very few strategies are found to act extortionately.

The work of~\cite{Press2012}, whilst showing that a clever approach to taking
advantage of another memory one strategy exists: this is incomplete. Whilst the
elegance of this result is very attractive, just as the simplicity of the
victory of Tit For Tat in Axelrod's original tournaments was, it is incomplete.
Extortionate strategies achieve a high number of wins but they do not
achieve a high score which corresponds to the fitness landscape in an
evolutionary sense. From the large number of interactions a payoff matrix \(S\)
can be measured where \(S_{ij}\) denotes the score (using standard values of
\((R, S, T, P) = (3, 0, 5, 1)\)) of the \(i\)th strategy
against the \(j\)th strategy. Using this, the replicator equation
describes the evolution of the system based on a population density fitness
function:

\begin{equation}\label{eqn:replicator_dynamics}
    \frac{dx}{dt} = x(S-x^TS x)
\end{equation}

Equation (\ref{eqn:replicator_dynamics}) is solved numerically through an
integration technique described in~\cite{Petzold1983} and
Figure~\ref{fig:replicator_dynamics} shows the evolution of the distribution of
the system: the various strategies are ranked by scores. It is clear to see that
only the high ranking strategies survive the evolutionary process (in fact,
only \input{./assets/img/replicator_dynamics/main.tex}
have a final distribution greater than \(10 ^ {-2}\)). This confirms the
findings of~\cite{Moran1707} in which sophisticated strategies resist
evolutionary invasion of shorter memory strategies. Recalling
Figure~\ref{fig:SSError_and_probabilities_in_full} this demonstrates that:

\begin{itemize}
    \item Cooperation emerges through the evolutionary process: the high scoring
        strategies do not exhibit extortionate behaviour towards each other.
    \item Extortionate strategies do not survive the evolutionary process.
\end{itemize}

\begin{figure}[!htbp]
    \centering
    \includegraphics[width=.8\textwidth]{./assets/img/replicator_dynamics/main.pdf}
    \caption{Numerical simulation of the replicator equation
    (\ref{eqn:replicator_dynamics}): strategies are ordered by score, only the strategies with a high score survive the evolutionary process.}
    \label{fig:replicator_dynamics}
\end{figure}

This work can be used to classify plays of the IPD\@: data can be collected from
actual interactions (in lab or in the field). Furthermore, this allows for a
classification method similar to the notion of fingerprinting presented
in~\cite{Ashlock2008}. Trained strategies can potentially be classified as
extortionate or not or it could be possible to even constrain the reinforcement
learning approaches that are becoming prevalent in the literature.
Alternatively, this mathematical approach for recognising extortion could be
used in sophisticated strategies to defend against invasion. Arguably, some of
the strategies considered here exhibit this behaviour, indeed as described
in~\cite{Harper2017}, the top ranking strategies in the full tournament are
obtained using evolutionary reinforcement learning techniques, thus, suspicion
of extortionate behaviour could in fact be an evolutionary trait.

\section*{Acknowledgements}

The following open source software libraries were used in this research:

\begin{itemize}
    \item The Axelrod ~\cite{Knight2016, Knight2018} library (IPD strategies and
        tournaments).
    \item The sympy library~\cite{Meurer2017} (verification of all symbolic
        calculations).
    \item The matplotlib~\cite{Droettboom2018} library (visualisation).
    \item The pandas~\cite{Structures2010}, dask~\cite{Dask2016} and
        NumPy~\cite{Oliphant2015} libraries (data manipulation).
    \item The SciPy~\cite{Jones2001} library (numerical integration of the
        replicator equation).
\end{itemize}

This work was performed using the computational facilities of the Advanced
Research Computing @ Cardiff (ARCCA) Division, Cardiff University.

\printbibliography

\newpage
\section*{Supplementary materials}

\includepdf{assets/pdf/proof_of_form_of_extortionate_strategies/main.pdf}

\newpage

Using the pair wise interactions the transition rates \(p,
q\) can be measured and the steady state probabilities inferred and compared to
the actual probabilities of each state.
This is done numerically by computing the singular eigenvector of the
matrix \(A\) \cite{Stewart2009}:

\[
    A =
    \begin{bmatrix}
        p_1 q_1 & p_1 (1 - q_1) & (1 - p_1) q_1 & (1 -p_1) (1 - q_1) \\
        p_2 q_2 & p_2 (1 - q_2) & (1 - p_2) q_2 & (1 -p_2) (1 - q_2) \\
        p_3 q_3 & p_3 (1 - q_3) & (1 - p_3) q_3 & (1 -p_3) (1 - q_3) \\
        p_4 q_4 & p_4 (1 - q_4) & (1 - p_4) q_4 & (1 -p_4) (1 - q_4) \\
    \end{bmatrix}
\]

Figure~\ref{fig:computed_probabilities_vs_theoretic_probabilities} shows a
regression line fitted to every pairwise interaction with a reported
\(\text{SSError}\) value (pairwise interactions with missing states were
omitted). This serves to validate the approach: a part from some edge cases the
relationship is consistent.

\begin{figure}[!htbp]
    \centering
    \includegraphics[width=.8\textwidth]{./assets/img/computed_probabilities_vs_theoretic_probabilities/main.pdf}
    \caption{The
        relationship between the steady state probabilities inferred from the
        measured transitions and the actual steady state probabilities. A linear
        regression line is included validating the approach.}
    \label{fig:computed_probabilities_vs_theoretic_probabilities}
\end{figure}


\end{document}

have a final distribution greater than \(10 ^ {-2}\)). This confirms the
findings of~\cite{Moran1707} in which sophisticated strategies resist
evolutionary invasion of shorter memory strategies. Recalling
Figure~\ref{fig:SSError_and_probabilities_in_full} this demonstrates that:

\begin{itemize}
    \item Cooperation emerges through the evolutionary process: the high scoring
        strategies do not exhibit extortionate behaviour towards each other.
    \item Extortionate strategies do not survive the evolutionary process.
\end{itemize}

\begin{figure}[!htbp]
    \centering
    \includegraphics[width=.8\textwidth]{./assets/img/replicator_dynamics/main.pdf}
    \caption{Numerical simulation of the replicator equation
    (\ref{eqn:replicator_dynamics}): strategies are ordered by score, only the strategies with a high score survive the evolutionary process.}
    \label{fig:replicator_dynamics}
\end{figure}

This work can be used to classify plays of the IPD\@: data can be collected from
actual interactions (in lab or in the field). Furthermore, this allows for a
classification method similar to the notion of fingerprinting presented
in~\cite{Ashlock2008}. Trained strategies can potentially be classified as
extortionate or not or it could be possible to even constrain the reinforcement
learning approaches that are becoming prevalent in the literature.
Alternatively, this mathematical approach for recognising extortion could be
used in sophisticated strategies to defend against invasion. Arguably, some of
the strategies considered here exhibit this behaviour, indeed as described
in~\cite{Harper2017}, the top ranking strategies in the full tournament are
obtained using evolutionary reinforcement learning techniques, thus, suspicion
of extortionate behaviour could in fact be an evolutionary trait.

\section*{Acknowledgements}

The following open source software libraries were used in this research:

\begin{itemize}
    \item The Axelrod ~\cite{Knight2016, Knight2018} library (IPD strategies and
        tournaments).
    \item The sympy library~\cite{Meurer2017} (verification of all symbolic
        calculations).
    \item The matplotlib~\cite{Droettboom2018} library (visualisation).
    \item The pandas~\cite{Structures2010}, dask~\cite{Dask2016} and
        NumPy~\cite{Oliphant2015} libraries (data manipulation).
    \item The SciPy~\cite{Jones2001} library (numerical integration of the
        replicator equation).
\end{itemize}

This work was performed using the computational facilities of the Advanced
Research Computing @ Cardiff (ARCCA) Division, Cardiff University.

\printbibliography

\newpage
\section*{Supplementary materials}

\includepdf{assets/pdf/proof_of_form_of_extortionate_strategies/main.pdf}

\newpage

Using the pair wise interactions the transition rates \(p,
q\) can be measured and the steady state probabilities inferred and compared to
the actual probabilities of each state.
This is done numerically by computing the singular eigenvector of the
matrix \(A\) \cite{Stewart2009}:

\[
    A =
    \begin{bmatrix}
        p_1 q_1 & p_1 (1 - q_1) & (1 - p_1) q_1 & (1 -p_1) (1 - q_1) \\
        p_2 q_2 & p_2 (1 - q_2) & (1 - p_2) q_2 & (1 -p_2) (1 - q_2) \\
        p_3 q_3 & p_3 (1 - q_3) & (1 - p_3) q_3 & (1 -p_3) (1 - q_3) \\
        p_4 q_4 & p_4 (1 - q_4) & (1 - p_4) q_4 & (1 -p_4) (1 - q_4) \\
    \end{bmatrix}
\]

Figure~\ref{fig:computed_probabilities_vs_theoretic_probabilities} shows a
regression line fitted to every pairwise interaction with a reported
\(\text{SSError}\) value (pairwise interactions with missing states were
omitted). This serves to validate the approach: a part from some edge cases the
relationship is consistent.

\begin{figure}[!htbp]
    \centering
    \includegraphics[width=.8\textwidth]{./assets/img/computed_probabilities_vs_theoretic_probabilities/main.pdf}
    \caption{The
        relationship between the steady state probabilities inferred from the
        measured transitions and the actual steady state probabilities. A linear
        regression line is included validating the approach.}
    \label{fig:computed_probabilities_vs_theoretic_probabilities}
\end{figure}


\end{document}
 times. All of this
interaction data is available at~\cite{vincent_knight_2018_1297075}. A good
match between the inferred Markov chain and the state distribution of the actual
interactions has been verified. Data for this is presented in the supplementary
materials.

Figure~\ref{fig:SSError_overall_in_stewart_plotkin} shows the \(\text{SSError}\)
values for all the strategies in the tournament, as reported
in~\cite{Stewart2012} the extortionate strategy (which has an expected
\(\text{SSError}\) approximately 0) gains a large number of wins.

\begin{figure}[!htbp]
    \centering
    \includegraphics[width=.8\textwidth]{./assets/img/SSError_overall_in_stewart_plotkin/main.pdf}
    \caption{\(\text{SSError}\) and state probabilities for the strategies
        of~\cite{Stewart2012}, ordered both by number of wins and overall score.
        Note that \(P(DC)\) is not shown as it corresponds to the transpose of
        \(P(CD)\). Cooperator and Defector are omitted as they do not visit all
        the states.}
    \label{fig:SSError_overall_in_stewart_plotkin}
\end{figure}

Here, the work of~\cite{Stewart2012} is extended by investigating a tournament
with \documentclass[a4paper]{article}

\usepackage{amsmath}
\usepackage{amssymb}
\usepackage[margin=1.5cm,
            includefoot,
            footskip=30pt]{geometry}
\usepackage{layout}
\usepackage{graphicx}
\usepackage{subcaption}

\usepackage{biblatex}
\usepackage{pdfpages}

\bibliography{main.bib}

\title{Suspicion: Recognising and evaluating the effectiveness
       of extortion in the Iterated Prisoner's Dilemma}
\author{Vincent A. Knight \and Nikoleta E. Glynatsi}
\date{\today}



\begin{document}

\maketitle

\begin{abstract}
    The Iterated Prisoner's Dilemma is a model for rational and evolutionary
    interactive behaviour. It has applications both in the study of human social
    behaviour as well as in biology.
    It is used to understand when and how a rational individual might
    accept an immediate cost to their own utility for the direct benefit of
    another.

    Much attention has been given to a class of strategies called
    Zero Determinant strategies. It has been theoretically shown that these
    strategies can ``extort'' any player.

    In this work, an approach to identify if observed strategies are playing in
    an extortionate way is described. Furthermore, experimental analysis of
    a large tournament with \documentclass[a4paper]{article}

\usepackage{amsmath}
\usepackage{amssymb}
\usepackage[margin=1.5cm,
            includefoot,
            footskip=30pt]{geometry}
\usepackage{layout}
\usepackage{graphicx}
\usepackage{subcaption}

\usepackage{biblatex}
\usepackage{pdfpages}

\bibliography{main.bib}

\title{Suspicion: Recognising and evaluating the effectiveness
       of extortion in the Iterated Prisoner's Dilemma}
\author{Vincent A. Knight \and Nikoleta E. Glynatsi}
\date{\today}



\begin{document}

\maketitle

\begin{abstract}
    The Iterated Prisoner's Dilemma is a model for rational and evolutionary
    interactive behaviour. It has applications both in the study of human social
    behaviour as well as in biology.
    It is used to understand when and how a rational individual might
    accept an immediate cost to their own utility for the direct benefit of
    another.

    Much attention has been given to a class of strategies called
    Zero Determinant strategies. It has been theoretically shown that these
    strategies can ``extort'' any player.

    In this work, an approach to identify if observed strategies are playing in
    an extortionate way is described. Furthermore, experimental analysis of
    a large tournament with \input{assets/tex/number_of_full_strategies/main.tex}
    strategies is considered. In this setting
    the most highly performing strategies do not play in an extortionate way
    against each other but do against lower performing strategies.
    This suggests that whilst the theory of Zero Determinant strategies
    indicates that memory is not of fundamental importance to the evolution of
    cooperative behaviour, this is incomplete.
\end{abstract}

\section{Introduction}\label{sec:introduction}

Agent based game theoretic models have become a stalwart of the underpinning
mathematics of interactive behaviours. One of the major pieces of work
in this area is the pair of original computer tournaments run by Robert
Axelrod~\cite{Axelrod1980, Axelrod1980a}. These tournaments pitted submitted
computer strategies against each other in plays of the Iterated Prisoner's
Dilemma. A common game where agents can choose to pay a slight cost to their
immediate utility in the hope of building a reputation. This has been used in
economic and evolutionary game theory to understand the evolution of cooperative
behaviour.

Recently, a class of strategies was described in~\cite{Press2012} that can
provably extort any given opponent. In~\cite{Hilbe2013, Moran1707} some
questions have already been asked about the true effectiveness of these
strategies in an evolutionary setting. Here another question is asked: is it
possible to recognise this extortionate behaviour? A mathematical procedure for
suspicion is presented: in the same way that the continued actions of an
extortionate individual might raise suspicion.

This work makes use of the Axelrod Python library~\cite{Knight2018, Knight2016}
with a large number of Prisoner Dilemma strategies available to give an
extensive numerical example of the ideas presented.  The approach is presented
in Section~\ref{sec:delta-zd-strategies}.  All of the code and data discussed
in Section~\ref{sec:numerical-experiments} is open sourced, archived and
written according to best scientific principles~\cite{Wilson2014}. The data
archive can be found at~\cite{vincent_knight_2018_1297075}.

\section{Recognising Extortion}\label{sec:delta-zd-strategies}

In~\cite{Press2012}, given a match between 2 memory-one strategies, the concept
of Zero Determinant (ZD) strategies is introduced. The main result of that paper
shows that given two memory one players \(p, q\in\mathbb{R}^4\) a linear
relationship between the players' scores could be forced by one of the players.

Using the notation of~\cite{Press2012}, assuming the utilities for player \(p\)
are given by \(S_x=(R, S, T, P)\) and for player \(q\) by \(S_y=(R, T, S, P)\)
and that the stationary scores of each player is given by \(S_X\) and \(S_Y\)
respectively. The main result of~\cite{Press2012} is that if

\begin{equation}\label{eqn:linear_relationship_for_p}
    \tilde p=\alpha S_x + \beta S_y + \gamma
\end{equation}

or

\begin{equation}\label{eqn:linear_relationship_for_q}
    \tilde q=\alpha S_x + \beta S_y + \gamma
\end{equation}

where \(\tilde p = (1 - p_1, 1 - p_2, p_3, p_4)\) and
\(\tilde q = (1 - q_1, 1 - q_2, q_3, q_4)\) then:

\begin{equation}
    \alpha S_X + \beta S_Y + \gamma = 0
\end{equation}

In~\cite{Press2012} a particular type of ZD strategy is defined: extortionate
strategies. If:

\begin{equation}\label{eqn:constraint_for_extortion}
    \gamma = - P(\alpha + \beta)
\end{equation}

then the player can ensure they get a score \(\chi\) times
larger than the opponent. This extortion coefficient is given by:

\begin{equation}\label{eqn:definition_of_chi}
    \chi=\frac{-\beta}{\alpha}
\end{equation}

Thus, if (\ref{eqn:constraint_for_extortion}) holds and \(\chi >1\) a player is
said to extort their opponent.
Here, the reverse problem is considered: given a
\(p\in\mathbb{R}^4\) how does one identify \(\alpha, \beta\) if they
exist and is the strategy in fact acting in an extortionate way?

These conditions correspond to:

\begin{align}
    \tilde p_1 & = \alpha R + \beta R - P (\alpha + \beta)
            \label{eqn:condition_for_tilde_p1}\\
    \tilde p_2 & = \alpha S + \beta T - P (\alpha + \beta)
            \label{eqn:condition_for_tilde_p2}\\
    \tilde p_3 & = \alpha T + \beta S - P (\alpha + \beta)
            \label{eqn:condition_for_tilde_p3}\\
    \tilde p_4 & = \alpha P + \beta P - P (\alpha + \beta)
            \label{eqn:condition_for_tilde_p4}
\end{align}

Equation (\ref{eqn:condition_for_tilde_p4}) ensures that \(p_4=\tilde p_4=0\).
Equations (\ref{eqn:condition_for_tilde_p1}-\ref{eqn:condition_for_tilde_p3})
can be used to eliminate \(\alpha, \beta\), giving:

\begin{equation}\label{eqn:planar_definition_of_extortion}
    \tilde p_1 = \frac{(R - P)(\tilde p_2 + \tilde p_3)}{S + T - 2P}
\end{equation}

with:

\begin{equation}\label{eqn:definition_of_chi}
    \chi = \frac{\tilde p_2 (P - T) + \tilde p_3 (S - P)}
                {\tilde p_2 (P - S) + \tilde p_3 (T - P)}
\end{equation}

Given a strategy \(p\in\mathbb{R}^{4\times 1}\) equations
(\ref{eqn:condition_for_tilde_p4}), (\ref{eqn:planar_definition_of_extortion}-\ref{eqn:definition_of_chi}) can be used to check if
a strategy is extortionate. The conditions correspond to:

\begin{align}
    p_1 & = \frac{(R-P)(p_2 + p_3) - R + T + S - P}{S + T - 2P}
     \label{eqn:condition_for_p1}\\
    p_4 & = 0 \label{eqn:condition_for_p4}\\
    1 & > p_2 + p_3\label{eqn:condition_for_chi}
\end{align}

The algebraic steps necessary to prove these results are available in the
supporting materials.

All extortionate strategies reside on a triangular (\ref{eqn:condition_for_chi})
plane (\ref{eqn:condition_for_p1}) in 3 dimensions (\ref{eqn:condition_for_p4}).
Using this formulation it can be seen that a necessary (but not sufficient)
condition for an extortionate strategy is that it cooperates on average less
than 50\% of the time when in a state of disagreement with the opponent.

As an example, consider the known extortionate strategy \(p=(8 / 9, 1 / 2, 1 /
3, 0)\) from~\cite{Stewart2012} which is referred to as \texttt{Extort-2}. In
this case, for the standard values of \((R, T, S, P)\) constraint
(\ref{eqn:condition_for_p1}) corresponds to:

\begin{equation}
    p_1 = \frac{2(p_2 + p_3) + 1}{3}
\end{equation}

It is clear that in this case all constraints hold.

This approach could in fact be used to confirm that a given strategy is acting
in an extortionate manner even if it is not a memory one strategy. However, in
practice, if a closed form for \(p\) is not known, then due to measurement
and/or numerical error this would not work.

This problem can be written in the following linear algebraic form where
\(x=(\alpha, \beta)\)
and \(p^*=(\tilde p_1 - 1, tilde_2 - 1, p_3)\):

\begin{equation}\label{eqn:linear_algebraic_equation_for_p}
    Cx= p^*
\end{equation}

\(C\) corresponds to equations
(\ref{eqn:condition_for_tilde_p1}-\ref{eqn:condition_for_tilde_p3}) and is
given by:

\begin{equation}\label{eqn:definition_of_C}
    C =
    \begin{bmatrix}
        R - P & R- P \\
        S - P & T- P \\
        T - P & S- P \\
    \end{bmatrix}
\end{equation}

Note that in general, equation (\ref{eqn:linear_algebraic_equation_for_p}) will
not necessarily have a solution. From the Rouch\'{e}-Capelli theorem if there is
a solution it is unique as \(\text{rank}(C)=2\) which is the dimension of the
variable \(x\). The best fitting \(x\) is found by minimizing:

\begin{equation}\label{eqn:r_squared}
    \text{SSError} = \|C x- p^*\|_2^2 = \sum_{i=1}^{3}\left((C\bar x)_i-p_i^*\right)^2
\end{equation}

Note that \(\text{SSError}\), which is the square of the Frobenius
norm~\cite{Golub2013}, becomes a measure of how close a strategy is to being an
extortionate strategy. Suspicion
of extortion then corresponds to a threshold on \(\text{SSError}\).

By observing interactions (human or otherwise), their memory one representation
can be inferred and this approach can be used to recognise extortionate
behaviour. The notion of comparing theoretic and actual plays of the IPD is not
novel, see for example~\cite{Rand2013}. Immediately it is noted that if the
environment is noisy~\cite{Wu1995} then no strategy can be considered to be
extortionate as \(p_4>0\).

In the next section, this idea will be illustrated by observing the interactions
that take place in a computer based tournament of the IPD\@.

\section{Numerical experiments}\label{sec:numerical-experiments}

In~\cite{Stewart2012} results from a tournament with
\input{./assets/tex/number_of_stewart_plotkin_strategies/main.tex} strategies,
was presented with specific consideration given to ZD strategies. This
tournament is reproduced here using the Axelrod-Python
project~\cite{Knight2016}. To obtain a good measure of the corresponding
transition rates for each strategy all matches have been run for
\input{assets/tex/number_of_turns/main.tex} turns and every match has been
repeated \input{assets/tex/number_of_repetitions/main.tex} times. All of this
interaction data is available at~\cite{vincent_knight_2018_1297075}. A good
match between the inferred Markov chain and the state distribution of the actual
interactions has been verified. Data for this is presented in the supplementary
materials.

Figure~\ref{fig:SSError_overall_in_stewart_plotkin} shows the \(\text{SSError}\)
values for all the strategies in the tournament, as reported
in~\cite{Stewart2012} the extortionate strategy (which has an expected
\(\text{SSError}\) approximately 0) gains a large number of wins.

\begin{figure}[!htbp]
    \centering
    \includegraphics[width=.8\textwidth]{./assets/img/SSError_overall_in_stewart_plotkin/main.pdf}
    \caption{\(\text{SSError}\) and state probabilities for the strategies
        of~\cite{Stewart2012}, ordered both by number of wins and overall score.
        Note that \(P(DC)\) is not shown as it corresponds to the transpose of
        \(P(CD)\). Cooperator and Defector are omitted as they do not visit all
        the states.}
    \label{fig:SSError_overall_in_stewart_plotkin}
\end{figure}

Here, the work of~\cite{Stewart2012} is extended by investigating a tournament
with \input{assets/tex/number_of_full_strategies/main.tex}
strategies.

The results of this analysis are shown in
Figure~\ref{fig:SSError_and_probabilities_in_full}. The top ranking strategies
by number of wins seem to be extortionate (but not against all strategies) and
it can be seen that a small sub group of strategies achieve mutual defection.
All the top ranking strategies according to score achieve mutual cooperation and
do not extort each other, however they
\textbf{do} exhibit extortionate behaviour towards a number of the lower ranking
strategies.

\begin{figure}[!htbp]
    \centering
    \includegraphics[width=.8\textwidth]{./assets/img/SSError_and_probabilities_in_full/main.pdf}
    \caption{\(\text{SSError}\) for the strategies for the full tournament. Only
    strategy interactions for which \(p_4=0\) and \(\chi>1\) are displayed.}
    \label{fig:SSError_and_probabilities_in_full}
\end{figure}

\section{Conclusion}\label{sec:conclusion}

This work defines an approach to measure whether or not a player is playing a
strategy that corresponds to an extortionate strategy as defined
in~\cite{Press2012}: a mathematical model for suspicion. Indeed, all
extortionate strategies have been
 classified as lying on a triangular plane.
This rigorous classification fails to be robust to small measurement error, thus
a statistical approach is proposed.
This is done through a linear algebraic approach for approximating the solution
of a linear system. Using this, a large number of pairwise interactions is
simulated and in fact very few strategies are found to act extortionately.

The work of~\cite{Press2012}, whilst showing that a clever approach to taking
advantage of another memory one strategy exists: this is incomplete. Whilst the
elegance of this result is very attractive, just as the simplicity of the
victory of Tit For Tat in Axelrod's original tournaments was, it is incomplete.
Extortionate strategies achieve a high number of wins but they do not
achieve a high score which corresponds to the fitness landscape in an
evolutionary sense. From the large number of interactions a payoff matrix \(S\)
can be measured where \(S_{ij}\) denotes the score (using standard values of
\((R, S, T, P) = (3, 0, 5, 1)\)) of the \(i\)th strategy
against the \(j\)th strategy. Using this, the replicator equation
describes the evolution of the system based on a population density fitness
function:

\begin{equation}\label{eqn:replicator_dynamics}
    \frac{dx}{dt} = x(S-x^TS x)
\end{equation}

Equation (\ref{eqn:replicator_dynamics}) is solved numerically through an
integration technique described in~\cite{Petzold1983} and
Figure~\ref{fig:replicator_dynamics} shows the evolution of the distribution of
the system: the various strategies are ranked by scores. It is clear to see that
only the high ranking strategies survive the evolutionary process (in fact,
only \input{./assets/img/replicator_dynamics/main.tex}
have a final distribution greater than \(10 ^ {-2}\)). This confirms the
findings of~\cite{Moran1707} in which sophisticated strategies resist
evolutionary invasion of shorter memory strategies. Recalling
Figure~\ref{fig:SSError_and_probabilities_in_full} this demonstrates that:

\begin{itemize}
    \item Cooperation emerges through the evolutionary process: the high scoring
        strategies do not exhibit extortionate behaviour towards each other.
    \item Extortionate strategies do not survive the evolutionary process.
\end{itemize}

\begin{figure}[!htbp]
    \centering
    \includegraphics[width=.8\textwidth]{./assets/img/replicator_dynamics/main.pdf}
    \caption{Numerical simulation of the replicator equation
    (\ref{eqn:replicator_dynamics}): strategies are ordered by score, only the strategies with a high score survive the evolutionary process.}
    \label{fig:replicator_dynamics}
\end{figure}

This work can be used to classify plays of the IPD\@: data can be collected from
actual interactions (in lab or in the field). Furthermore, this allows for a
classification method similar to the notion of fingerprinting presented
in~\cite{Ashlock2008}. Trained strategies can potentially be classified as
extortionate or not or it could be possible to even constrain the reinforcement
learning approaches that are becoming prevalent in the literature.
Alternatively, this mathematical approach for recognising extortion could be
used in sophisticated strategies to defend against invasion. Arguably, some of
the strategies considered here exhibit this behaviour, indeed as described
in~\cite{Harper2017}, the top ranking strategies in the full tournament are
obtained using evolutionary reinforcement learning techniques, thus, suspicion
of extortionate behaviour could in fact be an evolutionary trait.

\section*{Acknowledgements}

The following open source software libraries were used in this research:

\begin{itemize}
    \item The Axelrod ~\cite{Knight2016, Knight2018} library (IPD strategies and
        tournaments).
    \item The sympy library~\cite{Meurer2017} (verification of all symbolic
        calculations).
    \item The matplotlib~\cite{Droettboom2018} library (visualisation).
    \item The pandas~\cite{Structures2010}, dask~\cite{Dask2016} and
        NumPy~\cite{Oliphant2015} libraries (data manipulation).
    \item The SciPy~\cite{Jones2001} library (numerical integration of the
        replicator equation).
\end{itemize}

This work was performed using the computational facilities of the Advanced
Research Computing @ Cardiff (ARCCA) Division, Cardiff University.

\printbibliography

\newpage
\section*{Supplementary materials}

\includepdf{assets/pdf/proof_of_form_of_extortionate_strategies/main.pdf}

\newpage

Using the pair wise interactions the transition rates \(p,
q\) can be measured and the steady state probabilities inferred and compared to
the actual probabilities of each state.
This is done numerically by computing the singular eigenvector of the
matrix \(A\) \cite{Stewart2009}:

\[
    A =
    \begin{bmatrix}
        p_1 q_1 & p_1 (1 - q_1) & (1 - p_1) q_1 & (1 -p_1) (1 - q_1) \\
        p_2 q_2 & p_2 (1 - q_2) & (1 - p_2) q_2 & (1 -p_2) (1 - q_2) \\
        p_3 q_3 & p_3 (1 - q_3) & (1 - p_3) q_3 & (1 -p_3) (1 - q_3) \\
        p_4 q_4 & p_4 (1 - q_4) & (1 - p_4) q_4 & (1 -p_4) (1 - q_4) \\
    \end{bmatrix}
\]

Figure~\ref{fig:computed_probabilities_vs_theoretic_probabilities} shows a
regression line fitted to every pairwise interaction with a reported
\(\text{SSError}\) value (pairwise interactions with missing states were
omitted). This serves to validate the approach: a part from some edge cases the
relationship is consistent.

\begin{figure}[!htbp]
    \centering
    \includegraphics[width=.8\textwidth]{./assets/img/computed_probabilities_vs_theoretic_probabilities/main.pdf}
    \caption{The
        relationship between the steady state probabilities inferred from the
        measured transitions and the actual steady state probabilities. A linear
        regression line is included validating the approach.}
    \label{fig:computed_probabilities_vs_theoretic_probabilities}
\end{figure}


\end{document}

    strategies is considered. In this setting
    the most highly performing strategies do not play in an extortionate way
    against each other but do against lower performing strategies.
    This suggests that whilst the theory of Zero Determinant strategies
    indicates that memory is not of fundamental importance to the evolution of
    cooperative behaviour, this is incomplete.
\end{abstract}

\section{Introduction}\label{sec:introduction}

Agent based game theoretic models have become a stalwart of the underpinning
mathematics of interactive behaviours. One of the major pieces of work
in this area is the pair of original computer tournaments run by Robert
Axelrod~\cite{Axelrod1980, Axelrod1980a}. These tournaments pitted submitted
computer strategies against each other in plays of the Iterated Prisoner's
Dilemma. A common game where agents can choose to pay a slight cost to their
immediate utility in the hope of building a reputation. This has been used in
economic and evolutionary game theory to understand the evolution of cooperative
behaviour.

Recently, a class of strategies was described in~\cite{Press2012} that can
provably extort any given opponent. In~\cite{Hilbe2013, Moran1707} some
questions have already been asked about the true effectiveness of these
strategies in an evolutionary setting. Here another question is asked: is it
possible to recognise this extortionate behaviour? A mathematical procedure for
suspicion is presented: in the same way that the continued actions of an
extortionate individual might raise suspicion.

This work makes use of the Axelrod Python library~\cite{Knight2018, Knight2016}
with a large number of Prisoner Dilemma strategies available to give an
extensive numerical example of the ideas presented.  The approach is presented
in Section~\ref{sec:delta-zd-strategies}.  All of the code and data discussed
in Section~\ref{sec:numerical-experiments} is open sourced, archived and
written according to best scientific principles~\cite{Wilson2014}. The data
archive can be found at~\cite{vincent_knight_2018_1297075}.

\section{Recognising Extortion}\label{sec:delta-zd-strategies}

In~\cite{Press2012}, given a match between 2 memory-one strategies, the concept
of Zero Determinant (ZD) strategies is introduced. The main result of that paper
shows that given two memory one players \(p, q\in\mathbb{R}^4\) a linear
relationship between the players' scores could be forced by one of the players.

Using the notation of~\cite{Press2012}, assuming the utilities for player \(p\)
are given by \(S_x=(R, S, T, P)\) and for player \(q\) by \(S_y=(R, T, S, P)\)
and that the stationary scores of each player is given by \(S_X\) and \(S_Y\)
respectively. The main result of~\cite{Press2012} is that if

\begin{equation}\label{eqn:linear_relationship_for_p}
    \tilde p=\alpha S_x + \beta S_y + \gamma
\end{equation}

or

\begin{equation}\label{eqn:linear_relationship_for_q}
    \tilde q=\alpha S_x + \beta S_y + \gamma
\end{equation}

where \(\tilde p = (1 - p_1, 1 - p_2, p_3, p_4)\) and
\(\tilde q = (1 - q_1, 1 - q_2, q_3, q_4)\) then:

\begin{equation}
    \alpha S_X + \beta S_Y + \gamma = 0
\end{equation}

In~\cite{Press2012} a particular type of ZD strategy is defined: extortionate
strategies. If:

\begin{equation}\label{eqn:constraint_for_extortion}
    \gamma = - P(\alpha + \beta)
\end{equation}

then the player can ensure they get a score \(\chi\) times
larger than the opponent. This extortion coefficient is given by:

\begin{equation}\label{eqn:definition_of_chi}
    \chi=\frac{-\beta}{\alpha}
\end{equation}

Thus, if (\ref{eqn:constraint_for_extortion}) holds and \(\chi >1\) a player is
said to extort their opponent.
Here, the reverse problem is considered: given a
\(p\in\mathbb{R}^4\) how does one identify \(\alpha, \beta\) if they
exist and is the strategy in fact acting in an extortionate way?

These conditions correspond to:

\begin{align}
    \tilde p_1 & = \alpha R + \beta R - P (\alpha + \beta)
            \label{eqn:condition_for_tilde_p1}\\
    \tilde p_2 & = \alpha S + \beta T - P (\alpha + \beta)
            \label{eqn:condition_for_tilde_p2}\\
    \tilde p_3 & = \alpha T + \beta S - P (\alpha + \beta)
            \label{eqn:condition_for_tilde_p3}\\
    \tilde p_4 & = \alpha P + \beta P - P (\alpha + \beta)
            \label{eqn:condition_for_tilde_p4}
\end{align}

Equation (\ref{eqn:condition_for_tilde_p4}) ensures that \(p_4=\tilde p_4=0\).
Equations (\ref{eqn:condition_for_tilde_p1}-\ref{eqn:condition_for_tilde_p3})
can be used to eliminate \(\alpha, \beta\), giving:

\begin{equation}\label{eqn:planar_definition_of_extortion}
    \tilde p_1 = \frac{(R - P)(\tilde p_2 + \tilde p_3)}{S + T - 2P}
\end{equation}

with:

\begin{equation}\label{eqn:definition_of_chi}
    \chi = \frac{\tilde p_2 (P - T) + \tilde p_3 (S - P)}
                {\tilde p_2 (P - S) + \tilde p_3 (T - P)}
\end{equation}

Given a strategy \(p\in\mathbb{R}^{4\times 1}\) equations
(\ref{eqn:condition_for_tilde_p4}), (\ref{eqn:planar_definition_of_extortion}-\ref{eqn:definition_of_chi}) can be used to check if
a strategy is extortionate. The conditions correspond to:

\begin{align}
    p_1 & = \frac{(R-P)(p_2 + p_3) - R + T + S - P}{S + T - 2P}
     \label{eqn:condition_for_p1}\\
    p_4 & = 0 \label{eqn:condition_for_p4}\\
    1 & > p_2 + p_3\label{eqn:condition_for_chi}
\end{align}

The algebraic steps necessary to prove these results are available in the
supporting materials.

All extortionate strategies reside on a triangular (\ref{eqn:condition_for_chi})
plane (\ref{eqn:condition_for_p1}) in 3 dimensions (\ref{eqn:condition_for_p4}).
Using this formulation it can be seen that a necessary (but not sufficient)
condition for an extortionate strategy is that it cooperates on average less
than 50\% of the time when in a state of disagreement with the opponent.

As an example, consider the known extortionate strategy \(p=(8 / 9, 1 / 2, 1 /
3, 0)\) from~\cite{Stewart2012} which is referred to as \texttt{Extort-2}. In
this case, for the standard values of \((R, T, S, P)\) constraint
(\ref{eqn:condition_for_p1}) corresponds to:

\begin{equation}
    p_1 = \frac{2(p_2 + p_3) + 1}{3}
\end{equation}

It is clear that in this case all constraints hold.

This approach could in fact be used to confirm that a given strategy is acting
in an extortionate manner even if it is not a memory one strategy. However, in
practice, if a closed form for \(p\) is not known, then due to measurement
and/or numerical error this would not work.

This problem can be written in the following linear algebraic form where
\(x=(\alpha, \beta)\)
and \(p^*=(\tilde p_1 - 1, tilde_2 - 1, p_3)\):

\begin{equation}\label{eqn:linear_algebraic_equation_for_p}
    Cx= p^*
\end{equation}

\(C\) corresponds to equations
(\ref{eqn:condition_for_tilde_p1}-\ref{eqn:condition_for_tilde_p3}) and is
given by:

\begin{equation}\label{eqn:definition_of_C}
    C =
    \begin{bmatrix}
        R - P & R- P \\
        S - P & T- P \\
        T - P & S- P \\
    \end{bmatrix}
\end{equation}

Note that in general, equation (\ref{eqn:linear_algebraic_equation_for_p}) will
not necessarily have a solution. From the Rouch\'{e}-Capelli theorem if there is
a solution it is unique as \(\text{rank}(C)=2\) which is the dimension of the
variable \(x\). The best fitting \(x\) is found by minimizing:

\begin{equation}\label{eqn:r_squared}
    \text{SSError} = \|C x- p^*\|_2^2 = \sum_{i=1}^{3}\left((C\bar x)_i-p_i^*\right)^2
\end{equation}

Note that \(\text{SSError}\), which is the square of the Frobenius
norm~\cite{Golub2013}, becomes a measure of how close a strategy is to being an
extortionate strategy. Suspicion
of extortion then corresponds to a threshold on \(\text{SSError}\).

By observing interactions (human or otherwise), their memory one representation
can be inferred and this approach can be used to recognise extortionate
behaviour. The notion of comparing theoretic and actual plays of the IPD is not
novel, see for example~\cite{Rand2013}. Immediately it is noted that if the
environment is noisy~\cite{Wu1995} then no strategy can be considered to be
extortionate as \(p_4>0\).

In the next section, this idea will be illustrated by observing the interactions
that take place in a computer based tournament of the IPD\@.

\section{Numerical experiments}\label{sec:numerical-experiments}

In~\cite{Stewart2012} results from a tournament with
\documentclass[a4paper]{article}

\usepackage{amsmath}
\usepackage{amssymb}
\usepackage[margin=1.5cm,
            includefoot,
            footskip=30pt]{geometry}
\usepackage{layout}
\usepackage{graphicx}
\usepackage{subcaption}

\usepackage{biblatex}
\usepackage{pdfpages}

\bibliography{main.bib}

\title{Suspicion: Recognising and evaluating the effectiveness
       of extortion in the Iterated Prisoner's Dilemma}
\author{Vincent A. Knight \and Nikoleta E. Glynatsi}
\date{\today}



\begin{document}

\maketitle

\begin{abstract}
    The Iterated Prisoner's Dilemma is a model for rational and evolutionary
    interactive behaviour. It has applications both in the study of human social
    behaviour as well as in biology.
    It is used to understand when and how a rational individual might
    accept an immediate cost to their own utility for the direct benefit of
    another.

    Much attention has been given to a class of strategies called
    Zero Determinant strategies. It has been theoretically shown that these
    strategies can ``extort'' any player.

    In this work, an approach to identify if observed strategies are playing in
    an extortionate way is described. Furthermore, experimental analysis of
    a large tournament with \input{assets/tex/number_of_full_strategies/main.tex}
    strategies is considered. In this setting
    the most highly performing strategies do not play in an extortionate way
    against each other but do against lower performing strategies.
    This suggests that whilst the theory of Zero Determinant strategies
    indicates that memory is not of fundamental importance to the evolution of
    cooperative behaviour, this is incomplete.
\end{abstract}

\section{Introduction}\label{sec:introduction}

Agent based game theoretic models have become a stalwart of the underpinning
mathematics of interactive behaviours. One of the major pieces of work
in this area is the pair of original computer tournaments run by Robert
Axelrod~\cite{Axelrod1980, Axelrod1980a}. These tournaments pitted submitted
computer strategies against each other in plays of the Iterated Prisoner's
Dilemma. A common game where agents can choose to pay a slight cost to their
immediate utility in the hope of building a reputation. This has been used in
economic and evolutionary game theory to understand the evolution of cooperative
behaviour.

Recently, a class of strategies was described in~\cite{Press2012} that can
provably extort any given opponent. In~\cite{Hilbe2013, Moran1707} some
questions have already been asked about the true effectiveness of these
strategies in an evolutionary setting. Here another question is asked: is it
possible to recognise this extortionate behaviour? A mathematical procedure for
suspicion is presented: in the same way that the continued actions of an
extortionate individual might raise suspicion.

This work makes use of the Axelrod Python library~\cite{Knight2018, Knight2016}
with a large number of Prisoner Dilemma strategies available to give an
extensive numerical example of the ideas presented.  The approach is presented
in Section~\ref{sec:delta-zd-strategies}.  All of the code and data discussed
in Section~\ref{sec:numerical-experiments} is open sourced, archived and
written according to best scientific principles~\cite{Wilson2014}. The data
archive can be found at~\cite{vincent_knight_2018_1297075}.

\section{Recognising Extortion}\label{sec:delta-zd-strategies}

In~\cite{Press2012}, given a match between 2 memory-one strategies, the concept
of Zero Determinant (ZD) strategies is introduced. The main result of that paper
shows that given two memory one players \(p, q\in\mathbb{R}^4\) a linear
relationship between the players' scores could be forced by one of the players.

Using the notation of~\cite{Press2012}, assuming the utilities for player \(p\)
are given by \(S_x=(R, S, T, P)\) and for player \(q\) by \(S_y=(R, T, S, P)\)
and that the stationary scores of each player is given by \(S_X\) and \(S_Y\)
respectively. The main result of~\cite{Press2012} is that if

\begin{equation}\label{eqn:linear_relationship_for_p}
    \tilde p=\alpha S_x + \beta S_y + \gamma
\end{equation}

or

\begin{equation}\label{eqn:linear_relationship_for_q}
    \tilde q=\alpha S_x + \beta S_y + \gamma
\end{equation}

where \(\tilde p = (1 - p_1, 1 - p_2, p_3, p_4)\) and
\(\tilde q = (1 - q_1, 1 - q_2, q_3, q_4)\) then:

\begin{equation}
    \alpha S_X + \beta S_Y + \gamma = 0
\end{equation}

In~\cite{Press2012} a particular type of ZD strategy is defined: extortionate
strategies. If:

\begin{equation}\label{eqn:constraint_for_extortion}
    \gamma = - P(\alpha + \beta)
\end{equation}

then the player can ensure they get a score \(\chi\) times
larger than the opponent. This extortion coefficient is given by:

\begin{equation}\label{eqn:definition_of_chi}
    \chi=\frac{-\beta}{\alpha}
\end{equation}

Thus, if (\ref{eqn:constraint_for_extortion}) holds and \(\chi >1\) a player is
said to extort their opponent.
Here, the reverse problem is considered: given a
\(p\in\mathbb{R}^4\) how does one identify \(\alpha, \beta\) if they
exist and is the strategy in fact acting in an extortionate way?

These conditions correspond to:

\begin{align}
    \tilde p_1 & = \alpha R + \beta R - P (\alpha + \beta)
            \label{eqn:condition_for_tilde_p1}\\
    \tilde p_2 & = \alpha S + \beta T - P (\alpha + \beta)
            \label{eqn:condition_for_tilde_p2}\\
    \tilde p_3 & = \alpha T + \beta S - P (\alpha + \beta)
            \label{eqn:condition_for_tilde_p3}\\
    \tilde p_4 & = \alpha P + \beta P - P (\alpha + \beta)
            \label{eqn:condition_for_tilde_p4}
\end{align}

Equation (\ref{eqn:condition_for_tilde_p4}) ensures that \(p_4=\tilde p_4=0\).
Equations (\ref{eqn:condition_for_tilde_p1}-\ref{eqn:condition_for_tilde_p3})
can be used to eliminate \(\alpha, \beta\), giving:

\begin{equation}\label{eqn:planar_definition_of_extortion}
    \tilde p_1 = \frac{(R - P)(\tilde p_2 + \tilde p_3)}{S + T - 2P}
\end{equation}

with:

\begin{equation}\label{eqn:definition_of_chi}
    \chi = \frac{\tilde p_2 (P - T) + \tilde p_3 (S - P)}
                {\tilde p_2 (P - S) + \tilde p_3 (T - P)}
\end{equation}

Given a strategy \(p\in\mathbb{R}^{4\times 1}\) equations
(\ref{eqn:condition_for_tilde_p4}), (\ref{eqn:planar_definition_of_extortion}-\ref{eqn:definition_of_chi}) can be used to check if
a strategy is extortionate. The conditions correspond to:

\begin{align}
    p_1 & = \frac{(R-P)(p_2 + p_3) - R + T + S - P}{S + T - 2P}
     \label{eqn:condition_for_p1}\\
    p_4 & = 0 \label{eqn:condition_for_p4}\\
    1 & > p_2 + p_3\label{eqn:condition_for_chi}
\end{align}

The algebraic steps necessary to prove these results are available in the
supporting materials.

All extortionate strategies reside on a triangular (\ref{eqn:condition_for_chi})
plane (\ref{eqn:condition_for_p1}) in 3 dimensions (\ref{eqn:condition_for_p4}).
Using this formulation it can be seen that a necessary (but not sufficient)
condition for an extortionate strategy is that it cooperates on average less
than 50\% of the time when in a state of disagreement with the opponent.

As an example, consider the known extortionate strategy \(p=(8 / 9, 1 / 2, 1 /
3, 0)\) from~\cite{Stewart2012} which is referred to as \texttt{Extort-2}. In
this case, for the standard values of \((R, T, S, P)\) constraint
(\ref{eqn:condition_for_p1}) corresponds to:

\begin{equation}
    p_1 = \frac{2(p_2 + p_3) + 1}{3}
\end{equation}

It is clear that in this case all constraints hold.

This approach could in fact be used to confirm that a given strategy is acting
in an extortionate manner even if it is not a memory one strategy. However, in
practice, if a closed form for \(p\) is not known, then due to measurement
and/or numerical error this would not work.

This problem can be written in the following linear algebraic form where
\(x=(\alpha, \beta)\)
and \(p^*=(\tilde p_1 - 1, tilde_2 - 1, p_3)\):

\begin{equation}\label{eqn:linear_algebraic_equation_for_p}
    Cx= p^*
\end{equation}

\(C\) corresponds to equations
(\ref{eqn:condition_for_tilde_p1}-\ref{eqn:condition_for_tilde_p3}) and is
given by:

\begin{equation}\label{eqn:definition_of_C}
    C =
    \begin{bmatrix}
        R - P & R- P \\
        S - P & T- P \\
        T - P & S- P \\
    \end{bmatrix}
\end{equation}

Note that in general, equation (\ref{eqn:linear_algebraic_equation_for_p}) will
not necessarily have a solution. From the Rouch\'{e}-Capelli theorem if there is
a solution it is unique as \(\text{rank}(C)=2\) which is the dimension of the
variable \(x\). The best fitting \(x\) is found by minimizing:

\begin{equation}\label{eqn:r_squared}
    \text{SSError} = \|C x- p^*\|_2^2 = \sum_{i=1}^{3}\left((C\bar x)_i-p_i^*\right)^2
\end{equation}

Note that \(\text{SSError}\), which is the square of the Frobenius
norm~\cite{Golub2013}, becomes a measure of how close a strategy is to being an
extortionate strategy. Suspicion
of extortion then corresponds to a threshold on \(\text{SSError}\).

By observing interactions (human or otherwise), their memory one representation
can be inferred and this approach can be used to recognise extortionate
behaviour. The notion of comparing theoretic and actual plays of the IPD is not
novel, see for example~\cite{Rand2013}. Immediately it is noted that if the
environment is noisy~\cite{Wu1995} then no strategy can be considered to be
extortionate as \(p_4>0\).

In the next section, this idea will be illustrated by observing the interactions
that take place in a computer based tournament of the IPD\@.

\section{Numerical experiments}\label{sec:numerical-experiments}

In~\cite{Stewart2012} results from a tournament with
\input{./assets/tex/number_of_stewart_plotkin_strategies/main.tex} strategies,
was presented with specific consideration given to ZD strategies. This
tournament is reproduced here using the Axelrod-Python
project~\cite{Knight2016}. To obtain a good measure of the corresponding
transition rates for each strategy all matches have been run for
\input{assets/tex/number_of_turns/main.tex} turns and every match has been
repeated \input{assets/tex/number_of_repetitions/main.tex} times. All of this
interaction data is available at~\cite{vincent_knight_2018_1297075}. A good
match between the inferred Markov chain and the state distribution of the actual
interactions has been verified. Data for this is presented in the supplementary
materials.

Figure~\ref{fig:SSError_overall_in_stewart_plotkin} shows the \(\text{SSError}\)
values for all the strategies in the tournament, as reported
in~\cite{Stewart2012} the extortionate strategy (which has an expected
\(\text{SSError}\) approximately 0) gains a large number of wins.

\begin{figure}[!htbp]
    \centering
    \includegraphics[width=.8\textwidth]{./assets/img/SSError_overall_in_stewart_plotkin/main.pdf}
    \caption{\(\text{SSError}\) and state probabilities for the strategies
        of~\cite{Stewart2012}, ordered both by number of wins and overall score.
        Note that \(P(DC)\) is not shown as it corresponds to the transpose of
        \(P(CD)\). Cooperator and Defector are omitted as they do not visit all
        the states.}
    \label{fig:SSError_overall_in_stewart_plotkin}
\end{figure}

Here, the work of~\cite{Stewart2012} is extended by investigating a tournament
with \input{assets/tex/number_of_full_strategies/main.tex}
strategies.

The results of this analysis are shown in
Figure~\ref{fig:SSError_and_probabilities_in_full}. The top ranking strategies
by number of wins seem to be extortionate (but not against all strategies) and
it can be seen that a small sub group of strategies achieve mutual defection.
All the top ranking strategies according to score achieve mutual cooperation and
do not extort each other, however they
\textbf{do} exhibit extortionate behaviour towards a number of the lower ranking
strategies.

\begin{figure}[!htbp]
    \centering
    \includegraphics[width=.8\textwidth]{./assets/img/SSError_and_probabilities_in_full/main.pdf}
    \caption{\(\text{SSError}\) for the strategies for the full tournament. Only
    strategy interactions for which \(p_4=0\) and \(\chi>1\) are displayed.}
    \label{fig:SSError_and_probabilities_in_full}
\end{figure}

\section{Conclusion}\label{sec:conclusion}

This work defines an approach to measure whether or not a player is playing a
strategy that corresponds to an extortionate strategy as defined
in~\cite{Press2012}: a mathematical model for suspicion. Indeed, all
extortionate strategies have been
 classified as lying on a triangular plane.
This rigorous classification fails to be robust to small measurement error, thus
a statistical approach is proposed.
This is done through a linear algebraic approach for approximating the solution
of a linear system. Using this, a large number of pairwise interactions is
simulated and in fact very few strategies are found to act extortionately.

The work of~\cite{Press2012}, whilst showing that a clever approach to taking
advantage of another memory one strategy exists: this is incomplete. Whilst the
elegance of this result is very attractive, just as the simplicity of the
victory of Tit For Tat in Axelrod's original tournaments was, it is incomplete.
Extortionate strategies achieve a high number of wins but they do not
achieve a high score which corresponds to the fitness landscape in an
evolutionary sense. From the large number of interactions a payoff matrix \(S\)
can be measured where \(S_{ij}\) denotes the score (using standard values of
\((R, S, T, P) = (3, 0, 5, 1)\)) of the \(i\)th strategy
against the \(j\)th strategy. Using this, the replicator equation
describes the evolution of the system based on a population density fitness
function:

\begin{equation}\label{eqn:replicator_dynamics}
    \frac{dx}{dt} = x(S-x^TS x)
\end{equation}

Equation (\ref{eqn:replicator_dynamics}) is solved numerically through an
integration technique described in~\cite{Petzold1983} and
Figure~\ref{fig:replicator_dynamics} shows the evolution of the distribution of
the system: the various strategies are ranked by scores. It is clear to see that
only the high ranking strategies survive the evolutionary process (in fact,
only \input{./assets/img/replicator_dynamics/main.tex}
have a final distribution greater than \(10 ^ {-2}\)). This confirms the
findings of~\cite{Moran1707} in which sophisticated strategies resist
evolutionary invasion of shorter memory strategies. Recalling
Figure~\ref{fig:SSError_and_probabilities_in_full} this demonstrates that:

\begin{itemize}
    \item Cooperation emerges through the evolutionary process: the high scoring
        strategies do not exhibit extortionate behaviour towards each other.
    \item Extortionate strategies do not survive the evolutionary process.
\end{itemize}

\begin{figure}[!htbp]
    \centering
    \includegraphics[width=.8\textwidth]{./assets/img/replicator_dynamics/main.pdf}
    \caption{Numerical simulation of the replicator equation
    (\ref{eqn:replicator_dynamics}): strategies are ordered by score, only the strategies with a high score survive the evolutionary process.}
    \label{fig:replicator_dynamics}
\end{figure}

This work can be used to classify plays of the IPD\@: data can be collected from
actual interactions (in lab or in the field). Furthermore, this allows for a
classification method similar to the notion of fingerprinting presented
in~\cite{Ashlock2008}. Trained strategies can potentially be classified as
extortionate or not or it could be possible to even constrain the reinforcement
learning approaches that are becoming prevalent in the literature.
Alternatively, this mathematical approach for recognising extortion could be
used in sophisticated strategies to defend against invasion. Arguably, some of
the strategies considered here exhibit this behaviour, indeed as described
in~\cite{Harper2017}, the top ranking strategies in the full tournament are
obtained using evolutionary reinforcement learning techniques, thus, suspicion
of extortionate behaviour could in fact be an evolutionary trait.

\section*{Acknowledgements}

The following open source software libraries were used in this research:

\begin{itemize}
    \item The Axelrod ~\cite{Knight2016, Knight2018} library (IPD strategies and
        tournaments).
    \item The sympy library~\cite{Meurer2017} (verification of all symbolic
        calculations).
    \item The matplotlib~\cite{Droettboom2018} library (visualisation).
    \item The pandas~\cite{Structures2010}, dask~\cite{Dask2016} and
        NumPy~\cite{Oliphant2015} libraries (data manipulation).
    \item The SciPy~\cite{Jones2001} library (numerical integration of the
        replicator equation).
\end{itemize}

This work was performed using the computational facilities of the Advanced
Research Computing @ Cardiff (ARCCA) Division, Cardiff University.

\printbibliography

\newpage
\section*{Supplementary materials}

\includepdf{assets/pdf/proof_of_form_of_extortionate_strategies/main.pdf}

\newpage

Using the pair wise interactions the transition rates \(p,
q\) can be measured and the steady state probabilities inferred and compared to
the actual probabilities of each state.
This is done numerically by computing the singular eigenvector of the
matrix \(A\) \cite{Stewart2009}:

\[
    A =
    \begin{bmatrix}
        p_1 q_1 & p_1 (1 - q_1) & (1 - p_1) q_1 & (1 -p_1) (1 - q_1) \\
        p_2 q_2 & p_2 (1 - q_2) & (1 - p_2) q_2 & (1 -p_2) (1 - q_2) \\
        p_3 q_3 & p_3 (1 - q_3) & (1 - p_3) q_3 & (1 -p_3) (1 - q_3) \\
        p_4 q_4 & p_4 (1 - q_4) & (1 - p_4) q_4 & (1 -p_4) (1 - q_4) \\
    \end{bmatrix}
\]

Figure~\ref{fig:computed_probabilities_vs_theoretic_probabilities} shows a
regression line fitted to every pairwise interaction with a reported
\(\text{SSError}\) value (pairwise interactions with missing states were
omitted). This serves to validate the approach: a part from some edge cases the
relationship is consistent.

\begin{figure}[!htbp]
    \centering
    \includegraphics[width=.8\textwidth]{./assets/img/computed_probabilities_vs_theoretic_probabilities/main.pdf}
    \caption{The
        relationship between the steady state probabilities inferred from the
        measured transitions and the actual steady state probabilities. A linear
        regression line is included validating the approach.}
    \label{fig:computed_probabilities_vs_theoretic_probabilities}
\end{figure}


\end{document}
 strategies,
was presented with specific consideration given to ZD strategies. This
tournament is reproduced here using the Axelrod-Python
project~\cite{Knight2016}. To obtain a good measure of the corresponding
transition rates for each strategy all matches have been run for
\documentclass[a4paper]{article}

\usepackage{amsmath}
\usepackage{amssymb}
\usepackage[margin=1.5cm,
            includefoot,
            footskip=30pt]{geometry}
\usepackage{layout}
\usepackage{graphicx}
\usepackage{subcaption}

\usepackage{biblatex}
\usepackage{pdfpages}

\bibliography{main.bib}

\title{Suspicion: Recognising and evaluating the effectiveness
       of extortion in the Iterated Prisoner's Dilemma}
\author{Vincent A. Knight \and Nikoleta E. Glynatsi}
\date{\today}



\begin{document}

\maketitle

\begin{abstract}
    The Iterated Prisoner's Dilemma is a model for rational and evolutionary
    interactive behaviour. It has applications both in the study of human social
    behaviour as well as in biology.
    It is used to understand when and how a rational individual might
    accept an immediate cost to their own utility for the direct benefit of
    another.

    Much attention has been given to a class of strategies called
    Zero Determinant strategies. It has been theoretically shown that these
    strategies can ``extort'' any player.

    In this work, an approach to identify if observed strategies are playing in
    an extortionate way is described. Furthermore, experimental analysis of
    a large tournament with \input{assets/tex/number_of_full_strategies/main.tex}
    strategies is considered. In this setting
    the most highly performing strategies do not play in an extortionate way
    against each other but do against lower performing strategies.
    This suggests that whilst the theory of Zero Determinant strategies
    indicates that memory is not of fundamental importance to the evolution of
    cooperative behaviour, this is incomplete.
\end{abstract}

\section{Introduction}\label{sec:introduction}

Agent based game theoretic models have become a stalwart of the underpinning
mathematics of interactive behaviours. One of the major pieces of work
in this area is the pair of original computer tournaments run by Robert
Axelrod~\cite{Axelrod1980, Axelrod1980a}. These tournaments pitted submitted
computer strategies against each other in plays of the Iterated Prisoner's
Dilemma. A common game where agents can choose to pay a slight cost to their
immediate utility in the hope of building a reputation. This has been used in
economic and evolutionary game theory to understand the evolution of cooperative
behaviour.

Recently, a class of strategies was described in~\cite{Press2012} that can
provably extort any given opponent. In~\cite{Hilbe2013, Moran1707} some
questions have already been asked about the true effectiveness of these
strategies in an evolutionary setting. Here another question is asked: is it
possible to recognise this extortionate behaviour? A mathematical procedure for
suspicion is presented: in the same way that the continued actions of an
extortionate individual might raise suspicion.

This work makes use of the Axelrod Python library~\cite{Knight2018, Knight2016}
with a large number of Prisoner Dilemma strategies available to give an
extensive numerical example of the ideas presented.  The approach is presented
in Section~\ref{sec:delta-zd-strategies}.  All of the code and data discussed
in Section~\ref{sec:numerical-experiments} is open sourced, archived and
written according to best scientific principles~\cite{Wilson2014}. The data
archive can be found at~\cite{vincent_knight_2018_1297075}.

\section{Recognising Extortion}\label{sec:delta-zd-strategies}

In~\cite{Press2012}, given a match between 2 memory-one strategies, the concept
of Zero Determinant (ZD) strategies is introduced. The main result of that paper
shows that given two memory one players \(p, q\in\mathbb{R}^4\) a linear
relationship between the players' scores could be forced by one of the players.

Using the notation of~\cite{Press2012}, assuming the utilities for player \(p\)
are given by \(S_x=(R, S, T, P)\) and for player \(q\) by \(S_y=(R, T, S, P)\)
and that the stationary scores of each player is given by \(S_X\) and \(S_Y\)
respectively. The main result of~\cite{Press2012} is that if

\begin{equation}\label{eqn:linear_relationship_for_p}
    \tilde p=\alpha S_x + \beta S_y + \gamma
\end{equation}

or

\begin{equation}\label{eqn:linear_relationship_for_q}
    \tilde q=\alpha S_x + \beta S_y + \gamma
\end{equation}

where \(\tilde p = (1 - p_1, 1 - p_2, p_3, p_4)\) and
\(\tilde q = (1 - q_1, 1 - q_2, q_3, q_4)\) then:

\begin{equation}
    \alpha S_X + \beta S_Y + \gamma = 0
\end{equation}

In~\cite{Press2012} a particular type of ZD strategy is defined: extortionate
strategies. If:

\begin{equation}\label{eqn:constraint_for_extortion}
    \gamma = - P(\alpha + \beta)
\end{equation}

then the player can ensure they get a score \(\chi\) times
larger than the opponent. This extortion coefficient is given by:

\begin{equation}\label{eqn:definition_of_chi}
    \chi=\frac{-\beta}{\alpha}
\end{equation}

Thus, if (\ref{eqn:constraint_for_extortion}) holds and \(\chi >1\) a player is
said to extort their opponent.
Here, the reverse problem is considered: given a
\(p\in\mathbb{R}^4\) how does one identify \(\alpha, \beta\) if they
exist and is the strategy in fact acting in an extortionate way?

These conditions correspond to:

\begin{align}
    \tilde p_1 & = \alpha R + \beta R - P (\alpha + \beta)
            \label{eqn:condition_for_tilde_p1}\\
    \tilde p_2 & = \alpha S + \beta T - P (\alpha + \beta)
            \label{eqn:condition_for_tilde_p2}\\
    \tilde p_3 & = \alpha T + \beta S - P (\alpha + \beta)
            \label{eqn:condition_for_tilde_p3}\\
    \tilde p_4 & = \alpha P + \beta P - P (\alpha + \beta)
            \label{eqn:condition_for_tilde_p4}
\end{align}

Equation (\ref{eqn:condition_for_tilde_p4}) ensures that \(p_4=\tilde p_4=0\).
Equations (\ref{eqn:condition_for_tilde_p1}-\ref{eqn:condition_for_tilde_p3})
can be used to eliminate \(\alpha, \beta\), giving:

\begin{equation}\label{eqn:planar_definition_of_extortion}
    \tilde p_1 = \frac{(R - P)(\tilde p_2 + \tilde p_3)}{S + T - 2P}
\end{equation}

with:

\begin{equation}\label{eqn:definition_of_chi}
    \chi = \frac{\tilde p_2 (P - T) + \tilde p_3 (S - P)}
                {\tilde p_2 (P - S) + \tilde p_3 (T - P)}
\end{equation}

Given a strategy \(p\in\mathbb{R}^{4\times 1}\) equations
(\ref{eqn:condition_for_tilde_p4}), (\ref{eqn:planar_definition_of_extortion}-\ref{eqn:definition_of_chi}) can be used to check if
a strategy is extortionate. The conditions correspond to:

\begin{align}
    p_1 & = \frac{(R-P)(p_2 + p_3) - R + T + S - P}{S + T - 2P}
     \label{eqn:condition_for_p1}\\
    p_4 & = 0 \label{eqn:condition_for_p4}\\
    1 & > p_2 + p_3\label{eqn:condition_for_chi}
\end{align}

The algebraic steps necessary to prove these results are available in the
supporting materials.

All extortionate strategies reside on a triangular (\ref{eqn:condition_for_chi})
plane (\ref{eqn:condition_for_p1}) in 3 dimensions (\ref{eqn:condition_for_p4}).
Using this formulation it can be seen that a necessary (but not sufficient)
condition for an extortionate strategy is that it cooperates on average less
than 50\% of the time when in a state of disagreement with the opponent.

As an example, consider the known extortionate strategy \(p=(8 / 9, 1 / 2, 1 /
3, 0)\) from~\cite{Stewart2012} which is referred to as \texttt{Extort-2}. In
this case, for the standard values of \((R, T, S, P)\) constraint
(\ref{eqn:condition_for_p1}) corresponds to:

\begin{equation}
    p_1 = \frac{2(p_2 + p_3) + 1}{3}
\end{equation}

It is clear that in this case all constraints hold.

This approach could in fact be used to confirm that a given strategy is acting
in an extortionate manner even if it is not a memory one strategy. However, in
practice, if a closed form for \(p\) is not known, then due to measurement
and/or numerical error this would not work.

This problem can be written in the following linear algebraic form where
\(x=(\alpha, \beta)\)
and \(p^*=(\tilde p_1 - 1, tilde_2 - 1, p_3)\):

\begin{equation}\label{eqn:linear_algebraic_equation_for_p}
    Cx= p^*
\end{equation}

\(C\) corresponds to equations
(\ref{eqn:condition_for_tilde_p1}-\ref{eqn:condition_for_tilde_p3}) and is
given by:

\begin{equation}\label{eqn:definition_of_C}
    C =
    \begin{bmatrix}
        R - P & R- P \\
        S - P & T- P \\
        T - P & S- P \\
    \end{bmatrix}
\end{equation}

Note that in general, equation (\ref{eqn:linear_algebraic_equation_for_p}) will
not necessarily have a solution. From the Rouch\'{e}-Capelli theorem if there is
a solution it is unique as \(\text{rank}(C)=2\) which is the dimension of the
variable \(x\). The best fitting \(x\) is found by minimizing:

\begin{equation}\label{eqn:r_squared}
    \text{SSError} = \|C x- p^*\|_2^2 = \sum_{i=1}^{3}\left((C\bar x)_i-p_i^*\right)^2
\end{equation}

Note that \(\text{SSError}\), which is the square of the Frobenius
norm~\cite{Golub2013}, becomes a measure of how close a strategy is to being an
extortionate strategy. Suspicion
of extortion then corresponds to a threshold on \(\text{SSError}\).

By observing interactions (human or otherwise), their memory one representation
can be inferred and this approach can be used to recognise extortionate
behaviour. The notion of comparing theoretic and actual plays of the IPD is not
novel, see for example~\cite{Rand2013}. Immediately it is noted that if the
environment is noisy~\cite{Wu1995} then no strategy can be considered to be
extortionate as \(p_4>0\).

In the next section, this idea will be illustrated by observing the interactions
that take place in a computer based tournament of the IPD\@.

\section{Numerical experiments}\label{sec:numerical-experiments}

In~\cite{Stewart2012} results from a tournament with
\input{./assets/tex/number_of_stewart_plotkin_strategies/main.tex} strategies,
was presented with specific consideration given to ZD strategies. This
tournament is reproduced here using the Axelrod-Python
project~\cite{Knight2016}. To obtain a good measure of the corresponding
transition rates for each strategy all matches have been run for
\input{assets/tex/number_of_turns/main.tex} turns and every match has been
repeated \input{assets/tex/number_of_repetitions/main.tex} times. All of this
interaction data is available at~\cite{vincent_knight_2018_1297075}. A good
match between the inferred Markov chain and the state distribution of the actual
interactions has been verified. Data for this is presented in the supplementary
materials.

Figure~\ref{fig:SSError_overall_in_stewart_plotkin} shows the \(\text{SSError}\)
values for all the strategies in the tournament, as reported
in~\cite{Stewart2012} the extortionate strategy (which has an expected
\(\text{SSError}\) approximately 0) gains a large number of wins.

\begin{figure}[!htbp]
    \centering
    \includegraphics[width=.8\textwidth]{./assets/img/SSError_overall_in_stewart_plotkin/main.pdf}
    \caption{\(\text{SSError}\) and state probabilities for the strategies
        of~\cite{Stewart2012}, ordered both by number of wins and overall score.
        Note that \(P(DC)\) is not shown as it corresponds to the transpose of
        \(P(CD)\). Cooperator and Defector are omitted as they do not visit all
        the states.}
    \label{fig:SSError_overall_in_stewart_plotkin}
\end{figure}

Here, the work of~\cite{Stewart2012} is extended by investigating a tournament
with \input{assets/tex/number_of_full_strategies/main.tex}
strategies.

The results of this analysis are shown in
Figure~\ref{fig:SSError_and_probabilities_in_full}. The top ranking strategies
by number of wins seem to be extortionate (but not against all strategies) and
it can be seen that a small sub group of strategies achieve mutual defection.
All the top ranking strategies according to score achieve mutual cooperation and
do not extort each other, however they
\textbf{do} exhibit extortionate behaviour towards a number of the lower ranking
strategies.

\begin{figure}[!htbp]
    \centering
    \includegraphics[width=.8\textwidth]{./assets/img/SSError_and_probabilities_in_full/main.pdf}
    \caption{\(\text{SSError}\) for the strategies for the full tournament. Only
    strategy interactions for which \(p_4=0\) and \(\chi>1\) are displayed.}
    \label{fig:SSError_and_probabilities_in_full}
\end{figure}

\section{Conclusion}\label{sec:conclusion}

This work defines an approach to measure whether or not a player is playing a
strategy that corresponds to an extortionate strategy as defined
in~\cite{Press2012}: a mathematical model for suspicion. Indeed, all
extortionate strategies have been
 classified as lying on a triangular plane.
This rigorous classification fails to be robust to small measurement error, thus
a statistical approach is proposed.
This is done through a linear algebraic approach for approximating the solution
of a linear system. Using this, a large number of pairwise interactions is
simulated and in fact very few strategies are found to act extortionately.

The work of~\cite{Press2012}, whilst showing that a clever approach to taking
advantage of another memory one strategy exists: this is incomplete. Whilst the
elegance of this result is very attractive, just as the simplicity of the
victory of Tit For Tat in Axelrod's original tournaments was, it is incomplete.
Extortionate strategies achieve a high number of wins but they do not
achieve a high score which corresponds to the fitness landscape in an
evolutionary sense. From the large number of interactions a payoff matrix \(S\)
can be measured where \(S_{ij}\) denotes the score (using standard values of
\((R, S, T, P) = (3, 0, 5, 1)\)) of the \(i\)th strategy
against the \(j\)th strategy. Using this, the replicator equation
describes the evolution of the system based on a population density fitness
function:

\begin{equation}\label{eqn:replicator_dynamics}
    \frac{dx}{dt} = x(S-x^TS x)
\end{equation}

Equation (\ref{eqn:replicator_dynamics}) is solved numerically through an
integration technique described in~\cite{Petzold1983} and
Figure~\ref{fig:replicator_dynamics} shows the evolution of the distribution of
the system: the various strategies are ranked by scores. It is clear to see that
only the high ranking strategies survive the evolutionary process (in fact,
only \input{./assets/img/replicator_dynamics/main.tex}
have a final distribution greater than \(10 ^ {-2}\)). This confirms the
findings of~\cite{Moran1707} in which sophisticated strategies resist
evolutionary invasion of shorter memory strategies. Recalling
Figure~\ref{fig:SSError_and_probabilities_in_full} this demonstrates that:

\begin{itemize}
    \item Cooperation emerges through the evolutionary process: the high scoring
        strategies do not exhibit extortionate behaviour towards each other.
    \item Extortionate strategies do not survive the evolutionary process.
\end{itemize}

\begin{figure}[!htbp]
    \centering
    \includegraphics[width=.8\textwidth]{./assets/img/replicator_dynamics/main.pdf}
    \caption{Numerical simulation of the replicator equation
    (\ref{eqn:replicator_dynamics}): strategies are ordered by score, only the strategies with a high score survive the evolutionary process.}
    \label{fig:replicator_dynamics}
\end{figure}

This work can be used to classify plays of the IPD\@: data can be collected from
actual interactions (in lab or in the field). Furthermore, this allows for a
classification method similar to the notion of fingerprinting presented
in~\cite{Ashlock2008}. Trained strategies can potentially be classified as
extortionate or not or it could be possible to even constrain the reinforcement
learning approaches that are becoming prevalent in the literature.
Alternatively, this mathematical approach for recognising extortion could be
used in sophisticated strategies to defend against invasion. Arguably, some of
the strategies considered here exhibit this behaviour, indeed as described
in~\cite{Harper2017}, the top ranking strategies in the full tournament are
obtained using evolutionary reinforcement learning techniques, thus, suspicion
of extortionate behaviour could in fact be an evolutionary trait.

\section*{Acknowledgements}

The following open source software libraries were used in this research:

\begin{itemize}
    \item The Axelrod ~\cite{Knight2016, Knight2018} library (IPD strategies and
        tournaments).
    \item The sympy library~\cite{Meurer2017} (verification of all symbolic
        calculations).
    \item The matplotlib~\cite{Droettboom2018} library (visualisation).
    \item The pandas~\cite{Structures2010}, dask~\cite{Dask2016} and
        NumPy~\cite{Oliphant2015} libraries (data manipulation).
    \item The SciPy~\cite{Jones2001} library (numerical integration of the
        replicator equation).
\end{itemize}

This work was performed using the computational facilities of the Advanced
Research Computing @ Cardiff (ARCCA) Division, Cardiff University.

\printbibliography

\newpage
\section*{Supplementary materials}

\includepdf{assets/pdf/proof_of_form_of_extortionate_strategies/main.pdf}

\newpage

Using the pair wise interactions the transition rates \(p,
q\) can be measured and the steady state probabilities inferred and compared to
the actual probabilities of each state.
This is done numerically by computing the singular eigenvector of the
matrix \(A\) \cite{Stewart2009}:

\[
    A =
    \begin{bmatrix}
        p_1 q_1 & p_1 (1 - q_1) & (1 - p_1) q_1 & (1 -p_1) (1 - q_1) \\
        p_2 q_2 & p_2 (1 - q_2) & (1 - p_2) q_2 & (1 -p_2) (1 - q_2) \\
        p_3 q_3 & p_3 (1 - q_3) & (1 - p_3) q_3 & (1 -p_3) (1 - q_3) \\
        p_4 q_4 & p_4 (1 - q_4) & (1 - p_4) q_4 & (1 -p_4) (1 - q_4) \\
    \end{bmatrix}
\]

Figure~\ref{fig:computed_probabilities_vs_theoretic_probabilities} shows a
regression line fitted to every pairwise interaction with a reported
\(\text{SSError}\) value (pairwise interactions with missing states were
omitted). This serves to validate the approach: a part from some edge cases the
relationship is consistent.

\begin{figure}[!htbp]
    \centering
    \includegraphics[width=.8\textwidth]{./assets/img/computed_probabilities_vs_theoretic_probabilities/main.pdf}
    \caption{The
        relationship between the steady state probabilities inferred from the
        measured transitions and the actual steady state probabilities. A linear
        regression line is included validating the approach.}
    \label{fig:computed_probabilities_vs_theoretic_probabilities}
\end{figure}


\end{document}
 turns and every match has been
repeated \documentclass[a4paper]{article}

\usepackage{amsmath}
\usepackage{amssymb}
\usepackage[margin=1.5cm,
            includefoot,
            footskip=30pt]{geometry}
\usepackage{layout}
\usepackage{graphicx}
\usepackage{subcaption}

\usepackage{biblatex}
\usepackage{pdfpages}

\bibliography{main.bib}

\title{Suspicion: Recognising and evaluating the effectiveness
       of extortion in the Iterated Prisoner's Dilemma}
\author{Vincent A. Knight \and Nikoleta E. Glynatsi}
\date{\today}



\begin{document}

\maketitle

\begin{abstract}
    The Iterated Prisoner's Dilemma is a model for rational and evolutionary
    interactive behaviour. It has applications both in the study of human social
    behaviour as well as in biology.
    It is used to understand when and how a rational individual might
    accept an immediate cost to their own utility for the direct benefit of
    another.

    Much attention has been given to a class of strategies called
    Zero Determinant strategies. It has been theoretically shown that these
    strategies can ``extort'' any player.

    In this work, an approach to identify if observed strategies are playing in
    an extortionate way is described. Furthermore, experimental analysis of
    a large tournament with \input{assets/tex/number_of_full_strategies/main.tex}
    strategies is considered. In this setting
    the most highly performing strategies do not play in an extortionate way
    against each other but do against lower performing strategies.
    This suggests that whilst the theory of Zero Determinant strategies
    indicates that memory is not of fundamental importance to the evolution of
    cooperative behaviour, this is incomplete.
\end{abstract}

\section{Introduction}\label{sec:introduction}

Agent based game theoretic models have become a stalwart of the underpinning
mathematics of interactive behaviours. One of the major pieces of work
in this area is the pair of original computer tournaments run by Robert
Axelrod~\cite{Axelrod1980, Axelrod1980a}. These tournaments pitted submitted
computer strategies against each other in plays of the Iterated Prisoner's
Dilemma. A common game where agents can choose to pay a slight cost to their
immediate utility in the hope of building a reputation. This has been used in
economic and evolutionary game theory to understand the evolution of cooperative
behaviour.

Recently, a class of strategies was described in~\cite{Press2012} that can
provably extort any given opponent. In~\cite{Hilbe2013, Moran1707} some
questions have already been asked about the true effectiveness of these
strategies in an evolutionary setting. Here another question is asked: is it
possible to recognise this extortionate behaviour? A mathematical procedure for
suspicion is presented: in the same way that the continued actions of an
extortionate individual might raise suspicion.

This work makes use of the Axelrod Python library~\cite{Knight2018, Knight2016}
with a large number of Prisoner Dilemma strategies available to give an
extensive numerical example of the ideas presented.  The approach is presented
in Section~\ref{sec:delta-zd-strategies}.  All of the code and data discussed
in Section~\ref{sec:numerical-experiments} is open sourced, archived and
written according to best scientific principles~\cite{Wilson2014}. The data
archive can be found at~\cite{vincent_knight_2018_1297075}.

\section{Recognising Extortion}\label{sec:delta-zd-strategies}

In~\cite{Press2012}, given a match between 2 memory-one strategies, the concept
of Zero Determinant (ZD) strategies is introduced. The main result of that paper
shows that given two memory one players \(p, q\in\mathbb{R}^4\) a linear
relationship between the players' scores could be forced by one of the players.

Using the notation of~\cite{Press2012}, assuming the utilities for player \(p\)
are given by \(S_x=(R, S, T, P)\) and for player \(q\) by \(S_y=(R, T, S, P)\)
and that the stationary scores of each player is given by \(S_X\) and \(S_Y\)
respectively. The main result of~\cite{Press2012} is that if

\begin{equation}\label{eqn:linear_relationship_for_p}
    \tilde p=\alpha S_x + \beta S_y + \gamma
\end{equation}

or

\begin{equation}\label{eqn:linear_relationship_for_q}
    \tilde q=\alpha S_x + \beta S_y + \gamma
\end{equation}

where \(\tilde p = (1 - p_1, 1 - p_2, p_3, p_4)\) and
\(\tilde q = (1 - q_1, 1 - q_2, q_3, q_4)\) then:

\begin{equation}
    \alpha S_X + \beta S_Y + \gamma = 0
\end{equation}

In~\cite{Press2012} a particular type of ZD strategy is defined: extortionate
strategies. If:

\begin{equation}\label{eqn:constraint_for_extortion}
    \gamma = - P(\alpha + \beta)
\end{equation}

then the player can ensure they get a score \(\chi\) times
larger than the opponent. This extortion coefficient is given by:

\begin{equation}\label{eqn:definition_of_chi}
    \chi=\frac{-\beta}{\alpha}
\end{equation}

Thus, if (\ref{eqn:constraint_for_extortion}) holds and \(\chi >1\) a player is
said to extort their opponent.
Here, the reverse problem is considered: given a
\(p\in\mathbb{R}^4\) how does one identify \(\alpha, \beta\) if they
exist and is the strategy in fact acting in an extortionate way?

These conditions correspond to:

\begin{align}
    \tilde p_1 & = \alpha R + \beta R - P (\alpha + \beta)
            \label{eqn:condition_for_tilde_p1}\\
    \tilde p_2 & = \alpha S + \beta T - P (\alpha + \beta)
            \label{eqn:condition_for_tilde_p2}\\
    \tilde p_3 & = \alpha T + \beta S - P (\alpha + \beta)
            \label{eqn:condition_for_tilde_p3}\\
    \tilde p_4 & = \alpha P + \beta P - P (\alpha + \beta)
            \label{eqn:condition_for_tilde_p4}
\end{align}

Equation (\ref{eqn:condition_for_tilde_p4}) ensures that \(p_4=\tilde p_4=0\).
Equations (\ref{eqn:condition_for_tilde_p1}-\ref{eqn:condition_for_tilde_p3})
can be used to eliminate \(\alpha, \beta\), giving:

\begin{equation}\label{eqn:planar_definition_of_extortion}
    \tilde p_1 = \frac{(R - P)(\tilde p_2 + \tilde p_3)}{S + T - 2P}
\end{equation}

with:

\begin{equation}\label{eqn:definition_of_chi}
    \chi = \frac{\tilde p_2 (P - T) + \tilde p_3 (S - P)}
                {\tilde p_2 (P - S) + \tilde p_3 (T - P)}
\end{equation}

Given a strategy \(p\in\mathbb{R}^{4\times 1}\) equations
(\ref{eqn:condition_for_tilde_p4}), (\ref{eqn:planar_definition_of_extortion}-\ref{eqn:definition_of_chi}) can be used to check if
a strategy is extortionate. The conditions correspond to:

\begin{align}
    p_1 & = \frac{(R-P)(p_2 + p_3) - R + T + S - P}{S + T - 2P}
     \label{eqn:condition_for_p1}\\
    p_4 & = 0 \label{eqn:condition_for_p4}\\
    1 & > p_2 + p_3\label{eqn:condition_for_chi}
\end{align}

The algebraic steps necessary to prove these results are available in the
supporting materials.

All extortionate strategies reside on a triangular (\ref{eqn:condition_for_chi})
plane (\ref{eqn:condition_for_p1}) in 3 dimensions (\ref{eqn:condition_for_p4}).
Using this formulation it can be seen that a necessary (but not sufficient)
condition for an extortionate strategy is that it cooperates on average less
than 50\% of the time when in a state of disagreement with the opponent.

As an example, consider the known extortionate strategy \(p=(8 / 9, 1 / 2, 1 /
3, 0)\) from~\cite{Stewart2012} which is referred to as \texttt{Extort-2}. In
this case, for the standard values of \((R, T, S, P)\) constraint
(\ref{eqn:condition_for_p1}) corresponds to:

\begin{equation}
    p_1 = \frac{2(p_2 + p_3) + 1}{3}
\end{equation}

It is clear that in this case all constraints hold.

This approach could in fact be used to confirm that a given strategy is acting
in an extortionate manner even if it is not a memory one strategy. However, in
practice, if a closed form for \(p\) is not known, then due to measurement
and/or numerical error this would not work.

This problem can be written in the following linear algebraic form where
\(x=(\alpha, \beta)\)
and \(p^*=(\tilde p_1 - 1, tilde_2 - 1, p_3)\):

\begin{equation}\label{eqn:linear_algebraic_equation_for_p}
    Cx= p^*
\end{equation}

\(C\) corresponds to equations
(\ref{eqn:condition_for_tilde_p1}-\ref{eqn:condition_for_tilde_p3}) and is
given by:

\begin{equation}\label{eqn:definition_of_C}
    C =
    \begin{bmatrix}
        R - P & R- P \\
        S - P & T- P \\
        T - P & S- P \\
    \end{bmatrix}
\end{equation}

Note that in general, equation (\ref{eqn:linear_algebraic_equation_for_p}) will
not necessarily have a solution. From the Rouch\'{e}-Capelli theorem if there is
a solution it is unique as \(\text{rank}(C)=2\) which is the dimension of the
variable \(x\). The best fitting \(x\) is found by minimizing:

\begin{equation}\label{eqn:r_squared}
    \text{SSError} = \|C x- p^*\|_2^2 = \sum_{i=1}^{3}\left((C\bar x)_i-p_i^*\right)^2
\end{equation}

Note that \(\text{SSError}\), which is the square of the Frobenius
norm~\cite{Golub2013}, becomes a measure of how close a strategy is to being an
extortionate strategy. Suspicion
of extortion then corresponds to a threshold on \(\text{SSError}\).

By observing interactions (human or otherwise), their memory one representation
can be inferred and this approach can be used to recognise extortionate
behaviour. The notion of comparing theoretic and actual plays of the IPD is not
novel, see for example~\cite{Rand2013}. Immediately it is noted that if the
environment is noisy~\cite{Wu1995} then no strategy can be considered to be
extortionate as \(p_4>0\).

In the next section, this idea will be illustrated by observing the interactions
that take place in a computer based tournament of the IPD\@.

\section{Numerical experiments}\label{sec:numerical-experiments}

In~\cite{Stewart2012} results from a tournament with
\input{./assets/tex/number_of_stewart_plotkin_strategies/main.tex} strategies,
was presented with specific consideration given to ZD strategies. This
tournament is reproduced here using the Axelrod-Python
project~\cite{Knight2016}. To obtain a good measure of the corresponding
transition rates for each strategy all matches have been run for
\input{assets/tex/number_of_turns/main.tex} turns and every match has been
repeated \input{assets/tex/number_of_repetitions/main.tex} times. All of this
interaction data is available at~\cite{vincent_knight_2018_1297075}. A good
match between the inferred Markov chain and the state distribution of the actual
interactions has been verified. Data for this is presented in the supplementary
materials.

Figure~\ref{fig:SSError_overall_in_stewart_plotkin} shows the \(\text{SSError}\)
values for all the strategies in the tournament, as reported
in~\cite{Stewart2012} the extortionate strategy (which has an expected
\(\text{SSError}\) approximately 0) gains a large number of wins.

\begin{figure}[!htbp]
    \centering
    \includegraphics[width=.8\textwidth]{./assets/img/SSError_overall_in_stewart_plotkin/main.pdf}
    \caption{\(\text{SSError}\) and state probabilities for the strategies
        of~\cite{Stewart2012}, ordered both by number of wins and overall score.
        Note that \(P(DC)\) is not shown as it corresponds to the transpose of
        \(P(CD)\). Cooperator and Defector are omitted as they do not visit all
        the states.}
    \label{fig:SSError_overall_in_stewart_plotkin}
\end{figure}

Here, the work of~\cite{Stewart2012} is extended by investigating a tournament
with \input{assets/tex/number_of_full_strategies/main.tex}
strategies.

The results of this analysis are shown in
Figure~\ref{fig:SSError_and_probabilities_in_full}. The top ranking strategies
by number of wins seem to be extortionate (but not against all strategies) and
it can be seen that a small sub group of strategies achieve mutual defection.
All the top ranking strategies according to score achieve mutual cooperation and
do not extort each other, however they
\textbf{do} exhibit extortionate behaviour towards a number of the lower ranking
strategies.

\begin{figure}[!htbp]
    \centering
    \includegraphics[width=.8\textwidth]{./assets/img/SSError_and_probabilities_in_full/main.pdf}
    \caption{\(\text{SSError}\) for the strategies for the full tournament. Only
    strategy interactions for which \(p_4=0\) and \(\chi>1\) are displayed.}
    \label{fig:SSError_and_probabilities_in_full}
\end{figure}

\section{Conclusion}\label{sec:conclusion}

This work defines an approach to measure whether or not a player is playing a
strategy that corresponds to an extortionate strategy as defined
in~\cite{Press2012}: a mathematical model for suspicion. Indeed, all
extortionate strategies have been
 classified as lying on a triangular plane.
This rigorous classification fails to be robust to small measurement error, thus
a statistical approach is proposed.
This is done through a linear algebraic approach for approximating the solution
of a linear system. Using this, a large number of pairwise interactions is
simulated and in fact very few strategies are found to act extortionately.

The work of~\cite{Press2012}, whilst showing that a clever approach to taking
advantage of another memory one strategy exists: this is incomplete. Whilst the
elegance of this result is very attractive, just as the simplicity of the
victory of Tit For Tat in Axelrod's original tournaments was, it is incomplete.
Extortionate strategies achieve a high number of wins but they do not
achieve a high score which corresponds to the fitness landscape in an
evolutionary sense. From the large number of interactions a payoff matrix \(S\)
can be measured where \(S_{ij}\) denotes the score (using standard values of
\((R, S, T, P) = (3, 0, 5, 1)\)) of the \(i\)th strategy
against the \(j\)th strategy. Using this, the replicator equation
describes the evolution of the system based on a population density fitness
function:

\begin{equation}\label{eqn:replicator_dynamics}
    \frac{dx}{dt} = x(S-x^TS x)
\end{equation}

Equation (\ref{eqn:replicator_dynamics}) is solved numerically through an
integration technique described in~\cite{Petzold1983} and
Figure~\ref{fig:replicator_dynamics} shows the evolution of the distribution of
the system: the various strategies are ranked by scores. It is clear to see that
only the high ranking strategies survive the evolutionary process (in fact,
only \input{./assets/img/replicator_dynamics/main.tex}
have a final distribution greater than \(10 ^ {-2}\)). This confirms the
findings of~\cite{Moran1707} in which sophisticated strategies resist
evolutionary invasion of shorter memory strategies. Recalling
Figure~\ref{fig:SSError_and_probabilities_in_full} this demonstrates that:

\begin{itemize}
    \item Cooperation emerges through the evolutionary process: the high scoring
        strategies do not exhibit extortionate behaviour towards each other.
    \item Extortionate strategies do not survive the evolutionary process.
\end{itemize}

\begin{figure}[!htbp]
    \centering
    \includegraphics[width=.8\textwidth]{./assets/img/replicator_dynamics/main.pdf}
    \caption{Numerical simulation of the replicator equation
    (\ref{eqn:replicator_dynamics}): strategies are ordered by score, only the strategies with a high score survive the evolutionary process.}
    \label{fig:replicator_dynamics}
\end{figure}

This work can be used to classify plays of the IPD\@: data can be collected from
actual interactions (in lab or in the field). Furthermore, this allows for a
classification method similar to the notion of fingerprinting presented
in~\cite{Ashlock2008}. Trained strategies can potentially be classified as
extortionate or not or it could be possible to even constrain the reinforcement
learning approaches that are becoming prevalent in the literature.
Alternatively, this mathematical approach for recognising extortion could be
used in sophisticated strategies to defend against invasion. Arguably, some of
the strategies considered here exhibit this behaviour, indeed as described
in~\cite{Harper2017}, the top ranking strategies in the full tournament are
obtained using evolutionary reinforcement learning techniques, thus, suspicion
of extortionate behaviour could in fact be an evolutionary trait.

\section*{Acknowledgements}

The following open source software libraries were used in this research:

\begin{itemize}
    \item The Axelrod ~\cite{Knight2016, Knight2018} library (IPD strategies and
        tournaments).
    \item The sympy library~\cite{Meurer2017} (verification of all symbolic
        calculations).
    \item The matplotlib~\cite{Droettboom2018} library (visualisation).
    \item The pandas~\cite{Structures2010}, dask~\cite{Dask2016} and
        NumPy~\cite{Oliphant2015} libraries (data manipulation).
    \item The SciPy~\cite{Jones2001} library (numerical integration of the
        replicator equation).
\end{itemize}

This work was performed using the computational facilities of the Advanced
Research Computing @ Cardiff (ARCCA) Division, Cardiff University.

\printbibliography

\newpage
\section*{Supplementary materials}

\includepdf{assets/pdf/proof_of_form_of_extortionate_strategies/main.pdf}

\newpage

Using the pair wise interactions the transition rates \(p,
q\) can be measured and the steady state probabilities inferred and compared to
the actual probabilities of each state.
This is done numerically by computing the singular eigenvector of the
matrix \(A\) \cite{Stewart2009}:

\[
    A =
    \begin{bmatrix}
        p_1 q_1 & p_1 (1 - q_1) & (1 - p_1) q_1 & (1 -p_1) (1 - q_1) \\
        p_2 q_2 & p_2 (1 - q_2) & (1 - p_2) q_2 & (1 -p_2) (1 - q_2) \\
        p_3 q_3 & p_3 (1 - q_3) & (1 - p_3) q_3 & (1 -p_3) (1 - q_3) \\
        p_4 q_4 & p_4 (1 - q_4) & (1 - p_4) q_4 & (1 -p_4) (1 - q_4) \\
    \end{bmatrix}
\]

Figure~\ref{fig:computed_probabilities_vs_theoretic_probabilities} shows a
regression line fitted to every pairwise interaction with a reported
\(\text{SSError}\) value (pairwise interactions with missing states were
omitted). This serves to validate the approach: a part from some edge cases the
relationship is consistent.

\begin{figure}[!htbp]
    \centering
    \includegraphics[width=.8\textwidth]{./assets/img/computed_probabilities_vs_theoretic_probabilities/main.pdf}
    \caption{The
        relationship between the steady state probabilities inferred from the
        measured transitions and the actual steady state probabilities. A linear
        regression line is included validating the approach.}
    \label{fig:computed_probabilities_vs_theoretic_probabilities}
\end{figure}


\end{document}
 times. All of this
interaction data is available at~\cite{vincent_knight_2018_1297075}. A good
match between the inferred Markov chain and the state distribution of the actual
interactions has been verified. Data for this is presented in the supplementary
materials.

Figure~\ref{fig:SSError_overall_in_stewart_plotkin} shows the \(\text{SSError}\)
values for all the strategies in the tournament, as reported
in~\cite{Stewart2012} the extortionate strategy (which has an expected
\(\text{SSError}\) approximately 0) gains a large number of wins.

\begin{figure}[!htbp]
    \centering
    \includegraphics[width=.8\textwidth]{./assets/img/SSError_overall_in_stewart_plotkin/main.pdf}
    \caption{\(\text{SSError}\) and state probabilities for the strategies
        of~\cite{Stewart2012}, ordered both by number of wins and overall score.
        Note that \(P(DC)\) is not shown as it corresponds to the transpose of
        \(P(CD)\). Cooperator and Defector are omitted as they do not visit all
        the states.}
    \label{fig:SSError_overall_in_stewart_plotkin}
\end{figure}

Here, the work of~\cite{Stewart2012} is extended by investigating a tournament
with \documentclass[a4paper]{article}

\usepackage{amsmath}
\usepackage{amssymb}
\usepackage[margin=1.5cm,
            includefoot,
            footskip=30pt]{geometry}
\usepackage{layout}
\usepackage{graphicx}
\usepackage{subcaption}

\usepackage{biblatex}
\usepackage{pdfpages}

\bibliography{main.bib}

\title{Suspicion: Recognising and evaluating the effectiveness
       of extortion in the Iterated Prisoner's Dilemma}
\author{Vincent A. Knight \and Nikoleta E. Glynatsi}
\date{\today}



\begin{document}

\maketitle

\begin{abstract}
    The Iterated Prisoner's Dilemma is a model for rational and evolutionary
    interactive behaviour. It has applications both in the study of human social
    behaviour as well as in biology.
    It is used to understand when and how a rational individual might
    accept an immediate cost to their own utility for the direct benefit of
    another.

    Much attention has been given to a class of strategies called
    Zero Determinant strategies. It has been theoretically shown that these
    strategies can ``extort'' any player.

    In this work, an approach to identify if observed strategies are playing in
    an extortionate way is described. Furthermore, experimental analysis of
    a large tournament with \input{assets/tex/number_of_full_strategies/main.tex}
    strategies is considered. In this setting
    the most highly performing strategies do not play in an extortionate way
    against each other but do against lower performing strategies.
    This suggests that whilst the theory of Zero Determinant strategies
    indicates that memory is not of fundamental importance to the evolution of
    cooperative behaviour, this is incomplete.
\end{abstract}

\section{Introduction}\label{sec:introduction}

Agent based game theoretic models have become a stalwart of the underpinning
mathematics of interactive behaviours. One of the major pieces of work
in this area is the pair of original computer tournaments run by Robert
Axelrod~\cite{Axelrod1980, Axelrod1980a}. These tournaments pitted submitted
computer strategies against each other in plays of the Iterated Prisoner's
Dilemma. A common game where agents can choose to pay a slight cost to their
immediate utility in the hope of building a reputation. This has been used in
economic and evolutionary game theory to understand the evolution of cooperative
behaviour.

Recently, a class of strategies was described in~\cite{Press2012} that can
provably extort any given opponent. In~\cite{Hilbe2013, Moran1707} some
questions have already been asked about the true effectiveness of these
strategies in an evolutionary setting. Here another question is asked: is it
possible to recognise this extortionate behaviour? A mathematical procedure for
suspicion is presented: in the same way that the continued actions of an
extortionate individual might raise suspicion.

This work makes use of the Axelrod Python library~\cite{Knight2018, Knight2016}
with a large number of Prisoner Dilemma strategies available to give an
extensive numerical example of the ideas presented.  The approach is presented
in Section~\ref{sec:delta-zd-strategies}.  All of the code and data discussed
in Section~\ref{sec:numerical-experiments} is open sourced, archived and
written according to best scientific principles~\cite{Wilson2014}. The data
archive can be found at~\cite{vincent_knight_2018_1297075}.

\section{Recognising Extortion}\label{sec:delta-zd-strategies}

In~\cite{Press2012}, given a match between 2 memory-one strategies, the concept
of Zero Determinant (ZD) strategies is introduced. The main result of that paper
shows that given two memory one players \(p, q\in\mathbb{R}^4\) a linear
relationship between the players' scores could be forced by one of the players.

Using the notation of~\cite{Press2012}, assuming the utilities for player \(p\)
are given by \(S_x=(R, S, T, P)\) and for player \(q\) by \(S_y=(R, T, S, P)\)
and that the stationary scores of each player is given by \(S_X\) and \(S_Y\)
respectively. The main result of~\cite{Press2012} is that if

\begin{equation}\label{eqn:linear_relationship_for_p}
    \tilde p=\alpha S_x + \beta S_y + \gamma
\end{equation}

or

\begin{equation}\label{eqn:linear_relationship_for_q}
    \tilde q=\alpha S_x + \beta S_y + \gamma
\end{equation}

where \(\tilde p = (1 - p_1, 1 - p_2, p_3, p_4)\) and
\(\tilde q = (1 - q_1, 1 - q_2, q_3, q_4)\) then:

\begin{equation}
    \alpha S_X + \beta S_Y + \gamma = 0
\end{equation}

In~\cite{Press2012} a particular type of ZD strategy is defined: extortionate
strategies. If:

\begin{equation}\label{eqn:constraint_for_extortion}
    \gamma = - P(\alpha + \beta)
\end{equation}

then the player can ensure they get a score \(\chi\) times
larger than the opponent. This extortion coefficient is given by:

\begin{equation}\label{eqn:definition_of_chi}
    \chi=\frac{-\beta}{\alpha}
\end{equation}

Thus, if (\ref{eqn:constraint_for_extortion}) holds and \(\chi >1\) a player is
said to extort their opponent.
Here, the reverse problem is considered: given a
\(p\in\mathbb{R}^4\) how does one identify \(\alpha, \beta\) if they
exist and is the strategy in fact acting in an extortionate way?

These conditions correspond to:

\begin{align}
    \tilde p_1 & = \alpha R + \beta R - P (\alpha + \beta)
            \label{eqn:condition_for_tilde_p1}\\
    \tilde p_2 & = \alpha S + \beta T - P (\alpha + \beta)
            \label{eqn:condition_for_tilde_p2}\\
    \tilde p_3 & = \alpha T + \beta S - P (\alpha + \beta)
            \label{eqn:condition_for_tilde_p3}\\
    \tilde p_4 & = \alpha P + \beta P - P (\alpha + \beta)
            \label{eqn:condition_for_tilde_p4}
\end{align}

Equation (\ref{eqn:condition_for_tilde_p4}) ensures that \(p_4=\tilde p_4=0\).
Equations (\ref{eqn:condition_for_tilde_p1}-\ref{eqn:condition_for_tilde_p3})
can be used to eliminate \(\alpha, \beta\), giving:

\begin{equation}\label{eqn:planar_definition_of_extortion}
    \tilde p_1 = \frac{(R - P)(\tilde p_2 + \tilde p_3)}{S + T - 2P}
\end{equation}

with:

\begin{equation}\label{eqn:definition_of_chi}
    \chi = \frac{\tilde p_2 (P - T) + \tilde p_3 (S - P)}
                {\tilde p_2 (P - S) + \tilde p_3 (T - P)}
\end{equation}

Given a strategy \(p\in\mathbb{R}^{4\times 1}\) equations
(\ref{eqn:condition_for_tilde_p4}), (\ref{eqn:planar_definition_of_extortion}-\ref{eqn:definition_of_chi}) can be used to check if
a strategy is extortionate. The conditions correspond to:

\begin{align}
    p_1 & = \frac{(R-P)(p_2 + p_3) - R + T + S - P}{S + T - 2P}
     \label{eqn:condition_for_p1}\\
    p_4 & = 0 \label{eqn:condition_for_p4}\\
    1 & > p_2 + p_3\label{eqn:condition_for_chi}
\end{align}

The algebraic steps necessary to prove these results are available in the
supporting materials.

All extortionate strategies reside on a triangular (\ref{eqn:condition_for_chi})
plane (\ref{eqn:condition_for_p1}) in 3 dimensions (\ref{eqn:condition_for_p4}).
Using this formulation it can be seen that a necessary (but not sufficient)
condition for an extortionate strategy is that it cooperates on average less
than 50\% of the time when in a state of disagreement with the opponent.

As an example, consider the known extortionate strategy \(p=(8 / 9, 1 / 2, 1 /
3, 0)\) from~\cite{Stewart2012} which is referred to as \texttt{Extort-2}. In
this case, for the standard values of \((R, T, S, P)\) constraint
(\ref{eqn:condition_for_p1}) corresponds to:

\begin{equation}
    p_1 = \frac{2(p_2 + p_3) + 1}{3}
\end{equation}

It is clear that in this case all constraints hold.

This approach could in fact be used to confirm that a given strategy is acting
in an extortionate manner even if it is not a memory one strategy. However, in
practice, if a closed form for \(p\) is not known, then due to measurement
and/or numerical error this would not work.

This problem can be written in the following linear algebraic form where
\(x=(\alpha, \beta)\)
and \(p^*=(\tilde p_1 - 1, tilde_2 - 1, p_3)\):

\begin{equation}\label{eqn:linear_algebraic_equation_for_p}
    Cx= p^*
\end{equation}

\(C\) corresponds to equations
(\ref{eqn:condition_for_tilde_p1}-\ref{eqn:condition_for_tilde_p3}) and is
given by:

\begin{equation}\label{eqn:definition_of_C}
    C =
    \begin{bmatrix}
        R - P & R- P \\
        S - P & T- P \\
        T - P & S- P \\
    \end{bmatrix}
\end{equation}

Note that in general, equation (\ref{eqn:linear_algebraic_equation_for_p}) will
not necessarily have a solution. From the Rouch\'{e}-Capelli theorem if there is
a solution it is unique as \(\text{rank}(C)=2\) which is the dimension of the
variable \(x\). The best fitting \(x\) is found by minimizing:

\begin{equation}\label{eqn:r_squared}
    \text{SSError} = \|C x- p^*\|_2^2 = \sum_{i=1}^{3}\left((C\bar x)_i-p_i^*\right)^2
\end{equation}

Note that \(\text{SSError}\), which is the square of the Frobenius
norm~\cite{Golub2013}, becomes a measure of how close a strategy is to being an
extortionate strategy. Suspicion
of extortion then corresponds to a threshold on \(\text{SSError}\).

By observing interactions (human or otherwise), their memory one representation
can be inferred and this approach can be used to recognise extortionate
behaviour. The notion of comparing theoretic and actual plays of the IPD is not
novel, see for example~\cite{Rand2013}. Immediately it is noted that if the
environment is noisy~\cite{Wu1995} then no strategy can be considered to be
extortionate as \(p_4>0\).

In the next section, this idea will be illustrated by observing the interactions
that take place in a computer based tournament of the IPD\@.

\section{Numerical experiments}\label{sec:numerical-experiments}

In~\cite{Stewart2012} results from a tournament with
\input{./assets/tex/number_of_stewart_plotkin_strategies/main.tex} strategies,
was presented with specific consideration given to ZD strategies. This
tournament is reproduced here using the Axelrod-Python
project~\cite{Knight2016}. To obtain a good measure of the corresponding
transition rates for each strategy all matches have been run for
\input{assets/tex/number_of_turns/main.tex} turns and every match has been
repeated \input{assets/tex/number_of_repetitions/main.tex} times. All of this
interaction data is available at~\cite{vincent_knight_2018_1297075}. A good
match between the inferred Markov chain and the state distribution of the actual
interactions has been verified. Data for this is presented in the supplementary
materials.

Figure~\ref{fig:SSError_overall_in_stewart_plotkin} shows the \(\text{SSError}\)
values for all the strategies in the tournament, as reported
in~\cite{Stewart2012} the extortionate strategy (which has an expected
\(\text{SSError}\) approximately 0) gains a large number of wins.

\begin{figure}[!htbp]
    \centering
    \includegraphics[width=.8\textwidth]{./assets/img/SSError_overall_in_stewart_plotkin/main.pdf}
    \caption{\(\text{SSError}\) and state probabilities for the strategies
        of~\cite{Stewart2012}, ordered both by number of wins and overall score.
        Note that \(P(DC)\) is not shown as it corresponds to the transpose of
        \(P(CD)\). Cooperator and Defector are omitted as they do not visit all
        the states.}
    \label{fig:SSError_overall_in_stewart_plotkin}
\end{figure}

Here, the work of~\cite{Stewart2012} is extended by investigating a tournament
with \input{assets/tex/number_of_full_strategies/main.tex}
strategies.

The results of this analysis are shown in
Figure~\ref{fig:SSError_and_probabilities_in_full}. The top ranking strategies
by number of wins seem to be extortionate (but not against all strategies) and
it can be seen that a small sub group of strategies achieve mutual defection.
All the top ranking strategies according to score achieve mutual cooperation and
do not extort each other, however they
\textbf{do} exhibit extortionate behaviour towards a number of the lower ranking
strategies.

\begin{figure}[!htbp]
    \centering
    \includegraphics[width=.8\textwidth]{./assets/img/SSError_and_probabilities_in_full/main.pdf}
    \caption{\(\text{SSError}\) for the strategies for the full tournament. Only
    strategy interactions for which \(p_4=0\) and \(\chi>1\) are displayed.}
    \label{fig:SSError_and_probabilities_in_full}
\end{figure}

\section{Conclusion}\label{sec:conclusion}

This work defines an approach to measure whether or not a player is playing a
strategy that corresponds to an extortionate strategy as defined
in~\cite{Press2012}: a mathematical model for suspicion. Indeed, all
extortionate strategies have been
 classified as lying on a triangular plane.
This rigorous classification fails to be robust to small measurement error, thus
a statistical approach is proposed.
This is done through a linear algebraic approach for approximating the solution
of a linear system. Using this, a large number of pairwise interactions is
simulated and in fact very few strategies are found to act extortionately.

The work of~\cite{Press2012}, whilst showing that a clever approach to taking
advantage of another memory one strategy exists: this is incomplete. Whilst the
elegance of this result is very attractive, just as the simplicity of the
victory of Tit For Tat in Axelrod's original tournaments was, it is incomplete.
Extortionate strategies achieve a high number of wins but they do not
achieve a high score which corresponds to the fitness landscape in an
evolutionary sense. From the large number of interactions a payoff matrix \(S\)
can be measured where \(S_{ij}\) denotes the score (using standard values of
\((R, S, T, P) = (3, 0, 5, 1)\)) of the \(i\)th strategy
against the \(j\)th strategy. Using this, the replicator equation
describes the evolution of the system based on a population density fitness
function:

\begin{equation}\label{eqn:replicator_dynamics}
    \frac{dx}{dt} = x(S-x^TS x)
\end{equation}

Equation (\ref{eqn:replicator_dynamics}) is solved numerically through an
integration technique described in~\cite{Petzold1983} and
Figure~\ref{fig:replicator_dynamics} shows the evolution of the distribution of
the system: the various strategies are ranked by scores. It is clear to see that
only the high ranking strategies survive the evolutionary process (in fact,
only \input{./assets/img/replicator_dynamics/main.tex}
have a final distribution greater than \(10 ^ {-2}\)). This confirms the
findings of~\cite{Moran1707} in which sophisticated strategies resist
evolutionary invasion of shorter memory strategies. Recalling
Figure~\ref{fig:SSError_and_probabilities_in_full} this demonstrates that:

\begin{itemize}
    \item Cooperation emerges through the evolutionary process: the high scoring
        strategies do not exhibit extortionate behaviour towards each other.
    \item Extortionate strategies do not survive the evolutionary process.
\end{itemize}

\begin{figure}[!htbp]
    \centering
    \includegraphics[width=.8\textwidth]{./assets/img/replicator_dynamics/main.pdf}
    \caption{Numerical simulation of the replicator equation
    (\ref{eqn:replicator_dynamics}): strategies are ordered by score, only the strategies with a high score survive the evolutionary process.}
    \label{fig:replicator_dynamics}
\end{figure}

This work can be used to classify plays of the IPD\@: data can be collected from
actual interactions (in lab or in the field). Furthermore, this allows for a
classification method similar to the notion of fingerprinting presented
in~\cite{Ashlock2008}. Trained strategies can potentially be classified as
extortionate or not or it could be possible to even constrain the reinforcement
learning approaches that are becoming prevalent in the literature.
Alternatively, this mathematical approach for recognising extortion could be
used in sophisticated strategies to defend against invasion. Arguably, some of
the strategies considered here exhibit this behaviour, indeed as described
in~\cite{Harper2017}, the top ranking strategies in the full tournament are
obtained using evolutionary reinforcement learning techniques, thus, suspicion
of extortionate behaviour could in fact be an evolutionary trait.

\section*{Acknowledgements}

The following open source software libraries were used in this research:

\begin{itemize}
    \item The Axelrod ~\cite{Knight2016, Knight2018} library (IPD strategies and
        tournaments).
    \item The sympy library~\cite{Meurer2017} (verification of all symbolic
        calculations).
    \item The matplotlib~\cite{Droettboom2018} library (visualisation).
    \item The pandas~\cite{Structures2010}, dask~\cite{Dask2016} and
        NumPy~\cite{Oliphant2015} libraries (data manipulation).
    \item The SciPy~\cite{Jones2001} library (numerical integration of the
        replicator equation).
\end{itemize}

This work was performed using the computational facilities of the Advanced
Research Computing @ Cardiff (ARCCA) Division, Cardiff University.

\printbibliography

\newpage
\section*{Supplementary materials}

\includepdf{assets/pdf/proof_of_form_of_extortionate_strategies/main.pdf}

\newpage

Using the pair wise interactions the transition rates \(p,
q\) can be measured and the steady state probabilities inferred and compared to
the actual probabilities of each state.
This is done numerically by computing the singular eigenvector of the
matrix \(A\) \cite{Stewart2009}:

\[
    A =
    \begin{bmatrix}
        p_1 q_1 & p_1 (1 - q_1) & (1 - p_1) q_1 & (1 -p_1) (1 - q_1) \\
        p_2 q_2 & p_2 (1 - q_2) & (1 - p_2) q_2 & (1 -p_2) (1 - q_2) \\
        p_3 q_3 & p_3 (1 - q_3) & (1 - p_3) q_3 & (1 -p_3) (1 - q_3) \\
        p_4 q_4 & p_4 (1 - q_4) & (1 - p_4) q_4 & (1 -p_4) (1 - q_4) \\
    \end{bmatrix}
\]

Figure~\ref{fig:computed_probabilities_vs_theoretic_probabilities} shows a
regression line fitted to every pairwise interaction with a reported
\(\text{SSError}\) value (pairwise interactions with missing states were
omitted). This serves to validate the approach: a part from some edge cases the
relationship is consistent.

\begin{figure}[!htbp]
    \centering
    \includegraphics[width=.8\textwidth]{./assets/img/computed_probabilities_vs_theoretic_probabilities/main.pdf}
    \caption{The
        relationship between the steady state probabilities inferred from the
        measured transitions and the actual steady state probabilities. A linear
        regression line is included validating the approach.}
    \label{fig:computed_probabilities_vs_theoretic_probabilities}
\end{figure}


\end{document}

strategies.

The results of this analysis are shown in
Figure~\ref{fig:SSError_and_probabilities_in_full}. The top ranking strategies
by number of wins seem to be extortionate (but not against all strategies) and
it can be seen that a small sub group of strategies achieve mutual defection.
All the top ranking strategies according to score achieve mutual cooperation and
do not extort each other, however they
\textbf{do} exhibit extortionate behaviour towards a number of the lower ranking
strategies.

\begin{figure}[!htbp]
    \centering
    \includegraphics[width=.8\textwidth]{./assets/img/SSError_and_probabilities_in_full/main.pdf}
    \caption{\(\text{SSError}\) for the strategies for the full tournament. Only
    strategy interactions for which \(p_4=0\) and \(\chi>1\) are displayed.}
    \label{fig:SSError_and_probabilities_in_full}
\end{figure}

\section{Conclusion}\label{sec:conclusion}

This work defines an approach to measure whether or not a player is playing a
strategy that corresponds to an extortionate strategy as defined
in~\cite{Press2012}: a mathematical model for suspicion. Indeed, all
extortionate strategies have been
 classified as lying on a triangular plane.
This rigorous classification fails to be robust to small measurement error, thus
a statistical approach is proposed.
This is done through a linear algebraic approach for approximating the solution
of a linear system. Using this, a large number of pairwise interactions is
simulated and in fact very few strategies are found to act extortionately.

The work of~\cite{Press2012}, whilst showing that a clever approach to taking
advantage of another memory one strategy exists: this is incomplete. Whilst the
elegance of this result is very attractive, just as the simplicity of the
victory of Tit For Tat in Axelrod's original tournaments was, it is incomplete.
Extortionate strategies achieve a high number of wins but they do not
achieve a high score which corresponds to the fitness landscape in an
evolutionary sense. From the large number of interactions a payoff matrix \(S\)
can be measured where \(S_{ij}\) denotes the score (using standard values of
\((R, S, T, P) = (3, 0, 5, 1)\)) of the \(i\)th strategy
against the \(j\)th strategy. Using this, the replicator equation
describes the evolution of the system based on a population density fitness
function:

\begin{equation}\label{eqn:replicator_dynamics}
    \frac{dx}{dt} = x(S-x^TS x)
\end{equation}

Equation (\ref{eqn:replicator_dynamics}) is solved numerically through an
integration technique described in~\cite{Petzold1983} and
Figure~\ref{fig:replicator_dynamics} shows the evolution of the distribution of
the system: the various strategies are ranked by scores. It is clear to see that
only the high ranking strategies survive the evolutionary process (in fact,
only \documentclass[a4paper]{article}

\usepackage{amsmath}
\usepackage{amssymb}
\usepackage[margin=1.5cm,
            includefoot,
            footskip=30pt]{geometry}
\usepackage{layout}
\usepackage{graphicx}
\usepackage{subcaption}

\usepackage{biblatex}
\usepackage{pdfpages}

\bibliography{main.bib}

\title{Suspicion: Recognising and evaluating the effectiveness
       of extortion in the Iterated Prisoner's Dilemma}
\author{Vincent A. Knight \and Nikoleta E. Glynatsi}
\date{\today}



\begin{document}

\maketitle

\begin{abstract}
    The Iterated Prisoner's Dilemma is a model for rational and evolutionary
    interactive behaviour. It has applications both in the study of human social
    behaviour as well as in biology.
    It is used to understand when and how a rational individual might
    accept an immediate cost to their own utility for the direct benefit of
    another.

    Much attention has been given to a class of strategies called
    Zero Determinant strategies. It has been theoretically shown that these
    strategies can ``extort'' any player.

    In this work, an approach to identify if observed strategies are playing in
    an extortionate way is described. Furthermore, experimental analysis of
    a large tournament with \input{assets/tex/number_of_full_strategies/main.tex}
    strategies is considered. In this setting
    the most highly performing strategies do not play in an extortionate way
    against each other but do against lower performing strategies.
    This suggests that whilst the theory of Zero Determinant strategies
    indicates that memory is not of fundamental importance to the evolution of
    cooperative behaviour, this is incomplete.
\end{abstract}

\section{Introduction}\label{sec:introduction}

Agent based game theoretic models have become a stalwart of the underpinning
mathematics of interactive behaviours. One of the major pieces of work
in this area is the pair of original computer tournaments run by Robert
Axelrod~\cite{Axelrod1980, Axelrod1980a}. These tournaments pitted submitted
computer strategies against each other in plays of the Iterated Prisoner's
Dilemma. A common game where agents can choose to pay a slight cost to their
immediate utility in the hope of building a reputation. This has been used in
economic and evolutionary game theory to understand the evolution of cooperative
behaviour.

Recently, a class of strategies was described in~\cite{Press2012} that can
provably extort any given opponent. In~\cite{Hilbe2013, Moran1707} some
questions have already been asked about the true effectiveness of these
strategies in an evolutionary setting. Here another question is asked: is it
possible to recognise this extortionate behaviour? A mathematical procedure for
suspicion is presented: in the same way that the continued actions of an
extortionate individual might raise suspicion.

This work makes use of the Axelrod Python library~\cite{Knight2018, Knight2016}
with a large number of Prisoner Dilemma strategies available to give an
extensive numerical example of the ideas presented.  The approach is presented
in Section~\ref{sec:delta-zd-strategies}.  All of the code and data discussed
in Section~\ref{sec:numerical-experiments} is open sourced, archived and
written according to best scientific principles~\cite{Wilson2014}. The data
archive can be found at~\cite{vincent_knight_2018_1297075}.

\section{Recognising Extortion}\label{sec:delta-zd-strategies}

In~\cite{Press2012}, given a match between 2 memory-one strategies, the concept
of Zero Determinant (ZD) strategies is introduced. The main result of that paper
shows that given two memory one players \(p, q\in\mathbb{R}^4\) a linear
relationship between the players' scores could be forced by one of the players.

Using the notation of~\cite{Press2012}, assuming the utilities for player \(p\)
are given by \(S_x=(R, S, T, P)\) and for player \(q\) by \(S_y=(R, T, S, P)\)
and that the stationary scores of each player is given by \(S_X\) and \(S_Y\)
respectively. The main result of~\cite{Press2012} is that if

\begin{equation}\label{eqn:linear_relationship_for_p}
    \tilde p=\alpha S_x + \beta S_y + \gamma
\end{equation}

or

\begin{equation}\label{eqn:linear_relationship_for_q}
    \tilde q=\alpha S_x + \beta S_y + \gamma
\end{equation}

where \(\tilde p = (1 - p_1, 1 - p_2, p_3, p_4)\) and
\(\tilde q = (1 - q_1, 1 - q_2, q_3, q_4)\) then:

\begin{equation}
    \alpha S_X + \beta S_Y + \gamma = 0
\end{equation}

In~\cite{Press2012} a particular type of ZD strategy is defined: extortionate
strategies. If:

\begin{equation}\label{eqn:constraint_for_extortion}
    \gamma = - P(\alpha + \beta)
\end{equation}

then the player can ensure they get a score \(\chi\) times
larger than the opponent. This extortion coefficient is given by:

\begin{equation}\label{eqn:definition_of_chi}
    \chi=\frac{-\beta}{\alpha}
\end{equation}

Thus, if (\ref{eqn:constraint_for_extortion}) holds and \(\chi >1\) a player is
said to extort their opponent.
Here, the reverse problem is considered: given a
\(p\in\mathbb{R}^4\) how does one identify \(\alpha, \beta\) if they
exist and is the strategy in fact acting in an extortionate way?

These conditions correspond to:

\begin{align}
    \tilde p_1 & = \alpha R + \beta R - P (\alpha + \beta)
            \label{eqn:condition_for_tilde_p1}\\
    \tilde p_2 & = \alpha S + \beta T - P (\alpha + \beta)
            \label{eqn:condition_for_tilde_p2}\\
    \tilde p_3 & = \alpha T + \beta S - P (\alpha + \beta)
            \label{eqn:condition_for_tilde_p3}\\
    \tilde p_4 & = \alpha P + \beta P - P (\alpha + \beta)
            \label{eqn:condition_for_tilde_p4}
\end{align}

Equation (\ref{eqn:condition_for_tilde_p4}) ensures that \(p_4=\tilde p_4=0\).
Equations (\ref{eqn:condition_for_tilde_p1}-\ref{eqn:condition_for_tilde_p3})
can be used to eliminate \(\alpha, \beta\), giving:

\begin{equation}\label{eqn:planar_definition_of_extortion}
    \tilde p_1 = \frac{(R - P)(\tilde p_2 + \tilde p_3)}{S + T - 2P}
\end{equation}

with:

\begin{equation}\label{eqn:definition_of_chi}
    \chi = \frac{\tilde p_2 (P - T) + \tilde p_3 (S - P)}
                {\tilde p_2 (P - S) + \tilde p_3 (T - P)}
\end{equation}

Given a strategy \(p\in\mathbb{R}^{4\times 1}\) equations
(\ref{eqn:condition_for_tilde_p4}), (\ref{eqn:planar_definition_of_extortion}-\ref{eqn:definition_of_chi}) can be used to check if
a strategy is extortionate. The conditions correspond to:

\begin{align}
    p_1 & = \frac{(R-P)(p_2 + p_3) - R + T + S - P}{S + T - 2P}
     \label{eqn:condition_for_p1}\\
    p_4 & = 0 \label{eqn:condition_for_p4}\\
    1 & > p_2 + p_3\label{eqn:condition_for_chi}
\end{align}

The algebraic steps necessary to prove these results are available in the
supporting materials.

All extortionate strategies reside on a triangular (\ref{eqn:condition_for_chi})
plane (\ref{eqn:condition_for_p1}) in 3 dimensions (\ref{eqn:condition_for_p4}).
Using this formulation it can be seen that a necessary (but not sufficient)
condition for an extortionate strategy is that it cooperates on average less
than 50\% of the time when in a state of disagreement with the opponent.

As an example, consider the known extortionate strategy \(p=(8 / 9, 1 / 2, 1 /
3, 0)\) from~\cite{Stewart2012} which is referred to as \texttt{Extort-2}. In
this case, for the standard values of \((R, T, S, P)\) constraint
(\ref{eqn:condition_for_p1}) corresponds to:

\begin{equation}
    p_1 = \frac{2(p_2 + p_3) + 1}{3}
\end{equation}

It is clear that in this case all constraints hold.

This approach could in fact be used to confirm that a given strategy is acting
in an extortionate manner even if it is not a memory one strategy. However, in
practice, if a closed form for \(p\) is not known, then due to measurement
and/or numerical error this would not work.

This problem can be written in the following linear algebraic form where
\(x=(\alpha, \beta)\)
and \(p^*=(\tilde p_1 - 1, tilde_2 - 1, p_3)\):

\begin{equation}\label{eqn:linear_algebraic_equation_for_p}
    Cx= p^*
\end{equation}

\(C\) corresponds to equations
(\ref{eqn:condition_for_tilde_p1}-\ref{eqn:condition_for_tilde_p3}) and is
given by:

\begin{equation}\label{eqn:definition_of_C}
    C =
    \begin{bmatrix}
        R - P & R- P \\
        S - P & T- P \\
        T - P & S- P \\
    \end{bmatrix}
\end{equation}

Note that in general, equation (\ref{eqn:linear_algebraic_equation_for_p}) will
not necessarily have a solution. From the Rouch\'{e}-Capelli theorem if there is
a solution it is unique as \(\text{rank}(C)=2\) which is the dimension of the
variable \(x\). The best fitting \(x\) is found by minimizing:

\begin{equation}\label{eqn:r_squared}
    \text{SSError} = \|C x- p^*\|_2^2 = \sum_{i=1}^{3}\left((C\bar x)_i-p_i^*\right)^2
\end{equation}

Note that \(\text{SSError}\), which is the square of the Frobenius
norm~\cite{Golub2013}, becomes a measure of how close a strategy is to being an
extortionate strategy. Suspicion
of extortion then corresponds to a threshold on \(\text{SSError}\).

By observing interactions (human or otherwise), their memory one representation
can be inferred and this approach can be used to recognise extortionate
behaviour. The notion of comparing theoretic and actual plays of the IPD is not
novel, see for example~\cite{Rand2013}. Immediately it is noted that if the
environment is noisy~\cite{Wu1995} then no strategy can be considered to be
extortionate as \(p_4>0\).

In the next section, this idea will be illustrated by observing the interactions
that take place in a computer based tournament of the IPD\@.

\section{Numerical experiments}\label{sec:numerical-experiments}

In~\cite{Stewart2012} results from a tournament with
\input{./assets/tex/number_of_stewart_plotkin_strategies/main.tex} strategies,
was presented with specific consideration given to ZD strategies. This
tournament is reproduced here using the Axelrod-Python
project~\cite{Knight2016}. To obtain a good measure of the corresponding
transition rates for each strategy all matches have been run for
\input{assets/tex/number_of_turns/main.tex} turns and every match has been
repeated \input{assets/tex/number_of_repetitions/main.tex} times. All of this
interaction data is available at~\cite{vincent_knight_2018_1297075}. A good
match between the inferred Markov chain and the state distribution of the actual
interactions has been verified. Data for this is presented in the supplementary
materials.

Figure~\ref{fig:SSError_overall_in_stewart_plotkin} shows the \(\text{SSError}\)
values for all the strategies in the tournament, as reported
in~\cite{Stewart2012} the extortionate strategy (which has an expected
\(\text{SSError}\) approximately 0) gains a large number of wins.

\begin{figure}[!htbp]
    \centering
    \includegraphics[width=.8\textwidth]{./assets/img/SSError_overall_in_stewart_plotkin/main.pdf}
    \caption{\(\text{SSError}\) and state probabilities for the strategies
        of~\cite{Stewart2012}, ordered both by number of wins and overall score.
        Note that \(P(DC)\) is not shown as it corresponds to the transpose of
        \(P(CD)\). Cooperator and Defector are omitted as they do not visit all
        the states.}
    \label{fig:SSError_overall_in_stewart_plotkin}
\end{figure}

Here, the work of~\cite{Stewart2012} is extended by investigating a tournament
with \input{assets/tex/number_of_full_strategies/main.tex}
strategies.

The results of this analysis are shown in
Figure~\ref{fig:SSError_and_probabilities_in_full}. The top ranking strategies
by number of wins seem to be extortionate (but not against all strategies) and
it can be seen that a small sub group of strategies achieve mutual defection.
All the top ranking strategies according to score achieve mutual cooperation and
do not extort each other, however they
\textbf{do} exhibit extortionate behaviour towards a number of the lower ranking
strategies.

\begin{figure}[!htbp]
    \centering
    \includegraphics[width=.8\textwidth]{./assets/img/SSError_and_probabilities_in_full/main.pdf}
    \caption{\(\text{SSError}\) for the strategies for the full tournament. Only
    strategy interactions for which \(p_4=0\) and \(\chi>1\) are displayed.}
    \label{fig:SSError_and_probabilities_in_full}
\end{figure}

\section{Conclusion}\label{sec:conclusion}

This work defines an approach to measure whether or not a player is playing a
strategy that corresponds to an extortionate strategy as defined
in~\cite{Press2012}: a mathematical model for suspicion. Indeed, all
extortionate strategies have been
 classified as lying on a triangular plane.
This rigorous classification fails to be robust to small measurement error, thus
a statistical approach is proposed.
This is done through a linear algebraic approach for approximating the solution
of a linear system. Using this, a large number of pairwise interactions is
simulated and in fact very few strategies are found to act extortionately.

The work of~\cite{Press2012}, whilst showing that a clever approach to taking
advantage of another memory one strategy exists: this is incomplete. Whilst the
elegance of this result is very attractive, just as the simplicity of the
victory of Tit For Tat in Axelrod's original tournaments was, it is incomplete.
Extortionate strategies achieve a high number of wins but they do not
achieve a high score which corresponds to the fitness landscape in an
evolutionary sense. From the large number of interactions a payoff matrix \(S\)
can be measured where \(S_{ij}\) denotes the score (using standard values of
\((R, S, T, P) = (3, 0, 5, 1)\)) of the \(i\)th strategy
against the \(j\)th strategy. Using this, the replicator equation
describes the evolution of the system based on a population density fitness
function:

\begin{equation}\label{eqn:replicator_dynamics}
    \frac{dx}{dt} = x(S-x^TS x)
\end{equation}

Equation (\ref{eqn:replicator_dynamics}) is solved numerically through an
integration technique described in~\cite{Petzold1983} and
Figure~\ref{fig:replicator_dynamics} shows the evolution of the distribution of
the system: the various strategies are ranked by scores. It is clear to see that
only the high ranking strategies survive the evolutionary process (in fact,
only \input{./assets/img/replicator_dynamics/main.tex}
have a final distribution greater than \(10 ^ {-2}\)). This confirms the
findings of~\cite{Moran1707} in which sophisticated strategies resist
evolutionary invasion of shorter memory strategies. Recalling
Figure~\ref{fig:SSError_and_probabilities_in_full} this demonstrates that:

\begin{itemize}
    \item Cooperation emerges through the evolutionary process: the high scoring
        strategies do not exhibit extortionate behaviour towards each other.
    \item Extortionate strategies do not survive the evolutionary process.
\end{itemize}

\begin{figure}[!htbp]
    \centering
    \includegraphics[width=.8\textwidth]{./assets/img/replicator_dynamics/main.pdf}
    \caption{Numerical simulation of the replicator equation
    (\ref{eqn:replicator_dynamics}): strategies are ordered by score, only the strategies with a high score survive the evolutionary process.}
    \label{fig:replicator_dynamics}
\end{figure}

This work can be used to classify plays of the IPD\@: data can be collected from
actual interactions (in lab or in the field). Furthermore, this allows for a
classification method similar to the notion of fingerprinting presented
in~\cite{Ashlock2008}. Trained strategies can potentially be classified as
extortionate or not or it could be possible to even constrain the reinforcement
learning approaches that are becoming prevalent in the literature.
Alternatively, this mathematical approach for recognising extortion could be
used in sophisticated strategies to defend against invasion. Arguably, some of
the strategies considered here exhibit this behaviour, indeed as described
in~\cite{Harper2017}, the top ranking strategies in the full tournament are
obtained using evolutionary reinforcement learning techniques, thus, suspicion
of extortionate behaviour could in fact be an evolutionary trait.

\section*{Acknowledgements}

The following open source software libraries were used in this research:

\begin{itemize}
    \item The Axelrod ~\cite{Knight2016, Knight2018} library (IPD strategies and
        tournaments).
    \item The sympy library~\cite{Meurer2017} (verification of all symbolic
        calculations).
    \item The matplotlib~\cite{Droettboom2018} library (visualisation).
    \item The pandas~\cite{Structures2010}, dask~\cite{Dask2016} and
        NumPy~\cite{Oliphant2015} libraries (data manipulation).
    \item The SciPy~\cite{Jones2001} library (numerical integration of the
        replicator equation).
\end{itemize}

This work was performed using the computational facilities of the Advanced
Research Computing @ Cardiff (ARCCA) Division, Cardiff University.

\printbibliography

\newpage
\section*{Supplementary materials}

\includepdf{assets/pdf/proof_of_form_of_extortionate_strategies/main.pdf}

\newpage

Using the pair wise interactions the transition rates \(p,
q\) can be measured and the steady state probabilities inferred and compared to
the actual probabilities of each state.
This is done numerically by computing the singular eigenvector of the
matrix \(A\) \cite{Stewart2009}:

\[
    A =
    \begin{bmatrix}
        p_1 q_1 & p_1 (1 - q_1) & (1 - p_1) q_1 & (1 -p_1) (1 - q_1) \\
        p_2 q_2 & p_2 (1 - q_2) & (1 - p_2) q_2 & (1 -p_2) (1 - q_2) \\
        p_3 q_3 & p_3 (1 - q_3) & (1 - p_3) q_3 & (1 -p_3) (1 - q_3) \\
        p_4 q_4 & p_4 (1 - q_4) & (1 - p_4) q_4 & (1 -p_4) (1 - q_4) \\
    \end{bmatrix}
\]

Figure~\ref{fig:computed_probabilities_vs_theoretic_probabilities} shows a
regression line fitted to every pairwise interaction with a reported
\(\text{SSError}\) value (pairwise interactions with missing states were
omitted). This serves to validate the approach: a part from some edge cases the
relationship is consistent.

\begin{figure}[!htbp]
    \centering
    \includegraphics[width=.8\textwidth]{./assets/img/computed_probabilities_vs_theoretic_probabilities/main.pdf}
    \caption{The
        relationship between the steady state probabilities inferred from the
        measured transitions and the actual steady state probabilities. A linear
        regression line is included validating the approach.}
    \label{fig:computed_probabilities_vs_theoretic_probabilities}
\end{figure}


\end{document}

have a final distribution greater than \(10 ^ {-2}\)). This confirms the
findings of~\cite{Moran1707} in which sophisticated strategies resist
evolutionary invasion of shorter memory strategies. Recalling
Figure~\ref{fig:SSError_and_probabilities_in_full} this demonstrates that:

\begin{itemize}
    \item Cooperation emerges through the evolutionary process: the high scoring
        strategies do not exhibit extortionate behaviour towards each other.
    \item Extortionate strategies do not survive the evolutionary process.
\end{itemize}

\begin{figure}[!htbp]
    \centering
    \includegraphics[width=.8\textwidth]{./assets/img/replicator_dynamics/main.pdf}
    \caption{Numerical simulation of the replicator equation
    (\ref{eqn:replicator_dynamics}): strategies are ordered by score, only the strategies with a high score survive the evolutionary process.}
    \label{fig:replicator_dynamics}
\end{figure}

This work can be used to classify plays of the IPD\@: data can be collected from
actual interactions (in lab or in the field). Furthermore, this allows for a
classification method similar to the notion of fingerprinting presented
in~\cite{Ashlock2008}. Trained strategies can potentially be classified as
extortionate or not or it could be possible to even constrain the reinforcement
learning approaches that are becoming prevalent in the literature.
Alternatively, this mathematical approach for recognising extortion could be
used in sophisticated strategies to defend against invasion. Arguably, some of
the strategies considered here exhibit this behaviour, indeed as described
in~\cite{Harper2017}, the top ranking strategies in the full tournament are
obtained using evolutionary reinforcement learning techniques, thus, suspicion
of extortionate behaviour could in fact be an evolutionary trait.

\section*{Acknowledgements}

The following open source software libraries were used in this research:

\begin{itemize}
    \item The Axelrod ~\cite{Knight2016, Knight2018} library (IPD strategies and
        tournaments).
    \item The sympy library~\cite{Meurer2017} (verification of all symbolic
        calculations).
    \item The matplotlib~\cite{Droettboom2018} library (visualisation).
    \item The pandas~\cite{Structures2010}, dask~\cite{Dask2016} and
        NumPy~\cite{Oliphant2015} libraries (data manipulation).
    \item The SciPy~\cite{Jones2001} library (numerical integration of the
        replicator equation).
\end{itemize}

This work was performed using the computational facilities of the Advanced
Research Computing @ Cardiff (ARCCA) Division, Cardiff University.

\printbibliography

\newpage
\section*{Supplementary materials}

\includepdf{assets/pdf/proof_of_form_of_extortionate_strategies/main.pdf}

\newpage

Using the pair wise interactions the transition rates \(p,
q\) can be measured and the steady state probabilities inferred and compared to
the actual probabilities of each state.
This is done numerically by computing the singular eigenvector of the
matrix \(A\) \cite{Stewart2009}:

\[
    A =
    \begin{bmatrix}
        p_1 q_1 & p_1 (1 - q_1) & (1 - p_1) q_1 & (1 -p_1) (1 - q_1) \\
        p_2 q_2 & p_2 (1 - q_2) & (1 - p_2) q_2 & (1 -p_2) (1 - q_2) \\
        p_3 q_3 & p_3 (1 - q_3) & (1 - p_3) q_3 & (1 -p_3) (1 - q_3) \\
        p_4 q_4 & p_4 (1 - q_4) & (1 - p_4) q_4 & (1 -p_4) (1 - q_4) \\
    \end{bmatrix}
\]

Figure~\ref{fig:computed_probabilities_vs_theoretic_probabilities} shows a
regression line fitted to every pairwise interaction with a reported
\(\text{SSError}\) value (pairwise interactions with missing states were
omitted). This serves to validate the approach: a part from some edge cases the
relationship is consistent.

\begin{figure}[!htbp]
    \centering
    \includegraphics[width=.8\textwidth]{./assets/img/computed_probabilities_vs_theoretic_probabilities/main.pdf}
    \caption{The
        relationship between the steady state probabilities inferred from the
        measured transitions and the actual steady state probabilities. A linear
        regression line is included validating the approach.}
    \label{fig:computed_probabilities_vs_theoretic_probabilities}
\end{figure}


\end{document}

strategies.

The results of this analysis are shown in
Figure~\ref{fig:SSError_and_probabilities_in_full}. The top ranking strategies
by number of wins seem to be extortionate (but not against all strategies) and
it can be seen that a small sub group of strategies achieve mutual defection.
All the top ranking strategies according to score achieve mutual cooperation and
do not extort each other, however they
\textbf{do} exhibit extortionate behaviour towards a number of the lower ranking
strategies.

\begin{figure}[!htbp]
    \centering
    \includegraphics[width=.8\textwidth]{./assets/img/SSError_and_probabilities_in_full/main.pdf}
    \caption{\(\text{SSError}\) for the strategies for the full tournament. Only
    strategy interactions for which \(p_4=0\) and \(\chi>1\) are displayed.}
    \label{fig:SSError_and_probabilities_in_full}
\end{figure}

\section{Conclusion}\label{sec:conclusion}

This work defines an approach to measure whether or not a player is playing a
strategy that corresponds to an extortionate strategy as defined
in~\cite{Press2012}: a mathematical model for suspicion. Indeed, all
extortionate strategies have been
 classified as lying on a triangular plane.
This rigorous classification fails to be robust to small measurement error, thus
a statistical approach is proposed.
This is done through a linear algebraic approach for approximating the solution
of a linear system. Using this, a large number of pairwise interactions is
simulated and in fact very few strategies are found to act extortionately.

The work of~\cite{Press2012}, whilst showing that a clever approach to taking
advantage of another memory one strategy exists: this is incomplete. Whilst the
elegance of this result is very attractive, just as the simplicity of the
victory of Tit For Tat in Axelrod's original tournaments was, it is incomplete.
Extortionate strategies achieve a high number of wins but they do not
achieve a high score which corresponds to the fitness landscape in an
evolutionary sense. From the large number of interactions a payoff matrix \(S\)
can be measured where \(S_{ij}\) denotes the score (using standard values of
\((R, S, T, P) = (3, 0, 5, 1)\)) of the \(i\)th strategy
against the \(j\)th strategy. Using this, the replicator equation
describes the evolution of the system based on a population density fitness
function:

\begin{equation}\label{eqn:replicator_dynamics}
    \frac{dx}{dt} = x(S-x^TS x)
\end{equation}

Equation (\ref{eqn:replicator_dynamics}) is solved numerically through an
integration technique described in~\cite{Petzold1983} and
Figure~\ref{fig:replicator_dynamics} shows the evolution of the distribution of
the system: the various strategies are ranked by scores. It is clear to see that
only the high ranking strategies survive the evolutionary process (in fact,
only \documentclass[a4paper]{article}

\usepackage{amsmath}
\usepackage{amssymb}
\usepackage[margin=1.5cm,
            includefoot,
            footskip=30pt]{geometry}
\usepackage{layout}
\usepackage{graphicx}
\usepackage{subcaption}

\usepackage{biblatex}
\usepackage{pdfpages}

\bibliography{main.bib}

\title{Suspicion: Recognising and evaluating the effectiveness
       of extortion in the Iterated Prisoner's Dilemma}
\author{Vincent A. Knight \and Nikoleta E. Glynatsi}
\date{\today}



\begin{document}

\maketitle

\begin{abstract}
    The Iterated Prisoner's Dilemma is a model for rational and evolutionary
    interactive behaviour. It has applications both in the study of human social
    behaviour as well as in biology.
    It is used to understand when and how a rational individual might
    accept an immediate cost to their own utility for the direct benefit of
    another.

    Much attention has been given to a class of strategies called
    Zero Determinant strategies. It has been theoretically shown that these
    strategies can ``extort'' any player.

    In this work, an approach to identify if observed strategies are playing in
    an extortionate way is described. Furthermore, experimental analysis of
    a large tournament with \documentclass[a4paper]{article}

\usepackage{amsmath}
\usepackage{amssymb}
\usepackage[margin=1.5cm,
            includefoot,
            footskip=30pt]{geometry}
\usepackage{layout}
\usepackage{graphicx}
\usepackage{subcaption}

\usepackage{biblatex}
\usepackage{pdfpages}

\bibliography{main.bib}

\title{Suspicion: Recognising and evaluating the effectiveness
       of extortion in the Iterated Prisoner's Dilemma}
\author{Vincent A. Knight \and Nikoleta E. Glynatsi}
\date{\today}



\begin{document}

\maketitle

\begin{abstract}
    The Iterated Prisoner's Dilemma is a model for rational and evolutionary
    interactive behaviour. It has applications both in the study of human social
    behaviour as well as in biology.
    It is used to understand when and how a rational individual might
    accept an immediate cost to their own utility for the direct benefit of
    another.

    Much attention has been given to a class of strategies called
    Zero Determinant strategies. It has been theoretically shown that these
    strategies can ``extort'' any player.

    In this work, an approach to identify if observed strategies are playing in
    an extortionate way is described. Furthermore, experimental analysis of
    a large tournament with \input{assets/tex/number_of_full_strategies/main.tex}
    strategies is considered. In this setting
    the most highly performing strategies do not play in an extortionate way
    against each other but do against lower performing strategies.
    This suggests that whilst the theory of Zero Determinant strategies
    indicates that memory is not of fundamental importance to the evolution of
    cooperative behaviour, this is incomplete.
\end{abstract}

\section{Introduction}\label{sec:introduction}

Agent based game theoretic models have become a stalwart of the underpinning
mathematics of interactive behaviours. One of the major pieces of work
in this area is the pair of original computer tournaments run by Robert
Axelrod~\cite{Axelrod1980, Axelrod1980a}. These tournaments pitted submitted
computer strategies against each other in plays of the Iterated Prisoner's
Dilemma. A common game where agents can choose to pay a slight cost to their
immediate utility in the hope of building a reputation. This has been used in
economic and evolutionary game theory to understand the evolution of cooperative
behaviour.

Recently, a class of strategies was described in~\cite{Press2012} that can
provably extort any given opponent. In~\cite{Hilbe2013, Moran1707} some
questions have already been asked about the true effectiveness of these
strategies in an evolutionary setting. Here another question is asked: is it
possible to recognise this extortionate behaviour? A mathematical procedure for
suspicion is presented: in the same way that the continued actions of an
extortionate individual might raise suspicion.

This work makes use of the Axelrod Python library~\cite{Knight2018, Knight2016}
with a large number of Prisoner Dilemma strategies available to give an
extensive numerical example of the ideas presented.  The approach is presented
in Section~\ref{sec:delta-zd-strategies}.  All of the code and data discussed
in Section~\ref{sec:numerical-experiments} is open sourced, archived and
written according to best scientific principles~\cite{Wilson2014}. The data
archive can be found at~\cite{vincent_knight_2018_1297075}.

\section{Recognising Extortion}\label{sec:delta-zd-strategies}

In~\cite{Press2012}, given a match between 2 memory-one strategies, the concept
of Zero Determinant (ZD) strategies is introduced. The main result of that paper
shows that given two memory one players \(p, q\in\mathbb{R}^4\) a linear
relationship between the players' scores could be forced by one of the players.

Using the notation of~\cite{Press2012}, assuming the utilities for player \(p\)
are given by \(S_x=(R, S, T, P)\) and for player \(q\) by \(S_y=(R, T, S, P)\)
and that the stationary scores of each player is given by \(S_X\) and \(S_Y\)
respectively. The main result of~\cite{Press2012} is that if

\begin{equation}\label{eqn:linear_relationship_for_p}
    \tilde p=\alpha S_x + \beta S_y + \gamma
\end{equation}

or

\begin{equation}\label{eqn:linear_relationship_for_q}
    \tilde q=\alpha S_x + \beta S_y + \gamma
\end{equation}

where \(\tilde p = (1 - p_1, 1 - p_2, p_3, p_4)\) and
\(\tilde q = (1 - q_1, 1 - q_2, q_3, q_4)\) then:

\begin{equation}
    \alpha S_X + \beta S_Y + \gamma = 0
\end{equation}

In~\cite{Press2012} a particular type of ZD strategy is defined: extortionate
strategies. If:

\begin{equation}\label{eqn:constraint_for_extortion}
    \gamma = - P(\alpha + \beta)
\end{equation}

then the player can ensure they get a score \(\chi\) times
larger than the opponent. This extortion coefficient is given by:

\begin{equation}\label{eqn:definition_of_chi}
    \chi=\frac{-\beta}{\alpha}
\end{equation}

Thus, if (\ref{eqn:constraint_for_extortion}) holds and \(\chi >1\) a player is
said to extort their opponent.
Here, the reverse problem is considered: given a
\(p\in\mathbb{R}^4\) how does one identify \(\alpha, \beta\) if they
exist and is the strategy in fact acting in an extortionate way?

These conditions correspond to:

\begin{align}
    \tilde p_1 & = \alpha R + \beta R - P (\alpha + \beta)
            \label{eqn:condition_for_tilde_p1}\\
    \tilde p_2 & = \alpha S + \beta T - P (\alpha + \beta)
            \label{eqn:condition_for_tilde_p2}\\
    \tilde p_3 & = \alpha T + \beta S - P (\alpha + \beta)
            \label{eqn:condition_for_tilde_p3}\\
    \tilde p_4 & = \alpha P + \beta P - P (\alpha + \beta)
            \label{eqn:condition_for_tilde_p4}
\end{align}

Equation (\ref{eqn:condition_for_tilde_p4}) ensures that \(p_4=\tilde p_4=0\).
Equations (\ref{eqn:condition_for_tilde_p1}-\ref{eqn:condition_for_tilde_p3})
can be used to eliminate \(\alpha, \beta\), giving:

\begin{equation}\label{eqn:planar_definition_of_extortion}
    \tilde p_1 = \frac{(R - P)(\tilde p_2 + \tilde p_3)}{S + T - 2P}
\end{equation}

with:

\begin{equation}\label{eqn:definition_of_chi}
    \chi = \frac{\tilde p_2 (P - T) + \tilde p_3 (S - P)}
                {\tilde p_2 (P - S) + \tilde p_3 (T - P)}
\end{equation}

Given a strategy \(p\in\mathbb{R}^{4\times 1}\) equations
(\ref{eqn:condition_for_tilde_p4}), (\ref{eqn:planar_definition_of_extortion}-\ref{eqn:definition_of_chi}) can be used to check if
a strategy is extortionate. The conditions correspond to:

\begin{align}
    p_1 & = \frac{(R-P)(p_2 + p_3) - R + T + S - P}{S + T - 2P}
     \label{eqn:condition_for_p1}\\
    p_4 & = 0 \label{eqn:condition_for_p4}\\
    1 & > p_2 + p_3\label{eqn:condition_for_chi}
\end{align}

The algebraic steps necessary to prove these results are available in the
supporting materials.

All extortionate strategies reside on a triangular (\ref{eqn:condition_for_chi})
plane (\ref{eqn:condition_for_p1}) in 3 dimensions (\ref{eqn:condition_for_p4}).
Using this formulation it can be seen that a necessary (but not sufficient)
condition for an extortionate strategy is that it cooperates on average less
than 50\% of the time when in a state of disagreement with the opponent.

As an example, consider the known extortionate strategy \(p=(8 / 9, 1 / 2, 1 /
3, 0)\) from~\cite{Stewart2012} which is referred to as \texttt{Extort-2}. In
this case, for the standard values of \((R, T, S, P)\) constraint
(\ref{eqn:condition_for_p1}) corresponds to:

\begin{equation}
    p_1 = \frac{2(p_2 + p_3) + 1}{3}
\end{equation}

It is clear that in this case all constraints hold.

This approach could in fact be used to confirm that a given strategy is acting
in an extortionate manner even if it is not a memory one strategy. However, in
practice, if a closed form for \(p\) is not known, then due to measurement
and/or numerical error this would not work.

This problem can be written in the following linear algebraic form where
\(x=(\alpha, \beta)\)
and \(p^*=(\tilde p_1 - 1, tilde_2 - 1, p_3)\):

\begin{equation}\label{eqn:linear_algebraic_equation_for_p}
    Cx= p^*
\end{equation}

\(C\) corresponds to equations
(\ref{eqn:condition_for_tilde_p1}-\ref{eqn:condition_for_tilde_p3}) and is
given by:

\begin{equation}\label{eqn:definition_of_C}
    C =
    \begin{bmatrix}
        R - P & R- P \\
        S - P & T- P \\
        T - P & S- P \\
    \end{bmatrix}
\end{equation}

Note that in general, equation (\ref{eqn:linear_algebraic_equation_for_p}) will
not necessarily have a solution. From the Rouch\'{e}-Capelli theorem if there is
a solution it is unique as \(\text{rank}(C)=2\) which is the dimension of the
variable \(x\). The best fitting \(x\) is found by minimizing:

\begin{equation}\label{eqn:r_squared}
    \text{SSError} = \|C x- p^*\|_2^2 = \sum_{i=1}^{3}\left((C\bar x)_i-p_i^*\right)^2
\end{equation}

Note that \(\text{SSError}\), which is the square of the Frobenius
norm~\cite{Golub2013}, becomes a measure of how close a strategy is to being an
extortionate strategy. Suspicion
of extortion then corresponds to a threshold on \(\text{SSError}\).

By observing interactions (human or otherwise), their memory one representation
can be inferred and this approach can be used to recognise extortionate
behaviour. The notion of comparing theoretic and actual plays of the IPD is not
novel, see for example~\cite{Rand2013}. Immediately it is noted that if the
environment is noisy~\cite{Wu1995} then no strategy can be considered to be
extortionate as \(p_4>0\).

In the next section, this idea will be illustrated by observing the interactions
that take place in a computer based tournament of the IPD\@.

\section{Numerical experiments}\label{sec:numerical-experiments}

In~\cite{Stewart2012} results from a tournament with
\input{./assets/tex/number_of_stewart_plotkin_strategies/main.tex} strategies,
was presented with specific consideration given to ZD strategies. This
tournament is reproduced here using the Axelrod-Python
project~\cite{Knight2016}. To obtain a good measure of the corresponding
transition rates for each strategy all matches have been run for
\input{assets/tex/number_of_turns/main.tex} turns and every match has been
repeated \input{assets/tex/number_of_repetitions/main.tex} times. All of this
interaction data is available at~\cite{vincent_knight_2018_1297075}. A good
match between the inferred Markov chain and the state distribution of the actual
interactions has been verified. Data for this is presented in the supplementary
materials.

Figure~\ref{fig:SSError_overall_in_stewart_plotkin} shows the \(\text{SSError}\)
values for all the strategies in the tournament, as reported
in~\cite{Stewart2012} the extortionate strategy (which has an expected
\(\text{SSError}\) approximately 0) gains a large number of wins.

\begin{figure}[!htbp]
    \centering
    \includegraphics[width=.8\textwidth]{./assets/img/SSError_overall_in_stewart_plotkin/main.pdf}
    \caption{\(\text{SSError}\) and state probabilities for the strategies
        of~\cite{Stewart2012}, ordered both by number of wins and overall score.
        Note that \(P(DC)\) is not shown as it corresponds to the transpose of
        \(P(CD)\). Cooperator and Defector are omitted as they do not visit all
        the states.}
    \label{fig:SSError_overall_in_stewart_plotkin}
\end{figure}

Here, the work of~\cite{Stewart2012} is extended by investigating a tournament
with \input{assets/tex/number_of_full_strategies/main.tex}
strategies.

The results of this analysis are shown in
Figure~\ref{fig:SSError_and_probabilities_in_full}. The top ranking strategies
by number of wins seem to be extortionate (but not against all strategies) and
it can be seen that a small sub group of strategies achieve mutual defection.
All the top ranking strategies according to score achieve mutual cooperation and
do not extort each other, however they
\textbf{do} exhibit extortionate behaviour towards a number of the lower ranking
strategies.

\begin{figure}[!htbp]
    \centering
    \includegraphics[width=.8\textwidth]{./assets/img/SSError_and_probabilities_in_full/main.pdf}
    \caption{\(\text{SSError}\) for the strategies for the full tournament. Only
    strategy interactions for which \(p_4=0\) and \(\chi>1\) are displayed.}
    \label{fig:SSError_and_probabilities_in_full}
\end{figure}

\section{Conclusion}\label{sec:conclusion}

This work defines an approach to measure whether or not a player is playing a
strategy that corresponds to an extortionate strategy as defined
in~\cite{Press2012}: a mathematical model for suspicion. Indeed, all
extortionate strategies have been
 classified as lying on a triangular plane.
This rigorous classification fails to be robust to small measurement error, thus
a statistical approach is proposed.
This is done through a linear algebraic approach for approximating the solution
of a linear system. Using this, a large number of pairwise interactions is
simulated and in fact very few strategies are found to act extortionately.

The work of~\cite{Press2012}, whilst showing that a clever approach to taking
advantage of another memory one strategy exists: this is incomplete. Whilst the
elegance of this result is very attractive, just as the simplicity of the
victory of Tit For Tat in Axelrod's original tournaments was, it is incomplete.
Extortionate strategies achieve a high number of wins but they do not
achieve a high score which corresponds to the fitness landscape in an
evolutionary sense. From the large number of interactions a payoff matrix \(S\)
can be measured where \(S_{ij}\) denotes the score (using standard values of
\((R, S, T, P) = (3, 0, 5, 1)\)) of the \(i\)th strategy
against the \(j\)th strategy. Using this, the replicator equation
describes the evolution of the system based on a population density fitness
function:

\begin{equation}\label{eqn:replicator_dynamics}
    \frac{dx}{dt} = x(S-x^TS x)
\end{equation}

Equation (\ref{eqn:replicator_dynamics}) is solved numerically through an
integration technique described in~\cite{Petzold1983} and
Figure~\ref{fig:replicator_dynamics} shows the evolution of the distribution of
the system: the various strategies are ranked by scores. It is clear to see that
only the high ranking strategies survive the evolutionary process (in fact,
only \input{./assets/img/replicator_dynamics/main.tex}
have a final distribution greater than \(10 ^ {-2}\)). This confirms the
findings of~\cite{Moran1707} in which sophisticated strategies resist
evolutionary invasion of shorter memory strategies. Recalling
Figure~\ref{fig:SSError_and_probabilities_in_full} this demonstrates that:

\begin{itemize}
    \item Cooperation emerges through the evolutionary process: the high scoring
        strategies do not exhibit extortionate behaviour towards each other.
    \item Extortionate strategies do not survive the evolutionary process.
\end{itemize}

\begin{figure}[!htbp]
    \centering
    \includegraphics[width=.8\textwidth]{./assets/img/replicator_dynamics/main.pdf}
    \caption{Numerical simulation of the replicator equation
    (\ref{eqn:replicator_dynamics}): strategies are ordered by score, only the strategies with a high score survive the evolutionary process.}
    \label{fig:replicator_dynamics}
\end{figure}

This work can be used to classify plays of the IPD\@: data can be collected from
actual interactions (in lab or in the field). Furthermore, this allows for a
classification method similar to the notion of fingerprinting presented
in~\cite{Ashlock2008}. Trained strategies can potentially be classified as
extortionate or not or it could be possible to even constrain the reinforcement
learning approaches that are becoming prevalent in the literature.
Alternatively, this mathematical approach for recognising extortion could be
used in sophisticated strategies to defend against invasion. Arguably, some of
the strategies considered here exhibit this behaviour, indeed as described
in~\cite{Harper2017}, the top ranking strategies in the full tournament are
obtained using evolutionary reinforcement learning techniques, thus, suspicion
of extortionate behaviour could in fact be an evolutionary trait.

\section*{Acknowledgements}

The following open source software libraries were used in this research:

\begin{itemize}
    \item The Axelrod ~\cite{Knight2016, Knight2018} library (IPD strategies and
        tournaments).
    \item The sympy library~\cite{Meurer2017} (verification of all symbolic
        calculations).
    \item The matplotlib~\cite{Droettboom2018} library (visualisation).
    \item The pandas~\cite{Structures2010}, dask~\cite{Dask2016} and
        NumPy~\cite{Oliphant2015} libraries (data manipulation).
    \item The SciPy~\cite{Jones2001} library (numerical integration of the
        replicator equation).
\end{itemize}

This work was performed using the computational facilities of the Advanced
Research Computing @ Cardiff (ARCCA) Division, Cardiff University.

\printbibliography

\newpage
\section*{Supplementary materials}

\includepdf{assets/pdf/proof_of_form_of_extortionate_strategies/main.pdf}

\newpage

Using the pair wise interactions the transition rates \(p,
q\) can be measured and the steady state probabilities inferred and compared to
the actual probabilities of each state.
This is done numerically by computing the singular eigenvector of the
matrix \(A\) \cite{Stewart2009}:

\[
    A =
    \begin{bmatrix}
        p_1 q_1 & p_1 (1 - q_1) & (1 - p_1) q_1 & (1 -p_1) (1 - q_1) \\
        p_2 q_2 & p_2 (1 - q_2) & (1 - p_2) q_2 & (1 -p_2) (1 - q_2) \\
        p_3 q_3 & p_3 (1 - q_3) & (1 - p_3) q_3 & (1 -p_3) (1 - q_3) \\
        p_4 q_4 & p_4 (1 - q_4) & (1 - p_4) q_4 & (1 -p_4) (1 - q_4) \\
    \end{bmatrix}
\]

Figure~\ref{fig:computed_probabilities_vs_theoretic_probabilities} shows a
regression line fitted to every pairwise interaction with a reported
\(\text{SSError}\) value (pairwise interactions with missing states were
omitted). This serves to validate the approach: a part from some edge cases the
relationship is consistent.

\begin{figure}[!htbp]
    \centering
    \includegraphics[width=.8\textwidth]{./assets/img/computed_probabilities_vs_theoretic_probabilities/main.pdf}
    \caption{The
        relationship between the steady state probabilities inferred from the
        measured transitions and the actual steady state probabilities. A linear
        regression line is included validating the approach.}
    \label{fig:computed_probabilities_vs_theoretic_probabilities}
\end{figure}


\end{document}

    strategies is considered. In this setting
    the most highly performing strategies do not play in an extortionate way
    against each other but do against lower performing strategies.
    This suggests that whilst the theory of Zero Determinant strategies
    indicates that memory is not of fundamental importance to the evolution of
    cooperative behaviour, this is incomplete.
\end{abstract}

\section{Introduction}\label{sec:introduction}

Agent based game theoretic models have become a stalwart of the underpinning
mathematics of interactive behaviours. One of the major pieces of work
in this area is the pair of original computer tournaments run by Robert
Axelrod~\cite{Axelrod1980, Axelrod1980a}. These tournaments pitted submitted
computer strategies against each other in plays of the Iterated Prisoner's
Dilemma. A common game where agents can choose to pay a slight cost to their
immediate utility in the hope of building a reputation. This has been used in
economic and evolutionary game theory to understand the evolution of cooperative
behaviour.

Recently, a class of strategies was described in~\cite{Press2012} that can
provably extort any given opponent. In~\cite{Hilbe2013, Moran1707} some
questions have already been asked about the true effectiveness of these
strategies in an evolutionary setting. Here another question is asked: is it
possible to recognise this extortionate behaviour? A mathematical procedure for
suspicion is presented: in the same way that the continued actions of an
extortionate individual might raise suspicion.

This work makes use of the Axelrod Python library~\cite{Knight2018, Knight2016}
with a large number of Prisoner Dilemma strategies available to give an
extensive numerical example of the ideas presented.  The approach is presented
in Section~\ref{sec:delta-zd-strategies}.  All of the code and data discussed
in Section~\ref{sec:numerical-experiments} is open sourced, archived and
written according to best scientific principles~\cite{Wilson2014}. The data
archive can be found at~\cite{vincent_knight_2018_1297075}.

\section{Recognising Extortion}\label{sec:delta-zd-strategies}

In~\cite{Press2012}, given a match between 2 memory-one strategies, the concept
of Zero Determinant (ZD) strategies is introduced. The main result of that paper
shows that given two memory one players \(p, q\in\mathbb{R}^4\) a linear
relationship between the players' scores could be forced by one of the players.

Using the notation of~\cite{Press2012}, assuming the utilities for player \(p\)
are given by \(S_x=(R, S, T, P)\) and for player \(q\) by \(S_y=(R, T, S, P)\)
and that the stationary scores of each player is given by \(S_X\) and \(S_Y\)
respectively. The main result of~\cite{Press2012} is that if

\begin{equation}\label{eqn:linear_relationship_for_p}
    \tilde p=\alpha S_x + \beta S_y + \gamma
\end{equation}

or

\begin{equation}\label{eqn:linear_relationship_for_q}
    \tilde q=\alpha S_x + \beta S_y + \gamma
\end{equation}

where \(\tilde p = (1 - p_1, 1 - p_2, p_3, p_4)\) and
\(\tilde q = (1 - q_1, 1 - q_2, q_3, q_4)\) then:

\begin{equation}
    \alpha S_X + \beta S_Y + \gamma = 0
\end{equation}

In~\cite{Press2012} a particular type of ZD strategy is defined: extortionate
strategies. If:

\begin{equation}\label{eqn:constraint_for_extortion}
    \gamma = - P(\alpha + \beta)
\end{equation}

then the player can ensure they get a score \(\chi\) times
larger than the opponent. This extortion coefficient is given by:

\begin{equation}\label{eqn:definition_of_chi}
    \chi=\frac{-\beta}{\alpha}
\end{equation}

Thus, if (\ref{eqn:constraint_for_extortion}) holds and \(\chi >1\) a player is
said to extort their opponent.
Here, the reverse problem is considered: given a
\(p\in\mathbb{R}^4\) how does one identify \(\alpha, \beta\) if they
exist and is the strategy in fact acting in an extortionate way?

These conditions correspond to:

\begin{align}
    \tilde p_1 & = \alpha R + \beta R - P (\alpha + \beta)
            \label{eqn:condition_for_tilde_p1}\\
    \tilde p_2 & = \alpha S + \beta T - P (\alpha + \beta)
            \label{eqn:condition_for_tilde_p2}\\
    \tilde p_3 & = \alpha T + \beta S - P (\alpha + \beta)
            \label{eqn:condition_for_tilde_p3}\\
    \tilde p_4 & = \alpha P + \beta P - P (\alpha + \beta)
            \label{eqn:condition_for_tilde_p4}
\end{align}

Equation (\ref{eqn:condition_for_tilde_p4}) ensures that \(p_4=\tilde p_4=0\).
Equations (\ref{eqn:condition_for_tilde_p1}-\ref{eqn:condition_for_tilde_p3})
can be used to eliminate \(\alpha, \beta\), giving:

\begin{equation}\label{eqn:planar_definition_of_extortion}
    \tilde p_1 = \frac{(R - P)(\tilde p_2 + \tilde p_3)}{S + T - 2P}
\end{equation}

with:

\begin{equation}\label{eqn:definition_of_chi}
    \chi = \frac{\tilde p_2 (P - T) + \tilde p_3 (S - P)}
                {\tilde p_2 (P - S) + \tilde p_3 (T - P)}
\end{equation}

Given a strategy \(p\in\mathbb{R}^{4\times 1}\) equations
(\ref{eqn:condition_for_tilde_p4}), (\ref{eqn:planar_definition_of_extortion}-\ref{eqn:definition_of_chi}) can be used to check if
a strategy is extortionate. The conditions correspond to:

\begin{align}
    p_1 & = \frac{(R-P)(p_2 + p_3) - R + T + S - P}{S + T - 2P}
     \label{eqn:condition_for_p1}\\
    p_4 & = 0 \label{eqn:condition_for_p4}\\
    1 & > p_2 + p_3\label{eqn:condition_for_chi}
\end{align}

The algebraic steps necessary to prove these results are available in the
supporting materials.

All extortionate strategies reside on a triangular (\ref{eqn:condition_for_chi})
plane (\ref{eqn:condition_for_p1}) in 3 dimensions (\ref{eqn:condition_for_p4}).
Using this formulation it can be seen that a necessary (but not sufficient)
condition for an extortionate strategy is that it cooperates on average less
than 50\% of the time when in a state of disagreement with the opponent.

As an example, consider the known extortionate strategy \(p=(8 / 9, 1 / 2, 1 /
3, 0)\) from~\cite{Stewart2012} which is referred to as \texttt{Extort-2}. In
this case, for the standard values of \((R, T, S, P)\) constraint
(\ref{eqn:condition_for_p1}) corresponds to:

\begin{equation}
    p_1 = \frac{2(p_2 + p_3) + 1}{3}
\end{equation}

It is clear that in this case all constraints hold.

This approach could in fact be used to confirm that a given strategy is acting
in an extortionate manner even if it is not a memory one strategy. However, in
practice, if a closed form for \(p\) is not known, then due to measurement
and/or numerical error this would not work.

This problem can be written in the following linear algebraic form where
\(x=(\alpha, \beta)\)
and \(p^*=(\tilde p_1 - 1, tilde_2 - 1, p_3)\):

\begin{equation}\label{eqn:linear_algebraic_equation_for_p}
    Cx= p^*
\end{equation}

\(C\) corresponds to equations
(\ref{eqn:condition_for_tilde_p1}-\ref{eqn:condition_for_tilde_p3}) and is
given by:

\begin{equation}\label{eqn:definition_of_C}
    C =
    \begin{bmatrix}
        R - P & R- P \\
        S - P & T- P \\
        T - P & S- P \\
    \end{bmatrix}
\end{equation}

Note that in general, equation (\ref{eqn:linear_algebraic_equation_for_p}) will
not necessarily have a solution. From the Rouch\'{e}-Capelli theorem if there is
a solution it is unique as \(\text{rank}(C)=2\) which is the dimension of the
variable \(x\). The best fitting \(x\) is found by minimizing:

\begin{equation}\label{eqn:r_squared}
    \text{SSError} = \|C x- p^*\|_2^2 = \sum_{i=1}^{3}\left((C\bar x)_i-p_i^*\right)^2
\end{equation}

Note that \(\text{SSError}\), which is the square of the Frobenius
norm~\cite{Golub2013}, becomes a measure of how close a strategy is to being an
extortionate strategy. Suspicion
of extortion then corresponds to a threshold on \(\text{SSError}\).

By observing interactions (human or otherwise), their memory one representation
can be inferred and this approach can be used to recognise extortionate
behaviour. The notion of comparing theoretic and actual plays of the IPD is not
novel, see for example~\cite{Rand2013}. Immediately it is noted that if the
environment is noisy~\cite{Wu1995} then no strategy can be considered to be
extortionate as \(p_4>0\).

In the next section, this idea will be illustrated by observing the interactions
that take place in a computer based tournament of the IPD\@.

\section{Numerical experiments}\label{sec:numerical-experiments}

In~\cite{Stewart2012} results from a tournament with
\documentclass[a4paper]{article}

\usepackage{amsmath}
\usepackage{amssymb}
\usepackage[margin=1.5cm,
            includefoot,
            footskip=30pt]{geometry}
\usepackage{layout}
\usepackage{graphicx}
\usepackage{subcaption}

\usepackage{biblatex}
\usepackage{pdfpages}

\bibliography{main.bib}

\title{Suspicion: Recognising and evaluating the effectiveness
       of extortion in the Iterated Prisoner's Dilemma}
\author{Vincent A. Knight \and Nikoleta E. Glynatsi}
\date{\today}



\begin{document}

\maketitle

\begin{abstract}
    The Iterated Prisoner's Dilemma is a model for rational and evolutionary
    interactive behaviour. It has applications both in the study of human social
    behaviour as well as in biology.
    It is used to understand when and how a rational individual might
    accept an immediate cost to their own utility for the direct benefit of
    another.

    Much attention has been given to a class of strategies called
    Zero Determinant strategies. It has been theoretically shown that these
    strategies can ``extort'' any player.

    In this work, an approach to identify if observed strategies are playing in
    an extortionate way is described. Furthermore, experimental analysis of
    a large tournament with \input{assets/tex/number_of_full_strategies/main.tex}
    strategies is considered. In this setting
    the most highly performing strategies do not play in an extortionate way
    against each other but do against lower performing strategies.
    This suggests that whilst the theory of Zero Determinant strategies
    indicates that memory is not of fundamental importance to the evolution of
    cooperative behaviour, this is incomplete.
\end{abstract}

\section{Introduction}\label{sec:introduction}

Agent based game theoretic models have become a stalwart of the underpinning
mathematics of interactive behaviours. One of the major pieces of work
in this area is the pair of original computer tournaments run by Robert
Axelrod~\cite{Axelrod1980, Axelrod1980a}. These tournaments pitted submitted
computer strategies against each other in plays of the Iterated Prisoner's
Dilemma. A common game where agents can choose to pay a slight cost to their
immediate utility in the hope of building a reputation. This has been used in
economic and evolutionary game theory to understand the evolution of cooperative
behaviour.

Recently, a class of strategies was described in~\cite{Press2012} that can
provably extort any given opponent. In~\cite{Hilbe2013, Moran1707} some
questions have already been asked about the true effectiveness of these
strategies in an evolutionary setting. Here another question is asked: is it
possible to recognise this extortionate behaviour? A mathematical procedure for
suspicion is presented: in the same way that the continued actions of an
extortionate individual might raise suspicion.

This work makes use of the Axelrod Python library~\cite{Knight2018, Knight2016}
with a large number of Prisoner Dilemma strategies available to give an
extensive numerical example of the ideas presented.  The approach is presented
in Section~\ref{sec:delta-zd-strategies}.  All of the code and data discussed
in Section~\ref{sec:numerical-experiments} is open sourced, archived and
written according to best scientific principles~\cite{Wilson2014}. The data
archive can be found at~\cite{vincent_knight_2018_1297075}.

\section{Recognising Extortion}\label{sec:delta-zd-strategies}

In~\cite{Press2012}, given a match between 2 memory-one strategies, the concept
of Zero Determinant (ZD) strategies is introduced. The main result of that paper
shows that given two memory one players \(p, q\in\mathbb{R}^4\) a linear
relationship between the players' scores could be forced by one of the players.

Using the notation of~\cite{Press2012}, assuming the utilities for player \(p\)
are given by \(S_x=(R, S, T, P)\) and for player \(q\) by \(S_y=(R, T, S, P)\)
and that the stationary scores of each player is given by \(S_X\) and \(S_Y\)
respectively. The main result of~\cite{Press2012} is that if

\begin{equation}\label{eqn:linear_relationship_for_p}
    \tilde p=\alpha S_x + \beta S_y + \gamma
\end{equation}

or

\begin{equation}\label{eqn:linear_relationship_for_q}
    \tilde q=\alpha S_x + \beta S_y + \gamma
\end{equation}

where \(\tilde p = (1 - p_1, 1 - p_2, p_3, p_4)\) and
\(\tilde q = (1 - q_1, 1 - q_2, q_3, q_4)\) then:

\begin{equation}
    \alpha S_X + \beta S_Y + \gamma = 0
\end{equation}

In~\cite{Press2012} a particular type of ZD strategy is defined: extortionate
strategies. If:

\begin{equation}\label{eqn:constraint_for_extortion}
    \gamma = - P(\alpha + \beta)
\end{equation}

then the player can ensure they get a score \(\chi\) times
larger than the opponent. This extortion coefficient is given by:

\begin{equation}\label{eqn:definition_of_chi}
    \chi=\frac{-\beta}{\alpha}
\end{equation}

Thus, if (\ref{eqn:constraint_for_extortion}) holds and \(\chi >1\) a player is
said to extort their opponent.
Here, the reverse problem is considered: given a
\(p\in\mathbb{R}^4\) how does one identify \(\alpha, \beta\) if they
exist and is the strategy in fact acting in an extortionate way?

These conditions correspond to:

\begin{align}
    \tilde p_1 & = \alpha R + \beta R - P (\alpha + \beta)
            \label{eqn:condition_for_tilde_p1}\\
    \tilde p_2 & = \alpha S + \beta T - P (\alpha + \beta)
            \label{eqn:condition_for_tilde_p2}\\
    \tilde p_3 & = \alpha T + \beta S - P (\alpha + \beta)
            \label{eqn:condition_for_tilde_p3}\\
    \tilde p_4 & = \alpha P + \beta P - P (\alpha + \beta)
            \label{eqn:condition_for_tilde_p4}
\end{align}

Equation (\ref{eqn:condition_for_tilde_p4}) ensures that \(p_4=\tilde p_4=0\).
Equations (\ref{eqn:condition_for_tilde_p1}-\ref{eqn:condition_for_tilde_p3})
can be used to eliminate \(\alpha, \beta\), giving:

\begin{equation}\label{eqn:planar_definition_of_extortion}
    \tilde p_1 = \frac{(R - P)(\tilde p_2 + \tilde p_3)}{S + T - 2P}
\end{equation}

with:

\begin{equation}\label{eqn:definition_of_chi}
    \chi = \frac{\tilde p_2 (P - T) + \tilde p_3 (S - P)}
                {\tilde p_2 (P - S) + \tilde p_3 (T - P)}
\end{equation}

Given a strategy \(p\in\mathbb{R}^{4\times 1}\) equations
(\ref{eqn:condition_for_tilde_p4}), (\ref{eqn:planar_definition_of_extortion}-\ref{eqn:definition_of_chi}) can be used to check if
a strategy is extortionate. The conditions correspond to:

\begin{align}
    p_1 & = \frac{(R-P)(p_2 + p_3) - R + T + S - P}{S + T - 2P}
     \label{eqn:condition_for_p1}\\
    p_4 & = 0 \label{eqn:condition_for_p4}\\
    1 & > p_2 + p_3\label{eqn:condition_for_chi}
\end{align}

The algebraic steps necessary to prove these results are available in the
supporting materials.

All extortionate strategies reside on a triangular (\ref{eqn:condition_for_chi})
plane (\ref{eqn:condition_for_p1}) in 3 dimensions (\ref{eqn:condition_for_p4}).
Using this formulation it can be seen that a necessary (but not sufficient)
condition for an extortionate strategy is that it cooperates on average less
than 50\% of the time when in a state of disagreement with the opponent.

As an example, consider the known extortionate strategy \(p=(8 / 9, 1 / 2, 1 /
3, 0)\) from~\cite{Stewart2012} which is referred to as \texttt{Extort-2}. In
this case, for the standard values of \((R, T, S, P)\) constraint
(\ref{eqn:condition_for_p1}) corresponds to:

\begin{equation}
    p_1 = \frac{2(p_2 + p_3) + 1}{3}
\end{equation}

It is clear that in this case all constraints hold.

This approach could in fact be used to confirm that a given strategy is acting
in an extortionate manner even if it is not a memory one strategy. However, in
practice, if a closed form for \(p\) is not known, then due to measurement
and/or numerical error this would not work.

This problem can be written in the following linear algebraic form where
\(x=(\alpha, \beta)\)
and \(p^*=(\tilde p_1 - 1, tilde_2 - 1, p_3)\):

\begin{equation}\label{eqn:linear_algebraic_equation_for_p}
    Cx= p^*
\end{equation}

\(C\) corresponds to equations
(\ref{eqn:condition_for_tilde_p1}-\ref{eqn:condition_for_tilde_p3}) and is
given by:

\begin{equation}\label{eqn:definition_of_C}
    C =
    \begin{bmatrix}
        R - P & R- P \\
        S - P & T- P \\
        T - P & S- P \\
    \end{bmatrix}
\end{equation}

Note that in general, equation (\ref{eqn:linear_algebraic_equation_for_p}) will
not necessarily have a solution. From the Rouch\'{e}-Capelli theorem if there is
a solution it is unique as \(\text{rank}(C)=2\) which is the dimension of the
variable \(x\). The best fitting \(x\) is found by minimizing:

\begin{equation}\label{eqn:r_squared}
    \text{SSError} = \|C x- p^*\|_2^2 = \sum_{i=1}^{3}\left((C\bar x)_i-p_i^*\right)^2
\end{equation}

Note that \(\text{SSError}\), which is the square of the Frobenius
norm~\cite{Golub2013}, becomes a measure of how close a strategy is to being an
extortionate strategy. Suspicion
of extortion then corresponds to a threshold on \(\text{SSError}\).

By observing interactions (human or otherwise), their memory one representation
can be inferred and this approach can be used to recognise extortionate
behaviour. The notion of comparing theoretic and actual plays of the IPD is not
novel, see for example~\cite{Rand2013}. Immediately it is noted that if the
environment is noisy~\cite{Wu1995} then no strategy can be considered to be
extortionate as \(p_4>0\).

In the next section, this idea will be illustrated by observing the interactions
that take place in a computer based tournament of the IPD\@.

\section{Numerical experiments}\label{sec:numerical-experiments}

In~\cite{Stewart2012} results from a tournament with
\input{./assets/tex/number_of_stewart_plotkin_strategies/main.tex} strategies,
was presented with specific consideration given to ZD strategies. This
tournament is reproduced here using the Axelrod-Python
project~\cite{Knight2016}. To obtain a good measure of the corresponding
transition rates for each strategy all matches have been run for
\input{assets/tex/number_of_turns/main.tex} turns and every match has been
repeated \input{assets/tex/number_of_repetitions/main.tex} times. All of this
interaction data is available at~\cite{vincent_knight_2018_1297075}. A good
match between the inferred Markov chain and the state distribution of the actual
interactions has been verified. Data for this is presented in the supplementary
materials.

Figure~\ref{fig:SSError_overall_in_stewart_plotkin} shows the \(\text{SSError}\)
values for all the strategies in the tournament, as reported
in~\cite{Stewart2012} the extortionate strategy (which has an expected
\(\text{SSError}\) approximately 0) gains a large number of wins.

\begin{figure}[!htbp]
    \centering
    \includegraphics[width=.8\textwidth]{./assets/img/SSError_overall_in_stewart_plotkin/main.pdf}
    \caption{\(\text{SSError}\) and state probabilities for the strategies
        of~\cite{Stewart2012}, ordered both by number of wins and overall score.
        Note that \(P(DC)\) is not shown as it corresponds to the transpose of
        \(P(CD)\). Cooperator and Defector are omitted as they do not visit all
        the states.}
    \label{fig:SSError_overall_in_stewart_plotkin}
\end{figure}

Here, the work of~\cite{Stewart2012} is extended by investigating a tournament
with \input{assets/tex/number_of_full_strategies/main.tex}
strategies.

The results of this analysis are shown in
Figure~\ref{fig:SSError_and_probabilities_in_full}. The top ranking strategies
by number of wins seem to be extortionate (but not against all strategies) and
it can be seen that a small sub group of strategies achieve mutual defection.
All the top ranking strategies according to score achieve mutual cooperation and
do not extort each other, however they
\textbf{do} exhibit extortionate behaviour towards a number of the lower ranking
strategies.

\begin{figure}[!htbp]
    \centering
    \includegraphics[width=.8\textwidth]{./assets/img/SSError_and_probabilities_in_full/main.pdf}
    \caption{\(\text{SSError}\) for the strategies for the full tournament. Only
    strategy interactions for which \(p_4=0\) and \(\chi>1\) are displayed.}
    \label{fig:SSError_and_probabilities_in_full}
\end{figure}

\section{Conclusion}\label{sec:conclusion}

This work defines an approach to measure whether or not a player is playing a
strategy that corresponds to an extortionate strategy as defined
in~\cite{Press2012}: a mathematical model for suspicion. Indeed, all
extortionate strategies have been
 classified as lying on a triangular plane.
This rigorous classification fails to be robust to small measurement error, thus
a statistical approach is proposed.
This is done through a linear algebraic approach for approximating the solution
of a linear system. Using this, a large number of pairwise interactions is
simulated and in fact very few strategies are found to act extortionately.

The work of~\cite{Press2012}, whilst showing that a clever approach to taking
advantage of another memory one strategy exists: this is incomplete. Whilst the
elegance of this result is very attractive, just as the simplicity of the
victory of Tit For Tat in Axelrod's original tournaments was, it is incomplete.
Extortionate strategies achieve a high number of wins but they do not
achieve a high score which corresponds to the fitness landscape in an
evolutionary sense. From the large number of interactions a payoff matrix \(S\)
can be measured where \(S_{ij}\) denotes the score (using standard values of
\((R, S, T, P) = (3, 0, 5, 1)\)) of the \(i\)th strategy
against the \(j\)th strategy. Using this, the replicator equation
describes the evolution of the system based on a population density fitness
function:

\begin{equation}\label{eqn:replicator_dynamics}
    \frac{dx}{dt} = x(S-x^TS x)
\end{equation}

Equation (\ref{eqn:replicator_dynamics}) is solved numerically through an
integration technique described in~\cite{Petzold1983} and
Figure~\ref{fig:replicator_dynamics} shows the evolution of the distribution of
the system: the various strategies are ranked by scores. It is clear to see that
only the high ranking strategies survive the evolutionary process (in fact,
only \input{./assets/img/replicator_dynamics/main.tex}
have a final distribution greater than \(10 ^ {-2}\)). This confirms the
findings of~\cite{Moran1707} in which sophisticated strategies resist
evolutionary invasion of shorter memory strategies. Recalling
Figure~\ref{fig:SSError_and_probabilities_in_full} this demonstrates that:

\begin{itemize}
    \item Cooperation emerges through the evolutionary process: the high scoring
        strategies do not exhibit extortionate behaviour towards each other.
    \item Extortionate strategies do not survive the evolutionary process.
\end{itemize}

\begin{figure}[!htbp]
    \centering
    \includegraphics[width=.8\textwidth]{./assets/img/replicator_dynamics/main.pdf}
    \caption{Numerical simulation of the replicator equation
    (\ref{eqn:replicator_dynamics}): strategies are ordered by score, only the strategies with a high score survive the evolutionary process.}
    \label{fig:replicator_dynamics}
\end{figure}

This work can be used to classify plays of the IPD\@: data can be collected from
actual interactions (in lab or in the field). Furthermore, this allows for a
classification method similar to the notion of fingerprinting presented
in~\cite{Ashlock2008}. Trained strategies can potentially be classified as
extortionate or not or it could be possible to even constrain the reinforcement
learning approaches that are becoming prevalent in the literature.
Alternatively, this mathematical approach for recognising extortion could be
used in sophisticated strategies to defend against invasion. Arguably, some of
the strategies considered here exhibit this behaviour, indeed as described
in~\cite{Harper2017}, the top ranking strategies in the full tournament are
obtained using evolutionary reinforcement learning techniques, thus, suspicion
of extortionate behaviour could in fact be an evolutionary trait.

\section*{Acknowledgements}

The following open source software libraries were used in this research:

\begin{itemize}
    \item The Axelrod ~\cite{Knight2016, Knight2018} library (IPD strategies and
        tournaments).
    \item The sympy library~\cite{Meurer2017} (verification of all symbolic
        calculations).
    \item The matplotlib~\cite{Droettboom2018} library (visualisation).
    \item The pandas~\cite{Structures2010}, dask~\cite{Dask2016} and
        NumPy~\cite{Oliphant2015} libraries (data manipulation).
    \item The SciPy~\cite{Jones2001} library (numerical integration of the
        replicator equation).
\end{itemize}

This work was performed using the computational facilities of the Advanced
Research Computing @ Cardiff (ARCCA) Division, Cardiff University.

\printbibliography

\newpage
\section*{Supplementary materials}

\includepdf{assets/pdf/proof_of_form_of_extortionate_strategies/main.pdf}

\newpage

Using the pair wise interactions the transition rates \(p,
q\) can be measured and the steady state probabilities inferred and compared to
the actual probabilities of each state.
This is done numerically by computing the singular eigenvector of the
matrix \(A\) \cite{Stewart2009}:

\[
    A =
    \begin{bmatrix}
        p_1 q_1 & p_1 (1 - q_1) & (1 - p_1) q_1 & (1 -p_1) (1 - q_1) \\
        p_2 q_2 & p_2 (1 - q_2) & (1 - p_2) q_2 & (1 -p_2) (1 - q_2) \\
        p_3 q_3 & p_3 (1 - q_3) & (1 - p_3) q_3 & (1 -p_3) (1 - q_3) \\
        p_4 q_4 & p_4 (1 - q_4) & (1 - p_4) q_4 & (1 -p_4) (1 - q_4) \\
    \end{bmatrix}
\]

Figure~\ref{fig:computed_probabilities_vs_theoretic_probabilities} shows a
regression line fitted to every pairwise interaction with a reported
\(\text{SSError}\) value (pairwise interactions with missing states were
omitted). This serves to validate the approach: a part from some edge cases the
relationship is consistent.

\begin{figure}[!htbp]
    \centering
    \includegraphics[width=.8\textwidth]{./assets/img/computed_probabilities_vs_theoretic_probabilities/main.pdf}
    \caption{The
        relationship between the steady state probabilities inferred from the
        measured transitions and the actual steady state probabilities. A linear
        regression line is included validating the approach.}
    \label{fig:computed_probabilities_vs_theoretic_probabilities}
\end{figure}


\end{document}
 strategies,
was presented with specific consideration given to ZD strategies. This
tournament is reproduced here using the Axelrod-Python
project~\cite{Knight2016}. To obtain a good measure of the corresponding
transition rates for each strategy all matches have been run for
\documentclass[a4paper]{article}

\usepackage{amsmath}
\usepackage{amssymb}
\usepackage[margin=1.5cm,
            includefoot,
            footskip=30pt]{geometry}
\usepackage{layout}
\usepackage{graphicx}
\usepackage{subcaption}

\usepackage{biblatex}
\usepackage{pdfpages}

\bibliography{main.bib}

\title{Suspicion: Recognising and evaluating the effectiveness
       of extortion in the Iterated Prisoner's Dilemma}
\author{Vincent A. Knight \and Nikoleta E. Glynatsi}
\date{\today}



\begin{document}

\maketitle

\begin{abstract}
    The Iterated Prisoner's Dilemma is a model for rational and evolutionary
    interactive behaviour. It has applications both in the study of human social
    behaviour as well as in biology.
    It is used to understand when and how a rational individual might
    accept an immediate cost to their own utility for the direct benefit of
    another.

    Much attention has been given to a class of strategies called
    Zero Determinant strategies. It has been theoretically shown that these
    strategies can ``extort'' any player.

    In this work, an approach to identify if observed strategies are playing in
    an extortionate way is described. Furthermore, experimental analysis of
    a large tournament with \input{assets/tex/number_of_full_strategies/main.tex}
    strategies is considered. In this setting
    the most highly performing strategies do not play in an extortionate way
    against each other but do against lower performing strategies.
    This suggests that whilst the theory of Zero Determinant strategies
    indicates that memory is not of fundamental importance to the evolution of
    cooperative behaviour, this is incomplete.
\end{abstract}

\section{Introduction}\label{sec:introduction}

Agent based game theoretic models have become a stalwart of the underpinning
mathematics of interactive behaviours. One of the major pieces of work
in this area is the pair of original computer tournaments run by Robert
Axelrod~\cite{Axelrod1980, Axelrod1980a}. These tournaments pitted submitted
computer strategies against each other in plays of the Iterated Prisoner's
Dilemma. A common game where agents can choose to pay a slight cost to their
immediate utility in the hope of building a reputation. This has been used in
economic and evolutionary game theory to understand the evolution of cooperative
behaviour.

Recently, a class of strategies was described in~\cite{Press2012} that can
provably extort any given opponent. In~\cite{Hilbe2013, Moran1707} some
questions have already been asked about the true effectiveness of these
strategies in an evolutionary setting. Here another question is asked: is it
possible to recognise this extortionate behaviour? A mathematical procedure for
suspicion is presented: in the same way that the continued actions of an
extortionate individual might raise suspicion.

This work makes use of the Axelrod Python library~\cite{Knight2018, Knight2016}
with a large number of Prisoner Dilemma strategies available to give an
extensive numerical example of the ideas presented.  The approach is presented
in Section~\ref{sec:delta-zd-strategies}.  All of the code and data discussed
in Section~\ref{sec:numerical-experiments} is open sourced, archived and
written according to best scientific principles~\cite{Wilson2014}. The data
archive can be found at~\cite{vincent_knight_2018_1297075}.

\section{Recognising Extortion}\label{sec:delta-zd-strategies}

In~\cite{Press2012}, given a match between 2 memory-one strategies, the concept
of Zero Determinant (ZD) strategies is introduced. The main result of that paper
shows that given two memory one players \(p, q\in\mathbb{R}^4\) a linear
relationship between the players' scores could be forced by one of the players.

Using the notation of~\cite{Press2012}, assuming the utilities for player \(p\)
are given by \(S_x=(R, S, T, P)\) and for player \(q\) by \(S_y=(R, T, S, P)\)
and that the stationary scores of each player is given by \(S_X\) and \(S_Y\)
respectively. The main result of~\cite{Press2012} is that if

\begin{equation}\label{eqn:linear_relationship_for_p}
    \tilde p=\alpha S_x + \beta S_y + \gamma
\end{equation}

or

\begin{equation}\label{eqn:linear_relationship_for_q}
    \tilde q=\alpha S_x + \beta S_y + \gamma
\end{equation}

where \(\tilde p = (1 - p_1, 1 - p_2, p_3, p_4)\) and
\(\tilde q = (1 - q_1, 1 - q_2, q_3, q_4)\) then:

\begin{equation}
    \alpha S_X + \beta S_Y + \gamma = 0
\end{equation}

In~\cite{Press2012} a particular type of ZD strategy is defined: extortionate
strategies. If:

\begin{equation}\label{eqn:constraint_for_extortion}
    \gamma = - P(\alpha + \beta)
\end{equation}

then the player can ensure they get a score \(\chi\) times
larger than the opponent. This extortion coefficient is given by:

\begin{equation}\label{eqn:definition_of_chi}
    \chi=\frac{-\beta}{\alpha}
\end{equation}

Thus, if (\ref{eqn:constraint_for_extortion}) holds and \(\chi >1\) a player is
said to extort their opponent.
Here, the reverse problem is considered: given a
\(p\in\mathbb{R}^4\) how does one identify \(\alpha, \beta\) if they
exist and is the strategy in fact acting in an extortionate way?

These conditions correspond to:

\begin{align}
    \tilde p_1 & = \alpha R + \beta R - P (\alpha + \beta)
            \label{eqn:condition_for_tilde_p1}\\
    \tilde p_2 & = \alpha S + \beta T - P (\alpha + \beta)
            \label{eqn:condition_for_tilde_p2}\\
    \tilde p_3 & = \alpha T + \beta S - P (\alpha + \beta)
            \label{eqn:condition_for_tilde_p3}\\
    \tilde p_4 & = \alpha P + \beta P - P (\alpha + \beta)
            \label{eqn:condition_for_tilde_p4}
\end{align}

Equation (\ref{eqn:condition_for_tilde_p4}) ensures that \(p_4=\tilde p_4=0\).
Equations (\ref{eqn:condition_for_tilde_p1}-\ref{eqn:condition_for_tilde_p3})
can be used to eliminate \(\alpha, \beta\), giving:

\begin{equation}\label{eqn:planar_definition_of_extortion}
    \tilde p_1 = \frac{(R - P)(\tilde p_2 + \tilde p_3)}{S + T - 2P}
\end{equation}

with:

\begin{equation}\label{eqn:definition_of_chi}
    \chi = \frac{\tilde p_2 (P - T) + \tilde p_3 (S - P)}
                {\tilde p_2 (P - S) + \tilde p_3 (T - P)}
\end{equation}

Given a strategy \(p\in\mathbb{R}^{4\times 1}\) equations
(\ref{eqn:condition_for_tilde_p4}), (\ref{eqn:planar_definition_of_extortion}-\ref{eqn:definition_of_chi}) can be used to check if
a strategy is extortionate. The conditions correspond to:

\begin{align}
    p_1 & = \frac{(R-P)(p_2 + p_3) - R + T + S - P}{S + T - 2P}
     \label{eqn:condition_for_p1}\\
    p_4 & = 0 \label{eqn:condition_for_p4}\\
    1 & > p_2 + p_3\label{eqn:condition_for_chi}
\end{align}

The algebraic steps necessary to prove these results are available in the
supporting materials.

All extortionate strategies reside on a triangular (\ref{eqn:condition_for_chi})
plane (\ref{eqn:condition_for_p1}) in 3 dimensions (\ref{eqn:condition_for_p4}).
Using this formulation it can be seen that a necessary (but not sufficient)
condition for an extortionate strategy is that it cooperates on average less
than 50\% of the time when in a state of disagreement with the opponent.

As an example, consider the known extortionate strategy \(p=(8 / 9, 1 / 2, 1 /
3, 0)\) from~\cite{Stewart2012} which is referred to as \texttt{Extort-2}. In
this case, for the standard values of \((R, T, S, P)\) constraint
(\ref{eqn:condition_for_p1}) corresponds to:

\begin{equation}
    p_1 = \frac{2(p_2 + p_3) + 1}{3}
\end{equation}

It is clear that in this case all constraints hold.

This approach could in fact be used to confirm that a given strategy is acting
in an extortionate manner even if it is not a memory one strategy. However, in
practice, if a closed form for \(p\) is not known, then due to measurement
and/or numerical error this would not work.

This problem can be written in the following linear algebraic form where
\(x=(\alpha, \beta)\)
and \(p^*=(\tilde p_1 - 1, tilde_2 - 1, p_3)\):

\begin{equation}\label{eqn:linear_algebraic_equation_for_p}
    Cx= p^*
\end{equation}

\(C\) corresponds to equations
(\ref{eqn:condition_for_tilde_p1}-\ref{eqn:condition_for_tilde_p3}) and is
given by:

\begin{equation}\label{eqn:definition_of_C}
    C =
    \begin{bmatrix}
        R - P & R- P \\
        S - P & T- P \\
        T - P & S- P \\
    \end{bmatrix}
\end{equation}

Note that in general, equation (\ref{eqn:linear_algebraic_equation_for_p}) will
not necessarily have a solution. From the Rouch\'{e}-Capelli theorem if there is
a solution it is unique as \(\text{rank}(C)=2\) which is the dimension of the
variable \(x\). The best fitting \(x\) is found by minimizing:

\begin{equation}\label{eqn:r_squared}
    \text{SSError} = \|C x- p^*\|_2^2 = \sum_{i=1}^{3}\left((C\bar x)_i-p_i^*\right)^2
\end{equation}

Note that \(\text{SSError}\), which is the square of the Frobenius
norm~\cite{Golub2013}, becomes a measure of how close a strategy is to being an
extortionate strategy. Suspicion
of extortion then corresponds to a threshold on \(\text{SSError}\).

By observing interactions (human or otherwise), their memory one representation
can be inferred and this approach can be used to recognise extortionate
behaviour. The notion of comparing theoretic and actual plays of the IPD is not
novel, see for example~\cite{Rand2013}. Immediately it is noted that if the
environment is noisy~\cite{Wu1995} then no strategy can be considered to be
extortionate as \(p_4>0\).

In the next section, this idea will be illustrated by observing the interactions
that take place in a computer based tournament of the IPD\@.

\section{Numerical experiments}\label{sec:numerical-experiments}

In~\cite{Stewart2012} results from a tournament with
\input{./assets/tex/number_of_stewart_plotkin_strategies/main.tex} strategies,
was presented with specific consideration given to ZD strategies. This
tournament is reproduced here using the Axelrod-Python
project~\cite{Knight2016}. To obtain a good measure of the corresponding
transition rates for each strategy all matches have been run for
\input{assets/tex/number_of_turns/main.tex} turns and every match has been
repeated \input{assets/tex/number_of_repetitions/main.tex} times. All of this
interaction data is available at~\cite{vincent_knight_2018_1297075}. A good
match between the inferred Markov chain and the state distribution of the actual
interactions has been verified. Data for this is presented in the supplementary
materials.

Figure~\ref{fig:SSError_overall_in_stewart_plotkin} shows the \(\text{SSError}\)
values for all the strategies in the tournament, as reported
in~\cite{Stewart2012} the extortionate strategy (which has an expected
\(\text{SSError}\) approximately 0) gains a large number of wins.

\begin{figure}[!htbp]
    \centering
    \includegraphics[width=.8\textwidth]{./assets/img/SSError_overall_in_stewart_plotkin/main.pdf}
    \caption{\(\text{SSError}\) and state probabilities for the strategies
        of~\cite{Stewart2012}, ordered both by number of wins and overall score.
        Note that \(P(DC)\) is not shown as it corresponds to the transpose of
        \(P(CD)\). Cooperator and Defector are omitted as they do not visit all
        the states.}
    \label{fig:SSError_overall_in_stewart_plotkin}
\end{figure}

Here, the work of~\cite{Stewart2012} is extended by investigating a tournament
with \input{assets/tex/number_of_full_strategies/main.tex}
strategies.

The results of this analysis are shown in
Figure~\ref{fig:SSError_and_probabilities_in_full}. The top ranking strategies
by number of wins seem to be extortionate (but not against all strategies) and
it can be seen that a small sub group of strategies achieve mutual defection.
All the top ranking strategies according to score achieve mutual cooperation and
do not extort each other, however they
\textbf{do} exhibit extortionate behaviour towards a number of the lower ranking
strategies.

\begin{figure}[!htbp]
    \centering
    \includegraphics[width=.8\textwidth]{./assets/img/SSError_and_probabilities_in_full/main.pdf}
    \caption{\(\text{SSError}\) for the strategies for the full tournament. Only
    strategy interactions for which \(p_4=0\) and \(\chi>1\) are displayed.}
    \label{fig:SSError_and_probabilities_in_full}
\end{figure}

\section{Conclusion}\label{sec:conclusion}

This work defines an approach to measure whether or not a player is playing a
strategy that corresponds to an extortionate strategy as defined
in~\cite{Press2012}: a mathematical model for suspicion. Indeed, all
extortionate strategies have been
 classified as lying on a triangular plane.
This rigorous classification fails to be robust to small measurement error, thus
a statistical approach is proposed.
This is done through a linear algebraic approach for approximating the solution
of a linear system. Using this, a large number of pairwise interactions is
simulated and in fact very few strategies are found to act extortionately.

The work of~\cite{Press2012}, whilst showing that a clever approach to taking
advantage of another memory one strategy exists: this is incomplete. Whilst the
elegance of this result is very attractive, just as the simplicity of the
victory of Tit For Tat in Axelrod's original tournaments was, it is incomplete.
Extortionate strategies achieve a high number of wins but they do not
achieve a high score which corresponds to the fitness landscape in an
evolutionary sense. From the large number of interactions a payoff matrix \(S\)
can be measured where \(S_{ij}\) denotes the score (using standard values of
\((R, S, T, P) = (3, 0, 5, 1)\)) of the \(i\)th strategy
against the \(j\)th strategy. Using this, the replicator equation
describes the evolution of the system based on a population density fitness
function:

\begin{equation}\label{eqn:replicator_dynamics}
    \frac{dx}{dt} = x(S-x^TS x)
\end{equation}

Equation (\ref{eqn:replicator_dynamics}) is solved numerically through an
integration technique described in~\cite{Petzold1983} and
Figure~\ref{fig:replicator_dynamics} shows the evolution of the distribution of
the system: the various strategies are ranked by scores. It is clear to see that
only the high ranking strategies survive the evolutionary process (in fact,
only \input{./assets/img/replicator_dynamics/main.tex}
have a final distribution greater than \(10 ^ {-2}\)). This confirms the
findings of~\cite{Moran1707} in which sophisticated strategies resist
evolutionary invasion of shorter memory strategies. Recalling
Figure~\ref{fig:SSError_and_probabilities_in_full} this demonstrates that:

\begin{itemize}
    \item Cooperation emerges through the evolutionary process: the high scoring
        strategies do not exhibit extortionate behaviour towards each other.
    \item Extortionate strategies do not survive the evolutionary process.
\end{itemize}

\begin{figure}[!htbp]
    \centering
    \includegraphics[width=.8\textwidth]{./assets/img/replicator_dynamics/main.pdf}
    \caption{Numerical simulation of the replicator equation
    (\ref{eqn:replicator_dynamics}): strategies are ordered by score, only the strategies with a high score survive the evolutionary process.}
    \label{fig:replicator_dynamics}
\end{figure}

This work can be used to classify plays of the IPD\@: data can be collected from
actual interactions (in lab or in the field). Furthermore, this allows for a
classification method similar to the notion of fingerprinting presented
in~\cite{Ashlock2008}. Trained strategies can potentially be classified as
extortionate or not or it could be possible to even constrain the reinforcement
learning approaches that are becoming prevalent in the literature.
Alternatively, this mathematical approach for recognising extortion could be
used in sophisticated strategies to defend against invasion. Arguably, some of
the strategies considered here exhibit this behaviour, indeed as described
in~\cite{Harper2017}, the top ranking strategies in the full tournament are
obtained using evolutionary reinforcement learning techniques, thus, suspicion
of extortionate behaviour could in fact be an evolutionary trait.

\section*{Acknowledgements}

The following open source software libraries were used in this research:

\begin{itemize}
    \item The Axelrod ~\cite{Knight2016, Knight2018} library (IPD strategies and
        tournaments).
    \item The sympy library~\cite{Meurer2017} (verification of all symbolic
        calculations).
    \item The matplotlib~\cite{Droettboom2018} library (visualisation).
    \item The pandas~\cite{Structures2010}, dask~\cite{Dask2016} and
        NumPy~\cite{Oliphant2015} libraries (data manipulation).
    \item The SciPy~\cite{Jones2001} library (numerical integration of the
        replicator equation).
\end{itemize}

This work was performed using the computational facilities of the Advanced
Research Computing @ Cardiff (ARCCA) Division, Cardiff University.

\printbibliography

\newpage
\section*{Supplementary materials}

\includepdf{assets/pdf/proof_of_form_of_extortionate_strategies/main.pdf}

\newpage

Using the pair wise interactions the transition rates \(p,
q\) can be measured and the steady state probabilities inferred and compared to
the actual probabilities of each state.
This is done numerically by computing the singular eigenvector of the
matrix \(A\) \cite{Stewart2009}:

\[
    A =
    \begin{bmatrix}
        p_1 q_1 & p_1 (1 - q_1) & (1 - p_1) q_1 & (1 -p_1) (1 - q_1) \\
        p_2 q_2 & p_2 (1 - q_2) & (1 - p_2) q_2 & (1 -p_2) (1 - q_2) \\
        p_3 q_3 & p_3 (1 - q_3) & (1 - p_3) q_3 & (1 -p_3) (1 - q_3) \\
        p_4 q_4 & p_4 (1 - q_4) & (1 - p_4) q_4 & (1 -p_4) (1 - q_4) \\
    \end{bmatrix}
\]

Figure~\ref{fig:computed_probabilities_vs_theoretic_probabilities} shows a
regression line fitted to every pairwise interaction with a reported
\(\text{SSError}\) value (pairwise interactions with missing states were
omitted). This serves to validate the approach: a part from some edge cases the
relationship is consistent.

\begin{figure}[!htbp]
    \centering
    \includegraphics[width=.8\textwidth]{./assets/img/computed_probabilities_vs_theoretic_probabilities/main.pdf}
    \caption{The
        relationship between the steady state probabilities inferred from the
        measured transitions and the actual steady state probabilities. A linear
        regression line is included validating the approach.}
    \label{fig:computed_probabilities_vs_theoretic_probabilities}
\end{figure}


\end{document}
 turns and every match has been
repeated \documentclass[a4paper]{article}

\usepackage{amsmath}
\usepackage{amssymb}
\usepackage[margin=1.5cm,
            includefoot,
            footskip=30pt]{geometry}
\usepackage{layout}
\usepackage{graphicx}
\usepackage{subcaption}

\usepackage{biblatex}
\usepackage{pdfpages}

\bibliography{main.bib}

\title{Suspicion: Recognising and evaluating the effectiveness
       of extortion in the Iterated Prisoner's Dilemma}
\author{Vincent A. Knight \and Nikoleta E. Glynatsi}
\date{\today}



\begin{document}

\maketitle

\begin{abstract}
    The Iterated Prisoner's Dilemma is a model for rational and evolutionary
    interactive behaviour. It has applications both in the study of human social
    behaviour as well as in biology.
    It is used to understand when and how a rational individual might
    accept an immediate cost to their own utility for the direct benefit of
    another.

    Much attention has been given to a class of strategies called
    Zero Determinant strategies. It has been theoretically shown that these
    strategies can ``extort'' any player.

    In this work, an approach to identify if observed strategies are playing in
    an extortionate way is described. Furthermore, experimental analysis of
    a large tournament with \input{assets/tex/number_of_full_strategies/main.tex}
    strategies is considered. In this setting
    the most highly performing strategies do not play in an extortionate way
    against each other but do against lower performing strategies.
    This suggests that whilst the theory of Zero Determinant strategies
    indicates that memory is not of fundamental importance to the evolution of
    cooperative behaviour, this is incomplete.
\end{abstract}

\section{Introduction}\label{sec:introduction}

Agent based game theoretic models have become a stalwart of the underpinning
mathematics of interactive behaviours. One of the major pieces of work
in this area is the pair of original computer tournaments run by Robert
Axelrod~\cite{Axelrod1980, Axelrod1980a}. These tournaments pitted submitted
computer strategies against each other in plays of the Iterated Prisoner's
Dilemma. A common game where agents can choose to pay a slight cost to their
immediate utility in the hope of building a reputation. This has been used in
economic and evolutionary game theory to understand the evolution of cooperative
behaviour.

Recently, a class of strategies was described in~\cite{Press2012} that can
provably extort any given opponent. In~\cite{Hilbe2013, Moran1707} some
questions have already been asked about the true effectiveness of these
strategies in an evolutionary setting. Here another question is asked: is it
possible to recognise this extortionate behaviour? A mathematical procedure for
suspicion is presented: in the same way that the continued actions of an
extortionate individual might raise suspicion.

This work makes use of the Axelrod Python library~\cite{Knight2018, Knight2016}
with a large number of Prisoner Dilemma strategies available to give an
extensive numerical example of the ideas presented.  The approach is presented
in Section~\ref{sec:delta-zd-strategies}.  All of the code and data discussed
in Section~\ref{sec:numerical-experiments} is open sourced, archived and
written according to best scientific principles~\cite{Wilson2014}. The data
archive can be found at~\cite{vincent_knight_2018_1297075}.

\section{Recognising Extortion}\label{sec:delta-zd-strategies}

In~\cite{Press2012}, given a match between 2 memory-one strategies, the concept
of Zero Determinant (ZD) strategies is introduced. The main result of that paper
shows that given two memory one players \(p, q\in\mathbb{R}^4\) a linear
relationship between the players' scores could be forced by one of the players.

Using the notation of~\cite{Press2012}, assuming the utilities for player \(p\)
are given by \(S_x=(R, S, T, P)\) and for player \(q\) by \(S_y=(R, T, S, P)\)
and that the stationary scores of each player is given by \(S_X\) and \(S_Y\)
respectively. The main result of~\cite{Press2012} is that if

\begin{equation}\label{eqn:linear_relationship_for_p}
    \tilde p=\alpha S_x + \beta S_y + \gamma
\end{equation}

or

\begin{equation}\label{eqn:linear_relationship_for_q}
    \tilde q=\alpha S_x + \beta S_y + \gamma
\end{equation}

where \(\tilde p = (1 - p_1, 1 - p_2, p_3, p_4)\) and
\(\tilde q = (1 - q_1, 1 - q_2, q_3, q_4)\) then:

\begin{equation}
    \alpha S_X + \beta S_Y + \gamma = 0
\end{equation}

In~\cite{Press2012} a particular type of ZD strategy is defined: extortionate
strategies. If:

\begin{equation}\label{eqn:constraint_for_extortion}
    \gamma = - P(\alpha + \beta)
\end{equation}

then the player can ensure they get a score \(\chi\) times
larger than the opponent. This extortion coefficient is given by:

\begin{equation}\label{eqn:definition_of_chi}
    \chi=\frac{-\beta}{\alpha}
\end{equation}

Thus, if (\ref{eqn:constraint_for_extortion}) holds and \(\chi >1\) a player is
said to extort their opponent.
Here, the reverse problem is considered: given a
\(p\in\mathbb{R}^4\) how does one identify \(\alpha, \beta\) if they
exist and is the strategy in fact acting in an extortionate way?

These conditions correspond to:

\begin{align}
    \tilde p_1 & = \alpha R + \beta R - P (\alpha + \beta)
            \label{eqn:condition_for_tilde_p1}\\
    \tilde p_2 & = \alpha S + \beta T - P (\alpha + \beta)
            \label{eqn:condition_for_tilde_p2}\\
    \tilde p_3 & = \alpha T + \beta S - P (\alpha + \beta)
            \label{eqn:condition_for_tilde_p3}\\
    \tilde p_4 & = \alpha P + \beta P - P (\alpha + \beta)
            \label{eqn:condition_for_tilde_p4}
\end{align}

Equation (\ref{eqn:condition_for_tilde_p4}) ensures that \(p_4=\tilde p_4=0\).
Equations (\ref{eqn:condition_for_tilde_p1}-\ref{eqn:condition_for_tilde_p3})
can be used to eliminate \(\alpha, \beta\), giving:

\begin{equation}\label{eqn:planar_definition_of_extortion}
    \tilde p_1 = \frac{(R - P)(\tilde p_2 + \tilde p_3)}{S + T - 2P}
\end{equation}

with:

\begin{equation}\label{eqn:definition_of_chi}
    \chi = \frac{\tilde p_2 (P - T) + \tilde p_3 (S - P)}
                {\tilde p_2 (P - S) + \tilde p_3 (T - P)}
\end{equation}

Given a strategy \(p\in\mathbb{R}^{4\times 1}\) equations
(\ref{eqn:condition_for_tilde_p4}), (\ref{eqn:planar_definition_of_extortion}-\ref{eqn:definition_of_chi}) can be used to check if
a strategy is extortionate. The conditions correspond to:

\begin{align}
    p_1 & = \frac{(R-P)(p_2 + p_3) - R + T + S - P}{S + T - 2P}
     \label{eqn:condition_for_p1}\\
    p_4 & = 0 \label{eqn:condition_for_p4}\\
    1 & > p_2 + p_3\label{eqn:condition_for_chi}
\end{align}

The algebraic steps necessary to prove these results are available in the
supporting materials.

All extortionate strategies reside on a triangular (\ref{eqn:condition_for_chi})
plane (\ref{eqn:condition_for_p1}) in 3 dimensions (\ref{eqn:condition_for_p4}).
Using this formulation it can be seen that a necessary (but not sufficient)
condition for an extortionate strategy is that it cooperates on average less
than 50\% of the time when in a state of disagreement with the opponent.

As an example, consider the known extortionate strategy \(p=(8 / 9, 1 / 2, 1 /
3, 0)\) from~\cite{Stewart2012} which is referred to as \texttt{Extort-2}. In
this case, for the standard values of \((R, T, S, P)\) constraint
(\ref{eqn:condition_for_p1}) corresponds to:

\begin{equation}
    p_1 = \frac{2(p_2 + p_3) + 1}{3}
\end{equation}

It is clear that in this case all constraints hold.

This approach could in fact be used to confirm that a given strategy is acting
in an extortionate manner even if it is not a memory one strategy. However, in
practice, if a closed form for \(p\) is not known, then due to measurement
and/or numerical error this would not work.

This problem can be written in the following linear algebraic form where
\(x=(\alpha, \beta)\)
and \(p^*=(\tilde p_1 - 1, tilde_2 - 1, p_3)\):

\begin{equation}\label{eqn:linear_algebraic_equation_for_p}
    Cx= p^*
\end{equation}

\(C\) corresponds to equations
(\ref{eqn:condition_for_tilde_p1}-\ref{eqn:condition_for_tilde_p3}) and is
given by:

\begin{equation}\label{eqn:definition_of_C}
    C =
    \begin{bmatrix}
        R - P & R- P \\
        S - P & T- P \\
        T - P & S- P \\
    \end{bmatrix}
\end{equation}

Note that in general, equation (\ref{eqn:linear_algebraic_equation_for_p}) will
not necessarily have a solution. From the Rouch\'{e}-Capelli theorem if there is
a solution it is unique as \(\text{rank}(C)=2\) which is the dimension of the
variable \(x\). The best fitting \(x\) is found by minimizing:

\begin{equation}\label{eqn:r_squared}
    \text{SSError} = \|C x- p^*\|_2^2 = \sum_{i=1}^{3}\left((C\bar x)_i-p_i^*\right)^2
\end{equation}

Note that \(\text{SSError}\), which is the square of the Frobenius
norm~\cite{Golub2013}, becomes a measure of how close a strategy is to being an
extortionate strategy. Suspicion
of extortion then corresponds to a threshold on \(\text{SSError}\).

By observing interactions (human or otherwise), their memory one representation
can be inferred and this approach can be used to recognise extortionate
behaviour. The notion of comparing theoretic and actual plays of the IPD is not
novel, see for example~\cite{Rand2013}. Immediately it is noted that if the
environment is noisy~\cite{Wu1995} then no strategy can be considered to be
extortionate as \(p_4>0\).

In the next section, this idea will be illustrated by observing the interactions
that take place in a computer based tournament of the IPD\@.

\section{Numerical experiments}\label{sec:numerical-experiments}

In~\cite{Stewart2012} results from a tournament with
\input{./assets/tex/number_of_stewart_plotkin_strategies/main.tex} strategies,
was presented with specific consideration given to ZD strategies. This
tournament is reproduced here using the Axelrod-Python
project~\cite{Knight2016}. To obtain a good measure of the corresponding
transition rates for each strategy all matches have been run for
\input{assets/tex/number_of_turns/main.tex} turns and every match has been
repeated \input{assets/tex/number_of_repetitions/main.tex} times. All of this
interaction data is available at~\cite{vincent_knight_2018_1297075}. A good
match between the inferred Markov chain and the state distribution of the actual
interactions has been verified. Data for this is presented in the supplementary
materials.

Figure~\ref{fig:SSError_overall_in_stewart_plotkin} shows the \(\text{SSError}\)
values for all the strategies in the tournament, as reported
in~\cite{Stewart2012} the extortionate strategy (which has an expected
\(\text{SSError}\) approximately 0) gains a large number of wins.

\begin{figure}[!htbp]
    \centering
    \includegraphics[width=.8\textwidth]{./assets/img/SSError_overall_in_stewart_plotkin/main.pdf}
    \caption{\(\text{SSError}\) and state probabilities for the strategies
        of~\cite{Stewart2012}, ordered both by number of wins and overall score.
        Note that \(P(DC)\) is not shown as it corresponds to the transpose of
        \(P(CD)\). Cooperator and Defector are omitted as they do not visit all
        the states.}
    \label{fig:SSError_overall_in_stewart_plotkin}
\end{figure}

Here, the work of~\cite{Stewart2012} is extended by investigating a tournament
with \input{assets/tex/number_of_full_strategies/main.tex}
strategies.

The results of this analysis are shown in
Figure~\ref{fig:SSError_and_probabilities_in_full}. The top ranking strategies
by number of wins seem to be extortionate (but not against all strategies) and
it can be seen that a small sub group of strategies achieve mutual defection.
All the top ranking strategies according to score achieve mutual cooperation and
do not extort each other, however they
\textbf{do} exhibit extortionate behaviour towards a number of the lower ranking
strategies.

\begin{figure}[!htbp]
    \centering
    \includegraphics[width=.8\textwidth]{./assets/img/SSError_and_probabilities_in_full/main.pdf}
    \caption{\(\text{SSError}\) for the strategies for the full tournament. Only
    strategy interactions for which \(p_4=0\) and \(\chi>1\) are displayed.}
    \label{fig:SSError_and_probabilities_in_full}
\end{figure}

\section{Conclusion}\label{sec:conclusion}

This work defines an approach to measure whether or not a player is playing a
strategy that corresponds to an extortionate strategy as defined
in~\cite{Press2012}: a mathematical model for suspicion. Indeed, all
extortionate strategies have been
 classified as lying on a triangular plane.
This rigorous classification fails to be robust to small measurement error, thus
a statistical approach is proposed.
This is done through a linear algebraic approach for approximating the solution
of a linear system. Using this, a large number of pairwise interactions is
simulated and in fact very few strategies are found to act extortionately.

The work of~\cite{Press2012}, whilst showing that a clever approach to taking
advantage of another memory one strategy exists: this is incomplete. Whilst the
elegance of this result is very attractive, just as the simplicity of the
victory of Tit For Tat in Axelrod's original tournaments was, it is incomplete.
Extortionate strategies achieve a high number of wins but they do not
achieve a high score which corresponds to the fitness landscape in an
evolutionary sense. From the large number of interactions a payoff matrix \(S\)
can be measured where \(S_{ij}\) denotes the score (using standard values of
\((R, S, T, P) = (3, 0, 5, 1)\)) of the \(i\)th strategy
against the \(j\)th strategy. Using this, the replicator equation
describes the evolution of the system based on a population density fitness
function:

\begin{equation}\label{eqn:replicator_dynamics}
    \frac{dx}{dt} = x(S-x^TS x)
\end{equation}

Equation (\ref{eqn:replicator_dynamics}) is solved numerically through an
integration technique described in~\cite{Petzold1983} and
Figure~\ref{fig:replicator_dynamics} shows the evolution of the distribution of
the system: the various strategies are ranked by scores. It is clear to see that
only the high ranking strategies survive the evolutionary process (in fact,
only \input{./assets/img/replicator_dynamics/main.tex}
have a final distribution greater than \(10 ^ {-2}\)). This confirms the
findings of~\cite{Moran1707} in which sophisticated strategies resist
evolutionary invasion of shorter memory strategies. Recalling
Figure~\ref{fig:SSError_and_probabilities_in_full} this demonstrates that:

\begin{itemize}
    \item Cooperation emerges through the evolutionary process: the high scoring
        strategies do not exhibit extortionate behaviour towards each other.
    \item Extortionate strategies do not survive the evolutionary process.
\end{itemize}

\begin{figure}[!htbp]
    \centering
    \includegraphics[width=.8\textwidth]{./assets/img/replicator_dynamics/main.pdf}
    \caption{Numerical simulation of the replicator equation
    (\ref{eqn:replicator_dynamics}): strategies are ordered by score, only the strategies with a high score survive the evolutionary process.}
    \label{fig:replicator_dynamics}
\end{figure}

This work can be used to classify plays of the IPD\@: data can be collected from
actual interactions (in lab or in the field). Furthermore, this allows for a
classification method similar to the notion of fingerprinting presented
in~\cite{Ashlock2008}. Trained strategies can potentially be classified as
extortionate or not or it could be possible to even constrain the reinforcement
learning approaches that are becoming prevalent in the literature.
Alternatively, this mathematical approach for recognising extortion could be
used in sophisticated strategies to defend against invasion. Arguably, some of
the strategies considered here exhibit this behaviour, indeed as described
in~\cite{Harper2017}, the top ranking strategies in the full tournament are
obtained using evolutionary reinforcement learning techniques, thus, suspicion
of extortionate behaviour could in fact be an evolutionary trait.

\section*{Acknowledgements}

The following open source software libraries were used in this research:

\begin{itemize}
    \item The Axelrod ~\cite{Knight2016, Knight2018} library (IPD strategies and
        tournaments).
    \item The sympy library~\cite{Meurer2017} (verification of all symbolic
        calculations).
    \item The matplotlib~\cite{Droettboom2018} library (visualisation).
    \item The pandas~\cite{Structures2010}, dask~\cite{Dask2016} and
        NumPy~\cite{Oliphant2015} libraries (data manipulation).
    \item The SciPy~\cite{Jones2001} library (numerical integration of the
        replicator equation).
\end{itemize}

This work was performed using the computational facilities of the Advanced
Research Computing @ Cardiff (ARCCA) Division, Cardiff University.

\printbibliography

\newpage
\section*{Supplementary materials}

\includepdf{assets/pdf/proof_of_form_of_extortionate_strategies/main.pdf}

\newpage

Using the pair wise interactions the transition rates \(p,
q\) can be measured and the steady state probabilities inferred and compared to
the actual probabilities of each state.
This is done numerically by computing the singular eigenvector of the
matrix \(A\) \cite{Stewart2009}:

\[
    A =
    \begin{bmatrix}
        p_1 q_1 & p_1 (1 - q_1) & (1 - p_1) q_1 & (1 -p_1) (1 - q_1) \\
        p_2 q_2 & p_2 (1 - q_2) & (1 - p_2) q_2 & (1 -p_2) (1 - q_2) \\
        p_3 q_3 & p_3 (1 - q_3) & (1 - p_3) q_3 & (1 -p_3) (1 - q_3) \\
        p_4 q_4 & p_4 (1 - q_4) & (1 - p_4) q_4 & (1 -p_4) (1 - q_4) \\
    \end{bmatrix}
\]

Figure~\ref{fig:computed_probabilities_vs_theoretic_probabilities} shows a
regression line fitted to every pairwise interaction with a reported
\(\text{SSError}\) value (pairwise interactions with missing states were
omitted). This serves to validate the approach: a part from some edge cases the
relationship is consistent.

\begin{figure}[!htbp]
    \centering
    \includegraphics[width=.8\textwidth]{./assets/img/computed_probabilities_vs_theoretic_probabilities/main.pdf}
    \caption{The
        relationship between the steady state probabilities inferred from the
        measured transitions and the actual steady state probabilities. A linear
        regression line is included validating the approach.}
    \label{fig:computed_probabilities_vs_theoretic_probabilities}
\end{figure}


\end{document}
 times. All of this
interaction data is available at~\cite{vincent_knight_2018_1297075}. A good
match between the inferred Markov chain and the state distribution of the actual
interactions has been verified. Data for this is presented in the supplementary
materials.

Figure~\ref{fig:SSError_overall_in_stewart_plotkin} shows the \(\text{SSError}\)
values for all the strategies in the tournament, as reported
in~\cite{Stewart2012} the extortionate strategy (which has an expected
\(\text{SSError}\) approximately 0) gains a large number of wins.

\begin{figure}[!htbp]
    \centering
    \includegraphics[width=.8\textwidth]{./assets/img/SSError_overall_in_stewart_plotkin/main.pdf}
    \caption{\(\text{SSError}\) and state probabilities for the strategies
        of~\cite{Stewart2012}, ordered both by number of wins and overall score.
        Note that \(P(DC)\) is not shown as it corresponds to the transpose of
        \(P(CD)\). Cooperator and Defector are omitted as they do not visit all
        the states.}
    \label{fig:SSError_overall_in_stewart_plotkin}
\end{figure}

Here, the work of~\cite{Stewart2012} is extended by investigating a tournament
with \documentclass[a4paper]{article}

\usepackage{amsmath}
\usepackage{amssymb}
\usepackage[margin=1.5cm,
            includefoot,
            footskip=30pt]{geometry}
\usepackage{layout}
\usepackage{graphicx}
\usepackage{subcaption}

\usepackage{biblatex}
\usepackage{pdfpages}

\bibliography{main.bib}

\title{Suspicion: Recognising and evaluating the effectiveness
       of extortion in the Iterated Prisoner's Dilemma}
\author{Vincent A. Knight \and Nikoleta E. Glynatsi}
\date{\today}



\begin{document}

\maketitle

\begin{abstract}
    The Iterated Prisoner's Dilemma is a model for rational and evolutionary
    interactive behaviour. It has applications both in the study of human social
    behaviour as well as in biology.
    It is used to understand when and how a rational individual might
    accept an immediate cost to their own utility for the direct benefit of
    another.

    Much attention has been given to a class of strategies called
    Zero Determinant strategies. It has been theoretically shown that these
    strategies can ``extort'' any player.

    In this work, an approach to identify if observed strategies are playing in
    an extortionate way is described. Furthermore, experimental analysis of
    a large tournament with \input{assets/tex/number_of_full_strategies/main.tex}
    strategies is considered. In this setting
    the most highly performing strategies do not play in an extortionate way
    against each other but do against lower performing strategies.
    This suggests that whilst the theory of Zero Determinant strategies
    indicates that memory is not of fundamental importance to the evolution of
    cooperative behaviour, this is incomplete.
\end{abstract}

\section{Introduction}\label{sec:introduction}

Agent based game theoretic models have become a stalwart of the underpinning
mathematics of interactive behaviours. One of the major pieces of work
in this area is the pair of original computer tournaments run by Robert
Axelrod~\cite{Axelrod1980, Axelrod1980a}. These tournaments pitted submitted
computer strategies against each other in plays of the Iterated Prisoner's
Dilemma. A common game where agents can choose to pay a slight cost to their
immediate utility in the hope of building a reputation. This has been used in
economic and evolutionary game theory to understand the evolution of cooperative
behaviour.

Recently, a class of strategies was described in~\cite{Press2012} that can
provably extort any given opponent. In~\cite{Hilbe2013, Moran1707} some
questions have already been asked about the true effectiveness of these
strategies in an evolutionary setting. Here another question is asked: is it
possible to recognise this extortionate behaviour? A mathematical procedure for
suspicion is presented: in the same way that the continued actions of an
extortionate individual might raise suspicion.

This work makes use of the Axelrod Python library~\cite{Knight2018, Knight2016}
with a large number of Prisoner Dilemma strategies available to give an
extensive numerical example of the ideas presented.  The approach is presented
in Section~\ref{sec:delta-zd-strategies}.  All of the code and data discussed
in Section~\ref{sec:numerical-experiments} is open sourced, archived and
written according to best scientific principles~\cite{Wilson2014}. The data
archive can be found at~\cite{vincent_knight_2018_1297075}.

\section{Recognising Extortion}\label{sec:delta-zd-strategies}

In~\cite{Press2012}, given a match between 2 memory-one strategies, the concept
of Zero Determinant (ZD) strategies is introduced. The main result of that paper
shows that given two memory one players \(p, q\in\mathbb{R}^4\) a linear
relationship between the players' scores could be forced by one of the players.

Using the notation of~\cite{Press2012}, assuming the utilities for player \(p\)
are given by \(S_x=(R, S, T, P)\) and for player \(q\) by \(S_y=(R, T, S, P)\)
and that the stationary scores of each player is given by \(S_X\) and \(S_Y\)
respectively. The main result of~\cite{Press2012} is that if

\begin{equation}\label{eqn:linear_relationship_for_p}
    \tilde p=\alpha S_x + \beta S_y + \gamma
\end{equation}

or

\begin{equation}\label{eqn:linear_relationship_for_q}
    \tilde q=\alpha S_x + \beta S_y + \gamma
\end{equation}

where \(\tilde p = (1 - p_1, 1 - p_2, p_3, p_4)\) and
\(\tilde q = (1 - q_1, 1 - q_2, q_3, q_4)\) then:

\begin{equation}
    \alpha S_X + \beta S_Y + \gamma = 0
\end{equation}

In~\cite{Press2012} a particular type of ZD strategy is defined: extortionate
strategies. If:

\begin{equation}\label{eqn:constraint_for_extortion}
    \gamma = - P(\alpha + \beta)
\end{equation}

then the player can ensure they get a score \(\chi\) times
larger than the opponent. This extortion coefficient is given by:

\begin{equation}\label{eqn:definition_of_chi}
    \chi=\frac{-\beta}{\alpha}
\end{equation}

Thus, if (\ref{eqn:constraint_for_extortion}) holds and \(\chi >1\) a player is
said to extort their opponent.
Here, the reverse problem is considered: given a
\(p\in\mathbb{R}^4\) how does one identify \(\alpha, \beta\) if they
exist and is the strategy in fact acting in an extortionate way?

These conditions correspond to:

\begin{align}
    \tilde p_1 & = \alpha R + \beta R - P (\alpha + \beta)
            \label{eqn:condition_for_tilde_p1}\\
    \tilde p_2 & = \alpha S + \beta T - P (\alpha + \beta)
            \label{eqn:condition_for_tilde_p2}\\
    \tilde p_3 & = \alpha T + \beta S - P (\alpha + \beta)
            \label{eqn:condition_for_tilde_p3}\\
    \tilde p_4 & = \alpha P + \beta P - P (\alpha + \beta)
            \label{eqn:condition_for_tilde_p4}
\end{align}

Equation (\ref{eqn:condition_for_tilde_p4}) ensures that \(p_4=\tilde p_4=0\).
Equations (\ref{eqn:condition_for_tilde_p1}-\ref{eqn:condition_for_tilde_p3})
can be used to eliminate \(\alpha, \beta\), giving:

\begin{equation}\label{eqn:planar_definition_of_extortion}
    \tilde p_1 = \frac{(R - P)(\tilde p_2 + \tilde p_3)}{S + T - 2P}
\end{equation}

with:

\begin{equation}\label{eqn:definition_of_chi}
    \chi = \frac{\tilde p_2 (P - T) + \tilde p_3 (S - P)}
                {\tilde p_2 (P - S) + \tilde p_3 (T - P)}
\end{equation}

Given a strategy \(p\in\mathbb{R}^{4\times 1}\) equations
(\ref{eqn:condition_for_tilde_p4}), (\ref{eqn:planar_definition_of_extortion}-\ref{eqn:definition_of_chi}) can be used to check if
a strategy is extortionate. The conditions correspond to:

\begin{align}
    p_1 & = \frac{(R-P)(p_2 + p_3) - R + T + S - P}{S + T - 2P}
     \label{eqn:condition_for_p1}\\
    p_4 & = 0 \label{eqn:condition_for_p4}\\
    1 & > p_2 + p_3\label{eqn:condition_for_chi}
\end{align}

The algebraic steps necessary to prove these results are available in the
supporting materials.

All extortionate strategies reside on a triangular (\ref{eqn:condition_for_chi})
plane (\ref{eqn:condition_for_p1}) in 3 dimensions (\ref{eqn:condition_for_p4}).
Using this formulation it can be seen that a necessary (but not sufficient)
condition for an extortionate strategy is that it cooperates on average less
than 50\% of the time when in a state of disagreement with the opponent.

As an example, consider the known extortionate strategy \(p=(8 / 9, 1 / 2, 1 /
3, 0)\) from~\cite{Stewart2012} which is referred to as \texttt{Extort-2}. In
this case, for the standard values of \((R, T, S, P)\) constraint
(\ref{eqn:condition_for_p1}) corresponds to:

\begin{equation}
    p_1 = \frac{2(p_2 + p_3) + 1}{3}
\end{equation}

It is clear that in this case all constraints hold.

This approach could in fact be used to confirm that a given strategy is acting
in an extortionate manner even if it is not a memory one strategy. However, in
practice, if a closed form for \(p\) is not known, then due to measurement
and/or numerical error this would not work.

This problem can be written in the following linear algebraic form where
\(x=(\alpha, \beta)\)
and \(p^*=(\tilde p_1 - 1, tilde_2 - 1, p_3)\):

\begin{equation}\label{eqn:linear_algebraic_equation_for_p}
    Cx= p^*
\end{equation}

\(C\) corresponds to equations
(\ref{eqn:condition_for_tilde_p1}-\ref{eqn:condition_for_tilde_p3}) and is
given by:

\begin{equation}\label{eqn:definition_of_C}
    C =
    \begin{bmatrix}
        R - P & R- P \\
        S - P & T- P \\
        T - P & S- P \\
    \end{bmatrix}
\end{equation}

Note that in general, equation (\ref{eqn:linear_algebraic_equation_for_p}) will
not necessarily have a solution. From the Rouch\'{e}-Capelli theorem if there is
a solution it is unique as \(\text{rank}(C)=2\) which is the dimension of the
variable \(x\). The best fitting \(x\) is found by minimizing:

\begin{equation}\label{eqn:r_squared}
    \text{SSError} = \|C x- p^*\|_2^2 = \sum_{i=1}^{3}\left((C\bar x)_i-p_i^*\right)^2
\end{equation}

Note that \(\text{SSError}\), which is the square of the Frobenius
norm~\cite{Golub2013}, becomes a measure of how close a strategy is to being an
extortionate strategy. Suspicion
of extortion then corresponds to a threshold on \(\text{SSError}\).

By observing interactions (human or otherwise), their memory one representation
can be inferred and this approach can be used to recognise extortionate
behaviour. The notion of comparing theoretic and actual plays of the IPD is not
novel, see for example~\cite{Rand2013}. Immediately it is noted that if the
environment is noisy~\cite{Wu1995} then no strategy can be considered to be
extortionate as \(p_4>0\).

In the next section, this idea will be illustrated by observing the interactions
that take place in a computer based tournament of the IPD\@.

\section{Numerical experiments}\label{sec:numerical-experiments}

In~\cite{Stewart2012} results from a tournament with
\input{./assets/tex/number_of_stewart_plotkin_strategies/main.tex} strategies,
was presented with specific consideration given to ZD strategies. This
tournament is reproduced here using the Axelrod-Python
project~\cite{Knight2016}. To obtain a good measure of the corresponding
transition rates for each strategy all matches have been run for
\input{assets/tex/number_of_turns/main.tex} turns and every match has been
repeated \input{assets/tex/number_of_repetitions/main.tex} times. All of this
interaction data is available at~\cite{vincent_knight_2018_1297075}. A good
match between the inferred Markov chain and the state distribution of the actual
interactions has been verified. Data for this is presented in the supplementary
materials.

Figure~\ref{fig:SSError_overall_in_stewart_plotkin} shows the \(\text{SSError}\)
values for all the strategies in the tournament, as reported
in~\cite{Stewart2012} the extortionate strategy (which has an expected
\(\text{SSError}\) approximately 0) gains a large number of wins.

\begin{figure}[!htbp]
    \centering
    \includegraphics[width=.8\textwidth]{./assets/img/SSError_overall_in_stewart_plotkin/main.pdf}
    \caption{\(\text{SSError}\) and state probabilities for the strategies
        of~\cite{Stewart2012}, ordered both by number of wins and overall score.
        Note that \(P(DC)\) is not shown as it corresponds to the transpose of
        \(P(CD)\). Cooperator and Defector are omitted as they do not visit all
        the states.}
    \label{fig:SSError_overall_in_stewart_plotkin}
\end{figure}

Here, the work of~\cite{Stewart2012} is extended by investigating a tournament
with \input{assets/tex/number_of_full_strategies/main.tex}
strategies.

The results of this analysis are shown in
Figure~\ref{fig:SSError_and_probabilities_in_full}. The top ranking strategies
by number of wins seem to be extortionate (but not against all strategies) and
it can be seen that a small sub group of strategies achieve mutual defection.
All the top ranking strategies according to score achieve mutual cooperation and
do not extort each other, however they
\textbf{do} exhibit extortionate behaviour towards a number of the lower ranking
strategies.

\begin{figure}[!htbp]
    \centering
    \includegraphics[width=.8\textwidth]{./assets/img/SSError_and_probabilities_in_full/main.pdf}
    \caption{\(\text{SSError}\) for the strategies for the full tournament. Only
    strategy interactions for which \(p_4=0\) and \(\chi>1\) are displayed.}
    \label{fig:SSError_and_probabilities_in_full}
\end{figure}

\section{Conclusion}\label{sec:conclusion}

This work defines an approach to measure whether or not a player is playing a
strategy that corresponds to an extortionate strategy as defined
in~\cite{Press2012}: a mathematical model for suspicion. Indeed, all
extortionate strategies have been
 classified as lying on a triangular plane.
This rigorous classification fails to be robust to small measurement error, thus
a statistical approach is proposed.
This is done through a linear algebraic approach for approximating the solution
of a linear system. Using this, a large number of pairwise interactions is
simulated and in fact very few strategies are found to act extortionately.

The work of~\cite{Press2012}, whilst showing that a clever approach to taking
advantage of another memory one strategy exists: this is incomplete. Whilst the
elegance of this result is very attractive, just as the simplicity of the
victory of Tit For Tat in Axelrod's original tournaments was, it is incomplete.
Extortionate strategies achieve a high number of wins but they do not
achieve a high score which corresponds to the fitness landscape in an
evolutionary sense. From the large number of interactions a payoff matrix \(S\)
can be measured where \(S_{ij}\) denotes the score (using standard values of
\((R, S, T, P) = (3, 0, 5, 1)\)) of the \(i\)th strategy
against the \(j\)th strategy. Using this, the replicator equation
describes the evolution of the system based on a population density fitness
function:

\begin{equation}\label{eqn:replicator_dynamics}
    \frac{dx}{dt} = x(S-x^TS x)
\end{equation}

Equation (\ref{eqn:replicator_dynamics}) is solved numerically through an
integration technique described in~\cite{Petzold1983} and
Figure~\ref{fig:replicator_dynamics} shows the evolution of the distribution of
the system: the various strategies are ranked by scores. It is clear to see that
only the high ranking strategies survive the evolutionary process (in fact,
only \input{./assets/img/replicator_dynamics/main.tex}
have a final distribution greater than \(10 ^ {-2}\)). This confirms the
findings of~\cite{Moran1707} in which sophisticated strategies resist
evolutionary invasion of shorter memory strategies. Recalling
Figure~\ref{fig:SSError_and_probabilities_in_full} this demonstrates that:

\begin{itemize}
    \item Cooperation emerges through the evolutionary process: the high scoring
        strategies do not exhibit extortionate behaviour towards each other.
    \item Extortionate strategies do not survive the evolutionary process.
\end{itemize}

\begin{figure}[!htbp]
    \centering
    \includegraphics[width=.8\textwidth]{./assets/img/replicator_dynamics/main.pdf}
    \caption{Numerical simulation of the replicator equation
    (\ref{eqn:replicator_dynamics}): strategies are ordered by score, only the strategies with a high score survive the evolutionary process.}
    \label{fig:replicator_dynamics}
\end{figure}

This work can be used to classify plays of the IPD\@: data can be collected from
actual interactions (in lab or in the field). Furthermore, this allows for a
classification method similar to the notion of fingerprinting presented
in~\cite{Ashlock2008}. Trained strategies can potentially be classified as
extortionate or not or it could be possible to even constrain the reinforcement
learning approaches that are becoming prevalent in the literature.
Alternatively, this mathematical approach for recognising extortion could be
used in sophisticated strategies to defend against invasion. Arguably, some of
the strategies considered here exhibit this behaviour, indeed as described
in~\cite{Harper2017}, the top ranking strategies in the full tournament are
obtained using evolutionary reinforcement learning techniques, thus, suspicion
of extortionate behaviour could in fact be an evolutionary trait.

\section*{Acknowledgements}

The following open source software libraries were used in this research:

\begin{itemize}
    \item The Axelrod ~\cite{Knight2016, Knight2018} library (IPD strategies and
        tournaments).
    \item The sympy library~\cite{Meurer2017} (verification of all symbolic
        calculations).
    \item The matplotlib~\cite{Droettboom2018} library (visualisation).
    \item The pandas~\cite{Structures2010}, dask~\cite{Dask2016} and
        NumPy~\cite{Oliphant2015} libraries (data manipulation).
    \item The SciPy~\cite{Jones2001} library (numerical integration of the
        replicator equation).
\end{itemize}

This work was performed using the computational facilities of the Advanced
Research Computing @ Cardiff (ARCCA) Division, Cardiff University.

\printbibliography

\newpage
\section*{Supplementary materials}

\includepdf{assets/pdf/proof_of_form_of_extortionate_strategies/main.pdf}

\newpage

Using the pair wise interactions the transition rates \(p,
q\) can be measured and the steady state probabilities inferred and compared to
the actual probabilities of each state.
This is done numerically by computing the singular eigenvector of the
matrix \(A\) \cite{Stewart2009}:

\[
    A =
    \begin{bmatrix}
        p_1 q_1 & p_1 (1 - q_1) & (1 - p_1) q_1 & (1 -p_1) (1 - q_1) \\
        p_2 q_2 & p_2 (1 - q_2) & (1 - p_2) q_2 & (1 -p_2) (1 - q_2) \\
        p_3 q_3 & p_3 (1 - q_3) & (1 - p_3) q_3 & (1 -p_3) (1 - q_3) \\
        p_4 q_4 & p_4 (1 - q_4) & (1 - p_4) q_4 & (1 -p_4) (1 - q_4) \\
    \end{bmatrix}
\]

Figure~\ref{fig:computed_probabilities_vs_theoretic_probabilities} shows a
regression line fitted to every pairwise interaction with a reported
\(\text{SSError}\) value (pairwise interactions with missing states were
omitted). This serves to validate the approach: a part from some edge cases the
relationship is consistent.

\begin{figure}[!htbp]
    \centering
    \includegraphics[width=.8\textwidth]{./assets/img/computed_probabilities_vs_theoretic_probabilities/main.pdf}
    \caption{The
        relationship between the steady state probabilities inferred from the
        measured transitions and the actual steady state probabilities. A linear
        regression line is included validating the approach.}
    \label{fig:computed_probabilities_vs_theoretic_probabilities}
\end{figure}


\end{document}

strategies.

The results of this analysis are shown in
Figure~\ref{fig:SSError_and_probabilities_in_full}. The top ranking strategies
by number of wins seem to be extortionate (but not against all strategies) and
it can be seen that a small sub group of strategies achieve mutual defection.
All the top ranking strategies according to score achieve mutual cooperation and
do not extort each other, however they
\textbf{do} exhibit extortionate behaviour towards a number of the lower ranking
strategies.

\begin{figure}[!htbp]
    \centering
    \includegraphics[width=.8\textwidth]{./assets/img/SSError_and_probabilities_in_full/main.pdf}
    \caption{\(\text{SSError}\) for the strategies for the full tournament. Only
    strategy interactions for which \(p_4=0\) and \(\chi>1\) are displayed.}
    \label{fig:SSError_and_probabilities_in_full}
\end{figure}

\section{Conclusion}\label{sec:conclusion}

This work defines an approach to measure whether or not a player is playing a
strategy that corresponds to an extortionate strategy as defined
in~\cite{Press2012}: a mathematical model for suspicion. Indeed, all
extortionate strategies have been
 classified as lying on a triangular plane.
This rigorous classification fails to be robust to small measurement error, thus
a statistical approach is proposed.
This is done through a linear algebraic approach for approximating the solution
of a linear system. Using this, a large number of pairwise interactions is
simulated and in fact very few strategies are found to act extortionately.

The work of~\cite{Press2012}, whilst showing that a clever approach to taking
advantage of another memory one strategy exists: this is incomplete. Whilst the
elegance of this result is very attractive, just as the simplicity of the
victory of Tit For Tat in Axelrod's original tournaments was, it is incomplete.
Extortionate strategies achieve a high number of wins but they do not
achieve a high score which corresponds to the fitness landscape in an
evolutionary sense. From the large number of interactions a payoff matrix \(S\)
can be measured where \(S_{ij}\) denotes the score (using standard values of
\((R, S, T, P) = (3, 0, 5, 1)\)) of the \(i\)th strategy
against the \(j\)th strategy. Using this, the replicator equation
describes the evolution of the system based on a population density fitness
function:

\begin{equation}\label{eqn:replicator_dynamics}
    \frac{dx}{dt} = x(S-x^TS x)
\end{equation}

Equation (\ref{eqn:replicator_dynamics}) is solved numerically through an
integration technique described in~\cite{Petzold1983} and
Figure~\ref{fig:replicator_dynamics} shows the evolution of the distribution of
the system: the various strategies are ranked by scores. It is clear to see that
only the high ranking strategies survive the evolutionary process (in fact,
only \documentclass[a4paper]{article}

\usepackage{amsmath}
\usepackage{amssymb}
\usepackage[margin=1.5cm,
            includefoot,
            footskip=30pt]{geometry}
\usepackage{layout}
\usepackage{graphicx}
\usepackage{subcaption}

\usepackage{biblatex}
\usepackage{pdfpages}

\bibliography{main.bib}

\title{Suspicion: Recognising and evaluating the effectiveness
       of extortion in the Iterated Prisoner's Dilemma}
\author{Vincent A. Knight \and Nikoleta E. Glynatsi}
\date{\today}



\begin{document}

\maketitle

\begin{abstract}
    The Iterated Prisoner's Dilemma is a model for rational and evolutionary
    interactive behaviour. It has applications both in the study of human social
    behaviour as well as in biology.
    It is used to understand when and how a rational individual might
    accept an immediate cost to their own utility for the direct benefit of
    another.

    Much attention has been given to a class of strategies called
    Zero Determinant strategies. It has been theoretically shown that these
    strategies can ``extort'' any player.

    In this work, an approach to identify if observed strategies are playing in
    an extortionate way is described. Furthermore, experimental analysis of
    a large tournament with \input{assets/tex/number_of_full_strategies/main.tex}
    strategies is considered. In this setting
    the most highly performing strategies do not play in an extortionate way
    against each other but do against lower performing strategies.
    This suggests that whilst the theory of Zero Determinant strategies
    indicates that memory is not of fundamental importance to the evolution of
    cooperative behaviour, this is incomplete.
\end{abstract}

\section{Introduction}\label{sec:introduction}

Agent based game theoretic models have become a stalwart of the underpinning
mathematics of interactive behaviours. One of the major pieces of work
in this area is the pair of original computer tournaments run by Robert
Axelrod~\cite{Axelrod1980, Axelrod1980a}. These tournaments pitted submitted
computer strategies against each other in plays of the Iterated Prisoner's
Dilemma. A common game where agents can choose to pay a slight cost to their
immediate utility in the hope of building a reputation. This has been used in
economic and evolutionary game theory to understand the evolution of cooperative
behaviour.

Recently, a class of strategies was described in~\cite{Press2012} that can
provably extort any given opponent. In~\cite{Hilbe2013, Moran1707} some
questions have already been asked about the true effectiveness of these
strategies in an evolutionary setting. Here another question is asked: is it
possible to recognise this extortionate behaviour? A mathematical procedure for
suspicion is presented: in the same way that the continued actions of an
extortionate individual might raise suspicion.

This work makes use of the Axelrod Python library~\cite{Knight2018, Knight2016}
with a large number of Prisoner Dilemma strategies available to give an
extensive numerical example of the ideas presented.  The approach is presented
in Section~\ref{sec:delta-zd-strategies}.  All of the code and data discussed
in Section~\ref{sec:numerical-experiments} is open sourced, archived and
written according to best scientific principles~\cite{Wilson2014}. The data
archive can be found at~\cite{vincent_knight_2018_1297075}.

\section{Recognising Extortion}\label{sec:delta-zd-strategies}

In~\cite{Press2012}, given a match between 2 memory-one strategies, the concept
of Zero Determinant (ZD) strategies is introduced. The main result of that paper
shows that given two memory one players \(p, q\in\mathbb{R}^4\) a linear
relationship between the players' scores could be forced by one of the players.

Using the notation of~\cite{Press2012}, assuming the utilities for player \(p\)
are given by \(S_x=(R, S, T, P)\) and for player \(q\) by \(S_y=(R, T, S, P)\)
and that the stationary scores of each player is given by \(S_X\) and \(S_Y\)
respectively. The main result of~\cite{Press2012} is that if

\begin{equation}\label{eqn:linear_relationship_for_p}
    \tilde p=\alpha S_x + \beta S_y + \gamma
\end{equation}

or

\begin{equation}\label{eqn:linear_relationship_for_q}
    \tilde q=\alpha S_x + \beta S_y + \gamma
\end{equation}

where \(\tilde p = (1 - p_1, 1 - p_2, p_3, p_4)\) and
\(\tilde q = (1 - q_1, 1 - q_2, q_3, q_4)\) then:

\begin{equation}
    \alpha S_X + \beta S_Y + \gamma = 0
\end{equation}

In~\cite{Press2012} a particular type of ZD strategy is defined: extortionate
strategies. If:

\begin{equation}\label{eqn:constraint_for_extortion}
    \gamma = - P(\alpha + \beta)
\end{equation}

then the player can ensure they get a score \(\chi\) times
larger than the opponent. This extortion coefficient is given by:

\begin{equation}\label{eqn:definition_of_chi}
    \chi=\frac{-\beta}{\alpha}
\end{equation}

Thus, if (\ref{eqn:constraint_for_extortion}) holds and \(\chi >1\) a player is
said to extort their opponent.
Here, the reverse problem is considered: given a
\(p\in\mathbb{R}^4\) how does one identify \(\alpha, \beta\) if they
exist and is the strategy in fact acting in an extortionate way?

These conditions correspond to:

\begin{align}
    \tilde p_1 & = \alpha R + \beta R - P (\alpha + \beta)
            \label{eqn:condition_for_tilde_p1}\\
    \tilde p_2 & = \alpha S + \beta T - P (\alpha + \beta)
            \label{eqn:condition_for_tilde_p2}\\
    \tilde p_3 & = \alpha T + \beta S - P (\alpha + \beta)
            \label{eqn:condition_for_tilde_p3}\\
    \tilde p_4 & = \alpha P + \beta P - P (\alpha + \beta)
            \label{eqn:condition_for_tilde_p4}
\end{align}

Equation (\ref{eqn:condition_for_tilde_p4}) ensures that \(p_4=\tilde p_4=0\).
Equations (\ref{eqn:condition_for_tilde_p1}-\ref{eqn:condition_for_tilde_p3})
can be used to eliminate \(\alpha, \beta\), giving:

\begin{equation}\label{eqn:planar_definition_of_extortion}
    \tilde p_1 = \frac{(R - P)(\tilde p_2 + \tilde p_3)}{S + T - 2P}
\end{equation}

with:

\begin{equation}\label{eqn:definition_of_chi}
    \chi = \frac{\tilde p_2 (P - T) + \tilde p_3 (S - P)}
                {\tilde p_2 (P - S) + \tilde p_3 (T - P)}
\end{equation}

Given a strategy \(p\in\mathbb{R}^{4\times 1}\) equations
(\ref{eqn:condition_for_tilde_p4}), (\ref{eqn:planar_definition_of_extortion}-\ref{eqn:definition_of_chi}) can be used to check if
a strategy is extortionate. The conditions correspond to:

\begin{align}
    p_1 & = \frac{(R-P)(p_2 + p_3) - R + T + S - P}{S + T - 2P}
     \label{eqn:condition_for_p1}\\
    p_4 & = 0 \label{eqn:condition_for_p4}\\
    1 & > p_2 + p_3\label{eqn:condition_for_chi}
\end{align}

The algebraic steps necessary to prove these results are available in the
supporting materials.

All extortionate strategies reside on a triangular (\ref{eqn:condition_for_chi})
plane (\ref{eqn:condition_for_p1}) in 3 dimensions (\ref{eqn:condition_for_p4}).
Using this formulation it can be seen that a necessary (but not sufficient)
condition for an extortionate strategy is that it cooperates on average less
than 50\% of the time when in a state of disagreement with the opponent.

As an example, consider the known extortionate strategy \(p=(8 / 9, 1 / 2, 1 /
3, 0)\) from~\cite{Stewart2012} which is referred to as \texttt{Extort-2}. In
this case, for the standard values of \((R, T, S, P)\) constraint
(\ref{eqn:condition_for_p1}) corresponds to:

\begin{equation}
    p_1 = \frac{2(p_2 + p_3) + 1}{3}
\end{equation}

It is clear that in this case all constraints hold.

This approach could in fact be used to confirm that a given strategy is acting
in an extortionate manner even if it is not a memory one strategy. However, in
practice, if a closed form for \(p\) is not known, then due to measurement
and/or numerical error this would not work.

This problem can be written in the following linear algebraic form where
\(x=(\alpha, \beta)\)
and \(p^*=(\tilde p_1 - 1, tilde_2 - 1, p_3)\):

\begin{equation}\label{eqn:linear_algebraic_equation_for_p}
    Cx= p^*
\end{equation}

\(C\) corresponds to equations
(\ref{eqn:condition_for_tilde_p1}-\ref{eqn:condition_for_tilde_p3}) and is
given by:

\begin{equation}\label{eqn:definition_of_C}
    C =
    \begin{bmatrix}
        R - P & R- P \\
        S - P & T- P \\
        T - P & S- P \\
    \end{bmatrix}
\end{equation}

Note that in general, equation (\ref{eqn:linear_algebraic_equation_for_p}) will
not necessarily have a solution. From the Rouch\'{e}-Capelli theorem if there is
a solution it is unique as \(\text{rank}(C)=2\) which is the dimension of the
variable \(x\). The best fitting \(x\) is found by minimizing:

\begin{equation}\label{eqn:r_squared}
    \text{SSError} = \|C x- p^*\|_2^2 = \sum_{i=1}^{3}\left((C\bar x)_i-p_i^*\right)^2
\end{equation}

Note that \(\text{SSError}\), which is the square of the Frobenius
norm~\cite{Golub2013}, becomes a measure of how close a strategy is to being an
extortionate strategy. Suspicion
of extortion then corresponds to a threshold on \(\text{SSError}\).

By observing interactions (human or otherwise), their memory one representation
can be inferred and this approach can be used to recognise extortionate
behaviour. The notion of comparing theoretic and actual plays of the IPD is not
novel, see for example~\cite{Rand2013}. Immediately it is noted that if the
environment is noisy~\cite{Wu1995} then no strategy can be considered to be
extortionate as \(p_4>0\).

In the next section, this idea will be illustrated by observing the interactions
that take place in a computer based tournament of the IPD\@.

\section{Numerical experiments}\label{sec:numerical-experiments}

In~\cite{Stewart2012} results from a tournament with
\input{./assets/tex/number_of_stewart_plotkin_strategies/main.tex} strategies,
was presented with specific consideration given to ZD strategies. This
tournament is reproduced here using the Axelrod-Python
project~\cite{Knight2016}. To obtain a good measure of the corresponding
transition rates for each strategy all matches have been run for
\input{assets/tex/number_of_turns/main.tex} turns and every match has been
repeated \input{assets/tex/number_of_repetitions/main.tex} times. All of this
interaction data is available at~\cite{vincent_knight_2018_1297075}. A good
match between the inferred Markov chain and the state distribution of the actual
interactions has been verified. Data for this is presented in the supplementary
materials.

Figure~\ref{fig:SSError_overall_in_stewart_plotkin} shows the \(\text{SSError}\)
values for all the strategies in the tournament, as reported
in~\cite{Stewart2012} the extortionate strategy (which has an expected
\(\text{SSError}\) approximately 0) gains a large number of wins.

\begin{figure}[!htbp]
    \centering
    \includegraphics[width=.8\textwidth]{./assets/img/SSError_overall_in_stewart_plotkin/main.pdf}
    \caption{\(\text{SSError}\) and state probabilities for the strategies
        of~\cite{Stewart2012}, ordered both by number of wins and overall score.
        Note that \(P(DC)\) is not shown as it corresponds to the transpose of
        \(P(CD)\). Cooperator and Defector are omitted as they do not visit all
        the states.}
    \label{fig:SSError_overall_in_stewart_plotkin}
\end{figure}

Here, the work of~\cite{Stewart2012} is extended by investigating a tournament
with \input{assets/tex/number_of_full_strategies/main.tex}
strategies.

The results of this analysis are shown in
Figure~\ref{fig:SSError_and_probabilities_in_full}. The top ranking strategies
by number of wins seem to be extortionate (but not against all strategies) and
it can be seen that a small sub group of strategies achieve mutual defection.
All the top ranking strategies according to score achieve mutual cooperation and
do not extort each other, however they
\textbf{do} exhibit extortionate behaviour towards a number of the lower ranking
strategies.

\begin{figure}[!htbp]
    \centering
    \includegraphics[width=.8\textwidth]{./assets/img/SSError_and_probabilities_in_full/main.pdf}
    \caption{\(\text{SSError}\) for the strategies for the full tournament. Only
    strategy interactions for which \(p_4=0\) and \(\chi>1\) are displayed.}
    \label{fig:SSError_and_probabilities_in_full}
\end{figure}

\section{Conclusion}\label{sec:conclusion}

This work defines an approach to measure whether or not a player is playing a
strategy that corresponds to an extortionate strategy as defined
in~\cite{Press2012}: a mathematical model for suspicion. Indeed, all
extortionate strategies have been
 classified as lying on a triangular plane.
This rigorous classification fails to be robust to small measurement error, thus
a statistical approach is proposed.
This is done through a linear algebraic approach for approximating the solution
of a linear system. Using this, a large number of pairwise interactions is
simulated and in fact very few strategies are found to act extortionately.

The work of~\cite{Press2012}, whilst showing that a clever approach to taking
advantage of another memory one strategy exists: this is incomplete. Whilst the
elegance of this result is very attractive, just as the simplicity of the
victory of Tit For Tat in Axelrod's original tournaments was, it is incomplete.
Extortionate strategies achieve a high number of wins but they do not
achieve a high score which corresponds to the fitness landscape in an
evolutionary sense. From the large number of interactions a payoff matrix \(S\)
can be measured where \(S_{ij}\) denotes the score (using standard values of
\((R, S, T, P) = (3, 0, 5, 1)\)) of the \(i\)th strategy
against the \(j\)th strategy. Using this, the replicator equation
describes the evolution of the system based on a population density fitness
function:

\begin{equation}\label{eqn:replicator_dynamics}
    \frac{dx}{dt} = x(S-x^TS x)
\end{equation}

Equation (\ref{eqn:replicator_dynamics}) is solved numerically through an
integration technique described in~\cite{Petzold1983} and
Figure~\ref{fig:replicator_dynamics} shows the evolution of the distribution of
the system: the various strategies are ranked by scores. It is clear to see that
only the high ranking strategies survive the evolutionary process (in fact,
only \input{./assets/img/replicator_dynamics/main.tex}
have a final distribution greater than \(10 ^ {-2}\)). This confirms the
findings of~\cite{Moran1707} in which sophisticated strategies resist
evolutionary invasion of shorter memory strategies. Recalling
Figure~\ref{fig:SSError_and_probabilities_in_full} this demonstrates that:

\begin{itemize}
    \item Cooperation emerges through the evolutionary process: the high scoring
        strategies do not exhibit extortionate behaviour towards each other.
    \item Extortionate strategies do not survive the evolutionary process.
\end{itemize}

\begin{figure}[!htbp]
    \centering
    \includegraphics[width=.8\textwidth]{./assets/img/replicator_dynamics/main.pdf}
    \caption{Numerical simulation of the replicator equation
    (\ref{eqn:replicator_dynamics}): strategies are ordered by score, only the strategies with a high score survive the evolutionary process.}
    \label{fig:replicator_dynamics}
\end{figure}

This work can be used to classify plays of the IPD\@: data can be collected from
actual interactions (in lab or in the field). Furthermore, this allows for a
classification method similar to the notion of fingerprinting presented
in~\cite{Ashlock2008}. Trained strategies can potentially be classified as
extortionate or not or it could be possible to even constrain the reinforcement
learning approaches that are becoming prevalent in the literature.
Alternatively, this mathematical approach for recognising extortion could be
used in sophisticated strategies to defend against invasion. Arguably, some of
the strategies considered here exhibit this behaviour, indeed as described
in~\cite{Harper2017}, the top ranking strategies in the full tournament are
obtained using evolutionary reinforcement learning techniques, thus, suspicion
of extortionate behaviour could in fact be an evolutionary trait.

\section*{Acknowledgements}

The following open source software libraries were used in this research:

\begin{itemize}
    \item The Axelrod ~\cite{Knight2016, Knight2018} library (IPD strategies and
        tournaments).
    \item The sympy library~\cite{Meurer2017} (verification of all symbolic
        calculations).
    \item The matplotlib~\cite{Droettboom2018} library (visualisation).
    \item The pandas~\cite{Structures2010}, dask~\cite{Dask2016} and
        NumPy~\cite{Oliphant2015} libraries (data manipulation).
    \item The SciPy~\cite{Jones2001} library (numerical integration of the
        replicator equation).
\end{itemize}

This work was performed using the computational facilities of the Advanced
Research Computing @ Cardiff (ARCCA) Division, Cardiff University.

\printbibliography

\newpage
\section*{Supplementary materials}

\includepdf{assets/pdf/proof_of_form_of_extortionate_strategies/main.pdf}

\newpage

Using the pair wise interactions the transition rates \(p,
q\) can be measured and the steady state probabilities inferred and compared to
the actual probabilities of each state.
This is done numerically by computing the singular eigenvector of the
matrix \(A\) \cite{Stewart2009}:

\[
    A =
    \begin{bmatrix}
        p_1 q_1 & p_1 (1 - q_1) & (1 - p_1) q_1 & (1 -p_1) (1 - q_1) \\
        p_2 q_2 & p_2 (1 - q_2) & (1 - p_2) q_2 & (1 -p_2) (1 - q_2) \\
        p_3 q_3 & p_3 (1 - q_3) & (1 - p_3) q_3 & (1 -p_3) (1 - q_3) \\
        p_4 q_4 & p_4 (1 - q_4) & (1 - p_4) q_4 & (1 -p_4) (1 - q_4) \\
    \end{bmatrix}
\]

Figure~\ref{fig:computed_probabilities_vs_theoretic_probabilities} shows a
regression line fitted to every pairwise interaction with a reported
\(\text{SSError}\) value (pairwise interactions with missing states were
omitted). This serves to validate the approach: a part from some edge cases the
relationship is consistent.

\begin{figure}[!htbp]
    \centering
    \includegraphics[width=.8\textwidth]{./assets/img/computed_probabilities_vs_theoretic_probabilities/main.pdf}
    \caption{The
        relationship between the steady state probabilities inferred from the
        measured transitions and the actual steady state probabilities. A linear
        regression line is included validating the approach.}
    \label{fig:computed_probabilities_vs_theoretic_probabilities}
\end{figure}


\end{document}

have a final distribution greater than \(10 ^ {-2}\)). This confirms the
findings of~\cite{Moran1707} in which sophisticated strategies resist
evolutionary invasion of shorter memory strategies. Recalling
Figure~\ref{fig:SSError_and_probabilities_in_full} this demonstrates that:

\begin{itemize}
    \item Cooperation emerges through the evolutionary process: the high scoring
        strategies do not exhibit extortionate behaviour towards each other.
    \item Extortionate strategies do not survive the evolutionary process.
\end{itemize}

\begin{figure}[!htbp]
    \centering
    \includegraphics[width=.8\textwidth]{./assets/img/replicator_dynamics/main.pdf}
    \caption{Numerical simulation of the replicator equation
    (\ref{eqn:replicator_dynamics}): strategies are ordered by score, only the strategies with a high score survive the evolutionary process.}
    \label{fig:replicator_dynamics}
\end{figure}

This work can be used to classify plays of the IPD\@: data can be collected from
actual interactions (in lab or in the field). Furthermore, this allows for a
classification method similar to the notion of fingerprinting presented
in~\cite{Ashlock2008}. Trained strategies can potentially be classified as
extortionate or not or it could be possible to even constrain the reinforcement
learning approaches that are becoming prevalent in the literature.
Alternatively, this mathematical approach for recognising extortion could be
used in sophisticated strategies to defend against invasion. Arguably, some of
the strategies considered here exhibit this behaviour, indeed as described
in~\cite{Harper2017}, the top ranking strategies in the full tournament are
obtained using evolutionary reinforcement learning techniques, thus, suspicion
of extortionate behaviour could in fact be an evolutionary trait.

\section*{Acknowledgements}

The following open source software libraries were used in this research:

\begin{itemize}
    \item The Axelrod ~\cite{Knight2016, Knight2018} library (IPD strategies and
        tournaments).
    \item The sympy library~\cite{Meurer2017} (verification of all symbolic
        calculations).
    \item The matplotlib~\cite{Droettboom2018} library (visualisation).
    \item The pandas~\cite{Structures2010}, dask~\cite{Dask2016} and
        NumPy~\cite{Oliphant2015} libraries (data manipulation).
    \item The SciPy~\cite{Jones2001} library (numerical integration of the
        replicator equation).
\end{itemize}

This work was performed using the computational facilities of the Advanced
Research Computing @ Cardiff (ARCCA) Division, Cardiff University.

\printbibliography

\newpage
\section*{Supplementary materials}

\includepdf{assets/pdf/proof_of_form_of_extortionate_strategies/main.pdf}

\newpage

Using the pair wise interactions the transition rates \(p,
q\) can be measured and the steady state probabilities inferred and compared to
the actual probabilities of each state.
This is done numerically by computing the singular eigenvector of the
matrix \(A\) \cite{Stewart2009}:

\[
    A =
    \begin{bmatrix}
        p_1 q_1 & p_1 (1 - q_1) & (1 - p_1) q_1 & (1 -p_1) (1 - q_1) \\
        p_2 q_2 & p_2 (1 - q_2) & (1 - p_2) q_2 & (1 -p_2) (1 - q_2) \\
        p_3 q_3 & p_3 (1 - q_3) & (1 - p_3) q_3 & (1 -p_3) (1 - q_3) \\
        p_4 q_4 & p_4 (1 - q_4) & (1 - p_4) q_4 & (1 -p_4) (1 - q_4) \\
    \end{bmatrix}
\]

Figure~\ref{fig:computed_probabilities_vs_theoretic_probabilities} shows a
regression line fitted to every pairwise interaction with a reported
\(\text{SSError}\) value (pairwise interactions with missing states were
omitted). This serves to validate the approach: a part from some edge cases the
relationship is consistent.

\begin{figure}[!htbp]
    \centering
    \includegraphics[width=.8\textwidth]{./assets/img/computed_probabilities_vs_theoretic_probabilities/main.pdf}
    \caption{The
        relationship between the steady state probabilities inferred from the
        measured transitions and the actual steady state probabilities. A linear
        regression line is included validating the approach.}
    \label{fig:computed_probabilities_vs_theoretic_probabilities}
\end{figure}


\end{document}

have a final distribution greater than \(10 ^ {-2}\)). This confirms the
findings of~\cite{Moran1707} in which sophisticated strategies resist
evolutionary invasion of shorter memory strategies. Recalling
Figure~\ref{fig:SSError_and_probabilities_in_full} this demonstrates that:

\begin{itemize}
    \item Cooperation emerges through the evolutionary process: the high scoring
        strategies do not exhibit extortionate behaviour towards each other.
    \item Extortionate strategies do not survive the evolutionary process.
\end{itemize}

\begin{figure}[!htbp]
    \centering
    \includegraphics[width=.8\textwidth]{./assets/img/replicator_dynamics/main.pdf}
    \caption{Numerical simulation of the replicator equation
    (\ref{eqn:replicator_dynamics}): strategies are ordered by score, only the strategies with a high score survive the evolutionary process.}
    \label{fig:replicator_dynamics}
\end{figure}

This work can be used to classify plays of the IPD\@: data can be collected from
actual interactions (in lab or in the field). Furthermore, this allows for a
classification method similar to the notion of fingerprinting presented
in~\cite{Ashlock2008}. Trained strategies can potentially be classified as
extortionate or not or it could be possible to even constrain the reinforcement
learning approaches that are becoming prevalent in the literature.
Alternatively, this mathematical approach for recognising extortion could be
used in sophisticated strategies to defend against invasion. Arguably, some of
the strategies considered here exhibit this behaviour, indeed as described
in~\cite{Harper2017}, the top ranking strategies in the full tournament are
obtained using evolutionary reinforcement learning techniques, thus, suspicion
of extortionate behaviour could in fact be an evolutionary trait.

\section*{Acknowledgements}

The following open source software libraries were used in this research:

\begin{itemize}
    \item The Axelrod ~\cite{Knight2016, Knight2018} library (IPD strategies and
        tournaments).
    \item The sympy library~\cite{Meurer2017} (verification of all symbolic
        calculations).
    \item The matplotlib~\cite{Droettboom2018} library (visualisation).
    \item The pandas~\cite{Structures2010}, dask~\cite{Dask2016} and
        NumPy~\cite{Oliphant2015} libraries (data manipulation).
    \item The SciPy~\cite{Jones2001} library (numerical integration of the
        replicator equation).
\end{itemize}

This work was performed using the computational facilities of the Advanced
Research Computing @ Cardiff (ARCCA) Division, Cardiff University.

\printbibliography

\newpage
\section*{Supplementary materials}

\includepdf{assets/pdf/proof_of_form_of_extortionate_strategies/main.pdf}

\newpage

Using the pair wise interactions the transition rates \(p,
q\) can be measured and the steady state probabilities inferred and compared to
the actual probabilities of each state.
This is done numerically by computing the singular eigenvector of the
matrix \(A\) \cite{Stewart2009}:

\[
    A =
    \begin{bmatrix}
        p_1 q_1 & p_1 (1 - q_1) & (1 - p_1) q_1 & (1 -p_1) (1 - q_1) \\
        p_2 q_2 & p_2 (1 - q_2) & (1 - p_2) q_2 & (1 -p_2) (1 - q_2) \\
        p_3 q_3 & p_3 (1 - q_3) & (1 - p_3) q_3 & (1 -p_3) (1 - q_3) \\
        p_4 q_4 & p_4 (1 - q_4) & (1 - p_4) q_4 & (1 -p_4) (1 - q_4) \\
    \end{bmatrix}
\]

Figure~\ref{fig:computed_probabilities_vs_theoretic_probabilities} shows a
regression line fitted to every pairwise interaction with a reported
\(\text{SSError}\) value (pairwise interactions with missing states were
omitted). This serves to validate the approach: a part from some edge cases the
relationship is consistent.

\begin{figure}[!htbp]
    \centering
    \includegraphics[width=.8\textwidth]{./assets/img/computed_probabilities_vs_theoretic_probabilities/main.pdf}
    \caption{The
        relationship between the steady state probabilities inferred from the
        measured transitions and the actual steady state probabilities. A linear
        regression line is included validating the approach.}
    \label{fig:computed_probabilities_vs_theoretic_probabilities}
\end{figure}


\end{document}
have a stationary
probability value greater than \(10 ^ {-2}\)).

\begin{figure}[!htbp]
    \centering
    \includegraphics[width=.8\textwidth]{./assets/img/replicator_dynamics/main.pdf}
    \caption{Stationary distribution of the replicator dynamics
    (\ref{eqn:replicator_dynamics}): strategies are ordered by score. Note that
    strategies that make use of the knowledge of the length of the game are
    removed from this analysis as they have an evolutionary advantage.}
    \label{fig:replicator_dynamics}
\end{figure}

Figure~\ref{fig:compare-evolutionary-dynamics-to-sserror} plots the mean and
skew (a standard statistical measure on a distribution) of \(\SSe\) against the
stationary probabilities \(s\) of (\ref{eqn:replicator_dynamics}). Strategies
that perform strongly according to equation (\ref{eqn:replicator_dynamics}) seem
to be strategies that have a negative skew of \(\SSe\): indicating that they
often have a high value of \(\SSe\) (i.e. do not act extortionately) but have a
long left tail allowing them to adapt when necessary. A general linear model
obtained using recursive feature elimination is shown in
Table~\ref{tbl:compare-evolutionary-dynamics-to-sserror} with stronger
predictive power and confirming these conclusions.

\begin{figure}[!hbtp]
    \centering
    \includegraphics[width=\textwidth]{./assets/img/compare-evolutionary-dynamics-to-sserror/main.pdf}
    \caption{Mean, variance and skew of \(\SSe\) versus the stationary
    probabilities of (\ref{eqn:replicator_dynamics}). The plot of the skew
    clearly shows that all high probabilities have a negative skew.}
    \label{fig:compare-evolutionary-dynamics-to-sserror}
\end{figure}

\begin{table}[!hbtp]
    \begin{center}
    \tiny
    \begin{center}
\begin{tabular}{lclc}
\toprule
\textbf{Dep. Variable:}    &      $s_i$       & \textbf{  R-squared:         } &    0.648  \\
\textbf{Model:}            &       OLS        & \textbf{  Adj. R-squared:    } &    0.642  \\
\textbf{Method:}           &  Least Squares   & \textbf{  F-statistic:       } &    117.0  \\
\textbf{Date:}             & Fri, 21 Dec 2018 & \textbf{  Prob (F-statistic):} & 5.00e-43  \\
\textbf{Time:}             &     11:01:35     & \textbf{  Log-Likelihood:    } &   851.41  \\
\textbf{No. Observations:} &         195      & \textbf{  AIC:               } &   -1695.  \\
\textbf{Df Residuals:}     &         191      & \textbf{  BIC:               } &   -1682.  \\
\textbf{Df Model:}         &           3      & \textbf{                     } &           \\
\textbf{Covariance Type:}  &    nonrobust     & \textbf{                     } &           \\
\bottomrule
\end{tabular}
%\caption{OLS Regression Results}
\end{center}\begin{center}
\begin{tabular}{lcccccc}
\toprule
                                & \textbf{coef} & \textbf{std err} & \textbf{t} & \textbf{P$>$$|$t$|$} & \textbf{[0.025} & \textbf{0.975]}  \\
\midrule
\textbf{const}                  &       0.0007  &        0.001     &     1.137  &         0.257        &       -0.000    &        0.002     \\
\textbf{('SSE', 'mean')}   &      -0.0134  &        0.002     &    -8.369  &         0.000        &       -0.017    &       -0.010     \\
\textbf{('SSE', 'median')} &       0.0139  &        0.001     &    10.433  &         0.000        &        0.011    &        0.017     \\
\textbf{('SSE', 'var')}    &       0.0069  &        0.003     &     2.402  &         0.017        &        0.001    &        0.013     \\
\bottomrule
\end{tabular}
\end{center}\begin{center}
\begin{tabular}{lclc}
\toprule
\textbf{Omnibus:}       & 17.190 & \textbf{  Durbin-Watson:     } &    1.664  \\
\textbf{Prob(Omnibus):} &  0.000 & \textbf{  Jarque-Bera (JB):  } &   25.453  \\
\textbf{Skew:}          &  0.530 & \textbf{  Prob(JB):          } & 2.97e-06  \\
\textbf{Kurtosis:}      &  4.418 & \textbf{  Cond. No.          } &     23.7  \\
\bottomrule
\end{tabular}
\end{center}
    \end{center}
    \caption{General linear model. This shows that strategies with a low mean
    and high median are more likely to survive the evolutionary dynamics. This
    corresponds to negatively skewed distributions of \(\SSe\) which again
    highlights the importance of adaptability.}
    \label{tbl:compare-evolutionary-dynamics-to-sserror}
\end{table}

Figure~\ref{fig:sserror_distribution_for_selection_of_strategies} shows the
distribution of the \(\SSe\) for three selected strategies. It is evident that
Extort-2 almost always has the same low value of \(\SSe\) against all opponents
(which gives a positively skewed distribution), whereas EvolvedLookerUp2\_2\_2
and Tit For Tat have a wider distribution of values depending on the opponent
(which gives a negatively skewed distribution).

\begin{figure}[!hbtp]
    \centering
    \includegraphics[width=\textwidth]{./assets/img/sserror_distribution_for_selection_of_strategies/main.pdf}
    \caption{Distribution of \(\SSe\) values for 3 selected strategies. The
    first two distributions are negatively skewed and the third has a positive
    skew.}
    \label{fig:sserror_distribution_for_selection_of_strategies}
\end{figure}

\subsection{Finite Population Dynamics: Moran Process}

In~\cite{Moran1707} a large data set of pairwise fixation probabilities in the
Moran process is made available at~\cite{vincent_knight_2017_1040129}
Figure~\ref{fig:compare-fixation-to-sserror} shows linear models fitted to three
summary measures of \(\SSe\) and the mean (over population size \(N\) and
opponents) value of \(x_1\cdot N\). This specific measure of fixation is chosen
as \(x_1\) is usually compared to the neutral fixation probability of \(1 / N\).
As was noted in~\cite{Moran1707}, the specific case of \(N=2\) differs from all
other population sizes which is why it is presented in isolation.
We note that there is a significant relationship between the skew of
\(\SSe\) and the ability for a strategy to become fixed.
A general linear model obtained through recursive feature elimination is shown
in Table~\ref{tbl:compare-fixation-to-sserror} which confirms the conclusions.

\begin{figure}[!hbtp]
    \centering
    \includegraphics[width=\textwidth]{./assets/img/compare-fixation-to-sserror/main.pdf}
    \caption{The mean, variance and skew of
    \(\SSe\) against the normalised pairwise fixation probabilities
    from~\cite{Moran1707} (for a given strategy averaged over all opponents and
    population sizes). The clustering either side of a value of skew equal to
    0 show that strategies with above neutral
    fixation (\(N\cdot x_1>1\)) negative skew.}
    \label{fig:compare-fixation-to-sserror}
\end{figure}

\begin{table}[!hbtp]
    \begin{center}
    \tiny
    \begin{center}
\begin{tabular}{lclc}
\toprule
\textbf{Dep. Variable:}    &       mean       & \textbf{  R-squared:         } &    0.319  \\
\textbf{Model:}            &       OLS        & \textbf{  Adj. R-squared:    } &    0.310  \\
\textbf{Method:}           &  Least Squares   & \textbf{  F-statistic:       } &    36.53  \\
\textbf{Date:}             & Sat, 17 Nov 2018 & \textbf{  Prob (F-statistic):} & 9.74e-14  \\
\textbf{Time:}             &     23:33:54     & \textbf{  Log-Likelihood:    } &  -42.272  \\
\textbf{No. Observations:} &         159      & \textbf{  AIC:               } &    90.54  \\
\textbf{Df Residuals:}     &         156      & \textbf{  BIC:               } &    99.75  \\
\textbf{Df Model:}         &           2      & \textbf{                     } &           \\
\textbf{Covariance Type:}  &    nonrobust     & \textbf{                     } &           \\
\bottomrule
\end{tabular}
%\caption{OLS Regression Results}
\end{center}\begin{center}
\begin{tabular}{lcccccc}
\toprule
                                & \textbf{coef} & \textbf{std err} & \textbf{t} & \textbf{P$>$$|$t$|$} & \textbf{[0.025} & \textbf{0.975]}  \\
\midrule
\textbf{const}                  &       1.2815  &        0.056     &    22.993  &         0.000        &        1.171    &        1.392     \\
\textbf{('residual', 'mean')}   &      -1.0620  &        0.145     &    -7.323  &         0.000        &       -1.348    &       -0.776     \\
\textbf{('residual', 'median')} &       0.9037  &        0.106     &     8.535  &         0.000        &        0.695    &        1.113     \\
\bottomrule
\end{tabular}
\end{center}\begin{center}
\begin{tabular}{lclc}
\toprule
\textbf{Omnibus:}       &  2.302 & \textbf{  Durbin-Watson:     } &    1.716  \\
\textbf{Prob(Omnibus):} &  0.316 & \textbf{  Jarque-Bera (JB):  } &    1.850  \\
\textbf{Skew:}          & -0.199 & \textbf{  Prob(JB):          } &    0.397  \\
\textbf{Kurtosis:}      &  3.348 & \textbf{  Cond. No.          } &     11.2  \\
\bottomrule
\end{tabular}
\end{center}
    \end{center}
    \caption{General linear model. This shows that strategies with a high mean
        and low median are likely to be evolutionarily stable. This corresponds
        to negatively skewed distributions of \(\SSe\) which again highlights
        the importance of adaptability.}
    \label{tbl:compare-fixation-to-sserror}
\end{table}

These findings confirm the work of~\cite{Moran1707} in which sophisticated
strategies resist evolutionary invasion of shorter memory strategies. This also
confirms the work of~\cite{adami2013evolutionary, hilbe2015partners} which
proved that ZD strategies where not evolutionarily stable due to the fact that
they score poorly against themselves.

The work also provides strong evidence to the importance of adaptability:
strategies that offer a variety of behaviours corresponding to a higher standard
deviation of \(\SSe\) are significantly more likely to survive the
evolutionary process. This corresponds to the following quote
of~\cite{darwin1869origin}:

\begin{quote}
``It is not the most intellectual of the species that survives; it is not the
strongest that survives; but the species that survives is the one that is able
to adapt to and to adjust best to the changing environment in which it finds
itself.''
\end{quote}

\section{Discussion}\label{sec:conclusion}

This work defines an approach to measure whether or not a player is using an
extortionate strategy as defined in~\cite{Press2012}, or a strategy that behaves
similarly, broadening the definition of extortionate behavior. All extortionate
strategies have been classified as lying on a triangular plane. This rigorous
classification fails to be robust to small measurement error, thus a statistical
approach is proposed approximating the solution of a linear system.
This method
was applied to a large number of pairwise interactions.

The work of~\cite{Press2012}, while showing that a clever approach to taking
advantage of another memory-one strategy exists, is not the full story.
Though the elegance of this result is very attractive, just as the simplicity of
the victory of Tit For Tat in Axelrod's original tournaments was, it is
incomplete and in the author's opinions, has been oversimplified and
overgeneralized in subsequent work. Extortionate strategies achieve a high
number of wins but they do generally not achieve a high score and fail to be
evolutionarily stable.

Rather more sophisticated strategies are able to adapt to a variety of opponents
and act extortionately only against weaker strategies while cooperating with
like-minded strategies that are not susceptible to extortion. This adaptability
may be key to maintaining sustained cooperation, as some of these strategies
emerged naturally from evolutionary processes trained to maximize payoff in
IPD tournaments and fixation in population dynamics.

Following Axelrod's seminal work~\cite{Axelrod1980, Axelrod1980a}, it was
commonly thought that evolutionary cooperation required strategies that followed
a simple set of rules. The discovery/definition of extortionate
strategies~\cite{Press2012} seemingly showed that complex strategies could be
taken advantage of. In this manuscript it has been shown that not only is it
possible to detect and prevent extortionate behaviour but that more complex
strategies can be evolutionary stable. The complex strategies in question were
obtained through reinforcement learning approaches~\cite{Harper2017, Moran1707}.
Thus, this demonstrates that it is possible to recognise extortion, both
theoretically using \(\SSe\) but also that this ability can develop through
reinforcement learning. It seems human difficulty in directly developing
effective complex strategies has been incorrectly generalized to a weakness
in complex strategies themselves, which is demonstrable not the case. In fact,
complex strategies can be the most effective against a diverse set of opponents.

In closing, the authors wish to emphasize the role of comprehensive simulations to temper
theoretical results from overgeneralization, and perhaps more importantly, the
ability of simulations to provide insights that are difficult to obtain from theory.

\section*{Acknowledgements}

The following open source software libraries were used in this research:

\begin{itemize}
    \item The Axelrod ~\cite{Knight2016, Knight2018} library (IPD strategies and
        tournaments).
    \item The sympy library~\cite{Meurer2017} (verification of all symbolic
        calculations).
    \item The matplotlib~\cite{Droettboom2018} library (visualisation).
    \item The pandas~\cite{Structures2010}, dask~\cite{Dask2016} and
        NumPy~\cite{Oliphant2015} libraries (data manipulation).
    \item The SciPy~\cite{Jones2001} library (numerical integration of the
        replicator equation).
\end{itemize}

This work was performed using the computational facilities of the Advanced
Research Computing @ Cardiff (ARCCA) Division, Cardiff University.

\section*{Author contributions}

VK and NG conceived the idea. MH, JG, NG and VK were all involved in carrying
out the research and writing the manuscript.

\printbibliography

\includepdf{assets/pdf/proof_of_form_of_extortionate_strategies/main.pdf}
\includepdf{assets/img/sse_error_in_std_for_auxiliary/main.pdf}
\includepdf[pages={1-}]{assets/pdf/list_of_strategies/main.pdf}

\end{document}
