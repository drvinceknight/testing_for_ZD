\documentclass[a4paper]{article}

\usepackage{amsmath}
\usepackage[margin=1.5cm,
            includefoot,
            footskip=30pt]{geometry}
\usepackage{layout}

\title{Recognising and evaluating the effectiveness
       of extortion in the Iterated Prisoner's Dilemma}
\author{Vincent A. Knight \and Nikoleta E. Glynatsi}
\date{\today}


\begin{document}

\maketitle

\begin{abstract}
    The Iterated Prisoner's Dilemma is a model for rational and evolutionary
    interactive behaviour. It has applications both in the study of human social
    behaviour as well as in biology.

    This game is used to understand when and how a rational individual might
    accept an immediate cost to their own utility for the direct benefit of
    another.

    Much attention has been given to a class of strategies for this game, called
    Zero Determinant strategies. It has been theoretically shown that these
    strategies can ``extort'' any player.

    In this work, an approach to identify if observed strategies are playing in
    a Zero Determinant way is described. Furthermore, experimental analysis of
    a large tournament with 204 % TODO (Read in number of strategies)
    strategies is considered. In this setting
    the most highly performing strategies do not play in a Zero Determinant way.
    This suggests that whilst the theory of Zero Determinant strategies
    indicates that memory is not of fundamental importance to the evolution of
    cooperative behaviour, this is incomplete.
\end{abstract}

\section{Introduction}\label{sec:introduction}

% TODO Write short introduction and literature review.
% - Describe P&D, 
% - Describe SP, 
% - Describe Hilbe 2013 (evolution of extortion)
% - Describe Knight et al
% - statement about reproducibility of work

\section{Identifying Extortion}\label{sec:delta-zd-strategies}

% TODO Describe ZD

% TODO Describe example of finding alpha, gamma and beta: put this in terms of
% rank etc... (and the Rouché–Capelli theorem)

% TODO Add description of how this implies that the family of ZD strategies is
% in fact uniquely defined by 3 probabilities.

% TODO Discuss missing states

% TODO Discuss adding in constraint for extortion:
% $$-P\alpha-P\beta-\omega=0$$ which implies this is only possible when $p_4=0$
% This is "easy" as it means we can simply measure this from the data.
% Thus to identify if extortion: check $p_cc = 0$, $-\beta/\alpha>1$ and that
% $R^2 < $ some threshold.

\section{Numerical experiments}\label{sec:numerical-experiments}

% TODO Identify zero determinant, plot $R^2$ with the conditions as a
% marker/color
% TODO Show S&P as proof of concept: Show results scores/wins/ranks
% TODO Show full tournament: Show results scores/wins/ranks

% TODO Discuss pair wise and compare to CC probability

\section{Conclusion}\label{sec:conclusion}

% TODO In a large tournament, despite there being memory 1 strategies trained to
% perform well [] they do not play ZD.
% Potentially fallacy about the stability of playing short memory strategies.

\end{document}
